\upaper{25}{Las multitudes de mensajeros del espacio}
\author{Un elevado en autoridad}
\vs p025 0:1 Las multitudes de mensajeros del espacio ocupan un lugar intermedio dentro de la clasificación de la familia del Espíritu Infinito. Estos versátiles seres actúan de elementos de conexión entre los seres personales superiores y los espíritus servidores. Dentro de ellos se incluyen los siguientes órdenes de seres celestiales:
\vs p025 0:2 \li{1.}Los servitales de Havona.
\vs p025 0:3 \li{2.}Los conciliadores universales.
\vs p025 0:4 \li{3.}Los asesores técnicos.
\vs p025 0:5 \li{4.}Los custodios de archivos del Paraíso.
\vs p025 0:6 \li{5.}Los archivistas celestiales.
\vs p025 0:7 \li{6.}Los acompañantes morontiales.
\vs p025 0:8 \li{7.}Los acompañantes del Paraíso.
\vs p025 0:9 \pc De estos siete grupos, solamente tres, los servitales, los conciliadores y los acompañantes morontiales se crean como tales; los cuatro restantes resultan de órdenes angélicas que han conseguido esta denominación por sus logros. De acuerdo con las características de su naturaleza y la condición que han adquirido, las multitudes de mensajeros sirven de forma muy diversa en el universo de los universos, aunque siempre bajo la dirección de quienes gobiernan los lugares a los que se les ha asignado.
\usection{1. LOS SERVITALES DE HAVONA}
\vs p025 1:1 Aunque se denominan servitales, estas “criaturas intermedias” del universo central no son sirvientes en ningún sentido despectivo. En el mundo espiritual no hay tareas que se pudieran considerar serviles; todo servicio es sagrado y estimulante; tampoco los seres de orden superior menosprecian a los de orden más modesto.
\vs p025 1:2 \pc Los servitales de Havona tienen su origen en la unión creativa de los siete espíritus mayores y de sus compañeros, los siete directores supremos de la potencia. Tal colaboración creativa se constituye como el modelo más cercano para una larga lista de reproducciones de tipo doble, que se realizan en los universos evolutivos, y que se extienden desde la creación de una brillante estrella de la mañana mediante la conjunción de un hijo creador y de un espíritu materno creativo hasta la procreación sexual propia de mundos como Urantia.
\vs p025 1:3 El número de servitales es prodigioso y cada vez se crean más. Aparecen en grupos de mil en el tercer momento tras la asamblea de los espíritus mayores y de los directores supremos de la potencia, en un área compartida, situada en el sector más septentrional del Paraíso. Cada cuarto servital es de índole más física que los demás; o sea, de mil, setecientos cincuenta son aparentemente seres de índole auténticamente espiritual, pero doscientos cincuenta son de naturaleza semifísica. Estas \bibemph{cuartas criaturas} son, en cierto modo, del orden de seres materiales (material en el sentido de Havona), semejándose más a los directores de la potencia física que a los espíritus mayores.
\vs p025 1:4 \pc En las relaciones entre seres personales, lo espiritual domina sobre lo material, aunque ahora no parezca así en Urantia; y en la creación de los servitales de Havona, prevalece la ley del predominio del espíritu; la proporción establecida arroja un resultado de tres seres espirituales para uno semifísico.
\vs p025 1:5 \pc Todos los servitales recién creados, junto con los guías de los graduados que van apareciendo, pasan por cursos de formación que los guías de mayor rango dirigen continuamente en cada una de las siete vías circulatorias de Havona. Luego, los servitales se asignan a la actividad para la que están mejor adaptados y, puesto que son de dos tipos ---espiritual y semifísico---, hay muy pocos límites en la variedad de tareas que estos versátiles seres pueden realizar. A los grupos superiores o espirituales se les destina de forma selectiva al servicio del Padre, el Hijo y el Espíritu, y al trabajo de los siete espíritus mayores. Periódicamente, se les envía en gran número para servir en los mundos de estudio que circundan las esferas sedes de los siete suprauniversos. Estos mundos están dedicados a la formación final y al cultivo espiritual de las almas ascendentes del tiempo que se preparan para avanzar a las vías de Havona. Tanto a los servitales espirituales como a sus semejantes más físicos también se les designa como asistentes y colaboradores de los guías de los graduados para asistir en la instrucción de los varios órdenes de criaturas ascendentes, que han conseguido llegar a Havona e intentan alcanzar el Paraíso.
\vs p025 1:6 Los servitales de Havona y los guías de los graduados manifiestan una extraordinaria devoción hacia su trabajo y se profesan entre ellos un afecto conmovedor, un afecto que, a pesar de ser espiritual, solamente lo podríais comprender si lo comparáis con el fenómeno del amor humano. Cuando los servitales se separan de los guías, algo que ocurre con mucha frecuencia cuando se les envía en misiones más allá de los límites del universo central, se evidencia una gran tristeza divina. Pero los servitales van con regocijo y no con pena. En los seres espirituales, el gozo por la satisfacción de cumplir con un deber sublime es una emoción que eclipsa a cualquier otra. No puede existir aflicción frente a la conciencia del deber divino que se lleva a cabo fielmente. Cuando el alma ascendente del hombre esté frente al Juez Supremo, la resolución que tendrá importancia para la eternidad no se determinará por los éxitos materiales ni por la cantidad de logros; el veredicto que resuena en los tribunales superiores proclama: “Bien, siervo bueno y \bibemph{fiel;} por cuanto en lo poco has sido fiel, tendrás autoridad sobre las realidades del universo”.
\vs p025 1:7 En el servicio que se realiza en el suprauniverso, a los servitales siempre se les asigna a los dominios presididos por el espíritu mayor a quien ellos más se parecen en cuanto a prerrogativas espirituales de carácter general y especial. Realizan su labor solamente en los mundos de instrucción que rodean a las capitales de los siete suprauniversos. El último informe de Uversa muestra que casi 138 mil millones de servitales realizaban su ministerio en sus 490 satélites. Los servitales se encargan de una variedad interminable de actividades en relación con el cometido a realizar en estos mundos de instrucción, que constituyen las incomparables universidades del suprauniverso de Orvontón. Aquí son vuestros acompañantes; han descendido desde un plano al que llegaréis en vuestra andadura, para estudiaros e inspiraros con la realidad y la certidumbre de que acabaréis por graduaros de los universos del tiempo y llegar a los reinos de la eternidad. En estos contactos, los servitales adquieren esa experiencia previa en ese servicio a las criaturas ascendentes del tiempo que tan útil les resultará para su labor posterior en las vías de Havona como colaboradores de los guías de los graduados o ---como servitales trasladados--- en calidad misma de guías de los graduados.
\usection{2. LOS CONCILIADORES UNIVERSALES}
\vs p025 2:1 Por cada servital de Havona que se crea, se traen a la existencia a siete conciliadores universales, uno en cada suprauniverso. Esta acción creativa forma parte de un determinado procedimiento de respuesta de los suprauniversos, reflectora de las actuaciones que tienen lugar en el Paraíso.
\vs p025 2:2 En los mundos sedes de los suprauniversos, desempeñan su labor los siete reflejos de los espíritus mayores. Si bien, resulta difícil tratar de describir para la mente material la naturaleza de estos espíritus reflectores. Son verdaderos seres personales; aun así, cada uno de los miembros de un grupo del suprauniverso es un reflejo perfecto de solamente uno de los siete espíritus mayores. Y cada vez que estos espíritus se vinculan con los directores de la fuerza con el objetivo de crear un grupo de servitales de Havona, se produce una convergencia simultánea sobre uno de los espíritus reflectores en cada uno de los grupos de los suprauniversos y, de forma inmediata y completamente acabada, aparece un número igual de conciliadores universales en los mundos sedes de las supracreaciones. Si en la creación de los servitales tomara la iniciativa el espíritu mayor número siete, nadie con excepción de los espíritus reflectores del orden séptimo engendraría conciliadores y, coincidiendo con la creación de mil servitales del tipo característico de Orvontón, aparecerían en cada capital del suprauniverso mil conciliadores del orden séptimo. De estos episodios, que reflejan la naturaleza séptupla de los espíritus mayores, surgen los siete órdenes existentes de conciliadores en servicio en cada suprauniverso.
\vs p025 2:3 Antes de lograr su estatus paradisíacos, los conciliadores del Paraíso no prestan servicio de forma intercambiable entre los suprauniversos, sino que su servicio se limita al sector de universo en el que se crearon. Cada colectivo del suprauniverso, que abarca una séptima parte de los grupos creados, pasa, por consiguiente, un periodo muy largo de tiempo bajo la dirección de uno de los espíritus mayores, con exclusión de los otros, porque, aunque los siete se \bibemph{reflejan} en las capitales de los suprauniversos, solamente uno es \bibemph{dominante} en cada una de estos.
\vs p025 2:4 Cada una de estas siete supracreaciones está en efecto infundida de ese espíritu mayor que preside sobre sus destinos. Así pues, cada suprauniverso se convierte en un gigantesco espejo que refleja la naturaleza y el carácter del espíritu mayor que lo dirige, y todo esto se extiende adicionalmente a cada uno de los universos locales mediante la presencia y la actuación de los espíritus maternos creativos. El efecto de tal circunstancia sobre el desarrollo evolutivo es tan profundo que, en sus andaduras, los conciliadores, de manera colectiva, tras haber pasado por el suprauniverso, manifiestan cuarenta y nueve puntos de vista de tipo experiencial, u órdenes de percepción, cada cual desde distintos ángulos ---y, por consiguiente, incompletos---, aunque todos mutuamente compensatorios y tendentes en su conjunto a abarcar el círculo de la Supremacía.
\vs p025 2:5 \pc En cada suprauniverso, los conciliadores universales se encuentran misteriosamente divididos, de forma innata, en grupos de cuatro, en los que continúan sirviendo. En cada grupo, tres de ellos son seres personales espirituales y uno, como las cuartas criaturas de los servitales, un ser semimaterial. Este cuarteto constituye una comisión conciliadora y se compone de la siguiente manera:
\vs p025 2:6 \li{1.}\bibemph{El juez\hyp{}arbitro}. Aquel designado por unanimidad por los otros tres miembros de la comisión como el más competente y mejor cualificado para actuar como jefe judicial del grupo.
\vs p025 2:7 \li{2.}\bibemph{El abogado\hyp{}espíritu}. Aquel nombrado por el juez\hyp{}árbitro para presentar pruebas y salvaguardar los derechos de todos los seres personales implicados en cualquier asunto que se haya asignado a la comisión conciliadora para su dictamen.
\vs p025 2:8 \li{3.}\bibemph{El ejecutante divino}. El conciliador cualificado por atributos consustanciales a su naturaleza para ponerse en contacto con los seres materiales de los mundos del espacio y ejecutar las decisiones de la comisión. Los ejecutantes divinos, al ser cuartas criaturas ---seres cuasi materiales---, son casi, pero no del todo, visibles para la visión de corto alcance de las razas mortales.
\vs p025 2:9 \li{4.}\bibemph{El registrador}. El miembro restante de la comisión se convierte automáticamente en registrador, en secretario judicial. Él se asegura de que todos los informes estén preparados de forma satisfactoria para los archivos del suprauniverso y para la documentación del universo local. Si la comisión sirve en un mundo evolutivo, con la asistencia del ejecutante, se redacta un tercer informe para los archivos físicos del gobierno del sistema a cuya jurisdicción pertenece dicho mundo.
\vs p025 2:10 \pc Cuando se reúne en sesión, la comisión actúa como un grupo de tres ya que el abogado se ausenta durante la decisión judicial y participa en la formulación del veredicto solamente al concluir la audiencia. De ahí que estas comisiones se denominen a veces tríos arbitrales.
\vs p025 2:11 \pc Los conciliadores resultan de gran valor para que el universo de los universos marche sin complicaciones. Atravesando el espacio a la velocidad seráfica triple, sirven como tribunales ambulantes de los mundos, como comisiones dedicadas a tomar resoluciones judiciales rápidas sobre asuntos menores. Si no fuese por estas comisiones móviles y sumamente ecuánimes, los tribunales de las esferas estarían desesperadamente sobrecargados por desacuerdos de poca importancia.
\vs p025 2:12 Estos tríos arbitrales no dictaminan sobre asuntos que tendrán importancia para la eternidad; el alma, las posibilidades eternas de la criatura del tiempo, nunca se pone en peligro como consecuencia de los actos de este trío. Dichos conciliadores no se encargan de cuestiones que se extiendan más allá de la existencia temporal y del bienestar cósmico de las criaturas del tiempo. Si bien, una vez que la comisión ha aceptado la jurisdicción sobre un problema, sus resoluciones son finales y siempre unánimes; no hay apelación para la decisión del juez\hyp{}árbitro.
\usection{3. EL EXTENSO SERVICIO DE LOS CONCILIADORES}
\vs p025 3:1 Los conciliadores, como grupo, tienen su sede en la capital de su suprauniverso, donde se mantiene su colectivo de reserva principal. Sus reservas secundarias están emplazadas en las capitales de los universos locales. Los comisionados más jóvenes y de menos experiencia inician su servicio en los mundos de menor rango, mundos tales como Urantia, y se les promueve para que juzguen problemas de mayor trascendencia tras haber adquirido una mayor madurez.
\vs p025 3:2 El orden de los conciliadores es totalmente meritorio de confianza; ninguno de ellos se ha descarriado jamás. Aunque no sean infalibles en cuanto a su sabiduría y juicio, su fiabilidad es incuestionable y su fidelidad inequívoca. Se originan en las sedes de los suprauniversos donde terminan por regresar a medida que avanzan a través de los siguientes niveles de servicio en el universo:
\vs p025 3:3 \li{1.}\bibemph{Los conciliadores de los mundos}. Siempre que los seres personales encargados de la supervisión de algún mundo específico están sumidos en una gran perplejidad o se encuentran realmente estancados ante la forma adecuada de proceder en determinadas circunstancias, y si el asunto no tiene la suficiente importancia como para presentarlo ante los tribunales regularmente constituidos de dicho mundo, entonces, al recibo de la petición de dos seres personales, uno de cada parte en discordia, una comisión conciliadora comienza a actuar de forma inmediata.
\vs p025 3:4 Una vez que estos problemas administrativos y jurisdiccionales han recaído en las manos de los conciliadores para su estudio y juicio, su autoridad sobre ellos es suprema. No obstante, no emitirán ningún dictamen hasta que no se hayan oído todos los testimonios, y no hay en absoluto límite a su potestad para citar testigos de cualquier lugar o lugares. Y aunque sus dictámenes sean inapelables, ocurre a veces que los asuntos se desarrollan de tal manera que la comisión cierra el sumario al llegar a un punto determinado, concluye su resolución y transfiere toda la cuestión a los tribunales superiores del planeta.
\vs p025 3:5 Las resoluciones de los comisionados se asientan en los archivos planetarios y, si es necesario, el ejecutante divino las lleva a cabo. Su competencia es muy grande y en un mundo habitado poseen un amplio ámbito de acción. Conducen de manera magistral aquello que va en interés de lo que debería ser. Su labor se lleva a veces a término para el bienestar aparente del mundo y, a veces, su actuación en los mundos del tiempo y del espacio son difíciles de explicar. Aunque no ejecutan decretos que desafíen las leyes naturales ni las costumbres establecidas del planeta, con frecuencia efectúan actividades misteriosas y hacen cumplir los mandatos de los conciliadores conforme a las leyes superiores de la administración de los sistemas.
\vs p025 3:6 \li{2.}\bibemph{Los conciliadores de las sedes de los sistemas}. Tras servir en los mundos evolutivos, a estas comisiones de cuatro conciliadores se les promueve para asumir su responsabilidad en la sede del sistema. Aquí tienen mucho trabajo que hacer, y demuestran su amistad y comprensión hacia los hombres, los ángeles y otros seres espirituales. Los tríos arbitrales no se ocupan tanto de las desavenencias entre personas como de las controversias entre grupos y de los desacuerdos que surgen entre los distintos órdenes de criaturas. En las sedes de los sistemas viven tanto seres espirituales como materiales, al igual que seres como los hijos materiales que son una combinación de estas dos naturalezas.
\vs p025 3:7 En el momento en que los creadores dan origen a seres evolutivos con facultad de elección, en ese mismo momento se produce una desviación del funcionamiento fluido de la perfección divina; es seguro que van a surgir desacuerdos y que se deben arbitrar medidas para resolver con ecuanimidad estas diferencias sinceras de opinión. Deberíamos todos recordar que los creadores omnisapientes y todopoderosos podrían haber creado los universos locales tan perfectos como Havona. No se necesitan las comisiones conciliadoras en el universo central. Pero los creadores no optaron en su gran sabiduría por hacerlo así. Y aunque han dado origen a universos donde abundan las diferencias y son numerosas las dificultades, también se han proporcionado los mecanismos y los medios para poner en orden estas diferencias y armonizar toda esta confusión aparente.
\vs p025 3:8 \li{3.}\bibemph{Los conciliadores de las constelaciones}. Tras servir en los sistemas, a los conciliadores se les promociona para que juzguen los problemas de las constelaciones, encargándose de resolver las dificultades menores que surgen entre sus cien sistemas de mundos habitados. No muchos de los problemas que se producen en las sedes de la constelación caen bajo su jurisdicción, pero se mantienen atareados yendo de sistema en sistema recabando pruebas y preparando las declaraciones preliminares. Si la controversia es de buena fe, si las dificultades surgen de diferencias sinceras de opinión y de una disparidad legítima de puntos de vista, sin importar el escaso número de personas involucradas ni la aparente trivialidad del desacuerdo, siempre le es posible a la comisión conciliadora pronunciarse sobre el fondo de la cuestión.
\vs p025 3:9 \li{4.}\bibemph{Los conciliadores de los universos locales.} En esta labor, la de mayor amplitud de un universo, los comisionados suponen una gran ayuda tanto para los melquisedecs y los hijos magistrados como para los gobernantes de la constelación y las multitudes de seres personales que se encargan de la coordinación y administración de las cien constelaciones. Los órdenes diferentes de serafines y otros residentes de las esferas sedes de un universo local se sirven igualmente de la ayuda y las resoluciones de estos tríos arbitrales.
\vs p025 3:10 Resulta casi imposible explicar la naturaleza de las diferencias que pueden surgir en los pormenorizados asuntos de un sistema, una constelación o un universo. De hecho se producen dificultades, pero son muy distintas de las nimias pruebas y tribulaciones de la existencia material tal como se viven en los mundos evolutivos.
\vs p025 3:11 \li{5.}\bibemph{Los conciliadores de los sectores menores de un suprauniverso}. A partir de su labor ante los problemas de los universos locales, los comisionados pasan al estudio de cuestiones que surgen en los sectores menores de su suprauniverso. Cuanto más ascienden hacia el interior desde los planetas individuales, menos son los deberes materiales que el ejecutante divino tiene que realizar; gradualmente, asume un nuevo papel de intérprete de la misericordia\hyp{}justicia. Al mismo tiempo, al ser cuasi material, mantiene a la totalidad de la comisión, desde la comprensión, en contacto con los aspectos materiales de sus investigaciones.
\vs p025 3:12 \li{6.}\bibemph{Los conciliadores de los sectores mayores de un suprauniverso}. El carácter de la labor de los comisionados continúa cambiando a medida que avanzan en su tarea. Cada vez encuentran menos desacuerdos que dictaminar y más y más fenómenos misteriosos que explicar e interpretar. Nivel a nivel evolucionan desde su posición de árbitros en las disensiones a \bibemph{explicadores de misterios} ---jueces que se transforman en maestros interpretativos---. En otro momento eran árbitros de quienes por ignorancia permitían que se produjeran dificultades y desacuerdos; pero ahora se están convirtiendo en instructores de quienes son suficientemente inteligentes y tolerantes como para evitar choques entre mentes y guerras de opiniones. Cuanto más elevada sea la educación de una criatura, más respeto tendrá por el conocimiento, la experiencia y las opiniones de otros.
\vs p025 3:13 \li{7.}\bibemph{Los conciliadores del suprauniverso}. Aquí los conciliadores se hacen coiguales ---cuatro árbitros\hyp{}maestros que mutuamente se comprenden y desempeñan su labor de modo perfecto---. Al ejecutante divino se le libera de su facultad correctiva y se convierte en la voz física del trío espiritual. Llegado este momento, estos consejeros y maestros se han convertido en expertos conocedores de la mayor parte de los problemas y dificultades reales que se presentan en la dirección de los asuntos del suprauniverso. Así pues, se tornan mentores formidables y maestros de sabiduría para los peregrinos ascendentes que residen en las esferas de instrucción que rodean a los mundos sedes de los suprauniversos.
\vs p025 3:14 \pc Todos los conciliadores sirven bajo la supervisión general de los ancianos de días y bajo la dirección inmediata de los auxiliares de imagen hasta el momento en que se les promueve para servir en el Paraíso. Durante su estancia en el Paraíso dan cuenta al espíritu mayor que preside el suprauniverso del que son originarios.
\vs p025 3:15 En los registros del suprauniverso no constan aquellos conciliadores que sobrepasan su jurisdicción. Son comisiones que están bastante dispersas por todo el gran universo. El último informe registral en Uversa arroja un número de comisiones en operación en Orvontón de casi dieciocho billones ---más de setenta billones de individuos---. Pero estas representan solo una fracción muy pequeña de la multitud de conciliadores que se han creado en Orvontón; tal número es de una magnitud mucho más elevada y equivale al número total de servitales de Havona, teniendo en cuenta las transmutaciones de estos a guías de los graduados.
\vs p025 3:16 Periódicamente, a medida que aumenta el número de conciliadores del suprauniverso, se les traslada al consejo de perfección del Paraíso, del cual posteriormente emergen como colectivo coordinador erigido por el Espíritu Infinito para el universo de los universos; son un grupo maravilloso de seres que aumenta constantemente en número y eficacia. Al haber experimentado un entrenamiento de tipo experiencial en su ascenso al Paraíso, han adquirido a la vez un singular entendimiento de la realidad emergente del Ser Supremo, y recorren el universo de los universos en misiones especiales.
\vs p025 3:17 Los miembros de las comisiones conciliadoras nunca se separan. Los cuatro miembros del grupo sirven para siempre juntos tal como se unieron originariamente. Incluso en su glorificado servicio continúan actuando como cuartetos que han acumulado experiencia cósmica y sabiduría experiencial perfeccionada. Están vinculados para la eternidad como manifestación de la justicia suprema del tiempo y del espacio.
\usection{4. LOS ASESORES TÉCNICOS}
\vs p025 4:1 Estas mentes jurídicas y técnicas del mundo espiritual no se crearon como tales. El Espíritu Infinito seleccionó de entre los primeros supernafines y omniafines a un millón de las mentes más metódicas como núcleo de este grupo inmenso y versátil. Y desde aquel remoto momento, a todos los que aspiran a convertirse en asesores técnicos se les ha exigido experiencia genuina en la aplicación de las leyes de la perfección a los planes de la creación evolutiva.
\vs p025 4:2 \pc A los asesores técnicos se les recluta de entre los siguientes órdenes de seres personales:
\vs p025 4:3 \li{1.}Los supernafines.
\vs p025 4:4 \li{2.}Los seconafines.
\vs p025 4:5 \li{3.}Los terciafines.
\vs p025 4:6 \li{4.}Los omniafines.
\vs p025 4:7 \li{5.}Los serafines.
\vs p025 4:8 \li{6.}Ciertos tipos de mortales ascendentes.
\vs p025 4:9 \li{7.}Ciertos tipos de seres intermedios ascendentes.
\vs p025 4:10 \pc En estos momentos, sin contar a los mortales y a los seres intermedios, asignados a este grupo de forma transitoria, el número de asesores técnicos registrados en Uversa y operativos en Orvontón es ligeramente superior a sesenta y un billones.
\vs p025 4:11 Con frecuencia, los asesores técnicos desempeñan su labor de forma individual, pero están organizados para el servicio y mantienen una sede común en las esferas a las que están asignados en grupos de siete. En cada grupo, cinco de ellos al menos deben ser miembros permanentes, mientras que dos pueden ser temporales. Los mortales y las criaturas intermedias ascendentes sirven en estas comisiones de asesoramiento mientras perseveran en su ascenso al Paraíso, si bien, no ingresan en los cursos regulares de formación para los asesores técnicos ni se convierten jamás en miembros permanentes de este orden de seres.
\vs p025 4:12 A esos mortales y seres intermedios que sirven con los asesores de forma transitoria se les elige para tal labor porque son expertos conocedores de la ley universal y de la justicia suprema. A medida que viajáis hacia vuestra meta en el Paraíso, adquiriendo un conocimiento cada vez mayor y una mejor capacidad para poder usarlo de forma efectiva, se os proporcionará continuamente la oportunidad de transmitir vuestra sabiduría y vuestra experiencia acumulada a otros seres; durante todo vuestro trayecto hacia Havona representáis el papel de pupilo\hyp{}maestro. Os abriréis camino a través de los niveles ascendentes de esta inmensa universidad de la experiencia, impartiendo a los que están justo por debajo de vosotros el nuevo conocimiento que habéis adquirido en vuestra andadura. En el régimen universal no se considera que realmente seáis poseedores de conocimiento y verdad hasta que no hayáis demostrado vuestra habilidad y disposición para impartir a otros tal conocimiento y verdad.
\vs p025 4:13 Tras una formación prolongada y la adquisición de una experiencia real, a cualquiera de los espíritus servidores por encima del estatus de querubín se le permite recibir el nombramiento de asesor técnico de forma permanente. Todos los candidatos se unen voluntariamente a este tipo de servicio; pero una vez que han asumido tales responsabilidades, no pueden renunciar a ellas. Únicamente los ancianos de días están facultados para trasladar a estos asesores a otras ocupaciones.
\vs p025 4:14 \pc La formación de los asesores técnicos, iniciada en las facultades de los melquisedecs en los universos locales, continúa hasta los tribunales de los ancianos de días. Después de tal instrucción en el suprauniverso, proceden a las “escuelas de los siete círculos” localizadas en los mundos piloto de las vías de Havona. A partir de aquí, se les recibe en la “Facultad de la Ética de la Ley y de la Técnica de la Supremacía”, el lugar de perfeccionamiento en el Paraíso de los asesores técnicos.
\vs p025 4:15 Estos asesores son más que expertos jurídicos; son estudiosos y maestros de la ley \bibemph{aplicada,} esto es, las leyes del universo aplicadas a las vidas y destinos de todos los que habitan los extensos dominios de la inmensa creación. A medida que pasa el tiempo, se convierten en bibliotecas vivas de las leyes del tiempo y del espacio. De esta manera, evitan dificultades interminables y retrasos en la instrucción de los seres personales del tiempo sobre las formas y modos de procedimiento más aceptables para los gobernantes de la eternidad. Los asesores son capaces de guiar a los laboradores del espacio y capacitarlos para obrar en armonía con los imperativos del Paraíso; son los maestros de todas las criaturas en relación al modo de proceder de los creadores.
\vs p025 4:16 Tal biblioteca viva de ley aplicada no podría ser creada; estos seres deben evolucionar mediante la experiencia real. Las Deidades infinitas son existenciales, de ahí que su falta de experiencia quede compensada. Los asesores lo saben todo incluso antes de experimentarlo, pero no transmiten este conocimiento no experiencial a sus criaturas de menor rango.
\vs p025 4:17 \pc Los asesores técnicos se dedican a la labor de evitar retrasos, facilitar el progreso y guiar en la consecución de algún logro. Siempre existe la forma \bibemph{mejor} y \bibemph{más correcta} de hacer las cosas; siempre hay un plan de perfección, un método divino, y estos asesores saben cómo conducirnos a todos para que podamos hallar esa mejor manera.
\vs p025 4:18 Estos seres, excepcionalmente sabios y prácticos, siempre están estrechamente vinculados con el servicio y la labor de los censores universales. Los melquisedecs tienen a su disposición a un colectivo capacitado. Los gobernantes de los sistemas, las constelaciones, los universos y los sectores del suprauniverso están copiosamente provistos de estas mentes técnicas o de consulta jurídica del mundo espiritual. Hay un grupo especial que actúa asesorando jurídicamente a los portadores de vida, aconsejándoles sobre el grado de desviación permisible respecto al orden establecido de propagación de la vida. También instruyen a estos hijos en relación a sus prerrogativas y a su libertad de acción. Guían a todas las clases de seres en cuanto a los usos y métodos adecuados sobre cualquier acontecimiento del mundo espiritual. Pero no tratan de forma directa y personal con las criaturas materiales.
\vs p025 4:19 Además de aconsejar en relación a temas jurídicos, los asesores técnicos se dedican igualmente a interpretar con eficacia todas las leyes que guardan relación con los seres creados, ya sean físicas, mentales o espirituales. Están a disposición de los conciliadores universales y de todos aquellos que deseen conocer la verdad de la ley, es decir, que deseen saber cómo se puede esperar que la Supremacía de la Deidad reaccione ante una situación dada que contenga factores establecidos del orden físico, mental y espiritual. Tratan incluso de dilucidar la forma de proceder del Último.
\vs p025 4:20 Los asesores técnicos son seres escogidos y probados; nunca he sabido de ninguno que se haya descarriado. No tenemos datos en Uversa de que se les haya juzgado por desacato a las leyes divinas que tan eficazmente interpretan y tan elocuentemente exponen. No se conoce límite en cuanto al ámbito de su servicio, ni se le ha puesto límite a su desarrollo. Esos seres continúan como asesores incluso hasta las puertas del Paraíso; el universo completo de la ley y la experiencia está abierto para ellos.
\usection{5. LOS CUSTODIOS DE ARCHIVOS DEL PARAÍSO}
\vs p025 5:1 De entre los supernafines terciarios de Havona, se seleccionan algunos de los archivistas jefes de mayor rango para que sirvan de custodios de archivos, de conservadores de los registros regulares de la Isla de la Luz, de esos archivos que se diferencian de los registros vivos de la mente de los custodios del conocimiento, a veces denominados “bibliotecas vivas del Paraíso”.
\vs p025 5:2 Los ángeles archivistas de los planetas habitados constituyen la fuente de todos los expedientes individuales. En todos los universos, hay otros archivistas que desempeñan su actividad tanto en los archivos regulares como en los vivos. Desde Urantia hasta el Paraíso, se encuentran ambos tipos de archivos: en los universos locales hay más registros escritos y menos vivos; en el Paraíso, hay más registros vivos y menos escritos; en Uversa, ambos están igualmente disponibles.
\vs p025 5:3 Se ha de tomar nota de todo acontecimiento significativo que ocurra en la creación organizada y habitada y, aunque los sucesos que no tienen más que una importancia limitada, tan solo se archivan a nivel local, aquellos de mayor significación han de tratarse de forma conveniente. Desde los planetas, sistemas y constelaciones de Nebadón, todo lo que tenga relevancia para el universo se conserva en Lugar de Salvación; y, a partir de estas capitales de los universos, tales acontecimientos se guardan en un nivel superior pertinente a los asuntos del sector y de los gobiernos de los suprauniversos. El Paraíso también posee un resumen relevante de datos del suprauniverso y de Havona; y dicho relato histórico y acumulativo del universo de los universos se encuentra bajo la custodia de estos excelsos supernafines terciarios.
\vs p025 5:4 Aunque se ha enviado a algunos de estos seres a los suprauniversos para servir como jefes de archivos y dirigir el trabajo de los archivistas celestiales, nunca se ha transferido a ninguno de ellos del listado nominal permanente de su orden.
\usection{6. LOS ARCHIVISTAS CELESTIALES}
\vs p025 6:1 Estos archivistas desempeñan la tarea de hacer un duplicado de todos los registros. Realizan un original de tipo espiritual y su correspondiente semimaterial ---lo que se podría llamar una copia en papel carbón---. Pueden hacer esto debido a que poseen una habilidad singular para actuar de forma simultánea sobre la energía espiritual y la energía material. Estos seres no se crean como tales; son serafines ascendentes de los universos locales. Los consejos de los jefes de archivos emplazados en las sedes de los siete suprauniversos los reciben, agrupan y asignan a sus esferas de trabajo. Allí también se localizan las escuelas de formación de los archivistas celestiales. Los perfeccionadores de la sabiduría y los consejeros divinos dirigen la escuela de Uversa.
\vs p025 6:2 A medida que los archivistas avanzan en su servicio en el universo, continúan su sistema de duplicado, posibilitando por tanto que sus registros siempre estén disponibles para todas las clases de seres, desde aquellos de orden material hasta los espíritus elevados de luz. En vuestra experiencia de transición, a medida que ascendéis de este mundo material, siempre podréis consultar los archivos relativos a la historia y a las tradiciones de la esfera correspondiente a vuestro estatus para familiarizaros con ellas.
\vs p025 6:3 Los archivistas conforman un colectivo probado y de confianza. No he oído jamás que ninguno de ellos haya desertado, y nunca se ha sabido que hayan falsificado sus registros. Están sujetos a una inspección doble. Sus excelsos compañeros de Uversa y los mensajeros poderosos examinan en detalle estos registros y certifican la exactitud de los duplicados cuasi físicos de los originales espirituales.
\vs p025 6:4 Mientras que los archivistas emplazados en las esferas secundarias registradas en los universos de Orvontón se cuentan por billones de billones, los que han conseguido, al avanzar, su estatus en Uversa, casi no alcanzan los ocho millones. Estos archivistas de mayor rango o graduados son los custodios del suprauniverso y quienes emiten los archivos autentificados del tiempo y del espacio. Tienen su sede permanente en las moradas circulares que rodean el área documental de Uversa. Jamás entregan la custodia de estos registros a nadie; pueden estar ausentes a nivel individual, pero nunca en grandes números.
\vs p025 6:5 Al igual que esos supernafines que se han convertido en custodios de archivos, el colectivo de archivistas celestiales presta su servicio de forma permanente. Una vez que los serafines y los supernafines se incorporan a estas tareas, permanecerán respectivamente como archivistas celestiales y custodios de archivos hasta el día en que la plena manifestación personal del Dios Supremo dé inicio a un gobierno renovado.
\vs p025 6:6 En Uversa, estos archivistas celestiales de mayor rango pueden mostrar los registros de todo lo que haya tenido importancia cósmica en todo Orvontón, desde los remotos tiempos de la llegada de los ancianos de días, mientras que, en la Isla eterna, los custodios de archivos tutelan los registros de este lugar, dando testimonio de hechos ocurridos en el Paraíso desde los tiempos en que el Espíritu Infinito adquirió su ser personal.
\usection{7. LOS ACOMPAÑANTES MORONTIALES}
\vs p025 7:1 Estos hijos de los espíritus maternos de los universos locales son los amigos y colaboradores de todos los que viven la vida morontial ascendente. No son indispensables para el trabajo progresivo concreto que como criatura el ascendente ha de realizar ni tampoco reemplazan, en sentido alguno, la labor de los guardianes seráficos que con frecuencia acompañan a sus allegados mortales en su viaje al Paraíso. Los acompañantes morontiales son sencillamente gentiles anfitriones para quienes acaban de comenzar su largo ascenso hacia el interior. Igualmente, y con destreza, estos seres auspician el entretenimiento, labor en la que los directores de reversión los asisten hábilmente.
\vs p025 7:2 Aunque tendréis que realizar tareas importantes y paulatinamente más difíciles en los mundos morontiales de formación de Nebadón, siempre contaréis con temporadas regulares para el descanso y la reversión. A lo largo de vuestro viaje al Paraíso, siempre habrá tiempo para el ocio y el esparcimiento del espíritu; en vuestra andadura de luz y vida, siempre hay tiempo para la adoración y nuevos logros.
\vs p025 7:3 Estos acompañantes morontiales son colaboradores tan amigables que, cuando finalmente acabéis la última fase de vuestra experiencia morontial, al prepararos para embarcaros en la aventura espiritual del suprauniverso, verdaderamente lamentaréis que estas sociables criaturas no os puedan acompañar, dado que sirven exclusivamente en los universos locales. En todas las etapas de la andadura ascendente, todos los seres personales con quienes contactéis serán amigables y sociables, y hasta que no os encontréis con los acompañantes del Paraíso no conoceréis a otro grupo tan dedicado a la amistad y al compañerismo.
\vs p025 7:4 La labor de los acompañantes morontiales se describe de forma más detallada en el relato de los asuntos de vuestro universo local.
\usection{8. LOS ACOMPAÑANTES DEL PARAÍSO}
\vs p025 8:1 Los acompañantes del Paraíso son un grupo de origen heterogéneo o compuesto que se recluta de entre los serafines, seconafines, supernafines y omniafines. Aunque prestan su servicio durante un período que vosotros podríais considerar extraordinariamente largo, no lo hacen de forma indefinida. Por regla general, aunque no de forma invariable, al concluir este ministerio, reanudan el cometido que tenían cuando se les llamó para servir en el Paraíso.
\vs p025 8:2 Los espíritus maternos de los universos locales, los espíritus reflectores de los suprauniversos y Majestón del Paraíso nombran a miembros de las multitudes angélicas para este servicio. Uno de los siete espíritus mayores los convoca a la Isla central y los destina como acompañantes del Paraíso de forma transitoria. Y, además de la condición permanente en el Paraíso, este servicio temporal es el honor más grande que jamás se pueda conferir a espíritus servidores.
\vs p025 8:3 Estos ángeles seleccionados se dedican a servir de compañía y se les asigna como colaboradores de toda clase de seres que pueda suceder que se encuentren solos en el Paraíso, mayormente de mortales ascendentes, pero también de todos los demás seres que estén solos en la Isla central. Los acompañantes del Paraíso no tienen que hacer nada en especial en beneficio de aquellos con quienes fraternizan; son simplemente acompañantes. Casi todos los demás seres que los mortales encontraréis durante vuestra estancia en el Paraíso ---aparte de vuestros semejantes peregrinos--- tendrán algo determinado que hacer con vosotros o por vosotros; pero a estos acompañantes se les encarga solamente estar con vosotros y estar en comunicación con vosotros como compañeros de vuestras personas. Con frecuencia, los gentiles y brillantes ciudadanos del Paraíso los asisten en este ministerio.
\vs p025 8:4 Los mortales proceden de razas que son muy sociables. Los creadores saben bien que “no es bueno para el hombre que esté solo” y, por consiguiente, se toman medidas para que tengan compañía incluso en el Paraíso.
\vs p025 8:5 \pc Si vosotros, como mortales ascendentes, alcanzarais el Paraíso junto a alguien que os haya acompañado o haya estado estrechamente vinculado con vosotros durante vuestra andadura terrenal, o si sucede que vuestro guardián seráfico de destino llegara con vosotros u os estuviera esperando, entonces no se os asignaría ningún acompañante del Paraíso de forma permanente. Pero si llegáis solos, es seguro que uno de ellos os dará la bienvenida al despertaros del sueño último del tiempo en la Isla de la Luz. Pero incluso si se sabe que os acompañará alguien vinculado a vosotros en vuestro ascenso, se designarán acompañantes temporales para daros la bienvenida en estas orillas eternas y conduciros al lugar reservado ya listo para recibiros a vosotros y a vuestros allegados. Podéis estar seguros de que tendréis una calurosa acogida cuando, junto a las orillas eternas del Paraíso, resucitéis en la eternidad.
\vs p025 8:6 Los acompañantes del Paraíso encargados de esta recepción se nombran durante los días finales de la estancia de los ascendentes en la última vía de Havona. Ellos mismos examinan cuidadosamente el historial relativo a la etapa mortal y al memorable ascenso de estos a través de los mundos del espacio y de los círculos de Havona. Así pues, cuando saludan a los mortales del tiempo a su llegada, ya conocen bien la andadura de estos peregrinos y, de forma inmediata, se muestran como acompañantes compasivos y fascinantes.
\vs p025 8:7 Durante vuestra estancia en el Paraíso previa a la condición de finalizadores, si por cualquier razón tuvierais que separaros temporalmente de quien estuvo vinculado a vosotros en vuestra andadura como ascendente ---ya sea de origen mortal o seráfico---, se os asignará sin dilación un acompañante del Paraíso para aconsejaros y acompañaros. Una vez que se ha asignado a uno de estos acompañantes al mortal ascendente que reside solo en el Paraíso, dicho acompañante permanece con esa persona hasta que esta se reúna con sus allegados ascendentes o se incorpore como corresponde en el colectivo final.
\vs p025 8:8 \pc A los acompañantes del Paraíso se les nombra respetando un orden de espera; no obstante, nunca se coloca a un ascendente a cargo de un acompañante cuya naturaleza sea distinta por el suprauniverso del que procede a la suya. Si un mortal de Urantia llegara hoy al Paraíso, se le asignaría el primer acompañante en espera que sea originario de Orvontón o, de otro modo, que sea de la naturaleza del séptimo espíritu mayor. De ahí que los omniafines no desempeñen su labor con las criaturas ascendentes de los siete suprauniversos.
\vs p025 8:9 \pc Los acompañantes del Paraíso realizan otros muchos servicios: si un mortal ascendente llegara solo al universo central y, al atravesar Havona no superara alguna fase en la aventura de alcanzar la Deidad, a su debido tiempo tendría que volver a los universos del tiempo y, sin dilación, se haría un llamamiento a las reservas de acompañantes del Paraíso. Se designaría a uno de sus miembros para seguir al frustrado peregrino, estar con él y consolarlo y animarlo, y permanecer con él hasta su vuelta al universo central para reanudar su ascenso al Paraíso.
\vs p025 8:10 Si un peregrino fracasa en la aventura de alcanzar la Deidad mientras atraviesa Havona en compañía de un serafín ascendente, el ángel guardián de su andadura mortal podría optar por acompañar a su allegado mortal. Estos serafines siempre se ofrecen como voluntarios y se les permite acompañar a sus compañeros mortales de tantos años de vuelta al servicio del tiempo y del espacio.
\vs p025 8:11 No sucede lo mismo en el caso de dos ascendentes mortales que hayan estado estrechamente vinculados: si uno de ellos logra alcanzar a Dios mientras que el otro fracasa temporalmente, el que lo ha conseguido invariablemente opta por volver a las creaciones evolutivas con la decepcionada persona, pero esto no le está permitido. En cambio, se hace un llamamiento a las reservas de acompañantes del Paraíso y se elige a un voluntario para acompañar al desilusionado peregrino. Un ciudadano del Paraíso voluntario se vincula entonces al mortal que ha tenido éxito, el cual permanece en la Isla central aguardando el regreso a Havona de su compañero fracasado; mientras tanto, presenta, en alguna de las escuelas del Paraíso, el trepidante relato de su ascensión evolutiva.
\vsetoff
\vs p025 8:12 [Realizado con el auspicio de un elevado en autoridad de Uversa.]
