\upaper{141}{Inicio de la labor pública}
\author{Comisión de seres intermedios}
\vs p141 0:1 El 19 de enero del año 27 d. C., primer día de la semana, Jesús y los doce apóstoles se dispusieron a partir de su sede en Betsaida. Los doce desconocían los planes de su Maestro, salvo que subirían a Jerusalén para asistir a la fiesta de la Pascua en abril y que la intención era viajar cruzando el valle del Jordán. No salieron de la casa de Zebedeo hasta cerca del mediodía, porque las familias de los apóstoles y de otros de los discípulos habían venido para despedirlos y desearles lo mejor en la nueva labor que estaban a punto de comenzar.
\vs p141 0:2 Poco antes de partir, los apóstoles echaron de menos al Maestro, y Andrés fue a buscarlo. Tardó poco en hallarlo en la playa, sentado en una barca, llorando. A menudo, los doce habían visto a su Maestro en momentos de aparente aflicción y habían percibido en él breves periodos de seria preocupación mental, pero jamás ninguno de ellos lo había visto llorar. Andrés se sobresaltó, de alguna manera, al ver al Maestro tan afectado en vísperas de su partida para Jerusalén y se aventuró a acercarse a él para preguntarle: “Maestro, en este gran día, cuando estamos listos para salir hacia Jerusalén y proclamar el reino del Padre, ¿por qué lloras? ¿Alguno de nosotros te ha ofendido?”. Y, Jesús, yendo de regreso con Andrés para unirse a los doce, le respondió: “Ninguno de vosotros me ha causado tristeza alguna. Solo me apena que nadie de la familia de mi padre José se haya acordado de venir para manifestarnos sus buenos deseos en esta empresa que vamos a acometer”. En aquel momento, Ruth se encontraba en Nazaret visitando a su hermano José. Los otros miembros de su familia se habían apartado de él por orgullo, decepción, incomprensión y pequeños rencores, fruto de unos sentimientos heridos.
\usection{1. SALIDA DESDE GALILEA}
\vs p141 1:1 Cafarnaúm no estaba lejos de Tiberias, y la fama de Jesús había empezado a extenderse por toda Galilea e incluso por lugares más apartados. Jesús sabía que Herodes pronto prestaría atención a su labor; así que pensó que sería mejor viajar al sur y adentrarse en Judea con sus apóstoles. Un grupo de más de cien creyentes pretendían ir con ellos, pero Jesús les habló y les pidió que no acompañaran al grupo apostólico en su ruta a lo largo del valle del Jordán. Aunque accedieron a quedarse atrás, muchos de ellos seguirían al Maestro unos días más tarde.
\vs p141 1:2 El primer día, Jesús y los apóstoles llegaron únicamente hasta Tariquea, donde pasaron la noche. Al día siguiente, viajaron hasta un lugar específico en el Jordán, cerca de Pella, donde Juan había predicado hacía aproximadamente un año, y donde Jesús había recibido el bautismo. Allí se quedaron más de dos semanas, enseñando y predicando. Al final de la primera semana, se habían congregado varios centenares de personas en un campamento, cerca de donde Jesús y los doce paraban. Habían venido de Galilea, Fenicia, Siria, la Decápolis, Perea y Judea.
\vs p141 1:3 Jesús no hizo ninguna predicación pública. Andrés dividió a la multitud en grupos y asignó predicadores que se encargaran de las sesiones de la mañana y de la tarde; tras la cena, Jesús hablaba con los doce. No les enseñaba nada nuevo, sino que repasaba sus anteriores enseñanzas y contestaba a sus numerosas preguntas. Una de esas noches, les hizo algún comentario a los doce sobre los cuarenta días que había pasado en las colinas, cerca de allí.
\vs p141 1:4 Juan había bautizado a muchos de los que venían de Perea y Judea y estaban interesados en saber más de las enseñanzas de Jesús. Los apóstoles hicieron bastantes progresos en su instrucción de los discípulos de Juan, ya que, de ninguna manera, restaban valor a su predicación, y puesto que ni siquiera bautizaban a sus nuevos discípulos en aquel momento. Pero siempre resultaba un obstáculo para los seguidores de Juan el hecho de que si Jesús era todo lo que Juan había anunciado, no hubiese hecho nada por sacarlo de la cárcel. Los discípulos de Juan nunca pudieron entender por qué Jesús no había evitado la cruel muerte de su amado líder.
\vs p141 1:5 Noche tras noche, Andrés daba pormenorizadas indicaciones a sus compañeros apóstoles en cuanto a la tarea, delicada y difícil, de llevarse bien y sin contratiempos con los seguidores de Juan el Bautista. Durante este primer año del ministerio público de Jesús, más de las tres cuartas partes de sus seguidores habían sido anteriormente discípulos de Juan y habían recibido su bautismo. Todo el año 27 d. C. lo dedicaron a hacerse cargo, sosegadamente, de la labor de Juan en Perea y en Judea.
\usection{2. LA LEY DE DIOS Y LA VOLUNTAD DEL PADRE}
\vs p141 2:1 La noche antes de dejar Pella, Jesús dio más detalles a los apóstoles sobre el nuevo reino. El Maestro dijo: “Se os ha enseñado a aguardar la llegada del reino de Dios, y ahora yo vengo a anunciar que este reino, por tanto tiempo esperado, se ha acercado, que incluso ya está aquí entre nosotros. En todo reino debe haber un rey sentado en su trono que decrete las leyes del reino. Y, por ello, albergáis un concepto del reino de los cielos, como si fuese el gobierno glorificado del pueblo judío sobre todos los pueblos de la tierra, con el Mesías sentado en el trono de David promulgando, desde ese lugar de milagroso poder, las leyes de todo el mundo. Pero, hijos míos, no veis con los ojos de la fe ni oís con los oídos el entendimiento del espíritu. Yo declaro que el reino de los cielos es la toma de conciencia y el reconocimiento del gobierno de Dios en el corazón de los hombres. Es verdad que hay un Rey en este reino, y ese Rey es mi Padre y vuestro Padre. Ciertamente somos sus leales súbditos, pero, transcendiendo en mucho este hecho, está la verdad transformadora de que somos sus \bibemph{hijo}s. En mi vida esta verdad se hará manifiesta para todos. Nuestro Padre también se sienta en un trono, pero no en uno hecho a mano. El trono del Infinito es la morada eterna del Padre en el cielo de los cielos; él lo llena todo y proclama sus leyes a universos tras universos. Y, asimismo, el Padre gobierna en los corazones de sus hijos de la tierra por medio del espíritu enviado por él para que viva en las almas de los hombres mortales.
\vs p141 2:2 “Cuando sois súbditos de este reino, habréis de oír de hecho la ley del Gobernante del Universo; pero cuando, debido al evangelio del reino que yo he venido a anunciar, descubrís vosotros mismos por la fe que sois hijos de Dios, no os veréis desde ese momento como criaturas sujetas a la ley de un rey todopoderoso, sino como hijos privilegiados de un Padre amoroso y divino. De cierto, de cierto os digo, que cuando la voluntad del Padre es vuestra \bibemph{ley,} difícilmente estáis en el reino. Pero cuando la voluntad del Padre se vuelve realmente vuestra \bibemph{voluntad,} estaréis entonces de verdad en el reino, porque el reino se ha convertido, pues, en una experiencia arraigada en vosotros. Cuando la voluntad de Dios es vuestra ley, sois nobles súbditos esclavos; pero cuando creéis en este nuevo evangelio de la filiación divina, la voluntad de mi Padre se hace vuestra voluntad, y se os eleva a la alta posición de hijos libres de Dios, de hijos liberados del reino”.
\vs p141 2:3 Algunos de los apóstoles lograron entender algún aspecto de esta enseñanza, pero ninguno de ellos comprendió por completo el significado de tal extraordinaria aseveración, a no ser que fuera Santiago Zebedeo. Sin bien, estas palabras penetraron en sus corazones y se manifestarían para alegrar su ministerio durante su servicio en años posteriores.
\usection{3. ESTANCIA EN AMATUS}
\vs p141 3:1 El Maestro y sus apóstoles se quedaron casi tres semanas cerca de Amatus. Los apóstoles continuaron predicando dos veces al día a la multitud, y Jesús lo hacía cada \bibemph{sabbat} por la tarde. Fue imposible seguir con el esparcimiento de los miércoles; así pues, Andrés decidió que dos de los apóstoles descansarían cada día durante los seis días de la semana, mientras que todos estarían de servicio durante los oficios del \bibemph{sabbat}.
\vs p141 3:2 Pedro, Santiago y Juan realizaban la mayor parte de la predicación pública. Felipe, Natanael, Tomás y Simón llevaban a cabo gran parte de la labor personal y daban clases a determinados grupos de personas que habían mostrado su interés; los gemelos continuaron con su cargo de mantener el orden general de las muchedumbres, mientras que Andrés, Mateo y Judas formaron, entre los tres, una comisión gestora de asuntos generales, aunque cada cual efectuaba también una considerable labor religiosa.
\vs p141 3:3 Andrés estaba bastante ocupado con la tarea de resolver los constantes malentendidos y desacuerdos que surgían entre los discípulos de Juan y los discípulos más recientes de Jesús. Cada pocos días, se producían situaciones serias, pero Andrés, con la asistencia de sus compañeros apostólicos, se las arreglaba para convencer a las partes en conflicto a que alcanzaran algún tipo de acuerdo, al menos de forma temporal. Jesús se negaba a participar en estas charlas; tampoco daba consejo alguno sobre la mejor manera de solucionar tales dificultades. Nunca quiso ofrecer ni una sola recomendación sobre cómo zanjar estos desconcertantes problemas. Cuando Andrés acudía a Jesús con estas cuestiones, él siempre decía: “No es aconsejable que el anfitrión participe en las querellas familiares de sus invitados; un padre prudente nunca toma posiciones en las insignificantes disputas de sus hijos”.
\vs p141 3:4 \pc El Maestro daba muestras de una gran sabiduría y de una perfecta ecuanimidad en su trato con sus apóstoles y con todos sus discípulos. Jesús era en verdad un preceptor de hombres; ejercía una notable influencia sobre sus semejantes humanos debido a que, como persona, combinaba encanto y fuerza. Se evidenciaba una sutil autoridad en su vida ruda, nómada y sin hogar. Había una fascinación intelectual y una fuerza de atracción espiritual en su modo fidedigno de enseñar, en su lógica lúcida, en la fuerza de su razonamiento, en su sagaz percepción, en su agudeza de mente, en su incomparable porte y en su sublime tolerancia. Era sencillo, masculino, honesto y valeroso. Junto a toda esta influencia física e intelectual, que se ponía de manifiesto en la presencia del Maestro, había también esos encantos espirituales de la naturaleza de Jesús que se ha relacionado como parte de su persona: paciencia, ternura, mansedumbre, benignidad y humildad.
\vs p141 3:5 Jesús de Nazaret era, de hecho, una persona fuerte y vigorosa; era potestad intelectual y bastión espiritual. Su persona no solo apelaba a las mujeres de entre sus seguidores, a las mujeres con inclinaciones espirituales, sino también a Nicodemo, instruido e intelectual, y al recio soldado romano, el capitán que hacía guardia en la cruz, y que, una vez que vio morir al Maestro, dijo: “Verdaderamente este era Hijo de Dios”. Y los temperamentales y rudos pescadores galileos lo llamaban Maestro.
\vs p141 3:6 Las imágenes de Jesús han sido muy de lamentar. Estas pinturas de Cristo han ejercido una perniciosa influencia sobre la juventud; los mercaderes del templo habrían difícilmente huido ante él si hubiese sido el hombre que vuestros artistas han representado habitualmente. Tenía una dignificante masculinidad; era bueno, pero natural. Jesús no aparentaba ser un místico apacible, tierno, amable y cordial. Su forma de enseñar era apasionadamente dinámica. No solamente \bibemph{intentaba hacer el bien,} sino que realmente anduvo \bibemph{haciendo bienes}.
\vs p141 3:7 El Maestro nunca dijo: “Venid a mí todos los que sois indolentes y soñadores”. Sino que de cierto dijo muchas veces: “Venid a mí todos los que \bibemph{estáis trabajados,} y yo os proporcionaré descanso ---fuerza espiritual---”. El yugo del Maestro es en verdad fácil pero, incluso así, nunca lo impone; cada cual debe tomar ese yugo por su propia voluntad.
\vs p141 3:8 Jesús describió la conquista mediante el sacrificio, el sacrificio del orgullo y del egoísmo. Al manifestar misericordia, él quiso mostrar la liberación espiritual de todos los rencores, los agravios, la ira y el egoísmo de las ansias de poder y venganza. Y cuando dijo: “No confrontéis el mal”, explicó después que no quería decir que se tolerara el pecado ni que se confraternizara con la iniquidad. Él trataba más de enseñar a perdonar, “a no confrontar el maltrato hacia la propia persona, la maliciosa injuria a los sentimientos de la dignidad personal”.
\usection{4. ENSEÑANZAS SOBRE EL PADRE}
\vs p141 4:1 Mientras estaban en Amatus, Jesús pasó bastante tiempo con los apóstoles instruyéndolos en el nuevo concepto de Dios; repetidas veces les recalcó que \bibemph{Dios es un Padre,} y no un supremo tenedor de libros que se dedicara exclusivamente a anotar las acciones negativas de sus errados hijos de la tierra, a hacer un registro de pecados y maldad es para usarlos contra ellos, cuando más tarde el justo Juez de toda la creación los juzgara. Desde mucho tiempo atrás, los judíos habían concebido a Dios como el rey de todos, incluso como el Padre de la nación, pero nunca antes un número tan grande de mortales había albergado la idea de Dios como el Padre amoroso de \bibemph{cada una de las personas}.
\vs p141 4:2 Como respuesta a la pregunta de Tomás de “¿Quién es este Dios del reino?”. Jesús contestó: “Dios es \bibemph{tu} Padre, y la religión ---mi evangelio--- no es sino el leal reconocimiento de la verdad de que tú eres su hijo. Y estoy aquí entre vosotros en la carne para hacer patente en mi vida y en mis enseñanzas estas dos ideas”.
\vs p141 4:3 Igualmente, Jesús procuró liberar las mentes de sus apóstoles de la idea de ofrecer sacrificios de animales como obligación religiosa. Pero estos hombres, educados en la religión del sacrificio diario, tardaron en comprender lo que él quería decir. No obstante, el Maestro no se cansaba de impartirles sus enseñanzas. Cuando no lograba llegar a las mentes de todos los apóstoles mediante alguna ilustración, Jesús, para iluminarlos, replanteaba su mensaje con otro tipo de parábola.
\vs p141 4:4 \pc En ese mismo momento, Jesús empezó a enseñar a los doce más a fondo sobre su misión de “consolar a los afligidos y prestar auxilio a los enfermos”. El Maestro les habló mucho del hombre en su completitud ---la unión del cuerpo, la mente y el espíritu que conformaba al hombre o a la mujer individual---. Jesús les informó sobre los tres tipos de aflicción con los que se encontrarían y, luego, les instruyó sobre cómo deberían prestar sus cuidados a todos los que padecen el dolor de la enfermedad humana. Enseñó a los discípulos a reconocer:
\vs p141 4:5 \li{1.}Las dolencias de la carne ---esas aflicciones comúnmente consideradas como enfermedades físicas---.
\vs p141 4:6 \li{2.}Las mentes angustiadas ---esas aflicciones no físicas, con posterioridad consideradas como problemas y trastornos emocionales y mentales---.
\vs p141 4:7 \li{3.}La posesión de los espíritus malignos.
\vs p141 4:8 \pc En numerosas ocasiones, Jesús explicó a sus apóstoles la naturaleza, y algo del origen de estos espíritus malignos, en esos días también habitualmente llamados espíritus impuros. El Maestro conocía bien la diferencia entre la posesión de estos espíritus y la demencia, pero los apóstoles no. Si bien, debido al limitado conocimiento que poseían sobre la temprana historia de Urantia, a Jesús no le fue posible hacer que pudiesen comprender por completo esta cuestión. Pero, refiriéndose a estos espíritus malignos, les dijo repetidas veces: “No molestarán a los hombres cuando yo haya ascendido a la diestra de mi Padre en los cielos y haya después derramado mi espíritu sobre toda carne, en esos momentos en los que el reino vendrá con gran poder y gloria espiritual”.
\vs p141 4:9 Semana tras semana y mes tras mes, durante todo el año completo, los apóstoles prestarían cada vez más y más atención al ejercicio de su ministerio de curar al enfermo.
\usection{5. LA UNIDAD ESPIRITUAL}
\vs p141 5:1 Una de las charlas nocturnas más cruciales de las ocurridas en Amatus fue la relacionada con el estudio de la unidad espiritual. Santiago Zebedeo había preguntado: “Maestro, ¿cómo podremos aprender a ver las cosas de la misma forma y disfrutar por tanto de una mayor armonía entre nosotros?”. Cuando Jesús oyó esta pregunta, su espíritu se sintió tan agitado que respondió: “Santiago, Santiago, ¿cuándo os enseñé que debíais ver las cosas de igual manera? He venido al mundo para proclamar la libertad espiritual, para que los mortales tengan la facultad de vivir sus vidas individuales con originalidad y libertad ante Dios. No deseo que la armonía social y la paz fraternal se adquieran sacrificando la libertad de la persona y la creatividad espiritual. Lo que requiero de vosotros, apóstoles míos, es la \bibemph{unidad espiritual} ---algo que podréis experimentar en el gozo mismo de dedicaros de manera unida e incondicional a hacer la voluntad de mi Padre celestial---. No necesitáis ver las cosas ni pensar ni sentir de forma similar para \bibemph{ser espiritualmente iguales}. La unidad espiritual proviene de la conciencia de que cada uno de vosotros está habitado, y crecientemente regido, por el don espiritual del Padre celestial. Vuestra armonía apostólica ha de proceder del hecho de que la esperanza espiritual de cada uno de vosotros es idéntica en origen, naturaleza y destino.
\vs p141 5:2 “De este modo, podéis experimentar, en su grado de perfección, una unidad de propósito y entendimiento espiritual, que surge de la conciencia mutua de la identidad de cada uno de los espíritus del Paraíso que os habitan; y podréis gozar de esta profunda unidad espiritual en presencia misma de cualquier posible diversidad de vuestras actitudes individuales en cuanto a pensamiento intelectual, sentimiento temperamental y conducta social. Vuestras personas pueden ser estimulantemente diversas y notablemente diferentes, mientras que vuestra naturaleza espiritual y los frutos espirituales de la adoración divina y del amor fraternal pueden estar tan unificados, que todo aquel que contemple vuestras vidas tomará conocimiento, sin duda, de esta identidad de espíritu y unidad de alma; reconocerá que vosotros habéis estado conmigo y que habéis aprendido pues, y de forma adecuada, a hacer la voluntad del Padre de los cielos. Podéis lograr esta unidad en vuestro servicio a Dios, incluso si realizáis tal servicio según la forma de proceder de vuestros propios dones originales de mente, cuerpo y alma.
\vs p141 5:3 “Vuestra unidad de espíritu supone dos cosas, que siempre encontrarán su armonía en las vidas de cada uno de los creyentes: primero, poseéis un motivo común para vivir una vida de servicio; todos vosotros deseáis por encima de todo hacer la voluntad del Padre de los cielos. Segundo, todos tenéis una meta común que guía vuestra existencia; todos tenéis la intención de encontrar al Padre celestial, demostrando así al universo que os habéis hecho semejante a él”.
\vs p141 5:4 En numerosas ocasiones durante la formación de los doce, Jesús volvió a tratar este tema. Repetidas veces les dijo que no era su deseo que los que creyeran en él se volvieran dogmáticos y se unificasen ni incluso siguiendo la interpretación religiosa de hombres buenos. Una y otra vez advirtió a sus apóstoles que no formularan credos ni estableciesen tradiciones, como medio para guiar y dirigir a quienes creyeran en el evangelio del reino.
\usection{6. LA ÚLTIMA SEMANA EN AMATUS}
\vs p141 6:1 Casi al final de la última semana en Amatus, Simón Zelotes llevó a Jesús a un cierto Teherma, un persa que se encontraba haciendo negocios en Damasco. Teherma había oído hablar de Jesús y había venido a Cafarnaúm para verlo, y cuando supo allí que Jesús se había ido con sus apóstoles camino de Jerusalén por el valle del Jordán, se dispuso a buscarlo. Andrés se lo había presentado a Simón para que lo instruyera. Simón consideraba al persa un “adorador del fuego”, aunque Teherma no escatimó esfuerzos para explicarles que el fuego era solamente un símbolo visible del Uno Puro y Santo. Tras hablar con Jesús, el persa manifestó su intención de quedarse con ellos varios días para oír las enseñanzas y atender la predicación.
\vs p141 6:2 Cuando Simón Zelotes y Jesús estuvieron solos, Simón preguntó al Maestro: “¿Cómo es que yo no pude persuadirlo? ¿Por qué me opuso a mi resistencia y te prestó atención a ti con tanta prontitud? Jesús respondió: “Simón, Simón, ¿cuántas veces te he instruido a que te refrenes de cualquier intento por quitar algo \bibemph{del} corazón de los que buscan la salvación? ¿Cuántas veces te he dicho que te esfuerces por añadir algo \bibemph{en} estas almas hambrientas? Guía a los hombres hasta el reino, y sus verdades, grandes y vivas, disiparán pronto cualquier grave error. Cuando hayas presentado al hombre mortal la buena nueva de que Dios es su Padre, podrás, con mayor facilidad, persuadirlo de que él es realmente un hijo de Dios. Y, habiendo hecho eso, habrás llevado la luz de la salvación al que yace en la oscuridad. Simón, cuando el Hijo del Hombre vino primero a ti, ¿lo hizo denunciando a Moisés y a los profetas y proclamando un modo de vivir nuevo y mejor? No. Yo no vine para despojarte de lo que habías recibido de tus antecesores, sino para mostrarte la visión en perfección de lo que ellos solo percibieron en parte. Simón, ve, pues, a enseñar y a predicar el reino, y cuando tengas la total seguridad de que el hombre que has guiado hasta el reino permanezca en él, será entonces el momento preciso para que, si acude a ti con preguntas, le enseñes acerca del progresivo avance del alma en el reino divino”.
\vs p141 6:3 Simón quedó impresionado con estas palabras, pero hizo lo que Jesús le había indicado, y Teherma, el persa, se contó entre aquellos que entraron al reino.
\vs p141 6:4 \pc Aquella noche Jesús disertó para los apóstoles sobre la nueva vida en el reino. Dijo en parte: “Cuando entráis en el reino, renacéis. No podéis enseñar las cosas profundas del espíritu a aquellos que han nacido solamente de la carne; en primer lugar, aseguraos de que los hombres han nacido del espíritu antes de intentar instruirlos en los caminos superiores del espíritu. No pretendáis mostrar a los hombres las bellezas del templo antes de haberlos llevado primeramente a su interior. Presentad los hombres a Dios y, \bibemph{como} hijos de Dios, antes de platicarles sobre las doctrinas de la paternidad de Dios y de la filiación de los hombres. No contendáis con ellos ---sed pacientes siempre---. El reino no es vuestro; tan solo sois sus embajadores. Sencillamente, salid a proclamar: este es el reino de los cielos ---Dios es vuestro Padre y vosotros sois sus hijos---, y, si creéis incondicionalmente, esta buena nueva \bibemph{será} vuestra salvación eterna”.
\vs p141 6:5 Los apóstoles realizaron grandes avances durante su estancia en Amatus. Pero les decepcionó mucho que Jesús no les diera recomendaciones sobre cómo tratar con los discípulos de Juan. Incluso en el importante asunto del bautismo, lo único que Jesús dijo fue: “Juan ciertamente bautizó con agua, pero, cuando entréis en el reino de los cielos, vosotros seréis bautizados con el Espíritu”.
\usection{7. EN BETANIA, AL OTRO LADO DEL JORDÁN}
\vs p141 7:1 El 26 de febrero, Jesús, sus apóstoles y un gran grupo de seguidores viajaron a lo largo del Jordán hasta el vado cerca de Betania, en Perea, lugar en el que Juan hizo su primera proclamación sobre el reino venidero. Jesús se quedó allí con sus apóstoles durante cuatro semanas, enseñando y predicando antes de seguir su camino de ascenso a Jerusalén.
\vs p141 7:2 La segunda semana de su estancia en Betania, al otro lado del Jordán, Jesús llevó a Pedro, a Santiago y a Juan a las colinas que estaban cruzando el río y al sur de Jericó para descansar durante tres días. A estos tres discípulos, el Maestro les enseñó muchas verdades nuevas y avanzadas sobre el reino de los cielos. Al efecto de esta narrativa, hemos reorganizado y clasificado estas enseñanzas de la manera siguiente:
\vs p141 7:3 \pc Jesús quiso dejar bien claro que deseaba que sus discípulos, habiendo saboreado las realidades del espíritu del reino, vivieran en el mundo de tal modo que los hombres, al \bibemph{ver} sus vidas, tomaran conciencia de este y se sintieran, pues, guiados a preguntar a los creyentes sobre los caminos del reino. Todos los sinceros buscadores de la verdad siempre se alegran de \bibemph{oír} la buena nueva del don de la fe, que les garantiza la admisión al reino con sus realidades espirituales eternas y divinas.
\vs p141 7:4 El Maestro trató de inculcar a todos los maestros del evangelio del reino que su única labor era revelar al hombre individual que Dios era su Padre ---en guiarle para que tomara conciencia de su filiación y presentar, entonces, ese mismo hombre a Dios como su hijo en la fe---. Estas dos revelaciones esenciales se cumplen en Jesús. Él se convirtió, de hecho, en “el camino, la verdad y la vida”. La religión de Jesús se fundaba enteramente en vivir su vida de gracia en la tierra. Cuando Jesús partió de este mundo, no dejó atrás libros ni leyes ni forma alguna de organización humana, que repercutieran en la vida religiosa individual del ser humano.
\vs p141 7:5 Jesús manifestó claramente que había venido para establecer relaciones personales y eternas con los hombres, que debían siempre primar sobre cualquier otra relación humana. Y enfatizó que esta íntima hermandad espiritual había de extenderse a todos los hombres de todas las eras y condiciones sociales de entre todos los pueblos. La única recompensa que ofrecía a sus hijos era: en este mundo, gozo espiritual y comunión divina; en el mundo venidero, vida eterna en el progreso de las realidades espirituales divinas del Padre del Paraíso.
\vs p141 7:6 Jesús hizo un gran hincapié en lo que él denominaba las dos verdades de primordial importancia en las enseñanzas del reino, y que son: conseguir la salvación por medio de la fe, y solo fe, en conjunción con la revolucionaria enseñanza de lograr la libertad del hombre mediante el reconocimiento sincero de la verdad: “conoceréis la verdad y la verdad os hará libres”. Jesús era la verdad manifestada en la carne, y él prometió enviar a su espíritu de la verdad a los corazones de todos sus hijos tras su regreso al Padre de los cielos.
\vs p141 7:7 El Maestro impartía a estos apóstoles los fundamentos de la verdad para toda una era de la tierra. Ellos oían con frecuencia sus enseñanzas cuando en realidad lo que decía tenía el fin de inspirar e iluminar a otros mundos. Él mismo ejemplificaba un plan de vida nuevo y sin precedentes. Desde el punto de vista humano, él era ciertamente un judío, pero vivió su vida para todo el mundo como un mortal del planeta.
\vs p141 7:8 Para garantizar el reconocimiento de su Padre en el despliegue del plan del reino, Jesús explicó que había ignorado deliberadamente a los “grandes de la tierra”. Empezó su labor con los pobres, la misma clase social que había sido tan olvidada por la mayor parte de las religiones evolutivas de tiempos anteriores. No despreciaba a ningún hombre; su plan era para todo el mundo, incluso para todo un universo. Era tan atrevido y contundente en estas declaraciones que hasta Pedro, Santiago y Juan estuvieron tentados de pensar que posiblemente estaba fuera de sí.
\vs p141 7:9 Procuró impartir sutilmente a estos apóstoles la verdad de que había venido en esta misión de gracia, no para servir de ejemplo a unas pocas criaturas de la tierra, sino para establecer e ilustrar un modelo de vida humana para todos los pueblos de todos los mundos de la totalidad de su universo. Y este patrón se acercaba a la perfección más elevada, incluso a la bondad última del Padre universal. Pero los apóstoles no alcanzaban a entender el significado de sus palabras.
\vs p141 7:10 Afirmó que había venido en calidad de maestro, de maestro enviado desde el cielo para exponer la verdad espiritual a la mente material. Y esto fue exactamente lo que hizo; era maestro, no predicador. Desde el punto de vista humano, Pedro, como predicador, era mucho más eficiente que Jesús. La predicación de Jesús era tan efectiva debido a su excepcional persona, no tanto por su convincente oratoria o atractivo emocional. Jesús hablaba directamente al alma de los hombres. Era un maestro del espíritu del hombre, pero a través de la mente. Vivía con los hombres.
\vs p141 7:11 Fue en esta ocasión cuando Jesús insinuó a Pedro, Santiago y Juan que su labor en la tierra estaba en algunos aspectos limitado por el cometido que le había dado su “consejero de arriba”, en referencia a las instrucciones recibidas por Emanuel, su hermano del Paraíso, con anterioridad a su ministerio de gracia. Les dijo que él había venido para hacer la voluntad y solo la voluntad de su Padre. Estando, pues, motivado por un único e incondicional propósito, la maldad del mundo no le preocupaba como para crearle ansiedad.
\vs p141 7:12 Los apóstoles estaban empezando a apreciar la genuina afabilidad de Jesús. Aunque el Maestro era accesible, siempre vivía independiente, y a un nivel superior, de todos los seres humanos. Ni por un momento se dejó dominar por sentimientos puramente mortales ni estuvo sujeto al débil juicio de otros seres humanos. Hizo caso omiso de la opinión pública, y no se dejó influenciar por el elogio. Rara vez se detuvo para rectificar los malentendidos o se sintió ofendido por tergiversaciones. Nunca pidió consejo a los hombres; nunca pidió a nadie que orara por él.
\vs p141 7:13 Santiago se asombraba de cómo Jesús parecía ver el final desde el principio. El Maestro raramente parecía sorprenderse. Nunca se agitaba, enfadaba o desconcertaba. Nunca le pidió disculpas a nadie. A veces estaba triste, pero jamás se sintió desalentado.
\vs p141 7:14 Juan observó con gran nitidez que Jesús, a pesar de todos sus atributos divinos era, al fin y al cabo, humano. Jesús vivió como un hombre entre los hombres y entendía y amaba a los hombres, y sabía cómo guiarlos. En su vida personal, él era tan humano y, no obstante, tan intachable. Y careció siempre de todo egoísmo.
\vs p141 7:15 Aunque Pedro, Santiago y Juan no fueron capaces de entender suficientemente todo lo que Jesús dijo en esta ocasión, sus afables palabras perduraron en sus corazones y, tras la crucifixión y resurrección, brotaron generosamente en su memoria para enriquecer y alegrar su futuro ministerio. No es de extrañar que estos apóstoles no comprendieran por completo las palabras del Maestro, porque les estaba trazando el plan de una nueva era.
\usection{8. SU LABOR EN JERICÓ}
\vs p141 8:1 Durante las cuatro semanas de estancia en Betania, al otro lado del Jordán, Andrés, varias veces a la semana, designaba a un par de apóstoles para que fueran a Jericó por uno o dos días. Juan tenía muchos creyentes en Jericó, y la mayoría de ellos acogieron con satisfacción las enseñanzas más avanzadas de Jesús y sus apóstoles. Durante estas visitas a Jericó, los apóstoles empezaron a poner en práctica más expresamente las instrucciones dadas por Jesús de asistir a los enfermos; fueron a cada una de las casas de la ciudad e intentaron dar consuelo a quien se encontrase en aflicción.
\vs p141 8:2 Los apóstoles realizaron alguna labor pública en Jericó, pero de manera más sosegada y personal. Descubrieron en aquel momento que la buena nueva del reino resultaba de gran consuelo a los enfermos; que su mensaje proporcionaba sanación a los afligidos. Y fue en Jericó donde los doce primeramente llevaron enteramente a la práctica el cometido que les había dado Jesús de predicar la buena nueva del reino y asistir a los afligidos.
\vs p141 8:3 En el camino de subida a Jerusalén, se detuvieron en Jericó, ciudad en la que les alcanzó un grupo de personas llegadas de Mesopotamia para consultar con Jesús sobre sus enseñanzas. Los apóstoles tenían planeado estar allí solamente un día, pero, al llegar estos buscadores de la verdad de Oriente, Jesús permaneció con ellos tres días. Y todos regresaron a sus diferentes hogares a lo largo del Éufrates, gozosos en el conocimiento de las nuevas verdades del reino de los cielos.
\usection{9. SALEN PARA JERUSALÉN}
\vs p141 9:1 El lunes, último día de marzo, Jesús y los apóstoles comenzaron su viaje colina arriba en dirección a Jerusalén. Lázaro de Betania había bajado al Jordán dos veces para ver a Jesús, y se habían tomado todo tipo de medidas para que el Maestro y sus apóstoles establecieran su sede en la casa de Lázaro y sus hermanas, en Betania, durante todo el tiempo que quisieran permanecer en Jerusalén.
\vs p141 9:2 Los discípulos de Juan se quedaron en Betania, al otro lado del Jordán, enseñando y bautizando a las multitudes, así que a Jesús solo lo acompañaban los doce cuando llegó a la casa de Lázaro. Aquí, Jesús y los apóstoles estuvieron cinco días descansando y retomando fuerzas antes de proseguir con su viaje hasta Jerusalén para pasar la Pascua. En la vida de Marta y María, sería un magno acontecimiento tener al Maestro y sus apóstoles en la casa de su hermano y poder atender sus necesidades.
\vs p141 9:3 El domingo por la mañana del día 6 de abril, Jesús y los apóstoles fueron a Jerusalén; se trataba de la primera vez que el Maestro y los doce estaban allí todos reunidos.
