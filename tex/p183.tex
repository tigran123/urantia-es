\upaper{183}{Traición y arresto de Jesús}
\author{Comisión de seres intermedios}
\vs p183 0:1 Cuando Jesús terminó de despertar a Pedro, Santiago y Juan, les recomendó que se fueran a sus tiendas y trataran de dormir con el fin de que estuvieran listos para cumplir con sus obligaciones de la mañana siguiente. Pero, para entonces, los tres apóstoles estaban bien despiertos; aquellos breves descansos les habían reconfortado y, además, estaban sobreexcitados y ansiosos por la llegada al lugar de dos mensajeros que, en estado de agitación, preguntaron por David Zebedeo y se marcharon rápidamente a buscarlo en cuanto Pedro les informó de dónde estaba vigilando.
\vs p183 0:2 Aunque ocho de los apóstoles estaban profundamente dormidos, los griegos, acampados junto a ellos, sintiendo un mayor temor ante los problemas que podían surgir, habían apostado a un centinela para que les diera la voz de alarma en caso de peligro. Cuando estos dos mensajeros entraron de forma apresurada en el campamento, el centinela griego avisó a todos sus compatriotas, los cuales salieron de sus tiendas, vestidos y armados por completo. Todos en el campamento se despertaron en aquel momento, excepto los ocho apóstoles; Pedro quiso llamar a sus compañeros, pero Jesús manifiestamente se lo prohibió. Con calma, el Maestro les pidió que volvieran a sus tiendas, pero ellos se mostraron reacios a atender su petición.
\vs p183 0:3 Al no lograr dispersar a sus seguidores, el Maestro los dejó y caminó hasta la prensa de aceite de oliva más abajo, situada cerca de la entrada al parque de Getsemaní. Aunque los tres apóstoles, los griegos y otros allí acampados dudaron si ir tras él de inmediato, Juan Marcos cruzó corriendo por los olivos y se escondió en un pequeño cobertizo cerca de la prensa. Jesús se retiró del campamento y de sus amigos para que sus captores, cuando llegaran, pudieran arrestarlo a él sin causar ninguna molestia a sus apóstoles. El Maestro temía que sus apóstoles se despertaran y presenciaran el momento de su arresto, no fuera que la impactante traición de Judas soliviantara de tal manera su animosidad que ofrecieran resistencia a los soldados y fuesen detenidos junto con él. Temía que si eran arrestados pudieran igualmente perecer con él.
\vs p183 0:4 Aunque Jesús sabía que el plan de matarlo había partido de los consejos de los dirigentes de los judíos, también era consciente de que estas nefastas maquinaciones contaban con la total aprobación de Lucifer, Satanás y Caligastia. Y sabía bien que estos rebeldes de los mundos se complacerían en ver a todos los apóstoles morir con él.
\vs p183 0:5 Jesús se sentó en la prensa de aceite solo, y aguardó allí la llegada del traidor, y, en aquel instante, únicamente Juan Marcos y una innumerable multitud de observadores celestiales lo vieron.
\usection{1. LA VOLUNTAD DEL PADRE}
\vs p183 1:1 Existe el gran peligro de malinterpretar el significado de numerosas afirmaciones y de múltiples acontecimientos relacionados con la terminación de la andadura del Maestro en la carne. El cruel tratamiento infligido a Jesús por parte de ignorantes sirvientes y de guardias insensibles, el proceder injusto sufrido en sus juicios y la actitud despiadada de las autoridades religiosas no se deben confundir con el hecho de que Jesús, al someterse pacientemente a este sufrimiento y humillación, estaba verdaderamente haciendo la voluntad del Padre del Paraíso. Era, de hecho y en verdad, la voluntad del Padre que su Hijo bebiera de la copa de la experiencia humana hasta el fondo, desde su nacimiento hasta su muerte, pero el Padre de los cielos no instigó en absoluto el brutal comportamiento de esos seres humanos, supuestamente civilizados, que tan atrozmente torturaron al Maestro y que tan horriblemente sometieron a su persona, que no ofreció resistencia, a continuas vejaciones. Estas experiencias inhumanas y estremecedoras que Jesús tuvo que soportar en las horas finales de su vida como mortal no formaban parte, de ningún modo, de la voluntad divina del Padre, a la que su naturaleza humana se había comprometido a cumplir tan triunfalmente en el momento de la rendición final del hombre a Dios, tal como lo manifestó en la triple oración ofrecida en el jardín, mientras sus agotados apóstoles dormían físicamente exhaustos.
\vs p183 1:2 183:1.2 (1972.1) El Padre de los cielos deseaba que el Hijo de gracia terminara su andadura en la tierra \bibemph{de forma natural,} de la misma manera en la que todos los mortales deben terminar su vida en la tierra y en la carne. Los hombres y mujeres ordinarios no pueden esperar que sus últimas horas en la tierra y su posterior episodio de muerte se les haga más fáciles gracias a alguna indulgencia especial de Dios. Por ello, Jesús eligió dar su vida en la carne en consonancia con la manifestación de los acontecimientos naturales, negándose con tenacidad a librarse de la garra cruel del tenebroso derrotero de acontecimientos inhumanos, que lo empujaron de forma horrible e inflexible hasta su inconcebible humillación e ignominiosa muerte. Y cada brizna de esta impactante manifestación de odio y de esta inaudita demostración de crueldad se debió a la acción de hombres mortales malvados y perversos. El Dios de los cielos no lo quiso así, como tampoco la dictaron los archienemigos de Jesús, aunque si hicieron bastante para asegurarse de que humanos irreflexivos y pecaminosos rechazaran de aquella manera al Hijo de gracia. Hasta el padre del pecado apartó su rostro ante el intolerable horror de la escena de la crucifixión.
\usection{2. JUDAS EN LA CIUDAD}
\vs p183 2:1 Tras abandonar tan bruscamente la mesa durante el transcurso de la última cena, Judas se marchó directamente a casa de su primo y, luego, ambos fueron enseguida a ver al capitán de los guardias del templo. Judas pidió al capitán que congregara a los guardias y le informó de que estaba listo para llevarlos hasta Jesús. Al haber aparecido Judas allí algo antes de lo que se le esperaba, se produjo cierto retraso en el momento de partir para la casa de Marcos, donde Judas esperaba hallar a Jesús aún conversando con los apóstoles. El Maestro y los once salieron de la casa de Elías Marcos unos quince minutos antes de que el traidor y los guardias aparecieran. Para cuando los captores llegaron a la casa de Marcos, Jesús y los once ya estaban fuera de los muros de la ciudad, de camino al campamento del Monte de los Olivos.
\vs p183 2:2 A Judas le irritó bastante no haber encontrado a Jesús en la residencia de Marcos y en compañía de los once, ya que solo dos estaban armados para poder ofrecer cualquier oposición. Conocía que, por la tarde, cuando salieron del campamento, solo Simón Pedro y Simón Zelotes portaban sendas espadas; Judas esperaba prender a Jesús mientras la ciudad estaba tranquila, y con pocas posibilidades de que se produjera ninguna resistencia. El traidor temía que si esperaba a que volvieran al campamento habría allí más de sesenta fervientes discípulos; asimismo sabía que Simón Zelotes tenía almacenadas una gran cantidad de armas. A la vez, Judas sentía un mayor nerviosismo cada vez que pensaba cómo lo detestarían los once leales apóstoles, y temía que trataran de matarlo. No solo era desleal, sino un verdadero cobarde de corazón.
\vs p183 2:3 Al no hallar a Jesús en el aposento alto, Judas le pidió al capitán de los guardias que volviera al templo. En aquel momento, los dirigentes judíos habían comenzado a congregarse en la casa del sumo sacerdote, preparándose para la llegada de Jesús, ya que su trato con el traidor establecía que el arresto de Jesús se llevaría a cabo a la medianoche de aquel día. Judas explicó a sus acompañantes que no lo había podido encontrar en la casa de Marcos, y que sería necesario ir a Getsemaní para arrestarlo. El traidor añadió que había más de sesenta de sus fieles seguidores acampados con él, y que todos ellos estaban bien armados. Los dirigentes de los judíos recordaron a Judas que Jesús había predicado siempre la no confrontación, pero Judas respondió que no podían confiar en que todos los seguidores de Jesús observaran dicha enseñanza. Temiendo realmente por su vida, Judas se atrevió a pedir una compañía de cuarenta soldados armados. Y, dado que las autoridades judías no tenían bajo su jurisdicción a un grupo tan numeroso de hombres armados, fueron de inmediato a la fortaleza de Antonia para solicitar al comandante romano que les facilitara tal escolta; pero cuando este se enteró de que tenían la intención de detener a Jesús, se negó enseguida a acceder a esta solicitud y los remitió a su oficial superior. De ese modo, tomó más de una hora, yendo de una autoridad a otra, hasta que se vieron finalmente obligados a dirigirse al mismo Pilato para obtener permiso y poder recurrir a los guardias romanos armados. Era tarde cuando llegaron a la casa de Pilato, y él ya se había retirado con su esposa a sus aposentos privados. Pilato dudó si involucrarse en aquello y, mucho más, cuando su esposa le había pedido que no concediera tal petición. Pero, puesto que el oficial al mando del sanedrín judío estaba presente e hizo personalmente aquel requerimiento de asistencia, el gobernador consideró oportuno otorgarle la petición, pensando que podría, más adelante, arreglar cualquier mal que estos estuvieran dispuestos a cometer.
\vs p183 2:4 Por ello, cuando Judas Iscariote salió del templo, sobre las once y media de la noche, lo hacía acompañado de más de sesenta personas, incluyendo guardias del templo, soldados romanos y siervos de los sumos sacerdotes y de los líderes judíos que iban por curiosidad.
\usection{3. EL ARRESTO DEL MAESTRO}
\vs p183 3:1 A medida que esta compañía de soldados y guardias armados, portando antorchas y linternas, se iba acercando al jardín, Judas se adelantó bastante de ellos, con el fin de estar preparado para identificar rápidamente a Jesús y facilitar a los captores que lo apresaran antes de que sus acompañantes pudieran movilizarse en su defensa. Y también había otra razón por la que Judas decidió ir por delante de los enemigos del Maestro: pensó que así quizás pudiera parecer que él había llegado allí antes que los soldados, por lo que los apóstoles y otros que se hubieran reunido en torno a Jesús no lo relacionarían directamente con los guardias armados que venían pisándole los talones. Judas llegó incluso a pensar en fingir que se había dado prisa para avisarles de la llegada de los captores, pero este plan se vio frustrado cuando Jesús interrumpió esta acción saludando al traidor. El Maestro habló a Judas con amabilidad, pero lo recibió como a un traidor.
\vs p183 3:2 En cuanto Pedro, Santiago y Juan, junto con otras treinta personas también acampadas, vieron a estos hombres armados con antorchas caminar por la cumbre de la colina, supieron que estos soldados venían a arrestar a Jesús y todos ellos bajaron de prisa hasta la prensa de aceite donde el Maestro estaba sentado a solas, bajo la luz de la luna. Mientras que, por un lado, se acercaba la compañía de soldados, por el otro, lo hacían los tres apóstoles y sus acompañantes. Al adelantarse Judas para abordar al Maestro, ambos grupos permanecieron inmóviles, quedando el Maestro entre ellos, mientras Judas se preparaba para darle en la frente el beso de la traición.
\vs p183 3:3 El traidor tenía la esperanza de haber podido, tras conducir a los guardias hasta Getsemaní, sencillamente señalarles quién era Jesús o, como mucho, cumplir la promesa de saludarlo con un beso, y retirarse entonces, rápidamente, de allí. Judas temía bastante que todos los apóstoles estuvieran allí presentes, y centraran su ataque en él en represalia por haberse atrevido a traicionar a su amado maestro. Pero cuando el Maestro lo saludó como a un traidor, quedó tan confundido que no hizo ningún intento por huir.
\vs p183 3:4 Jesús hizo un último esfuerzo para impedir que Judas le traicionara, puesto que, antes de que el traidor llegara hasta él, se echó a un lado, y dirigiéndose al capitán de los romanos, el primer soldado de la izquierda, le dijo: “¿A quién buscáis?”. El capitán respondió: “A Jesús de Nazaret”. Entonces Jesús se puso de inmediato delante del oficial y, con la majestuosa calma del Dios de toda esta creación dijo: “Yo soy”. Muchos de estos hombres armados habían oído a Jesús enseñar en el templo, otros sabían de sus portentosas obras, y cuando lo escucharon dar a conocer su identidad con tanta valentía, los que estaban en primera fila retrocedieron y cayeron a tierra de repente. Se sobrecogieron ante su serenidad y la grandeza con la que se había identificado. No había, por lo tanto, ninguna necesidad de que Judas continuara con su traición. El Maestro se había revelado valerosamente a sus enemigos, y ellos podrían haberlo arrestado sin la ayuda de Judas. Pero el traidor tenía que hacer algo para dar razón de su presencia con este grupo armado; además, quería hacer gala de estar cumpliendo su parte de la traición acordada con los dirigentes de los judíos, y tener así derecho a la gran recompensa y a los honores con los que él creía lo colmarían como compensación por su promesa de entregar a Jesús en sus manos.
\vs p183 3:5 Conforme los guardias se recuperaban tras haber flaqueado al ver a Jesús por primera vez y oír su inaudita voz y, a medida que los apóstoles y discípulos se iban aproximando, Judas avanzó hacia Jesús y, besándolo en la frente, dijo: “¡Salve, Señor y maestro!”. Y al abrazar, pues, Judas a su Maestro, Jesús le dijo: “Amigo, ¿es que no has hecho ya bastante? ¿Tienes que traicionar también al Hijo del Hombre con un beso?”.
\vs p183 3:6 Los apóstoles y discípulos se quedaron literalmente atónitos ante lo que vieron. Durante un momento, nadie se movió. Luego Jesús, soltándose del abrazo traicionero de Judas, se adelantó a los guardias del templo y a los soldados y volvió a preguntarles: “¿A quién buscáis?”. Y, de nuevo, el capitán dijo: “A Jesús de Nazaret”. Y Jesús contestó nuevamente: “Os he dicho que yo soy. Si me buscáis, pues, a mí, dejad marchar a estos. Estoy listo para ir con vosotros”.
\vs p183 3:7 Jesús estaba preparado para volver a Jerusalén con los guardias, y el capitán y los soldados estaban totalmente dispuestos a permitir que los tres apóstoles y quienes los acompañaban siguieran en paz su propio camino. Pero antes de salir, estando Jesús allí esperando las órdenes del capitán, cierto Malco, el guardaespaldas sirio del sumo sacerdote, se acercó a Jesús con la intención de atarle las manos a la espalda, sin que el capitán romano lo hubiera ordenado. Cuando Pedro y los que estaban con él vieron que su Maestro estaba siendo sometido a esta indignidad, no fueron capaces de dominarse. Pedro desenvainó la espada y, con los demás, atacó a Malco. Pero antes de que pudieran intervenir los soldados en defensa del sirviente del sumo sacerdote, Jesús levantó la mano a Pedro en actitud de prohibición y, hablándole con severidad, le dijo: “Pedro, envaina tu espada, porque todos los que tomen espada, a espada perecerán. ¿Es que no entiendes que la voluntad del Padre es que yo beba de esta copa? ¿Y acaso no sabes también que aún así podría ordenar a más de doce legiones de ángeles y otros seres celestiales que me libren de las manos de estos pocos hombres?”.
\vs p183 3:8 Aunque Jesús logró detener tal demostración de resistencia física de sus seguidores, aquello fue suficiente para suscitar el temor del capitán de los guardias, que ahora, con la ayuda de sus soldados, sujetó con fuerza a Jesús y lo ató. Y mientras ataban sus manos con gruesas cuerdas, Jesús les dijo: “¿Contra mí habéis venido con espadas y palos como si fuera un ladrón? He estado con vosotros cada día en el templo, enseñando públicamente a la gente, y no me detuvisteis”.
\vs p183 3:9 Cuando Jesús quedó atado, el capitán, temiendo que sus seguidores intentaran rescatarlo, dio órdenes de prenderlos; pero los soldados no fueron lo suficientemente rápidos, ya que, al haber oído estas órdenes, los seguidores de Jesús huyeron a toda prisa de vuelta a la quebrada. Todo este tiempo, Juan Marcos había permanecido oculto en el cobertizo cercano. Cuando los soldados se encaminaron de regreso a Jerusalén con Jesús, Juan Marcos intentó salir sigilosamente del cobertizo para alcanzar a los apóstoles y discípulos en su huida; pero en ese momento, uno de los últimos soldados, que volvía de perseguir a los discípulos, pasó cerca y, viendo al joven en su manto de lino, corrió tras él, llegando casi a atraparlo. De hecho, el soldado se aproximó tanto a Juan como para agarrar su manto, pero el joven se deshizo de la vestimenta, pudiendo escapar desnudo, mientras el soldado se hacía con el manto vacío. Juan Marcos se dirigió precipitadamente hasta el sendero alto, lugar en el que se hallaba David Zebedeo. Cuando le contó lo que había ocurrido, ambos volvieron a toda prisa hasta las tiendas en las que dormían los apóstoles e informaron a los ocho de la traición al Maestro y de su arresto.
\vs p183 3:10 Cuando los ocho apóstoles se despertaban, los que habían huido arriba, a la quebrada, empezaban a regresar, y todos se congregaron junto a la prensa de aceite para deliberar sobre lo que debían hacer. Entretanto, Simón Pedro y Juan Zebedeo, que se habían ocultado entre los olivos, habían ido tras la turba de soldados, guardianes y sirvientes que llevaban a Jesús de regreso a Jerusalén como si se tratara de un peligroso criminal. Juan los seguía de cerca, mientras que Pedro lo hacía a más distancia. Después de zafarse del agarre del soldado, Juan Marcos se proveyó de un manto que encontró en la tienda de Simón Pedro y Juan Zebedeo. Sospechaba que los guardias llevarían a Jesús a la casa de Anás, el sumo sacerdote emérito; así pues, sorteando a la carrera los olivares, llegó allí antes que la muchedumbre, ocultándose próximo a la entrada de la puerta del palacio del sumo sacerdote.
\usection{4. CONVERSACIONES JUNTO A LA PRENSA DE ACEITE}
\vs p183 4:1 Santiago Zebedeo se vio separado de Simón Pedro y de su hermano Juan y se unió, en la prensa de aceite, a los demás apóstoles y a sus compañeros del campamento para deliberar sobre lo que debía hacerse en relación al arresto del Maestro.
\vs p183 4:2 Al estar liberado de cualquier responsabilidad respecto a la dirección del grupo de sus compañeros apóstoles, Andrés, en esta, la más grave crisis de sus vidas, se mantuvo en silencio. Tras una breve conversación informal, Simón Zelotes se erigió sobre el muro de piedra de la prensa y, haciendo un apasionado llamamiento a la lealtad al Maestro y a la causa del reino, instó a los apóstoles y a los demás discípulos a ir de prisa tras la turba que llevaba a Jesús y rescatarlo. La mayoría de los allí presentes habría estado dispuesto a seguir al combativo Simón de no haber sido por el consejo de Natanael, que, poniéndose de pie cuando este terminó de hablar, llamó la atención de todos sobre las enseñanzas, a menudo repetidas por Jesús, sobre la confrontación pasiva. Les recordó, además, que Jesús esa misma noche les había pedido que preservaran sus vidas para cuando les llegara el momento de salir al mundo a proclamar la buena nueva del evangelio del reino celestial. Y Natanael se vio reforzado en su postura por Santiago Zebedeo, que contó cómo Pedro y otros habían desenfundado sus espadas para defender al Maestro ante su arresto y cómo Jesús había mandado a Simón Pedro y a los demás a que las envainaran. Mateo y Felipe también aportaron sus ideas pero no salió nada definitivo de estas intervenciones, hasta que Tomás, haciéndoles hincapié en el hecho de que Jesús había aconsejado a Lázaro que no se arriesgara a morir, señaló que no podían hacer nada por salvar a su Maestro, ya que él se negaba a permitir que sus amigos lo defendieran e insistía en abstenerse de emplear sus poderes divinos para frustrar las intenciones de sus enemigos humanos. Tomás los convenció para que se dispersaran, cada uno por su lado, entendiendo que David Zebedeo permanecería en el campamento con la idea de mantener un punto de intercambio de información y un centro de mensajería para el grupo. A las dos y media de la mañana, el campamento se había quedado desierto; solo David permaneció disponible con tres o cuatro mensajeros; había enviado a los demás a conocer el lugar adónde se habían llevado a Jesús y lo que harían con él.
\vs p183 4:3 Cinco de los apóstoles ---Natanael, Mateo, Felipe y los gemelos--- fueron a Betfagé y Betania para esconderse. Tomás, Andrés, Santiago y Simón Zelotes lo hicieron en la ciudad. Simón Pedro y Juan Zebedeo continuaron hasta la casa de Anás.
\vs p183 4:4 Poco después del amanecer, Simón Pedro, deambulando, volvió al campamento de Getsemaní. Era la imagen del desconsuelo y de la más profunda desesperación. David lo envió con un mensajero para que se uniera a su hermano Andrés, que estaba en Jerusalén en la casa de Nicodemo.
\vs p183 4:5 Hasta el último momento de la crucifixión, Juan Zebedeo permaneció, tal como Jesús se lo había pedido, siempre próximo, y era él quien, de hora en hora, proporcionaba información a los mensajeros, que ellos se la llevaban a David, al campamento del jardín de Getsemaní, para luego remitirla a los apóstoles escondidos y a la familia de Jesús.
\vs p183 4:6 ¡En verdad, herirán al pastor y las ovejas serán dispersadas! Aunque todos ellos vagamente se daban cuenta de que Jesús les había prevenido respecto a esta misma situación, estaban tan sumamente conmocionados por la súbita desaparición del Maestro que eran incapaces de pensar con claridad.
\vs p183 4:7 Poco después del amanecer y, justo tras reunirse Pedro con su hermano, Judá, el hermano carnal de Jesús, llegó al campamento, casi sin aliento y con antelación al resto de la familia de Jesús, aunque solo para saber que el Maestro estaba arrestado y, de nuevo, bajó a toda prisa por la carretera de Jericó para informar de aquello a su madre y a sus hermanos y hermanas. David Zebedeo envió palabra a la familia de Jesús, mediante Judá, para que se reunieran en la casa de Marta y de María en Betania y aguardaran allí las noticias que sus mensajeros les llevarían de forma regular.
\vs p183 4:8 Aquella era la situación durante la última mitad de la noche del jueves y las primeras horas de la mañana del viernes en lo que se refiere a los apóstoles, a los discípulos más destacados y a la familia terrenal de Jesús. Y todos estos grupos y personas estaban en contacto unos con otros por medio del servicio de mensajeros, que David Zebedeo mantuvo operativo desde su sede en el campamento de Getsemaní.
\usection{5. DE CAMINO AL PALACIO DEL SUMO SACERDOTE}
\vs p183 5:1 Antes de salir del jardín con Jesús, se suscitó una disputa entre el capitán judío de los guardias del templo y el capitán romano de la compañía de soldados en cuanto al lugar al que se debería llevar a Jesús. El capitán de los guardias del templo dio órdenes para que lo llevaran a Caifás, el actual sumo sacerdote. El capitán de los soldados romanos mandó que lo llevaran al palacio de Anás, el antiguo sumo sacerdote y suegro de Caifás. Y lo hizo así porque los romanos acostumbraban a tratar directamente con Anás en todos los asuntos relativos a la aplicación de las leyes eclesiásticas judías. Las órdenes del capitán romano se obedecieron, y llevaron a Jesús a la casa de Anás para tener una audiencia previa.
\vs p183 5:2 Judas marchaba cerca de los capitanes, oyendo todo lo que se decía, pero sin tomar parte en aquella disputa, porque el capitán judío y el capitán romano ni siquiera hablaban con el traidor, tal era el desprecio que sentían por él.
\vs p183 5:3 En ese momento, Juan Zebedeo, recordando las instrucciones de su Maestro de permanecer siempre junto a él, se acercó de prisa a Jesús, que caminaba entre los dos capitanes. Al ver el capitán este movimiento de Juan, el comandante de los guardias del templo, dijo a su asistente: “Prende a este hombre y átalo. Es uno de sus seguidores”. Pero cuando el capitán romano oyó aquello y, mirando a su alrededor, vio a Juan, dio órdenes de que el apóstol se acercara a él, y de que nadie lo importunara. Luego, el capitán romano dijo al capitán judío: “Este hombre no es ni un traidor ni un cobarde. Lo vi en el jardín, y no desenfundó la espada para resistirse a nosotros. Tiene el valor de venir y acompañar a su Maestro, y nadie lo molestará. La ley romana permite a cualquier preso que tenga al menos un amigo que esté con él ante el tribunal del juicio, y no se le impedirá que se quede al lado del detenido, su Maestro”. Y cuando Judas escuchó aquello, se sintió tan avergonzado y humillado que empezó a quedarse rezagado hasta marchar detrás del grupo, llegando solo al palacio de Anás.
\vs p183 5:4 Y esto explica por qué se le permitió a Juan Zebedeo permanecer cerca de Jesús durante los duros momentos de esa noche y del día siguiente. Los judíos temían decirle nada a Juan o incomodarlo de alguna manera, porque había adquirido algo así como el estatus de asesor romano, asignado para hacer de observador de las actuaciones del tribunal eclesiástico judío. La posición de privilegio de Juan se afianzó aún más cuando el romano, al entregar a Jesús al capitán de los guardias del templo, en la puerta del palacio de Anás, dirigiéndose a su asistente, le dijo: “Acompaña a este preso y asegúrate de que los judíos no lo maten sin el consentimiento de Pilato. Vigila que no lo asesinen, y mira que se le permita a su amigo, el galileo, que se quede con él y observe todo lo que acontezca”. Y, de esa manera, pudo Juan estar cerca de Jesús hasta el mismo momento de su muerte en la cruz, aunque los otros diez apóstoles se vieran obligados a permanecer escondidos. Juan actuaba bajo protección romana, y los judíos no se atrevieron a molestarlo hasta después de la muerte del Maestro.
\vs p183 5:5 Y, durante todo el camino hasta el palacio de Anás, Jesús no abrió la boca. Desde su arresto hasta el momento de su aparición ante Anás, el Hijo del Hombre no dijo ni una sola palabra.
