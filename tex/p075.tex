\upaper{75}{La transgresión de Adán y de Eva}
\author{Solonia}
\vs p075 0:1 Tras más de cien años de esfuerzos en Urantia, Adán pudo comprobar que se estaban haciendo pocos progresos fuera del Jardín; en general, el mundo no parecía avanzar mucho. Lograr la mejora de las razas parecía estar muy distante, y las circunstancias parecían tan desoladoras que para solventarse se necesitaban tomar medidas no contempladas en los planes originales. Al menos esto es lo que pasaba a menudo por la mente de Adán, y así se lo expresó muchas veces a Eva. Adán y su pareja eran leales, pero estaban aislados de los de su mismo orden y profundamente consternados por la lamentable situación de su mundo.
\usection{1. EL PROBLEMA DE URANTIA}
\vs p075 1:1 La misión adánica en Urantia, un planeta experimental, aislado y marcado por la rebelión, era una labor colosal. El hijo y la hija material pronto se dieron cuenta de la dificultad y la complejidad de su cometido. A pesar de ello, intentaron con valentía resolver sus múltiples problemas; pero, cuando emprendieron la importante labor de apartar de las estirpes humanas a los seres deficientes y en declive degenerativo, se quedaron bastante consternados. No veían ninguna salida a este dilema, y tampoco podían contar con el asesoramiento de sus superiores de Jerusem o de Edentia. Estaban allí, aislados y teniendo que afrontar cada día alguna nueva complejidad, algún problema sin aparente solución.
\vs p075 1:2 En condiciones normales, la primera tarea del adán y eva planetarios sería la coordinación y la mezcla de las razas. Pero en Urantia, llevar a cabo tal proyecto parecía casi imposible. Las razas, aunque biológicamente aptas, jamás se habían depurado de sus estirpes deficientes y retrasadas.
\vs p075 1:3 Adán y Eva se vieron en una esfera que no estaba de manera alguna preparada para la proclamación de la hermandad del hombre, en un mundo que caminaba vacilante, que estaba inmerso en una deplorable oscuridad espiritual y afligido por una confusión agravada por el fracaso de la misión del gobierno anterior. La mente y los valores morales se hallaban en un nivel bajo y, en lugar de emprender la labor de llevar a efecto la unidad religiosa, debían recomenzar la tarea de instruir a los habitantes en las formas más simples de creencias religiosas. En lugar de encontrarse con un idioma ya listo para implantarse, tenían que hacer frente a la confusión existente en el mundo por la proliferación de cientos y cientos de dialectos locales. Ningún adán de servicio en un planeta se había asentado nunca en un mundo más difícil que este; los obstáculos parecían insuperables y los problemas se escapaban a la solución de criatura alguna.
\vs p075 1:4 Estaban aislados, y el enorme sentimiento de soledad que pesaba sobre ellos se intensificó aún más por la partida anticipada de los síndicos melquisedecs. Solo indirectamente, por medio de los órdenes angélicos, se podían comunicar con seres de fuera del planeta. Lentamente su valentía se debilitaba, su ánimo decaía y, algunas veces, casi les flaqueaba la fe.
\vs p075 1:5 Y esta es la verdadera imagen de la consternación que estas dos nobles almas sentían ante la labor con la que se enfrentaban. Ambos eran plenamente conscientes de la ingente tarea que conllevaba la realización de su misión planetaria.
\vs p075 1:6 Es posible que ninguno de los hijos materiales de Nebadón tuvo jamás que hacer frente a una tarea tan difícil y aparentemente imposible de llevar a cabo como la que tenían Adán y Eva ante la lamentable situación de Urantia. Pero algún día habrían tenido éxito si hubiesen sido más previsores y \bibemph{pacientes}. Ambos, especialmente Eva, eran demasiado impacientes; no estaban dispuestos a aquietarse ante aquella larguísima prueba de resistencia. Querían ver resultados inmediatos, y los consiguieron, pero los que obtuvieron de esta manera demostraron ser sumamente devastadores tanto para ellos como para su mundo.
\usection{2. LAS MAQUINACIONES DE CALIGASTIA}
\vs p075 2:1 Caligastia realizó frecuentes visitas al Jardín y mantuvo muchas conversaciones con Adán y Eva, pero estos se mostraron inflexibles ante todas sus arriesgadas sugerencias y alternativas de atajos. Tenían ante sí suficientes muestras de su rebelión como para no estar convenientemente inmunizados contra sus provocadoras propuestas. Ni incluso la joven progenie de Adán se vio afectada por las proposiciones de Daligastia. Y, por supuesto, ni Caligastia ni su colaborador tenían poder para influir en ninguna persona en contra de su voluntad y, mucho menos, para persuadir a los hijos de Adán a que hicieran el mal.
\vs p075 2:2 Es preciso recordar que Caligastia era todavía el príncipe planetario a titulo nominal de Urantia y, aunque descarriado, era un elevado hijo del universo local. No llegaría a ser finalmente depuesto hasta los tiempos de Cristo Miguel en Urantia.
\vs p075 2:3 Pero el príncipe caído fue persistente y determinado. Pronto renunció a convencer a Adán y, astutamente, decidió intentar atacar a Eva de forma indirecta. El maligno llegó a la conclusión de que la única posibilidad que tenía de triunfar en sus fines era utilizar hábilmente a las personas adecuadas que pertenecieran al estrato superior del grupo nodita, descendientes de los colaboradores de su antigua comitiva corpórea. Y, por consiguiente, se urdieron planes para tender una trampa a la madre de la raza violeta.
\vs p075 2:4 \pc Eva nunca tuvo la menor intención de hacer nada que contraviniese los planes de Adán ni que pusiese en peligro las responsabilidades planetarias que ambos habían contraído. Conociendo la propensión de la mujer a buscar resultados inmediatos en lugar de planear con previsión y obtener estos en un plazo más largo, los melquisedecs, antes de partir, habían instado encarecidamente a Eva acerca de los peculiares peligros que acuciaban su situación de aislamiento en el planeta; le habían advertido particularmente que nunca se apartase de su compañero, esto es, que no probara métodos personales ni secretos de fomentar sus mutuos cometidos. Eva había llevado a cabo escrupulosamente estas instrucciones durante más de cien años, y no se le ocurrió que las conversaciones cada vez más privadas y confidenciales que mantenía con un líder nodita llamado Serapatatia entrañaran peligro alguno. Todo el asunto se desarrolló de forma tan paulatina y natural que la tomó desprevenida.
\vs p075 2:5 Los habitantes del Jardín habían estado en contacto con los noditas desde los primeros días de Edén. De estos descendientes mixtos de los miembros rebeldes de la comitiva de Caligastia, habían recibido una valiosa ayuda y cooperación y, a través de ellos, el régimen edénico iba ahora a enfrentarse a su completa perdición y a su definitiva extinción.
\usection{3. LA TENTACIÓN DE EVA}
\vs p075 3:1 Adán acababa de terminar sus primeros cien años en la tierra cuando Serapatatia, a la muerte de su padre, asumió el liderazgo de la confederación occidental o siria de las tribus noditas. Serapatatia, un hombre de tez morena, era un brillante descendiente del antiguo jefe de la comisión de la salud de Dalamatia que se había casado con una de las privilegiadas mentes femeninas de la raza azul de aquellos remotos tiempos. A través de las eras, sus ancestros habían mantenido su autoridad y ejercido una gran influencia sobre las tribus noditas del oeste.
\vs p075 3:2 Serapatatia había hecho varias visitas al Jardín y estaba profundamente impresionado por la legitimidad de la causa de Adán. Y, al poco tiempo de asumir el mando de los noditas sirios, anunció su intención de adherirse a la misión de Adán y de Eva en el Jardín. La mayoría de su pueblo se unió a él en este proyecto, y Adán se sintió muy animado al saber que la más poderosa e inteligente de todas las tribus vecinas se había inclinado casi masivamente a apoyar su plan de mejora para el mundo; fue sin duda muy alentador. Y poco después de este gran acontecimiento, Adán y Eva recibieron como invitados, en su propio hogar, a Serapatatia y su nuevo equipo.
\vs p075 3:3 \pc Serapatatia se convirtió en uno de los más capaces y eficientes lugartenientes de Adán. Era totalmente honesto y sincero en todas sus actividades; nunca fue consciente, ni incluso más tarde, de que el astuto Caligastia lo estaba usando incidentalmente como un instrumento para sus fines.
\vs p075 3:4 \pc Pronto, Serapatatia se convirtió en el presidente adjunto de la comisión edénica sobre las relaciones tribales, y se establecieron numerosos planes para impulsar más decididamente la labor de ganarse a las tribus remotas para la causa del Jardín.
\vs p075 3:5 Serapatatia mantuvo muchas reuniones con Adán y Eva ---particularmente con Eva--- y conversaron sobre muchos planes para mejorar los procedimientos a seguir. Un día, durante una charla con Eva, a Serapatatia se le ocurrió que sería de gran utilidad, mientras se aguardaba el reclutamiento de un gran número de miembros de la raza violeta, poder hacer algo de forma inmediata que favoreciera el avance de las expectantes y necesitadas tribus. Serapatatia sostenía que, si los noditas, siendo la raza más progresiva y colaborativa, pudieran contar con un líder nacido de ellos y originado parcialmente en la estirpe violeta, esto constituiría un poderoso vínculo que uniría a estos pueblos más estrechamente al Jardín. Y todo ello se consideró con seriedad y honestidad pensando que sería beneficioso para el mundo, ya que este hijo, que sería criado y educado en el Jardín, ejercería una beneficiosa influencia sobre el pueblo de su padre.
\vs p075 3:6 Cabe destacar una vez más que Serapatatia era totalmente honrado y sincero en todas sus propuestas. Nunca llegó a sospechar que estaba haciéndole el juego a Caligastia y Daligastia. Serapatatia era enteramente leal al proyecto de formar una gran reserva de la raza violeta antes de intentar la mejora a escala mundial de los confundidos pueblos de Urantia, pero esto requeriría centenares de años para culminarse, y él era impaciente; quería ver resultados inmediatos ---algunos al menos durante su propia vida---. Le dijo claramente a Eva que Adán se desanimaba a menudo por lo poco que se había logrado para elevar el nivel del mundo.
\vs p075 3:7 \pc Estos planes fueron tomando forma clandestinamente durante más de cinco años hasta llegar al punto de que Eva consintió en tener una reunión secreta con Cano, la mente más brillante y el más activo líder de la colonia cercana de amigables noditas. Cano tenía al régimen adánico en buena consideración; de hecho, era el honesto líder espiritual de aquellos noditas vecinos que estaban a favor de relaciones de amistad con el Jardín.
\vs p075 3:8 El aciago encuentro se produjo durante las horas crepusculares de una tarde de otoño, no lejos de la casa de Adán. Eva no había conocido antes al hermoso y entusiasta Cano ---que era un magnífico ejemplar de ser humano dotado del físico superior y del extraordinario intelecto de sus remotos progenitores de la comitiva del príncipe---. Y Cano también creía plenamente en la legitimidad del proyecto de Serapatatia. (La poligamia se solía practicar fuera del Jardín.)
\vs p075 3:9 Influida por la adulación, el entusiasmo y una gran persuasión personal, Eva accedió en ese momento a embarcarse en aquella aventura, tantas veces sopesada, de añadir su propio y pequeño proyecto de salvación del mundo al plan divino más grande y de mucho mayor alcance. Y antes de darse cuenta de lo que estaba sucediendo, se había dado el fatídico paso. Todo estaba hecho.
\usection{4. RECONOCIMIENTO DE LA TRANSGRESIÓN}
\vs p075 4:1 La vida celestial del planeta estaba en estado de agitación. Adán notó que algo iba mal y pidió a Eva que se fuese con él aparte a algún lado del Jardín. Y, en ese momento, por vez primera, Adán oyó toda la historia del proyecto que durante tanto tiempo se había venido gestando para acelerar la mejora del mundo y que operaría simultáneamente en dos direcciones: el seguimiento del plan divino en conjunción con la realización de la iniciativa de Serapatatia.
\vs p075 4:2 Y mientras que el hijo y la hija materiales conversaban en el Jardín, bajo la luz de la luna, “la voz del Jardín” les reprochó la desobediencia. Y aquella voz no era otra que la mía propia que anunciaba a la pareja edénica que había transgredido el pacto del Jardín, que había desobedecido las instrucciones de los melquisedecs; que había incumplido su juramento de lealtad hacia el soberano del universo.
\vs p075 4:3 Eva había accedido a participar en la práctica del bien y del mal. El bien es el cumplimiento de los planes divinos; el pecado es una transgresión deliberada de la voluntad divina; el mal es la falta de conformación a los planes y procedimientos predispuestos, lo que resulta en la falta de armonía en el universo y en la confusión planetaria.
\vs p075 4:4 Cada vez que la pareja del Jardín había comido del fruto del árbol de la vida, el arcángel guardián les había advertido que no cedieran al consejo de Caligastia de combinar el bien y el mal. Se les había prevenido de la siguiente manera: “El día que entremezcléis bien y mal, ciertamente os convertiréis en mortales del mundo; moriréis con toda seguridad”.
\vs p075 4:5 Con motivo de su fatídica reunión secreta, Eva le había contado a Cano esta reiterada advertencia que se le había hecho; pero Cano, desconociendo la importancia o el significado de este apercibimiento, le había asegurado que los hombres y las mujeres con buenos motivos y legítimas intenciones no podían cometer ningún mal; que ella, sin duda, no moriría, sino que más bien viviría de nuevo en la persona de su vástago, que crecería para bendecir y estabilizar el mundo.
\vs p075 4:6 A pesar de que se concibió y llevó a cabo este proyecto para modificar el plan divino con total sinceridad y con solo las más elevadas intenciones por el bienestar del mundo, aquello constituyó un acto de maldad porque representaba el camino equivocado para lograr unos fines justos, porque se apartó del camino correcto, del plan divino.
\vs p075 4:7 En verdad, Eva había encontrado a Cano agradable a sus ojos, y reconoció todo lo que prometía su seductor a modo de “un nuevo y creciente conocimiento de las cuestiones humanas y de un más vivo entendimiento de la naturaleza humana, que complementaba su comprensión de la naturaleza adánica”.
\vs p075 4:8 Como era mi deber dadas aquellas dolorosas circunstancias, hablé esa misma noche en el Jardín con el padre y la madre de la raza violeta. Escuché el relato completo de todo lo que había llevado a la transgresión de la madre Eva y les asesoré y aconsejé en cuanto a su situación inmediata. Siguieron algunos de estos consejos; ignoraron otros. Esta conversación aparece en vuestros escritos como “el Señor Dios llamó a Adán y Eva en el Huerto y les preguntó: ‘¿Dónde estáis?'”. Era habitual que las generaciones venideras atribuyeran todo lo insólito y extraordinario, ya fuese natural o espiritual, directamente a la intervención personal de los Dioses.
\usection{5. CONSECUENCIAS DE LA TRANSGRESIÓN}
\vs p075 5:1 El desencanto de Eva fue realmente penoso. Adán se dio cuenta de la terrible situación y, aunque desconsolado y abatido, no albergó por su errada compañera sino sentimientos de piedad y conmiseración.
\vs p075 5:2 Al día siguiente del tropiezo de Eva, desesperado al comprender el fallo cometido, Adán buscó a Laotta, la brillante nodita que dirigía las escuelas occidentales del Jardín y, con premeditación, cometió la insensatez de Eva. Pero no tengáis una idea equivocada; Adán no fue seducido; sabía exactamente lo que hacía; decidió deliberadamente compartir el mismo destino que Eva. Amaba a su compañera con un afecto sobrehumano, y la idea de una posible velada sin ella en Urantia era algo más de lo que podía soportar.
\vs p075 5:3 Cuando se enteraron de lo que le había sucedido a Eva, los enfurecidos habitantes del Jardín se volvieron incontrolables; declararon la guerra al asentamiento nodita vecino. Salieron en tropel por las puertas de Edén y arremetieron contra la población desprevenida, exterminándola ---no se salvó ni un solo hombre, mujer o niño---. Y también pereció Cano, el padre del aún no nato Caín.
\vs p075 5:4 Al darse cuenta de lo que había sucedido, Serapatatia, abrumado por la consternación y fuera de sí por el temor y los remordimientos, se ahogó a sí mismo al día siguiente en el gran río.
\vs p075 5:5 Los hijos de Adán trataron de consolar a su afligida madre, mientras que su padre vagó en soledad durante treinta días. Al cabo de los cuales, se impuso el sentido común y Adán regresó a su casa y empezó a trazar las líneas de acción en cuanto a la planificación del futuro.
\vs p075 5:6 Muy a menudo, los hijos, inocentes, sufren las consecuencias de las insensateces de sus errados padres. Los honestos y nobles hijos e hijas de Adán y de Eva estaban desbordados; sentían un inexplicable pesar ante aquella inconcebible desgracia que, de manera tan repentina e implacable, les había sobrevenido. Cincuenta años tardarían los hijos mayores en recuperarse del dolor y de la tristeza de aquellos trágicos días, especialmente de la gran ansiedad que experimentaron durante esos días en los que su padre estuvo ausente del hogar, sin que su atormentada madre supiese de su paradero o suerte.
\vs p075 5:7 Y Eva sintió esos mismos treinta días como si fuesen largos años de dolor y sufrimiento. Esta noble alma jamás se recuperaría del todo de aquel insoportable período de aflicción mental y de pesar espiritual. Según sus propios recuerdos, ninguna de las privaciones y dificultades materiales por las que pasaría más adelante podrían llegar a compararse en lo más mínimo con aquellos días terribles y atroces noches de soledad e insoportable incertidumbre. Se enteró del impulsivo acto de Serapatatia y no sabía si su compañero se había causado a sí mismo la muerte o se le había apartado del mundo en castigo por su paso errado. Y, cuando Adán regresó, Eva sintió una sensación gratificante de gozo y gratitud, que jamás se borraría de ella durante su larga y difícil vida en común de duro trabajo y servicio.
\vs p075 5:8 \pc El tiempo transcurría, pero Adán no estuvo seguro de la naturaleza de su agravio hasta transcurridos setenta días desde la transgresión de Eva, cuando los síndicos melquisedecs regresaron a Urantia y asumieron la jurisdicción sobre los asuntos del mundo. Y entonces supo que habían fracasado.
\vs p075 5:9 \pc Pero aún se estaban gestando más problemas: la noticia de la aniquilación del asentamiento nodita aledaño a Edén no tardó en llegar a las tribus nativas de Serapatatia situadas al norte y, en ese momento, se estaba congregando un gran ejército para marchar contra el Jardín. Y este fue el comienzo de una larga y encarnizada guerra entre los adanitas y noditas; estas hostilidades continuaron muchos años después de que Adán y sus seguidores emigraran al segundo jardín en el valle del Éufrates. Hubo una intensa y persistente “enemistad entre aquel hombre y la mujer, entre la simiente de él y la simiente de ella”.
\usection{6. ADÁN Y EVA ABANDONAN EL JARDÍN}
\vs p075 6:1 Cuando Adán supo que los noditas se dirigían hacia ellos, solicitó la orientación de los melquisedecs, pero estos se negaron a aconsejarle; solo le dijeron que hiciera lo que considerara mejor, prometiéndole cooperar de forma amistosa, en todo lo posible, en cualquier curso de acción que tomara. A los melquisedecs se les había prohibido interferir en los planes personales de Adán y Eva.
\vs p075 6:2 Adán era consciente de que él y Eva habían fracasado, como la presencia de los síndicos melquisedecs así se lo indicaba; no obstante, él aún no sabía nada respecto a su situación personal y a la suerte que correría. Durante toda una noche, Adán mantuvo una reunión con mil doscientos partidarios leales que se comprometieron a seguir a su líder y, al día siguiente, al mediodía, estos peregrinos salieron de Edén en busca de un nuevo hogar. Adán no era partidario de la guerra, así que prefirió dejar a los noditas el primer jardín, sin oponer resistencia.
\vs p075 6:3 Al tercer día de salir del Jardín, la caravana edénica se detuvo por la llegada de los transportes seráficos procedentes de Jerusem. Y, por primera vez, a Adán y a Eva se les informó del destino que deparaba a sus hijos. Mientras que estos serafines aguardaban preparados, a los hijos que habían llegado a la edad de elegir (los veinte años) se les dio la posibilidad de permanecer en Urantia con sus padres o pasar bajo la tutela de los altísimos de Norlatiadec. Dos tercios de ellos optaron por ir a Edentia; casi un tercio prefirió quedarse en Urantia con sus padres. Se llevaron a Edentia a los que aún no tenían edad de elegir. Nadie que pudiese haber contemplado la dolorosa separación entre el hijo y la hija materiales y su progenie hubiese dejado de comprender que el camino de los transgresores es duro. Estos vástagos de Adán y de Eva están ahora en Edentia; no sabemos qué planes hay para ellos.
\vs p075 6:4 Embargada por una gran tristeza, la caravana se dispuso a continuar su viaje. ¡No podía haber nada más lamentable! ¡Haber llegado a un mundo con tan altas expectativas, haber sido recibidos con tan buenos auspicios y tener que salir después de Edén en desgracia, perdiendo además más de tres cuartos de sus hijos incluso antes de encontrar un nuevo lugar donde vivir!
\usection{7. ADÁN Y EVA DEGRADADOS A LA CONDICIÓN MORTAL}
\vs p075 7:1 Mientras que la caravana edénica estaba detenida, se informó a Adán y a Eva de la naturaleza de sus transgresiones y se les comunicó cuál sería su destino. Gabriel apareció para dictar sentencia. Y este fue el veredicto: Se declaraba al adán y eva planetarios en transgresión de su deber; habían violado el pacto de confianza depositado en ellos como gobernantes de este mundo habitado.
\vs p075 7:2 Aunque abatidos por el sentimiento de culpabilidad, Adán y Eva se sintieron muy aliviados al conocer que sus jueces de Lugar de Salvación los habían absuelto de todos los cargos de “desacato al gobierno del universo”. No se les había hallado culpables de rebelión.
\vs p075 7:3 A la pareja edénica se les informó que ellos mismos se habían degradado a la condición de los mortales del mundo; que en lo sucesivo debían comportarse como hombre y mujer de Urantia, mirando al futuro de las razas del planeta como el suyo propio.
\vs p075 7:4 Mucho antes de que Adán y Eva partieran de Jerusem, sus instructores les habían explicado a fondo las consecuencias de cualquier desviación fundamental de los planes divinos. Yo, personal y reiteradamente, les había advertido, tanto antes como después de que llegaran a Urantia, de que su descenso al estado mortal sería el indudable resultado, el castigo cierto, que conllevaría, de forma indefectible, el incumplimiento de su misión planetaria. No obstante, es esencial el entendimiento de la condición inmortal del orden material de la filiación para comprender con claridad las consecuencias que se derivaron de su transgresión.
\vs p075 7:5 \li{1.}Al igual que sus semejantes en Jerusem, Adán y Eva mantenían su estatus de inmortales mediante la conexión intelectual con la vía circulatoria de la gravedad\hyp{}mente del Espíritu. Cuando este sostén vital se rompe por disyunción mental, entonces, cualquiera que sea el nivel espiritual de la criatura, la condición de inmortalidad se pierde. La consecuencia inevitable del fallo intelectual de Adán y de Eva fue el estatus mortal seguido de la disolución física.
\vs p075 7:6 \li{2.}El hijo y la hija materiales de Urantia, teniendo sus personas semejando la de los seres humanos de este mundo, dependían además del mantenimiento de un sistema de circulación doble, uno de ellos derivado de su naturaleza física, el otro, de la supraenergía almacenada en el fruto del árbol de la vida. El arcángel custodio siempre había advertido a Adán y a Eva de que el incumplimiento de su responsabilidad culminaría en la degradación de su estatus, y, después de su transgresión, se les negó el acceso a esta fuente de energía.
\vs p075 7:7 \pc En verdad Caligastia consiguió que Adán y Eva cayeran en su trampa, pero no logró su propósito de llevarlos a rebelarse abiertamente contra el gobierno del universo. Lo que habían hecho era efectivamente perverso, pero nunca se les encontró culpables de despreciar la verdad, como tampoco de sumarse deliberadamente a la revuelta contra el justo gobierno del Padre Universal y de su hijo creador.
\usection{8. LA DENOMINADA “CAÍDA DEL HOMBRE”}
\vs p075 8:1 En efecto, Adán y Eva cayeron del alto estatus de hijos materiales al humilde estado del hombre mortal. Pero esto no fue la caída del hombre. La raza humana mejoró a pesar de las consecuencias inmediatas que trajo consigo la transgresión adánica. Aunque el plan divino diseñado para conceder la raza violeta a los pueblos de Urantia fracasara, las razas mortales se beneficiaron considerablemente de la limitada contribución que Adán y sus descendientes aportaron.
\vs p075 8:2 No ha existido ninguna “caída del hombre”. La historia de la raza humana es una historia de evolución progresiva, y el ministerio adánico mejoró notablemente a los pueblos del mundo respecto a su anterior condición biológica. Las estirpes mejor dotadas de Urantia contienen ahora elementos hereditarios procedentes de cuatro fuentes diferentes: la andonita, la sangik, la nodita y la adánica.
\vs p075 8:3 No se debe considerar a Adán como la causa de ninguna maldición que haya sobrevenido a la raza humana. Aunque realmente fracasó en llevar el plan divino a buen término, aunque de hecho transgredió su pacto con la Deidad, aunque él y su compañera fueron en efecto degradados en su estatus creatural, a pesar de todo, su aportación a la raza humana hizo mucho para avanzar la civilización en Urantia.
\vs p075 8:4 \pc Al tratar de valorar los resultados de la misión adánica en vuestro mundo, es de justicia reconocer la situación en la que se encontraba el planeta. Adán afrontaba una tarea prácticamente imposible de realizar cuando, con su bella compañera, se le transportó desde Jerusem a este oscuro y confundido planeta. Pero si hubiesen seguido el consejo de los melquisedecs y de sus colaboradores, y \bibemph{hubiesen sido más pacientes,} finalmente habrían alcanzado el éxito. Si bien, Eva escuchó la insidiosa propaganda a favor de la libertad personal y de la actuación planetaria. Se le indujo a experimentar con el plasma vital del orden material de filiación. Eva permitió que esta heredad de vida se llegara a mezclar prematuramente con el entonces ya mezclado orden, diseño de los portadores de vida, que previamente se había combinado con el de los seres reproductores en otro tiempo adscritos a la comitiva del príncipe planetario.
\vs p075 8:5 Nunca, en todo vuestro ascenso al Paraíso, obtendréis nada si por impaciencia intentáis eludir el plan divino establecido mediante atajos, inventivas personales u otros artificios para mejorar vuestro avance en el camino de la perfección, hacia la perfección y para la perfección eterna.
\vs p075 8:6 \pc En definitiva, probablemente jamás haya habido en ningún planeta de todo Nebadón un fallo de sabiduría más descorazonador. Pero no es extraño que estos tropiezos se den en relación a los asuntos de los universos evolutivos. Formamos parte de una gigantesca creación, y no es de extrañar que no todo obre en perfección; nuestro universo no fue creado en perfección. La perfección es nuestra meta eterna, no nuestro origen.
\vs p075 8:7 Si este fuese un universo mecanicista, si la Primera Gran Fuente y Centro fuese solamente una fuerza y no una persona también, si toda la creación fuese una inmensa acumulación de materia física dominada por precisas leyes que se caracterizaran por acciones energéticas invariables, entonces podría conseguirse la perfección, a pesar incluso de la incompletitud del universo. No habría desacuerdos; no habría fricciones. Pero en nuestro universo evolutivo de perfección e imperfección relativas, nos alegramos de que los desacuerdos y los equívocos sean posibles, pues así se evidencia el hecho y la acción en el universo del ser personal. Y si nuestra creación conlleva una existencia dominada por el ser personal, entonces podéis estar seguros de las posibilidades de la supervivencia, avance y realización de dicho ser personal; podemos tener confianza en el desarrollo, la experiencia y la aventura del ser personal. ¡Qué universo más glorioso, por cuanto que es personal y progresivo, no meramente mecanicista ni incluso pasivamente perfecto!
\vsetoff
\vs p075 8:8 [Exposición de Solonia, la “voz seráfica el Jardín”.]
