\upaper{122}{Nacimiento e infancia de Jesús}
\author{Comisión de seres intermedios}
\vs p122 0:1 Sería prácticamente imposible dar una explicación concluyente sobre las muchas razones que llevaron a elegir Palestina como el territorio en el que Miguel se daría de gracia y, en especial, por qué se escogió exactamente a la familia de José y María como el entorno inmediato para la aparición de este Hijo de Dios en Urantia.
\vs p122 0:2 Tras estudiar un informe especial realizado por los melquisedecs, con el asesoramiento de Gabriel, sobre las condiciones de los mundos aislados, Miguel, optó finalmente por Urantia, planeta en el que llevaría a cabo su último ministerio de gracia. Con posterioridad a esta decisión, Gabriel hizo una visita personal a Urantia y, como resultado de su estudio de los grupos humanos y de su examen de los rasgos espirituales, intelectuales, raciales y geográficos del mundo y de sus pueblos, concluyó que los hebreos poseían comparativamente algunas ventajas que justificaban su elección como la raza en la que Miguel efectuaría dicho ministerio. Al aprobar Miguel esta decisión, Gabriel designó y envió a Urantia a la Comisión de los Doce sobre la Familia ---seleccionada entre los órdenes de seres personales de mayor rango del universo--- a la que se le confió la tarea de investigar la vida de la familia judía. Cuando esta comisión acabó su labor, Gabriel se encontraba en Urantia y recibió el informe designando a tres posibles uniones, que en la opinión de la comisión, constituían familias igualmente propicias para el ministerio de gracia de Miguel y su proyectada encarnación.
\vs p122 0:3 De las tres parejas designadas, Gabriel eligió personalmente a José y a María y luego se apareció a María en persona para transmitirle la buena nueva de que había sido escogida para ser la madre terrenal del niño de gracia.
\usection{1. JOSÉ Y MARÍA}
\vs p122 1:1 José, el padre humano de Jesús (Josué ben José), fue un hebreo entre los hebreos, a pesar de poseer numerosas estirpes no judías que se habían añadido a su árbol genealógico ocasionalmente por medio de las líneas femeninas de sus progenitores. Los ancestros del padre de Jesús se remontaban a los días de Abraham y, mediante este venerable patriarca, a linajes más tempranos que llegaban hasta los sumerios y los noditas y, a través de las tribus meridionales del antiguo hombre azul, hasta Andón y Fonta. David y Salomón no se encontraban en la línea directa de los ancestros de José, ni el linaje de José se remontaba tampoco directamente a Adán. Los antepasados inmediatos de José eran artesanos: constructores, carpinteros, albañiles y herreros. El mismo José fue carpintero y después contratista. Su familia pertenecía a una larga e ilustre línea de la nobleza de la gente común, que de vez en cuando había adquirido prominencia por la aparición de personas extraordinarias que se habían distinguido con respecto a la evolución de la religión en Urantia.
\vs p122 1:2 María, la madre terrenal de Jesús, descendía de una larga línea de excepcionales ancestros en los que se incluían muchas de las mujeres más destacadas de la historia racial de Urantia. Aunque María era una mujer típica de su tiempo y generación, con un temperamento muy normal, contaba entre sus antecesoras a mujeres bien conocidas como Annón, Tamar, Rut, Betsabé, Ansie, Cloa, Eva, Enta y Ratta. Ninguna otra mujer judía de aquellos días poseía un linaje más ilustre de progenitores en común o que se remontara a unos orígenes más favorables. Entre los antecesores de María, al igual que entre los de José, predominaban personas fuertes pero corrientes, aunque de vez en cuando, en el curso de la civilización y en la evolución progresiva de la religión, surgieran numerosas figuras destacadas. Desde el punto de vista racial, difícilmente se podría considerar a María como judía. Era judía en cuanto a su cultura y creencias, pero en cuanto a su bagaje hereditario era más bien una mezcla de las razas siria, hitita, fenicia, griega y egipcia; su herencia racial era más extensa que la de José.
\vs p122 1:3 De todas las parejas que vivían en Palestina en la época prevista para el ministerio de gracia de Miguel, José y María combinaban de la manera más perfecta unos extensos vínculos raciales con unas dotes personales superiores a la media. El plan de Miguel era aparecer en la tierra como un hombre \bibemph{ordinario,} para que la gente común pudiera entenderlo y recibirlo; es por ello por lo que Gabriel eligió a unas personas como José y María para ser los padres de Miguel en su ministerio.
\usection{2. GABRIEL SE APARECE A ELISABET}
\vs p122 2:1 En realidad fue Juan el bautista quien comenzó la labor de vida de Jesús en Urantia. Zacarías, el padre de Juan, pertenecía al clero judío, mientras que su madre, Elisabet, era miembro de la rama más próspera del mismo gran grupo familiar al que también pertenecía María, la madre de Jesús. Zacarías y Elisabet, aunque llevaban muchos años casados, no tenían hijos.
\vs p122 2:2 \pc A finales del mes de junio del año 8 a. C., unos tres meses después de la boda de José y María, Gabriel se apareció un mediodía a Elisabet, tal como más tarde se presentaría ante María, y le dijo:
\vs p122 2:3 “Mientras tu marido, Zacarías, oficia ante el altar en Jerusalén, y mientras el pueblo reunido ora por la llegada de un libertador, yo, Gabriel, he venido para anunciarte que pronto darás a luz a un hijo que será el precursor de este maestro divino y a quien pondrás por nombre Juan. Crecerá dedicado al Señor tu Dios y, cuando llegue a los años de su madurez, alegrará tu corazón porque convertirá muchas almas a Dios, y proclamará también la venida del sanador de almas de tu pueblo y libertador espiritual de toda la humanidad. Tu parienta María será la madre de este hijo de la promesa, y yo también me apareceré a ella”.
\vs p122 2:4 Esta visión atemorizó mucho a Elisabet. Tras la partida de Gabriel, reflexionó sobre aquello, sopesando largamente las palabras del majestuoso visitante, pero no le refirió esta revelación a nadie salvo a su marido hasta que conversó después con María a comienzos de febrero del año siguiente.
\vs p122 2:5 \pc Sin embargo, Elisabet, durante cinco meses, ocultó su secreto incluso a su esposo. Cuando le desveló la historia de la visita de Gabriel, Zacarías fue muy escéptico y durante semanas puso en duda todo lo referido; solo accedió a creer sin mucho entusiasmo en esta visita de Gabriel a su esposa cuando ya no pudo poner en cuestión por más tiempo que se había quedado embarazada. Ante la futura maternidad de Elisabet, Zacarías se sintió muy turbado, sin dudar no obstante, a pesar de la edad avanzada en la que él mismo se encontraba, de la honradez de su esposa. No fue hasta unas seis semanas antes del nacimiento de Juan cuando Zacarías, como resultado de un impactante sueño, quedó totalmente convencido de que Elisabet iba a ser madre de un hijo de destino, de aquel que prepararía el camino para la llegada del Mesías.
\vs p122 2:6 Gabriel se apareció a María sobre mediados de noviembre del año 8 a. C., mientras trabajaba en su casa de Nazaret. Más tarde, cuando no le cupo duda de que iba a ser madre, convenció a José para que la dejara viajar a la ciudad de Judá, situada en las colinas, a más de seis kilómetros al oeste de Jerusalén, para visitar a Elisabet. Gabriel había comunicado a cada una de estas futuras madres de su respectiva aparición a la otra. Naturalmente, estaban deseosas de encontrarse, comparar sus vivencias y hablar del probable futuro de sus hijos. María permaneció durante tres semanas con su prima lejana. Elisabet hizo mucho por fortalecer la fe de María en la visión de Gabriel, así que esta volvió a su casa con una mayor dedicación a su futuro cometido de ser la madre del hijo de destino a quien muy pronto presentaría al mundo como un bebé indefenso, como cualquier otro pequeño ordinario y normal del mundo.
\vs p122 2:7 \pc Juan nació en la ciudad de Judá el 25 de marzo del año 7 a. C. Zacarías y Elisabet se regocijaron grandemente con la llegada de un hijo, como Gabriel había prometido y, cuando al octavo día presentaron al niño para la circuncisión, le pusieron por nombre Juan, tal como se les había previamente indicado que hicieran. Un sobrino de Zacarías había partido ya para Nazaret, llevando el mensaje de Elisabet a María de que le había nacido un hijo y de que su nombre sería Juan.
\vs p122 2:8 Desde su más tierna infancia, sus padres sensatamente le habían inculcado la idea de que cuando creciera se convertiría en un líder espiritual y en un maestro religioso. Y el corazón de Juan siempre fue un terreno abonado para la siembra de esta sugerente semilla. Incluso de niño se le encontraba con frecuencia en el templo durante los periodos en los que su padre oficiaba los servicios, quedándose profundamente impresionado ante la trascendencia de todo lo que veía.
\usection{3. LA ANUNCIACIÓN DE GABRIEL A MARÍA}
\vs p122 3:1 Una tarde, al ponerse el sol, antes de que José hubiese regresado a casa, Gabriel se apareció a María junto a una mesa de piedra de baja altura y, una vez que ella recobró la calma, le dijo: “Vengo por mandato de mi Maestro, a quien tú amarás y criarás. A ti, María, te traigo buenas nuevas al anunciarte que tu concepción está ordenada por el cielo y que, a su debido tiempo, te convertirás en la madre de un hijo; lo llamarás Joshua, y él inaugurará el reino del cielo en la tierra y entre los hombres. No hables de esto a nadie excepto a José y a Elisabet, tu parienta, a quien también me he aparecido, y quien pronto dará también a luz a un hijo cuyo nombre será Juan, y que preparará el camino para el mensaje de liberación que tu hijo proclamará a los hombres con gran fuerza y profunda convicción. Y no dudes de mi palabra, María, pues este hogar ha sido elegido como la morada humana del hijo de destino. Mi bendición reposa sobre ti, el poder de los Altísimos te fortalecerá y el Señor de toda la tierra te cubrirá con su sombra.
\vs p122 3:2 \pc Durante muchas semanas, hasta que no estuvo segura de su embarazo, María guardó secretamente en su corazón dicha visitación sin atreverse a desvelar estos extraordinarios acontecimientos a José. Cuando él oyó todo lo sucedido, aunque tenía una gran confianza en María, se turbó mucho y durante muchas noches no pudo conciliar el sueño. Al principio dudaba sobre la visitación de Gabriel; luego, cuando estaba casi convencido de que María había realmente oído la voz y había contemplado la forma del mensajero divino, su mente se debatió recapacitando cómo podían ocurrir estas cosas. ¿Cómo podía un hijo de destino divino ser progenie de unos seres humanos? José no podía reconciliar estas contradictorias ideas hasta que, tras varias semanas de reflexión, tanto él como María, a pesar de la dificultad de los judíos para concebir que el libertador esperado fuese de naturaleza divina, llegaron a la convicción de que se les había elegido para ser los padres del Mesías. Una vez que alcanzaron esta importante conclusión, María se apresuró a partir para conversar con Elisabet.
\vs p122 3:3 A su regreso, María fue a ver a sus padres, Joaquín y Ana. Sus dos hermanos y sus dos hermanas, al igual que sus padres, vieron siempre con escepticismo la misión divina de Jesús, aunque, por supuesto, en aquel momento, no sabían nada de la visitación de Gabriel. Si bien, María sí le confió a su hermana Salomé su convencimiento de que su hijo estaba destinado a ser un gran maestro.
\vs p122 3:4 \pc La anunciación de Gabriel a María tuvo lugar al día siguiente de la concepción de Jesús y constituyó el único hecho de naturaleza sobrenatural habido en relación al embarazo de María y a su alumbramiento del hijo de la promesa.
\usection{4. EL SUEÑO DE JOSÉ}
\vs p122 4:1 José no llegó a resignarse a la idea de que María iba a ser madre de un hijo extraordinario hasta después de tener un sueño del que quedó muy impresionado. En él se le aparecía un radiante mensajero celestial que, entre otras cosas, le dijo: “José, aparezco ante ti por mandato de Aquel que ahora reina en las alturas con el encargo de instruirte sobre el hijo que María va a tener, y que llegará a ser una gran luz en el mundo. En él estará la vida y su vida será la luz de la humanidad. Vendrá primero a su propia gente, pero apenas lo recibirán; mas a todos los que lo reciban les revelará que son los hijos de Dios”. Tras esto, José no volvió a dudar jamás del relato de María sobre la venida de Gabriel ni de la promesa de que el niño por nacer llegaría a ser un mensajero divino para el mundo.
\vs p122 4:2 \pc En todas estas visitaciones nada se dijo de la casa de David. Nada se dio a entender que Jesús llegaría a ser “libertador de los judíos”, ni incluso que sería el tan esperado Mesías. Jesús no era el Mesías que los judíos habían aguardado, sino el \bibemph{libertador del mundo}. Su misión era para con todas las razas y para todos los pueblos, no para un grupo en particular.
\vs p122 4:3 José no era del linaje del rey David. María tenía más descendencia davídica que José. Es cierto que José tuvo que ir a Belén, la ciudad de David, para empadronarse en el censo romano, pero esto se debió a que seis generaciones antes, el ascendente paterno de José de esa generación, al ser huérfano, fue adoptado por un tal Zadoc, descendiente directo de David; de ahí que también se considerara a José como perteneciente a la “casa de David”.
\vs p122 4:4 La mayoría de las llamadas profecías mesiánicas del Antiguo Testamento se aplicaron a Jesús mucho tiempo después de que él hubiese vivido su vida en la tierra. Durante siglos, los profetas hebreos habían anunciado la venida de un libertador y, durante generaciones sucesivas, se habían interpretado esas promesas en referencia a un nuevo gobernante judío que se sentaría en el trono de David y que, usando los supuestos métodos milagrosos de Moisés, procedería a establecer a los judíos de Palestina como una nación poderosa, libre de toda dominación extranjera. Igualmente, muchos pasajes figurados que se hallaban por todas las escrituras hebreas se aplicaron de forma errónea con posterioridad a la labor de vida de Jesús. Muchos de los dichos del Antiguo Testamento se alteraron para hacerlos coincidir con algunos episodios de la vida del Maestro en la tierra. Jesús mismo negó en alguna ocasión públicamente cualquier relación con la casa real de David. Incluso el pasaje “una joven dará a luz un hijo”, se modificó para que dijese “una virgen dará a luz un hijo”. Esto también ocurrió con las muchas genealogías tanto de José como de María compuestas con posterioridad a la andadura de Miguel en la tierra. Muchos de estos linajes contienen bastante de los ancestros del Maestro, pero en general no son genuinos y no se puede confiar en su fiabilidad. Con demasiada frecuencia, los primeros seguidores de Jesús cedieron a la tentación de hacer que todos los antiguos pronunciamientos proféticos parecieran tener cumplimiento en la vida de su Señor y Maestro.
\usection{5. LOS PADRES TERRENALES DE JESÚS}
\vs p122 5:1 José era hombre de buenas maneras, extremadamente concienzudo y, en todos los sentidos, fiel a las convenciones y prácticas religiosas de su pueblo. Hablaba poco pero pensaba mucho. José se sentía muy apenado por la triste situación del pueblo judío. De joven, entre sus ocho hermanos y hermanas, había tenido un carácter alegre, pero en los primeros años de su vida de casado (durante la niñez de Jesús) había sufrido algunos periodos de un cierto desánimo espiritual. Si bien, esta disposición de ánimo había experimentado un gran alivio justo antes de su muerte prematura, una vez que las condiciones económicas de su familia habían mejorado gracias a su ascenso desde el rango de carpintero hasta la posición de próspero contratista.
\vs p122 5:2 El temperamento de María era muy distinto al de su marido. Por lo general, era alegre, rara vez estaba abatida y poseía una actitud siempre animosa. María se permitía frecuentemente expresar con libertad sus sentimientos de emoción y nunca se le vio afligida hasta después de la muerte súbita de José. Y apenas se había recuperado de esta conmoción, tuvo que afrontar la ansiedad y los cuestionamientos suscitados por la extraordinaria andadura de su hijo mayor, que tan rápidamente se estaba desplegando ante su mirada atónita. Pero María, durante toda esta insólita experiencia, permaneció en calma, valerosa y muy prudente en sus relaciones con su singular y poco comprendido primogénito y con los hermanos y hermanas que le sobrevivieron.
\vs p122 5:3 De su padre, Jesús tenía bastante de su inusual dulzura y de su magnífica y compasiva comprensión de la naturaleza humana; de su madre, había heredado sus dotes de gran maestro y su formidable capacidad para indignarse justamente. En sus reacciones emocionales al entorno de su vida adulta, Jesús era en algún momento como su padre, meditativo y reverente; a veces se caracterizaba por una tristeza aparente; pero, con mayor frecuencia, se conducía de la misma manera optimista y resuelta de su madre. En conjunto, el temperamento de María tendió a prevalecer durante la andadura del hijo divino conforme crecía y daba pasos importantes hacia la edad adulta. En algunos detalles, Jesús era una mezcla de los rasgos de sus padres; en otros, poseía los de uno de ellos a diferencia de los del otro.
\vs p122 5:4 De José, Jesús obtuvo su estricta formación en los usos de los ceremoniales judíos y su excepcional conocimiento de las escrituras hebreas; de María, adquirió una perspectiva más amplia de la vida religiosa y un concepto más liberal de la libertad espiritual personal.
\vs p122 5:5 Las familias de José y de María eran bien educadas para su tiempo. José y María poseían una educación muy por encima de la media de su época y situación social. Él era un pensador; ella planificaba; estaba capacitada para adaptarse y llevar a cabo de forma inmediata cualquier tarea práctica. José era moreno de ojos negros; María, prácticamente rubia y de ojos marrones.
\vs p122 5:6 Si José hubiera vivido, sin duda se habría convertido en un firme creyente de la misión divina de su hijo mayor. María alternaba entre la creencia y la duda; estaba considerablemente influenciada por las posturas adoptadas por sus otros hijos y por sus amigos y parientes, pero, siempre, su actitud definitiva al respecto se veía afianzada por la memoria de la aparición de Gabriel inmediatamente tras concebir a su hijo.
\vs p122 5:7 María era una experta tejedora y en casi todas las artes del hogar era más hábil que la media de esos días; era una buena ama de casa con una excelente disposición para llevar su hogar. Tanto José como María eran buenos educadores y procuraron que sus hijos e hijas estuviesen bien versados en los conocimientos de aquellos tiempos.
\vs p122 5:8 \pc Cuando José era joven, el padre de María lo contrató para construir una ampliación de su casa, y fue al mediodía, durante un almuerzo, al llevarle María a José una taza de agua, cuando realmente comenzó el noviazgo de esta pareja que estaba destinada a convertirse en los padres de Jesús.
\vs p122 5:9 José y María se casaron de acuerdo con la costumbre judía, en la casa de María, en las inmediaciones de Nazaret, cuando José contaba con veintiún años de edad. Esta boda puso término a un noviazgo normal de casi dos años de duración. Poco después se mudaron a su nueva casa en Nazaret, que José había construido con la ayuda de dos de sus hermanos. La casa estaba situada próxima al pie de un terreno elevado con hermosas vistas al campo circundante. En esta casa, especialmente preparada, estos jóvenes y futuros padres se aprestaban en sus pensamientos a acoger al hijo de la promesa, sin darse cuenta de que este trascendental acontecimiento para el universo sucedería en Belén de Judea, cuando no se encontraban en su domicilio.
\vs p122 5:10 \pc La mayor parte de la familia de José se convirtió en creyente de las enseñanzas de Jesús, pero muy pocos de los allegados de María creyeron en él hasta después de su partida de este mundo. José se inclinaba más hacia el concepto espiritual del Mesías esperado, mientras que María y su familia, su padre en particular, se aferraba a la idea del Mesías como libertador secular y gobernante político. Los ascendentes de María se habían identificado de manera notoria con las actividades de los Macabeos, en aquel entonces aún recientes.
\vs p122 5:11 José apoyaba categóricamente el punto de vista oriental o babilónico de la religión judía; María se inclinaba firmemente hacia la interpretación occidental o helenística de la ley y de los profetas, más amplia y liberal.
\usection{6. LA CASA DE NAZARET}
\vs p122 6:1 La casa de Jesús no estaba lejos de la elevada colina situada en la parte norte de Nazaret, a cierta distancia de la fuente de la población, que se encontraba en la zona oriental de la ciudad. La familia de Jesús vivía en la periferia de Nazaret, lo que facilitó después a Jesús disfrutar de frecuentes paseos por el campo y subir a la cima de esta colina, la más alta del sur de Galilea, exceptuando la cordillera del Monte Tabor hacia el este y la colina de Naín, que tenía aproximadamente la misma altura. Su casa estaba ubicada hacia el sudeste del promontorio sur de esta colina y aproximadamente a medio camino entre la base de esta elevación y la carretera que conducía de Nazaret a Caná. Además de subir a la colina, el paseo preferido de Jesús era seguir el estrecho sendero que serpenteaba alrededor de la base de la colina en dirección nordeste hasta donde se unía con la carretera que iba a Séforis.
\vs p122 6:2 La casa de José y María era una construcción de piedra de una sola habitación con una azotea y un edificio contiguo para alojar a los animales. Los enseres consistían en una mesa de piedra de poca altura, platos y ollas de barro y piedra, un telar, una lámpara, varios taburetes pequeños y alfombras para dormir sobre el suelo de piedra. En el patio trasero, cerca del anexo de los animales, había un cobertizo que cubría el horno y el molino para moler grano. Se necesitaban dos personas para manejar este tipo de molino, una para moler y la otra para poner el grano. De niño, Jesús solía echar el grano en este molino mientras que su madre giraba la muela.
\vs p122 6:3 Años después, al crecer la familia, todos se sentaban en cuclillas alrededor de la mesa de piedra, ya agrandada, para disfrutar de sus comidas, sirviéndose los alimentos de un plato u olla común. Durante el invierno, a la hora de la cena, la mesa se iluminaba con una pequeña lámpara plana de arcilla, que se llenaba con aceite de oliva. Tras el nacimiento de Marta, José construyó una extensión a esta casa: una amplia habitación que se utilizaba de día como taller de carpintería y de noche como dormitorio.
\usection{7. EL VIAJE A BELÉN}
\vs p122 7:1 En el mes de marzo del año 8 a. C. (el mes en el que José y María se casaron), César Augusto decretó que se calculase el número de todos los habitantes del Imperio romano, que se hiciese un censo a fin de conseguir mejorar los tributos. Los judíos siempre habían tenido grandes prejuicios en contra de cualquier intento de “censar al pueblo” y esto, sumado a las graves dificultades internas de Herodes, rey de Judea, había contribuido a aplazar un año el censo en el reino judío. En todo el Imperio romano, este empadronamiento se llevó a cabo en el año 8 a. C., salvo en el reino palestino de Herodes, en el que se realizó en el año 7 a. C., un año más tarde.
\vs p122 7:2 No era necesario que María fuese a Belén para inscribirse ---José estaba autorizado para empadronar a toda su familia---, pero María, siendo una persona intrépida y dinámica, insistió en acompañarlo. Temía estar sola en caso de que el niño naciera en ausencia de José y, puesto que Belén no estaba lejos de la ciudad de Judá, María preveía la posibilidad de tener una agradable charla con su parienta Elisabet.
\vs p122 7:3 José prácticamente prohibió a María que lo acompañara, pero no sirvió de nada; cuando se empaquetó la comida para el viaje de tres o cuatro días, María preparó el doble de raciones y se dispuso a emprender el viaje. Si bien, antes de ponerse realmente en camino, ya José se había resignado a que María fuese con él y, al despuntar el alba, salieron de Nazaret con buen ánimo.
\vs p122 7:4 María y José eran pobres y, puesto que solamente tenían un animal de carga, María, que estaba en avanzado estado de gestación, iba montada en él con las provisiones, mientras que José caminaba guiando al animal. Construir y equipar la casa había sido un gran gasto para José, que además tenía que contribuir al mantenimiento de sus padres al estar su padre impedido desde hacía poco tiempo. Y así partió esta pareja judía desde su humilde hogar en dirección a Belén a primeras horas de la mañana del 18 de agosto del año 7 a. C.
\vs p122 7:5 Su primer día de viaje los llevó a las faldas del Monte Gilboa, donde acamparon por la noche junto al río Jordán y se enfrascaron en muchas conjeturas sobre qué clase de hijo les iba a nacer; José se aferraba a la noción de un maestro espiritual, mientras que María mantenía la idea de un Mesías judío, un libertador de la nación hebrea.
\vs p122 7:6 A primeras horas de la soleada mañana del 19 de agosto, José y María se pusieron de nuevo en camino. Al mediodía, almorzaron al pie del monte Sartaba, que domina el valle del Jordán, y continuaron su viaje, llegando a Jericó por la noche. Allí se detuvieron en un mesón del camino en las inmediaciones de la ciudad. Después de la cena y tras conversar ampliamente sobre la opresión del gobierno romano, Herodes, el empadronamiento y la comparación entre Jerusalén y Alejandría como centros del conocimiento y de la cultura judíos, los viajeros de Nazaret se retiraron para su descanso nocturno. Muy temprano, en la mañana del 20 de agosto, reanudaron su viaje, llegando a Jerusalén antes del mediodía; allí fueron a ver el templo y prosiguieron hacia su destino, llegando a Belén a media tarde.
\vs p122 7:7 El mesón estaba abarrotado y José trató de buscar alojamiento con parientes lejanos, pero todas las habitaciones de Belén estaban llenas a rebosar. Al volver al patio del mesón, le informaron de que los establos de las caravanas, escavados en el costado de la roca y localizados justo debajo del mesón, se encontraban libres de animales y limpios para alojar a los huéspedes. Dejando al asno en el patio, José cargó al hombro las bolsas de ropa y provisiones y bajó con María los escalones de piedra hasta su alojamiento. Se percataron de que estaban en lo que había sido un almacén de grano, enfrente de los establos y los pesebres. Se habían colgado cortinas de lona, y se consideraron afortunados de tener un lugar tan confortable para quedarse.
\vs p122 7:8 José tenía pensado salir enseguida para inscribirse, pero María estaba agotada; se sentía bastante angustiada y le rogó que permaneciera a su lado, algo que hizo.
\usection{8. EL NACIMIENTO DE JESÚS}
\vs p122 8:1 Durante toda esa noche, María estuvo inquieta por lo que ninguno de los dos pudo dormir mucho. Al abrir el día, los dolores de parto ya eran bien evidentes, y el mediodía del 21 de agosto del año 7 a. C., con la ayuda y el generoso cuidado de otras viajeras, María dio a luz un niño. Jesús de Nazaret había nacido en el mundo, se le envolvió en la ropa que María había traído para tal posible eventualidad y se le acostó en un pesebre cercano.
\vs p122 8:2 Y el niño prometido nació de la misma manera que todos los niños que habían llegado al mundo antes y que vendrían desde entonces; y al octavo día de su nacimiento, conforme a la práctica judía, fue circuncidado y se le puso por nombre Joshua (Jesús).
\vs p122 8:3 Al día siguiente del nacimiento de Jesús, José realizó el empadronamiento. Se encontró con un hombre con el que habían charlado dos noches antes en Jericó y este lo llevó a un amigo adinerado que tenía una habitación en el mesón, el cual le dijo que gustosamente intercambiaría su cuarto con el de la pareja de Nazaret. Aquella misma tarde se trasladaron al mesón, donde permanecieron durante casi tres semanas hasta que encontraron alojamiento en la casa de un pariente lejano de José.
\vs p122 8:4 Al segundo día del nacimiento de Jesús, María mandó a decir a Elisabet que su hijo había nacido y recibió respuesta de ella en la que invitaba a José a ir a Jerusalén para hablar de todos sus asuntos con Zacarías. Y allí se dirigió José, a la semana siguiente, a consultar con él. Tanto Zacarías como Elisabet habían llegado a la honesta convicción de que Jesús estaba realmente destinado a ser el libertador judío, el Mesías, y que su hijo Juan llegaría a ser un hombre de destino, el líder de sus ayudantes y su brazo derecho. Y como María sostenía estas mismas ideas, no resultó difícil convencer a José de quedarse en Belén, la ciudad de David, para que Jesús pudiera crecer y convertirse en el sucesor de David en el trono de todo Israel. Permanecieron, pues, en Belén más de un año, tiempo en el que José trabajó algo en su oficio de carpintero.
\vs p122 8:5 \pc En aquel mediodía en el que nació Jesús, los serafines de Urantia, congregados bajo sus directores, cantaron en efecto himnos de gloria sobre el pesebre de Belén, pero estas manifestaciones de gloria no fueron oídas por oídos humanos. Ningún pastor ni ninguna otra criatura mortal vinieron a rendir homenaje al niño de Belén hasta el día de la llegada de unos sacerdotes de Ur, que habían sido enviados desde Jerusalén por Zacarías.
\vs p122 8:6 Cierta vez, un extraño maestro religioso de su país había contado a estos sacerdotes de Mesopotamia que había tenido un sueño en el que se le comunicaba que la “luz de la vida” estaba a punto de aparecer en la tierra, entre los judíos, como niño. Y hacia allí se dirigieron los tres buscando esta “luz de la vida”. Tras muchas semanas de búsqueda inútil en Jerusalén y, estando a punto de regresar a Ur, Zacarías los conoció y les hizo saber que era a Jesús a quien buscaban. Zacarías los envió a Belén. Allí lo hallaron y dejaron sus regalos a María, su madre terrenal. En el momento de su visita, el niño tenía casi tres semanas de edad.
\vs p122 8:7 Estos sabios no vieron ninguna estrella que les sirviera de guía hasta Belén. La bella leyenda sobre la estrella de Belén se originó de esta manera: Jesús nació el mediodía del 21 de agosto del año 7 a. C. El 29 de mayo de ese mismo año, se dio una extraordinaria conjunción de Júpiter y Saturno en la constelación de Piscis. Y es un hecho astronómico notable que ocurrieran conjunciones similares el 29 de septiembre y el 5 de diciembre del mismo año. Sobre la base de estos sucesos extraordinarios, pero totalmente naturales, personas de generaciones venideras, bien intencionadas pero con exceso de celo, compusieron la interesante leyenda de la estrella de Belén y de los magos que guiados por ella acudieron al pesebre para contemplar y adorar al recién nacido. Las mentes de Oriente y del Oriente Próximo se deleitan en las fábulas y confeccionan continuamente mitos así de bellos sobre la vida de sus líderes religiosos y de sus héroes políticos. Cuando no existía la imprenta y la mayoría del conocimiento humano se trasmitía de boca en boca de una generación a la otra, era muy fácil que los mitos se convirtiesen en tradiciones y que las tradiciones acabaran por aceptarse como hechos.
\usection{9. PRESENTACIÓN EN EL TEMPLO}
\vs p122 9:1 Moisés había enseñado a los judíos que todos los hijos primogénitos pertenecían al Señor y que, en lugar de su sacrificio tal como era costumbre entre las naciones paganas, ese hijo podría vivir, siempre que sus padres lo redimieran mediante el pago de cinco siclos a cualquier sacerdote reconocido. También existía un decreto mosaico que ordenaba que una madre, tras un cierto período de tiempo, se presentara (o que alguien hiciera el debido sacrificio por ella) en el templo para purificarse. Era habitual realizar ambas ceremonias al mismo tiempo. Así pues, José y María se acercaron al templo de Jerusalén en persona para presentar a Jesús a los sacerdotes y llevar a cabo su redención, y hacer también ese debido sacrificio que asegurara la purificación ceremonial de María de la presunta impureza del alumbramiento.
\vs p122 9:2 \pc En los patios del templo había dos excepcionales personajes que no se apartaban de allí: Simeón, cantor, y Ana, poetisa. Simeón era de Judea, pero Ana, de Galilea. La pareja estaba con frecuencia uno en compañía del otro y eran amigos íntimos del sacerdote Zacarías, que les había confiado el secreto de Juan y de Jesús. Tanto Simeón como Ana anhelaban la venida del Mesías, y su confianza en Zacarías los llevó a creer que Jesús era el libertador esperado por los judíos.
\vs p122 9:3 Zacarías sabía el día en el que se esperaba que José y María apareciesen por el templo con Jesús y había acordado con Simeón y Ana que, durante la procesión de los niños primogénitos, les indicaría quién de ellos era Jesús.
\vs p122 9:4 Para esta ocasión, Ana había escrito un poema que Simeón procedió a cantar, para gran asombro de José, de María y de todos los que se encontraban reunidos en los patios del templo. Y este fue el himno de redención del hijo primogénito:
\vs p122 9:5 Bendito sea el Señor, Dios de Israel,
\vs p122 9:6 que ha visitado y redimido a su pueblo,
\vs p122 9:7 y nos levantó un poderoso Salvador
\vs p122 9:8 en la casa de David, su siervo.
\vs p122 9:9 Como habló por boca de sus santos profetas
\vs p122 9:10 ---salvación de nuestros enemigos y de la mano de todos los que nos odian;
\vs p122 9:11 para hacer misericordia con nuestros padres y acordarse de su santo pacto---.
\vs p122 9:12 del juramento que hizo a Abraham, nuestro padre,
\vs p122 9:13 que nos había de conceder, que, librados de nuestros enemigos,
\vs p122 9:14 sin temor lo serviríamos
\vs p122 9:15 en santidad y en justicia delante de él todos nuestros días.
\vs p122 9:16 Sí, y tú, niño de la promesa, serás llamado el profeta del Altísimo
\vs p122 9:17 porque irás delante de la presencia del Señor para establecer su reino,
\vs p122 9:18 para dar conocimiento de salvación a su pueblo
\vs p122 9:19 en remisión de sus pecados.
\vs p122 9:20 Regocijaos en la entrañable misericordia de nuestro Dios porque nos visitó desde lo alto la aurora
\vs p122 9:21 para dar luz a los que habitan en tinieblas y en sombra de muerte;
\vs p122 9:22 para encaminar nuestros pies por camino de paz.
\vs p122 9:23 Y, ahora, Señor, despides a tu siervo en paz, conforme a tu palabra,
\vs p122 9:24 porque han visto mis ojos tu salvación,
\vs p122 9:25 la cual has preparado en presencia de todos los pueblos;
\vs p122 9:26 luz para revelación a los gentiles
\vs p122 9:27 y gloria de tu pueblo Israel.
\vs p122 9:28 \pc En el camino de vuelta a Belén, José y María guardaban silencio, desconcertados y sobrecogidos. María estaba muy turbada por el saludo de despedida de Ana, la poetisa de avanzada edad, y José no estaba de acuerdo con este precipitado empeño por hacer de Jesús el Mesías esperado del pueblo judío.
\usection{10. HERODES PASA A LA ACCIÓN}
\vs p122 10:1 Pero los observadores de Herodes no actuaron sin falta de diligencia y, cuando le informaron de la visita de los sacerdotes de Ur a Belén, Herodes llamó a estos caldeos a su presencia. Les preguntó intencionadamente acerca del nuevo “rey de los judíos”, pero quedó poco contento. Le explicaron que el niño había nacido de una mujer que se había acercado a Belén con su marido para inscribirse en el censo. Insatisfecho con esta respuesta, Herodes los despidió con una bolsa de dinero mandándoles que encontraran al niño, para que así también él pudiese adorarle, ya que habían declarado que su reino sería espiritual y no terrenal. Pero como estos sabios no regresaron, Herodes empezó a recelar y, mientras estas cosas rondaban por su mente, sus informadores volvieron con una descripción completa de los últimos acontecimientos habidos en el templo. Le traían una copia de partes de la canción que Simeón había cantado en las ceremonias de redención de Jesús. Pero no habían podido seguir a José y a María. Y cuando le dijeron que no sabían adónde se había llevado la pareja al niño, Herodes se enfureció con ellos. Mandó entonces a exploradores para localizar a José y a María. Habiéndose enterado de que Herodes buscaba a la familia de Nazaret, Zacarías y Elisabet permanecieron alejados de Belén y se ocultó al niño en la casa de unos parientes de José.
\vs p122 10:2 José tenía miedo de buscar trabajo, y sus escasos ahorros estaban rápidamente desapareciendo. Incluso en el momento de las ceremonias de purificación en el templo, José se consideraba lo suficientemente pobre como para justificar su ofrenda de dos palominos para la purificación de María, tal como Moisés había prescrito para la purificación de las madres menesterosas.
\vs p122 10:3 Cuando, tras más de un año de búsqueda, los espías de Herodes aún no habían encontrado a Jesús y, ante la sospecha de que el niño todavía estaba oculto en Belén, Herodes expidió un mandato ordenando que se hiciera un registro sistemático de todas las casas de Belén y que se matara a todos los niños varones menores de dos años. De esta manera, esperaba acabar con quien se iba a convertir en el “rey de los judíos”. Y así, en un solo día, murieron dieciséis niños en Belén de Judea. Pero la intriga y el asesinato, incluso dentro de su misma familia cercana, eran algo común en la corte de Herodes.
\vs p122 10:4 La masacre de estos pequeños tuvo lugar a mediados de octubre del año 6 a. C., cuando Jesús tenía poco más de un año de edad. Pero incluso entre el personal de la corte de Herodes había creyentes en la venida del Mesías, y uno de ellos, al enterarse de la matanza de los niños de Belén, se puso en contacto con Zacarías, quien, a su vez, envió un mensajero a José; y, la noche antes de la masacre, José y María partieron con el niño de Belén para Alejandría, Egipto. Para evitar atraer la atención, viajaron solos a Egipto con Jesús. Llegaron a Alejandría ayudados de los fondos que Zacarías les proveyó, y allí José trabajó en su oficio mientras que María y Jesús se alojaban con parientes adinerados de la familia de José. Residieron en Alejandría durante dos años completos, y no regresaron a Belén hasta después de la muerte de Herodes.
