\upaper{176}{En el monte de los Olivos, a última hora de la tarde del martes}
\author{Comisión de seres intermedios}
\vs p176 0:1 Aquel martes, temprano por la tarde, al salir Jesús y los apóstoles del templo camino al campamento de Getsemaní, Mateo, llamando la atención sobre la arquitectura del templo, dijo: “Maestro, observa qué tipo de edificios. Mira las enormes piedras y los hermosos ornamentos; ¿cómo es posible que vayan a ser destruidos?”. Conforme se dirigían al Monte de los Olivos, Jesús dijo: “Veis estas piedras y este inmenso templo; de cierto, de cierto os digo que en los días que están pronto por llegar no quedará piedra sobre piedra que no sea derribada”. Estos comentarios acerca de la destrucción del templo sagrado despertó la curiosidad de los apóstoles que caminaban detrás del Maestro; no podían concebir nada que ocasionara la destrucción del templo a no ser que llegara el fin del mundo.
\vs p176 0:2 Para evitar a las multitudes que recorrían el valle del Cedrón en dirección a Getsemaní, Jesús y sus acompañantes tenían la idea de subir por la ladera occidental del Monte de los Olivos, durante un corto trayecto, y seguir luego por un sendero que llevaba hasta su propio campamento, cerca de Getsemaní, a poca distancia por encima de la zona de acampada pública. Al volverse para dejar la carretera que conducía a Betania, se fijaron en el templo, glorificado por los rayos del sol poniente; y, mientras se detuvieron en el monte, vieron encenderse las luces de la ciudad y contemplaron la belleza del templo iluminado; y allí se sentaron Jesús y los doce, bajo la suave luz de la luna llena. El Maestro estaba hablando con ellos cuando, de repente, Natanael hizo esta pregunta: “Dinos Maestro, ¿cómo sabremos cuándo serán estas cosas?”.
\usection{1. LA DESTRUCCIÓN DE JERUSALÉN}
\vs p176 1:1 Respondiendo a la pregunta de Natanael, Jesús dijo: “Sí, os diré de los tiempos en los que este pueblo habrá colmado la copa de su iniquidad; cuando la justicia caerá de momento sobre la ciudad de nuestros padres. Estoy a punto de dejaros; voy al Padre. Una vez que me haya ido, mirad que nadie os engañe, porque vendrán muchos libertadores y os llevarán por el camino errado. Cuando oigáis de guerras y rumores de guerras, no os turbéis, porque aunque estas cosas acontecerán, el fin de Jerusalén aún no está cerca. No os desasoseguéis por el hambre y los terremotos; ni debéis inquietaros cuando se os entregue a las autoridades civiles y seáis perseguidos por causa del evangelio. Seréis expulsados de la sinagoga e iréis a prisión por causa de mí, y a algunos de vosotros os matarán. Cuando os lleven ante los gobernadores y los dirigentes, será para dar testimonio de vuestra fe y para mostrar vuestra perseverancia en el evangelio del reino. Y cuando estéis ante los jueces, no os preocupéis por lo que habéis de decir, porque el espíritu os enseñará en aquella hora lo que debéis responder a vuestros adversarios. En estos días de aflicción, incluso vuestros propios parientes, bajo el mando de quienes han rechazado al Hijo del Hombre, os entregarán a la cárcel y a la muerte. Por algún tiempo, seréis odiados por todos por causa de mi nombre, pero a pesar de estas persecuciones yo no os abandonaré; mi espíritu no os desertará. ¡Sed pacientes! No dudéis de que este evangelio del reino triunfará sobre todos sus enemigos y terminará por proclamarse a todas las naciones”.
\vs p176 1:2 Jesús se detuvo mientras contemplaba la ciudad. El Maestro se dio cuenta de que el rechazo de los judíos del concepto espiritual del Mesías, su insistente y ciego aferramiento a la misión material del libertador esperado, acabaría en algún momento por llevarlos directamente a un enfrentamiento con los poderosos ejércitos romanos, y que esa contienda solo resultaría en la destrucción definitiva y completa de la nación judía. Cuando su pueblo rechazó su ministerio de gracia espiritual y se negó a recibir la luz del cielo, que de forma tan misericordiosa brillaba sobre ellos, sellaron con ello su perdición como pueblo independiente con una especial misión espiritual en la tierra. Incluso los líderes judíos reconocerían después que fue esta idea materialista del Mesías la que llevó directamente hasta los alborotos que provocaron finalmente su destrucción.
\vs p176 1:3 Debido a que Jerusalén llegaría a convertirse en la cuna del primitivo movimiento evangélico, Jesús no quería que sus maestros y predicadores murieran en la terrible derrota que el pueblo judío sufriría como consecuencia de la destrucción de Jerusalén; por ello, dio estas instrucciones a sus seguidores. A Jesús le preocupaba sobremanera que algunos de sus discípulos se vieran envueltos en las revueltas que estaban pronto a llegar y perecieran, por tanto, en la caída de Jerusalén.
\vs p176 1:4 Luego, Andrés preguntó: “Pero, Maestro, si la Ciudad Santa y el templo han de ser destruidos, y si tú no estás aquí para guiarnos, ¿cuándo debemos abandonar Jerusalén?”. Jesús dijo: “Podéis permanecer en la ciudad después de que yo me haya ido, incluso durante los tiempos de aflicción y amargas persecuciones, pero cuando veáis que Jerusalén están siendo finalmente rodeada por los ejércitos romanos, tras la revuelta de los falsos profetas, sabed que su destrucción ha llegado; huid enseguida a las montañas. Que nadie de los que están en la ciudad y a su alrededor se quede para salvar nada, ni nadie de los que estén fuera se atreva a entrar en ella. Habrá una gran tribulación, porque en esos días será la venganza de los gentiles. Y después de que hayáis abandonado la ciudad, este pueblo desobediente caerá a filo de espada y será llevado cautivo a todas las naciones; y Jerusalén será pisoteada por los gentiles. Entretanto, tened cuidado de que no os engañen. Si alguno os dice: ‘Mirad, aquí está el Libertador’ o ‘Mirad, allí está’, no le creáis, porque se levantarán muchos falsos maestros y muchos serán llevados por el camino errado; pero vosotros no os dejéis engañar porque os he dicho todo esto de antemano”.
\vs p176 1:5 Los apóstoles estuvieron sentados, en silencio, a la luz de la luna, durante un buen rato, mientras que estas asombrosas predicciones del Maestro se hacían eco en sus desconcertadas mentes. Y, acorde con esta advertencia, prácticamente todo el grupo de creyentes y discípulos huyó de Jerusalén en cuanto aparecieron las tropas romanas y encontraron refugio seguro en el norte, en Pella.
\vs p176 1:6 Pero, a pesar de esta clara advertencia, muchos de los seguidores de Jesús interpretaron estas predicciones en referencia a unos cambios que ocurrirían evidentemente en Jerusalén cuando el Mesías reapareciera y, con él, la instauración de la Nueva Jerusalén y el engrandecimiento de la ciudad hasta convertirse en la capital del mundo. En sus mentes, estos judíos relacionaban insistentemente la destrucción del templo con el “fin del mundo”. Creían que esta Nueva Jerusalén abarcaría toda Palestina; que al fin del mundo lo seguiría la aparición inmediata de los “nuevos cielos y la nueva tierra”. Y, así pues, no es de extrañar que Pedro dijera: “Maestro, sabemos que todas las cosas pasarán cuando aparezcan los nuevos cielos y la nueva tierra, pero, ¿cómo sabremos cuándo volverás para hacer todo esto realidad?”.
\vs p176 1:7 Cuando Jesús oyó esto, se quedó pensativo durante algún tiempo y entonces dijo: “Siempre erráis porque tratáis de relacionar la nueva enseñanza con la antigua; continuáis constantemente malinterpretando mis enseñanzas, porque insistís en interpretar el evangelio según vuestras creencias tradicionales. No obstante, trataré de daros una explicación”.
\usection{2. LA SEGUNDA VENIDA DEL MAESTRO}
\vs p176 2:1 En distintas ocasiones, Jesús había realizado algunas afirmaciones que habían llevado a pensar a quienes lo oían que, aunque se disponía a dejar este mundo en poco tiempo, de cierto regresaría para consumar la obra del reino celestial. Conforme crecía en sus seguidores la convicción de que los iba a dejar, y tras haber partido de este mundo, era muy natural que todos los creyentes se aferraran fuertemente a estas promesas de su vuelta. La doctrina de la segunda venida de Cristo se integró, pues, tempranamente, en las enseñanzas de los cristianos, y casi todas las generaciones de discípulos que siguieron creyeron con fervor en esta verdad y aguardaban confiados y con gran expectación su venida en algún momento.
\vs p176 2:2 Al tener que separarse de su Maestro e Instructor, estos primeros discípulos y apóstoles se aferraron aún más a esta promesa de su regreso, y no tardaron mucho en relacionar la destrucción vaticinada de Jerusalén con esta prometida segunda venida. Y así continuaron, pues, interpretando sus palabras, a pesar de que en su instrucción en el Monte de los Olivos, a última hora de la tarde, el Maestro se esforzó muy particularmente en evitar justamente tal error.
\vs p176 2:3 \pc Continuando con la respuesta a la pregunta de Pedro, Jesús dijo: “¿Por qué seguís aguardando a que el Hijo del Hombre se siente en el trono de David y pretendéis que se hagan realidad los sueños materialistas de los judíos? ¿Es que no os he dicho durante todos estos años que mi reino no es de este mundo? Las cosas que contempláis ahora desde aquí arriba están tocando su fin, pero este es un nuevo comienzo desde el que el evangelio del reino se extenderá a todo el mundo y esta salvación se esparcirá por todos los pueblos. Y cuando el reino haya llegado a su plenitud, tened la seguridad de que el Padre de los cielos no cejará en concederos una revelación ampliada de la verdad y una manifestación engrandecida de la rectitud, tal como ya otorgó a este mundo a quien se convirtió en el príncipe de la oscuridad y, luego, a Adán, a quien siguió Melquisedec y, en estos días, al Hijo del Hombre. Y, por tanto, mi Padre continuará manifestando su misericordia y mostrando su amor, incluso a este mundo malvado y en tinieblas. De igual manera, yo, una vez que mi Padre me haya investido con todo poder y autoridad, continuaré siguiendo vuestros caminos, guiándoos en los asuntos del reino por medio de la presencia de mi espíritu, que dentro de poco se derramará sobre toda carne. Y aunque estaré presente con vosotros en espíritu, os prometo también que alguna vez volveré a este mundo, donde he vivido esta vida en la carne consiguiendo, simultáneamente, revelar Dios al hombre y llevar al hombre hasta Dios. Muy pronto debo abandonaros y aceptar la labor que el Padre ha confiado en mis manos, pero tened buen ánimo porque, en algún momento, regresaré. Entretanto, mi espíritu de la verdad del universo os confortará y guiará.
\vs p176 2:4 “Ahora me veis en la debilidad y en la carne, pero cuando regrese, será con poder y en el espíritu. Los ojos humanos contemplan al Hijo del Hombre en la carne, pero solo los ojos espirituales podrán contemplar al Hijo del Hombre glorificado por su Padre y aparecer en la tierra en su propio nombre.
\vs p176 2:5 “Pero solo en los consejos del Paraíso se conoce el momento en el que el Hijo del Hombre reaparecerá; ni siquiera los ángeles del cielo saben cuándo acontecerá. Sin embargo, deberíais entender que, cuando este evangelio del reino se haya proclamado en todo el mundo para la salvación de todos los pueblos, y cuando haya llegado la plenitud de los tiempos de esta era, el Padre os otorgará una nueva dispensación de la verdad, o bien, el Hijo del Hombre regresará para decretar el fin de la era.
\vs p176 2:6 “Ahora bien, en cuanto a la lamentación de Jerusalén, de la que os he hablado, esta generación no pasará sin que se cumplan mis palabras; pero respecto a los tiempos de la nueva venida del Hijo del Hombre, nadie, ni en el cielo ni en la tierra, puede decirlo con seguridad. Pero debéis ser juiciosos en cuanto a la llegada de una era a su desarrollo pleno; debéis estar alertas para percibir los signos de los tiempos. Sabéis que cuando la rama de la higuera está tierna y brotan las hojas, el verano está cerca. Así también vosotros cuando el mundo haya pasado por el largo invierno del pensamiento materialista y veáis la llegada de la primavera espiritual de una nueva dispensación, sabréis que se aproxima el estío de una nueva visita.
\vs p176 2:7 “Pero, ¿cuál es el significado de lo que os enseño sobre la venida de los Hijos de Dios? ¿Es que no percibís que, cuando cada uno de vosotros sea llamado a deponer los afanes de vida y crucéis los portales de la muerte, se os juzgará de inmediato y estaréis ante la realidad de una nueva dispensación en la que el servicio es parte del propósito eterno del Padre infinito? Lo que el mundo entero debe afrontar literalmente al final de una era, vosotros, de forma individual, lo experimentaréis de cierto, personalmente, cuando alcancéis el término de vuestra vida natural y paséis con ello a enfrentaros a las condiciones y exigencias inherentes a la próxima revelación, que conlleva el progreso eterno en el reino del Padre”.
\vs p176 2:8 De todas las charlas impartidas por el Maestro a sus apóstoles, ninguna les resultó tan confusa como aquella del martes a última hora de la tarde, en el Monte de los Olivos, que trataba doblemente sobre la destrucción de Jerusalén y su propia segunda venida. Surgieron, pues, discordancias entre los relatos escritos posteriores, basados en los recuerdos de lo que el Maestro había dicho en aquella extraordinaria ocasión. Se omitieron, pues, muchos datos de esta charla, dando lugar a un gran número de tradiciones. Y, a comienzos del siglo II, un texto apocalíptico judío sobre el Mesías, redactado por un tal Selta, adscrito a la corte del emperador Calígula, se incluyó íntegramente en el Evangelio de Mateo y se añadió después (en parte) en los escritos de Marcos y de Lucas. Fue en este texto de Selta donde aparecía la parábola de las diez vírgenes. En ninguna otra parte de los textos evangélicos se produjo jamás tal errónea interpretación de las enseñanzas de aquel día. Pero no hubo tal confusión en el apóstol Juan.
\vs p176 2:9 Al reanudar su viaje hacia el campamento, estos trece hombres estaban sin habla y se encontraban bajo una gran tensión emocional. Judas se ratificó finalmente en su decisión de abandonar a sus compañeros. Era tarde cuando David Zebedeo, Juan Marcos y un cierto número de los discípulos más destacados recibieron a Jesús y a los doce en el nuevo campamento, pero los apóstoles no querían dormir; deseaban saber más cosas sobre la destrucción de Jerusalén, la partida del Maestro y el fin del mundo.
\usection{3. CONTINÚAN CONVERSANDO EN EL CAMPAMENTO}
\vs p176 3:1 Mientras se reunían alrededor del fuego del campamento unos veinte de ellos, Tomás preguntó: “Ya que tienes que regresar para terminar la obra del reino, ¿cuál debe ser nuestra actitud cuando estés lejos ocupado en los asuntos del Padre?”. Mirándolos a la luz de la lumbre, Jesús respondió:
\vs p176 3:2 \pc “Ni incluso tú, Tomás, alcanzas a comprender lo que he estado diciendo. ¿Es que no te he enseñado todo este tiempo que tu relación con el reino es espiritual e individual, que es completamente una cuestión de experiencia personal en el espíritu cuando te haces consciente, mediante la fe, de que eres hijo de Dios? ¿Qué más he de decir? La caída de las naciones, el desplome de los imperios, la desaparición de los judíos descreídos, la terminación de una era, incluso el fin del mundo, ¿qué tiene todo esto que ver con el evangelio y con quien ha consagrado su vida en la seguridad del reino eterno? Vosotros, conocedores de Dios y creyentes del evangelio, ya habéis recibido la garantía de la vida eterna. Puesto que habéis vivido vuestras vidas en el espíritu y para el Padre, nada os puede preocupar excesivamente. Los edificadores del reino, los ciudadanos reconocidos de los mundos celestiales, no deben alterarse por las agitaciones temporales ni turbarse por los cataclismos terrenales. ¿Qué os importa a quienes creéis en este evangelio del reino si las naciones caen derribadas, si la era se termina o si sucumben todas las cosas visibles, cuando sabéis que vuestra vida es el don del Hijo y que está eternamente segura en el Padre? Habiendo vivido la vida temporal por la fe y habiendo rendido los frutos del espíritu a través de vuestro noble y amoroso servicio por vuestros semejantes, podéis aspirar a dar confiadamente el siguiente paso en vuestra andadura eterna, con la misma fe en la supervivencia que os ha llevado a través de vuestra primera aventura terrenal en filiación con Dios.
\vs p176 3:3 “Cada generación de creyentes debe llevar a cabo su labor ante el futuro regreso del Hijo del Hombre exactamente de la misma manera que cada uno de ellos lleva adelante su labor de vida ante la inevitable y siempre acechante muerte natural. Cuando, por la fe, hayáis confirmado que sois hijos de Dios, no importa ninguna otra cosa, porque nada os podrá apartar de la seguridad de vuestra supervivencia. Pero, ¡no os equivoquéis! Esta fe en la supervivencia es una fe viva, y manifiesta crecientemente los frutos de ese divino espíritu que la inspiró primeramente en el corazón humano. El hecho de haber recibido en algún momento la filiación en el reino celestial, no os salva si rechazáis de forma intencionada y con insistencia esas verdades que se corresponden con los frutos espirituales que los hijos de Dios en la carne han de rendir cada vez más. Vosotros, que habéis estado conmigo en los asuntos del Padre en la tierra, incluso en este momento podéis desertar del reino si descubrís que no deseáis servir a la humanidad tal como el Padre ha establecido.
\vs p176 3:4 “De manera individual, y como generación de creyentes, oídme mientras os cuento una parábola: Hubo un cierto gran hombre que, antes de irse a un largo viaje a otro país, llamó a todos sus siervos de confianza y les entregó sus bienes. A uno dio cinco talentos, a otro dos y a otro uno. Y así sucesivamente al grupo completo de honrados mayordomos le confió a cada uno sus bienes conforme a su capacidad; y luego emprendió su viaje. Cuando su señor había partido, sus siervos se pusieron a trabajar para obtener ganancias de la riqueza que se les había confiado. De inmediato, el que había recibido cinco talentos fue y negoció con ellos y enseguida había ganado otros cinco talentos. Asimismo, el que había recibido dos talentos pronto tenía otros dos. Y así, todos los siervos consiguieron beneficios para su amo salvo aquel que había recibido un solo talento. Él se fue solo e hizo un hoyo en la tierra, escondiendo allí el dinero de su señor. Al poco tiempo, este regresó de forma inesperada y llamó a sus mayordomos para arreglar cuentas con ellos. Y cuando estaban en presencia de su amo, se acercó el que había recibido los cinco talentos con el dinero que se le había confiado y otros cinco talentos y dijo: ‘Señor, cinco talentos me entregaste para invertir, y me alegra darte otros cinco talentos que he ganado sobre ellos’. Y, entonces, su señor le dijo: ‘Bien hecho, mi buen y leal siervo, sobre poco has sido fiel; te nombraré mayordomo sobre mucho; entra así en el gozo de tu señor’. Y se acercó también el que había recibido dos talentos y dijo: ‘Señor, dos talentos me entregaste; aquí tienes, he ganado otros dos talentos sobre ellos’. Y su señor entonces le dijo: ‘Bien hecho, mi buen y leal mayordomo; tú también has sido fiel sobre poco y sobre mucho te pondré ahora; entra en el gozo de tu señor’. Y luego vino a rendir cuentas el que había recibido un talento. Este siervo se acercó diciendo: ‘Señor, yo te conocía y me daba cuenta de que eras un hombre astuto, porque esperabas beneficios de donde no habías trabajado personalmente; por lo cual, tuve miedo de arriesgar lo que se me había confiado. Escondí tu talento en lugar seguro en la tierra; aquí está; aquí tienes lo que es tuyo’. Pero su señor respondió: ‘mayordomo indolente y holgazán. Por tus propias palabras confiesas que sabías que exigiría de ti que me entregaras unos beneficios razonables, al igual que tus diligentes consiervos han hecho este día. Sabiendo esto, deberías haber dado, por tanto, mi dinero a los banqueros, y a mi regreso hubiera recibido lo que es mío con los intereses’. Y, entonces, le dijo este señor al mayordomo jefe: ‘Quitadle este talento a este inútil siervo y dadlo al que tiene diez talentos’.
\vs p176 3:5 “Al que tiene, más le será dado, y tendrá más; pero al que no tiene, aun lo que tiene le será quitado. No podéis quedaros quietos en los asuntos del reino eterno. Mi Padre quiere que todos sus hijos crezcan en la gracia y en el conocimiento de la verdad. Vosotros que conocéis estas verdades debéis dar cada vez más frutos del espíritu y manifestar una devoción creciente al servicio desinteresado de vuestros compañeros en esta misma tarea. Y recordad que en cuanto cuidasteis a uno de mis hermanos más pequeño, a mí lo hicisteis.
\vs p176 3:6 “Y así deberíais atender los asuntos del Padre, ahora y en lo sucesivo, incluso para siempre. Seguid adelante hasta que yo regrese. Haced fielmente lo que se os ha encomendado con lo que estaréis listos para el juicio final cuando os llame la muerte. Y habiendo vivido de esta forma para la gloria del Padre y la satisfacción del Hijo, entraréis con gozo y con grande y suma complacencia al servicio eterno del reino perpetuo”.
\vs p176 3:7 \pc La verdad está viva; el espíritu de la verdad guía por siempre a los hijos de la luz a los nuevos ámbitos de la realidad espiritual y del servicio divino. La verdad no se os da para que quede cristalizada bajo formas establecidas, inmutables y reverenciadas. Al pasar por vuestra experiencia personal, vuestra revelación de la verdad debe realzarse de tal manera que se desvelarán, ante todos los que contemplan vuestros frutos espirituales, una nueva belleza y unos verdaderos logros espirituales, por lo que se sentirán llevados a glorificar al Padre de los cielos. Solo aquellos servidores fieles, que crecen de esta manera en el conocimiento de la verdad y desarrollan pues la capacidad para reconocer de forma divina las realidades espirituales, pueden tener alguna vez la esperanza de “entrar plenamente en el gozo de su Señor”. Qué triste visión para las generaciones venideras de profesos seguidores de Jesús decir, refiriéndose a su mayordomía de la verdad divina: “Aquí, Maestro, está la verdad que nos encomendaste cien o mil años atrás. No hemos perdido nada; hemos preservado fielmente todo lo que nos diste; no hemos permitido ningún cambio en lo que nos enseñaste; aquí está la verdad que nos diste”. Pero esta excusa, que denota indolencia espiritual, no justifica al yermo mayordomo de la verdad en la presencia del Maestro. El Maestro os exigirá que rindáis cuenta de acuerdo con la verdad que se ha encomendado a vuestras manos.
\vs p176 3:8 En el próximo mundo, se os pedirá que deis razón de vuestros dones y de las responsabilidades que se os ha encomendado en este mundo. Ya sean muchos o pocos vuestros talentos innatos, se debe afrontar una rendición de cuentas que se llevará a cabo con justicia y misericordia. El mayordomo debe aceptar las consecuencias de sus deliberados actos si egoístamente emplea los dones a él otorgados con fines interesados y no presta consideración al deber superior de incrementar los frutos del espíritu, tal como se ponen de manifiesto en el siempre creciente servicio al hombre y en la adoración a Dios.
\vs p176 3:9 ¡Y cuánto se parece a todos esos mortales egoístas este siervo infiel, que recibió un solo talento, por cuanto culpó directamente a su señor de su propia holgazanería! Cuándo se ha de enfrentar a los errores que él mismo ha cometido, ¡qué propenso es el hombre a echarle las culpas a los demás y, a menudo, a quienes menos se lo merecen!
\vs p176 3:10 Aquella noche, cuando se retiraban a descansar, Jesús les dijo: “De gracia recibisteis, dad de gracia la verdad del cielo, y, al dar gratuitamente esta verdad se multiplicará y se mostrará la luz creciente de la gracia salvadora en la medida en la que vosotros la impartáis”.
\usection{4. EL REGRESO DE MIGUEL}
\vs p176 4:1 De todas las enseñanzas del Maestro, ningún aspecto se ha malinterpretado tanto como su promesa de volver algún día a este mundo en persona. No es de extrañar que Miguel estuviera interesado en regresar al planeta en el que había vivido su séptimo y último ministerio de gracia y residido como un mortal más. Es lógico creer que Jesús de Nazaret, ahora gobernante soberano de un inmenso universo, quiera volver, no solo una sino muchas veces, al mundo en el que tuvo una vida tan excepcional y donde logró, por sí mismo, que el Padre le concediera un ilimitado poder y autoridad sobre el universo. Urantia será eternamente una de las siete esferas en las que nació Miguel y alcanzó la soberanía del universo.
\vs p176 4:2 Efectivamente, Jesús expresó, en numerosas ocasiones y a muchas personas, su intención de regresar a este mundo. A medida que sus seguidores tomaron conciencia del hecho de que su Maestro no iba a convertirse en el libertador nacional que esperaban y, a medida que escucharon sus predicciones sobre la caída de Jerusalén y de la nación judía, comenzaron, de la forma más natural, a vincular su prometido retorno con estos sucesos catastróficos. Pero cuando los ejércitos romanos derribaron los muros de Jerusalén, destruyeron el templo y dispersaron a los judíos de Judea, y el Maestro no se había revelado aún en poder y gloria, sus seguidores comenzaron a concebir la noción de esa creencia que acabaría por relacionar la segunda venida de Cristo con el final de la era, incluso con el fin del mundo.
\vs p176 4:3 Una vez que había ascendido al Padre y se hubiera depositado en sus manos todo el poder en el cielo y en la tierra, Jesús prometió dos cosas. En primer lugar, prometió enviar al mundo, y en su lugar, a otro maestro, el espíritu de la verdad; y la cumplió el día de Pentecostés. En segundo lugar, indudablemente prometió a sus seguidores que regresaría a este mundo personalmente, en algún determinado momento. Pero no dijo cómo, dónde ni cuándo volvería a visitar este planeta donde vivió su vida de gracia en la carne. En cierta ocasión, indicó que al igual que los ojos de la carne lo habían contemplado mientras vivía aquí en la carne, tan solo se le podría percibir con los ojos espirituales a su vuelta (al menos en alguna de sus posibles visitas).
\vs p176 4:4 Muchos de nosotros nos inclinamos a creer que Jesús regresará a Urantia repetidas veces durante futuras eras. No contamos con su promesa concreta de que hará estas múltiples visitas, pero parece lo más probable que, quien lleva entre sus títulos logrados en el universo el de Príncipe Planetario de Urantia, pueda venir de vuelta en muchas ocasiones al mundo cuya conquista le confirió tan singular título.
\vs p176 4:5 Creemos sin lugar a dudas que Miguel volverá en persona a Urantia, pero no tenemos la más mínima certeza de cuándo ni cómo elegirá hacerlo. ¿Acontecerá esta prevista segunda venida a la tierra con motivo del juicio final de la era vigente, en conjunción o no con la aparición del hijo magistrado? ¿Vendrá en el marco de la terminación de alguna era aún por llegar a Urantia? ¿Vendrá sin anunciarse y como un acontecimiento aislado? No lo sabemos. Solo estamos seguros de algo, y es de que, cuando Miguel vuelva, todo el mundo probablemente lo sabrá, porque lo hará como gobernante supremo de un universo y no como un desconocido pequeño de Belén. Pero si todos los ojos lo han de contemplar, y solo los ojos espirituales podrán percibir su presencia, su venida deberá entonces dilatarse en el tiempo.
\vs p176 4:6 Haríais bien, pues, en desvincular el regreso personal del Maestro a la tierra con cualquier acontecimiento fijado o época establecida. Solo estamos seguros de algo: prometió que volvería. No tenemos idea de cuándo cumplirá tal promesa ni en relación a qué se cumplirá. Según tenemos entendido, puede aparecer en la tierra cualquier día, y puede que no lo haga hasta que hayan pasado una era tras otra, y todas hayan sido debidamente juzgadas por compañeros suyos del colectivo de hijos del Paraíso.
\vs p176 4:7 La segunda venida de Miguel a la tierra tiene un gran valor sentimental tanto para los seres intermedios como para los seres humanos. No obstante, para las criaturas intermedias no tiene relevancia inmediata. Tampoco tiene mayor importancia práctica para todos los seres humanos esta segunda venida, puesto que la muerte natural hace que se precipiten y vean envueltos, en el universo, en una consecución de acontecimientos que los llevarán directamente a la presencia de este mismo Jesús, gobernante soberano de nuestro universo. Los hijos de la luz están todos llamados a verlo, y no debería ser una gran preocupación que vayamos nosotros a él o que sea él quien venga primeramente a nosotros. Estad, pues, siempre listos para recibirlo en la tierra, así como él lo está siempre para recibiros en el cielo. Aguardamos confiados su gloriosa aparición, incluso su venida repetidas veces, pero completamente ignoramos cómo, cuándo o en relación a qué hecho está destinado a aparecer.
