\upaper{159}{Viaje por la Decápolis}
\author{Comisión de seres intermedios}
\vs p159 0:1 Cuando Jesús y los doce llegaron al parque de Magadán, vieron que los esperaban un grupo de casi cien evangelistas y discípulos, entre los que estaban el colectivo de mujeres, listos ya para comenzar de inmediato el viaje de enseñanza y predicación por las ciudades de la Decápolis.
\vs p159 0:2 Aquel jueves, 18 de agosto por la mañana, el Maestro reunió a sus seguidores y dio instrucciones para que cada uno de los apóstoles se uniera a uno de los doce evangelistas, y que salieran con los otros evangelistas en doce grupos para trabajar en las ciudades y aldeas de la Decápolis. Dispuso que el colectivo de mujeres y los otros discípulos se quedaran con él. Jesús fijó una duración de cuatro semanas para este viaje, indicando a sus seguidores a que regresaran a Magadán no más tarde del viernes, 16 de septiembre. Prometió que durante este periodo los visitaría a menudo. En el trascurso de este mes, los doce grupos ejercieron su labor en Gérasa, Gamala, Hipos, Zafón, Gadara, Abila, Edrei, Filadelfia, Hesbón, Dium, Escitópolis y muchas otras ciudades. A lo largo de todo el viaje, no se produjo ninguna curación milagrosa ni ningún otro acontecimiento de carácter extraordinario.
\usection{1. EL SERMÓN SOBRE EL PERDÓN}
\vs p159 1:1 Un día en Hipos, avanzada la tarde, Jesús impartió una lección sobre el perdón en respuesta a la pregunta de uno de los discípulos. El Maestro dijo:
\vs p159 1:2 \pc “Si un hombre de buen corazón tiene cien ovejas y se descarría una de ellas, ¿no deja de inmediato a las noventa y nueve y va a buscar a la que se ha descarriado? Y si es un buen pastor, ¿no va tras la oveja que perdió hasta encontrarla? Y, cuando la encuentra, la pone sobre sus hombros y, gozoso, de camino a su casa, llama a sus amigos y vecinos, y les dice: ‘Alegraos conmigo, porque he encontrado a mi oveja perdida’. Os digo que habrá más gozo en el cielo por un pecador que se arrepiente, que por noventa y nueve justos que no necesitan de arrepentimiento. De igual modo, no es la voluntad de mi Padre que está en los cielos que se pierda uno de estos pequeños, y mucho menos que perezca. En vuestra religión, Dios podrá recibir a los pecadores arrepentidos; en el evangelio del reino, el Padre va a buscarlos incluso antes de que hayan seriamente pensado en arrepentirse.
\vs p159 1:3 “El Padre de los cielos ama a sus hijos y, así pues, vosotros debéis aprender a amaros unos a otros; el Padre de los cielos perdona vuestros pecados y, por ello, debéis aprender a perdonaros unos a otros. Si tu hermano peca contra ti, ve a él y, con tacto y paciencia, muéstrale su falta. Y hazlo a solas con él. Y si te escucha, has ganado a tu hermano. Pero si tu hermano no te oye, si continúa pecando, ve de nuevo a él, lleva contigo a uno o dos mutuos amigos, para tener a uno o incluso dos testigos que confirmen tu testimonio y comprueben que has tratado con justicia y misericordia a este hermano que te ofendió. Ahora bien, si se niega a oír a vuestros hermanos, podrás contar toda la historia a la comunidad, y si se niega a oír a la hermandad, que sea el grupo el que tome las medidas que crea convenientes, y que este miembro rebelde se convierta en un marginado del reino. Aunque no puedas pretender juzgar las almas de vuestros semejantes y, aunque no podáis perdonar pecados o pretender tampoco atribuiros las competencias de los supervisores de las multitudes celestiales, se os ha entregado a vuestras manos, al mismo tiempo, el mantenimiento del orden temporal del reino en la tierra. Aunque no podáis inmiscuiros en los decretos divinos en cuanto a la vida eterna, debéis resolver las cuestiones de la conducta pertinentes al bienestar temporal de la hermandad en la tierra. Y así, en todos estos asuntos referentes a la disciplina de la hermandad, lo que dispongáis en la tierra se reconocerá en el cielo. Aunque no podáis decidir el destino eterno de la persona, sí podréis legislar en relación al proceder del grupo, porque, cuando dos o tres de vosotros estéis de acuerdo sobre algunas de estas cosas y me las pidáis, esas se harán por vosotros, siempre y cuando vuestra petición no contradiga la voluntad de mi Padre del cielo. Y todo esto es por siempre verdad, porque donde están dos o tres congregados en mi nombre, allí estoy yo en medio de ellos”.
\vs p159 1:4 \pc Simón Pedro era el apóstol que estaba al frente de los trabajadores en Hipos y, cuando oyó hablar a Jesús, preguntó: “Señor, ¿cuántas veces perdonaré a un hermano que peque contra mí? ¿Hasta siete?”. Jesús le respondió a Pedro: “No te digo hasta siete, sino aun setenta veces y siete. Por lo cual, el reino de los cielos es semejante a un rey que quiso hacer cuentas con sus mayordomos de palacio. Cuando comenzaron a hacerlas, trajeron ante su presencia a uno de sus sirvientes principales que confesó que debía a su rey diez mil talentos. Este funcionario de la corte del rey alegó que había pasado por tiempos difíciles, y que no tenía con qué pagar esta deuda. Así pues, el rey ordenó que se le confiscaran sus propiedades y que se vendieran a sus hijos para poder pagarla. Al escuchar este mayordomo principal tan severo decreto, se postró sobre su rostro ante el rey y le suplicó que tuviera misericordia y que le concediera más tiempo, diciendo, ‘Señor, ten algo más de paciencia conmigo, y yo te lo pagaré todo’. Y cuando el rey vio a este negligente siervo y a su familia, movido a la compasión, ordenó que lo soltaran y que se le perdonase enteramente el préstamo.
\vs p159 1:5 “Y este mayordomo principal, habiendo recibido, pues, misericordia y perdón de manos del rey, fue a ocuparse de sus asuntos, y encontrándose con uno de sus subordinados, que le debía meramente cien denarios, agarrándolo por el cuello, lo ahogaba y le dijo: ‘págame lo que me debes’. Y entonces su consiervo, postrándose ante él, le rogó diciendo: ‘ten paciencia conmigo, y pronto podré pagarte’. Pero el mayordomo principal no quiso mostrarse misericordioso con su compañero, sino que lo echó en la cárcel hasta que pagara la deuda. Viendo sus consiervos lo que pasaba, se entristecieron tanto que fueron y refirieron todo lo que había pasado a su señor y amo, al rey. Cuando este oyó el comportamiento de su mayordomo principal, hizo llamar ante él a este hombre ingrato y despiadado y le dijo: ‘eres un mayordomo malvado e indigno, te perdoné toda tu deuda sin condiciones porque me rogaste compasión. ¿Por qué no mostraste también misericordia con tu consiervo, como yo tuve misericordia de ti?’. Y el rey se enojó tanto que lo entregó a los carceleros para que lo encerraran hasta que pagara todo lo que debía. Y así también mi Padre celestial muestra una mayor abundancia de misericordia a los que muestren, generosamente, misericordia con sus semejantes. ¿Cómo podéis pedirle a Dios que tenga consideración por vuestros defectos, si estáis habituados a castigar a vuestros hermanos por ser culpables de vuestras mismas fragilidades humanas? Yo os digo a todos: de gracia recibisteis las buenas cosas del reino, dad pues de gracia a vuestros semejantes en la tierra”.
\vs p159 1:6 \pc De esa manera, ilustró Jesús los peligros y la injusticia de emitir un juicio personal sobre los semejantes. Se debe mantener la disciplina y administrar la justicia, pero en todos estos asuntos debe imperar la sabiduría de la hermandad. Jesús invistió de autoridad legislativa y judicial al \bibemph{grupo,} no al \bibemph{individuo}. Tampoco debe ejercerse esta autoridad de forma personal. Existe siempre el peligro de que el veredicto individual pueda verse sesgado por el prejuicio o distorsionado por la pasión. El juicio colectivo tiene mayor probabilidad de suprimir los peligros y eliminar las injusticias de la parcialidad personal. Jesús siempre procuró reducir al mínimo los componentes de la injusticia, la represalia y la venganza.
\vs p159 1:7 \pc [El empleo del término setenta y siete como ilustración de la misericordia y de la indulgencia se tomó de las Escrituras en referencia al júbilo de Lamec por las armas de metal de su hijo Tubal\hyp{}Caín, que, comparando estos superiores instrumentos con los de sus enemigos, exclamó: “Si Caín, con ningún arma en sus manos, fue vengado siete veces, yo seré ahora vengado setenta y siete”.]
\usection{2. EL PREDICADOR DESCONOCIDO}
\vs p159 2:1 Jesús fue a Gamala a visitar a Juan y a quienes trabajaban con él allí. Aquella noche, tras la sesión de preguntas y respuestas, Juan le dijo a Jesús: “Maestro, ayer fui a Astarot para ver a un hombre que enseñaba en tu nombre e incluso declaraba que podía echar fuera demonios. Si bien, él nunca ha estado con nosotros, ni nos sigue, y le prohibí hacer dichas cosas”. Entonces dijo Jesús: “No se lo prohíbas. ¿Es que no comprendes que este evangelio del reino se proclamará dentro de poco en todo el mundo? ¿Cómo esperas que todos los que crean en el evangelio se sometan a tu dirección? Regocíjate de que nuestras enseñanzas hayan empezado ya a manifestarse fuera de los límites de nuestra influencia personal. ¿Juan, es que no ves que quienes profesan hacer grandes obras en mi nombre acabarán por apoyar nuestra causa? De cierto será tardo en hablar mal de mí. Hijo mío, en cuestiones de esta naturaleza, sería mejor tener en cuenta que el que no está con nosotros, está contra nosotros. En las generaciones venideras, muchos, sin ser completamente dignos, harán numerosas cosas extrañas en mi nombre, pero yo no lo prohibiré. Te digo que incluso cuando se dé un vaso de agua fresca en mi nombre a un alma sedienta, los mensajeros del Padre dejarán siempre constancia de ese servicio de amor”.
\vs p159 2:2 Estas palabras desconcertaron bastante a Juan. ¿Es que no había oído al Maestro decir: “El que no está conmigo, está contra mí?”. Y es que no entendía que, en este caso, Jesús se estaba refiriendo a la relación personal del hombre con las enseñanzas espirituales del reino, mientras que, en el otro, lo hacía a las relaciones sociales externas de los creyentes en cuanto a la gestión y jurisdicción de un grupo de creyentes sobre la labor de otros grupos, que con el tiempo conformarían, a escala mundial, la hermandad futura.
\vs p159 2:3 Pero Juan, a menudo, narró este hecho en relación a su labor posterior en favor del reino. No obstante, los apóstoles se sintieron repetidas veces ofendidos con los que se atrevían a enseñar en nombre del Maestro. Siempre les pareció impropio que quienes nunca se habían sentado a los pies de Jesús osaran enseñar en su nombre.
\vs p159 2:4 Este hombre a quien Juan le prohibió enseñar y trabajar en nombre de Jesús, no prestó atención al mandato del apóstol. Siguió adelante con su labor y congregó a un número elevado de creyentes en Canatá, antes de continuar a Mesopotamia. Adén, así era su nombre, había llegado a creer en Jesús gracias al testimonio del demente que Jesús sanó cerca de Queresa y que, con tanta seguridad, había creído que los supuestos espíritus malignos que el Maestro había echado fuera de él, habían entrado en el hato de cerdos y los habían precipitado por un despeñadero provocándoles la muerte.
\usection{3. INSTRUCCIONES PARA MAESTROS Y CREYENTES}
\vs p159 3:1 Jesús estuvo en Edrei, lugar de trabajo de Tomás y de sus compañeros, un día y una noche y, a última hora de la tarde, en el transcurso de la conversación, dio a conocer los principios que debían guiar a quienes predican la verdad y servir de estímulo a quienes enseñan el evangelio del reino. Resumidos y reformulados en términos modernos, Jesús impartió las siguientes enseñanzas:
\vs p159 3:2 \pc Respetad siempre la persona del hombre. Jamás debe fomentarse ninguna causa justa por la fuerza; las victorias espirituales solo se ganan con el poder espiritual. Este mandato de no usar medios materiales alude tanto a la fuerza psíquica como a la física. No se deben emplear argumentos avasallantes ni superioridad mental para coaccionar a hombres y mujeres a que entren en el reino. No se debe arrollar la mente del hombre con el mero peso de la lógica ni intimidarla con la astucia de palabras elocuentes. Aunque no se debe eliminar por completo de las decisiones humanas el componente emocional, no se debe recurrir a este de forma directa en las enseñanzas de quienes quieren avanzar la causa del reino. Apelad en su lugar al espíritu divino que habita en las mentes de los hombres. No instéis al temor, a la piedad o al mero sentimiento. Al apelar a los hombres, sed justos; tened dominio de vosotros mismos y exhibid la moderación debida; demostrad hacia las personas de vuestros alumnos el adecuado respeto. Recordad lo que os he dicho: “He aquí que estoy a la puerta y llamo, y si alguno abre la puerta, entraré”.
\vs p159 3:3 Cuando traigáis a alguien al reino, no menoscabéis ni destruyáis su autoestima. Aunque un exceso de autoestima puede socavar la humildad personal y acabar en orgullo, vanagloria y arrogancia, la pérdida de la autoestima termina paralizando la voluntad. El propósito de este evangelio es restaurar el respeto de uno mismo en aquellos que lo han perdido y contenerlo en los que lo tienen. No cometáis el error de condenar solo las equivocaciones cometidas en las vidas de vuestros alumnos; acordaros de reconocer con generosidad lo más meritorio de sus vidas. No olvidéis que nada me detendrá para que aquellos que han perdido el respeto por sí mismos y que realmente desean recuperarlo lo recuperen.
\vs p159 3:4 Cuidaros de no herir la autoestima de esas almas tímidas y medrosas. No empleéis el sarcasmo en detrimento de esos hermanos míos simples de mente. No seáis cínicos con esos temerosos hijos míos. El ocio es pernicioso para el respeto de uno mismo; así, pues, alentad a vuestros hermanos a que se mantengan ocupados en la tarea que elijan, y haced todo lo posible para garantizar el trabajo a los desempleados.
\vs p159 3:5 No seáis nunca culpables de emplear tácticas indignas de intimidación para que hombres y mujeres entren en el reino. Un padre amoroso no infunde miedo en sus hijos para que se sometan a sus justas exigencias.
\vs p159 3:6 En algún momento, los hijos del reino serán conscientes de que el hecho de experimentar fuertes emociones no es señal de la guía del espíritu divino. Ni sentirse intensa y extrañamente impulsado a hacer algo o a ir a algún determinado lugar significa necesariamente que dichos impulsos sean reflejo de las directrices del espíritu morador.
\vs p159 3:7 Prevenid a todos los creyentes de los conflictos que tendrán que vencer al pasar de una vida tal como se vive en la carne a otra vida, de orden superior, tal como se vive en el espíritu. Para aquellos que vivan íntegramente en cualquiera de los dos entornos habrá poco conflicto o confusión, pero, en los momentos de transición entre estos dos niveles de vida, todos están abocados a experimentar mayor o menor grado de incertidumbre. Al entrar al reino, no podréis eludir sus responsabilidades ni evitar sus obligaciones, pero recordad: el yugo del evangelio es fácil y la carga de la verdad es ligera.
\vs p159 3:8 El mundo está lleno de almas que se mueren de hambre en presencia misma del pan de vida; los hombres mueren buscando al mismo Dios que vive en ellos. Y buscan los tesoros del reino con los corazones anhelantes y los pies cansados cuando tienen la fe viva a su alcance inmediato. La fe es para la religión lo que las velas son para un barco: aporta poder; no es una carga añadida a la vida. No hay sino una lucha para los que entran en el reino, y es la de pelear la buena batalla de la fe. El creyente tiene solo que librar una contienda, y es contra la duda ---la increencia---.
\vs p159 3:9 Al predicar el evangelio del reino, estáis simplemente enseñando amistad con Dios. Y esta fraternidad apela por igual a hombres y mujeres por el hecho de que ambos encontrarán en ella lo que verdaderamente más satisfaga sus aspiraciones e ideales particulares. Decid a mis hijos que si bien soy sensible a sus sentimientos y paciente con sus flaquezas, también soy inflexible con el pecado e intolerante con la iniquidad. En la misma medida en la que soy manso y humilde en la presencia de mi Padre, soy perennemente inexorable con la maleficencia deliberada y el pecado de rebelión contra la voluntad de mi Padre en el cielo.
\vs p159 3:10 No describáis a vuestro maestro como un hombre sufrido. Las futuras generaciones conocerán también el esplendor de nuestro gozo, el optimismo de nuestra buena voluntad y la inspiración de nuestro buen humor. Proclamamos un mensaje de buenas nuevas con un contagioso poder transformador. Nuestra religión palpita con nueva vida y nuevos significados. Quienes aceptan esta enseñanza se llenan de gozo y sus corazones se desbordan con el deseo de regocijarse para siempre. Todos los que tienen certeza de Dios vivencian constante y crecientemente la felicidad.
\vs p159 3:11 Enseñad a todos los creyentes a evitar el falso sentimiento de la conmiseración. No podéis desarrollar caracteres fuertes si sentís compasión por vosotros mismos; tratad honestamente de no caer en la engañosa tendencia de compartir vuestras desgracias con los demás. Otorgad vuestra solidaridad a los valientes e intrépidos, absteniéndoos, al mismo tiempo, de obsequiar una excesiva piedad a esas almas pusilánimes que se enfrentan a medias a las pruebas de la vida. No ofrezcáis consuelo a quienes flaquean ante sus problemas sin luchar. No os solidaricéis con vuestros semejantes por el mero hecho de que ellos se van a solidarizar con vosotros.
\vs p159 3:12 \pc Cuando mis hijos tomen conciencia de la certidumbre de la presencia divina, esa fe ampliará su mente, ennoblecerá su alma, fortalecerá su persona, incrementará su felicidad, intensificará su percepción espiritual y acrecentará su capacidad para amar y ser amados.
\vs p159 3:13 Enseñad a todos los creyentes que quienes entran en el reino no se vuelven por ello inmunes a los accidentes del tiempo ni a los desastres ordinarios de la naturaleza. Creer en el evangelio no impedirá que os veáis en problemas, pero sí garantizará que \bibemph{no tendréis miedo} cuando estos os acucien. Si os atrevéis a creer en mí y a continuar siguiéndome incondicionalmente, al hacerlo, sin lugar a dudas, entraréis en un sendero inevitablemente dificultoso. No os prometo liberaros de las aguas de la adversidad, pero sí os prometo que iré con vosotros a través de ellas.
\vs p159 3:14 \pc Y Jesús enseñó otras muchas cosas a este grupo de creyentes antes de que se retiraran a dormir aquella noche. Y los que oyeron estos dichos los atesoraron en sus corazones y, con frecuencia, los repitieron para la edificación de los apóstoles y discípulos que no estuvieron presentes cuando los dijo.
\usection{4. LA CHARLA CON NATANAEL}
\vs p159 4:1 Y entonces Jesús se dirigió a Abila, donde trabajaban Natanael y sus compañeros. Natanael estaba bastante preocupado por algunas de las afirmaciones de Jesús que parecían restar legitimidad a las escrituras hebreas, oficialmente reconocidas. Así pues, esa noche, tras el periodo habitual de preguntas y respuestas, Natanael alejó a Jesús de los demás y le preguntó: “Maestro, ¿podrías confiar en mí lo suficiente como para decirme la verdad acerca de las Escrituras? He observado que nos enseñas solo una parte de los escritos sagrados ---la mejor, tal como yo lo veo--- y deduzco que rechazas las enseñanzas de los rabinos en el sentido de que las palabras de la ley son las palabras mismas de Dios, y que han estado con Dios en el cielo incluso antes de los tiempos de Abraham y Moisés. ¿Cuál es la verdad sobre las Escrituras?”. Cuando Jesús oyó la pregunta de su desconcertado apóstol, respondió:
\vs p159 4:2 \pc “Natanael, estas acertado en tu juicio; yo no considero las Escrituras como lo hacen los rabinos. Hablaré contigo de esta cuestión a condición de que no les comentes nada a tus hermanos, ya que no todos están preparados para recibir tal instrucción. Las palabras de la ley de Moisés y las enseñanzas de las Escrituras no existían antes de Abraham. Solo hasta épocas recientes no se llegaron a recopilar tal como las conocemos ahora. Aunque contienen lo mejor de los pensamientos y anhelos más elevados del pueblo judío, también tienen cosas que distan bastante de representar el carácter y las enseñanzas del Padre de los cielos; por ello, entre sus mejores enseñanzas, tengo que elegir aquellas verdades que han de ser eco en el evangelio del reino.
\vs p159 4:3 “Estos escritos son obra de los hombres, algunos de ellos santos, otros, no tanto. Las enseñanzas de estos libros representan los puntos de vista y el grado de iluminación de los tiempos en los que se redactaron. Como revelación de la verdad, los últimos textos son más dignos de confianza que los primeros. Las Escrituras contienen errores y su origen es completamente humano, pero, no te equivoques, puesto que constituyen la mejor recopilación de la sabiduría religiosa y de la verdad espiritual que pueda hallarse, en este momento, en cualquier parte del mundo.
\vs p159 4:4 “Muchos de estos libros no se escribieron por las personas a quienes se les atribuyen, aunque esto, de ninguna manera, disminuye el valor de las verdades que en ellos se recoge. Considerando que la historia de Jonás no es un hecho, incluso si Jonás nunca hubiera existido, la profunda verdad de este relato, el amor de Dios por Nínive y por los denominados paganos no sería menos inestimable a los ojos de todos los que aman a sus semejantes. Las Escrituras son sagradas porque reflejan los pensamientos y los actos de los hombres que buscaban a Dios, y que dejaron constancia en ella de sus nociones más elevadas de la rectitud, la verdad y la santidad. Contienen mucho que es verdad, bastante, pero, a la luz de vuestras enseñanzas actuales, tú sabes que estos escritos dan también una imagen distorsionada del Padre de los cielos, del Dios amoroso que yo he venido a revelar a todos los mundos.
\vs p159 4:5 “Natanael, nunca te permitas creer, ni por un solo momento, lo que las Escrituras dicen respecto al Dios del amor que ordenó a vuestros antepasados salir al campo de batalla para exterminar a todos sus enemigos ---hombres, mujeres y niños---. Estas crónicas son palabras de hombres, hombres no muy santos, no son la palabra de Dios. Las Escrituras son siempre un reflejo, y lo serán siempre, del estatus intelectual, moral y espiritual de quienes las redactan. ¿Es que no has observado que los conceptos de Yahvé crecen en belleza y gloria a medida que los profetas los recogen desde Samuel hasta Isaías? E igualmente debes recordar que el propósito de las Escrituras es servir de instrucción religiosa y de guía espiritual. No son obra ni de historiadores ni de filósofos.
\vs p159 4:6 “Lo más lamentable de todo no consiste meramente en esta idea errónea de la perfección absoluta de los textos de las Escrituras ni en la infalibilidad de sus enseñanzas, sino en la interpretación, confusa y equivocada, que hacen de estos escritos sagrados los escribas y fariseos de Jerusalén, sometidos a la tradición. Y ahora, emplearán tanto la doctrina de la inspiración de las Escrituras como su tergiversación para oponerse a estas nuevas enseñanzas del evangelio del reino. Natanael, no te olvides nunca de que el Padre no limita la revelación de la verdad a una sola generación ni a un solo pueblo. Muchos buscadores fervientes de la verdad se han sentido, y seguirán sintiéndose, confundidos y decepcionados por estas doctrinas de la perfección de las Escrituras.
\vs p159 4:7 “La autoridad de la verdad es el espíritu mismo que habita en aquellos que lo manifiestan en sus vidas, y no las inertes palabras de los hombres menos iluminados y supuestamente inspirados de otra generación. E incluso si esos santos de antaño vivieran unas vidas inspiradas y plenas del espíritu, eso no significa que sus \bibemph{palabras} estuvieran asimismo inspiradas por el espíritu. Hoy en día no dejamos constancia por escrito de las enseñanzas de este evangelio del reino temiendo que, cuando yo me haya ido, os dividáis rápidamente en varios grupos y compitáis por la verdad como resultado de la diversidad de vuestras interpretaciones de mis enseñanzas. Durante esta generación, es mejor que \bibemph{vivamos} estas verdades evitando, al mismo tiempo, ponerlas por escrito.
\vs p159 4:8 “Toma buena nota de mis palabras, Natanael: nada que la naturaleza humana haya tocado puede considerarse infalible. Aunque la verdad divina pueda, de hecho, resplandecer a través de la mente humana, su pureza será siempre relativa y su divinidad parcial. La criatura puede anhelar ser infalible, pero solo los Creadores lo son.
\vs p159 4:9 “Pero el mayor error de las enseñanzas de las Escrituras consiste en la doctrina de que son libros sellados por el misterio y la sabiduría, que solo las mentes sabias de la nación pueden atreverse a interpretar. Las revelaciones de la verdad divina no están selladas salvo por la ignorancia humana, el fanatismo y la intolerancia de la estrechez de miras. Solo el prejuicio y la superstición oscurecen la luz de las Escrituras. El erróneo temor a lo sagrado ha impedido que la religión fuera tutelada por el sentido común. El miedo a la autoridad de los escritos sagrados del pasado impide de hecho que las almas honestas de hoy en día acepten la nueva luz del evangelio, la luz que aquellos mismos hombres conocedores de Dios de otra generación deseaban ver con tanta intensidad.
\vs p159 4:10 “Pero lo más triste de todo esto es el hecho de que algunos de los maestros de la santidad de este tradicionalismo conocen esta misma verdad. Entienden, en mayor o menor grado, estas limitaciones de las Escrituras, pero son cobardes morales, seres intelectualmente deshonestos. Conocen la verdad referente a los escritos sagrados, pero prefieren ocultar estos inquietantes hechos a la gente. Y así, pervierten y distorsionan las Escrituras, convirtiéndolas en la guía de unos detalles serviles de la vida diaria y en una autoridad en las cosas no espirituales, en lugar de apelar a las escrituras sagradas como las depositarias de la sabiduría moral, la inspiración religiosa y las enseñanzas espirituales de los hombres que en otras generaciones conocieron a Dios”.
\vs p159 4:11 \pc Natanael ganó en percepción intelectual, y quedó impresionado, por las afirmaciones del Maestro. Durante mucho tiempo, reflexionó sobre esta charla en las profundidades de su alma, pero no dijo nada a nadie respecto a ella hasta la ascensión de Jesús; e, incluso entonces, tuvo temor de transmitir la historia completa de la enseñanza del Maestro.
\usection{5. LA NATURALEZA POSITIVA DE LA RELIGIÓN DE JESÚS}
\vs p159 5:1 En Filadelfia, donde Santiago trabajaba, Jesús enseñó a los discípulos sobre la naturaleza positiva del evangelio del reino. Cuando, en el transcurso de sus comentarios, dio a entender que algunas partes de las Escrituras contenían más verdades que otras e instó a sus oyentes a que alimentaran su alma con el mejor de los alimentos espirituales, Santiago interrumpió al Maestro y le preguntó: “¿Tendrías la amabilidad de sugerirnos cómo podremos seleccionar los mejores pasajes de las Escrituras para nuestra edificación personal?”. Y Jesús respondió: “Sí, Santiago, cuando leáis las Escrituras, buscad aquellas enseñanzas que sean eternamente verdaderas y divinamente bellas, como las siguientes:
\vs p159 5:2 “Crea en mi, Oh Señor, un corazón limpio.
\vs p159 5:3 \pc “El Señor es mi pastor; nada me faltará.
\vs p159 5:4 \pc “Ama a tu prójimo como a ti mismo.
\vs p159 5:5 \pc “Porque yo el Señor soy tu Dios, quien te sostiene de tu mano derecha y te dice: ‘No temas, yo te ayudo’”.
\vs p159 5:6 \pc “Ni se adiestrarán más las naciones para la guerra”.
\vs p159 5:7 \pc Y esto ilustra la forma en la que Jesús, cada día, extraía lo más selecto de las Escrituras hebreas para instruir a sus seguidores y para integrarlo en las enseñanzas del nuevo evangelio del reino. Otras religiones habían planteado la idea de la proximidad de Dios al hombre, pero Jesús demostró que el cuidado que Dios impartía al hombre era el de un padre amoroso por el bienestar de unos hijos suyos que confían en él, e hizo entonces de esta enseñanza el fundamento de su religión. Y, por ello, la doctrina de la paternidad de Dios hizo imperativa la práctica de la hermandad de los hombres. La adoración de Dios y el servicio del hombre se convirtieron en la totalidad de su religión. Jesús tomó lo mejor de la religión judía y lo elevó a una meritoria posición en las nuevas enseñanzas del evangelio del reino.
\vs p159 5:8 Jesús implantó el espíritu de la acción positiva en las doctrinas pasivas de la religión judía. En lugar de la conformidad negativa a las exigencias ceremoniales, Jesús encomendó la actuación positiva de lo que su nueva religión requería de quienes la aceptaban. La religión de Jesús consistía no sencillamente en \bibemph{creer} en esas cosas que el evangelio requería, sino en \bibemph{hacerlas} realmente. No impartió la enseñanza de que la esencia de su religión estribaba en el servicio social, sino, más bien, que el servicio social era uno de los innegables efectos de estar en posesión del espíritu de la verdadera religión.
\vs p159 5:9 Jesús no dudaba en tomar la mejor mitad de los textos de las Escrituras, mientras rechazaba aquellos fragmentos que contuviesen menos verdad. Su gran exhortación, “Ama a tu prójimo como a ti mismo”, lo tomó del versículo en el que se lee: “No te vengarás de los hijos de tu pueblo, sino amarás a tu prójimo como a ti mismo”. Jesús extrajo la parte positiva de este texto mientras declinó hacer uso de la negativa. Se oponía incluso a la no confrontación negativa o meramente pasiva. Dijo: “Cuando un enemigo te hiera en una mejilla, no te quedes allí callado y pasivo, sino que, con una actitud positiva, vuélvele la otra mejilla, esto es, haz activamente lo que sea mejor para alejar a tu hermano errado de los senderos del mal y llevarlo a los mejores caminos de la vida recta”. Jesús instaba a sus seguidores a reaccionar de forma positiva y dinámica ante cualquier situación de la vida. El acto de volver la otra mejilla, o lo que dicha acción puede representar, exige iniciativa, precisa que la persona del creyente se exprese de forma enérgica, activa y valerosa.
\vs p159 5:10 Jesús no abogaba por someterse negativamente a las indignidades de aquellos que las imponían, intencionadamente, a quienes practicaban la no confrontación del mal, sino que proponía a sus seguidores que fueran sensatos y estuvieran alertas para reaccionar con rapidez y positividad al mal con el bien, y poder así realmente vencerlo. No olvidéis que el verdadero bien es invariablemente más poderoso que el mal más malicioso. El Maestro definió así el criterio positivo de la rectitud: “Si alguno quiere ser mi discípulo y venir en pos de mí, niéguese a sí mismo y se ocupe por completo de sus responsabilidades diarias”. Y él mismo vivió de forma que “anduvo haciendo bienes”. Y este aspecto del evangelio quedaría bien ilustrado por las numerosas parábolas que contaría más tarde a sus seguidores. Nunca recomendó a sus seguidores que soportaran con paciencia sus obligaciones, sino que cumplieran al máximo, con energía y entusiasmo, sus responsabilidades humanas y privilegios divinos por formar parte del reino de Dios.
\vs p159 5:11 Cuando Jesús formó a sus apóstoles para que si alguien les quitaba injustamente la capa, que le dieran también la otra vestimenta, no se refería literalmente tanto a una segunda capa como a la idea de hacer algo \bibemph{positivo} para salvar al malhechor, en lugar del antiguo consejo a la represalia ---”ojo por ojo”--- y todo lo que esto conllevaba. Jesús aborrecía la idea de la represalia o la de convertirse, simplemente, en un sufridor pasivo o en una víctima de la injusticia. En esta ocasión, les enseñó tres formas de enfrentarse, y resistirse, al mal:
\vs p159 5:12 \li{1.}Devolver el mal por el mal: el método positivo, pero injusto.
\vs p159 5:13 \li{2.}Sufrir el mal sin queja y sin confrontación: el método estrictamente negativo.
\vs p159 5:14 \li{3.}Devolver bien por mal, imponer la voluntad para convertirse en el dueño de la situación, para conquistar el mal con el bien: el método positivo y justo.
\vs p159 5:15 \pc Cierta vez, uno de los apóstoles le preguntó: “Maestro, ¿qué debo hacer si un extraño me obliga a llevar su carga por una milla?”. Jesús respondió: “No te sientes y suspires de alivio mientras riñes entre dientes al extraño. La rectitud no viene de estas actitudes pasivas. Si no se te ocurre nada más que hacer realmente positivo, puedes al menos llevar la carga una segunda milla. De cierto que eso servirá de reto al extraño injusto e impío”.
\vs p159 5:16 Los judíos sabían de un Dios que perdonaba a los pecadores arrepentidos y que trataba de olvidar sus fechorías, pero, hasta que vino Jesús, los hombres nunca habían tenido noticia de un Dios que iba en pos de las ovejas perdidas, que emprendía la búsqueda de los pecadores, que se regocijara cuando descubría que deseaban volver a la casa del Padre. Jesús expandió este rasgo positivo de la religión hasta incluirlo también en sus oraciones. Y convirtió la regla de oro negativa en una exhortación positiva que mostraba la equidad humana.
\vs p159 5:17 En sus enseñanzas, Jesús evitaba, invariablemente, dar detalles que resultaran de distracción. Rehuía el lenguaje artificioso y las meras imágenes poéticas o los juegos de palabras. De forma habitual, expresaba profundos significados con palabras sencillas. Con el objeto de presentar sus ilustraciones, Jesús expandía el significado ordinario de numerosos términos, tales como sal, levadura, pesca y niños pequeños. Hacía un uso muy efectivo de la antítesis, como cuando comparaba lo muy pequeño con lo infinito y otros muchos casos. Sus ejemplos eran sorprendentes, como cuando aludió “al ciego guiando al ciego”. Pero la virtud del uso de sus ilustraciones en sus enseñanzas residía en su naturalidad. Jesús trajo, desde el cielo a la tierra, la filosofía de la religión. Describió las necesidades básicas del alma otorgándole una nueva percepción y afecto.
\usection{6. REGRESO A MAGADÁN}
\vs p159 6:1 La misión en la Decápolis, de cuatro semanas de duración, resultó moderadamente exitosa. Se recibieron cientos de almas en el reino, y los apóstoles y evangelistas adquirieron una valiosa experiencia al tener que desarrollar su labor sin la inspiración de la inmediata presencia personal de Jesús.
\vs p159 6:2 El viernes, 16 de septiembre, todo el colectivo de trabajadores se reunió por previo acuerdo en el parque de Magadán. El día de \bibemph{sabbat} se celebró un consejo al que asistieron más de cien creyentes con el fin de examinar con detenimiento los planes futuros para la extensión de la labor del reino. Los mensajeros de David acudieron también y presentaron sus informes sobre los creyentes que había en Judea, Samaria, Galilea y en las comarcas limítrofes.
\vs p159 6:3 En aquel momento, pocos de los seguidores de Jesús valoraban suficientemente la gran utilidad de los servicios realizados por el grupo de mensajeros. Pero estos no solo mantenían a los creyentes en contacto entre sí y con Jesús y los apóstoles por toda Palestina, sino que durante estos días aciagos también recolectaban fondos para el sustento de Jesús y sus acompañantes al igual que para dar asistencia a las familias de los doce apóstoles y de los doce evangelistas.
\vs p159 6:4 Sobre esta época, Abner trasladó su centro de actividad de Hebrón a Belén. En este último lugar estaba igualmente la sede en Judea de los mensajeros de David, que mantenía un servicio nocturno de relevo de mensajeros entre Jerusalén y Betsaida. Cada día, estos corredores salían de Jerusalén a última hora de la tarde, se relevaban en Sicar y Escitópolis y llegaban a Betsaida a la mañana siguiente, a la hora del desayuno.
\vs p159 6:5 Jesús y sus acompañantes se prepararon entonces para tomar una semana de descanso antes de disponerse a emprender la postrera etapa de su labor por el reino. Aquel sería el último periodo de descanso que tendrían, puesto que la misión de Perea se convertiría en una campaña de predicación y enseñanza que continuó hasta el momento de llegar a Jerusalén junto al desarrollo de los episodios finales en la andadura de Jesús en la tierra.
