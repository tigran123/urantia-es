\upaper{163}{La ordenación de los setenta en Magadán}
\author{Comisión de seres intermedios}
\vs p163 0:1 Escasos días después de que Jesús y los doce regresaran a Magadán desde Jerusalén, llegaron, procedentes de Belén, Abner y un grupo de unos cincuenta discípulos. En ese momento, estaban congregados en el campamento de Magadán el colectivo de evangelistas, el de mujeres y unos ciento cincuenta discípulos de todas partes de Palestina, también verdaderos y probados. Después de dedicar unos pocos días a conversar con sus seguidores y a reorganizar el campamento, Jesús y los doce comenzaron a impartir un curso de formación intensiva a este grupo especial de creyentes. De este grupo de discípulos, bien formados y experimentados, Jesús elegiría posteriormente a los setenta maestros para enviarlos a proclamar el evangelio del reino. Esta instrucción comenzó con regularidad el viernes 4 de noviembre y prosiguió hasta el sábado 19 de noviembre.
\vs p163 0:2 Cada mañana, el Maestro daba una charla a este conjunto de creyentes. Pedro les enseñaba métodos de predicación pública; Natanael les instruía en el arte de la enseñanza; Tomás les explicaba cómo responder a las preguntas; mientras que Mateo se encargaba de gestionar los asuntos económicos del grupo. Los demás apóstoles también participaron en esta labor docente, según su experiencia particular y talentos naturales.
\usection{1. LA ORDENACIÓN DE LOS SETENTA}
\vs p163 1:1 El día del \bibemph{sabbat,} 19 de noviembre por la tarde, en este campamento de Magadán, Jesús ordenó a los setenta, y se puso a Abner al frente de estos predicadores y maestros del evangelio. Este colectivo de setenta maestros estaba formado por Abner y diez de los antiguos apóstoles de Juan, cincuenta y uno de los primeros evangelistas y otros ocho discípulos que se habían distinguido por su servicio al reino.
\vs p163 1:2 Sobre las dos de la tarde de ese \bibemph{sabbat,} entre aguacero y aguacero, un grupo de creyentes, que había aumentado a más de cuatrocientos por la llegada de David y de la mayoría del cuerpo de mensajeros, se congregó en la orilla del Lago de Galilea para presenciar la ordenación de los setenta.
\vs p163 1:3 Antes de imponer sus manos sobre las cabezas de cada uno de los setenta para consagrarlos como mensajeros del evangelio, Jesús les dirigió estas palabras: “La mies a la verdad es mucha, pero los obreros son pocos; rogad, pues, al Señor de la mies, que envíe a otros obreros más a su mies. Me dispongo a designaros como mensajeros del reino y a enviaros a judíos y gentiles como corderos en medio de lobos. Cuando toméis vuestros caminos, de dos en dos, no llevéis alforja ni ropa de más, porque salís a esta primera misión por un breve período de tiempo. No saludéis a nadie por el camino, limitaos a hacer vuestra labor. Cuando vayáis a alojaros en alguna casa, primeramente decid: “Paz sea a esta casa. Si quienes aman la paz viven allí, permaneced en ella; y si no, partid. Y quedaos en esa misma casa durante vuestra estancia en aquella ciudad, comiendo y bebiendo lo que os pongan por delante. Y haced esto porque el obrero es digno de su sustento. No os paséis de casa en casa porque os ofrezcan mejor alojamiento. Recordad: cuando andéis proclamando paz sobre la tierra y buena voluntad entre los hombres, deberéis luchar con enemigos implacables, que se engañan a sí mismos. Sed, pues, astutos como serpientes pero inocentes como palomas.
\vs p163 1:4 “Y adonde vayáis, predicad diciendo, ‘el reino de Dios se ha acercado’, y cuidad a quien esté enfermo de mente o de cuerpo. De gracia recibisteis las buenas cosas del reino; dad de gracia. Si la gente de alguna ciudad os recibe, hallará entrada amplia en el reino del Padre; pero si se niega a recibir este evangelio, proclamaréis vuestro mensaje al marcharos de esta incrédula comunidad, diciéndoles a esos mismos que rechazan vuestras enseñanzas: ‘Pese a que rechazáis la verdad, es cierto no obstante que el reino de Dios se ha acercado a vosotros”’. El que a vosotros oye, a mi me oye. Y el que me oye a mí, oye a Aquel que me envió. Quien rechace vuestro mensaje evangélico, a mí me rechaza. Y quien a mí me rechaza, rechaza a Aquel que me envió”.
\vs p163 1:5 Tras haber hablado estas cosas a los setenta, Jesús, empezando con Abner, impuso sus manos sobre la cabeza de cada uno de aquellos hombres, que estaban allí arrodillados formando un círculo.
\vs p163 1:6 Temprano en la mañana del día siguiente, Abner envió a los setenta mensajeros a todas las ciudades de Galilea, Samaria y Judea. Y estas treinta y cinco parejas salieron a predicar y a enseñar durante unas seis semanas, y todos ellos regresaron al nuevo campamento, asentado cerca de Pella, en Perea, el viernes 30 de diciembre.
\usection{2. EL JOVEN RICO Y OTROS DISCÍPULOS}
\vs p163 2:1 La comisión designada por Jesús para seleccionar a los aspirantes a los setenta rechazó a más de cincuenta discípulos que se habían ofrecido para ordenarse y convertirse en miembros de este grupo. La comisión la formaban Andrés, Abner y el jefe en funciones del colectivo de evangelistas. En todos los casos en los que esta comisión de tres miembros no alcanzaba un acuerdo unánime, llevaban al candidato ante Jesús y, aunque el Maestro no rechazó a ninguno de los que aspiraban a ser ordenados mensajeros del evangelio, hubo más de una docena que, cuando hablaban con él, perdían sus deseos de ser mensajeros del evangelio.
\vs p163 2:2 \pc Un ferviente discípulo vino a Jesús, diciendo: “Maestro, quiero ser uno de los nuevos apóstoles, pero mi padre es muy anciano y está al borde de la muerte; ¿se me permitiría regresar a mi casa para enterrarlo?”. Jesús le dijo a este hombre: “Hijo mío, las zorras tienen guaridas y las aves de los cielos nidos, pero el Hijo del hombre no tiene donde recostar la cabeza. Tú eres un discípulo fiel, y puedes continuar como tal al regresar a tu casa y cuidar a tus seres queridos, pero no es así con los mensajeros de mi evangelio. Ellos han renunciado a todo para seguirme y proclamar el reino. Si quieres ser ordenado como maestro, deja que otros entierren a los muertos, pero tú vete a anunciar la buena nueva”. Y este hombre se marchó muy desilusionado.
\vs p163 2:3 Otro discípulo vino al Maestro y dijo: “Quisiera ser ordenado mensajero, pero me gustaría ir a mi casa durante algún tiempo para consolar a mi familia”. Y Jesús le respondió: “Si quieres ser ordenado, debes estar dispuesto a dejarlo todo. Los mensajeros del evangelio no pueden tener sentimientos divididos. Ninguno que, habiendo puesto su mano en el arado mira hacia atrás, es digno de convertirse en mensajero del reino”.
\vs p163 2:4 \pc Entonces Andrés llevó hasta Jesús a un joven rico, un creyente devoto que deseaba ordenarse. Este joven, Matadormo, era miembro del sanedrín de Jerusalén; había oído enseñar a Jesús y, posteriormente, Pedro y los otros apóstoles lo habían formado en el evangelio del reino. Jesús informó a Matadormo de los requisitos necesarios para ser ordenado y le pidió que postergara su decisión hasta haberlo pensado más detenidamente. Temprano, en la mañana siguiente, cuando Jesús iba a dar un paseo, este joven lo abordó y le dijo: “Maestro, quisiera conocer de ti las certezas de la vida eterna. Ya que he cumplido todos los mandamientos desde mi juventud, desearía saber, ¿qué más debo hacer para lograr la vida eterna?”. Respondiendo a esta pregunta, Jesús le dijo: “Si guardas todos los mandamientos ---no adulterarás, no matarás, no hurtarás, no dirás falsos testimonios, no defraudarás, honrarás a tu padre y a tu madre--- haces bien, pero la salvación es la recompensa de la fe, no meramente de las obras. ¿Crees en este evangelio del reino?”. Y Matadormo contestó: “Sí, Maestro, de cierto creo todo lo que tú y tus apóstoles me habéis enseñado”. Y Jesús dijo: “Entonces, eres en verdad mi discípulo y un hijo del reino”.
\vs p163 2:5 Después, dijo el joven: “Pero, Maestro, no quiero resignarme a ser tu discípulo; desearía ser uno de tus nuevos mensajeros”. Cuando Jesús oyó esto, lo miró con mucho cariño y le dijo: “Te haré uno de mis mensajeros si estás dispuesto a pagar el precio debido y si haces una cosa que te falta”. Matadormo respondió: “Maestro, haré cualquier cosa si se me permite seguirte”. Jesús, besando en la frente al joven arrodillado ante él, añadió: “Si quieres ser mi mensajero, anda, vende todo lo que tienes y, cuando le hayas dado lo recaudado a los pobres o a tus hermanos, ven y sígueme, y tendrás tesoros en el reino de los cielos”.
\vs p163 2:6 Cuando Matadormo oyó esto, decayó su semblante. Se levantó y se fue triste porque tenía muchas posesiones. Este rico y joven fariseo se había criado creyendo que la riqueza era una señal del favor de Dios. Jesús sabía que lo ataba el amor a sí mismo y a sus riquezas. El Maestro quería liberarlo del \bibemph{amor} a la riqueza, no necesariamente de la propia riqueza. Aunque los discípulos de Jesús no se desprendían de todos sus bienes terrenales, los apóstoles y los setenta sí lo hacían. Matadormo pretendía ser uno de los setenta nuevos mensajeros y, por esta razón, Jesús le exigió que se deshiciera de todas sus posesiones temporales.
\vs p163 2:7 \pc Casi todos los seres humanos tienen algo acariciado a lo que aferrarse, pero que se exige como precio para ser admitido en el reino de los cielos. Si Matadormo se hubiese desprendido de su riqueza, esta se le habría puesto de nuevo en sus manos para que la administrara como tesorero de los setenta. Aunque más adelante, tras establecerse la Iglesia en Jerusalén, sí obedeció el mandato del Maestro, ya resultó ser demasiado tarde para formar parte de los setenta, y se convirtió en el tesorero de la Iglesia de Jerusalén, de la que estaba al frente Santiago, el hermano del Señor en la carne.
\vs p163 2:8 Así pues, siempre lo ha sido y por siempre lo será: los hombres deben tomar sus propias decisiones. Hay un cierto grado de libertad de elección que los mortales pueden ejercer. Las fuerzas del mundo espiritual no pretenden coaccionar al hombre; le permiten ir por el camino de sus propias opciones.
\vs p163 2:9 Jesús preveía que Matadormo, con sus riquezas, no podría de ninguna manera ordenarse y ser compañero de hombres que habían renunciado a todo por el evangelio; al mismo tiempo, veía que, sin sus riquezas, podría convertirse en su mejor líder. Pero, como los propios hermanos de Jesús, nunca llegaría a ser grande en el reino porque se privó a sí mismo de relacionarse estrecha y personalmente con el Maestro, algo que podría haber experimentado de haber estado dispuesto a hacer en aquel momento lo que Jesús le pedía, y que algunos años más tarde realizaría.
\vs p163 2:10 La riqueza no está directamente relacionada con la entrada en el reino de los cielos, pero \bibemph{el amor por la riqueza sí}. Las lealtades espirituales hacia el reino son irreconciliables con la servidumbre a las riquezas materiales. El hombre no puede ser supremamente leal a un ideal espiritual y entregarse, al mismo tiempo, a lo material.
\vs p163 2:11 Jesús nunca enseñó que era errado poseer riquezas. Solo exigió que los doce y los setenta destinaran todas sus posesiones terrenales a una causa común. Incluso así, aceptó que se obtuvieran beneficios al liquidar sus propiedades, como en el caso del apóstol Mateo. Jesús muchas veces ofreció algunas recomendaciones a sus discípulos acomodados, tal como había hecho con el hombre rico de Roma. El Maestro estimaba que la prudente inversión del exceso de beneficios era una forma legítima de asegurarse contras futuras e inevitables adversidades. Cuando las arcas apostólicas eran sobreabundantes, Judas colocaba fondos en depósito para poder usarlos posteriormente si tenían que afrontar necesidades por la disminución de los ingresos. Judas lo hacía tras consultar con Andrés. Jesús nunca tuvo nada que ver personalmente con la cuentas apostólicas salvo en el desembolso de las limosnas. Pero condenó muchas veces un tipo de abuso económico: la injusta explotación de personas débiles, iletradas y menos afortunadas por parte de semejantes suyos más fuertes, sagaces e inteligentes. Jesús manifestó que ese trato inhumano de hombres, mujeres y niños era incompatible con los ideales de la hermandad del reino de los cielos.
\usection{3. CHARLA SOBRE LA RIQUEZA}
\vs p163 3:1 En el momento en el que Jesús terminó de hablar con Matadormo, Pedro y algunos de los apóstoles ya se habían congregado a su alrededor y, conforme el joven rico se marchaba, Jesús se volvió y, mirando a los apóstoles, les dijo: “¡Veis lo difícil que es para los que tienen riquezas entrar enteramente en el reino de Dios! No se puede compartir la adoración espiritual con el apego a lo material; ninguno puede servir a dos señores. Tenéis el dicho de que ‘es más fácil pasar un camello por el ojo de una aguja, que los paganos hereden la vida eterna’. Y yo os digo que es tan fácil para ese camello pasar por el ojo de la aguja como para estos ricos satisfechos consigo mismos entrar al reino de los cielos”.
\vs p163 3:2 Cuando Pedro y los apóstoles oyeron estas palabras, se asombraron en extremo hasta tal punto que Pedro dijo: “¿Quién pues, Señor, podrá ser salvo? ¿Es que se excluirán del reino a todos los que tienen riquezas?”. Jesús respondió: “No, Pedro, pero todos los que depositen su confianza en las riquezas difícilmente podrán participar de la vida espiritual que lleva al progreso eterno. Pero incluso entonces, lo que es imposible para el hombre, no lo es para el Padre de los cielos, porque todas las cosas son posibles para Dios”.
\vs p163 3:3 Al marcharse ellos solos, Jesús se apenó de que Matadormo, a quien amaba bastante, no se quedara con ellos. Cuando llegaron al lago, se sentaron cerca del agua y Pedro, hablando por los doce (todos presentes en aquel momento), dijo: “Nos preocupan las palabras que le has dicho al joven rico. ¿Es que debemos exigir a todos los que quieran seguirte que se desprendan de todas sus riquezas terrenales?”. Jesús dijo: “No, Pedro, solo a aquellos que quieran convertirse en apóstoles, y deseen vivir conmigo como vosotros lo hacéis, y como una familia. Pero el Padre pide que los afectos de sus hijos sean puros y no estén divididos. Se debe renunciar a cualquier cosa o persona que se interponga entre vosotros y el amor a las verdades del reino. Si la riqueza que se posee no invade los recintos del alma, no tiene consecuencia en la vida espiritual de los que quieran entrar en el reino”.
\vs p163 3:4 Y, entonces, Pedro añadió: “Pero, Maestro, nosotros lo hemos dejado todo para seguirte, ¿qué tendremos nosotros?”. Y Jesús habló a los doce: “De cierto os digo que no hay nadie que haya dejado riquezas, casa, esposa, hermanos, padres o hijos por causa de mí y del reino de los cielos que no reciba cien veces más en este mundo, quizás con persecuciones, y la vida eterna en el mundo venidero. Pero muchos primeros serán los últimos, mientras que los últimos serán a menudo los primeros. El Padre se ocupa de sus criaturas conforme a sus necesidades y en obediencia a sus leyes justas, decretadas en misericordia y amor por el bienestar del universo.
\vs p163 3:5 “El reino de los cielos es semejante a un propietario, empleador de muchos, que salió por la mañana temprano a contratar obreros para su viña. Y habiendo convenido con los obreros en un denario al día, los envió a su viña. Saliendo cerca de las nueve, vio a otros que estaban en la plaza desocupados y les dijo: “Id también vosotros a mi viña, y os daré lo que sea justo”. Y ellos fueron enseguida a trabajar. Salió otra vez cerca de las doce y cerca de las tres e hizo lo mismo. Y yendo a la plaza sobre las cinco de la tarde, halló todavía a otros que estaban desocupados y les preguntó: “¿Por qué estáis aquí todo el día desocupados?”. Y los hombres le contestaron, “Porque nadie nos ha contratado”. Entonces les dijo el propietario: “Id también vosotros a trabajar a mi viña, y recibiréis lo que sea justo”.
\vs p163 3:6 “Cuando llegó la noche, este propietario de la viña le dijo a su mayordomo: ‘Llama a los trabajadores y págales su jornal, comenzando desde los últimos contratados hasta los primeros’. Llegaron los que habían ido cerca de las cinco y recibieron cada uno un denario, y así fue con cada uno de los demás trabajadores. Al llegar también los que habían sido contratados al comienzo del día, vieron cómo se les había pagado a los que habían venido los últimos y pensaron que recibirían más de lo acordado. Pero también recibieron cada uno solo un denario. Y cuando recibieron su paga, murmuraban contra el propietario diciendo: “Estos últimos han trabajado una sola hora y sin embargo los has tratado igual que a nosotros, que hemos soportado la carga y el calor sofocante del día”.
\vs p163 3:7 “Entonces, el propietario respondió: ‘Amigos míos, no os hago ninguna injusticia. ¿No convinisteis conmigo en trabajar por un denario al día? Tomad ahora lo que es vuestro e iros, porque quiero dar a los últimos que vinieron lo mismo que a vosotros. ¿No me está permitido hacer lo que quiero con lo mío? ¿O tenéis envidia de mi generosidad porque deseo hacer el bien y mostrar misericordia?’”.
\usection{4. DESPEDIDA DE LOS SETENTA}
\vs p163 4:1 Reinaba la emoción en el campamento de Magadán el día que los setenta salieron en su primera misión. Temprano esa mañana, en su última conversación con los setenta, Jesús hizo hincapié en lo siguiente:
\vs p163 4:2 \li{1.}El evangelio del reino debe proclamarse a todo el mundo, tanto a gentiles como a judíos.
\vs p163 4:3 \li{2.}Al atender a los enfermos, absteneros de enseñarles a esperar milagros.
\vs p163 4:4 \li{3.}Proclamad una hermandad espiritual de los hijos de Dios, no un reino visible de poder terrenal y de gloria temporal.
\vs p163 4:5 \li{4.}Evitad perder el tiempo en demasiadas relaciones sociales ni en otras trivialidades que puedan desviaros de vuestra dedicación incondicional a la predicación del evangelio.
\vs p163 4:6 \li{5.}Si la primera casa que elijáis para vuestra sede resulta ser un hogar digno, morad allí durante toda vuestra estancia en esa ciudad.
\vs p163 4:7 \li{6.}Dejad claro a todo creyente fiel que ha llegado el momento de romper abiertamente con los líderes religiosos de los judíos de Jerusalén.
\vs p163 4:8 \li{7.}Enseñad que todo el deber del hombre se resume en este único mandamiento: Ama al Señor tu Dios con toda tu mente y con toda tu alma y a tu prójimo como a ti mismo. (Debían enseñar este deber del hombre en lugar de las 613 reglas de vida enunciadas por los fariseos.)
\vs p163 4:9 \pc Cuando Jesús terminó de decirles estas cosas a los setenta en presencia de todos los apóstoles y de los discípulos, Simón Pedro se los llevó con él y les predicó su sermón de ordenación, que ampliaba las directrices que el Maestro les dio en el momento de imponerles las manos y designarlos como mensajeros del reino. Pedro instó a los setenta a que cultivaran en su experiencia las siguientes virtudes:
\vs p163 4:10 \li{1.}\bibemph{Consagración a la oración y a la adoración}. Orar siempre para que se envíen más obreros para recolectar la mies del evangelio. Explicó que, cuando se ora así, es como si se dijera: “Aquí estoy; envíame a mí”. Les instó a que no desatendiesen su práctica diaria de adoración.
\vs p163 4:11 \li{2.}\bibemph{Verdadero valor}. Les advirtió que hallarían hostilidades y que de cierto sufrirían persecuciones. Pedro les dijo que la misión que iban a emprender necesitaba de valor y aconsejó a quienes tuviesen miedo que se marcharan antes de comenzar. Pero nadie lo hizo.
\vs p163 4:12 \li{3.}\bibemph{Fe y confianza}. En esta corta misión, debían salir sin ningún aprovisionamiento; debían confiar en el Padre para la comida y el cobijo y para todo lo que necesitasen.
\vs p163 4:13 \li{4.}\bibemph{Fervor e iniciativa.} Debían poseer fervor y un sensato entusiasmo; debían ocuparse estrictamente de los asuntos de su Maestro. El saludo oriental conllevaba un ceremonial largo y elaborado; por ello, se les había dado instrucciones para que “no saludaran a nadie en el camino”, lo que era una forma común de exhortar a alguien a que se ocupara de sus asuntos sin perder el tiempo. Esto no significaba que no fuesen afectuosos en sus saludos.
\vs p163 4:14 \li{5.}\bibemph{Generosidad y cortesía}. El Maestro les había dado instrucciones para que evitaran perder innecesariamente el tiempo en ceremonias sociales, aunque les recomendó que actuaran cortésmente con todos los que se relacionaran. Debían mostrar total generosidad hacia quienes los albergaran en sus casas. Estaban estrictamente advertidos de no dejar un hogar humilde para alojarse en uno más confortable o de mayor prestigio.
\vs p163 4:15 \li{6.}\bibemph{Atención a los enfermos}. Pedro encomendó a los setenta a que buscaran a los enfermos de mente y cuerpo e hicieran todo lo que estuviera a su alcance para aliviar o curar sus enfermedades.
\vs p163 4:16 \pc Y una vez que se les encargó su misión y se les dieron estas instrucciones partieron, de dos en dos, para llevar a cabo esta misión en Galilea, Samaria y Judea.
\vs p163 4:17 Aunque los judíos tenían un extraño respeto por el número setenta, por el hecho de que las naciones paganas sumaban setenta, y aunque eran setenta los mensajeros que debían llevar el evangelio a todos los pueblos, por lo que podemos percibir, se trató de una mera coincidencia que este grupo constara de setenta miembros. Es cierto que Jesús habría aceptado a no menos de doce más de ellos, pero ellos no estaban dispuestos a pagar el precio de desprenderse de sus riquezas ni de su familia.
\usection{5. TRASLADO DEL CAMPAMENTO A PELLA}
\vs p163 5:1 Jesús y los doce se dispusieron entonces para establecer su última sede en Perea, cerca de Pella, a la altura del Jordán donde el Maestro había sido bautizado. Los últimos diez días de noviembre estuvieron ultimando planes en Magadán y el martes, 6 de diciembre, todo el grupo, de casi unas trescientas personas, partió al despuntar el día, con todos sus efectos, para pasar la noche cerca de Pella, junto al río. En ese mismo lugar, cerca del manantial, Juan el Bautista había instalado su campamento algunos años antes.
\vs p163 5:2 Tras el desmantelamiento del campamento de Magadán, David Zebedeo regresó a Betsaida y comenzó de inmediato a reducir su servicio de mensajería. El reino iniciaba una nueva etapa. Diariamente llegaban peregrinos de todas partes de Palestina e incluso de regiones remotas del Imperio romano. Ocasionalmente, acudían creyentes de Mesopotamia y de las tierras al este del Tigris. En consecuencia, el domingo 18 de diciembre, con la ayuda del cuerpo de mensajeros, David cargó los animales de carga con el equipamiento del campamento, almacenado entonces en casa de su padre. Con dicho material, él había gestionado con anterioridad el campamento de Betsaida cercano al lago. Despidiéndose de Betsaida por el momento, se dirigió al sur por la orilla del lago, y a lo largo del Jordán, hasta un punto a un kilómetro prácticamente al norte del campamento apostólico; y, en menos de una semana, estaba listo para ofrecer su hospitalidad a casi mil quinientos peregrinos. El campamento apostólico podía dar cobijo a unos quinientos. Aquella era la temporada de lluvias en Palestina, y se precisaba de este alojamiento para atender al creciente número de personas, mayormente fervientes devotos, que venían a Perea buscando a Jesús y oír sus enseñanzas.
\vs p163 5:3 David hizo todo esto por iniciativa propia, aunque se lo había comunicado a Felipe y a Mateo en Magadán. La mayor parte del cuerpo de mensajeros sirvió como ayudante en la dirección de este campamento; en aquel momento, eran menos de veinte los hombres que llevaban a cabo este servicio regular de mensajería. Hacia finales de diciembre y antes del regreso de los setenta, se habían reunido alrededor del Maestro casi ochocientos visitantes, que encontraron alojamiento en el campamento de David.
\usection{6. REGRESO DE LOS SETENTA}
\vs p163 6:1 El viernes 30 de diciembre, mientras Jesús se encontraba en las colinas cercanas con Pedro, Santiago y Juan, los setenta mensajeros fueron llegando, de dos en dos, a la sede de Pella, acompañados de un gran número de creyentes. Jesús volvió al campamento sobre las cinco de la tarde, estando los setenta reunidos en el área de formación. La cena se demoró durante más de una hora por el relato de las experiencias de estos entusiastas del evangelio del reino. Durante las semanas previas, los mensajeros de David ya habían informado a los apóstoles muchas de estas cosas, pero realmente inspiraba oír a estos maestros del evangelio, recientemente ordenados, narrar personalmente cómo había judíos y gentiles, sedientos de la verdad, que habían acogido su mensaje. Jesús, por fin, vio cómo los hombres salían a difundir la buena nueva sin su presencia personal, y supo entonces que podía dejar este mundo sin que esto supusiera un serio obstáculo para el progreso del reino.
\vs p163 6:2 Cuando los setenta comentaron que “hasta los demonios los obedecían”, se referían a las extraordinarias sanaciones que habían hecho en casos de víctimas con trastornos nerviosos. No obstante, sí hubo algunos pocos casos de auténtica posesión de los espíritus, a los que estos mensajeros del evangelio habían aliviado y, aludiendo a estos, Jesús dijo: “No es extraño que estos espíritus menores rebeldes se sometan a vosotros, pues yo vi a Satanás caer del cielo como un rayo. Pero no os alegréis tanto por ello, porque yo os declaro que, en cuanto regrese a mi Padre, enviaremos a nuestros espíritus a las mentes de los hombres para que estos pocos espíritus extraviados no puedan entrar más en las mentes de los desafortunados mortales. Me regocijo con vosotros de que tengáis influencia sobre los hombres, pero no os enaltezcáis a causa de esto, gozaos más bien de que vuestros nombres ya estén escritos en los listados del cielo, y de que avanzaréis por lo tanto en la andadura sin fin de la conquista espiritual”.
\vs p163 6:3 Y fue en ese instante, justo antes de cenar, cuando Jesús experimentó uno de esos raros momentos de éxtasis emocional, que en ocasiones sus seguidores habían presenciado. Dijo: “Te doy gracias, Padre mío, Señor del cielo y de la tierra porque, mientras que este gran evangelio estaba oculto a los sabios y a los arrogantes, el espíritu ha revelado estas glorias espirituales a estos hijos del reino. Sí, Padre mío, debe haber sido grato ante tus ojos hacer esto, y me regocija saber que la buena nueva se diseminará por todo el mundo, incluso después de que yo haya vuelto a ti y a la labor que me has encomendado. Estoy poderosamente conmovido porque me doy cuenta de que estás a punto de entregar en mis manos toda la autoridad, que nadie sino tú sabe realmente quién soy yo y que nadie sabe realmente quien eres tú sino yo, y aquellos a quienes yo te he revelado. Y cuando haya acabado de impartir esta revelación a mis hermanos en la carne, continuaré impartiéndola a tus criaturas en las alturas”.
\vs p163 6:4 Tras hablar con el Padre, Jesús se volvió a sus discípulos y mensajeros del evangelio y les dijo: “Dichosos los ojos que vean y los oídos que oigan estas cosas, porque os digo que muchos profetas y grandes hombres de eras antiguas desearon ver lo que vosotros ahora veis, y no lo vieron. Y muchas generaciones de hijos de la luz aún por venir, cuando oigan de estas cosas, os envidiarán a vosotros que las habéis oído y visto”.
\vs p163 6:5 Luego, dirigiéndose a todos los discípulos, dijo: “Habéis oído cuántas ciudades y aldeas han acogido la buena nueva del reino, y cómo los judíos y los gentiles han recibido a mis mensajeros del evangelio y maestros. Y benditas sean ciertamente estas comunidades que han escogido creer el evangelio del reino. Pero, ¡ay de los habitantes de Corazín, Betsaida\hyp{}Julias y Cafarnaúm!, que rechazaron la luz, esas ciudades que no recibieron bien a estos mensajeros. Yo declaro que si las obras portentosas que se hicieron en estos lugares se hubieran hecho en Tiro y Sidón, la gente de estas ciudades llamadas paganas ya hace tiempo que se habría arrepentido y se habría puesto ropas ásperas y cubierto de ceniza. Cuando llegue el día del juicio será de hecho más tolerable para Tiro y Sidón”.
\vs p163 6:6 \pc Al ser el día siguiente \bibemph{sabbat,} Jesús se fue aparte con los setenta y les dijo: “Verdaderamente me gocé con vosotros cuando regresasteis con las buenas nuevas sobre la acogida del evangelio del reino por tanta gente repartida por toda Galilea, Samaria y Judea. Pero, ¿por qué estabais tan sorprendentemente eufóricos? ¿Es que no esperabais que vuestro mensaje fuera poderoso? ¿Es que salisteis con tan poca fe en este evangelio que volvisteis sorprendidos de su efectividad? Y ahora bien, aunque no quisiera apagar vuestro espíritu de gozo, deseo advertiros seriamente contra las sutilezas del orgullo, del orgullo espiritual. Si pudieseis entender la caída de Lucifer, el inicuo, rehuiríais cualquier forma de orgullo espiritual.
\vs p163 6:7 “Habéis emprendido esta gran labor de enseñar al hombre mortal que él es un hijo de Dios. Os he mostrado el camino; salid a cumplir con vuestro deber y no os canséis de hacer el bien. A vosotros y a todos los que sigan vuestros pasos a través de las eras, os digo: yo siempre estoy cerca, y mi llamamiento\hyp{}invitación es y por siempre será: venid a mí todos los que estáis cansados y agobiados, y yo os haré descansar. Aceptad el yugo que os impongo, y aprended de mí, que soy honesto y leal, y encontraréis descanso espiritual para vuestras almas”.
\vs p163 6:8 \pc Y comprobaron que las palabras del Maestro eran verdad cuando pusieron a prueba sus promesas. Y, desde ese día, incontables miles de personas han experimentado y confirmado de igual manera la certitud de estas mismas promesas.
\usection{7. PREPARATIVOS PARA LA ÚLTIMA MISIÓN}
\vs p163 7:1 Durante los días siguientes, se vivieron momentos de ajetreo en el campamento de Pella; se estaban ultimando los preparativos para la misión de Perea. Jesús y sus acompañantes se disponían a emprender esta postrera misión: un viaje de tres meses por toda Perea, que acabaría cuando el Maestro entró en Jerusalén para llevar a cabo su última labor en la tierra. En todo este período, la sede de Jesús y los doce se mantuvo aquí, en el campamento de Pella.
\vs p163 7:2 Ya no era necesario que Jesús saliera a otros lugares para impartir sus enseñanzas, ahora la gente acudía a él en mayor número cada semana. Venían de todos sitios, no solo de Palestina, sino de todo el mundo romano y de Oriente Próximo. Aunque el Maestro participó junto con los setenta en el viaje por Perea, pasaba mucho tiempo en el campamento de Pella, enseñando a la multitud e instruyendo a los doce. Mientras duró este período de tres meses, al menos diez de los apóstoles permanecieron con Jesús.
\vs p163 7:3 El colectivo de mujeres se preparó igualmente para salir de dos en dos con los setenta, y ejercer su actividad en las ciudades más grandes de Perea. Este grupo, en un principio de doce mujeres, había últimamente formado a otro colectivo, más amplio, de cincuenta mujeres, en la tarea de visitar los hogares y en el arte de atender a los enfermos y a los afligidos. Perpetua, la esposa de Simón Pedro, formó parte de esta extensión del grupo de mujeres y se le encomendó liderar su trabajo, bajo las órdenes de Abner. Tras Pentecostés, Perpetua permaneció con su insigne marido, acompañándolo en todos sus viajes misioneros; y el día en el que Pedro fue crucificado en Roma, ella murió en la arena del anfiteatro víctima de las fieras salvajes. Entre los miembros de este nuevo colectivo de mujeres se contaban también las esposas de Felipe y Mateo y a la madre de Santiago y Juan.
\vs p163 7:4 Bajo el liderazgo personal de Jesús, la labor del reino llegaba a su etapa final. Y esta etapa se distinguía, por su profundidad espiritual, de aquella otra, de superficialidad, en la que las que las multitudes de orientación milagrera y buscadoras de portentos, lo habían seguido durante aquellos días previos de popularidad en Galilea. Sin embargo, aún existía un buen número de seguidores apegados a lo material, que no lograba comprender la verdad de que el reino de los cielos es la hermandad espiritual del hombre, fundada en el hecho eterno de la paternidad universal de Dios.
