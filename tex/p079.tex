\upaper{79}{La expansión andita en oriente}
\author{Arcángel}
\vs p079 0:1 Asia es el territorio patrio de la raza humana. Fue en una península sureña de este continente donde Andón y Fonta nacieron; en las altiplanicies de lo que ahora es Afganistán, su descendiente Badonán fundó un ancestral centro de cultura que perduró durante más de medio millón de años. Aquí, en este núcleo oriental de la especie humana, los pueblos sangiks se diferenciaron de las estirpes andonitas y Asia fue su primer hogar, su primer terreno de caza, su primer campo de batalla. El suroeste asiático fue testigo de las civilizaciones sucesivas de dalamatianos, noditas, adanitas y anditas y, desde estas regiones, se extendieron a todo el mundo los potenciales de la civilización moderna.
\usection{1. LOS ANDITAS DEL TURQUESTÁN}
\vs p079 1:1 Durante más de veinticinco mil años, hasta cerca del año 2000 a. C., el corazón de Eurasia fue, aunque disminuyendo de forma paulatina, predominantemente andita. En las tierras bajas del Turquestán, los anditas, rodeando los lagos interiores por el oeste, se desviaron en dirección a Europa, mientras que, desde las altiplanicies de esta región, se adentraron en el este. El Turquestán oriental (Sinkiang) y, en menor medida, el Tíbet, fueron las ancestrales entradas por las que estos pueblos de Mesopotamia penetraron en las montañas con rumbo a las tierras norteñas del hombre amarillo. La penetración andita de la India avanzó desde las altiplanicies del Turquestán hasta el Punyab y, desde las tierras de pastoreo iraníes, a través de Beluchistán. Estas emigraciones primitivas no fueron, en ningún sentido, conquistas, sino más bien el flujo continuo de las tribus anditas hacia el oeste de la India y China.
\vs p079 1:2 \pc Los centros de la heterogénea cultura andita subsistieron durante casi quince mil años en la cuenca del río Tarim en Sinkiang y, hacia el sur, en las altiplanicies del Tíbet, donde los anditas y andonitas se habían mezclado extensamente. El valle del Tarimera era el puesto más oriental de la verdadera cultura andita. Aquí construyeron sus asentamientos e iniciaron relaciones comerciales con los avanzados chinos del este y con los andonitas del norte. En esos días, la región del Tarim era una tierra fértil; la lluvia era abundante. Hacia el este estaban los pastizales abiertos de Gobi, donde los pastores iban paulatinamente pasando a la agricultura. Aunque en su día rivalizó con la misma Mesopotamia, esta civilización se extinguió cuando los vientos que traían la lluvia se desplazaron hacia el sureste.
\vs p079 1:3 \pc Hacia el año 8000 a. C., la aridez, en lento aumento, de las regiones altas de Asia central comenzó a empujar a los anditas a los lechos marinos y a las zonas costeras. Este incremento de la sequía no solamente los condujo a los valles de los ríos Nilo, Éufrates, Indo y Amarillo, sino que produjo un nuevo avance en la civilización andita. Una nueva clase de hombres, los comerciantes, comenzó a proliferar en gran número.
\vs p079 1:4 Cuando las condiciones climáticas hicieron la caza improductiva para los emigrantes anditas, estos no siguieron el rumbo evolutivo de las razas más antiguas convirtiéndose en pastores. El comercio y la vida urbana hicieron su aparición. Desde Egipto, Mesopotamia y el Turquestán hasta los ríos de China e India, las tribus más altamente civilizadas empezaron a agruparse en ciudades dedicadas a la manufactura y al comercio. Adonia, situada cerca de la actual ciudad de Ashjabad, se convirtió en la urbe comercial del Asia central. El comercio de piedras, metales, madera y alfarería tuvo un rápido desarrollo tanto de forma terrestre como fluvial.
\vs p079 1:5 Pero la sequía, cada vez más intensa, ocasionó paulatinamente el gran éxodo andita desde las tierras situadas al sur y al este del Mar Caspio. La ola migratoria empezó su cambio de rumbo de norte a sur, y la caballería de Babilonia comenzó su incursión en Mesopotamia.
\vs p079 1:6 El aumento de la aridez en Asia central contribuyó, además, a reducir la población y a hacer que estas personas fuesen menos belicosas; y, cuando el descenso de la lluvia en el norte forzó a los andonitas nómadas a dirigirse al sur, hubo un enorme éxodo de anditas desde el Turquestán. Fue el último desplazamiento de los llamados arios al Levante y a la India. Esto puso fin a esa gran dispersión de los descendientes mezclados de Adán, durante la cual todos los pueblos asiáticos y una gran parte de los pueblos de las islas del Pacífico mejoraron en cierta medida gracias a estas razas mejor dotadas.
\vs p079 1:7 Así pues, al dispersarse por el hemisferio oriental, los anditas fueron desposeídos de sus tierras natales de Mesopotamia y el Turquestán; este gran desplazamiento de los andonitas hacia el sur debilitó a los anditas de Asia central hasta llegar casi a desaparecer.
\vs p079 1:8 Pero, incluso en el siglo XX d. C., quedan vestigios de sangre andita entre los turanianos y tibetanos, como se puede observar por las personas rubias que en ocasiones se encuentran en estas regiones. En los primitivos anales chinos está registrada la presencia de nómadas pelirrojos al norte de los pacíficos asentamientos del río Amarillo, y aún se conservan pinturas que recogen de forma fidedigna la presencia, en otros tiempos pasados, en la cuenca del Tarim, del tipo rubio andita y del tipo moreno mongol.
\vs p079 1:9 La última gran expresión del latente genio militar de los anditas de Asia central se produjo en el año 1200 d. C., cuando los mongoles, bajo el mando de Gengis Kan, iniciaron la conquista de la mayor parte del continente asiático. Y como los anditas de antaño, estos guerreros proclamaron la existencia de “un solo Dios del cielo”. La temprana disolución de su imperio retrasó durante mucho tiempo el intercambio cultural entre Occidente y Oriente y mermó notablemente el desarrollo de un concepto monoteísta en Asia.
\usection{2. LA CONQUISTA ANDITA DE LA INDIA}
\vs p079 2:1 La India es el único lugar en el que todas las razas de Urantia se mezclaron; la invasión andita añadiría un último linaje. En las altiplanicies del noroeste de la India, aparecieron las razas sangiks y, en sus tempranos días, sin excepción, componentes de cada una de ellas penetraron en el subcontinente de la India, dejando tras de sí el cruce de raza más heterogéneo que jamás haya existido en Urantia. La India ancestral actuó como punto de congregación de las razas migratorias. Con anterioridad, la base de la península era algo más estrecha que ahora; una gran parte de los deltas del Ganges y del Indo se han formado en los últimos cincuenta mil años.
\vs p079 2:2 Las primeras mezclas de razas en la India estaban formadas por una combinación de razas migratorias roja y amarilla con los aborígenes andonitas. Este grupo se debilitó después al absorber racialmente a la mayor parte de los extintos pueblos verdes del este al igual que a un gran número de miembros de la raza naranja; mejoró ligeramente gracias a su reducida mezcla con el hombre azul, pero se deterioró sobremanera al cruzarse con una gran cantidad de componentes de la raza índigo. Si bien, los llamados nativos de la India apenas representan a esta primera población; son más bien los grupos peor dotados del sur y del este, que nunca fueron racialmente absorbidos del todo ni por los primeros anditas ni por sus primos arios, que aparecerían más adelante.
\vs p079 2:3 \pc Hacia el año 20\,000 a. C. la población del oeste de India se había impregnado de sangre adánica y nunca, en la historia de Urantia, se había dado en un solo pueblo una combinación de tantas razas diferentes. Pero, fue lamentable que predominaran los linajes sangiks secundarios, y resultó un verdadero desastre que tanto el hombre azul como el hombre rojo estuviesen, en buena medida, tan ausentes de este crisol racial de antaño. Una mayor cantidad de linajes sangiks primarios hubiese contribuido bastante al mejoramiento de lo que podría haber sido una civilización incluso más preeminente. Tal como se desarrolló la situación, el hombre rojo se estaba destruyendo a sí mismo en las Américas, el hombre azul se solazaba en Europa y los primeros descendientes de Adán (y la mayoría de los posteriores) demostraban escaso deseo de mezclarse con los pueblos de color más oscuro, ya fuese en la India, en África o en otras partes.
\vs p079 2:4 \pc Alrededor del año 15\,000 a. C., el aumento de la presión demográfica por todo el Turquestán e Irán dio lugar al primer movimiento migratorio realmente intenso de los anditas hacia la India. Durante más de quince siglos, estos pueblos superiores cruzaron en gran número las altiplanicies de Beluchistán, extendiéndose por los valles del Indo y del Ganges y desplazándose lentamente rumbo sur hacia el interior de Decán. Esta presión andita desde el noroeste llevó a que muchos de los pueblos menos dotados del sur y del este se dirigiesen a Birmania y al sur de China, pero no lo suficiente como para salvar a los invasores de la aniquilación racial.
\vs p079 2:5 El fracaso de la India por lograr la hegemonía sobre Eurasia se debió, en gran parte, a una cuestión de topografía; la presión demográfica que se ejercía desde el norte solo empujó a la mayoría de la población al sur, al menguante territorio de Decán, rodeado de mar por todas partes. De haber habido tierras colindantes para la emigración, las estirpes menos dotadas se hubiesen dispersado entonces por todas las direcciones, y las mejor dotadas habrían logrado una civilización de orden superior.
\vs p079 2:6 Y sucedió que estos primitivos conquistadores anditas realizaron un intento desesperado por preservar su identidad y detener la oleada de inmersión racial mediante el establecimiento de rígidas restricciones respecto al matrimonio interracial. No obstante, hacia el año 10\,000 a. C., los anditas se habían mestizados, pero esta absorción racial había mejorado considerablemente a todo el conjunto de la población.
\vs p079 2:7 \pc La mezcla de razas siempre es ventajosa en el sentido de que favorece la versatilidad de la cultura y contribuye al avance de la civilización, pero si predominan los componentes peor dotados de las estirpes raciales, tales logros serán efímeros. Una cultura multirracial puede preservarse únicamente si las estirpes mejores se reproducen con un margen de seguridad sobre las peores. La multiplicación incontrolada de los linajes menos deseables, con una decreciente reproducción de los deseables, es invariablemente letal para la civilización cultural.
\vs p079 2:8 Si los conquistadores anditas hubiesen sido tres veces más numerosos de los que eran, o si hubiesen expulsado o exterminado al tercio menos deseable de habitantes naranja\hyp{}verde\hyp{}índigo mezclados, la India se habría convertido, entonces, en uno de los centros más importantes del mundo de la civilización cultural e indudablemente habría atraído a una mayor cantidad de las oleadas posteriores de mesopotámicos que penetraron en el Turquestán y, desde allí, se dirigieron hacia el norte con rumbo a Europa.
\usection{3. LA INDIA DRAVIDIANA}
\vs p079 3:1 La mezcla de los conquistadores anditas de la India con el linaje nativo dio finalmente lugar a esa población mixta que se ha venido a llamar dravidiana. Los dravidianos primitivos y más puros poseían una gran capacidad para alcanzar logros culturales, que se fue debilitando constantemente conforme su herencia andita se iba atenuando de forma gradual. Y esto fue lo que hizo fracasar a la incipiente civilización de la India hace casi doce mil años. Pero la infusión de incluso esta pequeña cantidad de sangre adánica produjo una notable aceleración de su desarrollo social. Este linaje compuesto trajo de inmediato consigo la civilización más versátil de las existentes entonces en la tierra.
\vs p079 3:2 No mucho tiempo después de conquistar la India, los anditas dravidianos perdieron su contacto racial y cultural con Mesopotamia, pero la posterior apertura de las líneas marítimas y de las rutas de caravanas restablecieron estos vínculos; y, la India, en ningún momento durante los últimos diez años, ha perdido del todo el contacto con Mesopotamia al oeste y con China al este, aunque las barreras montañosas favorecieron sobremanera la interrelación con el oeste.
\vs p079 3:3 \pc La cultura superior y las inclinaciones religiosas de los pueblos de la India se remontan a los tempranos tiempos de la dominación dravidiana y se deben, en parte, al hecho del gran número de sacerdotes setitas que entraron en la India, tanto con la primera invasión andita como con las posteriores invasiones arias. La corriente monoteísta que circula por la historia de la India proviene, pues, de las enseñanzas impartidas por los adanitas en el segundo jardín.
\vs p079 3:4 En fecha tan temprana como el año 16\,000 a. C., un grupo de cien sacerdotes setitas entró en la India y estuvo a punto de llevar a cabo la conquista religiosa de la mitad occidental de este pueblo multirracial. Pero su religión no perduró; en cinco mil años, sus doctrinas sobre la Trinidad del Paraíso habían degenerado en el símbolo trino del dios del fuego.
\vs p079 3:5 Si bien, durante más de siete mil años, hacia el final de las emigraciones anditas, el nivel religioso de los habitantes de la India estaba muy por encima al del conjunto del mundo. Durante estos tiempos, parecía probable que la India daría origen a la principal civilización cultural, religiosa, filosófica y comercial del mundo. Si los anditas no hubiesen sido totalmente mestizados por los pueblos del sur, este destino se habría posiblemente cumplido.
\vs p079 3:6 \pc Los centros culturales dravidianos estaban situados en los valles fluviales, principalmente del Indo y el Ganges, y en el Decán a lo largo de los tres grandes ríos que fluyen a través de los Ghats Orientales hacia el mar. Los asentamientos que bordeaban la costa de los Ghats Occidentales les debían su posición preponderante a los intercambios marítimos con Sumeria.
\vs p079 3:7 Los dravidianos se contaban entre los primeros pueblos en construir ciudades y dedicarse a un importante negocio de exportación e importación, tanto por tierra como por mar. Hacia el año 7000 a. C., las caravanas de camellos efectuaban viajes regulares a la distante Mesopotamia. Las embarcaciones de los dravidianos navegaban a lo largo de la costa del mar Arábigo hasta las ciudades sumerias del Golfo Pérsico y se aventuraban en las aguas de la Bahía de Bengala hasta las Indias Orientales. De Sumeria, estos navegantes y mercaderes importaron un alfabeto junto al arte de escribir.
\vs p079 3:8 Estas relaciones comerciales contribuyeron enormemente a una mayor diversificación de esta cosmopolita cultura, dando lugar a la aparición temprana de muchos de los refinamientos e incluso lujos de la vida urbana. Cuando los arios, que aparecerían más tarde, entraron en la India no reconocieron en los dravidianos a sus primos anditas, ya mestizados con las razas sangik, pero sí hallaron una civilización muy adelantada. A pesar de sus limitaciones biológicas, los dravidianos fundaron una civilización superior que se había extendido por toda la India y que, hasta los tiempos modernos, ha perdurado en el Decán
\usection{4. LA INVASIÓN ARIA DE LA INDIA}
\vs p079 4:1 La segunda penetración andita de la India fue la invasión aria durante un período de casi quinientos años, a mediados del tercer milenio a. C. Esta emigración marcó el éxodo final de los anditas desde sus tierras natales en el Turquestán.
\vs p079 4:2 Los primeros centros arios estaban diseminados por la mitad septentrional de la India, particularmente en el noroeste. Estos invasores nunca completaron la conquista del país y esta negligencia causó posteriormente su perdición, puesto que su inferioridad numérica los hizo vulnerables a la absorción racial por parte de los dravidianos del sur, que más tarde invadirían toda la península a excepción de las provincias del Himalaya.
\vs p079 4:3 Los arios dejaron muy poca huella racial en la India, salvo en las provincias norteñas. En el Decán, su influencia fue más cultural y religiosa que racial. La denominada sangre aria persistió más en el norte de la India no solo por su propia presencia, más numerosa en estas regiones, sino también por el hecho de que más adelante se vieron reforzados por conquistadores, comerciantes y misioneros. Hasta el siglo I a. C., hubo una continua infiltración de sangre aria en la región del Punjab; su última afluencia fue en conjunción con las campañas de los pueblos helénicos.
\vs p079 4:4 Los arios y los dravidianos acabaron por entremezclarse en las llanuras del Ganges, dando lugar a un gran centro cultural, que se reforzaría más tarde con las contribuciones del noreste, procedentes de China.
\vs p079 4:5 En la India, cada cierto tiempo, florecían muchos tipos de organizaciones sociales, desde los sistemas semidemocráticos de los arios hasta las formas de gobierno despóticas y monárquicas. Pero el rasgo más característico de esa sociedad fue la subsistencia de las grandes castas sociales, instituidas por los arios en su afán por perpetuar su identidad racial. Este elaborado sistema de castas se ha mantenido hasta los tiempos actuales.
\vs p079 4:6 De las cuatro grandes castas existentes, todas, salvo la primera, se formaron en un vano intento por impedir el cruce racial de los conquistadores arios con sus súbditos de inferior orden. Pero la casta principal, la de los maestros\hyp{}sacerdotes, proviene de los setitas. Los brahmanes del siglo XX d. C. son los descendientes culturales directos de los sacerdotes del segundo jardín, aunque sus enseñanzas difieren sobremanera de las de sus ilustres predecesores.
\vs p079 4:7 Cuando los arios entraron en la India, traían consigo sus conceptos de la Deidad tal como se habían conservado en las persistentes tradiciones religiosas del segundo jardín. Pero los sacerdotes brahmanes nunca fueron capaces de resistir el impulso pagano que se había desarrollado a raíz de su repentino contacto con las religiones inferiores del Decán, tras la aniquilación racial de los arios. Por consiguiente, la inmensa mayoría de la población cayó en el yugo de las esclavizantes supersticiones de estas religiones inferiores; y así fue como la India no logró crear la elevada civilización que se preveía en tiempos anteriores.
\vs p079 4:8 El despertar espiritual del siglo VI a. C. no perduró en la India; se extinguió incluso antes de la invasión mahometana. Pero, algún día, quizás surja un gran Gautama que dirija a toda la India a la búsqueda del Dios vivo y, entonces, el mundo podrá observar el fruto de los potenciales culturales de un pueblo versátil, por tan largo tiempo comatoso bajo la paralizante influencia de una infradesarrollada visión espiritual.
\vs p079 4:9 La cultura descansa de hecho sobre unos cimientos biológicos, pero las castas por sí solas no pueden perpetuar la cultura aria, porque la religión, la verdadera religión, es la fuente indispensable de esa energía más elevada que impulsa al hombre a establecer una civilización superior basada en la fraternidad humana.
\usection{5. EL HOMBRE ROJO Y EL HOMBRE AMARILLO}
\vs p079 5:1 Mientras que la historia de la India es la historia de la conquista andita y de su integración final en los pueblos evolutivos más antiguos, la de Asia oriental es más bien la historia de los sangiks primarios, en particular del hombre rojo y del amarillo. Estas dos razas evitaron en gran medida ese mestizaje con el degradado linaje neandertal que tanto retrasó al hombre azul en Europa, preservando así el potencial superior del tipo sangik primario.
\vs p079 5:2 Aunque los primeros neandertales estaban dispersos por la totalidad de Eurasia, su rama oriental era la que se encontraba más contaminada con linajes animales degradados. Estos tipos subhumanos de seres fueron empujados en dirección sur por el quinto glaciar, la misma placa de hielo que por tan largo tiempo había obstaculizado la emigración sangik hacia Asia oriental. Y cuando el hombre rojo se desplazó hacia el noreste bordeando las altiplanicies de la India, halló el noreste de Asia libre de esta clase subhumana de seres. La organización tribal de las razas rojas se formó con anterioridad a la de cualquier otro pueblo, y fueron los primeros en emigrar del núcleo sangik de Asia central. Las estirpes pobremente dotadas del hombre de Neandertal fueron erradicadas o expulsadas del territorio continental por las posteriores migraciones de las tribus amarillas. Pero, antes de que llegasen las tribus amarillas, el hombre rojo había reinado con supremacía en Asia oriental durante casi cien mil años.
\vs p079 5:3 \pc Hace más de trescientos mil años, el grupo principal migratorio de la raza amarilla entró en China por la costa. Cada milenio se adentraron más y más hacia el interior, pero no entablaron contacto con sus hermanos tibetanos emigrantes hasta tiempos relativamente recientes.
\vs p079 5:4 La creciente presión demográfica hizo que la raza amarilla, que avanzaba hacia el norte, comenzara a invadir los territorios de caza del hombre rojo. Esta intrusión, unida al natural antagonismo racial, culminó en enfrentamientos cada vez más intensos, empezando así la lucha decisiva por las tierras fértiles más lejanas de Asia.
\vs p079 5:5 El relato de esta larguísima contienda entre las razas roja y amarilla fue un acontecimiento épico de la historia de Urantia. Durante más de doscientos mil años estas dos razas superiores libraron una encarnizada e incesante guerra. En las primeras batallas, los hombres rojos eran generalmente los vencedores y sus saqueos causaban estragos entre los asentamientos de la raza amarilla. Pero el hombre amarillo era un buen alumno en las artes bélicas y, rápidamente, manifestó una notable habilidad para vivir pacíficamente con sus compatriotas. Los chinos fueron los primeros en aprender que en la unión estaba la fuerza.
\vs p079 5:6 \pc Hace cien mil años las diezmadas tribus de la raza roja luchaban de espaldas al hielo en retroceso del último glaciar y, cuando el pasaje terrestre hacia el este, sobre el Estrecho de Bering, se hizo transitable, estas tribus no tardaron en abandonar las inhóspitas costas del continente asiático. Han transcurrido ochenta y cinco mil años desde que los últimos hombres rojos de linaje puro partieron de Asia, pero la larga lucha dejó su huella genética sobre la victoriosa raza amarilla. Los pueblos chinos del norte, junto con los siberianos andonitas, asimilaron una gran parte de la estirpe roja y resultaron bastante beneficiados de ella.
\vs p079 5:7 Los indios de América del Norte nunca entraron en contacto ni siquiera con los vástagos anditas de Adán y Eva, ya que habían sido desposeídos de sus tierras natales asiáticas unos cincuenta mil años antes de la venida de Adán. Durante la época de las emigraciones anditas, las razas rojas de linaje puro se expandieron por América del Norte como tribus nómadas, como cazadores que, en menor medida, practicaban la agricultura. Estas razas y grupos culturales permanecieron casi completamente aislados del resto del mundo desde su llegada a las Américas hasta el final del primer milenio de la era cristiana, cuando las razas blancas de Europa los descubrieron. Hasta ese momento, los esquimales eran lo más cercano al hombre blanco que las tribus norteñas de hombres rojos habían visto jamás.
\vs p079 5:8 Las razas roja y amarilla son las únicas estirpes humanas que llegaron a alcanzar un alto grado de civilización al margen de la influencia andita. El centro cultural amerindio de mayor antigüedad fue el de Onamonalontón, en California, aunque, para el año 35\,000 a. C., ya hacía tiempo que había desaparecido. En México, en América Central y en las montañas de América del Sur, las posteriores civilizaciones, más duraderas se fundaron por una raza predominantemente roja, pero que contenía una mezcla considerable de las razas amarilla, naranja y azul.
\vs p079 5:9 Estas civilizaciones fueron una consecuencia evolutiva de los sangiks, a pesar de haber vestigios de que la raza andita llegó a Perú. Exceptuando a los esquimales en América del Norte y a algunos anditas polinesios en América del Sur, los pueblos del hemisferio occidental no entablaron contacto con el resto del mundo hasta el fin del primer milenio después de Cristo. En el plan original de los melquisedecs para la mejora de las razas de Urantia se había previsto que un millón de descendientes de línea pura de Adán debían ir y hacer avanzar al hombre rojo de las Américas.
\usection{6. LOS ALBORES DE LA CIVILIZACIÓN CHINA}
\vs p079 6:1 Algún tiempo después de desplazar al hombre rojo a América del Norte, los chinos, en su expansión, desalojaron a los andonitas de los valles fluviales del este de Asia, forzándolos a dirigirse a Siberia, en el norte y, al Turquestán, en el oeste, donde pronto entrarían en contacto con la cultura superior de los anditas.
\vs p079 6:2 En Birmania y en la península de Indochina, las culturas de la India y de China se mezclaron y unificaron para dar origen a las sucesivas civilizaciones de esas regiones. Aquí, la desaparecida raza verde ha persistido en una mayor proporción que en cualquier otra parte del mundo.
\vs p079 6:3 Muchas razas diferentes ocuparon las islas del Pacífico. En general, las islas sureñas, entonces más grandes, estaban habitadas por pueblos que portaban un alto porcentaje de sangre de las razas verde e índigo. Las islas norteñas estaban en poder de los andonitas y, más tarde, de razas que poseían una gran proporción de estirpes amarilla y roja. Los ancestros del pueblo japonés no fueron expulsados de las tierras continentales hasta el año 12\,000 a. C., cuando los desalojó el fuerte empuje que las tribus chinas del norte ejercieron en el sur, a lo largo de la costa. Su éxodo final no se debió tanto a la presión demográfica como a la acción de un caudillo, a quien llegaron a considerar como alguien divino.
\vs p079 6:4 Tal como los pueblos de la India y el Levante, las victoriosas tribus del hombre amarillo se establecieron primeramente a lo largo de las costas y siguiendo el curso de los ríos. En años posteriores, los asentamientos costeros tuvieron un destino adverso a causa de las crecientes inundaciones y del curso cambiante de los ríos, que hicieron insostenible la situación de las ciudades de las tierras bajas.
\vs p079 6:5 Hace veinte mil años, los ancestros de los chinos habían construido una docena de poderosos centros de cultura y enseñanza primitiva, en especial a lo largo de los ríos Amarillo y Yangtzé. Y estos centros comenzaron a reforzarse con la llegada de un flujo continuo de pueblos mezclados de un orden superior provenientes de Sinkiang y el Tíbet. La emigración desde el Tíbet al valle del Yangtzé no fue de la magnitud de la del norte ni los centros tibetanos tan avanzados como los de la cuenca del Tarim; si bien, ambos movimientos migratorios llevaron cierta cantidad de sangre andita al este, a los asentamientos fluviales.
\vs p079 6:6 La superioridad de la ancestral raza amarilla se debió a cuatro grandes factores:
\vs p079 6:7 \li{1.}\bibemph{Genético}. A diferencia de sus primos de raza azul de Europa, tanto la raza roja como la amarilla habían evitado en buena medida mezclarse con estirpes humanas degradadas. Los chinos del norte, ya reforzados con pequeñas cantidades de los linajes superiores de la raza roja y de la andonita, se beneficiarían pronto de un considerable aporte de sangre andita. A los chinos del sur no les fue tan bien en este respecto; habían sufrido por mucho tiempo las consecuencias de la absorción de la raza verde, mientras que después se debilitarían aún más por la infiltración de enjambres de pueblos pobremente dotados, expulsados de la India a causa de la invasión andita\hyp{}dravidiana. Y, hoy en día, existe en China una clara diferencia entre las razas del norte y las del sur.
\vs p079 6:8 \li{2.}\bibemph{Social}. La raza amarilla aprendió tempranamente el valor de la paz entre ellos. Su pacifismo interno contribuyó de tal manera al aumento de la población que garantizó la diseminación de su civilización a muchos millones de personas. Desde el año 25\,000 al 5000 a. C., la mayor concentración de la civilización de Urantia se encontraba en el centro y norte de China. El hombre amarillo fue el primero en lograr solidaridad racial ---el primero en lograr una civilización cultural, social y política a gran escala---.
\vs p079 6:9 Los chinos del año 15\,000 a. C. eran militaristas agresivos; su excesiva reverencia por el pasado no los había debilitado y, siendo menos de doce millones, constituían un compacto colectivo que hablaba un idioma común. Durante esta época, crearon una auténtica nación, mucho más unida y homogénea que sus asociaciones políticas de los tiempos históricos.
\vs p079 6:10 \li{3.}\bibemph{Espiritual}. Durante la era de las emigraciones anditas, los chinos se contaban entre los pueblos más espirituales de la tierra. Su prolongada observancia del culto a la Verdad Única, proclamada por Singlangtón, los mantuvo por delante de la mayoría de las otras razas. El estímulo de una religión progresiva y avanzada es a menudo un factor decisivo en el desarrollo cultural; así pues, mientras que la India languidecía, China siguió avanzando bajo el vigorizante incentivo de una religión en la que la verdad se consagró como la Deidad suprema.
\vs p079 6:11 Esta adoración de la verdad estimulaba la investigación y la audaz exploración de las leyes de la naturaleza y de los potenciales de la humanidad. Incluso los chinos de hace seis mil años seguían aún siendo estudiantes ávidos y dinámicos en su búsqueda de la verdad.
\vs p079 6:12 \li{4.}\bibemph{Geográfico}. China está protegida al oeste por las montañas y, al este, por el Pacífico. Solamente en el norte está el camino abierto a cualquier ataque, y, desde los días del hombre rojo hasta la llegada de los posteriores descendientes de los anditas, el norte no estuvo ocupado por ninguna raza agresiva.
\vs p079 6:13 Y de no haber sido por las barreras montañosas y el posterior declive de su cultura espiritual, la raza amarilla habría sin duda atraído hacia sí a la mayor parte de la emigración andita desde el Turquestán e, indiscutiblemente, habría dominado rápidamente la civilización mundial.
\usection{7. LOS ANDITAS ENTRAN EN CHINA}
\vs p079 7:1 Hace unos quince mil años, los anditas, en un número considerable, atravesaron el desfiladero de Ti Tao y se dispersaron por el alto valle del río Amarillo entre los asentamientos chinos de Kansu. De allí penetraron por el este hasta Honan, donde se hallaban las comunidades más avanzadas. Esta infiltración desde el oeste era aproximadamente mitad andonita y mitad andita.
\vs p079 7:2 Los centros norteños de cultura situados a lo largo del río Amarillo siempre habían sido más avanzados que los asentamientos sureños del río Yangtzé. Pocos miles de años después de la llegada de estos escasos mortales mejor dotados, los asentamientos a lo largo del río Amarillo habían progresado por delante de los poblados del río Yangtzé y habían conseguido una posición avanzada en comparación con sus hermanos del sur, algo que han mantenido desde entonces.
\vs p079 7:3 \pc No se trataba de que hubiese un gran número de anditas ni que su cultura fuese tan superior, pero su mestizaje con ellos produjo una estirpe más versátil. Los chinos del norte recibieron justo la suficiente sangre andita como para estimular levemente sus mentes, innatamente capaces, pero no la suficiente como para prender en ellos la curiosidad inquieta y exploratoria tan característica de las razas blancas del norte. Esta infusión limitada de la herencia andita fue menos preocupante para la estabilidad innata del tipo sangik.
\vs p079 7:4 \pc Las oleadas de anditas que llegarían más tarde trajeron consigo algunos de los avances culturales de Mesopotamia; esto es particularmente cierto de las últimas oleadas migratorias provenientes del oeste, que mejoraron considerablemente las prácticas económicas y educativas de los chinos del norte; y, aunque su influjo sobre la cultura religiosa de la raza amarilla fue efímera, sus descendientes posteriores contribuyeron bastante al despertar espiritual que seguiría. Pero los relatos anditas sobre la belleza del Edén y Dalamatia ciertamente influyeron en las tradiciones chinas; las primitivas leyendas chinas situaban “la tierra de los dioses” en el oeste.
\vs p079 7:5 El pueblo chino no comenzó a construir ciudades ni a dedicarse a la manufactura hasta después del año 10\,000 a. C., con posterioridad a los cambios climáticos en el Turquestán y a la llegada de los últimos inmigrantes anditas. La infusión de esta nueva sangre no contribuyó mucho a la civilización del hombre amarillo, pero sí estimuló a un mayor y más rápido desarrollo de las tendencias latentes de las estirpes superiores chinas. Desde Honan hasta Shensi, el potencial de una civilización avanzada comenzaba a dar sus frutos. La metalurgia y todas las artes de la manufactura se remontan a estos días.
\vs p079 7:6 Las semejanzas entre algunos de los métodos chinos y mesopotámicos primitivos respecto al cálculo del tiempo, la astronomía y la administración del gobierno se debían a las relaciones comerciales entre estos dos centros remotamente situados. Los comerciantes chinos recorrían las rutas terrestres desde el Turquestán hasta Mesopotamia, incluso en los días de los sumerios. No se trataba de un intercambio unilateral ---el valle del Éufrates se benefició ampliamente de él, al igual que los pueblos de la llanura del Ganges---. Pero los cambios climáticos y las invasiones nómadas del tercer milenio a. C. redujeron, en gran medida, el volumen del comercio por los senderos de caravanas de Asia central.
\usection{8. LA POSTERIOR CIVILIZACIÓN CHINA}
\vs p079 8:1 Mientras el hombre rojo padeció las consecuencias de demasiadas guerras, no es del todo impropio decir que el desarrollo de la estructura estatal entre los chinos se retrasó por la exhaustividad de su conquista de Asia. Tenían un enorme potencial de solidaridad racial, pero no logró desarrollarse convenientemente porque les faltó el continuado incentivo que daba el constante peligro de una agresión exterior.
\vs p079 8:2 Al finalizar la conquista del Asia oriental, el antiguo estado militar se desintegró de forma gradual ---se olvidaron las guerras del pasado---. De las luchas épicas con la raza roja persistió la tradición borrosa de un ancestral enfrentamiento con los pueblos arqueros. Los chinos se entregaron tempranamente a actividades agrícolas, lo que también incrementó sus inclinaciones pacíficas, al tiempo que una población bien por debajo de la ratio tierra\hyp{}hombre para la agricultura contribuyó más aún al creciente carácter pacífico del país.
\vs p079 8:3 La conciencia de los logros del pasado (algo menor en la actualidad), el conservadurismo de un pueblo mayoritariamente agrícola y una vida familiar bien desarrollada se correspondieron con la veneración a los ancestros, que culminó en la costumbre de honrar a los hombres del pasado hasta tal punto que rayaba en la adoración. En Europa, entre las razas blancas, prevaleció una actitud muy similar durante unos quinientos años después de la desintegración de la civilización grecorromana.
\vs p079 8:4 La creencia y la adoración de la “Verdad Única”, tal como Singlangtón la enseñó, nunca desapareció del todo; pero, a medida que trascurrió el tiempo, la búsqueda de una verdad, nueva y superior, quedó relegada debido a la creciente tendencia a venerar la verdad establecida. Lentamente, el genio de la raza amarilla se desvió de la exploración de lo desconocido hacia la preservación de lo conocido. Y esta es la razón del estancamiento de la que había sido la civilización que más rápidamente había progresado en el mundo.
\vs p079 8:5 \pc Entre los años 4000 y 500 a. C., se consumó la reunificación política de la raza amarilla, pero ya se había efectuado la unión cultural de los centros de los ríos Amarillo y el Yangtzé. Esta reunificación política de los últimos grupos tribales no estuvo exenta de conflictos, pero la sociedad siguió con un mal concepto de la guerra. El culto a los ancestros, el incremento de los dialectos y la falta de llamamiento a la acción militar durante miles y miles de años lo habían convertido en un pueblo extremadamente pacífico.
\vs p079 8:6 A pesar de no cumplir con las expectativas del desarrollo temprano de un Estado avanzado, la raza amarilla sí progresó paulatinamente en la consecución de las artes de la civilización, especialmente en los ámbitos de la agricultura y la horticultura. Los problemas hidráulicos a los que se enfrentaron los agricultores en Shensi y Honan tuvieron que resolverse acudiendo a la cooperación colectiva. Esas dificultades de riego y de la conservación de los suelos contribuyeron, en no poca medida, al desarrollo de la interdependencia, con el consiguiente impulso de la paz entre los distintos grupos de agricultores.
\vs p079 8:7 Pronto, avances en la escritura, junto con la creación de escuelas, colaboraron a la difusión del conocimiento en una escala sin precedentes. Si bien, la complejidad del sistema de escritura ideográfica restringió el número de clases cultas, a pesar de la temprana aparición de la imprenta. Y, por encima de todo lo demás, continuó, a buen ritmo, el proceso de normalización social y de dogmatización religioso\hyp{}filosófica. El desarrollo religioso de la veneración de los ancestros se complicó más debido a un aluvión de supersticiones relacionadas con el culto a la naturaleza, pero, en la adoración imperial de Shang\hyp{}ti, quedaban, preservados, persistentes vestigios de un auténtico concepto de Dios.
\vs p079 8:8 La gran debilidad de la veneración de los ancestros está en el hecho de que promueve una filosofía de carácter retrospectivo. Aunque sea acertado cosechar sabiduría del pasado, es una insensatez considerar el pasado como fuente exclusiva de la verdad. La verdad es relativa y expansible, siempre \bibemph{vive} en el presente, llegando a tener una nueva expresión en cada generación de hombres ---incluso en cada vida humana---.
\vs p079 8:9 La gran fortaleza de la veneración de los ancestros es el valor que dicha actitud atribuye a la familia. La asombrosa estabilidad y subsistencia de la cultura china son una consecuencia de la posición primordial que se concedía a la familia, ya que la civilización depende directamente de su eficaz funcionamiento. Y, en China, la familia alcanzó una relevancia social, incluso una significación religiosa, que muy pocos otros pueblos lograron.
\vs p079 8:10 La devoción filial y la lealtad a la familia que el creciente culto de adoración a los ancestros exigía garantizó el fomento de relaciones familiares de orden superior y de grupos familiares perdurables, todo lo cual propició la aparición de los siguientes factores, que preservaban la civilización:
\vs p079 8:11 \li{1.}La conservación de propiedades y riquezas.
\vs p079 8:12 \li{2.}El intercambio de las experiencias de más de una generación.
\vs p079 8:13 \li{3.}La eficaz educación de los niños en las artes y las ciencias del pasado.
\vs p079 8:14 \li{4.}El desarrollo de un fuerte sentido del deber, el fortalecimiento de la moral y el incremento de la sensibilidad ética.
\vs p079 8:15 \pc El período formativo de la civilización china, que se inaugura con la llegada de los anditas, continúa hasta el gran despertar ético, moral y semirreligioso del siglo VI a. C. La tradición china preserva la borrosa crónica de su pasado evolutivo; la transición de la familia matriarcal a la paternal, el establecimiento de la agricultura, el desarrollo de la arquitectura, la iniciación de la industria ---todo esto se narra siguiendo una sucesión---. Y esta historia presenta, con mayor exactitud que cualquier otro relato similar, la imagen del magnífico ascenso de un pueblo superior desde los niveles de la barbarie. Durante este período, pasaron de una sociedad agrícola primitiva a una organización social superior que incluía ciudades, manufactura, metalurgia, intercambio comercial, gobierno, escritura, matemáticas, arte, ciencias e imprenta.
\vs p079 8:16 Y así ha persistido la ancestral civilización de la raza amarilla a través de los siglos. Han transcurrido casi cuarenta mil años desde que se produjeron los primeros avances importantes en la cultura china y, a pesar de muchos retrocesos, la civilización de los hijos de Han es la que más se acerca de todas a la expresión de una imagen ininterrumpida de progreso continuo hasta llegar a los tiempos del siglo XX. El desarrollo religioso y mecánico de las razas blancas ha sido de un elevado orden, pero estas razas nunca han destacado sobre los chinos en lealtad familiar, en ética de grupo o en moral personal.
\vs p079 8:17 Esta ancestral cultura ha contribuido mucho a la felicidad humana; millones de seres humanos han vivido y muerto, bendecidos por sus logros. Durante siglos esta gran civilización se ha apoyado en los laureles del pasado, pero está ahora volviendo a despertar para contemplar nuevamente la meta suprema de la existencia mortal, retomando la incansable lucha por el progreso sin fin.
\vsetoff
\vs p079 8:18 [Exposición de un arcángel de Nebadón.]
