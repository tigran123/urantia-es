\upaper{160}{Rodán de Alejandría}
\author{Comisión de seres intermedios}
\vs p160 0:1 El domingo, 18 de septiembre, por la mañana, Andrés anunció que no se haría ningún plan de trabajo para la semana siguiente. Todos los apóstoles, excepto Natanael y Tomás, fueron a sus casas para visitar a sus familias o a quedarse con amigos. Esa semana Jesús disfrutó de un período de descanso, casi total, pero Natanael y Tomás estuvieron ocupados con sus charlas con cierto filósofo griego de Alejandría, llamado Rodán. Recientemente, este griego se había convertido en discípulo de Jesús gracias a las enseñanzas de uno de los compañeros de Abner, que había llevado a cabo una misión en Alejandría. Rodán estaba entonces seriamente ocupado en la tarea de armonizar su filosofía de la vida con las nuevas enseñanzas religiosas de Jesús, y había venido a Magadán esperando poder hablar con el Maestro sobre estas cuestiones. Asimismo, deseaba asegurarse de tener, de primera mano, una versión genuina del evangelio por parte de Jesús o de alguno de sus apóstoles. Aunque el Maestro declinó entrar en conversaciones con Rodán, sí lo recibió gentilmente y dispuso, de inmediato, que Natanael y Tomás escucharan todo lo que Rodán tenía que decirles y ellos, a su vez, hablarles del evangelio.
\usection{1. LA FILOSOFÍA GRIEGA DE RODÁN}
\vs p160 1:1 El lunes por la mañana temprano, Rodán empezó a impartir una serie de diez discursos dirigidos a Natanael, a Tomás y a un grupo de unas dos docenas de creyentes que casualmente se encontraban en Magadán. Estas charlas, resumidas, combinadas y reformuladas en términos modernos, exponen, para su consideración, los siguientes pensamientos:
\vs p160 1:2 \pc La vida humana consiste en tres grandes estímulos: impulsos, deseos y alicientes. Un carácter fuerte, una poderosa personalidad, solo se adquiere al convertir los impulsos naturales en el arte social de la vida, transformando deseos ordinarios en esos superiores anhelos que llevan a logros perdurables y transfiriendo nuestras ideas convencionales y establecidas a los ámbitos más elevados de otras ideas inexploradas y de ideales por descubrir.
\vs p160 1:3 Cuanto más compleja llegue a ser la civilización, más difícil resultará el arte de vivir. Cuanto más rápidos se produzcan los cambios en los usos sociales, más complicada será la labor de desarrollar el carácter. Si el progreso ha de continuar, la humanidad, cada diez generaciones, debe aprender nuevamente este arte de vivir. Y si el hombre se vuelve tan ingenioso que incremente rápidamente las complejidades de la sociedad, habrá necesidad de reaprender dicho arte en menos tiempo, quizás en cada generación. Si la evolución del arte de vivir no logra acompasarse con los conocimientos de la existencia, la humanidad revertirá aceleradamente al simple impulso de vivir ---esto es, a la satisfacción de sus actuales deseos---. De esta manera, la humanidad continuará siendo inmadura; la sociedad no conseguirá crecer hasta su plena madurez.
\vs p160 1:4 La madurez social es equiparable al grado en el que el hombre esté dispuesto a renunciar a la gratificación de sus meros deseos pasajeros y ordinarios para contemplar anhelos superiores, ya que este esfuerzo por conseguirlos le proporcionará grandes satisfacciones conforme avanza paulatinamente hacia metas permanentes. Pero la verdadera insignia de la madurez social es la predisposición de un pueblo a renunciar a su derecho a vivir en la tranquilidad y en la complacencia, que le proporcionan los cómodos estándares de sus creencias establecidas y de sus ideas convencionales, mientras se entrega al inquietante incentivo, que le requiere más energía, de perseguir posibilidades inexploradas y al logro de las metas, aún por descubrir, de las sublimes realidades espirituales.
\vs p160 1:5 Los animales responden noblemente al impulso de la vida. En su mayoría, la humanidad también experimenta ese impulso animal, pero solo el hombre puede adquirir el arte de vivir. Los animales conocen únicamente este estímulo ciego e instintivo, que el hombre es capaz de trascender. El hombre puede optar por vivir en el plano elevado del arte inteligente, incluso en el plano de la felicidad celestial y del éxtasis espiritual. Los animales no se plantean preguntas sobre el propósito de la vida; por lo cual, nunca se angustian ni cometen suicidio. Entre los hombres, el suicidio da testimonio de que tales seres han salido de la etapa puramente animal de la existencia y del hecho de que, en su búsqueda, no han logrado alcanzar los niveles artísticos de la experiencia humana. Los animales no conocen el significado de la vida; el hombre no solo posee la capacidad para reconocer los valores y comprender los significados, sino que es también consciente del significado de los significados ---es consciente de su propia percepción---.
\vs p160 1:6 Cuando los hombres se atreven a renunciar a una vida de estímulos naturales para entregarse al trepidante arte de vivir y a la lógica incierta, cabe esperar que padezcan las dificultades emocionales que resultan de asumir tal riesgo ---conflictos, infelicidad e incertidumbres--- al menos hasta el momento en que logran cierto grado de madurez intelectual y emocional. El desaliento, la preocupación y la indolencia constituyen una clara evidencia de la inmadurez moral. La sociedad humana afronta dos problemas: lograr la madurez del individuo y la madurez de la raza. El ser humano maduro comienza pronto a mirar a todos los demás mortales con sentimientos de ternura y con emociones de tolerancia. Cuando han madurado, los hombres perciben a sus semejantes inmaduros con el mismo amor y consideración que los padres sienten por sus hijos.
\vs p160 1:7 Una vida victoriosa no consiste ni más ni menos que en el arte de dominar métodos que den respuesta a los problemas ordinarios. El primer paso en la solución de cualquier problema es localizarlo, aislarlo y reconocer con franqueza su naturaleza y gravedad. Se puede cometer la gran equivocación, cuando los problemas de la vida despiertan en nosotros profundos miedos, de negarse a aceptarlos. Igualmente, cuando reconocer nuestras dificultades supone mermar nuestra vanagloria, por mucho tiempo acariciada, admitir la envidia o abandonar prejuicios profundamente arraigados, puede suceder que la persona corriente prefiera aferrarse a sus antiguas ilusiones de seguridad y a los falsos sentimientos de certeza largamente atesorados. Solo una persona valiente es capaz de admitir honestamente, y afrontar sin temor, lo que una mente sincera y lógica descubre.
\vs p160 1:8 Para poder resolver un problema con sensatez y certeramente se precisa que la mente esté libre de parcialidad, pasión y de cualquier otro prejuicio puramente personal que pueda interferir con la búsqueda desinteresada de los verdaderos elementos que componen dicho problema. Se requiere valor y sinceridad para resolver estas dificultades que se presentan en la vida. Solo las personas honestas y valientes tienen el arrojo de seguir adelante ante el desconcertante y confuso laberinto de la vida y llegar adonde la mente lógica y sin temor les lleve. Y esta liberación de la mente y del alma nunca se llevará a cabo sin el impulso de un entusiasmo inteligente que linde con el fervor religioso. Se requiere el incentivo de un gran ideal para alentar al hombre a alcanzar una meta que está colmada de arduos problemas materiales y de múltiples riesgos intelectuales.
\vs p160 1:9 Aunque estéis convenientemente dotados para hacer frente a las situaciones difíciles de la vida, no esperéis conseguir mucho éxito a no ser que estéis equipados con esa sensatez de mente y ese encanto personal que os capaciten para ganar el apoyo y la cooperación fervientes de vuestros semejantes. No lograréis un elevado grado de éxito ni en vuestro trabajo secular ni en el religioso, a no ser que aprendáis a persuadir a vuestros semejantes y os ganéis su lealtad. Sencillamente, debéis poseer tacto y tolerancia.
\vs p160 1:10 \pc Pero aprendí de Jesús el mejor de los métodos para solucionar los problemas. Me refiero a algo que él practica asiduamente y que con tanta constancia os ha enseñado: aislarse para meditar y adorar. En este hábito tan frecuente de Jesús de apartarse de todos para estar en comunicación con el Padre de los cielos, se ha de encontrar el modo, ya no solo de acumular la fuerza y la sabiduría precisas para afrontar los conflictos ordinarios de la vida, sino también de adquirir la energía necesaria y solucionar los problemas de carácter moral y espiritual del orden más elevado. Pero, aunque esta forma de resolver los problemas sea conveniente, jamás compensará los defectos intrínsecos de la persona ni remediarán la ausencia del hambre y la sed de la verdadera rectitud.
\vs p160 1:11 Estoy profundamente impresionado con la costumbre de Jesús de marcharse periódicamente a solas para analizar los problemas de la vida; buscar nuevas fuentes de sabiduría y energías para hacer frente a las múltiples exigencias del servicio social; revitalizarse y profundizar en el supremo propósito de la vida, haciendo que la persona total tome conciencia de su contacto con la divinidad; llegar a captar nuevos y mejores métodos de adaptarse a las situaciones, siempre cambiantes, de la existencia en la vida; llevar a cabo esas vitales reconstrucciones y reajustes de las actitudes personales, que tan esenciales son para lograr una mejor percepción de todo lo que es valioso y real; y, hacer todo ello, con la mirada atentamente puesta en la gloria de Dios, susurrando sinceramente a los cielos la oración preferida de vuestro Maestro: “Que no se haga mi voluntad, sino la tuya”.
\vs p160 1:12 Esta práctica de adoración del Maestro trae consigo esa relajación que renueva la mente, esa iluminación que inspira el alma, ese arrojo que permite afrontar valientemente nuestros propios problemas, ese entendimiento de uno mismo que borra el temor extenuante y esa conciencia de unión con la divinidad que dota al hombre de la seguridad necesaria para atreverse a ser como Dios. La relajación que conlleva la adoración, o la comunión espiritual, tal como la practica el Maestro, alivia la tensión, suprime los conflictos y aumenta poderosamente todos los recursos de la persona. Y toda esta filosofía, más el evangelio del reino, constituyen la nueva religión, tal como yo la entiendo.
\vs p160 1:13 \pc El prejuicio ciega al alma y le impide reconocer la verdad. El prejuicio solo puede suprimirse mediante la sincera devoción del alma a una causa general que incluya a todos nuestros semejantes. El prejuicio está inseparablemente ligado al egoísmo y únicamente puede eliminarse si se abandona el interés propio a favor de la satisfacción de servir a una causa que no solo sea más grande que uno mismo, sino más grande que la humanidad ---intentar conocer a Dios, lograr la divinidad---. La madurez de la persona se hace patente en la transformación del deseo humano, que busca constantemente conseguir esos valores de mayor elevación y más divinamente reales.
\vs p160 1:14 En un mundo en continuo cambio, en medio de un orden social en evolución, es imposible mantener unas metas, fijas y estables, que os lleven a vuestro destino. La estabilidad de la persona solo puede experimentarse por quienes han descubierto y recibido con alegría al Dios vivo como una meta eterna a alcanzar por toda la eternidad. Y así, transferir la meta del tiempo a la eternidad, de la tierra al Paraíso, de lo humano a lo divino, requiere que el hombre se regenere, se convierta, nazca nuevamente; que se vuelva un hijo creado de nuevo del espíritu divino; que logre su entrada en la hermandad del reino de los cielos. Las filosofías y las religiones que no aborden estos ideales están desprovistas de madurez. La filosofía que yo imparto, y que está vinculada al evangelio que vosotros predicáis, representa la nueva religión de la madurez, el ideal de todas las generaciones futuras. Y esto es verdad porque nuestro ideal es final, infalible, eterno, universal, absoluto e infinito.
\vs p160 1:15 Mi filosofía me instó a buscar realidades que conllevasen un verdadero logro: la meta de la madurez. Pero mi incentivo resultó impotente; mi búsqueda carecía de una fuerza impulsora y adolecía de una dirección definida. Estas deficiencias se compensaron sobradamente gracias a este nuevo evangelio de Jesús, con su realce de las percepciones, con su elevación de los ideales y con sus metas perdurables. Sin dudas ni aprehensión, puedo ahora emprender con entusiasmo la aventura eterna.
\usection{2. EL ARTE DE VIVIR}
\vs p160 2:1 Existen solamente dos maneras en las que los mortales pueden convivir entre sí: la material o animal y la espiritual o humana. Mediante el uso de señales y sonidos, los animales pueden comunicarse entre sí de un modo limitado. Pero estas formas de comunicación no transmiten significados, valores o ideas. La singular diferencia entre el hombre y el animal es que el hombre puede comunicarse con sus semejantes mediante \bibemph{símbolos} que, sin lugar a dudas, designan e identifican significados, valores, ideas e incluso ideales.
\vs p160 2:2 Puesto que los animales no pueden comunicar ideas entre ellos, no pueden desarrollarse como persona. El hombre desarrolla su persona porque puede, por tanto, comunicarse con sus semejantes respecto a ideas y a ideales.
\vs p160 2:3 Es esta capacidad para comunicar y compartir significados la que constituye la cultura humana y permite al hombre, a través de las relaciones sociales, construir civilizaciones. El conocimiento y la sabiduría se acumulan debido a esta habilidad del hombre para comunicar tales ideas a las generaciones venideras. Y, de ahí, surge la actividad cultural de la raza: el arte, la ciencia, la religión y la filosofía.
\vs p160 2:4 La comunicación simbólica entre los seres humanos da lugar a la aparición de los grupos sociales. El más idóneo de todos los grupos sociales es la familia, más específicamente la de los \bibemph{dos padres}. El afecto personal es el nexo espiritual que mantiene la unidad de estas relaciones humanas. Una relación igualmente ideal es posible también entre dos personas del mismo sexo, tal como lo ilustra suficientemente la devoción que se profesan las auténticas amistades.
\vs p160 2:5 Estas relaciones de amistad y afecto mutuo son socializadoras y ennoblecedoras porque favorecen y facilitan los siguientes factores, que resultan esenciales en los niveles superiores del arte del vivir:
\vs p160 2:6 \li{1.}\bibemph{La mutua expresión personal y entendimiento}. La expresión de muchos nobles impulsos humanos perece porque no hay nadie para oírla. Ciertamente, no es bueno que el hombre esté solo. Un cierto grado de reconocimiento y reafirmación resulta esencial para el desarrollo del carácter humano. Sin el amor genuino que parte del hogar, ningún niño puede llegar a desarrollar por completo un carácter normal. El carácter es algo más que mente y sentimiento moral. De todas las relaciones sociales destinadas para desarrollar el carácter, los más eficientes e ideales son los lazos de amistad, y con estos el cariño y la comprensión, entre el hombre y la mujer, en el mutuo vínculo del matrimonio inteligente. El matrimonio, con las diversas relaciones que entraña, es unión social para propiciar esos hermosos impulsos y esos propósitos superiores que son indispensables para desarrollar un carácter fuerte. No dudo en glorificar, pues, la vida familiar porque vuestro Maestro ha elegido sabiamente la relación padre\hyp{}hijo como la piedra angular misma de su nuevo evangelio del reino. Y tal inigualable comunidad de afecto ---el hombre y la mujer en el cariñoso vínculo de los ideales más elevados del tiempo--- es una vivencia tan valiosa y gratificante que cualquier precio, cualquier sacrificio que se haga por adquirirla, merece la pena.
\vs p160 2:7 \li{2.}\bibemph{La unión de las almas} ---\bibemph{la activación de la sabiduría}---. Tarde o temprano, cualquier ser humano adopta algún concepto de este mundo y determinada visión del próximo. Ahora bien, es posible, mediante las relaciones personales, unificar estas perspectivas de la existencia temporal y de las expectativas eternas. Con ello, la mente incrementa sus valores espirituales al ganar gran parte de la percepción del otro. Así pues, los hombres enriquecen su alma aunando sus respectivas posesiones espirituales. Igualmente, el hombre queda, pues, facultado para eludir la tendencia, siempre presente, de ser víctima de la distorsión de su perspectiva, de la parcialidad de su punto de vista y de la estrechez de su juicio. Solamente, mediante el contacto cercano con otras mentes, es posible prevenir el temor, la envidia y la vanagloria. Llamo vuestra atención hacia el hecho de que el Maestro nunca os envía solos a trabajar para la expansión del reino, sino siempre de dos en dos. Y, puesto que la sabiduría trasciende el conocimiento, se desprende que, al unir su sabiduría, el grupo social, pequeño o grande, comparte mutuamente todo su conocimiento.
\vs p160 2:8 \li{3.}\bibemph{El entusiasmo por la vida}. El aislamiento tiende a agotar el acopio de energía del alma. La relación con nuestros semejantes es esencial para mantener la pasión por la vida e indispensable para continuar luchando valientemente esas batallas que resultan de la ascensión a los niveles superiores de la existencia humana. La amistad fomenta el gozo y glorifica los triunfos de la vida. Las relaciones humanas, cuando son estrechas y están fundadas en el cariño, tienden a despojar al sufrimiento de su dolor y a las adversidades de gran parte de su amargura. La presencia de un amigo realza la belleza y enaltece cualquier bondad. Mediante símbolos inteligentes, el hombre es capaz de vigorizar y ampliar la capacidad de sus amigos para apreciar más la vida. Una de las eminentes glorias de la amistad humana es este poder y posibilidad de alentar la imaginación mutua. Hay un gran poder espiritual inherente en la conciencia de la devoción incondicional a una causa común, en la lealtad mutua a una Deidad cósmica.
\vs p160 2:9 \li{4.}\bibemph{La defensa reforzada contra todo mal}. La relación entre las personas y el afecto mutuo representan un seguro eficaz contra el mal. Las dificultades, la aflicción, la decepción y la derrota son más dolorosas y descorazonadoras cuando se padecen a solas. Las relaciones personales no hacen que el mal se transforme en rectitud, pero contribuyen bastante a aliviar el dolor del aguijón. Vuestro Maestro dijo: “Dichosos los afligidos” ---si hay cerca un amigo que los consuele---. Hay una fuerza positiva en el conocimiento de que vivís para el bienestar de los demás, y de que ellos, a su vez, viven para vuestro bienestar y avance. En el aislamiento, el hombre languidece. Invariablemente, los seres humanos, cuando solo ven los acontecimientos transitorios del tiempo, se desalientan. El presente, cuando se desconecta del pasado y del futuro, se vuelve exasperadamente trivial. Únicamente vislumbrar el círculo de la eternidad puede inspirar al hombre a hacer todo lo que esté en su mano y motivarlo a sacar lo mejor de él. Y cuando el hombre da lo mejor de sí mismo, vive altruistamente por el bien de los demás, de sus semejantes, peregrinos como él del tiempo y la eternidad.
\vs p160 2:10 \pc Repito, que una relación personal tan inspiradora y ennoblecedora encuentra su potencial idóneo en el vínculo humano del matrimonio. En verdad, hay muchas cosas que se pueden obtener fuera del matrimonio, y muchos, muchos matrimonios no llegan a rendir de manera alguna estos frutos morales y espirituales. En muchas ocasiones, hay quienes acceden al matrimonio buscando otros valores, que son de inferior índole a los superiores atributos de la madurez humana. El matrimonio ideal debe fundarse en algo más estable que en las fluctuaciones de los sentimientos y en la veleidad de la simple atracción sexual; debe basarse en una mutua y genuina devoción personal. Y, de esta manera, si podéis conformar estas pequeñas unidades de relaciones humanas tan aptas y dignas de confianza, cuando estas se agrupen en un todo, el mundo percibirá una estructura social grande y glorificada: la civilización de la madurez humana. Tal raza humana podría comenzar a hacer realidad, de alguna manera, el ideal de vuestro Maestro de “paz en la tierra y buena voluntad entre los hombres”. Aunque dicha sociedad no sería perfecta ni estaría por completo libre del mal, se aproximaría al menos a la estabilización de la madurez.
\usection{3. LOS ATRACTIVOS DE LA MADUREZ}
\vs p160 3:1 Esforzarse por conseguir la madurez precisa de trabajo y este requiere energía. ¿De dónde viene el poder para llevar todo esto a cabo? Las cosas físicas pueden darse por hecho, pero el Maestro bien lo ha dicho: “No solo de pan vive el hombre”. Al habérsenos concedido un cuerpo normal y una salud razonablemente buena, debemos ahora buscar esos atractivos que sirvan de estímulos para despertar las fuerzas espirituales durmientes del hombre. Jesús nos ha enseñado que Dios vive en el hombre; ¿cómo podemos entonces alentar al hombre para que ponga en libertad ese poder divino e infinito aprisionado en su alma? ¿Cómo estimularemos al hombre a que libere a Dios para que salga de repente y renueve nuestras propias almas, y, en su tránsito hacia el exterior, contribuya, pues, a iluminar, enaltecer y bendecir a un sinnúmero de otras almas? ¿Cuál es la mejor manera de despertar para el bien a estos poderes que yacen latentes en vuestras almas? De una cosa estoy seguro: la excitación emocional no constituye el estímulo espiritual ideal. La excitación no aumenta la energía; más bien, agota las fuerzas de la mente y del cuerpo. ¿De dónde viene entonces la energía para hacer estas grandes cosas? Mirad a vuestro Maestro. En este preciso momento, está fuera en las colinas llenándose de fuerza mientras nosotros nos encontramos aquí gastando energía. El secreto para resolver este problema por completo puede encontrarse en la comunión espiritual, en la adoración. Desde el punto de vista humano, es la conjunción de la meditación y la relajación. La meditación realiza el contacto de la mente con el espíritu; la relajación determina la capacidad para la receptividad espiritual. Y este intercambio de la debilidad por la fuerza, del miedo por la valentía, de la propia mente por la voluntad de Dios constituye la adoración. Al menos, esa es la manera en la que el filósofo lo ve.
\vs p160 3:2 Cuando estos actos se repiten con frecuencia, se cristalizan en hábitos, en unos hábitos reverenciales que fortalecen, y que acaban por desarrollar un carácter espiritual, que nuestros semejantes reconocen finalmente como una \bibemph{persona madura}. Al principio, estas prácticas son difíciles y llevan tiempo, pero cuando se convierten en habituales proporcionan reposo y ahorro de tiempo a la vez. Cuánto más compleja sea la sociedad, cuánto más se multipliquen los atractivos de la civilización, más apremio habrá, para quienes conozcan a Dios, de instaurar tales prácticas habituales de protección, encaminadas a conservar y aumentar sus energías espirituales.
\vs p160 3:3 Otro requisito para lograr la madurez es la adaptación cooperativa de los grupos sociales a un entorno que está en cambio continuo. El hombre inmaduro suscita el antagonismo de sus semejantes, mientras que el maduro recibe su entusiasta cooperación, multiplicando así, por muchas veces, el fruto del esfuerzo que ha realizado en la vida
\vs p160 3:4 Mi filosofía me dice que hay épocas en las que debo luchar, si es preciso, para defender mi concepto de la rectitud, pero no dudo de que el Maestro, al ser una clase de persona de mayor madurez, ganaría fácilmente y de forma airosa una victoria semejante usando su método superior y atrayente del tacto y la tolerancia. Con mucha frecuencia, cuando contendemos por lo que es justo, resulta que tanto el victorioso como el vencido sufren la derrota. Ayer mismo oí al Maestro decir que “El hombre sensato, cuando intenta entrar por una puerta cerrada, no destruye la puerta sino que busca más bien la llave con la que abrirla”. Muy a menudo, nos enzarzamos en luchas meramente para convencernos a nosotros mismos de que no tenemos miedo.
\vs p160 3:5 Este nuevo evangelio del reino presta un gran servicio al arte de vivir en la medida en que proporciona un incentivo nuevo y más enriquecedor para vivir de acuerdo con valores superiores. Muestra una meta hacia un destino nuevo y excelso, hacia el propósito supremo de vida. Y estos conceptos nuevos que encaminan nuestra existencia a una meta eterna y divina son, por sí mismos, cruciales estímulos que motivan la respuesta de lo mejor que habita en la naturaleza superior del hombre. En cada cumbre del pensamiento intelectual, hay relajación para la mente, fuerza para el alma y comunión para el espíritu. Desde esta posición ventajosa de la vida elevada, el hombre es capaz de trascender las molestias materiales de los niveles inferiores del pensamiento: las preocupaciones, los celos, la envidia, la venganza y el orgullo de la persona inmadura. Estas almas que escalan alto se liberan a sí mismas de una multitud de conflictos contrapuestos por las nimiedades de la vida, volviéndose, pues, libres para tener conciencia de las corrientes más altas del pensamiento espiritual y de la comunicación celestial. Pero el propósito de la vida debe ser celosamente protegido de la tentación de buscar logros fáciles y transitorios; asimismo, este debe cultivarse de tal manera que se vuelva inmune a las nefastas amenazas del fanatismo.
\usection{4. EL EQUILIBRIO DE LA MADUREZ}
\vs p160 4:1 Aunque tengáis la mirada puesta en el logro de las realidades eternas, debéis también hacer provisiones para las necesidades de la vida temporal. Aunque el espíritu sea nuestra meta, la carne es un hecho. Ocasionalmente, nuestras necesidades materiales se pueden resolver de manera accidental, pero en general, debemos trabajar inteligentemente para satisfacerlas. Los dos problemas fundamentales de la vida son: ganarse la vida temporal y alcanzar la supervivencia eterna. Pero incluso el problema de ganarse la vida precisa de la religión para resolverse de la forma más idónea. Ambos problemas son sumamente personales. En realidad, la verdadera religión no obra al margen de la individualidad.
\vs p160 4:2 \pc Los elementos esenciales de la vida temporal, tal como yo los percibo, son:
\vs p160 4:3 \li{1.}Buena salud física.
\vs p160 4:4 \li{2.}Pensamiento claro y limpio.
\vs p160 4:5 \li{3.}Aptitud y destreza.
\vs p160 4:6 \li{4.}Riqueza: los bienes de la vida.
\vs p160 4:7 \li{5.}Capacidad para soportar la derrota.
\vs p160 4:8 \li{6.}Cultura: educación y sabiduría.
\vs p160 4:9 \pc Incluso las decisiones que se precisan para mantener eficientemente un cuerpo saludable se toman mejor cuando se ven desde el punto de vista religioso de las enseñanzas de nuestro Maestro: que el cuerpo y la mente del hombre son la morada del don de los Dioses, el espíritu de Dios que se vuelve el espíritu del hombre. La mente del hombre se convierte, pues, en la mediadora entre las cosas materiales y las realidades espirituales.
\vs p160 4:10 \pc Se necesita inteligencia para asegurarse para sí una parte de las cosas deseables en la vida. Es totalmente erróneo suponer que la lealtad en la realización de nuestro trabajo diario se recompense con las riquezas. Exceptuando aquellas que se adquieren de forma puntual y fortuita, las recompensas materiales de la vida temporal fluyen por ciertos cauces, bien organizados, y solo los que tienen acceso a estos pueden contar con que sus esfuerzos temporales se vean convenientemente retribuidos. La pobreza siempre será el destino de quienes buscan la riqueza de forma aislada e individual. Por lo tanto, para ser prósperos en este mundo, es esencial planificar con sensatez. El éxito no solo exige dedicación al trabajo, sino también actuar dentro de uno de dichos cauces de la riqueza material. Si no sabéis actuar con acierto en este sentido, podéis dedicar la vida a vuestra generación, sin obtener a cambio ninguna recompensa material; si os habéis beneficiado fortuitamente del flujo de la riqueza, podréis vivir rodeados de lujo sin haber hecho nada loable por vuestros semejantes.
\vs p160 4:11 La aptitud se hereda, mientras que la destreza se adquiere. La vida no es real para quien no sepa hacer alguna cosa bien, con pericia. La destreza es una de las verdaderas fuentes de satisfacción de la vida. La aptitud conlleva poseer el don de la previsión, tener una visión de futuro. No os dejéis engañar por la tentación de que obtendréis recompensas de acciones deshonestas; estad dispuestos a trabajar arduamente y con honradez para conseguir los beneficios que os correspondan. El hombre sensato es capaz de distinguir entre medios y fines; de lo contrario, a veces, el exceso de planificación para el futuro puede acabar con su propio elevado propósito. Si buscáis satisfacciones, tened como objetivo no solamente consumir sino producir.
\vs p160 4:12 Entrenad vuestra memoria para que mantenga la preciada custodia de esos capítulos de vuestra vida que os den fuerzas y que merezcan la pena ser recordados a voluntad para vuestra gratificación y edificación. Por consiguiente, construid para vosotros y en vosotros galerías personales que contengan belleza, bondad y grandeza artística. Pero los más nobles de todos los recuerdos son aquellos grandes momentos que atesoramos sobre alguna formidable amistad. Y, bajo el efecto liberador de la adoración espiritual, todos estos tesoros de la memoria irradian sus más inapreciables y magníficas influencias.
\vs p160 4:13 Pero en vuestra existencia, la vida se convertirá en un lastre si no aprendéis a hacer frente grácilmente a los fracasos. Hay un arte en la derrota que las almas nobles siempre adquieren; debéis saber cómo perder de forma animosa; debéis ser audaces ante la decepción. Nunca dudéis en admitir el fracaso. No tratéis de esconder la derrota tras sonrisas ilusorias y un radiante optimismo. Puede parecer una buena idea afirmar tener siempre éxito, pero el resultado último es devastador. Proceder así conduce directamente a la creación de un mundo de irrealidad y a encontrarse inevitablemente con grandes desilusiones.
\vs p160 4:14 El éxito puede generar valor y favorecer la confianza, pero la sabiduría se origina solamente cuando admitimos las consecuencias de nuestros fracasos. Los hombres que prefieren las ilusiones del optimismo a la realidad jamás llegarán a adquirir sabiduría. Solo podrán obtenerla aquellos que se enfrentan a los hechos y los adaptan a sus ideales. La sabiduría incluye tanto el hecho como el ideal y, por lo tanto, salva a sus devotos de esos dos extremos vacíos de la filosofía: el del hombre cuyo idealismo excluye los hechos y el del materialista desprovisto de visión espiritual. Esas almas aprensivas, que solo se mantienen en la lucha de la vida con la continua ayuda de sus falsas ilusiones de éxito, están destinadas a padecer el fracaso y la derrota cuando, en última instancia, despierten del mundo onírico de su propia imaginación.
\vs p160 4:15 Y es en esta cuestión de hacer frente y admitir el fracaso donde la visión de gran alcance de la religión ejerce su suprema influencia. El fracaso es simplemente un incidente educativo, un test instructivo para adquirir sabiduría, en la experiencia del hombre que busca a Dios y ha emprendido la aventura eterna de explorar un universo. Para tales hombres, la derrota no es sino un nuevo medio de alcanzar los niveles superiores del universo.
\vs p160 4:16 A la luz de la eternidad, la andadura de este hombre que busca a Dios puede resultar un gran éxito, aunque todo lo emprendido en su vida temporal pueda parecer un abrumador fracaso, con tal de que cada fracaso de vida haya podido generar la cultura de la sabiduría y el logro espiritual. No cometáis el error de confundir conocimiento, cultura y sabiduría. En la vida están relacionados, pero representan valores espirituales sumamente diferentes; la sabiduría siempre predomina sobre el conocimiento y constantemente glorifica la cultura.
\usection{5. LA RELIGIÓN DEL IDEAL}
\vs p160 5:1 Me habéis dicho que vuestro Maestro considera que la auténtica religión humana es la experiencia personal con las realidades espirituales. Yo la considero como la experiencia del hombre que responde a algo que le parece digno del tributo y la devoción de toda la humanidad. En este sentido, la religión simboliza la suprema devoción a lo que representa nuestro más elevado concepto de los ideales de la realidad y el mayor acercamiento de nuestras mentes hacia las posibilidades eternas del logro espiritual.
\vs p160 5:2 Cuando los hombres responden a la religión en un sentido tribal, nacional o racial es porque estiman que quienes no pertenecen a su grupo no son realmente humanos. En cualquier caso, siempre consideramos el destinatario de nuestra lealtad religiosa como digno de la veneración de todos los hombres. La religión nunca puede ser cuestión de mera creencia intelectual o de razonamiento filosófico; la religión es por y para siempre una forma de responder a las situaciones de la vida; es una especie de conducta. La religión incluye el pensar, el sentir y el actuar con veneración hacia una realidad que creemos merecedora de adoración universal.
\vs p160 5:3 Si, en tu experiencia, algo se ha convertido en religión, resulta evidente que ya te has vuelto un evangelista activo de esa religión, dado que valoras tu suprema percepción de la religión como digna de la adoración de toda la humanidad, de todas las inteligencias del universo. Si no eres un evangelista convencido y misionero de tu religión, te engañas a ti mismo, y lo que llamas religión no es sino una creencia tradicional o un simple sistema filosófico intelectual. Si tu religión es una experiencia espiritual, el destinatario de tu adoración debe serla realidad espiritual universal y el ideal de todas tus ideas espiritualizadas. Llamo intelectuales a todas las religiones basadas en el miedo, la emoción, la tradición y la filosofía, mientras que, verdaderas, a aquellas que se fundamentan en la verdadera experiencia espiritual. El destinatario de la devoción religiosa puede ser material o espiritual, verdadero o falso, real o irreal, humano o divino. Las religiones pueden ser, por consiguiente, buenas o malas.
\vs p160 5:4 La moral y la religión no son necesariamente lo mismo. Un conjunto de doctrinas morales, al tener un destinatario al que dirigir su adoración, puede llegar a ser una religión. Una religión, al perder su pasión universal por la lealtad y la devoción suprema, puede convertirse en sistema filosófico o en un código moral. Este algo, ser, estado político u orden de existencia o posibilidad de logro que constituyen el ideal supremo de la lealtad religiosa, y que es el receptor de la devoción religiosa de quienes adoran, es Dios, con independencia del nombre atribuido a este ideal de la realidad espiritual.
\vs p160 5:5 Las características sociales de una verdadera religión radican en el hecho de que esta busca, de forma invariable, convertir individualmente a las personas y transformar el mundo. La religión supone la existencia de ideales sin descubrir que trascienden en mucho las normas conocidas de la ética y de la moral, contenidas incluso en los usos sociales superiores de las instituciones de mayor madurez de la civilización. La religión trata de encontrar ideales por descubrir, realidades no exploradas, valores sobrehumanos, sabiduría divina y un auténtico logro espiritual. La verdadera religión hace todo esto; todas las otras creencias no son merecedoras de ese nombre. No podéis tener una religión genuinamente espiritual sin el ideal supremo y sublime de un Dios eterno. Una religión sin este Dios es una invención del hombre, una institución humana de creencias intelectuales inertes y de emotivas ceremonias sin sentido. Una religión puede afirmar que el objeto de su devoción es un gran ideal. Pero estos ideales son irreales e inalcanzables; tal pensamiento es ilusorio. Los únicos ideales susceptibles de lograrse humanamente son las realidades divinas de los valores infinitos que residen en el hecho espiritual del Dios eterno.
\vs p160 5:6 La palabra Dios ---la \bibemph{idea} de Dios a diferencia del \bibemph{ideal} de Dios--- puede llegar a formar parte de cualquier religión, al margen de lo pueril o falaz que esa religión pueda ser. Y esta idea de Dios puede llegar a ser cualquier cosa que los que la consideren elijan. Las religiones menores modelan sus ideas de Dios para satisfacer el estado natural del corazón humano; las religiones superiores requieren que el corazón humano se transforme para satisfacer las exigencias de los ideales de la verdadera religión.
\vs p160 5:7 \pc La religión de Jesús trasciende todos nuestros conceptos anteriores sobre la idea de la adoración, en cuanto que no solo caracteriza a su Padre como el ideal de la realidad infinita, sino que afirma, categóricamente, que esta fuente divina de valores y el centro eterno del universo es verdadera y personalmente accesible para cualquier criatura mortal que opte por entrar en el reino de los cielos en la tierra y que, al hacerlo, reconozca su filiación con Dios y la hermandad con los hombres. Sostengo que ese es el más elevado concepto de religión que el mundo haya conocido jamás, y declaro que no puede haber nunca ninguna noción de carácter superior, puesto que este evangelio engloba la infinitud de las realidades, la divinidad de los valores y la eternidad de los logros universales. Tal idea incluye la experiencia real de alcanzar este ideal supremo y último.
\vs p160 5:8 No estoy solamente fascinado por los consumados ideales de esta religión de vuestro Maestro, sino que me siento poderosamente movido a profesar mi creencia en su anuncio de que es factible alcanzar estos ideales de las realidades espirituales; que vosotros y yo podemos emprender esta aventura, larga y eterna, con la seguridad que él nos ofrece de la certeza de nuestra llegada última a las puertas del Paraíso. Hermanos míos, soy creyente, me he embarcado, y estoy de camino con vosotros, en esta aventura eterna. El Maestro dice que vino del Padre y que nos mostrará el camino. Estoy final y enteramente convencido de que dice la verdad, de que no se pueden alcanzar otros ideales de la realidad ni otros valores de la perfección que los del Padre eterno y universal.
\vs p160 5:9 Vengo, pues, a adorar, no simplemente al Dios de las existencias, sino al Dios de la posibilidad de todas las existencias futuras. Por consiguiente, vuestra devoción a un ideal supremo, si ese ideal es real, debe ser devoción a este Dios de los universos, pasados, presentes y futuros, de las cosas y de los seres. Y no hay otro Dios, porque no es posible que haya ningún otro Dios. Todos los otros dioses son quimeras de la imaginación, ilusiones de la mente mortal, distorsiones de la falsa lógica y engañosos ídolos de esos seres que los crean. Sí, podéis tener una religión sin este Dios, pero dicha religión estará falta de significado. Si intentáis sustituir la palabra Dios por este ideal del Dios vivo, os habréis engañado a vosotros mismos porque habéis antepuesto una idea a un ideal, a una realidad divina. Dichas creencias no son sino meras religiones ilusorias.
\vs p160 5:10 Veo en las enseñanzas de Jesús la religión en su máximo esplendor. Este evangelio nos capacita para buscar al verdadero Dios y encontrarlo. Pero, ¿estamos dispuestos a pagar el precio por entrar en el reino de los cielos? ¿Estamos dispuestos a nacer de nuevo? ¿A ser rehechos? ¿Deseamos someternos a este terrible y arduo proceso de destrucción de uno mismo y de reconstrucción del alma? Es que no ha dicho el Maestro: “El que quiera salvar su vida la perderá. No penséis que he venido a traer paz sino contienda para el alma”. Cierto, tras pagar el precio de nuestra dedicación a la voluntad del Padre, se experimenta una gran paz siempre que sigamos caminando por las sendas espirituales de la vida consagrada.
\vs p160 5:11 Y ahora estamos ciertamente abandonando los alicientes de un orden conocido de existencia para dedicarnos, sin reservas, a buscar los alicientes de un orden de existencia, desconocido e inexplorado, de una vida futura para aventurarnos en unos mundos espirituales que reflejan la idealización superior de la realidad divina. Y tratamos de encontrar esos símbolos significativos con los que transmitir a nuestros semejantes estos conceptos de la realidad de los ideales de la religión de Jesús, y no dejaremos de orar por ese día en que toda la humanidad comparta la emoción de la visión de esta suprema verdad. En este momento, el concepto central del Padre que albergamos en nuestros corazones es que Dios es espíritu; o, como lo manifestamos a nuestros semejantes, que Dios es amor.
\vs p160 5:12 La religión de Jesús requiere experiencia viva y espiritual. En otras religiones podrán concurrir creencias tradicionales, sentimientos de emoción, percepción filosófica y muchas otras cosas, pero en la enseñanza del Maestro es preciso alcanzar niveles que muestren nuestro progreso espiritual real.
\vs p160 5:13 Ser conscientes del impulso a ser como Dios no es verdadera religión. Sentir la emoción de adorar a Dios no es verdadera religión. Conocer las enseñanzas de renunciar a uno mismo para servir a Dios no es verdadera religión. Razonar sensatamente que esta religión es la mejor de todas no es la experiencia personal y espiritual de la religión. La verdadera religión alude al logro de un destino y de una realidad al igual que a la realidad e idealización de lo que aceptamos incondicionalmente por la fe. Y todo esto debe hacerse personal para nosotros a través de la revelación del Espíritu.
\vs p160 5:14 \pc Y de esta manera terminaron las disertaciones del filósofo griego, uno de los más grandes de su raza, que se había convertido en creyente del evangelio de Jesús.
