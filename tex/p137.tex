\upaper{137}{Tiempo de espera en Galilea}
\author{Comisión de seres intermedios}
\vs p137 0:1 El sábado 23 de febrero del año 26 d. C, por la mañana temprano, Jesús bajó de las colinas para reunirse con los seguidores de Juan acampados en Pella. Durante todo ese día, Jesús se mezcló con la muchedumbre. Asistió a un joven que se había herido en una caída y fue a la vecina aldea de Pella para llevárselo sano y salvo a sus padres.
\usection{1. ELECCIÓN DE LOS CUATRO PRIMEROS APÓSTOLES}
\vs p137 1:1 Durante este \bibemph{sabbat,} dos de los más destacados discípulos de Juan estuvieron bastante tiempo con Jesús. De todos los seguidores de Juan, alguien llamado Andrés era el que se había quedado más profundamente impresionado por Jesús; lo acompañó a Pella con el muchacho herido. En el camino de vuelta al emplazamiento de Juan, le hizo a Jesús muchas preguntas y, justo antes de llegar a su destino, se detuvieron para mantener una breve charla. En ese momento, Andrés dijo: “Te he observado desde que llegaste a Cafarnaúm, y creo que tú eres el nuevo Maestro, y aunque no entiendo todas tus enseñanzas, estoy decidido a seguirte; me gustaría sentarme a tus pies y aprender toda la verdad sobre el nuevo reino”. Y Jesús, dándole su caluroso beneplácito, acogió a Andrés como el primero de sus apóstoles, del grupo de los doce que trabajarían con él en la labor de instaurar el nuevo reino de Dios en el corazón de los hombres.
\vs p137 1:2 \pc Andrés era un observador silencioso, y creyente leal, de la misión de Juan, y tenía un hermano llamado Simón, muy capaz y entusiasta, que era uno de los discípulos más destacados de Juan. No sería inoportuno añadir que Simón era uno de los principales partidarios de Juan.
\vs p137 1:3 Poco después del regreso de Jesús y Andrés al campamento, Andrés buscó a Simón, su hermano, y, llevándolo aparte, le comunicó que estaba convencido de que Jesús era el gran Maestro, y que él ya se había comprometido a ser su discípulo. Continuó diciéndole que Jesús había aceptado su ofrecimiento de servicio, y le propuso que él (Simón) fuese también a Jesús y se ofreciese para servir en la fraternidad del nuevo reino. Simón dijo: “Desde que este hombre vino a trabajar a la factoría de Zebedeo, he pensado que era un enviado de Dios, pero, ¿qué hay de Juan? ¿Vamos a abandonarlo? ¿Crees que sería justo hacerlo?”. Acordaron, por tanto, ir enseguida a hablar con Juan, que se entristeció con la idea de perder a dos de sus más aptos hombres de confianza y más prometedores discípulos. No obstante, respondió con valentía a su consulta, diciendo: “Esto no es sino el principio; mi misión pronto acabará, y todos nos convertiremos en sus discípulos”. Entonces Andrés hizo señas a Jesús y le indicó aparte que su hermano deseaba sumarse al servicio del nuevo reino. Y al acoger a Simón como su segundo apóstol, Jesús le dijo: “Simón, tu entusiasmo es encomiable, pero es nocivo para el trabajo del reino. Te recomiendo que seas más comedido con lo que dices. Cambiaré tu nombre al de Pedro”.
\vs p137 1:4 \pc Los padres del joven herido, habitantes de Pella, habían pedido a Jesús que pasara la noche con ellos, haciendo uso de la casa como si fuera su propio hogar, y él había aceptado. Antes de dejar a Andrés y a su hermano, Jesús dijo: “Mañana temprano iremos a Galilea”.
\vs p137 1:5 \pc Cuando Jesús volvió a Pella para pasar la noche y, mientras que Andrés y Simón estaban aún comentando la naturaleza de su servicio en la instauración del reino venidero, llegaron al lugar Santiago y Juan, los hijos de Zebedeo, que regresaban de su larga e infructuosa búsqueda de Jesús en las colinas. Al oír decir a Simón Pedro cómo él y Andrés, su hermano, se habían convertido en los primeros instructores del nuevo reino en ser aceptados y que saldrían por la mañana temprano con su nuevo Maestro para Galilea, Santiago y Juan se entristecieron. Hacía tiempo que conocían a Jesús, y lo amaban. Lo habían estado buscando durante muchos días en las colinas, y, ahora, al volver, se habían enterado de que Jesús había preferido a otros antes que a ellos. Preguntaron adónde había ido Jesús y se apresuraron a ir en su busca.
\vs p137 1:6 Jesús estaba dormido cuando llegaron al lugar donde estaba albergado, pero lo despertaron, diciéndole: “¿Cómo es que mientras nosotros, que por tanto tiempo hemos convivido contigo, andábamos buscándote en las montañas, hayas preferido a otros antes que a nosotros, eligiendo a Andrés y a Simón como tus primeros acompañantes en el nuevo reino?”. Jesús les respondió: “Calmad vuestros corazones y preguntaos a vosotros mismos, ‘¿quién os dijo que fueseis en busca del Hijo del Hombre cuando él se ocupaba de los asuntos de su Padre?’”. Tras comentar los pormenores de su larga búsqueda en las colinas, Jesús les indicó además: “Debéis aprender a buscar el secreto del nuevo reino en vuestros corazones y no en las colinas. Lo que procurabais estaba ya en vuestras almas. Ciertamente, sois mis hermanos ---no necesitabais que yo os aceptara--- ya estabais en el reino, y debéis tener buen ánimo, mientras os preparáis para venir también con nosotros a Galilea mañana”. Juan se atrevió entonces a preguntarle: “Pero, Maestro, ¿seremos Santiago y yo compañeros tuyos en el nuevo reino, tal como lo son Andrés y Simón?”. Y Jesús, colocando una mano en el hombro de cada uno de ellos, les dijo: “Hermanos míos, ya estabais conmigo en el espíritu del reino, aun antes de que nadie más me pidiese que lo admitiera en él. Vosotros, hermanos míos, no necesitáis solicitar vuestra entrada en el reino; habéis estado en él conmigo desde el principio. Ante los hombres, es posible que otros se antepongan a vosotros, pero en mi corazón ya contaba con vosotros para los consejos del reino, incluso antes de que pensarais en pedírmelo. Y, aun así, vosotros podríais haber sido los primeros ante los hombres si no hubieseis estado ausentes, comprometidos en una tarea bienintencionada, pero que vosotros mismos os impusisteis, de salir a buscar a quien no estaba perdido. En el reino venidero, despreocupaos de las cosas que os inciten a la ansiedad, preocupaos, más bien, y en todo momento, de hacer únicamente la voluntad del Padre de los cielos”.
\vs p137 1:7 Santiago y Juan aceptaron de buen grado la reprimenda; nunca más tendrían celos de Andrés y Simón. Y, con sus otros dos compañeros apóstoles, se dispusieron a partir para Galilea a la mañana siguiente. Desde este día en adelante, se usaría la palabra “apóstol” para establecer una distinción entre la familia elegida de los hombres de confianza de Jesús y la inmensa multitud de discípulos creyentes que llegarían a seguirlo.
\vs p137 1:8 \pc Tarde esa noche, Santiago, Juan, Andrés y Simón sostuvieron una conversación con Juan el Bautista y, con lágrimas en los ojos pero la voz firme, el robusto profeta de Judea renunció a dos de sus discípulos principales para que se convirtieran en apóstoles del Príncipe galileo del reino venidero.
\usection{2. ELECCIÓN DE FELIPE Y NATANAEL}
\vs p137 2:1 En la mañana del domingo 24 de febrero del año 26 d. C., junto al río próximo a Pella, Jesús se despidió de Juan el Bautista, para no volver a verlo nunca más en la carne.
\vs p137 2:2 Ese día, conforme Jesús partía para Galilea con sus cuatro discípulos\hyp{}apóstoles, se formó un gran tumulto en el campamento de los seguidores de Juan. Se iba a producir la primera gran escisión. El día anterior, Juan había declarado expresamente a Andrés y a Esdras que Jesús era el Libertador. Andrés optó por seguir a Jesús, pero Esdras rechazó al afable carpintero de Nazaret, apelando a sus compañeros: “El profeta Daniel anuncia que el Hijo del Hombre vendrá sobre las nubes del cielo, con poder y gran gloria. Este carpintero galileo, este fabricante de embarcaciones de Cafarnaúm, no puede ser el Libertador. ¿Puede venir de Nazaret algún don divino? Este Jesús es pariente de Juan y, la mucha bondad de su corazón ha engañado a nuestro maestro. Permanezcamos al margen de este falso Mesías”. Cuando Juan reprendió a Esdras por estas palabras, este se alejó apresurado al sur llevándose a muchos discípulos. Y este grupo siguió bautizando en nombre de Juan, llegando a fundar con el tiempo un movimiento religioso, cuyos seguidores creían en Juan pero se negaban a aceptar a Jesús. Incluso hasta estos días persiste en Mesopotamia un resto de este grupo.
\vs p137 2:3 \pc Mientras se gestaba este conflicto entre los seguidores de Juan, Jesús y sus cuatro discípulos\hyp{}apóstoles ya habían recorrido un buen trecho para llegar a Galilea. Antes de cruzar el Jordán para ir a Nazaret, vía Naín, Jesús, mirando al frente, al camino, vio a un tal Felipe de Betsaida que se dirigía hacia ellos con un amigo. Jesús había conocido a Felipe en otro tiempo, y los cuatro nuevos apóstoles a su vez lo conocían bien. Iba de camino a Pella con su amigo Natanael para visitar a Juan y tener más información sobre la llegada del mencionado reino de Dios, y estuvo encantado de saludar a Jesús. Felipe admiraba a Jesús desde que este llegó por primera vez a Cafarnaúm. Pero Natanael, que vivía en Caná de Galilea, no lo conocía. Felipe se adelantó para saludar a sus amigos, mientras que Natanael descansaba bajo la sombra de un árbol al lado del camino.
\vs p137 2:4 Pedro se llevó a Felipe a un lado y le explicó que ellos, en alusión a él mismo, a Andrés, a Santiago y a Juan, se habían convertido en acompañantes de Jesús en el nuevo reino y le instó encarecidamente a ofrecer voluntariamente su servicio. Felipe se encontró ante un dilema. ¿Qué debía hacer? Aquí, sin previo aviso ---junto al camino que conducía al Jordán---, sentía la urgencia de tomar una decisión inmediata sobre la cuestión más trascendental de su vida. En ese momento, mientras él mantenía esta seria conversación con Pedro, Andrés y Juan, Jesús le indicaba a Santiago la ruta hacia Cafarnaún, pasando por Galilea. Finalmente, Andrés propuso a Felipe: “¿Por qué no preguntar al Maestro?”.
\vs p137 2:5 De pronto, Felipe comprendió que Jesús era en verdad un gran hombre, quizás el Mesías, y optó por atenerse a la decisión que Jesús tomara en esta cuestión; y se dirigió directamente hacia él, preguntándole: “Maestro, ¿debo unirme a Juan y a estos amigos míos que te siguen?”. Y Jesús le respondió: “Sígueme”. Felipe se emocionó ante la certeza de que había encontrado al Libertador.
\vs p137 2:6 \pc Felipe, entonces, hizo señas al grupo para que se quedasen donde estaban mientras él se apresuraba a dar noticia de su decisión a su amigo Natanael, que aún esperaba bajo el árbol de morera, dándole vueltas en su cabeza a las cosas que había oído sobre Juan el Bautista, el reino venidero y el Mesías esperado. Felipe interrumpió sus reflexiones, exclamando: “He encontrado al Libertador, a aquel de quien escribieron Moisés y los profetas y a quien Juan ha proclamado”. Natanael alzó la mirada e inquirió: “¿De dónde viene este maestro?”. Y Felipe respondió: “Es Jesús de Nazaret, el hijo de José, el carpintero, residente de Cafarnaúm desde no hace mucho”. Y entonces, algo consternado, Natanael preguntó: ¿De Nazaret puede salir algo bueno?”. Pero Felipe, tomándole del brazo, le dijo: “Ven y ve”.
\vs p137 2:7 Felipe llevó a Natanael hasta Jesús, que, mirando con benevolencia el rostro sincero de este escéptico, dijo: “¡Aquí hay un verdadero israelita en quien no hay engaño! Sígueme”. Y Natanael, volviéndose a Felipe, le dijo: “Tienes razón. Ciertamente él es preceptor de los hombres. Yo también lo seguiré, si soy digno”. Jesús asintió a Natanael, diciendo de nuevo: “Sígueme”.
\vs p137 2:8 \pc En ese momento, Jesús ya había congregado a la mitad de su futuro grupo de estrechos colaboradores, cinco que le habían conocido durante algún tiempo y, el otro, Natanael, que era un desconocido. Sin más demora, cruzaron el Jordán y, pasando por la aldea de Naín, llegaron a Nazaret avanzada la noche.
\vs p137 2:9 Todos pasaron la noche con José, en la casa en la que Jesús había pasado su adolescencia. Los acompañantes de Jesús no entendían bien la preocupación de su maestro por destruir cualquier vestigio que quedara allí de sus escritos, como los Diez Mandamientos y otras máximas y dichos. Pero esta forma de proceder, junto al hecho de que no volvieron a verle escribir nunca más ---salvo en el polvo o en la arena--- causó una profunda impresión en sus mentes.
\usection{3. VISITA A CAFARNAÚM}
\vs p137 3:1 Al día siguiente, Jesús envió a sus apóstoles a Caná, puesto que estaban todos invitados a la boda de una destacada joven de ese pueblo, mientras él se preparaba a hacerle una rápida visita a su madre en Cafarnaúm, deteniéndose en Magdala para ver a su hermano Judá.
\vs p137 3:2 Antes de salir de Nazaret, los nuevos acompañantes de Jesús relataron a José y a otros miembros de la familia de Jesús los magníficos acontecimientos del entonces pasado reciente y expresaron libremente su convencimiento de que Jesús era el libertador por tanto tiempo esperado. Y estos miembros de la familia de Jesús hablaron de todo esto, y José dijo: “Quizá, después de todo, nuestra madre tenía razón: tal vez nuestro extraño hermano sea el rey por venir”.
\vs p137 3:3 Judá había estado presente en el bautismo de Jesús y, con su hermano Santiago, se había convertido en un firme creyente de la misión de Jesús en la tierra. Pese a que tanto Santiago como Judá estaban muy desconcertados respecto a la naturaleza de la misión de su hermano, su madre había reavivado todas sus tempranas esperanzas de que Jesús era el Mesías, el hijo de David, y alentaba a sus hijos a que tuvieran fe en su hermano como libertador de Israel.
\vs p137 3:4 \pc Jesús llegó a Cafarnaúm el lunes por la noche, pero no fue a su propia casa, en la que vivían Santiago y su madre, sino directamente a la casa de Zebedeo. Todos sus amigos de Cafarnaúm notaron que se había producido en él un cambio grato y considerable. De nuevo, parecía estar relativamente contento, más parecido a como había sido en sus primeros años en Nazaret. Durante los años previos a su bautismo y a los períodos de retiro, justo antes y después, se había vuelto cada vez más serio y reservado. Ahora, a todos ellos les parecía que volvía a ser el mismo de antes. Tenía un porte majestuoso y un aspecto excelso, pero, una vez más, se mostraba libre de preocupaciones y feliz.
\vs p137 3:5 María estaba entusiasmada y esperanzada. Consideraba que la promesa de Gabriel estaba a punto de cumplirse. Esperaba que pronto toda Palestina se sobrecogiera y asombrara ante la revelación milagrosa de su hijo como rey celestial de los judíos. Si bien, a las numerosas preguntas que le hacían su madre, Santiago, Judá y Zebedeo, Jesús se limitaba a contestar con una sonrisa: “Es mejor quedarme aquí por algún tiempo; debo hacer la voluntad de mi Padre que está en los cielos”.
\vs p137 3:6 \pc El día siguiente, martes, se desplazaron todos a Caná para asistir al casamiento de Noemí, que se celebraría un día después. A pesar de que Jesús les había hecho reiteradas advertencias de que no dijeran nada a nadie de su misión “hasta que llegara la hora del Padre”, todos se empeñaron en difundir discretamente la noticia de que habían encontrado al Libertador. Estaban confiados en que Jesús asumiría su autoridad mesiánica en las próximas bodas de Caná, y que lo haría con un gran poder y un sublime esplendor. Recordaban lo que se les había comentado sobre los fenómenos que acompañaron a su bautismo, y creían que en el futuro su actuación en la tierra se caracterizaría por crecientes manifestaciones de portentos sobrenaturales y de demostraciones milagrosas. Por esta razón, toda la región se disponía a congregarse en Caná para la fiesta de bodas de Noemí y Johab, el hijo de Natán.
\vs p137 3:7 Hacía años que María no se encontraba tan dichosa. Se dirigió a Caná animosa, como una reina madre anhelante ante la coronación de su hijo. Nunca, desde que tenía trece años de edad, la familia y los amigos de Jesús le habían visto tan desenfadado y alegre, tan atento y comprensivo de los deseos y aspiraciones de sus acompañantes, tan conmovedoramente empático. Y, por lo tanto, murmuraban entre ellos, en pequeños grupos, preguntándose qué iba a suceder. ¿Qué sería lo próximo que haría esta persona tan poco convencional? ¿Cómo inauguraría la gloria del reino venidero? Y estaban todos entusiasmados con la idea de que presenciarían la revelación de la fuerza y el poder del Dios de Israel.
\usection{4. LAS BODAS DE CANÁ}
\vs p137 4:1 Hacia el mediodía del miércoles, habían llegado a Caná casi mil visitantes para asistir a las nupcias, más de cuatro veces el número fijado de invitados. La costumbre judía era celebrar los casamientos en miércoles, y las invitaciones se habían enviado a muchos sitios con un mes de antelación. Por la mañana y temprano por la tarde, aquello, más que unas bodas, parecía un acto público de recibimiento a Jesús. Todo el mundo quería saludar a este prácticamente famoso galileo, y él se mostraba de lo más cordial hacia todos, jóvenes y mayores, judíos y gentiles. Y todos estaban exultantes cuando Jesús accedió a encabezar el cortejo nupcial inicial.
\vs p137 4:2 Jesús ya era por completo consciente de su existencia humana, de su preexistencia divina y del estatus de sus naturalezas humana y divina en unión o fusión. Con perfecta compostura, él podía, en un determinado momento, adoptar un rol como ser humano o, de inmediato, asumir las prerrogativas personales de su naturaleza divina.
\vs p137 4:3 A medida que avanzaba el día, Jesús se fue percatando de que la gente estaba a la expectativa de que realizara algún prodigio; más concretamente, percibía que su familia y sus seis discípulos\hyp{}apóstoles esperaban que él anunciara su reino venidero, convenientemente, mediante alguna manifestación de carácter extraordinario y sobrenatural.
\vs p137 4:4 A primera hora de la tarde, María llamó a Santiago y, juntos, fueron a preguntarle a Jesús si no les podía decir en confianza a qué hora y en qué punto de las ceremonias de la boda había pensado manifestarse como “ser sobrenatural”. Apenas le mencionaron a Jesús estas cuestiones, vieron que habían suscitado su indignación característica. Solo dijo: “Si me amáis, estad dispuestos entonces a aguardar conmigo mientras espero la voluntad de mi Padre que está en los cielos”. Pero la elocuencia de su reprimenda se manifestaba más en la expresión de su rostro.
\vs p137 4:5 El Jesús humano se decepcionó bastante con el proceder de su madre, y reaccionó con seriedad ante su insinuación de que se permitiera realizar una demostración visible de su divinidad. Aquello era algo que él había decidido no hacer durante su reciente retiro en las montañas. Durante algunas horas, María estuvo muy deprimida, y le dijo a Santiago: “No puedo comprenderle; ¿qué significa todo esto? ¿Es que su extraña conducta no va a acabar nunca?”. Mientras Santiago y Judá trataron de consolar a su madre, Jesús se retiró una hora para estar en soledad. Pero volvió a la fiesta y de nuevo estuvo de buen ánimo y feliz.
\vs p137 4:6 \pc Las nupcias prosiguieron con una quietud expectante, pero la ceremonia llegó a su fin sin que el invitado de honor hiciera gesto alguno ni pronunciara una sola palabra. Entonces se rumoreó que el carpintero y fabricante de embarcaciones, anunciado por Juan como “el Libertador”, mostraría quién era en verdad durante los festejos de la noche, tal vez en la cena nupcial. Pero cualquier expectativa de tales demostraciones se borró de la mente de los seis discípulos\hyp{}apóstoles cuando los reunió un momento antes de dicha cena y muy seriamente les dijo: “No penséis que he venido aquí para obrar portentos y satisfacer a los curiosos o convencer a los incrédulos, sino más bien para aguardar la voluntad de nuestro Padre que está en los cielos”. Pero cuando María y los demás vieron a Jesús conversando con sus acompañantes, quedaron completamente convencidos de que algo extraordinario estaba a punto de ocurrir. Y todos se sentaron a disfrutar de la cena nupcial y de aquella noche en un ambiente festivo y de gran fraternidad.
\vs p137 4:7 \pc El padre del novio había previsto tener vino en abundancia para todos los invitados a la fiesta de bodas, pero, ¿cómo podía imaginarse que la boda de su hijo se convertiría en un acontecimiento tan íntimamente ligado a la esperada manifestación de Jesús como libertador mesiánico? Estaba encantado de tener el honor de contar entre sus invitados al célebre galileo, pero antes de que la cena nupcial terminase, los criados le dieron la preocupante noticia de que faltaba vino. En el momento en el que concluyó la cena formal y los invitados paseaban por el jardín, la madre del novio le confió a María que se habían agotado las existencias de vino. Y María, con seguridad, le dijo: “No os preocupéis. Hablaré con mi hijo; el nos ayudará”. Y, así pues, se atrevió a hablar con él, a pesar de la reprimenda recibida hacía solo unas horas.
\vs p137 4:8 Durante muchos años, María siempre había recurrido a la ayuda de Jesús ante cualquier crisis de su vida doméstica en Nazaret, de manera que resultaba muy lógico que pensase en él en este momento. Pero, en esta ocasión, existían otros motivos para que esta exigente madre acudiera a su hijo mayor. Jesús estaba solo, de pie, en un rincón del jardín cuando su madre se le acercó y le dijo: “Hijo mío, no tienen vino”. Y Jesús respondió: “Mi buena mujer, ¿qué tengo yo que ver con eso?”. María contestó: “Pero yo creo que ha llegado tu hora; ¿no puedes ayudarnos?”. Jesús respondió: “De nuevo manifiesto que no he venido para hacer las cosas de este modo. ¿Por qué me incomodas otra vez con tales asuntos?”. Entonces, rompiendo a llorar, María le imploró: “Pero, hijo mío, yo les prometí que nos ayudarías; ¿por favor, harías algo por mí?”. Habló Jesús entonces: “Mujer, ¿a qué viene que hagas tú tales promesas? Procura no hacer eso de nuevo. En todas las cosas, debemos aguardar la voluntad del Padre que está en los cielos”.
\vs p137 4:9 María, la madre de Jesús, estaba desconsolada; ¡se quedó atónita! Al verla allí frente a él, inmóvil, con las lágrimas resbalándole por el rostro, el corazón humano de Jesús se embargó de emoción hacia la mujer que le había dado a luz en la carne; e, inclinándose hacia adelante, dejó caer su mano tiernamente sobre su cabeza y le dijo: “Venga, venga madre María, no te aflijas por mis posibles duras palabras, pero ¿no te he dicho tantas veces que he venido solamente para hacer la voluntad de mi Padre celestial? Muy gustosamente haría lo que me pides si fuese parte de la voluntad de mi Padre---” y Jesús se detuvo de repente, vaciló. María pareció percatarse de que algo estaba sucediendo. Saltando, le rodeo el cuello con sus brazos, lo besó y fue corriendo hasta donde estaban los que servían, diciéndoles: “Haced todo lo que él os diga”. Pero Jesús no dijo nada. Se dio cuenta entonces que ya había dicho ---o más bien había pensado con un ferviente deseo--- en exceso.
\vs p137 4:10 María daba saltos de alegría. No sabía cómo se produciría el vino, pero estaba convencida de que por fin había podido persuadir a su hijo primogénito para que hiciera valer su autoridad, para que atreviera a dar un paso al frente y afirmar su estatus y mostrar su poder mesiánico. Y, debido a la presencia y a la conjunción de determinados poderes y de seres personales del universo, que todos los presentes ignoraban por completo, ella no iba a sufrir una decepción. El vino que María esperaba y que Jesús, el Dios\hyp{}hombre, humana y compasivamente deseaba, estaba próximo a hacerse realidad.
\vs p137 4:11 Había allí cerca seis tinajas de piedra, repletas de agua, con capacidad cada una de ellas para más de setenta y cinco litros. El agua estaba destinada a los ritos finales de purificación de la celebración nupcial. El alboroto de los criados en torno a estas enormes vasijas de piedra, bajo la ajetreada instrucción de su madre, atrajo la atención de Jesús, que al acercarse, observó que estaban sacando de allí jarras llenas de vino.
\vs p137 4:12 Paulatinamente, Jesús se fue percatando de lo que había sucedido. De todas las personas presentes en la fiesta de bodas de Caná, él era el más sorprendido. Otros habían esperado que obrara algún prodigio, pero aquello era justo lo que él no estaba dispuesto a hacer. Y, entonces, el Hijo del Hombre recordó la advertencia que le había hecho en las colinas su modelador del pensamiento personificado. Se acordó de que el modelador le había alertado sobre la imposibilidad de que ninguna fuerza o ser personal le privaría de su prerrogativa creadora de ser independiente del tiempo. En esta ocasión, los transformadores de la fuerza, los seres intermedios y todos los otros seres personales requeridos estaban congregados cerca del agua y de los demás componentes necesarios y, ante el deseo expreso del soberano creador del universo, la aparición instantánea del \bibemph{vino} era ineludible. Y la segunda razón de este hecho se confirmaba porque el modelador personificado había indicado que la realización del deseo del Hijo no conllevaba el incumplimiento de la voluntad del Padre.
\vs p137 4:13 Pero, de ningún modo, fue esto un milagro. Ninguna ley de la naturaleza se modificó, derogó o incluso se trascendió. No sucedió nada más que la supresión del \bibemph{tiempo} en conjunción con el acopio celestial de los elementos químicos exigidos para la elaboración del vino. En Caná, en esta ocasión, los agentes del creador hicieron vino tal como ellos lo hacen mediante procesos naturales ordinarios, \bibemph{excepto} que lo llevaron a cabo con independencia del tiempo y con la intervención de entidades de carácter sobrehumano, que acumularon en el espacio los indispensables componentes químicos.
\vs p137 4:14 Además, resultaba patente que la realización de este supuesto milagro no era contrario a la voluntad del Padre del Paraíso, ya que, de otro modo, no habría sucedido; Jesús se había sometido en todo a la voluntad del Padre.
\vs p137 4:15 \pc Cuando los criados sacaron este nuevo vino y se lo presentaron al padrino de boda, al “encargado del banquete”, este, tras probarlo, llamó al novio y le dijo: “Es costumbre servir primero el buen vino, y cuando los huéspedes han bebido mucho, poner el fruto inferior de la vid; sin embargo, tú has reservado el mejor vino para el final de la fiesta”.
\vs p137 4:16 María y los discípulos de Jesús se regocijaron enormemente ante aquel supuesto milagro, que pensaban que Jesús había realizado de forma intencionada, pero Jesús se retiró a un rincón resguardado del jardín y, durante breves momentos estuvo inmerso en profundos pensamientos. Finalmente, decidió que aquel incidente era ajeno a su voluntad en vista de las circunstancias y, que no siendo contrario a la voluntad de su Padre, había sido inevitable. Cuando regresó a donde estaba la gente, todos lo miraron con asombro; pensaban que él era el Mesías. Pero Jesús estaba penosamente perplejo; se daba cuenta de que solo creían en él por aquel extraño suceso que, inadvertidamente por su parte, habían contemplado. Jesús se retiró nuevamente a la azotea de la casa durante un rato para reflexionar sobre todo aquello.
\vs p137 4:17 Jesús tuvo en ese momento plena conciencia de que debía estar constantemente en guardia para que sus sentimientos de comprensión y piedad no diesen lugar a incidentes de esta naturaleza. Sin embargo, se produjeron muchos sucesos similares antes de que el Hijo del Hombre se despidiera finalmente de su vida mortal en la carne.
\usection{5. DE REGRESO A CAFARNAÚM}
\vs p137 5:1 Aunque muchos de los invitados se quedaron durante toda la semana de duración de los festejos nupciales, Jesús, con los recién elegidos discípulos\hyp{}apóstoles ---Santiago, Juan, Andrés, Pedro, Felipe y Natanael---, partió muy temprano, a la mañana siguiente, para Cafarnaúm, yéndose sin despedirse de nadie. La familia de Jesús y todos sus amigos de Caná quedaron muy consternados por lo repentino de su marcha, y Judá, el hermano menor de Jesús, fue en su busca. Jesús y sus apóstoles se dirigieron directamente a la casa de Zebedeo en Betsaida. En este viaje, Jesús habló de muchas cosas de importancia para el reino venidero con sus seis acompañantes, y les advirtió expresamente que no comentasen la transformación del agua en vino. Les aconsejó igualmente que evitaran las ciudades de Séforis y Tiberias en su labor futura.
\vs p137 5:2 Esa noche, tras la cena, en la casa de Zebedeo y Salomé, tuvo lugar una de las charlas más importantes de toda la andadura terrenal de Jesús. Solo los seis apóstoles estaban presentes en esta reunión; Judá llegó cuando estaban a punto de separarse. Estos seis hombres elegidos habían venido desde Caná a Betsaida con Jesús, con la sensación de caminar, en cierto modo, por las nubes. Se sentían llenos de vida por las expectativas que tenían ante sí y entusiasmados por haber sido escogidos como estrechos colaboradores del Hijo del Hombre. Pero cuando Jesús empezó a dejarles claro quién era él y cuál iba a ser su misión en la tierra y cómo acabaría posiblemente todo, se quedaron atónitos. No podían comprender lo que les estaba diciendo. Enmudecieron; incluso Pedro se sintió abrumado hasta lo indecible. Solo Andrés, como gran pensador, se atrevió a responder a las palabras de consejo de Jesús. Cuando Jesús percibió que no comprendían su mensaje, cuando vio que sus ideas sobre el Mesías judío estaban tan cristalizadas, los mandó a descansar mientras él caminaba y hablaba con su hermano Judá. Y antes de que Judá se despidiese de Jesús, él le dijo muy emocionado: “Mi padre\hyp{}hermano, nunca te he entendido. No sé con certeza si eres lo que madre nos ha enseñado, y no comprendo del todo el reino venidero, pero ciertamente sé que eres un poderoso hombre de Dios. Oí la voz en el Jordán, y creo en ti, sea quien seas. Y cuando terminó de hablar, partió hacia su propia casa en Magdala.
\vs p137 5:3 Esa noche Jesús no durmió. Se abrigó con una capa y se sentó en la orilla del lago a pensar, y así estuvo hasta el amanecer. Tras esas largas horas nocturnas en meditación, comprendió con claridad que jamás lograría que sus discípulos lo vieran de cualquier otra forma que no fuese la del Mesías por tanto tiempo esperado. Por último, reconoció que no había otra manera de exponer su mensaje del reino a no ser que cumpliera las predicciones de Juan y como aquel a quien los judíos esperaban. En definitiva, aunque él no era el Mesías de tipo davídico, en él se cumplían verdaderamente las profecías de los videntes antiguos de mayor propensión espiritual. No volvió nunca a negar del todo que no fuese el Mesías. Y decidió dejar que fuese la manifestación de la voluntad del Padre la que dilucidara finalmente esta compleja situación.
\vs p137 5:4 A la mañana siguiente, Jesús se reunió con sus amigos en el desayuno, pero estaban tristes. Conversó con ellos y, al terminar la comida, los reunió en torno a él y les dijo: “Es la voluntad de mi Padre que esperemos aquí algún tiempo. Habéis oído decir a Juan que había venido a fin de preparar el camino para el reino; así pues, nos corresponde a nosotros aguardar a que Juan termine su predicación. Cuando el precursor del Hijo del Hombre haya terminado su labor, comenzaremos nosotros a proclamar la buena nueva del reino”. Pidió a sus apóstoles que volvieran a sus redes de pesca mientras él se disponía para ir a la fábrica de barcos con Zebedeo. Les prometió que los vería al día siguiente en la sinagoga, donde él iba a hablar, y quedó para hablar con ellos ese \bibemph{sabbat} por la tarde.
\usection{6. LO ACONTECIDO EL DÍA DEL \bibemph{SABBAT}}
\vs p137 6:1 Tras su bautismo, Jesús hizo su primera aparición pública en la sinagoga de Cafarnaúm el día del \bibemph{sabbat} del 2 de marzo del año 26 d. C. La sinagoga estaba llena a rebosar. El relato del bautismo en el Jordán había llegado a magnificarse con las últimas noticias de Caná sobre la transformación del agua en vino. Jesús ofreció los asientos de honor a sus seis apóstoles; Santiago y Judá, sus hermanos en la carne, estaban también sentados con ellos. Su madre, que había regresado la tarde anterior a Cafarnaúm con Santiago, estaba igualmente presente, aunque sentada en la parte de las mujeres. Toda la audiencia estaba inquieta; esperaban ser testigos de alguna manifestación extraordinaria de poder sobrenatural, lo que constituiría un excelente testimonio de la naturaleza y la autoridad de quien les iba a hablar aquel día. Pero iban a sufrir una gran decepción.
\vs p137 6:2 Cuando se levantó Jesús, el jefe de la sinagoga le entregó el rollo de la Escritura, y él leyó un pasaje del profeta Isaías: “Así dice el Señor: ‘El cielo es mi trono y la tierra el estrado de mis pies. ¿Dónde está la casa que me habríais de edificar? ¿Dónde el lugar de mi reposo? Mi mano hizo todas estas cosas, así todas ellas llegaron a ser’, dice el Señor. ‘Pero yo miraré a aquel que es pobre y humilde de espíritu y que tiembla a mi palabra.’ Oíd la palabra del Señor, vosotros los que tembláis y teméis: ‘Vuestros hermanos os aborrecen y os echan fuera por causa de mi nombre,’ Pero sea el Señor glorificado. Él se os aparecerá en alegría, y todos los otros serán avergonzados. Una voz de alboroto de la ciudad, una voz del Templo, una voz del Señor dice: ‘¡Antes de que ella estuviera de parto, dio a luz; antes que le vinieran dolores, dio a luz un hijo! ¿Quién oyó cosa semejante? ¿Concebirá la tierra en un día? ¿O nacerá una nación de una sola vez? Porque así dice el Señor: ‘He aquí que yo extiendo sobre ella la paz como un río y la gloria incluso la de los gentiles será como un torrente que se desborda. Como aquel a quien consuela su madre, así os consolaré yo a vosotros, e incluso en Jerusalén recibiréis consuelo. Y cuando lo veáis, se alegrará vuestro corazón.’”
\vs p137 6:3 Cuando terminó de leer el pasaje, Jesús devolvió el rollo a su depositario. Antes de sentarse, dijo sencillamente: “Sed pacientes y veréis la gloria de Dios; de ese modo será con aquellos que esperan conmigo y aprenden así a hacer la voluntad de mi Padre que está los cielos”. Y la gente se fue a sus casas, preguntándose qué significaba todo aquello.
\vs p137 6:4 \pc Esa tarde Jesús y sus apóstoles, con Santiago y Judá, se subieron a una barca y se retiraron un poco de la orilla; allí echaron el ancla mientras él les hablaba del reino venidero. Y alcanzaron a comprender mejor sus palabras de lo que lo habían hecho la noche del jueves.
\vs p137 6:5 Jesús les pidió que se encargaran de sus tareas ordinarias hasta que “llegara la hora del reino”. Y para alentarlos, él mismo les sirvió como ejemplo al acudir él, con regularidad, a su trabajo en la fábrica de barcos. Al explicarles que cada día por la noche debían estudiar tres horas como preparación para su futura labor, Jesús les dijo: “Nos quedaremos por aquí hasta que el Padre me mande llamaros. Cada cual ha de volver a su actividad habitual como si nada hubiera ocurrido. No le habléis a ningún hombre de mí y recordad que mi reino no ha de venir con ruidos y ostentaciones, sino mediante el gran cambio que mi Padre forjará en vuestros corazones, y en el corazón de aquellos que serán llamados a unirse a vosotros en los consejos del reino. Ahora sois mis amigos; confío en vosotros y os amo; pronto os convertiréis en colaboradores personales míos. Sed pacientes, sed amables. Sed siempre obedientes a la voluntad del Padre. Estad siempre listos para cuando yo os llame a laborar por el reino. Aunque experimentaréis una gran alegría en vuestro servicio a mi Padre, debéis también estar preparados para las dificultades, porque os advierto que será solo por medio de muchas tribulaciones que muchos entrarán en el reino. Pero, aquellos que hayan encontrado el reino se llenarán de gozo y serán llamados los benditos de toda la tierra. Pero no alberguéis falsas esperanzas; el mundo flaqueará con mis palabras. Incluso vosotros, amigos míos, sois incapaces de percibir por completo lo que revelo a vuestras confusas mentes. No os engañéis; salimos a trabajar para una generación que busca signos. Exigirán la realización de prodigios para probar que mi Padre me ha enviado, y serán lentos en reconocer que la revelación del \bibemph{amor} de mi Padre que manifiesto en mi vida es el testimonio de mi misión.
\vs p137 6:6 Ese día, a última hora de la tarde, cuando volvieron a tierra, antes de que cada cual siguiera su camino, Jesús, de pie al borde del agua, oró: “Padre mío, te doy gracias por estos pequeños que han creído a pesar de sus dudas. Y, por su bien, escogí hacer tu voluntad. Que aprendan ahora a ser uno, tal como tú y yo somos uno”.
\usection{7. CUATRO MESES DE FORMACIÓN}
\vs p137 7:1 Durante cuatro largos meses ---marzo, abril, mayo y junio--- prosiguió este tiempo de espera; Jesús impartió a sus seis acompañantes y a Santiago, su propio hermano, más de cien sesiones formativas, largas e intensas, a la vez que animadas y alegres. Por enfermedad en su familia, Judá raras veces pudo asistir a estas clases. Santiago, el hermano de Jesús, no perdió la fe en él, pero María, durante estos meses de demora e inacción, casi llegó a desesperar por su hijo. Su fe, que había alcanzado tan altas cotas en Caná, se hundía hasta mínimos niveles. Solo recurría a su exclamación, tan repetida: “No consigo comprenderlo. No logro entender lo que significa todo esto”. Pero la mujer de Santiago contribuyó grandemente a fortalecer el ánimo de María.
\vs p137 7:2 Durante estos cuatro meses, los siete creyentes, uno de ellos su propio hermano en la carne, se iba familiarizando con Jesús; se estaban acostumbrando a la idea de vivir con este Dios\hyp{}hombre. Aunque lo llamaban “rabí”, estaban aprendiendo a no temerle. En su persona, Jesús poseía esa incomparable gracia que le permitía vivir entre ellos sin sentirse consternados por su divinidad. Les resultaba realmente fácil ser “amigos de Dios”, del Dios encarnado con la semejanza de un hombre mortal. Este tiempo de espera significó una dura prueba para todo el grupo de creyentes. No sucedió absolutamente nada de carácter milagroso; día tras día, ellos siguieron haciendo su trabajo ordinario, mientras que cada noche se sentaban a los pies de Jesús. Y se mantenían unidos gracias a su inigualable persona y a las afables palabras con las que les hablaba noche tras noche.
\vs p137 7:3 Este período de espera y enseñanza fue particularmente difícil para Simón Pedro. Este apóstol trató repetidamente de persuadir a Jesús para que emprendiera su predicación del reino en Galilea mientras Juan continuaba predicando en Judea. Pero Jesús siempre le decía: “Ten paciencia, Simón. Avanza. Ninguno de nosotros estará lo suficientemente preparado para cuando el Padre nos llame”. Y Andrés tranquilizaba a Pedro en ocasiones con sus consejos, más maduros y lógicos. Andrés estaba profundamente impresionado con la naturalidad humana de Jesús. Nunca se cansaba de contemplar cómo alguien que pudiese vivir tan cerca de Dios podría ser tan amigable y considerado con los hombres.
\vs p137 7:4 Durante todo este período, Jesús solo habló dos veces en la sinagoga. Tras muchas semanas de espera, los comentarios sobre su bautismo y el vino de Caná habían empezado a acallarse. Y Jesús se aseguró de que no ocurriera ningún otro supuesto milagro en este espacio de tiempo. Pero aunque vivían de forma tan discreta en Betsaida, hasta Herodes Antipas habían llegado noticias de estos extraños actos de Jesús, y este envió espías para averiguar lo que estaba ocurriendo. Pero Herodes estaba más preocupado por la predicación de Juan y decidió no molestar a Jesús, cuya labor continuaba apaciblemente en Cafarnaúm.
\vs p137 7:5 Durante este tiempo de espera, Jesús procuró enseñar a sus acompañantes la actitud a adoptar hacia los distintos grupos religiosos y partidos políticos de Palestina. Las palabras de Jesús siempre eran: “Tratamos de ganarlos a todos, pero no somos \bibemph{de} ninguno de ellos”.
\vs p137 7:6 \pc A los escribas y rabinos, considerados conjuntamente, se les llamaba fariseos. Se referían a sí mismos como “miembros”. En muchos aspectos, de entre todos los judíos, era el grupo más progresista al haber adoptado muchas enseñanzas, no claramente específicas de las escrituras hebreas, como la creencia en la resurrección de los muertos, una doctrina únicamente mencionada por Daniel, un profeta más reciente.
\vs p137 7:7 \pc Los saduceos estaban integrado por los sacerdotes y ciertos judíos ricos. No eran tan puristas en cuanto a los detalles del cumplimiento de la ley. Los fariseos y los saduceos, más que denominación religiosa, eran partidos religiosos.
\vs p137 7:8 \pc Los esenios constituían un verdadero movimiento religioso, originado durante la revuelta de los macabeos. Sus normas eran, en algunos aspectos, más rigurosas que las de los fariseos. Habían adoptado muchas creencias y prácticas persas, vivían como una hermandad en monasterios, se abstenían de casarse y todo lo poseían en común. Eran especialistas en la enseñanza sobre los ángeles.
\vs p137 7:9 \pc Los zelotes eran un grupo de vehementes patriotas judíos. Propugnaban que cualquier método estaba justificado en su lucha para librarse de la esclavitud del yugo romano.
\vs p137 7:10 \pc Los herodianos eran un partido exclusivamente político que defendía la emancipación del dominio directo de Roma mediante la restauración de la dinastía herodiana.
\vs p137 7:11 \pc En el mismo centro de Palestina vivían los samaritanos, con quienes “los judíos no tenían trato”, aun cuando tenían muchas opiniones similares respecto a las enseñanzas judías.
\vs p137 7:12 \pc Todos estos partidos y movimientos religiosos, incluyendo la pequeña hermandad nazarea, creían en una próxima llegada del Mesías. Todos aguardaban al libertador nacional. Pero Jesús dejó absolutamente claro que él y sus discípulos no se aliarían a ninguna de estas escuelas de pensamiento o práctica. El Hijo del Hombre no sería nazareo ni esenio.
\vs p137 7:13 Aunque Jesús más tarde dio instrucciones a sus apóstoles para que salieran, tal como había hecho Juan, a predicar el evangelio y enseñar a los creyentes, hizo hincapié en la proclamación de la “buena nueva del reino de los cielos”. Constantemente, recalcaba a sus acompañantes que debían “mostrar amor, compasión y comprensión”. Enseñó pronto a sus seguidores que el reino de los cielos era una experiencia espiritual que entrañaba la entronización de Dios en el corazón de los hombres.
\vs p137 7:14 Mientras estaban a la espera de emprender activamente su predicación pública, Jesús y los siete pasaban dos noches a la semana en la sinagoga, estudiando las escrituras hebreas. En años posteriores, tras periodos de intenso trabajo público, los apóstoles recordarían estos cuatro meses como los más valiosos y provechosos de los que habían disfrutado en compañía del Maestro. Jesús impartió a estos hombres toda la enseñanza que podían asimilar. No cometió la equivocación de enseñarles de más. No provocó su confusión revelándoles una verdad que sobrepasara su capacidad de comprensión.
\usection{8. EL SERMÓN SOBRE EL REINO}
\vs p137 8:1 El 22 de junio, día del \bibemph{sabbat,} poco antes de salir en su primer viaje de predicación, y unos diez días después de la encarcelación de Juan, Jesús tomó lugar en el púlpito de la sinagoga por segunda vez desde que había traído a sus apóstoles a Cafarnaúm.
\vs p137 8:2 Unos días antes de predicar este sermón sobre “El reino”, mientras Jesús trabajaba en la fábrica de barcos, Pedro le dio la noticia del arresto de Juan. Jesús, nuevamente, dejó sus herramientas, se quitó el delantal y le dijo a Pedro: “La hora del Padre ha llegado. Dispongámonos a proclamar el evangelio del reino”.
\vs p137 8:3 Jesús hizo su último trabajo en el banco de carpintería ese martes, 18 de junio del año 26 d. C. Pedro salió apresuradamente de la factoría y, hacia media tarde, había reunido a todos sus compañeros y, dejándolos en una arboleda junto a la costa, fue en busca de Jesús. Pero no pudo encontrarlo, porque el Maestro había ido a orar a otra arboleda. Y no lo vieron hasta entrada la noche cuando regresó a la casa de Zebedeo y pidió algo de comer. Al día siguiente, envió a su hermano Santiago para que se le permitiese el privilegio de hablar en la sinagoga el próximo \bibemph{sabbat}. Y el encargado de la sinagoga se sintió muy complacido de que Jesús deseara dirigir una vez más los oficios.
\vs p137 8:4 \pc Antes de dar este memorable sermón del reino de Dios, la primera y sublime actividad de su andadura pública, Jesús leyó los siguientes pasajes de las Escrituras: “Vosotros me seréis un reino de sacerdotes y gente santa. Yahvé es nuestro juez, Yahvé es nuestro legislador, Yahvé es nuestro rey; él mismo nos salvará. Yahvé es mi rey y mi Dios. Él es rey grande sobre toda la tierra. Su misericordia es sobre Israel en este reino. Bendita sea la gloria del Señor porque él es nuestro Rey”.
\vs p137 8:5 Cuando terminó de leer, dijo Jesús:
\vs p137 8:6 \pc “He venido para proclamar la instauración del reino del Padre. Y este reino consistirá en las almas devotas de judíos y gentiles, de ricos y pobres, de libres y esclavos, porque para mi Padre no hay acepción de personas; su amor y su misericordia son para todos.
\vs p137 8:7 “El Padre que está en los cielos envía a su espíritu para que habite en la mente de los hombres, y cuando yo haya terminado mi labor en la tierra, el espíritu de la verdad se derramará también sobre toda carne. El espíritu de mi Padre y el espíritu de la verdad os establecerán en el reino venidero del entendimiento espiritual y de la rectitud divina. Mi reino no es de este mundo. El Hijo del Hombre no liderará ejércitos en batalla para instaurar un trono poderoso o un reino de gloria terrenal. Cuando mi reino haya llegado, conoceréis al Hijo del Hombre como el Príncipe de la Paz, como la revelación del Padre eterno. Los hijos de este mundo pugnan por establecer y ampliar los reinos de este mundo, pero mis discípulos entrarán en el reino de los cielos por sus decisiones morales y por sus victorias espirituales; y cuando hayan entrado en él, hallarán gozo, rectitud y vida eterna.
\vs p137 8:8 “Aquellos que busquen primeramente entrar en el reino, y comienzan pues a procurar nobleza de carácter como la de mi Padre, no tardarán en poseer todas las demás cosas que les son necesarias. Pero os digo con toda sinceridad: a menos que tratéis de entrar en el reino con la fe y la crédula dependencia de un pequeño, de ningún modo seréis admitidos.
\vs p137 8:9 “Pero no os engañéis por quienes vienen diciendo que el reino está aquí o allí, porque el reino de mi Padre no guarda relación con las cosas visibles y materiales. Y este reino está ya entre vosotros, porque donde el espíritu de Dios enseña y guía al alma del hombre, allí está en verdad el reino de los cielos. Y este reino de Dios es rectitud, paz y gozo en el espíritu santo.
\vs p137 8:10 “Juan en efecto os bautizó en señal de arrepentimiento y para la remisión de vuestros pecados, pero cuando entréis en el reino celestial, seréis bautizados con el espíritu santo.
\vs p137 8:11 “En el reino de mi Padre no habrá judíos ni gentiles, sino solo aquellos que procuran la perfección mediante el servicio, porque declaro que el que quiera ser grande en el reino de mi Padre debe ser, primero, servidor de todos. Si deseáis servir a vuestros semejantes, os sentaréis conmigo en mi reino, al igual que yo, al servir en semejanza de las criaturas, me sentaré sin mucho tardar con mi Padre en su reino.
\vs p137 8:12 “Este nuevo reino es semejante a una semilla que se siembra en buena tierra. No da todos sus frutos enseguida. Hay un lapso de tiempo entre la instauración del reino en el alma del hombre y esa hora en la que el reino madura y fructifica por completo en perpetua rectitud y salvación eterna.
\vs p137 8:13 “Y este reino que yo os proclamo no es un reino de poder y abundancia. El reino de los cielos no es comida ni bebida, sino una vida de rectitud y gozo crecientes en el servicio, progresivamente perfecto, a mi Padre que está en los cielos. Porque, ¿acaso no ha dicho el Padre de sus hijos del mundo, ‘es mi voluntad que lleguéis a ser perfectos, como yo soy perfecto?’
\vs p137 8:14 “He venido a predicar la buena nueva del reino. No he venido a aumentar las pesadas cargas de quienes desean entrar en él. Anuncio el camino nuevo y mejor, y aquellos que puedan entrar en el reino venidero disfrutarán del descanso divino. Y sea lo que sea lo que os cueste en las cosas del mundo, sin importar el precio que paguéis para entrar en el reino de los cielos, recibiréis muchas veces más en gozo y en avance espiritual en este mundo, y en la vida eterna en la era por venir.
\vs p137 8:15 “La entrada en el reino del Padre no está a la espera de que los ejércitos marchen ni de que los reinos de este mundo caigan derrocados ni que el yugo de los cautivos se quiebre. El reino de los cielos ha llegado, y todo aquel que entre en él hallará libertad abundante y regocijo en la salvación.
\vs p137 8:16 “Este reino es un lugar de soberanía eterna. Quienes entren en él ascenderán hasta mi Padre; alcanzarán de cierto la diestra de su gloria en el Paraíso. Y todos aquellos que entren en el reino de los cielos serán hechos los hijos de Dios, y en la era por venir ascenderán hasta el Padre. Y yo no he venido a llamar a los presuntamente justos sino a los pecadores y a los que tienen hambre y sed de rectitud, de perfección divina.
\vs p137 8:17 “Juan vino a predicar el arrepentimiento como preparación para el reino; yo vengo ahora para proclamar la fe, el don de Dios, como el precio a pagar por entrar en el reino de los cielos. Con tan solo creer que mi Padre os ama con un amor infinito, estaréis ya en el reino de Dios”.
\vs p137 8:18 \pc Dicho esto, se sentó. Todos los que le oyeron quedaron asombrados ante sus palabras. Sus discípulos estaban maravillados. Pero la gente no estaba lista para recibir la buena nueva de labios de este Dios\hyp{}hombre. Alrededor de un tercio de los asistentes creyó en el mensaje a pesar de no entenderlo bien; sobre un tercio de ellos era propenso en su corazón a rechazar nociones tan puramente espirituales del reino esperado; mientras que el otro tercio restante no fue capaz de comprender sus enseñanzas, muchos realmente llegaron a creer que “estaba fuera de sí”.
