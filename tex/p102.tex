\upaper{102}{Fundamentos de la fe religiosa}
\author{Melquisedec}
\vs p102 0:1 Para el incrédulo materialista, el hombre es simplemente un accidente evolutivo. Sus esperanzas de supervivencia penden del hilo de la fantasía de la imaginación humana; sus temores, amores, deseos y creencias no son sino una reacción a la fortuita yuxtaposición de determinados átomos de materia inerte. Ninguna manifestación energética ni expresión de confianza pueden llevarle más allá de la tumba. La labor piadosa y la genialidad inspiradora de los mejores hombres están condenadas a desaparecer por la muerte, esa larga y solitaria noche del olvido eterno y de la extinción del alma. Una innombrable desesperación es la única recompensa del hombre por vivir y trabajar con dureza bajo el sol transitorio de la existencia mortal. Cada día de vida, aprietan de forma lenta y firmes las garras de un despiadado destino, que un universo material hostil e implacable ha decretado como una ofensa tremenda a todo lo en el deseo humano es hermoso, noble, elevado y bueno.
\vs p102 0:2 Pero este no es el fin ni el destino eterno del hombre; esta visión no es sino el grito de desesperación proferido por un alma errante que se ha visto perdida en la oscuridad espiritual, y que lucha con valentía ante los sofismas mecanicistas de una filosofía material, cegada por la confusión y la deformación de algún intrincado aprendizaje. Y toda esta oscura fatalidad y todo este destino de desesperanza se disipan para siempre con un valiente despliegue de fe por parte del más humilde e inerudito de los hijos de Dios en la tierra.
\vs p102 0:3 Esta fe salvadora tiene su nacimiento en el corazón humano cuando la conciencia moral del hombre se percata de que, en su experiencia como mortal, los valores humanos se pueden transformar de lo material a lo espiritual, de lo humano a lo divino, del tiempo a la eternidad\ldots
\usection{1. LAS SEGURIDADES DE LA FE}
\vs p102 1:1 La labor del modelador del pensamiento da razón a la transformación del sentido del deber primitivo y evolutivo del hombre en una fe superior y más segura en las realidades eternas de la revelación. Debe haber en el corazón del hombre sed de perfección para garantizarle la capacidad de comprender los caminos de la fe, que lo llevan hasta la suprema consecución. Si un hombre elige hacer la voluntad divina, conocerá el camino de la verdad. Es cierto, literalmente, que “hay que conocer las cosas humanas para poder amarlas, pero hay que amar las cosas divinas para poder conocerlas”. Pero la duda honesta y el sincero cuestionamiento no constituyen un pecado; estas actitudes simplemente representan una demora en el viaje progresivo hacia el logro de la perfección. Tener la confianza de un niño garantiza al hombre su entrada en el reino de la ascensión celestial, pero el avance depende totalmente del ejercicio vigoroso de la fe sólida y segura del hombre adulto.
\vs p102 1:2 La ciencia basa su razonamiento en los hechos observables del tiempo; la fe de la religión aboga por el plan espiritual de la eternidad. Lo que el conocimiento y la razón no pueden hacer por nosotros, la verdadera sabiduría nos exhorta a que permitamos que la fe lo logre por medio de la percepción religiosa y la transformación espiritual.
\vs p102 1:3 Como consecuencia del aislamiento sufrido por la rebelión, en Urantia, con demasiada frecuencia, la revelación de la verdad se ha visto mezclada con afirmaciones de cosmologías parciales y transitorias. La verdad permanece inalterable de generación en generación, pero sus correspondientes enseñanzas sobre el mundo físico varían de un día para otro y de un año al siguiente. No se debería menospreciar la verdad eterna porque se halle fortuitamente junto a ideas obsoletas sobre el mundo material. Mientras más ciencia conozcáis, menos seguros podréis estar; mientras más religión \bibemph{poseáis,} más ciertos estaréis.
\vs p102 1:4 Las seguridades de la ciencia proceden enteramente del intelecto; las seguridades de la religión brotan de los fundamentos mismos de \bibemph{todo el ser personal}. La ciencia apela al entendimiento de la mente; la lealtad y la devoción de la religión apelan al cuerpo, la mente y el espíritu, e incluso a la persona completa.
\vs p102 1:5 \pc Dios es tan real y absoluto que no se pueden ofrecer, en testimonio de su realidad, signos materiales probatorios ni demostraciones de los llamados milagros. Siempre lo conoceremos porque confiamos en él, y nuestra creencia en él se basa por completo en nuestra participación personal en las manifestaciones divinas de su realidad infinita.
\vs p102 1:6 \pc El modelador del pensamiento interior suscita en el alma del hombre indefectiblemente una auténtica sed de perfección a la vez que una inmensa curiosidad, que solo puede satisfacerse idóneamente en nuestra comunión con Dios, con la divina fuente de ese mismo modelador. El alma sedienta del hombre se niega a satisfacerse con algo que sea menos que la toma de conciencia personal del Dios vivo. Aunque Dios pueda ser mucho más que una persona moral excelsa y perfecta, no puede ser, en nuestro deseoso y finito concepto, nada menos.
\usection{2. RELIGIÓN Y REALIDAD}
\vs p102 2:1 Una mente que observa y un alma perceptiva saben reconocer la religión cuando la descubren en las vidas de sus semejantes. La religión no precisa de definición; todos conocemos sus frutos sociales, intelectuales, morales y espirituales. Y todo esto brota del hecho de que la religión es propiedad de la raza humana; no es producto de la cultura. Es cierto que la percepción de la religión sigue siendo aún humana y, de ahí, su sometimiento a la servidumbre de la ignorancia, a la esclavitud de la superstición, a los engaños de la sabiduría mundana y a los delirios de la falsa filosofía.
\vs p102 2:2 Una de las peculiaridades propias de la seguridad religiosa genuina es que, a pesar de la absolutidad de sus afirmaciones y de la firmeza de su posicionamiento, el espíritu de su expresión es tan sereno y moderado, que jamás refleja la más mínima impresión de superioridad en sus planteamientos ni de autoenaltecimiento. La sabiduría de la experiencia religiosa es algo así como una paradoja, porque es a la vez de origen humano y fruto del modelador. La fuerza religiosa no es fruto de las prerrogativas personales del ser humano, sino más bien consecuencia de la sublime alianza del hombre con la perdurable fuente de toda sabiduría. De ese modo, las palabras y los actos de la religión verdadera e impoluta se convierten en convincentemente fidedignas para todos los mortales iluminados.
\vs p102 2:3 Es difícil identificar y analizar los componentes de las experiencias religiosas, pero no lo es observar cómo los practicantes religiosos que las vivencian vivan y se comporten como si estuvieran ya ante la presencia del Eterno. Los devotos religiosos responden a esta vida temporal como si la inmortalidad estuviera ya a su alcance. En la vida de tales mortales, hay una creatividad auténtica y una espontaneidad expresiva que los separan para siempre de aquellos semejantes suyos que han absorbido solamente la sabiduría del mundo. Las personas religiosas parecen vivir realmente liberadas de la prisa acuciante y de las dolorosas tensiones de las vicisitudes intrínsecas al flujo transitorio del tiempo; demuestran una estabilidad como personas y una tranquilidad de carácter que no tienen explicación por las leyes de la fisiología, la psicología y la sociología.
\vs p102 2:4 \pc El tiempo es un elemento invariable en la adquisición del conocimiento; la religión hace que sus dones se hagan disponibles de inmediato, pese a que haya un importante factor de crecimiento en la gracia, un claro progreso en todas las etapas de la experiencia religiosa. El conocimiento entraña una búsqueda eterna; siempre estáis aprendiendo, pero jamás podréis llegar al pleno conocimiento de la verdad absoluta. Por si solo, el conocimiento nunca puede generar certeza absoluta, sino solo una creciente probabilidad y estimación; pero el alma religiosa, lúcida espiritualmente, \bibemph{sabe,} y sabe \bibemph{ahora}. Y, sin embargo, esta seguridad, profunda y positiva, no lleva a la persona religiosa sensata a interesarse menos por los avatares de la marcha de la sabiduría humana, vinculada en su lado materialista al lento avance de los logros de la ciencia.
\vs p102 2:5 Ni siquiera los descubrimientos de la ciencia se hacen verdaderamente \bibemph{reales} en la conciencia de las experiencias humanas hasta que no se desenmarañan y correlacionan, hasta que sus hechos relevantes no se convierten efectivamente en \bibemph{contenidos} mediante su encauzamiento en las corrientes de los pensamientos de la mente. El hombre mortal percibe incluso su entorno físico desde el plano mental, desde la perspectiva de su registro psicológico. No es extraño, por consiguiente, que el hombre albergue una interpretación sumamente unificada del universo y luego trate de identificar esta unidad energética de su ciencia con la unidad espiritual de su experiencia religiosa. La mente es unidad; la conciencia humana vive en el plano mental y percibe las realidades universales a través de los ojos de la mente de la que está dotado. La perspectiva de la mente no aporta la unidad existencial de la fuente de la realidad, de la Primera Fuente y Centro, pero sí puede mostrar al hombre, tal como alguna vez hará, la síntesis experiencial de la energía, la mente y el espíritu en y como el Ser Supremo. Si bien, la mente jamás puede lograr tal unificación de la diversidad de la realidad, a menos que sea firmemente consciente de las cosas materiales, de los contenidos intelectuales y de los valores espirituales; únicamente existe unidad en la armonía de esta triunidad de la realidad operativa, y únicamente en la unidad hay satisfacción personal en lo que respecta a la toma de conciencia de la uniformidad y de la coherencia cósmicas.
\vs p102 2:6 La mejor manera de hallar la unidad en la experiencia humana es a través de la filosofía. Y aunque el conjunto del pensamiento filosófico siempre deba basarse en los hechos materiales, el alma y la fuerza de la verdadera dinámica filosófica es la percepción espiritual humana.
\vs p102 2:7 \pc De forma natural, el hombre evolutivo no es un entusiasta del trabajo duro. Mantener en su experiencia de vida el ritmo con las imperiosas exigencias y los apremiantes impulsos de la creciente experiencia religiosa conlleva una incesante actividad en cuanto al desarrollo espiritual, a la expansión intelectual, a la ampliación del conocimiento de los hechos y al servicio social. No hay religión genuina al margen de un ser personal sumamente activo. Así pues, las personas más indolentes a menudo buscan escapar de los rigores de la verdadera actuación religiosa mediante alguna especie ingenioso autoengaño, prestándose a retirarse al falso refugio de las doctrinas y de los dogmas religiosos estereotipados. Pero la auténtica religión está viva. La cristalización intelectual de los conceptos religiosos equivale a la muerte espiritual. No podéis concebir la religión sin ideas, pero en cuanto la religión se ha reducido a una única \bibemph{idea,} ya no es religión; se ha convertido meramente en algún tipo de filosofía humana.
\vs p102 2:8 Asimismo, hay otros tipos de almas inestables y escasamente disciplinadas que están dispuestas a utilizar ideas sentimentales de la religión como vía de escape de las irritantes exigencias cotidianas. Cuando algunos mortales vacilantes y tímidos tratan de huir de la presión incesante de la vida evolutiva, la religión, tal como ellos la conciben, parece ofrecerles el cobijo más próximo, la mejor forma de escapatoria. Pero la misión de la religión es preparar al hombre para enfrentarse a las vicisitudes de la vida con valentía, e incluso heroicidad. La religión es el don supremo del hombre evolutivo, lo único que le permite seguir adelante y “sostenerse como viendo al invisible”. El misticismo, sin embargo, es a menudo una forma de retirarse de la vida, a la que se acogen esos humanos que no disfrutan con la actividad más vigorosa de vivir su vida religiosa en el espacio abierto de la sociedad y de las actividades comerciales humanas. La verdadera religión debe \bibemph{actuar}. El comportamiento será el resultado de la religión cuando el hombre realmente la posea o, más bien, cuando el hombre permita verdaderamente que la religión lo posea a él. La religión no se contentará jamás con el mero pensamiento o el sentimiento inactivo.
\vs p102 2:9 No estamos ciegos ante el hecho de que la religión actúa a menudo con insensatez, e incluso irreligiosamente, pero \bibemph{actúa}. Las aberraciones de las convicciones religiosas han llevado a persecuciones sangrientas, pero por siempre la religión hace algo; ¡es dinámica!
\usection{3. CONOCIMIENTO, SABIDURÍA Y PERCEPCIÓN}
\vs p102 3:1 Las deficiencias intelectuales o las carencias educativas obstaculizan inevitablemente los más elevados logros religiosos, porque un entorno de naturaleza espiritual tan depauperado substrae de la religión su principal cauce de contacto filosófico con el conocimiento del mundo científico. Los factores intelectuales de la religión son importantes, pero su excesivo desarrollo es a veces, igualmente, muy discapacitante y desconcertante. La religión debe obrar continuamente bajo una necesidad paradójica: la necesidad de hacer un uso efectivo del pensamiento, descartando al mismo tiempo su propia utilidad.
\vs p102 3:2 Las especulaciones religiosas son inevitables, pero siempre son perjudiciales; la especulación desvirtúa de forma invariable su objeto. Tiende a transformar la religión en algo material o humanista y, así pues, mientras que interfiere directamente con la claridad del pensamiento lógico, hace, indirectamente, que la religión aparezca en función del mundo temporal, el mismo mundo con el que debería estar continuamente en contraste. Por lo tanto, la religión siempre se caracterizará por paradojas, aquellas que resultan de la ausencia de una conexión experiencial entre el nivel material y el nivel espiritual del universo ---esto es, de la mota morontial, la sensibilidad suprafilosófica para discernir la verdad y percibir la unidad---.
\vs p102 3:3 Los sentimientos de índole material, las emociones humanas, conducen directamente a las acciones materiales, a los actos egoístas. La percepción religiosa, las motivaciones espirituales, llevan directamente a las acciones religiosas, a los actos desinteresados de servicio social y de generosidad.
\vs p102 3:4 El deseo religioso es la búsqueda anhelante de la realidad divina. La experiencia religiosa es la toma de conciencia de haber encontrado a Dios. Y, cuando un ser humano encuentra realmente a Dios, experimenta en su alma tal indescriptible estremecimiento por el triunfo de haberlo descubierto, que se siente impulsado a servir por amor y a acercarse a semejantes suyos de menor percepción espiritual, no para revelar que ha encontrado a Dios, sino más bien para hacer que la eterna bondad que inunda y desborda su propia alma, los renueve y ennoblezca. La auténtica religión lleva a incrementar el servicio social.
\vs p102 3:5 \pc La ciencia, el conocimiento, lleva a la conciencia de los \bibemph{hechos;} la religión, la experiencia, a la conciencia de los \bibemph{valores;} la filosofía, la sabiduría, a la conciencia \bibemph{integrada;} la revelación (sustituta de la mota morontial), a la conciencia de la \bibemph{verdadera realidad;} mientras que la coordinación de la conciencia de los hechos, los valores y la verdadera realidad supone ser conscientes de la realidad personal, de lo máximo del ser, junto con la creencia en la posibilidad de la supervivencia del ser personal.
\vs p102 3:6 \pc El conocimiento lleva a determinar un sitio para el hombre, da origen a los estratos sociales y a las castas. La religión lleva a servir a los demás, creando con ello la ética y el altruismo. La sabiduría lleva a una comunión de ideas y a una fraternidad con los semejantes de un orden superior y de mayor excelencia. La revelación libera al hombre y lo hace emprender la aventura eterna.
\vs p102 3:7 La ciencia clasifica a los hombres; la religión, los ama, como os amáis a vosotros mismos; la sabiduría les hace justicia en su diferencia; pero la revelación glorifica al hombre y desvela su capacidad para coaligarse con Dios.
\vs p102 3:8 La ciencia trata en vano de crear la hermandad de la cultura; la religión da origen a la hermandad del espíritu. La filosofía se esfuerza por conseguir la hermandad de la sabiduría; la revelación muestra la eterna hermandad, como en el caso del colectivo de finalizadores del Paraíso.
\vs p102 3:9 El conocimiento genera orgullo en el hecho de que eres una persona; la sabiduría te da la conciencia de lo que significa ser una persona; la religión te ofrece el reconocimiento del valor de ser una persona; la revelación te proporciona la garantía de que sobrevivirás como persona.
\vs p102 3:10 \pc La ciencia pretende identificar, analizar y clasificar las zonas segmentadas del ilimitado cosmos. La religión capta la idea del todo, del cosmos en su totalidad. La filosofía trata de identificar las segmentaciones materiales de la ciencia mediante el concepto de la percepción espiritual de la totalidad. En donde la filosofía fracasa en su intento, la revelación triunfa, afirmando que el círculo cósmico es universal, eterno, absoluto e infinito. Este cosmos del Infinito YO SOY no tiene fin ni límites y es todo incluyente ---sin tiempo, sin espacio e incondicionado---. Y damos testimonio de que el Infinito YO SOY es también el Padre de Miguel de Nebadón y el Dios de la salvación humana.
\vs p102 3:11 La ciencia muestra la Deidad como un \bibemph{hecho;} la filosofía expone la \bibemph{idea} de un Absoluto; la religión concibe a Dios como un amoroso \bibemph{ser personal}. La revelación afirma la \bibemph{unidad} del hecho de la Deidad, de la idea del Absoluto y del ser personal espiritual de Dios y, además, presenta este concepto como nuestro Padre ---el hecho universal de la existencia, la idea eterna de la mente y el espíritu infinito de la vida---.
\vs p102 3:12 La búsqueda del conocimiento configura la ciencia; la consecución de la sabiduría es la filosofía; el amor de Dios es la religión; la sed de la verdad \bibemph{es} una revelación. Pero el modelador del pensamiento interior vincula la sensación de realidad a la percepción espiritual del hombre respecto al cosmos.
\vs p102 3:13 \pc En la ciencia, la idea precede a la expresión de su consecución; en la religión, la experiencia de la consecución precede a la expresión de la idea. Existe una inmensa diferencia entre la voluntad evolutiva de creer y el resultado de la razón iluminada, de la percepción religiosa y de la revelación: \bibemph{la voluntad que cree}.
\vs p102 3:14 En la evolución, la religión a menudo lleva al hombre a crear sus conceptos de Dios; la revelación muestra el fenómeno de Dios desarrollando al hombre mismo, mientras que en la vida terrena de Cristo Miguel contemplamos el fenómeno de Dios revelándose a sí mismo al hombre. La evolución tiende a asemejar a Dios con el hombre; la revelación tiende a asemejar al hombre con Dios.
\vs p102 3:15 La ciencia únicamente se satisface con causas primeras; la religión, con un ser personal supremo y, la filosofía, con la unidad. La revelación afirma que las tres son una sola cosa, y que todas son buenas. En lo \bibemph{real eterno} está el bien del universo y no en la ilusoria realidad temporal de un pernicioso espacio. En la experiencia espiritual de todos los seres personales, es siempre verdad que lo real es lo bueno y lo bueno es lo real.
\usection{4. EL HECHO DE LA EXPERIENCIA}
\vs p102 4:1 Debido a la presencia del modelador del pensamiento en vuestras mentes, no es para vosotros más misterioso conocer la mente de Dios que estar seguros de la conciencia de conocer a otra mente cualquiera, humana o suprahumana. La religión y la conciencia social coinciden en esto: se basan en la conciencia de que hay otras mentes. El modo por el que podéis aceptar la idea del otro como vuestra es el mismo por el que podéis “dejar que la mente que estaba en Cristo esté también en vosotros”.
\vs p102 4:2 ¿Qué es la experiencia humana? Es simplemente cualquier interacción entre un yo activo e inquisitivo y cualquier otra realidad activa y externa. La magnitud de la experiencia viene determinada por la profundidad del concepto más la totalidad del reconocimiento de la realidad externa. El dinamismo de la experiencia equivale a la fuerza de la imaginación expectante más la agudeza del descubrimiento sensorial de las cualidades externas de la realidad con la que se contacta. El hecho de la experiencia se encuentra en la conciencia de uno mismo y de otras existencias ---otras cosas, otras mentes y otros espíritus---.
\vs p102 4:3 Muy pronto, el hombre se hace consciente de que no está solo en el mundo o en el universo. Se desarrolla en él una conciencia natural espontánea de sí mismo y de la existencia de otras mentes en el entorno del propio yo. La fe transforma esta experiencia natural en religión, en el reconocimiento de Dios como la realidad ---fuente, naturaleza y destino--- de la existencia de \bibemph{otras mentes}. Pero tal conocimiento de Dios es por siempre y para siempre una realidad de la experiencia personal. Si Dios no fuese un ser personal, no podría convertirse en parte viva de la auténtica experiencia religiosa de la persona humana.
\vs p102 4:4 El componente de error presente en la experiencia religiosa humana es directamente proporcional al contenido de materialismo que desvirtúa el concepto espiritual del Padre Universal. El avance del hombre en el universo como pre\hyp{}espíritu radica en el acto de liberarse de estas ideas erróneas de la naturaleza de Dios y de la realidad del espíritu puro y verdadero. La Deidad es más que espíritu, pero, para el hombre ascendente, el enfoque espiritual es el único posible.
\vs p102 4:5 \pc En realidad, la oración forma parte de la experiencia religiosa, pero las religiones modernas la han enfatizado indebidamente, descuidando en gran medida la comunión más esencial que se alcanza por medio de la adoración. La adoración profundiza y amplia el poder reflexivo de la mente. La oración podrá enriquecer la vida, pero la adoración ilumina el destino.
\vs p102 4:6 \pc La religión revelada es el elemento unificador de la existencia humana. La revelación unifica la historia, coordina la geología, la astronomía, la física, la química, la biología, la sociología y la psicología. La experiencia espiritual es la verdadera alma del cosmos del hombre.
\usection{5. SUPREMACÍA DEL POTENCIAL INTENCIONAL}
\vs p102 5:1 Aunque determinar el hecho de la creencia no equivale a determinar el hecho de aquello en lo que se cree, no obstante, el progreso evolutivo de la vida simple al estatus del ser personal demuestra ciertamente el hecho de la existencia del potencial del ser personal desde el principio. Y en los universos del tiempo, lo potencial es siempre supremo en relación a lo actual. En el cosmos evolutivo, lo potencial es lo que será, y lo que será es el despliegue de los mandatos intencionales de la Deidad.
\vs p102 5:2 Esta misma supremacía intencional se muestra en la evolución de la ideación de la mente cuando el miedo animal primitivo se transmuta en una permanente profundización de la reverencia hacia Dios y en un creciente asombro ante el universo. El hombre primitivo tenía más temor religioso que fe, y la supremacía de los potenciales espirituales sobre los actuales de la mente se hace evidente cuando este miedo asustadizo se transforma en una fe viva en las realidades espirituales.
\vs p102 5:3 Podéis explicar la religión evolutiva en términos psicológicos, pero no la religión de la experiencia personal, cuyo origen es espiritual. La moral humana puede reconocer valores, pero solo la religión puede conservar, enaltecer y espiritualizar estos valores. Si bien, a pesar de tal modo de actuar, la religión es algo más que una moral de tipo emocional. La religión es para la moral lo que el amor es para el deber, lo que la filiación es para la servidumbre, lo que la esencia es para la sustancia. La moral desvela un Rector todopoderoso, una Deidad a quien prestar servicio; la religión desvela un Padre amantísimo, un Dios a quien adorar y amar. Y, de nuevo, esto es así porque la potencialidad espiritual de la religión predomina sobre la realidad actual del deber de la moral evolutiva.
\usection{6. LA CERTEZA DE LA FE RELIGIOSA}
\vs p102 6:1 La eliminación filosófica del temor religioso y los constantes avances de la ciencia contribuyen de forma considerable al deceso de los dioses falsos; y, aunque esta pérdida de las deidades creadas por el hombre pueda momentáneamente oscurecer la visión espiritual, acaba por destruir esa ignorancia y superstición, que por tanto tiempo ocultaron al Dios vivo del amor eterno. La relación entre la criatura y el Creador es una experiencia viva, una fe religiosa dinámica, que no admite una definición concreta. Aislar parte de la vida y llamarla religión es desintegrar la vida y distorsionar la religión. Y es precisamente por esto por lo que el Dios de adoración exige total lealtad o ninguna.
\vs p102 6:2 Los dioses de los hombres primitivos pueden que no fueran más que sombras de ellos mismos; el Dios vivo es la luz divina cuyas interrupciones constituyen las sombras de la creación en todo el espacio.
\vs p102 6:3 \pc El devoto religioso preparado filosóficamente tiene fe en un Dios personal de salvación personal, en algo más que una realidad, un valor, un nivel de logro, un proceso excelso, una trasmutación, la ultimidad del tiempo\hyp{}espacio, una idealización, la manifestación personal de la energía, la entidad de la gravedad, una proyección humana, la idealización del yo, la erupción de la naturaleza, la propensión a la bondad, el impulso hacia adelante de la evolución o una sublime hipótesis. El creyente tiene fe en un Dios de amor. El amor es la esencia de la religión y la fuente de una civilización superior.
\vs p102 6:4 La fe transforma al Dios filosófico de la probabilidad en el Dios salvador de la certeza en la experiencia religiosa personal. El escepticismo podrá cuestionar las teorías de la teología, pero la confianza en la fiabilidad de la experiencia personal corrobora la verdad de esa creencia que ha llegado a convertirse en fe.
\vs p102 6:5 Se puede llegar a la convicción de Dios a través de razonamientos inteligentes, pero la persona se convierte en conocedora de Dios únicamente gracias a la fe, mediante la experiencia personal. En gran parte de lo relacionado con la vida, hay que contar con la probabilidad de su ocurrencia, pero, al contactar con la realidad cósmica, se puede experimentar certeza en el momento en el que tales contenidos y valores se abordan por medio de la fe viva. El alma que conoce a Dios se atreve a decir “yo sé”, incluso aunque se ponga en duda este conocimiento de Dios por el incrédulo, que niega esta certitud por no estar por completo sustentada en la lógica intelectual. Al escéptico se le ha de responder: “¿Cómo sabes que yo no sé?”.
\vs p102 6:6 \pc Aunque la razón siempre puede cuestionar la fe, la fe siempre puede ser un complemento de la razón y de la lógica. La razón plantea una probabilidad que la fe puede transforma en certeza moral, o incluso en experiencia espiritual. Dios es la primera verdad y el último hecho; así pues, toda verdad tiene en él su origen, mientras que todos los hechos existen con relación a él. Dios es la verdad absoluta. Se puede conocer a Dios como verdad, pero para comprender ---explicar--- a Dios, se debe explorar el hecho del universo de los universos. La inmensa brecha existente entre la experiencia de la verdad de Dios y el desconocimiento respecto al hecho de Dios solo puede salvarse por la fe viva. La razón por sí sola no puede lograr la armonía entre la verdad infinita y el hecho universal.
\vs p102 6:7 Las creencias quizás no puedan resistir la duda ni afrontar el temor, pero la fe siempre triunfa sobre la duda, porque la fe es positiva y está viva. Lo positivo siempre tiene ventaja sobre lo negativo, la verdad sobre el error, la experiencia sobre la teoría, las realidades espirituales sobre los hechos aislados del tiempo y del espacio. La prueba convincente de esta certeza espiritual consiste en los frutos sociales del espíritu que tales creyentes, personas de fe, generan como resultado de esta genuina experiencia espiritual. Jesús dijo: “Si amáis a vuestros semejantes como yo os he amado, todos los hombres sabrán que vosotros sois mis discípulos”.
\vs p102 6:8 \pc Para la ciencia, Dios es una posibilidad; para la psicología, una ventaja; para la filosofía, una probabilidad; para la religión, una certidumbre, una realidad de la experiencia religiosa. La razón requiere que una filosofía que no puede encontrar al Dios de la probabilidad respete esa fe religiosa que puede y que de hecho encuentra al Dios de la certeza. La ciencia tampoco debe ignorar, por razones de credulidad, la experiencia religiosa, al menos mientras persista en la suposición de que las dotes intelectuales y filosóficas del hombre surgieron de formas de inteligencia cada vez menores a medida que se retrocede más en el tiempo, teniendo en último término su origen en una vida primitiva totalmente desprovista de pensamiento y sentimiento.
\vs p102 6:9 Los hechos de la evolución no deben usarse para oponerse a la verdad de la realidad de la certeza de la experiencia espiritual de la vida religiosa del mortal conocedor de Dios. Los hombres inteligentes deberían dejar de razonar como niños e intentar utilizar la lógica congruente del adulto, la lógica que tolera el concepto de la verdad junto con la observación del hecho. El materialismo científico está en quiebra cuando, ante cada fenómeno recurrente del universo, persiste en consolidar sus objeciones vigentes, atribuyendo aquello admitido como superior a aquello igualmente admitido como inferior. La congruencia exige el reconocimiento de la labor de un Creador resolutivo.
\vs p102 6:10 La evolución orgánica es un hecho; la evolución deliberada o progresiva es una verdad que garantiza la congruencia de los fenómenos, por otra parte contradictorios, de los logros en constante ascenso de la evolución. Cuanto más avance el científico en la ciencia elegida, más renunciará a las teorías del hecho materialista en favor de la verdad cósmica, dominio de la Mente Suprema. El materialismo degrada la vida humana; el evangelio de Jesús eleva extraordinariamente de condición y enaltece sublimemente a todos los mortales. La existencia humana debe visualizarse como consistente en la misteriosa y fascinante experiencia del ser humano que alza sus manos al cielo para encontrarse con unas manos divinas y salvadoras que descienden de lo alto.
\usection{7. LA CERTIDUMBRE DE LO DIVINO}
\vs p102 7:1 El Padre Universal, teniendo existencia por sí mismo, se explica también en sí mismo; vive realmente en todo mortal racional. Pero no podéis estar seguros de Dios a no ser que lo conozcáis; la filiación es la única experiencia que hace cierta la paternidad. El universo está experimentando cambios por todos lados. Un universo cambiante es un universo dependiente; dicha creación no puede ser final ni absoluta. Un universo finito es completamente dependiente del Último y del Absoluto. Dios y el universo no son idénticos; uno es causa, el otro efecto. La causa es absoluta, infinita, eterna e invariable; el efecto es espacio\hyp{}temporal y trascendental, pero continuamente cambiante, en crecimiento constante.
\vs p102 7:2 Dios es el único hecho en el universo que tiene causa en sí mismo. Él es el secreto del orden, del plan y del propósito de la creación total de cosas y seres. El todo cambiante universo está regulado y estabilizado por leyes absolutamente invariables, por el patrón de conducta de un Dios invariable. El hecho de Dios, la ley divina, es invariable; la verdad de Dios, su relación con el universo, constituye una revelación relativa, que es por siempre adaptable a un universo en continua evolución.
\vs p102 7:3 \pc Aquellos que quieren inventar una religión sin Dios son como los que quieren recoger frutos sin árboles o tener hijos sin progenitores. No es posible tener efectos sin causas; solo el YO SOY es incausado. El hecho de la experiencia religiosa supone un Dios, y este Dios de la experiencia personal debe ser una Deidad personal. No podéis orar a una fórmula química, implorar a una ecuación matemática, adorar una hipótesis, poner la confianza en un postulado, estar en comunión con un proceso, servir a una abstracción o estar en amor fraternal con una ley.
\vs p102 7:4 Es verdad que muchos rasgos aparentemente religiosos pueden brotar de raíces no religiosas. Intelectualmente, el hombre puede negar a Dios y, sin embargo, ser moralmente bueno, leal, filial, honesto e incluso idealista. El hombre puede injertar muchas ramas puramente humanistas en su naturaleza espiritual básica y así comprobar presuntamente sus argumentos a favor de una religión sin Dios, pero dicho acto está desprovisto de valores de supervivencia, no conduce al conocimiento de Dios ni al ascenso a Dios. En tal experiencia humana solo aparecen frutos sociales, pero no espirituales. El injerto determina la naturaleza del fruto, pese a que el sostén vital se tome de raíces derivadas del don divino primigenio de la mente y del espíritu.
\vs p102 7:5 La marca intelectual distintiva de la religión es la certidumbre; su característica filosófica es la congruencia; sus frutos sociales son el amor y el servicio.
\vs p102 7:6 \pc La persona conocedora de Dios no es alguien ciego ante las dificultades o hace caso omiso de los obstáculos que le impidan encontrar a Dios en el laberinto de la superstición, la tradición y las tendencias materialistas de los tiempos modernos. Se ha enfrentado a todos estos elementos disuasorios y ha triunfado sobre estos, los ha superado mediante la fe viva y, a pesar de estos, ha alcanzado las altiplanicies de la experiencia espiritual. Pero es cierto que muchas de estas personas, seguras interiormente de Dios, temen expresar estos sentimientos de certidumbre debido a la doblez y a la astucia del gran número de aquellos que ponen objeciones y magnifican las dificultades sobre la creencia en Dios. No se necesita poseer una gran profundidad de intelecto para encontrar defectos, hacer preguntas o plantear objeciones. Pero sí se precisa una mente brillante para responder a esas preguntas y resolver dichas dificultades; la certidumbre de la fe es el mejor modo de abordar todos estos argumentos superficiales.
\vs p102 7:7 \pc Si la ciencia, la filosofía o la sociología se atreven a convertirse en dogmáticas en su pugna con los profetas de la verdadera religión, los hombres que conocen a Dios deben, entonces, responder a tal dogmatismo infundado con ese dogmatismo más juicioso de la certeza de la experiencia espiritual personal: “Sé lo que he experimentado porque soy hijo del YO SOY”. Si esta experiencia de la persona se ve cuestionada por el dogma, entonces este hijo del Padre que ha vivenciado, nacido de la fe, puede responder con ese dogma indisputable: declarando su genuina filiación con el Padre Universal.
\vs p102 7:8 Solo una realidad incondicionada, un absoluto, puede optar por ser consecuentemente dogmático. Aquellos que adoptan el dogmatismo, si son consecuentes, deben ser llevados más tarde o más temprano a los brazos del Absoluto de la energía, el Universal de la verdad y la Infinitud del amor.
\vs p102 7:9 Si los criterios no religiosos de la realidad cósmica pretenden cuestionar la certidumbre de la fe en razón de su condición no probada, entonces quien experimenta la verdad espiritual puede igualmente recurrir a cuestionar dogmáticamente los hechos de la ciencia y de las creencias filosóficas en razón de que no están probados; son, de la misma manera, experiencias de la conciencia del científico o del filósofo.
\vs p102 7:10 \pc De Dios, la más ineludible de todas las presencias, el más real de todos los hechos, la más vital de todas las verdades, el más amoroso de todos los amigos y el más divino de todos los valores, tenemos derecho a estar más ciertos que de cualquier otra experiencia en el universo.
\usection{8. EVIDENCIAS DE LA RELIGIÓN}
\vs p102 8:1 La mayor evidencia de la realidad y de la eficacia de la religión está en \bibemph{el hecho de la experiencia humana;} esto es, que el hombre, naturalmente temeroso y suspicaz, innatamente dotado de un fuerte instinto de autopreservación y del anhelo de sobrevivir a la muerte, está dispuesto plenamente a confiar los más profundos intereses de su presente y de su futuro a la protección y dirección de ese poder y de esa persona designados por su fe como Dios. Esa es la principal verdad de cualquier religión. Respecto a lo que ese poder o persona exige del hombre a cambio de su cuidado y salvación final, no hay dos religiones que estén de acuerdo; de hecho, todas más o menos discrepan.
\vs p102 8:2 Sobre la situación de cualquier religión en la escala evolutiva, la mejor manera de valorarla es por sus juicios morales y sus principios éticos. Cuanto más elevada sea la religión, más alienta y se siente alentada por una moral social y una cultura ética cada vez más excelentes. No podemos valorarla por el estatus de la civilización que la acompañe; sería mejor apreciar la verdadera naturaleza de una civilización por la pureza y la nobleza de su religión. Muchos de los más notables maestros religiosos del mundo eran prácticamente iletrados. La sabiduría del mundo no es necesaria para el desempeño de la fe salvadora en las realidades eternas.
\vs p102 8:3 La diferencia entre las religiones que han existido en las diversas eras depende por completo de la forma distinta en la que el hombre comprende la realidad y reconoce los valores morales, las relaciones éticas y las realidades espirituales.
\vs p102 8:4 \pc La ética es el espejo social o racial externo que refleja fielmente el progreso, por otra parte no observable, del desarrollo espiritual y religioso interno. El hombre siempre ha pensado en Dios en términos de lo mejor que sabía, de sus ideas más profundas y de sus ideales más elevados. Incluso la religión histórica ha creado siempre sus conceptos de Dios a partir de sus más altos valores reconocidos. Cualquier criatura inteligente da el nombre de Dios a lo más supremo y mejor que conoce.
\vs p102 8:5 La religión, cuando se ha ceñido a los términos de la razón y de la expresión intelectual, se ha atrevido siempre a criticar la civilización y el progreso evolutivo, juzgándolos en virtud de sus propios criterios respecto a la cultura ética y al progreso moral.
\vs p102 8:6 Aunque la religión personal precede a la evolución de la moral humana, es lamentable hacer constar que la religión institucional ha ido invariablemente a la zaga de las costumbres paulatinamente cambiantes de las razas humanas. La religión organizada ha resultado ser conservadoramente tardía. Por lo general, los profetas han llevado a los pueblos al desarrollo religioso; los teólogos normalmente los han retenido. La religión, al ser una cuestión de experiencia interior o personal, nunca puede desarrollarse con una gran anticipación respecto a la evolución intelectual de las razas.
\vs p102 8:7 Pero la religión nunca se realza apelando a lo milagroso. La búsqueda de los milagros se remonta a las primitivas religiones de la magia. La verdadera religión no tiene nada que ver con los supuestos milagros, y la religión revelada jamás acude a ellos para probar su legitimidad. La religión está por siempre y para siempre enraizada y fundamentada en la experiencia personal. Y vuestra religión más elevada, la vida de Jesús, consistió precisamente en tal experiencia personal: el hombre, el hombre mortal, buscando a Dios y encontrándolo plenamente durante su corta vida en la carne, mientras que, en la misma experiencia humana, apareció Dios buscando al hombre y encontrándolo para la plena satisfacción del alma perfecta de la supremacía infinita. Y esto es la religión, la más elevada que se haya revelado nunca en el universo de Nebadón: la vida terrenal de Jesús de Nazaret.
\vsetoff
\vs p102 8:8 [Exposición de un melquisedec de Nebadón.]
