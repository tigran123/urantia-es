\upaper{52}{Épocas planetarias de los mortales}
\author{Mensajero poderoso}
\vs p052 0:1 Desde el inicio de la vida en un planeta evolutivo hasta el momento en el que finalmente florece en la era de luz y vida, aparecen, dentro del contexto mundial, siete épocas relacionadas con el progreso de la vida humana. Estas sucesivas eras se definen con referencia a las misiones planetarias de los hijos divinos, las cuales, en un mundo habitado de tipo medio, ocurren en el siguiente orden:
\vs p052 0:2 \li{1.}El hombre antes de la llegada del príncipe planetario.
\vs p052 0:3 \li{2.}El hombre tras la llegada del príncipe planetario.
\vs p052 0:4 \li{3.}El hombre tras la llegada de Adán.
\vs p052 0:5 \li{4.}El hombre tras la llegada del hijo magistrado.
\vs p052 0:6 \li{5.}El hombre tras la llegada del hijo de gracia.
\vs p052 0:7 \li{6.}El hombre tras la llegada del hijo preceptor.
\vs p052 0:8 \li{7.}La era de luz y vida.
\vs p052 0:9 \pc En el momento en el que son físicamente adecuados para la vida, los mundos del espacio constan en el registro de los portadores de vida y, a su debido tiempo, se envía a estos, a dichos planetas, con el propósito de dar inicio a la vida. A todo el período desde el comienzo de la vida hasta la aparición del hombre se le denomina era prehumana y antecede a las épocas que consecutivamente acontecen en la vida de los mortales y a las que aludiremos en la siguiente narración.
\usection{1. EL HOMBRE PRIMITIVO}
\vs p052 1:1 Desde el momento en que el hombre emerge del nivel animal ---cuando puede decidir adorar al Creador--- y hasta la llegada del príncipe planetario, a las criaturas mortales volitivas se les llaman \bibemph{hombres primitivos}. Hay seis tipos fundamentales de razas de hombres primitivos, y estos primeros pueblos aparecen sucesivamente en el orden de los colores del espectro, comenzando con el rojo. El tiempo que lleva en tomar forma este desarrollo temprano de la vida varía considerablemente en los diferentes mundos, oscilando entre ciento cincuenta mil años y más de un millón de años del tiempo de Urantia.
\vs p052 1:2 Las razas evolutivas de color ---roja, naranja, amarilla, verde, azul e índigo--- comienzan a aparecer en la época en la que el hombre primitivo está desarrollando un lenguaje simple y está comenzando a ejercitar la imaginación creativa. Llegado ese momento, el hombre está bien acostumbrado a permanecer en posición erguida.
\vs p052 1:3 \pc Los hombres primitivos son grandes cazadores y fieros luchadores. La ley de esta era es la supervivencia física de los más aptos; el gobierno de estos tiempos es enteramente tribal. En muchos mundos, durante las primeras luchas raciales, algunas de las razas evolutivas desaparecen, tal como aconteció en Urantia. Generalmente, las que logran sobrevivir terminan por mezclarse con los pueblos adánicos, esto es, con la raza violeta más tarde importada.
\vs p052 1:4 A la luz de la civilización posterior, esta era del hombre primitivo constituye un episodio largo, sombrío y sangriento. La ética de la selva y la moral de los bosques primigenios no están en consonancia con las reglas de las dispensaciones posteriores de la religión revelada y su elevado desarrollo espiritual. En los mundos normales y no experimentales, esta época es muy diferente a la acontecida en Urantia, que está marcada por luchas dilatadas en el tiempo y extraordinariamente brutales. Cuando salgáis de ese primer mundo en el que vivís, empezaréis a daros cuenta de la causa de esta larga y dolorosa lucha que se entabla en los mundos evolutivos, y, a medida que avancéis en vuestra senda al Paraíso, más llegaréis a comprender la conveniencia de estos hechos, aparentemente extraños. No obstante, a pesar de todas las vicisitudes de las primeras eras de la aparición del ser humano, los logros del hombre primitivo representan un capítulo magnífico, incluso heroico, en los anales de los mundos evolutivos del tiempo y el espacio.
\vs p052 1:5 \pc El hombre evolutivo primitivo no es una criatura pintoresca. En general, estos mortales primitivos viven en cuevas o acantilados. También construyen cabañas rudimentarias en los grandes árboles. Antes de llegar a adquirir un elevado índice de inteligencia, los planetas en donde habitan se encuentran a veces invadidos de las clases más grandes de animales. Pero, temprano en esta era, los mortales aprenden a encender y a mantener el fuego y, con el incremento de la imaginación inventiva y de la mejora en las herramientas, el hombre evolutivo pronto somete a los animales de más tamaño y más difíciles de manejar. Las razas primitivas hacen igualmente un amplio uso de los animales voladores más grandes. Estas enormes aves son capaces de transportar a una o dos personas de tamaño medio durante un vuelo ininterrumpido de más de ochocientos kilómetros. En algunos planetas, estas aves son de gran utilidad al estar dotadas de un alto grado de inteligencia; a menudo, pueden incluso articular muchas palabras en los idiomas del mundo. Son bastante inteligentes, muy obedientes e increíblemente afectuosas. Hace tiempo que estas aves de pasajeros se extinguieron en Urantia, pero vuestros primeros ancestros disfrutaron de sus servicios.
\vs p052 1:6 \pc La adquisición por parte del hombre de juicio ético, de voluntad moral, coincide normalmente con la aparición del lenguaje primitivo. Al alcanzar el nivel humano, tras este emerger de la voluntad mortal, estos seres, se hacen receptivos a la morada temporal de los modeladores divinos y, cuando mueren, los arcángeles, cumplidamente, eligen a muchos de ellos como supervivientes y los confirman para su posterior resurrección y fusión con el espíritu. Los arcángeles acompañan siempre a los príncipes planetarios, y el juicio de una dispensación del mundo es simultáneo a la llegada del príncipe.
\vs p052 1:7 Todos los mortales en los que habita el modelador del pensamiento son adoradores en potencia; han sido “alumbrados por la verdadera luz” y poseen la capacidad de ir a la búsqueda de un encuentro mutuo con la divinidad. No obstante, la religión primitiva o biológica del hombre primitivo consiste, en su mayor parte, en la persistencia de un miedo animal instintivo acompañada de un asombro ignorante y de una superstición tribal. La supervivencia de la superstición en las razas de Urantia no dice mucho de vuestro desarrollo evolutivo ni es compatible con vuestros logros, por otra parte magníficos, en el ámbito del progreso material. Pero esta religión temprana del miedo responde a un propósito muy valioso porque sirve de freno al encendido temperamento de estas criaturas primitivas. Es la precursora de la civilización y el terreno en el que el príncipe planetario y sus asistentes plantarán después la semilla de la religión revelada.
\vs p052 1:8 \pc El príncipe planetario llega ordinariamente en unos cien mil años tras el momento en que el hombre logra la posición erguida; el soberano del sistema lo envía una vez que los portadores de vida le informan que la voluntad ha entrado en acción, aunque relativamente pocos han alcanzado este estado de desarrollo. Normalmente, el príncipe planetario y su comitiva visible tienen una buena acogida de parte de los mortales primitivos; de hecho, los miran a menudo con asombro y reverencia y, si no se les refrena, casi con adoración.
\usection{2. EL HOMBRE TRAS LA LLEGADA DEL PRÍNCIPE PLANETARIO}
\vs p052 2:1 Con la llegada del príncipe planetario, empieza una nueva dispensación. El gobierno hace su aparición en la tierra y se llega a una época tribal avanzada. Durante algunos miles de años de este régimen, se llevan a cabo grandes progresos sociales. Durante esta época, en condiciones normales, los mortales consiguen un elevado estado de civilización. No continúan en la barbarie por tan largo período de tiempo como en el caso de las razas de Urantia. Si bien, la rebelión obra tales cambios en un mundo habitado, que es difícil que podáis llegar a tener alguna idea de lo que significa tener un régimen así en un planeta normal.
\vs p052 2:2 La duración media de esta dispensación es de unos quinientos mil años, algunas veces más, otras veces menos. Durante esta era, el planeta se conecta e integra en las vías circulatorias del sistema, y se asigna a un contingente completo de ayudantes seráficos y de otros órdenes celestiales a su administración. Los modeladores del pensamiento acuden en un creciente número, y los guardianes seráficos aumentan su régimen de supervisión de los mortales.
\vs p052 2:3 Cuando el príncipe planetario llega a un mundo primitivo, es la religión evolutiva del miedo y la ignorancia la que prevalece. El príncipe y su comitiva imparten las primeras revelaciones de la verdad superior y de la organización del universo de una manera sencilla y, por lo general, en relación a los asuntos del sistema local. Antes de la llegada del príncipe planetario, la religión sigue un proceso totalmente evolutivo. Luego, progresa de la mano tanto de las revelaciones sucesivas como del crecimiento evolutivo. Cada dispensación, cada una de las épocas en las que transcurre la existencia mortal, es perceptora de un relato ampliado de la verdad espiritual y de la ética religiosa. El desarrollo de la capacidad de los habitantes de un mundo en cuanto a su receptividad espiritual determina, notablemente, su grado de avance espiritual y la magnitud de esta revelación de naturaleza religiosa.
\vs p052 2:4 En cada una de estas dispensaciones se asiste a un amanecer espiritual, y las diferentes razas y sus diversas tribus tienden a desarrollar sistemas especiales de pensamiento religioso y filosófico. Dos tendencias recorren uniformemente todas estas religiones raciales: los temores primigenios del hombre primitivo y las revelaciones posteriores del príncipe planetario. En algunos aspectos, los urantianos parecen no haber salido del todo de esta etapa evolutiva del planeta. Conforme prosigáis este estudio, os percataréis con más claridad de cómo el curso seguido por vuestro mundo se desvía del curso normal de progreso y desarrollo evolutivos.
\vs p052 2:5 \pc Pero el príncipe planetario no es el “Príncipe de la Paz”. Las luchas raciales y las guerras tribales continúan durante esta dispensación, aunque con frecuencia y severidad decrecientes. Esta es la gran era de la dispersión racial, que culmina en un período de intenso nacionalismo. El color es la base de la formación de grupos tribales y nacionales, y las diferentes razas a menudo desarrollan idiomas por separado. En su expansión, cada grupo de mortales tiende a buscar el aislamiento. Es la existencia de muchas lenguas la que favorece esta separación. Antes de la unificación de las distintas razas, sus implacables guerras resultan a veces en la erradicación de pueblos completos; los hombres naranjas y los verdes son particularmente propensos a dicha extinción.
\vs p052 2:6 En los mundos de tipo medio, durante la última parte del gobierno del príncipe, la vida nacional empieza a reemplazar a la organización tribal o, más bien, a superponerse a los grupos tribales ya formados. Si bien, el gran logro social de la época del príncipe es la gradual aparición de la vida familiar. Hasta este momento, las relaciones humanas han sido principalmente de índole tribal; ahora, comienza el hogar a tomar forma.
\vs p052 2:7 Esta es la dispensación en la que se consigue la igualdad de sexos. En algunos planetas, el hombre gobierna a la mujer; en otros, impera lo opuesto. Durante esta época, en los mundos normales se establece la igualdad plena entre sexos, lo que representa un paso previo al más consumado logro de los ideales de la vida familiar. Estos son los albores de la era de oro del hogar. Paulatinamente, la idea del gobierno tribal va dando paso a la doble noción de vida nacional y vida familiar.
\vs p052 2:8 Durante esta época, la agricultura hace su aparición. El crecimiento de la idea de la familia resulta incompatible con la vida itinerante y agitada del cazador. Poco a poco se va asentando la costumbre de vivir en un lugar fijo y labrar la tierra. La domesticación de los animales y el desarrollo del arte doméstico avanzan con rapidez. Al alcanzarse la cúspide de la evolución biológica, se llega a un alto nivel de civilización, pero hay poco desarrollo de orden mecánico; la invención es el rasgo característico de la siguiente era.
\vs p052 2:9 \pc Las razas se depuran y se elevan a un alto grado de perfección física y vigor intelectual antes del fin de esta era. El desarrollo primitivo de un mundo normal se fundamenta, significativamente, en el plan diseñado para promover el aumento de los tipos de mortales mejor dotados con una reducción proporcional de los peor dotados. Y el fracaso de vuestros antiguos pueblos al no discriminar entre estos dos tipos de mortales que explica la presencia de tantos seres con deficiencias y en declive degenerativo entre las razas actuales de Urantia.
\vs p052 2:10 Uno de los grandes logros de la era del príncipe consiste en restringir la multiplicación de seres con deficiencias mentales y socialmente inadaptados. Mucho antes de la época de la llegada de los segundos hijos, los adanes, la mayoría de los mundos se aplican seriamente a la tarea de la depuración de la raza, algo que los pueblos de Urantia aún no han emprendido con seriedad.
\vs p052 2:11 Este problema de la mejora de la raza no es una tarea de tanta envergadura si se emprende en esta era temprana de la evolución humana. El período anterior de luchas tribales y de dura competición por la supervivencia racial ha apartado a la mayoría de las estirpes anormales y con deficiencias. Una persona con un retraso mental profundo no tiene muchas posibilidades de sobrevivir en una organización social tribal primitiva y beligerante. Es el sentimiento equivocado de vuestras civilizaciones parcialmente perfeccionadas el que fomenta, protege y perpetúa las estirpes irremediablemente deficientes de las razas humanas evolutivas.
\vs p052 2:12 No es ternura ni altruismo prodigar una ineficaz conmiseración a seres humanos en declive degenerativo, a mortales irrecuperablemente anormales y deficientemente dotados. Incluso en el más normal de los mundos evolutivos, existen suficientes diferencias entre seres individuales y entre numerosos grupos sociales como para garantizar el pleno ejercicio de todas esas nobles cualidades nacidas del sentimiento altruista y del ministerio desinteresado a los mortales, sin perpetuar linajes de la humanidad en evolución socialmente inadaptados y en declive moral. Hay muchas oportunidades para el ejercicio de la tolerancia y del altruismo en favor de aquellos seres desafortunados y necesitados, que no han perdido irreparablemente su herencia moral ni han destruido para siempre su derecho espiritual de nacimiento.
\usection{3. EL HOMBRE TRAS LA LLEGADA DE ADÁN}
\vs p052 3:1 Cuando el ímpetu originario de la vida evolutiva ha culminado su etapa biológica, cuando el hombre ha alcanzado la cúspide del desarrollo animal, llega un orden segundo de filiación y se inaugura la segunda dispensación de gracia y ministerio. Esto es cierto en todos los mundos evolutivos. Cuando se ha alcanzado el más elevado nivel de vida evolutiva, cuando el hombre primitivo ha ascendido todo lo posible en la escala biológica, siempre aparecen en el planeta, enviados por el soberano del sistema, un hijo y una hija materiales.
\vs p052 3:2 Los modeladores del pensamiento cada vez se otorgan en su mayor parte a los hombres posadánicos, y un número en constante aumento de estos mortales adquiere la capacidad para fusionarse, posteriormente, con el modelador. Mientras obran en calidad de hijos descendentes de Dios, los adanes no poseen modeladores, pero sus vástagos planetarios ---por línea directa o cruzamiento--- se convierten, en su momento, en legítimos aspirantes para ser receptores de los mentores misteriosos. Al término de la era posadánica, el planeta dispone de su contingente completo de servidores celestiales; simplemente que no se otorgan los modeladores destinados a fusionarse de manera generalizada.
\vs p052 3:3 \pc El objetivo primordial del régimen adánico es influir sobre el hombre evolutivo para que termine su tránsito de la etapa de civilización en la que se encuentra de cazador y pastor a la de agricultor y horticultor, que se complementará con expansiones urbanas e industriales. Diez mil años de esta dispensación de mejoradores biológicos son suficientes para efectuar una magnífica transformación. Veinticinco mil años de tal administración, que conjunta la sabiduría del príncipe planetario y la de los hijos materiales, preparan normalmente, la esfera para la llegada de un hijo magistrado.
\vs p052 3:4 \pc En esta época generalmente se evidencia el fin del proceso de exclusión de los inaptos y la consiguiente depuración de las estirpes raciales; en los mundos normales, las anómalas tendencias de origen animal llegan a eliminarse casi por completo de las razas reproductoras de ese mundo.
\vs p052 3:5 La progenie adánica nunca se cruza con los linajes menos dotados de las razas evolutivas. El plan divino tampoco contempla el emparejamiento personal de los adanes y las evas planetarios con los pueblos evolutivos. Este proyecto de mejoramiento racial es labor de su progenie. Si bien, los vástagos de los hijos e hijas materiales se preparan para dicha actuación durante generaciones antes de emprender este ministerio en favor de la mezcla de razas.
\vs p052 3:6 El don del plasma vital adánico, otorgado a las razas mortales, resulta en la inmediata elevación de la capacidad intelectual al igual que en la aceleración del progreso espiritual. Habitualmente, se da también cierto mejoramiento físico. En un mundo de tipo medio, la dispensación posadánica es una era de grandes invenciones, del control de la energía y del desarrollo mecánico. Esta es la era en la que aparece una diversificada manufacturación de productos y se produce el control de las fuerzas naturales; es la edad de oro de la exploración y del sometimiento definitivo del planeta. Una gran parte del progreso material de los mundos ocurre durante este período en el que se inicia el desarrollo de las ciencias físicas, justamente una época como la que Urantia está ahora experimentando. Vuestro mundo sufre un retraso de toda una dispensación o más respecto al calendario regular previsto para los planetas.
\vs p052 3:7 Hacia el final de la dispensación adánica en un planeta normal, las razas están prácticamente mezcladas, de modo que ciertamente se puede anunciar que “De una sangre ha hecho Dios todas las naciones”, y que su hijo “ha hecho a todos los pueblos de un solo color”. El color de esta raza cruzada es de un tono algo aceitunado de tinte violeta, el “blanco” racial de las esferas.
\vs p052 3:8 \pc El hombre primitivo es en gran parte carnívoro; los hijos e hijas materiales no comen carne, pero, en pocas generaciones, sus vástagos normalmente tienden a una alimentación de tipo omnívoro; no obstante, grupos completos de sus descendientes continúan a veces sin comer carne. Este origen doble de las razas posadánicas explica la presencia, en estos linajes humanos mezclados, de vestigios anatómicos pertenecientes tanto a grupos de animales herbívoros como a carnívoros.
\vs p052 3:9 Al cabo de diez mil años de cruzamiento racial, los linajes resultantes muestran diferentes grados de mezcla anatómica; algunas razas portan más signos de ancestros no comedores de carne, mientras que otras poseen más rasgos distintivos y características físicas de sus progenitores evolutivos carnívoros. La mayoría de estas razas del mundo se vuelven pronto omnívoras, subsistiendo a base de una amplia variedad de alimentos tanto del reino animal como del reino vegetal.
\vs p052 3:10 \pc La época posadánica constituye la dispensación del internacionalismo. Al ir concluyendo la labor de la mezcla racial, el nacionalismo declina y la hermandad entre los hombres comienza verdaderamente a tomar forma. El gobierno representativo empieza a sustituir a la monarquía o forma paternalista de gobierno. El sistema educativo se extiende por todo el mundo y, paulatinamente, la lengua del pueblo violeta reemplaza los idiomas de las razas. Hasta que las razas no están bastante bien mezcladas y no se habla una lengua común es raro que se logre la paz y la cooperación globales.
\vs p052 3:11 Durante los últimos siglos de la era posadánica se desarrolla un renovado interés por el arte, la música y la literatura, y este despertar a escala mundial es la señal para la aparición del hijo magistrado. La cúspide del desarrollo de esta era se manifiesta en el interés generalizado respecto a las realidades intelectuales, a la verdadera filosofía. La religión se vuelve menos nacionalista; se convierte cada vez más en una cuestión planetaria. Estas eras se distinguen por el relato de nuevas revelaciones de la verdad. También los altísimos de las constelaciones empiezan a gobernar en los asuntos de los hombres. La verdad se revela hasta el nivel de la administración de las constelaciones.
\vs p052 3:12 Esta era se caracteriza por un gran avance ético; la meta de la sociedad es lograr la hermandad entre los hombres. La paz mundial ---el cese del conflicto racial y de la animosidad nacional--- señala que el planeta está preparado para la llegada del tercer orden de filiación: el hijo magistrado.
\usection{4. EL HOMBRE TRAS LA LLEGADA DEL HIJO MAGISTRADO}
\vs p052 4:1 En los planetas normales y leales, esta época se inicia con las razas mortales mezcladas y biológicamente aptas. No hay problemas de raza ni de color; literalmente, todas las naciones y razas son de una sola sangre. Florece la hermandad entre los hombres, y las naciones aprenden a vivir en paz y tranquilidad en el planeta. Dicho mundo se halla en vísperas de un desarrollo intelectual supremo y culminante.
\vs p052 4:2 \pc Cuando un mundo evolutivo así está preparado para la era de los magistrados, un miembro del alto orden de hijos avonales hace su aparición en misión como magistrado. El príncipe planetario y los hijos materiales tienen su origen en el universo local; el hijo magistrado procede del Paraíso.
\vs p052 4:3 Cuando los avonales del Paraíso llegan a las esferas de los mortales en actuaciones judiciales, solamente en calidad de jueces de una dispensación, nunca se encarnan. Pero cuando acuden en misión como magistrados, al menos en la primera que realizan, siempre se encarnan, aunque no experimentan el nacimiento ni tampoco mueren como los habitantes del mundo. Pueden vivir durante generaciones en aquellos casos en los que permanecen como gobernantes de determinados planetas. Cuando su misión concluye, abandonan la vida planetaria y regresan a su condición anterior de filiación divina.
\vs p052 4:4 Cada nueva dispensación expande el horizonte de la religión revelada y los hijos magistrados amplían la revelación de la verdad hasta incluir los asuntos del universo local y todos sus integrantes.
\vs p052 4:5 \pc Tras la primera llegada de un hijo magistrado, las razas llevan pronto a cabo su liberación económica. El trabajo diario que se precisa para garantizar la propia independencia correspondería a dos horas y media de vuestro tiempo. No entraña riesgo alguno liberar a estos mortales éticos e inteligentes de tal atadura. Estas cultivadas personas saben cómo usar su tiempo libre para el mejoramiento personal y el avance planetario. Esta era es testigo de un nuevo impulso en la depuración de los linajes raciales al restringirse la reproducción entre los individuos menos aptos y peor dotados.
\vs p052 4:6 El gobierno político y la administración social de las razas siguen mejorando y el autogobierno está bien consolidado hacia el final de esta era. Al decir autogobierno aludimos al más elevado tipo de gobierno representativo. Estos mundos solo promocionan y honran a los líderes y gobernantes más capacitados para asumir responsabilidades sociales y políticas.
\vs p052 4:7 Durante esta época, la mayoría de los mortales del mundo son morada de los modeladores. Pero incluso entonces, todavía no se otorgan a los mentores divinos de forma generalizada. Los modeladores destinados a la fusión todavía no se conceden a todos los mortales planetarios; aún hace falta que las criaturas de voluntad opten por los mentores misteriosos.
\vs p052 4:8 Durante las últimas eras de esta dispensación, la sociedad empieza a regresar a formas de vida más simples. La compleja naturaleza de una civilización que avanza está tocando su fin; los mortales están aprendiendo a vivir de un modo más natural y eficaz. Y esta tendencia se incrementa en cada una de las épocas venideras. Es la era del florecimiento del arte, la música y la enseñanza superior. Las ciencias físicas ya han alcanzado la cima de su desarrollo. En un mundo ideal, al final de esta época se asiste plenamente a un gran despertar religioso, a una lucidez espiritual a escala mundial. Este grado inmenso de vivificación de la naturaleza espiritual de las razas es la señal para la llegada del hijo de gracia y la inauguración de la quinta época de los mortales.
\vs p052 4:9 \pc Puede suceder que alguno de los muchos planetas no esté preparado para recibir a un hijo de gracia tras una sola misión en rango de magistrado; en tal caso, habrá una segunda e incluso una sucesión de misiones de hijos magistrados; cada una de ellas hará que las razas avancen de una dispensación a otra hasta que el planeta esté listo para el don del hijo de gracia. En la segunda misión y en las misiones venideras, los hijos magistrados pueden o no encarnarse. Pero cualquiera que sea el número de hijos magistrados que pueda hacer su aparición ---y estos pueden también venir, en ese mismo estatus, después del hijo de gracia---, la llegada de cada uno de ellos señala el fin de una dispensación y el comienzo de otra.
\vs p052 4:10 \pc Estas dispensaciones de los hijos magistrados abarcan una duración entre veinticinco y cincuenta mil años del tiempo de Urantia. A veces, alguno de estos periodos de tiempo es más corto y, en raras ocasiones, incluso más largo. Pero en la plenitud de los tiempos, uno de estos mismos hijos magistrados nacerá como el hijo de gracia del Paraíso.
\usection{5. EL HOMBRE TRAS LA LLEGADA DEL HIJO DE GRACIA}
\vs p052 5:1 Cuando se consigue cierta excelencia de desarrollo intelectual y espiritual en un mundo habitado, siempre llega un hijo de gracia del Paraíso. En los mundos normales no aparece en la carne hasta que las razas no han alcanzado los más elevados niveles de desarrollo intelectual y de logro ético. Si bien, en Urantia, el hijo de gracia, vuestro propio hijo creador, apareció al final de la dispensación adánica, pero ese no es el orden habitual de los acontecimientos en los mundos del espacio.
\vs p052 5:2 Cuando los mundos han alcanzado un grado de madurez que los hace propicios para la espiritualización, llega el hijo de gracia. Estos hijos siempre pertenecen al orden avonal o de los magistrados salvo en el caso, que sucede una sola vez en cada universo local, en el que el hijo creador se prepara para su último ministerio de gracia en un mundo evolutivo, tal como ocurrió cuando Miguel de Nebadón apareció en Urantia para darse como don a vuestras razas mortales. Solo un mundo entre casi diez millones puede disfrutar de dicho dádiva; todos los demás mundos avanzan espiritualmente gracias al ministerio de gracia de un hijo del Paraíso del orden de los avonales.
\vs p052 5:3 \pc El hijo de gracia llega a un mundo con una elevada cultura educativa y se encuentra con una raza espiritualmente formada y preparada para asimilar enseñanzas avanzadas y apreciar su misión. Es una época caracterizada por la búsqueda a escala mundial de cultura moral y de verdad espiritual. El gran anhelo que mueve a los mortales de esta dispensación es la comprensión de la realidad cósmica y la comunión con la realidad espiritual. Las revelaciones de la verdad se amplían hasta incluir el suprauniverso. Hacen su aparición sistemas de enseñanza enteramente nuevos y gobiernos que reemplazan a los rudimentarios regímenes de tiempos anteriores. La alegría de vivir adquiere un nuevo color y las respuestas a la vida se glorifican hasta alcanzar alturas de tonalidades y cualidades celestiales.
\vs p052 5:4 El hijo de gracia vive y muere para la elevación espiritual de las razas mortales del mundo. Instituye el “camino nuevo y vivo”; su vida es una encarnación de la verdad del Paraíso en carne mortal ---el espíritu mismo de la verdad--- en cuyo conocimiento los hombres se harán libres.
\vs p052 5:5 En Urantia, la institución de este “camino nuevo y vivo” constituyó un hecho al igual que una verdad. Debido al aislamiento de Urantia por la rebelión de Lucifer, había quedado en suspenso el sistema previsto por el que los mortales, al morir, pueden pasar directamente a las orillas de los mundos de las moradas. Antes de los días de Cristo Miguel, en Urantia todas las almas dormían hasta las resurrecciones dispensacionales o hasta las resurrecciones milenarias especiales. Ni incluso a Moisés se le permitió ir al otro lado hasta que no tuvo lugar una resurrección especial; Caligastia, el príncipe planetario caído, se oponía a tal liberación. Si bien, desde el día de Pentecostés, los mortales de Urantia pueden, de nuevo, continuar directamente a las esferas morontiales.
\vs p052 5:6 \pc Tras su resurrección, al tercer día después de abandonar su vida encarnada, el hijo de gracia asciende a la derecha del Padre Universal, le da testimonio de su misión y regresa al hijo creador en la sede central del universo local. En ese momento, el avonal de gracia y el creador Miguel envían su espíritu conjunto, el espíritu de la verdad, al mundo en el que se otorgó. Este es el momento en el que “el espíritu del hijo triunfante se derrama sobre toda carne”. El espíritu materno del universo también participa en este don del espíritu de la verdad y, en concurrencia con esto, se promulga el edicto para la concesión de los modeladores del pensamiento. A partir de entonces, todas las criaturas volitivas de mente normal de ese mundo recibirán un modelador en cuanto alcancen la edad de responsabilidad moral, de elección espiritual.
\vs p052 5:7 Si este avonal de gracia tuviera que regresar al mundo tras su misión, no se encarnaría sino que vendría “en gloria con las multitudes seráficas”.
\vs p052 5:8 \pc La era que sigue al hijo de gracia puede tener una duración entre diez mil y cien mil años. Estas eras dispensacionales no tienen un tiempo establecido de duración. Se trata de una época de gran progreso ético y espiritual. Bajo la influencia espiritual de estas épocas, el carácter humano sufre formidables transformaciones y experimenta un extraordinario desarrollo. Es factible poner en práctica la regla de oro. Las enseñanzas de Jesús tienen una aplicabilidad real en un mundo de mortales que han adquirido la formación previa de parte de los hijos anteriores a los hijos de gracia con sus dispensaciones para ennoblecer el carácter y aumentar la cultura.
\vs p052 5:9 Durante esta era casi se ha resuelto el problema de la enfermedad y de la delincuencia. La reproducción selectiva ya ha eliminado en gran medida el declive degenerativo. Prácticamente, se ha vencido a la enfermedad debido al gran grado de resistencia de las razas adánicas y a la aplicación inteligente y mundial de los descubrimientos de las ciencias físicas de las épocas precedentes. Durante este período, la duración media de la vida está muy por encima del equivalente de trescientos años del tiempo de Urantia.
\vs p052 5:10 A lo largo de esta época, hay una paulatina disminución de la supervisión gubernamental. El verdadero autogobierno comienza a actuar; cada vez se necesitan menos leyes restrictivas. Las ramas militares de la resistencia nacional van desapareciendo; la era de la armonía internacional está, en verdad, llegando. Existen muchas naciones, en su mayor parte determinadas por la distribución de la tierra, pero solamente una raza, un idioma y una religión. Los asuntos de los mortales se acercan mayormente, aunque no del todo, al estado ideal. ¡Verdaderamente es una era grande y gloriosa!
\usection{6. LA ERA POSTERIOR A LA MISIÓN DE GRACIA EN URANTIA}
\vs p052 6:1 El hijo de gracia es el Príncipe de Paz. Llega con el mensaje, “Paz en la tierra y buena voluntad para con los hombres”. En los mundos normales se trata de una dispensación de paz a nivel mundial; las naciones ya no se adiestran para la guerra. Pero estos beneficiosos efectos no estaban presentes a la llegada de Cristo Miguel, vuestro hijo de gracia. Urantia no avanza en un orden normal. Vuestro mundo no está en sintonía con la secuencia planetaria. Vuestro Maestro, cuando estaba en la tierra, previno a sus discípulos que su venida no traería a Urantia el habitual reino de paz. Les dijo claramente que habría “guerras y rumores de guerra”, y que las naciones se levantarían unas contra otras. En otro momento dijo: “No penséis que he venido a traer paz a la tierra”.
\vs p052 6:2 Incluso en los mundos evolutivos normales, no es tarea fácil llevar a cabo la hermandad del hombre a escala mundial. En un planeta confuso y sin orden como Urantia, para lograr algo así, se necesita mucho más tiempo y esfuerzo. Sin ayuda, la evolución social difícilmente puede conseguir resultados satisfactorios en una esfera espiritualmente aislada. La revelación religiosa es esencial para llevar a cabo dicha hermandad en Urantia. Y, aunque Jesús ha mostrado el camino para alcanzar de inmediato la hermandad espiritual, conseguir la hermandad social en vuestro mundo depende mucho de que se lleven a efecto las siguientes transformaciones personales y adaptaciones planetarias:
\vs p052 6:3 \li{1.}\bibemph{Fraternidad social}. La multiplicación de los contactos sociales internacionales e interraciales y de las relaciones fraternales por medio de viajes, comercio y juegos competitivos. El desarrollo de un lenguaje común y la multiplicación de personas plurilingües. El intercambio racial y nacional de estudiantes, maestros, industriales y filósofos religiosos.
\vs p052 6:4 \li{2.}\bibemph{Interacción intelectual fecunda}. La hermandad es imposible en un mundo cuyos habitantes son tan primitivos que no son capaces de reconocer la insensatez del desmesurado egoísmo. Debe producirse un intercambio de literatura nacional y racial. Cada raza debe familiarizarse con el pensamiento de todas las razas; cada nación debe conocer la forma de sentir de todas las naciones. La ignorancia engendra sospecha, y la sospecha es incompatible con la indispensable actitud de compasión y amor.
\vs p052 6:5 \li{3.}\bibemph{Despertar ético}. Solo una conciencia ética puede revelar la inmoralidad de la intolerancia humana y la pecaminosidad de la lucha fratricida. Solo una conciencia moral puede condenar los males de la envidia nacional y de los celos raciales. Solo los seres morales estarán siempre en búsqueda de esa percepción espiritual imprescindible para vivir la regla de oro.
\vs p052 6:6 \li{4.}\bibemph{Sabiduría política}. La madurez emocional es fundamental para el dominio de uno mismo. Solo la madurez emocional garantizará que los métodos internacionales de juicios civilizados sustituyan al arbitraje bárbaro de la guerra. Habrá algún día sabios estadistas que trabajarán por el bienestar de la humanidad procurando, al mismo tiempo, obrar en el interés de sus grupos nacionales o raciales. La astucia política de carácter egoísta es, en última instancia, suicida ---dañina para todas aquellas cualidades duraderas que garantizan la supervivencia planetaria del grupo---.
\vs p052 6:7 \li{5.}\bibemph{Percepción espiritual}. La hermandad del hombre se basa al fin y al cabo en el reconocimiento de la paternidad de Dios. La forma más rápida de lograr la hermandad de los hombres en Urantia consiste en llevar a efecto la transformación espiritual de la humanidad de hoy en día. El único modo de acelerar la tendencia natural de la evolución social es ejerciendo presión desde arriba, incrementando así la percepción moral y mejorando, al mismo tiempo, la capacidad del alma de cada uno de los mortales para comprender y amar a todos los demás. La comprensión mutua y el amor fraterno son elementos civilizadores trascendentales y factores poderosos en la consecución de la hermandad de los hombres.
\vs p052 6:8 \pc Si se os pudiera llevar desde vuestro mundo atrasado y confuso hasta un planeta normal que se encontrara ahora en la era posterior a la llegada del hijo de gracia, pensaríais que se os habría trasladado al cielo de vuestras tradiciones. Os resultaría difícil creer que estáis observando una esfera habitada por seres mortales siguiendo su curso evolutivo normal. Estos mundos están dentro de las vías circulatorias espirituales de su ámbito y disfrutan de todas las ventajas de las transmisiones del universo y de los servicios de la reflectividad del suprauniverso.
\usection{7. EL HOMBRE TRAS LA LLEGADA DEL HIJO PRECEPTOR}
\vs p052 7:1 El siguiente orden de filiación en llegar a un mundo evolutivo medio es el de los hijos preceptores de la Trinidad, los hijos divinos de la Trinidad del Paraíso. Y de nuevo nos damos cuenta de que Urantia no está en sintonía con sus esferas hermanas: vuestro Jesús prometió que regresaría. Ciertamente cumplirá su promesa, pero nadie sabe si su segunda venida antecederá o seguirá a la aparición de los hijos magistrados o de los hijos preceptores en Urantia.
\vs p052 7:2 Los hijos preceptores acuden en grupos a los mundos en camino de espiritualización. Un hijo preceptor planetario cuenta con la asistencia y el apoyo de setenta hijos primarios, doce hijos secundarios y tres de los más elevados y experimentados hijos del orden supremo de los dainales. Este colectivo permanece durante algún tiempo en el mundo, el suficiente como para llevar a cabo la transición de las épocas evolutivas a la era de luz y vida ---no menos de mil años de tiempo planetario y, con frecuencia, mucho más---. Dicha misión es el modo en el que la Trinidad hace su contribución a los esfuerzos realizados hasta ese momento por todos los seres personales divinos que han dispensado sus servicios en un mundo habitado.
\vs p052 7:3 \pc La revelación de la verdad se amplía ahora hasta incluir el universo central y el Paraíso. Las razas se vuelven espirituales en grado sumo. Un gran pueblo ha alcanzado su evolución y una gran época está en camino. Los sistemas de enseñanza, económicos y administrativos del planeta están experimentando transformaciones radicales. Se están estableciendo nuevos valores y relaciones. El reino de los cielos está manifestándose en la tierra y la gloria de Dios se derrama en el mundo.
\vs p052 7:4 En esta dispensación muchos mortales son trasladados de entre los vivos. A medida que avanza la era de los hijos preceptores de la Trinidad, la lealtad espiritual de los mortales del tiempo se hace cada vez más universal. La muerte natural se vuelve menos frecuente a medida que los modeladores se fusionan, cada vez con mayor frecuencia, con aquellos en los que moran durante la vida en la carne. El planeta se categoriza, finalmente, como perteneciente al orden modificado primario de ascensión de los mortales.
\vs p052 7:5 \pc La vida durante esta era es agradable y provechosa. El declive degenerativo y los linajes antisociales resultantes de la larga lucha evolutiva han desaparecido prácticamente. La duración de la vida se acerca a los quinientos años de Urantia; se evita el incremento de la raza humana mediante la regulación inteligente de la tasa reproductiva. Un orden enteramente nuevo de sociedad hace su aparición. Todavía existen grandes diferencias entre los mortales, pero el estado de la sociedad está más cerca de los ideales de la hermandad social y de la igualdad espiritual. El gobierno representativo está desapareciendo; el mundo está pasando a gobernarse mediante el autocontrol individual. La labor del gobierno se dirige principalmente a las tareas colectivas de la administración social y de la coordinación económica. La edad de oro está en marcha; la meta temporal de la larga e intensa lucha evolutiva planetaria está próxima. La recompensa de los tiempos pronto se hará realidad; la sabiduría de los Dioses está a punto de manifestarse.
\vs p052 7:6 Durante esta época, la administración física del mundo precisa alrededor de una hora diaria de parte de cualquier persona adulta; esto es, el equivalente de una hora de Urantia. El planeta está en estrecho contacto con los asuntos del universo, y sus habitantes examinan las últimas transmisiones con el mismo gran interés que vosotros manifestáis ahora por las últimas ediciones de vuestros periódicos diarios. Estas razas se ocupan de mil cosas de interés desconocidas en vuestro mundo.
\vs p052 7:7 \pc Cada vez más crece una genuina lealtad planetaria hacia el Ser Supremo. Generación tras generación, más miembros de la raza humana caminan en consonancia con aquellos que practican la justicia y viven la misericordia. De forma lenta pero segura, el mundo se gana para el servicio gozoso de los Hijos de Dios. En gran parte, las dificultades físicas y los problemas materiales se han resuelto; el planeta madura hacia una vida avanzada y una existencia más estable.
\vs p052 7:8 \pc Ocasionalmente, a lo largo de dicha dispensación, siguen llegando hijos preceptores a estos pacíficos mundos. Y no dejan el mundo hasta que se percatan de que el plan evolutivo en relación a ese planeta se está cumpliendo sin problema. Por lo general, un hijo magistrado en calidad de juez acompaña a los hijos preceptores en sus sucesivas misiones, mientras que otro hijo magistrado, del mismo rango, obra en el momento de su partida, y estas actuaciones judiciales continúan a través de las eras durante el transcurso del régimen dispuesto para los mortales del espacio y el tiempo.
\vs p052 7:9 Con cada misión periódica de los hijos preceptores de la Trinidad, ese excelso mundo se enaltece de forma consecutiva hasta alturas, siempre en ascenso, de sabiduría, espiritualidad e iluminación cósmica. Si bien, los nobles nativos de esta esfera son aún finitos y mortales. Nada es perfecto; sin embargo, en el funcionamiento de un mundo imperfecto y en la vida de sus habitantes humanos se están desarrollando cualidades que rozan la perfección.
\vs p052 7:10 \pc Los hijos preceptores de la Trinidad pueden regresar muchas veces al mismo mundo. Pero tarde o temprano, en relación con la finalización de alguna de sus misiones, se eleva al príncipe planetario a la condición de soberano planetario, y el soberano del sistema hace su aparición para proclamar el ingreso de dicho mundo en la era de luz y vida.
\vs p052 7:11 Juan dijo lo siguiente acerca del fin de la última misión de los hijos preceptores (al menos esa sería la cronología en un mundo normal): “Vi un cielo nuevo y una tierra nueva y la nueva Jerusalén descender del cielo, de parte de Dios, ataviada como una princesa hermoseada para su príncipe”.
\vs p052 7:12 Esta es la misma tierra renovada, ese planeta en avanzado estado de desarrollo, que el antiguo vidente visualizó cuando escribió: “‘Pues así como los nuevos cielos y la nueva tierra que voy a hacer perdurarán en mi presencia, así perduraréis vosotros y vuestros hijos. Así que de luna nueva en luna nueva y de \bibemph{sabbat} en \bibemph{sabbat} vendrá toda carne a adorar ante mí’, dice el Señor”.
\vs p052 7:13 Son los mortales de esta era los que se describen como “linaje escogido, real sacerdocio, nación santa, pueblo alabado; y daréis a conocer las alabanzas de aquel que os llamó de las tinieblas a su luz admirable”.
\vs p052 7:14 \pc Sea cual sea la particular historia natural de algún determinado planeta, al margen de que haya sido un mundo enteramente leal, contaminado por el mal o maldecido por el pecado ---sea cual sea su historia---, tarde o temprano, la gracia de Dios y el ministerio de los ángeles dan paso al día de la venida de los hijos preceptores de la Trinidad; y su partida, tras su misión final, inaugurará esta magnífica era de luz y vida.
\vs p052 7:15 Todos los mundos de Satania pueden unirse en la esperanza de aquel que escribió: “Pero nosotros esperamos, según sus promesas, cielo nuevo y tierra nueva, en los cuales mora la justicia. Por eso, amados, estando en espera de estas cosas, procurad con diligencia ser hallados por él en paz, sin mancha e irreprochables”.
\vs p052 7:16 \pc La partida del colectivo de los hijos preceptores, al final de su primer reinado o de algún reinado posterior, da paso al comienzo de la era de luz y vida ---el umbral de transición entre el tiempo y la antesala de la eternidad---. El logro, a nivel planetario, de esta era de luz y vida sobrepasa en mucho las expectativas más anheladas de los mortales de Urantia, que no tienen más visión a largo plazo respecto a la vida futura que aquella que sus creencias religiosas imparten, y que describen el cielo como el destino inmediato y como la morada final de los mortales supervivientes.
\vsetoff
\vs p052 7:17 [Auspiciado por un mensajero poderoso temporalmente adscrito a los asistentes de Gabriel.]
