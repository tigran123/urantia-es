\upaper{82}{Evolución del matrimonio}
\author{Jefe de los serafines}
\vs p082 0:1 El matrimonio ---el emparejamiento--- nace de la existencia de dos sexos. El matrimonio es el ajuste de la respuesta del hombre ante el otro sexo, mientras que la vida familiar es la suma resultante de todos estos ajustes evolutivos y adaptativos. El matrimonio es perdurable; no es intrínseco a la evolución biológica, pero constituye la base de toda evolución social y es por consiguiente seguro que continuará existiendo de alguna forma. El matrimonio ha dado el hogar a la humanidad, y el hogar es el colofón de toda la larga y ardua lucha evolutiva.
\vs p082 0:2 Aunque las instituciones religiosas, sociales y educativas son todas fundamentales para la supervivencia de la civilización cultural, \bibemph{la familia es el gran civilizador}. El niño aprende de su familia y vecinos la mayor parte de los elementos esenciales de la vida.
\vs p082 0:3 Los humanos de los viejos tiempos no poseían una civilización social muy rica, pero la que tenían la transmitieron de forma fiel y efectiva a la generación siguiente. Y debéis reconocer que la mayoría de estas civilizaciones del pasado continuaron evolucionando con un mínimo de otras influencias institucionales porque el hogar funcionaba eficazmente. Hoy en día, las razas humanas poseen un rico patrimonio social y cultural, y este se debe pasar con sabiduría y eficacia a las generaciones venideras. La familia como institución educativa debe mantenerse.
\usection{1. EL INSTINTO DE EMPAREJAMIENTO}
\vs p082 1:1 A pesar de la brecha personal existente entre el hombre y de la mujer, el impulso sexual es suficiente como para garantizar su unión en pro de la reproducción de la especie. Este instinto obraba con eficacia bastante tiempo antes de que los humanos experimentaran mucho de lo que más recientemente se ha llamado amor, devoción y lealtad conyugal. El emparejamiento es una propensión innata y, el matrimonio, su repercusión social evolutiva.
\vs p082 1:2 El interés y el deseo sexuales no eran pasiones predominantes en los pueblos primitivos; simplemente las daban por hecho. Toda su experiencia reproductora estaba exenta de embellecimientos imaginativos. La absorbente pasión sexual de los pueblos más altamente civilizados se debe principalmente a la mezcla de razas, en especial ahí donde la naturaleza evolutiva se ha estimulado por la imaginación asociativa y la apreciación de la belleza de los noditas y de los adanitas. Pero esta herencia andita la absorbieron las razas evolutivas en cantidades tan reducidas que no lograron proporcionar el suficiente dominio de sí mismo sobre las pasiones animales, despertadas y vivificadas debido a la dotación de una conciencia sexual más aguda y de impulsos más intensos de emparejamiento. De las razas evolutivas, el hombre rojo era el que tenía el código sexual más elevado.
\vs p082 1:3 \pc La regulación del sexo en relación al matrimonio muestra:
\vs p082 1:4 \li{1.}El progreso relativo de la civilización. Cada vez más, la civilización ha exigido que el sexo se satisfaga mediante canales apropiados y en conformidad con las costumbres.
\vs p082 1:5 \li{2.}La cantidad de linaje andita en cualquier población. Entre dichos grupos, el sexo se ha convertido en la expresión más alta y más baja tanto de la naturaleza física como de la emocional.
\vs p082 1:6 \pc Las razas sangik poseían pasiones animales normales, pero exhibían poca imaginación o valoración de la belleza y el atractivo físico del sexo opuesto. Lo que se llama atracción sexual es prácticamente inexistente incluso en las razas primitivas de hoy en día; estos pueblos sin mezclar tienen un claro instinto de emparejamiento, aunque una atracción sexual insuficiente como para crear graves problemas que precisen de un control social.
\vs p082 1:7 El instinto de emparejamiento es una de las fuerzas impulsoras físicas predominantes en los seres humanos; es la única emoción que, bajo la apariencia de gratificación individual, engaña eficazmente al hombre en su egoísmo para que ponga el bienestar y la perpetuación de la raza muy por encima de la comodidad individual y de la libertad de las responsabilidades personales.
\vs p082 1:8 Como institución, el matrimonio, desde sus primeros comienzos hasta los tiempos modernos, ilustra la evolución social de la tendencia biológica a la autoperpetuación. La perpetuación de la especie humana, en su avance, se hace realidad por la presencia de este impulso racial de emparejamiento, de un ímpetu llamado, en términos generales, “atracción sexual”. Esta gran necesidad biológica se convierte en el eje impulsor de toda índole de instintos, emociones y hábitos conexos: físicos, intelectuales, morales y sociales.
\vs p082 1:9 El acopio de alimentos era el estímulo impulsor del hombre salvaje, pero cuando la civilización garantiza comida en abundancia, el deseo sexual se convierte muchas veces en su impulso dominante y, por lo tanto, siempre precisa regularse socialmente. En los animales, la periodicidad instintiva rige su propensión al emparejamiento, pero, puesto que el hombre es, en gran medida, alguien que se controla a sí mismo, el deseo sexual no es periódico. Es por ello por lo que se hace necesario que la sociedad imponga a las personas un control sobre sí mismas.
\vs p082 1:10 Ninguna emoción o impulso humano, cuando está desenfrenado o se extralimita, puede producir tanto daño y pesar como este poderoso deseo sexual. Su sumisión inteligente a las regulaciones impuestas por la sociedad es la prueba suprema de la vigencia de cualquier civilización. Autocontrol, y más autocontrol, es el requerimiento creciente de una humanidad que evoluciona. El secretismo, la insinceridad y la hipocresía podrán enmascarar los problemas sexuales, pero no proporcionan soluciones, ni hacen que la ética avance.
\usection{2. TABÚES RESTRICTIVOS}
\vs p082 2:1 La historia de la evolución del matrimonio es sencillamente la historia del control del sexo mediante la presión de las restricciones sociales, religiosas y civiles. La naturaleza apenas reconoce a las personas; no tiene en cuenta los llamados sentimientos morales; está solo y exclusivamente interesada en la reproducción de la especie. La naturaleza hace abrumadoramente hincapié en la reproducción, pero deja con indiferencia los problemas que se originan para que sea la sociedad la que los resuelva, creando así una siempre presente y grave problemática para la humanidad evolutiva. Este conflicto social consiste en la guerra interminable entre los instintos básicos y el desarrollo de la ética.
\vs p082 2:2 \pc Entre las razas primitivas, existía poca o ninguna regulación de las relaciones entre los sexos. Debido a esta licencia sexual, no existía la prostitución. Hoy en día, los pigmeos y otros pueblos atrasados no poseen la institución del matrimonio; un estudio de ellos revela las sencillas costumbres de emparejamiento practicadas por las razas primitivas. Pero todos los pueblos antiguos deben siempre estudiarse y juzgarse a la luz de las normas morales de las costumbres que rigen en sus propios tiempos.
\vs p082 2:3 El amor libre, sin embargo, no ha tenido nunca una buena reputación por encima de la escala del salvajismo más absoluto. En el momento en que empiezan a tomar forma los grupos sociales, comienzan a desarrollarse los códigos y las restricciones maritales. El emparejamiento, pues, ha progresado a través de un sinnúmero de cambios, desde un estado de licencia sexual casi total hasta las normas del siglo XX que instan a una restricción sexual relativamente completa.
\vs p082 2:4 En las primeras etapas del desarrollo tribal, las costumbres y los tabúes restrictivos eran muy rudimentarios, aunque mantenían de hecho a los sexos separados ---algo que favorecía la tranquilidad, el orden y el trabajo---, y la larga evolución del matrimonio y del hogar había comenzado. Las costumbres respecto a la ropa, los adornos y las prácticas religiosas tuvieron su origen en estos primitivos tabúes que definieron el rango de libertades sexuales y acabaron por crear los conceptos de vicio, delito y pecado. Si bien, durante mucho tiempo, perduró el hábito de suspender todas las regulaciones sexuales los días de fiesta importantes, especialmente el Primero de Mayo.
\vs p082 2:5 \pc Las mujeres siempre han estado sometidas a unos tabúes más restrictivos que los hombres. Las primitivas costumbres concedían el mismo grado de libertad sexual a las mujeres no casadas que a los hombres, pero siempre se ha exigido a las esposas que sean fieles a sus maridos. El matrimonio antiguo no restringía demasiado las libertades sexuales del hombre, pero sí dictaba nuevos tabúes para el libertinaje sexual de la mujer. Las mujeres casadas siempre han llevado alguna marca que las distinguía como clase aparte, entre estas el peinado, la ropa, el velo, la reclusión, los adornos y los anillos.
\usection{3. TEMPRANAS COSTUMBRES MATRIMONIALES}
\vs p082 3:1 El matrimonio es la respuesta institucional del organismo social a la siempre presente tensión biológica del hombre en su incesante impulso a la reproducción ---la propagación de sí mismo---. El emparejamiento es generalmente natural y, a medida que se desarrolló la sociedad desde lo simple a lo complejo, se suscitó la correspondiente evolución de las costumbres respecto a este, la génesis de la institución marital. Dondequiera que la evolución social haya progresado a la etapa en la que se generan las costumbres, se encontrará el matrimonio como institución evolutiva.
\vs p082 3:2 Siempre ha habido y habrá dos ámbitos diferentes en el matrimonio: las costumbres, las leyes que regulan los aspectos externos del emparejamiento, y las relaciones por lo demás secretas y personales entre hombres y mujeres. Las personas siempre se han rebelado contra las regulaciones sexuales impuestas por la sociedad; y esta es la razón de tal milenaria problemática sexual: la conservación de uno mismo es individual, pero es el grupo el que la lleva a cabo; la perpetuación de uno mismo es social, pero es el impulso individual el que la garantiza.
\vs p082 3:3 Las costumbres, cuando se respetan, tienen suficiente poder para restringir y controlar el impulso sexual, tal como ha quedado demostrado en todas las razas. Las normas matrimoniales siempre han sido un verdadero indicador del poder real de las costumbres y de la integridad funcional del gobierno civil. Pero las tempranas costumbres sexuales y de emparejamiento eran un cúmulo de regulaciones incongruentes y rudimentarias. Los padres, los hijos, los parientes y la sociedad tenían intereses encontrados en cuanto a la reglamentación matrimonial. Si bien, a pesar de todo ello, esas razas que ensalzaron y practicaron el matrimonio evolucionaron con naturalidad a niveles superiores y sobrevivieron en un mayor número.
\vs p082 3:4 \pc En los tiempos primitivos, el matrimonio era el precio a pagar por la posición social; la posesión de una esposa era un símbolo de distinción. El salvaje consideraba su día de boda como el evento que marcaba el inicio de sus responsabilidades y edad adulta. En cierta época, el matrimonio se consideraba como un deber social; en otra, como una obligación religiosa; y, todavía en otra, como una exigencia política para proporcionar ciudadanos al Estado.
\vs p082 3:5 Muchas tribus primitivas demandaban proezas que tuvieran que ver con robos como requisito para el matrimonio; los pueblos que vinieron después sustituyeron tales incursiones y saqueos por pruebas atléticas y juegos competitivos. Los vencedores de estos certámenes recibían el primer premio ---elegir entre las novias disponibles---. Entre los cazadores de cabezas, un joven no se podía casar hasta tanto no poseyera al menos una cabeza, aunque estos cráneos algunas veces se podían comprar. A medida que la compra de esposas declinó, estas se ganaban en concursos de adivinanzas, una práctica que aún sobrevive entre numerosos grupos de los hombres negros.
\vs p082 3:6 Con el avance de la civilización, determinadas tribus ponían en manos de las mujeres las duras pruebas para casarse que acreditaban la resistencia masculina; de esta manera, podían dar su aprobación al hombre elegido. En dichas pruebas, se mostraba su capacidad para cazar, luchar y proveer para la familia. Durante mucho tiempo, se requirió que el novio fuese un miembro de la familia de la novia al menos un año, viviendo y trabajando allí para mostrar que era merecedor de la novia que deseaba.
\vs p082 3:7 La esposa debía ser capaz de desarrollar trabajos duros y concebir hijos. Se le exigía que llevara a cabo cierta labor agrícola dentro de un período dado. Y si había tenido algún hijo antes del matrimonio, era aún más valiosa, ya que su fertilidad estaba así asegurada.
\vs p082 3:8 \pc El hecho de que los pueblos primitivos consideraban el no casarse como una deshonra, e incluso como un pecado, explica el origen de los matrimonios entre niños; puesto que si uno se debía casar, cuanto antes lo hiciera, mejor. Por lo general, también se creía que las personas sin estar casadas no podían entrar al mundo de los espíritus, y esto era un incentivo más para los matrimonios infantiles, incluso desde el momento del nacimiento y, a veces, hasta antes del nacimiento, supeditado al sexo del niño. Los antiguos creían que incluso los muertos debían casarse. Se empleaban a los primitivos casamenteros para concertar los matrimonios entre los difuntos. Los padres disponían que estos intermediarios efectuaran el matrimonio de un hijo muerto con la hija muerta de otra familia.
\vs p082 3:9 Entre los pueblos posteriores, la pubertad era la edad común para casarse, si bien, dicha edad fue avanzando en proporción directa al progreso de la civilización. Temprano en la evolución social, surgieron órdenes peculiares y célibes tanto de hombres como de mujeres; estas órdenes se iniciaron y mantuvieron por personas más o menos carentes de deseos sexuales ordinarios.
\vs p082 3:10 Muchas tribus permitían a los miembros del grupo gobernante que tuviesen relaciones sexuales con la novia antes de que se entregase a su marido. Cada uno de estos hombres entregaba un obsequio a la muchacha, y este fue el origen de la costumbre de dar regalos de boda. Entre algunos grupos, se esperaba que la joven se ganara su dote, que consistía en los presentes recibidos como recompensa a su servicio sexual en el salón de exhibición de la novia.
\vs p082 3:11 Algunas tribus casaban a sus jóvenes con viudas y mujeres mayores y, cuando se después se quedaban viudos, se les permitía casarse con las jóvenes; de esta manera, se aseguraban, tal como así lo expresaban, de que no fuesen ambos padres necios, porque creían que eso ocurriría si se permitía que dos jóvenes se emparejasen. Otras tribus limitaban el emparejamiento a personas pertenecientes al mismo tramo de edad. Fue tal limitación del matrimonio a ciertos tramos etarios la que primeramente dio origen a las ideas sobre el incesto. (En la India, incluso en la actualidad, no hay restricciones de edad para contraer matrimonio.)
\vs p082 3:12 \pc Según determinadas costumbres, la viudez era muy de temer, porque se mataba a las viudas o se les autorizaba a suicidarse ante la tumba de sus maridos, ya que se suponía que debían ir al mundo de los espíritus junto con ellos. Casi indefectiblemente, a la viuda que sobrevivía se le culpaba de la muerte de su marido. Algunas tribus las quemaban vivas. Si una viuda seguía viviendo, la vida que llevaba era de continuo luto y de insoportables restricciones sociales, puesto que un segundo matrimonio estaba generalmente mal visto.
\vs p082 3:13 En los tiempos antiguos, se alentaban muchas prácticas que en este momento se consideran inmorales. No era infrecuente que las esposas primitivas se enorgulleciesen de los romances de sus maridos con otras mujeres. La castidad en las jóvenes era un gran impedimento para el matrimonio; tener hijos antes del matrimonio aumentaba notablemente la idoneidad de la joven para ser esposa, puesto que el hombre estaba seguro de tener una compañera fértil.
\vs p082 3:14 Muchas tribus primitivas aprobaban el matrimonio de prueba hasta que la mujer se quedaba embarazada, momento en el que se celebraba la habitual ceremonia nupcial; entre otros grupos, la boda no se celebraba hasta que no naciese el primer hijo. Si una esposa era estéril, los padres la tenían que rescatar, y el matrimonio se anulaba. Las costumbres demandaban que toda pareja tuviese hijos.
\vs p082 3:15 Estos matrimonios primitivos a prueba estaban enteramente libres de todo atisbo de formalidades; eran sencillamente genuinos test de fecundidad. Los contrayentes se casaban permanentemente tan pronto como se comprobaba la fertilidad. Cuando las parejas modernas se casan con la idea de un oportuno divorcio si no están completamente satisfechos con la vida conyugal, en realidad contraen un tipo de matrimonio de prueba; algo que está muy por debajo del estado de verdadera aventura de sus menos civilizados ancestros.
\usection{4. EL MATRIMONIO BAJO LAS COSTUMBRES DE LA PROPIEDAD PRIVADA}
\vs p082 4:1 El matrimonio ha estado siempre estrechamente vinculado con la propiedad y la religión. La propiedad ha sido el estabilizador del matrimonio; la religión, su moralizador.
\vs p082 4:2 El matrimonio primitivo era una inversión, simple especulación económica; era más un asunto comercial que un idilio amoroso. Los antiguos se casaban para el beneficio y el bienestar del grupo; por lo tanto, era el grupo, sus padres y los ancianos, quienes planeaban y organizaban los matrimonios. El hecho mismo de que el matrimonio era más permanente entre las tribus primitivas que en numerosos pueblos modernos confirma que las costumbres sobre la propiedad resultaban eficaces en la estabilización de tal institución.
\vs p082 4:3 A medida que avanzó la civilización y la propiedad privada adquirió un mayor reconocimiento en las costumbres, el robo se convirtió en el delito más grave. El adulterio se contempló como una forma de robo, una vulneración de los derechos de propiedad del marido; no se menciona pues expresamente en los primeros códigos y costumbres. La mujer empezaba siendo propiedad de su padre, que transfería su título al marido, y todas las relaciones sexuales legalizadas surgieron a partir de estos derechos preexistentes de la propiedad. El Antiguo Testamento trata a las mujeres como una forma de propiedad; el Corán imparte que es inferior. El hombre tenía el derecho de prestar su esposa a un amigo o invitado, y esta costumbre aún prevalece entre determinados pueblos.
\vs p082 4:4 Los celos sexuales modernos no son innatos; son el resultado de las costumbres en evolución. El hombre primitivo no tenía celos de su mujer; solo vigilaba su propiedad. La razón por la que la mujer estaba sujeta a unas consideraciones sexuales más estrictas que su marido se debía a que su infidelidad implicaba descendencia y herencia. Muy pronto en el curso de la civilización, el hijo ilegítimo cayó en deshonra. Al principio, solo se castigaba a la mujer por el adulterio; más adelante, las costumbres decretaron también el escarmiento de su compañero y, durante muchas eras, el marido ofendido o el padre protector tenían pleno derecho de matar al infractor masculino. Los pueblos modernos conservan estas costumbres, que permiten los llamados crímenes de honor bajo una ley no escrita.
\vs p082 4:5 Puesto que el tabú de la castidad se originó como un aspecto de las costumbres en cuanto a la propiedad, se aplicó al principio a las mujeres casadas pero no a las solteras. En años posteriores, era el padre, más que el pretendiente, quien exigía castidad; para el padre, una virgen era un bien comercial: se obtenía por ella un precio más alto. Conforme se requería más y más la castidad, era habitual pagar al padre unas tasas nupciales en reconocimiento por el servicio de criar debidamente a la novia en la castidad para su futuro marido. Una vez comenzada, esta idea de la castidad femenina se afianzó tanto en las razas que se convirtió en algo rutinario enjaular literalmente a las chicas, encerrarlas prácticamente durante años, para asegurar su virginidad. Y así las normas más recientes y las pruebas de virginidad exigidas dieron origen de forma automática a las clases de prostitutas profesionales, que eran, en general, novias rechazadas; aquellas mujeres a quienes las madres de los novios comprobaron que no eran vírgenes.
\usection{5. ENDOGAMIA Y EXOGAMIA}
\vs p082 5:1 Muy pronto, el salvaje observó que la mezcla de las razas mejoraba la calidad de la descendencia. No se trataba de que la endogamia fuese siempre mala, sino que la exogamia era siempre relativamente mejor; por ello, las costumbres tendieron a cristalizar la restricción de las relaciones sexuales entre los parientes cercanos. Se consideró que la exogamia aumentaba considerablemente las oportunidades selectivas para la variación y el avance evolutivos. Las personas nacidas de uniones exogámicas eran más versátiles y tenían una mayor capacidad para sobrevivir en un mundo hostil; los endógamos, así como sus costumbres, desaparecían paulatinamente. Fue un proceso lento; los salvajes no razonaban de forma consciente sobre estos problemas. Si bien, con posterioridad, los pueblos avanzados sí lo hicieron, y observaron también la debilidad generalizada que a veces provocaba la excesiva endogamia.
\vs p082 5:2 Aunque una endogamia de buenos linajes produjo a veces la formación de tribus fuertes, los casos espectaculares de los malos resultados observados por la endogamia de personas con deficiencias hereditarias llegó a impactar con mayor fuerza en la mente del hombre, por lo que, al desarrollarse las costumbre, se definieron cada vez más tabúes contra todos los matrimonios entre parientes cercanos.
\vs p082 5:3 \pc La religión ha sido, durante mucho tiempo, una eficiente barrera contra los matrimonios externos; muchas enseñanzas religiosas proscribían el matrimonio fuera de la fe. Habitualmente, la mujer favorecía la práctica de los matrimonios internos, el hombre la de los externos. La propiedad siempre ha tenido influencia sobre el matrimonio y, a veces, en un esfuerzo por conservar la propiedad en el clan, surgieron costumbres que obligaban a las mujeres a elegir a sus maridos dentro de las tribus de sus padres. Las reglas de este tipo llevaron a una gran multiplicación de los matrimonios entre primos. También se practicaba la endogamia en un intento por preservar los secretos de la artesanía; los expertos artesanos buscaban mantener el conocimiento de su arte dentro de la familia.
\vs p082 5:4 \pc Los grupos mejor dotados, cuando estaban aislados, volvían siempre al emparejamiento entre consanguíneos. Los noditas, durante más de ciento cincuenta mil años, fueron uno de los grandes grupos que practicaban los matrimonios internos. Las costumbres posteriores sobre este tipo de matrimonio se vieron enormemente influenciadas por las tradiciones de la raza violeta, en las cuales, al principio, los emparejamientos se hacían forzosamente entre hermano y hermana. Este tipo de matrimonio era frecuente en los antiguos Egipto, Siria y Mesopotamia y por todas las tierras una vez ocupadas por los anditas. Durante mucho tiempo, los egipcios acudieron a los matrimonios entre hermanos en un intento por mantener la pureza de la sangre real, una costumbre que perduró más tiempo aún en Persia. Entre los mesopotámicos, antes de los días de Abraham, los matrimonios entre primos eran obligatorios; los primos tenían derecho matrimonial prioritario a casarse entre ellos. Abraham mismo se casó con su media hermana, pero estas uniones dejaron de permitirse, con el paso del tiempo, bajo las costumbres de los judíos.
\vs p082 5:5 El primer paso para alejarse de los matrimonios entre hermanos y hermanas vinieron a raíz de la costumbre de tener más de una esposa, porque la esposa\hyp{}hermana predominaba con arrogancia sobre la otra esposa o esposas. Algunas costumbres tribales prohibían el matrimonio con la viuda de un hermano que hubiese fallecido, pero exigían que el hermano vivo engendrara los hijos de su difunto hermano. No existen instintos biológicos que vayan en contra de los distintos grados de matrimonios internos; tales restricciones son enteramente una cuestión de tabú.
\vs p082 5:6 \pc El matrimonio externo llegó a predominar porque el hombre lo prefería; tener una esposa de fuera le garantizaba una mayor libertad de la familia política. La familiaridad genera desprecio; así pues, a medida que la decisión personal comenzó a dominar en el emparejamiento, elegir a las parejas de fuera de la tribu se convirtió en una costumbre.
\vs p082 5:7 Finalmente, muchas tribus prohibieron el matrimonio dentro del clan; otras limitaron el emparejamiento a ciertas castas. El tabú contra el matrimonio con una mujer del mismo tótem dio impulso a la costumbre de robar mujeres de las tribus vecinas. Más adelante, se regularon los casamientos más en conformidad con el territorio en el que se residía que con el parentesco. Se dieron muchas etapas evolutivas desde el matrimonio interno hasta la práctica moderna del matrimonio externo. Pero, incluso después de establecerse el tabú contra los matrimonios internos para la gente común, se permitió a los jefes y a los reyes casarse con parientes cercanos con el fin de mantener la sangre real reunida y pura. Por lo general, las costumbres han permitido ciertas licencias a los gobernantes en temas sexuales.
\vs p082 5:8 La presencia posterior de los pueblos anditas tuvo mucho que ver con el aumento del deseo de las razas sangik por emparejarse fuera de su propia tribu. Pero este emparejamiento externo no se pudo extender hasta que los grupos vecinos no aprendieron a convivir en una paz relativa.
\vs p082 5:9 Por sí solo, el matrimonio externo promovió la paz; los matrimonios entre las tribus redujeron las hostilidades. El matrimonio externo llevó a la coordinación tribal y a las alianzas militares; llegó a predominar porque proporcionaba mayor fuerza; construyó naciones. El aumento de las relaciones comerciales también favoreció notablemente este tipo de matrimonio; la aventura y la exploración contribuyeron a la prolongación de los límites del emparejamiento y facilitaron en gran medida la fertilización cruzada de las culturas raciales.
\vs p082 5:10 Las incongruencias inexplicables, por otra parte, de las costumbres sobre los matrimonios raciales se deben, en gran parte, a esta tradición de matrimonios externos, en lo que estos conllevaban del robo y compra de esposas de otras tribus, todo lo cual resultó en una combinación de diferentes costumbres tribales. El hecho de que estos tabúes en relación al matrimonio interno eran sociológicos, no biológicos, está bien ilustrado por los tabúes sobre los matrimonios entre parientes, que comprendían muchos grados de relaciones con parientes políticos, en casos en los que no había ningún tipo de consanguineidad.
\usection{6. LAS MEZCLAS RACIALES}
\vs p082 6:1 Hoy ya no existen razas puras en el mundo. Los primeros pueblos evolutivos originarios de color solo tienen dos razas representativas que perduran en el mundo ---los hombres amarillos y los hombres negros---, e incluso estas dos razas están muy mezcladas con los extintos pueblos de color. Aunque la llamada raza blanca desciende predominantemente del ancestral hombre azul, está más o menos mezclada con todas las otras razas, como lo está el hombre rojo de las Américas.
\vs p082 6:2 De las seis razas sangik de color, tres eran primarias y tres secundarias. Aunque las razas primarias ---azul, roja y amarilla--- eran mejor dotadas en muchos aspectos a los tres pueblos secundarios, debe recordarse que estas razas secundarias tenían muchos rasgos deseables, que hubiesen elevado de manera considerable a los pueblos primarios de haber podido absorber racialmente sus mejores estirpes.
\vs p082 6:3 Los prejuicios actuales contra los “media casta”, los “híbridos” y los “mestizos” han surgido porque la mayor parte de los cruces raciales modernos se producen entre los linajes extremadamente poco dotados de las razas en cuestión. También se consigue una progenie poco satisfactoria cuando las estirpes en declive degenerativo de la misma raza contraen matrimonio entre ellos.
\vs p082 6:4 Si las razas actuales de Urantia pudiesen liberarse del infortunio que causan sus estratos más bajos de seres con anomalías, en declive degenerativo, antisociales, con deficiencia mental y marginados, existirían pocas objeciones para llevar a cabo una fusión racial limitada. Y si estas mezclas raciales pudieran realizarse entre los tipos más elevados de las distintas razas, existirían incluso menos objeciones.
\vs p082 6:5 La hibridación de los linajes mejor dotados y disímiles es el secreto de la creación de estirpes nuevas y más vigorosas. Y esto es cierto para las plantas, los animales y las especies humanas. La hibridación aumenta el vigor e incrementa la fertilidad. Las mezclas interraciales de los estratos medios o mejor dotados de diferentes pueblos aumentan notablemente el potencial \bibemph{creativo,} tal como se demuestra en la población actual de los Estados Unidos de América del Norte. Cuando estos emparejamientos se dan entre los estratos inferiores o peor dotados, la creatividad se reduce, tal como es el caso de los pueblos de hoy día del sur de la India.
\vs p082 6:6 La mezcla de las razas contribuye sobremanera a la aparición repentina de características \bibemph{nuevas,} y si esta hibridación resulta de la unión de las estirpes mejor dotadas, entonces, estas nuevas características conllevarán también \bibemph{una mejor} dotación biológica.
\vs p082 6:7 Mientras las razas actuales estén tan sobrecargadas con estirpes mal dotadas y en declive degenerativo, resultaría muy perjudicial efectuar mezclas raciales a gran escala, pero la mayoría de las objeciones a estos experimentos están basadas en prejuicios sociales y culturales más que en consideraciones biológicas. Incluso entre los linajes peor dotados, los híbridos representan con frecuencia una mejora con respecto a sus ancestros. La hibridación resulta en un perfeccionamiento de la especie debido a la función de los \bibemph{genes dominantes}. La mezcla racial aumenta la probabilidad de que un mayor número de rasgos \bibemph{dominantes} deseables estén presentes en el híbrido.
\vs p082 6:8 \pc Durante los últimos cien años ha habido más hibridación racial en Urantia de la que ha tenido lugar en miles de años. Se ha exagerado bastante el peligro de graves desarmonías como resultado del cruzamiento de los linajes humanos. Los principales problemas de los “mestizos” se deben a los prejuicios sociales.
\vs p082 6:9 El experimento Pitcairn, consistente en mezclar la raza blanca y la polinesia produjo buenos resultados, porque los hombres blancos y las mujeres polinesias pertenecían a estirpes raciales relativamente buenas. El cruce entre los tipos más elevados de las razas blanca, roja y amarilla hizo que surgieran enseguida muchas características nuevas y biológicamente eficaces. Estos tres pueblos pertenecen a las razas sangiks primarias. Las mezclas de las razas blanca y negra no son tan deseables en cuanto a sus consecuencias inmediatas, pero tampoco son estos vástagos mulatos tan objetables como el prejuicio social y racial trataba de hacerlos parecer. Físicamente, estos híbridos de blanco y negro son magníficos especímenes humanos, a pesar de su ligera inferioridad en algunos otros respectos.
\vs p082 6:10 \pc Cuando una raza sangik primaria se fusiona con una raza sangik secundaria, esta última mejora significativamente a expensas de la primera. Y en pequeña escala ---durante largos períodos de tiempo---, puede haber muy pocas objeciones serias a tal aportación que conlleva el sacrificio de las razas primarias para el mejoramiento de los grupos secundarios. Desde un punto de vista biológico, los sangiks secundarios eran en ciertos aspectos superiores a las razas primarias.
\vs p082 6:11 En definitiva, el verdadero peligro para la especie humana se encuentra en la multiplicación incontrolada de las estirpes peor dotadas y en declive degenerativo de los distintos pueblos civilizados, en lugar de cualquier supuesto peligro de sus cruces raciales.
\vsetoff
\vs p082 6:12 [Exposición del jefe de los serafines emplazado en Urantia.]
