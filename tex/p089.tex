\upaper{89}{Pecado, sacrificio y expiación}
\author{Brillante estrella vespertina}
\vs p089 0:1 El hombre primitivo se consideraba endeudado con los espíritus y necesitado de redención. Según la percepción de los salvajes, era de justicia que los espíritus hiciesen recaer sobre ellos incluso más mala suerte. Con el paso del tiempo, este concepto evolucionó hacia la doctrina del pecado y de la salvación. Se veía al alma como si llegara al mundo con un castigo ---el pecado original---. Había que rescatarla; se debía hallar un chivo expiatorio. El cazador de cabezas, además de practicar el sistema de culto nacido de la adoración a las calaveras, podía proveerse de un sustituto de su propia vida, una víctima propiciatoria.
\vs p089 0:2 Pronto el salvaje se dejó imbuir por la idea de que los espíritus sentían una suprema satisfacción cuando contemplaban la miseria, el sufrimiento y la humillación del ser humano. Al principio, el hombre tan solo se preocupó por los pecados de comisión, pero después lo angustiaron igualmente los de omisión. Y todo el sistema sacrificial que apareció más tarde surgió a partir de estas dos ideas. Este nuevo ritual tenía que ver con la observancia de las ceremonias propiciatorias de sacrificios. El hombre primitivo creía que había que hacer algo especial para ganarse el favor de los dioses; solo la civilización avanzada reconoce a un Dios de temperamento uniformemente ecuánime y benevolente. La propiciación era un seguro contra la mala suerte inminente más que una inversión en la felicidad futura. Y los rituales de evitación, exorcismo, coacción y propiciación se integraron unos en otros.
\usection{1. TABÚES}
\vs p089 1:1 La observancia del tabú significó el intento del hombre por eludir la mala suerte, por no ofender a los espectros espíritus evitando hacer algo. En un principio, los tabúes no eran religiosos, pero pronto adquirieron la aprobación de los espectros o espíritus, y cuando, consecuentemente, se afianzaron, se convirtieron en legisladores y en creadores de instituciones. El tabú es la fuente de las reglas ceremoniales y el ancestro del autocontrol primitivo. Fue la forma más primitiva de regulación social y, durante mucho tiempo, la única; todavía es un elemento regulador fundamental de la estructura social.
\vs p089 1:2 El respeto que estas prohibiciones imponían en la mente del salvaje equivalía exactamente a su temor de los poderes que supuestamente las imponían. Los tabúes surgieron, primero, como consecuencia de experiencias casuales con la mala suerte; luego, los propusieron los jefes y chamanes ---hombres fetiches que se pensaba estaban dirigidos por algún espectro espíritu, o incluso por un dios---. El temor a la represalia de los espíritus es tan intenso en la mente del hombre primitivo que a veces muere de miedo tras haber violado un tabú, y este dramático episodio refuerza, a su vez, enormemente, el poder del tabú sobre la mente de los supervivientes.
\vs p089 1:3 Entre las primeras prohibiciones, había restricciones sobre el robo de mujeres y de otras posesiones. A medida que la religión empezó a desempeñar una función más importante en la evolución del tabú, el asunto que se proscribía se consideró impuro en un principio y más tarde profano. En los registros de los hebreos, hay una gran abundancia de expresiones sobre cosas limpias e impuras, sagradas y profanas, pero sus creencias en este sentido eran mucho menos complejas y extensas que las de muchos otros pueblos.
\vs p089 1:4 Los siete mandamientos de Dalamatia y Edén, al igual que los diez preceptos de los hebreos, eran tabúes explícitos; todos estaban expresados de la misma forma negativa que la mayoría de las prohibiciones antiguas. Pero estos códigos más nuevos eran verdaderamente libertadores por el hecho de que sustituían miles de tabúes preexistentes. Y, además de esto, dichos mandamientos, más tardíos, prometían indudablemente algo a cambio de la obediencia.
\vs p089 1:5 Los primeros tabúes sobre los alimentos tuvieron su origen en el fetichismo y el totemismo. Para los fenicios, el cerdo era sagrado; para los hindúes, la vaca. El tabú egipcio sobre la carne de cerdo se ha perpetuado en la fe hebraica y en la islámica. Una variante del tabú sobre los alimentos era la idea de que cuando una mujer embarazada pensaba demasiado en algún alimento determinado, entonces el niño, al nacer, sería el reflejo de ese alimento. Dichas viandas serían tabúes para él.
\vs p089 1:6 Pronto, las formas de comer se convirtieron en tabúes, y así se dio origen a las normas de etiquetas antiguas y modernas en la mesa. Los sistemas de casta y los niveles sociales son restos residuales de prohibiciones remotas. Los tabúes resultaron sumamente eficaces para organizar la sociedad, pero también eran terriblemente gravosos; el sistema de proscripción negativa no solo mantenía reglas útiles y constructivas, sino también tabúes obsoletos, caducos e inservibles.
\vs p089 1:7 Sin embargo, ninguna sociedad civilizada podría criticar al hombre primitivo salvo por estos numerosos y generalizados tabúes, que nunca hubiesen perdurado a no ser por el consentimiento y respaldo de la religión primitiva. Muchos de los aspectos fundamentales de la evolución del hombre han tenido un elevado precio, han requerido una gran abundancia de esfuerzo, sacrificio y abnegación, pero estos logros, reflejados en el control de uno mismo, fueron los verdaderos escalones por los que el hombre subió la escalera ascendente de la civilización.
\usection{2. CONCEPTO DE PECADO}
\vs p089 2:1 El miedo al azar y el pavor a la mala suerte llevó de manera literal al hombre a inventar la religión primitiva como un supuesto seguro contra estas calamidades. De la magia y los espectros, la religión evolucionó, a través de los espíritus y fetiches, hasta llegar a los tabúes. Toda tribu primitiva tenía su árbol de fruta prohibida, literalmente era la manzana, pero figurativamente consistía en miles de ramas de las que colgaban pesadamente todo tipo de tabúes. Y el árbol prohibido siempre decía: “No lo harás”.
\vs p089 2:2 A medida que la mente del salvaje evolucionaba hasta que pudo concebir espíritus buenos y malos, y cuando el tabú recibió la solemne aprobación de la religión evolutiva, el escenario estuvo dispuesto para la aparición del nuevo concepto del \bibemph{pecado}. La idea del pecado se generalizó en el mundo antes de que la religión revelada hiciera su entrada. La mente primitiva llegó a considerar la muerte natural como algo lógico solo a través del concepto de pecado. El pecado era la transgresión del tabú y, la muerte, el castigo del pecado.
\vs p089 2:3 El pecado era un acto ritual, no racional; era una acción, no un pensamiento. Y este concepto total del pecado se vio favorecido por las persistentes tradiciones residuales de Dilmún y los días de un pequeño paraíso en la tierra. La tradición de Adán y del jardín de Edén también dio lugar a la ilusión de una “era de oro” de otro tiempo en el amanecer de las razas. Y todo esto confirmó la creencia manifestada después, de que el hombre había sido creado originariamente de una manera especial, que había comenzado su andadura siendo perfecto, y que la transgresión de los tabúes ---el pecado--- lo había llevado a sus últimas y lamentables penalidades.
\vs p089 2:4 La violación habitual de un tabú se volvió vicio; la ley primitiva lo convirtió en delito y, la religión, en pecado. Entre las primeras tribus, contravenir un tabú era una combinación de delito y pecado. Las calamidades de la comunidad se consideraban siempre como un castigo por el pecado de la tribu. Para los que creían que la prosperidad y la rectitud caminaban juntas, la notoria prosperidad de los malvados ocasionaba tanta turbación que fue necesario inventar infiernos para castigar a los transgresores de los tabúes; el número de estos lugares de castigo futuro ha oscilado entre uno y cinco.
\vs p089 2:5 La idea de la confesión y del perdón apareció tempranamente en la religión primitiva. En reuniones públicas, los hombres pedían perdón por los pecados que se proponían cometer la siguiente semana. La confesión era puramente un rito de exoneración, también un anuncio público de deshonra, un ritual en el que se gritaba “¡impuro, impuro!”. Luego seguían todas las formas rituales de la purificación. En su totalidad, los pueblos ancestrales practicaban estas ceremonias sin sentido. Muchas costumbres aparentemente higiénicas de las tribus primitivas eran en gran parte ceremoniales.
\usection{3. RENUNCIAMIENTO Y HUMILLACIÓN}
\vs p089 3:1 El renunciamiento fue el siguiente paso en la evolución religiosa; el ayuno era una práctica común. Pronto se convirtió en costumbre abstenerse de muchas formas de placer físico, en particular de naturaleza sexual. El rito del ayuno estaba profundamente enraizado en muchas religiones antiguas y se trasmitió a prácticamente todos los sistemas teológicos modernos de pensamiento.
\vs p089 3:2 Justo en el momento en que el hombre incivilizado se estaba recuperando de la dispendiosa práctica de quemar y enterrar los bienes con los muertos, justo cuando la estructura económica de las razas estaba empezando a tomar forma, apareció esta nueva doctrina religiosa del renunciamiento, y decenas de miles de almas sinceras empezaron a dejarse cautivar por la pobreza. Los bienes se consideraron como una desventaja espiritual. En los tiempos de Filón y Pablo, esta noción de los peligros espirituales de las posesiones materiales estaba ampliamente difundida y, desde entonces, ha tenido una marcada influencia en la filosofía europea.
\vs p089 3:3 La pobreza era solo una parte del ritual de la mortificación de la carne que, desafortunadamente, quedó incorporado en los escritos y enseñanzas de muchas religiones, principalmente del cristianismo. La penitencia es la forma negativa de este ritual de renunciamiento, con frecuencia insensato. Si bien, todo esto sirvió para enseñar al salvaje el \bibemph{control de sí mismo,} y significó un valioso avance en la evolución social. La abnegación y el autocontrol fueron dos de los logros sociales más importantes provenientes de la temprana religión evolutiva. El autocontrol proporcionó al hombre una nueva filosofía de vida; le enseñó el arte de engrandecer su fracción de vida reduciendo el denominador de las exigencias personales, en lugar de intentar siempre incrementar el numerador de las gratificaciones interesadas.
\vs p089 3:4 Estas antiguas ideas sobre la autodisciplina incluían la flagelación y toda clase de tortura física. Los sacerdotes del sistema de culto dedicado a la madre eran especialmente activos en la enseñanza de las virtudes del sufrimiento físico, dando ellos mismos ejemplo al someterse a la castración. Los hebreos, los hindúes y los budistas eran fervientes devotos de esta doctrina de la humillación física.
\vs p089 3:5 Durante todos los tiempos antiguos, el hombre buscaba de esta manera conseguir un reconocimiento añadido de abnegación en los libros de registro de sus dioses. Cuando se estaba bajo alguna tensión emocional, se acostumbraba a hacer votos de abnegación y tortura de uno mismo. Con el tiempo, estos votos se configuraron como pactos con los dioses y, en ese respecto, representaron un verdadero progreso evolutivo, dado que se suponía que los dioses harían algo concreto a cambio de esa tortura de sí mismo y de la mortificación de la carne. Los votos eran tanto negativos como positivos. En la actualidad, se pueden observar bastante bien estos prometimientos tan lesivos y extremos en determinados grupos de la India.
\vs p089 3:6 \pc Era natural que el sistema de culto surgido en torno al renunciamiento y la humillación hubiese prestado atención a la gratificación sexual. Este cúmulo de creencias y prácticas basado en la continencia se originó como un ritual seguido por los soldados antes de entrar en batalla; más tarde, se convirtió en la práctica de los “santos”. Tal sistema de culto toleraba el matrimonio solo como un mal menor a la fornicación. Muchas de las grandes religiones mundiales se han visto afectadas negativamente por esta ancestral creencia, pero ninguna de forma tan acusada como el cristianismo. El apóstol Pablo era un adepto de este sistema de culto y sus opiniones personales se reflejan en las enseñanzas que introdujo en la teología cristiana: “Bueno le sería al hombre no tocar mujer”. “Quisiera más bien que todos los hombres fuesen como yo”. “Digo, pues, a los solteros y a las viudas, que bueno les fuera quedarse como yo”. Pablo bien sabía que estas enseñanzas no eran parte del evangelio de Jesús, reconocimiento de su parte que queda ilustrado en su afirmación: “Pero esto lo digo más como concesión que como mandamiento”. Pero este sistema de culto llevó a Pablo a menospreciar a las mujeres. Y lo lamentable de todo esto es que sus opiniones personales hayan influido durante tanto tiempo en las enseñanzas de una gran religión mundial. Si los consejos de este maestro y fabricante de tiendas se tomasen literalmente y obedeciesen de forma generalizada, la raza humana llegaría a un fin repentino e ignominioso. Asimismo, la participación de una religión en el antiguo sistema de culto de la continencia conduce directamente a una guerra contra el matrimonio y el hogar, el auténtico pilar de la sociedad y la institución fundamental del progreso humano. Y no es de extrañar que todas estas creencias contribuyeran a la formación de sacerdocios célibes en las múltiples religiones de pueblos diferentes.
\vs p089 3:7 \pc Algún día el hombre deberá aprender a disfrutar de la libertad sin libertinaje, de la alimentación sin glotonería y del placer sin desenfreno. El autocontrol es una mejor política humana para regular el comportamiento que la extremada abnegación. Jesús tampoco enseñó nunca estos puntos de vista tan poco razonables a sus seguidores.
\usection{4. ORÍGENES DEL SACRIFICIO}
\vs p089 4:1 El sacrificio como parte de las devociones religiosas, al igual que otros muchos rituales de adoración, no tuvo un origen sencillo y único. La tendencia de inclinarse ante el poder y de postrarse en adoración reverente ante el misterio se prefiguraba en el servilismo del perro ante su amo. Solo hay un paso entre el impulso a la adoración y el acto de sacrificio. El hombre primitivo medía el valor de su sacrificio por el dolor que experimentaba. Cuando la idea de sacrificio se vinculó primeramente a la ceremonia religiosa, no se contemplaba ninguna ofrenda que no ocasionara dolor. Los primeros sacrificios eran actos tales como arrancarse el cabello, hacerse cortes en el cuerpo, mutilarse, extraerse los dientes y amputarse los dedos. A medida que avanzó la civilización, estas toscas ideas sobre el sacrificio adquirieron un nivel más elevado que incluían la abnegación, el ascetismo, el ayuno, las privaciones y la más tardía doctrina cristiana de santificación a través de la tristeza, el sufrimiento y la mortificación de la carne.
\vs p089 4:2 Pronto en la evolución de la religión, hubo dos concepciones del sacrificio: la idea del sacrificio como obsequio, que entrañaba una actitud de acción de gracias, y la del sacrificio como deuda, que conllevaba la idea de redención. Después, se desarrolló la noción de la substitución.
\vs p089 4:3 Más adelante aún, el hombre supuso que su sacrificio, de la naturaleza que fuese, podía obrar como portador de un mensaje dirigido a los dioses; podía ser como un grato olor para la deidad. Esto llevó al uso del incienso y de otros elementos estéticos en los rituales de sacrificio que desembocaron en fiestas sacrificiales, las cuales, con el tiempo, se volvieron cada vez más elaboradas y ornamentadas.
\vs p089 4:4 \pc A medida que la religión evolucionó, los ritos sacrificiales de conciliación y propiciación sustituyeron a los métodos más antiguos de evitación, apaciguamiento y exorcismo.
\vs p089 4:5 La idea más temprana del sacrificio fue la de un tributo de neutralidad gravado por los espíritus ancestrales; solo más tarde se desarrolló la idea de la expiación. A medida que el hombre se alejaba de la noción del origen evolutivo de la raza, a medida que las tradiciones de los días del príncipe planetario y de la estancia de Adán se tamizaron a través del tiempo, los conceptos del pecado y del pecado original se generalizaron, de modo que el sacrificio por pecados fortuitos y personales se convirtió en la doctrina del sacrificio para la expiación del pecado racial. La expiación del sacrificio era un mecanismo de seguro general que amparaba incluso ante el resentimiento y los celos de un dios desconocido.
\vs p089 4:6 Rodeado por tantos espíritus susceptibles y dioses codiciosos, el hombre primitivo se enfrentaba con tal multitud de deidades acreedoras que se precisaban todos los sacerdotes, ritos y sacrificios de una vida entera para liquidar la deuda espiritual. La doctrina del pecado original, o culpabilidad racial, hacía que cualquier persona comenzara su vida gravemente endeudado con los poderes espirituales.
\vs p089 4:7 \pc A los hombres se les hacen obsequios y sobornos, pero, cuando se hacen a los dioses, se les califica de “dedicados”, “consagrados” o se les llama “sacrificios”. El renunciamiento era la forma negativa de la propiciación; el sacrificio se convirtió en la forma positiva. El acto de propiciación incluía alabanzas, glorificación, halagos e incluso entretenimiento. Y son los restos de estas prácticas positivas del viejo sistema de culto de propiciación los que constituyen las manifestaciones modernas de la adoración divina; las actuales son simplemente la ritualización de estos antiguos métodos sacrificiales de propiciación positiva.
\vs p089 4:8 \pc El sacrificio de animales significaba mucho más para el hombre primitivo de lo que podía significar para las razas modernas. Estas incivilizadas personas consideraban a los animales como sus cercanos y verdaderos parientes. Conforme pasaba el tiempo, el hombre se hizo más astuto en sus sacrificios, dejando de ofrecer a sus animales de trabajo. En un principio sacrificaba lo \bibemph{mejor} de todo, incluyendo a sus animales domésticos.
\vs p089 4:9 No fue vana la jactancia de un soberano egipcio cuando afirmó que había sacrificado 113\,433 esclavos, 493\,386 cabezas de ganado, 88 barcos, 2756 imágenes de oro, 331\,702 tarros de miel y aceite, 228\,380 tinajas de vino, 680\,714 gansos, 6\,744\,428 barras de pan y 5\,740\,352 sacos de maíz. Y para poder hacer esto debió haber necesitado recaudar gravosos impuestos de sus esforzados súbditos.
\vs p089 4:10 Con el tiempo, la mera necesidad llevó a estos semisalvajes a comer la parte material de sus sacrificios, una vez que los dioses habían disfrutado del alma de los mismos. Y esta costumbre encontró su justificación con el pretexto de la antigua comida sagrada, un ministerio de comunión según la práctica moderna.
\usection{5. SACRIFICIOS Y CANIBALISMO}
\vs p089 5:1 Las ideas modernas sobre el canibalismo de los primeros tiempos están completamente equivocadas; formaba parte de las costumbres de la sociedad primitiva. Aunque tradicionalmente resulta terrible para la civilización moderna, el canibalismo era parte de la estructura social y religiosa de tal sociedad. Los intereses del grupo imponían la práctica del canibalismo. Se desarrolló por impulso de la necesidad y perduró por la esclavitud a la superstición y a la ignorancia. Era una costumbre social, económica, religiosa y militar.
\vs p089 5:2 El hombre primitivo era un caníbal; disfrutaba de la carne humana, y, por lo tanto, la entregaba como ofrenda alimenticia a los espíritus y a sus dioses primitivos. Puesto que los espíritus espectrales no eran sino hombres modificados y, puesto que la comida era la mayor necesidad del hombre, debía ser igualmente esta la mayor necesidad de los espíritus.
\vs p089 5:3 El canibalismo fue en cierta época prácticamente universal entre las razas en evolución. Los sangiks eran todos caníbales, pero, en un primer momento, los andonitas no lo eran, como tampoco los noditas ni los adanitas. Los anditas no lo fueron hasta que llegaron a mezclarse profusamente con las razas evolutivas.
\vs p089 5:4 El gusto por la carne humana crece. Habiendo comenzado por el hambre, la amistad, la venganza o el ritual religioso, el comer carne humana llega a establecerse como canibalismo. La antropofagia se produjo por la escasez de alimento, aunque esta fue, raras veces, la razón fundamental. Los esquimales y los andonitas primitivos, sin embargo, fueron pocas veces caníbales excepto en tiempos de hambrunas. El hombre rojo, especialmente en América Central, era caníbal. En cierto momento, era una práctica generalizada que las madres primitivas mataran y se comieran a sus propios hijos para recobrar la fuerza perdida en el parto, y, en Queensland, todavía es frecuente matar y devorar al hijo primogénito. En tiempos recientes, muchas tribus africanas han recurrido deliberadamente al canibalismo, como medida de guerra, para aterrorizar a sus vecinos con tal atrocidad.
\vs p089 5:5 Se produjo canibalismo como resultado de la degradación de los que fueron linajes superiormente dotados, pero este mayormente prevaleció entre las razas evolutivas. La antropofagia apareció en un momento en el que los hombres experimentaban sensaciones intensas y agrias hacia sus enemigos. Comer carne humana se volvió parte de una solemne ceremonia de venganza; se creía así que se podía acabar con el espectro del enemigo o fusionarlo con el de quien la consumía. En otro tiempo, estaba bastante generalizada la idea de que los magos conseguían sus poderes ingiriendo carne humana.
\vs p089 5:6 Determinados grupos de devoradores de hombres consumían solamente a los miembros de sus propias tribus, en una especie de endogamia pseudoespiritual que se suponía acentuaba la solidaridad tribal. Pero también se comían a los enemigos para vengarse, con la idea de apropiarse de su fuerza. Se consideraba un honor para el alma de un amigo o de un compañero de la tribu que se comiese su cuerpo, mientras que devorar a un enemigo no era sino un castigo justo. La mente del salvaje no tenía la pretensión de ser coherente.
\vs p089 5:7 En algunas tribus, los padres ancianos trataban de que sus hijos se los comiesen; en otras, era costumbre abstenerse de comer a los parientes cercanos; sus cuerpos se vendían o intercambiaban por los de extraños. Existió un importante comercio de mujeres y niños que se hacían engordar para su matanza. Cuando ni la enfermedad ni la guerra lograban regular el crecimiento de la población, el excedente se devoraba sin contemplaciones.
\vs p089 5:8 \pc El canibalismo ha ido desapareciendo paulatinamente debido a los siguientes factores:
\vs p089 5:9 \li{1.}En ocasiones se convirtió en una ceremonia comunal y en la asunción de responsabilidad colectiva por infligir la pena de muerte a un compañero de tribu. La culpa de sangre deja de ser un crimen cuando todos ---la sociedad--- participan en ella. El último acto de canibalismo en Asia fue comerse a los criminales ejecutados.
\vs p089 5:10 \li{2.}Muy pronto se volvió un ritual religioso, pero el aumento del miedo a los espectros no siempre operó para reducir la antropofagia.
\vs p089 5:11 \li{3.}Con el tiempo avanzó hasta el punto en el que solo se comían ciertas partes u órganos del cuerpo, esas partes que se suponía contenían el alma o porciones del espíritu. El beber la sangre pasó a ser común, y era costumbre mezclar las porciones “comestibles” del cuerpo con medicamentos.
\vs p089 5:12 \li{4.}Se limitó a los hombres; a las mujeres se les prohibió ingerir carne humana.
\vs p089 5:13 \li{5.}Luego, se limitó a los jefes, sacerdotes y chamanes.
\vs p089 5:14 \li{6.}Después se convirtió en tabú entre las tribus mejor dotadas. El tabú de ingerir carne humana tuvo su origen en Dalamatia y se propagó lentamente por todo el mundo. Los noditas instaban a la cremación como medio de combatir el canibalismo, dado que en otro tiempo era práctica habitual desenterrar a los cuerpos sepultados para comérselos.
\vs p089 5:15 \li{7.}El sacrificio humano supuso el principio del fin del canibalismo. Al haberse convertido la carne humana en el alimento de hombres de rango superior, de los jefes, esta acabó por reservarse para los espíritus de aún mayor rango; y, de este modo, las ofrendas de sacrificios humanos pusieron fin eficazmente al canibalismo, excepto entre las tribus peor dotadas. Cuando el sacrificio humano se estableció por completo, la antropofagia se convirtió en tabú; la carne humana se destinó únicamente como alimento de los dioses; el hombre solo podía comer en ocasiones ceremoniales un trozo pequeño, un sacramento.
\vs p089 5:16 \pc Finalmente, para fines sacrificiales, se generalizó el uso de sustitutos e, incluso entre las tribus más atrasadas, comer perros redujo notablemente la antropofagia. El perro fue el primer animal que se domesticó y se tenía en gran estima como animal doméstico y como alimento.
\usection{6. EVOLUCIÓN DEL SACRIFICIO HUMANO}
\vs p089 6:1 El sacrificio humano fue una consecuencia indirecta del canibalismo al igual que su cura. El proporcionar escoltas espirituales al mundo de los espíritus llevó también a la reducción de la antropofagia, dado que nunca se tuvo la costumbre de comer la carne de los muertos por sacrificios. En algún momento de su historia, ninguna raza ha estado totalmente libre de alguna forma de sacrificio humano, aunque los andonitas, los noditas y los adanitas fueron los menos predispuestos al canibalismo.
\vs p089 6:2 El sacrifico humano ha sido prácticamente generalizado; perduró en las costumbres religiosas de los chinos, hindúes, egipcios, hebreos, mesopotámicos, griegos, romanos y muchos otros pueblos, incluso persiste en tiempos recientes entre las tribus atrasadas de África y Australia. La civilización de los posteriores indios americanos surgió del canibalismo y, por lo tanto, abundaba en sacrificios humanos, particularmente en Sudamérica y Centroamérica. Los caldeos estaban entre los primeros que abandonaron los sacrificios humanos para ocasiones ordinarias, sustituyéndolos por animales. Alrededor de dos mil años atrás, un emperador japonés de buen corazón introdujo imágenes de arcilla para que tomaran el lugar de los sacrificios humanos; si bien, estos sacrificios no desaparecieron en el norte de Europa hasta hace menos de mil años. En algunas tribus atrasadas, el sacrificio humano se sigue llevando a cabo por voluntarios, como algún tipo de suicidio religioso o ritual. Una vez un chamán ordenó el sacrificio de un anciano muy respetado de cierta tribu. El pueblo se sublevó y se negaron a obedecer. Ante lo cual, el anciano hizo que su propio hijo lo matara; los antiguos creían verdaderamente en esta costumbre.
\vs p089 6:3 \pc No hay constancia de un suceso más trágico y lamentable que ilustre el desgarrador enfrentamiento entre las antiguas costumbres religiosas de larga tradición y las exigencias contrarias de la civilización en avance, que el relato hebreo sobre Jefté y su única hija. Como era habitual, este hombre, bien intencionado, había hecho una insensata promesa: había negociado con el “dios de las batallas”, conviniendo en pagar un precio determinado por la victoria sobre sus enemigos. Y este precio consistía en sacrificar a quien primeramente saliese de su casa para recibirlo cuando él volviese. Jefté pensó, por consiguiente, que sería uno de sus fieles esclavos el que se acercaría para saludarle, pero resultó ser su hija, la única que tenía. Y, así pues, incluso en aquella fecha tan tardía y en un pueblo supuestamente civilizado, se ofreció a esta bella doncella, tras dos meses de lamentaciones por su sino, como sacrificio humano por parte de su padre, con la aprobación de sus compañeros de tribu. Y todo ello se llevó a cabo a pesar de las estrictas estipulaciones de Moisés contra las ofrendas de sacrificios humanos. Pero los hombres y las mujeres son fuertemente propensos a hacer promesas necias e innecesarias; promesas que los hombres de la antigüedad asumían como altamente sagradas.
\vs p089 6:4 \pc En los tiempos antiguos, cuando se comenzaba a construir un edificio de cierta importancia, era costumbre matar a un ser humano como “sacrificio al echar los cimientos”. Esto hacía que algún espíritu espectro vigilara y protegiera la construcción. Cuando los chinos se preparaban para fundir una campana, las costumbres disponían que se sacrificase al menos a una doncella con el fin de mejorar el tono de la campana; la elegida se arrojaba viva al metal derretido.
\vs p089 6:5 Durante mucho tiempo, la práctica seguida por numerosos grupos consistía en emparedar a esclavos vivos en importantes murallas. En tiempos posteriores, las tribus del norte de Europa sustituyeron esta costumbre de sepultar a personas vivas en los muros de los nuevos edificios por la de emparedar la sombra de algún transeúnte. Los chinos enterraban en las murallas a los albañiles que morían al construirlas.
\vs p089 6:6 Un reyezuelo de Palestina, al construir los muros de Jericó, “echó el cimiento sobre Abiram, su primogénito, y puso las puertas sobre su hijo menor, Segub”. En esa fecha tan tardía, este padre no solo puso a sus dos hijos vivos en los huecos de los cimientos de las puertas de la ciudad, sino que su acto quedó registrado como “conforme a la palabra del Señor”. Moisés había prohibido tales sacrificios, pero, poco después de su muerte, los israelitas los retomaron. La ceremonia del siglo XX consistente en depositar baratijas y recuerdos en la base de un nuevo edificio es una reminiscencia de estos sacrificios primitivos que se realizaban al echar los cimientos.
\vs p089 6:7 \pc Durante mucho tiempo, en muchos pueblos existía la costumbre de dedicar los primeros frutos a los espíritus. Y estas prácticas, ahora más o menos simbólicas, son restos de ceremonias primitivas que implicaban la realización de sacrificios humanos. La idea de ofrecer al primogénito como sacrificio estaba extendida entre los antiguos, particularmente entre los fenicios que fueron los últimos en desistir de ella. Se solía decir en el momento del sacrificio, “vida por vida”. Ahora vosotros decís ante la muerte, “polvo al polvo”.
\vs p089 6:8 La visión de Abraham obligado a sacrificar a su hijo Isaac, aunque resulte chocante para la susceptibilidad del mundo civilizado, no era una idea nueva ni extraña para los hombres de aquellos días. Hacía tiempo que era una práctica frecuente que los padres, en momentos de gran tensión emocional, sacrificaran a sus hijos primogénitos. Numerosos pueblos tienen tradiciones similares, porque alguna vez existió la creencia, profunda y generalizada, de que era necesario ofrecer un sacrificio humano cuando sucedía algo extraordinario o poco común.
\usection{7. MODIFICACIONES DEL SACRIFICIO HUMANO}
\vs p089 7:1 Moisés intentó poner fin a los sacrificios humanos introduciendo el rescate como sustituto. Estableció un plan sistemático que permitía a su pueblo escapar de los peores efectos de sus apresuradas e insensatas promesas. Se podían redimir tierras, propiedades e hijos de acuerdo a unas tasas establecidas, que se pagaban a los sacerdotes. Pronto, los grupos que dejaron de sacrificar a sus hijos primogénitos tuvieron una gran ventaja respecto a sus vecinos menos adelantados que continuaron con actos atroces. Muchas de estas tribus atrasadas no solo se debilitaron grandemente por la pérdida de los hijos, sino que, a menudo, incluso se rompía la línea sucesoria de líderes.
\vs p089 7:2 Una consecuencia de la desaparición del sacrificio de los hijos fue la costumbre de manchar el dintel de la casa con sangre para proteger a los hijos primogénitos. Esto se hacía frecuentemente en relación a una de las fiestas sagradas del año, y esta ceremonia en otro tiempo prevaleció en la mayor parte del mundo, desde México hasta Egipto.
\vs p089 7:3 Incluso después de que la mayoría de los grupos habían puesto fin a la matanza ritual de niños, era costumbre abandonar a un niño en tierra inhóspita o en un pequeño bote en el agua. Si el niño sobrevivía, se pensaba que los dioses habían intervenido para protegerlo, como en las tradiciones de Sargón, Moisés, Ciro y Rómulo. Luego vino la práctica de consagrar a los hijos primogénitos como sagrados o propiciatorios, a los que se les permitía crecer y luego exiliarse en lugar de matarlos; este fue el origen de la colonización. Los romanos se acogieron a esta costumbre en sus proyectos de expansión.
\vs p089 7:4 \pc Muchas de las peculiares relaciones entre la laxitud sexual y el primitivo culto de adoración se originaron en conexión con el sacrificio humano. En tiempos antiguos, si una mujer se encontraba con cazadores de cabezas, podía redimir su vida entregándose sexualmente a ellos. Luego, una doncella consagrada sacrificialmente a los dioses podía optar por redimirse dedicando su cuerpo de por vida al servicio sexual sagrado del templo; por este medio, podría ganar el dinero de su redención. Los antiguos consideraban como sublime mantener relaciones sexuales con una mujer comprometida de esta forma con el rescate de su vida. Era un acto religioso confraternizar con estas doncellas sagradas y, además, todo este ritual les brindaba una justificación aceptable para su común gratificación sexual. Era una sutil especie de autoengaño que tanto a las doncellas como a sus consortes les complacía practicar entre ellos. Las costumbres siempre van rezagadas respecto al avance evolutivo de la civilización, autorizando, pues, las prácticas sexuales más tempranas e incivilizadas de las razas en evolución.
\vs p089 7:5 Con el tiempo, la prostitución en los templos se extendió por toda Europa del sur y Asia. Entre los distintos pueblos, el dinero que las prostitutas del templo ganaban se consideraba sagrado ---un don excelso destinado a los dioses---. Las mujeres de más elevado orden se prodigaban en los mercados sexuales del templo, dedicando sus ingresos a todo tipo de servicio sagrado y de obras de bien público. Muchas mujeres de las mejores clases reunían su dote sirviendo sexualmente en el templo con carácter temporal, y la mayoría de los hombres preferían tener a estas mujeres como esposas.
\usection{8. REDENCIÓN Y PACTOS}
\vs p089 8:1 La redención sacrificial y la prostitución en el templo eran en realidad modificaciones del sacrificio humano. Luego vino el sacrificio simulado de las hijas. Esta ceremonia consistía en un derramamiento de sangre, acompañado de la dedicación a la virginidad de por vida, y se trató de una reacción de índole moral a la antigua prostitución en el templo. En tiempos más recientes, las vírgenes se dedicaron al cuidado de los fuegos sagrados de los templos.
\vs p089 8:2 Con el paso del tiempo, los hombres pensaron que la ofrenda de alguna parte del cuerpo podía reemplazar al antiguo y total sacrificio humano. La mutilación física también se consideró como un sustituto aceptable. Se sacrificaba el cabello, las uñas, la sangre e incluso los dedos de las manos y de los pies. Posteriormente, el ancestral rito, prácticamente generalizado, de la circuncisión fue una extensión del sistema de culto del sacrificio parcial; era de orden puramente sacrificial; no se le adscribía ninguna razón higiénica. A los hombres se les circuncidaban y a las mujeres se les perforaban las orejas.
\vs p089 8:3 Seguidamente, se convirtió en costumbre atarse los dedos en lugar de amputárselos. Afeitarse la cabeza y cortarse el cabello eran igualmente formas de devoción religiosa. El hacerse eunuco fue en un principio una modificación de la idea del sacrificio humano. En África se sigue practicando la perforación de la nariz y de los labios, y el tatuaje es un desarrollo artístico de las más primitivas y rudas cicatrices por heridas que se infligían en el cuerpo.
\vs p089 8:4 \pc La costumbre del sacrificio acabó por asociarse, como resultado de enseñanzas más avanzadas, con la idea del pacto. Por fin, se llegó a creer que los dioses alcanzaban verdaderos acuerdos con el hombre; y esto fue un paso importante en la estabilización de la religión. La ley, un pacto, reemplazó a la suerte, al miedo y a la superstición.
\vs p089 8:5 El hombre jamás hubiese ni siquiera soñado con concertar un contrato con la Deidad hasta que su concepto de Dios no había progresado a un nivel en el que percibiese a los regidores del universo como dignos de confianza. Y la temprana idea del hombre sobre Dios era tan antropomórfica que era incapaz de concebir una Deidad en la que se pudiese confiar hasta que él mismo no se volviese relativamente confiable, moral y ético.
\vs p089 8:6 Pero la idea de hacer una alianza con los dioses finalmente había llegado. \bibemph{El hombre evolutivo acabó por adquirir tal dignidad moral como para atreverse a pactar con sus dioses}. Y, así, el acto de ofrecer sacrificios se convirtió paulatinamente en el juego del convenio filosófico del hombre con Dios. Y todo esto significó un nuevo recurso para asegurarse contra la mala suerte o, más bien, un método mejorado de adquirir una indudable prosperidad. No alberguéis la errada idea de que estos sacrificios primitivos eran un don gratuito a los dioses, una ofrenda espontánea de gratitud o de acción de gracias; no eran manifestaciones de auténtica adoración.
\vs p089 8:7 \pc Las primeras formas de oración no eran ni más ni menos que un regateo con los espíritus, una argumentación con los dioses. Era un tipo de trueque en el que la súplica y la persuasión se sustituyeron por algo más tangible y costoso. El desarrollo del comercio entre las razas había inculcado un espíritu mercantilista y había desarrollado la astucia en los trueques; y, entonces, estas conductas comenzaron a aparecer en los métodos de adoración del hombre. Y, así, al igual que algunos hombres eran mejores comerciantes que otros, también se consideraba que habían quien oraba mejor que los demás. La oración del hombre justo era altamente apreciada. Un hombre justo era quien había saldado todas sus cuentas con los espíritus, quien había cumplido satisfactoriamente todas sus obligaciones rituales hacia los dioses.
\vs p089 8:8 La oración primitiva difícilmente se puede considerar adoración; era un acto de negociación para pedir salud, riqueza y vida. Y, en muchos sentidos, las oraciones no han cambiado demasiado con el paso del tiempo. Aún se leen en voz alta en los libros, se recitan solemnemente y se las coloca escritas en las ruedas o se cuelgan de los árboles, allí donde los vientos al soplar eximen al hombre de la molestia de gastar su propio aliento.
\usection{9. SACRIFICIOS Y SACRAMENTOS}
\vs p089 9:1 El sacrificio humano, en el transcurso de la evolución de los rituales urantianos, ha avanzado de ser un cruento acto de antropofagia hasta estratos superiores y más simbólicos. A partir de los primeros ritos sacrificiales, se generaron las posteriores ceremonias sacramentales. En tiempos más recientes, solo era el sacerdote el que ingería una pequeña porción del sacrificio caníbal o bebía una gota de sangre humana, y luego los demás consumían el animal sustitutorio. Estas ideas primitivas de rescate, redención y pactos evolucionaron hasta convertirse en los servicios sacramentales de los últimos días. Y toda esta evolución ceremonial ha ejercido un poderoso influjo socializador.
\vs p089 9:2 Con relación al sistema de culto de la Madre de Dios, en México y en otros lugares, llegó a establecerse un ceremonial sacramental de tortas y vino en lugar de la carne y la sangre de los más antiguos sacrificios humanos. Los hebreos practicaron durante mucho tiempo este ritual como parte de sus celebraciones de Pascua, y fue dicho ceremonial el que daría origen a la posterior versión cristiana del sacramento.
\vs p089 9:3 Las antiguas hermandades sociales se basaban en el rito de la ingestión de sangre; la temprana fraternidad judía fue cuestión de un sacrificio cruento. Pablo comenzó a establecer un nuevo sistema de culto cristiano sobre “la sangre del pacto eterno”. Y aunque puede que sobrecargó innecesariamente al cristianismo con enseñanzas sobre la sangre y el sacrificio, puso fin de una vez y por todas a las doctrinas de la redención mediante sacrificios humanos o animales. Su argumentación teológica muestra que incluso la revelación debe someterse al control escalonado de la evolución. De acuerdo con Pablo, Cristo fue el sacrificio humano último y todo suficiente; el Juez divino está ahora satisfecho de forma plena y para siempre.
\vs p089 9:4 Y así, tras largas eras, el sistema de culto del sacrificio se ha transformado en el sistema de culto del sacramento. En consecuencia, los ceremoniales sacramentales de las religiones modernas son los legítimos sucesores de aquellas tempranas e impactantes ceremonias de sacrificios humanos y de los rituales caníbales, aún más primitivos. Muchos todavía dependen de la sangre para la búsqueda de la salvación, pero, por lo menos, esta se ha vuelto figurativa, simbólica y mística.
\usection{10. EL PERDÓN DE LOS PECADOS}
\vs p089 10:1 El hombre primitivo solo lograba tener conciencia del favor de Dios mediante los sacrificios. El hombre moderno debe desarrollar nuevos métodos de alcanzar la autoconciencia de la salvación. La conciencia del pecado persiste en la mente mortal, pero los patrones de pensamiento respecto a la salvación de este se han vuelto obsoletos y anticuados. Persiste la realidad de la necesidad espiritual, pero el desarrollo intelectual ha acabado con los antiguos medios de garantizar la paz y el consuelo de la mente y del alma.
\vs p089 10:2 \pc \bibemph{El pecado se ha de redefinir como una deslealtad deliberada a la Deidad}. Hay diferentes grados de deslealtad: la lealtad parcial por indecisión, la lealtad dividida por conflicto, la lealtad agonizante por indiferencia y la muerte de la lealtad por devoción a ideales impíos.
\vs p089 10:3 \pc El sentido o el sentimiento de culpa es la conciencia de la violación de las costumbres; esto no es necesariamente pecado. No hay pecado real en ausencia de una deslealtad consciente a la Deidad.
\vs p089 10:4 La posibilidad de reconocer la sensación de culpa es una insignia que distingue a la humanidad y la hace trascender. No señala al hombre como mezquino, sino más bien lo particulariza como criatura de grandeza potencial y de gloria en continuo ascenso. Tal sentimiento de indignidad constituye el estímulo inicial que debería llevar, de forma rápida y segura, hacia aquellas conquistas de la fe que conducen a la mente mortal hacia esos magníficos niveles de nobleza moral, percepción cósmica y vida espiritual; por ello, todos los contenidos de la existencia humana cambian de lo temporal a lo eterno, y todos los valores se elevan de lo humano a lo divino.
\vs p089 10:5 La confesión del pecado es la valiente repulsa de la deslealtad, pero, de ninguna manera, mitiga las consecuencias en el espacio\hyp{}tiempo de tal deslealtad. Si bien, la confesión ---el reconocimiento sincero de la naturaleza del pecado--- es esencial para el crecimiento religioso y el desarrollo espiritual.
\vs p089 10:6 El perdón de los pecados por parte de la Deidad constituye la renovación de la relación de lealtad tras un período en el que el ser humano es consciente de la interrupción de dicha relación como consecuencia de su deliberada rebelión. No se tiene que procurar el perdón, sino tan solo recibirlo, siendo consciente del restablecimiento del nexo de lealtad entre criatura y Creador. Y todos los hijos leales de Dios son felices, sirven con amor y por siempre avanzan de forma ascendente hacia el Paraíso.
\vsetoff
\vs p089 10:7 [Exposición de una brillante estrella vespertina de Nebadón.]
