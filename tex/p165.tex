\upaper{165}{Comienza la misión de Perea}
\author{Comisión de seres intermedios}
\vs p165 0:1 El martes, 3 de enero del año 30 d. C., Abner, el antiguo jefe de los doce apóstoles de Juan el Bautista, un nazareo, que había estado en otro tiempo al frente de la escuela nazarea de En\hyp{}gadi y era ahora jefe de los setenta mensajeros del reino, reunió a sus compañeros y les dio unas instrucciones finales antes de enviarlos en misión a todas las ciudades y aldeas de Perea. Esta misión continuó durante casi tres meses y fue el último ministerio que ejercería el Maestro. Tras este, Jesús fue directamente a Jerusalén para pasar por los postreros momentos de su vida en la carne. Los setenta, asistidos en su labor periódicamente por Jesús y los doce apóstoles, trabajaron en las siguientes ciudades y pueblos y en unas cincuenta aldeas más: Zafón, Gadara, Macad, Arbela, Ramat, Edrei, Bosora, Caspin, Mispé, Gérasa, Ragaba, Sucot, Amatus, Adam, Penuel, Capitolias, Dion, Hatita, Gada, Filadelfia, Jogbeha, Galaad, Bet\hyp{}nimra, Tiro, Eleale, Livias, Hesbón, Callirhoe, Bet\hyp{}peor, Sitim, Sibma, Medeba, Bet\hyp{}meón, Areópolis y Aroer.
\vs p165 0:2 Durante todo este viaje por Perea, el colectivo de mujeres, integrado en ese momento por sesenta y dos miembros, asumió la mayor parte del cuidado de los enfermos. El desarrollo de los más elevados aspectos espirituales del evangelio del reino entraba en su último período y no se obró pues ningún milagro. En ninguna otra parte de Palestina, habían trabajado los apóstoles y los discípulos de Jesús con tanta intensidad, y en ninguna otra región las mejores clases de ciudadanos aceptaron de forma tan generalizada las enseñanzas del Maestro.
\vs p165 0:3 En esa época, la población de Perea era gentil y judía casi en partes iguales, ya que de forma generalizada los judíos habían salido de estas tierras durante los tiempos de Judas Macabeo. Perea era la provincia más bella y pintoresca de toda Palestina. Los judíos acostumbraban a referirse a ella como “la tierra al otro lado del Jordán”.
\vs p165 0:4 A lo largo de todo este período, Jesús repartió su tiempo entre el campamento de Pella y las salidas con los doce para prestar su asistencia a los setenta en las distintas ciudades en la que estos enseñaban y predicaban. Siguiendo las instrucciones de Abner, los setenta bautizaban a los creyentes, a pesar de que Jesús no les había dicho que lo hicieran.
\usection{1. EN EL CAMPAMENTO DE PELLA}
\vs p165 1:1 Hacia mediados de enero, se habían congregado más de mil doscientas personas en Pella, y Jesús, cuando se encontraba en el campamento, les solía enseñar al menos una vez al día, normalmente a las nueve de la mañana y si la lluvia no se lo impedía. Pedro y los demás apóstoles lo hacían todas las tardes. Jesús reservaba las horas del anochecer para las acostumbradas sesiones de preguntas y respuestas con los doce y otros discípulos avanzados. Como término medio, estos grupos de la noche ascendían a cincuenta personas.
\vs p165 1:2 A mediados del mes de marzo, momento en el que Jesús comenzó su viaje hacia Jerusalén, una nutrida multitud de más de cuatro mil personas asistía cada mañana a oír sus predicaciones o a las de Pedro. El Maestro escogió acabar su labor en la tierra cuando el interés por su mensaje había alcanzado un punto culminante, el más álgido de esta segunda etapa, no milagrosa, del progreso del reino. Aunque las tres cuartas partes de esta multitud eran buscadores de la verdad, había también un gran número de fariseos de Jerusalén y de otras partes, junto con muchos incrédulos y críticos.
\vs p165 1:3 Jesús y los doce apóstoles dedicaron gran parte de su tiempo a la multitud que se había congregado en el campamento de Pella. Los doce prestaron poca o ninguna atención al trabajo realizado fuera de esta sede; solo salían con Jesús ocasionalmente para visitar a los compañeros de Abner. Este conocía bastante bien la región de Perea porque, en ella, su anterior maestro, Juan el Bautista, había realizado gran parte de su labor. Una vez comenzada la misión de Perea, Abner y los setenta no regresarían nunca más al campamento de Pella.
\usection{2. EL SERMÓN DEL BUEN PASTOR}
\vs p165 2:1 Un grupo de más de trescientas jerosolimitanos, fariseos y otros había seguido a Jesús cuando se dirigía al norte, a Pella, tras salir precipitadamente y alejarse de la jurisdicción de los dirigentes judíos al término de la fiesta de la Dedicación; y Jesús, en presencia de estos maestros y líderes judíos al igual que ante los doce apóstoles, predicó el sermón de “El buen pastor”. Después de media hora de charla informal, dirigiéndose a unas cien personas, Jesús dijo:
\vs p165 2:2 \pc “Esta noche tengo muchas cosas que deciros y, puesto que muchos de vosotros sois mis discípulos y otros mis implacables enemigos, expondré mis enseñanzas en una parábola para que cada cual pueda tomar para sí lo que su corazón le dicte.
\vs p165 2:3 “Esta noche, ante mí, hay hombres que estarían dispuestos a morir por mí y por este evangelio del reino, y algunos de ellos lo harán en los próximos años; y aquí hay también algunos, esclavos de la tradición, que me habéis seguido desde Jerusalén, y que, con vuestros sombríos y falaces líderes, queréis matar al Hijo del Hombre. La vida que vivo ahora en la carne os juzgará a ambos, a los buenos y a los falsos pastores. Si el falso pastor fuera ciego, no tendría pecado, pero vosotros decís que veis; profesáis ser maestros de Israel; por ello, vuestro pecado permanece.
\vs p165 2:4 “El buen pastor reúne a su rebaño en el redil por la noche en tiempos de peligro. Y cuando llega la mañana, entra en el redil por la puerta, y cuando llama, las ovejas conocen su voz. El pastor que no entra por la puerta en el redil de las ovejas, sino por otra parte, ese es ladrón y salteador. Al buen pastor, el portero le abre la puerta del redil, y las ovejas salen porque conocen su voz. Y cuando ha sacado fuera todas las propias, va delante de ellas; y marca el camino y las ovejas lo siguen porque conocen su voz. Pero al extraño no seguirán, sino que huirán de él, porque no conocen su voz. Esta multitud aquí congregada a nuestro alrededor es como el rebaño sin pastor, pero, cuando les hablamos, conocen la voz del pastor, y nos siguen; al menos quienes tienen hambre de verdad y sed de rectitud lo hacen. Algunos entre vosotros no sois de mi redil; no conocéis mi voz y no me seguís. Y porque sois falsos pastores, las ovejas no conocen vuestra voz y no os seguirán”.
\vs p165 2:5 \pc Y cuando Jesús contó esta parábola, nadie le preguntó nada. Tras un momento, empezó a hablar de nuevo y continuó comentando la parábola:
\vs p165 2:6 “Los que queráis ser los pastores subalternos de los rebaños de mi Padre, no solo debéis ser dignos líderes, sino que también habréis de \bibemph{alimentar} al rebaño con buena comida; no seréis verdaderos pastores a menos que llevéis a vuestros rebaños a pastos verdes junto a aguas de reposo.
\vs p165 2:7 “Y ahora pues, no sea que algunos de vosotros comprendáis esta parábola con demasiada facilidad, os digo que yo soy ambas cosas: la puerta del redil del Padre y el buen pastor de los rebaños. Cualquier pastor que quiera entrar en el redil sin mí no lo conseguirá, y las ovejas no oirán su voz. Yo soy la puerta, junto a otros que ejercen su ministerio conmigo. Cualquier alma que entre al camino eterno que yo he creado y decretado será salva y logrará llegar a los pastos eternos del Paraíso.
\vs p165 2:8 “Pero yo también soy el verdadero pastor que está dispuesto a dar su vida por las ovejas. El ladrón no asalta el redil sino para hurtar, matar y destruir, pero yo he venido para que tengáis vida, y para que la tengáis en abundancia. El asalariado ve venir el peligro y huye, y las ovejas se dispersan y mueren; pero el verdadero pastor no huye cuando el lobo viene; protegerá su rebaño y, si es necesario, pone su vida por sus ovejas. De cierto, de cierto os digo, amigos y enemigos, que yo soy el verdadero pastor; conozco a mis ovejas y las mías me conocen a mí. No huiré ante el peligro. Finalizaré este servicio y haré por completo la voluntad de mi Padre, y no abandonaré al rebaño que el Padre ha confiado a mi cuidado.
\vs p165 2:9 “Tengo, además, otras muchas ovejas que no son de este redil, y estas palabras no son solo válidas para este mundo. Estas otras ovejas también oyen y conocen mi voz, y yo he prometido al Padre que reuniré a todas en un mismo rebaño, en una única hermandad de los hijos de Dios. Y entonces todos vosotros conoceréis la voz de un único pastor, del pastor verdadero, y todos reconoceréis la paternidad de Dios.
\vs p165 2:10 “Y sabréis, por tanto, por qué el Padre me ama y ha puesto todos sus rebaños de estos dominios en mis manos para que yo los cuide; es porque el Padre sabe que yo no flaquearé en la salvaguarda del redil, no vacilaré en dar mi vida al servicio de sus muchos rebaños. Pero, prestad atención, que si doy mi vida, la volveré a tomar. Ningún hombre ni ninguna otra criatura me la puede quitar. Tengo el derecho y el poder de dar mi vida y el mismo derecho y poder para tomarla. Vosotros no podéis entender esto, pero esa autoridad recibí de mi Padre, incluso antes de que el mundo fuera”.
\vs p165 2:11 \pc Al oír estas palabras, sus apóstoles quedaron confundidos y sus discípulos se asombraron, mientras que los fariseos de Jerusalén y de sus alrededores salieron en la noche diciendo: “Está fuera de sí o tiene demonio”. Si bien, algunos de los maestros de Jerusalén dijeron: “Habla como quien tiene autoridad; además, ¿quién ha visto alguna vez a alguien con un demonio abrir los ojos de un ciego de nacimiento y hacer todas las cosas maravillosas que este hombre ha hecho?”.
\vs p165 2:12 Al día siguiente, aproximadamente la mitad de estos maestros judíos profesaron su creencia en Jesús, mientras que la otra mitad regresó consternada a Jerusalén y a sus hogares.
\usection{3. EL SERMÓN DEL \bibemph{SABBAT} EN PELLA}
\vs p165 3:1 Hacia finales de enero, eran casi tres mil los asistentes que se congregaron aquel \bibemph{sabbat} por la tarde. El sábado 28 de enero, Jesús predicó un memorable sermón sobre “La confianza y la preparación espiritual”. Tras unos comentarios previos de Simón Pedro, el Maestro dijo:
\vs p165 3:2 \pc “Lo que he dicho repetidas veces a mis apóstoles y a mis discípulos, lo proclamo a esta multitud: guardaos de la levadura de los fariseos, que es la hipocresía, nacida del prejuicio y nutrida en las ataduras a la tradición, aunque muchos de estos fariseos son honestos de corazón y algunos de ellos se han quedado aquí como discípulos míos. En breve, todos entenderéis mis enseñanzas porque nada hay encubierto que no haya de ser descubierto. Todo lo que ahora está oculto de vosotros se os hará saber cuando el Hijo del Hombre haya terminado su misión en la tierra y en la carne.
\vs p165 3:3 “Pronto, muy pronto, lo que nuestros enemigos planean ahora en secreto y en las tinieblas, a la luz se oirá y será proclamado desde las azoteas. Pero yo os digo, amigos míos, cuando intenten matar al Hijo del Hombre, no les tengáis miedo. No temáis a los que, aunque puedan matar el cuerpo, ya no tendrán después ningún poder sobre vosotros. Os insto a que no temáis a nadie, ni en el cielo ni en la tierra, pero gozaos en el conocimiento de Aquel que tiene poder para liberaros de toda injusticia y de presentaros libres de culpa ante el asiento de juicio de un universo.
\vs p165 3:4 “¿No se venden cinco pajarillos por un cuarto? Con todo, cuando estas aves revolotean en busca de alimento, ninguna de ellas existe sin el conocimiento del Padre, la fuente de toda vida. Para los guardianes seráficos aun vuestros cabellos están todos contados. Y si todo esto es verdad, ¿por qué habéis de vivir con temor de las muchas nimiedades que puedan aparecer en vuestras vidas diarias? Así que, no tengáis miedo; más valéis vosotros que muchos pajarillos.
\vs p165 3:5 “A cualquiera que haya tenido la valentía de confesar su fe en mi evangelio delante de los hombres, yo lo reconoceré pronto delante de los ángeles del cielo, pero a cualquiera que niegue deliberadamente la verdad de mis enseñanzas delante de los hombres, el guardián de destino les negará incluso delante de los ángeles del cielo.
\vs p165 3:6 “Cualquier cosa que digáis sobre el Hijo del Hombre, os será perdonado; pero el que ose blasfemar contra Dios raramente será perdonado. Cuando los hombres, intencionadamente, llegan hasta el extremo de atribuir los actos de Dios a las fuerzas del mal, será muy difícil que estos rebeldes conscientes pidan perdón por sus pecados.
\vs p165 3:7 “Y cuando nuestros enemigos os entreguen ante los jefes de las sinagogas y ante otras altas autoridades, no os preocupéis por lo que tenéis que decir y no os angustiéis por cómo habéis de contestar a sus preguntas, porque el espíritu que mora en vosotros ciertamente os enseñará, en esa misma hora, lo que debéis decir en honor del evangelio del reino.
\vs p165 3:8 “¿Cuánto tiempo estaréis en el valle de la indecisión? ¿Por qué vaciláis entre dos opiniones? ¿Por qué deberían tener dudas los judíos o los gentiles en aceptar la buena nueva de que son hijos del Dios eterno? ¿Cuánto tiempo nos llevará convenceros de que aceptéis con gozo vuestra herencia espiritual? Vine a este mundo para revelaros al Padre y llevaros hasta él. Ya he hecho lo primero, pero no puedo hacer lo segundo sin vuestra aprobación; el Padre nunca obliga a nadie a entrar en el reino. La invitación ha sido siempre y siempre será: el que quiera, que venga y que participe libremente del agua de la vida”.
\vs p165 3:9 \pc Cuando Jesús terminó de hablar, muchos se fueron para ser bautizados por los apóstoles en el Jordán, mientras que él escuchaba las preguntas de los que se quedaron.
\usection{4. REPARTICIÓN DE LA HERENCIA}
\vs p165 4:1 Conforme los apóstoles bautizaban a los creyentes, el Maestro hablaba con los que permanecieron allí. Y un joven le dijo: “Maestro, mi padre murió dejándonos muchas propiedades a mí y a mi hermano, pero mi hermano se niega a darme lo que me pertenece. ¿Podrías decirle a mi hermano que parta conmigo la herencia? A Jesús le indignó levemente que aquel joven, de mente materialista, trajera tal cuestión monetaria como objeto de comentario; pero aprovechó la ocasión para impartir nuevas enseñanzas. Jesús le dijo: “Hombre, ¿quién me ha puesto sobre vosotros como partidor? ¿De dónde has sacado la idea de que yo presto atención a los asuntos materiales de este mundo?”. Y, entonces, volviéndose a los que estaban a su alrededor, dijo: “Mirad, guardaos de toda avaricia, porque la vida del hombre no consiste en la abundancia de los bienes que posee. La felicidad no viene del poder de la riqueza ni la alegría surge de los bienes. La riqueza, en sí misma, no es una maldición, pero el amor a la riqueza lleva muchas veces a tal apego a las cosas de este mundo, que el alma queda ciega a los bellos atractivos de las realidades espirituales del reino de Dios en la tierra y a los gozos de la vida eterna en el cielo.
\vs p165 4:2 \pc “Os contaré la historia de cierto rico cuya heredad había producido mucho; y cuando se hizo muy rico, empezó a pensar dentro de sí, diciendo: ‘¿Qué haré con toda mi riqueza? Es tanta la que poseo en este momento que no tengo dónde guardarla’. Y tras meditar sobre este problema, se dijo: ‘Esto haré; derribaré mis graneros, y los edificaré más grandes, y así tendré lugar abundante para guardar mis frutos y mis bienes. Entonces diré luego a mi alma: alma, mucha riqueza tienes guardada para muchos años; ahora descansa, come, bebe y regocíjate, porque eres rica y tus bienes se han incrementado’.
\vs p165 4:3 “Pero este hombre rico era también un necio. Al proveer a las necesidades materiales de su mente y de su cuerpo, no había sabido almacenar tesoros en el cielo para la satisfacción de su espíritu y la salvación de su alma. Y ni incluso llegaría a gozar del placer de gastar la riqueza que había acaparado porque aquella misma noche le fue pedida su alma. Esa noche llegaron unos bandidos que asaltaron su casa y lo mataron, y después de saquear sus graneros, incendiaron lo que quedó. Y la propiedad que se salvó de los ladrones se la disputaron entre ellos sus herederos. Este hombre hizo tesoros para sí en la tierra, pero no era rico para con Dios”.
\vs p165 4:4 \pc Aquella fue la manera en la que Jesús procedió con el joven y su herencia, porque sabía que su problema era la codicia. Si no hubiese sido así, el Maestro no habría intervenido; él nunca se inmiscuía en los asuntos temporales ni incluso de sus apóstoles, y mucho menos de sus discípulos.
\vs p165 4:5 Cuando Jesús terminó de contar su historia, otro hombre se levantó y le preguntó: “Maestro, sé que tus apóstoles han vendido todas sus posesiones terrenales para seguirte y que tienen todas las cosas en común al igual que hacen los esenios, pero ¿es que quieres que todos nosotros, que somos tus discípulos, hagamos lo mismo? ¿Es pecado poseer riqueza obtenida honradamente?”. Y Jesús respondió a esta pregunta: “Amigo mío, no es pecado poseer riqueza honrosa; pero sí lo es si conviertes la riqueza de posesiones materiales en \bibemph{tesoros} que absorban tus intereses y aparten tus deseos de dedicarte a los logros espirituales del reino. No hay ningún pecado en tener posesiones conseguidas con honradez en la tierra, con tal de que tu \bibemph{tesoro} esté en el cielo, porque donde esté tu tesoro, estará también allí tu corazón. Existe una gran diferencia entre la riqueza que lleva a la avaricia y al egoísmo y aquella que se tiene y se distribuye en espíritu de fideicomiso por quienes tienen abundancia de bienes de este mundo, y que tan ampliamente contribuyen al mantenimiento de los que dedican todas sus energías al trabajo del reino. A muchos de los que estáis aquí sin dinero se os alimenta y alberga en la cercana ciudad hecha de tiendas gracias a las aportaciones, que para estos fines hombres y mujeres generosos y de recursos han entregado a vuestro anfitrión, David Zebedeo.
\vs p165 4:6 “Pero, en definitiva, nunca os olvidéis de que la riqueza no perdura. Con excesiva frecuencia, el amor por la riqueza oscurece e incluso destruye la visión espiritual. No dejéis de reconocer el peligro de que la riqueza puede convertirse en dueña de vosotros en lugar de estar a vuestro servicio”.
\vs p165 4:7 \pc Jesús no enseñó ni aprobaba la negligencia, la ociosidad ni la indiferencia en proveer a las necesidades físicas de la familia de uno, ni la dependencia de las limosnas. Pero ciertamente enseñó que las cosas materiales y temporales deben subordinarse al bienestar del alma y al progreso de la naturaleza espiritual en el reino de los cielos.
\vs p165 4:8 \pc Más tarde, mientras la gente bajó al río para presenciar los bautismos, el primer hombre fue a ver a Jesús privadamente para hablarle de su herencia, puesto que pensaba que Jesús lo había tratado con dureza; y cuando el Maestro lo oyó de nuevo, le respondió: “Hijo mío, ¿por qué desaprovechas la oportunidad de alimentarte del pan de la vida en un día como este y complacer así esa disposición tuya a la avaricia? ¿Es que no sabes que las leyes judías sobre la herencia dictarán con justicia si vas con tu reclamación al tribunal de la sinagoga? ¿Es que no ves que con mi labor trato de asegurarme de que conozcas tu herencia celestial? ¿Has leído en la Escritura: ‘Hay quien se hace rico a fuerza de engaño y avaricia, y esta es la parte de su recompensa: cuando dice, ya he logrado reposo, ahora voy a comer de mis bienes, sin embargo no sabe qué tiempo le va a venir, morirá y se lo dejará a otros’? ¿No has leído en los mandamientos: ‘No codiciarás’? Ni ¿‘Han comido hasta saciarse y engordaron, y entonces se volvieron a dioses ajenos’? ¿No has leído en los Salmos que ‘El Señor desprecia al codicioso’ y que ‘Mejor es lo poco del justo que las riquezas de muchos pecadores’ y ‘Si aumentan las riquezas, no pongas el corazón en ellas’? ¿No has leído lo que dice Jeremías: ‘No se alabe el rico en sus riquezas’?; y Ezequiel habló la verdad cuando dijo: ‘Hacen halagos con sus bocas, pero sus corazones están en pos de sus propias y egoístas ganancias’”.
\vs p165 4:9 Jesús despidió al joven, diciéndole: “Hijo mío, ¿de qué le servirá al hombre ganar todo el mundo, si pierde su alma?”.
\vs p165 4:10 A otro que se hallaba cerca y que preguntó a Jesús cómo se consideraría a los ricos en el día del juicio, él le respondió: “No he venido para juzgar ni a ricos ni a pobres, si bien, se juzgará a todos los hombres por las vidas que vivan. Al margen de cualquier otra cosa que ataña a los ricos en el juicio, quienes adquieren grandes riquezas deben responder al menos a tres preguntas, las cuales son:
\vs p165 4:11 “1. ¿Cuánta riqueza has acumulado?
\vs p165 4:12 “2. ¿Cómo has conseguido esta riqueza?
\vs p165 4:13 “3. ¿Cómo has usado tu riqueza?”.
\vs p165 4:14 \pc Luego, Jesús se fue a su tienda para descansar un rato antes de la cena. Cuando los apóstoles terminaron de bautizar, volvieron también y habrían hablado con él sobre la riqueza en la tierra y los tesoros en el cielo, pero él estaba dormido.
\usection{5. CHARLAS CON LOS APÓSTOLES SOBRE LA RIQUEZA}
\vs p165 5:1 Aquella noche, tras la cena, cuando Jesús y los doce se juntaron para tener su encuentro diario, Andrés preguntó: “Maestro, mientras nosotros bautizábamos a los creyentes, tú le dijiste muchas cosas a la multitud que se había quedado contigo, pero que ninguno de nosotros oyó. ¿Te importaría decirnos a nosotros eso mismo para nuestro bien?”. En respuesta a la petición de Andrés, Jesús dijo:
\vs p165 5:2 \pc “Sí, Andrés, os hablaré de estos temas de la riqueza y de vuestro propio sustento, pero las palabras que yo os hablo a vosotros, mis apóstoles, han de ser algo diferentes a las que dirigí a los discípulos y a la multitud, puesto que vosotros lo habéis dejado todo, no solo para seguirme a mí, sino para ser ordenados como embajadores del reino. Ya tenéis una experiencia de años conmigo, y sabéis que el Padre cuyo reino proclamáis no os abandonará. Habéis dedicado vuestras vidas al ministerio del reino; así pues no os angustiéis ni os preocupéis por las cosas de la vida temporal, por lo que comeréis, por vuestro cuerpo ni por vuestro vestido. El bienestar del alma es más que alimento y bebida; el progreso en el espíritu está muy por encima de la necesidad de vestido. Cuando estéis tentados a dudar si tendréis vuestro pan, mirad las aves del cielo, que no siembran ni siegan ni recogen en graneros; y, sin embargo, el Padre celestial da alimento a las que lo que buscan. Y ¿no valéis vosotros mucho más que ellas? Además, nada puede hacer vuestra angustia ni vuestras abrumadoras dudas por proveer a vuestras necesidades materiales. ¿Y quién de vosotros podrá, por mucho que se angustie, añadir a su estatura un codo o un día a vuestra vida? Puesto que estos asuntos no están en vuestras manos, ¿por qué os inquietáis pensando en ellos?
\vs p165 5:3 “Considerad los lirios del campo, cómo crecen: no trabajan ni hilan; pero os digo que ni aun Salomón con toda su gloria se vistió como uno de ellos. Si Dios viste así la hierba del campo, que hoy está viva y mañana se corta y echa al fuego, cuanto más os vestirá él a vosotros, embajadores del reino celestial. ¡Oh, vosotros, hombres de poca fe! Si os dedicáis fervientemente a la proclamación del evangelio del reino, desechad las dudas de vuestras mentes de que no hallaréis sostén para vosotros ni para las familias a las que habéis renunciado. Si realmente entregáis vuestras vidas al evangelio, viviréis mediante el evangelio. Si sois solamente discípulos devotos, debéis ganaros vuestro propio pan y contribuir al mantenimiento de todos los que enseñan y predican y curan. Si estáis preocupados por vuestro pan y vuestra agua, ¿de qué manera sois diferentes de las naciones del mundo que con tanto afán buscan surtir tales necesidades? Dedicaos a vuestra labor, convencidos de que tanto el Padre como yo sabemos que tenéis necesidad de todas ellas. Os aseguro, de una vez y para siempre, que si consagráis vuestras vidas al trabajo del reino, todas vuestras verdaderas necesidades serán abastecidas. Buscad lo más grande y hallareis entonces lo más pequeño; pedid lo celestial y se incluirá lo material. Ciertamente la sombra sigue a la sustancia.
\vs p165 5:4 “Sois solamente un pequeño grupo, pero si tenéis fe, si el temor no os hace tropezar, os declaro que a mi Padre le ha placido daros este reino. Habéis guardado vuestros tesoros en bolsas enceradas que no envejecen, donde los ladrones no hurtan ni la polilla destruye. Como dije a la gente, donde esté vuestro tesoro, allí estará también vuestro corazón.
\vs p165 5:5 “Pero en la labor que tenemos de inmediato por delante, y en la que os queda a vosotros una vez que yo vuelva al Padre, pasaréis por penosas pruebas. Debéis estar vigilantes contra el temor y las dudas. Tened, cada uno de vosotros, vuestras mentes ceñidas y vuestras lámparas encendidas. Tened la actitud de hombres que aguardan atentamente a que su señor regrese de las bodas, para que, cuando llegue y llame, le abráis enseguida. A estos siervos vigilantes su señor los bendecirá cuando venga y los halle fielmente velando en tan gran momento. Entonces, el señor hará que se sienten a la mesa mientras él los sirve. De cierto, de cierto os digo que está a punto de llegar una crisis a vuestras vidas, y es preciso que estéis vigilantes y preparados.
\vs p165 5:6 “Comprendéis bien que ningún hombre permitiría que asaltasen su casa si supiera a qué hora había de llegar el ladrón. Velad también vosotros mismos, porque a la hora que menos sospechéis y de la manera que no penséis, el Hijo del Hombre partirá”.
\vs p165 5:7 \pc Durante unos minutos, los doce se quedaron sentados en silencio. Habían oído antes algunas de estas advertencias, pero no en el contexto en el que Jesús se las expuso en ese momento.
\usection{6. RESPUESTA A LA PREGUNTA DE PEDRO}
\vs p165 6:1 Mientras estaban sentados pensativos, Simón Pedro preguntó: “¿Dices esta parábola a nosotros, tus apóstoles, o a todos los discípulos?”. Y Jesús contestó:
\vs p165 6:2 \pc “Es en los momentos de dificultades cuando se revela el alma del hombre; las pruebas desvelan lo que realmente se lleva en el corazón. Cuando se ha probado al siervo y ha demostrado su valía, el señor lo pondrá a cargo de su casa y, seguro de él, confiará a este leal mayordomo la alimentación y la crianza de sus hijos. De la misma manera, sabré pronto en quién se podrá confiar para el cuidado de mis hijos cuando yo haya regresado al Padre. Al igual que el señor de la casa pondrá los siervos leales y probados a cargo de los asuntos de su familia, así glorificaré yo a aquellos que soporten los difíciles problemas que pronto encontrarán en los asuntos relativos a mi reino.
\vs p165 6:3 “Pero si el siervo indolente dice en su corazón: ‘Mi señor tarda en llegar’, y comienza a maltratar a sus consiervos, y a comer y a beber con los borrachos, vendrá el señor de aquel siervo en la hora que este no espera y, hallándole desleal, lo expulsará en deshonra. Por lo tanto, hacéis bien en prepararos para ese día en el que se os visitará de repente y de forma inesperada. Recordad que se os ha dado mucho y que será mucho lo que se os demande. Os vienen pruebas muy difíciles. De un bautismo me es necesario ser bautizado, y estaré vigilante hasta que se cumpla. Vosotros predicáis paz en la tierra, pero mi misión no traerá paz en los asuntos materiales de los hombres; no por un tiempo, al menos. La división es el único resultado posible cuando dos miembros de una familia creen en mí y tres de ellos rechazan este evangelio. Los amigos, los parientes y los seres queridos están destinados a estar unos contra otros a causa del evangelio que vosotros predicáis. Es verdad que cada uno de estos creyentes tendrá una paz grande y duradera en su corazón, pero no llegará la paz en la tierra hasta que todos no tengan voluntad de creer y de ser parte de la gloriosa herencia de la filiación con Dios. No obstante, salid al mundo y proclamad este evangelio a todas las naciones, a cada hombre, mujer y niño”.
\vs p165 6:4 \pc Y así terminó aquel día de \bibemph{sabbat} pleno y atareado. Al día siguiente, Jesús y los doce fueron a las ciudades del norte de Perea para visitar a los setenta, que trabajaban en esas regiones bajo la supervisión de Abner.
