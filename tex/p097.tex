\upaper{97}{Evolución del concepto de Dios entre los hebreos}
\author{Melquisedec}
\vs p097 0:1 Los líderes espirituales de los hebreos lograron lo que nadie había logrado antes: desantropomorfizar su concepto de Dios, sin convertirlo en una Deidad abstracta únicamente comprensible para los filósofos. Incluso la gente corriente era capaz de considerar a Yahvé, en el concepto que había ido madurando, como Padre, si no de la persona individual, al menos de la raza.
\vs p097 0:2 El concepto del ser personal de Dios, aunque impartido con claridad en Salem en los días de Melquisedec, era vago y confuso en el momento de la huida de Egipto, llegando a evolucionar solo paulatinamente en la mente hebraica de generación en generación como respuesta a las enseñanzas de los líderes espirituales. En cuanto a su gradual evolución, la percepción de la persona de Yahvé fue mucho más continua que la de cualquier otro atributo de la Deidad. Desde Moisés hasta Malaquías hubo en la mente hebrea un crecimiento teórico de Dios como persona, prácticamente sin interrupción y, tal concepto, con el tiempo, se realzó y glorificó gracias a las enseñanzas de Jesús sobre el Padre de los cielos.
\usection{1. SAMUEL: EL PRIMERO DE LOS PROFETAS HEBREOS}
\vs p097 1:1 La presión hostil ejercida por los pueblos que rodeaban a Palestina enseñó pronto a los jeques hebreos que no tenían esperanza de sobrevivir a menos que confederaran sus organizaciones tribales en un gobierno centralizado. Y esta centralización del poder ejecutivo proporcionó a Samuel una mejor oportunidad de actuar como maestro y reformador.
\vs p097 1:2 Samuel surgió de una larga línea de maestros salemitas que habían continuado manteniendo las verdades de Melquisedec como parte de su forma de adoración. Este maestro era un hombre varonil y resuelto. Fue únicamente su gran devoción, junto con su extraordinaria determinación, la que le permitió soportar la oposición casi generalizada a la que tuvo que enfrentarse, cuando quiso que Israel retornara al culto del supremo Yahvé de los tiempos de Moisés. E incluso entonces su triunfo fue parcial; solo logró recuperar para el culto, según esa percepción superior de Yahvé, a la mitad más inteligente de los hebreos; la otra mitad siguió adorando a los dioses tribales del país y a Yahvé, a quien percibían de forma más superficial.
\vs p097 1:3 Samuel era un hombre rudo y dispuesto, un reformador práctico capaz de salir un día con los suyos y derribar una veintena de emplazamientos dedicados a Baal. Los avances que hizo fueron por la fuerza viva de la coacción; predicó poco, enseñó incluso menos, pero sí actuó. Un día se burlaba del sacerdote de Baal; el siguiente, cortaba en pedazos a un rey cautivo. Creía con devoción en el Dios único, y tenía una idea clara de ese Dios como creador del cielo y de la tierra: “Del Señor son las columnas de la tierra, y él afirmó sobre ellas el mundo”.
\vs p097 1:4 Pero la gran aportación de Samuel al desarrollo del concepto de la Deidad fue su clamorosa declaración de que Yahvé era \bibemph{invariable,} que por siempre era la personificación misma de una inequívoca perfección y divinidad. En estos tiempos, Yahvé se concebía como un Dios inestable, caprichosamente celoso, que continuamente se lamentaba de haber hecho esto o aquello; pero ahora, por primera vez desde que los hebreos salieron impetuosamente de Egipto, escucharon estas sorprendentes palabras: “La Fuerza de Israel no mentirá ni se arrepentirá, porque no es hombre para que se arrepienta”. Se proclamaba la estabilidad en la relación con la divinidad. Samuel confirmó el pacto de Melquisedec con Abraham y declaró que el Señor Dios de Israel era la fuente de toda verdad, estabilidad y constancia. Siempre los hebreos habían mirado a su Dios como hombre, suprahombre, excelso espíritu de origen desconocido; pero ahora oían cómo este espíritu del Horeb se reconocía como un Dios invariable de perfección creadora. Samuel estaba contribuyendo a que el concepto evolutivo de Dios escalara cotas por encima del estado cambiante de la mente de los hombres y de las vicisitudes de la existencia humana. Con sus enseñanzas, el Dios de los hebreos empezaba a ascender desde una idea equivalente a la que se tenía de dioses tribales hasta el ideal de un Creador todopoderoso e invariable y \bibemph{Supervisor} de toda la creación.
\vs p097 1:5 Y predicó de nuevo la historia de la honestidad de Dios, su fiabilidad en el cumplimiento de un pacto. Dijo Samuel: “El Señor no desamparará a su pueblo”. “Él ha hecho con nosotros un pacto eterno, bien ordenado en todo y seguro”. Y así, resonó en toda Palestina el llamamiento a volver a la adoración del Yahvé supremo. Este enérgico maestro no se cansaba de proclamar: “¡Por tanto, tú te has engrandecido, oh Señor Dios, por cuanto no hay como tú, ni hay Dios fuera de ti!”.
\vs p097 1:6 \pc Hasta entonces, los hebreos habían considerado el favor de Yahvé principalmente en términos de prosperidad material. En Israel, causó un gran impacto, y casi le cuesta su vida a Samuel, cuando se atrevió a proclamar: “Jehová empobrece y enriquece, abate y enaltece. Él levanta del polvo al pobre; alza del basurero al menesteroso, para hacerlo sentar con príncipes y heredar el trono de la gloria”. Desde los tiempos de Moisés, no se habían anunciado promesas tan reconfortantes para los humildes y menos afortunados y, entre los pobres, miles de personas desesperadas comenzaron a tener la esperanza de que podrían mejorar su estatus espiritual.
\vs p097 1:7 Pero Samuel no avanzó mucho más allá de la noción de un dios tribal. Anunció un Yahvé hacedor de todos los hombres, pero que se ocupó principalmente de los hebreos, su pueblo elegido. Incluso así, como en los días de Moisés, una vez más el concepto de Dios describía una Deidad santa y recta. “No hay nadie santo como el Señor. ¿Quién cómo este santo Señor Dios?”.
\vs p097 1:8 Conforme pasaban los años, el encanecido y viejo líder avanzó en su comprensión de Dios, pues declaró: “El Señor es el Dios que todo lo sabe y a él le toca pesar las acciones. El Señor juzgará los confines de la tierra; se mostrará misericordioso con el misericordioso y recto para con el hombre íntegro”. Incluso aquí se halla el comienzo de la misericordia, aunque limitada a los que son misericordiosos. Más tarde, llegaría a dar un paso más cuando, en la adversidad de su pueblo, les exhortó: “Caigamos ahora en manos del Señor, porque sus misericordias son muchas”. “No hay límites en el Señor para salvar a muchos o a pocos”.
\vs p097 1:9 \pc Y este gradual desarrollo del concepto del carácter de Yahvé continuó bajo el ministerio de los sucesores de Samuel. Trataron de presentar a Yahvé como un Dios que cumplía sus pactos, pero apenas pudieron seguir el paso marcado por Samuel; no consiguieron desarrollar la idea de la misericordia de Dios, tal como Samuel la había concebido en sus últimos años. Se dio una continuada vuelta hacia atrás, hacia el reconocimiento de otros dioses, a pesar de mantener a Yahvé por encima de todos ellos. “Tuyo, oh Señor, es el reino, y tú eres excelso sobre todos”.
\vs p097 1:10 La idea dominante de esta época era el poder divino; los profetas de esta era predicaban una religión destinada a potenciar al rey en su trono hebreo. “Tuya es, oh Señor, la grandeza y el poder y la gloria y la victoria y la majestad. En tu mano está la fuerza y el poder y el dar grandeza y poder a todos”. Y este era el estatus de la idea de Dios durante los tiempos de Samuel y de sus inmediatos sucesores.
\usection{2. ELÍAS Y ELISEO}
\vs p097 2:1 En el siglo X a. C., la nación hebrea se dividió en dos reinos. En ambas divisiones políticas, muchos maestros de la verdad se esforzaron por contener la marea regresiva de decadencia espiritual que se había instalado, y que continuó catastróficamente tras la guerra de separación. Pero este afán por hacer avanzar la religión hebraica no prosperó hasta el momento en que Elías, un decidido y valiente guerrero, comenzó, en nombre de la rectitud, a impartir sus enseñanzas. En el reino del norte, Elías restituyó una noción de Dios equiparable a la que había estado vigente en los días de Samuel. Elías tuvo pocas oportunidades de elaborar un concepto avanzado de Dios; se mantenía ocupado, tal como lo había estado Samuel, derribando los altares de Baal y demoliendo los ídolos de los falsos dioses. Y llevó adelante sus reformas a pesar de la oposición de un monarca idólatra; su labor fue aún más gigantesca y difícil que aquella a la que Samuel se había enfrentado.
\vs p097 2:2 Cuando Elías fue arrebatado, Eliseo, su fiel compañero, asumió su tarea y, con la inestimable ayuda del poco conocido Micaías, mantuvo la luz de la verdad viva en Palestina.
\vs p097 2:3 Pero estos no eran tiempos para avanzar en la conceptualización de la Deidad. Los hebreos ni siquiera habían ascendido al ideal mosaico. La era de Elías y de Eliseo concluyó al tiempo que las clases acomodadas volvían a la adoración del Yahvé supremo y se presenció el restablecimiento de la idea del Creador Universal, en el punto aproximado en el que Samuel la había dejado.
\usection{3. YAHVÉ Y BAAL}
\vs p097 3:1 La interminable controversia que existía entre los creyentes en Yahvé y los seguidores de Baal fue más un enfrentamiento socioeconómico de ideologías que una diferencia de creencias religiosas.
\vs p097 3:2 \pc Los habitantes de Palestina diferían en su actitud hacia la propiedad privada de la tierra. Las tribus árabes del sur o tribus itinerantes (los yahvistas) consideraban la tierra como algo inalienable ---como un regalo de la Deidad al clan---. Sostenían que la tierra no se podía vender ni hipotecar. “Yahvé habló y dijo: ‘La tierra no se venderá, porque la tierra es mía'“.
\vs p097 3:3 Los cananeos del norte (los baalitas), más asentados, compraban, vendían e hipotecaban sus tierras a voluntad. La palabra Baal significa propietario. El sistema de culto de Baal se fundó sobre la base de dos doctrinas principales: en la primera, se reconocía el intercambio, los contratos y los pactos sobre la propiedad ---el derecho a comprar y vender tierra---; en la segunda, se suponía que Baal enviaba lluvia ---era el Dios de la fertilidad del suelo---. Las buenas cosechas dependían del favor de Baal. Este sistema de culto se preocupaba en gran medida de la \bibemph{tierra,} su propiedad y la fertilidad.
\vs p097 3:4 En general, los baalitas poseían casas, tierras y esclavos. Eran terratenientes aristócratas y vivían en las ciudades. Cada Baal tenía su lugar secreto, su sacerdocio y sus “mujeres santas”, o prostitutas rituales.
\vs p097 3:5 A partir de esta diferencia fundamental en relación a la tierra, se produjo un enconado antagonismo en las actitudes sociales, económicas, morales y religiosas de los cananeos y los hebreos. Esta controversia socioeconómica no se convirtió en una cuestión claramente religiosa hasta los tiempos de Elías. Desde los días de este combativo profeta, dicha cuestión se dirimió con un enfrentamiento en términos más estrictamente religiosos ---Yahvé contra Baal---, y acabó con el triunfo de Yahvé y el consiguiente impulso hacia el monoteísmo.
\vs p097 3:6 Elías desplazó la controversia entre Yahvé y Baal desde el problema de las tierras hasta el aspecto religioso de las ideologías hebrea y cananea. Cuando Acab asesinó a Nabot mediante un complot para obtener la propiedad de sus tierras, Elías hizo de las viejas costumbres sobre las tierras un asunto moral y puso en marcha una intensa campaña contra los baalitas. Se trataba también de una lucha de la gente del campo contra la dominación por parte de las ciudades. Principalmente bajo la influencia de Elías, Yahvé se convirtió en Elohim. El profeta comenzó como reformador agrario y acabó por exaltar la Deidad. Había muchos baales, Yahvé era \bibemph{uno solo} ---el monoteísmo finalmente triunfó sobre el politeísmo---.
\usection{4. AMÓS Y OSEAS}
\vs p097 4:1 Amós dio un paso importante respecto a la transición del dios tribal ---del dios al que por tanto tiempo se había servido con sacrificios y ceremonias, el Yahvé de los primeros hebreos--- a un Dios que castigaría el crimen y la inmoralidad incluso de su propio pueblo. Amós apareció, procedente de las colinas del sur, para denunciar la criminalidad, la embriaguez, la opresión y la inmoralidad de las tribus del norte. Desde los tiempos de Moisés no se habían proclamado verdades tan clamorosas en Palestina.
\vs p097 4:2 Amós no solo renovó o reformó sino que definió nuevos conceptos relativos a la Deidad. Muchas veces, dio a conocer acerca de Dios lo que sus predecesores ya habían anunciado y valerosamente combatió la creencia en un Ser Divino que toleraba el pecado de su supuesto pueblo elegido. Por primera vez desde los días de Melquisedec, los oídos del hombre oyeron la denuncia del doble rasero de la justicia y la moralidad nacional. Por primera vez en su historia, los hebreos se enteraron de que su propio Dios, Yahvé, no toleraría la delincuencia y el pecado en sus vidas como tampoco lo haría en el caso de otros pueblos. Amós concibió el Dios severo y justo de Samuel y Elías, pero también a un Dios para quien los hebreos no eran diferentes de cualquier otra nación cuando se trataba de castigar el mal. Era un ataque directo a la doctrina egoísta del “pueblo elegido”, lo que produjo un amargo resentimiento en muchos hebreos de aquellos días.
\vs p097 4:3 Dijo Amós: “El que formó los montes y creó el viento, buscad a aquel que hizo las Pléyades y el Orión, que torna la sombra de la muerte en amanecer y hace oscurecer el día como noche”. Y, al denunciar a contemporáneos suyos semirreligiosos, hipócritas y a veces inmorales, buscaba poner de manifiesto la justicia inexorable de un Yahvé invariable cuando dijo de los malvados: “Aunque caven hasta el infierno, de allá los tomará mi mano; y aunque suban hasta el cielo, de allá los haré descender”. “Y si van en cautiverio delante de sus enemigos, allí mandaré la espada y los matará”. Amós sobresaltó aún más a quienes lo oían cuando, señalándolos con un dedo acusador y reprobador, declaró en nombre de Yahvé: “De seguro que no olvidaré jamás ninguna de vuestras obras”. “Y yo mandaré que la casa de Israel sea zarandeada entre todas las naciones, como se zarandea el grano en una criba”.
\vs p097 4:4 Amós proclamó que Yahvé era el “Dios de todas las naciones” y advirtió a los israelitas que el ritual no debía reemplazar a la rectitud. Y, antes de que este valeroso maestro muriese lapidado, había esparcido suficiente levadura de la verdad como para salvar la doctrina del Yahvé supremo; había propiciado un mayor desarrollo de la revelación de Melquisedec.
\vs p097 4:5 \pc Oseas siguió a Amós y a su doctrina de un Dios universal de justicia resucitando el concepto mosaico de un Dios de amor. Oseas predicó el perdón mediante el arrepentimiento, no mediante el sacrificio. Proclamó un evangelio de amorosa benevolencia y misericordia divina, diciendo: “Te desposaré conmigo para siempre; sí, te desposaré conmigo en justicia, juicio, benignidad y misericordia. Te desposaré conmigo incluso en fidelidad”. “Los amaré de pura gracia, porque mi ira se apartó de ellos”.
\vs p097 4:6 Oseas continuó fielmente las advertencias morales de Amós, diciendo de Dios: “Los castigaré cuando lo desee”. Pero los israelitas consideraron lo que dijo una crueldad que lindaba con la traición: “Diré a aquellos que no fueron mi pueblo, ‘vosotros sois mi pueblo'; y ellos dirán, ‘tú eres nuestro Dios'”. Continuó predicando arrepentimiento y perdón, diciendo: “Yo los sanaré de su rebelión, los amaré de pura gracia, porque mi ira se apartó de ellos”. Oseas proclamó siempre la esperanza y el perdón. Por siempre, el núcleo de su mensaje fue: “Tendré compasión de mi pueblo. No conocerás, pues, otro dios fuera de mí ni otro salvador sino a mí”.
\vs p097 4:7 \pc Amós avivó la conciencia nacional de los hebreos a favor del reconocimiento de que Yahvé no aprobaría el delito ni el pecado entre ellos, aunque supuestamente fuesen el pueblo elegido, mientras que Oseas pulsó las primeras notas de los posteriores acordes de la compasión divina y de la amorosa benevolencia, que tan espléndidamente Isaías y sus compañeros entonarían.
\usection{5. EL PRIMER ISAÍAS}
\vs p097 5:1 Eran tiempos en los que algunos anunciaban un inminente castigo por los pecados personales y los crímenes nacionales de los clanes del norte mientras que otros predecían calamidades como represalia por las transgresiones del reino del sur. Como consecuencia de este despertar en las naciones hebreas de la conciencia y de la sensibilización, apareció el primer Isaías.
\vs p097 5:2 Isaías comenzó a predicar la naturaleza eterna de Dios, su sabiduría infinita, la invariable perfección de su lealtad. Representó al Dios de Israel diciendo: “Ajustaré el juicio a cordel, y a nivel la justicia”. “En el día en que Jehová te dé reposo de tu pena, de tus temores y de la dura servidumbre en que te hicieron servir”. “Entonces tus oídos oirán detrás de ti la palabra que diga: “Este es el camino, andad por él'”. “He aquí, Dios es mi salvación; me aseguraré y no temeré; porque mi fortaleza y mi canción es el Señor”. “‘Venid luego’, dice el Señor, ‘y estemos a cuenta: aunque vuestros pecados sean como la grana, como la nieve serán emblanquecidos; aunque sean rojos como el carmesí, vendrán a ser como blanca lana’”.
\vs p097 5:3 Al dirigirse a los hebreos, almas sedientas y atemorizadas, este profeta dijo: “¡Levántate, resplandece, porque ha venido tu luz, y la gloria del Señor ha nacido sobre ti!”. “El espíritu del Señor está sobre mí, porque me ha ungido. Me ha enviado a predicar buenas noticias a los pobres, a vendar a los quebrantados de corazón, a publicar libertad a los cautivos y a los prisioneros apertura de la cárcel”. ““En gran manera me gozaré en el Señor, mi alma se alegrará en mi Dios, porque me vistió con vestiduras de salvación, me rodeó de manto de justicia”. “En toda angustia de ellos él fue angustiado, y el ángel de su presencia los salvó. En su amor y en su clemencia los redimió”.
\vs p097 5:4 \pc A este Isaías le siguieron Miqueas y Abdías, que confirmaron y embellecieron su evangelio, que tan gratificante era para el alma. Y estos dos valientes mensajeros se atrevieron a denunciar los rituales hebreos en manos de los sacerdotes, y atacaron sin miedo a todo el sistema sacrificial.
\vs p097 5:5 Miqueas denunció: “Los jefes juzgan por cohecho, los sacerdotes enseñan por precio, los profetas adivinan por dinero”. Enseñó que llegaría el día en el que se estaría libre de la superstición y de las supercherías sacerdotales: “Se sentará cada uno debajo de su vid y no habrá quien les infunda temor, porque toda la gente vivirá, cada cual según su entendimiento de Dios”.
\vs p097 5:6 El mensaje central de Miqueas fue siempre: “¿Me presentaré ante Dios con holocaustos? ¿Se agradará el Señor de millares de carneros o de diez mil arroyos de aceite? ¿Daré mi primogénito por mi rebelión, el fruto de mis entrañas por el pecado de mi alma? Él me ha declarado, oh hombre, lo que es bueno; y lo que pide el Señor de ti es solamente hacer justicia, amar misericordia y humillarte ante tu Dios”. Fue un gran momento, y, sin duda, fueron tiempos agitados para el hombre mortal que, hace más de dos mil quinientos años, oyó estos mensajes libertadores; hubo incluso quien los creyó. Y si no hubiese sido por la pertinaz resistencia de los sacerdotes, estos maestros habrían eliminado todo el ceremonial sangriento del ritual de adoración de los hebreos.
\usection{6. JEREMIÁS EL AUDAZ}
\vs p097 6:1 Aunque diversos maestros continuaron divulgando el evangelio de Isaías, le correspondió a Jeremías dar con valentía el siguiente paso en la internacionalización de Yahvé, Dios de los hebreos.
\vs p097 6:2 Sin temor, Jeremías declaró que Yahvé no estaba del lado de los hebreos en sus contiendas militares con otras naciones. Afirmó que Yahvé era el Dios de toda la tierra, de todas las naciones y de todos los pueblos. Las enseñanzas de Jeremías representaron el punto culminante de la creciente ola surgida respecto a la internacionalización del Dios de Israel; de una vez por todas, este audaz e intrépido predicador proclamó que Yahvé era el Dios de todas las naciones, y que no había Osiris para los egipcios, Bel para los babilonios, Asur para los asirios ni Dagón para los filisteos. Y, de esta manera, la religión de los hebreos participó en ese renacimiento del monoteísmo que se dio por todo el mundo, en ese momento y tras él; por fin, el concepto de Yahvé había ascendido como Deidad a un nivel de dignificación planetaria e incluso cósmica. Pero hubo muchos conciudadanos de Jeremías que tenían dificultades para concebir a Yahvé al margen de la nación hebrea.
\vs p097 6:3 Jeremías predicó también sobre el Dios justo y amante descrito por Isaías, declarando: “Sí, con amor eterno te he amado; por eso, te prolongué mi amorosa benevolencia”. “Pues no se complace en afligir o entristecer a los hijos de los hombres”.
\vs p097 6:4 Este audaz profeta dijo: “Justo es nuestro Señor, grande en consejo y magnífico en hechos. Sus ojos están abiertos sobre todos los caminos de los hijos de los hombres, para dar a cada uno según sus caminos y según el fruto de sus obras”. Pero se consideró como traición por blasfemia cuando, durante el asedio de Jerusalén, él dijo: “Y ahora yo he puesto todas estas tierras en mano de Nabucodonosor, rey de Babilonia, mi siervo”. Y cuando Jeremías aconsejó que se rindiera la ciudad, los sacerdotes y los gobernantes civiles le arrojaron al hoyo fangoso de una sombría mazmorra.
\usection{7. EL SEGUNDO ISAÍAS}
\vs p097 7:1 El fin de la nación hebrea y su cautiverio en Mesopotamia podrían haber sido de gran utilidad para la expansión de su teología, si no hubiese sido por la decidida acción de su sacerdocio. Su nación había caído ante los ejércitos de Babilonia y su Yahvé de ámbito nacional había padecido la predicación internacionalista de los líderes espirituales. Fue el resentimiento por la pérdida de su dios nacional lo que llevó a los sacerdotes judíos a llegar a los extremos de inventar fábulas y multiplicar los sucesos de apariencia milagrosa en la historia hebrea, en un intento por restituir a los judíos como pueblo elegido entre las demás naciones, incluso a pesar de la idea nueva y ampliada de un Dios de ámbito internacional.
\vs p097 7:2 Durante su cautiverio, los judíos se vieron influenciados por las tradiciones y las leyendas babilónicas, aunque cabe señalar que, sin lugar a dudas, mejoraron el tono moral y el significado espiritual de las historias caldeas que adoptaron, pese a que sistemáticamente distorsionaron estas leyendas para poner de manifiesto el honor y la gloria de la ascendencia y la historia de Israel.
\vs p097 7:3 Estos sacerdotes y escribas hebreos tenían una sola idea en la mente: la rehabilitación de la nación judía, la glorificación de las tradiciones hebreas y la exaltación de su historia racial. Si existe alguna animosidad por el hecho de que los sacerdotes implantaran sus equivocadas ideas en un segmento tan grande del mundo occidental, conviene recordar que no lo hicieron de forma intencionada; nunca alegaron que escribían por inspiración; nunca declararon que escribían un libro sagrado. Estaban simplemente redactando un texto concebido para reforzar el mermado ánimo de sus semejantes cautivos. Claramente, tenían el objetivo de alentar el espíritu y el estado de ánimo nacional de sus compatriotas. En días posteriores, hubo quienes reunieron estos y otros escritos en un libro guía de enseñanzas supuestamente infalibles.
\vs p097 7:4 El sacerdocio judío utilizó ampliamente estos escritos tras la época del cautiverio, pero su influencia sobre sus semejantes en cautividad se encontró con grandes dificultades por la presencia de un profeta joven e indomable, el segundo Isaías, que se había convertido por completo al Dios de la justicia, el amor, la rectitud y la misericordia del viejo Isaías. También creía, junto con Jeremías, que Yahvé se había convertido en el Dios de todas las naciones. Predicó estos planteamientos de la naturaleza de Dios con tal profundo efecto que hizo conversos por igual entre judíos y sus captores. Y este joven predicador dejó constancia de sus enseñanzas, que los sacerdotes, hostiles e implacables, trataron de desvincular de toda relación con él; si bien, el enorme respeto por la belleza y la grandeza de estas enseñanzas llevó a su inclusión en los escritos del primer Isaías. Y, por consiguiente, los escritos de este segundo Isaías se pueden encontrar en el libro de dicho nombre, incluidos en los capítulos que van desde el cuarenta al cincuenta y cinco.
\vs p097 7:5 \pc Ningún profeta ni maestro religioso desde Maquiventa hasta el tiempo de Jesús consiguió elaborar un concepto tan elevado de Dios como el que proclamó el segundo Isaías durante estos días de cautiverio. El Dios que anunció este líder espiritual no era pequeño, antropomorfo ni obra del hombre. “He aquí que las islas le son como polvo que se desvanece”. “Y como son más altos los cielos que la tierra, así son mis caminos más altos que vuestros caminos y mis pensamientos más que vuestros pensamientos”.
\vs p097 7:6 Por fin, Maquiventa Melquisedec contemplaba maestros humanos que proclamaban un verdadero Dios al hombre mortal. Como Isaías el primero, este líder espiritual predicaba un Dios creador y sustentador universal. “Yo he creado la tierra y creé sobre ella al hombre. No la creé en vano, sino para que fuera habitada la creé”. “Yo soy el primero y el último; y fuera de mí no hay Dios”. Hablando en nombre del Señor Dios de Israel, este nuevo profeta dijo: “Los cielos se desvanecerán y la tierra se envejecerá, pero mi justicia permanecerá perpetuamente y mi salvación por generación y generación”. “No temas, porque yo estoy contigo; no desmayes, porque yo soy tu Dios”. “Y no hay más Dios que yo, Dios justo y salvador. No hay otro fuera de mí”.
\vs p097 7:7 Y, como ha consolado a miles y miles de personas desde entonces, a los cautivos judíos les consoló oír estas palabras: “’Así dice el Señor, ‘yo te he creado, te he redimido, te puse nombre, mío eres tú’”. “Cuando pases por las aguas, yo estaré contigo porque a mis ojos eres de gran estima”. “¿Se olvidará la mujer de su hijo lactante y dejar de compadecerse de su hijo? ¡Aunque ella lo olvide, yo nunca me olvidaré de mis hijos! He aquí que en las palmas de las manos los tengo esculpidos; con la sombra de mi mano los cubrí”. “Deje el impío su camino y el hombre inicuo sus pensamientos, y vuélvase al Señor, el cual tendrá de él misericordia, y al Dios nuestro, el cual será amplio en perdonar”.
\vs p097 7:8 Escuchad de nuevo el evangelio de esta nueva revelación del Dios de Salem: “Como pastor apacentará su rebaño. En su brazo llevará a los corderos, junto a su pecho los llevará. Él da esfuerzo al cansado y multiplica las fuerzas al que no tiene ningunas. Los que esperan en el Señor tendrán nuevas fuerzas, levantarán las alas como las águilas, correrán y no se cansarán, caminarán y no se fatigarán”.
\vs p097 7:9 Este Isaías difundió extensamente el evangelio del concepto ampliado de un Yahvé supremo. Rivalizó con Moisés en elocuencia al describir al Señor Dios de Israel como el Creador Universal. Se expresó poéticamente al ilustrar los atributos infinitos del Padre Universal. Jamás se han vuelto a hacer afirmaciones tan bellas sobre el Padre celestial. Como los salmos, los escritos de Isaías se encuentran entre las definiciones más sublimes y verdaderas de la idea espiritual de Dios que se hayan manifestado ante oídos humanos antes de la llegada de Miguel a Urantia. Escuchad su descripción de la Deidad: “Soy el alto y el sublime, el que habita la eternidad”. “Yo soy el primero y yo soy el último, y fuera de mí no hay Dios”. “Y he aquí que no se ha acortado la mano del Señor para salvar, ni se ha endurecido su oído para oír”. Para el pueblo judío se trataba de una doctrina nueva cuando este benévolo pero imperioso profeta perseveró en su predicación de la constancia divina, de la fidelidad de Dios. Afirmó que “Dios no olvidará, no abandonará”.
\vs p097 7:10 Este atrevido maestro proclamó que el hombre estaba muy estrechamente vinculado a Dios, diciendo: “Todos los llamados de mi nombre, para gloria mía los he creado, y mis alabanzas publicarán. Yo, yo soy quien borro sus rebeliones por amor de mí mismo, y no me acordaré de sus pecados”.
\vs p097 7:11 Oíd a este gran hebreo acabar con el concepto de un Dios nacional mientras proclama en gloria la divinidad del Padre Universal, de quien dice: “El cielo es mi trono y la tierra el estrado de mis pies”. Y el Dios de Isaías era no obstante santo, majestuoso, justo e inescrutable. El concepto del Yahvé furioso, vengativo y celoso de los beduinos del desierto se ha desvanecido casi por completo. Una nueva noción del Yahvé supremo y universal ha surgido en la mente del hombre mortal, para nunca más perderse de la percepción humana. El entendimiento de la justicia divina ha comenzado a erradicar la magia primitiva y el miedo biológico. Por fin, se da a conocer al hombre un universo de ley y orden y un Dios universal de atributos fiables y últimos.
\vs p097 7:12 Y este predicador de un Dios celestial nunca dejó de proclamar a un \bibemph{Dios de amor}. “Yo habito en la altura y la santidad, pero habito también con el quebrantado y el humilde de espíritu”. Y tal gran maestro pronunció todavía más palabras de consuelo a sus contemporáneos: “Jehová te pastoreará siempre y saciará tu alma. Serás como un huerto de riego, como un manantial de aguas, cuyas aguas nunca se agotan. Y si el enemigo viene como un río crecido, el espíritu del Señor levantará una defensa contra él”. Y, para la bendición de la humanidad, brilló, una vez más, el evangelio de Melquisedec, que deshacía el temor, y la religión de Salem, que hacía brotar la confianza.
\vs p097 7:13 E Isaías, clarividente y valeroso, eclipsó eficientemente al Yahvé nacionalista gracias a su sublime representación de la majestad y de la omnipotencia universal del Yahvé supremo, Dios de amor, soberano del universo y Padre amoroso de toda la humanidad. Desde esos días memorables, en Occidente, el concepto más elevado de Dios denota justicia universal, misericordia divina y rectitud eterna. En un magnífico lenguaje y con una delicadeza sin igual, este gran maestro describió al Creador todopoderoso como un Padre amantísimo.
\vs p097 7:14 Este profeta del cautiverio predicó a su gente y a aquellas personas de otras muchas naciones que lo oyeron a orillas del río de Babilonia. Y este segundo Isaías contribuyó, en buena medida, a contrarrestar los muchos conceptos equivocados y egoístamente raciales sobre la misión del Mesías prometido. Si bien, en este aspecto, sus esfuerzos no se vieron recompensados del todo. Si los sacerdotes no se hubiesen dedicado a la labor de desarrollar un desacertado nacionalismo, las enseñanzas de los dos Isaías habrían allanado el camino al reconocimiento y a la acogida del Mesías prometido.
\usection{8. HISTORIA SAGRADA E HISTORIA PROFANA}
\vs p097 8:1 La costumbre de considerar el relato de las experiencias de los hebreos como historia sagrada y los actos del resto del mundo como historia profana es responsable de gran parte de la confusión imperante en la mente humana respecto a la interpretación de la historia. Y esta dificultad se plantea porque no hay una historia laica judía. Después de que los sacerdotes exiliados en Babilonia diseñasen una nueva crónica de los actos supuestamente milagrosos de Dios en su trato con los hebreos ---la historia sagrada de Israel tal como se plasma en el Antiguo Testamento---, destruyeron cuidadosamente y por completo los registros existentes sobre las cuestiones hebreas: libros tales como “Los hechos de los reyes de Israel” y “Los hechos de los reyes de Judá”, junto con algunos otros textos más o menos precisos de la historia hebrea.
\vs p097 8:2 Para comprender la demoledora presión y el ineludible apremio de la historia laica que tanto aterrorizaba a los judíos cautivos, subyugados por extranjeros, y que les llevó a un intento por reescribir y refundir por completo su propia historia, tenemos que acercarnos de forma sucinta a los antecedentes de su confusa experiencia nacional. Conviene recordar que los judíos no lograron desarrollar una adecuada filosofía no teológica de la vida. Se debatían entre su concepto original egipcio de la recompensa divina por la rectitud y los terribles castigos por el pecado. La historia de Job fue en cierto modo una protesta contra esta filosofía equivocada. El pesimismo manifiesto del Eclesiastés fue una sensata reacción a estas creencias demasiado optimistas en la Providencia.
\vs p097 8:3 Pero quinientos años de supremacía de soberanos extranjeros resultó muy agraviante incluso para los pacientes y sufridos judíos. Los profetas y los sacerdotes comenzaron a lamentarse: “¿Hasta cuándo, oh Señor, hasta cuándo?”. Cuando los judíos escudriñaban honestamente las Escrituras, su desconcierto era aún mayor. Un vidente de antaño había prometido que Dios protegería y redimiría a su “pueblo elegido”. Amós había amenazado con que Dios abandonaría a Israel a menos que restableciesen sus normas de rectitud nacional. El escriba del Deuteronomio había descrito la Gran Elección ---entre el bien y el mal, entre la bendición y la maldición---. El primer Isaías había predicado un benevolente rey\hyp{}libertador. Jeremías había proclamado una era de rectitud interna ---el pacto escrito en las tablas del corazón---. El segundo Isaías había hablado de la salvación a través del sacrificio y la redención. Ezequiel proclamó libertad por medio de la dedicación al servicio y Esdras prometió prosperidad mediante el cumplimiento de la ley. Pero a pesar de todo ello, permanecían cautivos y la liberación se postergaba. Entonces Daniel presentó el relato de una “crisis” inminente: la destrucción de la gran imagen y el establecimiento inmediato del reino perpetuo de la rectitud, o reino mesiánico.
\vs p097 8:4 Y todas estas falsas esperanzas dieron lugar a tal grado de decepción y frustración raciales que los confundidos líderes de los judíos no lograron reconocer y aceptar la misión y el ministerio de un divino hijo del Paraíso, cuando este llegó poco después a ellos con la semejanza de un hombre mortal ---encarnado como Hijo del Hombre---.
\vs p097 8:5 \pc La totalidad de las religiones modernas ha cometido el grave error de intentar dar una interpretación milagrosa a determinadas épocas de la historia humana. Aunque es verdad que muchas veces Dios ha tendido providencialmente su mano paterna y ha intervenido en el flujo de los asuntos humanos, es un error considerar los dogmas teológicos y las supersticiones religiosas como un sedimento sobrenatural aparecido, mediante alguna acción milagrosa, en dicho flujo de la historia humana. El hecho de que “los altísimos gobiernan los reinos de los hombres” no convierte la historia laica en una pretendida historia sagrada.
\vs p097 8:6 Los autores del Nuevo Testamento y los escritores cristianos más tardíos complicaron aún más la tergiversación de la historia hebrea a través de sus intentos bien intencionados por trascendentalizar a los profetas judíos. En consecuencia, se ha utilizado calamitosamente la historia hebrea tanto por parte de los escritores judíos como de los escritores cristianos. La historia laica de los hebreos se ha dogmatizado considerablemente. Se ha convertido en una historia sagrada ficcionalizada y está inextricablemente ligada a los conceptos morales y las enseñanzas religiosas de las llamadas naciones cristianas.
\vs p097 8:7 \pc Una breve narración de los momentos más importantes de la historia hebrea ilustrará de qué manera alteraron los sacerdotes judíos en Babilonia los hechos registrados, hasta llegar a transformar la historia laica de la cotidianidad de su pueblo en una historia sagrada de su invención.
\usection{9. LA HISTORIA HEBREA}
\vs p097 9:1 Nunca hubo doce tribus de israelitas ---en Palestina solo se establecieron tres o cuatro tribus---. La nación hebrea nació como resultado de la unión de los llamados israelitas y cananeos. “Así los hijos de Israel comenzaron a habitar entre los cananeos. Y tomaron a sus hijas por mujeres y dieron sus hijas a los hijos de los cananeos”. Los hebreos nunca expulsaron a los cananeos de Palestina, a pesar de que, en la crónica de los sacerdotes respecto a estos hechos, se afirme sin vacilación que sí lo hicieron.
\vs p097 9:2 La conciencia israelita tuvo su origen en la zona montañosa de Efraín; la conciencia judía surgida con posterioridad se originó en el clan meridional de Judá. Los judíos (o judaítas) siempre trataron de difamar y manchar la historia de los israelitas del norte (o eframitas).
\vs p097 9:3 \pc La fatua historia hebrea comienza con la movilización de los clanes del norte por parte de Saúl para resistir un ataque de los amonitas contra la tribu hermana de los galaaditas del este del Jordán. Con un ejército de algo más de tres mil hombres venció al enemigo, y fue esta hazaña la que hizo que las tribus de las colinas lo proclamaran rey. Cuando los sacerdotes exiliados reescribieron esta historia, aumentaron el número de hombres del ejército de Saúl a 330\,000 y añadieron “Judá” a la lista de las tribus que habían intervenido en la batalla.
\vs p097 9:4 Justo después de derrotar a los amonitas, sus tropas, por elección popular, nombraron rey a Saúl. En este acontecimiento no participaron ni sacerdotes ni profetas. Si bien, más adelante, los sacerdotes incorporaron a sus crónicas la idea de que el profeta Samuel coronó rey a Saúl de acuerdo con directrices divinas. Esto lo hicieron a fin de establecer una “línea de descendencia divina” para el reinado judaíta de David.
\vs p097 9:5 La mayor de todas las tergiversaciones de la historia judía tuvo que ver con David. Tras la victoria de Saúl sobre los amonitas (que él atribuyó a Yahvé), los filisteos se alarmaron y comenzaron a atacar a los clanes del norte. David y Saúl nunca llegaron a ponerse de acuerdo. David, con seiscientos hombres, concertó una alianza con los filisteos y marchó a la costa de Esdraelón. En Gat los filisteos ordenaron a David que se alejara de allí; temían que se uniese a Saúl. David se retiró; los filisteos atacaron y derrotaron a Saúl. No podrían haberlo hecho si David hubiese sido leal a Israel. El ejército de David era un variopinto colectivo de descontentos, formado, en su mayor parte, por inadaptados sociales y fugitivos de la justicia.
\vs p097 9:6 Ante los ojos de los cananeos circundantes, la trágica derrota de Saúl en Gilboa por los filisteos llevó a Yahvé a su escalafón más bajo respecto a los demás dioses. Por lo común, la derrota de Saúl se hubiese atribuido a una apostasía hacia Yahvé, pero esta vez los revisores judaítas lo achacaron a errores rituales. Precisaban de la tradición de Saúl y Samuel como antecedentes para el reinado de David.
\vs p097 9:7 Con su pequeño ejército, David se estableció en la ciudad no hebrea de Hebrón. En poco tiempo, sus compatriotas lo proclamaron rey del nuevo reino de Judá. Judá estaba integrado mayormente por grupos no hebreos: los ceneos, los calebitas, los jebuseos y otros cananeos. Eran nómadas ---pastores--- y, por lo tanto, eran afines a la idea hebrea sobre la propiedad de la tierra. Poseían las ideologías de los clanes del desierto.
\vs p097 9:8 \pc La diferencia entre la historia sagrada y la historia profana queda bien ilustrada por los dos relatos distintos del nombramiento de David como rey, hallados en el Antiguo Testamento. Los sacerdotes, inadvertidamente, dejaron una parte de la historia laica de cómo sus seguidores inmediatos (su ejército) lo hicieron rey. Estos mismos redactarían posteriormente el largo y prosaico relato de la historia sagrada en el que se describe cómo el profeta Samuel, siguiendo directrices divinas, escogió a David entre sus hermanos y, mediante ceremonias solemnes y elaboradas, procedió a ungirle formalmente rey de los hebreos, para luego proclamarle sucesor de Saúl.
\vs p097 9:9 Cuántas veces los sacerdotes, tras preparar sus ficticias narrativas sobre los contactos milagrosos entre Dios e Israel, se olvidaron de suprimir por completo algunas de las aseveraciones sencillas y realistas ya existentes en esos relatos.
\vs p097 9:10 \pc David intentó mejorar su posición política, casándose primero con la hija de Saúl, luego con la viuda de Nabal, el rico edomita, y, luego, con la hija de Talmai, el rey de Geshur. Tomó seis esposas entre las mujeres de Jebús, sin olvidar a Betsabé, la esposa del hitita.
\vs p097 9:11 Y por medio de estos métodos y junto con dichas personas, David elaboró la ficción de un reino divino de Judá como sucesor de la herencia y las tradiciones del reino septentrional del Israel efraimita, en vías de desaparición. La tribu de David, la cosmopolita Judá, era más gentil que judía; no obstante, los ancianos oprimidos de Efraín bajaron y “le ungieron como rey de Israel”. Tras una amenaza militar, David hizo una alianza con los jebuseos y estableció la capital del reino unido en Jebús (Jerusalén), que era una ciudad fuertemente amurallada a medio camino entre Judá e Israel. Los filisteos se vieron movidos a la acción y atacaron pronto a David. Después de una encarnizada batalla, estos cayeron derrotados y, una vez más, se instauró a Yahvé como “el Señor Dios de las Huestes”.
\vs p097 9:12 Pero Yahvé debía, forzosamente, compartir parte de esta gloria con los dioses cananeos, ya que el grueso del ejército de David no era hebreo. Y, por lo tanto, aparece en vuestros escritos (quedando desapercibida por los revisores judaítas) esta reveladora afirmación: “Yahvé Jehová me abrió brecha entre mis enemigos. Por esto llamó el nombre de aquel lugar Baal\hyp{}perazim”. Y así lo hicieron porque el ochenta por ciento de los soldados de David eran baalitas.
\vs p097 9:13 David explicó la derrota de Saúl en Gilboa alegando que Saúl había atacado Gabaón, una ciudad cananea cuyo pueblo tenía un tratado de paz con los eframitas. Por ello, Yahvé lo abandonó. Incluso en los tiempos de Saúl, David había defendido la ciudad cananea de Keila contra los filisteos, y luego situó su capital en una ciudad cananea. En conformidad con la política de compromiso con los cananeos, David entregó a siete de los descendientes de Saúl a los gabaonitas para que fuesen ahorcados.
\vs p097 9:14 Tras la derrota de los filisteos, David tomó posesión del “arca de Yahvé”, la llevó a Jerusalén y, en su reino, convirtió en oficial la adoración de Yahvé. A continuación, impuso onerosos tributos a las tribus vecinas de los edomitas, moabitas, amonitas y sirios.
\vs p097 9:15 A través de su corrupta maquinaria política, David comenzó a apoderarse personalmente de las tierras al norte, infringiendo las costumbres hebreas, y pronto obtuvo el control de los aranceles de las caravanas, anteriormente recaudados por los filisteos. Entonces, se cometieron una serie de atrocidades, que culminaron con el asesinato de Urías. Todas las apelaciones judiciales se resolvían en Jerusalén; “los ancianos” ya no podían hacer justicia. No es de extrañar que estallara una revuelta. Hoy en día se calificaría a Absalón de demagogo; su madre era cananea. Había media docena de candidatos al trono además de Salomón, el hijo de Betsabé.
\vs p097 9:16 \pc Tras la muerte de David, Salomón purgó la maquinaria política de todas las influencias del norte, pero continuó toda la tiranía y la tributación del régimen de su padre. Salomón llevó a la nación a la quiebra debido a los lujos de su corte y a sus complejos planes de construcción tales como la casa del Líbano, el palacio de la hija del faraón, el templo de Yahvé, el palacio del rey y la restauración de las murallas de muchas ciudades. Salomón creó una gran marina mercante hebrea, operada por marineros sirios, que comerciaba con todo el mundo. El número de mujeres de su harén ascendía a casi mil.
\vs p097 9:17 \pc Hacia esta época, el templo de Yahvé en Silo estaba desacreditado, y toda la adoración de la nación estaba centrada en Jebús, en la magnífica capilla real. El reino del norte regresó más a la adoración de Elohim. Gozaban del favor de los faraones, que más tarde esclavizarían a Judá, colocando al reino del sur bajo tributación.
\vs p097 9:18 Hubo altibajos ---guerras entre Israel y Judá---. Después de cuatro años de guerra civil y tres dinastías, Israel cayó bajo el mandato de los tiranos de las ciudades, que comenzaron a comerciar con las tierras. Incluso el rey Omri trató de comprar los terrenos de Semer. Pero el fin se desencadenó rápidamente cuando Salmanasar III decidió dominar la costa mediterránea. El rey Acab, de Efraín, reunió a otros diez grupos y resistió en Karkar; la batalla terminó en tablas. Detuvieron a los asirios, pero los aliados fueron diezmados. Este gran combate ni siquiera se menciona en el Antiguo Testamento.
\vs p097 9:19 Se produjeron nuevos problemas cuando el rey Acab intentó comprar la heredad de Nabot. Su esposa fenicia falsificó el nombre de Acab en unos escritos en los que se ordenaba que se confiscara la tierra de Nabot bajo el cargo de haber blasfemado contra los nombres de “Elohim y del rey”. Él y sus hijos fueron rápidamente ejecutados. El enérgico Elías apareció en escena denunciando a Acab por el asesinato de los Nabot. Así pues, Elías, uno de los más grandes profetas, comenzó sus enseñanzas como defensor de las viejas costumbres sobre las tierras y en contra de la actitud de los baalitas de venderlas, contra el intento de las ciudades por dominar el campo. Pero la reforma no tuvo éxito hasta que el terrateniente Jehú no unió fuerzas con el caudillo gitano Jonadab para exterminar a los profetas (agentes de la propiedad) de Baal en Samaria.
\vs p097 9:20 \pc Una nueva vida apareció cuando Joás y su hijo Jeroboám libraron a Israel de sus enemigos. Pero, para entonces, gobernaba en Samaria una nobleza facinerosa cuyas expolios rivalizaban con los de la dinastía davídica de los viejos tiempos. El Estado y la Iglesia obraban del mismo modo. El intento de reprimir la libertad de expresión llevó a Elías, Amós y Oseas a escribir de forma secreta, y este fue el verdadero comienzo de las biblias judía y cristiana.
\vs p097 9:21 \pc Pero el reino del norte no desapareció de la historia hasta que el rey de Israel conspiró con el rey de Egipto y se negó a pagar más tributos a Asiria. Empezó entonces un asedio que duró tres años al que siguió la total dispersión del reino del norte. Efraín (Israel) dejó, pues, de existir. En Judá ---los judíos, “el resto de Israel”--- había comenzado la concentración de la propiedad de las tierras en las manos de unos pocos, tal como dijo Isaías: “juntando casa a casa y añadiendo hacienda a hacienda”. En Jerusalén, en poco tiempo, hubo un templo de Baal al lado del templo de Yahvé. Este reino de terror terminó mediante una revuelta a favor del monoteísmo liderada por el rey niño Joás, que libró una cruzada de treinta y cinco años por Yahvé.
\vs p097 9:22 El siguiente rey, Amasías, tuvo problemas con la sublevación de los contribuyentes edomitas y sus vecinos. Después de una clara victoria, atacó a sus vecinos del norte y fue derrotado con igual claridad. Entonces se rebelaron los campesinos; asesinaron al rey y colocaron en el trono a su hijo de dieciséis años. Se trataba de Azarías, llamado Uzías por Isaías. Después de Uzías, las cosas fueron de mal en peor, y Judá se aseguró su existencia durante cien años pagando tributos a los reyes de Asiria. El primer Isaías les dijo que Jerusalén, por ser la ciudad de Yahvé, jamás caería. Pero Jeremías no dudó en proclamar su caída.
\vs p097 9:23 \pc La verdadera perdición de Judá se debió a un grupo de políticos corruptos y ricos que actuaba en coordinación bajo el gobierno de Manasés, un rey niño. La economía cambiante favoreció la vuelta a la adoración de Baal, cuyos creyentes negociaban con la propiedad privada de las tierras en contra de la ideología de los seguidores de Yahvé. La caída de Asiria y la preponderancia de Egipto trajeron durante algún tiempo la liberación a Judá, y los campesinos tomaron el poder. Bajo Josías, estos aniquilaron al grupo de políticos corrompidos de Jerusalén.
\vs p097 9:24 Pero esta era tuvo un final trágico cuando Josías se atrevió a marchar para interceptar el poderoso ejército de Necao, que desde Egipto avanzaba por la costa con el fin de ayudar a Asiria contra Babilonia. Josías fue aniquilado, y Judá cayó bajo la tributación de Egipto. El partido político de Baal volvió al poder en Jerusalén y así comenzó la \bibemph{verdadera} esclavitud a manos de los egipcios. A continuación, tuvo lugar un período en el que los políticos de Baal controlaban tanto los tribunales como el sacerdocio. La adoración de Baal era un sistema económico y social que se ocupaba de derechos de la propiedad al igual que de la fertilidad del suelo.
\vs p097 9:25 Con el derrocamiento de Necao por parte de Nabucodonosor, Judá cayó bajo el régimen de Babilonia y se le dieron diez años de gracia, pero pronto se rebeló. Cuando Nabucodonosor fue contra ellos, los judaítas comenzaron reformas sociales, como la liberación de los esclavos, para influir en Yahvé. Cuando el ejército babilonio se retiró temporalmente, los hebreos se regocijaron porque el método mágico de sus reformas les había puesto en libertad. Fue durante este período cuando Jeremías les habló de una fatalidad inminente y, enseguida, volvió Nabucodonosor.
\vs p097 9:26 Y, así, el fin de Judá llegó de repente. Se destruyó la ciudad y se condujo al pueblo a Babilonia. La lucha entre Yahvé y Baal terminó en cautiverio. Y este conmovió a la población restante de Israel hasta convertirlos al monoteísmo.
\vs p097 9:27 \pc En Babilonia, los judíos llegaron a la conclusión de que no podían existir en Palestina como un reducido grupo, con sus propias costumbres sociales y económicas peculiares, y, para que sus ideologías prevaleciesen, debían convertir a los gentiles. Así fue como se originó su nuevo concepto de destino ---la idea de que los judíos deben convertirse en los servidores elegidos de Yahvé---. Realmente, la religión judía del Antiguo Testamento evolucionó en Babilonia, durante el cautiverio.
\vs p097 9:28 La doctrina de la inmortalidad también tomó forma en Babilonia. Los judíos habían pensado que la idea de la vida futura restaba énfasis a su evangelio sobre la justicia social. Ahora, por primera vez, la teología desplazaba a la sociología y a la economía. La religión estaba configurándose como un sistema de pensamiento y de conducta humanos, separándose cada vez más de la política, la sociología y la economía.
\vs p097 9:29 \pc Y, de esta manera, la verdad sobre el pueblo judío desvela que gran parte de lo que se ha considerado como historia sagrada resulta ser poco más que la crónica de una ordinaria historia profana. El judaísmo fue el terreno donde el cristianismo creció, pero los judíos no eran un pueblo milagroso.
\usection{10. LA RELIGIÓN HEBREA}
\vs p097 10:1 Sus líderes habían enseñado a los israelitas que ellos eran un pueblo elegido, no por gratificación particular hacia ellos ni por el acaparamiento del favor divino, sino por el excepcional servicio de llevar la verdad del Dios único a todas las naciones. Y habían prometido a los judíos que, si cumplían con su destino, se convertirían en los líderes espirituales de todos los pueblos, y que el Mesías venidero reinaría sobre ellos y sobre todo el mundo como el Príncipe de la Paz.
\vs p097 10:2 Cuando los persas liberaron a los judíos, estos regresaron a Palestina solo para caer en la servidumbre de su propio código de leyes, sacrificios y rituales bajo el control de los sacerdotes. Y al igual que los clanes hebreos rechazaron la maravillosa historia de Dios presentada en la oración de despedida de Moisés a cambio de los rituales de sacrificio y arrepentimiento, del mismo modo, estas personas remanentes de la nación hebrea rechazaron las magníficas nociones del segundo Isaías a cambio de las reglas, regulaciones y rituales de su creciente sacerdocio.
\vs p097 10:3 El egoísmo nacional, la fe falsa en una desacertada idea del Mesías prometido y la servidumbre y la tiranía, cada vez más intensas, a las que el sacerdocio sometía, silenciaron para siempre las voces de los líderes espirituales (excepto las de Daniel, Ezequiel, Hageo y Malaquías); y, desde aquel día hasta los tiempos de Juan el Bautista, todo Israel experimentó un gradual retroceso espiritual. Pero los judíos no olvidaron jamás su percepción del Padre Universal; incluso hasta el siglo XX d. C. continuaron manteniendo esta noción de la Deidad.
\vs p097 10:4 Desde Moisés hasta Juan el Bautista se dio una línea ininterrumpida de leales maestros que pasaron la antorcha de la luz monoteísta de generación en generación, a la vez que censuraban continuamente a los gobernantes sin escrúpulos, denunciaban el afán mercantilista de los sacerdotes y exhortaban constantemente al pueblo a que profesasen la adoración del Yahvé supremo, el Señor Dios de Israel.
\vs p097 10:5 \pc Como nación, los judíos acabaron por perder su identidad política, pero la religión hebrea, basada en su sincera creencia en el Dios único y universal, continúa viviendo en el corazón de los exiliados dispersos. Y esta religión sobrevive porque ha podido conservar adecuadamente los valores más elevados de sus seguidores. Ciertamente, la religión judía preservó los ideales de un pueblo, pero no consiguió promover el progreso ni alentar el descubrimiento creativo y filosófico en los ámbitos de la verdad. La religión judía tenía muchos defectos ---era deficiente en filosofía y estaba prácticamente carente de cualidades estéticas---, pero sí conservó los valores morales y, por lo tanto, perduró. En relación a otros conceptos de la Deidad, el Yahvé supremo estaba claramente definido; era vívido, personal y moral.
\vs p097 10:6 Los judíos amaban la justicia, la sabiduría, la verdad y la rectitud como pocos pueblos lo han hecho, pero contribuyeron menos que cualquier otro a la comprensión intelectual y al entendimiento espiritual de estas cualidades divinas. Aunque la teología hebrea no llegó a extenderse, desempeñó un papel importante en el desarrollo de otras dos religiones mundiales: el cristianismo y el mahometismo.
\vs p097 10:7 La religión judía subsistió también debido a sus instituciones. Es difícil que una religión sobreviva si es una práctica privada de personas aisladas. Este ha sido siempre el error de los líderes religiosos: ver las maldades de la religión institucionalizada y tratar de anular la forma de actuar en grupo. En lugar de suprimir todo el ritual, sería mejor reformarlo. En este respecto, Ezequiel estuvo más acertado que sus contemporáneos; aunque se unió a ellos al insistir en la responsabilidad moral personal, también se propuso establecer el fiel cumplimiento de un ritual superior y purificado.
\vs p097 10:8 \pc Y, de este modo, los sucesivos maestros de Israel realizaron la mayor hazaña en la evolución de la religión jamás llevada a cabo en Urantia: la transformación gradual pero continua desde el primitivo concepto de un Yahvé demoniaco y salvaje, el dios celoso y el espíritu cruel del fulminante volcán del Sinaí, hasta al concepto más reciente, excelso y sublime del Yahvé supremo, creador de todas las cosas y Padre amoroso y misericordioso de toda la humanidad. Y esta noción hebraica de Dios resultó ser la concepción humana más elevada del Padre Universal hasta el momento en el que se amplió aún más y de forma tan espléndida por las enseñanzas personales y el ejemplo de vida de su hijo, Miguel de Nebadón.
\vsetoff
\vs p097 10:9 [Exposición de un melquisedec de Nebadón.]
