\upaper{71}{El desarrollo del estado}
\author{Melquisedec}
\vs p071 0:1 El Estado es un valioso avance de la civilización; representa el beneficio de la sociedad tras los estragos y sufrimientos de la guerra. Incluso el arte de gobernar no es sino un método acumulativo de regular la antagónica disputa de fuerzas entre las tribus y las naciones en pugna.
\vs p071 0:2 El Estado moderno es la institución que sobrevivió a la larga lucha por el poder del grupo. Con el tiempo, el poder superior prevaleció y trajo consigo un ente fáctico ---el Estado--- junto con el mito moral de la absoluta obligación del ciudadano de vivir y morir por él. Pero el Estado no es de procedencia divina; ni siquiera es producto de la acción volitiva de la inteligencia humana; es una institución puramente evolutiva y de origen completamente natural.
\usection{1. ETAPA EMBRIONARIA DEL ESTADO}
\vs p071 1:1 El Estado es un organismo de regulación social y territorial, y el Estado más fuerte, más eficaz y perdurable está formado por una sola nación cuya población posee una lengua, unas costumbres y unas instituciones comunes.
\vs p071 1:2 Los primeros Estados eran pequeños y surgieron todos como consecuencia de las conquistas. No tuvieron su origen en asociaciones voluntarias. Un gran número de ellos se fundaron por conquistadores nómadas, los cuales se abatían sobre pastores pacíficos o sobre agricultores ya establecidos para someterlos y esclavizarlos. Los Estados, que resultaban de las conquistas, estaban forzosamente estratificados; las clases eran inevitables, y las luchas de clases sociales siempre se han definido de forma selectiva.
\vs p071 1:3 \pc Las tribus norteñas de hombres rojos americanos nunca lograron una verdadera estructura de Estado. Nunca avanzaron más allá de una confederación poco compacta de tribus, una forma de Estado muy primitiva. La que más cerca estuvo de configurarse como tal fue la federación iroquesa, pero este grupo de seis naciones nunca llegó a ejercer del todo la función de Estado, y no logró subsistir a causa de la ausencia de ciertos elementos esenciales para la vida nacional moderna como los que se indican a continuación:
\vs p071 1:4 \li{1.}La adquisición y herencia de la propiedad privada.
\vs p071 1:5 \li{2.}La existencia de ciudades además de la agricultura y la industria.
\vs p071 1:6 \li{3.}Animales domésticos de utilidad.
\vs p071 1:7 \li{4.}Una organización familiar efectiva. Los hombres rojos se ceñían a la familia materna y a la herencia del sobrino.
\vs p071 1:8 \li{5.}Un territorio concreto.
\vs p071 1:9 \li{6.}Un mandatario jefe fuerte.
\vs p071 1:10 \li{7.}Esclavización de los cautivos ---o los adoptaban o los masacraban---.
\vs p071 1:11 \li{8.}Conquistas firmes.
\vs p071 1:12 \pc Los hombres rojos eran demasiado democráticos; tenían un buen gobierno, pero fracasó. Con el tiempo habrían desarrollado un Estado, si no hubiesen tropezado de forma prematura con la civilización más avanzada del hombre blanco, que utilizaba las modalidades de gobierno de los griegos y los romanos.
\vs p071 1:13 \pc El éxito del Estado romano se basó en:
\vs p071 1:14 \li{1.}La familia paterna.
\vs p071 1:15 \li{2.}La agricultura y la domesticación de animales.
\vs p071 1:16 \li{3.}La concentración de la población ---las ciudades---.
\vs p071 1:17 \li{4.}La propiedad y la tierra privadas.
\vs p071 1:18 \li{5.}La esclavitud ---las clases de ciudadanía---.
\vs p071 1:19 \li{6.}La conquista y reorganización de los pueblos débiles y atrasados.
\vs p071 1:20 \li{7.}Un territorio concreto con carreteras.
\vs p071 1:21 \li{8.}Gobernantes individuales y fuertes.
\vs p071 1:22 \pc La gran debilidad de la civilización romana, y un factor que contribuyó al definitivo desmoronamiento del Imperio, fue la medida, supuestamente reformadora y avanzada, de emancipar a los jóvenes a los veintiún años de edad y de liberar incondicionalmente a las jóvenes para que pudieran casarse con el hombre que eligiesen o ir a otros lugares para darse a la inmoralidad. El daño ocasionado a la sociedad no radicó en estas reformas por sí mismas, sino más bien en la forma repentina y generalizada en la que se adoptaron. La caída de Roma evidencia lo que cabe esperar cuando un Estado se ve sometido a una expansión demasiado rápida acompañada de una decadencia interna.
\vs p071 1:23 \pc El Estado embrionario fue posible gracias al declive de los vínculos de sangre en favor de los territoriales; y, por lo general, estas federaciones tribales se consolidaron firmemente mediante las conquistas. Aunque, la principal característica del verdadero Estado es una soberanía que trasciende a todas las irrelevantes luchas y discrepancias entre los grupos, todavía persisten en las organizaciones estatales posteriores, como remanentes del pasado, numerosas clases y castas. Los últimos y mayores Estados territoriales sostuvieron prolongados y duros enfrentamientos con estos grupos de clanes consanguíneos más pequeños, resultando de esto la valiosa aportación del gobierno tribal a la transición de la autoridad familiar a la estatal. Algún tiempo después, de los gremios y de otras asociaciones industriales surgieron muchos clanes.
\vs p071 1:24 El fracaso de la integración del Estado da lugar al retroceso a las condiciones de los métodos de gobierno previos al Estado mismo, tal como el feudalismo de la Europa de la Edad Media. Durante estas épocas de oscurantismo, el Estado territorial colapsó y hubo una vuelta a los grupos pequeños de castillos, a la reaparición de las etapas de desarrollo del clan y la tribu. Existen todavía ahora en Asia y África semiestados, aunque no todos son producto de retrocesos evolutivos; muchos de ellos son núcleos embrionarios de los Estados del futuro.
\usection{2. EVOLUCIÓN DEL GOBIERNO REPRESENTATIVO}
\vs p071 2:1 La democracia, a pesar de ser un ideal, es fruto de la civilización, no de la evolución. ¡Id despacio!, ¡elegid cuidadosamente!, porque los peligros de la democracia son:
\vs p071 2:2 \li{1.}La glorificación de la mediocridad.
\vs p071 2:3 \li{2.}La elección de gobernantes abyectos e ignorantes.
\vs p071 2:4 \li{3.}La falta de reconocimiento de los hechos esenciales de la evolución social.
\vs p071 2:5 \li{4.}El peligro del sufragio universal en manos de mayorías sin formación e indolentes.
\vs p071 2:6 \li{5.}La esclavitud a la opinión pública; la mayoría no siempre tiene la razón.
\vs p071 2:7 \pc La opinión pública, la opinión común, siempre ha retrasado la sociedad; no obstante, es valiosa porque, aunque demora la evolución social, preserva de hecho la civilización. La educación de la opinión pública es el único método seguro y efectivo de acelerar la civilización; la fuerza es solamente un recurso temporal, y el desarrollo cultural se agilizará cada vez más conforme las balas den paso a las votaciones. La opinión pública, las costumbres, es el principio generador básico y elemental de la evolución social y el desarrollo del Estado, pero, para que tenga valor de Estado, su expresión no ha de ser violenta.
\vs p071 2:8 La medida de avance de la sociedad se determina directamente por el grado en el que la opinión pública puede regir el comportamiento personal y la reglamentación estatal sin recurrir a la violencia. El gobierno realmente civilizado apareció cuando se invistió a la opinión pública con las competencias del sufragio personal. Las elecciones populares pueden no tomar siempre decisiones acertadas, pero sí representan la forma correcta incluso de equivocarse. La evolución no produce de inmediato una perfección en grado superlativo, sino más bien una adaptación práctica relativa y progresiva.
\vs p071 2:9 \pc Existen diez pasos, o etapas, que conducen a la evolución de un gobierno representativo práctico y eficiente, a saber:
\vs p071 2:10 \li{1.}\bibemph{La libertad de la persona}. La esclavitud, la servidumbre y toda forma de cautiverio humano han de desaparecer.
\vs p071 2:11 \li{2.}\bibemph{La libertad de la mente}. A no ser que se eduque a un pueblo libre ---que se le enseñe a pensar de forma inteligente y a planificar con sabiduría---, la libertad normalmente hace más mal que bien.
\vs p071 2:12 \li{3.}\bibemph{El imperio de la ley}. Solo se puede disfrutar de la libertad cuando la voluntad y los caprichos de los dirigentes humanos se reemplazan por disposiciones legislativas en conformidad con unas leyes fundamentales aceptadas.
\vs p071 2:13 \li{4.}\bibemph{La libertad de expresión}. Es inconcebible un gobierno representativo sin que las aspiraciones y las opiniones humanas puedan expresarse con entera libertad.
\vs p071 2:14 \li{5.}\bibemph{La seguridad de la propiedad}. Ningún gobierno persiste por mucho tiempo si no prevé el derecho a alguna manera de disfrute de la propiedad personal. El hombre ambiciona tener derecho a usar, gestionar, otorgar, vender, arrendar y legar su propiedad personal.
\vs p071 2:15 \li{6.}\bibemph{El derecho de petición}. Un gobierno representativo asume el derecho de los ciudadanos a ser escuchados. La prerrogativa de petición es consustancial a la ciudadanía libre.
\vs p071 2:16 \li{7.}\bibemph{El derecho a gobernar}. No basta con ser escuchados; el poder de la petición debe avanzar hasta la gestión misma del gobierno.
\vs p071 2:17 \li{8.}\bibemph{El sufragio universal}. El gobierno representativo implica un electorado inteligente, eficiente y universal. La naturaleza de este tipo de gobierno siempre estará determinada por el carácter y la aptitud de aquellos que lo forman. Con el progreso de la civilización, el sufragio, siempre que permanezca universal para ambos sexos, será debidamente modificado, reagrupado y diferenciado de alguna manera.
\vs p071 2:18 \li{9.}\bibemph{El control de los servidores públicos}. Ningún gobierno civil será útil y eficaz sin que la ciudadanía posea y utilice métodos sensatos de guiar y controlar a los cargos y servidores públicos.
\vs p071 2:19 \li{10.}\bibemph{Una representación inteligente y capacitada}. La supervivencia de la democracia depende del éxito del gobierno representativo; y esto está supeditado a la práctica de elegir para cargos públicos solamente a aquellas personas que estén específicamente formadas, que sean intelectualmente competentes, socialmente leales y moralmente idóneas. Únicamente siguiendo estas disposiciones podrá subsistir un gobierno del pueblo, por el pueblo y para el pueblo.
\usection{3. LOS IDEALES DEL ESTADO}
\vs p071 3:1 La forma política o administrativa de un gobierno es irrelevante siempre que proporcione los elementos esenciales del progreso civil: la libertad, la seguridad, la educación y la coordinación social. No se trata de lo que el Estado es sino de lo que hace para influenciar el curso de la evolución social. Y, al fin y al cabo, ningún Estado puede estar por encima de los valores morales de su ciudadanía, tal como sus líderes elegidos ejemplifican. La ignorancia y el egoísmo aseguran la caída hasta del tipo más elevado de gobierno.
\vs p071 3:2 Por mucho que haya que lamentar, el egoísmo nacional ha sido esencial para la supervivencia social. La doctrina del pueblo elegido ha sido, hasta las épocas modernas, un factor fundamental en la unión de las tribus y en la formación de las naciones. Pero ningún Estado puede alcanzar niveles ideales de operatividad hasta que no logre vencer todas las formas de intolerancia; la cual es permanentemente hostil al progreso humano. Y la mejor forma de combatir la intolerancia es mediante la coordinación de la ciencia, el comercio, el entretenimiento y la religión.
\vs p071 3:3 \pc El Estado ideal opera bajo el impulso de tres fuerzas coordinadas y poderosas:
\vs p071 3:4 \li{1.}La lealtad y el cariño derivados de la realización de la hermandad de los hombres.
\vs p071 3:5 \li{2.}El patriotismo inteligente basado en ideales profundos.
\vs p071 3:6 \li{3.}La percepción cósmica interpretada en función de los hechos, las necesidades y los objetivos planetarios.
\vs p071 3:7 \pc Las leyes del Estado ideal son poco numerosas, y han dejado a un lado la era negativa de los tabúes para entrar en la era del progreso positivo de la libertad individual como consecuencia de un mejor autocontrol. El Estado de elevado rango no solo compele a sus ciudadanos a trabajar, sino que también los incita a usar de forma provechosa y edificante su creciente tiempo libre, que resulta de su liberación del agotador trabajo gracias al avance de la era de las máquinas. El ocio ha de producir al igual que consumir.
\vs p071 3:8 Ninguna sociedad llega demasiado lejos en su desarrollo si permite la haraganería o tolera la pobreza. Pero nunca se podrá erradicar la pobreza y la dependencia si se apoya sin reservas a linajes deficientes y en declive degenerativo, y se les permite reproducirse sin restricción.
\vs p071 3:9 Una sociedad moral debe aspirar a salvaguardar la autoestima de su ciudadanía y a proporcionar a cualquier persona normal suficientes oportunidades para su realización personal. Un plan de logro social de tal índole daría como resultado una sociedad cultural de orden superior. La evolución social se debería alentar por medio de una supervisión de parte del gobierno que ejerza un mínimo de control de regulación. El mejor Estado es aquel que coordina más y gobierna menos.
\vs p071 3:10 Los ideales del Estado han de alcanzarse mediante la evolución, mediante el crecimiento lento de la conciencia cívica, esto es, el reconocimiento de la obligación y el privilegio del servicio social. Tras el fin de la administración de políticos con afán de lucro, los hombres asumen en un principio las cargas del gobierno como un deber, pero más tarde buscan este servicio como un privilegio, como el más grande de los honores. La naturaleza de los ciudadanos que se ofrecen voluntariamente para aceptar las responsabilidades del Estado refleja fielmente la categoría de cualquier nivel de civilización.
\vs p071 3:11 En una verdadera mancomunidad, la tarea de gobernar ciudades y provincias se efectúa de la mano de los expertos y se lleva a cabo como cualquier otra forma de asociaciones comerciales y económicas de personas.
\vs p071 3:12 En los Estados avanzados, el servicio político se precia como la mayor entrega que la ciudadanía puede dar. La más grande aspiración de los ciudadanos de mayor sabiduría y nobleza es lograr el reconocimiento civil, ser elegidos o designados para algún puesto de confianza en el gobierno, y estos gobiernos otorgan sus máximos honores en reconocimiento por los servicios prestados a sus servidores civiles y sociales. A continuación se conceden honores, en el orden que se menciona, a los filósofos, a los educadores, a los científicos, a los industriales y a los militares. Los padres son debidamente recompensados mediante la excelencia de sus hijos; los líderes puramente religiosos, al ser embajadores del reino espiritual, reciben su verdadera recompensa en otro mundo.
\usection{4. LA CIVILIZACIÓN EN PROGRESO}
\vs p071 4:1 La economía, la sociedad y el gobierno deben evolucionar si han de perdurar. El estancamiento en un mundo evolutivo es sintomático de decadencia; solo persisten aquellas instituciones que avanzan siguiendo la corriente evolutiva.
\vs p071 4:2 \pc El programa de una civilización en avance y expansión incluye:
\vs p071 4:3 \li{1.}El mantenimiento de las libertades individuales.
\vs p071 4:4 \li{2.}La protección del hogar.
\vs p071 4:5 \li{3.}El fomento de la seguridad económica.
\vs p071 4:6 \li{4.}La prevención de las enfermedades.
\vs p071 4:7 \li{5.}La educación obligatoria.
\vs p071 4:8 \li{6.}El empleo obligatorio.
\vs p071 4:9 \li{7.}El uso provechoso del tiempo libre.
\vs p071 4:10 \li{8.}La asistencia a los desfavorecidos.
\vs p071 4:11 \li{9.}La mejora de la raza humana.
\vs p071 4:12 \li{10.}La promoción de las ciencias y las artes.
\vs p071 4:13 \li{11.}La promoción de la filosofía: la sabiduría
\vs p071 4:14 \li{12.}El aumento de la percepción cósmica: la espiritualidad.
\vs p071 4:15 \pc Y este progreso en las artes de la civilización lleva directamente a la realización, de parte de los tenaces mortales, de los objetivos humanos y divinos más elevados ---la consecución social de la hermandad del hombre y la condición personal de ser conscientes de Dios, algo que se manifiesta en el supremo deseo de toda persona de hacer la voluntad del Padre de los cielos---.
\vs p071 4:16 La aparición de la auténtica fraternidad significa que ha llegado un orden social en el que todos los hombres se complacen en llevar las cargas de los demás; desean realmente practicar la regla de oro. Pero esta sociedad ideal no se puede llevar a cabo mientras que el débil o el malvado estén al acecho para sacar ventaja de manera injusta y nefasta de aquellos que se sienten principalmente movidos por su dedicación al servicio de la verdad, la belleza y la bondad. En una situación así solo existe un camino viable: los seguidores de la “regla de oro” pueden establecer una sociedad progresiva en la que puedan vivir de acuerdo con sus ideales, manteniendo, al mismo tiempo, una conveniente defensa contra aquellos congéneres sumidos en la ignorancia, que podrían tratar o bien de sacar provecho de sus inclinaciones pacíficas o de destruir su civilización en avance.
\vs p071 4:17 El idealismo nunca puede sobrevivir en un planeta en evolución si los idealistas de cada generación se dejan exterminar por los grupos más innobles de la humanidad. La gran prueba del idealismo es la siguiente: ¿Puede una sociedad avanzada mantener unos preparativos militares que la proteja de todos los ataques de sus belicosos vecinos, sin caer en la tentación de emplear esta fuerza militar en operaciones ofensivas contra otros pueblos para su propio provecho o engrandecimiento nacional? La supervivencia nacional exige un estado de preparación, y solo el idealismo religioso puede impedir que tal preparación se prostituya y se convierta en agresión. Solo el amor, la fraternidad, puede impedir que los fuertes opriman a los débiles.
\usection{5. EVOLUCIÓN DE LA COMPETITIVIDAD}
\vs p071 5:1 La competitividad es esencial para el progreso social, pero si no está regulada genera violencia. En la sociedad actual, la competitividad va desplazando lentamente a la guerra en el sentido de que determina el lugar de la persona en el mercado laboral, y así decreta la supervivencia de la industria. (El asesinato y la guerra difieren en su estatus ante las costumbres; el asesinato se prohibió desde los primeros tiempos de la sociedad; mientras que las guerras no están todavía prohibidas por el conjunto de la humanidad.)
\vs p071 5:2 El Estado ideal emprende la regulación de la conducta social solo lo suficiente como para acabar con la violencia de la competitividad individual y evitar la injusticia en la iniciativa personal. He aquí el gran problema del Estado: ¿Cómo se puede garantizar la paz y la calma en la industria, pagar los impuestos para apoyar al poder estatal y, al mismo tiempo, evitar que la tributación obstaculice la industria e impedir que el Estado se convierta en parasitario o tiránico?
\vs p071 5:3 A lo largo de las eras primitivas de cualquier mundo, la competitividad es fundamental en una civilización en progreso. A medida que la evolución del hombre avanza, la cooperación se hace cada vez más efectiva. En las civilizaciones avanzadas, la cooperación es más eficaz que la competitividad. La competitividad estimula al hombre primitivo. La evolución primitiva se caracteriza por la supervivencia de seres biológicamente aptos, pero las civilizaciones siguientes reciben un mayor impulso a través de la cooperación inteligente, la fraternidad comprensiva y la hermandad espiritual.
\vs p071 5:4 Es cierto que la competitividad es sumamente infructuosa y altamente ineficaz, pero no se debe favorecer ningún intento por erradicar esta duplicación innecesaria del esfuerzo, si ello comporta incluso la más leve supresión de cualquiera de las libertades individuales básicas.
\usection{6. EL ÁNIMO DE LUCRO}
\vs p071 6:1 En nuestros días, una economía motivada por el ánimo de lucro está abocada al fracaso, a no ser que adquiera una dimensión de servicio. Una competencia despiadada cuya base radique en un interés personal de estrechas miras destruye, en última instancia, todo aquello que trata de mantener. El afán exclusivo de lucro e interés propio es incompatible con los ideales cristianos, y lo es mucho más con las enseñanzas de Jesús.
\vs p071 6:2 En la economía, el ánimo de lucro es para la motivación al servicio lo que el temor es para el amor en la religión. Pero el afán de lucro no debe eliminarse o erradicarse de forma repentina; mantiene trabajando arduamente a numerosos mortales que de otra forma se darían a la holgazanería. No es preciso, sin embargo, que los objetivos de este impulsor del vigor social tengan siempre un carácter interesado.
\vs p071 6:3 El ánimo de lucro en las actividades económicas es deleznable por completo y totalmente indigno de un orden avanzado de sociedad; no obstante, es un factor indispensable durante las primeras fases de la civilización. No se debe despojar a los hombres de esta motivación al beneficio propio hasta que no posean claramente unos fines sin ánimo de lucro, de orden superior, en sus aspiraciones económicas y servicio social ---el impulso supremo que parte de una sabiduría en grado sumo, de una fascinante hermandad y de un magnífico logro espiritual---.
\usection{7. LA EDUCACIÓN}
\vs p071 7:1 El Estado perdurable se fundamenta en la cultura, se rige por los ideales y está motivado por el servicio. El propósito de la educación debe ser la adquisición de destrezas, la búsqueda de la sabiduría, la realización del yo y el logro de valores espirituales.
\vs p071 7:2 En el Estado ideal, la educación continúa durante toda la vida, y la filosofía se convierte en algún momento en el principal afán de sus ciudadanos. Los ciudadanos de este orden de comunidad emprenden la búsqueda de la sabiduría para mejorar su percepción del sentido de las relaciones humanas, de los contenidos de la realidad, de la nobleza de los valores, de los objetivos de la vida y de las glorias del destino cósmico.
\vs p071 7:3 Los urantianos deben adquirir el concepto de una sociedad cultural nueva y superior. La educación dará el salto a unos nuevos niveles de valor cuando desaparezca el sistema económico exclusivamente motivado por el afán de lucro. Durante demasiado tiempo, la educación ha sido localista, militarista, ha exaltado el ego y procurado el éxito; con el tiempo, se ha de convertir en global, idealista, impulsora del desarrollo personal y la aprehensión cósmica.
\vs p071 7:4 No hace mucho que la educación pasó del control del clero al de los abogados y de los hombres de negocios. Algún día se deberá entregar a los filósofos y a los científicos. Los maestros deben ser seres libres, verdaderos líderes, con el fin de que la filosofía, la búsqueda de la sabiduría, pueda convertirse en el principal propósito de la educación.
\vs p071 7:5 La educación es una tarea de por vida, y ha de continuar a lo largo de toda ella para que la humanidad pueda, de forma gradual, experimentar los niveles ascendentes de la sabiduría humana, los cuales son:
\vs p071 7:6 \li{1.}El conocimiento de las cosas.
\vs p071 7:7 \li{2.}La comprensión de los contenidos.
\vs p071 7:8 \li{3.}La apreciación de los valores.
\vs p071 7:9 \li{4.}La nobleza del trabajo: el deber.
\vs p071 7:10 \li{5.}La motivación de los objetivos: la moralidad.
\vs p071 7:11 \li{6.}El amor al servicio: el carácter.
\vs p071 7:12 \li{7.}La percepción cósmica: la percepción espiritual.
\vs p071 7:13 \pc Y entonces, por medio de estos logros, muchos ascenderán hasta la postrera consecución mental humana: ser conscientes de Dios.
\usection{8. NATURALEZA DEL ESTADO}
\vs p071 8:1 El único rasgo sagrado de cualquier gobierno humano es la división del Estado en ámbitos: la función ejecutiva, la legislativa y la judicial. El universo se administra de acuerdo con este sistema de separación de funciones y de autoridad. Al margen de este concepto divino en cuanto a una eficiente regulación social o gobierno civil, poco importa qué forma de Estado pueda elegir el pueblo a condición de que la ciudadanía continúe progresando hacia la meta de un mayor autocontrol y servicio social. La percepción intelectual, el juicio económico, la habilidad social y la fuerza moral de un pueblo se reflejan todas fielmente en el Estado.
\vs p071 8:2 La evolución del Estado conlleva el progreso de un nivel a otro, tal como se indica a continuación:
\vs p071 8:3 \li{1.}La creación de un gobierno triple compuesto de las ramas ejecutiva, legislativa y judicial.
\vs p071 8:4 \li{2.}La libertad de las actividades sociales, políticas y religiosas.
\vs p071 8:5 \li{3.}La abolición de todas las formas de esclavitud y cautiverio humanos.
\vs p071 8:6 \li{4.}La posibilidad de la ciudadanía de controlar la recaudación de impuestos.
\vs p071 8:7 \li{5.}El establecimiento de una educación universal: prolongación del aprendizaje desde la cuna hasta la tumba.
\vs p071 8:8 \li{6.}El ajuste adecuado entre los gobiernos locales y el nacional.
\vs p071 8:9 \li{7.}El fomento de la ciencia y la victoria sobre las enfermedades.
\vs p071 8:10 \li{8.}El debido reconocimiento de la igualdad de los sexos y el funcionamiento cooperativo de hombres y mujeres en el hogar, en la escuela y en la Iglesia, con servicios especializados de las mujeres en la industria y en el gobierno.
\vs p071 8:11 \li{9.}La eliminación del trabajo duro esclavizante por la invención de las máquinas y el consecuente dominio de la mecanización.
\vs p071 8:12 \li{10.}La conquista de los dialectos: el triunfo de una lengua universal.
\vs p071 8:13 \li{11.}El fin de las guerras: resolución internacional de las diferencias nacionales y raciales por parte de los tribunales continentales de las naciones presididos por un tribunal supremo planetario, que se reclute exclusivamente de los presidentes, jubilados periódicamente, de los órganos jurisdiccionales continentales. Los tribunales continentales son potestativos; la corte mundial tiene carácter consultivo: moral.
\vs p071 8:14 \li{12.}La tendencia mundial de la búsqueda de la sabiduría: la exaltación de la filosofía. La evolución de una religión mundial, que anunciará la entrada del planeta en las tempranas fases de su asentamiento en luz y vida.
\vs p071 8:15 \pc Estas son las condiciones previas de un gobierno de carácter progresivo y las características distintivas del Estado ideal. Urantia está lejos de la consecución de estos elevados ideales, pero las razas civilizadas han comenzado el camino ---la humanidad avanza hacia destinos evolutivos de un orden superior---.
\vsetoff
\vs p071 8:16 [Auspiciado por un Melquisedec de Nebadón.]
