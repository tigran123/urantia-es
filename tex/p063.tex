\upaper{63}{La primera familia humana}
\author{Portador de vida}
\vs p063 0:1 Urantia quedó registrada como mundo habitado cuando los dos primeros seres humanos ---los gemelos--- tenían once años, y antes de que se hubiesen convertido en los padres del primogénito de la segunda generación de auténticos seres humanos. El mensaje del arcángel, remitido desde Lugar de Salvación, en esta ocasión de reconocimiento formal del planeta, acababa con estas palabras:
\vs p063 0:2 “La mente humana ha aparecido en el planeta 606 de Satania, y a estos padres de la nueva raza se les pondrá por nombre \bibemph{Andón} y \bibemph{Fonta}. Todos los arcángeles oran para que prontamente se pueda dotar a estas criaturas con la inhabitación personal del don del espíritu del Padre Universal”.
\vs p063 0:3 \pc Andón es un nombre originario de Nebadón que significa “la primera criatura semejante al Padre en mostrar sed humana de perfección”. Fonta significa “la primera criatura semejante al Hijo en mostrar sed humana de perfección”. Andón y Fonta nunca tuvieron conocimiento de estos nombres hasta que se les dio en el momento de fusionarse con sus modeladores del pensamiento. Durante toda su estancia como mortales en Urantia el uno al otro se llamaban Sonta\hyp{}an y Sonta\hyp{}en; Sonta\hyp{}an significa “amado por la madre” y, Sonta\hyp{}en, “amado por el padre”. Ellos mismos se pusieron estos nombres y sus significados denotan el respeto y el cariño mutuo que se profesaban.
\usection{1. ANDÓN Y FONTA}
\vs p063 1:1 En muchos aspectos, Andón y Fonta fueron la pareja de seres humanos más extraordinaria que jamás ha vivido sobre la faz de la Tierra. Estos dos magníficos seres, los verdaderos padres de toda la humanidad, fueron superiores en todos los sentidos a muchos de sus descendientes inmediatos, y radicalmente diferentes a todos sus ancestros, tanto próximos como lejanos.
\vs p063 1:2 Los padres de esta primera pareja humana eran al parecer poco diferentes del promedio de su tribu, aunque figuraban entre sus miembros más inteligentes; fue el primer grupo que aprendió a lanzar piedras y a emplear palos en sus peleas. También utilizaban espículas puntiagudas de piedra, sílex y hueso.
\vs p063 1:3 Mientras aún vivía con sus padres, Andón, usando tendones de animales, había atado un trozo afilado de sílex en el extremo de un palo, y al menos en doce ocasiones hizo un buen uso de esta arma para salvar su vida y la de su hermana, igualmente intrépida y curiosa, que constantemente lo acompañaba en todos sus viajes de exploración.
\vs p063 1:4 La decisión de Andón y Fonta de huir de la tribu de los primates supone una capacidad mental muy por encima de la inteligencia de orden inferior que caracterizaba a tantos de sus descendientes posteriores, los cuales se rebajaron para aparearse con sus primos retrasados de las tribus simias. Pero el sentimiento vago de ser algo más que meros animales se debía a que poseían un ser personal, y se incrementaba por la presencia interior de los modeladores del pensamiento.
\usection{2. HUIDA DE LOS GEMELOS}
\vs p063 2:1 Tras haber decidido huir hacia el norte, Andón y Fonta sucumbieron a sus temores durante algún tiempo, en particular al temor de disgustar a su padre y a su familia cercana. Imaginaron que serían atacados por parientes hostiles y admitieron, pues, la posibilidad de encontrar la muerte a manos de los miembros de su tribu, ya resentidos con ellos. De pequeños, los gemelos habían pasado la mayor parte de su tiempo en compañía el uno del otro y, por dicha razón, nunca habían contado con demasiado afecto entre sus primos animales de la tribu de los primates. Tampoco el hecho de haber construido su habitáculo arbóreo separado y muy superior había mejorado su prestigio en la tribu.
\vs p063 2:2 Y fue en este nuevo habitáculo entre las copas de los árboles, una noche, tras haber sido despertados por una violenta tormenta y encontrarse abrazados en el temor y el afecto, cuando tomaron final y enteramente la decisión de huir de aquel entorno tribal y de su vivienda arbórea.
\vs p063 2:3 Ya habían preparado un rudimentario refugio en la copa de un árbol a una media jornada de camino hacia el norte. Era su escondite secreto y seguro para pasar el primer día lejos de sus bosques natales. A pesar de que los gemelos compartían con los primates el mismo miedo fatal a permanecer en el suelo durante la noche, se atrevieron a salir en dirección norte poco antes del anochecer. Aunque se precisaba un valor fuera de lo común para emprender este viaje nocturno, incluso con luna llena, comprendieron acertadamente que así era menos probable que sus parientes y los miembros de su tribu les echaran de menos y los persiguieran. Y, poco después de la medianoche, lograron llegar de forma segura al emplazamiento fijado de antemano.
\vs p063 2:4 En su viaje al norte, hallaron un yacimiento de sílex al descubierto y, al encontrar muchas piedras con formas idóneas para distintos usos, se abastecieron de ellas para el futuro. Cuando Andón intentó astillar estas piedras de sílex y adaptarlas mejor para determinados fines, descubrió que producían chispas y concibió la idea de hacer fuego. Pero esta idea no se afianzó en él con firmeza en aquel momento; el clima era todavía saludable y había poca necesidad de fuego.
\vs p063 2:5 Pero el sol del otoño descendía cada vez más en el cielo y, a medida que se dirigían al norte, las noches eran más y más frías. Ya se habían visto obligados a hacer uso de pieles de animales para abrigarse. Antes de llevar una luna lejos de su tierra natal, Andón indicó a su compañera que creía que podía hacer fuego con el sílex. Durante dos meses, intentaron usar la chispa de esta piedra para lograrlo, pero fracasaron. Todos los días, esta pareja golpeaba las piedras de sílex y trataba de prenderle fuego a la madera. Por fin una tarde, hacia la puesta de sol, la clave para lograrlo se les desveló cuando se le ocurrió a Fonta escalar un árbol cercano para conseguir un nido de pájaro abandonado. El nido estaba seco y era sumamente inflamable, y se encendió pues con una abundante llama en cuanto la chispa cayó sobre él. Quedaron tan sorprendidos y asustados de su éxito que casi perdieron el fuego, pero lo salvaron añadiendo el combustible adecuado. Comenzó entonces la primera búsqueda de leña por parte de los padres de toda la humanidad.
\vs p063 2:6 Aquel fue uno de los momentos más felices de su corta pero agitada vida. Pasaron toda la noche despiertos viendo arder su fuego, vagamente percatándose de que habían hecho un descubrimiento que les permitiría desafiar el clima y ser, por tanto, independientes para siempre de sus parientes animales de las tierras del sur. Tras tres días descansando y disfrutando del fuego, prosiguieron su viaje.
\vs p063 2:7 A menudo, los ancestros primates de Andón habían avivado el fuego que los rayos encendían, pero nunca antes las criaturas de la tierra habían tenido un método de hacer fuego a voluntad. Si bien, tuvo que pasar mucho tiempo antes de que los gemelos supieran que, para prenderlo, el musgo seco y otros materiales servían de igual manera que los nidos de los pájaros.
\usection{3. LA FAMILIA DE ANDÓN}
\vs p063 3:1 Habían transcurrido casi dos años desde la noche en que los gemelos partieron de su hogar cuando nació su primer hijo. Le pusieron de nombre Sontad. Él fue la primera criatura nacida en Urantia a la que se le envolvió en ropas de abrigo en el momento de nacer. Comenzaba la raza humana y, con este nuevo desarrollo evolutivo, apareció el instinto de cuidar convenientemente a los cada vez más débiles bebés, algo que caracterizaba el desarrollo gradual de una mente de índole intelectual a diferencia de una más puramente animal.
\vs p063 3:2 Andón y Fonta tuvieron un total de diecinueve hijos, y vivieron para gozar de la compañía de casi cincuenta nietos y media docena de biznietos. La familia residía en cuatro refugios rocosos contiguos, o semicuevas, de las que tres estaban interconectadas por medio de corredores, que se habían excavado en la caliza blanda con utensilios de sílex ideados por los hijos de Andón.
\vs p063 3:3 Estos primeros andonitas mostraban un espíritu de clan muy acentuado; cazaban en grupo y jamás se alejaban demasiado del entorno en el que vivían. Parecían darse cuenta de que eran un grupo de seres vivos aislado y excepcional y que debían, por consiguiente, evitar llegar a separarse. Este sentimiento de estrecho parentesco respondía, sin duda, al enaltecido ministerio mental de los espíritus asistentes.
\vs p063 3:4 \pc Andón y Fonta trabajaron continuamente para alimentar y mejorar el clan. Ambos murieron a la edad de cuarenta y dos años víctimas de un terremoto, al desprenderse un saliente rocoso. Cinco hijos suyos y once nietos perecieron con ellos, y casi una veintena de sus descendientes sufrieron serias lesiones.
\vs p063 3:5 A la muerte de sus padres, Sontad, pese a tener un pie gravemente herido, asumió de inmediato el liderazgo del clan, contando para ello con la eficaz ayuda de su mujer, la mayor de sus hermanas. Su primera tarea fue subir rodando unas piedras para dar debida sepultura a sus padres, hermanos, hermanas e hijos muertos. No se le debe otorgar un significado injustificado a este acto de enterramiento. Sus ideas sobre la supervivencia tras la muerte eran muy vagas e indefinidas, en gran parte extraídas de su fantástica y abigarrada vida onírica.
\vs p063 3:6 \pc La familia de Andón y Fonta se mantuvo unida hasta la vigésima generación, cuando la lucha por el alimento junto a las desavenencias sociales provocaron el comienzo de su dispersión.
\usection{4. LOS CLANES ANDÓNICOS}
\vs p063 4:1 El hombre primitivo ---los andonitas--- tenía los ojos negros y la tez morena; era algo así como una mezcla entre la raza amarilla y la roja. La melanina es una sustancia colorante que se encuentra en la piel de todos los seres humanos. Es el pigmento original de la piel andónica. En su aspecto general y en el color de la piel, estos primeros andonitas eran más parecidos a los esquimales de hoy día que a ningún otro tipo de seres humanos vivos. Fueron las primeras criaturas en usar la piel de los animales para protegerse del frío; tenían algo más de pelo en el cuerpo que los humanos actuales.
\vs p063 4:2 La vida tribal de los ancestros animales de estos primeros hombres prefiguraba los principios de múltiples convenciones sociales y, con el desenvolvimiento de las emociones y el incremento de la capacidad cerebral de estos seres, se produjo un desarrollo inmediato de la organización social y una nueva división de la labor en el clan. Eran extremadamente imitativos, pero su instinto lúdico estaba apenas desarrollado y su sentido del humor era casi inexistente. El hombre primitivo sonreía ocasionalmente, pero nunca se prodigaba en risas. El humor fue un legado de la posterior raza adámica. Estos primeros seres humanos no eran tan sensibles al dolor ni tan reactivos a las situaciones desagradables como muchos de los mortales evolutivos que les seguirían. El parto no representaba ninguna experiencia dolorosa o angustiante ni para Fonta ni para su progenie inmediata.
\vs p063 4:3 \pc Constituían una magnífica tribu. Los varones luchaban heroicamente por la seguridad de sus compañeras y de su prole; las mujeres se dedicaban cariñosamente a sus hijos. Si bien, su sentido de patria se limitaba enteramente a su clan inmediato. Eran muy leales a sus familias y estaban dispuestos a morir sin dudarlo para defender a sus hijos, pero eran incapaces de concebir la idea de hacer un mundo mejor para sus nietos. El altruismo aún no había nacido en el corazón humano, pese a que todas las emociones esenciales para el nacimiento de la religión estaban ya presentes en estos aborígenes de Urantia.
\vs p063 4:4 Estos primeros hombres profesaban un afecto entrañable hacia sus compañeros y tenían, indudablemente, un concepto real, pero rudimentario, de la amistad. En épocas más tardías, durante sus constantes y recurrentes batallas contra las tribus inferiores, era común ver a uno de estos hombres primitivos luchar valientemente con una mano mientras que seguía combatiendo, intentando proteger y salvar a un compañero de guerra herido. Muchos de los rasgos más nobles y sumamente humanos presentes en el desarrollo evolutivo posterior se prefiguran emotivamente en estos pueblos primitivos.
\vs p063 4:5 \pc El clan andónico original mantuvo una línea ininterrumpida de líderes hasta la vigésimo séptima generación, en la que, al no aparecer ningún vástago varón entre los descendientes directos de Sontad, dos miembros rivales del clan, aspirantes a su liderazgo, se ensalzaron en una lucha por la supremacía.
\vs p063 4:6 Antes de la extensa dispersión de los clanes andónicos, había llegado a evolucionar un lenguaje bien elaborado a partir de sus primeros intentos por comunicarse entre ellos. Este lenguaje continuó desarrollándose y recibía aportaciones casi diarias a causa de los nuevos inventos y de las adaptaciones al entorno que este pueblo activo, inquieto y curioso efectuaba. Y este lenguaje se convirtió en la expresión verbal de Urantia, en la lengua de la familia humana primitiva, hasta la posterior aparición de las razas de color.
\vs p063 4:7 \pc Con el transcurso del tiempo, los clanes andónicos incrementaron su número y el contacto entre estas familias en expansión provocó desavenencias y equívocos. Solo dos cosas ocupaban las mentes de estos pueblos: cazar para conseguir alimentos y pelear para vengarse de alguna injusticia o de algún insulto, real o supuesto, a manos de las tribus vecinas.
\vs p063 4:8 Las disputas familiares aumentaron, estallaron las guerras tribales y se produjeron graves pérdidas entre los mejores integrantes de los grupos más capaces y avanzados. Ciertas pérdidas fueron irreparables; se perdieron definitivamente para el mundo algunas de las más preciadas estirpes en cuanto a capacidad e inteligencia. Estas guerras constantes entre los clanes amenazaron con extinguir a esta temprana raza y a su civilización primitiva.
\vs p063 4:9 Es imposible persuadir a seres tan primitivos a que vivan juntos y en paz durante mucho tiempo. El hombre desciende de animales beligerantes y, cuando la gente ruda se relaciona estrechamente, se irrita y se ofende entre sí. Los portadores de vida conocen esta tendencia de las criaturas evolutivas y, en este sentido, disponen que los seres humanos en desarrollo se separen por lo menos en tres razas diferentes y apartes y, con mayor frecuencia, en seis.
\usection{5. LA DISPERSIÓN DE LOS ANDONITAS}
\vs p063 5:1 Las tempranas razas andonitas no se adentraron mucho en Asia, y, en un principio, no entraron en África. La geografía de aquellos tiempos las hizo dirigirse hacia el norte, y estos seres viajaron más y más hacia el norte, hasta que se vieron obstaculizados por el hielo del tercer glaciar, en su lento avance.
\vs p063 5:2 Antes de que esta extensa capa de hielo alcanzara Francia y las Islas Británicas, los descendientes de Andón y Fonta habían avanzado por Europa en dirección oeste y habían establecido más de mil asentamientos separados a lo largo de los grandes ríos que desembocaban en las entonces cálidas aguas del Mar del Norte.
\vs p063 5:3 Estas tribus andónicas fueron los primeros pobladores ribereños de Francia; vivieron a lo largo del río Somme durante decenas de miles de años. El Somme es el único río que quedó inalterado por los glaciares y, en aquellos días, confluía en el mar de forma similar a la de hoy en día. Esto explica por qué se encuentran tantas pruebas de la presencia de los descendientes andónicos por todo el trazado del valle fluvial.
\vs p063 5:4 Estos aborígenes de Urantia no eran arborícolas, aunque en situaciones de emergencia todavía se subían en las copas de los árboles. Vivían habitualmente a lo largo de los ríos bajo el cobijo de los salientes acantilados y en las grutas de las laderas, que les proporcionaban una buena visión de los caminos de acceso y les protegía de los elementos atmosféricos. Podían así disfrutar del confort de sus hogueras sin que el humo les causara demasiadas molestias. Tampoco eran realmente cavernícolas, aunque, en tiempos posteriores, las últimas capas de hielo, en su avance hacia el sur, empujaron a sus descendientes a las cuevas. Preferían acampar en el borde de los bosques y junto a los riachuelos.
\vs p063 5:5 Pronto se volvieron extraordinariamente habilidosos en camuflar sus habitáculos, parcialmente resguardados, y demostraron una gran pericia en la construcción de cabañas de piedra con forma de bóveda, que utilizaban como dormitorio, y a las que accedían por la noche a gatas. La entrada de estas cabañas se cerraba rodando una piedra hasta situarla delante de ella; se trataba de una piedra grande que se había colocado a tal fin en el interior antes de poner finalmente en su sitio las piedras del techo.
\vs p063 5:6 Los andonitas eran cazadores intrépidos y eficaces y, exceptuando bayas silvestres y algunos frutos de los árboles, únicamente se alimentaban de carne. Al igual que Andón, que había inventado el hacha de piedra, de la misma manera, sus descendientes descubrieron pronto la lanza y el arpón e hicieron un uso efectivo de ellos. Por fin una mente apta para crear herramientas actuaba conjuntamente con una mano capaz de aplicarlas, y estos primeros humanos se volvieron sumamente diestros en la fabricación de herramientas de sílex. Viajaban por todas partes buscándolo, prácticamente tal como los humanos de hoy día se trasladan hasta los confines de la tierra en busca de oro, platino y diamantes.
\vs p063 5:7 Y estas tribus andónicas manifestaron, de otras muchas maneras, un grado de inteligencia que sus descendientes retrógrados no lograron adquirir en medio millón de años, aunque de hecho una y otra vez redescubrieran distintos métodos de encender el fuego.
\usection{6. ONAGAR: EL PRIMER MAESTRO DE LA VERDAD}
\vs p063 6:1 A medida que se extendía la dispersión andónica, el grado cultural y espiritual de los clanes empeoró durante casi diez mil años hasta los días de Onagar, que asumió el liderazgo de estas tribus, trajo la paz entre ellas y, por primera vez, las guió a todas a la adoración de “El Dador del Aliento a hombres y a animales”.
\vs p063 6:2 \pc La filosofía de Andón había sido bastante confusa; casi se libró de convertirse en adorador del fuego por el gran bienestar que le procuraba su fortuito descubrimiento. Sin embargo, la razón lo desvió de su propio descubrimiento, orientándolo hacia el sol como fuente superior de luz y calor, inspiradora de un gran temor reverencial; pero el astro estaba demasiado distante, y Andón no llegó a convertirse en adorador del sol.
\vs p063 6:3 Pronto, los andonitas desarrollaron miedo a elementos atmosféricos como el trueno, el relámpago, la lluvia, la nieve, el granizo y el hielo. Pero el hambre era un estímulo constante y recurrente en esos tempranos días y, como subsistían mayormente gracias a los animales, acabaron por elaborar un culto de adoración hacia ellos. Para Andón, los animales más grandes, destinados a la alimentación, eran símbolos de fuerza creativa y de poder sustentante. Cada cierto tiempo, acostumbraban a designar a algunos de estos animales más grandes como objetos de adoración. Durante el auge en este sentido de un animal particular, dibujaban toscos esbozos de él en las paredes de las cuevas y, más tarde, a medida que las artes progresaban, se esculpían en este dios animal diversos ornamentos.
\vs p063 6:4 Muy pronto, los pueblos andónicos adquirieron la costumbre de abstenerse de comer la carne del animal objeto de veneración tribal. Al poco tiempo, para impresionar más convenientemente la mente de los jóvenes, elaboraron una ceremonia de veneración que se llevaba a cabo en torno al cuerpo de uno de estos animales reverenciados; todavía más tarde, esta celebración primitiva se convirtió en las ceremonias sacrificiales más complejas de sus descendientes. Y este es el origen de los sacrificios como parte del culto de adoración. Moisés elaboró esta idea en el ritual hebreo y el apóstol Pablo la conservó, en esencia, como la doctrina de la expiación de los pecados mediante el “derramamiento de sangre”.
\vs p063 6:5 El alimento era lo más importante en las vidas de estos seres humanos primitivos, tal como se muestra en la oración que Onagar, su gran maestro, enseñó a esta gente sencilla, y que decía así:
\vs p063 6:6 “Oh Aliento de la Vida, danos hoy el alimento de cada día, líbranos de la maldición del hielo, sálvanos de nuestros enemigos del bosque, y recíbenos con misericordia en el Gran Más Allá”.
\vs p063 6:7 \pc Onagar mantuvo su sede en las orillas septentrionales del ancestral Mediterráneo, en la región del actual Mar Caspio, en un asentamiento llamado Obán, que era lugar de parada y punto de desvío al oeste de la ruta que se dirigía al norte procedente de las tierras meridionales de Mesopotamia. Desde Obán, Onagar envió maestros a remotos asentamientos para propagar sus nuevas doctrinas sobre una sola Deidad y su concepto de la vida futura, que él llamaba el Gran Más Allá. Estos emisarios de Onagar fueron los primeros misioneros del mundo: también fueron los primeros seres humanos en cocinar la carne, los primeros que usaron el fuego de forma regular para preparar la comida. La cocinaban en la punta de unos palos y también sobre piedras calientes; más tarde, asaron grandes trozos de carne al fuego, pero sus descendientes volvieron casi enteramente al consumo de la carne cruda.
\vs p063 6:8 Onagar nació hace 983\,323 años (desde 1934 d. C.), y vivió hasta los sesenta y nueve años de edad. La historia de los logros de este gran pensador y líder espiritual de los tiempos previos al príncipe planetario constituye un fascinante relato sobre la organización de estos pueblos primitivos en una verdadera sociedad. Instituyó un eficiente gobierno tribal no igualado en muchos milenios por las generaciones que seguirían. Nunca jamás, hasta la llegada del príncipe planetario, volvió a existir en la tierra una civilización espiritual tan elevada. Esta gente sencilla disfrutaba de una verdadera religión, aunque fuese primitiva, pero sus degradados descendientes la perderían más tarde.
\vs p063 6:9 Aunque tanto Andón como Fonta, al igual que muchos de sus descendientes, habían recibido modeladores del pensamiento, no fue hasta los días de Onagar cuando los modeladores y los serafines guardianes acudieron en gran número a Urantia. De hecho, aquella fue la edad de oro del hombre primitivo.
\usection{7. SUPERVIVENCIA DE ANDÓN Y FONTA}
\vs p063 7:1 Andón y Fonta, los magníficos fundadores de la raza humana, obtuvieron su reconocimiento en el momento en el que Urantia se sometió a juicio con la llegada del príncipe planetario y, a su debido tiempo, salieron del régimen de los mundos de las moradas con el estatus de ciudadanos de Jerusem. Aunque nunca se les ha permitido regresar a Urantia, son conscientes de la historia de la raza que fundaron. Se afligieron por la traición de Caligastia, se entristecieron con el fracaso de Adán, pero se alegraron sobremanera cuando se recibió la noticia de que Miguel había elegido a su mundo como escenario para su último ministerio de gracia.
\vs p063 7:2 Andón y Fonta se fusionaron en Jerusem con sus modeladores del pensamiento, al igual que lo hicieron algunos de sus hijos, entre ellos Sontad; pero la mayoría de ellos, incluso sus descendientes inmediatos, solo consiguieron fusionarse con el espíritu.
\vs p063 7:3 Poco después de llegar a Jerusem, Andón y Fonta recibieron permiso del soberano del sistema para regresar al primer mundo de las moradas, al objeto de servir con los seres personales morontiales que dan la bienvenida a los peregrinos del tiempo llegados de Urantia a las esferas celestiales. Y se les ha asignado esta labor de forma indefinida. Intentaron enviar sus saludos a Urantia en relación con estas revelaciones, pero su petición les fue prudentemente denegada.
\vs p063 7:4 \pc Y esta es la crónica del episodio más heroico y fascinante de toda la historia de Urantia, el relato de la evolución, de la lucha por la vida, de la muerte y de la supervivencia eterna de los inigualables progenitores de toda la humanidad.
\vsetoff
\vs p063 7:5 [Exposición de un portador de vida residente en Urantia.]
