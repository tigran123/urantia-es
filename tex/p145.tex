\upaper{145}{Cuatro días memorables en Cafarnaúm}
\author{Comisión de seres intermedios}
\vs p145 0:1 Jesús y los apóstoles llegaron a Cafarnaúm el martes, 13 de enero, al caer la noche. Como era costumbre, establecieron su sede en la casa de Zebedeo en Betsaida. Al haber sido Juan el Bautista llevado a la muerte, Jesús se preparó para poner en marcha, de forma manifiesta y pública, su primer viaje de predicación en Galilea. La noticia del regreso de Jesús se difundió rápidamente por toda la ciudad y, temprano, al día siguiente, María, la madre de Jesús, se dio prisa en ir a Nazaret, para ver a su hijo José.
\vs p145 0:2 Jesús pasó miércoles, jueves y viernes en casa de Zebedeo, instruyendo a sus apóstoles y preparándolos para su primer extenso viaje de predicación pública. Asimismo, recibió y enseñó, tanto de manera individual como en grupo, a muchas sinceras personas que acudieron a él. A través de Andrés, hizo las gestiones para hablar en la sinagoga el próximo día del sabbat.
\vs p145 0:3 El viernes, avanzada la noche, Ruth, la hermana menor de Jesús, le hizo, secretamente, una visita. Estuvieron juntos casi una hora en una barca anclada a corta distancia de la orilla. Ningún ser humano llegó a conocer jamás este hecho, salvo Juan Zebedeo, a quien se le advirtió que no se lo comentara a nadie. De entre los miembros de la familia de Jesús, Ruth fue la única que creyó constante y firmemente en la divinidad de su misión en la tierra, desde el momento de su temprana toma de conciencia espiritual hasta todo su trascendental ministerio, muerte, resurrección y ascensión; y Ruth falleció finalmente, llegando a los mundos del más allá sin haber dudado jamás del carácter sobrenatural de la misión de su padre\hyp{}hermano en la carne. La pequeña Ruth, significó el mayor consuelo de Jesús, en lo que respecta a su familia de la tierra, durante toda la difícil prueba de su juicio, rechazo y crucifixión.
\usection{1. LA CAPTURA DE PECES}
\vs p145 1:1 El viernes por la mañana de esta misma semana, cuando Jesús enseñaba en la orilla del lago, la gente se agolpó a su alrededor tan cerca del borde del agua, que hizo señas a algunos pescadores, ocupantes de una barca cercana, para que vinieran a su rescate. Tras subir a la barca, continuó enseñando a la muchedumbre allí reunida durante más de dos horas. La barca portaba el nombre de “Simón”. Había sido el antiguo bote de pesca de Simón Pedro y Jesús la había construido con sus propias manos. Esa mañana en particular, habían estado utilizando dicha barca David Zebedeo y dos acompañantes, que acababan de llegar a la playa tras una infructuosa noche de pesca. Se encontraban limpiando y remendando sus redes cuando Jesús les pidió que acudieran en su ayuda.
\vs p145 1:2 Cuando acabó de impartir su enseñanza a la gente, Jesús le dijo a David: “Por haberos causado retraso al tener que venir a ayudarme, trabajaré ahora con vosotros. Vamos a pescar. Remad mar adentro hasta aguas profundas y arrojad vuestras redes”. Pero Simón, uno de los ayudantes de David, respondió: “Maestro, es inútil. Hemos estado bregando toda la noche y no hemos pescado nada; pero, en tu palabra, saldremos y echaremos las redes”. Y Simón accedió a seguir las instrucciones de Jesús, obedeciendo el gesto que su señor, David, le hizo. Cuando llegaron al sitio indicado por Jesús, echaron las redes y recogieron tal cantidad de peces que tuvieron miedo de que las redes se rompiesen; era tanta la pesca que llamaron a unos compañeros suyos que estaban en la orilla para que fuesen a prestarles ayuda. Cuando llenaron de peces las tres barcas, de tal manera que casi se hundían, este Simón cayó a las rodillas de Jesús, diciendo: “Apártate de mí, Maestro, porque soy un hombre pecador”. El asombro se apoderó de Simón y de cuantos con él estaban, a causa de los peces que habían capturado. Desde ese día, David Zebedeo, este Simón y sus compañeros dejaron sus redes y siguieron a Jesús.
\vs p145 1:3 Pero, de ninguna manera, se trató de una pesca milagrosa; Jesús era un gran estudioso de la naturaleza y un pescador experimentado, y conocía los hábitos de los peces del mar de Galilea. En esta ocasión, simplemente hizo que estos hombres se dirigieran al lugar donde se podían encontrar normalmente los peces en esta hora del día. Si bien, los seguidores de Jesús siempre lo considerarían un milagro.
\usection{2. EN LA SINAGOGA POR LA TARDE}
\vs p145 2:1 El siguiente \bibemph{sabbat,} en el oficio de tarde de la sinagoga, Jesús predicó su sermón sobre “La voluntad del Padre de los cielos”. Por la mañana, Simón Pedro lo había hecho sobre “El reino”. Allí mismo, en la reunión del jueves por la noche, Andrés había impartido enseñanzas con el tema de “El nuevo camino”. En ese preciso momento, había en Cafarnaúm más personas que creían en Jesús que en ninguna otra ciudad de la tierra.
\vs p145 2:2 Cuando Jesús enseñó en la sinagoga ese \bibemph{sabbat} por la tarde, conforme a la costumbre, tomó el primer texto de la Ley, y leyó del libro del Éxodo: “Y serviréis al Señor, vuestro Dios, y él bendecirá tu pan y tus aguas, y apartaré de ti toda enfermedad”. Escogió el segundo texto de los Profetas, leyendo esta vez de Isaías: “¡Álzate y brilla, que llega tu luz, y la gloria del Señor amanece sobre ti! Las tinieblas cubren la tierra y la terrible oscuridad a los pueblos; mas sobre ti amanecerá el espíritu del Señor y sobre ti será vista su gloria. Caminarán los gentiles a esta luz y muchas grandes mentes se rendirán a su resplandor”.
\vs p145 2:3 Con este sermón, Jesús trató de dejar claro que la religión es una \bibemph{experiencia personal}. Entre otras cosas, el Maestro dijo:
\vs p145 2:4 “Bien sabéis que, aunque un padre de buen corazón ama enteramente toda su familia, su afecto hacia esta como grupo se debe a su intenso cariño por cada miembro individual de esa familia. No debéis nunca más acercaros al Padre de los cielos como un hijo de Israel, sino como un \bibemph{hijo de Dios}. Como grupo, sois, de hecho, hijos de Israel, pero individualmente, cada uno de vosotros es un hijo de Dios. Yo he venido, no para revelar el Padre a los hijos de Israel, sino para traer una experiencia personal genuina de este conocimiento de Dios y de la revelación de su amor y misericordia al creyente individual. Los profetas os han enseñado que Yahvé cuida de su pueblo, que Dios ama a Israel. Pero yo he venido en medio de vosotros para proclamar una verdad mayor, una verdad que muchos de los últimos profetas también llegaron a entender, y esta es que Dios \bibemph{os} ama ---a cada uno de vosotros--- como seres individuales. Durante todas estas generaciones habéis tenido una religión nacional o racial; ahora yo he venido para daros una religión personal.
\vs p145 2:5 “Pero incluso esta verdad no es una idea nueva. Muchos de entre vosotros de mentalidad espiritual la habéis conocido, puesto que algunos de los profetas os han instruido así. No habéis leído en las Escrituras, donde el profeta Jeremías dice: ‘En aquellos días no dirán más: ‘Los padres comieron las uvas agrias y los hijos sufren de dentera’, pues cada uno morirá por su propia maldad; quien coma uvas agrias sufrirá de dentera. He aquí que vienen días, en los cuales yo haré un nuevo pacto con mi pueblo, no como el pacto que hice con sus padres el día en que los saqué de la tierra de Egipto, sino conforme al nuevo camino. Incluso escribiré mi ley en sus corazones. Yo seré su Dios y ellos serán mi pueblo. En ese día, no dirá más ninguno a su prójimo, ¿conoces al Señor? ¡No! Porque todos me conocerán personalmente, desde el más pequeño de ellos hasta el más grande.
\vs p145 2:6 “¿Es que no habéis leído estas promesas? ¿Es que no creéis en las Escrituras? ¿Es que no comprendéis que las palabras del profeta se cumplen en lo que veis hoy? ¿Es que no exhortó Jeremías a que hicierais de la religión un asunto del corazón, a que os relacionarais con Dios de forma individual? ¿Es que no os dijo el profeta que el Dios de los cielos escudriña el corazón de cada uno? Y, ¿no se os advirtió que el corazón humano es naturalmente engañoso más que todas las cosas y, con frecuencia, terriblemente perverso?
\vs p145 2:7 “¿Acaso no habéis leído también en las Escrituras donde Ezequiel enseñó a vuestros padres que la religión debe convertirse en una realidad, en vuestra experiencia individual? Nunca más tendréis por qué usar el proverbio: ‘Los padres comieron las uvas agrias y los hijos sufren de dentera’. ‘Por mi vida’, dice el Señor Dios, ‘He aquí que todas las almas son mías; como el alma del padre, así el alma del hijo es mía. El alma que peque esa morirá’. Y, entonces, Ezequiel llegó a predecir este día cuando habló en nombre de Dios diciendo: ‘Os daré también un corazón nuevo, y pondré un espíritu nuevo dentro de vosotros’.
\vs p145 2:8 “Nunca más temáis que Dios castigará a una nación por el pecado de una sola persona; tampoco castigará el Padre de los cielos a uno de sus hijos creyentes por los pecados de una nación, pese a que cada miembro de cualquier familia pueda sufrir a menudo las consecuencias materiales de los errores de esta y de las transgresiones del grupo. ¿Es que no os dais cuenta de que la esperanza de una nación mejor ---o de un mundo mejor--- está ligada al progreso y a la lucidez de cada cual?”.
\vs p145 2:9 Entonces, el Maestro planteó que el Padre de los cielos, una vez que el hombre reconoce esta libertad espiritual, desea que sus hijos en la tierra comiencen esa ascensión eterna en su andadura hacia el Paraíso, que consiste en la respuesta consciente de la criatura al impulso divino del espíritu interior por encontrar al Creador, por conocer a Dios y tratar de llegar a ser como él.
\vs p145 2:10 \pc A los apóstoles les resultó de gran ayuda este sermón. Todos ellos se dieron cuenta, en mayor medida, de que el evangelio del reino era un mensaje dirigido a cada uno de forma individual, y no a la nación.
\vs p145 2:11 Aun cuando la gente de Cafarnaúm estaba familiarizada con las enseñanzas de Jesús, se quedó asombrada con su sermón en aquel día del \bibemph{sabbat,} porque ciertamente les hablaba como quien tiene autoridad y no como los escribas.
\vs p145 2:12 \pc Justo en el momento en el que Jesús terminó de hablar, un hombre de la congregación, que se había sentido muy perturbado por sus palabras, cayó preso de un violento ataque epiléptico y se puso a gritar a grandes voces. Al cesar las convulsiones, cuando estaba recobrando la consciencia, habló en un estado onírico, diciendo: “¿Qué tenemos nosotros contigo, Jesús de Nazaret? Tú eres el santo de Dios; ¿has venido a destruirnos?”. Jesús mandó a la gente a que se callara y, tomando al joven de la mano, dijo: “¡Sal de ese estado!”; y se despertó de inmediato.
\vs p145 2:13 Este joven no estaba poseído de ningún espíritu impuro o demonio; era víctima de una epilepsia ordinaria. Pero se le había enseñado que su aflicción se debía a la posesión de un espíritu maligno. Él creía esta enseñanza y se comportaba en consecuencia en todo lo que pensaba o decía de su dolencia. Todos creían que estos fenómenos eran fruto directo de la presencia de espíritus impuros. Por consiguiente, creyeron que Jesús había sacado a un demonio de este hombre. Pero, en aquel momento, Jesús no curó la epilepsia del joven. No sería hasta más tarde, en ese mismo día, tras la puesta de sol, cuando este joven halló la sanación. Mucho después del día de Pentecostés, el apóstol Juan, que sería el último en escribir sobre los hechos de Jesús, evitó cualquier alusión a estos supuestos actos de “expulsar a los demonios”, y lo hizo así en vista del hecho de que, después de Pentecostés, nunca volvieron a darse casos de posesión demoníaca.
\vs p145 2:14 Como resultado de este incidente común, se extendió rápidamente por todo Cafarnaúm la noticia de que Jesús, en la sinagoga, al concluir su sermón de la tarde, había sacado a un demonio de un hombre y lo había curado milagrosamente. El \bibemph{sabbat} era la ocasión propicia para que tal impactante rumor se difundiera con tanta celeridad y efectividad. Este hecho se propagó por poblaciones más pequeñas en torno a Cafarnaúm, y mucha gente lo creyó.
\vs p145 2:15 \pc La esposa y la suegra de Simón Pedro se encargaban en gran parte de la cocina y de los quehaceres domésticos en la gran casa de Zebedeo, en la que Jesús y los doce tenían su sede. La casa de Pedro estaba cerca de la de Zebedeo; y Jesús y sus amigos se detuvieron allí, en su trayecto desde la sinagoga, porque la suegra de Pedro llevaba enferma varios días con escalofríos y fiebre. Pues bien, sucedió que, mientras Jesús estaba de pie cerca de la enferma, sosteniéndole la mano, acariciándole la frente y hablándole palabras de consuelo y aliento, la fiebre la dejó. Jesús todavía no había tenido tiempo de explicar a sus apóstoles que no se había obrado ningún milagro en la sinagoga; y con este suceso tan reciente y vívido en su mente, y al recordar además el episodio del agua y el vino en Caná, tomaron esta coincidencia como otro milagro, y algunos de ellos se apresuraron a divulgar la noticia por toda la ciudad.
\vs p145 2:16 Amata, la suegra de Pedro, padecía de fiebre palúdica. Jesús no la sanó milagrosamente en aquel momento. Su curación no se llevó a cabo hasta varias horas después, tras la puesta de sol, y se realizó en el contexto del hecho extraordinario ocurrido en el patio delantero de la casa de Zebedeo.
\vs p145 2:17 \pc Y estos son los típicos casos que muestran la forma en la que una generación que buscaba prodigios y un pueblo constantemente deseoso de milagros se servían de estas coincidencias como pretexto para proclamar que Jesús había obrado otro milagro.
\usection{3. CURACIÓN AL ATARDECER}
\vs p145 3:1 Cuando Jesús y sus apóstoles se preparaban para compartir su cena, casi al término de este memorable \bibemph{sabbat,} todo Cafarnaúm y sus alrededores se encontraban en estado de alboroto por estas presuntas curaciones milagrosas; y todos los que estaban enfermos o afligidos se dispusieron para ir a ver a Jesús o hacer que sus amigos lo llevaran allí en cuanto se pusiera el sol. De acuerdo con las enseñanzas judías, durante las horas sagradas del \bibemph{sabbat} ni siquiera se permitía ir a buscar la salud.
\vs p145 3:2 Entonces, en cuanto el sol se escondió detrás del horizonte, decenas de hombres, mujeres y niños afligidos empezaron a encaminarse hacia la casa de Zebedeo en Betsaida. Un hombre y su hija paralítica partieron en cuanto el sol desapareció tras la casa de su vecino.
\vs p145 3:3 Los hechos acaecidos durante todo el día habían preparado las condiciones para este extraordinario acontecimiento, que tendría lugar en el ocaso. Incluso el texto empleado por Jesús en su sermón aquella tarde había dado a entender que la enfermedad debía ser desterrada; y ¡él había hablado con tal inusitada autoridad y poder! ¡Su mensaje había sido tan contundente! No había aludido a la autoridad humana, sino que había hablado directamente a la conciencia y a las almas de los hombres. No había recurrido a la lógica, a argucias legales o a expresiones ingeniosas, pero había apelado de forma poderosa, directa, clara y personal a los corazones de quienes lo oían.
\vs p145 3:4 \pc Ese \bibemph{sabbat} fue un gran día en la vida terrenal de Jesús, sí, y en la vida de todo un universo. Para el universo local, la pequeña ciudad judía de Cafarnaúm era, a efectos prácticos, la verdadera capital de Nebadón. El grupo de judíos que se hallaba en la sinagoga de Cafarnaúm no fue el único en oír aquella significativa afirmación con la que Jesús terminó su sermón: “El odio es la sombra del miedo; la venganza, la máscara de la cobardía”. Como tampoco olvidarían los asistentes sus dichosas palabras, cuando declaró: “El hombre es hijo de Dios, no hijo del diablo”.
\vs p145 3:5 \pc Poco después de la puesta del sol, mientras Jesús y sus apóstoles se quedaron, tras la cena, un rato más en la mesa, la esposa de Pedro oyó voces en el patio delantero y, al ir a la puerta, vio que una gran muchedumbre de personas enfermas se estaban congregando allí y que la carretera que venía de Cafarnaún estaba atestada de personas buscando la sanación de manos de Jesús. Al observar aquello, fue enseguida a informar a su marido, que a su vez se lo contó a Jesús.
\vs p145 3:6 Cuando el Maestro salió a la puerta delantera de la casa de Zebedeo, sus ojos se encontraron con una gran cantidad de personas aquejadas y afligidas. Vio a casi mil seres humanos enfermos y dolientes; al menos ese era el número de personas que se había reunido ante él. No todos los presentes estaban con padecimientos; algunos habían venido para ayudar a sus seres queridos en este intento por curarse.
\vs p145 3:7 La visión de estos mortales afligidos, hombres, mujeres y niños sufriendo en gran parte debido a los errores y malas acciones de sus propios hijos a los que se les había confiado la administración del universo, conmovió inusualmente el corazón humano de Jesús y suscitó la misericordia divina de este benevolente hijo creador. Pero Jesús era bien consciente de que no podía erigir un movimiento espiritual perdurable sobre los pilares de prodigios puramente materiales. Constantemente, su norma había consistido en abstenerse de exhibir sus prerrogativas creadoras. Desde Caná, ni lo sobrenatural ni lo milagroso habían estado presentes en sus enseñanzas; sin embargo, esta doliente multitud hizo que su corazón sintiese conmiseración y apeló con fuerzas a su cariño y comprensión.
\vs p145 3:8 Una voz que venía del patio delantero exclamó: “Maestro, di la palabra, restáuranos la salud, sánanos de nuestras enfermedades y salva nuestras almas”. Tan pronto como estas palabras fueron pronunciadas, una inmensa comitiva de serafines, controladores físicos, portadores de vida y seres intermedios, como los que siempre acompañaban a este creador encarnado de un universo, se dispusieron a actuar con poder creativo en el caso de que su soberano diera la señal. En toda la andadura terrenal de Jesús, este fue uno de esos momentos en los que la sabiduría divina y la compasión humana se entrelazaron de tal manera con la decisión del Hijo del Hombre, que él buscó refugio apelando a la voluntad de su Padre.
\vs p145 3:9 Cuando Pedro imploró al Maestro que atendiera sus gritos de auxilio, Jesús, posando su mirada sobre aquella muchedumbre afligida, contestó: “He venido al mundo para revelar al Padre e instaurar su reino. Con tal fin he vivido mi vida hasta esta hora. Si, por lo tanto, fuese la voluntad de Aquel que me envió y no resulta contrario a mi dedicación a la proclamación del evangelio del reino de los cielos, querría ver a mis hijos curados y\ldots ”, pero las siguientes palabras de Jesús se perdieron en el tumulto.
\vs p145 3:10 Jesús había confiado la responsabilidad de esta decisión de curar a la multitud a la resolución de su Padre. Evidentemente, la voluntad del Padre no medió ninguna objeción, porque apenas el Maestro había acabado de pronunciar estas palabras, el conjunto de seres personales celestiales, que prestaba servicio bajo el mando del modelador de pensamiento personificado de Jesús, se puso vigorosamente en movimiento. La inmensa comitiva descendió en medio de esta heterogénea multitud de dolientes mortales y, en un momento de tiempo, 683 hombres, mujeres y niños quedaron sanados, se curaron completamente de todas sus enfermedades físicas y de otros trastornos materiales. Nunca antes de ese día se había sido testigo en la tierra de algo así, ni tampoco después. Y para aquellos de nosotros que estuvimos presentes, la contemplación de este caudal creativo de curaciones fue de cierto emocionante.
\vs p145 3:11 \pc Pero de entre todos los seres que quedaron impresionados ante esta repentina e inesperada irrupción de curación sobrenatural, Jesús fue quien más lo fue. En un momento, cuando su interés y compasión humanos se centraban en la escena de sufrimiento y aflicción que se extendía ante él, se olvidó de tener presentes en su mente humana las advertencias de su modelador personificado, en cuanto a la imposibilidad de limitar bajo ciertas condiciones y en ciertas circunstancias el elemento temporal de las prerrogativas creadoras de un hijo creador. Jesús deseaba ver a estos sufrientes mortales curados si con ello no se violaba la voluntad de su Padre. El modelador personificado de Jesús determinó de forma instantánea que dicho acto de energía creativa no transgredía en aquel momento la voluntad del Padre del Paraíso y, mediante esa decisión ---en razón del deseo de Jesús de sanar manifestado previamente--- el acto creativo \bibemph{fue}. Lo que un \bibemph{hijo creador} desea y su Padre \bibemph{quiere} ES. En toda la vida que le quedaba a Jesús por vivir en la tierra no tendría lugar una masiva curación física semejante.
\vs p145 3:12 \pc Como cabía esperar, la noticia de esta curación, ocurrida en Betsaida de Cafarnaúm al atardecer, se propagó por todo Galilea, Judea y demás regiones. Una vez más, se suscitaron los temores de Herodes, que envió observadores para que le informaran sobre la labor y las enseñanzas de Jesús y comprobaran si se trataba del antiguo carpintero de Nazaret o de Juan el Bautista, que había resucitado de la muerte.
\vs p145 3:13 En lo sucesivo y, mayormente debido a esta involuntaria demostración de curación física, durante el resto de su andadura en la tierra, se convirtió tanto en médico como en predicador. Aunque es verdad que continuó enseñando, su labor personal consistió principalmente en atender a los enfermos y a los angustiados, mientras que sus apóstoles realizaban la tarea de la predicación pública y de bautizar a los creyentes.
\vs p145 3:14 Pero la mayoría de los destinatarios de esta curación física de carácter sobrenatural o creativo, en esta demostración a la puesta de sol, de energía divina, no llegarían a beneficiarse espiritualmente, y de manera permanente, de esta extraordinaria manifestación de misericordia. Un pequeño número de ellos se sintió edificado por ese ministerio físico, pero el reino espiritual no avanzó en el corazón de los hombres por esta sorprendente irrupción de sanación creativa al margen del tiempo.
\vs p145 3:15 Los prodigios curativos que, ocasionalmente, estuvieron presentes durante la misión de Jesús en la tierra no formaban parte de su plan de proclamar el reino. Fue fruto del hecho mismo de que hubiese en la tierra un ser divino con prerrogativas creadoras, prácticamente ilimitadas, en conjunción con una inusitada combinación de misericordia divina y de compasión humana. Pero estos llamados milagros trajeron muchos problemas a Jesús, porque incitaban a la atención pública y a los prejuicios contra él.
\usection{4. LA NOCHE SIGUIENTE}
\vs p145 4:1 Durante toda la noche siguiente a esta gran eclosión de curaciones, la muchedumbre, gozosa y feliz, inundó la casa de Zebedeo, y los apóstoles de Jesús, emotivamente, sintieron un entusiasmo que llegaba a sobrepasarles. Desde un punto de vista humano, aquel fue probablemente el día más grande de todos los grandes días pasados al lado de Jesús. Nunca, ni antes ni después, sus esperanzas habían alcanzado tales alturas de innegable expectación. Solo unos días antes, y cuando aún estaban dentro de las fronteras de Samaria, Jesús les había dicho que ya había llegado la hora de proclamar el reino en \bibemph{poder,} y ahora sus ojos habían contemplado lo que suponían significaba el cumplimiento de esa promesa. Estaban llenos de entusiasmo ante lo que estaba por venir, si esta formidable manifestación de poder curativo era solo el principio. Desterraron las dudas que aún albergaban sobre la divinidad de Jesús. Literalmente, se dejaron embriagar por el éxtasis de su confundido encantamiento.
\vs p145 4:2 Pero cuando fueron en busca de Jesús, no pudieron encontrarlo. El Maestro estaba muy perturbado por lo que había sucedido. Los hombres, mujeres y niños que habían resultado sanados de sus distintas enfermedades se quedaron hasta bien entrada la noche, esperando el regreso de Jesús para poder darle las gracias. Los apóstoles no podían entender el proceder del Maestro a medida que pasaban las horas y permanecía en su retiro; su gozo hubiese sido pleno y perfecto a no ser por su continuada ausencia. Cuando Jesús regresó con ellos, era tarde y prácticamente todos los que se habían beneficiado de esta curación habían vuelto a sus casas. Jesús rechazó los parabienes y la adoración de los doce y de los otros que habían permanecido allí para saludarlo, diciendo solamente: “No os regocijéis de que mi Padre tenga el poder de sanar el cuerpo, sino que más bien de que lo tiene para salvar el alma. Vámonos a descansar, porque mañana hemos de ocuparnos de los asuntos del Padre”.
\vs p145 4:3 Y, de nuevo, estos doce hombres se fueron a descansar, decepcionados, desconcertados y tristes; excepto los gemelos, pocos de ellos pudieron dormir mucho aquella noche. Tan pronto como el Maestro hacía algo que les alborozaba el alma y les alegraba el corazón, él parecía hacer trizas de forma inmediata las esperanzas de los apóstoles y demolía los pilares de su arrojo y entusiasmo. Al mirarse entre sí, estos perplejos pescadores tenían un solo pensamiento: “No alcanzamos a entenderlo. ¿Qué significa todo esto?”.
\usection{5. EL DOMINGO DE MADRUGADA}
\vs p145 5:1 Jesús tampoco durmió mucho aquel sábado por la noche. Se daba cuenta de que el mundo estaba lleno de malestares físicos y aquejado de dificultades materiales, y consideró sobre el gran peligro que conllevaba verse obligado a dedicar tanto de su tiempo al cuidado de los enfermos y afligidos, que su misión de instaurar el reino espiritual en el corazón de los hombres se vería obstaculizada o al menos subordinada al ministerio de las cosas físicas. Debido a estos pensamientos y a otros similares, que ocuparon la mente mortal de Jesús durante la noche, ese domingo por la mañana se levantó mucho antes de despuntar el día y se fue solo a uno de sus sitios preferidos para estar en comunión con su Padre. Esa mañana temprano, Jesús oró por sabiduría y juicio para no permitir que su compasión humana, unida a su misericordia divina, le instara, en presencia del sufrimiento mortal, a ocupar todo su tiempo con el ministerio físico en detrimento del espiritual. Aunque no deseaba de ninguna manera eludir la atención al enfermo, era conocedor de que también debía hacer una labor más importante de enseñanza espiritual y de formación religiosa.
\vs p145 5:2 Jesús iba tantas veces a las colinas a orar, porque no había allí ninguna estancia privada idónea para sus devociones personales.
\vs p145 5:3 Pedro no pudo dormir esa noche; así pues, muy temprano, poco después de que Jesús se había ido a orar, despertó a Santiago y a Juan, y los tres fueron al encuentro del Maestro. Tras buscarlo durante más de una hora, encontraron a Jesús y le suplicaron que les dijera la razón de su extraña conducta. Deseaban saber por qué parecía estar turbado por el portentoso derramamiento del espíritu de sanación, cuando toda la gente estaba jubilosa y los apóstoles tan exultantes.
\vs p145 5:4 Durante más de cuatro horas, Jesús procuró explicar a estos tres apóstoles lo que había sucedido, dándole detalles e informándoles de los riesgos de tales manifestaciones. Jesús les confió el motivo por el que había salido a orar. Trató de hacer entender a sus acompañantes personales las verdaderas razones por las que el reino del Padre no podía edificarse sobre la realización de prodigios ni sobre las curaciones físicas. Pero ellos no llegaban a comprender sus enseñanzas.
\vs p145 5:5 Entretanto, temprano el domingo por la mañana, otras muchedumbres de almas afligidas y muchas personas animadas por la curiosidad empezaron a congregarse alrededor de la casa de Zebedeo. Pedían a gritos que querían ver a Jesús. Andrés y los apóstoles se quedaron tan desconcertados que, mientras que Simón Zelotes hablaba a las personas concentradas allí, Andrés, con algunos de sus compañeros, fue a buscar a Jesús. Cuando Andrés lo encontró, en compañía de los otros tres apóstoles, le dijo: “Maestro, ¿por qué nos dejas solos con la multitud? Mira que todos te buscan; nunca antes ha habido tantos queriendo que les enseñes. Ahora mismo, la casa está rodeada de quienes han venido de cerca y de lejos debido a tus formidables obras. ¿Es que no vas a volver con nosotros para atenderlos?”.
\vs p145 5:6 Cuando Jesús escuchó estas palabras, le contestó: “Andrés, ¿es que no te he enseñado a ti y a estos otros que mi misión en la tierra es la revelación del Padre y, mi mensaje, la proclamación del reino de los cielos? ¿Cómo es que quieres, pues, que yo me aparte de mi misión para gratificar a los curiosos y para satisfacer a los que buscan signos y prodigios? ¿Es que no hemos estado entre esta gente durante todos estos meses y han acudido en multitudes para oír la buena nueva del reino? ¿Por qué vienen ahora a acosarnos? ¿Es porque buscan curar sus cuerpos físicos y no por haber recibido la verdad espiritual para la salvación de sus almas? Cuando los hombres se ven atraídos hacia nosotros por manifestaciones extraordinarias, la mayoría de ellos no viene buscando la verdad y la salvación, sino más bien a curar sus dolencias físicas y librarse de sus dificultades materiales.
\vs p145 5:7 “Todo este tiempo he estado en Cafarnaúm y, tanto en la sinagoga como junto al mar, he proclamado la buena nueva del reino a todos los que tenían oídos para oír y corazones para recibir la verdad. No es la voluntad de mi Padre que vuelva con vosotros para atender a estos curiosos ni ocuparme del ministerio de las cosas materiales, descuidando las espirituales. Os he ordenado que prediquéis el evangelio y auxiliéis a los enfermos, pero no voy a condescender con la sanación de los cuerpos en lugar de enseñar la verdad. No, Andrés, no voy a volver con vosotros. Id y decidle a la gente que crea en lo que les hemos enseñado y que se regocijen en la libertad de los hijos de Dios, y disponeos para nuestra partida a las otras ciudades de Galilea, donde el camino está ya listo para la predicación de la buena nueva del reino. Fue con este propósito por el que yo vine del Padre. Id, pues, y preparaos para salir de inmediato mientras aguardo aquí vuestro regreso”.
\vs p145 5:8 Cuando Jesús acabó de hablar, Andrés y sus compañeros apóstoles volvieron con tristeza a la casa de Zebedeo, despidieron a la multitud allí congregada y rápidamente se prepararon para el viaje, como Jesús había ordenado. Y, así, en la tarde del domingo, 18 de enero del año 28 d. C., Jesús y los apóstoles emprendieron su primer viaje de predicación, realmente público y manifiesto, en las ciudades de Galilea. Durante este primer viaje, predicaron el evangelio del reino en muchas ciudades, pero no fueron a Nazaret.
\vs p145 5:9 Ese domingo por la tarde, poco después de que Jesús y sus apóstoles hubiesen salido para Rimón, sus hermanos Santiago y Judá fueron a verlo, deteniéndose brevemente en la casa de Zebedeo. Sobre el mediodía, de aquel mismo día, Judá había ido a buscar a su hermano Santiago y, al encontrarlo, le había insistido en que acudiesen a ver a Jesús. Si bien, cuando Santiago por fin accedió a ir con Judá, Jesús ya había partido.
\vs p145 5:10 Los apóstoles se mostraron reacios a dejar atrás el gran interés que habían suscitado en Cafarnaúm. Pedro calculaba que no menos de mil creyentes podrían haberse bautizados en el reino. Jesús les escuchó con paciencia, pero no consintió en volver. Por un rato, imperó el silencio y, entonces, Tomás se dirigió a sus compañeros apóstoles, diciéndoles: “¡Vámonos! El Maestro ha hablado. No importa si no comprendemos del todo los misterios del reino de los cielos, de una cosa estamos seguros: seguimos a un maestro que no busca la gloria para sí mismo”. Y, a disgusto, salieron para predicar la buena nueva en las ciudades de Galilea.
