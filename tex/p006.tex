\upaper{6}{El Hijo Eterno}
\author{Consejero divino}
\vs p006 0:1 El Hijo Eterno es la expresión perfecta y final del “primer” concepto personal y absoluto del Padre Universal. En consecuencia, siempre y como quiera que el Padre se exprese de forma personal y absoluta, lo hace a través de su Hijo Eterno, que siempre ha sido, ahora es, y siempre será, el Verbo vivo y divino. Y este Hijo Eterno reside en el centro de todas las cosas, en vinculación con el Padre Eterno y Universal y envolviendo estrechamente su presencia personal.
\vs p006 0:2 Hablamos del “primer” pensamiento de Dios y, con el fin de tener acceso a los cauces de pensamiento del intelecto humano, aludimos a un imposible origen temporal del Hijo Eterno. Distorsionamos así el lenguaje para intentar, de la mejor manera, comunicarnos y avenirnos a las mentes sujetas al tiempo de las criaturas mortales. Desde una perspectiva secuencial, el Padre Universal no pudo nunca haber tenido un primer pensamiento ni el Hijo Eterno pudo nunca haber tenido un principio. Pero se me ha instruido que describa para las mentes temporalmente limitadas de los mortales las realidades de la eternidad recurriendo a estas nociones simbólicas y que designe, por medio de dichos conceptos secuenciales, las relaciones de la eternidad.
\vs p006 0:3 El Hijo Eterno es la manifestación personal del concepto universal e infinito de la realidad divina, del espíritu incondicionado y del ser personal absoluto del Padre del Paraíso. Y, de ese modo, el Hijo constituye la revelación divina de la identidad creadora del Padre Universal. El ser personal perfecto del Hijo desvela que el Padre es, en realidad, el origen eterno y universal de todos los contenidos y valores de lo espiritual, lo volitivo, lo intencional y lo personal.
\vs p006 0:4 En nuestro empeño por posibilitar que la mente finita del tiempo se forme un concepto secuencial de las relaciones de los seres eternos e infinitos de la Trinidad del Paraíso, acudimos a ciertas licencias conceptuales como la de referirnos al “primer concepto personal, universal e infinito del Padre”. Me es imposible transmitir a la mente humana cualquier idea que describa adecuadamente las relaciones eternas de las Deidades; por consiguiente, empleo términos que ofrezcan a la mente finita algún tipo de idea de la relación de estos seres eternos, en edades posteriores del tiempo. Creemos que el Hijo surgió del Padre; se nos enseña que ambos son incondicionalmente eternos. Es evidente, por tanto, que ninguna criatura temporal puede llegar a comprender jamás del todo este misterio de un Hijo que tiene su origen en el Padre y que, sin embargo, se equipara en eternidad con él.
\usection{1. IDENTIDAD DEL HIJO ETERNO}
\vs p006 1:1 El Hijo Eterno es el Hijo primigenio y unigénito de Dios. Él es el Dios Hijo, la segunda persona de la Deidad y el compañero creador de todas las cosas. Así como el Padre es la Primera Gran Fuente y Centro, el Hijo Eterno es la Segunda Gran Fuente y Centro.
\vs p006 1:2 El Hijo Eterno es el centro espiritual y el administrador divino del gobierno espiritual del universo de los universos. El Padre Espiritual es primero creador y luego rector; el Hijo Eterno es primero cocreador y luego \bibemph{administrador espiritual}. “Dios es espíritu” y el Hijo constituye la revelación personal de ese espíritu. La Primera Fuente y Centro es el absoluto de la volición; la Segunda Fuente y Centro es el absoluto del ser personal.
\vs p006 1:3 El Padre Universal nunca obra personalmente como creador excepto en conjunción con el Hijo o con la acción equiparada del Hijo. Si el escritor del Nuevo Testamento se hubiera referido al Hijo Eterno, habría dicho la verdad cuando escribió: “En el principio era el Verbo, y el Verbo era con Dios, y el Verbo era Dios. Todas las cosas por él fueron hechas, y sin él nada de lo que ha sido hecho fue hecho”.
\vs p006 1:4 Cuando un hijo del Hijo Eterno apareció en Urantia, los que confraternizaron con este ser divino en forma humana aludían a él como “Aquel que era desde el principio, a quien hemos oído, a quien hemos visto con nuestros ojos, a quien hemos contemplado, a quien han palpado nuestras manos, realmente es el Verbo de Vida”. Y este hijo de gracia nació del Padre al igual que lo hizo el Hijo Primigenio, como se sugiere en una de sus oraciones terrestres: “Ahora pues, Padre mío, glorifícame tú al lado tuyo, con la gloria que tuve contigo antes de que el mundo existiera”.
\vs p006 1:5 \pc Al Hijo Eterno se le conoce con diferentes nombres en los distintos universos. En el universo central, se le conoce como la Fuente de Igual Rango, el Cocreador y el Compañero Absoluto. En Uversa, sede central de este suprauniverso, designamos al Hijo “Centro Espiritual Coigual y Administrador Espiritual Eterno”. En Lugar de Salvación, sede central de vuestro universo local, a este Hijo se le conoce como la Segunda Fuente y Centro Eterno. Los melquisedecs se refieren a él como el Hijo de Hijos. En vuestro mundo, pero no en vuestro sistema de esferas habitadas, se ha confundido a este Hijo Primigenio con uno de los hijos creadores homólogos, Miguel de Nebadón, que se dio de gracia a las razas mortales de Urantia.
\vs p006 1:6 Aunque se puede llamar Hijo de Dios, como corresponde, a cualquier hijo del Paraíso, estamos acostumbrados a denominar Hijo Eterno a este Hijo Primigenio, Segunda Fuente y Centro, cocreador con el Padre Universal del universo central de poder y perfección y cocreador de todos los demás hijos divinos que surgen de las Deidades infinitas.
\usection{2. LA NATURALEZA DEL HIJO ETERNO}
\vs p006 2:1 El Hijo Eterno es tan invariable e infinitamente meritorio de confianza como el Padre Universal. Es también tan espiritual como el Padre, un espíritu igual de ilimitado que el Padre. Para vosotros que tenéis un origen humilde, el Hijo os puede parecer más personal que el Padre Universal al estar un escalafón más cercano y accesible.
\vs p006 2:2 El Hijo Eterno es el Verbo Eterno de Dios. Es totalmente semejante al Padre; en efecto, el Hijo Eterno es el Dios Padre manifestado de manera personal en el universo de los universos. Y así fue y es y será por siempre cierto del Hijo Eterno y de todos los hijos creadores homólogos: “El que ha visto al Hijo, ha visto al Padre”.
\vs p006 2:3 En su naturaleza, el Hijo es totalmente como el Padre espiritual. Cuando adoramos al Padre Universal, en realidad adoramos al mismo tiempo al Dios Hijo y al Dios Espíritu. El Dios Hijo es en su naturaleza tan divinamente real y eterno como el Dios Padre.
\vs p006 2:4 El Hijo no solo posee toda la rectitud infinita y suprema del Padre sino que refleja también toda la santidad de carácter del Padre. El Hijo comparte la perfección del Padre y comparte, junto con él, la responsabilidad de ayudar a todas las criaturas imperfectas en su afán espiritual por alcanzar la perfección divina.
\vs p006 2:5 El Hijo Eterno posee todo el carácter divino y los atributos espirituales del Padre. El Hijo constituye, en ser personal y en espíritu, la plenitud de la absolutidad de Dios y revela estas cualidades al dirigir personalmente el gobierno espiritual del universo de los universos.
\vs p006 2:6 Dios es, realmente, un espíritu universal; Dios es espíritu; y esta naturaleza espiritual del Padre converge y se hace personal en la Deidad del Hijo Eterno. En el Hijo, todas las características espirituales parecen haberse ampliado grandemente al diferenciarse de la universalidad de la Primera Fuente y Centro. Y tal como el Padre comparte su naturaleza espiritual con el Hijo, así también comparten ambos, plena e incondicionalmente, el espíritu divino con el Actor Conjunto, el Espíritu Infinito.
\vs p006 2:7 En el amor a la verdad y en la creación de la belleza, el Padre y el Hijo son iguales, salvo que el Hijo \bibemph{parece} más dedicado a la realización de la belleza exclusivamente espiritual de los valores universales.
\vs p006 2:8 En bondad divina, no percibo diferencia alguna entre el Padre y el Hijo. El Padre ama a los hijos de su universo como un padre; el Hijo Eterno contempla a todas las criaturas como padre y hermano.
\usection{3. EL MINISTERIO DEL AMOR DEL PADRE}
\vs p006 3:1 El Hijo comparte la justicia y la rectitud de la Trinidad, pero sobrepasa estos rasgos divinos mediante su infinita manifestación personal del amor y de la misericordia del Padre; el Hijo es la revelación del amor divino a los universos. Tal como Dios es amor, el Hijo es misericordia. El Hijo no puede amar más que el Padre, pero puede mostrar misericordia a las criaturas de otra manera, porque no solo es un creador primordial como el Padre, sino que es también el Hijo Eterno del mismo Padre, participando así de la experiencia filial de todos los otros hijos del Padre Universal.
\vs p006 3:2 El Hijo Eterno es el gran benefactor de la misericordia para toda la creación. La misericordia es la esencia del carácter espiritual del Hijo. Los mandatos del Hijo Eterno, cuando salen de las vías de la Segunda Fuente y Centro por donde circula el espíritu, se modelan con tonos de misericordia.
\vs p006 3:3 Para comprender el amor del Hijo Eterno, debéis primero tomar conciencia de su origen divino ---el Padre, que es amor---, y luego contemplar el despliegue de este afecto infinito en el extenso ministerio del Espíritu Infinito y de su casi ilimitada multitud de seres personales en servicio.
\vs p006 3:4 El ministerio del Hijo Eterno está destinado a revelar el Dios de amor al universo de los universos. Este hijo divino no se ocupa de la innoble tarea de intentar persuadir a su benigno Padre de que ame a sus modestas criaturas y de que sea misericordioso con los infractores de los tiempos. ¡Qué equivocación concebir al Hijo Eterno suplicando al Padre Universal que sea misericordioso con las modestas criaturas de los mundos materiales del espacio! Estos son conceptos burdos y grotescos acerca de Dios. Más bien debierais daros cuenta de que todo el ministerio misericordioso de los Hijos de Dios constituye la revelación directa del corazón del Padre, alguien universalmente amoroso e infinitamente compasivo. El amor del Padre es la fuente real y eterna de la misericordia del Hijo.
\vs p006 3:5 Dios es amor, el Hijo es misericordia. La misericordia es la aplicación del amor, el amor del Padre en acción en la persona de su Hijo Eterno. El amor de este Hijo universal es asimismo universal. Tal como se comprende el amor en un planeta donde existe el sexo, el amor de Dios se compara más al amor de un padre, mientras que el amor del Hijo Eterno se asemeja más al afecto de una madre. Estas son de hecho toscas ilustraciones, pero las empleo con la esperanza de transmitir a la mente humana la idea de que existe una diferencia, no en contenido divino sino en cualidad y forma de expresión, entre el amor del Padre y el amor del Hijo
\usection{4. LOS ATRIBUTOS DEL HIJO ETERNO}
\vs p006 4:1 El Hijo Eterno alienta el nivel espiritual de la realidad cósmica; la potencia espiritual del Hijo es absoluta en relación con todas las manifestaciones del universo. Mediante su dominio absoluto de la gravedad espiritual, ejerce una perfecta potestad sobre la vinculación mutua entre toda la energía espiritual indiferenciada y toda la realidad espiritual actualizada. Todo espíritu puro no fraccionado y todos los seres y valores espirituales responden al infinito poder de atracción del Hijo Primordial del Paraíso. Y si se presenciara en un futuro eterno la aparición de un universo ilimitado, la gravedad espiritual y la potencia espiritual del Hijo Primigenio resultarían totalmente adecuadas para la dirección espiritual y la administración eficaz de esa creación sin límites.
\vs p006 4:2 \pc El Hijo es omnipotente solamente en el ámbito espiritual. En la eficaz organización eterna de la administración del universo, no hay lugar para que una misma función se repita de forma pródiga e innecesaria; no son dadas las Deidades a la vana duplicación de las tareas del universo.
\vs p006 4:3 \pc La omnipresencia del Hijo Primigenio constituye la unidad espiritual del universo de los universos. La cohesión espiritual de toda la creación descansa en la presencia activa y en todo lugar del espíritu divino del Hijo Eterno. Cuando concebimos la presencia espiritual del Padre, nos resulta difícil diferenciarla en nuestro pensamiento de la presencia espiritual del Hijo Eterno. El espíritu del Padre reside eternamente en el espíritu del Hijo.
\vs p006 4:4 El Padre ha de ser espiritualmente omnipresente, pero tal omnipresencia parece ser inseparable de la actividad espiritual por doquier del Hijo Eterno. Sí creemos, sin embargo, que en cualquier circunstancia en la que se dé esta doble presencia de la naturaleza espiritual del Padre y del Hijo, el espíritu del Hijo se correlaciona con el del Padre.
\vs p006 4:5 En su contacto con el ser personal, el Padre actúa en la vía circulatoria del ser personal. En su contacto personal y perceptible con la creación espiritual, el Padre se manifiesta en la totalidad de las fracciones de su Deidad, y estas fracciones del Padre tienen una labor singular, única y exclusiva en cualquier momento y lugar en que aparezcan en los universos. En todas estas situaciones, el espíritu del Hijo se correlaciona con la labor espiritual de la presencia repartida del Padre Universal.
\vs p006 4:6 Espiritualmente, el Hijo Eterno es omnipresente. El espíritu del Hijo Eterno está ciertamente con vosotros y en torno a vosotros, pero no está dentro de vosotros ni forma parte de vosotros como el mentor misterioso. Desde su morada interior, la fracción del Padre modela la mente humana conforme a sus actitudes progresivamente divinas, de modo que esa mente ascendente responde cada vez más al poder de atracción espiritual de la todopoderosa vía de la Segunda Fuente y Centro por donde circula la gravedad espiritual.
\vs p006 4:7 \pc El Hijo Primigenio es universal y espiritualmente consciente de sí mismo. En sabiduría, el Hijo es completamente igual al Padre. En los ámbitos del conocimiento, de la omnisciencia, no podemos distinguir entre las Fuentes Primera y Segunda; como el Padre, el Hijo lo sabe todo; ningún acontecimiento universal le sorprende jamás; él comprende el fin desde el comienzo.
\vs p006 4:8 \pc El Padre y el Hijo conocen realmente el número y el paradero de todos los espíritus y seres espiritualizados del universo de los universos. El Hijo no solo conoce todas las cosas en virtud de su propio espíritu omnipresente, sino que el Hijo, al igual que el Padre y el Actor Conjunto, es plenamente conocedor de toda la inmensa información procedente de la reflectividad del Ser Supremo, cuyo sistema de información le hace asimismo consciente en cualquier momento de todo lo que ocurre en todos los mundos de los siete suprauniversos. Y hay otros modos en los que el Hijo del Paraíso es omnisciente.
\vs p006 4:9 \pc El Hijo Eterno, como ser personal espiritual, amoroso, misericordioso y benefactor es total e infinitamente igual al Padre Universal, mientras que en todos esos contactos personales misericordiosos y amorosos con los seres ascendentes de los mundos más modestos, el Hijo Eterno es tan generoso y considerado, tan paciente y magnánimo, como lo son sus hijos del Paraíso en los universos locales, que con tanta frecuencia se dan de gracia a los mundos evolutivos del tiempo.
\vs p006 4:10 Resulta innecesario extenderse más sobre los atributos del Hijo Eterno. Teniendo en cuenta las excepciones observadas, basta con examinar los atributos espirituales del Dios Padre para entender y evaluar correctamente los atributos del Dios Hijo.
\usection{5. LIMITACIONES DEL HIJO ETERNO}
\vs p006 5:1 El Hijo Eterno no obra de forma personal en los ámbitos físicos ni tampoco lo hace, excepto mediante el Actor Conjunto, en los niveles del ministerio de la mente en los seres creados. Pero estas condiciones no limitan en modo alguno al Hijo Eterno en el pleno y libre ejercicio de todos sus atributos divinos de omnisciencia, omnipresencia y omnipotencia \bibemph{espirituales}.
\vs p006 5:2 El Hijo Eterno no se difunde de forma personal por los potenciales del espíritu consustanciales a la infinitud del Absoluto de la Deidad, pero, a medida que estos potenciales se actualizan, entran en el dominio todopoderoso de la vía circulatoria de la gravedad espiritual del Hijo.
\vs p006 5:3 El ser personal es la dádiva exclusiva del Padre Universal. El Hijo Eterno deriva del Padre su persona, pero no otorga, sin el Padre, el ser personal. El Hijo da origen a una inmensa multitud de espíritus, pero estos vástagos suyos no son seres personales. Cuando el Hijo crea a un ser personal, lo hace en conjunción con el Padre o con el Creador Conjunto, que puede actuar por el Padre en tales relaciones. El Hijo Eterno es, pues, cocreador de seres personales, pero no otorga el ser personal a ningún ser, y de sí mismo, por sí solo, nunca crea seres personales. Esta limitación de acción, sin embargo, no priva al Hijo de su capacidad de crear cualquier tipo o todos los tipos de realidad distinta de la realidad personal.
\vs p006 5:4 El Hijo Eterno está limitado en la transmisión de las prerrogativas de creador. El Padre, al eternizar al Hijo Primigenio, le otorgó el poder y el privilegio para unirse, con posterioridad, con el Padre en el acto divino de dar origen a otros hijos que poseyeran atributos creativos, y esto es lo que han hecho y ahora hacen. Pero una vez que se ha dado origen a estos hijos de igual rango entre sí, al parecer cesan de transmitirse estas prerrogativas. El Hijo Eterno transmite poderes de creación solo a su manifestación personal primera o directa. Así pues, cuando el Padre y el Hijo se unen para hacer personal a un hijo creador, consiguen su propósito; pero el hijo creador que tuvo su existencia de este modo jamás puede transmitir ni delegar sus prerrogativas de creación a los distintos órdenes de hijos que él pueda crear con posterioridad, no obstante, en los más elevados hijos del universo local aparece un reflejo muy limitado de los atributos creativos de un hijo creador.
\vs p006 5:5 El Hijo Eterno, como ser infinito y exclusivamente personal, no puede fraccionar su naturaleza, no puede distribuir y dar porciones individualizadas de su yo a otras entidades o personas como lo hacen el Padre Universal y el Espíritu Infinito. Pero el Hijo puede darse a sí mismo de gracia y, de hecho, se otorga como espíritu ilimitado para impregnar toda la creación y atraer hacia sí, de forma incesante, a todos los seres personales espirituales y a todas las realidades espirituales.
\vs p006 5:6 Recordad siempre que el Hijo Eterno es, para toda la creación, la representación personal del Padre espiritual. El Hijo es personal y nada más que personal en un sentido relativo a la Deidad; este ser personal divino y absoluto no puede disgregarse ni fraccionarse. El Dios Padre y el Dios Espíritu son verdaderamente personales, pero, además de ser seres personales divinos son también todo lo demás.
\vs p006 5:7 Aunque el Hijo Eterno no puede participar personalmente en la concesión de los modeladores del pensamiento, en el pasado eterno sí participó en consejo con el Padre Universal, aprobando el plan y prometiendo su ilimitada cooperación, cuando el Padre, al preparar la dádiva de los modeladores del pensamiento, propuso al Hijo: “Hagamos al hombre mortal a nuestra propia imagen”. Y así, al igual que una fracción espiritual del Padre mora en ti, la presencia espiritual del Hijo te envuelve, a la vez que ambos actúan eternamente como uno solo para tu avance espiritual.
\usection{6. LA MENTE ESPIRITUAL}
\vs p006 6:1 El Hijo Eterno es espíritu y tiene mente, pero no una mente ni un espíritu que la mente mortal pueda comprender. El hombre mortal percibe la mente en los niveles finito, cósmico, material y personal. El hombre también observa los fenómenos mentales en organismos vivos que actúan en un nivel por debajo del ámbito del ser personal (el animal), pero le resulta difícil alcanzar a comprender la naturaleza de la mente cuando se vincula a seres supramateriales o cuando es parte de seres personales exclusivamente espirituales. Sin embargo, la mente debe definirse de manera distinta cuando hace referencia al nivel espiritual de la existencia y cuando se emplea para denotar las capacidades espirituales de la inteligencia. El tipo de mente directamente ligada al espíritu no es comparable ni con la mente que coordina el espíritu y la materia ni con la mente que solo está ligada a la materia.
\vs p006 6:2 El espíritu es siempre consciente, mental y su identidad posee distintas facetas. Sin manifestación de alguna faceta mental, no habría ninguna conciencia espiritual en la fraternidad de los seres espirituales. El equivalente de la mente, la capacidad de conocer y ser conocido, es connatural a la Deidad. La Deidad puede ser personal, prepersonal, suprapersonal o impersonal; pero la Deidad no existe nunca sin mente, es decir, nunca carece, al menos, de la capacidad de comunicarse con entidades, seres o personas similares.
\vs p006 6:3 La mente del Hijo Eterno es como la del Padre, pero distinta de cualquier otra mente del universo y, con la mente del Padre, es antecesora a las diversas y extendidas mentes del Actor Conjunto. La mente del Padre y el Hijo, ese intelecto que es anterior a la mente absoluta de la Tercera Fuente y Centro, tal vez se ilustre mejor en relación con la premente del modelador del pensamiento, porque, aunque estas fracciones del Padre están completamente fuera de las vías del Actor Conjunto, circulatorias de la mente, poseen alguna forma de premente; conocen y son conocidos; gozan del equivalente al pensamiento humano.
\vs p006 6:4 El Hijo Eterno es completamente espiritual; el hombre es casi enteramente material; por tanto, gran parte de lo que es pertinente al ser personal del Hijo Eterno, a sus siete esferas espirituales que circundan el Paraíso y a la naturaleza de las creaciones impersonales del Hijo del Paraíso, tendrá que esperar para vuestro reconocimiento que alcancéis el estado espiritual, tras completar vuestra ascensión morontial en el universo local de Nebadón. Y, luego, según paséis por el suprauniverso en dirección a Havona, se dilucidarán muchos de estos misterios espirituales ocultos conforme comencéis a ser dotados de la “mente del espíritu” ---de percepción espiritual---.
\usection{7. EL SER PERSONAL DEL HIJO ETERNO}
\vs p006 7:1 Mediante el método de la trinitización, el Padre Universal escapó de las ataduras del ser personal incondicionado al otorgar al Hijo Eterno un ser personal infinito y ha continuado, desde entonces, dándose de gracia en interminable profusión sobre su universo siempre en expansión de creadores y criaturas. El Hijo es un \bibemph{ser personal absoluto;} Dios es un \bibemph{ser personal paternal} ---la fuente del ser personal, quien otorga el ser personal, la causa del ser personal---. Todo persona personal deriva su ser personal del Padre Universal así como el Hijo Primigenio eternamente deriva su ser personal del Padre del Paraíso.
\vs p006 7:2 El ser personal del Hijo del Paraíso es absoluto y puramente espiritual, y este ser personal absoluto es también el modelo divino y eterno, el primero, de la dádiva del ser personal por parte del Padre al Actor Conjunto y, posteriormente, de su dádiva del ser personal a sus innumerables criaturas del extenso universo.
\vs p006 7:3 El Hijo Eterno es en verdad un benefactor misericordioso, un espíritu divino, un poder espiritual y un ser personal real. El Hijo es la naturaleza espiritual y personal de Dios manifestada a los universos ---la esencialidad de la Primera Fuente y Centro despojada de lo no personal, de lo extradivino, de lo no espiritual y de lo puramente potencial---. Pero es imposible transmitir a la mente humana una imagen verbal de la belleza y grandeza del excelso ser personal del Hijo Eterno. Todo lo que tiende a oscurecer al Padre Universal ejerce una influencia casi igual para impedir que se reconozca conceptualmente al Hijo Eterno. Debéis esperar llegar al Paraíso para entender por qué fui incapaz de describir el carácter de este ser personal absoluto para que la mente finita lo comprendiera.
\usection{8. LA REALIDAD DEL HIJO ETERNO}
\vs p006 8:1 En lo concerniente a la identidad, naturaleza y otros atributos personales, el Hijo Eterno es el equivalente pleno, el complemento perfecto y el homólogo eterno del Padre Universal. En el mismo sentido que Dios es el Padre Universal, el Hijo es la Madre Universal. Y todos nosotros, de cualquier rango, formamos su familia universal.
\vs p006 8:2 Para ser completamente conscientes del carácter del Hijo, debéis examinar la revelación del carácter divino del Padre: son perpetua e inseparablemente uno. Como seres personales divinos son prácticamente indistinguibles para los órdenes más modestos de inteligencia, si bien, no es tan difícil reconocerlos por separado para los que tienen su origen en los actos creativos de las Deidades mismas. Los seres nacidos en el universo central y en el Paraíso disciernen al Padre y al Hijo no solo como una unidad personal con potestad universal, sino también como dos seres personales separados que obran en ámbitos determinados de la administración del universo.
\vs p006 8:3 Como personas, podéis concebir al Padre Universal y al Hijo Eterno como seres separados, porque en verdad lo son. No obstante, en la administración de los universos están tan entrelazados y correlacionados que no siempre se les puede distinguir. Cuando, en los asuntos de los universos, nos encontramos con el Padre y el Hijo vinculados entre sí de forma que nos pueda confundir, no siempre resulta beneficioso intentar aislar sus intervenciones; recordad simplemente que Dios es el pensamiento iniciador y que el Hijo es el verbo manifestado en plenitud. En cada universo local, esta inseparabilidad se hace personal en la divinidad del hijo creador, que representa al Padre y al Hijo ante las criaturas de diez millones de mundos habitados.
\vs p006 8:4 El Hijo Eterno es infinito, pero es accesible a través de las personas de sus hijos del Paraíso y mediante el paciente ministerio del Espíritu Infinito. Sin el beneficio del ministerio de gracia de los hijos del Paraíso y el servicio amoroso de las criaturas del Espíritu Infinito, los seres de origen material difícilmente podrían tener la esperanza de llegar al Hijo Eterno. Y es de igual modo cierto que, con la ayuda y dirección de estas instancias intermedias celestiales, el mortal con conciencia de Dios ciertamente arribará al Paraíso y algún día estará ante la presencia personal de este majestuoso Hijo de Hijos.
\vs p006 8:5 \pc A pesar de que el Hijo Eterno constituye el modelo para la consecución del ser personal humano, os será más fácil alcanzar a comprender la realidad del Padre y del Espíritu porque el Padre es quien en verdad otorga vuestro ser personal humano, y el Espíritu Infinito es la fuente absoluta de vuestra mente mortal. Pero según ascendáis en vuestro progreso espiritual por la senda del Paraíso, el ser personal del Hijo Eterno se os hará cada vez más real, y la realidad de su mente, infinitamente espiritual, se hará más discernible para vuestra mente en su progreso espiritual.
\vs p006 8:6 Nunca podrá el concepto del Hijo Eterno resplandecer en vuestra mente material ni en la morontial posterior. Hasta que no os hagáis espíritus y comencéis vuestra ascensión espiritual, no comenzaréis a comprender el ser personal del Hijo Eterno con la misma nitidez con la que concebís el ser personal del hijo creador originado en el Paraíso quien, en persona y como persona, una vez se encarnó y vivió en Urantia como un hombre entre los hombres.
\vs p006 8:7 A lo largo de vuestras experiencias en el universo local, el hijo creador, cuyo ser personal es comprensible para el hombre, debe compensaros por vuestra incapacidad para alcanzar a comprender en su plena significación al Hijo Eterno, más exclusivamente espiritual, no por ello menos personal. Conforme avancéis a través de Orvontón y Havona, a medida que dejéis atrás las imágenes vívidas y los recuerdos intensos del hijo creador de vuestro universo local, se compensará el tránsito de esta experiencia material y morontial al concebir con cada vez mayor amplitud y comprender cada vez más intensamente al Hijo Eterno del Paraíso, cuya realidad y proximidad aumentarán cada vez más conforme avancéis hacia el Paraíso.
\vs p006 8:8 \pc El Hijo Eterno es un ser personal magno y glorioso. Aunque está fuera del alcance de la mente humana y material poder comprender la realidad del ser personal de este ser infinito, no lo dudéis, él es una persona. Yo sé de lo que hablo. Son prácticamente innumerables las veces que me he encontrado ante la presencia divina de este Hijo Eterno para luego marchar al universo y poner por obra su clemente mandato.
\vsetoff
\vs p006 8:9 [Redactado por un consejero divino designado para desarrollar esta descripción del Hijo Eterno del Paraíso.]
