\upaper{126}{Los dos años cruciales}
\author{Comisión de seres intermedios}
\vs p126 0:1 De todas las experiencias de Jesús en la tierra, las vividas durante su decimocuarto y decimoquinto años de vida fueron las más cruciales. Estos dos años, después de que comenzara a tomar conciencia de su divinidad y de su destino, y antes de que lograra un alto grado de comunicación con su modelador interior, fueron los más difíciles de su intensa vida en Urantia. Para Jesús, este período de dos años significó una gran prueba, una verdadera tentación. Ningún joven humano en su paso por las primeras confusiones y los problemas de adaptación de la adolescencia experimentó nunca una prueba más decisiva que la que Jesús atravesó durante su transición desde la niñez a la edad adulta temprana.
\vs p126 0:2 Este importante período en el desarrollo juvenil de Jesús comenzó con la finalización de la visita a Jerusalén y su regreso a Nazaret. Al principio, María se encontraba feliz con la idea de haber recuperado a su hijo una vez más, de que Jesús hubiese vuelto al hogar para ser un hijo obediente ---no es que antes no lo fuera--- y que en lo sucesivo sería más receptivo a los planes que ella había dispuesto para su vida futura. Pero María no iba a sentir por mucho tiempo el solaz de aquella ilusión materna ni de su desapercibido orgullo familiar; muy pronto estaría aún incluso más desencantada. El muchacho estaba cada vez más en la compañía de su padre; acudía cada vez menos a ella con sus problemas, a la vez que paulatinamente tenían más dificultades para entender su frecuente alternancia entre los asuntos de este mundo y su reflexión sobre la relación con los asuntos de su Padre. Francamente, no lo entendían, pero sentían por él un amor genuino.
\vs p126 0:3 \pc Conforme Jesús crecía, su compasión y amor por el pueblo judío se hacían más profundos, pero, con el paso de los años, se iba desarrollando progresivamente en su mente un sentimiento de justa indignación contra la presencia en el templo del Padre de los sacerdotes designados políticamente. Tenía un gran respeto por los fariseos sinceros y los escribas honestos, pero sentía un gran desprecio por los fariseos hipócritas y los teólogos deshonestos; miraba con desdén a todos aquellos líderes religiosos que no eran sinceros. Cuando examinaba con detenimiento a los dirigentes de Israel, se sentía tentado de mostrarse favorable a la posibilidad de convertirse en el Mesías esperado por los judíos, pero nunca sucumbió a tal tentación.
\vs p126 0:4 \pc El relato de sus proezas entre los sabios del templo de Jerusalén resultó gratificante para todo Nazaret, particularmente para sus antiguos maestros de la escuela de la sinagoga. Durante algún tiempo, los elogios hacia Jesús estuvieron en los labios de todos. El pueblo entero se hacía eco de la sabiduría del joven y su encomiable comportamiento, y pronosticaban que estaba destinado a convertirse en un gran líder de Israel; por fin saldría de Nazaret de Galilea un maestro verdaderamente grande. Y todos esperaban deseosos el momento en el que cumpliera los quince años para que se le pudiera permitir leer las Escrituras de forma regular en la sinagoga el día del \bibemph{sabbat}.
\usection{1. SU DECIMOCUARTO AÑO (AÑO 8 D. C.)}
\vs p126 1:1 Este es el año natural de su decimocuarto cumpleaños. Se había convertido en un buen fabricante de yugos y trabajaba bien tanto la lona como el cuero. También se estaba volviendo rápidamente un experto carpintero y ebanista. Este verano hizo frecuentes visitas a la cima de la colina situada al noroeste de Nazaret para orar y meditar. Paulatinamente, iba tomando más conciencia de la naturaleza de su ministerio de gracia en la tierra.
\vs p126 1:2 Hacía poco más de cien años, esta colina había sido el “alto lugar de Baal”, y ahora era la tumba de Simeón, un renombrado santo varón de Israel. Desde la cumbre de la colina de Simeón, Jesús podía divisar Nazaret y sus alrededores. Desde ahí contemplaba Megido y recordaba la historia del ejército egipcio que consiguió allí su primera gran victoria en Asia, y cómo, posteriormente, un ejército semejante derrotó a Josías, el rey de Judea. No lejos de allí podía divisar Taanac, donde Débora y Barac derrotaron a Sísara. En la distancia podía ver las colinas de Dotán donde, según se le había enseñado, los hermanos de José lo vendieron como esclavo a los egipcios. Luego, al dirigir la mirada hacia Ebal y Gerizim, evocaba las tradiciones de Abraham, Jacob y Abimelec. Y así rememoraba y pensaba detenidamente en los acontecimientos históricos y tradicionales del pueblo de su padre José.
\vs p126 1:3 Prosiguió sus cursos avanzados de lectura bajo la guía de los profesores de la sinagoga e, igualmente, continuó con la educación familiar de sus hermanos y hermanas conforme estos llegaban a la edad adecuada.
\vs p126 1:4 Al principio de este año, José dispuso que los ingresos procedentes de sus propiedades de Nazaret y Cafarnaúm se reservaran para pagar el largo plan de estudios de Jesús en Jerusalén; se había previsto que Jesús iría a Jerusalén en agosto del año siguiente cuando cumpliera los quince años.
\vs p126 1:5 Hacia comienzos de este año, José y María albergaban ya, con frecuencia, sus dudas sobre el destino de su hijo primogénito. Era de hecho un muchacho brillante y amable, pero muy difícil de comprender, muy complicado de dilucidar; y, además, nunca había sucedido nada extraordinario o milagroso. Decenas de veces, su madre, orgullosa, estuvo en ansiosa expectativa, esperando ver a su hijo efectuar algún acto milagroso o sobrehumano, pero sus esperanzas siempre acababan en una dolorosa decepción. Y todo esto la desalentaba, incluso la descorazonaba. Los devotos religiosos de aquellos días creían realmente que los profetas y los hombres de la promesa demostraban siempre su vocación y establecían su autoridad divina realizando milagros y obrando prodigios. Sin embargo, Jesús no hacía nada de estas cosas; por lo que la confusión de sus padres aumentaba constantemente cuando reflexionaban sobre su futuro.
\vs p126 1:6 La mejora de las condiciones económicas de la familia de Nazaret se reflejaba de múltiples maneras en el hogar y, especialmente, en el incremento del número de tablillas blancas y lisas que se usaban como pizarras para escribir con un carboncillo. A Jesús también se le permitió reanudar sus clases de música; era muy aficionado a tocar el arpa.
\vs p126 1:7 \pc En verdad se puede decir que, a lo largo de este año, Jesús “creció en gracia ante los hombres y ante Dios”. Las perspectivas de la familia parecían buenas y el futuro se presentaba prometedor.
\usection{2. LA MUERTE DE JOSÉ}
\vs p126 2:1 Todo iba bien hasta aquel fatídico martes, 25 de septiembre, cuando un mensajero de Séforis trajo a esta casa de Nazaret la trágica noticia de que José, mientras trabajaba en la residencia del gobernador, había resultado gravemente herido por la caída de una grúa. De camino a la casa de José, el mensajero se había detenido en el taller para informar a Jesús del accidente de su padre, y los dos juntos fueron a la casa para comunicarle a María la triste noticia. Jesús deseaba ir de inmediato al lado de su padre, pero María no quería hacer oído a nada que no fuese su deber de apresurarse para estar junto a su marido. Al no conocer la seriedad de sus heridas, dispuso que Santiago, que tenía entonces diez años de edad, la acompañara a Séforis y que Jesús permaneciera en la casa con los niños más pequeños hasta su regreso. Si bien, antes de que llegara María, José ya había fallecido como consecuencia de sus lesiones. Lo trajeron a Nazaret y, al día siguiente, fue enterrado con sus padres.
\vs p126 2:2 \pc Justo en el momento en el que las perspectivas eran buenas y el futuro se vislumbraba prometedor, el infortunio se cebó cruelmente con el padre de esta familia de Nazaret. Los asuntos de este hogar se vieron trastocados, y todos los planes relacionados con Jesús y su futura educación quedaron deshechos. Este joven carpintero, que acababa de cumplir catorce años, despertó a la conciencia de que no solo tenía que llevar a cabo la tarea que su Padre celestial le había encomendado de revelar la naturaleza divina en la tierra y en la carne, sino que su joven naturaleza humana debía igualmente asumir la responsabilidad de ocuparse de su madre, viuda, y de sus siete hermanos y hermanas, no contando con la hermana que aún estaba por nacer. El joven de Nazaret se tornó entonces en el único apoyo y consuelo de esta familia tan súbitamente afligida. De este modo, se permitió que esos hechos de orden natural ocurrieran en Urantia, y que iban a obligar al joven de destino a contraer bien pronto unas obligaciones enormes, pero sumamente instructivas y prácticas, al tener que convertirse en el padre de una familia humana, en el padre de sus propios hermanos y hermanas, en el sostén y protector de su madre, en el custodio del hogar de su padre, el único hogar que llegaría a conocer mientras permaneció en este mundo.
\vs p126 2:3 Jesús aceptó gustosamente las responsabilidades que recayeron sobre él de forma tan repentina y cumplió con ellas fielmente hasta el final. Al menos, aunque de forma trágica, se había dado solución a un gran problema y a un inconveniente que se preveía en su vida: no se suponía ya que debía ir a Jerusalén para estudiar con los rabinos. Siempre fue verdad que Jesús “no se instruyó a los pies de nadie”. Estaba siempre dispuesto a aprender incluso del más humilde de los pequeños, pero su autoridad para enseñar la verdad nunca provino de fuentes humanas.
\vs p126 2:4 Aún no sabía nada de la visita de Gabriel a su madre antes de su nacimiento; únicamente la conoció por Juan el día de su bautismo, al comienzo de su ministerio público.
\vs p126 2:5 \pc A medida que trascurrían los años, este joven carpintero de Nazaret juzgaba cada vez más el valor de las instituciones sociales y de las costumbres religiosas aplicando un test invariable: ¿Qué hacen por el alma humana? ¿Acercan Dios al hombre? ¿Acercan el hombre a Dios? Aunque este joven no se olvidaba completamente de los aspectos recreativos y sociales de la vida, dedicaba su tiempo y sus energías, cada vez más, a conseguir sus dos únicos objetivos: cuidar de su familia y prepararse para hacer en la tierra la voluntad divina de su Padre.
\vs p126 2:6 \pc Este año, los vecinos adquirieron la costumbre de pasarse por su casa en las noches de invierno para escuchar a Jesús tocar el arpa, oír sus historias (el muchacho era un hábil narrador) y escucharlo leer las Escrituras en griego.
\vs p126 2:7 Las cuestiones económicas de la familia continuaron yendo bastante bien, porque había una gran suma de dinero en el momento de la muerte de José. Jesús no tardó en demostrar que poseía un agudo sentido para los negocios y sagacidad en los temas pecuniarios. Era desprendido, pero austero; ahorrativo, aunque generoso; y demostró ser un administrador prudente y eficaz de la herencia de su padre.
\vs p126 2:8 Mas a pesar de todo lo que Jesús y los vecinos de Nazaret pudieran hacer para traer alegría a la casa, María, e incluso los niños, estaban sumidos en la pena. José ya no estaba. Había sido un marido y un padre extraordinario, y todos lo extrañaban. Y su muerte les parecía todavía más trágica al pensar que no habían podido hablar con él o recibir su última bendición.
\usection{3. SU DECIMOQUINTO AÑO (AÑO 9 D. C.)}
\vs p126 3:1 A mediados de este decimoquinto año ---contamos el tiempo de acuerdo con el calendario del siglo XX y no según el año judío---, Jesús ya había asumido firmemente la gestión de los asuntos de su familia. Antes de finalizar este año, los ahorros casi habían desaparecido y tuvieron que enfrentarse a la necesidad de vender una de las casas de Nazaret que José poseía en sociedad con su vecino Jacob.
\vs p126 3:2 La noche del miércoles 17 de abril, del año 9, nació Rut, la pequeña de la familia y, en la medida de sus posibilidades, Jesús procuró sustituir a su padre, alentando y cuidando a su madre durante esta difícil y especialmente triste experiencia. Por casi una veintena de años (hasta que empezó su ministerio público), ningún padre podría haber amado y cuidado a su hija con mayor cariño y devoción que lo hizo Jesús con la pequeña Rut. Fue igualmente un buen padre para todos los demás miembros de su familia.
\vs p126 3:3 \pc A lo largo de este año, Jesús compuso la oración que enseñaría después a sus apóstoles, y que muchos conocen con el nombre de “La oración del Señor”. En cierto modo, se desarrolló a partir de devociones en familia, que tenían muchas fórmulas de alabanza y varias oraciones formales. Tras la muerte de su padre, Jesús intentó enseñar a los niños mayores a que, en sus oraciones, se expresaran de forma personal ---como él tanto disfrutaba haciendo---, pero no alcanzaban a comprender sus consideraciones e invariablemente recurrían de nuevo a las oraciones aprendidas de memoria. En su empeño por estimular a sus hermanos y hermanas de mayor edad a que oraran de forma personal, Jesús trataba de guiarlos sugiriéndoles expresiones y, en poco tiempo, ocurrió que, sin intención alguna por su parte, todos ellos usaban una forma de orar basada en gran medida en las expresiones que Jesús les había sugerido.
\vs p126 3:4 Jesús acabó por renunciar a la idea de que cada miembro de la familia dijera sus oraciones de forma espontánea y, un día de octubre, al final de la tarde, se sentó cerca de la pequeña pero robusta lámpara situada sobre una mesa de piedra de baja altura y, en una tablilla de cedro pulido de menos de cincuenta centímetros por cada lado, escribió, con un trozo de carboncillo, la oración que llegaría a convertirse en algo habitual en las peticiones de la familia.
\vs p126 3:5 \pc Este año Jesús estuvo muy preocupado por los desconcertantes pensamientos que lo acuciaban. Sus responsabilidades familiares habían tenido el efecto de alejarlo, en gran medida, de su idea de realizar de inmediato cualquier plan en respuesta al mandato recibido, en la aparición tenida en Jerusalén, para que “se ocupara de los asuntos de su Padre”. Jesús concluía con razón que velar por la familia de su padre terrenal debía primar sobre cualquier otro deber, que mantener a su familia debía ser su primera prioridad.
\vs p126 3:6 \pc En el transcurso de este año, Jesús encontró un pasaje en el llamado Libro de Enoc que le influyó más tarde para adoptar la expresión “Hijo del Hombre” y que designaría su misión de gracia en Urantia. Había examinado a fondo la idea del Mesías judío y estaba firmemente convencido de que él no iba a ser ese Mesías. Anhelaba ayudar al pueblo de su padre, pero nunca pensó liderar los ejércitos judíos y acabar con la dominación extranjera que sufría Palestina. Sabía que nunca se sentaría en el trono de David en Jerusalén. Tampoco creía que su misión consistía en ser el libertador espiritual o el maestro moral exclusivamente del pueblo judío. De ninguna manera, pues, podía ser su misión en la vida dar cumplimiento a los profundos deseos y a las supuestas profecías mesiánicas de las escrituras hebreas; al menos, no de la manera en la que los judíos entendían estas predicciones de los profetas. Asimismo, estaba seguro de que nunca aparecería como el Hijo del Hombre tal como lo describía el profeta Daniel.
\vs p126 3:7 Pero, cuando le llegara la hora de salir al mundo como su maestro, ¿cómo se llamaría a sí mismo? ¿Qué afirmaciones debería hacer en relación a su misión? ¿Con qué nombre lo llamarían las personas que se convertirían en creyentes de sus enseñanzas?
\vs p126 3:8 \pc Mientras le daba vueltas a estos problemas en su cabeza, halló en la biblioteca de la sinagoga de Nazaret, entre los libros apocalípticos que había estado estudiando, un manuscrito llamado “El libro de Enoc”, y aunque estaba seguro de que el Enoc de los tiempos pasados no lo había escrito, le resultó fascinante, y lo leyó y releyó muchas veces. Había un pasaje que le impresionó particularmente; se trataba de aquel en el que aparecía la expresión “Hijo del Hombre”. El autor de dicho libro continuaba hablando de este Hijo del Hombre, describiendo la labor que habría que realizar en la tierra y explicando que, antes de descender a ella para traer la salvación a la humanidad, había recorrido los atrios de la gloria celestial con su Padre, el Padre de todos; y había dado la espalda a toda esta grandeza y gloria para descender a la tierra y proclamar la salvación a los mortales necesitados. Conforme Jesús leía estos pasajes (sabiendo bien que gran parte del misticismo oriental que se había entremezclado con estas enseñanzas era erróneo), percibía en su corazón y aceptaba en su mente que, de todas las predicciones mesiánicas de las escrituras hebreas y de todas las teorías sobre el libertador judío ninguna estaba tan cerca de la verdad como este relato inserto en el Libro de Enoc, tan solo parcialmente fidedigno; y allí y entonces tomó la determinación de adoptar de forma inicial el nombre de “Hijo del Hombre”. Y esto fue lo que hizo al comenzar después su ministerio público. Jesús poseía una inequívoca capacidad para reconocer la verdad y nunca dudaba en abrazarla, fuese cual fuese la fuente de la que emanase.
\vs p126 3:9 Para entonces, Jesús tenía bien resueltas muchas cuestiones relacionadas con su futura labor en el mundo, pero no mencionó nada de estas cosas a su madre, que continuaba aún aferrada a la idea de que él era el Mesías judío.
\vs p126 3:10 Jesús experimentó en estos años jóvenes un gran desconcierto. Habiendo resuelto de alguna manera la naturaleza de su misión en la tierra, “ocuparse de los asuntos de su Padre” ---anunciar la naturaleza amorosa de su Padre a toda la humanidad--- empezó a reflexionar una vez más sobre las múltiples afirmaciones de las escrituras en relación a la venida de un libertador nacional, fuese un rey o un maestro judío. ¿A qué se referían estas profecías? ¿No era él judío?, ¿o sí lo era? ¿Era él o no de la casa de David? Su madre afirmaba que sí; su padre había determinado que no. Él había concluido que no lo era. Pero, ¿habían confundido los profetas la naturaleza y la misión del Mesías?
\vs p126 3:11 Después de todo, ¿podría su madre tener razón? En la mayoría de las cuestiones, cuando en el pasado habían surgido diferencias de opiniones, era ella quien había tenido la razón. Si fuese él un nuevo maestro y \bibemph{no} el Mesías, ¿cómo podría reconocer al Mesías judío si este apareciese en Jerusalén durante el tiempo de su misión en la tierra? y, también, ¿cuál debería ser su relación con este Mesías judío? Y tras haber emprendido su misión de vida, ¿cuál debería ser su relación con su familia?, ¿con la comunidad y la religión judías?, ¿con el Imperio romano?, ¿con los gentiles y sus religiones? El joven galileo daba vueltas en su mente a cada una de estas cruciales cuestiones, y las analizaba con detenimiento mientras continuaba trabajando en el banco de carpintero, ganándose laboriosamente la vida y sustentando a su madre y a otras ocho bocas hambrientas.
\vs p126 3:12 \pc Antes de terminar este año, María vio que los fondos de la familia disminuían. Pasó la venta de las palomas a Santiago. Poco después, compraron una segunda vaca y, con la ayuda de Miriam, comenzaron a vender leche a sus vecinos de Nazaret.
\vs p126 3:13 \pc Sus profundos períodos de meditación, sus frecuentes desplazamientos a la cima de la colina para orar y las muchas extrañas ideas que de vez en cuando exponía inquietaban mucho a su madre. En ocasiones pensaba que el joven estaba fuera de sí, y luego apaciguaba sus temores, recordando que, después de todo, era un hijo de la promesa y, de alguna manera, diferente de los otros jóvenes.
\vs p126 3:14 Pero Jesús estaba aprendiendo a no expresar todos sus pensamientos, a no comunicar todas sus ideas al mundo, ni siquiera a su propia madre. A partir de este año, Jesús, sistemáticamente, fue dejando de desvelar lo que pasaba por su mente; es decir, hablaba menos de aquellas cosas que las personas normales no podían alcanzar a comprender y que podían llevar a que lo considerasen como alguien raro o diferente de la gente ordinaria. En apariencia, se volvió una persona corriente y convencional, aunque sí anhelaba encontrar a alguien que pudiera entender sus problemas. Ansiaba tener un amigo fiel en quien confiar, pero estos problemas eran demasiado complejos para ser comprendidos por alguien de naturaleza humana. El carácter único de esta excepcional situación le compelía a soportar el peso de su carga en soledad.
\usection{4. EL PRIMER SERMÓN EN LA SINAGOGA}
\vs p126 4:1 Con la llegada de su decimoquinto cumpleaños, Jesús podía ocupar oficialmente el púlpito de la sinagoga el día del \bibemph{sabbat}. En muchas ocasiones anteriores, cuando faltaban oradores, se le había pedido que leyera las Escrituras, pero había llegado el día en el que, de acuerdo con la ley, Jesús podía oficiar el servicio. Así pues, el primer \bibemph{sabbat} tras cumplir los quince años, el jazán dispuso que Jesús oficiara los servicios matutinos de la sinagoga, y, cuando todos los fieles de Nazaret se habían congregado, el joven, habiendo escogido unos pasajes de las Escrituras, se levantó y comenzó a leer:
\vs p126 4:2 \pc “El espíritu del Señor Dios está sobre mí, por cuanto el Señor me ha ungido; me ha enviado para traer buenas nuevas a los mansos, para vendar a los quebrantados de corazón, para pregonar la libertad a los cautivos y para poner en libertad a los presos espirituales; para predicar el año de la buena voluntad de Dios y el día del juicio de nuestro Dios; para consolar a todos los afligidos, para darles esplendor en lugar de ceniza, el aceite de gozo en lugar de luto, un canto de alabanza en lugar del espíritu angustiado, para que puedan ser llamados árboles de justicia, el plantío del Señor, para gloria suya”.
\vs p126 4:3 “Buscad el bien y no el mal para que podáis vivir, y así el Señor, el Dios de los ejércitos, estará con vosotros. Aborreced el mal y amad el bien; y poned juicio en la puerta. Quizás el Señor Dios tendrá piedad del remanente de José”.
\vs p126 4:4 “Lavaos y limpiaos; quitad la maldad de vuestras obras de delante de mis ojos; dejad de hacer lo malo y aprended a hacer el bien; buscad la justicia, socorred al agraviado. Defended al huérfano y amparad a la viuda”.
\vs p126 4:5 “¿Con qué me presentaré ante el Señor, para inclinarme ante el Señor de toda la tierra? ¿Vendré ante él con holocaustos, con becerros de un año? ¿Se agradará el Señor de millares de carneros, de decenas de millares de ovejas o de ríos de aceite? ¿Daré mi primogénito por mi transgresión, el fruto de mi cuerpo por el pecado de mi alma? ¡No!, porque el Señor nos ha mostrado, oh hombres, lo que es bueno. ¿Y qué pide el Señor de vosotros sino que obréis con justicia, que améis la misericordia y que caminéis humildemente con vuestro Dios?”.
\vs p126 4:6 “¿Con quién, pues, haréis semejante a Dios que está sentado sobre el círculo de la tierra? Levantad en alto vuestros ojos y mirad quién ha creado todos estos mundos, quién saca y cuenta su ejército y los llama a todos por su nombre. Él obra estas cosas por la grandeza de su fuerza y por el poder de su dominio, ninguno fallará. Él da esfuerzo al débil y multiplica las fuerzas de los que están cansados. No temáis, porque estoy con vosotros; no desmayéis, porque yo soy vuestro Dios. Os esforzaré y os ayudaré; sí, os sustentaré con la diestra de mi justicia, porque yo soy el Señor vuestro Dios. Y os sostendré de la mano derecha, diciéndoos: no temáis, porque yo os ayudaré”.
\vs p126 4:7 “Y vosotros sois mis testigos, dice el Señor, y mi siervo que yo he escogido para que todos me conozcáis y creáis y entendáis que yo soy el Eterno. Yo, yo mismo, soy el Señor, y fuera de mí no hay quien salve”.
\vs p126 4:8 \pc Y tras haber terminado de leer, se sentó, y la gente se fue a sus casas reflexionando sobre las palabras que tan lleno de gracia les había leído. Sus conciudadanos nunca le habían visto tan magníficamente solemne; nunca le habían oído hablar con aquella voz tan seria y sincera; nunca lo habían visto tan pujante y decidido, con tanta autoridad.
\vs p126 4:9 Este \bibemph{sabbat} por la tarde, Jesús subió a la colina de Nazaret con Santiago y, cuando regresaron a casa, con un carboncillo, sobre dos tablillas pulidas, escribió los Diez Mandamientos en griego. Más tarde, Marta coloreó y adornó estas tablillas, y, durante mucho tiempo, estuvieron colgadas en la pared, por encima del pequeño banco de trabajo de Santiago.
\usection{5. ANTE LAS DIFICULTADES ECONÓMICAS}
\vs p126 5:1 Poco a poco, Jesús y su familia volvieron a la vida sencilla de sus primeros años. Sus ropas e incluso su comida se hicieron más modestas. Tenían abundante leche, mantequilla y queso. Según la estación, disfrutaban de los productos de su huerto, pero cada mes que trascurría necesitaban practicar una mayor austeridad. Su desayuno era muy sencillo; reservaban su mejor comida para la cena. Sin embargo, la falta de recursos entre estos judíos no implicaba inferioridad social.
\vs p126 5:2 Este joven prácticamente alcanzaba ya a comprender de qué modo vivían los hombres de su tiempo. Y, la medida en la que entendía la vida en el hogar, en el campo y en el taller se mostraría después en sus enseñanzas, las cuales revelan plenamente su estrecho contacto con la experiencia humana, en todas sus facetas.
\vs p126 5:3 El jazán de Nazaret continuaba aferrado a la idea de que Jesús se convertiría en un gran maestro, probablemente en el sucesor del renombrado Gamaliel de Jerusalén.
\vs p126 5:4 \pc Aparentemente, todos los planes de Jesús en cuanto a su andadura se vieron frustrados. Según el curso de los acontecimientos, el futuro no parecía muy prometedor. Si bien, no flaqueó ni se desalentó. Siguió con su vida, día tras día, desempeñando bien su deber presente y cumpliendo fielmente las responsabilidades \bibemph{inmediatas} de su situación en la vida. La vida de Jesús constituye un perdurable consuelo para todos aquellos idealistas desencantados.
\vs p126 5:5 El salario normal de un día de trabajo de un carpintero iba reduciéndose lentamente. Hacia finales de este año, trabajando desde el amanecer a la puesta de sol, Jesús podía ganar el equivalente de veinticinco céntimos. Al año siguiente, les resultó difícil pagar los impuestos civiles, por no mencionar las cuotas de la sinagoga y el impuesto de medio siclo que se daba al templo. Durante este año, el recaudador intentó sacarle a Jesús otros impuestos, amenazándole incluso con llevarse su arpa.
\vs p126 5:6 Temiendo que el ejemplar de las Escrituras en griego pudiera ser descubierto y confiscado por los cobradores de impuestos, Jesús, el día de su decimoquinto cumpleaños, lo donó a la biblioteca de la sinagoga de Nazaret, como su ofrenda de madurez al Señor.
\vs p126 5:7 \pc La mayor conmoción de su decimoquinto año le sobrevino cuando Jesús fue a Séforis para recibir la resolución de Herodes en relación a la apelación interpuesta ante él, por la disputa sobre la cantidad de dinero que se le debía a José en el momento de su muerte por accidente. Jesús y María tenían esperanzas de recibir una considerable suma de dinero, pero el tesorero de Séforis les había ofrecido una cantidad insignificante. Los hermanos de José habían apelado ante el mismo Herodes, y ahora se encontraba Jesús en el palacio cuando oyó a Herodes decretar que a su padre no se le debía nada. Debido a esta decisión tan injusta, Jesús nunca más volvió a confiar en Herodes Antipas. No es de extrañar que en una ocasión se refiriera a él como “ese zorro”.
\vs p126 5:8 Durante este año y los siguientes, el minucioso trabajo en el banco de carpintero privó a Jesús de la posibilidad de mezclarse con los viajeros de las caravanas. Su tío ya se había hecho cargo de la tienda de suministros de la familia, y Jesús trabajaba enteramente en el taller de la casa, donde estaba cerca para ayudar a María con la familia. Por esta época, comenzó a enviar a Santiago a la zona de las caravanas para recabar información sobre lo que sucedía en el mundo, buscando con ello mantenerse al corriente de las noticias del día.
\vs p126 5:9 A medida que crecía hacia la edad adulta, pasó por los mismos conflictos y confusiones de cualquier joven ordinario de épocas anteriores y posteriores. Y la ardua experiencia de mantener a su familia resultó en efecto una salvaguarda frente al hecho de tener demasiado tiempo que dedicar a la meditación ociosa o a la complacencia mística.
\vs p126 5:10 \pc En este año, Jesús arrendó una gran porción de terreno justo al norte de su casa, que se parceló para convertirlo en un huerto familiar. Cada uno de los niños mayores tenía su propio huerto individual, y entre ellos se entabló una entusiasta competencia respecto a sus labores agrícolas. Durante la temporada de cultivo de las hortalizas, el hermano mayor pasaba diariamente algún tiempo trabajando con sus hermanos y hermanas menores en el huerto, y, cuando lo hacía, numerosas veces albergó el deseo de vivir todos juntos en una granja en el campo donde poder disfrutar de la libertad de una vida sin obstáculos. Pero no se hallaban en el campo; y Jesús, que era un joven muy práctico a la vez que idealista, abordó los problemas con energía e inteligencia conforme se le presentaban, e hizo todo lo que estuvo a su alcance para que él y su familia se amoldasen a las realidades de su situación y, dentro de su condición, poder lograr la mayor satisfacción posible de sus anhelos individuales y colectivos.
\vs p126 5:11 Hubo un momento en el que Jesús abrigó vagamente la esperanza de reunir los recursos suficientes, siempre que pudieran cobrar la importante cantidad de dinero que le debían a su padre por sus trabajos en el palacio de Herodes, para acometer la compra de una pequeña granja. Había considerado seriamente trasladar a su familia al campo, pero ante la negativa de Herodes de pagarles lo que se le adeudaba a José, renunciaron a sus pretensiones de ser propietarios de una casa en el campo. Estando así las cosas, se las arreglaron para disfrutar de muchas de las experiencias de la vida rural; en aquel momento tenían, además de las palomas, tres vacas, cuatro ovejas, una manada de pollos, un asno y un perro. Incluso los más pequeños tenían sus propias obligaciones que realizar de forma habitual, dentro del bien organizado sistema que guiaba la vida hogareña de esta familia de Nazaret.
\vs p126 5:12 \pc Al término de su decimoquinto año de vida, concluyó la travesía de Jesús por un arriesgado y difícil período de la existencia humana, ese momento de transición entre los años más placenteros de la infancia y la conciencia de la inminente edad adulta, con su aumento de responsabilidades y oportunidades para adquirir una mayor experiencia en el desarrollo de un carácter noble. Había acabado su periodo de crecimiento mental y físico, y empezaba ya la verdadera andadura de este joven de Nazaret.
