\upaper{144}{En Gilboa y en la Decápolis}
\author{Comisión de seres intermedios}
\vs p144 0:1 Pasaron los meses de septiembre y octubre retirados en un campamento apartado en las laderas del monte Gilboa. Jesús pasó el mes de septiembre aquí a solas con sus apóstoles, enseñándoles y formándoles en las verdades del reino.
\vs p144 0:2 Había varias razones que justificaban este retiro de Jesús y sus apóstoles a un lugar aislado en las fronteras entre Samaria y la Decápolis. Los dignatarios religiosos de Jerusalén eran muy hostiles; Herodes Antipas mantenía todavía a Juan recluido en la cárcel, temiendo tanto ponerlo en libertad como ejecutarlo, mientras continuaba con la sospecha de que había algún tipo de vinculación entre Juan y Jesús. Era, pues, en estas condiciones, desaconsejable actuar abiertamente en Judea o en Galilea. Había una tercera razón: el lento aumento de la tensión entre los líderes de los discípulos de Juan y los apóstoles de Jesús, que se había ido agravando conforme el número de creyentes crecía.
\vs p144 0:3 Jesús era consciente de que el período preliminar de enseñanza y predicación llegaba a su fin, que el siguiente paso suponía el inicio de la labor última y plena de su vida en la tierra, y no quería que esto pudiese de ninguna manera resultar difícil o humillante para Juan el Bautista. Por ello, Jesús, había tomado la decisión de alejarse durante algún tiempo para que los apóstoles pudieran repasar lo aprendido y realizar, después, un trabajo discreto en las ciudades de la Decápolis, hasta que a Juan se le llevara a la muerte o quedara libre y se uniese a ellos para trabajar en común.
\usection{1. EL CAMPAMENTO DE GILBOA}
\vs p144 1:1 A medida que pasaba el tiempo, más crecía la devoción de los doce hacia Jesús y más comprometidos estaban con la labor del reino. Su devoción era, mayormente, una cuestión de lealtad personal. No llegaban a entender las numerosas facetas de su enseñanza, como tampoco comprendían del todo la naturaleza de Jesús ni el significado de su ministerio de gracia en la tierra.
\vs p144 1:2 Jesús dejó bien claro a sus apóstoles que eran tres las razones por las que se habían retirado a un lugar apartado:
\vs p144 1:3 1. Confirmar su entendimiento, y fe, del evangelio del reino.
\vs p144 1:4 2. Dejar que la oposición existente contra su obra en Judea y Perea se apaciguara.
\vs p144 1:5 3. Aguardar el destino que le deparaba a Juan el Bautista.
\vs p144 1:6 Mientras estaban en Gilboa, Jesús les refirió a los doce muchas cosas acerca de su vida temprana y experiencias en el monte Hermón; también les desveló algo de lo que sucedió en las colinas durante los cuarenta días inmediatamente posteriores a su bautismo. Y les encargó explícitamente que no dijeran nada a nadie sobre esto hasta que no hubiese regresado al Padre.
\vs p144 1:7 Durante esas semanas de septiembre descansaron, conversaron, recordaron sus experiencias desde que Jesús los llamó por primera vez al servicio, y se comprometieron de manera activa para aunar todo lo que el Maestro les había enseñado hasta entonces. En cierta medida, todos percibían que esta sería su última oportunidad de disfrutar de un descanso prolongado. Se percataron de que su próxima tarea pública en Judea o en Galilea señalaría el comienzo de la proclamación final del reino venidero, pero tenían poca o ninguna idea precisa en cuanto a cómo sería el reino cuando llegara. Juan y Andrés creían que el reino ya había venido; Pedro y Santiago pensaban que aún estaba por llegar; Natanael y Tomás confesaban, con franqueza, que estaban desconcertados; Mateo, Felipe y Simón Zelotes se encontraban inseguros y confusos al respecto; los gemelos estaban felizmente ignorantes de la controversia; y Judas Iscariote permanecía callado, evasivo.
\vs p144 1:8 Gran parte del tiempo, Jesús estaba solo en la montaña cercana al campamento. En ocasiones, se llevaba a Pedro, a Santiago o a Juan con él, pero con mucha frecuencia se marchaba a solas para orar o estar en comunión. Tras el bautismo de Jesús y los cuarenta días en las colinas de Perea, no sería muy apropiado considerar estos momentos de comunión con su Padre como oración, ni tampoco sería congruente considerarlos como adoración; pero sería totalmente correcto aludir a ellos como momentos de comunión personal con su Padre.
\vs p144 1:9 A lo largo de todo el mes de septiembre, las charlas se centraron en el tema de la oración y de la adoración. Después de haber tratado de la adoración durante algunos días, Jesús, finalmente, dio su memorable discurso sobre la oración, respondiendo a la petición de Tomás: “Maestro, enséñanos a orar”.
\vs p144 1:10 Juan había enseñado a sus discípulos una oración, una oración para hallar salvación en el reino venidero. Aunque Jesús nunca prohibió a sus seguidores que usaran la forma de orar de Juan, los apóstoles se percataron muy pronto de que su Maestro no aprobaba del todo la práctica de orar usando fórmulas establecidas y ceremoniosas. No obstante, los creyentes estaban constantemente solicitando que se les enseñara a orar y los doce sentían un enorme deseo de saber qué tipo de peticiones serían de la aprobación de Jesús. Y fue principalmente debido a esta necesidad, por parte de la gente común de aprender una forma sencilla de orar, por lo que Jesús, en respuesta a esta solicitud de Tomás, accedió a sugerírsela. Jesús impartió esta enseñanza una tarde durante su tercera semana de estancia en el monte Gilboa.
\usection{2. CHARLA SOBRE LA ORACIÓN}
\vs p144 2:1 “Es cierto que Juan os enseñó un sencillo modo de orar: ‘¡Oh Padre, límpianos del pecado, muéstranos tu gloria, revélanos tu amor y permite que tu espíritu santifique nuestro corazón por siempre jamás, amén!’. Os la impartió para que tuvieseis algo que enseñar a las multitudes. No pretendía que emplearais fórmulas establecidas que fuesen la expresión de vuestras propias almas cuando orarais.
\vs p144 2:2 La oración es enteramente un modo personal y espontáneo de expresar la actitud del alma hacia el espíritu; la oración debe ser comunión filial y manifestación fraternal. La oración, cuando está modelada por el espíritu, nos lleva a progresar y a colaborar con él. La oración ideal es una forma de comunión espiritual que conduce a la adoración inteligente. La verdadera oración es la actitud sincera de quien busca alcanzar el cielo para lograr sus ideales.
\vs p144 2:3 “La oración es el aliento del alma y debe llevaros a perseverar en vuestro intento por conocer la voluntad del Padre. Si alguno de vosotros tiene un vecino y va a verle a la media noche para decirle: ‘amigo, préstame tres panes, porque un amigo mío ha venido a mí de viaje y no tengo nada que ofrecerle’; y tu vecino responde, ‘no me molestes; mi puerta ya está cerrada y mis hijos están conmigo en la cama. No puedo levantarme a dártelo’, pero perseveras, explicándole que tu amigo tiene hambre, y que no tienes comida que darle. Yo te digo que, aunque tu vecino no quiere levantarse para darte pan porque eres su amigo, no obstante, por tu inoportunidad se levantará y te dará todos los panes que necesites. Y si, entonces, con perseverancia se gana el favor de un simple mortal, imaginaos cuanto más logrará vuestra perseverancia en el espíritu el pan de vida de las manos voluntariosas del Padre de los cielos. Por eso os digo de nuevo: Pedid, y se os dará; buscad, y hallaréis; llamad, y se os abrirá. Porque todo aquel que pide, recibe; y el que busca, halla; y al que llama a la puerta de la salvación, se le abrirá.
\vs p144 2:4 “¿Qué padre hay de vosotros, que si su hijo le pide con insensatez, dudaría en darle conforme a su sabiduría paterna más que según la petición equivocada de su hijo? Si el niño necesita pan, ¿acaso le dará una piedra solo porque imprudentemente la pidió? O si necesita un pescado, ¿acaso le dará una culebra de agua solo porque pueda haber aparecido en la red con el pescado y el niño neciamente pide la serpiente? Pues si vosotros, siendo mortales y finitos, sabéis cómo responder a las peticiones y dar buenas y convenientes dádivas a vuestros hijos, ¿cuánto más vuestro Padre celestial dará el espíritu y numerosas otras bendiciones a los que se las pidan? Los hombres deben siempre orar sin desalentarse.
\vs p144 2:5 “Os contaré la historia de un juez que vivía en una ciudad inicua. Este juez ni temía a Dios ni respetaba a los hombres. Había también en aquella ciudad una viuda necesitada, la cual venía a menudo a este injusto juez y le decía: “Defiéndeme de mi adversario.’ Por algún tiempo, él no quiso darle oído; pero después de esto dijo dentro de sí: ‘Aunque no temo a Dios ni tengo consideración por los hombres, sin embargo, porque esta viuda me es molesta, le haré justicia, no sea que viniendo de continuo me agote la paciencia’. Os narro estas historias para animaros a perseverar en la oración; no para insinuar que vuestras peticiones puedan hacer cambiar al Padre de lo alto, que es justo y recto. Vuestra perseverancia, sin embargo, no es para ganar el favor de Dios, sino para modificar vuestra actitud aquí en la tierra y agrandar la capacidad de vuestra alma para recibir el espíritu.
\vs p144 2:6 “Pero cuando oráis, lo hacéis con muy poca fe. La auténtica fe es capaz de mover las montañas de dificultades materiales que puedan hallarse en el camino de la expansión del alma y del progreso espiritual”.
\usection{3. LA ORACIÓN DEL CREYENTE}
\vs p144 3:1 Pero los apóstoles no se quedaron aún satisfechos; deseaban que Jesús les diese una oración modelo para poder enseñársela a los nuevos discípulos. Tras escuchar esta charla sobre la oración, Santiago Zebedeo dijo: “Muy bien, Maestro, pero querríamos una forma de orar no ya para nosotros mismos sino para los nuevos creyentes, que con tanta frecuencia nos ruegan: ‘Enseñadnos cómo orar al Padre de los cielos de la mejor manera’”.
\vs p144 3:2 Cuando Santiago había terminado de hablar, Jesús dijo: “Si aún deseáis ese tipo de oración, os daré la que enseñé en Nazaret a mis hermanos y hermanas”:
\vsetoff
\vs p144 3:3 Padre nuestro que estás en los cielos,
\vs p144 3:4 \hsetoff Santificado sea tu nombre.
\vs p144 3:5 Venga tu reino; hágase tu voluntad
\vs p144 3:6 \hsetoff así en la tierra como en el cielo.
\vs p144 3:7 Danos hoy nuestro pan para mañana;
\vs p144 3:8 \hsetoff Reconforta nuestras almas con el agua de la vida.
\vs p144 3:9 Y perdónanos nuestras deudas
\vs p144 3:10 \hsetoff como nosotros también hemos perdonado a nuestros deudores.
\vs p144 3:11 Sálvanos de la tentación, líbranos del mal,
\vs p144 3:12 \hsetoff y haznos cada vez más perfectos como tú lo eres.
\vsetoff
\vs p144 3:13 No es de extrañar que los apóstoles solicitaran a Jesús que les enseñase una oración modelo para presentársela a los creyentes. Juan el Bautista había enseñado algunas oraciones a sus seguidores; todos los grandes maestros les habían dado forma para sus pupilos. Los maestros religiosos de los judíos tenían unas veinticinco o treinta oraciones establecidas, que recitaban en las sinagogas e incluso en las esquinas de las calles. Jesús era particularmente reacio a orar en público. Hasta este momento, los doce le habían oído orar solo algunas pocas veces. Veían que pasaba noches enteras orando o en adoración, y sentían mucha curiosidad por saber el modo o la forma de sus peticiones. Les era muy difícil dar respuesta a las multitudes cuando les pedían que les enseñaran a orar como Juan había hecho con sus discípulos.
\vs p144 3:14 Jesús instruyó a los doce que debían orar siempre en secreto; que cuando lo hiciesen debían buscar la solitaria paz de la naturaleza de los alrededores o ir a su habitación y cerrar las puertas.
\vs p144 3:15 Tras la muerte de Jesús y su ascensión al Padre, para muchos creyentes se convirtió en costumbre añadir al final de esta oración, llamada la oración del Señor, lo siguiente: “En el nombre del Señor Jesucristo”. Aún más tarde, se perdieron dos líneas en su copiado y se agregó además una cláusula: “porque tuyo es el Reino, el poder y la gloria, por siempre jamás”.
\vs p144 3:16 Jesús dio esta oración a los apóstoles para que se dijera de forma colectiva, tal como se hacía en su casa de Nazaret. Nunca impartió ninguna oración sistematizada para que se orase de forma personal, sino solo peticiones que se emplearan a nivel de grupo, familia o reuniones sociales. E incluso así, nunca se prestó a hacerlo de forma voluntaria.
\vs p144 3:17 Jesús enseñó que la oración para que fuera efectiva debía ser:
\vs p144 3:18 1. Desinteresada: no solamente para uno mismo.
\vs p144 3:19 2. Creíble: conforme a la fe.
\vs p144 3:20 3. Sincera: honesta de corazón.
\vs p144 3:21 4. Inteligente: según la propia luz.
\vs p144 3:22 5. Confiada: en sumisión a la voluntad omnisapiente del Padre.
\vs p144 3:23 Cuando Jesús pasaba noches enteras en la montaña en oración, lo hacía principalmente para pedir por sus discípulos, en particular por los doce. El Maestro oraba muy poco para sí mismo, aunque se sumía bastante en una adoración de entendimiento de su Padre del Paraíso y de comunión con él.
\usection{4. MÁS SOBRE LA ORACIÓN}
\vs p144 4:1 Durante los días que siguieron a su charla sobre la oración, los apóstoles continuaron preguntando al Maestro sobre esta práctica de veneración tan sumamente importante. La instrucción sobre la oración y la adoración que Jesús impartió a los apóstoles en estos días se puede resumir y reformular, en términos modernos, de la siguiente manera:
\vs p144 4:2 La repetición ferviente y anhelante de cualquier petición, cuando dicha oración es la expresión sincera de un hijo de Dios y se pronuncia con fe, al margen de lo desaconsejable que sea o de la imposibilidad de darle una respuesta inmediata, nunca deja de expandir la capacidad de recepción espiritual del alma.
\vs p144 4:3 En toda oración, recordad que la filiación es un \bibemph{don}. Ningún niño ha de hacer nada para \bibemph{obtener} el estatus de hijo o de hija. El niño terrenal llega a existir gracias a la voluntad de sus padres. No obstante, el hijo de Dios llega a la gracia y a la nueva vida del espíritu mediante la voluntad del Padre celestial. Así pues, el reino de los cielos ---la filiación divina--- debe \bibemph{recibirse} como si se fuese un niño pequeño. La rectitud ---o el desarrollo progresivo del carácter--- se adquiere, pero la filiación se recibe por la gracia y a través de la fe.
\vs p144 4:4 La oración llevó a Jesús a la sublime comunión de su alma con los Gobernantes Supremos del universo de los universos. La oración llevará a los mortales de la tierra a la comunión con el Padre, que es verdadera adoración. La capacidad de recepción espiritual del alma determina la cantidad de bendiciones celestiales que se pueda poseer personalmente y reconocer conscientemente como respuesta a la oración.
\vs p144 4:5 La oración y la adoración, compañera suya, es un modo de distanciarse de la rutina diaria de la vida, del monótono ajetreo de la existencia material. Constituye un medio de aproximarse a la realización espiritualizada de uno mismo y al logro de la individualidad intelectual y religiosa.
\vs p144 4:6 La oración es un antídoto contra una introspección lesiva. Al menos, la oración, como el Maestro la enseñó, tiene dicho efecto benefactor para el alma. Jesús empleó permanentemente el poder beneficioso de la oración para el bien de sus semejantes. Por lo general, el Maestro oraba en plural, no en singular. Jesús únicamente pidió por sí mismo en las grandes crisis de su vida terrenal.
\vs p144 4:7 La oración es el aliento de la vida del espíritu en medio de la civilización material de las razas humanas. La adoración constituye la salvación para las generaciones de mortales que están en búsqueda de placeres.
\vs p144 4:8 Orar puede semejarse a la recarga de las baterías espirituales del alma, por lo que la adoración puede compararse al hecho de sintonizar el alma para que sea capaz de captar las transmisiones del espíritu infinito del Padre Universal en el universo.
\vs p144 4:9 La oración es la mirada sincera y anhelante del hijo a su Padre espiritual; es un proceso psicológico en el que se intercambia la voluntad humana por la voluntad divina. La oración forma parte del plan divino para remodelar lo que es en lo que debería ser.
\vs p144 4:10 Una de las razones por las que Pedro, Santiago y Juan, que tan a menudo acompañaban a Jesús en sus largas vigilias nocturnas, nunca le escucharon orar era porque su Maestro no solía expresar sus oraciones con palabras. Prácticamente hacía sus oraciones en espíritu y con el corazón ---silenciosamente---.
\vs p144 4:11 De todos los apóstoles, Pedro y Santiago eran los que más cerca estuvieron de comprender las enseñanzas del Maestro sobre la oración y la adoración.
\usection{5. OTRAS FORMAS DE ORACIÓN}
\vs p144 5:1 Ocasionalmente, durante el resto de su estancia en la tierra, Jesús puso en conocimiento de los apóstoles otras formas diferentes de orar, pero solo lo hizo para ilustrar diferentes cuestiones, prescribiéndoles que no enseñaran a las multitudes estas “oraciones para ilustrar parábolas”. Muchas de ellas tenían su origen en otros planetas habitados, pero Jesús no reveló este hecho a los doce. Entre estas oraciones, estaban las siguientes:
\vsetoff
\vs p144 5:2 Padre nuestro, en quien los mundos del universo consisten,
\vs p144 5:3 \hsetoff enaltecido sea tu nombre y tu todo glorioso carácter.
\vs p144 5:4 Tu presencia nos circunda, y tu gloria se manifiesta
\vs p144 5:5 \hsetoff imperfecta a través de nosotros, mientras se muestra en perfección en lo alto.
\vs p144 5:6 Danos hoy las fuerzas vivificadoras de la luz,
\vs p144 5:7 \hsetoff y no dejes que nos extraviemos en las pérfidas sendas de nuestra imaginación,
\vs p144 5:8 porque tuya es la morada gloriosa, el poder sempiterno,
\vs p144 5:9 \hsetoff y, para nosotros, el don eterno del amor infinito de tu Hijo.
\vs p144 5:10 Que así sea, en perpetua verdad.
\separatorline
\vs p144 5:12 Nuestro Padre y Creador, que estás en el centro del universo,
\vs p144 5:13 \hsetoff concédenos tu naturaleza y danos tu carácter.
\vs p144 5:14 Haznos tus hijos e hijas por la gracia
\vs p144 5:15 \hsetoff y glorifica tu nombre por medio de nuestros logros eternos.
\vs p144 5:16 Danos tu espíritu modelador y rector para que viva y habite en nosotros
\vs p144 5:17 \hsetoff y podamos hacer tu voluntad en esta esfera tal como los ángeles cumplen tus mandatos en luz.
\vs p144 5:18 Sostennos hoy en nuestro progreso por el sendero de la verdad.
\vs p144 5:19 \hsetoff Líbranos de la inercia, del mal y de cualquier transgresión pecaminosa.
\vs p144 5:20 Sé paciente con nosotros así como nosotros mostramos amorosa benevolencia hacia nuestros semejantes.
\vs p144 5:21 \hsetoff Derrama el espíritu de tu misericordia en nuestros corazones mortales.
\vs p144 5:22 Llévanos de la mano, paso a paso, por el incierto laberinto de la vida,
\vs p144 5:23 \hsetoff y cuando venga nuestro fin, recibe en tu seno nuestro fiel espíritu.
\vs p144 5:24 Pero cúmplase tu voluntad y no nuestros deseos.
\separatorline
\vs p144 5:26 Nuestro Padre celestial, perfecto y recto,
\vs p144 5:27 \hsetoff guía y dirige hoy nuestra andadura.
\vs p144 5:28 Santifica nuestros pasos y da armonía a nuestros pensamientos.
\vs p144 5:29 \hsetoff Por siempre, llévanos por los caminos del progreso eterno.
\vs p144 5:30 Llénanos de sabiduría hasta nuestra máxima capacidad
\vs p144 5:31 \hsetoff y vivifícanos con tu energía infinita.
\vs p144 5:32 Inspíranos con la conciencia divina de
\vs p144 5:33 \hsetoff la presencia y guía de las multitudes seráficas.
\vs p144 5:34 Guíanos, para siempre, en nuestro ascenso por la senda de la luz;
\vs p144 5:35 \hsetoff sostennos bien en el día del gran juicio.
\vs p144 5:36 Haznos como tú eres en tu gloria eterna.
\vs p144 5:37 \hsetoff Y, en lo alto, acógenos para que te sirvamos sin fin.
\separatorline
\vs p144 5:39 Padre nuestro que habitas en el misterio,
\vs p144 5:40 \hsetoff revélanos tu carácter santo.
\vs p144 5:41 Deja que tus hijos de la tierra hoy
\vs p144 5:42 \hsetoff vean el camino, la luz y la verdad.
\vs p144 5:43 Muéstranos el sendero del eterno progreso,
\vs p144 5:44 \hsetoff Y danos la voluntad para caminar en él.
\vs p144 5:45 Instaura en nosotros tu reino divino
\vs p144 5:46 \hsetoff y otórganos así el dominio pleno de nosotros mismos.
\vs p144 5:47 No permitas que nos extraviemos por las sendas de la oscuridad y de la muerte;
\vs p144 5:48 \hsetoff llévanos sempiternamente junto a las aguas de vida.
\vs p144 5:49 Escucha estas nuestras oraciones por tu propia causa;
\vs p144 5:50 \hsetoff complácete en hacernos cada vez más como tú eres.
\vs p144 5:51 Y, al final, por tu hijo divino,
\vs p144 5:52 \hsetoff acógenos en tus brazos eternos.
\vs p144 5:53 Pero que no se haga nuestra voluntad sino la tuya.
\separatorline
\vs p144 5:55 Glorioso Padre y Madre, unidos como un solo progenitor,
\vs p144 5:56 \hsetoff haz que seamos leales a tu naturaleza divina.
\vs p144 5:57 Obra para que tu propio ser viva de nuevo en nosotros y a través de nosotros
\vs p144 5:58 \hsetoff por el don y la dádiva de tu espíritu divino,
\vs p144 5:59 manifestándote así imperfectamente en esta esfera
\vs p144 5:60 \hsetoff mientras lo haces perfecta y majestuosamente en lo alto.
\vs p144 5:61 Impártenos cada día el dulce ministerio de la hermandad
\vs p144 5:62 \hsetoff y llévanos en cada momento por el sendero del servicio amoroso.
\vs p144 5:63 Sé por siempre invariablemente paciente con nosotros
\vs p144 5:64 \hsetoff así como nosotros mostramos tu paciencia en nuestros hijos.
\vs p144 5:65 Danos la sabiduría divina que hace bien todas las cosas
\vs p144 5:66 \hsetoff y el amor infinito que es generosamente benigno con todas las criaturas.
\vs p144 5:67 Concédenos tu paciencia y tu clemencia,
\vs p144 5:68 \hsetoff para que nuestra caridad dé cabida a los débiles del mundo.
\vs p144 5:69 y cuando nuestra andadura acabe, haz que honre y glorifique tu nombre,
\vs p144 5:70 \hsetoff que agrade a tu buen espíritu y que satisfaga a quienes asisten nuestras almas.
\vs p144 5:71 Pero que no sea según nuestro deseo, nuestro amoroso Padre, sino conforme al tuyo para el eterno bien de tus hijos mortales.
\vs p144 5:72 \hsetoff Que así sea.
\separatorline
\vs p144 5:74 Nuestra Fuente fiel en plenitud y Centro todopoderoso,
\vs p144 5:75 \hsetoff reverenciado y santificado sea el nombre de tu Hijo, en todo generosamente benigno.
\vs p144 5:76 Tus dádivas y tus bendiciones han descendido sobre nosotros,
\vs p144 5:77 \hsetoff fortalécenos para cumplir tu voluntad y obedecer tus mandatos.
\vs p144 5:78 Danos en cada momento el sustento del árbol de la vida;
\vs p144 5:79 \hsetoff revitalízanos cada día con las aguas vivas del río de la vida.
\vs p144 5:80 Paso a paso, aléjanos de la oscuridad y llévanos a la luz divina.
\vs p144 5:81 \hsetoff Renueva nuestras mentes por la acción transformadora del espíritu interior,
\vs p144 5:82 y cuando por último nos llegue el fin como mortales,
\vs p144 5:83 \hsetoff recíbenos en ti y danos cobijo en la eternidad.
\vs p144 5:84 Corónanos con las diademas celestiales del servicio fecundo,
\vs p144 5:85 \hsetoff y glorificaremos al Padre, al Hijo y a la Influencia Santa.
\vs p144 5:86 Que así sea, por todo un universo sin final.
\separatorline
\vs p144 5:88 Padre nuestro que habitas en los lugares secretos del universo,
\vs p144 5:89 \hsetoff sea tu nombre honrado, tu misericordia venerada y tu juicio respetado.
\vs p144 5:90 Que el sol de la rectitud brille sobre nosotros al mediodía,
\vs p144 5:91 \hsetoff en tanto que te suplicamos que guíes nuestros pasos descarriados en el crepúsculo.
\vs p144 5:92 Llévanos de la mano por los caminos que tú mismo escojas,
\vs p144 5:93 \hsetoff y no nos abandones cuando la senda sea ardua y las horas oscuras.
\vs p144 5:94 No nos olvides tal como nosotros tantas veces te abandonamos y olvidamos.
\vs p144 5:95 \hsetoff Pero sé misericordioso y ámanos como nosotros deseamos amarte a ti.
\vs p144 5:96 Míranos con benevolencia y perdónanos con misericordia
\vs p144 5:97 \hsetoff como nosotros perdonamos en justicia a quienes nos afligen y hieren.
\vs p144 5:98 Que el amor, la devoción y el ministerio de gracia del Hijo majestuoso
\vs p144 5:99 \hsetoff nos proporcionen vida sempiterna con tu misericordia y amor interminables.
\vs p144 5:100 Que el Dios de los universos nos otorgue la plenitud de su espíritu;
\vs p144 5:101 \hsetoff Danos la gracia de rendirnos a las directrices de este espíritu.
\vs p144 5:102 Por el ministerio amoroso de las fieles multitudes seráficas
\vs p144 5:103 \hsetoff que el Hijo nos guíe y lleve hasta el término de la era.
\vs p144 5:104 Haznos por siempre más y más semejantes a ti
\vs p144 5:105 \hsetoff y, cuando llegue nuestro fin, recíbenos en el acogimiento eterno del Paraíso.
\vs p144 5:106 Que así sea, en el nombre del Hijo de gracia
\vs p144 5:107 \hsetoff y para el honor y gloria del Padre Supremo.
\vsetoff
\vs p144 5:108 Aunque no estaban autorizados para exponer estas lecciones sobre la oración en sus enseñanzas públicas, todas estas revelaciones resultaron muy beneficiosas para sus experiencias religiosas a nivel personal. Jesús utilizó estos y otros modelos de oración como ilustraciones para la instrucción privada de los doce, y se ha concedido un permiso específico para incluir estas siete muestras de oraciones en la presente narración.
\usection{6. REUNIONES CON LOS APÓSTOLES DE JUAN}
\vs p144 6:1 A principios de octubre, Felipe y algunos de sus compañeros estaban en una aldea cercana comprando alimentos, cuando se encontraron con algunos de los apóstoles de Juan el Bautista. A raíz de este encuentro fortuito en el mercado, y durante tres semanas, se llevaron a cabo, en el campamento de Gilboa, reuniones entre los apóstoles de Jesús y los de Juan. Juan había nombrado recientemente, siguiendo el precedente de Jesús, a doce apóstoles. Lo había hecho así en respuesta a la petición de Abner, jefe de sus leales seguidores. Jesús estuvo presente en el campamento de Gilboa durante la primera semana de estas reuniones conjuntas, pero se ausentó las últimas dos semanas.
\vs p144 6:2 A comienzos de la segunda semana de este mismo mes, Abner había reunido a todos sus compañeros en el campamento de Gilboa y estaba listo para entrar en diálogo con los apóstoles de Jesús. Durante tres semanas, estos veinticuatro hombres celebraron encuentros tres veces al día, seis días por semana. La primera semana, Jesús se unió a ellos entre sus sesiones de la mañana, tarde y noche. Querían que el Maestro estuviese con ellos y presidiese las deliberaciones entre ambos grupos, pero él se negó rotundamente a participar en sus discusiones, aunque accedió a hablar con ellos en tres ocasiones. Estas charlas que Jesús impartió a los veinticuatro trataban de la compasión, la cooperación y la tolerancia.
\vs p144 6:3 Andrés y Abner se turnaban en la presidencia de dichas reuniones conjuntas de ambos grupos apostólicos. Estos hombres tenían muchas cuestiones difíciles que debatir y numerosos problemas que resolver. Repetidas veces, remitían sus problemas a Jesús, aunque únicamente para oírle decir: “Solo me preocupan vuestros problemas personales y puramente religiosos. Represento al Padre ante cada persona \bibemph{de forma individual,} no ante el grupo. Si tenéis dificultades personales en vuestras relaciones con Dios, venid a mí, y yo os escucharé y os aconsejaré para hallar alguna solución. Pero cuando se trate de coordinar interpretaciones humanas discrepantes sobre cuestiones religiosas y la socialización de la religión, sois vosotros los que os tenéis que encargar de solucionar vuestras dificultades, según vuestros propios criterios. Si bien, os manifiesto mi comprensión e interés y, cuando lleguéis a alguna conclusión sobre estos asuntos espiritualmente intrascendentales, siempre que lleguéis todos a un acuerdo, os prometo con antelación mi total aprobación y mi entusiasta cooperación. Y ahora, para no ser un obstáculo en vuestras deliberaciones, os dejaré por dos semanas. No os inquietéis por mí, porque volveré a vosotros. Estaré en los asuntos de mi Padre, ya que tenemos otros mundos, además de este”.
\vs p144 6:4 Tras estas palabras, Jesús descendió por la ladera de la montaña, y no volvieron a verlo en dos semanas completas. Y nunca supieron dónde había ido ni qué había hecho durante esos días. Trascurrió algún tiempo antes de que los veinticuatro pudiesen calmarse y considerar seriamente sus problemas; estaban muy desconcertados por la ausencia del Maestro. Sin embargo, en una semana, estaban de nuevo debatiéndolos a fondo, aunque sin poder acudir a Jesús para que les asistiera.
\vs p144 6:5 La primera cuestión sobre la que llegaron todos a un acuerdo fue respecto a la adopción de la oración que Jesús les había enseñado recientemente. Se votó por unanimidad a favor de que esta fuese la única oración que ambos grupos de apóstoles enseñarían a los creyentes.
\vs p144 6:6 A continuación, decidieron que, mientras Juan viviera, ya fuese en prisión o fuera de ella, los dos grupos de doce apóstoles continuarían con su labor, y que cada tres meses mantendrían estas reuniones de los dos grupos durante una semana en lugares que se acordarían periódicamente.
\vs p144 6:7 Si bien, el problema más serio con el que se encontraron fue el relativo al bautismo, dificultades que llegaron a agravarse aún más al negarse Jesús a pronunciarse sobre aquel tema. Por último, convinieron que mientras que Juan viviese o hasta el momento en el que todos ellos tomaran de forma conjunta otra decisión, solo los apóstoles de Juan bautizarían a los creyentes, y solo los apóstoles de Jesús instruirían finalmente a los nuevos discípulos. Por consiguiente, desde ese momento y hasta la muerte de Juan, dos de los apóstoles de este acompañaron a Jesús y sus apóstoles para bautizar a los creyentes, puesto que en asamblea se había votado por unanimidad que el bautismo se convertiría en el paso inicial necesario para demostrar la adhesión pública del creyente a los asuntos del reino.
\vs p144 6:8 Se acordó después que, en el caso de la muerte de Juan, sus apóstoles se presentarían ante Jesús y se someterían a sus directrices, y no bautizarían de nuevo salvo por autorización de Jesús o de sus apóstoles.
\vs p144 6:9 Y luego se votó que, en la eventualidad de la muerte de Juan, los apóstoles de Jesús comenzarían a bautizar con agua como símbolo del bautismo del Espíritu divino. Se dejó a criterio propio si el \bibemph{arrepentimiento} debía o no vincularse a la predicación del bautismo; ninguna decisión al respecto tenía carácter vinculante para el grupo en su conjunto. Los apóstoles de Juan predicaban: “Arrepentíos y sed bautizados”. Los apóstoles de Jesús proclamaban: “Creed y sed bautizados”.
\vs p144 6:10 Y este es el relato del primer intento de los seguidores de Jesús por coordinar labores dispares, consensuar diferencias de opinión, organizar tareas a nivel de grupo, regular ceremoniales públicos y socializar prácticas religiosas personales.
\vs p144 6:11 Se examinaron numerosas otras cuestiones de menor importancia que se solucionaron por unanimidad. Durante estas dos semanas, estos veinticuatro hombres tuvieron una experiencia verdaderamente extraordinaria, al verse obligados a afrontar problemas y a resolver calmadamente sus dificultades sin Jesús. Aprendieron a discrepar, a debatir, a hacer valer sus argumentos, a orar y a zanjar sus diferencias y, en todo esto, a ser comprensivos con el punto de vista de la otra persona y ejercer al menos algún grado de tolerancia por las opiniones sinceras de los demás.
\vs p144 6:12 En la tarde en la que se debatían por último los asuntos económicos, Jesús regresó, oyó sus deliberaciones, escuchó sus decisiones y dijo: “Estas son, pues, vuestras conclusiones, y asistiré a cada uno de vosotros a llevar a cabo el objetivo que os habéis fijado en vuestras mutuas decisiones”.
\vs p144 6:13 Juan fue ejecutado dos meses y medio más tarde y, en todo este período, los apóstoles de Juan se quedaron con Jesús y los doce. Todos trabajaron juntos y bautizaron a los creyentes durante este tiempo de actividad en las ciudades de la Decápolis. El campamento de Gilboa se desmanteló el 2 de noviembre del año 27 d. C.
\usection{7. EN LAS CIUDADES DE LA DECÁPOLIS}
\vs p144 7:1 Durante los meses de noviembre y diciembre, Jesús y los veinticuatro trabajaron discretamente en las ciudades griegas de la Decápolis, principalmente en Escitópolis, Gerasa, Abila y Gadara. En realidad, se trataba del final del periodo emprendido para asumir la labor y la organización religiosa de Juan. Siempre, la religión socializada de una nueva revelación ha de pagar las consecuencias de hacer concesiones a las formas y usos establecidos de la religión que le precedió y a la que trata de rescatar. El bautismo fue el precio que los seguidores de Jesús tuvieron que pagar por contar entre ellos al grupo religioso de los seguidores de Juan el Bautista. Estos, al unirse a los seguidores de Jesús, renunciaron a casi todo salvo al bautismo con agua.
\vs p144 7:2 Jesús impartió poca enseñanza pública en esta misión en las ciudades de la Decápolis. Pasó un tiempo considerable enseñando a los veinticuatro y mantuvo muchas sesiones formativas de carácter extraordinario con los doce apóstoles de Juan. Con el tiempo, llegaron a entender las razones por las que Jesús no fue a visitar a Juan en la cárcel ni trató de lograr su liberación. Sin bien, nunca pudieron entender por qué no realizaba prodigios, por qué se negaba a manifestar signos externos de su autoridad divina. Antes de ir al campamento de Gilboa, ellos habían creído en Jesús mayormente por el testimonio de Juan, pero estaban pronto empezando a creer en él en virtud de su propio contacto con el Maestro y sus enseñanzas.
\vs p144 7:3 Durante estos dos meses, el grupo trabajó gran parte del tiempo en parejas, esto es, uno de los apóstoles de Jesús salía con otro de Juan. El apóstol de Juan bautizaba, el de Jesús instruía, mientras que ambos predicaban el evangelio del reino según lo entendían. Y ganaron muchas almas entre estos gentiles y judíos apóstatas.
\vs p144 7:4 Abner, el jefe de los apóstoles de Juan, se convirtió en un fervoroso creyente de Jesús y, más tarde, se le nombró líder de un grupo de setenta maestros, a los que el Maestro encomendó predicar el evangelio.
\usection{8. EN EL CAMPAMENTO CERCANO A PELLA}
\vs p144 8:1 Durante la última parte del mes de diciembre, se fueron todos próximos al Jordán, cerca de Pella, en donde volvieron a enseñar y predicar. Tanto judíos como gentiles venían a este campamento para oír el evangelio. Una tarde, mientras Jesús enseñaba a la multitud, algunos amigos cercanos a Juan le llevaron al Maestro el último mensaje que llegaría a tener de Juan.
\vs p144 8:2 Juan llevaba ya un año y medio en prisión y, la mayoría de este tiempo, Jesús había realizado su labor muy discretamente; no era extraño, pues, que a Juan le interesase saber acerca del reino. Los amigos de Juan interrumpieron las enseñanzas de Jesús para decirle: “Juan el Bautista nos envía para que te preguntemos: ¿Eres tú en verdad el Libertador, o esperaremos a otro?”
\vs p144 8:3 Jesús hizo una pausa para responder a los amigos de Juan: “Volved y decid a Juan que no se le ha olvidado. Id y haced saber a Juan lo que oís y veis, que se predica la buena nueva a los pobres”. Y cuando habló otras cosas con los mensajeros de Juan, Jesús se volvió de nuevo a la multitud y les dijo: “No penséis que Juan duda del evangelio del reino. Solo pregunta para aliviar las propias dudas de sus discípulos, que son también los míos. No es que Juan sea débil. Yo os pregunto a vosotros, que habéis oído a Juan predicar antes de que Herodes lo encarcelara: ¿Qué salisteis a ver en él? ¿Una caña sacudida por el viento? ¿Un hombre de ánimo variable y cubierto de vestiduras delicadas? Los que tienen vestiduras preciosas y viven en deleites están en las cortes de los reyes y en las mansiones de los ricos. Entonces ¿qué salisteis a ver? ¿Un profeta? Sí, os digo, y más que profeta. De Juan está escrito: “He aquí que yo envío mi mensajero delante de tu faz, el cual preparará el camino delante de ti”.
\vs p144 8:4 “De cierto, de cierto os digo, que entre los nacidos de mujeres no hay nadie más grande que Juan el Bautista; y, sin embargo, el más pequeño en el reino de los cielos es mayor que él porque ha nacido del espíritu y sabe que se ha convertido en un hijo de Dios”.
\vs p144 8:5 Muchos de los que oyeron a Jesús ese día se sometieron voluntariamente al bautismo de Juan, profesando de este modo, públicamente, su entrada en el reino. Y los apóstoles de Juan permanecieron firmemente unidos a Jesús desde aquel día en adelante. Este hecho significó la verdadera unión de los seguidores de Juan y de Jesús.
\vs p144 8:6 Una vez que los mensajeros conversaron con Abner, partieron hacia Maqueronte para contar todo esto a Juan que, con las palabras de Jesús y el mensaje de Abner, sintió un gran consuelo y vio su fe fortalecida.
\vs p144 8:7 Esa tarde, Jesús continuó con sus enseñanzas, diciendo: “Pero ¿a qué compararé esta generación? Muchos de entre vosotros no recibiréis ni el mensaje de Juan ni mis enseñanzas. Sois semejantes a los muchachos que juegan en la plaza del mercado y gritan a sus compañeros, diciendo: “Os tocamos flauta y no bailasteis; gemimos y no llorasteis”. Y así con algunos de vosotros. Vino Juan que ni comía ni bebía y dijeron que tenía un demonio. Vino el Hijo del Hombre, que come y bebe, y dicen: “Este es un hombre comilón y bebedor de vino, amigo de publicanos y pecadores”. Verdaderamente la sabiduría se ha acreditado por sus hijos.
\vs p144 8:8 “Parecería que el Padre de los cielos ha escondido algunas de estas verdades de los sabios y soberbios, y se las ha revelado a los niños. Pero el Padre hace todas las cosas bien; el Padre se revela al universo escogiendo su propia manera. Venid por tanto a mí todos los que estáis trabajados y cargados, y hallaréis descanso para vuestras almas. Llevad mi yugo divino sobre vosotros y experimentaréis la paz de Dios, que sobrepasa todo entendimiento”.
\usection{9. LA MUERTE DE JUAN EL BAUTISTA}
\vs p144 9:1 Por orden de Herodes Antipas, Juan el Bautista fue ejecutado la noche del 10 de enero del año 28 d. C. Al siguiente día, algunos de los discípulos de Juan, que habían ido a Maqueronte, se enteraron de su ejecución y yendo ante Herodes, pidieron su cuerpo, que depositaron en una tumba, enterrándolo más tarde en Sebaste, la tierra de Abner. Al otro día, el 12 de enero, se dirigieron al norte, al campamento de los apóstoles de Juan y Jesús cerca de Pella, y les dieron la noticia de la muerte de Juan. Cuando Jesús la oyó, despidió a la multitud y, llamando a los veinticuatro, les dijo: “Juan ha muerto. Herodes lo ha decapitado. Esta noche, en asamblea, arreglad vuestros asuntos debidamente. No habrá más demoras. Ha llegado la hora de proclamar el reino manifiestamente y con poder. Mañana nos vamos a Galilea”.
\vs p144 9:2 Así pues, el 13 de enero del año 28 d. C., temprano por la mañana, Jesús y los apóstoles, acompañados por unos veinticinco discípulos, se encaminaron a Cafarnaúm y se alojaron esa noche en la casa de Zebedeo.
