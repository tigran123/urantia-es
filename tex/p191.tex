\upaper{191}{Apariciones a los apóstoles y a otros líderes}
\author{Comisión de seres intermedios}
\vs p191 0:1 El domingo de la resurrección fue un día terrible en la vida de los apóstoles; diez de ellos pasaron la mayor parte del tiempo en el aposento alto con las puertas cerradas y atrancadas. Podían haber huido de Jerusalén, pero temían ser arrestados por los agentes del sanedrín si los encontraban en la calle. En Betfagé, a solas, Tomás cavilaba insistentemente sobre sus problemas. Le hubiera venido mejor quedarse con los otros apóstoles, y haberlos ayudado a encauzar sus deliberaciones por sendas más provechosas.
\vs p191 0:2 A lo largo de todo el día, Juan había defendido la idea de que Jesús había resucitado de entre los muertos. Relató que el Maestro había afirmado, en no menos de cinco ocasiones diferentes, que resucitaría y, que por lo menos en tres de ellas, se había referido al tercer día. La actitud de Juan ejercía una gran influencia sobre ellos, en particular sobre su hermano Santiago y sobre Natanael, e incluso podría haber sido mayor de no ser el miembro más joven del grupo.
\vs p191 0:3 El aislamiento de los apóstoles guardaba mucha relación con las dificultades por las que atravesaban. Juan Marcos los mantenía al corriente de lo que acontecía en torno al templo y les informaba sobre los muchos rumores que crecientemente se extendían por toda la ciudad; si bien, no se les ocurrió recabar noticias de los diferentes grupos de creyentes a quienes Jesús ya se había aparecido. Hasta ese momento, los mensajeros de David habían prestado esa clase de servicio, pero todos estaban cumpliendo con su última tarea como heraldos de la resurrección para los grupos de creyentes que vivían lejos de Jerusalén. Por primera vez en todos estos años, los apóstoles tomaron conciencia de lo dependientes que habían sido de los mensajeros de David para informarse diariamente sobre los asuntos del reino.
\vs p191 0:4 Durante todo ese día, Pedro, como era característico en él, se debatía emocionalmente entre la fe y la duda con respecto a la resurrección del Maestro. Pedro no podía olvidarse de la visión de los paños mortuorios, depositados allí, en la tumba, como si el cuerpo de Jesús se hubiera evaporado simplemente desde dentro. “No obstante”, razonaba Pedro, “si ha resucitado y puede mostrarse a las mujeres, ¿por qué no lo hace ante nosotros, sus apóstoles?”. Pedro se entristecía cuando pensaba que quizás no venía a ellos por su propia presencia entre los apóstoles, por haberlo negado aquella noche en el patio de Anás. Y, entonces, se animaba al recordar las palabras de las mujeres: “Id y decid a mis apóstoles ---y a Pedro---”. Pero, el hecho de sentirse alentado por este mensaje suponía que debía creer que las mujeres habían visto y oído realmente al Maestro resucitado. De este modo, Pedro estuvo fluctuando entre la fe y la duda a lo largo del día, hasta poco después de las ocho de la noche, cuando se decidió a salir al patio. Pedro pensó alejarse de los apóstoles y no ser un impedimento para la llegada de Jesús por haber negado al Maestro.
\vs p191 0:5 En un principio, Santiago Zebedeo abogó porque todos fueran a la tumba; estaba decididamente a favor de hacer algo para llegar al fondo del misterio. Fue Natanael quien, en respuesta a la insistencia de Santiago, evitó que se expusieran en público, recordándoles la advertencia de Jesús de que no pusieran innecesariamente su vida en peligro, en aquellos momentos. Hacia el mediodía, Santiago se había calmado como los demás y se mantuvo en vigilante espera. Se comunicó poco; se encontraba enormemente decepcionado porque Jesús no se había aparecido a ellos, y desconocía las numerosas apariciones efectuadas por el Maestro a otros grupos y personas.
\vs p191 0:6 Ese día Andrés se dedicó en buena parte a escuchar lo que se decía. Estaba muy desconcertado por la situación y tenía sus propias y grandes dudas, pero al menos disfrutaba al sentirse de alguna manera libre de la responsabilidad de dirigir a los demás apóstoles. Estaba, de hecho, agradecido de que el Maestro lo hubiera liberado del peso de aquel liderazgo antes de que estos tiempos preocupantes se abatieran sobre ellos.
\vs p191 0:7 En más de una ocasión, durante las largas y agotadoras horas de este trágico día, la única influencia positiva para el grupo eran con frecuencia los comentarios de Natanael, aportando sus habituales consejos filosóficos. Él fue quien en realidad sirvió de guía a los diez apóstoles durante todo el día. Ni una sola vez se manifestó respecto a su convicción o incredulidad en cuanto a la resurrección del Maestro. Pero conforme avanzaba el día, más se inclinaba a creer que Jesús había cumplido su promesa de resucitar.
\vs p191 0:8 Simón Zelotes se sentía demasiado descorazonado como para participar en estas conversaciones. La mayoría del tiempo estaba reclinado en un diván, en un rincón de la habitación, de cara a la pared; no habló ni media docena de veces a lo largo de todo el día. Su concepto del reino se había desmoronado, y no podía vislumbrar cómo la resurrección del Maestro podía cambiar materialmente la situación. Su decepción era muy personal y demasiado lacerante como para poder recuperarse de esta en breve plazo, ni siquiera frente a un hecho tan formidable como la resurrección.
\vs p191 0:9 Aunque resulte extraño, Felipe, ordinariamente poco expresivo, habló mucho ese día, a lo largo de la tarde. Tuvo poco que decir durante la mañana, pero se pasó toda la tarde haciendo preguntas a los otros apóstoles. Con frecuencia, las inquisitivas preguntas de Felipe irritaban a Pedro, pero los demás las tomaron con afabilidad. Felipe estaba especialmente deseoso de saber si, en el caso de que Jesús hubiera realmente resucitado de su sepultura, llevaría en su cuerpo las señales físicas de la crucifixión.
\vs p191 0:10 Mateo estaba extremadamente confundido; oía los comentarios de sus compañeros, pero, durante la mayor parte del tiempo le daba vueltas en su mente al problema de la futura economía del grupo. Al margen de la supuesta resurrección de Jesús, Judas ya no estaba, David le había entregado bruscamente los fondos a él y no disponían de la autoridad de un líder. Antes de que Mateo tuviera tiempo de considerar seriamente las polémicas de los demás sobre la resurrección, ya había visto cara a cara al Maestro.
\vs p191 0:11 Los gemelos Alfeo apenas tomaron parte en estas serías deliberaciones; estaban bastante ocupados con sus acostumbrados menesteres. Uno de ellos expresó la actitud de ambos cuando, en respuesta a una pregunta de Felipe, dijo: “No entendemos esto de la resurrección, pero nuestra madre dice que ha hablado con el Maestro, y nosotros la creemos”.
\vs p191 0:12 Tomás estaba en medio de uno de sus típicos y desesperados episodios de depresión. Durmió parte del día y caminó por las colinas el resto del tiempo. Sentía el impulso de volver a unirse a sus compañeros, pero su deseo de estar solo era más fuerte.
\vs p191 0:13 El Maestro postergó su primera aparición morontial a los apóstoles por distintos motivos. En primer lugar, quería que dispusieran de tiempo, una vez que hubieran sabido que había resucitado, para que recapacitaran acerca de lo que les había dicho con respecto a su muerte y resurrección cuando aún estaba con ellos en la carne. El Maestro pretendía que Pedro se enfrentara a sus peculiares dificultades antes de manifestarse ante todos ellos. En segundo lugar, deseaba que Tomás estuviera con ellos en el momento de esta su primera aparición. Juan Marcos localizó a Tomás en la casa de Simón, en Betfagé, temprano ese domingo por la mañana e informó a los apóstoles sobre este hecho hacia las once. En cualquier momento de este día, Tomás habría vuelto si Natanael o dos cualesquiera de los apóstoles hubieran ido a buscarlo. Él quería realmente volver, pero al haberse marchado tal como lo hizo la noche anterior, era demasiado orgulloso como para regresar tan pronto por su propia iniciativa. Al día siguiente, se sintió tan deprimido que le llevó casi una semana tomar la decisión de regresar. Los apóstoles lo esperaban y él, a su vez, esperaba que sus hermanos fueran a buscarlo para pedirle que volviera. En consecuencia, Tomás permaneció alejado de sus compañeros hasta el sábado siguiente, a últimas horas de la tarde, cuando, al hacerse de noche, Pedro y Juan fueron a Betfagé y lo trajeron de vuelta con ellos. Y esta es también la razón por la que no fueron enseguida a Galilea después de que Jesús se apareciera primeramente ante ellos; no querían irse sin Tomás.
\usection{1. LA APARICIÓN A PEDRO}
\vs p191 1:1 Eran casi las ocho y media de la noche de ese domingo, cuando Jesús se apareció a Simón Pedro en el jardín de la casa de Marcos. Se trataba de su octava manifestación morontial. Pedro había vivido bajo el pesado lastre de las dudas y la culpa desde su negación del Maestro. Durante todo el día del sábado y ese domingo, Pedro había luchado contra el miedo de que quizás, nunca más, volvería a ser un apóstol. Se estremecía ante el destino sufrido por Judas y llegó incluso a creer que él, también, había traicionado a su Maestro. Toda esa tarde pensó que tal vez su presencia con los apóstoles impedía que Jesús se les apareciera; siempre que, por supuesto, él hubiera realmente resucitado de entre los muertos. Y fue a un Pedro sumido en este estado de mente y de alma a quien Jesús se apareció cuando el abatido apóstol paseaba entre flores y arbustos.
\vs p191 1:2 Cuando Pedro pensó en la cariñosa mirada del Maestro al pasar junto a él en el patio de Anás, y le pasaba por su mente el maravilloso mensaje que le habían traído temprano esa mañana las mujeres que vinieron de la tumba vacía, “id y contadles a mis apóstoles ---y a Pedro---”, al considerar estos signos de misericordia, su fe empezó a sobreponerse a sus dudas, y se quedó quieto, apretando los puños, mientras decía en voz alta: “Yo creo que ha resucitado de entre los muertos; iré y se lo diré a mis hermanos”. Y, al decir esto repentinamente apareció ante él la forma de un hombre, que, con un timbre de voz que le resultó familiar, le habló, diciéndole: “Pedro, el enemigo quería hacerse contigo, pero no les iba a dejar que te llevaran de mí. Yo sabía que no renegaste de mí desde el corazón, por lo que te perdoné incluso antes de que me lo pidieras; pero ahora debes dejar de pensar en ti mismo y en los problemas del presente, mientras te preparas para llevar la nueva buena del evangelio a los que se hallan en tinieblas. Nunca más debes preocuparte por lo que puedas recibir del reino, sino aplícate más bien a lo que tú puedes dar a los que viven en extrema pobreza espiritual. Apréstate, Simón, para la batalla de un nuevo día, para la lucha contra la oscuridad espiritual y contra las malévolas dudas de las mentes materiales de los hombres”.
\vs p191 1:3 Pedro y el Jesús morontial caminaron por el jardín y hablaron de cosas pasadas, presentes y futuras durante casi cinco minutos. Luego, el Maestro desapareció de su vista, diciendo: “Adiós, Pedro, hasta que te vea con tus hermanos”.
\vs p191 1:4 Durante un momento, Pedro se sintió exultante ante la repentina constatación de que había hablado con el Maestro resucitado, y de que podía tener la certeza de que aún era un embajador del reino. Acababa de oír al Maestro glorificado que lo alentaba a continuar predicando el evangelio. Y con todo esto brotando de su corazón, se apresuró al aposento alto y, en presencia de sus compañeros apóstoles, exclamó con la respiración entrecortada por la emoción: “He visto al Maestro en el jardín. He hablado con él, y me ha perdonado”.
\vs p191 1:5 Las palabras de Pedro de que había visto a Jesús en el jardín impresionaron profundamente a sus compañeros apóstoles y, cuando estaban a punto de deponer sus dudas, Andrés se levantó y los previno de que no se dejaran influir demasiado por las palabras de su hermano. Andrés dio a entender que Pedro, ya antes, había visto cosas que resultaron no ser reales. Aunque Andrés no hizo alusión directa a la visión que Pedro había tenido de noche en el mar de Galilea, cuando aseguró que había visto al Maestro caminando hacia ellos sobre las aguas, sí dijo lo suficiente como para hacer ver a todos los allí presentes que tenía este suceso en mente. Simón Pedro se sintió muy dolido por las insinuaciones de su hermano y, enseguida, cabizbajo, guardó silencio. Los gemelos se sintieron muy apenados por Pedro, y ambos se acercaron a expresarle su solidaridad y a decirle que ellos lo creían, reiterando a la vez que su propia madre también había visto al Maestro.
\usection{2. LA PRIMERA APARICIÓN A LOS APÓSTOLES}
\vs p191 2:1 Poco después de las nueve de esa noche, tras la partida de Cleofas y Santiago, mientras los gemelos Alfeo consolaban a Pedro y mientras Natanael reprendía a Andrés, y estando los diez apóstoles congregados en el aposento alto con las puertas cerradas y los cerrojos echados por miedo a que los arrestaran, el Maestro, en forma morontial, apareció de repente en medio de ellos, diciendo: “La paz esté con vosotros. ¿Por qué os asustáis tanto cuando aparezco, como si vierais a un espíritu? ¿Es que no os hablé sobre estas cosas cuando aún estaba con vosotros en la carne? ¿Es que no os dije que los principales sacerdotes y los gobernantes de los judíos me entregarían para darme muerte; que uno de vosotros mismos me traicionaría y que resucitaría al tercer día? ¿A qué vienen todas vuestras dudas y todas esas polémicas sobre las noticias que os han traído las mujeres, Cleofas y Santiago, e incluso Pedro? ¿Cuánto tiempo continuaréis dudando de mis palabras y negándoos a creer en mis promesas? Y, ahora, que realmente me veis, ¿creeréis? Aún así, uno de vosotros no está. Cuando os hayáis reunido una vez más, y después de que todos vosotros sepáis con certeza que el Hijo del Hombre ha resucitado de su sepultura, iros de aquí a Galilea. Tened fe en Dios; tened fe unos en otros; y entraréis así en el nuevo servicio del reino de los cielos. Yo permaneceré en Jerusalén con vosotros hasta que os dispongáis a ir a Galilea. Mi paz os dejo”.
\vs p191 2:2 Cuando el Jesús morontial les habló estas cosas, desapareció en un instante de la vista de ellos. Todos ellos postraron sus rostros, alabando a Dios y venerando a su Maestro, que acababa de desvanecerse. Se trataba de la novena aparición morontial del Maestro.
\usection{3. CON LAS CRIATURAS MORONTIALES}
\vs p191 3:1 Jesús pasó todo el siguiente día, lunes, con las criaturas morontiales, presentes en aquel momento en Urantia. Como partícipes del tránsito morontial experimentado por el Maestro, habían venido a Urantia más de un millón de directores y colaboradores morontiales, junto con mortales, en estado de transición, de diversos órdenes procedentes de los siete mundos de estancia de Satania. El Jesús morontial estuvo con estas espléndidas inteligencias durante cuarenta días. Los instruyó y aprendió de sus directores sobre la vida de transición morontial tal como la experimentan los mortales de los mundos habitados de Satania a su paso por las esferas morontiales del sistema.
\vs p191 3:2 En torno a la medianoche de este lunes, la forma morontial del Maestro se adaptó para que pudiera hacer su tránsito a la segunda etapa de su progreso morontial. Cuando volvió a aparecer a sus hijos mortales en la tierra, era un ser morontial de la segunda etapa. A medida que el Maestro avanzaba en su andadura morontial, les resultaba, técnicamente, cada vez más difícil a las inteligencias morontiales y a sus acompañantes, los transformadores de la energía, conseguir que el Maestro se hiciera visible a ojos mortales y materiales.
\vs p191 3:3 Jesús hizo su tránsito a la tercera etapa morontial el viernes 14 de abril; a la cuarta etapa, el lunes 17; a la quinta etapa el sábado 22; a la sexta etapa el jueves 27; a la séptima etapa el martes 2 de mayo; a la ciudadanía de Jerusem, el domingo 7; y recibió el acogimiento de los altísimos de Edentia, el domingo 14.
\vs p191 3:4 De este modo, Miguel de Nebadón acabó su experiencia y servicio en el universo, puesto que, en el marco de sus previos ministerios de gracia, ya había experimentado, en su totalidad, la vida de los mortales ascendentes del tiempo y del espacio, desde su estancia en la sede de la constelación hasta incluso ---y a través de--- su servicio en la sede central del suprauniverso. Y, mediante esta misma serie de experiencias morontiales por las que pasó, el hijo creador de Nebadón finalizó y concluyó realmente y con éxito su séptimo y último ministerio de gracia en el universo.
\usection{4. LA DÉCIMA APARICIÓN (EN FILADELFIA)}
\vs p191 4:1 La décima aparición morontial de Jesús ante la visión humana ocurrió el martes 11 de abril, poco antes de las ocho, en Filadelfia. En este lugar se mostró a Abner y a Lázaro y a unos ciento cincuenta de sus compañeros, incluidos más de cincuenta del colectivo de evangelistas de los setenta. Esta aparición tuvo lugar justo tras la apertura de una reunión especial en la sinagoga, convocada por Abner para tratar sobre la crucifixión de Jesús y el informe más reciente de su resurrección, llevado hasta ellos por un mensajero de David. Debido a que el Lázaro resucitado era en ese momento miembro de este grupo de creyentes, no les resultó difícil creer en la noticia de que Jesús había resucitado de entre los muertos.
\vs p191 4:2 Se estaba inaugurando la reunión de la mano de Abner y Lázaro, que se encontraban ambos juntos de pie en el pulpito de la sinagoga, cuando toda la congregación de creyentes vio aparecer súbitamente la forma del Maestro. Se adelantó unos pasos desde el lugar en el que había aparecido, entre Abner y Lázaro, ninguno de los cuales lo había visto, y saludando a los asistentes, dijo:
\vs p191 4:3 \pc “La paz esté con vosotros. Todos sabéis que tenemos un Padre en el cielo y que hay un único evangelio del reino ---la buena nueva de la dádiva de la vida eterna que los hombres reciben por la fe---. Al regocijaros en vuestra lealtad al evangelio, orad al Padre de la verdad para que se derrame en vuestros corazones un amor nuevo y más grande por vuestros hermanos. Debéis amar a todos los hombres tal como yo os he amado; debéis servir a todos los hombres tal como yo os he servido. Con comprensión, empatía y afecto fraternal, acoged a todos vuestros hermanos que están dedicados a la proclamación de la buena nueva, ya sean judíos o gentiles, griegos o romanos, persas o etíopes. Juan proclamó el reino con antelación; vosotros predicáis el evangelio en poder; los griegos ya enseñan la buena nueva; y yo pronto enviaré el espíritu de la verdad a las almas de todos estos, mis hermanos, que tan desinteresadamente dedican su vida a iluminar a aquellos semejantes suyos que habitan en las tinieblas espirituales. Todos vosotros sois los hijos de la luz; por eso, que no os sean de tropiezo los enredosos malentendidos que causan la desconfianza y la intolerancia humana. Si, por la gracia de la fe, se os ha ennoblecido para amar a los no creyentes, ¿no debéis amar igualmente a vuestros compañeros creyentes de la creciente familia de la fe? Recordad que en la medida en la que os amáis unos a otros, sabrán todos los hombres que sois mis discípulos.
\vs p191 4:4 “Id, pues, al mundo entero a proclamar este evangelio de la paternidad de Dios y de la hermandad de los hombres a todas las naciones y razas, y sed por siempre prudentes cuando elijáis la manera en la que presentaréis la buena nueva a las diferentes razas y tribus de la humanidad. De gracia habéis recibido este evangelio del reino, y de gracia daréis la buena nueva a todas las naciones. No temáis resistir al mal, porque yo estoy siempre con vosotros, incluso hasta el fin de los tiempos. Y mi paz os dejo”.
\vs p191 4:5 \pc En cuanto dijo, “Mi paz os dejo”, desapareció de la vista de todos. Salvo en el caso de una de sus apariciones en Galilea, donde lo vieron más de quinientos creyentes al mismo tiempo, en este grupo de Filadelfia había el mayor número de mortales que lo hubieran visto en una misma ocasión.
\vs p191 4:6 Temprano por la mañana siguiente, mientras que los apóstoles permanecían en Jerusalén en espera de que Tomas se repusiera emocionalmente, estos creyentes de Filadelfia salieron a proclamar que Jesús de Nazaret había resucitado de entre los muertos.
\vs p191 4:7 Jesús pasó el día siguiente, miércoles, sin interrupciones, en fraternidad con sus compañeros morontiales y, a media tarde, recibió a los delegados morontiales, visitantes de los mundos de estancia de todos los sistemas locales de esferas habitadas de toda la constelación de Norlatiadek. Y todos se regocijaron al saber que su creador era ahora uno de los de su propio orden de inteligencias del universo.
\usection{5. LA SEGUNDA APARICIÓN ANTE LOS APÓSTOLES}
\vs p191 5:1 Tomás pasó una semana solo, apartado de todo, en las colinas situadas en torno al Monte de los Olivos. Durante ese tiempo, solamente vio a quienes se estaban quedando en la casa de Simón y a Juan Marcos. Sobre las nueve del sábado, 15 de abril, los dos apóstoles lo encontraron y se lo llevaron de vuelta a su lugar de reunión en la casa de Marcos. Al día siguiente, Tomás oyó el relato acerca de las distintas apariciones del Maestro, pero se negó tajantemente a creer. Opinaba que Pedro, con su entusiasmo, les había hecho creer que habían visto al Maestro. Natanael trató de razonar con él, pero fue inútil. Su tozudez emocional junto a su acostumbrada tendencia a dudar y, tal estado mental, sumado a su sentimiento de pesar por haberlos abandonado, contribuyó a crearle una sensación de aislamiento que ni siquiera el mismo Tomás era capaz de comprender del todo. Se había retirado de sus compañeros, había tomado su propio camino y, ahora, a pesar de haber regresado a ellos, tendía, de forma inconsciente, a adoptar una actitud de discrepancia. Tardaba en rendirse; detestaba claudicar. Sin pretenderlo, disfrutaba realmente de la atención de la que era objeto; involuntariamente, le complacía el esfuerzo que realizaban todos sus hermanos por convencerlo y hacerlo cambiar. Los había echado de menos durante toda una semana, y sus insistentes atenciones le producían una gran satisfacción.
\vs p191 5:2 Estaban tomando la cena, poco después de las seis, con Pedro sentado a un lado de Tomás y Natanael al otro, cuando el incrédulo apóstol dijo: “No voy a creer a menos que vea al Maestro con mis propios ojos y pueda poner el dedo en la marca de los clavos”. Así pues, cuando estaban sentados cenando, con las puertas firmemente cerradas y atrancadas, el Maestro morontial se apareció de repente dentro de la curvatura interior de la mesa, que tenía forma de U, y, quedándose directamente delante de Tomás, dijo:
\vs p191 5:3 “La paz esté con vosotros. He aguardado durante toda una semana completa para poder aparecer cuando todos vosotros estuvierais presentes y pudierais oír de nuevo mí comisión de que salgáis a todo el mundo a predicar este evangelio del reino. Una vez más os digo: Como me envió el Padre al mundo, así también yo os envío. Como yo he revelado al Padre, así revelaréis vosotros el amor divino, no meramente con palabras, sino en vuestra vida cotidiana. Os envío, no para que améis las almas de los hombres, sino para que \bibemph{améis a los hombres}. No debéis proclamar simplemente los gozos del cielo, sino también mostrar, en vuestra experiencia diaria, esas realidades espirituales de la vida divina, pues vosotros ya tenéis vida eterna, como dádiva de Dios, por medio de la fe. Cuando tengáis fe, cuando el poder de lo alto, el espíritu de la verdad, haya venido sobre vosotros, no ocultaréis vuestra luz, aquí, tras puertas cerradas. Haréis conocer el amor y la misericordia de Dios a toda la humanidad. Por miedo, huís ahora de los hechos de una ingrata experiencia, pero cuando hayáis sido bautizados con el espíritu de la verdad, saldréis con valentía y gozo a encontrar las nuevas experiencias de proclamar la buena nueva de la vida eterna en el reino de Dios. Podréis permanecer aquí y en Galilea por un breve espacio de tiempo, mientras os reponéis de la conmoción del paso desde la falsa seguridad que ofrece la autoridad del tradicionalismo al nuevo orden consistente en la autoridad de los hechos, de la verdad y de la fe en las realidades supremas de la experiencia viva. Vuestra misión en el mundo se funda en el hecho de que he vivido entre vosotros una vida dedicada a la revelación de Dios, en la verdad de que vosotros y todos los demás hombres sois hijos de Dios, y consistirá en la vida que vosotros viviréis entre los hombres ---la experiencia real y viva de amar a los hombres y servirlos, como yo os he amado y servido---. Que la fe revele al mundo vuestra luz; que la revelación de la verdad abra los ojos enceguecidos por la tradición; que vuestro amoroso servicio logre destruir los prejuicios engendrados por la ignorancia. Si os acercáis a vuestros semejantes con comprensión y empatía y entrega desinteresada, los guiaréis al conocimiento salvífico del amor del Padre. Los judíos ensalzan la bondad; los griegos exaltan la belleza; los hindúes predican la adoración; los lejanos ascetas enseñan veneración; los romanos exigen lealtad; pero yo demando de mis discípulos vida, es más, una vida dedicada al servicio amoroso de vuestros hermanos en la carne”.
\vs p191 5:4 Cuando el Maestro dijo estas cosas, miró el rostro de Tomás y dijo: “Y tú, Tomás, que dijiste que no creerías a no ser que me vieras y pusieras el dedo en la marca de los clavos de mis manos, ahora me has contemplado y oído mis palabras; y aunque no veas ninguna marca de clavos en mis manos, ya que he resucitado con una forma que tú también tendrás cuando dejes este mundo, ¿qué les dirás a tus hermanos? Reconocerás la verdad, porque ya en tu corazón habías comenzado a creer, incluso cuando afirmabas tan rotundamente tu escepticismo. Tus dudas, Tomás, siempre se vuelven más obcecadas en tu mente justo antes de que finalmente sucumben ante la verdad. Tomás, te ruego que no seas incrédulo sino creyente; y yo sé que creerás con todo tu corazón”.
\vs p191 5:5 Cuando Tomás oyó estas palabras, cayó de rodillas ante el Maestro morontial y exclamó: “¡Creo! ¡Señor mío y Maestro mío!”. Luego, Jesús le dijo a Tomás: “Tomás, porque me has visto y oído creíste. Bienaventurados los que en las eras aún por venir creerán, aunque no me hayan visto con los ojos de la carne ni me hayan oído con oídos humanos”.
\vs p191 5:6 Y, entonces, conforme la forma del Maestro se acercó a la cabecera de la mesa, se dirigió a todos ellos diciendo: “Y, ahora, id todos a Galilea, donde en breve me apareceré a vosotros”. Una vez dicho esto, desapareció de la vista de todos.
\vs p191 5:7 \pc En ese momento, los once apóstoles estaban completamente convencidos de que Jesús había resucitado de entre los muertos y, muy temprano, por la mañana siguiente, antes del romper el alba, partieron para Galilea.
\usection{6. APARICIÓN EN ALEJANDRÍA}
\vs p191 6:1 Mientras los once apóstoles iban de camino a Galilea, finalizando ya su viaje, el martes 18 de abril, sobre las ocho y media de la noche, Jesús se apareció a Rodán y a otros ochenta creyentes más de Alejandría. Se trataba de la duodécima aparición del Maestro en forma morontial. Jesús se apareció a estos griegos y judíos cuando uno de los mensajeros de David acababa de informar acerca de su crucifixión. Este mensajero, el quinto corredor de relevo entre Jerusalén y Alejandría, había llegado a Alejandría bien entrada la tarde, y cuando entregó su mensaje a Rodán, se decidió convocar a los creyentes para que recibieran la trágica noticia de labios del mensajero mismo. Hacia las ocho de la noche, el mensajero, Natán de Busiris, acudió ante este grupo y les expuso detalladamente todo lo que el anterior corredor le había contado a él. Natán terminó su conmovedor relato con estas palabras: “Pero David, que nos envía estas palabras, informa de que el Maestro, al predecir su muerte, afirmó que resucitaría”. Estaba Natán aún hablando, cuando el Maestro morontial apareció ante la mirada de todos. Y al sentarse Natán, Jesús dijo:
\vs p191 6:2 “La paz esté con vosotros. Aquello a lo que mi Padre me envió al mundo a instaurar no pertenece a una raza ni a una nación ni a un grupo especial de maestros o predicadores. Este evangelio del reino pertenece por igual a judíos y gentiles, a ricos y pobres, a libres y esclavos, a hombres y mujeres e incluso a los niños pequeños. Y todos debéis proclamar este evangelio de amor y verdad por medio de las vidas que viváis en la carne. Os amaréis unos a otros con un afecto nuevo y extraordinario, tal como yo os he amado a vosotros. Serviréis a la humanidad con una devoción nueva y magnífica, tal como yo os he servido a vosotros, y cuando los hombres vean la medida en la que los amáis, y cuando observen el fervor con el que os servís, percibirán que os habéis convertido en partícipes, por la fe, del reino de los cielos, y seguirán al espíritu de la verdad que verán en vuestras vidas, hasta encontrar la salvación eterna.
\vs p191 6:3 “Como el Padre me envió a este mundo, así os envío yo ahora a vosotros. Todos estáis llamados a llevar la buena nueva a los que habitan en tinieblas. Este evangelio del reino pertenece a todos los que crean en él; no se le pondrá bajo la custodia de meros sacerdotes. Pronto, el espíritu de la verdad vendrá sobre vosotros, y él os guiará a toda la verdad. Id por tanto a todo el mundo a predicar este evangelio, y mirad que yo estoy con vosotros siempre, hasta incluso el fin de los tiempos”.
\vs p191 6:4 Una vez que el Maestro dijo estas cosas, desapareció de la vista de todos. Estos creyentes permanecieron toda la noche allí, juntos, recordando sus experiencias como creyentes del reino y escuchando las muchas palabras de Rodán y de sus acompañantes. Todos creyeron que Jesús había resucitado de entre los muertos. Imaginad la sorpresa del heraldo de la resurrección enviado por David, que llegó dos días más tarde, cuando respondieron a su anuncio diciendo: “Sí, lo sabemos, porque lo hemos visto. Se nos apareció anteayer”.
