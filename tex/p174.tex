\upaper{174}{Martes por la mañana en el templo}
\author{Comisión de seres intermedios}
\vs p174 0:1 Sobre las siete de la mañana de ese martes, Jesús se reunió en la casa de Simón con los apóstoles, el colectivo de mujeres y unas dos docenas de otros prominentes discípulos. En esta reunión, se despidió de Lázaro, dándole instrucciones que tan pronto lo llevarían a huir a Filadelfia, en Perea, donde más tarde se unió al movimiento misionero con sede en dicha ciudad. Jesús también dijo adiós al anciano Simón, e impartió unos consejos de despedida al colectivo de mujeres, a las que ya nunca se dirigiría de nuevo de manera formal.
\vs p174 0:2 Aquella mañana saludó personalmente a cada uno de los doce. A Andrés le dijo: “No desmayes por lo que está al llegar. Mantén a tus hermanos fuertemente unidos y cuida de que no te vean abatido”. A Pedro le dijo: “No deposites tu confianza en el brazo de la carne ni en las armas de acero. Asiéntate sobre los cimientos espirituales de las rocas eternas”. A Santiago le dijo: “No flaquees ante las apariencias externas. Permanece firme en tu fe, y pronto conocerás la realidad de aquello en lo que crees”. A Juan le dijo: “Sé amable; ama incluso a tus enemigos; sé tolerante. Y recuerda que te he encomendado muchas cosas”. A Natanael le dijo: “No juzgues por las apariencias; permanece firme en tu fe incluso cuando todo parezca tambalearse; sé fiel a tu misión como embajador del reino”. A Felipe le dijo: “Sé inconmovible ante lo que se avecina. Permanece inquebrantable, incluso si no puedes ver el camino. Sé leal a tu promesa de consagración”. A Mateo le dijo: “No olvides la misericordia que te recibió en el reino. Que nadie te prive de tu premio eterno. Así como has soportado las propensiones de la naturaleza mortal, está dispuesto a ser constante”. A Tomás le dijo: “Por muy difícil que sea, ahora debes andar por fe y no por vista. No dudes de que podré acabar la obra que he comenzado, y de que con el tiempo veré a todos mis fieles embajadores en el reino de más allá”. A los gemelos Alfeo les dijo: “No permitáis que las cosas que no podáis entender os sobrecojan. Sed fieles a los afectos de vuestros corazones y no depositéis vuestra confianza ni en grandes hombres ni en la actitud mudable de la gente. Apoyad a vuestros hermanos”. A Simón Zelotes le dijo: “Simón, puede que la decepción te abrume, pero tu espíritu ha de elevarse sobre todo lo que te pueda sobrevenir. Lo que no pudiste aprender de mí, mi espíritu te lo enseñará. Busca las verdaderas realidades del espíritu y no te sientas atraído por las sombras irreales y materiales”. Y, a Judas Iscariote, le dijo: “Judas, te amo y ruego para que ames a tus hermanos. No te canses de hacer el bien; y quiero advertirte de que te cuides de los caminos resbaladizos de la adulación y de los dardos envenenados del ridículo”.
\vs p174 0:3 Y cuando acabó de decir estas palabras de saludo, partió para Jerusalén con Andrés, Pedro, Santiago y Juan, mientras los demás apóstoles se dedicaban a montar el campamento de Getsemaní, adonde irían aquella noche, y donde establecerían su sede durante el resto de la vida del Maestro en la carne. Hacia mediados del camino de descenso del Monte de los Olivos, Jesús hizo una pausa y charló con los cuatro apóstoles más de una hora.
\usection{1. EL PERDÓN DIVINO}
\vs p174 1:1 Durante varios días, Pedro y Santiago habían estado debatiendo sus diferencias de opinión respecto a las enseñanzas del Maestro sobre el perdón de los pecados. Ambos convinieron en plantearle la cuestión a Jesús, y Pedro se sirvió oportunamente de aquella ocasión para procurar la guía del Maestro. Por ello, Simón Pedro interrumpió la conversación, que abordaba las diferencias entre la alabanza y la adoración, y preguntó: “Maestro, Santiago y yo no nos ponemos de acuerdo en cuanto a tu doctrina sobre el perdón de los pecados. Santiago afirma que tú enseñas que el Padre nos perdona incluso antes de que se lo pidamos, y yo mantengo que el arrepentimiento y la confesión deben anteceder al perdón. ¿Quién de nosotros tiene razón? ¿Qué dices tú?”.
\vs p174 1:2 Tras un breve momento de silencio, Jesús miró significativamente a los cuatro y respondió: “Hermanos míos, erráis en vuestras opiniones porque no comprendéis la naturaleza de esa relación entrañable y amorosa que existe entre la criatura y el Creador, entre el hombre y Dios. No lográis captar ese entendimiento y empatía que un padre sensato siente hacia su hijo inmaduro y, en ocasiones, equivocado. De hecho, es dudoso que unos padres inteligentes y cariñosos puedan verse jamás llamados a perdonar a un hijo común y corriente. Una relación de entendimiento unida a una actitud amorosa evita en efecto todos esos distanciamientos que, más tarde, precisarían de la reconciliación del arrepentimiento del hijo con el perdón del padre.
\vs p174 1:3 “Una parte de cualquier padre vive en el hijo. El padre es preeminente y superior en entendimiento en todas las cuestiones que tienen que ver con la relación entre el hijo y el padre. El padre es capaz de ver la inmadurez del hijo desde la perspectiva de su más avanzada madurez paterna, desde la experiencia más adulta de un allegado de mayor edad. En el caso del hijo terrenal y del Padre celestial, el padre divino es infinita y divinamente compasivo con sus hijos y posee la facultad de comprenderlos amorosamente. El perdón divino es inevitable; es inherente e inalienable a la infinita comprensión de Dios, a su perfecto conocimiento de todo lo que concierne al juicio equivocado y a la decisión errada del hijo. La justicia divina es tan eternamente ecuánime que invariablemente conlleva entendimiento y misericordia.
\vs p174 1:4 “Cuando un hombre sensato comprende los motivos que impulsan a sus semejantes, los amará. Y cuando amáis a vuestro hermano, ya lo habéis perdonado. Esta facultad de comprender la naturaleza humana y de perdonar sus aparentes malos actos es semejarse a Dios. Y, si sois unos padres sensatos, podréis, pues, comprender a vuestros hijos y perdonarlos, aun cuando algún malentendido pasajero os haya podido supuestamente separar. El hijo, siendo inmaduro y carente de un mayor entendimiento de la profunda relación entre hijo y padre, no puede evitar sentirse a menudo culpable por el alejamiento del beneplácito pleno del padre, pero el verdadero padre nunca es consciente de esta separación. La experiencia del pecado ocurre en la conciencia de la criatura; no es parte de la conciencia de Dios.
\vs p174 1:5 “Vuestra incapacidad o reticencia para perdonar a vuestros semejantes mide vuestra inmadurez, vuestro fracaso en el logro, en su nivel adulto, de la compasión, el entendimiento y el amor. Guardáis rencores y acariciáis la venganza en proporción directa a vuestra ignorancia de la naturaleza interior y de los verdaderos anhelos de vuestros hijos y de vuestros semejantes. El amor es consecuencia del impulso divino e interno de la vida. Se funda en el entendimiento, se nutre del servicio desinteresado y se perfecciona mediante la sabiduría”.
\usection{2. PREGUNTAS DE LOS DIRIGENTES JUDÍOS}
\vs p174 2:1 A última hora de la tarde del lunes, se había celebrado un consejo entre el sanedrín y unos otros cincuenta líderes, elegidos entre los escribas, los fariseos y los saduceos. En esta reunión, se alcanzó el consenso de que sería peligroso arrestar a Jesús en público, porque seguía firme el cariño de la gente común hacia él. La mayoría de ellos también opinaba que se debería tomar la firme decisión de desacreditarlo en presencia de la multitud antes de arrestarlo y llevarlo a juicio. Así pues, se designaron a varios grupos de eruditos para que, al día siguiente por la mañana, hicieran acto de presencia en el templo y tenderle una trampa con preguntas difíciles, además de avergonzarlo de otras maneras delante de la gente. Por fin, los fariseos y los saduceos aunaban sus fuerzas para desautorizar a Jesús ante los ojos de las multitudes venidas para la Pascua.
\vs p174 2:2 El martes por la mañana, cuando Jesús llegó al patio del templo e inició sus enseñanzas, apenas había dicho unas pocas palabras cuando un grupo de los estudiantes más jóvenes de las academias, a quienes se les había preparado de antemano para este propósito, se adelantó hacia Jesús y, mediante su portavoz, le dijeron: “Maestro, sabemos que impartes tu doctrina con rectitud y que proclamas los caminos de la verdad, y que únicamente sirves a Dios, porque no temes a ningún hombre, y no haces acepción de personas. Solo somos estudiantes, y nos gustaría conocer la verdad sobre una cuestión que nos preocupa; esta es nuestra dificultad: ¿Es lícito dar tributo al césar? ¿Daremos, o no daremos tributo?”. Jesús, percatándose de su hipocresía y malicia, les dijo: “¿Por qué venís para tentarme? Mostradme la moneda del tributo, y yo os contestaré”. Y cuando ellos le presentaron un denario, él lo miró y dijo: “¿De quién es la imagen y la inscripción de esta moneda?”. Cuando le contestaron, “Del césar”, Jesús les dijo: “Dad al césar lo que es del césar, y a Dios lo que es de Dios”.
\vs p174 2:3 Su respuesta hizo que estos jóvenes escribas y sus cómplices herodianos se fueran de su presencia, y la gente, incluidos los saduceos, disfrutaron de su turbación. Hasta los jóvenes que habían intentado tenderle esta trampa quedaron maravillados por la inesperada astucia de la respuesta del Maestro.
\vs p174 2:4 El día anterior, los líderes habían querido hacerlo tropezar ante la multitud en temas de autoridad eclesiástica y, habiendo fracasado, se proponían ahora implicarlo en algún comentario que dañara la autoridad civil. Tanto Pilato como Herodes estaban en Jerusalén en aquel momento y los enemigos de Jesús suponían que, si él se atrevía a desaconsejar que se pagara el tributo al césar, podrían llevarlo de momento ante las autoridades romanas y acusarlo de sedición. Por otro lado, si aconsejaba con claridad que había que pagar el tributo, sopesaban correctamente que sus palabras herirían enormemente el orgullo nacional de los judíos allí presentes, enajenando así la buena voluntad y el cariño de la multitud.
\vs p174 2:5 En todo esto, los enemigos de Jesús fueron derrotados puesto que había una norma bien conocida del sanedrín, creada para guiar a los judíos dispersos por las naciones gentiles, la cual decía que el “derecho de acuñar monedas comportaba el derecho a cobrar impuestos”. De este modo, Jesús había evitado caer en su trampa. Si hubiera contestado “no” a su pregunta, su respuesta habría equivalido a incitar a la rebelión; y, si hubiera contestado “sí”, habría colisionado con los sentimientos nacionalistas, profundamente arraigados, de aquellos días. El Maestro no eludió la pregunta; simplemente empleó su sabiduría y dio una respuesta doble. Jesús nunca fue evasivo, sino siempre juicioso en su trato con los que querían hostigarlo y acabar con él.
\usection{3. LOS SADUCEOS Y LA RESURRECCIÓN}
\vs p174 3:1 Antes de que Jesús pudiera comenzar sus enseñanzas, otro grupo pasó al frente para preguntarle; esta vez, eran unos saduceos cultos y astutos. Su portavoz, acercándose a él, dijo: “Maestro, Moisés dijo que si un hombre casado moría sin dejar hijos, su hermano debía casarse con la mujer de este y dar descendencia a su hermano fallecido. Pues bien, sucedió que cierto hombre, que tenía seis hermanos, murió sin dejar hijos; el siguiente hermano se casó con su mujer, pero también murió pronto sin dejar descendencia. Igualmente, el segundo hermano tomó a la mujer como esposa, pero también murió sin hijos. Y, así sucesivamente, hasta que los seis hermanos se casaron con ella, y los seis murieron sin tener descendencia. Finalmente, tras ellos, falleció también la mujer. Ahora bien, nos gustaría preguntarte esto: en la resurrección, ¿de cuál de ellos será ella esposa, ya que los siete la tuvieron por mujer?”.
\vs p174 3:2 Jesús sabía, como también lo sabía la gente, que estos saduceos no eran sinceros cuando hicieron tal pregunta, porque era improbable que se diera realmente un caso así; y, además, en ese momento, entre los judíos, la costumbre de que los hermanos de un hombre fallecido trataran de engendrar hijos por él era palabra muerta prácticamente. No obstante, Jesús consintió en responder a aquella pregunta maliciosa, diciendo: “Erráis al preguntarme esto porque ignoráis las Escrituras y el poder vivo de Dios. Sabéis que los hijos de este mundo pueden casarse y darse en matrimonio, pero no parece que entendáis que aquellos que sean tenidos dignos de alcanzar los mundos venideros, mediante la resurrección de los justos, no se casan ni se dan en matrimonio. Quienes experimentan la resurrección de los muertos son más como los ángeles del cielo, y nunca mueren. Estos resucitados son eternamente hijos de Dios; son los hijos de la luz resucitados al camino de progreso de la vida eterna. Y aun vuestro padre Moisés entendió esto porque, en su experiencia vivida con la zarza ardiente, él oyó al Padre decir: ‘Yo \bibemph{soy} el Dios de Abraham, el Dios de Isaac y el Dios de Jacob’. Y, así, junto con Moisés, yo os declaro que mi Padre no es Dios de muertos sino de vivos. En él vivís, os reproducís y poseéis vuestra existencia mortal”.
\vs p174 3:3 Cuando Jesús acabó de dar respuesta a dichas preguntas, los saduceos se alejaron, y algunos de los fariseos se olvidaron tanto de sí mismos que exclamaron: “Maestro, ciertamente has contestado bien a estos saduceos incrédulos”. Los saduceos no osaron preguntarle nada más, y la gente común se maravilló de la sabiduría de su doctrina.
\vs p174 3:4 \pc Jesús solo se refirió a Moisés en su enfrentamiento con los saduceos porque esta secta político\hyp{}religiosa reconocía únicamente la validez de los denominados cinco libros de Moisés; no creían que las enseñanzas de los profetas pudieran admitirse como base de los dogmas doctrinales. El Maestro, en su respuesta, confirmaba inequívocamente el hecho de la supervivencia de las criaturas mortales por vía de la resurrección, pero, de ninguna manera, dio su aprobación a las creencias fariseas de la resurrección literal del cuerpo humano. Lo que Jesús quería destacar era, que el Padre había dicho: “Yo \bibemph{soy} el Dios de Abraham, Isaac y Jacob’, no que yo \bibemph{era} el Dios de ellos”.
\vs p174 3:5 Los saduceos habían planeado humillarlo mediante el \bibemph{ridículo,} sabiendo muy bien que perseguirlo públicamente lograría, sin duda, que la multitud se solidarizaría más con él.
\usection{4. EL GRAN MANDAMIENTO}
\vs p174 4:1 Otro grupo de saduceos tenía instrucciones de enredar a Jesús con preguntas acerca de los ángeles, pero cuando vieron la suerte corrida por los compañeros suyos al querer tenderle una trampa con preguntas sobre la resurrección, decidieron, muy prudentemente, guardar silencio; se retiraron sin más. El plan convenido por los fariseos, escribas, saduceos y herodianos, ahora coaligados, era plantearle cuestiones insidiosas a Jesús a lo largo de todo el día, esperando así desacreditar a Jesús ante la gente y, al mismo tiempo, impedir en lo posible que tuviera tiempo de impartir sus preocupantes enseñanzas.
\vs p174 4:2 Entonces se presentó un grupo de fariseos para hostigarlo con sus preguntas, y su portavoz, haciendo señas a Jesús, dijo: “Maestro, soy intérprete de la ley, y me gustaría preguntarte cuál es, en tu opinión, el mandamiento más grande”. Jesús respondió: “Existe tan solo un mandamiento, que es el más grande de todos, y ese mandamiento es: ‘Oye, Israel: el Señor nuestro Dios, el Señor uno es; y amarás al Señor tu Dios con todo tu corazón y con toda tu alma, con toda tu mente y con todas tus fuerzas’. Este es el primero y gran mandamiento. Y el segundo mandamiento es semejante; de hecho, proviene directamente de él, y es: ‘Amarás a tu prójimo como a ti mismo’. No hay otro mandamiento mayor que ellos; de estos dos mandamientos dependen toda la Ley y los Profetas”.
\vs p174 4:3 Cuando el intérprete de la ley percibió que Jesús había respondido no solo conforme al más elevado concepto de la religión judía, sino que lo había hecho asimismo sabiamente en presencia de la multitud allí congregada, decidió ser valeroso y elogiar públicamente la respuesta del Maestro. Por lo tanto, dijo: “En verdad, Maestro, bien has dicho que Dios es uno y que no hay otro fuera de él; y que amarlo de todo corazón, con todo el entendimiento y con todas las fuerzas, y amar asimismo al prójimo como a uno mismo, es el primero y gran mandamiento; y estamos de acuerdo en que este gran mandamiento debe ser mucho más que todos los holocaustos y sacrificios”. Cuando el intérprete de la ley contestó con tanta prudencia, Jesús bajó la mirada hacia él y dijo: “Amigo mío, veo que no estás lejos del reino de Dios”.
\vs p174 4:4 \pc Jesús habló la verdad cuando, en referencia a este intérprete de la ley, dijo: “no estás lejos del reino”, porque aquella misma noche marchó al campamento del Maestro, cerca de Getsemaní, profesó su fe en el evangelio del reino y fue bautizado por Josías, uno de los discípulos de Abner.
\vs p174 4:5 \pc Otros dos o tres grupos de escribas y fariseos estaban presentes y pretendían formularle más preguntas, pero se vieron desarmados por la respuesta de Jesús al intérprete de la ley, o bien los disuadió el bochorno de todos los que habían tratado de tenderle una trampa. Tras esto, ya nadie se atrevió a preguntarle nada más en público.
\vs p174 4:6 Al no haber más preguntas y, acercándose la hora del mediodía, Jesús no retomó sus enseñanzas sino que simplemente se limitó a preguntarle a los fariseos y a sus acompañantes. Jesús les dijo: “Puesto que no tenéis más preguntas que hacerme, me gustaría haceros yo a vosotros una. ¿Qué pensáis del Libertador? Esto es, ¿de quién es hijo?”. Tras una breve pausa, uno de los escribas contestó: “El Mesías es el hijo de David”. Y puesto que Jesús sabía que se había debatido mucho, incluso entre sus propios discípulos, si él era o no hijo de David, preguntó de nuevo: “Si en efecto el Libertador es hijo de David, ¿cómo puede ser que, en el salmo que atribuís a David, él mismo, hablando en el espíritu, dice: “El Señor dijo a mi Señor: siéntate a mi diestra hasta que ponga a tus enemigos por estrado de tus pies’? Si David lo llama Señor, ¿cómo, pues, es su hijo?”. Aunque los dirigentes judíos, los escribas y los sumos sacerdotes no contestaron e igualmente se abstuvieron de hacerle a él otras preguntas para confundirlo. Nunca respondieron a la pregunta que Jesús les planteó, pero, tras la muerte del Maestro, intentaron evitar aquella cuestión problemática cambiando la interpretación de este salmo para que hiciera referencia a Abraham en lugar de al Mesías. Otros trataron de evadirse de dicho dilema negando que David fuera el autor de este denominado salmo mesiánico.
\vs p174 4:7 Poco tiempo atrás, los fariseos habían disfrutado de la forma en la que el Maestro había silenciado a los saduceos; ahora los saduceos celebraban el fracaso de los fariseos; pero esta rivalidad era solo pasajera; rápidamente se olvidaron de sus diferencias tradicionales y se unieron para impedir las enseñanzas y las obras de Jesús. Pero, a lo largo de todos estos hechos, la gente común lo oía gustosamente.
\usection{5. LOS GRIEGOS INQUISITIVOS}
\vs p174 5:1 Sobre el mediodía, mientras Felipe compraba suministros para el nuevo campamento que se estaba montando ese día cerca de Getsemaní, lo abordó una delegación de extranjeros, un grupo de griegos creyentes de Alejandría, Atenas y Roma, cuyo portavoz dijo al apóstol: “Los que te conocen nos han indicado que eres uno de los apóstoles; así pues, venimos a ti, Señor, solicitando ver a Jesús, tu Maestro”. A Felipe lo tomó por sorpresa hallar allí a estos prominentes gentiles griegos que preguntaban por Jesús en la plaza, y, puesto que Jesús había encargado a los doce tan explícitamente que no impartieran ninguna enseñanza pública durante la semana de Pascua, se sintió algo confuso respecto a la mejor forma de afrontar aquella situación. También le desconcertó el hecho de que estos hombres eran gentiles extranjeros. Si hubieran sido judíos o gentiles de las proximidades que les resultaran conocidos no habría estado tan visiblemente dubitativo. Esto fue lo que hizo: pidió a los griegos que se quedaran justo donde estaban. Al marcharse de prisa, ellos supusieron que había ido a buscar a Jesús, pero en realidad salió corriendo a la casa de José, donde sabía que estaban almorzando Andrés y los demás apóstoles; y llamando a Andrés para que saliera, le explicó por qué había ido hasta allí y, entonces, acompañado por Andrés, regresó al lugar donde esperaban los griegos.
\vs p174 5:2 Dado que Felipe había prácticamente terminado de comprar, él y Andrés, con los griegos, se dirigieron de vuelta a la casa de José, donde Jesús los recibió; y se sentaron cerca de él mientras Jesús hablaba a sus apóstoles y a un cierto número de destacados discípulos que se habían congregado para almorzar. Jesús dijo:
\vs p174 5:3 \pc “Mi Padre me envió a este mundo para revelar su amorosa benevolencia a los hijos de los hombres, pero aquellos a los que primero vine se han negado a recibirme. Es verdad que muchos de vosotros habéis creído mi evangelio por vosotros mismos, pero los hijos de Abraham y sus líderes están a punto de rechazarme, y con ello, rechazarán a Aquel que me envió. He anunciado públicamente el evangelio de la salvación a este pueblo; les he hablado de la filiación que trae gozo, libertad y una vida más abundante en el espíritu. Mi Padre ha hecho muchas obras maravillosas entre estos atemorizados hijos de los hombres. Pero el profeta Isaías hizo en verdad alusión a este pueblo cuando escribió: ‘Señor, ¿quién ha creído en nuestras enseñanzas? Y ¿sobre quien se ha manifestado el Señor?’ En verdad los líderes de mi pueblo han enceguecido deliberadamente sus ojos para no ver y han endurecido su corazón por miedo a creer y ser salvos. Todos estos años he tratado de curar su incredulidad para que pudieran recibir la salvación eterna del Padre. Sé que no todos me han fallado; algunos de vosotros habéis realmente creído mi mensaje. En esta sala, hay ahora una veintena de hombres que una vez fueron miembros del sanedrín, o que ocupaban altos puestos en los consejos de la nación, aunque algunos de vosotros sois aún reacios a confesar la verdad públicamente, no sea que os expulsen de la sinagoga. Algunos estáis tentados de amar más la gloria de los hombres que la de Dios. Pero debo mostraros paciencia porque temo por la seguridad y la lealtad de incluso algunos de los que por tanto tiempo han estado cerca de mí y han vivido a mi lado.
\vs p174 5:4 “Observo que en esta sala de banquetes hay congregados judíos y gentiles aproximadamente en igual número, y os hablaré como a los primeros y a los últimos de un grupo así a los que instruiré en los asuntos del reino antes de ir a mi Padre”.
\vs p174 5:5 Estos griegos habían atendido fielmente las enseñanzas de Jesús en el templo. Al anochecer del lunes habían mantenido una reunión en la casa de Nicodemo, que duró hasta el amanecer, y treinta de ellos habían optado por entrar en el reino.
\vs p174 5:6 Estando en aquel momento de pie ante ellos, Jesús percibió que terminaba una dispensación y que comenzaba otra. Y al volver su atención a los griegos, el Maestro dijo:
\vs p174 5:7 \pc “El que cree en este evangelio, no cree simplemente en mí sino en Aquel que me envió. Cuando me miráis, veis no solo al Hijo del Hombre, sino también a Aquel que me envió. Yo soy la luz del mundo, y quienquiera que crea en mis enseñanzas ya no permanecerá más en tinieblas. Si vosotros, los gentiles, me oís, recibiréis las palabras de vida y entraréis enseguida en la gozosa libertad de la verdad sobre la filiación con Dios. Si mis compatriotas, los judíos, deciden rechazarme y desprecian mis enseñanzas, yo no los juzgaré, porque no he venido para juzgar al mundo sino para ofrecerle salvación. No obstante, quienes me rechacen y no reciban mis enseñanzas se someterán a su debido tiempo a juicio por mi Padre y por aquellos a quienes él haya asignado para juzgar a los que no acepten el don de la misericordia ni las verdades de la salvación. Recordad, todos vosotros, que yo no hablo por mi propia cuenta; el Padre que me envió, él me dio el mandamiento de que os declarara fielmente lo que había de revelar a los hijos de los hombres. Y estas palabras que el Padre me dijo que hablara al mundo son palabras de verdad divina, de perpetua misericordia y de vida eterna.
\vs p174 5:8 “Pero a judíos y a gentiles yo os anuncio que está al llegar la hora en la que el Hijo del Hombre será glorificado. Sabéis bien que si el grano de trigo no cae en tierra y muere, allí queda él solo; pero si muere en buena tierra, brota de nuevo a la vida y da mucho fruto. El que ama su vida egoístamente está en peligro de perderla; pero el que esté dispuesto a dar su vida por mí y por el evangelio disfrutará de una existencia más abundante en la tierra y en el cielo, de la vida eterna. Si alguno verdaderamente me sigue, incluso después que yo haya ido al Padre, os convertiréis en mis discípulos y en los sinceros servidores de vuestros semejantes mortales.
\vs p174 5:9 “Sé que mi hora se acerca y estoy turbado. Veo que mi pueblo está decidido a despreciar el reino, pero me complace recibir a estos gentiles, buscadores de la verdad, que hoy vienen aquí preguntando por el camino de la luz. No obstante, mi corazón sufre por mi pueblo, y mi alma está consternada por lo que está ante mí. ¿Qué puedo decir cuando miro hacia adelante y vislumbro lo que está a punto de sucederme? ¿Acaso diré: Padre, sálvame de esta terrible hora? ¡No! Para esto mismo he venido al mundo y he llegado incluso a esta hora. Más bien diré, y rogaré que os unáis a mí: Padre, glorifica tu nombre, que se haga tu voluntad”.
\vs p174 5:10 Cuando Jesús dijo estas palabras, su modelador personificado, que había habitado en él en tiempos prebautismales, apareció ante él, y conforme hizo, perceptiblemente una pausa, este espíritu, ahora poderoso, representando al Padre, habló a Jesús de Nazaret, diciéndole: “He glorificado mi nombre muchas veces en tus ministerios de gracia, y lo glorificaré otra vez”.
\vs p174 5:11 Aunque los judíos y gentiles allí reunidos no oyeron ninguna voz, sí percibieron que el Maestro había dejado de hablar, mientras le llegaba un mensaje de alguna fuente sobrehumana. Todos ellos dijeron, hablando cada cual al que tenía al lado: “Un ángel le ha hablado”.
\vs p174 5:12 Entonces Jesús continuó hablando: “Todo esto no ha ocurrido por causa mía, sino por causa de vosotros. Sé ciertamente que el Padre me recibirá y aceptará mi misión en vuestro nombre, pero es necesario que os animéis y estéis preparados para la prueba de fuego que os sobrevendrá. Os aseguro que la victoria acabará por coronar vuestro esfuerzo para iluminar unidos al mundo y liberar a la humanidad. Llega la hora del juicio para el viejo orden: he derribado al Príncipe de este mundo; y todos los hombres serán libres por la luz del espíritu que yo derramaré sobre toda carne después de haber ascendido a mi Padre de los cielos.
\vs p174 5:13 “Y ahora pues, os anuncio que cuando sea levantado de la tierra y de vuestras vidas, a todos os atraeré hacia mí mismo y a la fraternidad de mi Padre. Habéis creído que el Libertador permanecería para siempre en la tierra, pero yo os digo que los hombres rechazarán al Hijo del Hombre y que él regresará al Padre. Solo estaré con vosotros un corto tiempo; solo un poco de tiempo estará la luz viva en medio de esta entenebrecida generación. Andad entretanto que tenéis esta luz para que no os sorprendan las inminentes tinieblas y confusión. El que anda en tinieblas no sabe dónde va; pero si elegís andar en la luz, llegaréis a ser, de cierto, hijos liberados de Dios. Y, ahora todos vosotros, venid conmigo en tanto que volvemos al templo, y yo les diré palabras de despedida a los sumos sacerdotes, a los escribas, a los fariseos, a los saduceos, a los herodianos y a los ignorantes dirigentes de Israel”.
\vs p174 5:14 Habiendo dicho estas cosas, Jesús llevó al grupo por las angostas calles de Jerusalén de vuelta al templo. Acababan de oír al Maestro decir que aquel sería su discurso de despedida en este lugar, y le siguieron en silencio y profunda meditación.
