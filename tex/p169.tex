\upaper{169}{Últimas enseñanzas en Pella}
\author{Comisión de seres intermedios}
\vs p169 0:1 A última hora de la noche del lunes, 6 de marzo, Jesús y los diez apóstoles llegaron al campamento de Pella. Aquella era su última semana de estancia allí y Jesús tuvo una gran actividad enseñando a la multitud e instruyendo a los apóstoles. Por las tardes, predicaba a las multitudes y, por las noches, respondía a las preguntas de los apóstoles y de algunos de los discípulos más avanzados que residían en el campamento.
\vs p169 0:2 La noticia de la resurrección de Lázaro había llegado al campamento dos días antes de la llegada del Maestro, y todo el grupo estaba emocionado. Desde la alimentación de los cinco mil no había sucedido nada que estimulara tanto la imaginación de la gente. Y fue, pues, en este mismo punto álgido de la segunda etapa de su ministerio público del reino cuando Jesús planeó impartir en Pella, durante una sola y corta semana, sus enseñanzas y comenzar entonces su recorrido por el sur de Perea, a lo que seguidamente le seguirían los episodios finales y trágicos de su última semana en Jerusalén.
\vs p169 0:3 \pc Los fariseos y los sumos sacerdotes empezaban a formular sus cargos y a concretar sus acusaciones. Se oponían a las enseñanzas del Maestro por las siguientes causas:
\vs p169 0:4 \li{1.}Es amigo de publicanos y pecadores; recibe a los impíos e incluso come con ellos.
\vs p169 0:5 \li{2.}Es un blasfemo; habla de Dios como si fuera su Padre y se cree igual a Dios.
\vs p169 0:6 \li{3.}Es un quebrantador de la ley. Sana enfermedades en \bibemph{sabbat} y menosprecia la ley sagrada de Israel de muchas otras maneras.
\vs p169 0:7 \li{4.}Está coaligado con los diablos. Obra prodigios y aparentes milagros por el poder de Beelzebú, el príncipe de estos diablos.
\usection{1. LA PARÁBOLA DEL HIJO PERDIDO}
\vs p169 1:1 El jueves por la tarde, Jesús le habló a la multitud sobre la “gracia de la salvación”. A lo largo de este sermón, volvió a contar la historia de la oveja perdida y de la moneda perdida, y después añadió su parábola predilecta, la del hijo pródigo. Jesús dijo:
\vs p169 1:2 \pc “Los profetas, desde Samuel hasta Juan, os han exhortado a que busquéis a Dios, a que busquéis la verdad. Siempre os han dicho: ‘buscad al Señor mientras pueda ser hallado’. Y cualquier enseñanza semejante debe tomarse en serio. Pero yo he venido para mostraros que, mientras que estáis buscando a Dios, Dios está asimismo buscándoos a vosotros. Muchas veces os he contado la historia del buen pastor que dejó a las noventa y nueve ovejas en el redil para ir tras la que estaba perdida, y cómo, cuando encontró a la oveja extraviada, la puso sobre su hombro y la llevó tiernamente de vuelta al redil. Y cuando la oveja perdida fue devuelta al redil, recordaréis que el buen pastor llamó a sus amigos y los invitó a que gozaran con él porque había encontrado a la oveja que estaba extraviada. Os digo de nuevo que así habrá más gozo en el cielo por un pecador que se arrepiente que por noventa y nueve justos que no necesitan de arrepentimiento. El hecho de que las almas estén \bibemph{perdidas} solo hace acrecentar el interés del Padre celestial. Yo he venido a este mundo para hacer la voluntad de mi Padre, y se ha dicho con justicia del Hijo del Hombre que es amigo de publicanos y pecadores.
\vs p169 1:3 “Se os ha enseñado que la aceptación divina viene tras vuestro arrepentimiento y debido a todas vuestras obras de sacrificio y penitencia, pero yo os aseguro que el Padre os acepta incluso antes de que os hayáis arrepentido y envía al Hijo y a sus compañeros para encontraros y traeros con gozo de vuelta al redil, al reino de la filiación y del progreso espiritual. Todos vosotros sois como ovejas que se han extraviado, y yo he venido para buscar y salvar a quienes estén perdidos.
\vs p169 1:4 “Y debéis también recordar la historia de la mujer que, habiéndose hecho un collar con diez monedas de plata para su adorno personal, perdió una de ellas, y cómo encendió la lámpara y barrió y buscó con diligencia hasta encontrarla. Y en cuanto la encontró, reunió a sus amigas y vecinas, y les dijo: “Gozaos conmigo porque he encontrado la moneda que había perdido’. Así os digo nuevamente que hay gozo delante de los ángeles del cielo por un pecador que se arrepiente y vuelve al redil del Padre. Y os cuento esta historia para haceros ver que el Padre y su Hijo salen en \bibemph{busca} de quienes están perdidos, y en esta búsqueda empleamos cualquier recurso que pueda asistirnos en nuestros solícito empeño por encontrar a los que están perdidos, a aquellos que precisen de salvación. Y así, mientras el Hijo del Hombre sale al desierto a buscar a la oveja extraviada, también busca la moneda que se ha perdido en la casa. La oveja deambula lejos, involuntariamente; la moneda se cubre con el polvo del tiempo y queda oculta por la acumulación sobre ella de las cosas de los hombres.
\vs p169 1:5 “Ahora me gustaría contaros la historia del irreflexivo hijo de un agricultor acomodado, que \bibemph{deliberadamente} dejó la casa de su padre y se fue a tierra extranjera, donde padeció muchas tribulaciones. Recordáis que la oveja se extravió sin intención, pero este joven abandonó su casa con premeditación. Esto fue lo que ocurrió:
\vs p169 1:6 \pc “Un hombre tenía dos hijos; el menor de ellos era alegre y despreocupado, tratando siempre de pasarlo bien y de eludir sus responsabilidades, mientras que su hermano mayor era serio, sobrio, buen trabajador y dispuesto a asumir responsabilidades. Pues bien, estos dos hermanos no congeniaban; discutían y reñían continuamente. El más joven era animoso y vivaz, pero indolente y poco fiable; el mayor era constante y laborioso, pero, al mismo tiempo, egocéntrico, huraño y vanidoso. Al más joven le gustaba la diversión, pero rehuía el trabajo; el mayor estaba siempre dispuesto a trabajar y pocas veces se divertía. Esta relación se volvió tan tensa que el hijo menor fue a su padre y le dijo: padre, dame la tercera parte de los bienes tuyos que me corresponde y deja que me vaya al mundo para buscar mi propia fortuna’. Y cuando el padre oyó su ruego, sabiendo lo infeliz que era en el hogar y con su hermano mayor, repartió sus bienes, dándole al joven su parte.
\vs p169 1:7 “En pocas semanas, el joven juntó todos sus fondos y salió de viaje a un país lejano, y como no encontró nada que hacer que fuera provechoso y al mismo tiempo placentero, pronto desperdició toda su herencia viviendo perdidamente. Y cuando lo malgastó todo, vino una larga hambruna en aquel país y comenzó a pasar necesidad. Y cuando sufrió hambre y fue grande su desesperación encontró empleo con uno de los ciudadanos de ese país, el cual lo envió a su hacienda para que apacentara cerdos. Y el joven gustosamente habría llenado su vientre con las algarrobas que comían los cerdos, pero nadie le daba nada.
\vs p169 1:8 “Un día, cuando tenía mucha hambre, volviendo en sí, dijo: ‘¡Cuántos jornaleros en casa de mi padre tienen abundancia de pan mientras que yo perezco de hambre, apacentando cerdos aquí en un país extranjero! Me levantaré e iré a mi padre y le diré: padre, he pecado contra el cielo y contra ti. Ya no soy digno de ser llamado tu hijo; hazme como a uno de tus jornaleros’. Y habiendo tomado esta decisión, se levantó y se dirigió a la casa de su padre.
\vs p169 1:9 “Pues bien, el padre había sufrido mucho por su hijo; había extrañado a este joven, alegre, pero inconsciente. Este padre amaba a su hijo y siempre estaba a la expectativa de su llegada, de modo que el día en el que el hijo se iba aproximando a la casa y, cuando aún estaba lejos, lo vio y, movido por el amor y la compasión, corrió hacia él y lo saludó cariñosamente, abrazándolo y besándolo. Y, tras este encuentro, el hijo alzó la mirada al rostro lloroso de su padre y le dijo: ‘Padre, he pecado contra el cielo y ante tus ojos; ya no soy digno de ser llamado tu hijo’, pero el muchacho no pudo acabar de confesarse ante él, porque el padre, exultante de alegría, dijo a los siervos, que en ese momento llegaban corriendo: ‘Traedle rápido su mejor vestido, el que yo tengo guardado, y vestidle, y poned el anillo de hijo en su dedo y traed sandalias para sus pies’.
\vs p169 1:10 “Y luego el feliz padre llevó a casa a su hijo, agotado y con los pies doloridos. En cuanto llegó, llamó a sus siervos y les dijo: ‘Traed el becerro gordo y matadlo, y comamos y hagamos fiesta, porque este, mi hijo, muerto era y ha revivido; se había perdido y es hallado’. Y, reunidos alrededor del padre, todos se regocijaron con él por la vuelta de su hijo.
\vs p169 1:11 “En aquel momento, mientras estaban de celebración, el hijo mayor vino después de un día de trabajo en el campo y, al acercarse a la casa, oyó la música y las danzas. Y cuando llegó a la puerta trasera, llamó a uno de los criados y le preguntó a qué venía aquel festejo. Y, entonces, el criado le dijo: ‘Después de mucho tiempo perdido, tu hermano ha vuelto a casa, y tu padre, gozoso, ha hecho matar el becerro gordo por haberlo recibido bueno y salvo. Entra para que puedas tú también saludar a tu hermano y darle la bienvenida por su vuelta a la casa de tu padre’.
\vs p169 1:12 “Pero cuando el hermano mayor oyó aquello, se sintió tan dolido y se enojó tanto que no quiso entrar en la casa. Cuando el padre supo de su indignación por la bienvenida dada a su hermano menor, salió para rogarle que entrara. Pero el hijo mayor no cedió a los intentos del padre por convencerlo. Y, respondiendo, dijo a su padre: ‘Tantos años hace que te sirvo, no habiéndote desobedecido jamás en cosa alguna, por muy pequeña que fuera, y nunca me has dado ni un cabrito para gozarme con mis amigos. Me he quedado aquí y te he cuidado todos estos años y nunca te has gozado de mi fiel servicio, pero cuando vino este hijo tuyo, que ha derrochado tus bienes con rameras, te apresuras a matar para él al becerro gordo y haces fiesta para él.
\vs p169 1:13 “Puesto que este padre amaba verdaderamente a sus dos hijos, intentó razonar con el mayor: ‘Hijo mío, tú siempre has estado conmigo y todas mis cosas son tuyas. Podrías haber tenido un cabrito en cualquier momento y compartir tu alegría con amigos de haberlos tenido. Pero sería conveniente que te unieras ahora a mí y nos regocijemos y hagamos fiesta por la vuelta de tu hermano. Considera, hijo mío, que tu hermano se había perdido y ha sido hallado; ¡ha regresado vivo a nosotros!’”.
\vs p169 1:14 \pc Se trataba de una de las parábolas más conmovedoras y convincentes de todas las que Jesús contaba para hacer hincapié a los que lo oían que el Padre estaba siempre dispuesto a recibir a todo el que quisiera entrar en el reino de los cielos.
\vs p169 1:15 A Jesús le encantaba contar estas tres historias al mismo tiempo. Decía la historia de la oveja perdida para mostrar que, cuando los hombres se separan involuntariamente de la senda de la vida, el Padre es consciente de que se han \bibemph{perdido,} y sale con sus Hijos, los verdaderos pastores del rebaño, a buscar a las ovejas extraviadas; luego narraba la historia de la moneda perdida en la casa para ilustrar la cuidadosa \bibemph{búsqueda} divina de todos los que están desorientados, dudosos o, de alguna otra manera, cegados espiritualmente por las preocupaciones materiales y las cosas acumuladas en la vida; y, por último, abordaba el relato de la parábola del hijo perdido, el recibimiento del hijo pródigo que regresa, para enseñar de qué manera tan completa \bibemph{se restaura de nuevo} al hijo perdido en la casa y en el corazón de su padre.
\vs p169 1:16 Muchas, muchas veces, durante sus años de enseñanzas, Jesús contó y volvió a contar esta historia del hijo pródigo. Esta parábola y la historia del buen samaritano eran sus métodos predilectos de enseñar el amor al Padre y la dedicación de los hombres a su prójimo.
\usection{2. LA PARÁBOLA DEL MAYORDOMO ASTUTO}
\vs p169 2:1 A últimas horas de la tarde, Simón Zelotes, comentando sobre una de las declaraciones de Jesús, dijo: “Maestro, ¿qué quisiste decir hoy cuando te referiste a que muchos de los hijos del mundo son más astutos en su generación que los hijos del reino, puesto que son hábiles en trabar amistad con las riquezas injustas”. Jesús le respondió:
\vs p169 2:2 \pc “Algunos de vosotros, antes de entrar al reino, erais muy astutos en el trato con vuestros socios de negocios. Aunque fueseis injustos y con frecuencia desleales, erais, al mismo tiempo, prudentes y con visión de futuro en cuanto que realizabais vuestros negocios con la mirada puesta únicamente en vuestro beneficio presente y en vuestra seguridad futura. ¿No deberíais igualmente ordenar ahora vuestra vida en el reino de igual manera para que os proporcione gozo en el presente, mientras os aseguráis el disfrute futuro de los tesoros almacenados en el cielo? Si tan diligentes erais en lograr beneficios para vosotros mismos cuando estabais a vuestro propio servicio, ¿por qué mostrar menos diligencia en ganar almas para el reino ahora que sois sirvientes de la hermandad del hombre y mayordomos de Dios?
\vs p169 2:3 “Todos podéis tomar ejemplo de la historia de un hombre rico que tenía un mayordomo astuto pero injusto. Este mayordomo no solo presionaba a los clientes de su amo en beneficio propio, sino que también había malgastado y derrochado sus fondos sin dudarlo. Cuando esto llegó finalmente a oídos del amo, este llamó al mayordomo a su presencia y le preguntó sobre lo que se rumoreaba acerca de él, y le exigió que diera cuenta inmediatamente de su mayordomía y se preparara para entregar los asuntos de su amo a otro mayordomo.
\vs p169 2:4 “Entonces, este mayordomo infiel dijo para sí: ‘¿Qué haré ahora que estoy a punto de perder mi mayordomía. Me faltan fuerzas para cavar; mendigar, me da vergüenza. Ya sé lo que haré para asegurarme de que, cuando me quiten esta mayordomía, me reciban en sus casas todos los que hacen negocios con mi amo’. Así pues, llamando a cada uno de los deudores de su señor, dijo al primero: ‘¿Cuánto debes a mi amo?’ Este respondió: ‘Cien barriles de aceite’. A continuación, dijo el mayordomo: ‘toma la tablilla de cera de tu cuenta, siéntate pronto, y cámbialo a cincuenta’. Después dijo a otro deudor: ‘Y tú ¿cuánto debes?’. Y él le respondió: ‘cien medidas de trigo’. Entonces le dijo el mayordomo: ‘toma tu cuenta y escribe ochenta’. Y así hizo con numerosos otros deudores. Y de esta manera, este mayordomo deshonesto quiso hacerse de amigos para cuando se le despojara de la mayordomía. Incluso su señor y amo, cuando se enteró después de estas cosas, se vio forzado a admitir que su mayordomo infiel había al menos actuado sagazmente, por cómo había buscado proveerse para los días futuros de escasez y adversidad.
\vs p169 2:5 “Y esa es la manera en la que los hijos de este mundo demuestran en ocasiones más sabiduría en sus planes para el futuro que los hijos de la luz. Os digo a vosotros que clamáis estar acumulando tesoros en el cielo: tomad ejemplo de los que ganan amigos por medio de las riquezas injustas y, del mismo modo, dirigid vuestra vida para entablar amistad eterna con las fuerzas de la rectitud para que, cuando todas las cosas terrenales os falten, seáis recibido con júbilo en las moradas eternas.
\vs p169 2:6 “Yo afirmo que el que es fiel en lo poco, también en lo más es fiel; y el que en lo muy poco es injusto, también en lo más es injusto. Si no habéis demostrado previsión y honestidad en los asuntos de este mundo, ¿cómo podéis tener la esperanza de ser fieles y prudentes cuando se os confíe la mayordomía de las verdaderas riquezas del reino celestial? Si no sois buenos mayordomos y tesoreros fieles, si en lo ajeno no fuisteis fieles, ¿quién será tan necio como para daros el gran tesoro que será vuestro?
\vs p169 2:7 “Y de nuevo os digo que ningún hombre puede servir a dos señores; o bien odiará a uno y amará al otro, o bien estimará al uno y menospreciará al otro. No podéis servir a Dios y a las riquezas”.
\vs p169 2:8 \pc Cuando los fariseos allí presentes oyeron estas cosas, se burlaban y mofaban de él por ser muy dados a amasar riquezas. Estos asistentes poco amigables procuraron implicarlo en argumentaciones inútiles, pero él se negó a debatir con sus enemigos. Cuando los fariseos comenzaron a reñir entre ellos, su vocerío hizo venir a una gran parte de la multitud que se encontraba acampada en las proximidades; y cuando estos comenzaron, a su vez, a discutir entre ellos, Jesús se retiró, y se fue a su tienda a descansar por esa noche.
\usection{3. EL RICO Y EL MENDIGO}
\vs p169 3:1 Cuando la reunión se volvió demasiado tumultuosa, Simón Pedro, levantándose, se impuso, diciendo: “Hombres y hermanos, es inapropiado que discutáis de esa manera entre vosotros. El Maestro ha hablado, y haríais bien en meditar sus palabras. Y no es nueva la doctrina que él os proclama. ¿Es que no habéis oído también el relato alegórico de los nazareos sobre el rico y el mendigo? Algunos entre nosotros hemos oído a Juan el Bautista decir en voz bien alta esta parábola, que sirve de advertencia a quienes aman las riquezas y codician la fortuna desleal. Y aunque esta antigua parábola no se conforma al evangelio que predicamos, todos haríais bien en prestar atención a su enseñanza, hasta el momento en el que comprendáis la nueva luz del reino de los cielos. La historia, tal como Juan la contó, era así:
\vs p169 3:2 “Había un hombre rico llamado Dives, que se vestía de púrpura y de lino fino, y que cada día vivía en júbilo y de forma esplendorosa. Y había también un mendigo llamado Lázaro, que estaba echado a la puerta del rico, lleno de llagas y ansiaba saciarse de las migajas que caían de la mesa del rico; sí, aun los perros venían y le lamían las llagas. Y aconteció que murió el mendigo y fue llevado por los ángeles para descansar en el seno de Abraham. Y entonces, al poco tiempo, murió también el rico y fue enterrado con gran pompa y esplendor real. Al partir de este mundo alzó sus ojos en el Hades, estando en tormentos, y vio de lejos a Abraham y a Lázaro en su seno. Y entonces Dives gritó: ‘Padre Abraham, ten misericordia de mí y envía a Lázaro para que moje la punta de su dedo en agua y refresque mi lengua, porque siento una gran angustia por mi castigo’. Y Abraham replicó: ‘Hijo mío, acuérdate que recibiste tus bienes en tu vida, y Lázaro sufrió males. Pero ahora todo eso ha cambiado, pues Lázaro es consolado aquí y tú atormentado. Además de todo esto, una gran sima está puesta entre nosotros y vosotros, de manera que los que quieran pasar de aquí a vosotros no pueden, ni de allá pasar acá”. Entonces, Dives le dijo a Abraham: ‘Te ruego que envíes a Lázaro de vuelta a la casa de mi padre, porque tengo cinco hermanos, para que les testifique a fin de que no vengan ellos también a este lugar de tormento’. Pero Abraham dijo: ‘Hijo mío, a Moisés y a los profetas tienen; ¡que los oigan a ellos!’. Y después contestó Dives: ‘¡No, no, padre Abraham! Pero si alguno de los muertos va a ellos, se arrepentirán’. Y entonces Abraham le dijo: ‘Si no oyen a Moisés y a los profetas, tampoco se persuadirán aunque alguno se levante de los muertos’”.
\vs p169 3:3 Una vez que Pedro les refiriera esta antigua parábola de la hermandad nazarea, y dado que la multitud se había tranquilizado, Andrés se levantó y los despidió por esa noche. Aunque tanto los apóstoles como sus discípulos a menudo le hacían preguntas a Jesús sobre la parábola de Dives y Lázaro, él nunca accedió a hacer comentarios sobre la misma.
\usection{4. EL PADRE Y SU REINO}
\vs p169 4:1 A Jesús siempre le resultó difícil explicar a los apóstoles que, aunque ellos proclamaban la instauración del reino de Dios, el Padre de los cielos \bibemph{no era un rey}. En los días en los que Jesús vivió en la tierra y enseñó en la carne, la gente de Urantia conocía mayormente a los reyes y emperadores como los gobernantes de las naciones, y los judíos, desde hacía mucho tiempo, habían previsto la llegada del reino de Dios. Por estas y otras razones, el Maestro creyó conveniente designar a la hermandad espiritual del hombre con el término de reino de los cielos y al jefe espiritual de esta hermandad con el de \bibemph{Padre de los cielos}. Jesús nunca aludió a su Padre como a un rey. En sus conversaciones privadas con los apóstoles siempre se refería a sí mismo como el Hijo del Hombre y como su hermano mayor. Describía a todos sus seguidores como servidores de la humanidad y mensajeros del evangelio del reino.
\vs p169 4:2 Jesús nunca dio a sus apóstoles una lección planificada sobre la persona y los atributos del Padre de los cielos. Nunca pidió a los hombres que creyeran en su Padre; asumía que era así. Jesús nunca condescendió a ofrecer argumentaciones que probaran la realidad del Padre. Sus enseñanzas sobre el Padre se centraban en su aseveración de que él y el Padre son uno; que el que ha visto al Hijo, ha visto al Padre; que el Padre, como el Hijo, conoce todas las cosas; que nadie realmente conoce al Padre, sino el Hijo y aquel a quien el Hijo lo quiera revelar; que quien conoce al Hijo también conoce al Padre; y que el Padre lo envió al mundo para revelar sus naturalezas unificadas y mostrar su labor conjunta. Nunca se pronunció más sobre su Padre, excepto a la mujer de Samaria del pozo de Jacob, cuando declaró: “Dios es espíritu”.
\vs p169 4:3 \pc Sabéis de Dios por Jesús, observando la divinidad de su vida, no en función de sus enseñanzas. De la vida del Maestro, cada cual puede asimilar esa idea de Dios que represente la medida de su capacidad para percibir las realidades espirituales y divinas, las verdades reales y eternas. Lo finito jamás podrá comprender al Infinito, salvo cuando el Infinito convergió en la persona espacio\hyp{}temporal, que expresa la experiencia finita de la vida humana de Jesús de Nazaret.
\vs p169 4:4 Jesús era bien consciente de que Dios solo puede conocerse mediante las realidades de la experiencia; nunca se le puede entender mediante meras enseñanzas que apelen a la mente. Jesús enseñó a sus apóstoles que, aunque jamás podrían llegar a entender del todo a Dios, sin duda podrían \bibemph{conocerlo,} tal como habían conocido al Hijo del Hombre. Se puede conocer a Dios, no entendiendo lo que Jesús dijo, sino conociendo lo que Jesús fue. Jesús mismo \bibemph{fue} la revelación de Dios.
\vs p169 4:5 \pc Excepto cuando citaba las escrituras hebreas, Jesús se refirió a la Deidad usando solo dos nombres: Dios y Padre. Y cuando el Maestro hacía referencia a su Padre como Dios, normalmente empleaba la palabra hebrea que significaba el Dios plural (la Trinidad) y no la palabra Yahvé, que representaba la progresiva conceptualización del Dios tribal de los judíos.
\vs p169 4:6 Jesús nunca llamó rey al Padre, y se lamentaba profundamente que la esperanza que los judíos albergaban de la restauración de su reino y la proclamación de Juan de un reino venidero habían hecho necesario que él denominara reino de los cielos a la hermandad espiritual que había previsto inaugurar. Con una excepción ---la declaración de que “Dios es espíritu”--- Jesús nunca aludió a la Deidad a no ser en términos que describieran su propia relación personal con la Primera Fuente y Centro del Paraíso.
\vs p169 4:7 Jesús empleó la palabra “Dios” para designar la \bibemph{idea} de la Deidad y la palabra “Padre” para nombrar la \bibemph{experiencia} de conocer a Dios. Cuando el término “Padre” se emplea para significar Dios, debería entenderse en su sentido más amplio posible. El vocablo “Dios” no puede definirse y representa, por lo tanto, el concepto infinito del Padre, mientras que el término “Padre”, al poder definirse parcialmente, se puede emplear para representar el concepto humano del Padre divino en su relación con el hombre durante el trascurso de su existencia mortal.
\vs p169 4:8 Para los judíos, Elohim era el Dios de dioses, mientras que Yahvé era el Dios de Israel. Jesús aceptó el concepto de Elohim y llamó Dios a este grupo supremo de seres. En lugar de la noción de Yahvé, como deidad racial, introdujo la idea de la paternidad de Dios y la hermandad mundial del hombre. Exaltó el concepto de Yahvé, de Padre deificado de la raza, a la idea del Padre de todos los hijos de los hombres, del Padre divino de cada uno de los creyentes. Y enseñó además que este Dios de los universos y este Padre de todos los hombres eran exactamente lo mismo que la Deidad del Paraíso.
\vs p169 4:9 Jesús nunca afirmó que era la manifestación de Elohim (Dios) en la carne. Nunca declaró que era la revelación de Elohim (Dios) a los mundos. Nunca enseñó que el que le había visto a él había visto a Elohim (Dios). Pero se proclamó a sí mismo como la revelación del Padre en la carne y ciertamente dijo que todos los que le habían visto a él habían visto al Padre. Como Hijo divino afirmó que representaba solo al Padre.
\vs p169 4:10 Él era, realmente, el Hijo incluso del Dios Elohim; pero, en su semejanza de hombre mortal y para los hijos mortales de Dios, optó por limitar la revelación de su vida en la tierra a la descripción del carácter de su Padre, en la medida en la que tal revelación pudiera ser comprensible para el hombre mortal. En lo que respecta al carácter de las otras personas de la Trinidad del Paraíso, tendremos que contentarnos con la enseñanza de que son totalmente como el Padre, cuya imagen personal se ha revelado en la vida de su Hijo encarnado, Jesús de Nazaret.
\vs p169 4:11 \pc Aunque Jesús reveló la verdadera naturaleza del Padre celestial en su vida en la tierra, enseñó pocas cosas sobre él. En verdad, dijo solo dos cosas: que Dios en sí mismo es espíritu y que, en cualquier cuestión referida a las relaciones con sus criaturas, él es un Padre. Esa noche, Jesús se pronunció definitivamente sobre su relación con Dios, cuando declaró: “Salí del Padre y he venido al mundo; otra vez dejo el mundo y regreso al Padre”.
\vs p169 4:12 Pero, ¡cuidado!, Jesús nunca dijo: “El que me ha oído a mí, ha oído a Dios”. Aunque sí dijo: “El que me ha visto a mí, ha \bibemph{visto} al Padre”. Oír las enseñanzas de Jesús no equivale a conocer a Dios; pero \bibemph{ver} a Jesús es una vivencia que es en sí misma la revelación del Padre al alma. El Dios de los universos gobierna una gigantesca creación, pero es el Padre de los cielos quien envía a su espíritu para que habite en vuestra mente.
\vs p169 4:13 Jesús es una lente espiritual con la semejanza de un hombre mortal, que hace visible a la criatura material a Aquel que es invisible. Él es vuestro hermano mayor que, en la carne, hace que \bibemph{conozcáis} a un Ser de atributos infinitos a quien ni siquiera las multitudes celestiales pueden llegar a comprender por completo. Sin embargo, todas estas verdades deben ser parte de la experiencia personal del \bibemph{creyente individual}. Dios que es espíritu solo puede conocerse como experiencia espiritual. Dios puede revelarse a los hijos finitos de los mundos materiales por el Hijo divino de los ámbitos espirituales solo como \bibemph{Padre}. Podéis conocer al Eterno como un Padre; podéis adorarlo como el Dios de los universos, como el Creador infinito de todas las existencias.
