\upaper{81}{Desarrollo de la civilización moderna}
\author{Arcángel}
\vs p081 0:1 A pesar de las vicisitudes que sobrevinieron por el malogro de los planes en cuanto al mejoramiento del mundo, previstos en las misiones de Caligastia y Adán, la evolución orgánica básica de la especie humana continuó impulsando a las razas en la escala del progreso humano y del desarrollo racial. La evolución puede retrasarse, pero no detenerse.
\vs p081 0:2 La influencia ejercida por la raza violeta, aunque el número de miembros de esta raza fuese inferior al planeado, produjo un avance de la civilización que, desde los días de Adán, superó con creces el progreso que la humanidad había realizado a lo largo de toda su existencia previa de casi un millón de años.
\usection{1. LA CUNA DE LA CIVILIZACIÓN}
\vs p081 1:1 Durante aproximadamente treinta y cinco mil años después de los días de Adán, la cuna de la civilización estuvo en el sudoeste de Asia. Se extendía desde el valle del Nilo hacia el este y ligeramente hacia el norte cruzando Arabia septentrional a través de Mesopotamia hasta llegar al Turquestán. El \bibemph{clima} fue el factor decisivo en el establecimiento de la civilización en esa zona.
\vs p081 1:2 Fueron los grandes cambios climáticos y geológicos ocurridos en el norte de África y en el oeste de Asia los que pusieron término a las primeras emigraciones de los adanitas, retirándolos de Europa por la expansión del Mediterráneo y desviando el flujo migratorio al norte y al este hasta el Turquestán. En el momento de la conclusión de estas elevaciones de tierra y de los cambios climáticos que los acompañaron, alrededor del año 15\,000 a. C., la civilización se encontraba en un estado de estancamiento mundial salvo por los fermentos culturales y las reservas biológicas de los anditas, que todavía estaban confinados en el este por las montañas de Asia y en el oeste por los bosques en expansión de Europa.
\vs p081 1:3 La evolución climática estaba entonces a punto de lograr algo en lo que todas las otras iniciativas habían fracasado, esto es, obligar al hombre eurasiático a abandonar la caza en favor de las ocupaciones más avanzada del pastoreo y de la agricultura. La evolución puede ser lenta, pero es sumamente efectiva.
\vs p081 1:4 Puesto que los primeros agricultores hacían uso de los esclavos de manera tan generalizada, se les despreciaba por parte del cazador y por el pastor. Durante eras, se consideraba servil labrar la tierra de ahí la idea de que la labranza es una maldición, cuando se trata de la mayor de las bendiciones. Incluso en los días de Caín y Abel, los sacrificios provenientes del pastoreo se tenían en mayor estima que las ofrendas de la agricultura.
\vs p081 1:5 El hombre habitualmente se convertía agricultor tras ser cazador y pasar transitoriamente por una era de pastoreo, y esto también le aconteció a los anditas, pero con mayor frecuencia el apremio evolutivo de la necesidad climática hacía que tribus enteras pasaran directamente de la condición de cazadores hasta la de prósperos agricultores. Pero este fenómeno de pasar de forma inmediata de la caza a la agricultura solo ocurrió en aquellas regiones en las que había un alto grado de mezcla racial con el linaje violeta.
\vs p081 1:6 Los pueblos evolutivos (particularmente los chinos) aprendieron pronto a plantar semillas y a cultivar cosechas, tras observar la germinación de las semillas que se habían humedecido accidentalmente o que se habían puesto en las tumbas como alimento para los difuntos. Pero, por todo el suroeste de Asia, a lo largo de los fértiles fondos fluviales y de las llanuras colindantes, los anditas estaban llevando a cabo técnicas agrícolas mejoradas tal como las habían heredado de sus ancestros, que habían hecho de la agricultura y de la horticultura sus principales actividades dentro de las lindes del segundo jardín.
\vs p081 1:7 Durante miles de años, los descendientes de Adán habían cultivado trigo y cebada, tal como estos cultivos se habían perfeccionado en el Jardín, a lo largo de las altiplanicies del borde superior de Mesopotamia. Aquí, los descendientes de Adán y de Adánez se encontraron, comerciaron y se entremezclaron socialmente.
\vs p081 1:8 Estos cambios obligados de las condiciones de vida fueron los causantes de que una parte tan grande de la raza humana se volviera omnívora. Y la combinación de un régimen de trigo, arroz y vegetal con la carne de los rebaños supuso un gran paso hacia adelante en la salud y en el vigor de estos ancestrales pueblos.
\usection{2. INSTRUMENTOS CIVILIZADORES}
\vs p081 2:1 El crecimiento de la cultura depende del desarrollo de instrumentos civilizadores. Y los utilizados por el hombre para salir del salvajismo fueron eficaces justo en la medida en que liberaban su capacidad de realizar tareas superiores.
\vs p081 2:2 Vosotros que ahora vivís en medio de un escenario posterior de florecimiento de la cultura y del comienzo de un avance en los asuntos sociales, vosotros que realmente disponéis de algo de tiempo libre para \bibemph{pensar} sobre la sociedad y la civilización, no debéis ignorar el hecho de que vuestros primitivos ancestros tenían poco o casi ningún tiempo de ocio para poder centrarse en la reflexión y en el pensamiento social.
\vs p081 2:3 \pc Los cuatro primeros avances sociales de la civilización humana fueron:
\vs p081 2:4 \li{1.}El dominio del fuego.
\vs p081 2:5 \li{2.}La domesticación de los animales.
\vs p081 2:6 \li{3.}La esclavización de los cautivos.
\vs p081 2:7 \li{4.}La propiedad privada.
\vs p081 2:8 \pc Aunque el fuego, el primer gran descubrimiento, acabaría por abrir las puertas del mundo científico, tuvo poco valor en este aspecto para el hombre primitivo, que se negaba a reconocer causas naturales como explicaciones de fenómenos habituales.
\vs p081 2:9 Cuando se preguntaba de dónde venía el fuego, la sencilla historia de Andón y el pedernal se sustituyó pronto por la leyenda de que un tal Prometeo lo había robado del cielo. Los antiguos buscaban una explicación sobrenatural para todos los fenómenos naturales que no estuviesen dentro de los límites de su comprensión personal; y, en tiempos recientes, hay quienes siguen haciendo lo mismo. Ha llevado eras despersonalizar los fenómenos naturales, y aún no ha concluido. Pero la búsqueda sincera, honesta y valiente de las causas verdaderas dio origen a la ciencia moderna: transformó la astrología en astronomía, la alquimia en química y la magia en medicina.
\vs p081 2:10 \pc Durante la era previa a la máquina, la única forma que tenía el hombre de poder realizar su labor era usar animales. Su domesticación le puso en las manos herramientas vivas, y el uso inteligente de ellas allanó el camino tanto para la agricultura como para el transporte. Y, sin estos animales, el hombre no podría haberse elevado desde su estado primitivo hasta los niveles de la civilización posterior.
\vs p081 2:11 La mayoría de los animales más adecuados para su domesticación se encontraban en Asia, especialmente en las regiones centrales y sudoccidentales. Por dicha razón, la civilización progresó más rápidamente en esos territorios que en otras partes del mundo. Muchos de estos animales se habían domesticados dos veces con anterioridad, y en la era andita lo fueron de nuevo. Pero el perro había permanecido con los cazadores desde que el hombre azul lo adoptó muchísimo tiempo antes.
\vs p081 2:12 Los anditas de Turquestán fueron los primeros pueblos que domesticaron a un gran número de caballos, y esta fue otra razón por la que su cultura fue preponderante por tanto tiempo. Para el año 5000 a. C., los granjeros de Mesopotamia, del Turquestán y chinos habían comenzado a criar ovejas, cabras, vacas, camellos, caballos, aves y elefantes. Empleaban como animales de carga al buey, al camello, al caballo y al yak. En cierta época, el hombre mismo fue uno de ellos. Un jefe de la raza azul tuvo en determinada ocasión a un grupo de cien mil hombres como porteadores de cargas.
\vs p081 2:13 \pc La institución de la esclavitud y de la propiedad privada del suelo llegó con la agricultura. La esclavitud elevó el nivel de vida del amo y facilitó más tiempo de ocio para la cultura social.
\vs p081 2:14 El salvaje es un esclavo de la naturaleza, pero la civilización científica está lentamente dotando de cada vez mayor libertad a la humanidad. Mediante los animales, el fuego, el viento, el agua, la electricidad y otras fuentes de energías por descubrir, el hombre se ha liberado y continuará liberándose a sí mismo de la necesidad del duro e incesante trabajo. A pesar de los problemas transitorios producidos por la invención prolífica de maquinarias, resultan incalculables los beneficios fundamentales que se derivarán de tales inventos mecánicos. La civilización nunca puede florecer, y mucho menos establecerse, hasta que el hombre no disponga de \bibemph{tiempo de ocio} para pensar, planear, imaginar formas nuevas de hacer las cosas.
\vs p081 2:15 \pc Al principio, el hombre simplemente tomaba posesión de su refugio; vivía bajo los salientes de las rocas o en cuevas. Posteriormente, adaptó materiales naturales como la madera y la piedra para erigir cabañas familiares. Por último, entró en la etapa creadora de la construcción de casas, aprendió a fabricar ladrillo y otros materiales de construcción.
\vs p081 2:16 Entre las razas más modernas, los pueblos de las altiplanicies del Turquestán fueron los primeros en construir sus hogares de madera, no muy diferentes de las primitivas cabañas de troncos de los pioneros norteamericanos. En las llanuras, las viviendas humanas se hacían de ladrillos; más tarde, se hicieron de ladrillos cocidos.
\vs p081 2:17 Las más antiguas razas ribereñas construían sus cabañas colocando unos palos altos en la tierra en forma de círculo; se unían luego los extremos superiores haciendo un armazón para la cabaña, que se entrelazaba con cañas transversales, de forma que todo el conjunto parecía un enorme canasto invertido. Esta estructura se podía entonces embadurnar de arcilla y, tras secarse al sol, constituía una vivienda muy práctica y resistente a la intemperie.
\vs p081 2:18 A partir de estas primeras cabañas, se desarrolló la idea de tejer todo tipo de cestos por separado. En uno de los grupos, surgió la idea de hacer cerámica al observar los efectos del untamiento de estas estructuras de palo con arcilla húmeda. La práctica de endurecer la cerámica mediante cocción se descubrió cuando una de estas cabañas primitivas recubiertas de arcilla se incendió de forma accidental. Las artes de la antigüedad se originaban muchas veces en hechos fortuitos intrínsecos a la vida diaria de estos pueblos primitivos. Al menos, esto fue casi enteramente cierto del progreso evolutivo de la humanidad hasta la llegada de Adán.
\vs p081 2:19 Aunque la comitiva del príncipe había introducido por vez primera la alfarería hace aproximadamente medio millón de años, la fabricación de recipientes de arcilla había prácticamente cesado durante más de ciento cincuenta mil años. Solamente los noditas presumerios de la costa del golfo las seguían fabricando. El arte de la alfarería se restableció en los tiempos de Adán. La propagación de este arte fue paralela a la extensión de las áreas desérticas de África, Arabia y Asia central, y se difundió sucesivamente de forma mejorada desde Mesopotamia a todo el hemisferio oriental.
\vs p081 2:20 No siempre se pueden rastrear estas civilizaciones de la era andita acudiendo a las etapas de su alfarería o de otras artes. El rumbo fluido de la evolución humana se vio enormemente alterado por los regímenes de Dalamatia y de Edén. Sucede con frecuencia que las vasijas y los utensilios más tardíos son inferiores a las tempranas producciones de los pueblos anditas más puros.
\usection{3. CIUDADES, MANUFACTURA Y COMERCIO}
\vs p081 3:1 La destrucción climática de los ricos pastizales abiertos de caza y de las tierras de pastoreo del Turquestán, comenzada hacia el año 12\,000 a. C., obligó a los hombres de esas regiones a recurrir a nuevos tipos de labores y de manufactura rudimentaria. Algunos se ocuparon del cultivo de rebaños domésticos, otros se volvieron agricultores o recolectaron alimentos de origen acuático, pero los tipos mejor dotados de intelectos anditas optaron por el comercio y la manufactura. Incluso se hizo costumbre que una tribu completa se dedicase a una sola clase de actividad. Desde el valle del Nilo hasta el Hindu Kush y desde el Ganges hasta el río Amarillo, la ocupación principal de las tribus superiores llegó a ser el cultivo del suelo, con el comercio de forma secundaria.
\vs p081 3:2 El aumento del comercio y de la manufactura de la materia prima para su transformación en distintos artículos de comercio redundó, directamente, en la creación de esas primitivas comunidades semipacíficas, que tanto influyeron en la difusión de la cultura y de las artes de la civilización. Antes de la era de un generalizado comercio mundial, las comunidades sociales eran tribales ---grupos familiares expandidos---. El comercio trajo consigo la relación fraternal entre distintos tipos de seres humanos, contribuyendo así a un intercambio más rápido y fecundo de la cultura.
\vs p081 3:3 Hace unos doce mil años, la era de las ciudades independientes estaba en sus comienzos. Y estas primeras ciudades comerciales y manufactureras estaban siempre rodeadas de zonas de agricultura y de ganadería. Aunque es verdad que la elevación del nivel de vida impulsó la industria, no debéis formaros ideas equivocadas sobre los refinamientos de la temprana vida urbana. Las razas primitivas no eran excesivamente ordenadas ni limpias, y la comunidad media primitiva se elevaba entre más de treinta a casi setenta centímetros cada veinticinco años como resultado de la mera acumulación de la suciedad y de los desechos. Algunas de estas antiguas ciudades también se elevaban por encima de las tierras aledañas con mucha rapidez porque sus chozas de barro no cocido eran de corta duración, y se acostumbraba a construir nuevas viviendas directamente encima de las ruinas de las viejas.
\vs p081 3:4 \pc La utilización generalizada de los metales fue uno de los rasgos de esta era de las primeras ciudades industriales y comerciales. Ya habéis hallado en el Turquestán una cultura del bronce que data de antes del año 9000 a. C., y los anditas aprendieron pronto a trabajar asimismo el hierro, el oro y el cobre. Pero las condiciones eran muy diferentes lejos de los centros de civilización más avanzados. No hubo períodos distintos, tales como la Edad de Piedra, de Bronce y de Hierro; los tres existieron al mismo tiempo en localidades diferentes.
\vs p081 3:5 El oro fue el primer metal que el hombre buscó; era fácil de trabajar y, al principio, se usó solamente como adorno. Luego, se empleó el cobre, pero no de forma general hasta que se mezcló con el estaño para producir el bronce, de mayor dureza. El descubrimiento de la mezcla de cobre con estaño para hacer bronce lo realizó uno de los descendientes de Adánez del Turquestán, cuya mina de cobre de las altiplanicies se encontraba precisamente situada al lado de un yacimiento de estaño.
\vs p081 3:6 \pc Con la aparición de una rudimentaria manufactura y de una incipiente industria, el comercio rápidamente se convirtió en el más potente influjo en la difusión de la civilización cultural. La apertura de las rutas comerciales por tierra y por mar facilitó en gran medida el viaje y la mezcla de las culturas al igual que de las civilizaciones. Hacia el año 5000 a. C., el caballo se usaba comúnmente en todos los territorios civilizados y semicivilizados. Estas razas posteriores, no solo habían domesticado al caballo sino que también tenían diferentes clases de carretas y carruajes. Mucho tiempo antes se había empleado la rueda, pero ahora el uso de los vehículos provistos de ellas se generalizó en el comercio y en la guerra.
\vs p081 3:7 Los comerciantes itinerantes y los exploradores, en su búsqueda de distintos destinos, hicieron más por el avance de la civilización histórica que el conjunto de todas las demás factores. Las conquistas militares, la colonización y las iniciativas misioneras, promovidas por las religiones que vendrían después, fueron otros elementos que contribuyeron a la difusión de la cultura; pero todos ellos eran secundarios respecto a las relaciones comerciales, siempre en creciente aumento debido al desarrollo de las artes y de las ciencias de la industria.
\vs p081 3:8 La infusión del linaje adánico en las razas humanas no solo aceleró el ritmo de la civilización, sino que también estimuló notablemente su propensión hacia la aventura y la exploración, por lo que la mayor parte de Eurasia y el norte de África se ocupó enseguida con los descendientes híbridos de los anditas, que se multiplicaban rápidamente.
\usection{4. RAZAS HÍBRIDAS}
\vs p081 4:1 Al acercarnos a los comienzos de los tiempos históricos, toda Eurasia, el norte de África y las islas del Pacífico están poblados por las razas mixtas de la humanidad. Y las razas de hoy en día son el resultado de una repetida mezcla de cinco básicos linajes humanos de Urantia.
\vs p081 4:2 Cada una de las razas de Urantia se identifica por ciertos rasgos físicos distintivos. Los adanitas y los noditas eran de cabeza alargada; los andonitas, de cabeza ancha. Las razas sangiks eran de cabeza de tamaño medio, aunque los hombres amarillos y azules tendían a tenerla ancha. Las razas azules, cuando se mezclaban con el linaje andonita, eran claramente de cabeza ancha. Los sangiks secundarios tenían las cabezas entre medianas y alargadas.
\vs p081 4:3 Aunque estas dimensiones craneales son útiles a la hora de descifrar los orígenes raciales, es más fiable acudir al esqueleto entero. En el desarrollo temprano de las razas de Urantia, había originariamente cinco tipos distintos de estructuras esqueléticas:
\vs p081 4:4 \li{1.}El andónico: los aborígenes de Urantia.
\vs p081 4:5 \li{2.}El sangik primario: las razas roja, amarilla y azul.
\vs p081 4:6 \li{3.}El sangik secundario: las razas naranja, verde e índigo.
\vs p081 4:7 \li{4.}Los noditas: los descendientes de los dalamatianos.
\vs p081 4:8 \li{5.}Los adanitas: la raza violeta.
\vs p081 4:9 \pc Puesto que estos cinco grandes grupos raciales se entremezclaron de forma considerable, la continua hibridación tendió a ocultar al tipo andónico en favor del predominio de la herencia sangik. Los lapones y los esquimales son una mezcla de andonitas y de la raza sangik azul. Su estructura esquelética es la próxima al tipo andónico aborigen. Pero los adanitas y los noditas se han mezclado tanto con otras razas que tan solamente se les puede detectar como un tipo caucasoide generalizado.
\vs p081 4:10 En general, por lo tanto, a medida que los restos humanos de los últimos veinte mil años se vayan desenterrando, será imposible distinguir con claridad los cinco tipos originarios. El estudio de estas estructuras esqueléticas revelará que en la actualidad la humanidad está dividida en aproximadamente tres clases:
\vs p081 4:11 \li{1.}\bibemph{La caucasoide:} la mezcla andita de los linajes noditas y adanitas, modificada posteriormente por la unión con los sangiks primarios y (con algunos) secundarios y por un considerable cruce andónico. Las razas blancas occidentales, junto con algunos pueblos indios y turanianos, están incluidas en este grupo. El elemento que unifica esta división es la mayor o menor proporción de herencia andita.
\vs p081 4:12 \li{2.}\bibemph{La mongoloide:} el tipo sangik primario, que comprende a las razas roja, amarilla y azul originarias. Los chinos y los amerindios pertenecen a este grupo. En Europa, el tipo mongoloide se ha modificado mediante una mezcla con los sangiks secundarios y los andonitas, e incluso más debido a la infusión andita. Los malayos y otros pueblos indonesios se incluyen en esta clasificación, aunque contienen un alto porcentaje de sangre sangik secundaria.
\vs p081 4:13 \li{3.}\bibemph{La negroide:} el tipo sangik secundario, que incluía originariamente a las razas naranja, verde e índigo. El mejor ejemplo de este tipo es el hombre negro, y se puede encontrar en África, la India e Indonesia, en todos los lugares en donde se emplazaron las razas sangiks secundarias.
\vs p081 4:14 \pc En la China del norte existe cierta unión de los tipos caucasoide y mongoloide; en el Levante, se han entremezclado los caucasoides y los negroides; en la India, al igual que en América del Sur, se encuentran representadas las tres clases. Y las características del esqueleto de los tres tipos supervivientes aún persisten y ayudan a identificar la herencia posterior de las razas humanas de hoy.
\usection{5. LA SOCIEDAD CULTURAL}
\vs p081 5:1 La evolución biológica y la civilización cultural no están necesariamente correlacionadas; la evolución orgánica, en cualquier era, puede proseguir sin trabas en medio de la decadencia cultural. Pero cuando se analizan períodos prolongados de la historia humana, se puede observar que, a la larga, la evolución y la cultura llegan a relacionarse como causa y efecto. La evolución puede avanzar en ausencia de la cultura, pero la civilización cultural no florece sin el debido trasfondo de una previa progresión racial. Adán y Eva no introdujeron arte alguno de la civilización que fuera relevante para el progreso de la sociedad humana, pero la sangre adánica incrementó la aptitud innata de las razas y ciertamente aceleró el ritmo del desarrollo económico y el avance industrial. El ministerio de Adán mejoró la capacidad cerebral de las razas, por lo que dio un notable impulso al proceso de la evolución natural.
\vs p081 5:2 Por medio de la agricultura, la domesticación de los animales y una mejor arquitectura, la humanidad escapó paulatinamente de lo peor de la incesante lucha por la vida y se lanzó a la búsqueda de cómo dulcificarla; así comenzó su afán por hallar niveles cada vez más elevados de confort material. Mediante la manufactura y la industria, el hombre está aumentando gradualmente el elemento placer de la vida mortal.
\vs p081 5:3 Pero la sociedad cultural no es una gran asociación benéfica en la que todos los hombres nacen con el privilegio heredado de pertenecer a ella de forma gratuita y completamente igualitaria. Es más bien una eminente agrupación siempre en avance de trabajadores del mundo, que admite en sus filas solo a los más grandes, a aquellos que se esfuerzan y luchan por hacer del mundo un lugar mejor en el que sus hijos y los hijos de sus hijos puedan vivir y seguir avanzando en eras venideras. Y tal agrupación, como parte de la civilización, exige altos precios de admisión, impone disciplinas estrictas y rigurosas, establece severas penalizaciones a todos los disidentes y no conformistas, mientras que concede pocas licencias o privilegios personales salvo los de una mayor seguridad contra los peligros comunes y los riesgos raciales.
\vs p081 5:4 La asociación social es una forma de seguro de supervivencia, cuya utilidad los seres humanos conocen; por lo tanto, la mayoría de las personas está dispuesta a pagar esas cuotas de sacrificio de sí mismo al igual que la reducción de la libertad personal que la sociedad requiere de sus miembros, a cambio de esta avanzada protección colectiva. En resumen, el mecanismo social de hoy en día es un plan de seguro sobre la base de ensayo y error, destinado a proporcionar cierto grado de seguridad y protección contra una vuelta a las terribles condiciones antisociales que caracterizaban las primeras experiencias de la raza humana.
\vs p081 5:5 La sociedad se convierte así en un sistema cooperativo que asegura la libertad civil a través de las instituciones, la libertad económica a través del capital y la invención, la libertad social a través de la cultura y la ausencia de violencia a través de normativas policiales.
\vs p081 5:6 \bibemph{El poder no da la razón, pero de cierto hace que se cumplan los derechos comúnmente reconocidos de cada generación venidera}. La misión primordial del gobierno es la definición del derecho, la reglamentación justa y equitativa de las diferencias de clase y la aplicación de la igualdad de oportunidades bajo las reglas de la ley. Todo derecho humano va acompañado de un deber social; el privilegio del grupo es un mecanismo de seguridad que indefectiblemente exige el pago íntegro de rigurosas primas por el servicio del grupo. Y los derechos del grupo, al igual que los del individuo, deben protegerse, incluida la regulación de la propensión al sexo.
\vs p081 5:7 La libertad sujeta a la normativa del grupo es el objetivo legítimo de la evolución social. Una libertad sin ningún tipo de restricción es el sueño vano y descabellado de mentes humanas inestables e irresponsables.
\usection{6. MANTENIMIENTO DE LA CIVILIZACIÓN}
\vs p081 6:1 Aunque la evolución biológica ha continuado siempre hacia adelante, gran parte de la evolución cultural partió del valle del Éufrates en oleadas que fueron sucesivamente debilitándose a medida que trascurría el tiempo, hasta que finalmente la totalidad de los descendientes adánicos de puro linaje había salido para enriquecer las civilizaciones de Asia y Europa. Las razas no se mezclaron por completo, pero sus civilizaciones sí lo hicieron en un grado considerable. La cultura se extendió lentamente por todo el mundo. Y esta civilización debe mantenerse y fomentarse, porque hoy en día no existen nuevas fuentes de cultura ni anditas que vigoricen y estimulen el lento progreso evolutivo de la civilización.
\vs p081 6:2 \pc La civilización que se está desarrollando actualmente en Urantia surgió y se fundamenta en los siguientes factores:
\vs p081 6:3 \li{1.}\bibemph{Las circunstancias naturales}. La naturaleza y el grado de civilización material están en gran medida determinados por los recursos naturales disponibles. El clima, el tiempo atmosférico y numerosas condiciones físicas son factores fundamentales en la evolución de la cultura.
\vs p081 6:4 Al inicio de la era andita solamente había dos extensos territorios de caza abiertos y fértiles en todo el mundo. Uno de ellos se localizaba en América del Norte y estaba completamente ocupado por los amerindios; el otro se hallaba al norte del Turquestán y estaba parcialmente ocupado por una raza andónico\hyp{}amarilla. La raza y el clima fueron los elementos que influyeron decisivamente en la evolución de una cultura superior en el suroeste de Asia. Los anditas eran un gran pueblo, pero el elemento principal que determinó el curso de su civilización fue el aumento de la aridez en Irán, el Turquestán y Xinjiang, y que los \bibemph{forzó} a inventar y adoptar métodos nuevos y avanzados de arrancar un medio de vida a sus tierras, cada vez menos fértiles.
\vs p081 6:5 La configuración de los continentes y otras condiciones relativas a la disposición del suelo tienen un gran influjo para la paz o para la guerra. Muy pocos urantianos han tenido jamás una oportunidad tan favorable para el desarrollo continuo y sin hostigamiento como la que disfrutaron los pueblos de América del Norte, protegidos por inmensos océanos prácticamente por todos lados.
\vs p081 6:6 \li{2.}\bibemph{Los bienes de capital}. La cultura nunca se desarrolla bajo condiciones de pobreza; el ocio es esencial para el progreso de la civilización. Una persona puede adquirir un carácter que tenga valores morales y espirituales en ausencia de la riqueza material, pero una civilización cultural solamente se puede derivar de esas condiciones de prosperidad material que promuevan ocio en combinación con aspiraciones.
\vs p081 6:7 Durante los tiempos primitivos, la vida en Urantia era una cuestión seria y sobria. Y fue para escapar de esa lucha incesante y de inacabable trabajo por lo que la humanidad tendió constantemente a dirigirse al salubre clima de los trópicos. Aunque estas zonas más cálidas mitigaban de alguna manera la intensa lucha por la existencia, las razas y las tribus que buscaron este alivio raras veces utilizaron su inmerecido ocio para hacer avanzar la civilización. El progreso social invariablemente procede de los pensamientos y planes de esas razas que, mediante su trabajo inteligente, han aprendido a extraer de la tierra su sustento de vida con un menor esfuerzo y con jornadas de trabajo más reducidas, pudiendo así gozar de un margen merecido y provechoso momento de ocio.
\vs p081 6:8 \li{3.}\bibemph{El conocimiento científico.} Los aspectos materiales de la civilización siempre deben esperar la acumulación de los datos científicos. Tuvo que pasar mucho tiempo tras el descubrimiento del arco y de la flecha y de la utilización de los animales por sus capacidades físicas antes de que el hombre aprendiera a explotar el viento y el agua, seguidos del empleo del vapor y de la electricidad. Pero los instrumentos de la civilización mejoraron lentamente. A la tejeduría, la alfarería, la domesticación de los animales y la metalurgia les siguió una era de escritura e imprenta.
\vs p081 6:9 El conocimiento es poder. La invención precede siempre a la aceleración del desarrollo cultural a escala mundial. La ciencia y la invención fueron las más beneficiadas de la imprenta, y la interacción de todas estas actividades culturales e inventivas ha incrementado enormemente el ritmo del avance cultural.
\vs p081 6:10 La ciencia enseña al hombre a hablar el nuevo lenguaje de las matemáticas y capacita sus pensamientos siguiendo líneas de precisión y rigurosidad. Y la ciencia estabiliza igualmente la filosofía a través de la eliminación del error, mientras que, al destruir la superstición, purifica la religión.
\vs p081 6:11 \li{4.}\bibemph{Los recursos humanos.} La acción del hombre es indispensable para la difusión de la civilización. En igualdad de condiciones, un pueblo numeroso dominará la civilización de una raza menos numerosa. En consecuencia, la incapacidad de aumentar las cifras de población hasta un cierto punto impide la plena realización del destino nacional, pero llega un momento en el incremento de la población en el que un crecimiento mayor sería suicida. La multiplicación de la población más allá de la proporción óptima normal entre tierra y hombre significa una disminución del nivel de vida o una expansión inmediata de las lindes territoriales por medio de la penetración pacífica o de la conquista militar: la ocupación por la fuerza.
\vs p081 6:12 En ocasiones os conmocionan los estragos de la guerra, pero debéis reconocer que es necesario que nazca un gran número de mortales para dar una mejor oportunidad al desarrollo social y moral; con tal fertilidad planetaria pronto ocurre el grave problema de la sobrepoblación. La mayor parte de los mundos habitados son pequeños. Urantia es de tamaño medio, quizás algo menor. La óptima estabilización de la población nacional mejora la cultura y previene la guerra. Y es sabia aquella nación que sabe cuándo detener su crecimiento.
\vs p081 6:13 Pero el continente más rico en yacimientos naturales y más avanzado en equipo mecánico hará poco progreso si la inteligencia de su pueblo está en declive. El conocimiento se obtiene a través de la educación, pero la sabiduría, que es indispensable a la verdadera cultura, tan solo puede garantizarse a través de la experiencia y mediante hombres y mujeres que sean innatamente inteligentes. Un pueblo así es capaz de aprender de la experiencia; puede llegar a ser verdaderamente sabio.
\vs p081 6:14 \li{5.}\bibemph{La eficacia de los recursos materiales.} Hay muchas cosas que dependen de la sabiduría ejercitada en la utilización de los recursos naturales, en el conocimiento científico, en los bienes de capital y en los potenciales humanos. El factor principal en la civilización primitiva era la \bibemph{fuerza} que ejercían los sabios dirigentes sociales; al hombre primitivo se le imponía literalmente la civilización por parte de sus contemporáneos superiores. En gran medida, las minorías bien organizadas y mejor dotadas han gobernado este mundo.
\vs p081 6:15 El poder no da la razón, pero de cierto crea lo que es y ha sido en la historia. Solo últimamente se ha llegado en Urantia a ese punto en el que la sociedad está dispuesta a debatir la ética del poder y la razón.
\vs p081 6:16 \li{6.}\bibemph{La eficacia del lenguaje}. La difusión de la civilización ha de aguardar al lenguaje. Las lenguas vivas y en crecimiento aseguran la expansión del pensamiento y la planificación civilizados. Durante las eras primitivas, se hicieron importantes avances en el lenguaje. Hoy día, hay una gran necesidad de un mayor desarrollo lingüístico para facilitar la expresión del pensamiento evolutivo.
\vs p081 6:17 El lenguaje evolucionó a partir de las asociaciones grupales, en las que cada grupo local desarrolló su propio sistema de intercambio de palabras. El lenguaje creció a través de gestos, signos, gritos, sonidos imitativos, entonación y acento hasta llegar a la vocalización de los posteriores alfabetos. La lengua es el instrumento más importante y práctico que posee el hombre para pensar, pero no puede florecer hasta ese momento en el que los grupos sociales dispongan de algún tiempo de ocio. La tendencia a jugar con la lengua hace que se desarrollen palabras nuevas: el argot. Si la mayoría adopta un argot particular, entonces su uso lo convierte en lenguaje. El origen de los dialectos se explica por la tolerancia hacia el “lenguaje infantil” del grupo familiar.
\vs p081 6:18 Las diferencias entre las lenguas han sido siempre el gran obstáculo a la expansión de la paz. La conquista de los dialectos debe preceder a la diseminación de una cultura en una raza, en un continente o en todo el mundo. Un lenguaje universal promueve la paz, asegura la cultura y aumenta la felicidad. Incluso si las lenguas del mundo se reducen a unas pocas, el dominio de estas por los pueblos que lideran la cultura influirá poderosamente en el logro de la paz y la prosperidad a escala mundial.
\vs p081 6:19 Aunque se ha realizado muy poco progreso en Urantia en el desarrollo de un idioma internacional, se ha conseguido bastante en cuanto al establecimiento de intercambios comerciales a nivel internacional. Y se deben fomentar todas estas relaciones internacionales, ya se traten de idiomas, comercio, arte, ciencia, juegos competitivos o religión.
\vs p081 6:20 \pc \bibemph{7. La eficacia de los dispositivos mecánicos.} El progreso de la civilización está directamente relacionado con el desarrollo y la posesión de herramientas, máquinas y canales de distribución. Unas herramientas mejores, unas máquinas ingeniosas y eficientes, determinan la supervivencia de grupos enfrentados entre sí en el contexto de una civilización en avance.
\vs p081 6:21 En los primeros días, la única fuerza que se aplicaba al cultivo del suelo era la del hombre. Fue necesaria una larga lucha para sustituirle por el buey, porque le quitaba el trabajo. Últimamente, las máquinas han empezado a desplazar al hombre, y cada uno de estos avances contribuye directamente al progreso de la sociedad porque libera la fuerza humana en bien del desempeño de tareas más valiosas.
\vs p081 6:22 La ciencia, guiada por la sabiduría, puede convertirse en la gran libertadora social del hombre. La era mecánica puede resultar catastrófica solo para aquella nación cuyo nivel intelectual sea demasiado bajo como para hallar esos métodos convenientes y técnicas acertadas que le permitan adaptarse satisfactoriamente a las dificultades transitorias derivadas de la pérdida repentina de un gran número de empleos por la invención, demasiado rápida, de nuevos tipos de maquinaria economizadoras de mano de obra.
\vs p081 6:23 \li{8.}\bibemph{El papel de las avanzadillas}. La herencia social permite al hombre apoyarse en los hombros de todos aquellos que lo precedieron y que han contribuido de alguna manera a la suma de la cultura y el conocimiento. En esta labor de pasar la antorcha cultural a la generación siguiente, el hogar por siempre será la institución básica. A continuación, viene el esparcimiento y la vida social, con la escuela como último lugar, pero igualmente indispensable en una sociedad compleja y altamente organizada.
\vs p081 6:24 Los insectos nacen plenamente formados y equipados para la vida ---de hecho, una existencia muy limitada y puramente instintiva---. El niño humano nace sin formación; por lo tanto, el hombre, mediante el control de la formación educativa de la generación más joven, posee el poder de modificar en buena parte el curso evolutivo de la civilización.
\vs p081 6:25 Las mayores influencias del siglo XX que contribuyen al fomento de la civilización y al avance de la cultura son el notable aumento de los viajes internacionales y las mejoras sin precedentes en los métodos de comunicación. Pero el desarrollo de la educación no ha seguido el ritmo de expansión de la estructura social; tampoco se ha desarrollado un reconocimiento moderno de la ética en correspondencia con el crecimiento de un criterio puramente intelectual y científico. Y la civilización moderna está estancada respecto a su desarrollo espiritual y a la salvaguardia de la institución del hogar.
\vs p081 6:26 \li{9.}\bibemph{Los ideales raciales}. Los ideales de una generación forjan las vías del destino de su descendencia inmediata. La \bibemph{calidad} de las avanzadillas sociales determinará si la civilización irá hacia adelante o hacia atrás. El hogar, las iglesias y las escuelas de una generación determinan previamente la índole de la tendencia de la generación siguiente. El ímpetu moral y espiritual de una raza o de una nación determina en buena medida la velocidad cultural de esa civilización.
\vs p081 6:27 Los ideales elevan la fuente del caudal social. Ningún caudal se elevará más alto que su fuente sea cual fuese el método de presión o el control direccional que se pueda emplear. La fuerza impulsora de incluso los aspectos más materiales de una civilización cultural reside en el menos material de los logros de la sociedad. La inteligencia puede regir el mecanismo de la civilización, la sabiduría puede dirigirlo, pero el idealismo espiritual es la energía que realmente hace ascender y avanzar la cultura humana de un nivel de realización a otro.
\vs p081 6:28 En un principio, la vida fue una lucha por la existencia; ahora, por un nivel de vida; luego será por la calidad del pensamiento, el próximo objetivo terrenal de la existencia humana.
\vs p081 6:29 \li{10.}\bibemph{La coordinación de los especialistas.} La civilización ha avanzado enormemente mediante la temprana división del trabajo y su posterior consecuencia: la especialización. Ahora, la civilización depende de la coordinación eficaz de los diferentes especialistas, y, a medida que la sociedad se expande, es necesario encontrar un método que los aúna.
\vs p081 6:30 Los especialistas sociales, artísticos, técnicos e industriales continuarán multiplicando e incrementando su capacidad y destreza. Y esta diversificación de las capacidades y disparidad de ocupaciones acabarán por debilitar y desintegrar a la sociedad humana, si no se desarrollan medios efectivos de coordinación y cooperación. Pero una inteligencia capaz de tal inventiva y de tal especialización debería ser enteramente competente para elaborar métodos adecuados de control y de regulación de todos los problemas derivados del rápido crecimiento de la invención y del ritmo acelerado de la expansión cultural.
\vs p081 6:31 \li{11.}\bibemph{Los mecanismos para encontrar empleo}. La próxima era de desarrollo social incorporará una mejor y más eficaz cooperación y coordinación de la especialización, en constante aumento y expansión. Y, a medida que la mano de obra se va diversificando cada vez más, habrá que elaborar métodos para dirigir a las personas a un empleo adecuado. Las máquinas no son la única causa de desempleo entre los pueblos civilizados de Urantia. La complejidad económica y el incremento continuo de la especialización industrial y profesional se añaden a los problemas de inserción laboral.
\vs p081 6:32 No es suficiente con entrenar a los hombres para el trabajo; en una sociedad compleja también se deben proporcionar métodos eficientes para encontrar un empleo. Antes de formar a los ciudadanos en sistemas altamente especializados de ganarse la vida, se debería hacerlo en uno o más métodos de tareas, oficios o vocaciones corrientes que se podrían utilizar en caso de que se hallen transitoriamente sin empleo en su especialidad laboral. Ninguna civilización puede sobrevivir un largo plazo si alberga a grandes grupos de desempleados. Con el tiempo, incluso los mejores ciudadanos se verán perturbados y desmoralizados si tienen que aceptar la ayuda del erario público. Incluso la caridad privada se convierte en perniciosa cuando se concede durante mucho tiempo a ciudadanos aptos.
\vs p081 6:33 Una sociedad tan altamente especializada no verá con buenos ojos las ancestrales prácticas comunales y feudales de los pueblos antiguos. Es verdad que muchos servicios comunes se pueden socializar de forma aceptable y provechosa, pero los seres humanos altamente formados y extremadamente especializados se pueden regir mejor mediante la cooperación inteligente. La coordinación modernizada y la regulación fraternal producirán una cooperación de mayor duración que los viejos y más primitivos métodos del comunismo o las instituciones regulatorias de carácter dictatorial basadas en la fuerza.
\vs p081 6:34 \li{12.}\bibemph{La disposición a cooperar.} Uno de los grandes obstáculos para el progreso de la sociedad humana es el conflicto entre los intereses y el bienestar de los grupos humanos más grandes y socializados y de los de las asociaciones humanas más pequeñas con mentalidad contraria y asocial, sin mencionar a individuos aislados de mente antisocial.
\vs p081 6:35 Ninguna civilización nacional perdura mucho tiempo a menos que sus métodos educativos e ideales religiosos inspiren un alto tipo de patriotismo inteligente y de devoción nacional. Sin este orden de patriotismo inteligente y solidaridad cultural, todas las naciones tienden a desintegrarse como resultado de celos provinciales y egoísmos locales.
\vs p081 6:36 El mantenimiento de una civilización a escala mundial depende de que los seres humanos aprendan a vivir juntos en paz y fraternidad. Sin una coordinación efectiva, la civilización industrial se ve amenazada por los peligros de la especialización extrema: la monotonía, la restricción y la tendencia a generar desconfianza y celos.
\vs p081 6:37 \li{13.}\bibemph{El liderazgo eficaz y competente}. En la civilización muchísimo depende de un espíritu cooperativo eficiente. Diez hombres valen poco más que uno solo para levantar un gran peso a menos que lo hagan todos juntos ---todos al mismo tiempo---. Este trabajo de equipo ---cooperación social--- depende del liderazgo. Las civilizaciones culturales del pasado y del presente se han basado en la cooperación inteligente de la ciudadanía con líderes competentes y avanzados; y hasta que el hombre no evolucione a unos niveles superiores, la civilización continuará dependiendo de un liderazgo lúcido y enérgico.
\vs p081 6:38 Las grandes civilizaciones nacen de la juiciosa correlación de riqueza material, grandeza intelectual, valor moral, habilidad social y percepción cósmica.
\vs p081 6:39 \li{14.}\bibemph{Los cambios sociales}. La sociedad no es una institución divina; es un fenómeno de la evolución progresiva; y la civilización en avance siempre sufre retrasos cuando sus líderes son lentos en realizar esos cambios en la organización social esenciales para mantener el mismo ritmo que los desarrollos científicos de la era. Por todo ello, no se deben despreciar las cosas simplemente por ser viejas, ni tampoco se debe adoptar una idea de forma incondicional simplemente por ser nueva e innovadora.
\vs p081 6:40 El hombre no debería tener miedo a experimentar con los mecanismos de la sociedad. Pero estas aventuras de adecuación cultural deben siempre dirigirse por aquellos que están plenamente familiarizados con la historia de la evolución social; y estos innovadores deben siempre guiarse por los sabios consejos de los que han tenido experiencia práctica en el ámbito del experimento social o económico previsto. \bibemph{No se debería intentar ningún gran cambio social o económico de forma repentina}. El tiempo es esencial para cualquier tipo de adaptación humana: física, social o económica. Solo los ajustes morales y espirituales pueden hacerse de improviso, e incluso estos requieren del paso del tiempo para el pleno desarrollo de sus repercusiones sociales y materiales. Los ideales de la raza son los principales apoyo y garantía durante los tiempos críticos en los que la civilización está en tránsito de un nivel a otro.
\vs p081 6:41 \li{15.}\bibemph{La prevención de los colapsos transitorios}. La sociedad es producto de era tras era de ensayo y error; es lo que ha sobrevivido de las adaptaciones y readaptaciones selectivas en las sucesivas etapas de la milenaria elevación de la humanidad desde el nivel animal hasta el nivel humano de estatus planetario. El gran peligro de cualquier civilización ---en cualquier momento--- es la amenaza de colapso durante el periodo de transición desde los métodos establecidos del pasado a los procedimientos del futuro, nuevos y mejores, pero sin probar.
\vs p081 6:42 El liderazgo es vital para el progreso. La sabiduría, la percepción y la visión de futuro son indispensables para que las naciones perduren. La civilización nunca está realmente en peligro hasta que comiencen a desaparecer los líderes competentes que la guían. Y la cifra de tal capaz dirección no ha superado nunca el uno por ciento de la población.
\vs p081 6:43 Fue por estos peldaños de la escala evolutiva por los que la evolución ha ascendido hasta ese punto en el que se pudieron iniciar aquellas influencias poderosas que han culminado, en el siglo XX, en la rápida expansión de la cultura. Y solo mediante el cumplimiento de estos factores esenciales puede el hombre esperar mantener su civilización actual, mientras provee por su desarrollo continuo y su segura supervivencia.
\vs p081 6:44 \pc Esta es la esencia de la larguísima lucha de los pueblos de la tierra por establecer la civilización desde los tiempos de Adán. La cultura de hoy día es el resultado final de esta denodada evolución. Antes del descubrimiento de la imprenta, el progreso fue relativamente lento, puesto que las generaciones no podían beneficiarse tan rápidamente de los logros de sus predecesoras. Pero, en la actualidad, la sociedad humana se impulsa hacia adelante bajo la fuerza del ímpetu acumulado de todas las eras a través de las que la civilización ha luchado.
\vsetoff
\vs p081 6:45 [Auspiciado por un arcángel de Nebadón.]
