\upaper{164}{En la fiesta de la Dedicación}
\author{Comisión de seres intermedios}
\vs p164 0:1 Mientras se instalaba el campamento de Pella, Jesús, llevándose con él a Natanael y a Tomás, fueron en secreto a Jerusalén para asistir a la fiesta de la Dedicación. Hasta que no cruzaron el Jordán por el vado de Betania, los dos apóstoles no se dieron cuenta de que su Maestro se dirigía a Jerusalén. Cuando supieron que realmente pretendía estar presente en la fiesta de la Dedicación, protestaron encarecidamente y emplearon todo tipo de argumentos para tratar de disuadirlo. Pero sus esfuerzos fueron en vano; Jesús estaba determinado a ir a Jerusalén. A todas sus súplicas y advertencias, que le hacían hincapié en la locura y en el peligro de ponerse a sí mismo en manos del sanedrín, él solo contestaba: “Quiero dar a estos maestros de Israel otra oportunidad para que vean la luz, antes de que llegue mi hora”.
\vs p164 0:2 Se dirigieron pues hacia Jerusalén, aunque los dos apóstoles continuaron expresándole sus temores y dudas sobre la prudencia de tal supuestamente atrevido acto. Llegaron a Jericó hacia las cuatro y media de la tarde con la idea de alojarse allí por la noche.
\usection{1. LA HISTORIA DEL BUEN SAMARITANO}
\vs p164 1:1 Al anochecer, se congregaron alrededor de Jesús y de los dos apóstoles un considerable grupo de personas para hacerles preguntas. Los apóstoles contestaron a muchas de ellas, mientras que el Maestro lo hizo a las demás. En el trascurso de la noche, un intérprete de la Ley, tratando de enredarlo en alguna situación comprometida, dijo: “Me gustaría preguntarte, maestro, ¿haciendo qué cosa heredaré la vida eterna?”. Jesús contestó: “¿Qué está escrito en la ley y los profetas; cómo lees tú las Escrituras?”. El intérprete de la Ley, conociendo las enseñanzas tanto de Jesús como de los fariseos, respondió: “Amarás al Señor tu Dios con todo tu corazón, con toda tu alma, con toda tu mente y con todas tus fuerzas y a tu prójimo como a ti mismo”. Entonces Jesús le dijo: “Bien has respondido; esto, si realmente lo haces, te conducirá a la vida eterna”.
\vs p164 1:2 Pero, al hacer esa pregunta, el intérprete de la Ley no era del todo sincero y, queriendo justificarse a sí mismo, mientras esperaba también poner a Jesús en una situación embarazosa, se aventuró a hacerle otra pregunta. Acercándose algo más a él, dijo: “Pero, Maestro, me gustaría que me dijeras, ¿quién es concretamente mi prójimo?”. El intérprete de la Ley formuló esta pregunta con la esperanza de tenderle una trampa e inducirlo a hacer algún tipo de afirmación que contraviniera la ley judía, que definía al prójimo como “los hijos de su propio pueblo”. Los judíos consideraban a todos los demás como “perros gentiles”. Este hombre estaba algo familiarizado con las enseñanzas de Jesús y era bien consciente, por tanto, de que el Maestro pensaba de forma diferente; así pues, esperaba incitarlo a decir algo que pudiera interpretarse como un ataque a la ley sagrada.
\vs p164 1:3 Pero Jesús percibió las razones que movían a este hombre y, en lugar de caer en su enredo, procedió a contar a los presentes una historia que cualquiera en Jericó pudiera perfectamente valorar. Jesús dijo: “Un hombre que descendía de Jerusalén a Jericó cayó en manos de bandidos despiadados, los cuales lo despojaron, le robaron, lo hirieron y se fueron dejándolo medio muerto. Enseguida aconteció que descendió un sacerdote por aquel camino y cuando llegó al hombre, viendo su lamentable estado, pasó de largo por el otro lado de la carretera. Y asimismo un levita también, llegando cerca de aquel lugar, al verlo pasó al otro lado. Y luego, sobre esa hora, un samaritano que bajaba a Jericó, llegó junto al hombre herido; y cuando vio cómo le habían robado y cómo lo habían golpeado, fue movido a la compasión y, acercándose a él, vendó sus heridas, echándoles aceite y vino, y, montándolo en su propia cabalgadura, lo trajo aquí, al mesón, y cuidó de él. Y al otro día sacó algún dinero y, dándoselo al mesonero, le dijo: “Cuida bien de mi amigo, y todo lo que gastes de más yo te lo pagaré cuando regrese”. Ahora bien, yo te pregunto: ¿Quién, pues, de estos tres te parece que fue el prójimo del que cayó en manos de los ladrones?”. Cuando el intérprete de la Ley vio que había caído en su propia trampa, respondió: “El que usó misericordia con él”. Y Jesús le dijo: “Ve y haz tú lo mismo”.
\vs p164 1:4 El intérprete de la Ley contestó, “El que usó misericordia con él”, para evitar pronunciar la odiada palabra “samaritano”. El hombre se vio obligado a dar su propia respuesta a la pregunta “¿quién es mi prójimo?”, algo que Jesús deseaba que hiciera, y que, si él se la hubiese dado expresamente, le habría acarreado directamente una acusación de herejía. Jesús no solo desconcertó a aquel desleal intérprete de la Ley, sino que contó a los presentes una historia que era, al mismo tiempo, un hermoso llamamiento a todos sus seguidores y una contundente amonestación dirigida a todos los judíos por su actitud hacia los samaritanos. Y esta historia ha continuado impulsando el amor fraternal entre todos aquellos que han creído después en el evangelio de Jesús.
\usection{2. EN JERUSALÉN}
\vs p164 2:1 Jesús había asistido a la fiesta de los Tabernáculos para poder proclamar el evangelio a los peregrinos de todas partes del imperio; ahora iba a la fiesta de la Dedicación solo con un propósito: conceder al sanedrín y a los líderes judíos otra oportunidad de ver la luz. El acontecimiento más importante ocurrido durante estos pocos días tuvo lugar el viernes por la noche en la casa de Nicodemo. Allí se habían congregado unos veinticinco líderes judíos, creyentes de las enseñanzas de Jesús. En este grupo había catorce hombres que en aquel momento eran o habían sido últimamente miembros del sanedrín. Eber, Matadormo y José de Arimatea asistieron a esta reunión.
\vs p164 2:2 En esta ocasión, quienes habían acudido a oír a Jesús eran todos hombres eruditos, y tanto ellos como los dos apóstoles se sorprendieron de la extensión y la profundidad de los comentarios realizados por el Maestro ante tal distinguido grupo. Desde la época en la que había enseñado en Alejandría, en Roma y en las islas del Mediterráneo, Jesús no había mostrado tal conocimiento ni percepción de los asuntos de los hombres, tanto profanos como religiosos.
\vs p164 2:3 Cuando terminó aquel corto encuentro, todos se fueron maravillados por la persona del Maestro, encantados con sus gentiles maneras y encariñados del hombre. Habían tratado de aconsejar a Jesús en relación con su deseo de ganar para el reino a los restantes miembros del sanedrín. El Maestro escuchó con atención, aunque en silencio, todas sus propuestas. Era bien consciente de que ninguno de esos planes tendría éxito. Llegó a la conclusión de que la mayoría de los líderes judíos jamás aceptaría el evangelio del reino; no obstante, quería darles a todos ellos esta otra oportunidad de elección. Pero cuando salió aquella noche, con Natanael y Tomás, para alojarse en el Monte de los Olivos, aún no había decidido qué método seguir para que el sanedrín prestara atención a su labor.
\vs p164 2:4 Esa noche, Natanael y Tomás durmieron poco; estaban muy sorprendidos por lo que habían oído en la casa de Nicodemo. Estuvieron pensando bastante sobre el último comentario de Jesús en relación a la propuesta de los miembros antiguos y actuales del sanedrín de ir con él ante los setenta miembros de esta corte suprema de la ley judía. El Maestro había dicho: “No, hermanos míos, no serviría de nada. Recaería sobre vuestras cabezas una ira aún mayor, y no lograríais aminorar, en lo más mínimo, el odio que sienten hacia mí. Ocupaos, cada cual, de los asuntos del Padre conforme el espíritu os guíe, mientras yo les hago reconsiderar sobre el reino tal como el Padre me muestre”.
\usection{3. CURACIÓN DEL MENDIGO CIEGO}
\vs p164 3:1 La mañana siguiente, los tres se acercaron a la casa de Marta en Betania para desayunar, y enseguida se dirigieron a Jerusalén. Aquel \bibemph{sabbat} por la mañana, al aproximarse Jesús y sus dos apóstoles al templo, se encontraron con un mendigo bien conocido, ciego de nacimiento, que estaba sentado en su lugar habitual. Aunque estos indigentes no pedían ni recibían limosnas el día del \bibemph{sabbat,} sí se les permitía sentarse en los sitios donde acostumbraban. Jesús se detuvo y miró al mendigo. Al mirar a este hombre, que había nacido ciego, se le ocurrió cómo atraer la atención del sanedrín y de los demás líderes judíos y maestros religiosos de nuevo hacia su misión en la tierra.
\vs p164 3:2 Mientras el Maestro estaba de pie delante del ciego, profundamente absorto en sus pensamientos, Natanael, reflexionando sobre la posible causa de la ceguera de aquel hombre, preguntó: “Maestro, ¿quién pecó, este o sus padres, para que haya nacido ciego?”.
\vs p164 3:3 \pc Los rabinos enseñaban que el pecado era el causante de todos los casos de ceguera de nacimiento. No solo se concebían y nacían los niños en pecado, sino que un niño podía nacer ciego como castigo por algún pecado cometido por su padre. Enseñaban incluso que el mismo niño podría pecar antes de venir al mundo y que, además, algún pecado o exceso de la madre estando embarazada podría producir estos defectos.
\vs p164 3:4 En todas estas regiones se continuaba creyendo en la reencarnación. Los antiguos maestros judíos, junto con Platón, Filón y muchos de los esenios, toleraban la teoría de que el hombre podía cosechar en una encarnación lo que habían sembrado en alguna existencia previa; asimismo se pensaba que durante la vida se expiaban los pecados cometidos en vidas anteriores. Al Maestro le resultaba difícil hacer creer a los hombres que sus almas no habían tenido existencias previas.
\vs p164 3:5 Sin embargo, por incongruente que parezca, aunque se suponía que esta ceguera era resultado del pecado, los judíos consideraban que era altamente encomiable dar limosnas a los mendigos ciegos. Estos ciegos solían cantar constantemente a los viandantes: “Oh, corazones tiernos, haced mérito ayudando al ciego”.
\vs p164 3:6 \pc Jesús decidió comentar con Natanael y Tomás este caso del ciego, no solamente porque había decidido usarlo aquel día como medio para hacer que su misión fuese más prominente a la atención de los líderes judíos, sino también porque siempre animaba a sus apóstoles a que buscaran las verdaderas causas de cualquier fenómeno, ya fuese natural o espiritual. Con frecuencia, les había advertido que evitaran la tendencia común de atribuir causas espirituales a los acontecimientos físicos ordinarios.
\vs p164 3:7 Jesús estaba determinado a valerse de ese mendigo ciego, llamado Josías, para la labor que planeaba llevar a cabo ese día, pero antes de hacer nada por él, procedió a responder a la pregunta de Natanael. El Maestro dijo: “No es que pecó este, ni sus padres, para que las obras de Dios se manifiesten en él. Le sobrevino la ceguera debido al curso natural de los acontecimientos, pero es necesario que hagamos ahora las obras de Quien me envió, mientras dura el día, porque la noche vendrá y nos será imposible hacer lo que estamos a punto de realizar. Cuando yo estoy en el mundo, luz soy del mundo, pero en poco tiempo ya no estaré con vosotros”.
\vs p164 3:8 Cuando Jesús acabó de hablar, dijo a Natanael y Tomás: “Creemos la visión para este ciego en este día de \bibemph{sabbat,} para que los escribas y los fariseos tengan la gran ocasión que buscan de acusar al Hijo del Hombre”. Dicho esto, inclinándose, escupió en tierra y mezcló esta con la saliva y, hablando de todo esto para que el ciego lo oyera, se acercó a Josías y le puso el lodo sobre sus ojos ciegos, diciendo: “Hijo mío, ve a lavarte este lodo en el estanque de Siloé y de inmediato recibirás la visión”. Y cuando Josías se lavó en el estanque de Siloé, regresó a sus amigos y familia, ya viendo.
\vs p164 3:9 Como siempre había sido mendigo, no sabía hacer otra cosa; así que, una vez pasados los primeros momentos de emoción, tras recibir la vista, volvió al sitio en el que solía mendigar limosnas. Sus amigos, sus vecinos y todos los que lo habían conocido anteriormente, al notar que veía, decían: “¿No es este Josías el mendigo ciego?”. Algunos afirmaban que era él, mientras que otros comentaban: “No, se parece a él, pero este hombre puede ver”. Si bien, cuando le preguntaban a él mismo, respondía: “Yo soy”.
\vs p164 3:10 Cuando querían saber cómo era que podía ver, él les respondía: “Un hombre llamado Jesús pasó por aquí, y cuando hablaba de mí con sus amigos, hizo lodo con su saliva, me lo untó en los ojos y mandó que me los lavara en el estanque de Siloé. Hice lo que me dijo e inmediatamente recibí la vista. Y eso fue hace solo algunas horas. Todavía no conozco el significado de mucho de lo que veo”. Cuando la gente, que empezaba a acumularse a su alrededor, inquiría dónde podrían encontrar al extraño hombre que lo había sanado, Josías solo contestaba que no lo sabía.
\vs p164 3:11 \pc Se trata de uno de los milagros más extraños de todos los realizados por el Maestro. Este ciego no pidió que lo curaran. No sabía que el Jesús que lo había mandado a Siloé a lavarse los ojos, y que le había prometido la visión, era el mismo profeta de Galilea que había predicado en Jerusalén durante la fiesta de los Tabernáculos. Este hombre tenía poca fe en recibir la vista, pero la gente de aquellos días tenía una gran fe en la eficacia de la saliva de un gran hombre o de un santo; y de la conversación que Jesús había mantenido con Natanael y Tomás, Josías había llegado a la conclusión de que quien iba a ser su benefactor era un gran hombre, un docto maestro o un santo profeta; por ello, hizo lo que Jesús le pidió.
\vs p164 3:12 Jesús recurrió a la tierra y la saliva y le ordenó que se lavara en el estanque simbólico de Siloé por tres razones:
\vs p164 3:13 \li{1.}No se trató de un milagro realizado en respuesta a la fe de una persona. Jesús tenía sus propios propósitos para llevar a cabo este prodigio, pero lo dispuso de manera que este hombre pudiera obtener de él algún beneficio perdurable.
\vs p164 3:14 \li{2.}Como el ciego no había pedido ser curado, y puesto que su fe era débil, se realizaron estos actos materiales con el fin de alentarlo. Él ciertamente creía en la superstición de la eficacia de la saliva, y sabía que el estanque de Siloé era un lugar semisagrado. Pero no habría ido allí de no haber sido necesario lavar el lodo de su unción. Había suficiente ceremonial en esta acción como para inducirle a hacer algo.
\vs p164 3:15 \li{3.}Pero Jesús tenía una tercera razón para recurrir a estos medios materiales en cuanto a este extraordinario acto: obró este milagro estrictamente en cumplimiento de su propia decisión y con el deseo de enseñar a sus seguidores de ese día y de todas las eras por venir que se abstuvieran de despreciar o ignorar los recursos materiales en la sanación de los enfermos. Quería enseñarles que debían dejar de considerar los milagros como la única manera de curar las enfermedades humanas.
\vs p164 3:16 \pc Jesús devolvió la visión a este hombre realizando un milagro aquel \bibemph{sabbat} por la mañana y en Jerusalén, cerca del templo, con el principal propósito de retar públicamente con este acto al sanedrín y a todos los maestros judíos y líderes religiosos. Era su manera de romper abiertamente con los fariseos. Siempre fue positivo en todo lo que hizo. Y fue con el fin de presentar estas cuestiones ante el sanedrín por lo que Jesús había llevado a sus dos apóstoles junto a este mendigo temprano por la tarde de ese día del \bibemph{sabbat,} y provocó deliberadamente aquellos comentarios que obligaron a los fariseos a prestar atención al milagro.
\usection{4. JOSÍAS ANTE EL SANEDRÍN}
\vs p164 4:1 A media tarde, la curación de Josías había originado tales desavenencias alrededor del templo que los líderes del sanedrín optaron por convocar el consejo en su lugar habitual de reunión en el templo. Y lo hicieron contraviniendo una norma permanente que prohibía la reunión del sanedrín el día del \bibemph{sabbat}. Jesús sabía que el quebrantamiento del \bibemph{sabbat} sería uno de los cargos principales que se le imputarían contra él cuando le llegara el juicio final, y quería que lo llevaran ante el sanedrín para ser juzgado por la acusación de haber sanado a un ciego ese mismo día del \bibemph{sabbat,} cuando en el periodo de sesiones del alto tribunal judío, que le juzgaba por este acto de misericordia, deliberaba sobre estas cuestiones también el día del \bibemph{sabbat,} en flagrante violación de una leyes impuestas por ellos mismos.
\vs p164 4:2 Pero no requirieron a Jesús ante ellos; temían hacerlo. En cambio, sí enviaron de inmediato a por Josías. Tras algunas preguntas previas, el portavoz del sanedrín (en presencia de unos cincuenta miembros) mandó a Josías que les contara lo que le había sucedido. Desde su curación esa mañana, Josías se había enterado por Tomás, Natanael y otros que los fariseos estaban furiosos de que su curación hubiese ocurrido en \bibemph{sabbat} y eso, con toda probabilidad, traería problemas a todos los que se habían visto implicados en ella; pero Josías aún no se daba cuenta de que Jesús era aquel a quien llamaban el Libertador. Así pues, cuando los fariseos lo interrogaron, dijo: “Este hombre llegó, puso lodo sobre mis ojos, me dijo que me lavara en Siloé, y ahora realmente veo”.
\vs p164 4:3 Uno de los fariseos más ancianos, después de pronunciar un largo discurso, dijo: “Ese hombre no procede de Dios, porque veis que no guarda el \bibemph{sabbat}. Infringe la ley primeramente al hacer el lodo, después al enviar a este mendigo a lavarse en Siloé el día del \bibemph{sabbat}. Tal hombre no puede ser un maestro enviado por Dios”.
\vs p164 4:4 Entonces uno de los más jóvenes, que creía secretamente en Jesús, dijo: “Si Dios no lo envió, ¿cómo puede hacer estas cosas? Sabemos que un pecador ordinario no puede obrar estos milagros. Todos conocemos a este mendigo y sabemos que nació ciego, y que ahora ve. ¿Aún decís que este profeta realiza estos prodigios por el poder del príncipe de los diablos?”. Y por cada fariseo que se atrevía a acusar y denunciar a Jesús, había otro que se levantaba y planteaba preguntas complicadas y embarazosas, por lo que se suscitó una gran división entre ellos. El presidente vio hacia dónde estaban derivando y, a fin de calmar el debate, se dispuso él mismo a interrogar de nuevo a aquel hombre. Volviéndose a Josías, le dijo: “¿Qué tienes tú que decir de este hombre, de este Jesús, que según afirmas te abrió los ojos?”. Y Josías respondió: “Creo que es un profeta”.
\vs p164 4:5 Los líderes estaban muy alterados y, no sabiendo qué más hacer, decidieron llamar a los padres de Josías para saber si realmente había nacido ciego. Se resistían a creer que aquel mendigo se hubiese curado.
\vs p164 4:6 Era bien conocido en Jerusalén no solo que se le prohibía la entrada a Jesús en todas las sinagogas, sino también que a todos los que creían en sus enseñanzas se les había expulsados de ellas, excomulgados de la congregación de Israel; y esto significaba que se le denegaban todos los derechos y privilegios de cualquier tipo en todo el pueblo judío, salvo el derecho de adquirir las cosas necesarias para la vida.
\vs p164 4:7 Por lo tanto, cuando los padres de Josías, almas pobres y temerosas, aparecieron ante el augusto sanedrín, tenían miedo de hablar con libertad. El portavoz del tribunal dijo: “¿Es este vuestro hijo? y ¿nació ciego si entendemos bien? Si eso es verdad, ¿cómo pues ve ahora?”. Y entonces el padre de Josías, respaldado por la madre, contestó: “Sabemos que este es nuestro hijo y que nació ciego; pero cómo es que ve ahora, no lo sabemos, o quién le haya abierto los ojos, nosotros tampoco lo sabemos; edad tiene, preguntadle a él; él hablará por sí mismo”.
\vs p164 4:8 Entonces, llamaron a Josías ante ellos una segunda vez. Sus planes de celebrar un juicio oficial no iban bien, y algunos empezaban a sentirse incómodos por estar haciendo aquello el día del \bibemph{sabbat;} por consiguiente, cuando volvieron a llamar a Josías, intentaron enredarlo atacándole de forma diferente. El oficial del tribunal le habló al antiguo ciego, diciéndole: “¿Por qué no le das gloria a Dios por esto? ¿Por qué no nos dices toda la verdad sobre lo que pasó? Nosotros sabemos que ese hombre es pecador. ¿Por qué te niegas a ver la verdad? Sabes que tanto tú como este hombre sois culpables de quebrantar la ley del \bibemph{sabbat}. ¿Por qué no expías tu pecado, admitiendo que es Dios quien te sanó, si insistes en afirmar que tus ojos se abrieron en este día?”.
\vs p164 4:9 Pero Josías no era torpe ni falto de humor; así que respondió al oficial del tribunal: “Si es pecador, no lo sé; una cosa sé, que habiendo yo sido ciego, ahora veo”. Y como no podían atraparlo, volvieron a preguntarle: “¿Cómo te abrió los ojos?¿ ¿Qué te hizo realmente? ¿Qué te dijo? ¿Te pidió que creyeras en él?”.
\vs p164 4:10 Josías respondió con cierta impaciencia: “Os he dicho exactamente como sucedió todo, y si no creísteis en mi testimonio, ¿por qué lo queréis oír otra vez? ¿Queréis también vosotros haceros sus discípulos?”. Cuando Josías habló estas cosas, se creó una gran confusión, llegando casi a la violencia. Los líderes se abalanzaron sobre Josías exclamando airadamente: “Tú puedes decir que eres su discípulo, pero nosotros, somos discípulos de Moisés y somos los maestros de las leyes de Dios. Nosotros sabemos que Dios habló a través de Moisés, pero, respecto a ese, no sabemos de dónde ha salido”.
\vs p164 4:11 Entonces Josías, de pie sobre un banco, gritó en voz alta a todos los que podían oír, diciendo: “Escuchad, vosotros que clamáis ser maestros de todo Israel, mientras yo os declaro que esto es lo maravilloso, que vosotros confeséis que no sepáis de dónde ha salido este hombre, y sin embargo sabéis con seguridad, por el testimonio que habéis oído, que a mí me abrió los ojos. Todos sabemos que Dios no hace tales obras para los impíos; que Dios únicamente haría tal cosa a petición de un creyente auténtico: de quien es santo y recto. Sabéis que desde el principio del mundo no se ha oído que alguno abriera los ojos a uno que nació ciego. ¡Miradme pues, todos vosotros, y daos cuenta de lo que se ha hecho en este día en Jerusalén! Yo os digo, que si este hombre no viniera de Dios, no podría hacer esto”. Y conforme el sanedrín se alejaba en ira y confusión, le gritaron: “Naciste totalmente en pecado, ¿y ahora te atreves a enseñarnos a nosotros? Quizá no naciste realmente ciego, y aunque tus ojos se abrieron en este día de \bibemph{sabbat,} eso se hizo por el poder de los diablos”. Y fueron de inmediato a la sinagoga a expulsar a Josías.
\vs p164 4:12 Josías llegó a este juicio con vagas ideas sobre Jesús y la naturaleza de su curación. Gran parte de este atrevido testimonio que tanta inteligencia y arrojo expuso ante el tribunal supremo de todo Israel se formó en su mente conforme aquel juicio se desarrollaba de forma irregular e injusta.
\usection{5. ENSEÑANZA EN EL PÓRTICO DE SALOMÓN}
\vs p164 5:1 Mientras esta sesión del sanedrín, que, en quebrantamiento del \bibemph{sabbat,} seguía su curso en una de las cámaras del templo, Jesús caminaba muy cerca de allí, enseñando a la gente en el pórtico de Salomón, esperando a ser citado ante el sanedrín y poder así hablarles de la buena nueva de la libertad y del gozo de la filiación divina en el reino de Dios. Pero temían enviar a buscarlo. Siempre les desconcertaban estas apariciones repentinas y públicas de Jesús en Jerusalén. Jesús les daba ahora la oportunidad que con tanto empeño habían buscado, pero tenían miedo de llamarlo ante el sanedrín incluso como testigo, y mucho más de arrestarlo.
\vs p164 5:2 Era pleno invierno en Jerusalén, y la gente se medio cobijaba en el pórtico de Salomón; y mientras Jesús esperaba, las multitudes le hicieron muchas preguntas y él les enseñó durante más de dos horas. Algunos de los maestros judíos trataron de atraparlo preguntándole públicamente: “¿Hasta cuándo nos tendrás en suspenso? Si tú eres el Mesías, dínoslo abiertamente”. Jesús les dijo: “Os he hablado muchas veces de mí y de mi Padre, pero no queréis creerme. ¿Es que no podéis ver que las obras que yo hago en nombre de mi Padre, ellas dan testimonio de mí? Pero muchos de vosotros no creéis porque no sois de mi rebaño. El maestro de la verdad atrae solamente a quienes tienen hambre de verdad y sed de rectitud. Mis ovejas oyen mi voz y yo las conozco y me siguen. A todos los que siguen mis enseñanzas, yo les doy vida eterna y no perecerán jamás, ni nadie los arrebatará de mi mano. Mi Padre, que me dio a estos hijos, mayor que todos es, y nadie los puede arrebatar de la mano de mi Padre. El Padre y yo somos uno”. Algunos judíos incrédulos fueron corriendo hasta donde aún se estaba construyendo el edificio del templo y tomaron piedras para arrojárselas a Jesús, pero los creyentes los contuvieron.
\vs p164 5:3 Jesús continuó sus enseñanzas: “Muchas obras de amor os he mostrado de mi Padre, y os pregunto ahora: ¿Por cuál de esas buenas obras pensáis apedrearme?”. Entonces respondió uno de los fariseos: “Por buena obra no te apedreamos, sino por la blasfemia, porque tú, siendo hombre, te atreves a hacerte Dios”. Y Jesús respondió: “Acusas al Hijo del Hombre de blasfemia porque te niegas a creer en mí cuando yo declaro que Dios me ha enviado. Si no hago las obras de Dios, no me creas, pero si las hago, aunque no me creas a mí, cree al menos en las obras. Pero para que estés seguro de lo que proclamo, te afirmo de nuevo que el Padre es en mí y yo en el Padre y que, así como el Padre vive en mí, así viviré yo en quien cree este evangelio”. Y cuando la gente oyó estas palabras, muchos de entre ellos se apresuraron a buscar piedras para arrojárselas, pero él se fue por los recintos del templo; y encontrándose con Natanael y Tomás, que habían asistido a la sesión del sanedrín, esperó con ellos cerca del templo hasta que saliese Josías de la cámara del consejo.
\vs p164 5:4 Jesús y los dos apóstoles no fueron a buscar a Josías a su casa hasta que oyeron que lo habían expulsado de la sinagoga. Cuando llegaron allí, Tomás lo llamó al patio, y Jesús, hablándole, dijo: “Josías, ¿crees tú en el Hijo de Dios?”. Y Josías respondió: “Dime quién es, para que crea en él”. Y Jesús le dijo: “Lo has visto y lo has oído; el que habla ahora contigo es”. Y Josías dijo: “Creo, Señor”, y postrándose, lo adoró.
\vs p164 5:5 Cuando Josías supo que había sido expulsado de la sinagoga, en un principio se sintió muy abatido, pero se animó mucho cuando Jesús le pidió que se preparara de inmediato para ir con ellos al campamento de Pella. Ciertamente, se había expulsado de la sinagoga judía a este sencillo hombre de Jerusalén, pero he aquí que el creador de un universo lo guió para que formara parte de la nobleza espiritual de ese día y generación.
\vs p164 5:6 Y, entonces, Jesús salió de Jerusalén, para no regresar nuevamente allí hasta que se acercara el momento de disponerse a dejar este mundo. El Maestro regresó a Pella con los dos apóstoles y con Josías. Este hombre demostró ser uno de los fructíferos destinatarios de la labor milagrosa del Maestro, porque se convirtió de por vida en predicador del evangelio del reino.
