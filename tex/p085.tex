\upaper{85}{Orígenes de la adoración}
\author{Brillante estrella vespertina}
\vs p085 0:1 La religión primitiva tuvo un origen biológico, un desarrollo evolutivo natural, al margen de connotaciones morales y de cualquier influencia espiritual. Los animales superiores tienen miedos, pero no ilusiones, y, por consiguiente, ninguna religión. El hombre crea sus religiones primitivas a partir de sus miedos e ilusiones.
\vs p085 0:2 En la evolución de la especie humana, la adoración en sus manifestaciones primitivas aparece mucho antes de que la mente del hombre esté en condiciones de elaborar los conceptos más complejos sobre la vida de ahora y del más allá que merezcan llamarse religión. La naturaleza de la religión primitiva era por completo intelectual y se fundamentaba enteramente en circunstancias asociativas. Los objetos de adoración eran en su totalidad los que les resultaban sugerentes; consistían en las cosas de la naturaleza que estaban a su alcance, o que tenían una gran importancia en la experiencia corriente de los ingenuos primitivos urantianos.
\vs p085 0:3 Cuando la religión evolucionó más allá de la adoración a la naturaleza, se enraizó su origen espiritual, pero siguió, no obstante, estando siempre condicionada por el entorno social. Conforme se desarrolló esta forma de adoración, el hombre conceptualizó una división de labores en el mundo supramortal; había espíritus de la naturaleza para los lagos, los árboles, las cataratas, la lluvia y para centenares de otros fenómenos terrestres ordinarios.
\vs p085 0:4 En un momento o en otro, el hombre mortal ha adorado todo lo que ha encontrado sobre la faz de la tierra, inclusive a sí mismo. También ha adorado todo lo imaginable en el cielo y bajo la superficie de la tierra. El hombre primitivo temía todas las manifestaciones de poder; rendía culto a cualquier fenómeno natural que no pudiese comprender. La observación de las fuerzas poderosas de la naturaleza como las tormentas, las inundaciones, los terremotos, los aludes, los volcanes, el fuego, el calor y el frío impresionaban enormemente a la mente humana a medida que se iba desarrollando. Las cosas inexplicables de la vida todavía se califican como “actos de Dios” y “dispensaciones misteriosas de la Providencia”.
\usection{1. ADORACIÓN DE LAS PIEDRAS Y LAS COLINAS}
\vs p085 1:1 La piedra fue lo primero que el hombre en evolución adoró. Incluso en la actualidad los kateríes del sur de la India rinden culto a una piedra, al igual que numerosas tribus del norte de este continente. Jacob durmió sobre una piedra porque la veneraba; incluso llegó a ungirla. Raquel ocultaba un cierto número de piedras sagradas en su tienda.
\vs p085 1:2 Las piedras causaban impresión al hombre primitivo como si fuesen algo extraordinario debido, primeramente, a la repentina manera en la que aparecían sobre la superficie de un campo cultivado o pastizal. Los hombres no tenían en cuenta ni la erosión ni el resultado de remover la tierra. Las piedras también impresionaban profundamente a los pueblos primitivos por su frecuente parecido con los animales. Las numerosas formaciones rocosas de las montañas, que tanto se asemejan a las caras de animales e incluso a la de los hombres, captaban la atención del hombre civilizado. Pero la mayor influencia la ejercían las piedras meteóricas, grandiosamente llameantes, que los humanos primitivos contemplaban cruzando la atmósfera a toda velocidad. Las estrellas fugaces maravillaban al hombre primitivo, que creía con facilidad que tales abrasadoras rayas señalaban el paso de un espíritu en su camino a la tierra. No era de extrañar que los hombres se sintieran atraídos a la adoración de dichos fenómenos, especialmente una vez que descubrieron los meteoros. Y esto llevó a una mayor reverencia hacia todas las otras piedras. En Bengala, hay muchas personas que rinden culto a un meteoro que cayó a la tierra en el año 1880 d. C.
\vs p085 1:3 Todos los clanes y tribus antiguos tenían sus piedras sagradas, y la mayoría de los pueblos modernos manifiestan cierto grado de veneración por algunos tipos de piedra ---sus joyas---. En la India, se reverenciaba un grupo de cinco piedras; en Grecia, un conjunto de treinta; entre los hombres rojos se veneraba normalmente un círculo de piedras. Los romanos siempre arrojaban una piedra al aire cuando invocaban a Júpiter. En la India, incluso hoy en día, se puede usar una piedra como testigo. En algunas regiones, la piedra puede emplearse como talismán de la ley y, por su influencia, un delincuente puede ser forzado a comparecer ante un tribunal. Pero los simples mortales no siempre identifican la Deidad con un objeto al que se le rinde ceremonia. Tales fetiches son muchas veces meros símbolos del verdadero objeto de adoración.
\vs p085 1:4 Los antiguos tenían un curioso respeto por los agujeros en las piedras. Se suponía que estas rocas porosas eran extraordinariamente eficaces para curar enfermedades. No se perforaban las orejas para portar piedras, aunque sí se colocaban estas para mantener los orificios de las orejas abiertos. Incluso en los tiempos modernos, las personas supersticiosas hacen agujeros en las monedas. En África, las piedras fetiches despiertan un gran entusiasmo entre los nativos. De hecho, todas las tribus y pueblos atrasados aún sienten una veneración supersticiosa hacia ellas. Incluso en la actualidad, el culto a las piedras está ampliamente generalizado por todo el mundo. La lápida es un símbolo que subsiste a partir de las imágenes y de los ídolos que se esculpían en la piedra, y que estaban relacionados con las creencias en los espectros y en los espíritus de sus congéneres fallecidos.
\vs p085 1:5 La adoración de las colinas siguió a la de las piedras, y las primeras colinas que se veneraron eran grandes formaciones rocosas. Pronto se convirtió en costumbre creer que los dioses habitaban las montañas, de manera que los lugares altos de tierra se adoraban por esta razón añadida. Con el paso del tiempo, se relacionaban algunas montañas con ciertos dioses y, por lo tanto, se volvieron sagradas. Los ignorantes y supersticiosos aborígenes creían que las cuevas conducían al inframundo, con sus espíritus y demonios malignos, en contraste con las montañas, que se identificaban con los conceptos, desarrollados más tarde, de las deidades y espíritus buenos.
\usection{2. ADORACIÓN DE LAS PLANTAS Y LOS ÁRBOLES}
\vs p085 2:1 En un primer momento, se temió a las plantas, pero después se las adoró debido a las bebidas embriagantes que se obtenían de ellas. El hombre primitivo creía que la embriaguez lo volvía divino. Se suponía que había algo de insólito y sagrado en tal experiencia. Incluso en los tiempos modernos, las bebidas alcohólicas se conocen como “espirituosas”.
\vs p085 2:2 El hombre primitivo miraba con temor y respeto supersticioso a la semilla cuando germinaba. El apóstol Pablo no fue el primero en extraer de esta imagen profundas lecciones espirituales y basarse en ella para predicar creencias religiosas.
\vs p085 2:3 El culto de adoración a los árboles está entre los ritos religiosos más antiguos. Todas las bodas primitivas se celebraban bajo los árboles, y cuando las mujeres deseaban hijos, se las encontraba a veces en el bosque abrazando afectivamente a un robusto roble. Se veneraban muchas plantas y árboles debido a sus poderes medicinales reales o imaginarios. El salvaje creía que todas sus propiedades químicas se debían a la actividad directa de las fuerzas sobrenaturales.
\vs p085 2:4 Las ideas sobre los espíritus de los árboles variaban considerablemente entre las distintas tribus y razas. Algunos árboles estaban habitados por espíritus bondadosos; otros albergaban espíritus falaces y crueles. Los finlandeses creían que la mayoría de los árboles estaban poblados por espíritus buenos. Por mucho tiempo, los suizos desconfiaron de los árboles, creyendo que contenían espíritus engañosos. Los habitantes de la India y de Rusia oriental creen que los espíritus de los árboles son crueles. Los patagones aún adoran a los árboles, tal como lo hicieron los primitivos semitas. Mucho después de que los hebreos dejaran de adorar a los árboles, continuaron venerando a sus distintas deidades en las arboledas. Excepto en la China, alguna vez existió un culto generalizado del \bibemph{árbol de la vida}.
\vs p085 2:5 La creencia de que el agua o los metales preciosos que se hallan bajo la superficie de la tierra pueden detectarse por medio de varilla adivinatoria de madera es una reliquia de los antiguos cultos a los árboles. El mayo, el árbol de navidad y la práctica supersticiosa de tocar madera son una perpetuación de algunas de las antiguas costumbres de adoración de los árboles.
\vs p085 2:6 Muchas de estas primeras formas de veneración de la naturaleza se mezclaron con los métodos de adoración que evolucionarían más tarde, pero la adoración más antigua, activada por el asistente de la mente, estaba obrando mucho antes de que el más reciente despertar de la naturaleza religiosa de la humanidad fuese plenamente receptiva al estímulo de las influencias espirituales.
\usection{3. ADORACIÓN DE LOS ANIMALES}
\vs p085 3:1 El hombre primitivo tenía un sentimiento peculiar y de solidaridad hacia los animales superiores. Sus ancestros habían vivido e incluso se habían apareado con ellos. En el sur de Asia, se creyó pronto que las almas de los hombres regresaban a la tierra en forma de animal. Esta creencia era una continuación de la práctica, todavía más antigua, de adorar a los animales.
\vs p085 3:2 Los hombres primitivos reverenciaban a los animales por su fuerza y astucia. Creían que el agudo olfato y la vista de gran alcance de ciertas criaturas eran muestras de su guía espiritual. En algún momento u otro y por alguna raza u otra, se ha adorado a los animales. Entre tales elementos de culto, había seres que se consideraban mitad humano y mitad animal, tales como los centauros y las sirenas.
\vs p085 3:3 Los hebreos adoraron a las serpientes hasta los días del rey Ezequías, y los hindúes aún conservan vínculos de amistad con sus serpientes domésticas. La adoración de los chinos por el dragón es un vestigio del \bibemph{culto de amor a la serpiente}. La sabiduría de la serpiente era el símbolo de la medicina griega y todavía se emplea como emblema por los médicos modernos. El arte del encantamiento de la serpiente se ha transmitido desde los días del culto de amor a la serpiente de las chamanes femeninas, las cuales, como resultado de las mordeduras diarias de las serpientes, se volvían inmunes, de hecho, se convertían en auténticas adictas al veneno y no podían vivir sin tal ponzoña.
\vs p085 3:4 La adoración de los insectos y de otros animales se promovió por una interpretación posterior de la regla de oro ---trata a los demás (a cualquier forma de vida) como querrías que te trataran a ti---. Alguna vez, los antiguos creyeron que las alas de los pájaros producían los vientos, por lo que temían y adoraban a la vez a todas las criaturas aladas. Los nórdicos primitivos pensaban que los eclipses se debían a un lobo que devoraba una parte del sol o de la luna. A menudo, los hindúes muestran a Visnú con cabeza de caballo. Muchas veces, un símbolo animal representa un dios olvidado o un culto desaparecido. Pronto, en la religión evolutiva, el cordero se volvió el típico animal para el sacrificio y, la paloma, el símbolo de la paz y del amor.
\vs p085 3:5 En la religión, el simbolismo puede ser bueno o malo justo en la medida en la que el símbolo desplace o no la idea originaria de adoración. Y el simbolismo no se debe confundir con la idolatría explícita, en la que se adora al objeto material de forma directa y real.
\usection{4. ADORACIÓN DE LOS ELEMENTOS NATURALES}
\vs p085 4:1 La humanidad ha rendido culto a la tierra, al aire, al agua y al fuego. Las razas primitivas veneraban los manantiales y adoraban los ríos. Incluso en la actualidad, en Mongolia, prospera un influyente culto al río. En Babilonia, el bautismo se convirtió en una ceremonia religiosa, y los griegos practicaban el baño ritual anual. Era fácil para los antiguos imaginar que los espíritus vivían en los manantiales burbujeantes, en las fuentes que brotaban, en los ríos que fluían y en los tumultuosos torrentes. El movimiento de las aguas impresionaba intensamente a estas simples mentes que creían en la vivacidad de los espíritus y en los poderes sobrenaturales. A veces, se le negaba el socorro a una persona que se ahogaba por temor a ofender a algún dios del río.
\vs p085 4:2 Hay numerosas cosas y acontecimientos que han obrado como estímulos religiosos para distintos pueblos en diferentes eras. Todavía se adora el arcoíris en muchas de las tribus montañesas de la India. Tanto en la India como en África el arco iris se considera una gigantesca serpiente celestial; los hebreos y los cristianos lo consideran un “arco de promesa”. Asimismo, las influencias, que en alguna parte del mundo se consideran beneficiosas, se pueden ver como malévolas en otras regiones. El viento del este es un dios en América del Sur porque trae lluvias; en la India, es un demonio porque trae polvo y es causa de sequías. Los antiguos beduinos creían que un espíritu de la naturaleza producía los remolinos de arena e, incluso en tiempos de Moisés, la creencia en los espíritus de la naturaleza era lo suficientemente fuerte como para que se perpetuasen en la teología hebrea bajo la forma de ángeles del fuego, del agua y del aire.
\vs p085 4:3 Las nubes, la lluvia y el granizo se han temido y adorado por parte de numerosas tribus primitivas y en muchos de los primeros cultos a la naturaleza. Las tempestades con truenos y relámpagos intimidaban al hombre primitivo. Tanto le impresionaban estas perturbaciones de los elementos que se consideraba el trueno como la voz de un dios airado. La adoración del fuego y el temor a los relámpagos estaban relacionados entre sí y muy extendidos entre muchos grupos primitivos.
\vs p085 4:4 Fuego y magia se entremezclaban en la mente de los atemorizados mortales primitivos. Un seguidor de la magia recordará vívidamente un casual resultado positivo en la práctica de sus fórmulas mágicas, mientras que despreocupadamente olvidará una veintena de resultados negativos, de absolutos fracasos. La reverencia al fuego alcanzó su punto culminante en Persia, donde persistió durante mucho tiempo. Algunas tribus adoraban el fuego como una deidad por sí misma; otros lo reverenciaban como símbolo llameante del espíritu purificante y depurador de sus veneradas deidades. Las vírgenes vestales tenían la obligación de vigilar los fuegos sagrados y, en el siglo XX, aún se encienden velas como parte del ritual de muchos servicios religiosos.
\usection{5. ADORACIÓN DE LOS CUERPOS CELESTES}
\vs p085 5:1 La adoración de rocas, colinas, árboles y animales evolucionó de manera natural a través de la veneración nacida del temor a los elementos naturales hasta la deificación del sol, la luna y las estrellas. En la India y en otros lugares, se pensaba que las estrellas eran las almas glorificadas de los grandes hombres que habían abandonado la vida en la carne. Los adoradores caldeos de las estrellas se consideraban a sí mismos hijos del padre cielo y de la madre tierra.
\vs p085 5:2 La adoración de la luna precedió a la del sol. La veneración de la luna llegó a su apogeo durante la era de la caza, mientras que la adoración del sol se convirtió en la principal ceremonia religiosa en las épocas agrícolas que siguieron. La adoración del sol tuvo por primera vez un gran arraigo en la India, y allí subsistió durante un mayor periodo de tiempo. En Persia, la veneración del sol dio origen al posterior culto mitraico. Entre muchos pueblos, el sol se consideraba el ancestro de sus reyes. Los caldeos colocaban al sol en el centro de “los siete círculos del universo”. Las civilizaciones que vendrían después rindieron homenaje al sol dándole su nombre al primer día de la semana.
\vs p085 5:3 Se suponía que el dios Sol era el padre místico de los hijos de destino nacidos de vírgenes que, según se pensaba, se concedían en ocasiones a las razas privilegiadas como salvadores. Estos niños sobrenaturales siempre se dejaban a la deriva en algún río sagrado para ser rescatados de alguna forma extraordinaria, tras lo cual crecían hasta convertirse en personas milagrosas y libertadores de sus pueblos.
\usection{6. ADORACIÓN DEL HOMBRE}
\vs p085 6:1 Habiendo adorado todo lo que se hallaba sobre la faz de la tierra y arriba en los cielos, el hombre no vaciló en rendirse homenaje a sí mismo adorándose igualmente. El salvaje de mente sencilla no hace una distinción clara entre animales, hombres y dioses.
\vs p085 6:2 El hombre primitivo creía que todas las personas poco comunes eran sobrehumanas, y sentía hacia ellos un temor reverencial; en cierta medida, los adoraba literalmente. Incluso el tener gemelos se consideraba o una gran suerte o muy mala suerte. Los lunáticos, los epilépticos y los discapacitados mentales se adoraban por sus semejantes de mente normal, que creían que seres con alguna anomalía así estaban habitados por los dioses. Se rendía culto a los sacerdotes, a los reyes y a los profetas; se pensaba que los hombres santos de la antigüedad estaban inspirados por las deidades.
\vs p085 6:3 Los jefes tribales morían y se les \bibemph{deificaba}. Más tarde, \bibemph{se santificaba} a las almas distinguidas que fallecían. La evolución no asistida nunca dio origen a dioses que fuesen superiores a los espíritus glorificados, excelsos y desarrollados de los humanos muertos. En la evolución temprana, la religión crea sus propios dioses. En el curso de la revelación, los Dioses elaboran la religión. La religión evolutiva crea sus dioses a imagen y semejanza del hombre mortal; la religión revelada procura que el hombre mortal se desarrolle y transforme a imagen y semejanza de Dios.
\vs p085 6:4 Los dioses espectros, que se suponían eran de origen humano, se deben distinguir de los dioses de la naturaleza, porque la adoración de la naturaleza trajo consigo un panteón ---espíritus de la naturaleza elevados a la posición de dioses---. El culto a la naturaleza continuó desarrollándose a la par del culto a los espectros que apareció después, y cada cual ejerció su influencia sobre el otro. Muchos sistemas religiosos adoptaron una doble noción de la deidad: dioses de la naturaleza y dioses espectros. En algunas teologías, estos conceptos se entrelazan de forma confusa, como queda demostrado en Thor, un héroe espectro que también era el señor del relámpago.
\vs p085 6:5 Pero la adoración del hombre por el hombre alcanzó su punto álgido cuando los gobernantes temporales ordenaban a sus súbditos que le rindieran dicha veneración y, para fundamentar tales exigencias, afirmaban haber descendido de la deidad.
\usection{7. LOS ASISTENTES DE LA ADORACIÓN Y DE LA SABIDURÍA}
\vs p085 7:1 Puede parecer que la adoración de la naturaleza surgió de forma natural y espontánea en la mente de los hombres y mujeres primitivos, y así fue; pero durante todo este tiempo, en estas mismas mentes primitivas, obraba el sexto espíritu asistente, otorgado a estos pueblos para ejercer su influencia en esta etapa de la evolución humana. Y este espíritu estimulaba continuamente el impulso a la adoración en la especie humana, por muy primitivas que fuesen sus primeras manifestaciones. Claramente, el espíritu de adoración dio origen a dicho impulso humano, aunque el temor a los animales motivó su expresión, y su práctica temprana se centró en motivos de la naturaleza.
\vs p085 7:2 Debéis recordar que el sentimiento, y no el pensamiento, fue el elemento que guió y rigió el desarrollo evolutivo. Para la mente primitiva, hay poca diferencia entre temer, rehuir, homenajear y adorar.
\vs p085 7:3 Cuando el estímulo a la adoración está asesorado y dirigido por la sabiduría ---o pensamiento meditativo y vivencial---, empieza seguidamente a convertirse en el fenómeno de la auténtica religión. Cuando el séptimo asistente, el espíritu de la sabiduría, logra llevar a cabo su ministerio con eficacia, entonces el hombre, a través de la adoración, comienza a alejarse de la naturaleza y de los motivos naturales para acercarse al Dios de la naturaleza y Creador eterno de todas las cosas naturales.
\vsetoff
\vs p085 7:4 [Exposición de una brillante estrella vespertina de Nebadón.]
