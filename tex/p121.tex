\upaper{121}{La época del ministerio de gracia de Miguel}
\author{Comisión de seres intermedios}
\vs p121 0:1 Soy el ser intermedio secundario, adscrito con anterioridad al apóstol Andrés. Actúo bajo la supervisión de una comisión de doce miembros de la Hermandad de los Seres Intermedios Unidos de Urantia, auspiciada conjuntamente por el presidente de nuestro orden y el melquisedec designado. Estoy facultado para dejar constancia de la narración de los hechos de la vida de Jesús de Nazaret tal como mi orden de criaturas de la tierra los observó y el ser humano bajo mi custodia temporal en parte los relató. Sabiendo cómo su maestro había sido particularmente cuidadoso en evitar dejar testimonios escritos tras él, Andrés se negó categóricamente a que se multiplicasen las copias de su narrativa escrita. Una actitud idéntica por parte de los otros apóstoles de Jesús demoró de manera considerable la redacción de los evangelios.
\usection{1. OCCIDENTE EN EL SIGLO I D. C.}
\vs p121 1:1 Jesús no vino a este mundo en una era de decadencia espiritual. En el momento de su nacimiento, Urantia estaba experimentando un resurgimiento del pensamiento espiritual y de la vida religiosa nunca antes experimentado ni durante todo el periodo posadánico que le precedió ni durante el que le siguió desde entonces. Cuando Miguel se encarnó en Urantia, en el mundo se daban las más ventajosas condiciones para el desarrollo del ministerio de gracia del hijo creador que hubieran imperado ni hasta ese momento ni desde entonces. En los siglos inmediatamente anteriores a esta época, la cultura y la lengua griega se habían difundido por el Occidente y el Oriente Próximo y, los judíos, al ser una raza levantina de naturaleza en parte occidental y en parte oriental estaban excepcionalmente dotados para servirse de ese contexto cultural y lingüístico al objeto de la adecuada difusión de una nueva religión tanto hacia el este como hacia el oeste. Estas circunstancias, muy favorables, se vieron beneficiadas además por la política tolerante de los romanos en el mundo mediterráneo.
\vs p121 1:2 Toda esta combinación de influencias a escala mundial queda bien ilustrada en la labor de Pablo que, siendo un hebreo entre los hebreos por su bagaje religioso y un ciudadano romano, predicó el evangelio de un mesías judío en lengua griega.
\vs p121 1:3 Nada equiparable a la civilización de los tiempos de Jesús se ha visto en Occidente ni con anterioridad a aquellos días ni desde entonces. La civilización europea se unificó y coordinó bajo un triple y extraordinario influjo:
\vs p121 1:4 \li{1.}El sistema político y social romano.
\vs p121 1:5 \li{2.}La cultura y la lengua griega ---y, hasta cierto punto, su filosofía---.
\vs p121 1:6 \li{3.}La rápida propagación de las enseñanzas religiosas y morales judías.
\vs p121 1:7 \pc Cuando nació Jesús, todo el entorno del mediterráneo era un imperio unificado. Por primera vez en la historia del mundo, había buenas carreteras que conectaban entre sí muchos centros principales. Los mares estaban libres de piratas, y era un gran momento para el rápido desarrollo del comercio y los viajes. Europa no volvió a gozar de un período así hasta el siglo XIX d. C.
\vs p121 1:8 A pesar de la paz interna y de la prosperidad aparente del mundo greco\hyp{}romano, la mayoría de los habitantes del imperio se consumía en la miseria y en la pobreza. La reducida clase alta era rica; pero la gente ordinaria de la humanidad pertenecía a una clase baja desdichada y empobrecida. En aquellos días, no existía una clase media feliz y próspera; esta acababa de hacer su aparición en la sociedad romana.
\vs p121 1:9 Las primeras luchas entre los estados romano y parto, ambos en expansión, habían concluido en el entonces reciente pasado, dejando a Siria en manos de los romanos. En los tiempos de Jesús, Palestina y Siria gozaban de un período de prosperidad, de paz relativa y de amplias relaciones comerciales con los territorios tanto los situados al este como al oeste.
\usection{2. EL PUEBLO JUDÍO}
\vs p121 2:1 Los judíos formaban parte de la raza semita más antigua. En ella se incluía también a los babilonios, a los fenicios y a los enemigos más recientes de Roma: a los cartagineses. Durante la primera parte del siglo I d. C., los judíos constituían, de entre los pueblos semitas, el grupo más influyente, ocupando una posición geográfica particularmente estratégica en el mundo, por lo que se refiere a cómo el comercio se organizaba y regía en ese momento.
\vs p121 2:2 Muchas de las grandes carreteras que unían a las naciones de la antigüedad pasaban por Palestina, que de este modo se convirtió en el punto de encuentro, o encrucijada, de tres continentes. Por Palestina pasaban continuamente viajeros, comerciantes al igual que los ejércitos de Babilonia, Asiria, Egipto, Siria, Grecia, Partia y Roma. Desde tiempos inmemoriales, muchas rutas de caravanas procedentes del Oriente atravesaban alguna parte de esta región en dirección a los pocos buenos puertos del extremo oriental del Mediterráneo, desde donde los barcos transportaban sus cargamentos a todo el occidente marítimo. Más de la mitad de este comercio de caravanas cruzaba la pequeña ciudad de Nazaret de Galilea.
\vs p121 2:3 Aunque Palestina fue el germen de la cultura religiosa judía y la cuna del cristianismo, los judíos se repartieron por el mundo; vivían en muchos países distintos y comerciaban en todas las provincias de los estados romano y parto.
\vs p121 2:4 Grecia aportó el idioma y la cultura, Roma construyó las carreteras y unificó un imperio, pero la dispersión de los judíos, con sus más de doscientas sinagogas y comunidades religiosas bien organizadas y diseminadas por todas partes del mundo romano, facilitó la creación de centros culturales en los que se dio acogida por primera vez al nuevo evangelio del reino de los cielos, que de aquí se propagó seguidamente a las regiones más distantes del mundo.
\vs p121 2:5 Cada sinagoga hebrea toleraba a un grupo marginal de creyentes gentiles, de hombres “devotos” o “temerosos de Dios”, y fue en este grupo de prosélitos donde Pablo hizo la mayor parte de sus primeros conversos al cristianismo. Incluso el templo de Jerusalén disponía de un patio ornamentado para los gentiles. Había una relación muy estrecha entre la cultura, el comercio y el culto de adoración de Jerusalén y Antioquía. En Antioquía se llamó por primera vez “cristianos” a los discípulos de Pablo.
\vs p121 2:6 La centralización del culto de adoración judío en el templo de Jerusalén significó por igual la clave de la supervivencia de su monoteísmo y la promesa de que se cultivaría y proyectaría al mundo un concepto, nuevo y ampliado, de un solo Dios de todas las naciones y Padre de todos los mortales. El servicio del templo de Jerusalén representaba la supervivencia de un concepto cultural religioso frente a la aciaga llegada por parte de los gentiles de una sucesión de mandatarios nacionales y persecutores raciales.
\vs p121 2:7 \pc El pueblo judío de aquel momento, aunque bajo el protectorado romano, gozaba de un grado considerable de autonomía y, recordando los entonces recientes actos heroicos de liberación de Judas Macabeo y de sus inmediatos sucesores, vibraba ante la expectativa de la aparición inminente de un libertador todavía más grande: el tan esperado Mesías.
\vs p121 2:8 La clave de la supervivencia de Palestina, el reino de los judíos, como estado semiindependiente, consistía en la política exterior del gobierno romano, que deseaba mantener el control sobre la carretera palestina de tránsito entre Siria y Egipto, al igual que sobre las estaciones terminales occidentales de las rutas de las caravanas entre Oriente y Occidente. Roma no deseaba que surgiese ninguna potencia en el Levante que pudiera poner freno a su futura expansión en estas regiones. La política de intrigas que tenía como objeto el enfrentamiento entre la Siria seléucida y el Egipto ptolemaico necesitaba favorecer en Palestina un estado separado e independiente. La política romana, la decadencia de Egipto y el progresivo debilitamiento de los seléucidas ante el creciente poder de Partia explican por qué, durante varias generaciones un grupo pequeño y poco poderoso de judíos pudo mantener su independencia en contra de los seléucidas del norte y de los ptolomeos del sur. Los judíos atribuían esta libertad e independencia fortuitas del régimen político de los pueblos más poderosos que les rodeaban al hecho de que ellos eran “el pueblo elegido”, esto es, a la intervención directa de Yahvé. Esta actitud de superioridad racial hizo que les resultara muy difícil soportar el protectorado romano cuando este finalmente recayó sobre su tierra. Pero, incluso en esa triste hora, los judíos se negaron a admitir que su misión en el mundo era espiritual, no política.
\vs p121 2:9 \pc En los tiempos de Jesús, los judíos se mostraban extraordinariamente temerosos y recelosos porque estaban gobernados por un extranjero, Herodes el Idumeo, que había tomado la jefatura de Judea congraciándose hábilmente con los gobernantes romanos. Aunque profesara su lealtad a la observación del ceremonial hebreo, Herodes procedió a construir templos a muchos dioses extranjeros.
\vs p121 2:10 Las relaciones amistosas de Herodes con los gobernantes romanos propiciaron un mundo seguro al viajero judío, abriéndose el camino a una mayor penetración de los judíos, con el nuevo evangelio del reino de los cielos, hasta zonas remotas del Imperio romano y hasta distintas naciones extranjeras aliadas. El reinado de Herodes también hizo una gran contribución a la posterior mezcla de las filosofías hebrea y helenística.
\vs p121 2:11 Herodes construyó el puerto de Cesarea, lo que ayudó además a hacer de Palestina el cruce de caminos del mundo civilizado. Herodes murió en el año 4 a. C., y su hijo Herodes Antipas gobernó en Galilea y Perea durante la juventud y el ministerio de Jesús, hasta el año 39 d. C. Al igual que su padre, Antipas fue un gran constructor. Reconstruyó muchas de las ciudades de Galilea, incluyendo el importante centro comercial de Séforis.
\vs p121 2:12 Los dirigentes religiosos de Jerusalén y los maestros rabínicos no tenían una buena opinión de los galileos. Cuando Jesús nació, Galilea era más gentil que judía.
\usection{3. ENTRE LOS GENTILES}
\vs p121 3:1 Aunque las condiciones sociales y económicas del estado romano no eran de un orden muy elevado, la paz y la prosperidad internas, ampliamente generalizadas, fueron propicias para el ministerio de gracia de Miguel. En el siglo I d. C., la sociedad del mundo mediterráneo estaba compuesta de estratos bien definidos:
\vs p121 3:2 \li{1.}\bibemph{La aristocracia}. Las clases altas con dinero y poder oficial, los grupos privilegiados y gobernantes.
\vs p121 3:3 \li{2.}Los \bibemph{grupos comerciales}. Los comerciantes más poderosos y los banqueros, los grupos empresariales ---los grandes importadores y exportadores---, los comerciantes internacionales.
\vs p121 3:4 \li{3.}La \bibemph{pequeña clase media}. Este grupo, aunque reducido de hecho, era muy influyente y aportó el pilar moral de la Iglesia cristiana primitiva, que alentó a estos grupos a que continuaran realizando sus distintos oficios y actividades comerciales. Entre los judíos, muchos de los fariseos pertenecían a esta clase de comerciantes.
\vs p121 3:5 \li{4.}\bibemph{El proletariado libre}. Este grupo tenía poco o ningún estatus social. Aunque orgullosos de su libertad, tenían una posición de desventaja porque estaban forzados a competir por el trabajo con los esclavos. Las clases altas los miraban con desprecio, admitiendo que solo servían para “fines de reproducción”.
\vs p121 3:6 \li{5.}\bibemph{Los esclavos}. La mitad de la población del estado romano estaba compuesta por esclavos; muchos de ellos sobresalían por su talento y se abrían camino rápidamente en el proletariado libre e incluso entre los comerciantes. En su mayor parte eran personas corrientes o muy pobremente dotadas.
\vs p121 3:7 Las conquistas militares de los romanos se caracterizaban por esclavizar a personas de valía. El poder de los amos sobre sus esclavos era ilimitado. La Iglesia cristiana primitiva se componía en gran parte de estos esclavos y de las clases bajas.
\vs p121 3:8 Los esclavos de mayor valía solían recibir sueldos que ahorraban para comprar su libertad. Muchos de estos esclavos emancipados accedían a puestos importantes en el Estado, en la Iglesia y en el mundo del comercio. Fue precisamente este motivo el que llevó a la Iglesia cristiana primitiva a ser tan tolerante con esta forma moderada de esclavitud.
\vs p121 3:9 \pc En el Imperio romano del siglo I d. C., no había problemas sociales generalizados. La mayoría de la población se veía a sí misma como parte del grupo en el que le había tocado nacer. Siempre había una puerta abierta para que las personas capaces y de talento pudieran escalar del estrato inferior de la sociedad romana al superior, pero la mayoría se contentaba con su rango social. No tenían conciencia de clase ni tampoco consideraban injustas o equivocadas las distinciones existentes. El cristianismo no fue, en sentido alguno, un movimiento económico que pretendiera aliviar la miseria de las clases oprimidas.
\vs p121 3:10 Aunque en todo el Imperio romano, la mujer gozaba de mayor libertad que en Palestina, donde su situación era más restrictiva, la devoción a la familia y la afectividad natural de los judíos sobrepasaban con creces a las del mundo de los gentiles.
\usection{4. LA FILOSOFÍA DE LOS GENTILES}
\vs p121 4:1 Desde un punto de vista moral, los gentiles eran algo inferiores a los judíos, pero había, en los corazones de los más nobles, terreno abundante para la bondad natural y un potencial de afecto humano que hacía posible que la semilla del cristianismo germinara y produjera una abundante cosecha de caracteres morales y de logro espiritual. En aquel momento, el mundo de los gentiles estaba dominado por cuatro grandes filosofías, todas ellas derivadas en mayor o menor grado del más temprano platonismo de los griegos. Estas escuelas filosóficas eran las siguientes:
\vs p121 4:2 \li{1.}\bibemph{La epicúrea}. Esta escuela de pensamiento estaba dedicada a la búsqueda de la felicidad. Los mejores epicúreos no eran dados a excesos sensuales. Esta doctrina contribuyó, al menos, a la liberación de los romanos de una forma más nefasta de fatalismo; enseñaba que los hombres podían hacer algo para mejorar su estatus en la tierra. Combatía con eficacia las supersticiones nacidas de la ignorancia.
\vs p121 4:3 \li{2.}\bibemph{La estoica}. El estoicismo era la filosofía superior de las clases más elevadas. Los estoicos creían que toda la naturaleza estaba regida por una razón\hyp{}hado. Enseñaban que el alma del hombre era divina, que estaba recluida en un cuerpo maligno de naturaleza física. El alma del hombre alcanzaba la libertad viviendo en armonía con la naturaleza, con Dios; así pues, la virtud era su propia recompensa. El estoicismo ascendió hasta una moral sublime, hasta unos ideales jamás trascendidos por ningún otro sistema de filosofía puramente humano. Aunque los estoicos profesaban ser “descendientes de Dios”, no lograron conocerlo y, por lo tanto, no lograron encontrarlo. El estoicismo siguió siendo una filosofía; nunca se convirtió en una religión. Sus seguidores trataron de sintonizar su mente con la armonía de la Mente Universal; pero no lograron verse a sí mismos como hijos de un Padre amoroso. Pablo mostró su gran preferencia por el estoicismo cuando escribió: “He aprendido a contentarme, cualquiera que sea mi situación”.
\vs p121 4:4 \li{3.}\bibemph{La cínica}. Aunque los cínicos trazaban el origen de su filosofía hasta Diógenes de Atenas, una gran parte de su doctrina estaba basada en restos de las enseñanzas de Maquiventa Melquisedec. El cinismo había sido anteriormente más una religión que una filosofía. Al menos los cínicos democratizaron su filosofía religiosa. En los campos y en los mercados, los cínicos pregonaban continuamente su doctrina de que “el hombre podría salvarse a sí mismo si lo quería”. Predicaban la sencillez y la virtud e instaban a los hombres a enfrentarse a la muerte con valentía. Estos sacerdotes cínicos itinerantes hicieron mucho por preparar al pueblo, hambriento espiritualmente, para los misioneros cristianos que llegarían más tarde. Su plan de predicación popular seguía bastante al de las epístolas de Pablo en cuanto a su estructura y estilo.
\vs p121 4:5 \li{4.}\bibemph{La escéptica}. El escepticismo afirmaba que el conocimiento era falaz, y que la convicción y la certeza eran imposibles. Se trataba de una actitud puramente negativa y nunca llegó a extenderse demasiado.
\vs p121 4:6 \pc Todas estas filosofías eran semirreligiosas; a menudo eran motivadoras, éticas y ennoblecedoras; pero en general estaban muy lejos del alcance de la gente común. Con la posible excepción del cinismo, eran filosofías para el fuerte y el sabio, y no religiones de salvación ni incluso para los pobres y los débiles.
\usection{5. LAS RELIGIONES DE LOS GENTILES}
\vs p121 5:1 Durante todas las eras anteriores, la religión había sido principalmente un asunto de la tribu o de la nación; pocas veces un tema de preocupación individual. Los dioses eran tribales o nacionales, nunca personales. Se trataba de unos sistemas religiosos que daban poca satisfacción a las aspiraciones espirituales individuales de la gente corriente.
\vs p121 5:2 En los tiempos de Jesús, en las religiones de Occidente se incluían:
\vs p121 5:3 \li{1.}\bibemph{Los cultos paganos}. Combinaban la mitología helénica y latina, el patriotismo y la tradición.
\vs p121 5:4 \li{2.}\bibemph{El culto de adoración al emperador}. Esta deificación del hombre como símbolo del Estado indignaba seriamente a los judíos y a los primeros cristianos, y llevó directamente a las encarnizadas persecuciones de estas dos iglesias por parte del gobierno romano.
\vs p121 5:5 \li{3.}\bibemph{La astrología}. Esta pseudociencia de Babilonia se convirtió en todo el Imperio grecorromano en una religión. Ni incluso en el siglo XX le ha sido posible al hombre liberarse del todo de esta creencia supersticiosa.
\vs p121 5:6 \li{4.}\bibemph{Las religiones de misterio}. En aquel mundo tan espiritualmente hambriento afluyeron los cultos de misterio, nuevas y extrañas religiones provenientes del Levante, que cautivaban al hombre común con la promesa de la salvación \bibemph{individual}. Estas religiones se convirtieron rápidamente en la creencia generalizada de las clases más bajas del mundo grecorromano. E hicieron mucho por preparar el camino para la rápida difusión de las enseñanzas cristianas, inmensamente superiores, que impartían un concepto majestuoso de la Deidad, junto a una teología fascinante para los inteligentes y el profundo ofrecimiento de la salvación para todos, incluyendo al hombre medio de aquellos días, ignorante pero espiritualmente hambriento.
\vs p121 5:7 \pc Las religiones de misterio representaban el fin de la era de las creencias nacionales y dieron lugar al nacimiento de numerosos sistemas de culto personales. Había muchos de ellos, pero todos tenían características comunes:
\vs p121 5:8 \li{1.}Una leyenda mítica, un misterio ---de ahí su nombre---. Como norma, el misterio aludía a la vida, muerte y vuelta a la vida de un dios, tal como se ilustra en las enseñanzas del mitraísmo, que durante cierto tiempo, fue contemporáneo, y compitió, con el creciente sistema de culto cristiano de Pablo.
\vs p121 5:9 \li{2.}Las religiones de misterio eran interraciales y sin límites nacionales. Eran personales y fraternales, y dieron origen a hermandades religiosas y a numerosas organizaciones sectarias.
\vs p121 5:10 \li{3.}En cuanto a sus servicios religiosos, se caracterizaban por elaboradas ceremonias de iniciación e impresionantes sacramentos de adoración. Sus ritos y rituales secretos eran a veces espantosos y repugnantes.
\vs p121 5:11 \li{4.}Pero fuese cual fuese la naturaleza de sus ceremonias o el grado de sus excesos, estos cultos mistéricos invariablemente prometían a sus devotos la \bibemph{salvación,} “la liberación del mal, la supervivencia tras la muerte y una vida perdurable en los reinos de la dicha, más allá de este mundo de pesares y esclavitud”.
\vs p121 5:12 \pc Pero, no cometáis el error de confundir las enseñanzas de Jesús con los cultos mistéricos. Su popularidad revela la búsqueda del hombre por sobrevivir y, por tanto, una verdadera sed y hambre de una religión personal y de una rectitud individual. Aunque estos sistemas de cultos no lograron dar una adecuada satisfacción a este anhelo, sí prepararon el camino para la posterior aparición de Jesús, que verdaderamente trajo a este mundo el pan y el agua de vida.
\vs p121 5:13 Procurando aprovechar la adhesión generalizada a los mejores tipos de las religiones de misterio, Pablo efectuó ciertas adaptaciones de las enseñanzas de Jesús para hacerlas más aceptables a un mayor número de posibles conversos. Pero incluso estos arreglos de Pablo de las enseñanzas de Jesús (el cristianismo) resultaron ser superiores a lo mejor de estas religiones en que:
\vs p121 5:14 \li{1.}Pablo enseñaba una redención moral, una salvación ética. El cristianismo proponía una nueva vida y proclamaba un nuevo ideal. Pablo prescindió de los ritos mágicos y de los encantamientos ceremoniales.
\vs p121 5:15 \li{2.}El cristianismo representaba una religión que trataba de resolver definitivamente el problema humano, porque no solo ofrecía rescatar del dolor e incluso de la muerte, sino que prometía también liberar del pecado y la consiguiente dotación de un carácter recto con cualidades de supervivencia eterna.
\vs p121 5:16 \li{3.}Los cultos mistéricos se basaban en mitos. El cristianismo, tal como Pablo lo predicaba, se fundaba en un hecho histórico: Miguel, el Hijo de Dios, se daba de gracia a la humanidad.
\vs p121 5:17 \pc Entre los gentiles, la moral no estaba necesariamente relacionada ni con la filosofía ni con la religión. Fuera de Palestina, no siempre se suponía que los sacerdotes de una religión debían llevar una vida moral. La religión judía, luego las enseñanzas de Jesús y, más tarde, la evolución del cristianismo de Pablo constituyeron las primeras religiones europeas que pusieron de relieve la moral y la ética, exigiendo que los creyentes prestaran atención a las dos.
\vs p121 5:18 Jesús nació en Palestina, en esta generación de hombres, dominados por unos sistemas filosóficos deficientes y perplejos ante complejos sistemas de culto religiosos. Y a esta misma generación le ofrecería con el tiempo su evangelio de una religión personal: la filiación con Dios.
\usection{6. LA RELIGIÓN HEBREA}
\vs p121 6:1 Hacia finales del siglo I a. C., el pensamiento religioso de Jerusalén se vio enormemente influenciado y, en cierta manera, modificado, por las enseñanzas culturales griegas e incluso por la filosofía griega. En la larga disputa entre las perspectivas de las escuelas oriental y occidental de pensamiento hebreo, Jerusalén y el resto de Occidente así como el Levante adoptaron, en general, el punto de vista occidental judío o el helenístico modificado.
\vs p121 6:2 En la Palestina de los tiempos de Jesús imperaban tres idiomas: la gente corriente hablaba algún dialecto del arameo; los sacerdotes y rabinos hablaban hebreo; las clases educadas y los estratos más altos de la población judía hablaban griego por lo general. La pronta traducción en Alejandría de las escrituras hebreas al griego fue, en buena medida, responsable del posterior predominio de la corriente griega en la cultura y la teología judías. En esta misma lengua no tardarían en aparecer los escritos de los maestros cristianos. El renacimiento del judaísmo data de la traducción griega de las escrituras hebreas. Esta influencia fue crucial y determinó que el sistema de culto cristiano de Pablo derivase más tarde hacia el oeste en lugar de hacerlo hacia el este.
\vs p121 6:3 Aunque las enseñanzas de los epicúreos habían influenciado muy escasamente las creencias judías helenizadas, la filosofía de Platón y las doctrinas de la abnegación de los estoicos sí lo habían hecho y en un grado muy considerable. El gran avance del estoicismo se ejemplifica en el Cuarto Libro de los Macabeos; el influjo tanto de la filosofía platónica como de las doctrinas estoicas se ilustra en la Sabiduría de Salomón. Los judíos helenizados interpretaban las escrituras hebreas de una forma tan alegórica, que no encontraron dificultad en hacer conformar la teología hebrea con su reverenciada filosofía de Aristóteles. Si bien, todo esto llevó a una nefasta confusión hasta que Filón de Alejandría se hizo cargo de la problemática, armonizando y sistematizando la filosofía griega y la teología hebrea en un sistema sólido y bastante coherente de creencias y prácticas religiosas. Y, más tarde, fueron estas enseñanzas que combinaban la filosofía griega y la teología hebrea las que predominaban en Palestina cuando Jesús vivía e impartía sus enseñanzas, y serían las que servirían a Pablo como base para edificar su más avanzado y preclaro sistema de culto del cristianismo.
\vs p121 6:4 Filón fue un gran maestro; desde Moisés no había existido un hombre que ejerciera una influencia tan profunda sobre el pensamiento ético y religioso del mundo occidental. Ha habido siete excelentes maestros humanos que se prestaron a combinar los mejores elementos de los sistemas coetáneos de enseñanzas éticas y religiosas: Sethard, Moisés, Zoroastro, Lao\hyp{}tse, Buda, Filón y Pablo.
\vs p121 6:5 Aunque no todas, Pablo identificó muchas de las incoherencias en las que había incurrido Filón, en su intento de combinar la filosofía mística griega y las doctrinas estoicas romanas con la teología legalista de los hebreos, y las supo eliminar de los fundamentos de su teología precristiana. Filón abrió el camino para que Pablo pudiera restaurar en mayor plenitud el concepto de la Trinidad del Paraíso, que había estado por mucho tiempo latente en la teología judía. En una única cuestión Pablo no estuvo a la altura de Filón o no superó las enseñanzas de este judío, rico y educado, de Alejandría, y fue en la doctrina de la expiación; Filón educaba en la idea de la liberación de una doctrina del perdón únicamente basada en el derramamiento de sangre. Es posible que Filón también vislumbrara la realidad y la presencia de los modeladores del pensamiento con mayor claridad que Pablo. Pero la teoría de Pablo del pecado original ---las doctrinas de la culpa hereditaria y del mal innato y su redención--- era de origen parcialmente mitraico y tenía poco en común con la teología hebrea, con la filosofía de Filón o con las enseñanzas de Jesús. Algunas facetas de las enseñanzas de Pablo en relación al pecado original y a la expiación eran suyas originales.
\vs p121 6:6 El evangelio de Juan, la última de las narraciones sobre la vida terrena de Jesús, se dirigía a los pueblos occidentales y presenta su relato, teniendo en cuenta el punto de vista de los cristianos alejandrinos posteriores, también seguidores de las enseñanzas de Filón.
\vs p121 6:7 \pc Aproximadamente en la época de Cristo, ocurrió en Alejandría un extraño cambio de actitud hacia los judíos y, desde este anterior baluarte judío, se propagó una virulenta ola de persecuciones que se extendió incluso hasta Roma, de donde miles de ellos fueron deportados. Pero esta campaña que daba una imagen falsa de los judíos fue breve; muy pronto el gobierno imperial restableció por completo, en todo el imperio, las libertades que se les había truncado.
\vs p121 6:8 Por todo el amplio mundo, donde fuese que se encontraran los judíos dispersos por el comercio o por las persecuciones, todos perseveraban en mantener sus corazones centrados en el templo sagrado de Jerusalén. La teología judía sobrevivió realmente tal como se había interpretado y practicado en Jerusalén, a pesar de haber sido rescatada varias veces del olvido por la oportuna intervención de algunos maestros de Babilonia.
\vs p121 6:9 Hasta dos millones y medio de estos judíos dispersos solían dirigirse a Jerusalén para la celebración de sus festivales religiosos nacionales. Y al margen de las diferencias teológicas o filosóficas entre los judíos orientales (de Babilonia) y los occidentales (helénicos), todos coincidían en considerar a Jerusalén como el centro de su culto de adoración y de su anhelada espera de la llegada del Mesías.
\usection{7. JUDÍOS Y GENTILES}
\vs p121 7:1 En los tiempos de Jesús, los judíos habían llegado a un arraigado concepto de su origen, historia y destino. Entre ellos y el mundo gentil habían erigido un espeso muro de separación; miraban las costumbres de los gentiles con enorme desprecio. Adoraban la letra de la ley y se complacían en una forma de santurronería basada en el vano orgullo de su línea de descendencia. Se habían formado nociones preconcebidas sobre el Mesías prometido, y la mayoría de estas expectativas preveían a un mesías que llegaría como parte de su historia nacional y racial. Para los hebreos de aquellos días, la teología judía estaba irrevocablemente determinada, fijada para siempre.
\vs p121 7:2 Las enseñanzas y las prácticas de Jesús en relación a la tolerancia y a la bondad contravenían la actitud que, desde tiempo inmemoriales, tenían los judíos hacia otros pueblos a los que consideraban paganos. Durante generaciones, los judíos habían cultivado una actitud hacia el mundo exterior que les hacía imposible aceptar las enseñanzas del Maestro sobre la fraternidad espiritual del hombre. No estaban dispuestos a compartir a Yahvé en términos de igualdad con los gentiles, y eran igualmente contrarios a aceptar como el Hijo de Dios a quien enseñaba unas doctrinas tan nuevas y extrañas.
\vs p121 7:3 Los escribas, los fariseos y los sacerdotes mantenían al pueblo judío en una terrible sujeción al ritualismo y al legalismo, una sujeción mucho más real que la debida al gobierno político romano. Los judíos del tiempo de Jesús no solo estaban bajo el sometimiento de la \bibemph{ley,} sino que estaban igualmente atados a las exigencias serviles de las tradiciones, que afectaban e invadían todos los ámbitos de su vida personal y social. Estas minuciosas reglas de conducta se aplicaban y gobernaban a cualquier judío leal, y no es de extrañar que rechazaran de inmediato a cualquiera de los suyos que osara ignorar sus sagradas \bibemph{tradiciones} y que se atreviese a desacatar las reglas de conducta social por tanto tiempo veneradas. Era difícil que pudiesen tener una opinión favorable de las enseñanzas de quien no dudaba en entrar en conflicto con los dogmas establecidos, según creían, por el mismo Padre Abraham. Moisés les había dado la ley y no estaban dispuestos a ponerla en tela de juicio.
\vs p121 7:4 Durante el siglo I d. C., la interpretación oral de la ley por los maestros reconocidos ---los escribas--- había llegado a primar sobre la misma ley escrita. Y todo esto hizo que algunos líderes religiosos de los judíos pudieran fácilmente disponer a la gente en contra de la aceptación de un nuevo evangelio.
\vs p121 7:5 Estas circunstancias imposibilitaban a los judíos cumplir su destino divino como mensajeros del nuevo evangelio de libertad religiosa y espiritual. No pudieron romper las cadenas de la tradición. Jeremías había hablado de una “ley que se escribirá en el corazón de los hombres”, Ezequiel había hecho referencia a un “nuevo espíritu que habitará en el alma del hombre” y el salmista había orado a Dios que “dentro de él creara un corazón limpio y renovara un espíritu recto”. Pero cuando la religión judía de las buenas obras y de la esclavitud a la ley fue víctima del estancamiento de la inercia tradicionalista, la acción de la evolución religiosa se desplazó al oeste, hacia los pueblos europeos.
\vs p121 7:6 Y así fue como un pueblo diferente fue llamado a llevar al mundo una teología en avance, un sistema de enseñanza que englobaba la filosofía de los griegos, la ley de los romanos, la moral de los hebreos y el evangelio de la santidad de la persona y de la libertad espiritual elaborado por Pablo sobre la base de las enseñanzas de Jesús.
\vs p121 7:7 \pc El sistema de culto cristiano de Pablo tenía como marca de nacimiento la moral de los hebreos. Los judíos concebían la historia como la providencia de Dios: Yahvé obrando. Los griegos aportaron a las nuevas enseñanzas unos conceptos más claros de la vida eterna. En cuanto a su teología y su filosofía, las doctrinas de Pablo tuvieron la influencia no solo de las enseñanzas de Jesús sino también de Platón y Filón. En cuanto a su ética, se inspiró no solamente en Cristo sino también en los estoicos.
\vs p121 7:8 El evangelio de Jesús, tal como quedó plasmado en el sistema de culto paulino del cristianismo de Antioquía, se mezcló con las enseñanzas siguientes:
\vs p121 7:9 \li{1.}El razonamiento filosófico de los prosélitos griegos del judaísmo, incluyendo algunos de sus conceptos sobre la vida eterna.
\vs p121 7:10 \li{2.}Las atractivas enseñanzas de los cultos de misterio imperantes en esa época, especialmente las doctrinas mitraicas de la redención, la expiación y la salvación por medio del sacrificio hecho por algún dios.
\vs p121 7:11 \li{3.}La sólida moral de la religión judía establecida.
\vs p121 7:12 \pc En los tiempos de Jesús, el Imperio romano del Mediterráneo, el reino parto y los pueblos limítrofes poseían ideas rudimentarias y primitivas sobre la geografía del mundo, la astronomía, la salud y la enfermedad y, naturalmente, se extrañaban de las novedosas y sorprendentes afirmaciones del carpintero de Nazaret. La idea de la posesión de espíritus, buenos y malos, se aplicaba no simplemente a los seres humanos, sino también, como muchos pensaban, a las rocas y a los árboles. Era una época de encantamientos, y se creía que los milagros eran algo común.
\usection{8. REGISTROS ESCRITOS ANTERIORES}
\vs p121 8:1 En lo posible, y acorde con nuestro mandato, hemos procurado hacer uso y, hasta cierto punto, armonizar los registros escritos existentes en Urantia que tuviesen que ver con la vida de Jesús. Aunque hemos podido acceder al texto perdido del apóstol Andrés y nos hemos beneficiado de la colaboración de una inmensa multitud de seres celestiales que estaban en la tierra en los tiempos del ministerio de gracia de Miguel (en particular de su modelador ahora en estado personal), también ha sido nuestra intención utilizar los llamados evangelios de Mateo, Marcos, Lucas y Juan.
\vs p121 8:2 Estos textos del Nuevo Testamento tuvieron su origen en las circunstancias siguientes:
\vs p121 8:3 \li{1.}\bibemph{El evangelio de Marcos}. Juan Marcos escribió el relato más temprano (a excepción de las notas de Andrés), más breve y más sencillo sobre la vida de Jesús y presentó al Maestro como servidor, como hombre entre los hombres. Aunque Marcos era un muchacho que estuvo presente en muchas de las escenas que describe, su relato es, en realidad, el evangelio según Simón Pedro. Fue primeramente acompañante de él y, más tarde, de Pablo. Marcos escribió este relato a instancias de Pedro y ante la insistente petición de la Iglesia de Roma. Conociendo la firmeza con la que el Maestro se negaba a redactar sus enseñanzas mientras vivía en la tierra en la carne, Marcos, como los apóstoles y otros discípulos principales, dudó en ponerlas por escrito. Pero Pedro se percató de que la Iglesia de Roma necesitaba la ayuda de esta narración escrita, y Marcos accedió a prepararla. Tomó muchas notas antes de que Pedro muriese en el año 67 d. C., y, de acuerdo con el borrador aprobado por Pedro, comenzó a escribir el evangelio para dicha Iglesia. Lo comenzó a escribir poco después de la muerte de Pedro y lo terminó hacia fines del año 68 d. C. Lo redactó basándose enteramente en su propia memoria y en la de Pedro. Desde entonces, el texto sufrió bastantes cambios; se eliminaron numerosos pasajes y, más adelante, se añadió material para reemplazar la última quinta parte del evangelio original, que se perdió del primer manuscrito incluso antes de ser copiado. El relato de Marcos, junto con las notas de Andrés y Mateo, sirvió de base para la redacción de todas las narraciones evangélicas posteriores que describían la vida y enseñanzas de Jesús.
\vs p121 8:4 \li{2.}\bibemph{El evangelio de Mateo}. El llamado evangelio según Mateo es el relato de la vida del Maestro escrito para la edificación de los cristianos judíos. El autor de este texto trata continuamente de demostrar que mucho de lo que hizo Jesús en su vida fue “para que se cumpliera lo que había dicho el profeta”. El evangelio de Mateo describe a Jesús como hijo de David, y lo presenta mostrando un gran respeto por la ley y los profetas.
\vs p121 8:5 El apóstol Mateo no escribió este evangelio. Lo escribió Isador, uno de sus discípulos. Para esta labor, Isador dispuso no solo de los recuerdos personales que tenía Mateo de estos acontecimientos, sino también de cierta anotación que este había hecho de los dichos de Jesús inmediatamente tras la crucifixión. Mateo escribió este texto en arameo; Isador escribió en griego. Al atribuir el evangelio a Mateo, no hubo intención de engaño. En esos tiempos, aquella era la costumbre en la que los pupilos honraban a sus maestros.
\vs p121 8:6 El texto original de Mateo se revisó y amplió en el año 40 d. C., justo antes de que este partiera de Jerusalén para predicar el evangelio. Se trataba de un documento privado; la última copia se destruyó en el incendio de un monasterio sirio en el año 416 d. C.
\vs p121 8:7 Isador huyó de Jerusalén en el año 70 d. C., tras el asedio de la ciudad por los ejércitos de Tito, llevándose con él a Pella una copia de las notas de Mateo. En el año 71, mientras vivía en Pella, Isador escribió el evangelio según Mateo. Tenía también con él las primeras cuatro quintas partes de la narración de Marcos.
\vs p121 8:8 \li{3.}El \bibemph{evangelio de Lucas}. Lucas, el médico de Antioquía de Pisidia fue un converso gentil de Pablo, y escribió una historia muy diferente de la vida del Maestro. En el año 47 d. C., comenzó a seguir a Pablo y a conocer la vida y las enseñanzas de Jesús. Lucas mantiene mucho de la “gracia del Señor Jesucristo” en su texto al haber recopilado los hechos que relata de Pablo y de otras personas. Lucas presenta al Maestro como “amigo de publicanos y pecadores”. Hasta después de la muerte de Pablo, Lucas no redactó sus muchas notas en forma de evangelio. Lucas escribió en Acaya durante el año 82. Planeaba escribir tres libros sobre la historia de Cristo y del cristianismo, pero murió en el año 90, justo antes de terminar la segunda de estas obras: los “Hechos de los Apóstoles”.
\vs p121 8:9 Lucas reunió el material para su evangelio tomándolo, en primer lugar, de la historia de la vida de Jesús tal como Pablo se la había relatado; por lo tanto, el evangelio de Lucas es, de alguna manera, el evangelio según Pablo. Pero Lucas tenía otras fuentes de información. No solamente entrevistó a decenas de testigos presenciales de los numerosos episodios de la vida de Jesús de los que tomó nota, sino que también tenía con él una copia del evangelio de Marcos, esto es, de sus cuatro primeras quintas partes, de la narración de Isador y de un breve texto redactado en el año 78 d. C., en Antioquía, por un creyente llamado Cedes. Lucas contaba igualmente con una copia fragmentada y muy revisada de algunas notas supuestamente realizadas por el apóstol Andrés.
\vs p121 8:10 \li{4.}\bibemph{El evangelio de Juan}. El evangelio según Juan relata gran parte de la obra de Jesús en Judea y alrededor de Jerusalén no mencionada en los otros textos. Se trata del llamado evangelio según Juan el hijo de Zebedeo y, aunque Juan no la escribió, ciertamente la inspiró. Desde su primera redacción, se ha revisado repetidas veces para dar la apariencia de que el mismo Juan la había escrito. Cuando se redactó este texto, Juan tenía los otros evangelios y observó que había muchas omisiones; por consiguiente, en el año 101 d. C. alentó a su acompañante Natán, un judío griego de Cesarea, a que acometiera la tarea de escribir. Juan aportaba el material de memoria y en referencia a los tres evangelios ya existentes. Juan no contaba con ningún registro escrito por él mismo sobre el tema, aunque sí escribió la epístola denominada “La primera de San Juan”, que era una carta de presentación del trabajo realizado por Natán bajo su dirección.
\vs p121 8:11 \pc Todos estos autores describieron con honestidad a Jesús tal como ellos lo habían visto, lo recordaban o habían sabido de él, y en la medida en la que sus conceptos de estos distantes acontecimientos se vieron influenciados por su posterior adopción de la teología paulina del cristianismo. Y estos textos, aunque imperfectos, han bastado para cambiar el curso de la historia de Urantia durante casi dos mil años.
\vsetoff
\vs p121 8:12 [\bibemph{Agradecimientos:} Para llevar a cabo mi cometido de reelaborar las enseñanzas de Jesús de Nazaret y de volver a narrar su obra, me he servido ampliamente de todas las fuentes documentales e informativas del planeta. Mi motivo principal ha sido preparar un texto que no solamente iluminara a la generación de hombres ahora vivos, sino que pudiera ser igualmente de utilidad a todas las generaciones futuras. De la inmensa reserva de información que se me ha facilitado, he seleccionado aquella más idónea para el logro de este propósito. En la medida de lo posible, he obtenido mi información de fuentes exclusivamente humanas. He recurrido a esos documentos de orden sobrehumano únicamente en el caso de que tales fuentes humanas fuesen insatisfactorias. Cuando alguna mente humana ha expresado razonablemente bien las ideas y los conceptos de la vida y de las enseñanzas de Jesús, siempre he dado preferencia a tales patrones de pensamiento supuestamente humanos. Aunque he tratado de adaptar la expresión verbal para conformarla, del mejor modo, con nuestro concepto del significado verdadero y de la auténtica importancia de la vida y de las enseñanzas del Maestro, en la medida de lo posible, me he ajustado, en todas mis narraciones, al concepto real y al patrón de pensamiento humanos. Sé bien que esos conceptos que han tenido origen en la mente humana resultarán más viables y provechosos para las demás mentes humanas. Cuando no he podido encontrar los conceptos necesarios en textos o expresiones humanas, he recurrido entonces a la memoria de mi propio orden de criaturas terrenas, los seres intermedios. Y cuando esta fuente secundaria de información se ha mostrado insuficiente, he acudido sin vacilar a fuentes de información extraplanetarias.
\vs p121 8:13 Las notas que he reunido, y que me han servido para preparar este relato de la vida y las enseñanzas de Jesús ---aparte de la memoria del texto del apóstol Andrés--- contienen joyas del pensamiento y conceptos superiores de las enseñanzas de Jesús, recogidos de más de dos mil seres humanos que vivieron en la tierra desde los tiempos de Jesús hasta el momento de la composición de estas revelaciones o, más exactamente, de su reelaboración. El permiso de revelación solo se ha utilizado cuando el registro y los conceptos humanos no proveían un patrón de pensamiento satisfactorio. Mi cometido en cuanto a la revelación me prohíbe recurrir a fuentes extrahumanas de información o de expresión, en tanto no pueda atestiguar que mi empeño por encontrar la expresión conceptual necesaria, en base a fuentes exclusivamente humanas, han resultado infructuosos.
\vs p121 8:14 Aunque yo, con la colaboración de once seres intermedios, compañeros y colaboradores míos, y bajo la supervisión del melquisedec asignado, he compuesto este relato de acuerdo con mi idea de su más adecuada disposición y en respuesta a mi elección, por su inmediatez, de alguna expresión, no obstante, la mayoría de las ideas e incluso algunas de las adecuadas expresiones que he utilizado se originaron en la mente de los hombres de muchas razas que han vivido en la tierra durante las generaciones trascurridas y que siguen viviendo en el momento de la realización de esta labor. En muchos aspectos, he servido más de recopilador y revisor que realmente de narrador. Sin vacilación, he hecho uso de las ideas y conceptos, preferiblemente humanos, que me permitieran crear el retrato más adecuado de la vida de Jesús y me facultaran para reelaborar sus inigualables enseñanzas, usando el modo de expresión más especialmente provechoso y universalmente significativo. En nombre de la Hermandad de los Seres Intermedios Unidos de Urantia, reconozco, con mi mayor agradecimiento, nuestra deuda con todos los recursos textuales y conceptuales utilizados de aquí en adelante para elaborar nuestra nueva exposición de la vida de Jesús en la tierra].
