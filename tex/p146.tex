\upaper{146}{El primer viaje de predicación por Galilea}
\author{Comisión de seres intermedios}
\vs p146 0:1 El primer viaje de predicación pública que se realizó en Galilea se inició el domingo 18 de enero, del año 28 d. C., y tuvo una duración de unos dos meses. Concluyó el 17 de marzo, tras lo cual regresaron a Cafarnaúm. En este viaje, Jesús y los doce apóstoles, con la ayuda de antiguos apóstoles de Juan, predicaron el evangelio y bautizaron a los creyentes en Rimón, Jotapata, Ramá, Zabulón, Irón, Giscala, Corazín, Madón, Caná, Naín y Endor. En dichas ciudades permanecieron algún tiempo para impartir sus enseñanzas, mientras que en otras muchas, más pequeñas, anunciaron el evangelio del reino a su paso por ellas.
\vs p146 0:2 Fue la primera vez que Jesús permitió a sus acompañantes predicar sin limitaciones. Durante el transcurso de ese viaje, los previno en tres ocasiones; les advirtió que se mantuvieran alejados de Nazaret y que fueran discretos al cruzar Cafarnaúm y Tiberias. Para los apóstoles, tener por fin la libertad de predicar y enseñar sin restricciones fue motivo de gran satisfacción, y se entregaron con fervor y júbilo a la labor de predicar el evangelio, atender a los enfermos y bautizar a los creyentes.
\usection{1. PREDICACIÓN EN RIMÓN}
\vs p146 1:1 En antaño, la pequeña ciudad de Rimón había estado dedicada a la adoración de Ramán, un dios babilónico del aire. En las creencias de los habitantes de Rimón subsistían muchas de las enseñanzas antiguas babilónicas y de las posteriores zoroástricas; así pues, Jesús y los veinticuatro emplearon gran parte de su tiempo tratando de aclararles la diferencia entre estas viejas creencias y el nuevo evangelio del reino. Pedro impartió aquí uno de los grandes sermones del inicio de su andadura con el título de “Aarón y el becerro de oro”.
\vs p146 1:2 Aunque muchos de los ciudadanos de Rimón se hicieron creyentes de las enseñanzas de Jesús, años después estos ocasionarían serios problemas a sus hermanos. Es muy difícil, en el corto espacio de tiempo de una sola vida, convencer a estos adoradores de la naturaleza a que se unan plenamente a la fraternidad de quienes adoran un ideal espiritual.
\vs p146 1:3 \pc Gran parte de las mejores ideas babilónicas y persas sobre la luz y la oscuridad, el bien y el mal, el tiempo y la eternidad, se incorporarían más tarde a las doctrinas del llamado cristianismo, y esto hizo que entre los pueblos del Oriente Próximo se aceptaran las enseñanzas cristianas con mayor celeridad. De la misma manera, la inclusión de muchas de las teorías de Platón sobre el espíritu ideal o los patrones invisibles de todas las cosas visibles y materiales, posteriormente adaptadas por Filón a la teología hebrea, hizo que, entre los griegos occidentales, se acogieran con mayor facilidad las enseñanzas cristianas de Pablo.
\vs p146 1:4 \pc Aquí en Rimón fue donde Todán oyó hablar del evangelio del reino por primera vez, llevando luego este mensaje hasta Mesopotamia y mucho más allá. Fue uno de los primeros en predicar la buena nueva a quienes habitaban al otro lado del Éufrates.
\usection{2. EN JOTAPATA}
\vs p146 2:1 Aunque la gente corriente de Jotapata oyó con agrado a Jesús y a sus apóstoles, y bastantes personas aceptaron el evangelio del reino, en esta pequeña ciudad, la misión se caracteriza en particular por la charla que Jesús dio a los veinticuatro durante la segunda noche de su estancia allí. Natanael se sentía confundido sobre las palabras del Maestro acerca de la oración, la acción de gracias y la adoración y, en respuesta a su pregunta, Jesús explicó con mayor detenimiento sus enseñanzas. Resumida en terminología moderna, se pueden destacar de esta conversación las siguientes consideraciones:
\vs p146 2:2 \li{1.}La iniquidad, si se alberga de forma consciente y pertinaz en el corazón del hombre, destruye paulatinamente la conexión del alma humana orante con las vías circulatorias espirituales que comunican al hombre y a su Hacedor. Naturalmente, Dios oye la petición de su hijo, pero cuando en el corazón humano se alojan, deliberada y persistentemente, pensamientos inicuos, sobreviene, de forma paulatina, una pérdida de comunión personal entre el hijo terrenal y su Padre celestial.
\vs p146 2:3 \li{2.}Esa oración que es incongruente con las leyes de Dios conocidas y establecidas es una abominación para las Deidades del Paraíso. Si un hombre no escucha a los Dioses cuando hablan a su creación mediante las leyes del espíritu, la mente y la materia, el acto mismo de tal desdén voluntario y consciente de la criatura hace que los seres espirituales aparten sus oídos de las peticiones personales de estos mortales transgresores y desobedientes. Jesús mencionó a sus apóstoles las palabras del profeta Zacarías: “Pero no quisieron escuchar, sino que volvieron la espalda y se taparon los oídos para no oír. Sí, endurecieron su corazón como diamante, no fuese que oyeran mi ley y mis palabras que yo les enviaba por mi espíritu a través de los profetas; por tanto, los frutos de su malévolo pensamiento vinieron como una gran ira sobre sus culpables cabezas. Y, aconteció, que clamaron por misericordia, pero no encontraron oído que los escucharan”. Y, entonces, Jesús citó el proverbio del hombre sabio que decía: “Incluso la oración le es abominable al que aparta su oído para no escuchar la ley divina”.
\vs p146 2:4 \li{3.}Al abrir los mortales el lado humano del canal de comunicación Dios\hyp{}hombre, se hace de inmediato accesible, a las criaturas de los mundos, el inagotable flujo del ministerio divino. Cuando el hombre oye al espíritu de Dios que le habla en su corazón humano, inherente, a esa vivencia, Dios oye simultáneamente su oración. Incluso el perdón de los pecados obra de esa misma inequívoca manera. El Padre de los cielos os ha perdonado aun antes de que hayáis pensado en pedírselo, pero tal perdón no está disponible en vuestra experiencia religiosa personal hasta que vosotros no perdonéis a vuestros semejantes. El perdón de Dios, como \bibemph{hecho,} no está condicionado a que tú perdones a tus semejantes, pero como \bibemph{vivencia} precisamente lo está. Y esta realidad de la sincronía del perdón divino y del perdón humano se reconoció así y están entrelazados en la oración que Jesús enseñó a los apóstoles.
\vs p146 2:5 \li{4.}Hay una ley elemental de justicia en el universo que la misericordia no es capaz de eludir. No es posible que una criatura totalmente egoísta de los reinos del tiempo y del espacio pueda recibir las generosas glorias del Paraíso. Ni siquiera el amor infinito de Dios puede forzar a una criatura mortal a la salvación de la vida eterna, si esta no elige sobrevivir. La misericordia es amplia en otorgar sus dádivas, pero, con todo, existen mandatos de la justicia que ni siquiera el amor en combinación con la misericordia puede realmente abolir. Jesús citó de nuevo un pasaje de las escrituras hebreas: “Os llamé, pero no quisisteis escuchar; tendí mi mano, pero no hubo quien atendiera, sino que desechasteis mi consejo y rechazasteis mi reprensión, y, por vuestra actitud rebelde, cuando me llaméis no responderé. Habéis rechazado el camino de la vida; y aunque me busquéis con diligencia cuando sobre vosotros venga la tribulación, no me hallaréis”.
\vs p146 2:6 \li{5.}Quienes deseen recibir misericordia, han de mostrar misericordia; no juzguéis, para no ser juzgados, porque con el espíritu con el que juzguéis a los demás así seréis juzgados. La misericordia no deroga por entero la equidad del universo. En último término, se hará cierto: “El que cierra su oído al clamor del pobre tampoco será oído cuando clame”. La sinceridad de cualquier oración es la garantía de que será oída; la sabiduría espiritual y la consonancia con el universo de una petición determinan el momento, el modo y el grado de la respuesta. Un padre sensato no responde \bibemph{literalmente} a las oraciones descabelladas de sus ignorantes e inexpertos hijos, aunque estos puedan obtener gran complacencia y una verdadera satisfacción del alma al realizar estas desatinadas peticiones.
\vs p146 2:7 \li{6.}Cuando os dediquéis por completo a hacer la voluntad del Padre de los cielos, se os dará respuesta a todas vuestras peticiones, porque oraréis en plena conformidad con la voluntad del Padre, y la voluntad del Padre siempre se manifiesta en todo su inmenso universo. Lo que el hijo verdadero desea y el Padre Infinito quiere ES. Tal oración no puede permanecer sin respuesta, y ningún otro tipo de petición será enteramente respondido.
\vs p146 2:8 \li{7.}El clamor de los rectos constituye el acto de fe del hijo de Dios que abre las puertas de los depósitos de bondad, verdad y misericordia del Padre; y estas buenas dádivas llevan mucho a la espera de que el hijo se aproxime y los posea de manera personal. La oración no modifica la actitud divina hacia el hombre, pero sí la actitud del hombre hacia el Padre inmutable. El \bibemph{motivo} de la oración es el que da paso al oído divino y no la condición social o económica ni el superficial estatus religioso de quien ora.
\vs p146 2:9 \li{8.}No puede emplearse la oración para evitar las dilaciones del tiempo ni superar los impedimentos del espacio. La oración no está concebida para el engrandecimiento del yo ni para lograr una injusta ventaja sobre vuestros semejantes. Un alma profundamente egoísta es incapaz de orar, en el sentido estricto de la palabra. Jesús dijo: “Deléitate supremamente en el carácter de Dios y él ciertamente te concederá los deseos sinceros de tu corazón”. “Encomienda al Señor tu camino, confía en él y él hará”. “Porque el Señor oye el clamor del menesteroso y atenderá la oración del desvalido”.
\vs p146 2:10 \li{9.}“Yo he venido del Padre; si, por tanto, dudáis alguna vez sobre lo que habréis de pedirle, hacedlo en mi nombre y yo presentaré vuestra petición según vuestras necesidades y deseos verdaderos y en conformidad con la voluntad del Padre”. Guardaos del serio peligro de orar de forma egocéntrica. Evitad orar en demasía por vosotros mismos; hacedlo más por el avance espiritual de vuestros hermanos. Eludid la oración materialista; orad en el espíritu y por la abundancia de los dones del espíritu.
\vs p146 2:11 \li{10.}Cuando oréis por los enfermos y los afligidos, no esperéis que vuestras peticiones sustituyan la atención, amorosa e inteligente, de las necesidades de estos dolientes. Orad por el bienestar de vuestras familias, amigos y compañeros, pero hacedlo muy especialmente por aquellos que os maldicen, y pedid con amor por quienes os persiguen. “Pero no os diré cuándo tenéis que orar. Solo el espíritu que habita en vosotros puede moveros a expresar esas peticiones que reflejen vuestra relación interior con el Padre de los espíritus”.
\vs p146 2:12 \li{11.}Hay muchos que solo recurren a la oración cuando están en dificultades, que tal acto no es sino una manera irreflexiva e ilusoria de orar. Sí, hacéis bien en orar cuando estéis angustiados, pero tened presente que, como hijo, debéis hablar con el Padre, incluso cuando vuestra alma se sienta complacida. Que vuestras verdaderas peticiones sean siempre en secreto. Que los hombres no oigan vuestras oraciones personales. Las oraciones de acción de gracias resultan convenientes para la adoración en grupo, pero la oración del alma es algo personal. No hay más que un modo de orar válido para todos los hijos de Dios, y es: “Que se haga, no obstante, tu voluntad”.
\vs p146 2:13 \li{12.}Todos los creyentes de este evangelio deben orar encarecidamente por la expansión del reino de los cielos. De todas las oraciones de las escrituras hebreas, Jesús comentó con encomio la siguiente oración del salmista: “Crea en mí, oh Dios, un corazón limpio, y renueva un espíritu recto dentro de mí. Líbrame de los pecados que me son ocultos y preserva a tu siervo de las soberbias”. Jesús habló detenidamente sobre la relación entre la oración y el lenguaje descuidado y ofensivo, citando: “Pon, oh Señor, ante mi boca un centinela; guarda la puerta de mis labios”. “Pocos hombres”, dijo Jesús, “pueden domar la lengua, pero el espíritu interior puede transformar este órgano ingobernable en una amable voz de tolerancia y en un estimulante dador de misericordia”.
\vs p146 2:14 \li{13.}Jesús enseñó que la oración que pide la guía divina en el sendero de la vida terrenal seguía en relevancia a la que lo hace por el conocimiento de la voluntad del Padre. En realidad, esto significa orar por la sabiduría divina. Jesús nunca enseñó que, mediante la oración, se pudiesen adquirir conocimientos y determinadas habilidades de carácter humano. Pero ciertamente enseñó que la oración contribuye a aumentar la propia capacidad para recibir la presencia del espíritu divino. Cuando Jesús instruyó a sus acompañantes para que oraran en el espíritu y en la verdad, explicó que se refería a orar con sinceridad y conforme a la lucidez de cada cual, a orar sin reservas y con inteligencia, ferviente y firmemente.
\vs p146 2:15 \li{14.}Jesús advirtió a sus seguidores que no pensaran que las oraciones serían más eficaces mediante ornamentadas repeticiones, lenguaje elocuente, el ayuno, la penitencia o los sacrificios. Pero sí exhortó a sus creyentes a que emplearan la oración como un medio conducente, por medio de la acción de gracias, a la verdadera adoración. Jesús se lamentó de que las oraciones y culto de sus seguidores contuviesen tan poco espíritu de agradecimiento. En esta ocasión, citó un pasaje de las Escrituras, que decía: “Bueno es dar gracias al Señor y cantar alabanzas en nombre del Altísimo, anunciar por la mañana su misericordia y su fidelidad cada noche, por cuanto Dios me ha alegrado con sus obras. En todo daré gracias conforme a la voluntad de Dios”.
\vs p146 2:16 \li{15.}Y entonces dijo Jesús: “No estéis constantemente angustiados por vuestras necesidades diarias. No os inquietéis por los problemas de vuestra existencia terrenal; pero en todas estas cosas, mediante la oración y la súplica, con un espíritu de sincera acción de gracias, poned vuestras necesidades ante vuestro Padre de los cielos”. Citó luego de las Escrituras: “El nombre de Dios alabaré con cántico, le ensalzaré con la acción de gracias. Y agradará al Señor más que sacrificio de buey o becerro que tiene cuernos y pezuñas”.
\vs p146 2:17 \li{16.}Jesús enseñó a sus seguidores que, cuando hubiesen orado al Padre, debían permanecer durante un tiempo en un silencio receptivo para facilitar al espíritu morador una más óptima comunicación con el alma escuchante. El espíritu del Padre habla mejor al hombre cuando la mente humana está en una actitud de verdadera adoración. Adoramos a Dios con la asistencia de este espíritu morador del Padre y por la iluminación de la mente humana a través del ministerio de la verdad. La adoración, enseñó Jesús, hace al hombre cada vez más semejante al ser al que está adorando. La adoración es una vivencia transformadora por medio de la que lo finito se va paulatinamente aproximando hasta alcanzar, por último, la presencia de lo Infinito.
\vs p146 2:18 \pc Y Jesús dijo numerosas otras verdades a sus apóstoles sobre la comunión del hombre con Dios, pero no muchos de ellos fueron capaces de asimilar totalmente sus enseñanzas.
\usection{3. PARADA EN RAMÁ}
\vs p146 3:1 En Ramá, Jesús tuvo una memorable charla con el anciano filósofo griego, que enseñaba que la ciencia y la filosofía eran suficientes para satisfacer las necesidades de la experiencia humana. Jesús escuchó con paciencia y comprensión a este maestro griego, admitiendo la verdad de las muchas de las cosas que decía pero señalándole, cuando había acabado de hablar, que no había conseguido explicar en su exposición sobre la existencia humana el “de dónde, por qué y hacia dónde”, y añadió: “Donde tú lo dejas, comenzamos nosotros. La religión es una revelación dada al alma humana, que trata de unas realidades espirituales que la mente, por sí misma, jamás podría descubrir ni entender por completo. Los afanes intelectuales pueden revelar los hechos de la vida, pero el evangelio del reino desvela las \bibemph{verdades} del ser. Has hablado de las sombras materiales de la verdad; ¿querrás ahora escuchar mientras te digo acerca de las realidades eternas y espirituales que proyectan estas sombras temporales pasajeras de los hechos materiales y de la existencia mortal?”. Durante más de una hora, Jesús enseñó a este griego las verdades salvadoras del evangelio del reino. El anciano filósofo fue receptivo al modo de hablar del Maestro y, siendo de corazón honesto y sincero, creyó de inmediato en este evangelio de salvación.
\vs p146 3:2 Los apóstoles estaban algo desconcertados por la facilidad con la que Jesús aceptaba muchas de las afirmaciones del griego, aunque, más tarde, Jesús les dijo en privado: “Hijos míos, no os asombréis por mi tolerancia hacia la filosofía del griego. La verdadera y genuina certeza interior no teme en lo más mínimo el análisis externo ni la verdad se resiente ante la crítica cabal. No olvidéis que la intolerancia es la máscara que encubre las dudas ocultas sobre la verdad de las propias creencias. Nunca se debe uno inquietar por la actitud de su prójimo cuando alberga confianza absoluta en la verdad de lo que sin reservas cree. La valentía es la confianza en la profunda honestidad de lo que se profesa creer. El hombre que es sincero no teme al examen crítico de sus verdaderas convicciones ni de sus nobles ideales”.
\vs p146 3:3 \pc La segunda noche en Ramá, Tomás le hizo a Jesús la siguiente pregunta: “Maestro ¿cómo puede un nuevo creyente de tus enseñanzas saber realmente, estar realmente cierto, de la verdad de este evangelio del reino?”.
\vs p146 3:4 Jesús respondió a Tomás: “Vuestra certeza de que habéis entrado en la familia del reino del Padre y de que sobreviviréis eternamente con los hijos del reino es enteramente una cuestión de experiencia personal ---la fe en la palabra de la verdad---. La certeza espiritual es el equivalente de vuestra experiencia religiosa personal de las realidades eternas de la verdad divina y es, dicho de otro modo, igual a vuestro inteligente entendimiento de las realidades de la verdad, más vuestra fe espiritual y menos vuestras honestas dudas.
\vs p146 3:5 “El Hijo está naturalmente dotado de la vida del Padre. Y al haber sido dotados del espíritu vivo del Padre, sois, pues, hijos de Dios. Sobrevivís a vuestra vida en el mundo material de la carne porque estáis identificados con el espíritu vivo del Padre, el don de la vida eterna. Muchos, ciertamente, ya tenían esta vida antes de que yo viniera del Padre, y muchos más recibieron este espíritu porque creyeron en mis palabras; pero yo os declaro que, cuando regrese al Padre, él enviará su espíritu al corazón de todos los hombres.
\vs p146 3:6 “Aunque no podéis observar la labor del espíritu divino en vuestras mentes, hay un modo práctico de descubrir el grado en el que habéis cedido la dirección de los poderes de vuestra alma a las enseñanzas y a las directrices de este espíritu morador del Padre celestial, y se trata del grado de amor que sentís por vuestros semejantes. Este espíritu del Padre participa del amor del mismo Padre y, conforme rige al hombre, lo lleva invariablemente hacia la adoración divina y hacia la amorosa consideración de vuestros semejantes. En un principio, creéis que sois hijos de Dios porque mis enseñanzas os han hecho más conscientes de que estáis internamente guiados por la presencia en vosotros de nuestro Padre; pero pronto se derramará sobre toda carne el espíritu de la verdad, y vivirá entre los hombres e impartirá su enseñanza a todos los hombres, al igual que yo vivo ahora entre vosotros y os hablo las palabras de la verdad. Y este espíritu de la verdad, que representa los dones de vuestra alma, os ayudará a saber que sois hijos de Dios. Dará testimonio de forma indefectible con la presencia interior del Padre, vuestro espíritu, que morará entonces en todos los hombres así como ahora lo hace en algunos, diciéndoos que sois realmente hijos de Dios.
\vs p146 3:7 “Cualquier hijo de la tierra que siga la guía de este espíritu acabará por conocer la voluntad de Dios, y aquel que se rinda a la voluntad de mi Padre, vivirá para siempre. El camino que va desde la vida terrenal hasta la heredad eterna no se os ha revelado, pero hay un camino, que siempre ha existido, y yo he venido para abríroslo nuevo y vivo. Quien entra en el reino ya tiene vida eterna ---nunca perecerá---. Pero entenderéis mejor muchas de estas cosas cuando yo haya regresado a mi Padre y podáis echar una mirada atrás a vuestras experiencias actuales”.
\vs p146 3:8 Y todo aquel que oyó estas gloriosas palabras tuvieron gran regocijo. Respecto a la supervivencia de los rectos, las enseñanzas judías habían sido confusas y vagas y, para los seguidores de Jesús, resultaba reconfortante y alentador oír estas palabras tan claras y ciertas que aseguraban la supervivencia eterna de todo creyente sincero.
\vs p146 3:9 \pc Los apóstoles continuaron predicando y bautizando a los creyentes, mientras continuaban con la costumbre de ir de casa en casa, confortando a los abatidos y atendiendo al enfermo y al afligido. El cuerpo apostólico se había ampliado por el hecho de que cada uno de los apóstoles de Jesús tenía ahora, como acompañante, a uno de los de Juan. Abner era el compañero de Andrés; y esta forma de organizarse prevaleció hasta que fueron a Jerusalén para la siguiente Pascua.
\vs p146 3:10 \pc Durante su estancia en Zabulón, la instrucción especial dada por Jesús consistió, principalmente, en una ampliación de charlas previas sobre las obligaciones mutuas del reino e incluía enseñanzas encaminadas a precisar las diferencias entre la experiencia religiosa personal y la concordia en las obligaciones religiosas sociales. Se trataba de una de las escasas ocasiones en las que el Maestro les hablaría sobre los aspectos sociales de la religión. Durante toda su vida en la tierra, Jesús dio poca instrucción a sus seguidores acerca de la socialización de la religión.
\vs p146 3:11 La población de Zabulón era una raza mixta, ni judía ni gentil, y pocos de ellos realmente creyeron en Jesús, pese a haber oído de la curación de los enfermos en Cafarnaúm.
\usection{4. EL EVANGELIO EN IRÓN}
\vs p146 4:1 En Irón, al igual que en muchas de las ciudades incluso más pequeñas de Galilea y Judea, había una sinagoga y, durante las primeras épocas de su ministerio, Jesús acostumbraba hablar en ellas el día del \bibemph{sabbat.} A veces, lo hacía en el servicio matutino, y Pedro o alguno de los otros apóstoles predicaban por la tarde. Jesús y los apóstoles también a menudo predicaban y enseñaban durante los días de la semana en las asambleas nocturnas de la sinagoga. Aunque la hostilidad de los líderes religiosos de Jerusalén hacia Jesús se había incrementado, estos no ejercían un control directo sobre las sinagogas que estaban fuera de Jerusalén. No fue hasta más tarde en el ministerio público de Jesús cuando estos líderes pudieron crear tal corriente de opinión en su contra, que provocaron el cierre casi total de las sinagogas a sus enseñanzas. Si bien, en este momento todas las sinagogas de Galilea y Judea estaban abiertas para él.
\vs p146 4:2 En aquellos días, había en Irón importantes minas de minerales y, dado que Jesús nunca había compartido la vida de un minero, pasó en ellas una gran parte de su tiempo de estancia en dicha población. Mientras los apóstoles visitaban hogares y predicaban en lugares públicos, Jesús trabajaba en las minas, en el subsuelo, con estos obreros. La fama de Jesús como sanador se había extendido incluso hasta esta remota aldea y numerosos enfermos y afligidos buscaron ayuda de su parte, y muchos se beneficiaron en gran medida de su labor curativa. Pero el Maestro, en ninguno de estos casos, salvo en el del leproso, obró los denominados milagros.
\vs p146 4:3 \pc Avanzada ya la tarde de su tercer día en Irón, al regresar Jesús de las minas, pasó casualmente por una angosta calle secundaria en dirección al lugar en el que se alojaba. Conforme se aproximaba a la mísera vivienda de cierto leproso, este enfermo, que había oído de su fama como sanador, se atrevió a abordarle cuando pasaba por su puerta, diciendo mientras se arrodillaba ante él: “Señor, Señor, si quieres, puedes limpiarme. He oído el mensaje de tus instructores y quisiera entrar al reino si pudiera quedar limpio”. Y el leproso habló de esta manera porque, entre los judíos, se les prohibía a los leprosos incluso asistir a la sinagoga o participar en cultos públicos de oración. Este hombre creía, verdaderamente, que, a menos que curara su lepra, no lo recibirían en el reino venidero. Y cuando Jesús lo vio en su aflicción y oyó sus palabras en las que se aferraba a la poca fe que le quedaba, su corazón humano se conmovió, y la mente divina sintió compasión. Al mirarlo Jesús, el hombre se postró sobre su rostro y adoró. Entonces el Maestro extendió la mano y lo tocó, diciendo: “Quiero, sé limpio”. Y al instante se curó; y ya no estaba aquejado de lepra.
\vs p146 4:4 Cuando Jesús lo levantó sobre sus pies, le mandó: “Mira, no lo digas a nadie, sino ve tranquilo a tus asuntos, muéstrate al sacerdote y ofrece por tu purificación lo que Moisés mandó para testimonio a ellos”. Pero este hombre no hizo lo que Jesús le había ordenado. Por el contrario, comenzó a divulgar a toda la población que Jesús le había curado de la lepra y, puesto que todos lo conocían en la aldea, pudieron comprobar claramente que estaba limpio de su enfermedad. No fue a los sacerdotes como Jesús le había advertido. Como resultado de la difusión de la noticia de esta curación, el Maestro se sintió tan acosado por los enfermos que se vio forzado a levantarse temprano al día siguiente y abandonar la aldea. Aunque Jesús no entró de nuevo en la ciudad, permaneció dos días en las afueras, cerca de las minas, y continuó instruyendo a los mineros creyentes sobre el evangelio del reino.
\vs p146 4:5 Esta limpieza del leproso fue el primero de los llamados milagros, de los que Jesús había obrado de forma intencionada y deliberada hasta ese momento. Y se trataba de un auténtico caso de lepra.
\vs p146 4:6 \pc Desde Irón se dirigieron a Giscala y pasaron allí dos días proclamando el evangelio. Luego, salieron para Corazín donde estuvieron casi una semana predicando la buena nueva, aunque sin conseguir ganar muchos creyentes para el reino. De todos los lugares en los que había enseñado, Jesús jamás había encontrado una población como esta que rechazara su mensaje de forma tan generalizada. La estancia en Corazín fue bastante deprimente para la mayoría de los apóstoles, y a Andrés y a Abner les resultó difícil lograr que sus compañeros mantuviesen el ánimo. Y así, atravesando discretamente Cafarnaúm, fueron a la aldea de Madón, y no les fue mejor. En las mentes de gran parte de los apóstoles, reinaba la idea de que no habían tenido éxito en estas ciudades recientemente visitadas debido a la insistencia de Jesús en que se abstuvieran en sus enseñanzas y predicaciones de referirse a él como sanador. ¡Cuánto deseaban que limpiara a otro leproso de su enfermedad o que manifestara de algún otro modo su poder para atraer así la atención de la gente! Pero el Maestro se mantuvo impasible ante sus entusiastas sugerencias.
\usection{5. DE VUELTA EN CANÁ}
\vs p146 5:1 El grupo apostólico se alegró sobremanera cuando Jesús anunció: “Mañana vamos a Caná”. Sabían que en Caná tendrían un público favorable porque era muy conocido allí. Y estaban realizando bien su labor de atraer a la gente al reino, cuando, al tercer día, llegó a Caná un prominente ciudadano de Cafarnaúm, de nombre Tito, que era un creyente poco convencido, y cuyo hijo estaba gravemente enfermo. Había oído que Jesús estaba en Caná, y se había apresurado a verlo. En Cafarnaúm, los creyentes pensaban que Jesús podía curar cualquier enfermedad.
\vs p146 5:2 Cuando este noble encontró a Jesús en Caná, le rogó que se dirigiera de prisa a Cafarnaúm para curar a su hijo enfermo. Mientras los apóstoles lo rodeaban expectantes, con la respiración entrecortada, Jesús, mirando al padre del muchacho, dijo: “¿Hasta cuándo habré de soportaros? El poder de Dios está en medio de vosotros, pero si no veis señales y prodigios, no creeréis”. Pero el noble suplicó a Jesús: “Mi Señor, yo sí creo, pero ven antes de que mi hijo perezca, porque cuando le dejé ya estaba a punto de morir”. Y una vez que Jesús inclinara la cabeza por un momento en silenciosa meditación, dijo de repente: “Vuelve a casa; tu hijo vivirá”. Tito creyó la palabra de Jesús y se apresuró a regresar a Cafarnaúm. Y, mientras lo hacía, sus siervos salieron a su encuentro, diciendo: “Regocíjate, pues tu hijo ha mejorado y vive”. Entonces, Tito les preguntó a qué hora había comenzado a mejorar, y cuando los siervos le respondieron, “ayer hacia la hora séptima se le pasó la fiebre”, el padre entendió que sobre aquella hora Jesús le había dicho: “Tu hijo vivirá”. Y, desde ese momento, Tito creyó con todo su corazón, y con él toda su familia. Este hijo llegó a ser un gran servidor del reino y, más tarde, sacrificó su vida con los que sufrían en Roma. Aunque toda la casa de Tito, sus amigos e incluso los apóstoles vieron ese suceso como un milagro, no lo fue. Al menos, no fue un milagro de curación de una enfermedad física. Se trató simplemente de un caso de preconocimiento del curso de la ley natural, justo el tipo de conocimiento al que Jesús recurrió a menudo tras su bautismo.
\vs p146 5:3 Jesús, de nuevo, se vio obligado a salir apresuradamente de Caná debido a la indebida atención que se le había dado al segundo suceso de esta clase, ocurrido durante su ministerio en esta aldea. Los lugareños recordaban el caso del agua y el vino, y, ahora, que se suponía que había curado al hijo del noble desde una gran distancia, vinieron a él, no solamente trayendo a los enfermos y afligidos, sino que también le enviaban mensajeros pidiendo que curara a los dolientes a distancia. Y, cuando Jesús vio que toda el área rural estaba en estado de sobreexcitación, dijo: “Marchémonos a Naín”.
\usection{6. NAÍN Y EL HIJO DE LA VIUDA}
\vs p146 6:1 Esta gente creía en los signos; era una generación ansiosa de prodigios. Hacia esta época, la gente de la Galilea central y meridional veía a Jesús y su ministerio personal desde esta propensión a los milagros. Cientos de personas honestas, que sufrían de alteraciones puramente nerviosas y estaban afligidas por trastornos emocionales venían ante la presencia de Jesús, y luego regresaban a sus casas anunciando a sus amigos que Jesús los había sanado. Y estas personas ignorantes y de mente sencilla consideraban estos casos de curación mental como curación física, como curas milagrosas.
\vs p146 6:2 \pc Cuando Jesús salió de Caná para dirigirse a Naín, una gran multitud de creyentes y curiosos los siguió. Estaban empeñados en presenciar milagros y prodigios, y no iban a verse decepcionados. Cuando llegaron cerca de la puerta de la ciudad, Jesús y sus apóstoles se encontraron con un cortejo fúnebre de camino a un cementerio de las inmediaciones, que llevaba al hijo único de una madre viuda de Naín. Esta mujer era muy respetada y la mitad de la aldea seguía a los portadores del féretro con el cuerpo aparentemente sin vida de este muchacho. Cuando el cortejo pasó por donde estaban Jesús y sus seguidores, la viuda y sus amigos reconocieron al Maestro y le suplicaron que le devolviera la vida al hijo. Su expectativa ante los milagros de Jesús era tanta que pensaban que él podía curar cualquier enfermedad humana y, ¿por qué no podría un sanador así incluso resucitar a los muertos? Jesús, sintiéndose abrumado por las insistentes súplicas, se adelantó y, levantando la tapa del ataúd, examinó al muchacho. Al descubrir que el joven no estaba realmente muerto, se dio cuenta de la tragedia que su presencia podía impedir, por lo que, volviéndose a la madre, le dijo: “No llores. Tu hijo no está muerto; está dormido. Volverá a ti”. Y, tomando al joven de la mano, le dijo: “Despierta y levántate”. Y el joven, que supuestamente estaba muerto, al momento se incorporó y comenzó a hablar, y Jesús los envió de vuelta a sus casas.
\vs p146 6:3 Jesús trató de calmar a la multitud, e intentó, en vano, explicarles que el joven no estaba realmente muerto, que él no lo había hecho regresar de la tumba; pero era inútil. El ánimo de la multitud que lo seguía, y de toda la aldea de Naín, se soliviantó hasta llegar al más alto grado de frenesí. Muchos fueron presa del miedo, otros del pánico, mientras que otros empezaron a orar y a lamentarse por sus pecados. Y no fue hasta mucho después del anochecer cuando pudo por fin lograrse que la ruidosa multitud se dispersara. Y, evidentemente, a pesar de que Jesús les había dicho que el muchacho no estaba muerto, todos insistían en que se había obrado un milagro y que incluso se había resucitado a un muerto. Aunque Jesús les dijo que el muchacho estaba simplemente sumido en un sueño profundo, explicaron que aquella era su forma de hablar e hicieron notar el hecho de que él siempre, con su gran modestia, procuraba ocultar sus milagros.
\vs p146 6:4 Y se extendió la voz por toda Galilea y Judea de que Jesús había resucitado de la muerte al hijo de la viuda, y muchos de los que oyeron este relato, creyeron en su veracidad. Jesús no fue capaz de hacer entender, ni incluso a sus propios apóstoles, que el hijo de la viuda no estaba en verdad muerto cuando le mandó que se despertara y levantara. Pero ciertamente a ellos les causó tal impresión que lo omitirían en todos los textos posteriores, salvo Lucas, que dejó constancia de este hecho tal como le había sido relatado a él. Y Jesús, viéndose de nuevo tan asediado por su fama como médico, que al día siguiente, temprano, partió para Endor.
\usection{7. EN ENDOR}
\vs p146 7:1 En Endor, durante unos días, Jesús logró evadirse de las ruidosas multitudes que buscaban la curación física. Durante su estancia en este lugar, el Maestro relató, a fin de instruir a los apóstoles, la historia del rey Saúl y de la bruja de Endor. Manifestó con claridad a sus apóstoles que los seres intermedios descarriados y rebeldes, que con frecuencia se habían hecho pasar por los supuestos espíritus de los muertos, estarían pronto bajo control y no podrían hacer nunca más estas extrañas cosas. Les dijo a sus seguidores que, tras su regreso al Padre, y una vez que se derramase su espíritu sobre toda carne, ya no volverían estos seres semiespirituales ---los llamados espíritus impuros--- a poseer a aquellos mortales de mente débil y malévola.
\vs p146 7:2 Asimismo, Jesús explicó a sus apóstoles que los espíritus de los seres humanos muertos no retornan al mundo de donde proceden para comunicarse con sus semejantes vivos. Al espíritu en avance de un hombre mortal solo le sería posible volver a la tierra tras el cierre de una era de dispensación e, incluso así, solo en casos excepcionales, y como miembro del gobierno espiritual del planeta.
\vs p146 7:3 Tras descansar dos días, Jesús dijo a sus apóstoles: “Mañana regresaremos a Cafarnaúm y nos quedaremos enseñando mientras que las zonas rurales se aquietan. Allí, de dónde venimos, ya se habrán recuperado prácticamente de esta clase de conmoción.
