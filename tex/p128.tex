\upaper{128}{Edad adulta temprana de Jesús}
\author{Comisión de seres intermedios}
\vs p128 0:1 Al comenzar los primeros años de su edad adulta, Jesús de Nazaret había vivido, y continuaba viviendo, una vida humana normal y corriente en la tierra. Jesús vino a este mundo exactamente como viene cualquier otro niño; no tuvo nada que ver con la elección de sus padres. Sí escogió este mundo concreto como el planeta en el que llevaría a cabo su séptimo y último ministerio de gracia, su encarnación con la semejanza de un hombre mortal; pero, por lo demás, vino al mundo de una manera natural, creciendo como un niño más del planeta y luchando contra las vicisitudes de su entorno tal como lo hacen los demás mortales en este y en otros mundos similares.
\vs p128 0:2 Tened siempre presente el doble propósito del ministerio de gracia de Miguel en Urantia:
\vs p128 0:3 \li{1.}Dominar la experiencia de vivir plenamente la vida de una criatura humana en carne mortal, culminar su soberanía de Nebadón.
\vs p128 0:4 \li{2.}Revelar a los habitantes mortales de los mundos del tiempo y del espacio al Padre Universal y guiar con mayor eficacia a estos mismos mortales a una mejor comprensión de él.
\vs p128 0:5 \pc Todos los demás beneficios para las criaturas y ventajas para el universo eran adicionales y secundarios ante estos principales propósitos de su ministerio de gracia como mortal.
\usection{1. SU VIGÉSIMO PRIMER AÑO (AÑO 15 D. C.)}
\vs p128 1:1 Al alcanzar la edad adulta, Jesús se dispuso seriamente y con plena conciencia a la labor de completar su aprendizaje de la vida de las formas más humildes de sus criaturas inteligentes, consiguiendo así, de modo final y plenamente, su derecho a gobernar de manera incondicional el universo por él mismo creado. Emprendió esta formidable tarea enteramente consciente de su doble naturaleza. Si bien, él ya había combinado eficazmente estas dos naturalezas en una sola: la de Jesús de Nazaret.
\vs p128 1:2 Josué ben José sabía perfectamente que era un hombre, un hombre mortal, nacido de mujer. Esto se demuestra al elegir \bibemph{Hijo del Hombre} como su primer apelativo. Verdaderamente participó de la carne y de la sangre e, incluso ahora, cuando preside con autoridad soberana los destinos de un universo, todavía ostenta entre sus numerosos y bien merecidos títulos el de Hijo del Hombre. Es literalmente cierto que el Verbo creador ---el Hijo Creador--- del Padre Universal “se hizo carne y habitó en Urantia como un hombre del mundo”. Trabajaba, se cansaba, descansaba y dormía. Tuvo hambre y satisfizo su apetito con alimentos; tuvo sed y sació su sed con agua. Experimentó toda clase de sentimientos y emociones humanas; fue “tentado en todo según vuestra semejanza”, y sufrió y murió.
\vs p128 1:3 Adquirió conocimientos, acumuló experiencia y los aunó para su propia sabiduría, como hacen otros mortales del mundo. Hasta después de su bautismo no recurrió a ningún poder sobrenatural. No utilizó ningún medio que no formara parte de sus dotes humanas como hijo de José y de María.
\vs p128 1:4 Se despojó de los atributos inherentes a su existencia prehumana. Con anterioridad al comienzo de su obra pública, se limitó a sí mismo por completo en cuanto al conocimiento de los acontecimientos y de los hombres. Fue un verdadero hombre entre hombres.
\vs p128 1:5 \pc Es para siempre una gran verdad que: “Tenemos un sumo gobernante que puede compadecerse de nuestras debilidades. Tenemos un Soberano que en todas las cosas fue tentado y probado como nosotros, solo que él no pecó”. Y puesto que él mismo ha sufrido, siendo tentado y probado, es sobradamente capaz de comprender y asistir a los que se encuentran confundidos y angustiados.
\vs p128 1:6 \pc El carpintero de Nazaret comprendía ahora plenamente la labor que tenía ante sí, pero decidió vivir la vida humana siguiendo su cauce natural. Y, en algunas de estas cuestiones, es realmente un ejemplo para sus criaturas mortales, pues tal como está escrito: “Haya en vosotros este sentir que hubo también en Cristo Jesús, el cual, siendo de la naturaleza de Dios, no estimaba extraño ser igual a Dios. Pero se restó importancia y, tomando la forma de una criatura, nació con semejanza a la humanidad. Y apareciendo pues en su porte como hombre, se humilló a sí mismo y se hizo obediente hasta la muerte, incluso a la muerte de cruz”.
\vs p128 1:7 Vivió su vida mortal tal como la de los demás miembros de la familia humana pueden vivir la suya, como “aquel que en los días en la carne ofreció con tanta asiduidad oraciones y súplicas, con gran clamor y lágrimas, a Aquel que lo podía librar de todo mal, y sus oraciones fueron oídas porque creía”. Por lo cual debía ser \bibemph{en todo} semejante a sus hermanos, para venir a ser un soberano misericordioso y comprensivo para ellos.
\vs p128 1:8 Nunca dudó de su naturaleza humana; era manifiesta y siempre estaba presente en su conciencia. Pero en cuanto a su naturaleza divina, siempre albergó dudas e hizo conjeturas; al menos esto fue así justo hasta el momento de su bautismo. La toma de conciencia de su divinidad fue un proceso lento y, desde la perspectiva humana, constituyó una revelación que tuvo un desarrollo natural. Esta revelación y toma de conciencia de su divinidad empezaron en Jerusalén con el primer suceso sobrenatural de su existencia humana, cuando aún no tenía trece años de edad; y esta experiencia de llevar a cabo la autoconciencia de su naturaleza divina se completó en el momento del segundo hecho sobrenatural experimentado, mientras estaba en la carne, en conjunción con el bautismo por parte de Juan, que marcó el comienzo de su andadura pública de servicio y enseñanza.
\vs p128 1:9 Entre estas dos visitas celestiales, una en su decimotercero año de vida y la otra en su bautismo, no ocurrió nada sobrenatural ni sobrehumano en la vida de este hijo creador encarnado. No obstante, él, el niño de Belén, el muchacho, el joven y el hombre de Nazaret, era en realidad el creador de todo un universo; si bien, durante su vida humana hasta el día en que Juan lo bautizó, no utilizó ni una sola vez este poder ni se sirvió de la guía de seres personales celestiales, aparte de la de su serafín guardián. Y nosotros que damos testimonio de esto, sabemos de lo que hablamos.
\vs p128 1:10 Y, sin embargo, a lo largo de todos estos años de su vida en la carne, era verdaderamente divino. Era de hecho un hijo creador del Padre del Paraíso. Una vez que había emprendido su andadura pública, tras completar oficialmente su experiencia puramente mortal para adquirir la soberanía, no vaciló en admitir públicamente que era el Hijo de Dios. No dudó en declarar: “Yo soy el Alfa y la Omega, el principio y el fin, el primero y el último”. Años más tarde, no expresó su disconformidad cuando lo llamaron Señor de la Gloria, Gobernante de un universo, el Señor Dios de toda la creación, el Santo de Israel, el Señor de todo, nuestro Señor y nuestro Dios, Dios con nosotros, el que tiene un nombre sobre todo nombre y en todos los mundos, la Omnipotencia de un universo, la Mente Universal de esta creación, el Único en quien están ocultos todas las riquezas de la sabiduría y del conocimiento, la plenitud de Aquel que llena todas las cosas, el Verbo eterno del Dios eterno, Aquel que era antes de todas las cosas y en quien todas las cosas consisten, el Creador de los cielos y de la tierra, el Sustentador de un universo, el Juez de toda la tierra, el Dador de la vida eterna, el Verdadero Pastor, el Libertador de los mundos y Quien encabeza nuestra salvación.
\vs p128 1:11 \pc Nunca puso objeciones a ninguno de estos títulos cuando se lo aplicaron tras emerger de su vida puramente humana y entrar en los años siguientes en los que ya era autoconsciente de su ministerio divino en la humanidad, y por la humanidad, y para la humanidad en este mundo y para los demás mundos. Jesús solo puso objeción a un título que le aplicaron. Cuando cierta vez lo llamaron Emanuel, simplemente replicó: “No soy yo; es mi hermano mayor”.
\vs p128 1:12 Jesús siempre, incluso tras emerger a una vida de mayor preponderancia en la tierra, se sometió sumisamente a la voluntad del Padre que está en el cielo.
\vs p128 1:13 Después de su bautismo, Jesús no tuvo inconveniente en permitir que lo reverenciaran sus creyentes sinceros y sus seguidores agradecidos. Incluso cuando luchaba contra la pobreza y trabajaba arduamente con sus manos para proporcionar las necesidades básicas de la vida a su familia, su conciencia de ser un Hijo de Dios iba en aumento; sabía que era el hacedor de los cielos y de aquella misma tierra en la que estaba ahora viviendo su existencia humana. Y las multitudes de seres celestiales de todo el inmenso universo que le contemplaban sabían igualmente que este hombre de Nazaret era su amado soberano y su creador\hyp{}padre. En todos estos años, en el universo de Nebadón se instaló una profunda expectación; continuamente todas las miradas celestiales se centraban en Urantia: en Palestina.
\vs p128 1:14 \pc Este año, Jesús se dirigió a Jerusalén con José para celebrar la Pascua. Al haber llevado ya a Santiago al templo para su consagración, consideró que era su deber hacer lo mismo con José. Jesús nunca mostró el menor grado de favoritismo en el trato con su familia. Fue con José a Jerusalén tomando la ruta acostumbrada del valle del Jordán, pero volvió a Nazaret por el camino que pasaba por Amatus, al este del Jordán. Al bajar por el Jordán, Jesús relató a José la historia de los judíos y, en el viaje de regreso, le habló de las experiencias de las renombradas tribus de Rubén, Gad y Gilead que tradicionalmente habían vivido en estas regiones situadas al este del río.
\vs p128 1:15 José hizo muchas preguntas a Jesús con la intención de que hablara de su misión de vida, pero a la mayoría de ellas, Jesús solo le contestaba: “Aún no ha llegado mi hora”. Sin embargo, durante estas conversaciones de índole personal, Jesús dejó deslizar muchas palabras que José recordó durante los agitados sucesos de los siguientes años. Jesús, junto con José, pasó esta Pascua en Betania, con sus tres amigos, como acostumbraba a hacer cuando asistía en Jerusalén a estas fiestas conmemorativas.
\usection{2. SU VIGÉSIMO SEGUNDO AÑO (AÑO 16 D. C.)}
\vs p128 2:1 Este fue uno de esos años en el que los hermanos y hermanas de Jesús se enfrentaban a las pruebas y tribulaciones propias de las dificultades y reajustes que tienen lugar en la adolescencia. Jesús tenía ahora hermanos y hermanas con edades comprendidas entre los siete y los dieciocho años de edad, y se mantenía atareado ayudándoles a adaptarse al nuevo despertar de su vida intelectual y emocional. Así pues, tuvo que abordar los problemas de la adolescencia a medida que se iban presentando en las vidas de sus hermanos y hermanas menores.
\vs p128 2:2 Este año, Simón se graduó de la escuela y empezó a trabajar con Jacob, el albañil, el antiguo compañero de juegos de la niñez de Jesús y su siempre dispuesto defensor. Como resultado de varias conversaciones familiares, se decidió que no era aconsejable que todos los muchachos se dedicaran a la carpintería. Pensaron que si diversificaban sus oficios estarían en posición de hacer contratos para construir edificios enteros. Además, desde que tres de ellos habían trabajado como carpinteros a tiempo completo, no todos habían estado ocupados.
\vs p128 2:3 Jesús continuó este año con el acabado de interiores y la ebanistería, pero pasaba la mayor parte de su tiempo en la tienda de reparaciones de las caravanas. Santiago estaba empezando a alternarse con él para atenderla. En la última parte del año, cuando el trabajo de carpintería escaseaba en Nazaret, Jesús dejó a Santiago a cargo de la tienda y a José en el banco de carpintero de la casa, mientras que él se fue a Séforis para trabajar con un herrero. Estuvo trabajando seis meses en el metal, adquiriendo una considerable destreza en el yunque.
\vs p128 2:4 \pc Antes de ocupar su nuevo empleo en Séforis, Jesús mantuvo una de sus periódicas charlas familiares y erigió solemnemente a Santiago, que acababa de cumplir dieciocho años, como jefe provisional de la familia. Prometió a su hermano un apoyo fiel y su plena cooperación, y exigió de cada uno de los miembros de esta la promesa formal de obedecer a Santiago. A partir de este día, Santiago asumió toda la responsabilidad de los asuntos económicos de la familia; Jesús entregaba a su hermano su paga semanal. Nunca más se hizo con las riendas del hogar, ya en manos de Santiago. Mientras trabajaba en Séforis podría haber regresado caminando todas las noches a su casa si hubiera sido necesario, pero permaneció alejado deliberadamente, atribuyendo al tiempo y a otras razones el no hacerlo; si bien, su verdadero motivo era formar a Santiago y a José para que pudieran hacerse responsables de la familia. Había comenzado el lento proceso de separación de su familia. Jesús volvía a Nazaret cada \bibemph{sabbat} y, a veces, durante la semana cuando la ocasión lo requería, para observar el funcionamiento del nuevo plan, dar consejos y ofrecer sugerencias útiles.
\vs p128 2:5 \pc El vivir la mayoría del tiempo en Séforis durante seis meses proporcionó a Jesús una nueva oportunidad para llegar a familiarizarse mejor con el punto de vista de los gentiles respecto a la vida. Trabajó con ellos, vivió con ellos y, de todas las formas posibles, hizo un minucioso y concienzudo examen de sus hábitos de vida y su mentalidad.
\vs p128 2:6 Los valores morales de esta ciudad de residencia de Herodes Antipas eran inferiores incluso a los de Nazaret, la ciudad de las caravanas, que tras una estancia de seis meses en Séforis, Jesús no tuvo inconveniente en buscar un pretexto para regresar a Nazaret. La empresa para la que trabajaba iba a acometer unas obras públicas tanto en Séforis como en la nueva ciudad de Tiberíades, y Jesús era reticente a tener cualquier tipo de empleo que estuviera bajo la supervisión de Herodes Antipas. Y también existían otras razones que aconsejaban, en opinión de Jesús, regresar a Nazaret. Cuando volvió a la tienda de reparaciones, no asumió de nuevo la dirección personal de los asuntos familiares. Trabajó junto con Santiago en el taller y esto, en la medida de lo posible, le permitió continuar supervisando el hogar. La gestión de los gastos familiares y la administración del presupuesto doméstico, que estaban en manos de Santiago, permanecieron sin cambio.
\vs p128 2:7 Fue precisamente esta planificación, sensata y cuidadosa, la que preparó el camino para que Jesús se retirara finalmente de toda participación activa en los asuntos de su familia. Cuando Santiago llevaba dos años de experiencia como jefe provisional de la familia ---y dos años completos antes de que él (Santiago) contrajera matrimonio---, se designó a José a cargo de los fondos de la casa y se le confió la gestión general del hogar.
\usection{3. SU VIGÉSIMO TERCER AÑO (AÑO 17 D. C.)}
\vs p128 3:1 Este año se distendió levemente la presión económica que soportaban, ya que había cuatro miembros de la familia con trabajo. Miriam obtenía considerables ingresos gracias a la venta de leche y mantequilla; Marta se había convertido en una experta tejedora. Habían pagado más de un tercio del precio de compra de la tienda de reparaciones. La situación era tal que Jesús dejó de trabajar durante tres semanas para llevar a Simón a la Pascua de Jerusalén. Desde la muerte de su padre, se trataba del período más largo que había disfrutado sin quehaceres cotidianos.
\vs p128 3:2 Viajaron a Jerusalén camino de la Decápolis y cruzaron Pella, Gerasa, Filadelfia, Hesbón y Jericó. Regresaron a Nazaret por la ruta de la costa, pasando junto a Lida, Jope, Cesarea y, desde allí, rodeando el Monte Carmelo, se dirigieron a Tolemaida y Nazaret. Este viaje permitió a Jesús conocer bastante bien todo el territorio de Palestina situado al norte de la región de Jerusalén.
\vs p128 3:3 En Filadelfia, Jesús y Simón conocieron a un mercader de Damasco a quien le llegó a agradar tanto la pareja de hermanos de Nazaret que insistió para que se detuvieran con él en su sede de Jerusalén. Mientras Simón asistía al templo, Jesús conversó durante bastante tiempo con este hombre de mundo, bien educado y muy viajado. Este mercader poseía más de cuatro mil camellos de caravana; tenía intereses en todo el mundo romano y ahora estaba de camino a Roma. Le propuso a Jesús que fuese a Damasco para participar en su negocio de importaciones de productos orientales, pero Jesús le explicó que no consideraba su propuesta motivo de justificación para alejarse tanto de su familia en ese momento. Si bien, en el camino de regreso a casa pensó mucho en estas ciudades lejanas y en países incluso más remotos del extremo oeste y del lejano oriente, países de los que tan frecuentemente había oído hablar a los viajeros y a los conductores de las caravanas.
\vs p128 3:4 Simón disfrutó mucho de su visita a Jerusalén. Fue debidamente aceptado en la ciudadanía de Israel durante la consagración pascual de los nuevos hijos del mandamiento. Mientras Simón asistía a las ceremonias pascuales, Jesús se mezcló con las multitudes de visitantes y entabló muchas interesantes conversaciones de orden personal con numerosos prosélitos gentiles.
\vs p128 3:5 El más destacado de todos estos contactos fue quizás el mantenido con un joven helenista llamado Esteban. Este joven visitaba Jerusalén por vez primera y se encontró casualmente con Jesús el jueves por la tarde de la semana de la Pascua. Mientras los dos paseaban contemplando el palacio asmoneo, Jesús comenzó una conversación informal con él que desembocó en un mutuo interés entre ellos, algo que les llevó a una charla de cuatro horas sobre la forma de vivir y el verdadero Dios y su culto. Esteban quedó enormemente impresionado con lo que Jesús le dijo y nunca olvidó sus palabras.
\vs p128 3:6 Se trataba del mismo Esteban que posteriormente se convertiría en creyente de las enseñanzas de Jesús, y cuyo arrojo en la predicación de este temprano evangelio provocaría la ira de los judíos, que le apedrearon hasta la muerte. Una parte de la extraordinaria valentía de Esteban al proclamar su visión del nuevo evangelio fue consecuencia directa de este primer encuentro con Jesús. Pero Esteban nunca supuso en lo más mínimo que el galileo con quien había hablado unos quince años antes era la misma persona que más tarde él daría a conocer como el Salvador del mundo, y por quien iba a morir tan pronto, convirtiéndose así en el primer mártir de la fe cristiana que recientemente empezaba a desarrollarse. Y cuando Esteban dio su vida como precio por su severa crítica contra el templo judío y sus prácticas tradicionales, se hallaba allí un tal Saulo, ciudadano de Tarso. Cuando Saulo vio cómo este griego moría por su fe, se suscitaron en su corazón unos sentimientos que acabaron por llevarle a apoyar la causa por la que Esteban había muerto; más tarde se convirtió en el combativo e indomable Pablo, el filósofo, e incluso en el único fundador de la religión cristiana.
\vs p128 3:7 \pc El domingo tras la semana de Pascua, Simón y Jesús emprendieron su viaje de vuelta a Nazaret. Simón nunca olvidaría lo que Jesús le había enseñado en este viaje. Siempre lo había amado, pero ahora sentía que había empezado a conocer a su padre\hyp{}hermano. Tuvieron muchas charlas íntimas mientras recorrían el país y preparaban sus comidas al borde del camino. Llegaron a casa el jueves al mediodía y, esa noche, Simón mantuvo despierta a la familia hasta tarde relatándoles sus experiencias.
\vs p128 3:8 María se disgustó mucho cuando Simón le informó de que Jesús había pasado la mayoría del tiempo en Jerusalén “conversando con los extranjeros, particularmente con aquellos procedentes de países lejanos”. La familia de Jesús nunca pudo comprender su gran interés por las personas, su afán por hablar con ellos, por conocer su manera de vivir y por averiguar lo que pensaban.
\vs p128 3:9 \pc La familia de Nazaret estaba cada vez más abstraída en sus problemas inmediatos y humanos; con poca frecuencia, se mencionaba la futura misión de Jesús, y él mismo escasamente hablaba de su andadura futura. Su madre raramente se planteaba que era un hijo de la promesa. Iba lentamente renunciando a la idea de que Jesús tenía que cumplir una misión divina en la tierra, pero a veces, cuando se detenía a recordar la visitación de Gabriel antes de que el niño naciera, se reavivaba su fe.
\usection{4. EL EPISODIO DE DAMASCO}
\vs p128 4:1 Jesús pasó los cuatro últimos meses de este año en Damasco como invitado del mercader que conoció por primera vez en Filadelfia, yendo de camino a Jerusalén. Un representante de este mercader había buscado a Jesús a su paso por Nazaret y lo acompañó hasta Damasco. El mercader, en parte judío, pensaba dedicar una extraordinaria suma de dinero a la institución de una escuela de filosofía religiosa en Damasco. Tenía la intención de crear un centro de estudios que rivalizara con el de Alejandría. Y le propuso a Jesús que comenzara de inmediato un largo periplo por los centros educativos del mundo, como paso preliminar para convertirse en el director de este nuevo proyecto. Esta fue una de las mayores tentaciones que Jesús tuvo que afrontar en el curso de su andadura puramente humana.
\vs p128 4:2 Al poco tiempo, este mercader llevó ante Jesús a un grupo de doce mercaderes y banqueros que acordaron financiar el proyecto de esta nueva escuela. Jesús manifestó su profundo interés por la escuela propuesta y los ayudó a planificar su organización, pero siempre expresó el temor de que sus otras obligaciones previas, sin hacer alusión a ellas, le impedirían aceptar la dirección de una empresa tan ambiciosa. Su futuro benefactor era perseverante y empleó provechosamente a Jesús en su casa haciendo algunas traducciones, mientras que él, su esposa y sus hijos e hijas procuraban persuadirlo para que aceptara el honor que se le brindaba. Pero Jesús no lo hizo. Sabía muy bien que su misión en la tierra no debía apoyarse en instituciones de enseñanza; sabía que no debía comprometerse en lo más mínimo a la dirección de las “asambleas de los hombres”, por muy bien intencionadas que fuesen.
\vs p128 4:3 Quien fue rechazado por los líderes religiosos de Jerusalén, incluso después de haber demostrado su autoridad, era reconocido y aclamado como un experimentado maestro por los empresarios y los banqueros de Damasco, y todo esto cuando era un carpintero anónimo y desconocido de Nazaret.
\vs p128 4:4 Él jamás mencionó este ofrecimiento a su familia; y, al final de este año, lo encontramos de nuevo en Nazaret desempeñando sus obligaciones diarias, como si nunca hubiese sido tentado por las halagadoras propuestas de sus amigos de Damasco. Estos hombres tampoco relacionaron al futuro ciudadano de Cafarnaúm, que transformó por completo todo el judaísmo, con aquel carpintero de Nazaret, que se había atrevido a rechazar el honor que la aportación de sus riquezas podía haberle granjeado.
\vs p128 4:5 \pc Jesús procuró con gran inteligencia e intencionadamente aislar diversos episodios de su vida para que, a los ojos del mundo, nunca se llegasen a relacionar como si las hubiese realizado una misma persona. Muchas veces, en años venideros, escuchó narrar la misma historia del extraño galileo que rehusó la oportunidad de fundar en Damasco una escuela que compitiera con Alejandría.
\vs p128 4:6 Al tratar de separar ciertos componentes de su experiencia terrenal, uno de los propósitos que le impulsaba era evitar la conformación de una andadura tan versátil y espectacular, que pudiera llevar a las futuras generaciones a venerar al maestro en lugar de cumplir la verdad que él había vivido y enseñado. Jesús no quería que la recreación de un historial de sus logros pudiese distraer la atención de sus enseñanzas. Comprendió muy pronto que sus seguidores se sentirían tentados de elaborar una religión \bibemph{acerca} de él, que pudiera rivalizar con el evangelio del reino que pretendía proclamar al mundo. Por consiguiente, durante toda su memorable trayectoria de vida, trató sistemáticamente de suprimir todo lo que él pensaba que pudiera contribuir a esta tendencia humana natural de exaltar al maestro, en lugar de afirmar sus enseñanzas.
\vs p128 4:7 Este mismo motivo también justifica por qué permitió que se le conociera con apelativos diferentes durante los distintos períodos por los que atravesó en su variada vida en la tierra. Además, no quería ejercer ninguna influencia indebida sobre su familia u otras personas para no inducirles a creer en él en contra de sus sinceras convicciones personales. Siempre se negó a beneficiarse de forma indebida e injusta de la mente humana. No quería que los hombres creyeran en él a menos que sus corazones se mostrarán receptivos a las realidades espirituales reveladas en sus enseñanzas.
\vs p128 4:8 \pc Hacia finales de este año, todo marchaba bastante bien en el hogar de Nazaret. Los niños crecían y María se iba acostumbrando a que Jesús estuviese lejos de la casa. Él continuaba entregando sus ingresos a Santiago para el mantenimiento de la familia; tan solo se reservaba una pequeña parte para sus gastos personales más inmediatos.
\vs p128 4:9 A medida que pasaban los años, más difícil resultaba ser conscientes de que este hombre era un Hijo de Dios en la tierra. Parecía haberse convertido en una criatura más del mundo, en un hombre entre los hombres. El Padre celestial había ordenado que su ministerio de gracia se desarrollara precisamente de este modo.
\usection{5. SU VIGÉSIMO CUARTO AÑO (AÑO 18 D. C.)}
\vs p128 5:1 Este fue el primer año en el que Jesús estuvo relativamente libre de responsabilidades familiares. Santiago dirigía con gran acierto los asuntos del hogar con la ayuda de los consejos y los ingresos de Jesús.
\vs p128 5:2 \pc A la semana siguiente de la Pascua de este año, un joven de Alejandría llegó a Nazaret para concertar un encuentro entre Jesús y un grupo de judíos de Alejandría que tendría lugar más tarde en el año, en algún lugar de la costa de Palestina. La reunión se fijó para mediados de junio, y Jesús se desplazó hasta Cesarea para encontrarse con cinco judíos prominentes de Alejandría que le rogaron que se estableciera en su ciudad como maestro religioso, ofreciéndole como incentivo, para empezar, el puesto de ayudante del jazán en la sinagoga principal de la ciudad.
\vs p128 5:3 Los portavoces de esta comisión explicaron a Jesús que Alejandría estaba destinada a convertirse en la sede principal de la cultura judía para el mundo entero; que la tendencia helenista de las cuestiones judías se había impuesto prácticamente a la escuela de pensamiento babilónico. Recordaron a Jesús los inquietantes rumores de rebelión que corrían por Jerusalén y por toda Palestina, y le aseguraron que cualquier sublevación de los judíos palestinos equivaldría a un suicidio nacional, que la mano de hierro de Roma aplastaría la rebelión en tres meses, y que Jerusalén sería destruida y el templo derribado hasta que no quedara piedra sobre piedra.
\vs p128 5:4 Jesús escuchó todo lo que tenían que decir, les agradeció su confianza, y, al declinar su invitación para ir a Alejandría, les dijo en esencia: “Aún no ha llegado mi hora”. Se quedaron perplejos por su aparente indiferencia al honor que habían procurado concederle. Antes de despedirse de Jesús, le ofrecieron una suma de dinero como prueba de la estima de sus amigos de Alejandría y en compensación por el tiempo y los gastos de su traslado hasta Cesarea para deliberar con ellos. Pero se negó igualmente a aceptar el dinero, diciendo: “La casa de José nunca ha recibido limosnas, y no podemos comernos el pan de otro siempre que mis brazos estén fuertes y mis hermanos puedan trabajar”.
\vs p128 5:5 Sus amigos de Egipto zarparon para su tierra, y en años posteriores, cuando oyeron rumores sobre el constructor de embarcaciones de Cafarnaúm que tanta conmoción estaba creando en Palestina, pocos de ellos supusieron que se trataba del niño de Belén ya adulto ni del mismo galileo de extraño comportamiento que había declinado, sin muchos formalismos, su invitación para convertirse en un gran maestro en Alejandría.
\vs p128 5:6 \pc Jesús regresó a Nazaret. Los seis meses que restaban de este año fueron los menos accidentados de toda su andadura; disfrutó de un alivio transitorio de su lista habitual de problemas que resolver y de dificultades que superar. Estaba en intensa comunión con su Padre celestial e hizo enormes progresos en el dominio de su mente humana.
\vs p128 5:7 Pero en los mundos del tiempo y del espacio, los asuntos humanos nunca se desarrollan sin dificultades por mucho tiempo. En diciembre, Santiago tuvo una conversación privada con Jesús para explicarle que estaba muy enamorado de Esta, una joven de Nazaret y que, en algún momento, les gustaría casarse si se pudiesen hacer los arreglos necesarios. Jesús señaló el hecho de que José pronto cumpliría dieciocho años, y que sería una buena experiencia para él tener la oportunidad de servir como jefe provisional de la familia. Jesús dio su consentimiento para el matrimonio de Santiago dos años más tarde, siempre que, durante el intervalo de tiempo, capacitara debidamente a José para asumir la dirección del hogar.
\vs p128 5:8 Y entonces empezaron a ocurrir algunas cosas ---la boda estaba en el pensamiento de todos---. El logro conseguido por Santiago al obtener el consentimiento de Jesús para su casamiento alentó a Miriam a plantear sus planes ante su hermano\hyp{}padre. Jacob, el joven albañil, que se había en antaño autoproclamado defensor de Jesús y que era ahora socio comercial de Santiago y José, hacía tiempo que deseaba pedir la mano de Miriam. Después de que Miriam expusiera su idea a Jesús, este le solicitó que Jacob viniera a él para hacer una petición formal y prometió dar su bendición al matrimonio, en cuanto ella estimara que Marta estaba capacitada para asumir sus deberes de hija mayor.
\vs p128 5:9 \pc Mientras estaba en Nazaret, Jesús seguía dando clases en la escuela vespertina tres veces a la semana. Los \bibemph{sabbats,} en la sinagoga, leía a menudo las escrituras, conversaba con su madre, enseñaba a los niños y, en general, se conducía como un digno y respetado ciudadano de Nazaret, perteneciente a la comunidad de Israel.
\usection{6. SU VIGÉSIMO QUINTO AÑO (AÑO 19 D. C.)}
\vs p128 6:1 Este año comenzó con toda la familia de Nazaret en buena salud; en dicho año, se completó la escolarización regular de todos los niños, con la excepción de alguna tarea que Marta debía hacer para Rut.
\vs p128 6:2 \pc Jesús era uno de los hombres más robustos y cultivados que habían aparecido en la tierra desde los tiempos de Adán. Su desarrollo físico era espléndido. Su mente era activa, aguda y penetrante ---había alcanzado colosales proporciones en comparación con la capacidad mental media de sus contemporáneos--- y su espíritu era de cierto humanamente divino.
\vs p128 6:3 \pc La economía familiar se hallaba en las mejores condiciones desde que tuvieron que desprenderse de las propiedades de José. Se habían efectuado los últimos pagos de la tienda de reparaciones de caravanas; no debían nada a nadie y, por primera vez en años, contaban con algunos fondos. Siendo así, y puesto que había llevado a sus otros hermanos a Jerusalén para asistir a sus primeras ceremonias de la Pascua, Jesús decidió acompañar a Judá (que acababa de graduarse de la sinagoga) en su primera visita al templo.
\vs p128 6:4 Fueron a Jerusalén por el valle del Jordán y regresaron por la misma ruta; Jesús temía que surgieran problemas si atravesaba Samaria con su joven hermano. Ya en Nazaret, Judá había tenido a veces algún leve problema, debido a su impetuoso temperamento junto a sus fuertes convicciones patrióticas.
\vs p128 6:5 Llegaron a Jerusalén a su debido tiempo e iban de camino para hacer una primera visita al templo, cuya sola visión había agitado y emocionado a Judá hasta lo más profundo de su alma, cuando se encontraron fortuitamente con Lázaro de Betania. Mientras Jesús charlaba con Lázaro y trataba de hacer los arreglos necesarios para celebrar la Pascua juntos, Judá les causó serias dificultades. A corta distancia de allí había un guardia romano que hizo unos comentarios indebidos a una muchacha judía que pasaba por allí. Judá, preso de una terrible indignación, no tardó en dar rienda suelta a su ira por este impropio comportamiento, dirigiéndose al soldado, que pudo oírlo. Los legionarios romanos eran muy sensibles a todo lo que pudiera constituir una falta de respeto por parte de los judíos y, por tanto, el guardia arrestó a Judá de inmediato. Esto resultó excesivo para el joven patriota y, antes de que Jesús pudiera alertarlo con una mirada de advertencia, ya había prorrumpido en una prolija denuncia de sentimientos antirromanos reprimidos, lo que no hizo sino empeorar la situación. Judá, con Jesús a su lado, fue llevado enseguida a la prisión militar.
\vs p128 6:6 Jesús trató de conseguir ya fuese una audiencia inmediata para Judá o su liberación a tiempo para poder celebrar la Pascua aquella tarde, pero no obtuvo resultado alguno. Puesto que el día siguiente era la “reunión santa” en Jerusalén, ni siquiera los romanos se atrevían a alegar cargos contra un judío. En consecuencia, Judá continuó encarcelado hasta la mañana del segundo día tras su arresto y Jesús se quedó con él en la prisión. No estuvieron presentes en el templo en la ceremonia de recepción de los hijos de la ley a la plena ciudadanía de Israel. Judá no pasaría por este acto formal hasta varios años después, cuando se encontraba de nuevo en Jerusalén durante la Pascua en relación con su labor propagandística a favor de los zelotes, la organización patriótica a la que pertenecía y en la que era muy activo.
\vs p128 6:7 A la mañana siguiente de su segundo día en la cárcel, Jesús compareció ante el juez militar de instrucción en nombre de Judá. Al disculparse por la juventud de su hermano y efectuar además una exposición aclaratoria, pero sensata, sobre el carácter provocativo del incidente que había llevado al arresto de su hermano, Jesús gestionó el caso de tal manera que el juez instructor manifestó la opinión de que el joven judío pudiera haber tenido alguna excusa válida que justificara su estallido de violencia. Después de advertir a Judá que no se permitiera de nuevo cometer tal temeridad, al despedirlos, dijo a Jesús: “Harías bien en vigilar al muchacho; es probable que os cause a todos muchos problemas”. Y el juez romano decía la verdad. Judá de cierto originó bastantes trastornos a Jesús, y siempre eran de la misma naturaleza: enfrentamientos con las autoridades civiles a causa de sus arrebatos patrióticos irreflexivos e imprudentes.
\vs p128 6:8 Jesús y Judá caminaron hasta Betania para pasar la noche; allí explicaron las razones por no haber podido acudir a su cita para la cena de Pascua y, al día siguiente, partieron para Nazaret. Jesús no le dijo nada a su familia sobre el arresto de su joven hermano en Jerusalén, pero unas tres semanas después de su regreso, tuvo una larga conversación con Judá sobre este incidente. Tras esta, el mismo Judá contó lo sucedido a la familia. Nunca olvidaría la paciencia e indulgencia demostradas por su hermano\hyp{}padre durante toda esta difícil experiencia.
\vs p128 6:9 Esta sería la última Pascua a la que Jesús asistiría con un miembro de su propia familia. Cada vez más, el Hijo del Hombre iba separándose de los estrechos vínculos que le unían a los de su propia sangre y carne.
\vs p128 6:10 \pc Este año, sus periodos de profunda meditación se vieron a menudo interrumpidos por Rut y sus compañeros de juego. Y Jesús siempre estaba dispuesto a posponer sus reflexiones sobre su futura labor en aras del mundo y el universo para poder ser partícipe del júbilo infantil y de la alegría vigorosa de estos jóvenes, que nunca se cansaban de escuchar a Jesús narrar las experiencias de sus diversos viajes a Jerusalén. También disfrutaban mucho con sus relatos sobre los animales y la naturaleza.
\vs p128 6:11 Los niños eran siempre bien recibidos en la tienda de reparaciones. Al lado de esta, Jesús les ponía arena, tacos de madera y piedras, y multitud de jóvenes acudía allí en tropel para entretenerse. Cuando se cansaban de sus juegos, los más intrépidos echaban una ojeada dentro de la tienda y, si su encargado no estaba ocupado, se atrevían a entrar diciendo: “Tío Joshua, sal y cuéntanos una buena historia”. Entonces lo hacían salir tirándole de las manos hasta que se sentaba en su piedra preferida situada en la esquina de la tienda, con los niños formando un semicírculo en el suelo delante de él. Y ¡cómo disfrutaban estos pequeñuelos con su tío Joshua! Aprendían a reírse, y a reírse entusiasmados. Uno o dos de los más pequeños tenían la costumbre de subirse a sus rodillas y se sentaban allí, contemplando, maravillados, los rasgos expresivos de su rostro al narrar los relatos. Los niños amaban a Jesús, y Jesús amaba a los niños.
\vs p128 6:12 A sus amigos les resultaba difícil comprender su variada actividad intelectual, cómo podía pasar de forma tan repentina y completa de un profundo análisis de la política, de la filosofía o de la religión hasta el nivel lúdico y desenfadado de estos niños de cinco a diez años de edad. Conforme sus propios hermanos y hermanas crecían, conforme gozaba de más tiempo de ocio y antes de que llegaran los nietos, prestaba una gran atención a estos pequeños. Pero no vivió suficiente tiempo en la tierra como para disfrutar mucho de esos nietos.
\usection{7. SU VIGÉSIMO SEXTO AÑO (AÑO 20 D. C.)}
\vs p128 7:1 Al comenzar este año, Jesús de Nazaret se volvió claramente consciente de que poseía en potencia un amplio rango de facultades. Pero estaba totalmente convencido, igualmente, de que su persona no iba a usar estas capacidades como Hijo del Hombre, al menos hasta que llegase su hora.
\vs p128 7:2 En esta época, aunque recapacitó mucho sobre la relación con su Padre celestial, poco decía al respecto. Y cierta vez, estando en la cima de la colina, expresó en una oración el resultado de toda esta reflexión diciendo: “Sea yo quien fuese y sea cual fuese el poder que yo pueda o no pueda tener, he estado siempre y siempre estaré sujeto a la voluntad de mi Padre del Paraíso”. Y, sin embargo, mientras este hombre caminaba por Nazaret yendo y viniendo de su trabajo, era literalmente cierto ---en cuanto a un inmenso universo--- que “en él estaban escondidos todos los tesoros de la sabiduría y del conocimiento”.
\vs p128 7:3 \pc Todo este año, los asuntos de la familia se desarrollaron sin dificultades, excepto en lo referido a Judá. Durante años, Santiago tuvo problemas con su hermano menor, que no estaba dispuesto a ponerse a trabajar ni se podía contar con él para que contribuyera a los gastos domésticos. Aunque vivía en la casa, no tomaba conciencia de que tenía que aportar al sostén de su familia.
\vs p128 7:4 Jesús era un hombre de paz y, en ocasiones, le disgustaban los actos beligerantes y los numerosos arrebatos patrióticos de Judá. Santiago y José estaban a favor de expulsarlo de la casa, pero Jesús nunca accedió a ello. Cuando llegaban verdaderamente al límite de su paciencia, Jesús simplemente les recomendaba: “Tened paciencia. Sed sabios en vuestros consejos y elocuentes en vuestras vidas, para que vuestro hermano menor pueda conocer primero el camino mejor y luego se sienta obligado a seguiros en este”. Las recomendaciones de Jesús, prudentes y amorosas, evitaron una ruptura de la familia; permanecieron juntos, pero Judá, hasta después de su matrimonio, nunca consiguió templar sus sentidos.
\vs p128 7:5 María raramente hablaba de la futura misión de Jesús. Siempre que se hacía referencia a este asunto, Jesús solo respondía: “Aún no ha llegado mi hora”. Jesús casi había terminado la difícil tarea de desacostumbrar a su familia de la dependencia a su inmediata presencia personal. Se estaba preparando con rapidez para el día en el que podría dejar permanentemente este hogar de Nazaret y comenzar de forma más activa el preludio de su verdadero ministerio en favor de los hombres.
\vs p128 7:6 Nunca perdáis de vista el hecho de que la misión fundamental de Jesús en su séptimo ministerio de gracia era adquirir la experiencia de sus criaturas, lograr la soberanía de Nebadón. Y al acumular esta misma experiencia, realizaba, para Urantia y para todo su universo local, la revelación suprema del Padre del Paraíso. De forma adicional a estos fines, también emprendió la tarea de desenmarañar los complejos asuntos de este planeta en su relación con la rebelión de Lucifer.
\vs p128 7:7 \pc Jesús gozó este año de más tiempo de ocio de lo habitual, y dedicó mucho tiempo a formar a Santiago en la gestión de la tienda de reparaciones y a José en la dirección de los asuntos del hogar. María tenía la sensación de que se estaba preparando para dejarlos. ¿Dejarlos para ir adónde? ¿Para hacer qué? Casi había cesado en su idea de que Jesús era el Mesías. No podía entenderlo; simplemente no lograba comprender a su hijo primogénito.
\vs p128 7:8 Este año, Jesús pasó una gran cantidad de tiempo de forma individual con cada uno de los miembros de su familia. Los llevaba a dar largos y frecuentes paseos por el campo y por la colina. Antes de la cosecha, llevó a Judá a casa de su tío el granjero, al sur de Nazaret, pero Judá no permaneció mucho tiempo tras la cosecha. Huyó de allí y Simón lo encontró más tarde con los pescadores en el lago. Cuando Simón lo trajo de vuelta a casa, Jesús habló sobre algunas cosas con el joven fugado y, puesto que quería ser pescador, fue hasta Magdala con él y lo puso en manos de un pariente suyo pescador; y, Judá, desde aquel momento, trabajó razonablemente bien y con regularidad hasta que contrajo matrimonio, continuando como pescador después de casarse.
\vs p128 7:9 \pc Por fin había llegado el día en el que todos los hermanos de Jesús habían elegido, y se habían asentado, en sus respectivas profesiones. Llegaba el momento en el que Jesús habría de dejar el hogar.
\vs p128 7:10 \pc En noviembre hubo una doble boda. Santiago se casó con Esta y Miriam lo hizo con Jacob. Aquello fue en verdad un motivo de júbilo. Incluso María estaba feliz una vez más, salvo cuando, en ocasiones, se daba cuenta de que Jesús se estaba preparando para marcharse. Sentía en sí el peso de una gran incertidumbre: Ojalá Jesús quisiera sentarse y hablar con ella de todo esto con libertad, como cuando era niño, pero se había vuelto poco comunicativo; mantenía un profundo silencio respecto al futuro.
\vs p128 7:11 Santiago y su esposa Esta se trasladaron a una casa, pequeña y agradable, regalo del padre de ella, situada en la parte oeste de la ciudad. Aunque Santiago continuaba con su ayuda al mantenimiento de la casa de su madre, debido a su matrimonio, su aportación se redujo a la mitad, y Jesús nombró formalmente a José como jefe de la familia. Judá enviaba ahora con toda lealtad su contribución mensual a la casa. Las bodas de Santiago y de Miriam tuvieron una influencia muy beneficiosa sobre él y, al marcharse para las zonas de pesca al día siguiente del doble enlace, le aseguró a José que podía contar con él “para cumplir plenamente con mi deber e incluso más si es necesario”. Y mantuvo su promesa.
\vs p128 7:12 Miriam, vecina de al lado de María, vivía en la casa de Jacob, dado que el anciano Jacob yacía ya descansando con sus padres. Marta ocupó el lugar de Miriam en la casa, y, antes de que terminara el año, la nueva organización funcionaba sin dificultad alguna.
\vs p128 7:13 \pc Al día siguiente de la doble boda, Jesús mantuvo una importante conversación con Santiago. Le contó confidencialmente que se estaba preparando para marcharse. Cedió a Santiago toda la titularidad de la tienda de reparaciones, renunció de manera formal y solemnemente como jefe de la casa de José y erigió a su hermano Santiago, de forma muy emotiva, como “jefe y protector de la casa de mi padre”. Redactó un pacto secreto, que ambos firmaron, en el que se estipulaba que, a cambio del obsequio de la tienda de reparaciones, Santiago asumiría en lo sucesivo plena responsabilidad de la economía de la familia, eximiendo así a Jesús de cualquier otra obligación en estas cuestiones. Después de firmar el contrato y, tras organizar el presupuesto de manera que la familia pudiera hacer frente a sus gastos reales sin ninguna contribución de Jesús, este dijo a Santiago: “Si bien, hijo mío, todos los meses seguiré enviándote algo hasta que haya llegado mi hora, pero utiliza lo que te envíe según las circunstancias lo exijan. Dedica mis ingresos a las necesidades o los placeres de la familia como estimes oportuno. Úsalos en caso de enfermedad o para afrontar urgencias inesperadas que puedan sobrevenir a cualquier miembro de la familia”.
\vs p128 7:14 Y así se preparaba Jesús para emprender la segunda faceta de su vida adulta, separado de su hogar, antes de comenzar a ocuparse públicamente de los asuntos de su Padre.
