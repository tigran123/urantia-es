\upaper{162}{En la fiesta de los Tabernáculos}
\author{Comisión de seres intermedios}
\vs p162 0:1 Tal como Jesús había planeado, partió para Jerusalén con los diez apóstoles. Lo hicieron a través de Samaria, el camino más corto. Así pues, tomaron la costa oriental del lago y, siguiendo el camino de Escitópolis, se adentraron en las fronteras de Samaria. Cerca del anochecer, Jesús envió a Felipe y Mateo a una aldea situada en las laderas orientales del monte Gilboa a buscar alojamiento para todo el grupo. Sucedió que estos aldeanos tenían serios prejuicios contra los judíos, incluso más de lo que era normal entre los samaritanos, y estos sentimientos estaban enardecidos en aquel momento particular debido a la cantidad de judíos que iban a la fiesta de los Tabernáculos. Estas personas apenas sabían nada de Jesús y se negaron a facilitarle el alojamiento porque él y sus acompañantes eran judíos. Cuando Mateo y Felipe, mostrando su indignación, hicieron saber a estos samaritanos que le estaban negando el hospedaje al Santo de Israel, los aldeanos, enfurecidos, los persiguieron arrojándoles palos y piedras hasta echarlos de su pequeña localidad.
\vs p162 0:2 Una vez que Felipe y Mateo estaban de vuelta al lado de sus compañeros y les informaron de cómo los habían expulsado de la aldea, Santiago y Juan se aproximaron a Jesús y le dijeron: “Maestro, te rogamos que nos des tu permiso para mandar que descienda el fuego del cielo y consuma a estos insolentes e impenitentes samaritanos”. Pero cuando Jesús oyó estas palabras de venganza, se volvió a los hijos de Zebedeo y los reprendió duramente diciéndoles: “Vosotros no sabéis qué forma de proceder es esa. La venganza no forma parte del reino de los cielos. En lugar de discutir, dirijámonos a la pequeña aldea situada junto al vado del Jordán”. De ese modo, debido a su actitud sectaria, estos samaritanos se negaron a sí mismos el honor de brindar su hospitalidad al Hijo Creador de un universo.
\vs p162 0:3 Jesús y los diez se detuvieron para pasar la noche en aquella aldea cercana al vado del Jordán. Al día siguiente, temprano, cruzaron el río y continuaron hacia Jerusalén por la carretera que discurría por el este del Jordán, llegando a Betania a última hora de la tarde del miércoles. Tomás y Natanael lo hicieron el viernes, al haberse demorado por sus conversaciones con Rodán.
\vs p162 0:4 \pc Jesús y los doce se quedaron en las inmediaciones de Jerusalén hasta finales del mes siguiente (octubre), sobre unas cuatro semanas y media. El mismo Jesús solo visitó la ciudad, y de forma breve, durante los días de la fiesta de los Tabernáculos. Pasó una parte considerable del mes de octubre en Belén con Abner y sus compañeros.
\usection{1. LOS PELIGROS DE LA VISITA A JERUSALÉN}
\vs p162 1:1 Mucho antes de huir de Galilea, los seguidores de Jesús le habían suplicado que fuera a Jerusalén para anunciar el evangelio del reino con el fin de que su mensaje obtuviese la estima pública de haberse predicado en el centro de la cultura y del conocimiento judíos; pero cuando Jesús hizo acto de presencia en Jerusalén para impartir sus enseñanzas, temieron por su vida. Sabiendo que el sanedrín quería juzgarle en Jerusalén, y recordando la reciente y reiterada afirmación del Maestro de que debía someterse a la muerte, los apóstoles se quedaron realmente desconcertados ante su repentina decisión de asistir a la fiesta de los Tabernáculos. A sus anteriores ruegos para que fuese a Jerusalén, Jesús había respondido: “Aún no ha llegado mi hora”. Ahora, al manifestar ellos su miedo, solo contestaba: “Pero ya ha llegado mi hora”.
\vs p162 1:2 Durante la fiesta de los Tabernáculos, Jesús se arriesgó a ir a Jerusalén en diversas ocasiones y enseñó públicamente en el templo, a pesar de los intentos de sus apóstoles por disuadirlo. Y aunque por mucho tiempo le habían instado a que proclamara su mensaje en Jerusalén, en ese preciso momento temían verlo entrar en la ciudad; eran muy conscientes de que los escribas y los fariseos estaban empeñados en darle muerte.
\vs p162 1:3 La valiente aparición de Jesús en Jerusalén confundió más que nunca a sus seguidores. Muchos de sus discípulos y hasta Judas Iscariote, un apóstol, habían llegado incluso a pensar que Jesús había huido a Fenicia por miedo a los líderes judíos y a Herodes Antipas. No alcanzaban a comprender el sentido de los traslados del Maestro. Su presencia en Jerusalén en la fiesta de los Tabernáculos, a pesar de las recomendaciones en contra de sus seguidores, bastó para poner fin para siempre a las murmuraciones sobre su temor y cobardía.
\vs p162 1:4 Durante la fiesta de los Tabernáculos, fueron miles los creyentes de todas las partes del Imperio romano que vieron a Jesús y lo oyeron enseñar, y muchos se desplazaron incluso a Betania para hablar con él sobre el progreso del reino en sus regiones natales.
\vs p162 1:5 Eran muchas las razones por las que Jesús pudo predicar públicamente en los patios del templo durante los días de la fiesta, y la principal de ellas fue la suspicacia de los oficiales del sanedrín por la división encubierta de opiniones en sus propias filas. Era un hecho que muchos de los miembros del sanedrín o bien creían secretamente en Jesús o eran muy reacios a prenderlo durante la fiesta, en un momento en el que había un número tan grande de personas en Jerusalén, muchas de las cuales o bien creían en él o, al menos, eran favorables al movimiento espiritual que él promovía.
\vs p162 1:6 La actividad de Abner y de sus compañeros por toda Judea había hecho bastante para que se llegase a consolidar una actitud propicia al reino, hasta el punto que los enemigos de Jesús no se atrevían a oponerse demasiado abiertamente a él. Esta fue una de las razones por las que Jesús pudo acudir públicamente a Jerusalén y marcharse de allí con vida. Uno o dos meses antes, seguramente lo habrían condenado a muerte.
\vs p162 1:7 Si bien, la valentía y audacia de Jesús al aparecer públicamente en Jerusalén dejó impresionados a sus enemigos; no se esperaban tal atrevimiento. Algunas veces, durante este mes, el sanedrín había hecho débiles intentos por arrestar al Maestro, pero habían resultado baldíos. Sus enemigos estaban tan desconcertados por la inesperada aparición pública de Jesús en Jerusalén que dedujeron que las autoridades romanas debían haberle prometido protección. Sabiendo que Felipe (el hermano de Herodes Antipas) era prácticamente seguidor de Jesús, los miembros del sanedrín se imaginaron que Felipe había garantizado a Jesús que lo guardaría de sus enemigos. No obstante, antes de que estos se percataran de su equivocación al creer que su repentina y atrevida presencia en Jerusalén se debía a un acuerdo tácito con los oficiales romanos, Jesús ya había salido de su jurisdicción.
\vs p162 1:8 Cuando partieron de Magadán, solo los doce apóstoles sabían que Jesús tenía la intención de asistir a la fiesta de los Tabernáculos. Los otros seguidores del Maestro se sorprendieron bastante cuando apareció en los patios del templo y empezó a enseñar públicamente, y las autoridades judías quedaron impactadas hasta lo indescriptible cuando se les informó de que Jesús impartía sus enseñanzas en el templo.
\vs p162 1:9 Aunque sus discípulos no preveían que Jesús asistiera a la fiesta, la inmensa mayoría de los peregrinos, llegados de lejos y que habían oído hablar de él, abrigaban la esperanza de poder verlo en Jerusalén. Y no se vieron decepcionados porque, en algunas ocasiones, enseñó en el pórtico de Salomón y en otras partes de los patios del templo. En realidad, estas enseñanzas significaban el anuncio, de manera oficial o explícita, de la divinidad de Jesús al pueblo judío y a todo el mundo.
\vs p162 1:10 Las multitudes que escuchaban las enseñanzas del Maestro tenían opiniones encontradas. Algunos decían que era un hombre bueno; otros, que era un profeta. Había quien comentaba que era realmente el Mesías, mientras que para otros era un malicioso entrometido, que llevaba a la gente por el mal camino con sus extrañas doctrinas. Sus enemigos dudaban si denunciarlo abiertamente por temor a sus leales creyentes, mientras que sus amigos temían apoyarlo públicamente por miedo a los líderes judíos, conscientes de que el sanedrín estaba resuelto a darle muerte. Pero incluso sus enemigos se maravillaban de sus enseñanzas, y más considerando que no se había educado en las escuelas de los rabinos.
\vs p162 1:11 Cada vez que Jesús iba a Jerusalén, los apóstoles caían presa del terror. E incluso sentían más miedo un día tras otro al escuchar sus pronunciamientos, cada vez más valientes, sobre la naturaleza de su misión en la tierra. No estaban acostumbrados a oírle realizar afirmaciones tan categóricas y asombrosas ni incluso cuando predicaba entre sus amigos.
\usection{2. PRIMER DISCURSO EN EL TEMPLO}
\vs p162 2:1 La primera tarde que Jesús enseñaba en el templo, había un considerable número de personas sentadas escuchando sus palabras sobre la libertad del nuevo evangelio y del gozo de quienes creen en la buena nueva, y uno de estos asistentes, lleno de curiosidad, le interrumpió para preguntar: “Maestro, ¿cómo es que puedes citar las Escrituras y enseñar a la gente con tanta soltura, cuando se nos ha dicho que no has recibido formación de los rabinos?”. Jesús le contestó: “Ningún hombre me ha enseñado las verdades que yo os proclamo. Y esta doctrina no es mía, sino de Aquel que me envió. El que quiera realmente hacer la voluntad de mi Padre, de cierto conocerá si la doctrina es de Dios o la hablo por mi propia cuenta. El que habla por su propia cuenta busca su propia gloria; pero cuando yo anuncio las palabras del Padre, busco por tanto la gloria del que me envió. Pero antes de que tratéis de entrar en la nueva luz, ¿no debéis mejor seguir la luz que ya tenéis? Moisés os dio la ley, sin embargo, ¿cuántos de entre vosotros la cumple de verdad? En esta ley, Moisés os manda, diciendo: ‘No matarás’; mas a pesar de este mandato, algunos de vosotros queréis matar al Hijo del Hombre”.
\vs p162 2:2 \pc Cuando la muchedumbre oyó estás palabras, empezaron a discutir entre ellos. Algunos decían que estaba loco; otros, que tenía un demonio. Otros que en efecto él era el profeta de Galilea a quien los escribas y fariseos buscaban para matarlo desde hacía tiempo. Había quien aseguraba que las autoridades religiosas tenían miedo de acosarlo; otros pensaban que no lo prendían porque creían en él. Tras un intenso debate, se adelantó alguien desde la multitud y le preguntó a Jesús: “¿Por qué quieren matarte los dirigentes religiosos?”. Y él respondió: “Esos dirigentes quieren matarme porque les molesta mis enseñanzas sobre las buenas nuevas del reino, un evangelio que libera a los hombres de las onerosas tradiciones de una religión formalista y ritual, que estos maestros están decididos a mantener a cualquier precio. Según la ley, circuncidan el día de \bibemph{sabbat,} pero quieren matarme porque yo cierta vez en \bibemph{sabbat} libré a un hombre que era víctima de una aflicción. Vienen tras de mí en \bibemph{sabbat} para espiarme, pero quieren matarme porque, en otra ocasión, decidí curar a un hombre gravemente aquejado en \bibemph{sabbat}. Quieren matarme porque saben bien que, si creéis sinceramente y os atrevéis a aceptar mis enseñanzas, su tradicional sistema religioso será derribado, destruido para siempre. Así pues, se les despojará de autoridad sobre aquello a lo que han dedicado su vida ante su rotunda negativa a aceptar este evangelio del reino de Dios nuevo y más glorioso. Y ahora apelo a cada uno de vosotros: no juzguéis según las apariencias externas, sino juzgad más bien con el espíritu verdadero de estas enseñanzas; juzgad con justo juicio”.
\vs p162 2:3 Entonces dijo otro de los asistentes: “Sí, Maestro, buscamos al Mesías, pero cuando venga, sabemos que aparecerá de manera misteriosa. Sabemos de dónde eres tú. Tú has estado entre tus hermanos desde el principio. El libertador llegará en poder para restaurar el trono del reino de David. ¿Declaras que eres realmente el Mesías?”. Y Jesús contestó: “Dices que me conoces y que sabes de dónde soy. Ojalá que tus afirmaciones fuesen ciertas, porque sin duda hallarías vida abundante en ese conocimiento. Pero yo declaro que no he venido por mí mismo; el Padre me ha enviado, y quien me envió es verdadero y fiel. Al negaros a oírme os estáis negando a recibir a aquel que me envía. Vosotros, si acogéis este evangelio, llegaréis a conocer a aquel que me envió. Yo conozco al Padre, porque del Padre he venido para anunciároslo y revelároslo”.
\vs p162 2:4 Los agentes de los escribas querían prenderlo, pero tenían miedo de la multitud, porque había muchos que creían en él. La labor de Jesús, desde su bautismo, había llegado a ser bien conocida por todo el pueblo judío, y conforme mucha de esta gente relataba estas cosas, decían entre ellos: “Aunque este maestro sea de Galilea, y aunque no responda a todas nuestras expectativas sobre el Mesías, nos preguntamos si el libertador, cuando venga, hará realmente más maravillas que las que ha hecho este Jesús de Nazaret”.
\vs p162 2:5 Cuando los fariseos y sus agentes oyeron a la gente hablar de este modo, consultaron con sus líderes y resolvieron que debían hacer algo sin dilación para detener estas apariciones públicas de Jesús en los patios del templo. En general, los líderes de los judíos estaban determinados a evitar un enfrentamiento con Jesús, creyendo que las autoridades romanas le habían prometido inmunidad. No podían justificar de otra manera su arrojo al venir a Jerusalén en aquel momento; pero los funcionarios del sanedrín no daban del todo crédito a aquel rumor. Suponían que los gobernadores romanos no harían tal cosa a escondidas y sin el conocimiento del máximo órgano de gobierno de la nación judía.
\vs p162 2:6 Por esta razón, se envió a Eber, el oficial asignado por el sanedrín, y a dos ayudantes para que arrestaran a Jesús. Al avanzar Eber hacia Jesús, el Maestro le dijo: “Acércate sin miedo. Aproxímate y escucha mis enseñanzas. Sé que te han enviado para apresarme, pero debes entender que nada le ocurrirá al Hijo del Hombre hasta que no llegue su hora. No estás predispuesto en mi contra; solo vienes por mandato de tus jefes, e incluso no hay duda de que esos dirigentes de los judíos piensan que hacen un servicio a Dios cuando, encubiertamente, procuran mi muerte.
\vs p162 2:7 “No siento animosidad contra ninguno de vosotros. El Padre os ama, y yo ansío, pues, rescataros del yugo del prejuicio y de la oscuridad de la tradición. Os ofrezco la libertad de la vida y el gozo de la salvación. Anuncio un nuevo camino vivo, la liberación del mal y el quebrantamiento de vuestras ataduras al pecado. He venido para que tengáis vida y la tengáis eternamente. Queréis libraros de mí y de mis inquietantes enseñanzas. ¡Si os dierais realmente cuenta del poco tiempo que estaré con vosotros! En breve iré a Aquél que me envió a este mundo. Y luego muchos de vosotros me buscaréis arduamente pero no me hallaréis, porque adonde estoy a punto de ir, vosotros no podéis ir. Pero todos aquellos que en verdad quieran encontrarme accederán en algún momento a la vida que conduce a la presencia de mi Padre”.
\vs p162 2:8 Algunos de los que se mofaban de él dijeron entre sí: “¿Adónde irá este, que no lo hallaremos? ¿Es que se irá a vivir entre los griegos? ¿Se quitará la vida? ¿Qué significa cuando anuncia que se alejará pronto de nosotros y que adonde él va no podemos ir nosotros?
\vs p162 2:9 Eber y sus ayudantes se negaron a arrestar a Jesús y regresaron sin él al punto de encuentro asignado. Así pues, cuando los sumos sacerdotes y los fariseos los reprendieron por no haber llevado a Jesús con ellos, Eber solamente respondió: “Temimos prenderlo en medio de la multitud, porque hay muchos que creen en él. Además, jamás habíamos oído a ningún hombre hablar como este hombre. Hay algo fuera de lo común en este maestro. Todos haríais bien en ir a oírlo”. Y cuando los dirigentes principales oyeron estas palabras, se quedaron atónitos y le dijeron a Eber en tono de burla: “¿Es que tú también estás descarriado? ¿Vas a creer en ese embaucador? ¿Has oído decir que algunos de nuestros eruditos o de nuestros dirigentes hayan creído en él? ¿Es que alguno de los escribas o de los fariseos se ha visto engañado por sus ingeniosas enseñanzas? ¿Cómo es que te has dejado influenciar por el comportamiento de esa multitud ignorante que no conoce la ley ni los profetas? ¿Es que no sabes que esa gente sin instrucción está maldita?”. Y entonces respondió Eber: “No obstante, mis señores, este hombre habla a la multitud palabras de misericordia y de esperanza. Alegra a los abatidos, y sus palabras trajeron consuelo incluso a nuestras almas. ¿Qué mal puede haber en estas enseñanzas aunque no sea el Mesías de las Escrituras? Y aun así, ¿es que nuestra ley no precisa de la equidad? ¿Es que condenamos a un hombre antes de oírlo?”. Y el jefe del sanedrín se encolerizó contra Eber y, volviéndose hacia él, le dijo: “¿Te has vuelto loco? ¿Acaso eres tú también de Galilea? Escudriña las Escrituras y verás que de Galilea no se ha levantado ningún profeta, ni mucho menos el Mesías”.
\vs p162 2:10 El sanedrín se disolvió en confusión, y Jesús se retiró a Betania para pasar la noche.
\usection{3. LA MUJER SORPRENDIDA EN ADULTERIO}
\vs p162 3:1 Fue durante esta visita a Jerusalén cuando Jesús trató con una mujer de mala reputación, que fue llevada a su presencia por los acusadores de esta y los enemigos de él. La tergiversada narrativa que poseéis de este episodio apunta a que fueron los escribas y los fariseos quienes la llevaron a él y Jesús se dirigió a ellos acusando de posible inmoralidad a estos mismos líderes religiosos de los judíos. No obstante, Jesús era conocedor de que, aunque los escribas y fariseos estaban ciegos espiritualmente y llenos de prejuicios intelectuales por su lealtad a la tradición, se contaban entre los hombres más cabalmente morales de aquel día y generación.
\vs p162 3:2 Esto fue lo que realmente ocurrió: temprano, durante la tercera mañana de la fiesta, conforme Jesús se acercaba al templo, se encontró con un grupo de agentes a sueldo del sanedrín que arrastraban a una mujer. Al aproximarse a él, su portavoz dijo: “Maestro, esta mujer ha sido sorprendida en el acto mismo de adulterio, y en la Ley nos mandó Moisés apedrear a tales mujeres. Tú, pues, ¿qué dices que deberíamos hacer con ella?”.
\vs p162 3:3 La intención de los enemigos de Jesús era que si respaldaba la ley de Moisés, que exigía se apedreara a la infractora confesa, se viese involucrado en problemas con los dirigentes romanos, que negaban a los judíos el derecho de imponer la pena de muerte sin la aprobación de un tribunal romano. Si prohibía apedrear a la mujer, lo acusarían ante el sanedrín de ponerse por encima de Moisés y de la ley judía. Si permanecía callado, lo acusarían de cobardía. Pero el Maestro solventó la situación de tal modo que toda aquella trama urdida se vino abajo por su propio y sórdido peso.
\vs p162 3:4 Esta mujer, en otro tiempo hermosa, era la esposa de un ciudadano de estrato inferior de Nazaret, alguien que le había ocasionado a Jesús problemas durante sus días jóvenes. El hombre, tras casarse con esta mujer, la había forzado de forma ignominiosa a que ganase el sustento para ambos a expensas de comerciar con su cuerpo. Había venido a la fiesta de Jerusalén para que su esposa se prostituyese vendiendo sus atractivos físicos. Había hecho un trato con los asalariados de los dirigentes judíos y traicionar pues a su propia esposa dedicada a la prostitución. Por lo tanto, vinieron con la mujer y su compañero transgresor con el fin de atrapar a Jesús si realizaba alguna declaración que, en el caso de su detención, pudiera argüirse en contra de él.
\vs p162 3:5 Jesús, escrutando la multitud, vio al marido, de pie, tras los demás. Sabía qué clase de hombre era y se dio cuenta de que estaba implicado en aquel despreciable trato. Primeramente, Jesús caminó acercándose adonde se encontraba este degenerado marido y escribió algunas palabras en la arena, que hicieron que se fuese precipitadamente. Luego, regresó al lado de la mujer y escribió de nuevo en la tierra unas palabras dirigidas a los que la iban a acusar, y al leerlas se alejaron también, uno a uno. Y cuando el Maestro escribió por tercera vez en la arena, el acompañante de la mujer en aquella mala acción también se marchó, de manera que, cuando el Maestro se alzó después de escribir, no vio a nadie sino a la mujer de pie ante él. Jesús le dijo: “Mujer, ¿dónde están los que te acusaban? ¿Es que no ha quedado nadie para apedrearte?”. Y la mujer, levantando la mirada, respondió: “Nadie, Señor”. Entonces dijo Jesús: “Yo sé de ti; tampoco yo te condeno. Vete en paz”. Y esta mujer, Hildana, abandonó a su malvado marido y se unió a los discípulos del reino.
\usection{4. LA FIESTA DE LOS TABERNÁCULOS}
\vs p162 4:1 La presencia de personas procedentes de todo el mundo conocido, desde España hasta la India, hacía de la fiesta de los Tabernáculos una ocasión idónea para que Jesús pudiese proclamar públicamente, y por vez primera, íntegramente su evangelio en Jerusalén. Durante toda esta fiesta, la gente solía vivir al aire libre, en tiendas hechas de ramas. Era la fiesta de la Recogida de la Cosecha y al hacerse, como de hecho se hacía, en el fresco ambiente de los meses de otoño, generalmente acudían más judíos de otros lugares del mundo que a la Pascua, a finales del invierno, o a Pentecostés, a comienzos del verano. Por fin, los apóstoles vieron a su Maestro anunciando con arrojo su misión en la tierra ante el mundo entero, por así decirlo.
\vs p162 4:2 Se trataba de la fiesta de las fiestas, porque, en tales momentos, se podía realizar cualquier otro sacrificio que no se hubiera hecho en otras festividades. En esta ocasión, se presentaban las ofrendas en el templo; era una combinación del disfrute de las vacaciones con los ritos solemnes del culto religioso. Era un momento de gozo nacional, mezclado con los sacrificios, los cantos levíticos y los toques solemnes de las trompetas de plata de los sacerdotes. Por la noche, la impresionante escena del templo y de sus multitudes de peregrinos se iluminaba esplendorosamente gracias a los grandes candelabros que ardían radiantes en el patio de las mujeres y al brillo de las innumerables antorchas colocadas por los patios del templo. Salvo el castillo romano de Antonia, la ciudad por entero se adornaba vistosamente. Este castillo se asomaba desde lo alto, sombrío, en contraste con aquella escena festiva y de devoción. ¡Era grande el odio que los judíos sentían hacia aquel recordatorio siempre presente del yugo romano!
\vs p162 4:3 Durante la fiesta se sacrificaban setenta bueyes, como símbolo de las setenta naciones paganas. La ceremonia del derramamiento del agua simbolizaba a su vez el derramamiento del espíritu divino. Dicha ceremonia tenía lugar a la salida del sol, tras la procesión de los sacerdotes y de los levitas. Los fieles descendían por los escalones que conducían desde el patio de Israel hasta el patio de las mujeres, mientras que sonaban, en toques sucesivos, las trompetas de plata. Y luego los fieles marchaban hasta la puerta de la Hermosa por la que se accedía al patio de los gentiles. Aquí, la gente se volvía mirando al oeste, para repetir sus cantos y continuar la procesión hacia el agua simbólica.
\vs p162 4:4 \pc El último día de la fiesta, casi cuatrocientos cincuenta sacerdotes, con su respectivo número de levitas, oficiaban la ceremonia. Al despuntar el día, se congregaron los peregrinos venidos de todas las partes de la ciudad. Cada uno de ellos llevaba en la mano derecha un manojo de ramas de mirto, sauce y palma, mientras que, en la izquierda, una rama de manzana del paraíso ---la cidra, o la “fruta prohibida”---. Para esta ceremonia matutina, los peregrinos se dividían en tres grupos. Uno de estos permanecía en el templo para asistir a los sacrificios de la mañana; otro grupo bajaba, en procesión, desde Jerusalén hasta cerca de Maza para cortar las ramas de sauce que adornarían el altar sacrificial, mientras que el tercero marchaba desde el templo, siguiendo al sacerdote a cargo del agua, que al son de las trompetas de plata y portando la jarra de oro que contendría el agua simbólica, salía a través de Ofel hasta llegar cerca de Siloé, donde se hallaba la puerta de la Fuente. Una vez que dicha jarra se llenaba en el estanque de Siloé, la procesión marchaba de vuelta al templo, entrando por la puerta del Agua y dirigiéndose directamente al patio de los sacerdotes. Aquí, al sacerdote que llevaba la jarra de agua se le unía el que portaba el vino para la libación. Entonces, estos dos sacerdotes se encaminaban a los embudos de plata conducentes a la base del altar y derramaban en ellos el contenido de las jarras. La realización de este rito del derramamiento del vino y del agua era la señal para que los peregrinos allí reunidos entonaran el cántico de los salmos desde el 113 al 118 inclusive, alternándose con los levitas. Y, a medida que repetían estos versos, ondeaban sus manojos de ramas hacia el altar. Luego, se efectuaban los sacrificios de ese día, vinculados a la repetición del salmo correspondiente. El último día de la fiesta se cantaba el salmo 82, a partir del quinto verso.
\usection{5. SERMÓN SOBRE LA LUZ DEL MUNDO}
\vs p162 5:1 Al anochecer del penúltimo día de la fiesta, cuando las luces de los candelabros y de las antorchas iluminaban brillantemente aquel entorno, Jesús se puso de pie en medio de la multitud allí congregada, y les habló diciendo:
\vs p162 5:2 \pc “Yo soy la luz del mundo. El que me sigue, no andará en tinieblas sino que tendrá la luz de la vida. Cuando os atrevéis a ser mis acusadores y a la vez mis propios jueces, afirmáis que, si doy testimonio acerca de mí mismo, mi testimonio no puede ser válido. Pero la criatura no puede juzgar a su Creador. Aunque yo doy testimonio acerca de mí mismo, mi testimonio es sempiternamente válido, porque sé de dónde vengo, quién soy y adónde voy. Vosotros, que queréis matar al Hijo del Hombre, no sabéis de dónde vengo, quién soy ni adónde voy. Vosotros juzgáis según la carne; no percibís las realidades del espíritu. Yo no juzgo a ningún hombre, ni siquiera a mi archienemigo. Pero si yo juzgo, mi juicio es recto y según la verdad, porque yo no juzgaría solo, sino con mi Padre que me envió al mundo, y que es la fuente del verdadero juicio. Vosotros incluso admitís el testimonio de dos personas de confianza; pues, bien, yo doy testimonio de estas verdades; también lo hace mi Padre de los cielos. Y cuando ayer yo os dije esto mismo, en vuestra oscuridad me preguntasteis, ‘¿dónde está tu Padre?’. Es verdad que ni me conocéis a mí, ni a mi Padre; si a mí me conocierais, también a mi Padre conoceríais.
\vs p162 5:3 “Ya os he dicho que me voy, y que me buscaréis pero no me hallaréis, porque adonde yo voy, vosotros no podéis ir. Vosotros, que rechazáis esta luz, sois de abajo; yo soy de arriba. Vosotros, que preferís andar en las tinieblas, sois de este mundo; yo no soy de este mundo, y vivo en la luz eterna del Padre de las luces. Ya habéis tenido bastantes oportunidades para saber quién soy yo, pero tendréis otra prueba que confirma la identidad del Hijo del Hombre. Yo soy la luz de la vida, y quien rechace deliberadamente y con entendimiento esta luz salvadora morirá en sus pecados. Hay muchas cosas que tengo que deciros, pero sois incapaces de recibir mis palabras. Sin embargo, aquel que me envió es verdadero y fiel; mi Padre ama incluso a sus hijos errados. Y todo lo que mi Padre ha hablado, también yo lo proclamo al mundo.
\vs p162 5:4 “Cuando el Hijo del Hombre sea elevado, entonces todos sabréis que yo soy él, y que nada hago por mí mismo, sino según me enseñó el Padre. Y así hablo estas palabras para vosotros y para vuestros hijos, porque el que me envió conmigo está ahora; no me ha dejado solo, porque yo hago siempre lo que le agrada”.
\vs p162 5:5 \pc Al enseñar Jesús estas cosas a los peregrinos en los patios del templo, muchos creyeron. Y nadie se atrevió a prenderlo.
\usection{6. DISCURSO SOBRE EL AGUA DE LA VIDA}
\vs p162 6:1 En el último y gran día de la fiesta, cuando la procesión que venía del estanque de Siloé pasó por los patios del templo y, justo después de que los sacerdotes derramaran el agua y el vino en el altar, Jesús, de pie entre los peregrinos, dijo: “Si alguien tiene sed que venga a mí y beba. Del Padre de lo alto traigo a este mundo el agua viva. El que cree en mí se llenará del espíritu que esta agua representa, incluso las Escrituras dicen: ‘de su interior brotarán ríos de agua viva’. Cuando el Hijo del Hombre haya terminado su obra en la tierra, se derramará sobre toda carne el espíritu vivo de la verdad. Quienes reciban este espíritu jamás conocerán la sed espiritual”.
\vs p162 6:2 Jesús no interrumpió el servicio de culto para decir estas palabras. Se dirigió a los fieles inmediatamente tras el canto del Halel, la lectura alternada de los Salmos en la que se ondeaban las ramas ante el altar. Justo aquí se hacía una pausa mientras se preparaban los sacrificios, y fue en ese momento cuando los peregrinos oyeron la fascinante voz del Maestro proclamar que él era el dador del agua viva a cualquier alma que tuviese sed de espíritu.
\vs p162 6:3 Al concluir este temprano servicio matutino, Jesús continuó enseñando a la multitud, diciéndole: “¿Es que no habéis leído en las Escrituras: ‘He aquí que como las aguas se derraman sobre la tierra seca y se extienden sobre el sequedal, así derramaré yo el espíritu de santidad sobre tus hijos y mi bendición sobre los hijos de tus hijos’? ¿Por qué tenéis sed del ministerio del espíritu mientras dais a vuestras almas de beber el agua de las tradiciones de los hombres, un agua derramada de las jarras rotas del servicio ceremonial? Lo que estáis viendo que pasa aquí, en este templo, es la forma en la que vuestros padres quisieron simbolizar la dádiva del espíritu divino otorgada a los hijos de la fe, y habéis hecho bien en perpetuar estos símbolos, incluso hasta este día. Pero ahora, a esta generación, a través del ministerio de gracia de su Hijo, le ha llegado la revelación del Padre de los espíritus, y a todo esto ciertamente le seguirá la dádiva del espíritu del Padre y del Hijo que se derramará sobre los hijos de los hombres. Para el que tenga fe, este espíritu, al serle concedido, se convertirá en el verdadero maestro del camino que lleva a la vida perdurable, a las verdaderas aguas de la vida del reino de los cielos sobre la tierra y al Paraíso del Padre, que está más allá”.
\vs p162 6:4 Y Jesús prosiguió respondiendo a las preguntas de la multitud y de los fariseos. Algunos pensaban que era un profeta; otros, que era el Mesías; otros más, considerando que venía de Galilea, decían que no podía ser el Cristo y que el Mesías debía restaurar el trono de David. No obstante, no se atrevieron a prenderlo.
\usection{7. DISCURSO SOBRE LA LIBERTAD ESPIRITUAL}
\vs p162 7:1 En la tarde del último día de la fiesta y tras el intento fallido por parte de los apóstoles por convencerlo para que huyera de Jerusalén, Jesús fue de nuevo al templo para enseñar. Al encontrar a un gran grupo de creyentes congregados en el pórtico de Salomón, les habló diciendo:
\vs p162 7:2 \pc “Si vosotros permanecéis en mi palabra y estáis dispuestos a hacer la verdad de mi Padre, seréis pues verdaderamente mis discípulos. Conoceréis la verdad y la verdad os hará libres. Sé que me contestaréis: somos los hijos de Abraham y no somos esclavos de nadie. ¿Cómo se nos hará libres? Pero yo no os hablo de sometimiento al dominio de fuera; me refiero a la libertad del alma. De cierto, de cierto os digo que todo aquel que comete pecados, esclavo es del pecado. Y sabéis que el esclavo no es muy probable que quede en la casa del amo para siempre. También sabéis que el hijo sí queda en la casa de su padre. Así que, si el Hijo os libera, os hará hijos, seréis realmente libres.
\vs p162 7:3 “Yo sé que sois de la simiente de Abraham; sin embargo, vuestros líderes intentan matarme para que mi palabra no tenga cabida en vosotros ni transforme vuestros corazones. Sus almas están selladas por el prejuicio y cegadas por la soberbia de la venganza. Yo os proclamo la verdad que el Padre eterno me ha mostrado, mientras que estos engañosos maestros tratan de hacer las cosas que han aprendido de sus padres terrenales. Y cuando respondéis que Abraham es vuestro padre, os digo que, si fuerais hijos de Abraham, las obras de Abraham haríais. Entre vosotros, algunos creéis en mis enseñanzas, pero otros intentáis matarme a mí, que os he hablado la verdad, la cual he oído de Dios. Pero Abraham no trató así la verdad de Dios. Algunos de vosotros estáis decididos a hacer las obras del diablo. Si Dios fuera vuestro Padre, me conoceríais y amaríais la verdad que os revelo. ¿Es que no veis que he venido del Padre, que Dios me ha enviado, que no estoy haciendo esta labor de mí mismo? ¿Por qué no escucháis mis palabras? ¿Es acaso porque habéis elegido haceros hijos del mal? Si sois los hijos de la oscuridad, no podréis caminar en la luz de la verdad que yo os revelo. Los hijos del mal siguen solamente los caminos de su padre, que fue un mentiroso y no permaneció en la verdad, porque no hay verdad en él. Pero ahora viene el Hijo del Hombre que dice y vive la verdad, y muchos de vosotros os negáis a creer.
\vs p162 7:4 “¿Quién de vosotros me acusa de pecado? Y si, entonces, digo y vivo la verdad que me ha mostrado mi Padre, ¿por qué no me creéis? El que es de Dios oye con alegría las palabras de Dios; por esto no la oís muchos de vosotros, porque no sois de Dios. Vuestros maestros se atrevieron incluso a decir que hago mis obras por el poder del príncipe de los demonios. Alguien aquí cerca acaba de decir que yo tengo un demonio, que soy hijo del diablo. Pero los que entre vosotros que actuáis honestamente con vuestras propias almas, sabéis muy bien que yo no soy un diablo. Sabéis que honro al Padre mientras vosotros me deshonráis a mí. Yo no busco mi propia gloria, sino la gloria de mi Padre del Paraíso. Y yo no os juzgo, porque hay quien que juzga por mí.
\vs p162 7:5 “De cierto, de cierto os digo a quienes creéis en el evangelio, que, si el hombre guarda esta palabra de verdad viva en su corazón, nunca verá la muerte. Y, así pues, justo a mi lado, hay un escriba que dice que esta afirmación da testimonio de que tengo un diablo, dado que Abraham murió, al igual que los profetas. Y él pregunta: ‘¿Eres tú acaso mayor que Abraham y que los profetas que te atreves a decir aquí que el que guarda tu palabra jamás probará la muerte? ¿Quién te crees que eres para atreverte a proferir tales blasfemias?’. Y a todo esto os digo que, si me glorifico a mí mismo, mi gloria es como si fuera nada. Pero es el Padre el que me glorificará, ese mismo Padre a quien vosotros llamáis Dios. Pero no habéis conseguido conocer a este vuestro Dios y mi Padre, y yo he venido para que estéis juntos, para mostraros cómo convertiros verdaderamente en los hijos de Dios. Aunque vosotros no conozcáis al Padre, yo ciertamente lo conozco. Incluso Abraham se gozó de ver mi día, y por la fe lo vio y se gozó”.
\vs p162 7:6 \pc Cuando los judíos incrédulos y los agentes del sanedrín, que en aquel momento se habían congregado allí, oyeron estas palabras, se alborotaron y gritaron: “No tienes cincuenta años, y sin embargo dices que has visto a Abraham; ¡eres hijo del diablo!”. Jesús fue incapaz de continuar su discurso. Al marcharse solo dijo: “De cierto, de cierto os digo, que antes que Abraham fuera, yo soy”. Muchos de estos descreídos corrieron por piedras para arrojárselas, y los agentes del sanedrín intentaron arrestarlo, pero el Maestro se fue rápidamente por los corredores del templo y escapó a un lugar de encuentro secreto, cerca de Betania, donde lo esperaban Marta, María y Lázaro.
\usection{8. CONVERSACIÓN CON MARTA Y MARÍA}
\vs p162 8:1 Se había dispuesto que Jesús se alojara con Lázaro y sus hermanas en casa de un amigo, mientras los apóstoles se dispersaban en pequeños grupos por distintos lugares. Se habían tenido que tomar estas precauciones porque las autoridades judías estaban de nuevo resueltas a prenderlo.
\vs p162 8:2 Durante muchos años, estos tres hermanos se habían acostumbrado a dejarlo todo para escuchar las enseñanzas de Jesús siempre que sucedía que los visitaba. Al morir sus padres, Marta había asumido la responsabilidad de los asuntos domésticos y, así pues, en esta ocasión, mientras Lázaro y María se sentaban a los pies de Jesús, recreándose en sus reconfortantes enseñanzas, Marta se dispuso a servir la cena. Conviene precisar que Marta se distraía innecesariamente con numerosos quehaceres superficiales y que se dejaba abrumar por muchas preocupaciones triviales; pero ese era su temperamento.
\vs p162 8:3 Mientras que Marta se ocupaba con estos pretendidos deberes, le molestaba el hecho de que María no hiciera nada por ayudarla. Por eso, fue adonde estaba Jesús y le dijo: “Maestro, ¿no te da cuidado que mi hermana me deje servir sola? Dile, pues, que venga y me ayude”. Jesús respondió: “Marta, Marta, ¿por qué estás siempre tan afanada por tantas cosas y preocupada por tantas nimiedades? Solo una cosa merece verdaderamente la pena, y puesto que María ha escogido la parte buena y necesaria, no se la quitaré. Pero, ¿cuándo aprenderéis ambas a vivir como os he enseñado: sirviendo ambas en cooperación y reconfortando vuestras almas al unísono? ¿Es que no sois capaces de aprender que hay una hora para todo, que los asuntos menores de la vida deben dar paso a las cosas más grandes del reino celestial?”.
\usection{9. EN BELÉN CON ABNER}
\vs p162 9:1 A lo largo de la semana que siguió a la fiesta de los Tabernáculos, se reunían en Betania una gran cantidad de creyentes para recibir instrucción de los doce apóstoles. Si bien, al no estar Jesús presente, el sanedrín no tomó ninguna medida para obstaculizar estos encuentros. Él estuvo esos días trabajando con Abner y sus compañeros en Belén. Jesús había partido para Betania un día después de la clausura de la fiesta, y no volvería a enseñar en el templo durante aquella visita a Jerusalén.
\vs p162 9:2 \pc En ese momento, Abner tenía su sede en Belén, y desde allí se habían enviado muchos trabajadores a las ciudades de Judea y del sur de Samaria e incluso a Alejandría. A los pocos días de su llegada, Jesús y Abner concretaron los planes para unificar el quehacer de los dos grupos de apóstoles.
\vs p162 9:3 Durante toda la visita a Jerusalén en ocasión de la fiesta de los Tabernáculos, Jesús había repartido su tiempo casi por igual entre Betania y Belén. En Betania, pasó bastante tiempo con sus apóstoles; en Belén, instruyó a Abner y a los demás antiguos apóstoles de Juan. Y fue este estrecho contacto el que los llevó finalmente a creer en él. A estos antiguos apóstoles de Juan el Bautista, les influyó el arrojo mostrado en sus enseñanzas públicas en Jerusalén al igual que el entendimiento y la comprensión que experimentaron en su enseñanza privada en Belén. Estas circunstancias finalmente convencieron por completo a cada uno de los compañeros de Abner a que aceptaran incondicionalmente el reino y todo lo que dicho paso conllevaba.
\vs p162 9:4 \pc Antes de dejar Belén por última vez, el Maestro hizo planes para que todos se unieran a él en una labor conjunta que precedería al fin de su andadura terrenal en la carne. Se acordó que Abner y sus compañeros se reunirían con Jesús y los doce en un futuro cercano en el parque de Magadán.
\vs p162 9:5 Según este acuerdo, a comienzos de noviembre Abner y sus once apóstoles se unieron con Jesús y los doce, y trabajaron con ellos como un solo equipo hasta el mismo momento de la crucifixión.
\vs p162 9:6 A finales de octubre, Jesús y los doce se alejaron de las inmediaciones de Jerusalén. El domingo 30 de octubre, Jesús y sus acompañantes dejaron la ciudad de Efraín, donde Jesús por sí solo había descansado unos días, y, tomando la carretera al oeste del Jordán, llegaron directamente al parque de Magadán avanzada la tarde del miércoles 2 de noviembre.
\vs p162 9:7 Los apóstoles se sintieron enormemente aliviados de tener al Maestro de vuelta en suelo amigo; nunca más le instarían a que fuera a Jerusalén para proclamar el evangelio del reino.
