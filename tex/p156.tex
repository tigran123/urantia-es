\upaper{156}{Estancia en Tiro y Sidón}
\author{Comisión de seres intermedios}
\vs p156 0:1 El viernes, 10 de junio por la tarde, Jesús y sus acompañantes llegaron a los alrededores de Sidón. Allí se detuvieron en casa de una mujer acomodada, antigua paciente del hospital de Betsaida cuando el Maestro estaba en el momento álgido de su popularidad. Los evangelistas y los apóstoles se alojaron con amigos suyos en un vecindario cercano y descansaron durante el \bibemph{sabbat} en un entorno reconfortante. Pasaron casi dos semanas y media en Sidón y en sus inmediaciones antes de disponerse a visitar las ciudades costeras del norte.
\vs p156 0:2 Aquel \bibemph{sabbat} de junio fue un día de mucha tranquilidad. Los evangelistas y los apóstoles estaban totalmente absortos en sus meditaciones sobre las charlas que el Maestro les había impartido de camino a Sidón. Todos alcanzaban a entender algo de lo que les había dicho, pero ninguno de ellos llegaba a captar del todo el significado de sus enseñanzas.
\usection{1. LA MUJER SIRIA}
\vs p156 1:1 Cerca de la casa de Karuska, donde se alojaba el Maestro, vivía una mujer siria que había oído hablar mucho de Jesús como gran sanador y maestro y, ese \bibemph{sabbat} por la tarde, fue a verlo con su joven hija. La niña, de unos doce años de edad, estaba aquejada de una grave alteración nerviosa, caracterizada por convulsiones y otras preocupantes manifestaciones.
\vs p156 1:2 Jesús había pedido a sus acompañantes que no dijeran a nadie de su presencia en casa de Karuska, explicando que quería descansar. Aunque ellos habían obedecido las instrucciones de su Maestro, la sirviente de Karuska había ido a casa de esta mujer siria, Norana, para informarle de que Jesús se alojaba en la casa de su ama y había instado a esta madre ansiosa a que trajera a su afligida hija para que la curara. Esta madre daba por supuesto que su hija estaba poseída por un demonio, por un espíritu impuro.
\vs p156 1:3 Cuando Norana llegó con su hija, los gemelos Alfeo le explicaron, a través de un intérprete, que el Maestro estaba descansando y que no se le podía molestar. A esto, Norana respondió que se quedarían allí hasta que el Maestro terminara de descansar. Pedro intentó razonar también con ella y convencerla de que se volviera a su casa. Le explicó que Jesús estaba agotado por tanta enseñanza y curaciones, y que había venido a Fenicia para pasar un tiempo de tranquilidad y reposo. Pero fue inútil. Norana no consentía en irse. Ante los ruegos de Pedro, ella únicamente contestaba: “No me marcharé hasta que no haya visto a vuestro Maestro. Yo sé que él puede expulsar al demonio que está en mi hija, y no me iré hasta que el sanador haya puesto su mirada en ella”.
\vs p156 1:4 Entonces Tomás trató de hacer salir a la mujer, pero tampoco lo logró. Ella le dijo: “Tengo fe en que vuestro Maestro pueda expulsar a este demonio que atormenta a mi hija. He oído de sus portentosas obras en Galilea, y creo en él. ¿Qué os pasa a vosotros, sus discípulos, que echáis a los que vienen buscando la ayuda de vuestro Maestro?”. Y cuando habló de ese modo, Tomás se retiró.
\vs p156 1:5 Luego se presentó Simón Zelotes y discutió con Norana. Simón le dijo: “Mujer, eres una gentil de habla griega; no es justo que esperes que el Maestro tome el pan de los hijos del pueblo predilecto y se lo eche a los perros”. Pero Norana no se sintió ofendida por la rudeza de Simón. Solo respondió: “Sí, maestro, entiendo tus palabras. A los ojos de los judíos, yo no soy sino un perro, pero en lo que concierne a tu Maestro, soy un perro creyente. Estoy empeñada en que él vea a mi hija, porque estoy convencida de que con tan solo mirarla la curará. Y ni incluso tú, buen hombre, te opondrías a quitarle a los perros el beneficio de recibir las migajas de pan que puedan caer de la mesa de los niños”.
\vs p156 1:6 Justo en ese momento, ante todos ellos, la pequeña fue presa de una violenta convulsión, y la madre les gritó: “Veis como mi pequeña está poseída por un espíritu maligno. Si esta imperiosa necesidad nuestra no tiene ningún efecto en vosotros, sí apelará a vuestro Maestro, de quien me han dicho que es amante de todos los hombres y hasta se atreve a curar a los gentiles cuando creen. No sois dignos de ser sus discípulos. No me iré hasta que mi hija no se haya curado”.
\vs p156 1:7 Jesús, que había oído toda esta conversación por una ventana abierta, salió de inmediato para gran sorpresa de ellos, y le dijo: “Oh mujer, tu fe es grande, tan grande que no puedo negarte lo que deseas; vete en paz. Tu hija ya ha recobrado la salud”. Y la pequeña se puso bien desde esa hora. Cuando Norana y la niña se despidieron, Jesús les rogó que no contaran a nadie lo sucedido; y aunque sus acompañantes sí cumplieron lo que se les pidió, la madre y la niña divulgaron incesantemente la noticia de la curación de la pequeña por toda la región, e incluso en Sidón, tanto que Jesús creyó aconsejable, días más tarde, alojarse en otro lugar.
\vs p156 1:8 \pc Al día siguiente, al impartir sus enseñanzas a los apóstoles, Jesús comentó la sanación de la hija de la mujer siria diciéndoles: “Y siempre ha sido así; veis por vosotros mismos cómo los gentiles son capaces de demostrar una fe salvadora en las enseñanzas del evangelio del reino de los cielos. En verdad, en verdad os digo que los gentiles tomarán el reino del Padre si los hijos de Abraham no muestran fe suficiente para entrar en él”.
\usection{2. ENSEÑANZAS EN SIDÓN}
\vs p156 2:1 Al entrar en Sidón, Jesús y sus acompañantes pasaron por un puente, el primero que muchos de ellos había visto jamás. Conforme caminaban por él, Jesús, entre otras cosas, dijo: “Este mundo no es más que un puente; podéis cruzarlo, pero no penséis en construir vuestra morada en él”.
\vs p156 2:2 \pc Mientras los veinticuatro comenzaban su labor en Sidón, Jesús se alojó hacia el norte de la ciudad, en casa de Justa y de Berenice, su madre. Jesús enseñaba a los veinticuatro cada mañana en la casa de Justa y, durante las tardes y las noches, salían a distintos lugares de Sidón a enseñar y predicar.
\vs p156 2:3 Los apóstoles y los evangelistas estaban muy animados por el modo en el que los gentiles de Sidón acogían su mensaje; durante su breve estancia allí, se sumaron muchos al reino. Este período de unas seis semanas en Fenicia resultó muy provechoso en cuanto a la labor de ganar almas, pero los más recientes escritores judíos de los evangelios prestaban poca atención al cálido recibimiento de las enseñanzas de Jesús por parte de dichos gentiles, justo en este preciso instante en el que tantos de su propio pueblo eran hostiles hacia él.
\vs p156 2:4 En muchos aspectos, estos creyentes gentiles sentían un mayor aprecio por las enseñanzas de Jesús que los judíos. Una mayoría de estos sirofenicios de habla griega no solo llegó a entender que Jesús era como Dios sino también que Dios era como Jesús. Estos llamados paganos adquirieron un buen entendimiento de las enseñanzas del Maestro sobre la consistencia de las leyes de este mundo y de todo el universo. Lograron comprender la idea de que Dios no hace acepción de personas, razas o naciones; que con el Padre universal no hay favoritismos; que el universo es por siempre y por completo observante de la ley e indefectiblemente digno de confianza. Estos gentiles no temían a Jesús; se atrevieron a aceptar su mensaje. A través de los tiempos, los hombres podrían haber comprendido a Jesús, pero han tenido miedo a hacerlo.
\vs p156 2:5 \pc Jesús dejó claro a los veinticuatro que él no había huido de Galilea porque le faltara valor para hacer frente a sus enemigos. Ellos entendieron que él no estaba aún preparado para un enfrentamiento abierto con la religión establecida, y que no buscaba convertirse en mártir. En una de estas charlas en la casa de Justa, el Maestro dijo a sus discípulos por primera vez que “el cielo y la tierra pasarán, pero mis palabras de verdad no pasarán”.
\vs p156 2:6 \pc El tema tratado por Jesús durante la estancia en Sidón fue el progreso espiritual. Les dijo que no podían detenerse, que debían seguir avanzando en la rectitud, porque de otra manera retrocederían al mal y al pecado. Les instó a que “se olvidaran de las cosas del pasado, mientras seguían adelante hasta llegar a abrazar las realidades más grandes del reino”. Les rogó que no se contentaran con ser meros niños en el evangelio sino que se esforzaran por crecer y lograr el mayor desarrollo posible de filiación divina en la comunión del espíritu y en la fraternidad de los creyentes.
\vs p156 2:7 Dijo Jesús: “Mis discípulos no solo deben dejar de hacer el mal, sino aprender a hacer el bien; no solo debéis estar limpios de todo pecado deliberado, sino incluso asimismo rechazar cualquier sentimiento de culpa. Si confesáis vuestros pecados, estos os serán perdonados; por esto procurad tener siempre una conciencia sin ofensa”.
\vs p156 2:8 Jesús disfrutaba bastante del agudo sentido del humor característico de estos gentiles. Fue el sentido del humor de Norana, la mujer siria, al igual que su gran y perseverante fe, lo que tanto enterneció el corazón del Maestro y apeló a su misericordia. Jesús lamentaba profundamente que su pueblo ---los judíos--- carecieran de humor. Una vez le dijo a Tomás: “Mi pueblo se toma a sí mismo demasiado en serio; casi no aprecian el humor. La opresiva religión de los fariseos nunca podría haberse originado en un pueblo con sentido del humor. Asimismo les falta coherencia; cuelan el mosquito y se tragan el camello”.
\usection{3. VIAJE HACIA EL NORTE POR LA COSTA}
\vs p156 3:1 El martes 28 de junio, el Maestro y sus acompañantes salieron de Sidón y se dirigieron costa arriba hacia Porfireón y Heldua. Durante esta semana de enseñanza y predicación, tuvieron una buena acogida por parte de los gentiles y muchos de ellos se sumaron al reino. Los apóstoles predicaban en Porfireón, a la vez que los evangelistas enseñaban en Heldua. Mientras los veinticuatro se dedicaban a esta labor, Jesús los dejó durante tres o cuatro días para dirigirse a la ciudad costera de Beirut. Allí conversó con un sirio llamado Malac, un creyente que había estado en Betsaida el año anterior.
\vs p156 3:2 El miércoles 6 de julio, todos ellos regresaron a Sidón y se quedaron en la casa de Justa hasta el domingo por la mañana, momento en el que partieron por la costa, en dirección sur, hacia Tiro, a través de Sarepta. Llegaron a Tiro el lunes 11 de julio. Para entonces, los apóstoles y los evangelistas ya se estaban acostumbrando a trabajar entre los llamados gentiles, que, en su mayoría, descendían en realidad de las tempranas tribus cananeas de origen semita aún anterior. Todos estos pueblos hablaban la lengua griega. A los apóstoles y a los evangelistas les sorprendió sobremanera observar el entusiasmo de estos gentiles por oír el evangelio y la facilidad con la que muchos de ellos creían.
\usection{4. EN TIRO}
\vs p156 4:1 Desde el 11 al 24 de julio estuvieron en Tiro impartiendo sus enseñanzas. Cada uno de los apóstoles se llevó con él a uno de los evangelistas y, de este modo, de dos en dos, enseñaron y predicaron por toda la ciudad y sus inmediaciones. La multicultural población de este activo puerto marítimo los oía con agrado, y muchos se bautizaron en la hermandad social del reino. Jesús estableció su sede en la casa de un judío llamado José, un creyente que vivía a unos cinco o seis kilómetros al sur de Tiro, no lejos de la tumba de Hiram, que en los tiempos de David y Salomón había sido rey de la ciudad\hyp{}estado de Tiro.
\vs p156 4:2 Diariamente, durante este período de dos semanas, los apóstoles y evangelistas accedían a Tiro por el paso elevado de Alejandro para celebrar pequeñas reuniones y, cada noche, la mayoría de ellos regresaba al campamento de la casa de José, al sur de la ciudad. Todos los días, los creyentes dejaban la ciudad para hablar con Jesús en su lugar de descanso. El Maestro habló en Tiro solo una vez y a primera hora de la tarde del 20 de julio. Enseñó a los creyentes acerca del amor del Padre por toda la humanidad y acerca de la misión del Hijo de revelar al Padre a todas las razas de los hombres. Existía tanto interés por el evangelio del reino entre estos gentiles que ese día se abrieron para él las puertas del templo de Melcart y, cabe destacar, que, en años posteriores, se construyó una Iglesia cristiana en el mismo sitio de este antiguo templo.
\vs p156 4:3 Muchos de los principales fabricantes de púrpura de Tiro, el tinte que hizo famosa a dicha ciudad y a Sidón en todo el mundo, y que tanto contribuyó a su comercio a escala mundial y consiguiente enriquecimiento, creyeron en el reino. Cuando, poco tiempo después, empezó a reducirse el abastecimiento de los animales marinos de los que se extraía dicho tinte, sus fabricantes salieron en busca de los nuevos hábitats de estos moluscos. Y, de este modo, al emigrar a los confines de la tierra, llevaron con ellos el mensaje de la paternidad de Dios y de la hermandad del hombre ---el evangelio del reino---.
\usection{5. ENSEÑANZAS DE JESÚS EN TIRO}
\vs p156 5:1 Ese miércoles y a la caída de la tarde, en el trascurso de su sermón, Jesús contó primeramente a sus seguidores la historia del lirio blanco que alza alta su cabeza pura y nevada hacia el sol, mientras que sus raíces se hunden en el fango y el lodo del suelo oscuro. “Del mismo modo”, dijo, “el hombre mortal, aunque su origen y su ser estén enraizados en el suelo animal de la naturaleza humana, puede, por la fe, elevar su naturaleza espiritual al sol de la verdad celestial y realmente rendir los nobles frutos del espíritu”.
\vs p156 5:2 Durante este mismo sermón, Jesús utilizó la primera y única parábola relacionada con su propio oficio ---la carpintería---. Cuando los instaba a “construir buenos cimientos sobre los que pudiera crecer un carácter noble dotado espiritualmente”, él dijo: “Para rendir los frutos del espíritu, debéis haber nacido del espíritu. Si queréis vivir una vida llena del espíritu entre vuestros semejantes, debéis dejaros enseñar y guiar por el espíritu. Pero no cometáis el error del carpintero insensato que malgasta su valioso tiempo cuadrando, midiendo y alisando su madera carcomida por la polilla y podrida en su interior, para, después, tras haberse entregado esforzadamente a su trabajo con esta viga defectuosa, tiene que rechazarla por ser inadecuada para los cimientos del edificio que va a construir y que va a tener que soportar la agresión del tiempo y las tormentas. Que todo hombre se asegure de que los cimientos intelectuales y morales de su carácter sean de tal manera que puedan soportar adecuadamente la superestructura de la expansiva y ennoblecedora naturaleza espiritual, que, de ese modo, transformará la mente humana y, luego, en conjunción con esa mente de nueva creación, propiciará el desarrollo del alma, cuyo destino es la inmortalidad. Vuestra naturaleza espiritual ---el alma conjuntamente creada--- es crecimiento vivo, pero la mente y los principios morales de la persona son el terreno del que deben brotar estas manifestaciones superiores del desarrollo humano y del destino divino. El terreno del alma evolutiva es humano y material, pero el destino de esta criatura, compuesta de mente y espíritu, es espiritual y divino”.
\vs p156 5:3 A últimas horas de la tarde de ese mismo día, Natanael le preguntó a Jesús: “Maestro, ¿por qué le pedimos a Dios que no nos guíe a la tentación, cuando sabemos bien por tu revelación del Padre que él jamás hace tales cosas?”. Jesús le respondió:
\vs p156 5:4 “No es extraño que hagas estas preguntas, puesto que estás comenzando a conocer al Padre tal como yo lo conozco, y no tan vagamente como lo percibían los antiguos profetas hebreos. Sabes bien que nuestros ancestros estaban predispuestos a ver a Dios en casi cualquier cosa que sucediera. Buscaban la mano de Dios en todo acontecimiento natural y en cualquier episodio insólito de la experiencia humana. Asociaban a Dios tanto con el bien como con el mal. Creían que había ablandado el corazón de Moisés y endurecido el corazón del faraón. Cuando el hombre se sentía impulsado a hacer algo, ya fuese bueno o malo, tenía por costumbre justificar estas inusuales emociones diciendo: ‘El Señor me habló y me dijo, haz esto y aquello, o ve aquí y allá’. Por consiguiente, al caer el hombre, con tanta frecuencia e intensidad, en la tentación, nuestros antepasados se habituaron a creer que Dios los guiaba a ella para probarlos, castigarlos o fortalecerlos. Sabes que los hombres, demasiadas veces, se ven llevados a la tentación por los impulsos de su propio egoísmo y de su naturaleza animal. Cuando te sientas así tentado, te recomiendo que, al tiempo que reconozcas con honestidad y sinceridad la tentación por lo que es en sí, reconduzcas de manera inteligente las energías de tu espíritu, de tu mente y de tu cuerpo, que tratan de manifestarse, por senderos superiores y hacia metas más idealistas. Con ello, podrás transformar tus tentaciones en el más elevado e inspirador servicio a los mortales y evitarás, casi en su totalidad, esos vanos y debilitadores conflictos entre la naturaleza animal y la naturaleza espiritual.
\vs p156 5:5 “Pero me gustaría avisarte de la insensatez de tratar de vencer la tentación tratando de suplantar este deseo por otro supuestamente superior, sencillamente mediante la fuerza de la voluntad humana. Si en verdad deseas triunfar sobre las tentaciones de tu naturaleza inferior, debes llegar a un punto ventajoso espiritualmente en el que real y auténticamente hayas desarrollado interés y amor por esas formas superiores e ideales de conducta, que tu mente desea sustituir por dichos otros hábitos de conducta de inferior orden y menos idealistas, que reconoces como tentación. De esta manera, tu transformación espiritual te liberará en lugar de sobrecargarte con la falaz represión de los deseos mortales. Lo viejo y lo inferior se olvidarán a través del amor por lo nuevo y lo superior. La belleza siempre triunfa sobre la fealdad en los corazones de aquellos que están iluminados por el amor a la verdad. En la energía del afecto espiritual nuevo y sincero existe un gran poder que logra erradicar las viejas tentaciones. Nuevamente te digo, no te dejes vencer por el mal, sino que vence más bien al mal con el bien”.
\vs p156 5:6 Los apóstoles y los evangelistas continuaron con sus preguntas hasta bien entrada la noche y, de las muchas respuestas que Jesús les dio, reformulamos en términos modernos los siguientes pensamientos:
\vs p156 5:7 Los elementos esenciales para lograr el éxito material son ambición decidida, juicio inteligente y experimentada sabiduría. El liderazgo depende de la habilidad natural, del buen criterio, de la fuerza de voluntad y de la determinación; el destino espiritual, de la fe, el amor y la consagración a la verdad ---el hambre y la sed de rectitud---, el ferviente deseo de encontrar a Dios y semejarse a él.
\vs p156 5:8 No os desaniméis cuando descubrís que sois humanos. La naturaleza humana puede tender al mal pero no es intrínsecamente pecaminosa. No os sintáis desalentados si no podéis olvidar por completo algunas de vuestras experiencias más desafortunadas. Los errores que no podáis olvidar en el tiempo se olvidarán en la eternidad. Aliviad el peso del alma adquiriendo una más amplia visión de vuestro destino, considerando la expansión de vuestra andadura en el universo.
\vs p156 5:9 No cometáis el error de valorar el alma por las imperfecciones de la mente o por los apetitos del cuerpo. No juzguéis al alma ni evaluéis su destino por algún único y lamentable suceso humano. Vuestro destino espiritual solo está condicionado por vuestros deseos y propósitos espirituales.
\vs p156 5:10 La religión es una vivencia, exclusivamente espiritual, del alma inmortal evolutiva del hombre que conoce a Dios, pero la fortaleza moral y la energía espiritual son fuerzas poderosas que pueden emplearse para abordar situaciones sociales difíciles y resolver complejos problemas económicos. Estos atributos morales y espirituales enriquecen y dan mayor significación a todos los niveles de la vida humana.
\vs p156 5:11 Si solo aprendéis a amar a quienes os aman a vosotros, estáis destinados a vivir una vida angosta y miserable. De hecho, el amor humano puede ser recíproco, pero el amor divino busca su satisfacción al darse a los demás. Cuanto menor sea el amor que albergue la naturaleza de cualquier criatura, más amor necesitará, y más satisfará el amor divino esa necesidad. El amor nunca es egoísta, y no puede otorgarse a sí mismo. El amor divino no puede contenerse en sí mismo; debe darse desinteresadamente.
\vs p156 5:12 Los creyentes del reino deben tener una fe inquebrantable, creer con toda el alma en el triunfo cierto de la rectitud. Los constructores del reino no deben albergar dudas sobre la verdad del evangelio de la salvación eterna. Los creyentes deben aprender, crecientemente, a apartarse del ajetreo de la vida ---escapar del hostigamiento de la existencia material---, mientras que revitalizan sus almas, inspiran sus mentes y renuevan sus espíritus mediante la comunión y la adoración.
\vs p156 5:13 Quienes conocen a Dios no se desaniman ante los infortunios ni se dejan abatir por las decepciones. Los creyentes son inmunes a la depresión que resulta de trastornos puramente materiales; quienes viven en el espíritu no se dejan contrariar por los sucesos del mundo material. Estos aspirantes a la vida eterna llevan a la práctica un modo vigorizante y constructivo de enfrentarse a las vicisitudes y acosos de la vida mortal. Cada día que un verdadero creyente vive, halla \bibemph{más fácil} hacer lo que es recto.
\vs p156 5:14 La vida espiritual incrementa fuertemente el verdadero respeto por sí mismo. Pero este respeto no significa admiración por uno mismo, sino que se conjuga con el amor y el servicio a los demás. No es posible respetarse a sí mismo más de lo que se puede amar al prójimo; el uno es la medida de la capacidad del otro.
\vs p156 5:15 Con el paso del tiempo, el verdadero creyente será más capaz de atraer a sus semejantes hacia el amor de la verdad eterna. ¿No sois más imaginativos hoy en vuestra revelación de la bondad a la humanidad de lo que fuisteis ayer? ¿Sabéis mejor recomendar la rectitud este año que el pasado? ¿Os hacéis más y más creativos en el modo en el que guiais a las almas sedientas al reino espiritual?
\vs p156 5:16 ¿Son vuestros ideales suficientemente elevados como para garantizaros la salvación eterna, a la vez que vuestras ideas son tan prácticas como para llegar a ser un ciudadano útil, que actúa en la tierra en cooperación con vuestros semejantes mortales? En el espíritu, tenéis vuestra ciudadanía en el cielo; en la carne, todavía sois ciudadanos de los reinos terrestres. Dad a los césares las cosas materiales y, a Dios, las espirituales.
\vs p156 5:17 Vuestra fe en la verdad y vuestro amor por el prójimo miden la capacidad espiritual del alma evolutiva, pero la fuerza del carácter humano mide vuestra capacidad para refrenaros del resentimiento y de la indignación ante la aflicción más profunda. La derrota es el verdadero espejo en el que podéis honestamente ver vuestro auténtico yo.
\vs p156 5:18 A medida que os hacéis mayor y os volvéis más experimentados en los asuntos del reino, ¿tendréis más tacto en vuestro trato con mortales problemáticos y más tolerancia hacia esas personas obstinadas con la que convivís? El tacto es necesario para mejorar las relaciones sociales. La tolerancia es el rasgo característico de la grandeza del alma. Si poseéis estos dones, poco frecuentes y encantadores, conforme pase el tiempo, os volveréis más alertas y mejores expertos en vuestro encomiable empeño por evitar innecesarios malentendidos sociales. Estas sensatas almas evitan en sí mismas muchas de las aflicciones como el desequilibrio emocional, que padecen quienes se niegan a crecer y se resisten a envejecer con dignidad.
\vs p156 5:19 Evitad la falta de honradez y de equidad cuando dediquéis vuestros esfuerzos a predicar la verdad y a proclamar el evangelio. No busquéis un reconocimiento no ganado y ni ansiéis una inmerecida compasión. De gracia recibisteis el amor de las fuentes divinas y humanas al margen de vuestros méritos, y amad de gracia a cambio. Pero en todas las demás cosas relacionadas con el honor y la adulación, buscad solo aquello que honradamente os pertenezca.
\vs p156 5:20 El mortal que es consciente de Dios está seguro de su salvación; no teme a la vida; es honrado y constante. Sabe cómo tolerar valientemente sufrimientos inevitables; no se queja cuando se enfrenta a ineludibles penurias.
\vs p156 5:21 El verdadero creyente no se cansa de hacer el bien simplemente porque se vea obstaculizado. Las dificultades estimulan el fervor del amante de la verdad, mientras que los obstáculos hacen que se redoblen los esfuerzos de este tenaz constructor del reino.
\vs p156 5:22 \pc Y Jesús les impartió otras muchas enseñanzas antes de disponerse a salir de Tiro.
\vs p156 5:23 El día previo a su salida de Tiro para regresar a la región del mar de Galilea, Jesús convocó a sus acompañantes y mandó a los doce evangelistas a que regresasen por una ruta diferente a la que él y los doce apóstoles tomarían. Y tras despedirse allí de Jesús, estos evangelistas jamás volverían a tener un contacto tan estrecho con él.
\usection{6. REGRESO DE FENICIA}
\vs p156 6:1 Hacia el mediodía del domingo, 24 de julio, Jesús y los doce dejaron la casa de José, situada al sur de Tiro, y bajaron a Tolemaida por la costa. Allí se quedaron durante un día, consolando con sus palabras al grupo de creyentes que residía allí. Pedro les predicó a últimas horas de la tarde del 25 de julio.
\vs p156 6:2 El martes salieron de Tolemaida y se dirigieron por el este, tierra adentro, por la carretera de Tiberias, hasta cerca de Jotapata. El miércoles se detuvieron en Jotapata e instruyeron más a fondo a los creyentes sobre las cosas del reino. El jueves salieron de Jotapata, en dirección norte, por el sendero de Nazaret al monte Líbano, hasta llegar, a través de Ramá, a la aldea de Zabulón. El viernes mantuvieron reuniones con los creyentes en Ramá, población en la que pasaron el \bibemph{sabbat}. Llegaron a Zabulón el domingo 31. Hacia la noche, se reunieron de nuevo y partieron a la mañana siguiente.
\vs p156 6:3 Una vez que salieron de Zabulón, viajaron hasta el cruce con la carretera Magdala a Sidón, cerca de Giscala y, desde allí, se encaminaron por la costa occidental del lago de Galilea a Genesaret, en el sur de Cafarnaúm, donde habían acordado que se encontrarían con David Zebedeo, y decidir sobre el próximo paso a dar en su labor de predicar el evangelio del reino.
\vs p156 6:4 Durante su breve encuentro con David, supieron que había muchos líderes de los creyentes congregados al otro lado del lago, cerca de Queresa y, por lo tanto, avanzada aquella misma tarde, tomaron una barca para cruzar el lago. Estuvieron todo un día descansando en las colinas y, al día siguiente, se dirigieron al parque cercano donde el Maestro había dado de comer a los cinco mil. Allí reposaron tres días, al tiempo que impartían charlas diarias a las que asistían unos cincuenta hombres y mujeres, los creyentes restantes de Cafarnaúm y de sus inmediaciones, del que había sido un nutrido grupo.
\vs p156 6:5 \pc Mientras Jesús estuvo ausente de Cafarnaúm y de Galilea, durante su período de estancia en Fenicia, sus enemigos supusieron que todo el movimiento se había disuelto y llegaron a la conclusión de que su salida precipitada mostraba que estaban tan aterrorizados que probablemente nunca volverían de nuevo a darles problemas. La intensa oposición a la que habían estado sometidas sus enseñanzas había casi desaparecido. Una vez más, los creyentes comenzaban a reunirse públicamente y se estaba logrando una paulatina consolidación de estos verdaderos y probados creyentes del evangelio, los supervivientes de la gran criba recién ocurrida.
\vs p156 6:6 Felipe, el hermano de Herodes, se había convertido en creyente, aunque poco convencido, de Jesús y le envió palabra al Maestro de que era libre de vivir y realizar su labor en sus dominios.
\vs p156 6:7 La orden de cerrar las sinagogas de toda Palestina a las enseñanzas de Jesús y a sus seguidores había tenido consecuencias adversas para los escribas y los fariseos. Al retirarse Jesús para no ser él mismo objeto de controversias, se produjo de inmediato una reacción de todo el pueblo judío; se generó un sentimiento de indignación generalizada contra los fariseos y los líderes del sanedrín de Jerusalén. Muchos de los jefes de las sinagogas comenzaron, subrepticiamente, a abrir sus sinagogas a Abner y a sus compañeros, argumentando que estos maestros eran seguidores de Juan, y no discípulos de Jesús.
\vs p156 6:8 Incluso Herodes Antipas experimentó un cambio de actitud y, al enterarse de que Jesús residía al otro lado del lago, en el territorio de su hermano Felipe, envió a este un mensaje informándole de que, aunque había firmado órdenes para que se le arrestara en Galilea, no había autorizado su detención en Perea, dando, pues, a conocer que no se molestaría a Jesús si permanecía fuera de Galilea. Herodes comunicó esta misma resolución a los judíos de Jerusalén.
\vs p156 6:9 Y aquella era la situación hacia el comienzo de agosto del año 29 d.C., cuando el Maestro regresó de su misión en Fenicia y comenzó a reorganizar a su grupo de seguidores, en aquel momento dispersos, probados y exhaustos, para este último e intenso año de su misión en la tierra.
\vs p156 6:10 Los términos del enfrentamiento estaban bien definidos conforme el Maestro y sus acompañantes se preparaban para proclamar una nueva religión, la religión del espíritu del Dios vivo que habita en las mentes de los hombres.
