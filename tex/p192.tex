\upaper{192}{Apariciones en Galilea}
\author{Comisión de seres intermedios}
\vs p192 0:1 Cuando los apóstoles dejaron Jerusalén y se encaminaron a Galilea, los líderes judíos ya estaban bastante calmados. Puesto que Jesús se apareció solo a su familia de creyentes del reino y, dado que los apóstoles estaban escondidos y no predicaban públicamente, los dirigentes de los judíos llegaron a la conclusión de que el movimiento evangélico había resultado finalmente aplastado del todo. Evidentemente, estaban desconcertados por los rumores, cada vez más extendidos, de que Jesús había resucitado de entre los muertos, pero confiaban en que los guardias a los que habían sobornado lograrían contrarrestar todas estas noticias, repitiendo la historia de que un grupo de seguidores suyos se había llevado el cuerpo.
\vs p192 0:2 Desde este momento, hasta la llegada de la creciente oleada de persecuciones que dispersó a los apóstoles, se reconoció generalmente a Pedro como cabeza del cuerpo apostólico. Jesús jamás le había concedido tal autoridad, y sus hermanos apóstoles nunca lo habían escogido formalmente para tal puesto de responsabilidad en el grupo; él lo asumió de manera natural y lo ostentó de común acuerdo, siendo además su principal predicador. En lo sucesivo, la predicación pública se convirtió en la labor más importante de los apóstoles. Tras regresar a Galilea, Matías, elegido para ocupar el lugar de Judas, se convirtió en su tesorero.
\vs p192 0:3 Durante la semana en la que permanecieron en Jerusalén, María la madre de Jesús, pasó mucho tiempo con las mujeres creyentes que se alojaban en la casa de José de Arimatea.
\vs p192 0:4 Ese lunes, temprano por la mañana, cuando los apóstoles partieron para Galilea, Juan Marcos fue tras ellos. Los siguió hasta la salida de la ciudad y, una vez recorrido un buen trecho tras pasar Betania, se unió a ellos resueltamente, sintiéndose seguro de que ya no lo enviarían de vuelta.
\vs p192 0:5 De camino a Galilea, los apóstoles se detuvieron varias veces para contar la historia de su Maestro resucitado a la gente y, por este motivo, no llegaron a Betsaida hasta muy avanzada la noche del miércoles. El jueves, no antes del mediodía, estaban todos despiertos y listos para desayunar juntos.
\usection{1. APARICIÓN JUNTO AL LAGO}
\vs p192 1:1 El viernes 21 de abril, sobre las seis de la mañana, el Maestro morontial hizo su decimotercera aparición, la primera que realizaba en Galilea, a los diez apóstoles en el momento en el que su barca se aproximaba a la orilla, cerca del lugar habitual de desembarque de Betsaida.
\vs p192 1:2 Estando los apóstoles en espera en la casa de Zebedeo desde la tarde hasta las primeras horas de la noche del jueves, Simón Pedro les sugirió que fueran a pescar. Todos los apóstoles aceptaron y decidieron acompañarlo. Durante toda esa noche, estuvieron faenando con las redes, pero no lograron pescar nada. No le dieron mucha importancia al hecho de no haber podido hacer ninguna captura de peces, porque tenían muchas experiencias interesantes de las que hablar, cosas que tan recientemente les habían sucedido en Jerusalén. Pero, al despuntar el alba, tomaron la determinación de volver a Betsaida. Al ir aproximándose a la orilla, vieron a alguien en la playa, cerca del embarcadero, de pie junto a una hoguera. En un principio, creyeron que era Juan Marcos, que había venido a recibirlos de vuelta con su pesca, pero, cuando estaban más cerca de la orilla, observaron que estaban equivocados ---el hombre era demasiado alto para ser Juan---. A nadie se le había ocurrido pensar que la persona que estaba allí era el Maestro. No entendían muy bien por qué Jesús quería encontrarse con ellos en los lugares donde habían estado juntos en el pasado y al aire libre, en contacto con la naturaleza, alejados del entorno cerrado de la ciudad de Jerusalén, tan trágicamente relacionada con el miedo, la traición y la muerte. Jesús les había dicho que, si iban a Galilea, allí se vería con ellos, y estaba a punto de cumplir esa promesa.
\vs p192 1:3 Conforme echaban el ancla y se preparaban para subir al pequeño bote para dirigirse a tierra firme, el hombre de la playa alzando la voz, les preguntó: “Muchachos, ¿habéis pescado algo?”. Y cuando le contestaron, “No”, él añadió, “Echad la red a la derecha de la barca, y hallaréis peces”. Aunque no sabían que era Jesús quien se lo había recomendado, ellos, de común acuerdo, echaron la red siguiendo estas instrucciones, y, enseguida, esta se llenó de peces, de tal manera, que apenas si podían sacarla. Juan Zebedeo era muy perspicaz y, al ver la red tan cargada, comprendió que era el Maestro quien les había hablado. Cuando le vino ese pensamiento a la mente, se inclinó sobre Pedro y le susurró: “Es el Maestro”. Pedro, habiendo sido siempre un hombre de actos irreflexivos y muy impulsivo en su lealtad al Maestro, en cuanto Juan le susurró aquello al oído, se levantó con rapidez y se tiró al agua para poder llegar lo antes posible junto a él. En la barca pequeña, que arrastraba la red con los peces, sus hermanos lo siguieron de cerca y alcanzaron la orilla.
\vs p192 1:4 A estas horas, Juan Marcos ya se había levantado y, viendo a los apóstoles que llegaban a la costa con la red repleta de peces, corrió por la playa para recibirlos; y, cuando vio a once hombres en lugar de diez, dedujo que el desconocido era Jesús resucitado y, mientras los diez, atónitos, permanecieron en silencio, el joven se apresuró a ir hacia el Maestro, y se arrodilló a sus pies, diciendo: “Señor mío y Maestro mío”. Y entonces Jesús habló, no como lo había hecho en Jerusalén, cuando los saludaba diciendo “La paz esté con vosotros”, sino que, en un tono afable, se dirigió a Juan Marcos, diciéndole: “Bien, Juan, me alegro de verte de nuevo en la tranquila Galilea, donde podemos disfrutar de una buena charla. Quédate con nosotros, Juan, y desayuna”.
\vs p192 1:5 Mientras Jesús hablaba con el joven, los diez estaban tan impresionados y sorprendidos que se olvidaron de acarrear hasta la playa la red con los peces. En ese momento, Jesús les dijo: “Traed vuestros peces y preparad algunos para el desayuno. El fuego ya está prendido y tenemos bastante pan”.
\vs p192 1:6 Mientras Juan Marcos le rendía homenaje al Maestro, Pedro se conmocionó un momento a la vista de las resplandecientes brasas de carbón en la playa; la escena le recordó vívidamente la hoguera del patio de Anás aquella medianoche en la que había renegado del Maestro. No obstante, se recuperó de aquel pensamiento y, arrodillándose a los pies del Maestro, exclamó: “¡Señor mío y Maestro mío!”.
\vs p192 1:7 Pedro se unió luego a sus compañeros que ya sacaban la red. Cuando descargaron en tierra su captura, contaron los peces, y había ciento cincuenta de los grandes. Y de nuevo se cometió el error de llamar pesca milagrosa a lo sucedido sin tratarse de ningún milagro. Fue sencillamente un acto de precognición del Maestro. Él sabía que los peces estaban allí y les indicó adónde debían echar la red.
\vs p192 1:8 Jesús les habló, diciendo: “Venid ahora, todos vosotros, a desayunar. Los gemelos también deben sentarse mientras yo converso con vosotros; Juan Marcos preparará el pescado”. Juan Marcos trajo siete peces de buen tamaño, que el Maestro colocó sobre las brasas y, cuando estuvieron cocinados, el muchacho se los sirvió a los diez. Entonces, Jesús partió el pan y se lo entregó a Juan que, a su vez, se lo sirvió a los hambrientos apóstoles. Cuando todos estuvieron servidos, Jesús le pidió a Juan Marcos que se sentara mientras él mismo le servía al muchacho el pescado y el pan. Y mientras comían, Jesús habló con ellos, recordando muchas de las experiencias vividas en Galilea y junto a este mismo lago.
\vs p192 1:9 \pc Aquella fue la tercera vez que Jesús se manifestaba a los apóstoles como grupo. Cuando Jesús se dirigió a ellos por primera vez, preguntándoles si habían pescado algo, no sospecharon que se trataba de él, porque era algo común entre los pescadores del mar de Galilea, cuando desembarcaban, que los comerciantes de pescado de Tariquea se dirigieran a ellos, dispuestos normalmente a comprarles la pesca fresca para llevarla a los secaderos.
\vs p192 1:10 \pc Jesús habló con los diez apóstoles y Juan Marcos durante más de una hora y, entonces, de dos en dos, paseó con ellos yendo y viniendo por la playa mientras conversaban ---pero no eran las mismas parejas que en un principio había enviado a enseñar---. Los once apóstoles habían venido juntos desde Jerusalén, pero Simón Zelotes, a medida que se aproximaban a Galilea, más decaído se encontraba, por lo que, cuando llegaron a Betsaida, abandonó a sus hermanos y regresó a su casa.
\vs p192 1:11 Esa mañana, antes de despedirse de ellos, Jesús les pidió que dos de los apóstoles se ofrecieran para ir a buscar a Simón Zelotes, y lo trajeran de vuelta ese mismo día. Y esto hicieron Pedro y Andrés.
\usection{2. CHARLAS DE DOS EN DOS CON LOS APÓSTOLES}
\vs p192 2:1 Cuando acabaron de desayunar y, mientras los demás estaban sentados junto al fuego, Jesús hizo señas a Pedro y a Juan para que lo acompañaran a pasear por la playa. Mientras caminaban, Jesús le dijo a Juan: “Juan, ¿me amas?”. Y cuando Juan contestó: “Sí, Maestro, con todo mi corazón”, el Maestro dijo: “Entonces, Juan, renuncia a tu intolerancia y aprende a amar a los hombres tal como yo te he amado a ti. Consagra tu vida a demostrar que el amor es la cosa más grande del mundo. Es el amor de Dios el que mueve a los hombres a buscar la salvación. El amor es el predecesor de toda bondad espiritual, la esencia de lo verdadero y de lo bello”.
\vs p192 2:2 Jesús se volvió entonces a Pedro y le preguntó: “Pedro, ¿me amas?”. Pedro contestó: “Señor, sabes que te amo con toda mi alma”. Entonces, le dijo Jesús: “Si me amas, Pedro, apacienta a mis corderos. No desatiendas a los débiles, a los pobres, a los jóvenes y a los niños. Predica el evangelio sin temor ni favoritismos; recuerda siempre que Dios no hace acepción de personas. Sirve a tus semejantes tal como yo te he servido a ti; perdona a tus compañeros mortales tal como yo te he perdonado a ti. Que la experiencia te enseñe a valorar la meditación y el poder de la reflexión inteligente”.
\vs p192 2:3 Tras caminar algo más por allí, el Maestro se volvió otra vez a Pedro y le preguntó: “Pedro, ¿realmente me amas?”. Y, entonces, Simón dijo: “Sí, Señor, tú sabes que te amo”. Y nuevamente dijo Jesús: “Pastorea bien a mis ovejas. Sé un pastor bueno y verdadero para el rebaño. No traiciones la confianza que ponen en ti. Que no te tome por sorpresa la mano del enemigo. Mantente en guardia en todo momento ---vigila y ora---”.
\vs p192 2:4 Después de continuar unos pasos más, Jesús se volvió a Pedro, por tercera vez, y le preguntó: “Pedro, ¿me amas de verdad?”.Y entonces Pedro, algo entristecido por la aparente desconfianza del Maestro, le dijo embargado por la emoción: “Señor, tú lo sabes todo, y sabes por tanto que te amo real y verdaderamente”. Entonces, dijo Jesús: “Apacienta mis ovejas. No las abandones. Sé un ejemplo y una inspiración para todos tus compañeros pastores. Ama al rebaño tal como yo te he amado a ti y dedícate a su bien tal como yo he dedicado mi vida al tuyo. Y sígueme hasta incluso el fin”.
\vs p192 2:5 Pedro tomó estas últimas palabras de forma literal ---que debía continuar siguiéndole--- y volviéndose a Jesús, señaló a Juan preguntando: “Si te sigo, ¿qué será de este hombre?”. Y, entonces, dándose cuenta de que Pedro había interpretado mal sus palabras, Jesús dijo: “Pedro, no te preocupes por lo que tus hermanos hagan. Si yo deseo que Juan se quede aquí después de que tú te hayas ido, incluso hasta que yo vuelva, ¿qué a ti? Solo asegúrate de seguirme”.
\vs p192 2:6 \pc Este comentario se extendió entre los hermanos y se vio como una afirmación de que Juan no moriría antes de que el Maestro regresara, como muchos pensaban y esperaban, para instaurar el reino en poder y gloria. Esta interpretación de lo que había dicho Jesús contribuyó bastante a que Simón Zelotes volviera de nuevo al servicio y continuara con su labor.
\vs p192 2:7 \pc Cuando regresaron donde estaban los otros, Jesús dio un paseo con Andrés y Santiago con la intención de charlar con ellos. Cuando habían caminado una corta distancia, Jesús le dijo a Andrés: “Andrés, ¿confías en mí?”. Y cuando el antiguo jefe de los apóstoles oyó la pregunta que Jesús le hacía, se quedó quieto y contestó: “Sí, Maestro, ciertamente confío en ti, y tú sabes que sí”. Entonces, le dijo Jesús: “Andrés, si confías en mí, confía más en tus hermanos ---incluso en Pedro---. Hace tiempo, te confié el liderazgo de tus hermanos. Ahora debes tú confiar en los demás cuando yo te deje para ir al Padre. En el momento en el que tus hermanos comiencen a dispersarse por todos lados a causa de las encarnizadas persecuciones, sé un consejero respetuoso y sensato para Santiago, mi hermano en la carne, cuando le impongan sobre sus hombros unas pesadas cargas, a las que no está capacitado por falta de experiencia para soportarlas. Y, entonces, sigue confiando, porque yo no te fallaré. Cuando hayas terminado en la tierra, vendrás a mí”.
\vs p192 2:8 Luego Jesús se volvió a Santiago, y le preguntó: “Santiago ¿confías en mí?”. Y, sin dudar, Santiago respondió: “Sí, Maestro, confío en ti con todo mi corazón”. Entonces, Jesús le dijo: “Santiago, si confiaras más en mí, serías menos impaciente con tus hermanos. Si confías en mí, serás más compasivo con la hermandad de los creyentes. Aprende a sopesar las consecuencias de tus palabras y de tus actos. Recuerda que se siega según se siembra. Ora por tu tranquilidad de espíritu y cultiva la paciencia. Estas gracias, junto con la fe viva, te sostendrán cuando llegue la hora de beber la copa del sacrificio. Pero nunca desmayes; cuando hayas terminado en la tierra, vendrás también para estar conmigo”.
\vs p192 2:9 \pc Luego, Jesús habló con Tomás y Natanael. A Tomás le dijo: “Tomás, ¿me sirves?”. Tomás respondió: “Sí, Señor, yo te sirvo ahora y siempre”. Entonces, le dijo Jesús: “Si quieres servirme, sirve a mis hermanos en la carne tal como yo te he servido a ti. Y no desfallezcas a la hora de obrar el bien, sino que persevera como quien ha sido consagrado por Dios para realizar este servicio de amor. Cuando hayas terminado tu servicio conmigo en la tierra, servirás conmigo en la gloria. Tomás, debes dejar de dudar; debes crecer en la fe y en el conocimiento de la verdad. Cree en Dios como si fueras un niño, pero deja de actuar de forma tan infantil. Ten arrojo; sé fuerte en la fe y poderoso en el reino de Dios”.
\vs p192 2:10 Entonces, el Maestro le dijo a Natanael: “Natanael, ¿me sirves?”. Y el apóstol contestó: “Sí, Maestro, y con mi total afecto”. Entonces Jesús le dijo: “Si, por lo tanto, me sirves con todo tu corazón, asegúrate de que te dedicas con infatigable afecto al bien de mis hermanos de la tierra. Incluye la amistad en tus consejos y añade el amor a tu filosofía. Sirve a tus semejantes tal como yo te he servido a ti. Sé leal cuando te ocupes de los hombres, tal como me he ocupado de ti. Sé menos crítico; espera menos de algunas personas, y solo así sufrirás menos decepciones. Y cuando tu labor aquí abajo haya finalizado, servirás conmigo en lo alto”.
\vs p192 2:11 \pc Tras esto, el Maestro habló con Mateo y Felipe. A Felipe le dijo: “Felipe, ¿tú me obedeces?”. Felipe contestó: “Sí, Señor, te obedeceré incluso con mi vida”. Entonces, Jesús le dijo: “Si quieres obedecerme, ve, por tanto, a las tierras de los gentiles y proclama este evangelio. Los profetas os han dicho que la obediencia es mejor que los sacrificios. Por la fe te has convertido en un hijo del reino, conocedor de Dios. Hay tan solo una ley que obedecer, y es el mandamiento de salir a proclamar el evangelio del reino. No temas más a los hombres; no tengas miedo de predicar la buena nueva de la vida eterna a esos semejantes tuyos que languidecen en la oscuridad y están hambrientos de la luz de la verdad. Nunca más, Felipe, tendrás que ocuparte de dinero ni de bienes. Eres libre ahora de predicar la buena nueva al igual que tus hermanos. Y yo iré delante de ti y estaré contigo incluso hasta el fin”.
\vs p192 2:12 Y entonces, dirigiéndose a Mateo, el Maestro le preguntó: “Mateo, ¿abrigas en tu corazón el deseo de obedecerme?”. Mateo respondió: “Sí, Señor, estoy enteramente dedicado a hacer tu voluntad”. Luego, el Maestro le dijo: “Mateo, si quieres obedecerme, sal a enseñar este evangelio del reino a todos los pueblos. Nunca más tendrás que servir a tus hermanos en las cosas materiales de la vida; en adelante, debes también proclamar la buena nueva de la salvación espiritual. Desde ahora, ten puesta tu mirada solo en obedecer tu comisión de predicar este evangelio del reino del Padre. Al igual que yo he hecho la voluntad del Padre en la tierra, así cumplirás tú esta encomienda divina. Recuerda que tanto los judíos como los gentiles son tus hermanos. No sientas temor de nadie cuando proclames las verdades salvíficas del evangelio del reino de los cielos. Y al lugar donde yo voy, pronto vendrás tú”.
\vs p192 2:13 \pc Después caminó y habló con los gemelos Alfeo, Santiago y Judas, y, dirigiéndose a los dos, preguntó: “Santiago y Judas, ¿creéis en mí?”. Y cuando ambos respondieron: “Sí, Maestro, de cierto creemos”, él les dijo: “Pronto os dejaré. Veis que ya os he dejado en la carne. Permaneceré solo un corto período de tiempo en esta forma antes de ir al Padre. Creéis en mí; sois mis apóstoles y siempre lo seréis. Continuad creyendo y recordando vuestra relación conmigo cuando me haya ido, después de volver, posiblemente, a la tarea que realizabais antes de venir a vivir conmigo. Jamás permitáis que el cambio a un trabajo corriente pueda influir en vuestra lealtad. Tened fe en Dios hasta el fin de vuestros días en la tierra. Nunca os olvidéis de que, cuando eres un hijo de Dios por la fe, todo trabajo honrado del mundo es sagrado. Nada de lo que hace un hijo de Dios puede ser ordinario. Haced, por consiguiente, vuestro trabajo, a partir de ahora, como si fuera para Dios. Y cuando hayáis terminado en este mundo, tengo otros mundos mejores, donde trabajaréis igualmente para mí. Y en toda esta labor, en este mundo y en los otros, yo trabajaré con vosotros, y mi espíritu vivirá en vosotros”.
\vs p192 2:14 \pc Eran casi las diez cuando Jesús volvió de su charla con los gemelos Alfeo y, al dejar a los apóstoles, él les dijo: “Adiós, hasta que os encuentre a todos mañana al mediodía en el monte donde fuisteis ordenados”. Cuando dijo estas cosas, desapareció de su vista.
\usection{3. EN EL MONTE DE LA ORDENACIÓN}
\vs p192 3:1 El sábado 22 de abril, al mediodía, los once apóstoles se congregaron, como estaba previsto, en la colina cerca de Cafarnaúm, y Jesús se apareció entre ellos. Este encuentro tuvo lugar en el mismo monte donde el Maestro los escogió como apóstoles suyos y como embajadores del reino del Padre en la tierra. Y aquella fue la decimocuarta manifestación morontial del Maestro.
\vs p192 3:2 En esta ocasión, los once apóstoles se arrodillaron formando un círculo en torno al Maestro. Le oyeron repetir la comisión dada entonces a cada uno y lo vieron recrear el momento de la ordenación tal como fue cuando los eligió para la especial labor del reino. Para ellos, todo fue como un recordatorio de su anterior consagración al servicio del Padre, exceptuando la oración del Maestro. Cuando el Maestro ---el Jesús morontial--- oró ahora, lo hizo en un tono majestuoso y con palabras de poder jamás oídas antes por ellos. Su Maestro hablaba en ese momento con los gobernantes de los universos como quien, en su propio universo, tuviera encomendado a sus manos todo el poder y la autoridad. Y estos once hombres jamás olvidarían esta experiencia cuando el Jesús morontial les consagró de nuevo a sus anteriores promesas como embajadores del reino. El Maestro estuvo solamente una hora en este monte con sus embajadores y, tras despedirse afectuosamente de ellos, desapareció de su vista.
\vs p192 3:3 \pc Y nadie vio a Jesús durante toda una semana. Los apóstoles no tenían en realidad idea alguna de qué hacer, desconociendo además si el Maestro había ido al Padre. En este estado de incertidumbre, se quedaron en Betsaida. Temían ir de pesca por si llegaba y no lo veían. Durante toda esa semana completa, Jesús estuvo ocupado con las criaturas morontiales que se encontraban en la tierra y con cuestiones relativas al tránsito morontial que él estaba experimentando en este mundo.
\usection{4. ENCUENTRO JUNTO AL LAGO}
\vs p192 4:1 La noticia de las apariciones de Jesús se estaba extendiendo por toda Galilea y, cada día, eran más los creyentes que llegaban a la casa de Zebedeo para inquirir sobre la resurrección del Maestro y conocer la verdad de estas supuestas apariciones. A comienzos de la semana, Pedro informó a los creyentes de que se celebraría una reunión pública junto al lago el siguiente \bibemph{sabbat} a las tres de la tarde.
\vs p192 4:2 Por consiguiente, el sábado 29 de abril, a las tres de la tarde, más de quinientos creyentes de las inmediaciones de Cafarnaúm se congregaron en Betsaida para oír a Pedro predicar su primer sermón público desde la resurrección. Estaba en su mejor momento y, después de concluir su atrayente discurso, a pocos de sus oyentes aún les quedaban dudas de que el Maestro había resucitado de entre los muertos.
\vs p192 4:3 Pedro acabó su sermón diciendo: “Afirmamos que Jesús de Nazaret no está muerto; declaramos que ha resucitado de la tumba; proclamamos que lo hemos visto y hemos hablado con él”. Justo en el momento en el que acababa de pronunciar esta declaración de fe, allí a su lado, a la vista de toda la gente, apareció el Maestro en forma morontial y, hablándoles con una cadencia que les era familiar, dijo: “La paz sea con vosotros, y mi paz os dejo”. Tras aparecer así y decir estas cosas, desapareció de su vista. Aquella era la decimoquinta manifestación morontial de Jesús resucitado.
\vs p192 4:4 \pc Debido a ciertas cosas que Jesús les dijo a los once mientras conversaban con él en el monte de la ordenación, los apóstoles entendieron que su Maestro haría en breve una aparición pública ante un grupo de creyentes galileos y que, una vez acontecida esta, ellos deberían regresar a Jerusalén. Por lo tanto, al día siguiente, domingo 30 de abril, los once salieron temprano de Betsaida en dirección a Jerusalén. Mientras caminaban aguas abajo a lo largo del Jordán, iban en gran medida enseñando y predicando, así que no llegaron a la casa de los Marcos en Jerusalén hasta el miércoles, 3 de mayo, ya tarde.
\vs p192 4:5 \pc Para Juan Marcos, fue una triste vuelta a casa. Justo unas pocas horas antes de llegar, su padre, Elías Marcos, había muerto de repente de una hemorragia cerebral. Aunque la certeza de la resurrección de la muerte ayudó a consolar la pena de los apóstoles, al mismo tiempo, lloraron sinceramente la pérdida de este buen amigo, que había sido su inquebrantable defensor incluso en momentos de grandes dificultades y decepciones. Juan Marcos intentó en todo lo posible dar consuelo a su madre y, hablando en su nombre, invitó a los apóstoles a que siguieran considerando aquella casa como su propio hogar. Y los once hicieron del aposento alto su sede hasta después del día de Pentecostés.
\vs p192 4:6 \pc Los apóstoles, intencionadamente, habían entrado en Jerusalén después de la caída de la noche, para no ser vistos por las autoridades judías. Tampoco se mostraron en público en relación con el funeral de Elías Marcos. Durante todo el siguiente día, permanecieron en este memorable aposento alto sosegadamente aislados.
\vs p192 4:7 El jueves por la noche, los apóstoles celebraron una formidable reunión en el aposento alto, y todos se comprometieron a salir a predicar públicamente el nuevo evangelio del Señor resucitado salvo Tomás, Simón Zelotes y los gemelos Alfeo. Ya habían comenzado a dar los primeros pasos que transformarían el evangelio del reino ---la filiación con Dios y la hermandad con el hombre--- en la proclamación de la resurrección de Jesús. Natanael se opuso a esta modificación de la esencia del mensaje público, pero no pudo nada hacer frente a la elocuencia de Pedro ni pudo vencer el entusiasmo de los discípulos, en especial el de las mujeres creyentes.
\vs p192 4:8 Y así, bajo el pujante liderazgo de Pedro y antes de que el Maestro ascendiera al Padre, sus bien intencionados representantes iniciaron ese sutil proceso de transformar, de forma gradual y cierta, la religión \bibemph{de} Jesús en una forma de religión, nueva y modificada, \bibemph{sobre} Jesús.
