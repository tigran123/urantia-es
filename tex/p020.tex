\upaper{20}{Los Hijos de Dios del Paraíso}
\author{Perfeccionador de la sabiduría}
\vs p020 0:1 En cuanto a su labor en el suprauniverso de Orvontón, los Hijos de Dios se clasifican según los siguientes apartados generales:
\vs p020 0:2 \li{1.}Hijos descendentes de Dios.
\vs p020 0:3 \li{2.}Hijos ascendentes de Dios.
\vs p020 0:4 \li{3.}Hijos trinitizados de Dios.
\vs p020 0:5 \pc En los órdenes de filiación que descienden se incluyen los seres personales creados directamente por medios divinos. Los hijos que ascienden, tales como las criaturas mortales, adquieren esta condición participando de forma experiencial en el método creativo conocido como evolución. Los hijos trinitizados forman un grupo de origen compuesto que incluye a todos los seres acogidos por la Trinidad del Paraíso, aunque no hayan tenido su origen directamente en esta.
\usection{1. LOS HIJOS DESCENDENTES DE DIOS}
\vs p020 1:1 Todos los hijos descendentes de Dios tienen orígenes elevados y divinos. Descienden para dedicarse al ministerio de servir a los mundos y sistemas del tiempo y del espacio y facilitar allí el progreso, la subida al Paraíso, de las modestas criaturas de origen evolutivo, de los hijos ascendentes de Dios. De los numerosos órdenes de hijos descendentes, describiremos en estas narraciones a siete de ellos. A estos hijos, que provienen de las Deidades de la Isla central de luz y vida, se les llaman \bibemph{Hijos de Dios del Paraíso} y comprenden los siguientes tres órdenes:
\vs p020 1:2 \li{1.}Los hijos creadores: los migueles.
\vs p020 1:3 \li{2.}Los hijos magistrados: los avonales.
\vs p020 1:4 \li{3.}Los hijos preceptores de la Trinidad: los dainales.
\vs p020 1:5 \pc Los restantes cuatro órdenes de filiación que descienden se conocen como los \bibemph{Hijos de Dios de los universos locales:}
\vs p020 1:6 \li{4.}Los hijos melquisedecs.
\vs p020 1:7 \li{5.}Los hijos vorondadecs.
\vs p020 1:8 \li{6.}Los hijos lanonandecs.
\vs p020 1:9 \li{7.}Los portadores de vida.
\vs p020 1:10 \pc Los melquisedecs son vástagos conjuntos del hijo creador de un universo local, del espíritu materno creativo y del padre Melquisedec. Tanto los vorondadecs como los lanonandecs deben su ser a un hijo creador y a su espíritu materno creativo acompañante. A los vorondadecs se les conoce mejor como los altísimos, los padres de las constelaciones; a los lanonandecs, como los soberanos de los sistemas y como los príncipes planetarios. El triple orden de los portadores de vida debe su ser a un hijo creador y un espíritu materno creativo en vinculación con uno de los tres ancianos de días del suprauniverso bajo su jurisdicción. Pero la naturaleza y actividad de estos Hijos de Dios de los universos locales se describen de forma más apropiada en los escritos que abordan asuntos relacionados con las creaciones locales.
\vs p020 1:11 \pc Los Hijos de Dios del Paraíso tienen un triple origen: los hijos primarios o creadores deben su ser al Padre Universal y al Hijo Eterno; los hijos secundarios o hijos magistrados son hijos del Hijo Eterno y del Espíritu Infinito; los hijos preceptores de la Trinidad son vástagos del Padre, del Hijo y del Espíritu. Desde la perspectiva del servicio, la adoración y la súplica, los hijos del Paraíso son como uno; su espíritu es uno, y su trabajo es idéntico en cualidad y completitud.
\vs p020 1:12 Así como los órdenes de días del Paraíso han demostrado ser administradores divinos, del mismo modo los órdenes de los hijos del Paraíso se han revelado como benefactores divinos: creadores, servidores, otorgadores, jueces, maestros y reveladores de la verdad. Recorren el universo de los universos desde las orillas de la Isla eterna hasta los mundos habitados del tiempo y del espacio, prestando en el universo central y en los suprauniversos múltiples servicios no desvelados en estas narraciones. Están organizados de forma distinta dependiendo de la naturaleza y ubicación de estos servicios, pero en un universo local tanto los hijos magistrados como los hijos preceptores sirven bajo la dirección del hijo creador que lo preside.
\vs p020 1:13 Los hijos creadores parecen poseer un atributo espiritual central a sus personas que dirigen y que pueden otorgarse de gracia, como hizo vuestro propio hijo creador al derramar su espíritu sobre toda carne mortal en Urantia. Cada hijo creador está dotado de este poder espiritual de atracción en su propio entorno; él es personalmente consciente de toda acción y emoción de cualquier hijo descendente de Dios que sirva en sus dominios. He aquí un reflejo divino, una duplicación en un universo local, de ese absoluto poder de atracción espiritual del Hijo Eterno, que le permite extenderse para realizar y mantener contacto con todos sus hijos del Paraíso, sin importar en qué lugar del universo de los universos se encuentren.
\vs p020 1:14 Los hijos creadores del Paraíso no solo sirven en calidad de hijos en su ministerio al descender para servir y darse a sí mismos, sino que cuando completan su andadura de gracia, cada uno obra como Padre del universo de su propia creación, mientras que los otros Hijos de Dios continúan su ministerio de gracia y elevación espiritual concebido para hacer que los planetas, uno a uno, reconozcan de voluntad el amoroso gobierno del Padre Universal, hasta culminar en la consagración de la criatura a la voluntad del Padre del Paraíso y en la lealtad planetaria a la soberanía sobre el universo de su hijo creador.
\vs p020 1:15 En un hijo creador séptuplo, Creador y criatura se combinan uniéndose por siempre en la comprensión, la compasión y la misericordia. Todo el orden de Miguel, o hijos creadores, es tan singular que el estudio de su naturaleza y actividad se reserva para el próximo escrito de esta serie, mientras que esta narrativa se ocupará mayormente de los dos órdenes restantes de filiación del Paraíso: los hijos magistrados y los hijos preceptores de la Trinidad.
\usection{2. LOS HIJOS MAGISTRADOS}
\vs p020 2:1 Cada vez que un concepto original y absoluto del ser definido por el Hijo Eterno se une con un ideal nuevo y divino de servicio amoroso concebido por el Espíritu Infinito, se produce un hijo de Dios nuevo y primigenio, un hijo magistrado del Paraíso. Estos hijos conforman el orden de los avonales, a diferencia del orden de Miguel, o hijos creadores. Aunque no son creadores en el sentido personal, están en toda su labor estrechamente vinculados a los migueles. Los avonales son servidores y jueces planetarios; son los magistrados de los reinos del tiempo y del espacio: de todas las razas, para todos los mundos y en todos los universos.
\vs p020 2:2 Tenemos razones para pensar que el número total de hijos magistrados en el gran universo es de aproximadamente mil millones. Son un orden autónomo bajo la dirección de un consejo supremo del Paraíso formado por avonales experimentados que se eligen de los servicios de todos los universos. Si bien, cuando los hijos magistrados están asignados a un universo local, y en servicio de este, se encuentran bajo la dirección del hijo creador de ese universo local.
\vs p020 2:3 Los avonales son hijos del Paraíso dedicados al ministerio y a misiones de gracia en cada uno de los planetas de los universos locales. Ya que cada hijo avonal tiene un ser personal exclusivo, puesto que no hay dos que sean idénticos, realizan una labor individual sin parangón en los planetas donde residen, donde con frecuencia se encarnan con la semejanza de un hombre mortal y a veces nacen en los mundos evolutivos de madres terrestres.
\vs p020 2:4 \pc Además de su servicio en los niveles administrativos de gobierno más elevados, los avonales tienen una labor triple en los mundos habitados:
\vs p020 2:5 \li{1.}\bibemph{Actuaciones judiciales}. Actúan al final de las dispensaciones planetarias. Con el tiempo, se pueden realizar muchísimas ---hasta centenares--- de estas misiones en cada uno de los mundos, y pueden acudir innumerables veces al mismo mundo o a otros mundos para terminar una dispensación y liberar a los supervivientes dormidos.
\vs p020 2:6 \li{2.}\bibemph{Misiones en rango de magistrado}. Este tipo de visita planetaria ocurre generalmente antes de la llegada de un hijo de gracia. En estas misiones, el avonal aparece como un adulto del planeta mediante un modo de encarnación que no incluye su nacimiento como mortal. Después de esta primera visita, habitualmente como magistrados, los avonales pueden servir en dicha competencia en el mismo planeta tanto antes como después de la aparición del hijo de gracia. En estas otras misiones como magistrado, el avonal puede aparecer o no en forma visible y material, pero en ninguna de ellas nacerá en el mundo como un niño indefenso.
\vs p020 2:7 \li{3.}\bibemph{Misiones de gracia}. Al menos una vez, todos los hijos avonales se dan de gracia a alguna raza de mortales de algún mundo evolutivo. Las visitas de carácter judicial son numerosas, las misiones en rango de magistrado pueden ser plurales, pero en cada planeta aparece solo un hijo de gracia. Los avonales en misiones de gracia nacen de una mujer al igual que hizo Miguel de Nebadón al encarnarse en Urantia.
\vs p020 2:8 \pc No hay límite en cuanto a la cantidad de veces que los hijos avonales pueden servir en rango de magistrado y en misiones de gracia, pero, generalmente, cuando se ha tenido esa experiencia siete veces, esta se suspende en favor de aquellos que hayan realizado tal servicio con menor frecuencia. Estos hijos que han experimentado múltiples ministerios de gracia se asignan entonces al elevado consejo personal de un hijo creador, llegando así a participar en la administración de los asuntos del universo.
\vs p020 2:9 En toda la labor que realizan en y para los mundos habitados, los hijos magistrados se ayudan de dos órdenes de criaturas de los universos locales, de los melquisedecs y de los arcángeles, mientras que en las misiones de gracia también los acompañan las brillantes estrellas vespertinas, originadas de igual manera en las creaciones locales. Cuando actúan en los planetas, los hijos secundarios del Paraíso, los avonales, encuentran apoyo en el pleno poder y autoridad de un hijo primario del Paraíso, el hijo creador del universo local en el que realiza su servicio. Para todos los efectos, su labor en las esferas habitadas es tan eficaz y adecuada como lo habría sido el servicio de un hijo creador en esos mundos habitados por mortales.
\usection{3. ACTUACIONES JUDICIALES}
\vs p020 3:1 Los avonales se conocen como hijos magistrados porque son los altos magistrados de los mundos, los árbitros de las sucesivas dispensaciones de los mundos del tiempo. Presiden el despertar de los supervivientes dormidos, juzgan al planeta, ponen fin a una dispensación en la que se postergó la justicia, ejecutan los mandatos de una era de misericordia probatoria, reasignan a las criaturas del espacio que realizan un ministerio planetario a las tareas de la nueva dispensación y regresan a la sede de su universo local al completar su misión.
\vs p020 3:2 Cuando juzgan los destinos de una era, los avonales decretan el destino de las razas evolutivas, pero aunque puedan disponer en su sentencia la extinción de la identidad de criaturas personales, ellos no ejecutan sentencias. Los veredictos de esta naturaleza jamás los ejecutan otros que no sean las autoridades de un suprauniverso.
\vs p020 3:3 La llegada de un avonal del Paraíso a un mundo evolutivo con el propósito de poner término a una dispensación y de inaugurar una nueva era en el progreso del planeta no es necesariamente ni una misión como magistrado ni una misión de gracia. Las misiones en calidad de magistrado a veces, y las misiones de gracia siempre, son encarnaciones; o sea, durante estos cometidos los avonales sirven en un planeta en forma material ---literalmente---. Sus otras visitas tienen carácter “formal”, y en esta labor un avonal no se encarna para servir en el planeta. Si un hijo magistrado viene solamente como árbitro de una dispensación, llega al planeta como ser espiritual, invisible para las criaturas materiales del mundo. Dichas visitas de carácter formal ocurren repetidas veces durante la larga historia de un mundo habitado.
\vs p020 3:4 Los hijos avonales pueden servir de jueces planetarios antes de tener la experiencia de hacerlo como magistrados o realizando un ministerio de gracia. En cada una de estas misiones, sin embargo, el hijo encarnado juzgará la era planetaria que termina; lo mismo hace un hijo creador cuando está encarnado en una misión de gracia con la semejanza de un hombre mortal. Cuando un hijo del Paraíso viene a un mundo evolutivo y se convierte en uno de sus habitantes, su presencia pone término a una dispensación y decreta el juicio del mundo.
\usection{4. MISIONES EN RANGO DE MAGISTRADO}
\vs p020 4:1 Antes de la aparición en el planeta de un hijo de gracia, un avonal en calidad de magistrado visita generalmente el mundo habitado. Si se trata de una primera venida como magistrado, el avonal siempre se encarna como ser material; aparece en el planeta al que está asignado como varón maduro de las razas mortales, un ser plenamente visible para las criaturas mortales de su día y generación y capaz de tener con ellas un contacto físico. Durante toda su encarnación como magistrado, la conexión del hijo avonal con las fuerzas espirituales locales y universales es total e ininterrumpida.
\vs p020 4:2 A un planeta pueden acudir muchos avonales, tanto antes como después de la aparición de un hijo de gracia. Lo pueden visitar muchas veces el mismo avonal u otros avonales en calidad de árbitros de una dispensación, si bien, dichas misiones formales de enjuiciamiento no son ni ministerios de gracia ni misiones como magistrados, y los avonales en estas ocasiones nunca se encarnan. Incluso cuando a un planeta se le bendice con repetidas visitas de magistrados, los avonales no siempre se someten a la encarnación mortal; y cuando en verdad sirven con la semejanza de un hombre mortal, siempre aparecen como seres adultos del mundo; no nacen de una mujer.
\vs p020 4:3 Cuando se encarnan en misiones de gracia o como magistrados, los hijos del Paraíso tienen modeladores experimentados, modeladores diferentes para cada encarnación. Los modeladores que habitan en la mente de los Hijos de Dios encarnados no pueden tener jamás la esperanza de obtener su ser personal mediante la fusión con los seres humanos\hyp{}divinos en los que moran; no obstante, con frecuencia adquieren su ser personal mediante el mandato del Padre Universal. Dichos modeladores forman el consejo de dirección de Lugar de la Divinidad para la administración, identificación y envío de los mentores misteriosos a los mundos habitados. También reciben y autorizan a los modeladores que regresan al “seno del Padre” cuando la muerte disuelve su tabernáculo terrenal. De esta manera, los fieles modeladores de los jueces del mundo se erigen como jefes excelsos de su clase.
\vs p020 4:4 \pc Urantia nunca ha albergado a un hijo avonal en el cargo de magistrado. Si Urantia hubiese seguido el plan general de los mundos habitados, habría resultado bendecida con esta misión en algún momento entre los días de Adán y el ministerio de gracia de Cristo Miguel. Pero la secuencia regular de los hijos del Paraíso en vuestro planeta se modificó totalmente con la aparición de vuestro hijo creador en su último ministerio de gracia hace mil novecientos años.
\vs p020 4:5 Todavía es posible que un avonal se encarne y visite Urantia con el encargo de cumplir su misión como magistrado, pero en cuanto a la aparición futura de los hijos del Paraíso, ni siquiera “los ángeles en el cielo saben el momento ni la forma de dichas venidas”, porque un mundo en el que un miguel se haya dado como don gratuito se convierte en el pupilo individual y personal de un hijo mayor y, como tal, está completamente sujeto a sus propios designios y resoluciones. Y en vuestro mundo, esto se complica todavía más por la promesa que hizo Miguel de regresar. Aparte de los malentendidos que pueda haber sobre la estancia de Miguel de Nebadón en Urantia, algo es ciertamente incuestionable: su promesa de volver a vuestro mundo. En vista de esta posibilidad, tan solo el tiempo podrá revelar en qué orden se darán las futuras venidas de los Hijos de Dios del Paraíso a Urantia.
\usection{5. EL MINISTERIO DE GRACIA DE LOS HIJOS DE DIOS DEL PARAÍSO}
\vs p020 5:1 El Hijo Eterno es la Palabra eterna de Dios; es la expresión perfecta del “primer” pensamiento absoluto e infinito de su Padre eterno. Cuando una duplicación personal o una extensión divina de este Hijo Primigenio comienza una misión de gracia encarnándose como un mortal, se hace literalmente verdad que la divina “Palabra se hace carne” y que la Palabra mora así entre los modestos seres de origen animal.
\vs p020 5:2 En Urantia existe la creencia generalizada de que el propósito del ministerio de gracia de un hijo del Paraíso es el de influir de alguna manera sobre la actitud del Padre Universal. Pero vuestra propia lucidez debería indicaros que esto no es verdad. Las misiones de gracia de los hijos avonales y de los hijos migueles son una parte necesaria del proceso experiencial diseñado para hacer de estos hijos unos magistrados y gobernantes comprensivos y dignos de confianza de la gente y de los planetas del tiempo y del espacio. El séptuplo ministerio de gracia constituye para todos los hijos creadores del Paraíso la suprema meta a alcanzar. Y a todos los hijos magistrados los motiva este espíritu de servicio que con tanta abundancia caracteriza a los hijos creadores primarios y al Hijo Eterno del Paraíso.
\vs p020 5:3 Algún orden de hijos del Paraíso debe otorgarse a los mundos habitados por mortales para posibilitar la llegada de los modeladores del pensamiento a la mente de \bibemph{todos} los seres humanos normales de esas esferas. Los modeladores no acuden a todos los seres humanos de buena fe hasta que no se haya derramado el espíritu de la verdad sobre toda carne; y el envío del espíritu de la verdad depende del regreso a la sede central del universo de un hijo del Paraíso que haya realizado con éxito una misión de gracia como mortal en un mundo evolutivo.
\vs p020 5:4 Durante el curso de la larga historia de un planeta habitado, tienen lugar muchos juicios dispensacionales, y puede ocurrir más de una visita de un magistrado, pero habitualmente solamente una vez servirá un hijo de gracia en esta esfera. Solamente se requiere para cada mundo habitado que uno solo de estos hijos viva allí su vida mortal plenamente, desde su nacimiento hasta su muerte. Tarde o temprano, sea cual fuere su estatus espiritual, todo mundo habitado por mortales está destinado a recibir a un hijo magistrado en misión de gracia, exceptuando el planeta de aquel universo local en el que un hijo creador elige darse a sí mismo como mortal.
\vs p020 5:5 \pc Mientras mejor comprendáis la misión de los hijos de gracia, más os daréis cuenta del elevado interés que se asigna a Urantia en la historia de Nebadón. Vuestro pequeño e insignificante planeta tiene importancia para el universo local, sencillamente porque es el hogar mortal de Jesús de Nazaret. Fue el escenario del ministerio de gracia final y triunfante de vuestro hijo creador: el lugar en el que Miguel alcanzó la soberanía personal suprema del universo de Nebadón.
\vs p020 5:6 En la sede de su universo local, un hijo creador, en especial tras completar su vida de gracia como mortal, pasa mucho de su tiempo aconsejando e instruyendo en la facultad de hijos coiguales, los hijos magistrados y algunos otros más. En amor y devoción, en tierna misericordia y afectuosa consideración, estos hijos magistrados se dan como don gratuito a los mundos del espacio. Y, de ninguna manera, son estos ministerios de gracia planetarios inferiores a los realizados por los migueles como mortales. Es verdad que vuestro hijo creador eligió para la aventura final de su experiencia creatural un mundo que había sido excepcionalmente desafortunado. Pero ningún planeta puede jamás hallarse en condición alguna como para requerir la dádiva de sí mismo de un hijo creador y efectuar así su rehabilitación espiritual. Cualquier hijo de los que realizan estos ministerios de gracia hubiese resultado también suficiente, porque en toda su labor en los mundos de un universo local, los hijos magistrados son tan divinamente eficaces y omnisapientes como podría serlo el hijo creador, su hermano en el Paraíso.
\vs p020 5:7 \pc Aunque la posibilidad de desastres está siempre presente para estos hijos del Paraíso en su encarnación, estoy aún por conocer algún dato que confirme algún fallo o falta en la misión de gracia de un hijo magistrado o de un hijo creador. Ambos tienen un origen tan cercano a la perfección absoluta como para fracasar. Estos de hecho asumen el riesgo, en realidad se convierten en criaturas mortales de carne y hueso y, por consiguiente, adquieren una vivencia creatural sin parangón, pero según mis observaciones siempre triunfan. Nunca dejan de alcanzar el objetivo de su misión. La historia de su ministerio planetario constituye en todo Nebadón el capítulo más noble y fascinante en la historia de vuestro universo local.
\usection{6. LAS ANDADURAS DE GRACIA COMO MORTALES DE LOS HIJOS DEL PARAÍSO}
\vs p020 6:1 La forma por la que un hijo del Paraíso se prepara para su encarnación humana como hijo de gracia, naciendo de una madre en el planeta elegido, es un misterio universal; y todo esfuerzo para descubrir cómo se realiza este procedimiento, propio de Lugar del Hijo, está condenado a un fracaso seguro. Dejad que el conocimiento sublime de la vida mortal de Jesús de Nazaret penetre en vuestras almas, y no desperdiciéis vuestro pensamiento en especulaciones inútiles sobre cómo se llevó a cabo esta misteriosa encarnación de Miguel de Nebadón. Regocijémonos todos con el conocimiento y la certeza de que tales logros son posibles para la naturaleza divina y no perdamos tiempo en conjeturas vanas para averiguar la forma empleada por la sabiduría divina para efectuar este fenómeno.
\vs p020 6:2 \pc En una misión de gracia como mortal, un hijo del Paraíso siempre nace de mujer y crece como un niño varón de su mundo, tal como Jesús hizo en Urantia. Todos estos hijos viven desde su infancia hasta su edad adulta pasando por la juventud como cualquier otro ser humano. En todos los aspectos, se convierten en mortales de la raza en la que nacen. Hacen peticiones al Padre como lo hacen los hijos del mundo en el que prestan su supremo servicio. Desde un punto de vista material, estos hijos humanos y divinos viven vidas comunes con una sola excepción: no engendran vástagos en los mundos donde residen; esta es una restricción universal impuesta a todos los órdenes de los hijos de gracia del Paraíso.
\vs p020 6:3 Al igual que Jesús trabajó en vuestro mundo como el hijo del carpintero, otros hijos del Paraíso lo hacen en diferentes ámbitos laborales en los planetas en los que se dan a sí mismos. Os sería difícil pensar en algún trabajo que algún hijo del Paraíso no haya realizado en el curso de su ministerio de gracia en alguno de los planetas evolutivos del tiempo.
\vs p020 6:4 Cuando un hijo de gracia ha aprendido a vivir la vida mortal, cuando ha llegado a estar en sintonía perfecta con su modelador interior, acto seguido comienza esa parte de su misión planetaria destinada a iluminar la mente e inspirar el alma de sus hermanos en la carne. Como maestros, estos hijos están dedicados exclusivamente al desarrollo de la lucidez espiritual de las razas mortales en los mundos donde residen.
\vs p020 6:5 \pc Las andaduras de gracia como mortales de los migueles y de los avonales, aunque equiparables en la mayoría de los aspectos, no son del todo idénticas. Un hijo magistrado jamás proclama “el que ha visto al hijo, ha visto al Padre”, como lo hizo vuestro hijo creador cuando estaba encarnado en Urantia. Pero un avonal sí declara “el que me ha visto a mí, ha visto al Hijo Eterno de Dios”. Los hijos magistrados no descienden directamente del Padre Universal ni tampoco se encarnan en obediencia a la voluntad del Padre; siempre se dan a sí mismos como \bibemph{hijos} del Paraíso en obediencia a la voluntad del Hijo Eterno del Paraíso.
\vs p020 6:6 \pc Cuando estos hijos de gracia, creadores o magistrados, entran en el portal de la muerte, reaparecen al tercer día. Pero no deberíais contemplar la idea de que siempre hallan el trágico fin devenido al hijo creador que residió en vuestro mundo hace mil novecientos años. Aquella vivencia inusitada y extremadamente cruel por la que pasó Jesús de Nazaret ha hecho que Urantia sea conocida en la zona como “el mundo de la cruz”. No es necesario que se trate de manera tan inhumana a un hijo de Dios, y en la gran mayoría de los planetas se les ha recibido de manera más considerada, permitiéndoles terminar sus andaduras como mortales, llegar a su madurez, juzgar a los supervivientes dormidos e inaugurar una nueva dispensación, sin sufrir una muerte violenta. Un hijo de gracia debe enfrentarse a la muerte, debe pasar por las mismas experiencias que los mortales del planeta, pero en el plan divino no se requiere que su muerte sea ni violenta ni fuera de lo común.
\vs p020 6:7 Cuando los hijos de gracia no hallan la muerte de manera violenta, abandonan de forma voluntaria sus vidas y pasan por los portales de la muerte, no para satisfacer las exigencias de la “justicia severa” o de la “ira divina”, sino más bien para completar su misión, “para beber la copa” de su andadura de encarnación y de su experiencia personal en todo lo que constituye la vida de una criatura mortal tal como se vive en sus planetas. La misión de gracia es una necesidad planetaria y universal, y la muerte física no es nada más que una parte necesaria de dicha misión.
\vs p020 6:8 Cuando la encarnación mortal se termina, el avonal en servicio se encamina al Paraíso, donde el Padre Universal lo acepta, regresa al universo local asignado y el hijo creador lo recibe. Acto seguido, el avonal y el hijo creador envían al espíritu de la verdad, conjunto a ellos, para obrar en el corazón de las razas mortales que moran en el mundo en el que se realizó esta misión de gracia. En las eras previas a la soberanía de un universo local, este es el espíritu conjunto de ambos hijos, llevado a efecto por el espíritu creativo. Difiere un tanto del espíritu de la verdad que caracteriza las eras del universo local tras el séptimo ministerio de gracia de un miguel.
\vs p020 6:9 Al completarse el último de estos servicios de un hijo creador, el espíritu de la verdad, previamente enviado a todos los mundos de gracia de ese universo local en los que se ha otorgado un avonal, cambia de naturaleza, convirtiéndose de forma más literal en el espíritu de Miguel soberano. Este fenómeno tiene lugar de forma simultánea con la liberación del espíritu de la verdad para servir en el planeta en el que Miguel vivió como mortal. De ahí en adelante, todo mundo honrado por el ministerio de un magistrado recibirá al mismo Confortador espiritual del hijo creador séptuplo, en colaboración con el hijo magistrado, que ese mundo habría recibido si el mismo soberano del universo local se hubiese encarnado personalmente allí como hijo de gracia.
\usection{7. LOS HIJOS PRECEPTORES DE LA TRINIDAD}
\vs p020 7:1 Estos hijos del Paraíso, que son seres sumamente personales y espirituales, deben su existencia a la Trinidad del Paraíso. En Havona se les conoce como el orden de los dainales. En Orvontón constan como hijos preceptores de la Trinidad, llamados así por su origen. En Lugar de Salvación se les denomina a veces hijos espirituales del Paraíso.
\vs p020 7:2 El número de los hijos preceptores está constantemente en aumento. El último censo universal realizado arrojó para estos hijos de la Trinidad, que realizan su labor en el universo central y en los suprauniversos, unas cifras que superaban en algo los veintiún mil millones, y esto excluyendo las reservas del Paraíso que suman más de un tercio de todos los hijos preceptores de la Trinidad.
\vs p020 7:3 El orden dainal de filiación no es parte orgánica de las administraciones de los universos locales o de los suprauniversos. Sus miembros no son ni creadores ni rescatadores, ni tampoco jueces ni gobernantes. No se preocupan tanto de la administración del universo como de la lucidez moral y del avance espiritual. Son educadores universales dedicados al despertar espiritual y a la guía moral de todos los planetas. Su ministerio está íntimamente interrelacionado con el de los seres personales del Espíritu Infinito y estrechamente vinculado con la ascensión de los seres creaturales al Paraíso.
\vs p020 7:4 Estos hijos de la Trinidad participan de las naturalezas combinadas de las tres Deidades del Paraíso, si bien en Havona parecen reflejar más la naturaleza del Padre Universal. En los suprauniversos parecen representar la naturaleza del Hijo Eterno, mientras que en las creaciones locales parecen mostrar las características del Espíritu Infinito. En todos los universos, son el servicio y el criterio de la sabiduría manifestados.
\vs p020 7:5 A diferencia de sus hermanos del Paraíso, los migueles y los avonales, los hijos preceptores de la Trinidad no reciben formación previa en el universo central. Se les envía directamente a las sedes de los suprauniversos y desde allí se les nombra para servir en algún universo local. En su ministerio en estos planetas evolutivos, hacen uso de la influencia espiritual combinada de un hijo creador y de los hijos magistrados adjuntos a este, porque los dainales no disponen en sí mismos ni por sí mismos de poder de atracción espiritual.
\usection{8. EL MINISTERIO DE LOS DAINALES EN LOS UNIVERSOS LOCALES}
\vs p020 8:1 Los hijos espirituales del Paraíso son seres singulares de origen en la Trinidad y las únicas criaturas trinitarias que están tan completamente vinculadas con la dirección de los universos de doble origen. Se dedican afectuosamente al ministerio de instrucción de las criaturas mortales y de los órdenes menores de seres espirituales. Comienzan su labor en los sistemas locales y, de acuerdo con su experiencia y logro, progresan interiormente mediante su servicio en las constelaciones en las tareas más elevadas de la creación local. Tras ser certificados, pueden convertirse en embajadores espirituales en representación de los universos locales donde sirven.
\vs p020 8:2 No conozco el número exacto de hijos preceptores que hay en Nebadón; existen muchos miles de ellos. Muchos de los jefes de departamentos en las escuelas de los melquisedecs pertenecen a este orden, mientras que la suma total de miembros de la Universidad de Lugar de Salvación, regularmente constituida, asciende a más de cien mil, incluyendo a estos hijos. Hay un gran número emplazado en los distintos mundos de formación morontial, pero no están totalmente dedicados al avance espiritual e intelectual de las criaturas mortales; se preocupan por igual de la instrucción de los seres seráficos y de otros nativos de las creaciones locales. Muchos de sus ayudantes se seleccionan de entre los seres trinitizados por criaturas.
\vs p020 8:3 Los hijos preceptores son los encargados de administrar todos los exámenes y dirigir todas las pruebas para la instrucción y certificación de todas las facetas menores del servicio del universo, desde la labor de centinela en puestos avanzados hasta la de estudiante de las estrellas. Imparten un curso de formación de una era de duración que se extiende desde cursos planetarios hasta la alta Facultad de la Sabiduría localizada en Lugar de Salvación. Todos los que completan estas aventuras en sabiduría y verdad, ya sea un mortal en ascenso o un querubín con aspiraciones, obtienen el reconocimiento de su esfuerzo y logro.
\vs p020 8:4 En todos los universos, todos los Hijos de Dios admiran a estos hijos preceptores de la Trinidad, siempre fieles y universalmente eficientes. Son los maestros excelsos de todos los seres personales espirituales; son realmente los maestros probados y auténticos de entre los mismos Hijos de Dios. Pero no me es posible informaros de los interminables detalles de los deberes y funciones que los hijos preceptores llevan a cabo. El inmenso ámbito de actividad que realiza este orden dainal de filiación se comprenderá mejor en Urantia cuando hayáis avanzado más en inteligencia, y después de que el aislamiento espiritual de vuestro planeta haya terminado.
\usection{9. EL SERVICIO PLANETARIO DE LOS DAINALES}
\vs p020 9:1 Cuando el progreso de los acontecimientos en un mundo evolutivo indica que ha llegado el momento propicio para el inicio de una era espiritual, los hijos preceptores de la Trinidad se ofrecen siempre como voluntarios para prestar dicho servicio. No estáis familiarizados con este orden de filiación porque Urantia nunca ha conocido una era espiritual, un milenio de lucidez cósmica. Pero los hijos preceptores, incluso ahora, visitan vuestro mundo con el fin de planear su futura residencia en vuestra esfera. Aparecerán en Urantia, una vez que sus habitantes se hayan liberado prácticamente de las ataduras del animalismo y de las cadenas del materialismo.
\vs p020 9:2 Los hijos preceptores de la Trinidad no tienen nada que ver con la terminación de las dispensaciones planetarias. Ni juzgan a los muertos ni trasladan a los vivos; si bien, en cada misión planetaria van acompañados de un hijo magistrado que realiza estos servicios. Los hijos preceptores se ocupan enteramente de la iniciación de una era espiritual, de los albores de una era de realidades espirituales en un planeta evolutivo. Hacen realidad los homólogos espirituales del conocimiento material y de la sabiduría del tiempo.
\vs p020 9:3 Los hijos preceptores por lo general permanecen en los planetas que visitan durante mil años de tiempo planetario. Asistido por setenta colaboradores de su orden, un hijo preceptor preside ese reinado milenario planetario. Los dainales no se encarnan ni se materializan en forma visible para los seres mortales; por tanto, el contacto con el mundo que visitan se mantiene mediante la actividad de las brillantes estrellas vespertinas, seres personales del universo local vinculados a los hijos preceptores de la Trinidad.
\vs p020 9:4 Los dainales pueden regresar muchas veces a un mundo habitado y, tras su misión final, el planeta se conduce hacia el estatus de esfera asentada en luz y vida, la meta evolutiva de todos los mundos habitados por mortales de la presente era del universo. El colectivo final de los mortales tiene mucho que ver con las esferas asentadas en luz y vida, y su actividad planetaria está relacionada con la de los hijos preceptores. En efecto, todo el orden de filiación dainal está íntimamente relacionado con todas las etapas de la actividad de los finalizadores en las creaciones evolutivas del tiempo y del espacio.
\vs p020 9:5 \pc Los hijos preceptores de la Trinidad parecen estar tan totalmente identificados con el régimen de progreso del mortal a través de las etapas primitivas de ascenso evolutivo, que frecuentemente tendemos a especular sobre su posible vinculación con los finalizadores en su andadura no desvelada en los universos futuros. Observamos que los administradores de los suprauniversos son en parte seres personales de origen en la Trinidad y en parte criaturas evolutivas ascendentes acogidas por la Trinidad. Creemos firmemente que los hijos preceptores y los finalizadores se dedican ahora a adquirir experiencia de su vinculación en el tiempo, lo cual quizás sea una formación previa en preparación para su estrecha relación en algún destino futuro no revelado. En Uversa creemos que, cuando los suprauniversos finalmente se asienten en luz y vida, estos hijos preceptores del Paraíso, tan completamente conocedores de los problemas de los mundos evolutivos y, por tanto tiempo, tan relacionados con la andadura de los mortales evolutivos, probablemente sean transferidos a su conjunción eterna con el colectivo final del Paraíso.
\usection{10. EL MINISTERIO UNIDO DE LOS HIJOS DEL PARAÍSO}
\vs p020 10:1 Todos los Hijos de Dios del Paraíso son divinos en su origen y su naturaleza. La labor de cada uno de estos hijos del Paraíso en bien de cada uno de los mundos es como si cada uno de ellos, en su ministerio, fuese el primero y único hijo de Dios.
\vs p020 10:2 Los hijos del Paraíso constituyen el obsequio divino de las naturalezas actuantes de las tres personas de la Deidad a las regiones del tiempo y del espacio. Los hijos creadores, magistrados y preceptores son los dones de las Deidades eternas a los hijos de los mortales y a todas las demás criaturas del universo con potencial de ascensión. Estos Hijos de Dios son benefactores divinos dedicados, sin cesar, a la tarea de ayudar a las criaturas del tiempo para que alcancen el elevado objetivo espiritual de la eternidad.
\vs p020 10:3 En los hijos creadores, el amor del Padre Universal se combina con la misericordia del Hijo Eterno y se desvela a los universos locales en el poder creativo, el ministerio amoroso y la soberanía comprensiva de los migueles. En los hijos judiciales, la misericordia del Hijo Eterno, unida con el ministerio del Espíritu Infinito, se revela a las regiones evolutivas en la andadura de estos avonales magistrados, que sirven y se dan de gracia. En los hijos preceptores de la Trinidad el amor, la misericordia y el ministerio de las tres Deidades del Paraíso se coordinan en los más elevados niveles de valor espacio\hyp{}temporal y se presentan a los universos como verdad viva, bondad divina y verdadera belleza espiritual.
\vs p020 10:4 En los universos locales, estos órdenes de filiación colaboran en la revelación de las Deidades del Paraíso a las criaturas del espacio; como Padre de un universo local, un hijo creador representa el carácter infinito del Padre Universal. Como hijos de misericordia que se dan de gracia, los avonales revelan la naturaleza incomparable del Hijo Eterno en su infinita compasión. Como verdaderos maestros de los seres personales ascendentes, los hijos dainales de la Trinidad desvelan el aspecto magisterial del Espíritu Infinito. Los migueles, los avonales y los dainales en su cooperación, divinamente perfecta, contribuyen a la actualización y revelación del ser personal y la soberanía del Dios Supremo en y para los universos del tiempo y del espacio. En la armonía de su actividad trina, estos Hijos de Dios del Paraíso operan siempre en la vanguardia de los seres personales de la Deidad conforme siguen la inacabable expansión de la divinidad de la Primera Gran Fuente y Centro desde la sempiterna Isla del Paraíso hacia las profundidades desconocidas del espacio.
\vsetoff
\vs p020 10:5 [Exposición de un perfeccionador de la sabiduría de Uversa.]
