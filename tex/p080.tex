\upaper{80}{La expansión andita en occidente}
\author{Arcángel}
\vs p080 0:1 Aunque el hombre azul europeo no alcanzó por sí mismo una gran civilización cultural, sí sentó la base biológica, la cual, al mezclarse sus estirpes adanizadas con los posteriores invasores anditas, daría origen a uno de los linajes más poderosos y propicios para el logro de una enérgica civilización, como la que nunca había aparecido en Urantia desde los tiempos de la raza violeta y de sus sucesores anditas.
\vs p080 0:2 Los pueblos blancos modernos contienen las estirpes supervivientes del linaje adánico, que llegaron a mezclarse con las razas sangiks, esto es, con algunos componentes de la roja y de la amarilla, pero, más especialmente, con los de la azul. En todas las razas blancas, hay un considerable porcentaje de linaje andonita original e, incluso más, de las primeras estirpes noditas.
\usection{1. LOS ADANITAS ENTRAN EN EUROPA}
\vs p080 1:1 Antes de que se expulsara a los últimos anditas del valle del Éufrates, muchos de sus hermanos habían entrado en Europa como aventureros, maestros, comerciantes y guerreros. Durante los primeros días de la raza violeta, la depresión mediterránea estaba protegida por el istmo de Gibraltar y por el puente terrestre siciliano. Una parte del más temprano comercio marítimo del hombre se estableció en estos lagos del interior, donde los hombres azules del norte y los saharianos del sur se encontraron con los noditas y los adanitas del este.
\vs p080 1:2 En la depresión oriental del Mediterráneo, los noditas habían instaurado uno de sus centros culturales más extensos y, desde ellos, habían penetrado ligeramente en el sur de Europa, aunque, más particularmente, en el norte de África. Bien pronto, los sirios nodita\hyp{}andonitas de cabezas anchas introdujeron la alfarería y la agricultura en sus asentamientos del delta del Nilo, en lenta elevación. También importaron ovejas, cabras, ganado y otros animales domésticos e implantaron métodos notablemente mejorados de metalurgia; Siria era entonces el centro de esa industria.
\vs p080 1:3 Durante más de treinta mil años, Egipto recibió un flujo constante de mesopotámicos, que llevaron consigo su arte y su cultura para enriquecer la del valle del Nilo. Pero la entrada de un gran número de pueblos del Sahara deterioró enormemente la temprana civilización que se asentó a lo largo del Nilo, por lo que Egipto llegó a su nivel cultural más bajo hace unos quince mil años.
\vs p080 1:4 Si bien, en tiempos pasados, había pocas cosas que obstaculizaran la emigración de los adanitas hacia el oeste. El Sahara era tierra de pastoreo abierta, poblada de pastores y agricultores. Estos saharianos no se dedicaban a la manufactura ni tampoco construían ciudades. Formaban un grupo índigo\hyp{}negro que portaban numerosas estirpes de las extintas razas verde y naranja. Pero recibieron una cantidad muy limitada de herencia violeta antes de que la elevación del suelo y el cambio de los vientos húmedos dispersaran los restos de esta próspera y pacífica civilización.
\vs p080 1:5 La mayoría de las razas humanas compartió la sangre de Adán, pero algunas la obtuvieron más que otras. Las razas mezcladas de la India y los pueblos más oscuros de África no eran atractivos para los adanitas. Se habrían mezclado sin restricción con el hombre rojo de no haber estado tan alejado en las Américas, y tenían una buena disposición hacia el hombre amarillo, pero resultaba igualmente difícil acceder a la distante Asia. Así pues, cuando se sentían impulsados por la aventura o el altruismo, o cuando fueron expulsados del valle del Éufrates, era muy natural que eligiesen la unión con las razas azules de Europa.
\vs p080 1:6 Los hombres azules, entonces dominantes en Europa, no poseían prácticas religiosas que repeliesen a los primeros emigrantes adanitas, y existía una gran atracción sexual entre la raza violeta y la raza azul. Los mejores hombres azules consideraban un gran honor que se les permitiese procrear con las adanitas. Todo hombre azul tenía la ambición de volverse tan hábil y artístico como para ganar el afecto de alguna mujer adanita, y era la más elevada aspiración de las mujeres azules mejor dotadas recibir las atenciones de un adanita.
\vs p080 1:7 Lentamente, estos hijos migratorios de Edén se unieron con los tipos superiores de la raza azul, vigorizando sus prácticas culturales mientras que exterminaban de modo implacable a las persistentes estirpes de la raza neandertal. Esta táctica de mezclar las razas, en combinación con la eliminación de las estirpes peor dotadas, produjo una veintena o más de grupos viriles y avanzados de hombres azules de orden superior, uno de los cuales habéis denominado cromañón.
\vs p080 1:8 Por estas y otras razones, entre ellas y no menos importante la de rutas migratorias más favorables, las primeras oleadas de cultura mesopotámica se abrieron camino casi exclusivamente en Europa. Y fueron estas las circunstancias que determinaron los antecedentes de la civilización europea moderna.
\usection{2. CAMBIOS CLIMÁTICOS Y GEOLÓGICOS}
\vs p080 2:1 La expansión inicial de la raza violeta en Europa se vio interrumpida por algunos cambios climáticos y geológicos bastante repentinos. Con el repliegue de los campos de hielo del norte, los vientos húmedos del oeste cambiaron hacia el norte, convirtiendo gradualmente a las grandes regiones de pastizales abiertos del Sahara en un estéril desierto. Esta sequía dispersó a los moradores de baja estatura, morenos de ojos negros, pero de cabeza alargada, que habitaban la gran llanura del Sahara.
\vs p080 2:2 Los grupos índigos más puros se dirigieron al sur, a los bosques de África central, donde han permanecido desde entonces. Aquellos más mezclados se dispersaron en tres direcciones: las tribus más dotadas del oeste emigraron a España y, desde allí, a las zonas adyacentes de Europa, formando el núcleo de las posteriores razas mediterráneas morenas de cabeza alargada. Los menos avanzados del este de la llanura del Sahara emigraron a Arabia y desde allí, a través de la Mesopotamia del norte y de la India, hasta la lejana Ceilán. El grupo central se desplazó al norte y al este del valle del Nilo y entró en Palestina.
\vs p080 2:3 Es este substrato sangik secundario el que apunta a la existencia de algún grado de parentesco entre los pueblos modernos diseminados desde el Decán a través de Irán, Mesopotamia y a lo largo de las dos orillas del mar Mediterráneo.
\vs p080 2:4 \pc Por el tiempo de estos cambios climáticos ocurridos en África, Inglaterra se separó del continente y Dinamarca emergió del mar, mientras que el istmo de Gibraltar, que protegía la cuenca occidental del Mediterráneo, se hundió como resultado de un terremoto, por lo que este lago interior se elevó rápidamente hasta el nivel del Océano Atlántico. Enseguida, el puente terrestre siciliano se sumergió, creando un solo Mar Mediterráneo y conectándolo con el Océano Atlántico. Este cataclismo natural anegó decenas de asentamientos humanos y ocasionó la mayor pérdida de vidas por inundación de toda la historia del mundo.
\vs p080 2:5 Esta inundación de la cuenca del Mediterráneo limitó inmediatamente los movimientos de los adanitas hacia el oeste, mientras que la gran afluencia de saharianos los llevó a buscar salidas para su creciente población hacia el norte y el este de Edén. A medida que los descendientes de Adán viajaban hacia el norte desde los valles del Tigris y del Éufrates, se hallaron con barreras montañosas y con el Mar Caspio, entonces más extenso. Y durante muchas generaciones, los adanitas cazaron, pastorearon y cultivaron la tierra alrededor de sus asentamientos dispersos por todo el Turquestán. Lentamente, este magnífico pueblo amplió su territorio hacia Europa. Es en ese momento cuando los adanitas entraron en Europa desde el este y se encontraron con la cultura del hombre azul, miles de años por detrás de la de Asia, dado que esta región no había tenido prácticamente casi ningún contacto con Mesopotamia.
\usection{3. EL HOMBRE AZUL DE CROMAÑÓN}
\vs p080 3:1 Los centros ancestrales de la cultura del hombre azul estaban situados a lo largo de todos los ríos de Europa, pero solo el Somme fluye ahora por el mismo cauce que tenía durante los tiempos preglaciales.
\vs p080 3:2 Aunque decimos que el hombre azul estaba presente en el continente europeo, había multitud de tipos raciales. Incluso, hace treinta y cinco mil años, las razas azules europeas eran ya un pueblo bastante mezclado portador de estirpes de las razas roja y amarilla, mientras que en las costas atlánticas y en las regiones de la actual Rusia, habían absorbido una considerable cantidad de sangre andonita y, hacia el sur, estaban en contacto con los pueblos saharianos. Pero tendría poca utilidad tratar de enumerar los numerosos grupos raciales.
\vs p080 3:3 La civilización europea de este temprano período postadánico era una singular mezcla del vigor y del arte de los hombres azules con la imaginación creativa de los adanitas. Los hombres azules eran una raza de mucho brío, pero deterioraron enormemente el estado cultural y espiritual de los adanitas. A estos últimos les fue sumamente difícil imponer su religión a los cromañones debido a la tendencia de muchos de ellos a seducir y engañar a las doncellas. Durante diez mil años, la religión en Europa estuvo en un nivel muy bajo en comparación con el desarrollo alcanzado en la India y en Egipto.
\vs p080 3:4 Los hombres azules eran enteramente honestos en todos sus asuntos y estaban completamente libres de los malos hábitos sexuales de los mezclados adanitas. Respetaban la virginidad y tan solo practicaban la poligamia cuando la guerra ocasionaba falta de varones.
\vs p080 3:5 Los pueblos de Cromañón eran una raza valiente y con visión de futuro. Poseían un eficiente sistema de educación de los niños. Ambos padres participaban en estas labores, y se contaba plenamente con la ayuda de los hijos mayores. Se formaba cuidadosamente a todos los niños en el cuidado de las cuevas, en las artes y en el trabajo del pedernal. En eras tempranas, las mujeres conocían bien las artes domésticas y una agricultura rudimentaria, mientras que los hombres eran cazadores hábiles y guerreros valerosos.
\vs p080 3:6 Los hombres azules eran cazadores, pescadores y recolectores de alimentos. Eran expertos fabricantes de barcos. Hacían hachas de piedra, talaban árboles, erigían cabañas de troncos, parcialmente bajo tierra y con techos de pieles. En Siberia hay aún poblados que construyen cabañas similares. Los cromañones del sur vivían generalmente en cuevas y grutas.
\vs p080 3:7 Durante los rigores del invierno, no era raro que los centinelas de guardia nocturna en las entradas de las cuevas murieran de frío. Eran valerosos, pero, sobre todo, artistas; la mezcla adánica aceleró de repente su imaginación creativa. El arte del hombre azul tuvo su punto álgido hace unos quince mil años, antes de los días en los que las razas de piel más oscura llegaran al norte desde África a través de España.
\vs p080 3:8 \pc Hace unos quince mil años, los bosques alpinos se estaban expandiendo abundantemente. Los cazadores europeos se vieron empujados hacia los valles fluviales y las zonas costeras por las mismas adversidades climáticas que habían transformado las favorables tierras de caza del mundo en desiertos secos y áridos. Conforme los vientos húmedos cambiaban hacia el norte, las grandes llanuras de pastizales abiertos de Europa se fueron cubriendo de bosques. Estas grandes modificaciones climáticas, relativamente repentinas, hicieron que las razas de Europa pasaran de cazar en los espacios abiertos a pastorear y, en cierta medida, a pescar y a labrar la tierra.
\vs p080 3:9 Estos cambios, aunque propiciaron avances culturales, produjeron algunas regresiones biológicas. Durante la anterior era de caza, las tribus superiores habían contraído matrimonio con los tipos de cautivos de guerra mejor dotados e invariablemente habían aniquilado a quienes consideraban inferiores. Pero, a medida que comenzaron a establecer asentamientos y a dedicarse a la agricultura y al comercio, empezaron a conservar como esclavos a muchos de los cautivos pobremente dotados. Y fue la progenie de estos esclavos la que posteriormente llegaría a deteriorar todo el linaje de Cromañón. Este retroceso de la cultura continuó hasta recibir un nuevo impulso procedente del este cuando la invasión final y masiva de mesopotámicos barrió Europa, absorbiendo rápidamente la estirpe y la cultura de Cromañón y dando inicio a la civilización de las razas blancas.
\usection{4. LAS INVASIONES ANDITAS DE EUROPA}
\vs p080 4:1 Aunque los anditas entraron en multitudes y en un flujo constante en Europa, se produjeron siete invasiones principales; los últimos en llegar lo hicieron a caballo, en tres grandes oleadas. Algunos penetraron en Europa por las islas del Egeo, remontando el valle del Danubio, pero la mayoría de las primeras y más puras estirpes emigraron al noroeste de Europa por la ruta del norte, a través de los pastizales del Volga y del Don.
\vs p080 4:2 Entre la tercera y la cuarta invasión, una horda de andonitas se adentró en Europa desde el norte; venían de Siberia por la ruta de los ríos rusos y el Báltico, aunque fueron de inmediato asimilados por las tribus anditas del norte.
\vs p080 4:3 Las primeras expansiones de la raza violeta más pura fueron mucho más pacíficas que las de sus posteriores descendientes anditas, que eran semimilitaristas y amantes de la conquista. Los adanitas eran pacíficos y, los noditas, belicosos. La unión de estos linajes, al mezclarse después con las razas sangiks, dio origen a los capaces y agresivos anditas, que llevaron a cabo auténticas conquistas militares.
\vs p080 4:4 \pc Pero el caballo fue el factor evolutivo determinante en el dominio de los anditas en occidente. El caballo les proporcionó, al dispersarse, la ventaja, inexistente hasta ese momento, de la movilidad, permitiendo a los últimos grupos de jinetes anditas avanzar rápidamente alrededor del Mar Caspio hasta invadir Europa por completo. Todas las oleadas anteriores de anditas se habían movilizado con tanta lentitud que tendían a disgregarse a grandes distancias de Mesopotamia. Pero estas oleadas posteriores avanzaron tan rápidamente que alcanzaron Europa en grupos cohesionados, manteniendo aún cierto grado de cultura superior.
\vs p080 4:5 El mundo habitado en su totalidad, al margen de China y la región del Éufrates, había tenido un muy limitado progreso cultural durante diez mil años, cuando los incansables jinetes anditas hicieron su aparición en el sexto y séptimo milenio a. C. Al avanzar en dirección oeste a través de las llanuras rusas, absorbiendo racialmente lo mejor del hombre azul y exterminando lo peor, se fueron mezclando hasta formar un solo pueblo. Estos fueron los ancestros de las llamadas razas nórdicas, los antepasados de los pueblos escandinavos, germanos y anglosajón.
\vs p080 4:6 \pc No pasó mucho tiempo antes de que las estirpes azules mejor dotadas fuesen totalmente absorbidas racialmente por los anditas en toda Europa septentrional. Solo en Laponia (y hasta cierto punto en Bretaña) pudieron preservar los andonitas más antiguos cierta apariencia de su identidad.
\usection{5. LA CONQUISTA ANDITA DE EUROPA SEPTENTRIONAL}
\vs p080 5:1 Las tribus de Europa septentrional estaban siendo, continuamente, reforzadas y mejoradas por el flujo migratorio constante llegado de Mesopotamia a través de las regiones del Turquestán y el sur de Rusia y, cuando las últimas oleadas de jinetes anditas se extendieron por Europa, había ya más hombres con herencia andita en esa zona que en el resto del mundo.
\vs p080 5:2 Durante tres mil años, el cuartel general militar de los anditas del norte se situó en Dinamarca. Desde este punto central, en oleadas sucesivas, partían a sus conquistas, siendo ya cada vez menos de raza andita y más de raza blanca, a medida que el paso de los siglos era testigo de la mezcla final de los conquistadores mesopotámicos con los pueblos conquistados.
\vs p080 5:3 \pc Aunque el hombre azul había sido absorbido racialmente en el norte y llegaría finalmente a sucumbir ante la caballería de los atacantes blancos que penetraron por el sur, las tribus de raza blanca mezclada, en su avance, se encontraron con la resistencia tenaz y persistente de los cromañones, pero su inteligencia de orden superior y sus reservas biológicas en continuo aumento les permitieron erradicar a esta raza más antigua.
\vs p080 5:4 Las batallas decisivas entre el hombre blanco y el hombre azul se libraron en el valle del Somme. Aquí, lo más selecto de la raza azul luchó encarnizadamente contra los anditas que avanzaban en dirección sur y, por más de quinientos años, estos cromañones defendieron con éxito su territorio, antes de sucumbir ante la estrategia militar superior de los invasores blancos. Thor, el victorioso comandante de los ejércitos del norte en la batalla final del Somme, se convirtió en el héroe de las tribus blancas septentrionales y, más tarde, fue reverenciado por algunos de ellos como un dios.
\vs p080 5:5 \pc Las plazas fuertes del hombre azul que más resistieron y de forma más prolongada fueron las del sur de Francia, pero fue junto al Somme donde se puso fin a su última gran resistencia militar. Posteriormente, la conquista se llevó a cabo mediante la infiltración del comercio, la presión demográfica a lo largo de los ríos y los continuados casamientos con las estirpes mejor dotadas, junto con el despiadado aniquilamiento de las peor dotadas.
\vs p080 5:6 Cuando el consejo tribal de los ancianos anditas juzgaba inapto a un cautivo pobremente dotado, se le entregaba, mediante una elaborada ceremonia, a los sacerdotes chamanes que lo conducían al río y le administraban los ritos de iniciación para las “felices praderas de caza” ---la inmersión letal---. De esta manera, los invasores blancos de Europa exterminaron a todos aquellos pueblos con los que se encontraban que no fuesen absorbidos de forma rápida para sus propias filas y, fue así como el hombre azul llegó, y lo hizo pronto, a su fin.
\vs p080 5:7 \pc El hombre azul de Cromañón constituyó la base biológica de las razas europeas modernas, pero sobrevivió únicamente en la medida en la que fue racialmente absorbido por los poderosos conquistadores de sus territorios, llegados más tarde. La estirpe azul aportó a la raza blanca de Europa robustez y mucho vigor físico, pero el humor y la imaginación de los mezclados pueblos europeos procedían de los anditas. Esta unión de la raza andita y azul, que dio origen a las razas blancas nórdicas, trajo consigo un lapsus inmediato de la civilización andita, un retraso de índole transitoria. Finalmente, la superioridad latente de estos bárbaros nórdicos se manifestó y resultó en la civilización europea de hoy día.
\vs p080 5:8 \pc Hacia el 5000 a. C., las razas blancas en evolución dominaban toda Europa septentrional, incluyendo el norte de Alemania, el norte de Francia y las Islas Británicas. Durante algún tiempo, Europa central estuvo bajo el control del hombre azul y de los andonitas de cabeza redonda. Estos últimos se hallaban principalmente en el valle del Danubio; los anditas nunca llegarían a desalojarlos de allí del todo.
\usection{6. LOS ANDITAS A LO LARGO DEL NILO}
\vs p080 6:1 Desde los tiempos de las últimas emigraciones anditas, la cultura declinó en el valle del Éufrates, y el inmediato centro de la civilización se trasladó al valle del Nilo. Egipto sucedió a Mesopotamia como sede del grupo humano más avanzado de la tierra.
\vs p080 6:2 El valle del Nilo comenzó a sufrir inundaciones poco antes de que estas ocurrieran en los valles de la Mesopotamia, pero les fueron menos perjudiciales. Este primer contratiempo estuvo más que compensado por el flujo continuo de emigrantes anditas, de manera que la cultura de Egipto, aunque de hecho proveniente de la región del Éufrates, pareció avanzar. Si bien, en el año 5000 a. C., durante el período de las inundaciones de Mesopotamia, había siete grupos diferentes de seres humanos en Egipto; todos ellos, salvo uno, venían de Mesopotamia.
\vs p080 6:3 \pc Cuando se produjo el último éxodo desde el valle del Éufrates, Egipto tuvo la suerte de hacerse con una gran cantidad de los más diestros artistas y artesanos. Estos artesanos anditas se encontraron como en casa ya que estaban muy familiarizados con la vida fluvial, sus inundaciones, el riego y las temporadas de sequía. Disfrutaban del emplazamiento protegido del valle del Nilo; estaban allí mucho menos sometidos a las incursiones y a los ataques hostiles de lo que estaban a lo largo del Éufrates. Y aportaron considerablemente al desarrollo metalúrgico de los egipcios. Aquí trabajaron los minerales de hierro procedentes del monte Sinaí en lugar de aquellos de las regiones del Mar Negro.
\vs p080 6:4 \pc Muy pronto, los egipcios agruparon a sus deidades locales en un elaborado sistema nacional de dioses. Desarrollaron una extensa teología y tenían un sacerdocio igualmente extenso pero dificultoso. Diversos líderes trataron de revivir los restos de las tempranas enseñanzas religiosas de los setitas, pero estos intentos tuvieron corta vida. Los anditas construyeron las primeras estructuras de piedra en Egipto. La primera y más excelente de las pirámides en este material fue erigida por Imhotep, un genio arquitectónico andita, mientras ejercía como primer ministro. Previamente, las edificaciones se habían hecho de ladrillo y, aunque en diferentes partes del mundo ya se había usado la piedra, esta fue la primera vez que se hizo en Egipto. Pero, desde los días de este gran arquitecto, el arte de la construcción fue decayendo de manera constante.
\vs p080 6:5 Esta época brillante de la cultura se interrumpió debido a las guerras internas que se libraban a lo largo del Nilo, y el país fue pronto invadido, como lo había sido Mesopotamia, por las tribus inferiores de la inhóspita Arabia y por los negros del sur. Como resultado, el avance social decayó continuamente durante más de quinientos años.
\usection{7. LOS ANDITAS DE LAS ISLAS MEDITERRÁNEAS}
\vs p080 7:1 Durante la decadencia de la cultura mesopotámica, subsistió por algún tiempo en las islas del Mediterráneo oriental una civilización superior.
\vs p080 7:2 Alrededor del año 12\,000 a. C., una brillante tribu de anditas emigró a Creta. Se trataba de la única isla que sería tan tempranamente colonizada por un grupo de orden superior, y pasaron casi dos mil años antes de que los descendientes de estos marineros se expandieran a las islas vecinas. Este grupo estaba constituido por los anditas de cabeza estrecha, de estatura más pequeña, que se habían vinculado en matrimonio con los miembros del colectivo vanita de los noditas del norte. Todos ellos medían menos de un metro ochenta de altura y habían sido literalmente expulsados del continente por sus compañeros más altos, aunque menos dotados. Los emigrantes que llegaron a Creta eran grandes expertos en la tejeduría, metalurgia, alfarería, fontanería y en el uso de la piedra como material de construcción. Practicaban la escritura y vivían del pastoreo y la agricultura.
\vs p080 7:3 Casi dos mil años después de la colonización de Creta, un grupo de los descendientes de alta estatura de Adánez se abrió paso a través de las islas norteñas hasta llegar a Grecia. Venían casi directamente desde sus altiplanicies natales al norte de Mesopotamia. Sato, un descendiente directo de Adánez y Ratta, condujo a estos progenitores de los griegos hasta occidente.
\vs p080 7:4 El grupo que llegó finalmente a establecerse en Grecia estaba compuesto por trescientos setenta y cinco personas elegidas y de orden superior, integrantes del resto de la segunda civilización de los adanecitas. Estos hijos posteriores de Adánez portaban las entonces valiosísimas estirpes de las emergentes razas blancas. Poseían un elevado nivel intelectual y, desde el punto de vista físico, eran los hombres más hermosos que habían existido desde los días del primer Edén.
\vs p080 7:5 \pc Enseguida Grecia y la región de las islas egeas sucedieron a Mesopotamia y a Egipto como centro occidental del comercio, el arte y la cultura. Pero tal como había sucedido en Egipto, prácticamente de la misma manera, todo el arte y la ciencia del mundo egeo provenían de Mesopotamia, salvo por la cultura de los precursores adanecitas de los griegos. Todo el arte y el ingenio de estas generaciones de personas es legado directo del linaje de Adánez, el primer hijo de Adán y Eva, y de su extraordinaria segunda esposa, una hija descendiente en línea ininterrumpida del puro linaje nodita de la comitiva del príncipe Caligastia. No es de extrañar que los griegos tuviesen tradiciones mitológicas que aludiesen a su descendencia directa de los dioses y de seres suprahumanos.
\vs p080 7:6 La región egea pasó por cinco etapas culturales distintas, cada cual menos espiritual que la anterior y, en poco tiempo, la última era gloriosa del arte sucumbió bajo el peso de la rápida multiplicación de seres pobremente dotados, descendientes de los esclavos de la zona del Danubio, que habían llevado generaciones posteriores de griegos.
\vs p080 7:7 Fue en Creta, durante esta época, cuando alcanzó su punto álgido \bibemph{el culto a la madre} por parte de los descendientes de Caín. Esta práctica religiosa glorificaba a Eva; se adoraba a la “gran madre”. Había imágenes de ella por todas partes. Se erigieron miles de santuarios públicos por toda Creta y Asia Menor. Y este culto subsistió hasta los tiempos de Cristo, llegando a ser más tarde incorporado en la religión cristiana primitiva bajo la forma de la glorificación y adoración de María, la madre terrenal de Jesús.
\vs p080 7:8 \pc Hacia el año 6500 a. C., se había producido una gran decadencia en la herencia espiritual de los anditas. Los descendientes de Adán estaban bastante dispersos y se habían visto prácticamente absorbidos por las razas humanas más antiguas y más numerosas. Y este declive de la civilización andita, unido a la desaparición de sus estándares religiosos, dejó a las razas del mundo espiritualmente empobrecidas y en un estado deplorable.
\vs p080 7:9 \pc Hacia el año 5000 a. C., las tres estirpes más puras de los descendientes de Adán estaban en Sumeria, el norte de Europa y Grecia. Toda Mesopotamia se iba deteriorando lentamente debido al flujo de razas mezcladas y más oscuras que se infiltraban desde Arabia. Y la llegada de estos pueblos peor dotados contribuyó aún más a la dispersión del residuo biológico y cultural de los anditas. De todo el creciente fértil, los pueblos más intrépidos se esparcieron en dirección este, hacia las islas. Estos emigrantes cultivaban tanto el cereal como la hortaliza, y llevaron consigo animales domésticos.
\vs p080 7:10 Sobre el año 5000 a. C., una imponente multitud de mesopotámicos avanzados dejaron el valle del Éufrates y se establecieron en la isla de Chipre. Dos mil años más tarde, esta civilización caería aniquilada por las hordas bárbaras del norte.
\vs p080 7:11 \pc Otra gran colonia se estableció en el Mediterráneo, cerca del lugar que más adelante sería Cartago. Y, desde el norte de África, un gran número de anditas entraron en España. Posteriormente, se mezclaron en Suiza con sus hermanos, que con anterioridad, desde las islas egeas, habían ido a Italia.
\vs p080 7:12 \pc Cuando Egipto siguió a Mesopotamia en su declive cultural, muchas de las familias más capaces y adelantadas huyeron a Creta, engrandeciendo así, de forma considerable, a esta civilización, ya avanzada. Y cuando la llegada de grupos peor dotados procedentes de Egipto amenazó después la civilización cretense, las familias de más cultura se trasladaron hacia el oeste, a Grecia.
\vs p080 7:13 \pc Los griegos no solo fueron grandes maestros y artistas, sino también los más grandes comerciantes y colonizadores del mundo. Antes de sucumbir a la avalancha de mediocridad que acabaría por engullir su arte y su comercio, lograron establecer, hacia el oeste, tantos puestos avanzados de cultura, que un gran número de los logros de la civilización griega primitiva persistió en los pueblos subsiguientes del sur de Europa, y muchos de los mezclados descendientes de estos adanecitas se fueron incorporando en las tribus de las tierras continentales contiguas.
\usection{8. LOS ANDONITAS DEL VALLE DEL DANUBIO}
\vs p080 8:1 Los pueblos anditas del valle del Éufrates emigraron al norte, hacia Europa, para mezclarse con los hombres azules y, al oeste, hacia las regiones mediterráneas, para unirse con los remanentes de los saharianos ya mezclados y los hombres azules del sur. Estas dos ramificaciones de la raza blanca estaban, y están ahora, muy distanciadas por los supervivientes montañeses de cabeza ancha de las primeras tribus andonitas, que, durante largo tiempo, habían habitado estas regiones centrales.
\vs p080 8:2 Estos descendientes de Andón estaban dispersos por la mayoría de las regiones montañosas de Europa central y del suroeste. A menudo, se vieron reforzados por los llegados de Asia Menor, región que ocupaban de forma predominante. Los antiguos hititas procedían directamente del linaje andonita; su piel pálida y sus cabezas anchas eran típicas de esa raza. Los ancestros de Abraham portaban esta estirpe, que contribuyó mucho a los rasgos faciales de sus posteriores descendientes judíos. Estos, aunque su cultura y religión provenían de los anditas, hablaban una lengua muy diferente. Su idioma era claramente andonita.
\vs p080 8:3 Las tribus que vivían en casas erigidas sobre pilotes o pilares de troncos en los lagos de Italia, Suiza y Europa del sur eran expansiones periféricas de las emigraciones africana, egea y, más particularmente, danubianas.
\vs p080 8:4 Los danubianos eran andonitas, agricultores y pastores que habían entrado en Europa a través de la península balcánica y que se dirigían lentamente hacia el norte a través del valle del Danubio. Trabajaban la cerámica y cultivaban la tierra, y preferían vivir en los valles. El asentamiento más septentrional de los danubianos estaba en Lieja, Bélgica. Estas tribus declinaron rápidamente a medida que se alejaban del centro y fuente de su cultura. Su mejor cerámica se produjo en los asentamientos más tempranos.
\vs p080 8:5 Los danubianos se convirtieron en adoradores de la madre como resultado de la labor de los misioneros de Creta. Más adelante, estas tribus se fusionaron con grupos de marineros andonitas, también adoradores de la madre, que llegaron por barco desde la costa del Asia Menor. Así pues, gran parte de Europa central se colonizó pronto por estos tipos mezclados de raza blanca de cabeza ancha, que practicaban este culto de adoración y el rito religioso de cremación de sus difuntos; los practicantes de dicho culto tenían la costumbre de incinerar a sus muertos en cabañas de piedra.
\usection{9. LAS TRES RAZAS BLANCAS}
\vs p080 9:1 Hacia el final de las emigraciones anditas, las mezclas raciales en Europa llegaron a generalizarse en tres razas blancas, como se detalla a continuación:
\vs p080 9:2 \li{1.}\bibemph{La raza blanca septentrional}. Esta raza, a la que se denominó nórdica, estaba compuesta principalmente por el hombre azul más el andita, aunque también portaba una cantidad considerable de sangre andonita, junto con proporciones más pequeñas de sangre sangik roja y amarilla. Este tipo de raza blanca contenía, pues, los cuatro linajes humanos más deseables. Pero la herencia principal provenía del hombre azul. El típico nórdico primitivo tenía la cabeza alargada y era alto y rubio. Si bien, hace mucho tiempo que tal raza se mezcló por completo con todas las otras ramificaciones de los pueblos blancos.
\vs p080 9:3 La cultura primitiva de Europa, que los invasores nórdicos hallaron, era la de los danubianos mezclados con el hombre azul y en proceso de retrogresión. La cultura nórdico\hyp{}danesa y la cultura danubiano\hyp{}andonita se encontraron y se mezclaron en el Rin, como se evidencia por la existencia en la actualidad de dos grupos raciales en Alemania.
\vs p080 9:4 Los nórdicos continuaron el comercio del ámbar desde la costa báltica, desarrollando un gran intercambio comercial a través del Paso del Brennero con los habitantes de cabeza ancha del valle del Danubio. Este contacto prolongado con los danubianos llevó a estos habitantes norteños a la adoración de la madre y, durante varios miles de años, la ceremonia de incinerar a los muertos fue casi generalizada en toda Escandinavia. Esto explica por qué los restos de las primitivas razas blancas, aunque están enterrados por toda Europa, no se pueden encontrar ---tan solo se hallan sus cenizas en urnas de piedra y arcilla---. Estos hombres blancos también construían viviendas; nunca vivieron en cuevas. Y de nuevo esto explica por qué hay tan pocos vestigios de la temprana cultura del hombre blanco, a pesar de que el hombre anterior de Cromañón esté bien preservado por haber sido sus restos sellados firmemente en cuevas y grutas. Por así decirlo, un día hay en Europa septentrional una cultura primitiva de danubianos en retrogresión y del hombre azul y, al día siguiente, de forma repentina, aparece la cultura, inmensamente superior, del hombre blanco.
\vs p080 9:5 \li{2.}\bibemph{La raza blanca central}. Aunque este grupo incluye estirpes azules, amarillas y anditas, es predominantemente andonita. Estos pueblos son de cabeza ancha, morenos y fornidos. Conforman una especie de cuña entre la raza nórdica y la mediterránea, con la parte ancha apoyada en Asia y el vértice penetrando en el este de Francia.
\vs p080 9:6 Durante casi veinte mil años, los anditas habían empujado a los andonitas cada vez más lejos, hacia el norte de Asia central. Hacia el año 3000 a. C., el aumento de la aridez llevó a estos andonitas de regreso al Turquestán. Este obligado desplazamiento de los andonitas hacia el sur continuó por más de mil años y, separándose alrededor del Mar Caspio y del Mar Negro, penetraron en Europa a través tanto de los Balcanes como de Ucrania. En esta invasión iban los restantes grupos de descendientes de Adánez y, durante la segunda mitad del período en el que ocurrió, se unió un gran número de anditas iraníes al igual que muchos de los descendientes de los sacerdotes setitas.
\vs p080 9:7 Para el año 2500 a. C., llevados por este impulso hacia el oeste, los andonitas llegaron a Europa. Y esta ocupación de toda Mesopotamia, Asia Menor y la cuenca del Danubio por los bárbaros de las colinas del Turquestán constituyó el retraso cultural más grave y duradero ocurrido hasta ese momento. Estos invasores definitivamente andonizaron el carácter de las razas centroeuropeas, que desde entonces permanecieron típicamente alpinas.
\vs p080 9:8 \li{3.}\bibemph{La raza blanca meridional}. Esta raza mediterránea de piel morena era una mezcla del hombre andita y del hombre azul, con menor proporción de linaje andonita que en el norte. Este grupo también absorbió una cantidad considerable de sangre sangik secundaria a través de los saharianos. En tiempos posteriores, esta ramificación meridional de la raza blanca recibió la infusión de poderosos elementos anditas procedentes del Mediterráneo oriental.
\vs p080 9:9 Sin embargo, los anditas no dominaron las costas del Mediterráneo hasta los tiempos de las grandes invasiones nómadas del año 2500 a. C. El transporte y el comercio terrestres estuvieron prácticamente interrumpidos durante estos siglos cuando los nómadas invadieron las zonas orientales mediterráneas. Esta obstrucción del viaje por tierra dio lugar a un gran crecimiento del transporte y el comercio marítimos; el comercio por el Mediterráneo estaba en pleno auge hace unos cuatro mil quinientos años. Y este desarrollo del tráfico marítimo resultó en la expansión repentina de los descendientes de los anditas por todo el territorio costero de la cuenca mediterránea.
\vs p080 9:10 Estas mezclas raciales sentaron las bases para el nacimiento de la raza europea meridional, la más mezclada de todas. Y, desde esos días, dicha raza ha experimentado, además, otras mezclas, fundamentalmente con los pueblos azules\hyp{}amarillos\hyp{}anditas de Arabia. De hecho, esta raza mediterránea se mezcló, sin restricción alguna, con los pueblos de los alrededores de tal modo que es prácticamente imposible distinguirla como una categoría separada; si bien, en general, sus integrantes son de baja estatura, cabeza alargada y morenos.
\vs p080 9:11 En el norte, los anditas, mediante la guerra y el matrimonio, erradicaron al hombre azul, pero, en el sur, el hombre azul sobrevivió en mayor número. Los vascos y los bereberes representan la supervivencia de las dos ramificaciones de esta raza, pero incluso estos pueblos se mezclaron por completo con los saharianos.
\vs p080 9:12 \pc Esta era la imagen de la mezcla de razas presentes en Europa central hacia el año 3000 a. C. A pesar de la transgresión parcial de Adán, sí hubo una mezcla de las estirpes más dotadas.
\vs p080 9:13 \pc Estos eran los tiempos de la Edad Nueva de Piedra que se solapaba con la inminente Edad del Bronce. En Escandinavia, estaba la Edad del Bronce asociada con la adoración a la madre; en el sur de Francia y en España, la Edad Nueva de Piedra se asociaba con la adoración al sol. Ese fue el momento de la construcción de los templos al sol, circulares y sin techo. Las razas blancas europeas eran constructoras dinámicas, que disfrutaban erigiendo grandes piedras en homenaje a este astro, al igual que harían más tarde sus descendientes en Stonehenge. La práctica de adorar al sol indica que se trataba de un gran período para la agricultura en Europa meridional.
\vs p080 9:14 Las supersticiones de esta era, relativamente reciente, de adoración al sol perduran incluso hoy en día en las costumbres populares de Bretaña. Aunque cristianizados hace más de mil quinientos años, estos bretones aún conservan amuletos de la Edad Nueva de Piedra para repeler el mal de ojo. Todavía tienen piedras del trueno en la chimenea como protección contra el relámpago. Los bretones no se mezclaron nunca con los nórdicos de Escandinavia. Son los supervivientes de los habitantes andonitas primigenios de Europa occidental, mezclados con el linaje mediterráneo.
\vs p080 9:15 \pc Pero es un error pretender clasificar a los pueblos blancos como nórdicos, alpinos y mediterráneos. En conjunto, se han producido demasiada mescolanza como para admitir tal agrupamiento. En algún momento, existía una división bastante bien definida de la raza blanca en tales grupos, pero la mezcla acaecida desde entonces hace que ya no sea posible identificar con claridad estas distinciones. Incluso en el año 3000 a. C., los antiguos grupos sociales, al igual que los habitantes actuales de América del Norte, no constituían una sola raza.
\vs p080 9:16 Esta cultura europea continuó creciendo durante cinco mil años y, en cierto modo, entremezclándose. Pero la barrera del idioma impidió la plena reciprocidad de las diversas naciones occidentales. Durante el último siglo, esta cultura ha estado experimentando su mejor oportunidad de mezclarse con la población cosmopolita de América del Norte; y el futuro de ese continente estará determinado por la calidad de los componentes raciales a los que se les permitan entrar en su población presente y futura, al igual que por el nivel de la cultura social que se mantenga.
\vsetoff
\vs p080 9:17 [Exposición de un arcángel de Nebadón.]
