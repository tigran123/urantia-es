\upaper{91}{Evolución de la oración}
\author{Jefe de los seres intermedios}
\vs p091 0:1 La oración, como instrumento de la religión, evolucionó desde expresiones previas no religiosas monologadas y dialogadas. Con el logro de la conciencia de sí mismo por parte del hombre primitivo, se dio la inevitable consecuencia de llegar a ser consciente del otro, manifestándose el doble potencial de respuesta social y de reconocimiento de Dios.
\vs p091 0:2 Las formas más tempranas de la oración no iban dirigidas a la Deidad. Estas manifestaciones se parecían mucho a lo que se le puede decir a un amigo al emprender alguna importante iniciativa: “Deséame suerte”. El hombre primitivo estaba esclavizado a la magia; la suerte, tanto buena como mala, era parte integrante de todas las cuestiones de su vida. En un primer momento, estas peticiones de suerte eran monólogos ---simplemente un tipo de pensamiento en voz alta por parte del sirviente de la magia---. Luego, estos creyentes en la suerte recabaron el apoyo de sus amigos y familiares y, enseguida, se llevó a cabo alguna forma de ceremonia que incluía a todo el clan o tribu.
\vs p091 0:3 Cuando evolucionó la noción que se tenía de los espectros y de los espíritus, estas peticiones se destinaron a entes sobrehumanos y, al tener conciencia de los dioses, tales manifestaciones alcanzaron niveles de verdaderas oraciones. Sirven como ejemplo las primitivas oraciones religiosas de algunas tribus australianas, que precedieron a su creencia en espíritus y en personas sobrehumanas.
\vs p091 0:4 En la actualidad, la tribu toda de la India lleva a cabo esta práctica de orar a nadie en particular, tal como hacían los pueblos primitivos antes de los tiempos en los que adquirieron una conciencia religiosa. Pero, entre los todas, esto constituye un retroceso de su religión, degradada a este nivel primitivo. Los ritos actuales de los sacerdotes lecheros de los todas no representan una ceremonia religiosa, dado que estas oraciones impersonales no contribuyen en nada a conservar e incrementar sus valores sociales, morales o espirituales.
\vs p091 0:5 La oración prerreligiosa formaba parte de las prácticas del maná de los melanesios, las creencias en el \bibemph{oudah} de los pigmeos africanos y las supersticiones sobre \bibemph{manitú} de los indios norteamericanos. Las tribus baganda de África acaban de salir de la etapa maná de la oración. En esta temprana confusión evolutiva, los hombres oran a los dioses ---locales y nacionales---, fetiches, amuletos, espectros, gobernantes y gente corriente.
\usection{1. LA ORACIÓN PRIMITIVA}
\vs p091 1:1 La labor de la religión evolutiva primitiva es conservar y aumentar los valores sociales, morales y espirituales esenciales que lentamente van cobrando forma. La humanidad no percibe esta misión de la religión de forma consciente, sino que se realiza principalmente mediante la acción de la oración. Orar es un esfuerzo involuntario, aunque personal y colectivo, de cualquier grupo por garantizar (hacer realidad) esta preservación de los valores superiores. A no ser por la salvaguarda de la oración, todos los días sagrados reverterían rápidamente a la condición de simples días festivos.
\vs p091 1:2 \pc La religión y sus instrumentos, de los cuales la oración es el principal, se coaligan solamente con aquellos valores que tienen un reconocimiento social de carácter general, esto es la aprobación del grupo. Por lo tanto, cuando el hombre primitivo intentaba satisfacer sus emociones más básicas o cumplir, sin paliativos, sus ambiciones egoístas, se le privaba del consuelo de la religión y de la ayuda de la oración. Si la persona trataba de llevar a cabo algo antisocial, estaba obligado a buscar la ayuda de la magia no religiosa, recurrir a los hechiceros y quedar de este modo despojado de la asistencia de la oración. En consecuencia, la oración se convirtió muy pronto en un poderoso elemento para la promoción de la evolución social, el progreso moral y el logro espiritual.
\vs p091 1:3 Pero la mente primitiva no era lógica ni tampoco coherente. Los hombres primitivos no entendían que las cosas materiales no pertenecían al ámbito de la oración. Estas almas de mente sencilla consideraban que el alimento, el refugio, la lluvia, la caza y otros bienes materiales mejoraban el bienestar social, y, así pues, comenzaron a orar por estas bendiciones de carácter físico. Aunque esto constituía una perversión de la oración, contribuía a su esfuerzo de conseguir fines materiales mediante acciones sociales y éticas. Tal aberración de la oración, aunque degradaba los valores espirituales de los pueblos, elevaba, no obstante, de forma directa, sus costumbres económicas, sociales y éticas.
\vs p091 1:4 En el tipo de mente más primitivo, la oración solo es un monólogo. Tempranamente se convierte en diálogo y asciende rápidamente hasta el nivel de adoración colectiva. La oración supone que las encantaciones premágicas de la religión primitiva han evolucionado hasta ese estadio en el que la mente humana reconoce la realidad de poderes o seres benefactores que pueden mejorar los valores sociales y aumentar los ideales morales y, además, que esas influencias son sobrehumanas y distintas del ego humano autoconsciente y de sus semejantes mortales. La verdadera oración no aparece, por tanto, hasta que la acción del ministerio religioso se contempla como \bibemph{personal}.
\vs p091 1:5 \pc La oración se relaciona escasamente con el animismo, pero estas creencias pueden existir paralelamente a los sentimientos religiosos emergentes. Muchas veces, la religión y el animismo han tenido orígenes totalmente independientes.
\vs p091 1:6 \pc Para aquellos mortales que no se han liberado de la primitiva servidumbre al miedo, existe el peligro real de que la oración pueda llevar a un sombrío sentido del pecado, a unos remordimientos no justificados de culpa, real o imaginaria. Pero, en los tiempos modernos, no es probable que haya muchas personas que dediquen suficiente tiempo a la oración como para llegar a esta nociva cavilación sobre su indignidad o pecaminosidad. Los peligros concomitantes a la distorsión y perversión de la oración consisten en la ignorancia, la superstición, la cristalización, la desvitalización, el materialismo y el fanatismo.
\usection{2. EVOLUCIÓN DE LA ORACIÓN}
\vs p091 2:1 Las primeras oraciones fueron simplemente deseos verbalizados, la expresión de deseos sinceros. Luego, la oración se convirtió en un modo de conseguir la cooperación de los espíritus. Y, después, realizó la labor superior de asistir a la religión en la conservación de todos los valores dignos.
\vs p091 2:2 La oración y la magia surgieron como resultado de la respuesta adaptativa del hombre al entorno urantiano. Pero, al margen de esta relación de índole general, ambas tienen poco en común. La oración siempre ha señalado una acción positiva del ego orante; siempre ha sido psíquica y a veces espiritual. La magia normalmente ha representado un intento de manipular la realidad sin menoscabar el ego del manipulador, el practicante de la magia. A pesar de sus orígenes independientes, la magia y la oración se han interrelacionado a menudo en las fases posteriores de su desarrollo. En ocasiones, la magia ha escalado, al elevar sus metas, desde sus prácticas, a través de ritos y encantamientos, hasta el umbral de la verdadera oración. La oración, en algunos casos, se ha vuelto tan materialista que ha degenerado en un modo pseudomágico de eludir el gasto de ese esfuerzo que se requiere para la solución de los problemas urantianos.
\vs p091 2:3 \pc Cuando el hombre aprendió que la oración no podía ejercer coacción sobre los dioses, se volvió entonces más bien una petición, en la búsqueda de favores. Pero la oración más verdadera es en realidad una comunión entre el hombre y su Hacedor.
\vs p091 2:4 \pc La aparición de la idea del sacrificio en cualquier religión indefectiblemente va en detrimento de una mayor eficacia de la verdadera oración, en cuanto que los hombres tratan de sustituir sus ofrendas de posesiones materiales por aquellas en las que sus propias voluntades se consagran a hacer la voluntad de Dios.
\vs p091 2:5 Cuando la religión se despoja de un Dios personal, las oraciones adquieren niveles teológicos y filosóficos. Cuando el concepto más elevado de Dios que posee una religión es el de una Deidad impersonal, tal como sucede en el idealismo panteístico, aunque facilite la base para determinadas formas de comunión mística, resulta fatal para la verdadera oración, que siempre representa la comunión del hombre con un ser personal y superior.
\vs p091 2:6 Durante los tempranos tiempos de la evolución racial e incluso en los tiempos actuales, en la experiencia día tras día del mortal común, la oración es, en gran medida, el hecho de la relación del hombre con su propio subconsciente. Pero también existe un ámbito de la oración en el que la persona intelectualmente alerta y espiritualmente en avance logra un mayor o menor contacto con los niveles supraconscientes de la mente humana, que es el ámbito del modelador del pensamiento interior. Además, hay un determinado aspecto espiritual de la verdadera oración que concierne a su recepción y reconocimiento por parte de las fuerzas espirituales del universo, y que es enteramente diferente de cualquier proceso intelectual y humano.
\vs p091 2:7 La oración contribuye significativamente al desarrollo del sentimiento religioso de la mente humana en evolución. Tiene un poderoso efecto en la prevención del aislamiento de la persona.
\vs p091 2:8 La oración constituye una práctica vinculada a las religiones naturales de la evolución racial, que forma parte, al mismo tiempo, de los valores vivenciales de las religiones superiores ---o religiones reveladas---, caracterizadas por su excelencia ética.
\usection{3. LA ORACIÓN Y EL ÁLTER EGO}
\vs p091 3:1 Cuando empiezan a hacer uso del lenguaje, los niños son propensos a pensar en voz alta, a expresar sus pensamientos en palabras, incluso si no hay nadie presente para oírlos. Con la llegada de la imaginación creativa, los niños manifiestan una tendencia a conversar con compañeros imaginarios. De esta manera, un yo incipiente busca comunicarse con un \bibemph{álter ego} ficticio. Mediante este método, él aprende pronto a convertir sus conversaciones monologadas en pseudodiálogos, en los que este álter ego responde a su pensamiento verbal y a la expresión de sus deseos. Gran parte del pensamiento del adulto se lleva a cabo mentalmente en forma de conversación.
\vs p091 3:2 La forma incipiente y primitiva de orar era muy parecida a las recitaciones semimágicas de la tribu de los todas de hoy en día; eran oraciones que no se dirigían a nadie en particular. Si bien, tal modo de orar tiende a evolucionar en un tipo de diálogo comunicativo debido a la gradual aparición de la idea del álter ego. Con el paso del tiempo, el concepto del álter ego se enaltece hasta un estatus superior adquiriendo dignidad divina y surge la oración como instrumento de la religión. A través de muchas etapas y durante largas eras, este tipo primitivo de oración está llamado a desarrollarse antes de lograr el nivel de oración inteligente y verdaderamente ética
\vs p091 3:3 Como se percibe a través de generaciones sucesivas de mortales orantes, el álter ego evoluciona desde los espectros, fetiches y espíritus hasta los dioses politeístas y, en última instancia, hasta el Dios Único, un ser divino que personifica los más altos ideales y las más nobles aspiraciones del yo que ora. Y, por tanto, la oración obra como el instrumento de mayor poder de la religión en la conservación de los más altos valores e ideales de estos orantes. Desde el momento del nacimiento del álter ego hasta la aparición del concepto de un Padre divino y celestial, la oración es siempre una práctica socializadora, moralizadora y espiritualizadora.
\vs p091 3:4 La simple oración de fe pone de manifiesto un gran desarrollo evolutivo en la experiencia humana por la que las antiguas conversaciones con el símbolo ficticio del álter ego de la religión primitiva se han enaltecido, hasta el nivel de la comunión con el espíritu del Infinito y de una auténtica conciencia de la realidad del Dios eterno y Padre del Paraíso de toda la creación inteligente.
\vs p091 3:5 Aparte de todo lo que constituye el superyó en el hecho de orar, debe recordarse que la oración ética es un modo magnífico de elevar el ego y reforzar el yo para conseguir una vida mejor y unos logros superiores. La oración alienta al ego humano a mirar a dos lados para buscar ayuda: asistencia material en el depósito subconsciente de la experiencia humana e inspiración y guía en las fronteras supraconscientes donde se realiza el contacto de lo material con lo espiritual, esto es, con el mentor misterioso.
\vs p091 3:6 La oración siempre ha sido y siempre será una experiencia humana doble: un proceso psicológico y un método espiritual en correlación. Y estas dos capacidades de la oración nunca se pueden separar del todo.
\vs p091 3:7 La oración iluminada debe reconocer no solo a un Dios externo y personal sino también a una Divinidad interna e impersonal, o modelador interior. Es muy conveniente que el hombre, cuando ore, trate de comprender el concepto del Padre Universal en el Paraíso, pero el método más eficaz, por motivos prácticos, será retomar al concepto de un álter ego cercano, tal como acostumbraba a hacer la mente primitiva, y luego reconocer que la idea de este álter ego ha evolucionado de mera ficción a la verdad de que Dios habita en el hombre mortal, mediante la presencia real del modelador para que el hombre pueda hablar frente a frente, por así decirlo, con un álter ego real y genuino y divino que reside en él y que es la presencia y esencia mismas del Dios vivo, del Padre Universal.
\usection{4. LA ORACIÓN ÉTICA}
\vs p091 4:1 Ninguna oración puede ser ética cuando el peticionario busca su propio provecho sobre el de sus semejantes. La oración interesada y materialista es incompatible con las religiones éticas, que se basan en un amor desprendido y divino. Toda oración no ética revierte a los niveles primitivos de la pseudomagia y es impropia de civilizaciones en avance y de religiones preclaras. La oración interesada transgrede el espíritu de toda ética fundamentada en una justicia amorosa.
\vs p091 4:2 La oración no debe degradarse hasta el punto de convertirse en un sustituto de la acción. Toda oración ética constituye un estímulo para la adopción de medidas y la guía para la lucha progresiva por los fines idealistas al alcance del superyó.
\vs p091 4:3 En todas vuestras oraciones, sed \bibemph{justos;} no esperéis que Dios muestre favoritismos, que os ame más a vosotros que a sus otros hijos, a vuestros amigos, vecinos e incluso a vuestros enemigos. Si bien, la oración de las religiones naturales o evolutivas no es ética en un principio, como lo es en las religiones reveladas posteriores. Toda oración, sea esta individual o comunal, puede ser interesada o altruista. Esto es, la oración puede centrarse en el yo o en los demás. Cuando la oración no procura nada para aquel que ora ni para sus semejantes, entonces, esta actitud del alma tiende hacia niveles de verdadera adoración. Las oraciones egocéntricas incluyen confesiones y suplicaciones y, a menudo, consisten en peticiones de favores materiales. La oración es algo más ética cuando se ocupa del perdón y busca sabiduría para lograr un mejor dominio de sí mismo.
\vs p091 4:4 La oración altruista fortalece y consuela; la oración materialista comporta decepción y desencanto a medida que, con el avance de los descubrimientos científicos, se demuestra que el hombre vive en un universo físico de ley y orden. La niñez de una persona o de una raza se caracteriza por la oración primitiva, interesada y materialista. Y, en cierto modo, todas estas peticiones son eficaces en cuanto que invariablemente conducen a esas iniciativas y actuaciones que contribuyen a lograr respuestas a tales oraciones. La verdadera oración de fe siempre aporta a la mejora del modo de vida, incluso cuando las peticiones no sean merecedoras de un reconocimiento espiritual. Si bien, la persona espiritualmente avanzada debe actuar con gran cautela al tratar de disuadir a la mente primitiva o inmadura respecto a tales oraciones.
\vs p091 4:5 \pc Recordad que, incluso si la oración no cambia a Dios, con mucha frecuencia efectúa cambios grandes y perdurables en aquel que ora con fe y espera con confianza. La oración ha sido el ancestro de gran parte de la paz mental, la alegría, la calma, el valor, el autodominio y la equidad habidos entre los hombres y las mujeres de las razas en evolución.
\usection{5. REPERCUSIONES SOCIALES DE LA ORACIÓN}
\vs p091 5:1 En la adoración de los antepasados, la oración conduce al cultivo de los ideales ancestrales. Pero la oración, como componente de la adoración a la Deidad, trasciende todas las otras prácticas de tal naturaleza, puesto que encamina al cultivo de los ideales divinos. Conforme el concepto del álter ego de la oración se convierte en supremo y divino, de igual manera los ideales del hombre se elevan en consonancia desde lo meramente humano hasta niveles sublimes y divinos, y el resultado de cualquier oración de esta índole es el fortalecimiento del carácter humano y la profunda unificación de la persona humana.
\vs p091 5:2 Pero no es necesario que la oración sea siempre individual. Orar en grupo o en congregación es muy eficaz porque tiene repercusiones altamente socializadoras. Cuando un grupo se dedica a la oración comunitaria para el acrecentamiento moral y la elevación espiritual, estas prácticas religiosas tienen su efecto en las personas que componen tal grupo; todos ellos se vuelven mejores debido a esta misma participación. Incluso toda una ciudad o una nación pueden beneficiarse de tales prácticas oracionales. La confesión, el arrepentimiento y la oración han motivado a personas, ciudades y naciones y razas enteras a realizar grandes esfuerzos de reforma y valientes actos conducentes a audaces logros.
\vs p091 5:3 \pc Si verdaderamente deseáis superar la costumbre de criticar a un amigo, la forma más rápida y certera de lograr dicho cambio de actitud consiste en establecer la costumbre de orar por esa persona cada día de vuestra vida. Pero las repercusiones sociales de tales oraciones dependen en gran parte de dos condiciones:
\vs p091 5:4 \li{1.}La persona por la que se ora debe saber que se está orando por ella.
\vs p091 5:5 \li{2.}La persona que ora debe relacionarse socialmente y de forma estrecha con la persona por la que está orando.
\vs p091 5:6 \pc La oración es un medio por el que, más pronto o más tarde, cualquier religión se institucionaliza. Y, con el tiempo, la oración se vincula a numerosas mediaciones secundarias, algunas útiles, otras innegablemente perniciosas, tales como sacerdotes, libros sagrados, rituales de adoración y ceremoniales.
\vs p091 5:7 Pero las mentes de mayor iluminación espiritual deberían ser pacientes y tolerantes con los intelectos menos dotados que ansían simbolismo para movilizar su frágil percepción espiritual. El fuerte no debe mirar con desdén al débil. Aquellos que son conscientes de Dios sin necesidad de simbolismos no deben negarle el ministerio de gracia de los símbolos a aquellos que tienen dificultades para adorar a la Deidad y venerar la verdad, la belleza y la bondad sin formulismos ni ritos. En la adoración orante, la mayoría de los mortales conciben algún símbolo del objeto\hyp{}meta de sus devociones.
\usection{6. ÁMBITOS DE ACCIÓN DE LA ORACIÓN}
\vs p091 6:1 La oración, salvo que esté en conjunción con la voluntad y la actuación de las fuerzas personales espirituales y de los supervisores materiales del mundo, no puede tener un efecto directo sobre el entorno físico. Aunque hay un límite muy preciso en cuanto al ámbito de acción de las peticiones oracionales, dicho límite no es igualmente aplicable a la \bibemph{fe} de aquellos que oran.
\vs p091 6:2 La oración no es un medio para curar males reales y orgánicos, pero ha contribuido enormemente al disfrute de abundante salud y a la sanación de numerosas dolencias mentales, emocionales y nerviosas. Incluso en el caso de auténticas enfermedades bacterianas, la oración, en numerosas ocasiones, ha añadido eficacia a otros procedimientos curativos. La oración ha transformado a muchos enfermos irritables y quejumbrosos en ejemplos de paciencia y los ha convertido en una inspiración para todos los demás humanos víctimas del sufrimiento.
\vs p091 6:3 Sin importar lo difícil que pueda ser reconciliar las dudas científicas sobre la eficacia de la oración con el permanente impulso de buscar ayuda y guía de fuentes divinas, no olvidéis jamás que la oración sincera de fe es una fuerza poderosa que favorece la felicidad personal, el autocontrol individual, la armonía social, el progreso moral y los logros espirituales.
\vs p091 6:4 La oración, incluso como práctica puramente humana, como diálogo con el propio álter ego, constituye el modo más eficiente de acercarse a la consecución de aquellos poderes de reserva de la naturaleza humana que están almacenados y conservados en los ámbitos inconscientes de la mente del hombre. La oración es una práctica psicológica sana, aparte de sus implicaciones religiosas y de su significación espiritual. Es un hecho de la existencia humana que la mayoría de las personas, cuando están sometidas a presiones muy intensas, oren de alguna manera para que se les provea ayuda.
\vs p091 6:5 \pc No seas tan indolente como para pedirle a Dios que resuelva tus dificultades, pero jamás vaciles en pedirle sabiduría y fuerza espiritual para que te guíe y sustente mientras abordas por ti mismo, con resolución y valor, los problemas que te acucian.
\vs p091 6:6 \pc La oración ha sido un factor imprescindible en el progreso y la preservación de la civilización religiosa, y todavía puede contribuir abundantemente a un mayor mejoramiento y espiritualización de la sociedad si aquellos que oran solo lo hacen a la luz de los hechos científicos, de la sabiduría filosófica, de la sinceridad intelectual y de la fe espiritual. Orad tal como Jesús enseñó a sus discípulos: honestamente, de forma desinteresada, con equidad y sin dudar.
\vs p091 6:7 Pero la eficacia de la oración en la experiencia espiritual personal de quien ora no depende en modo alguno de la comprensión intelectual del devoto religioso, de su percepción filosófica, de su nivel social, de su estatus cultural ni de otros conocimientos humanos. Los atributos psicológicos y espirituales de la oración de fe son inmediatos, personales y vivenciales. No existe ningún otro medio por el que cualquier hombre, con independencia de todos sus otros logros humanos, pueda de forma tan eficaz y directa acercarse al umbral de ese ámbito en el que puede comunicarse con su Hacedor, en el que la criatura se pone en contacto con la realidad del Creador, con el modelador del pensamiento que habita en su interior.
\usection{7. MISTICISMO, ÉXTASIS E INSPIRACIÓN}
\vs p091 7:1 El misticismo, como medio de cultivar la conciencia de la presencia de Dios, es completamente encomiable, pero cuando su práctica conduce al aislamiento social y desemboca en el fanatismo religioso, es reprobable. Muy a menudo, lo que el místico exacerbado considera inspiración divina no es sino una convulsión de su propia mente profunda. El contacto de la mente mortal con su modelador interior, aunque a menudo se vea favorecido por una fervorosa meditación, se facilita, con mayor frecuencia, con el servicio incondicional y amoroso que parte del ministerio desinteresado hacia los demás.
\vs p091 7:2 Los grandes maestros religiosos y los profetas de eras pasadas no eran místicos extremos. Eran hombres y mujeres que conocían a Dios y cuya mejor manera de servirle era mediante el ministerio desinteresado a sus semejantes mortales. Con frecuencia, Jesús se llevaba a los apóstoles a solas durante breves periodos de tiempo para dedicarse a la meditación y a la oración, pero, en gran parte, los mantenía en una relación de servicio con las multitudes. El alma del hombre precisa del ejercicio espiritual al igual que del alimento espiritual.
\vs p091 7:3 El éxtasis religioso es admisible cuando proviene del ejercicio del sano juicio, pero tales estados son más a menudo una consecuencia de factores puramente emocionales que de manifestaciones de carácter espiritual profundo. Las personas religiosas no deben considerar cualquier vívido presentimiento de índole psicológica o cualquier vivencia emocional intensa como revelación divina ni como comunicación espiritual. El éxtasis espiritual genuino se suele asociar a una gran calma exterior y a un control emocional casi perfecto. Si bien, la verdadera visión profética es un presentimiento suprapsicológico. Estas prácticas no son pseudoalucinaciones ni tampoco el éxtasis característico del trance.
\vs p091 7:4 La mente humana puede actuar en respuesta a la supuesta inspiración cuando es sensible a las convulsiones del subconsciente o al estímulo del supraconsciente. En cualquiera de los dos casos, la persona tiene la impresión de que tales expansiones de la capacidad de la conciencia son más o menos ajenas. El entusiasmo místico irrefrenable y el éxtasis religioso desenfrenado no son las credenciales de la inspiración, unas credenciales supuestamente divinas.
\vs p091 7:5 La prueba práctica de todas estas experiencias religiosas extrañas de misticismo, éxtasis e inspiración consiste en observar si estos fenómenos hacen que una persona:
\vs p091 7:6 \li{1.}Disfrute de una salud física mejor y más integral.
\vs p091 7:7 \li{2.}Actúe de manera más eficaz y práctica en su vida mental.
\vs p091 7:8 \li{3.}Socialice su experiencia religiosa con mayor plenitud y gozo.
\vs p091 7:9 \li{4.}Espiritualice más completamente su vida diaria, en tanto que desempeña fielmente los deberes habituales de la rutinaria existencia mortal.
\vs p091 7:10 \li{5.}Aumente su amor y aprecio de la verdad, la belleza y la bondad.
\vs p091 7:11 \li{6.}Preserve los valores sociales, morales, éticos y espirituales reconocidos en ese momento.
\vs p091 7:12 \li{7.}Incremente su percepción espiritual: su conciencia de Dios.
\vs p091 7:13 \pc Si bien, la oración no está realmente relacionada con estas singulares experiencias religiosas. Cuando la oración se vuelve excesivamente estética, cuando consiste casi exclusivamente en la contemplación hermosa y dichosa de la divinidad paradisíaca, pierde gran parte de su influencia socializadora y tiende hacia el misticismo y el aislamiento de sus devotos. Orar en privado entraña cierto peligro, que se corrige y evita haciéndolo en grupo, o realizando prácticas devocionales comunitarias.
\usection{8. LA ORACIÓN COMO EXPERIENCIA PERSONAL}
\vs p091 8:1 La oración contiene un elemento de verdadera espontaneidad, dado que el hombre primitivo se vio a sí mismo orando mucho tiempo antes de tener la noción clara de un Dios. El hombre primitivo solía orar en dos situaciones diferentes: cuando estaba en necesidad imperiosa experimentaba el impulso de buscar ayuda; y cuando estaba exultante, se permitía expresar su alegría de manera impulsiva.
\vs p091 8:2 \pc La oración no evolucionó de la magia; cada cual surgió de forma independiente. La magia era un intento de adaptar la Deidad a las circunstancias; la oración es el esfuerzo de adaptar la persona a la voluntad de la Deidad. La verdadera oración es a la vez moral y religiosa; la magia no es ninguna de las dos.
\vs p091 8:3 \pc La oración puede convertirse en una costumbre establecida; muchas personas oran porque otras lo hacen. Incluso hay quienes oran porque tienen miedo de que les pueda ocurrir algo horrible si no llevan a cabo sus peticiones de forma regular.
\vs p091 8:4 Para algunas personas, la oración es una expresión serena de gratitud; para otros, una manifestación colectiva de alabanza, de devociones sociales; algunas veces, es la imitación de una religión ajena, mientras que la auténtica oración es la comunicación sincera y confiada de la naturaleza espiritual de la criatura con el omnipresente espíritu del Creador.
\vs p091 8:5 La oración puede ser una expresión espontánea de la conciencia de Dios o un vano recitado de fórmulas teológicas. Puede ser la alabanza extática del alma conocedora de Dios o la sumisa reverencia de algún atemorizado mortal. A veces, es una manifestación lastimosa de ansias espirituales y, en ocasiones, el grito flagrante de expresiones piadosas. La oración puede ser una alabanza gozosa o una humilde súplica de perdón.
\vs p091 8:6 La oración puede ser una petición infantil por lo imposible o la madura súplica por crecimiento moral y fuerza espiritual. La petición puede ser por el pan de cada día o puede consistir en el anhelo incondicional de encontrar a Dios y hacer su voluntad. Puede tratarse de un ruego totalmente interesado o un gesto real y magnífico por el logro de una fraternidad solidaria.
\vs p091 8:7 La oración puede ser un grito furioso de venganza o una intercesión misericordiosa por el enemigo. Puede ser la expresión del deseo de cambiar a Dios o un medio poderoso de cambiarse a uno mismo. Puede ser la súplica sumisa de un pecador perdido ante un Juez supuestamente severo o la expresión jubilosa de un hijo del Padre celestial vivo y misericordioso libre de sometimientos.
\vs p091 8:8 \pc Al hombre moderno le desconcierta la idea de hablar de sus cosas con Dios de manera puramente personal. Muchos han cesado de orar de forma regular; solo lo hacen cuando están bajo alguna presión excepcional ---en situaciones de urgencia---. El hombre no debe tener miedo a hablar con Dios, pero solo una persona con un espíritu infantil puede intentar persuadir, o pretender cambiar, a Dios.
\vs p091 8:9 \pc Pero la verdadera oración ciertamente logra alcanzar la realidad. Incluso cuando las corrientes de aire son ascendentes, ningún pájaro puede volar salvo que extienda sus alas. La oración eleva al hombre porque es un medio de avanzar mediante la utilización de las corrientes espirituales ascendentes del universo.
\vs p091 8:10 La oración genuina contribuye al crecimiento espiritual, modifica las actitudes y produce esa satisfacción que proviene de la comunión con la divinidad. Es un brote espontáneo de la conciencia de Dios.
\vs p091 8:11 Dios responde a la oración del hombre otorgándole una mayor revelación de la verdad, un mejor aprecio de la belleza y un más amplio concepto de la bondad. La oración es un gesto subjetivo, pero pone en contacto con poderosas realidades objetivas en los niveles espirituales de la experiencia humana; es un significativo acto humano para alcanzar valores sobrehumanos. Es el más potente estímulo para el crecimiento espiritual.
\vs p091 8:12 En la oración, las palabras son irrelevantes; constituyen meramente un canal intelectual por el que puede que fluya el río de la súplica espiritual. Las palabras tienen un valor meramente autosugestivo cuando se ora en privado y sociosugestivo cuando se hace en grupo. Dios responde a la actitud del alma, no a las palabras.
\vs p091 8:13 La oración no es un medio de huir de los conflictos sino más bien un estímulo para crecer ante el conflicto mismo. Orad solo por valores, no por cosas; por crecimiento, no para ser gratificados.
\usection{9. CONDICIONES PARA LA EFICACIA DE LA ORACIÓN}
\vs p091 9:1 Si quieres que tu oración sea efectiva debes tener en cuenta las leyes que imperan con respecto a las peticiones:
\vs p091 9:2 \li{1.}Debes prepararte para orar con eficacia sabiéndote enfrentar con sinceridad y valor a los problemas de la realidad del universo. Debes poseer vigor cósmico.
\vs p091 9:3 \li{2.}Debes haber agotado con franqueza tu capacidad humana de compromiso. Debes haber sido diligente.
\vs p091 9:4 \li{3.}Debes entregar todo deseo de mente y todo anhelo del alma al abrazo transformador del crecimiento espiritual. Debes haber experimentado el realce de los contenidos y la elevación de los valores.
\vs p091 9:5 \li{4.}Debes elegir hacer la voluntad divina sin reservas. Debes evitar caer en el punto muerto de la indecisión.
\vs p091 9:6 \li{5.}No solo reconoces la voluntad del Padre y eliges cumplirla, sino que te has consagrado incondicionalmente, y dedicado con fuerzas, a hacer verdaderamente la voluntad del Padre.
\vs p091 9:7 \li{6.}La sabiduría divina dirigirá con exclusividad tu oración para poder solucionar los particulares problemas humanos con los que te encuentres en tu ascensión al Paraíso, la realización de la perfección divina.
\vs p091 9:8 \li{7.}Y debes tener fe, fe viva.
\vsetoff
\vs p091 9:9 [Exposición del jefe de los seres intermedios de Urantia.]
