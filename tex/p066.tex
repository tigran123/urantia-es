\upaper{66}{El príncipe planetario de Urantia}
\author{Melquisedec}
\vs p066 0:1 La llegada de un hijo lanonandec a un mundo ordinario indica que la voluntad, la capacidad de elegir el camino de la supervivencia eterna, se ha desarrollado en la mente del hombre primitivo. Si bien, el príncipe planetario llegó a Urantia casi medio millón de años tras la aparición de la voluntad humana.
\vs p066 0:2 Caligastia, el príncipe planetario, llegó a Urantia hace unos quinientos mil años en simultaneidad con la aparición de las seis razas de color o razas sangik. A su llegada había en la tierra casi quinientos millones de seres humanos primitivos, bien dispersos por Europa, Asia y África. La sede del príncipe, que se estableció en Mesopotamia, estaba situada aproximadamente en el centro de la población mundial.
\usection{1. EL PRÍNCIPE CALIGASTIA}
\vs p066 1:1 Caligastia era un hijo lanonandec, el número 9344 del orden secundario. Contaba con experiencia en la administración de los asuntos del universo local en general y, durante eras posteriores, la adquirió en la gestión del sistema local de Satania en particular.
\vs p066 1:2 Con anterioridad al reinado de Lucifer en Satania, Caligastia había estado adscrito al consejo de los asesores de los portadores de vida de Jerusem. Lucifer lo ascendió a un puesto como miembro de sus asistentes personales, y desempeñó razonablemente bien cinco misiones sucesivas que requerían de él honestidad y ser merecedor de confianza.
\vs p066 1:3 \pc Muy pronto, Caligastia intentó conseguir el cargo de príncipe planetario; pero, repetidas veces, cada vez que su solicitud se sometía a la aprobación de los consejos de la constelación, no lograba la conformidad de los padres de la constelación. Caligastia parecía particularmente deseoso de ser enviado en calidad de gobernante planetario a un mundo decimal o de modificación de la vida. Antes de que finalmente consiguiera ser destinado a Urantia, su petición había sido revocada varias veces.
\vs p066 1:4 Caligastia salió de Jerusem en dirección al mundo, cuyo gobierno se le había confiado, con un envidiable expediente de lealtad y dedicación al bienestar del universo del que era residente y originario, a pesar de una cierta natural inquietud, sumada a una tendencia a estar en desacuerdo con el orden establecido en ciertas cuestiones menores.
\vs p066 1:5 Me encontraba en Jerusem cuando el brillante Caligastia partió de la capital del sistema. Ningún príncipe planetario había emprendido jamás una andadura para ostentar el gobierno de un mundo con tan amplia experiencia previa ni con mejores perspectivas que las que poseía Caligastia en aquel memorable día hace medio millón de años. Una cosa es cierta: al llevar a cabo mi tarea de difundir aquel acontecimiento en las transmisiones del universo local, jamás albergué ni por un solo momento la más mínima idea de que este noble lanonandec pudiese traicionar, en tan poco tiempo, su sagrado deber como custodio planetario y manchar, de forma tan terrible, el buen nombre de su elevado orden de filiación del universo. Realmente llegué a creer que Urantia se contaba entre los cinco o seis planetas más afortunados de toda Satania al considerar que una mente tan original, brillante y experimentada asumiría el mando de los asuntos mundiales. No me percaté entonces de que Caligastia, de forma insidiosa, se estaba enamorando de sí mismo; no comprendí muy bien en aquel momento las sutilezas del orgullo personal.
\usection{2. LA COMITIVA DEL PRÍNCIPE}
\vs p066 2:1 No se envió al príncipe planetario de Urantia a su misión solo, sino que lo acompañó el acostumbrado colectivo de asistentes y de ayudantes de su labor de administración.
\vs p066 2:2 Daligastia, asistente adjunto al príncipe planetario, estaba al frente de este grupo. Daligastia era igualmente un hijo lanonandec secundario, el número 319\,407 de dicho orden. Tenía el rango de asistente en el momento de ser asignado como adjunto a Caligastia.
\vs p066 2:3 La comitiva planetaria estaba integrada por una gran cantidad de cooperadores angélicos y por una multitud de otros seres celestiales cuya misión era hacer progresar los intereses y promocionar el bienestar de las razas humanas. Pero, desde vuestro punto de vista, el grupo más interesante de todos era el de los miembros corpóreos de la comitiva del príncipe, aludidos a veces como \bibemph{los cien de Caligastia}.
\vs p066 2:4 \pc Caligastia eligió a estos cien miembros rematerializados entre más de 785\,000 ciudadanos ascendentes de Jerusem, que se ofrecieron como voluntarios para emprender la aventura urantiana. Cada uno de los cien elegidos procedía de un planeta diferente; ninguno de ellos era de Urantia.
\vs p066 2:5 Estos voluntarios jerusemitas se trajeron directamente desde la capital del sistema hasta Urantia mediante transporte seráfico y, a su llegada, se mantuvieron envueltos en el serafín hasta que se les pudo proporcionar formas personales de doble naturaleza, acorde con el servicio planetario especial que iban a realizar; se trataba de verdaderos cuerpos de carne y hueso que estaban también sintonizados con las vías circulatorias vitales del sistema.
\vs p066 2:6 \pc Algo antes de la llegada de estos cien ciudadanos de Jerusem, los dos portadores de vida supervisores, residentes en Urantia, habiendo previamente perfeccionado sus planes, solicitaron permiso a Jerusem y Edentia para trasplantar el plasma vital de cien supervivientes seleccionados del linaje de Andón y Fonta a los cuerpos materiales previstos para los miembros corpóreos del príncipe. Esta solicitud se concedió en Jerusem con la aprobación de Edentia.
\vs p066 2:7 Por consiguiente, los portadores de vida escogieron a cincuenta hombres y a cincuenta mujeres de las sucesivas generaciones de Andón y Fonta, en representación de los supervivientes de las mejores estirpes de aquella excepcional raza. Salvo por una o dos excepciones, estos andonitas, contribuidores al progreso de la raza, se desconocían entre sí. Gracias a la dirección de los modeladores del pensamiento en coordinación con la guía seráfica, se les había reunido, desde lugares muy separados, en el umbral de la sede planetaria del príncipe. Aquí se pusieron a los cien sujetos humanos en manos de una comisión de voluntarios, altamente cualificada, procedente de Avalón, que dirigió la extracción material de una porción del plasma vital de estos descendientes de Andón. Este material vivo se transfirió, a su vez, a los cuerpos materiales creados para utilización de los cien miembros jerusemitas de la comitiva del príncipe. Entretanto, estos ciudadanos recién llegados de la capital del sistema se mantenían dormidos en el transporte seráfico.
\vs p066 2:8 \pc Estos hechos, junto con la creación específica de los cuerpos especiales para los cien de Caligastia, dieron origen a numerosas leyendas; gran parte de las cuales se confundieron más tarde con las tradiciones acerca del establecimiento de Adán y de Eva en el planeta.
\vs p066 2:9 Todo el proceso de reconstitución personal duró exactamente diez días, contando desde la llegada de los transportes seráficos, que portaban a los cien voluntarios procedente de Jerusem, hasta que recuperaron la conciencia como seres triples del mundo.
\usection{3. DALAMATIA: LA CIUDAD DEL PRÍNCIPE}
\vs p066 3:1 La sede del príncipe planetario se encontraba en la región del Golfo Pérsico de aquellos días, en la zona correspondiente a la futura Mesopotamia.
\vs p066 3:2 El clima y el paisaje de la Mesopotamia de aquellos tiempos eran, en todos los sentidos, favorables para las iniciativas de la comitiva del príncipe y de sus asistentes, muy diferentes de las condiciones que a veces han imperado desde entonces. Era necesario disponer de un clima favorable como parte del entorno natural ideado para incitar a los primitivos urantianos a realizar ciertos progresos iniciales en cultura y civilización. La gran tarea de aquellas eras consistía en transformar al hombre de cazador a pastor, con la esperanza de que, más tarde, se convirtiese en agricultor amante de la paz y con sentido de hogar.
\vs p066 3:3 \pc La sede del príncipe planetario en Urantia era típica de las estaciones situadas en esferas jóvenes y en desarrollo. El núcleo del asentamiento del príncipe era una ciudad muy sencilla, pero bella, rodeada por una muralla de doce metros de alto. A este centro mundial de la cultura se le llamó Dalamatia en honor a Daligastia.
\vs p066 3:4 La ciudad constaba de diez subdivisiones, con las imponentes residencias sedes de los diez consejos de la comitiva corpórea situados en el centro de cada una de ellas. En la parte central de la ciudad, estaba el templo del Padre invisible. La sede administrativa del príncipe y de sus colaboradores estaba dispuesta en doce cámaras que se agrupaban lindantes entre sí en torno al templo mismo.
\vs p066 3:5 Todos los edificios de Dalamatia eran de una sola planta, salvo la sede del consejo, que tenía dos, y el templo central del Padre de todos, que, aunque pequeño, tenía tres plantas de altura.
\vs p066 3:6 La ciudad estaba construida de ladrillos, el mejor material de construcción de aquellos primeros días. Se hizo uso de muy poca piedra o madera. Siguiendo el ejemplo de Dalamatia, la construcción de viviendas y la arquitectura de los poblados circundantes mejoraron de forma considerable.
\vs p066 3:7 \pc Cerca de la sede del príncipe habitaban seres humanos de todos los colores y estratos. De estas tribus cercanas se reclutaron los primeros estudiantes de las escuelas del príncipe. Y aunque estas tempranas escuelas de Dalamatia eran toscas, proporcionaban todo lo que pudiera ser de utilidad para los hombres y mujeres de aquella era primitiva.
\vs p066 3:8 La comitiva corpórea del príncipe se rodeaba continuamente de los individuos mejor dotados de las tribus circundantes y, tras haber formado y alentado a estos estudiantes, se les enviaba de vuelta a estas como maestros y líderes de sus respectivos pueblos.
\usection{4. LOS PRIMEROS DÍAS DE LOS CIEN}
\vs p066 4:1 La llegada de la comitiva del príncipe causó una honda impresión. Aunque se precisaron casi mil años para que esta noticia se difundiese ampliamente, las enseñanzas y la conducta de los cien nuevos moradores de Urantia tuvieron una enorme influencia en las tribus próximas a la sede mesopotámica. Y una gran parte de vuestra mitología surgió posteriormente de las confusas leyendas sobre estos tempranos días en los que a los miembros de la comitiva del príncipe se les volvió a dar su forma personal en Urantia como suprahombres.
\vs p066 4:2 La tendencia de los mortales a considerar a estos maestros extraplanetarios como dioses obstaculizaba seriamente la influencia positiva que estos pudieran ejercer; si bien, al margen del procedimiento usado para su aparición en la tierra, los cien de Caligastia ---cincuenta hombres y cincuenta mujeres--- no recurrieron a métodos sobrenaturales ni a actuaciones de carácter sobrehumano.
\vs p066 4:3 Pero la comitiva corpórea era, no obstante, sobrehumana. Empezaron su misión en Urantia como seres extraordinarios de naturaleza triple:
\vs p066 4:4 \li{1.}Eran materiales y relativamente humanos, pues eran depositarios del genuino plasma vital de una de las razas humanas: el plasma vital andónico de Urantia.
\vs p066 4:5 Estos cien miembros de la comitiva del príncipe estaban divididos, por igual, según el sexo y de acuerdo con su estatus anterior como mortales. Cada una de las personas que integraban este grupo era capaz de convertirse en coprogenitora de algún orden nuevo de ser físico, pero se les había instruido específicamente que no recurrieran a la paternidad salvo en determinadas circunstancias. Es habitual que la comitiva corpórea del príncipe planetario dé nacimiento a sus sucesores en algún momento antes de la retirada de su particular servicio planetario. Esto sucede normalmente a la llegada del adán y eva planetarios o poco después.
\vs p066 4:6 Estos seres especiales tenían, por tanto, poca o ninguna idea del tipo de criatura material que se engendraría como resultado de su unión sexual. Y, en realidad, nunca lo sabrían; antes de que se diera este paso como continuidad de su labor en el mundo, todo el sistema social se alteró debido a la rebelión, y aquellos que más adelante tuvieron un papel parental habían sido aislados de las corrientes vitales del sistema.
\vs p066 4:7 En cuanto al color de la piel y al idioma, estos miembros materializados de la comitiva de Caligastia seguían a la raza andónica. Tal como lo hacían los mortales del mundo, ingerían alimentos, aunque con la diferencia de que los cuerpos re\hyp{}creados de este grupo se satisfacían plenamente con una dieta alimenticia sin carne. Este fue uno de los motivos que determinaron que fijasen su residencia en una región cálida con abundancia de frutas y frutos de cáscara. La práctica de subsistir a base de una dieta sin carne data de los tiempos de los cien de Caligastia. Esta costumbre se difundió por todas partes, llegando a influir en los hábitos alimenticios de muchas tribus circundantes o grupos provenientes de las razas evolutivas que, en otro tiempo, habían sido exclusivamente consumidoras de carne.
\vs p066 4:8 \li{2.}Los cien se reconstituyeron en Urantia como hombres y mujeres únicos de un orden elevado y especial. Eran seres materiales pero sobrehumanos.
\vs p066 4:9 Aunque este grupo gozaba de una ciudadanía de carácter provisional en Jerusem, aún permanecían sin fusionarse con sus modeladores del pensamiento; y, cuando se ofrecieron como voluntarios y se les aceptó para el servicio planetario en conexión con los órdenes descendentes de filiación, sus modeladores se separaron de ellos. Pero estos jerusemitas eran seres efectivamente sobrehumanos ---poseían almas en crecimiento ascendente---. Durante la vida mortal en la carne, el alma tiene un estado embrionario; nace (resucita) en la vida morontial y experimenta un crecimiento a su paso por los sucesivos mundos morontiales. Las almas de los cien de Caligastia se desarrollaron de esta manera, al adquirir experiencias de carácter progresivo desde los siete mundos de las estancias hasta alcanzar la ciudadanía en Jerusem.
\vs p066 4:10 Conforme a las instrucciones recibidas, la comitiva no participó en reproducción sexual alguna, aunque sí estudió con minuciosidad su constitución personal e investigó detenidamente todos los aspectos que cabían imaginar de la unión del intelecto (la mente) y de la morontia (el alma). Y durante el trigésimo tercer año de su estancia en Dalamatia, mucho antes de que la muralla se terminase, los números dos y siete del grupo danita descubrieron de forma casual un fenómeno asociado a la unión de sus yos morontiales (supuestamente ni sexuales ni materiales); el resultado de este inesperado suceso fue la primera de las criaturas intermedias primarias. Este ser nuevo era enteramente visible para la comitiva planetaria y sus colaboradores celestiales, aunque no para los hombres y mujeres de las distintas tribus humanas. Con autorización del príncipe planetario, toda la comitiva consiguió dar origen a seres similares, siguiendo las instrucciones de la pareja pionera danita. De este modo, la comitiva del príncipe acabó por dar existencia al colectivo original de 50\,000 seres intermedios primarios.
\vs p066 4:11 Estas criaturas de tipo intermedio efectuaban un excelente servicio al implicarse en los asuntos de la sede mundial. Eran invisibles a los seres humanos, pero a los moradores primitivos de Dalamatia se les informó de estos semiespíritus invisibles y, durante eras, estos constituyeron la totalidad del mundo espiritual para estos mortales en evolución.
\vs p066 4:12 \li{3.}Los cien de Caligastia eran personalmente inmortales ---no perecían---. Por sus formas materiales circulaban los aportes contrarrestantes de las corrientes de vida del sistema; y, si no hubiesen perdido el contacto con las vías circulatorias vitales debido a la rebelión, habrían seguido viviendo por tiempo indefinido hasta la llegada del próximo hijo de Dios o hasta la posterior liberación de sus cometidos para reanudar su interrumpido trayecto a Havona y al Paraíso.
\vs p066 4:13 Estos aportes de efecto antídoto de las corrientes de vida de Satania se derivaban del fruto del árbol de la vida, un arbusto de Edentia que los altísimos de Norlatiadek enviaron a Urantia a la llegada de Caligastia. En los días de Dalamatia, este árbol crecía en el patio central del templo del Padre invisible, y su fruto permitía que los seres materiales de la comitiva del príncipe, por otra parte mortales, pudieran vivir de forma indefinida siempre que tuviesen acceso a él.
\vs p066 4:14 Aunque carecía de eficacia para las razas evolutivas, este extra sustento era más que suficiente para dotar de vida sin fin a los cien de Caligastia e igualmente a los cien andonitas modificados que estaban en colaboración con ellos.
\vs p066 4:15 \pc En relación a esto, es necesario explicar que, en el momento en que los cien andonitas proporcionaron su plasma germinal humano a los miembros de la comitiva del príncipe, los portadores de vida introdujeron en sus cuerpos mortales el aporte de las vías circulatorias del sistema; y esto les permitió continuar viviendo conjuntamente con la comitiva, siglo tras siglo, desafiando la muerte física.
\vs p066 4:16 Con el tiempo, los cien andonitas tuvieron conocimiento de su contribución a la nueva anatomía de sus superiores, y estos mismos cien hijos de las tribus de Andón se mantuvieron en la sede como ayudantes personales de la comitiva corpórea del príncipe.
\usection{5. LA ORGANIZACIÓN DE LOS CIEN}
\vs p066 5:1 Los cien estaban organizados para el servicio en diez consejos autónomos de diez miembros cada uno. Cuando dos o más de dos de estos diez consejos se reunían en sesión conjunta, era Daligastia quien presidía tales reuniones de enlace. Estos diez grupos estaban constituidos de la siguiente manera:
\vs p066 5:2 \li{1.}\bibemph{El consejo de alimentación y de bienestar material}. Ang presidía este grupo. Este capaz colectivo impulsaba las cuestiones relacionadas con los alimentos, el agua, la ropa y el progreso material de la especie humana. Instruían en la excavación de pozos, el control de los manantiales y el riego. A aquellos procedentes de las cotas más altas y del norte les enseñaban mejores métodos de tratar las pieles para su uso como prendas de vestir; más adelante, los maestros de las artes y las ciencias introducirían la tejeduría.
\vs p066 5:3 Se realizaron grandes progresos en métodos de almacenaje de alimentos. La comida se conservaba mediante la cocción, el secado y el ahumado, convirtiéndose así en la primera forma de propiedad. Al hombre se le enseñó a prevenir el riesgo de la hambruna, que periódicamente diezmaba al mundo.
\vs p066 5:4 \li{2.}\bibemph{La junta de domesticación y utilización de los animales}. Este consejo se dedicaba a la tarea de seleccionar y criar aquellos animales mejor adaptados para ayudar a los seres humanos a llevar cargas y a transportarlos, a proporcionar alimento y, más adelante, a servir en el cultivo de la tierra. Bon dirigía este capaz colectivo.
\vs p066 5:5 Se domesticaron diversos tipos de animales provechosos, ahora extintos, junto a otros que han continuado como animales domésticos hasta los tiempos presentes. El hombre llevaba mucho tiempo conviviendo con el perro, y el hombre azul ya había conseguido domar al elefante. La vaca había mejorado tanto gracias a su esmerada cría que se convirtió en una fuente valiosa de alimento; la mantequilla y el queso se volvieron artículos comunes de la dieta alimenticia humana. Se enseñó a los hombres a emplear a los bueyes para llevar sus cargas; sin embargo, el caballo no se domesticó hasta algún tiempo después. Los miembros de este colectivo fueron los primeros que instruyeron a los hombres a servirse de la rueda para facilitar la tracción.
\vs p066 5:6 En estos días, se usaron por vez primera las palomas mensajeras; se llevaban en los viajes largos con el fin de enviar mensajes o llamadas de auxilio. El grupo de Bon consiguió amaestrar a los grandes fándores como aves de transporte, pero estos se extinguieron hace más de treinta mil años.
\vs p066 5:7 \li{3.}\bibemph{Los asesores de la conquista de animales depredadores}. No bastaba con que el hombre primitivo intentase domesticar a determinados animales, también tuvo que aprender a protegerse de la destrucción que podían causar los restantes animales hostiles del mundo animal. Dan capitaneaba este grupo.
\vs p066 5:8 Las murallas de las ciudades antiguas tenían el propósito de ofrecer protección contra las fieras feroces al igual que contra los ataques por sorpresa de humanos hostiles. Sin protección de una muralla, los que vivían en el bosque dependían de habitáculos construidos en los árboles, de refugios de piedra y del mantenimiento de hogueras durante la noche. Por tanto, era muy natural que estos maestros dedicaran mucho tiempo a instruir a sus alumnos en la mejora de las viviendas humanas. Se hicieron grandes progresos en el sometimiento de los animales mediante mejores técnicas y el uso de trampas.
\vs p066 5:9 \li{4.}\bibemph{El profesorado a cargo de la difusión y conservación del conocimiento}. Este grupo organizaba y dirigía las actividades puramente educativas de aquellas tempranas eras. Era Fad quien lo presidía. Los métodos didácticos de Fad consistían en supervisar el trabajo, al mismo tiempo que daba instrucciones respecto a la mejora de los métodos empleados. Fad desarrolló el primer alfabeto e introdujo un sistema de escritura. Este alfabeto tenía veinticinco caracteres. Como material de escritura, estos pueblos primitivos utilizaban cortezas de árboles, tablillas de arcilla, placas de piedra, una clase de pergamino hecho de cuero martillado y un tosco tipo material semejante al papel, confeccionado de nidos de avispas. La biblioteca de Dalamatia, destruida al poco tiempo de la deslealtad de Caligastia, contenía más de dos millones de archivos escritos individuales y se le conocía como “la casa de Fad”.
\vs p066 5:10 El hombre azul era partidario del alfabeto escrito y, en este sentido, consiguió los mayores avances. El hombre rojo prefería la escritura pictográfica, mientras que las razas amarillas se inclinaron por el uso de símbolos para las palabras y las ideas, de forma muy parecida a los que actualmente emplean. Pero el alfabeto y muchas otras cosas se perdieron en el mundo durante la confusión que trajo consigo la rebelión. La deserción de Caligastia destruyó la esperanza de que el mundo dispusiera de una lengua universal, al menos durante incalculables eras.
\vs p066 5:11 \li{5.}\bibemph{La comisión de manufactura y comercio}. Este consejo, que estaba liderado por Nod, se ocupaba de fomentar la manufactura dentro de las tribus y de promover el intercambio comercial entre los diferentes grupos pacíficos. El colectivo alentó toda forma de fabricación primitiva y contribuyó directamente a elevar el nivel de vida de los hombres primitivos, proporcionándoles muchos productos nuevos para atraer su atención. También amplió considerablemente el comercio de una sal mejorada producida por el consejo de las ciencias y las artes.
\vs p066 5:12 El crédito comercial se practicó por vez primera entre estos instruidos grupos educados en las escuelas de Dalamatia. Obtenían fichas de un centro de intercambio de crédito que se aceptaban en lugar de los mismos objetos de trueque. Durante cientos de miles de años, el mundo no llegó a superar estos métodos comerciales.
\vs p066 5:13 \pc \bibemph{6. El colegio de religión revelada}. Este cuerpo desempeñó su actividad con lentitud. La civilización en Urantia se forjó literalmente entre el yunque de la necesidad y el martillo del temor. Pero, antes de que su labor se viese interrumpida por la confusión resultante de la revuelta secesionista, este grupo había realizado notables avances en sus intentos por sustituir el temor a las criaturas (el culto a los espectros) por el temor al Creador. Hap encabezaba este consejo.
\vs p066 5:14 Ningún miembro de la comitiva del príncipe deseaba presentar una revelación que dificultara la evolución; la revelación se expuso solamente como culminación una vez que habían agotado las fuerzas de la evolución. Pero Hap sí cedió al deseo de los habitantes de la ciudad de establecer alguna forma de servicio religioso. Su grupo proporcionó a los dalamatianos los siete cánticos del culto de adoración y también les dio una expresión diaria de alabanza; finalmente, les enseñó “la oración del Padre”, que decía:
\vs p066 5:15 \pc “Padre de todos, a cuyo hijo honramos, míranos con favor. Libéranos de todo temor, salvo del temor de ti. Haz que contentemos a nuestros maestros divinos y pon por siempre la verdad en nuestros labios. Líbranos de la violencia y de la ira; concédenos respeto por nuestros ancianos y por lo que pertenece a nuestros semejantes. Danos en esta estación pastos verdes y abundante rebaño que alegren nuestros corazones. Oramos para que se apresure la venida prometida de aquel que nos exaltará. Hágase tu voluntad en este mundo tal como otros la hacen en los mundos del más allá”.
\vs p066 5:16 \pc Aunque la comitiva del príncipe estaba limitada al uso de medios naturales y métodos ordinarios para mejorar las razas, hicieron la promesa del don adánico que traería una raza nueva, la meta del crecimiento evolutivo, que seguiría una vez se alcanzase la cúspide del desarrollo biológico.
\vs p066 5:17 \li{7.}\bibemph{Los guardianes de la salud y la vida}. Este consejo se encargaba de introducirles en la sanidad y de promover una higiene primitiva. Fue Lut quien los lideraba.
\vs p066 5:18 Una gran parte de las enseñanzas impartidas por los miembros de este consejo se perdieron durante la confusión de las eras que siguieron, y no se llegarían a redescubrir hasta el siglo XX. Enseñaron a la humanidad que cocinar ---hervir y asar los alimentos--- era un modo de evitar las enfermedades; también, que tal forma de cocinar reducía bastante la mortalidad infantil y facilitaba un destete temprano.
\vs p066 5:19 Muchas de las primeras enseñanzas sobre la salud dadas por los guardianes de Lut persistieron entre las tribus de la tierra hasta los días de Moisés, aunque bastante distorsionadas y cambiadas.
\vs p066 5:20 El gran obstáculo con el que se encontraron en sus medidas para promover la higiene entre estos pueblos ignorantes consistía en el hecho de que las verdaderas causas de muchas enfermedades eran demasiado pequeñas para ser vistas a simple vista, al igual que en el supersticioso respeto que tenían por el fuego. Se precisaron miles de años para persuadirlos de que quemaran sus desechos. Mientras tanto, se les instó a que enterraran la basura en descomposición. El gran adelanto sanitario de esta época vino de la difusión del conocimiento sobre las propiedades saludables y sanadoras de la luz del sol.
\vs p066 5:21 Antes de la llegada del príncipe, el baño había sido una ceremonia exclusivamente religiosa. De hecho, resultaba difícil convencer a los hombres primitivos para que lavasen sus cuerpos como práctica sanitaria. Finalmente, Lut persuadió a los maestros religiosos para que incluyeran abluciones en las ceremonias de purificación que habían de practicarse al mediodía, una vez por semana, durante las oraciones de adoración del Padre de todo.
\vs p066 5:22 Estos guardianes de la salud intentaron implantar también el apretón de manos para sellar la amistad personal y como muestra de lealtad al grupo, en sustitución de la práctica de intercambiar saliva o beber sangre. Pero cuando se encontraban sin la presión imperiosa de las enseñanzas de sus líderes superiores, estos pueblos primitivos no tardaban en volver a sus antiguas prácticas ignorantes y supersticiosas, que tan perniciosas resultaban para la salud y tanto favorecían la proliferación de enfermedades.
\vs p066 5:23 \li{8.}\bibemph{El consejo planetario de las artes y las ciencias.} Este colectivo, liderado por Mek, contribuyó bastante a mejorar los métodos de manufactura del hombre primitivo y a elevar su concepto de la belleza.
\vs p066 5:24 En todo el mundo, las artes y las ciencias se encontraban en un punto bajo; pero se enseñaron a los dalamatianos los rudimentos de la física y la química. La alfarería avanzó, todas las artes decorativas mejoraron y los ideales de la belleza humana se vieron considerablemente realzados. Pero hasta después de la llegada de la raza violeta, la música no hizo muchos progresos.
\vs p066 5:25 Estos hombres primitivos no consentían en experimentar con la energía del vapor, a pesar de los repetidos requerimientos de sus maestros; eran incapaces de vencer su gran temor a la potencia explosiva del vapor encerrado. Finalmente, sin embargo, se les persuadió para que trabajaran los metales y el fuego, aunque, para el hombre primitivo, un pedazo de metal al rojo vivo era un objeto aterrador.
\vs p066 5:26 Mek contribuyó sobremanera a elevar la cultura de los andonitas y a mejorar las artes del hombre azul. La mezcla de este con el linaje de Andón generó un tipo de hombre artísticamente dotado, y muchos de ellos se convirtieron en grandes escultores. No trabajaban ni la piedra ni el mármol, pero sus obras de arcilla, endurecidas al horno, adornaron los jardines de Dalamatia.
\vs p066 5:27 Se hicieron grandes progresos en las artes del hogar, aunque la mayoría de estas se perdieron durante las largas y oscuras épocas de la rebelión para nunca volverse a descubrir hasta los tiempos modernos.
\vs p066 5:28 \li{9.}\bibemph{Los gobernadores de las relaciones tribales avanzadas}. Este era un grupo cuyo cometido era elevar la sociedad humana al nivel de estado. Su jefe era Tut.
\vs p066 5:29 Estos líderes contribuyeron mucho a propiciar el matrimonio intertribal. Fomentaron la práctica del cortejo y, tras la reflexión debida y tener la oportunidad para conocerse, el casamiento. Las danzas guerreras de índole puramente militar fueron refinadas y puestas al servicio de fines sociales valiosos. Se impulsaron muchos juegos competitivos, pero estos pueblos de la antigüedad eran personas serias; el humor no era algo que caracterizara a estas tribus primitivas. Pocas de estas prácticas sobrevivieron a la fragmentación cultural que siguió a la insurrección planetaria.
\vs p066 5:30 Tut y sus colaboradores se esforzaron sumamente por promover asociaciones grupales de naturaleza pacífica, por regular y humanizar la guerra, por coordinar las relaciones intertribales y por mejorar los gobiernos tribales. En las proximidades de Dalamatia, se desarrolló una cultura de mayor avance, y estas relaciones sociales mejoradas resultaron muy eficaces por su influencia en tribus más lejanas. Si bien, el modelo de civilización que imperaba en la sede del príncipe era completamente diferente al de la sociedad rudimentaria que se desarrollaba en otras partes, del mismo modo que la sociedad del siglo XX de la Ciudad del Cabo, en Sudáfrica, es totalmente distinta a la cruda cultura de los diminutos bosquimanos del norte.
\vs p066 5:31 \li{10.}\bibemph{El tribunal supremo de coordinación tribal y cooperación racial}. Van dirigía este consejo supremo, que servía de tribunal de apelaciones para todas las otras nueve comisiones especiales a cargo de la supervisión de los asuntos humanos. Este consejo tenía amplias funciones pues se le habían confiado todas las cuestiones terrestres de interés no asignadas específicamente a los otros grupos. Este selecto colectivo había tenido la aprobación de los padres de la constelación de Edentia, antes de que se le autorizara para asumir las funciones de tribunal supremo de Urantia.
\usection{6. EL REINADO DEL PRÍNCIPE}
\vs p066 6:1 El nivel cultural de un mundo se mide por el patrimonio social de sus habitantes nativos, y el grado de expansión cultural depende exclusivamente de la capacidad que tengan para comprender ideas nuevas y avanzadas.
\vs p066 6:2 La sujeción a la tradición crea estabilidad y cooperación al enlazar sentimentalmente el pasado con el presente; pero igualmente reprime las iniciativas y esclaviza el poder creativo de la persona. Cuando llegaron los cien de Caligastia y comenzaron a proclamar esta nueva doctrina de la iniciativa individual dentro de los grupos sociales de esos días, todo el mundo estaba atrapado en su apego a la tradición. Pero estas provechosas normas se vieron interrumpidas con tanta celeridad que las razas nunca llegaron a liberarse del todo de la servidumbre a las costumbres; esta actitud aún impera excesivamente en Urantia.
\vs p066 6:3 Los cien de Caligastia ---graduados de los mundos de las moradas de Satania--- conocían bien las artes y la cultura de Jerusem, pero tales conocimientos carecían prácticamente de valor en un planeta salvaje, poblado por humanos primitivos. Estos seres sabios sabían que no se debía emprender la transformación \bibemph{repentina,} o elevación masiva, de las razas primitivas de aquellos días. Eran conscientes de la lenta evolución de la especie humana y, con sensatez, se abstuvieron de cualquier intento extremo por modificar el modo de vida del hombre en la tierra.
\vs p066 6:4 Cada una de las diez comisiones planetarias se implicó en el avance \bibemph{lento} y natural de los intereses que les competía. Su plan consistía en atraer a las mentes más brillantes de las tribus circundantes y, tras haberlos formado, se les enviaba de vuelta a sus respectivos pueblos como emisarios de la mejora social.
\vs p066 6:5 Nunca se enviaban emisarios extranjeros a raza alguna salvo por petición expresa del pueblo en cuestión. Los que desarrollaban la labor de mejora y progreso de alguna tribu o raza determinada eran siempre nativos de alguna de ellas. Los cien no intentaban imponer los hábitos y costumbres de una raza, aunque esta fuese superior, sobre ninguna tribu. Siempre obraban con paciencia para elevar y hacer progresar las costumbres de cada raza que habían resistido la prueba del tiempo. La gente sencilla de Urantia llevaba consigo sus costumbres sociales a Dalamatia, no para cambiarlas por prácticas nuevas y mejores, sino para mejorarlas mediante el contacto con una cultura superior y la relación con mentes superiores. El proceso fue lento pero efectivo.
\vs p066 6:6 Los maestros de Dalamatia pretendían añadir a la selección puramente natural de la evolución biológica una selección social consciente. No trastornaron la sociedad humana, pero sí aceleraron de forma notable su evolución normal y natural. Su propósito era el progreso mediante la evolución y no la revolución mediante la revelación. A la raza humana le había llevado muchas eras adquirir la poca religión y principios morales que tenía, y estos superhombres sabían que no se debía arrebatar estos limitados avances a la humanidad. Cuando seres superiores e instruidos emprenden la mejora de razas atrasadas mediante una sobrecarga de enseñanzas y de conocimiento, siempre se produce confusión y consternación.
\vs p066 6:7 Cuando los misioneros cristianos van al corazón de África, donde se supone que los hijos y las hijas deben permanecer bajo el control y la dirección de sus padres durante la vida de estos, solo provocan la confusión y la ruptura de toda autoridad cuando persiguen, en una sola generación, reemplazar esta práctica enseñando que los hijos han de liberarse de toda restricción paterna al cumplir los veintiún años.
\usection{7. LA VIDA EN DALAMATIA}
\vs p066 7:1 La sede del príncipe, aunque de belleza exquisita y diseñada para infundir en el hombre primitivo un sobrecogedor sentimiento de reverencia, era en general modesta. Los edificios no eran particularmente grandes, pues el motivo por el que se habían traído de fuera a estos maestros era impulsar con el tiempo el desarrollo de la agricultura mediante la introducción de la ganadería. Los terrenos que se habían dispuesto dentro de las murallas de la ciudad eran suficientes como para proporcionar, por medio del pastoreo y la horticultura, el sostenimiento de una población de unos veinte mil habitantes.
\vs p066 7:2 Los interiores del templo central de adoración y las diez imponentes residencias destinadas a los consejos de los grupos supervisores de los suprahombres eran realmente bellas obras de arte. Y aunque los edificios residenciales eran modelos de orden y limpieza, todo era muy sencillo y bastante básico en comparación con el desarrollo que se produciría con el tiempo. En esta sede de la cultura, no se empleó ningún método que no perteneciera a Urantia de forma natural.
\vs p066 7:3 La comitiva corpórea del príncipe disponía de moradas sencillas y modélicas, que mantenían como hogares diseñados para inspirar e impresionar favorablemente a los estudiantes observadores que residían temporalmente en el centro social y sede educativa del mundo.
\vs p066 7:4 \pc El bien definido sistema de vida familiar y de residencia de toda la familia junta en una vivienda situada en un lugar relativamente estable data de estos tiempos de Dalamatia y se debía, principalmente, al ejemplo y a las enseñanzas de los cien y de sus alumnos. El hogar como núcleo social no tuvo éxito hasta que los suprahombres y supramujeres de Dalamatia guiaron a la humanidad a amar a sus nietos y a los hijos de sus nietos y a hacer planes para ellos. El hombre salvaje ama a sus hijos, pero el hombre civilizado ama también a sus nietos.
\vs p066 7:5 Los miembros de la comitiva del príncipe vivían juntos obrando como padres y madres. Si bien es cierto que no tenían hijos propios, los cincuenta hogares modelo de Dalamatia nunca dieron albergue a menos de quinientos pequeños adoptados, que habían sido seleccionados de las familias mejor dotadas de las razas andónicas y sangik; muchos de estos niños eran huérfanos. Se beneficiaban de la disciplina y formación de estos superpadres; y, luego, tras tres años en las escuelas del príncipe (en las que ingresaban entre los trece y los quince años de edad), podían optar por el matrimonio y estaban preparados para ser enviados como emisarios del príncipe a las menesterosas tribus de sus razas respectivas.
\vs p066 7:6 \pc Fad auspició el plan docente de Dalamatia, que se llevó a cabo como escuela de manufactura en la que los alumnos aprendían de manera práctica efectuando diariamente tareas de utilidad. Este plan educativo no ignoraba la función de los pensamientos y las emociones en el desarrollo del carácter, pero priorizaba la formación de tipo manual. La instrucción era individual y colectiva. Tanto hombres como mujeres, por separado y conjuntamente, impartían clases a los alumnos. La mitad de esta instrucción de carácter grupal se impartía haciendo distinción de sexo; la otra mitad era mixta. Se enseñaba a los estudiantes destrezas manuales de modo individual y se les reunía para socializar en grupos o clases. Se les formaba para que fraternizaran con grupos más jóvenes y de mayor edad y con adultos, al igual que para trabajar en equipo con los de su misma edad. También se les familiarizó respecto a asociaciones tales como grupos familiares, equipos de juego y clases escolares.
\vs p066 7:7 Entre los últimos estudiantes formados en Mesopotamia para trabajar con sus respectivas razas se encontraban los andonitas de las altiplanicies de la India occidental junto con representantes de los hombres rojos y de los hombres azules; más adelante también se admitió a un reducido número de la raza amarilla.
\vs p066 7:8 \pc Hap presentó a las razas primitivas un código moral que se conocía con el nombre de “La Vía del Padre”, y que constaba de los siguientes siete mandamientos:
\vs p066 7:9 \li{1.}No temerás ni servirás a otro Dios salvo al Padre de todos.
\vs p066 7:10 \li{2.}No desobedecerás al hijo del Padre, soberano del mundo, ni mostrarás falta de respeto por sus colaboradores sobrehumanos.
\vs p066 7:11 \li{3.}No mentirás cuando comparezcas ante los jueces del pueblo.
\vs p066 7:12 \li{4.}No matarás a hombres, mujeres o niños.
\vs p066 7:13 \li{5.}No robarás los bienes ni el ganado de tu prójimo.
\vs p066 7:14 \li{6.}No tocarás a la esposa de tu amigo.
\vs p066 7:15 \li{7.}No mostrarás falta de respeto por tus padres ni por los ancianos de la tribu.
\vs p066 7:16 \pc Durante casi trescientos mil años esta fue la ley de Dalamatia. Y muchas de las piedras sobre las que se inscribió yacen en la actualidad bajo las aguas frente a las costas de Mesopotamia y Persia. Se convirtió en una costumbre tener presente uno de estos mandamientos cada día de la semana, empleándose como saludo y como acción de gracias a la hora de la comida.
\vs p066 7:17 \pc En estos días, el tiempo se medía por meses lunares, un periodo cifrado en veintiocho días. A excepción del día y la noche, esta constituía la única medida de tiempo conocida por estos pueblos primitivos. Los maestros de Dalamatia introdujeron la semana de siete días cuyo origen provenía del hecho de que el número siete era la cuarta parte de veintiocho. El significado del número siete en el suprauniverso les dio sin duda la ocasión de añadir una nota espiritual al cálculo ordinario del tiempo. Pero el período semanal no tiene origen natural.
\vs p066 7:18 \pc El campo alrededor de la ciudad, en un radio de ciento sesenta kilómetros, estaba bastante bien organizado. En las inmediaciones de la ciudad, cientos de graduados de las escuelas del príncipe se dedicaban a la ganadería o llevaban a cabo, de alguna otra manera, las enseñanzas recibidas de parte de la comitiva del príncipe y de sus numerosos ayudantes humanos. Algunos se ocupaban de la agricultura y de la horticultura.
\vs p066 7:19 No se encargó a la humanidad al duro trabajo de la agricultura como castigo por un supuesto pecado. El mandato “con el sudor de tu frente comerás el fruto de la tierra” no fue una sentencia condenatoria impuesta al hombre por su participación en la insensatez de la rebelión de Lucifer bajo el liderazgo del traidor Caligastia. El cultivo de la tierra es connatural al establecimiento y avance de la civilización en los mundos evolutivos, y este mandato fue el eje de la enseñanza del príncipe planetario y de su comitiva durante los trescientos mil años que transcurrieron entre su llegada a Urantia y aquellos días trágicos en los que Caligastia unió su suerte a la del rebelde Lucifer. La labranza de la tierra no es una maldición, sino más bien la mayor bendición que se puede conceder a todos los que pueden así disfrutar de la más humana de las actividades humanas.
\vs p066 7:20 Al estallar la rebelión, en Dalamatia residían casi seis mil habitantes. Esta cifra incluye a los estudiantes regulares, pero no tiene en cuenta a los visitantes y observadores, cuyo número siempre sobrepasaba el millar. Pero poca idea os podréis hacer del sorprendente progreso de aquellos tiempos lejanos; prácticamente todos los extraordinarios logros de los humanos de aquellos días se destruyeron por la terrible confusión y la deplorable oscuridad espiritual que siguieron al devastador engaño y sedición de Caligastia.
\usection{8. LOS INFORTUNIOS DE CALIGASTIA}
\vs p066 8:1 Al recordar la larga trayectoria de Caligastia, nos encontramos con una sola particularidad de su conducta que podría haber llamado nuestra atención: era extremadamente individualista. Tenía predisposición a colocarse de parte de cualquier grupo de protesta y solía solidarizarse con quienes expresaban con moderación críticas soterradas. Descubrimos la aparición temprana de esta tendencia a inquietarse ante la autoridad superior y a contrariarse levemente ante cualquier supervisión. A pesar de sentirse ligeramente agraviado por el asesoramiento de algún superior y mostrarse algo intranquilo ante la autoridad, siempre que se sometió a alguna prueba, había demostrado su lealtad hacia los gobernantes del universo y su obediencia a los mandatos de los padres de la constelación. Realmente nunca se le había hallado en falta hasta el momento de su indigna traición a Urantia.
\vs p066 8:2 Cabe destacar que tanto a Lucifer como a Caligastia se les había indicado pacientemente y advertido con amor respecto a su predisposición a la crítica y al sutil desarrollo de su envanecimiento y del exagerado sentimiento de prepotencia que se derivaba de este. No obstante, todos estos intentos por ayudarles habían sido tergiversados y considerados como una crítica sin fundamento y como una intromisión injustificada en la libertad personal. Según la opinión de Caligastia y de Lucifer, sus bien dispuestos y generosos asesores obraban siguiendo los mismos censurables motivos que comenzaban a dominar sus propios pensamientos distorsionados y sus equivocados planes. Los juzgaban según el desarrollo de su propio egoísmo.
\vs p066 8:3 \pc Desde la llegada del príncipe Caligastia, la civilización planetaria progresó con bastante normalidad durante casi trescientos mil años. Al margen de ser una esfera de modificación de la vida y, por lo tanto, estar sometida a numerosas irregularidades y a inusuales episodios de fluctuación evolutiva, Urantia, en su trayectoria planetaria, avanzó de forma muy satisfactoria hasta los tiempos de la rebelión de Lucifer y de la traición simultánea de Caligastia. Este fatal error al igual que el fracaso más tarde de Adán y de Eva en el cumplimiento de su misión planetaria modificó de forma definitiva toda la historia posterior del planeta.
\vs p066 8:4 El príncipe de Urantia se hundió en la oscuridad en el momento de la rebelión de Lucifer, ocasionando la prolongada confusión que reinaría en Urantia. Con posterioridad, se le privó de su autoridad como soberano mediante la acción correlacionada de los gobernantes de la constelación y de otras autoridades del universo. El príncipe fue partícipe de las inevitables vicisitudes por las que pasó Urantia hasta el momento de la estancia de Adán en el planeta y contribuyó de algún modo al fracaso del plan destinado a la elevación de las razas mortales, que se llevaría a cabo con la ayuda de la infusión de sangre vital de la nueva raza violeta: los descendientes de Adán y Eva.
\vs p066 8:5 En los días de Abraham, se restringió enormemente la capacidad del príncipe caído para interferir en los asuntos humanos debido a la encarnación mortal de Maquiventa Melquisedec; y, posteriormente, durante la vida de Miguel en la carne, se despojó finalmente a este príncipe traidor de toda autoridad sobre Urantia.
\vs p066 8:6 \pc Aunque la doctrina de un diablo personal en Urantia tenía alguna base en la presencia planetaria del traidor e inicuo Caligastia, era sin embargo enteramente ficticia en el sentido de que predicaba que dicho “diablo” podía influenciar la mente humana normal en contra de su facultad de elección libre y natural. Incluso antes del ministerio de gracia de Miguel en Urantia, ni Caligastia ni Daligastia pudieron jamás controlar a los mortales ni forzar a ningún ser normal a cometer alguna acción contraria a la voluntad humana. La libre voluntad del hombre es suprema en los asuntos morales; incluso el modelador del pensamiento interior se niega a imponer en el hombre cualquier pensamiento o a que realice una sola acción que sea contraria a la elección de su voluntad.
\vs p066 8:7 Y ahora, este rebelde del mundo, despojado de cualquier poder para causar mal a sus antiguos súbditos, aguarda el dictamen final de los ancianos de días de Uversa respecto a todos los que participaron en la rebelión de Lucifer.
\vsetoff
\vs p066 8:8 [Exposición de un melquisedec de Nebadón.]
