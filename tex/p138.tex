\upaper{138}{Formación de los mensajeros del reino}
\author{Comisión de seres intermedios}
\vs p138 0:1 Esa tarde, una vez que predicó su sermón sobre “El reino”, Jesús congregó a los seis apóstoles y comenzó a revelarles su plan de ir visitando las ciudades de las inmediaciones del mar de Galilea. Sus hermanos Santiago y Judá se sintieron muy dolidos por no haber sido llamados a esta reunión. Hasta ese momento, se habían considerado a sí mismos como parte del círculo cercano de los acompañantes de Jesús. Pero Jesús había decidido no tener parientes cercanos en el grupo de los jefes apostólicos del reino. Esta no admisión de Santiago y Judá en el grupo de los elegidos, junto con su aparente alejamiento de su madre desde lo sucedido en Caná, creó una brecha entre Jesús y su familia, que se fue agrandando cada vez más. Tal situación continuó a lo largo de todo su ministerio público ---ellos casi llegaron a rechazarlo--- y estas diferencias no llegarían a desaparecer por completo hasta después de la muerte y resurrección de Jesús. Su madre se debatía constantemente, por un lado, entre una actitud fluctuante de fe y de esperanza y, por otro, entre crecientes sentimientos de decepción, humillación y desconsuelo. Solo Ruth, la más pequeña, permaneció inquebrantablemente leal a su padre\hyp{}hermano.
\vs p138 0:2 Hasta después de la resurrección, toda la familia de Jesús tuvo muy poco que ver con su ministerio. No hay profeta sin honra salvo en su propia tierra, ni sin reconocimiento y aprecio salvo en su propia familia.
\usection{1. INSTRUCCIONES FINALES}
\vs p138 1:1 Al día siguiente, domingo 23 de junio del año 26 d. C., Jesús impartió sus últimas instrucciones a los seis. Les mandó que salieran de dos en dos para enseñar la buena nueva del reino. Les prohibió que bautizaran y les advirtió que no predicaran públicamente. Añadió que, más adelante, sí les sería permitido predicar en público, pero que, durante un tiempo y por distintos motivos, deseaba que adquiriesen experiencia práctica en el trato personal con otras personas. Jesús se propuso que su primer viaje apostólico fuese por entero una \bibemph{labor de índole personal}. Aunque este anuncio fue una especie de decepción para los apóstoles, reconocieron, al menos en parte, la razón que movía a Jesús para dar comienzo de esta manera a la proclamación del reino, y siguieron sus instrucciones animosos y con confiado entusiasmo. Los envió de dos en dos; Santiago y Juan a Queresa, Andrés y Pedro a Cafarnaúm, en tanto que a Felipe y a Natanael a Tariquea.
\vs p138 1:2 Antes de comenzar estas primeras dos semanas de servicio, Jesús les anunció que deseaba ordenar a doce apóstoles para que continuasen el trabajo del reino tras su partida, y autorizó a cada uno de ellos a elegir, entre sus primeros conversos, a algún hombre para que fuese miembro del cuerpo previsto de apóstoles. Juan expresó su opinión, preguntando: “Pero, Maestro, ¿vendrán estos seis hombres entre nosotros y compartirán todas las cosas igualmente con nosotros que hemos estado contigo desde el Jordán y hemos oído todas tus enseñanzas como preparación para esta, nuestra primera tarea por el reino?”. Y Jesús le contestó: “Sí, Juan, los hombres que elijáis serán uno solo con nosotros, y vosotros les enseñaréis todo lo que atañe al reino tal como yo os lo he enseñado”. Tras decir esto, Jesús los dejó.
\vs p138 1:3 Los seis no acudieron a realizar su labor hasta no haber comentado entre ellos largamente las indicaciones de Jesús de que cada uno eligiese a un nuevo apóstol. Finalmente, se impuso el criterio de Andrés, y se marcharon a sus actividades. En esencia, Andrés dijo: “El Maestro tiene razón; somos demasiados pocos para poder ocuparnos de todo el trabajo por hacer. Necesitamos más instructores, y el Maestro ha depositado una gran confianza en nosotros al encomendarnos que escojamos a estos seis nuevos apóstoles”. Esa mañana, al separarse para atender cada cual su tarea, albergaban ocultamente en sus corazones un cierto desaliento. Sabían que iban a extrañar a Jesús y, además de su temor y cortedad de ánimo, no se habían imaginado que la inauguración del reino de los cielos se haría de aquel modo.
\vs p138 1:4 Se había dispuesto que los seis llevaran a cabo aquella labor durante dos semanas, tras la que regresarían a la casa de Zebedeo para mantener una reunión. Entretanto, Jesús fue a Nazaret para hablar con José, Simón y otros miembros de su familia que vivían en los alrededores. Jesús hizo todo lo humanamente posible, compatible con su dedicación a la realización de la voluntad de su Padre, para mantener la confianza y el afecto de su familia. Él, en esta cuestión, cumplió perfectamente con sus obligaciones e incluso más.
\vs p138 1:5 Mientras los apóstoles llevaban a cabo esta misión, Jesús pensó mucho en Juan, ya en aquel momento encarcelado. Sintió la gran tentación de usar sus latentes poderes para liberarlo, pero una vez más se resignó a “aguardar la voluntad del Padre”.
\usection{2. LA ELECCIÓN DE LOS SEIS}
\vs p138 2:1 Este primer viaje misionero de los seis fue sumamente satisfactorio. Todos ellos descubrieron el gran valor del contacto directo y personal con las demás personas. Volvieron a Jesús siendo totalmente conscientes de que, al fin y al cabo, la religión es exclusivamente, y por completo, una cuestión de \bibemph{experiencia personal}. Empezaron a percibir el hambre de la gente común por oír palabras que les consolaran religiosamente y les alentara espiritualmente. Cuando se reunieron con Jesús, todos querían hablar al mismo tiempo, pero Andrés tomó las riendas de la conversación y, conforme los iba llamando uno a uno, cada cual presentaba debidamente su informe al Maestro y ofrecía sus propuestas de nombres para los seis nuevos apóstoles.
\vs p138 2:2 Jesús, después de que cada cual hubiese expuesto su elección de los nuevos apóstoles, sometió a votación, entre todos, los nombramientos realizados; de este modo, los seis nuevos apóstoles fueron oficialmente aceptados por los seis más antiguos. Entonces, Jesús anunció que todos ellos visitarían a estos candidatos para llamarlos al servicio.
\vs p138 2:3 Los apóstoles recién seleccionados fueron:
\vs p138 2:4 \li{1.}\bibemph{Mateo Leví,} recaudador de aduanas de Cafarnaúm, que tenía su despacho exactamente al este de la ciudad, cerca de la frontera con Batanea. Andrés fue quién lo seleccionó.
\vs p138 2:5 \li{2.}\bibemph{Tomás Dídimo,} pescador de Tariquea y con anterioridad carpintero y albañil de Gadara. Felipe lo eligió.
\vs p138 2:6 \li{3.}\bibemph{Santiago Alfeo,} pescador y agricultor de Kheresa, fue escogido por Santiago Zebedeo.
\vs p138 2:7 \li{4.}\bibemph{Judas Alfeo,} el hermano gemelo de Santiago Alfeo, también pescador, fue elegido por Juan Zebedeo.
\vs p138 2:8 \li{5.}\bibemph{Simón el Zelote} era un alto funcionario del movimiento patriótico de los zelotes, cargo al que renunció para unirse a los apóstoles de Jesús. Antes de unirse a los zelotes, Simón había sido comerciante. Pedro fue quien lo eligió.
\vs p138 2:9 \li{6.}\bibemph{Judas Iscariote} era el hijo único de unos padres judíos adinerados, habitantes de Jericó. Se había incorporado al grupo de Juan el Bautista, y sus padres saduceos lo habían repudiado. Estaba buscando trabajo en estas regiones cuando los apóstoles de Jesús lo encontraron y, mayormente por su experiencia con las cuestiones pecuniarias, Natanael lo invitó a unirse a ellos. De entre los apóstoles, Judas Iscariote era el único originario de Judea.
\vs p138 2:10 \pc Jesús pasó todo un día con los seis, respondiendo a sus preguntas y escuchando los detalles de sus informes; tenían muchas vivencias interesantes y provechosas que relatar. Reconocían ahora la sabiduría del plan del Maestro de enviarlos a trabajar de una manera sosegada y personal, antes de poner en marcha cualquier iniciativa de carácter público de mayor envergadura.
\usection{3. EL LLAMAMIENTO DE MATEO Y SIMÓN}
\vs p138 3:1 Al día siguiente Jesús y los seis hicieron una visita a Mateo, el recaudador de aduanas. Mateo los aguardaba; ya había hecho el balance de las cuentas y estaba listo para traspasar los asuntos de su despacho a su hermano. Al aproximarse a la caseta de peajes, Andrés se adelantó con Jesús, que, mirándole a los ojos, dijo a Mateo: “Sígueme”; y él se levantó y se fue a su casa con Jesús y los apóstoles.
\vs p138 3:2 Mateo le dijo a Jesús que él había organizado un banquete para esa noche, que deseaba ofrecerles tal cena a su familia y amigos si Jesús la aprobaba y accedía ser el invitado de honor. Y Jesús asintió, dando su consentimiento. Entonces Pedro llevó a Mateo aparte y le comentó que él había invitado a un tal Simón a que se uniera a los apóstoles, asegurándose de que Mateo diera su aprobación para que Simón fuese también convidado al festín.
\vs p138 3:3 \pc Al mediodía, tras almorzar en casa de Mateo, fueron todos con Pedro a visitar a Simón el Zelote, a quien encontraron en su antiguo negocio, del que se encargaba ahora su sobrino. Cuando Pedro llevó a Jesús hasta Simón, el Maestro saludó al apasionado patriota y solamente le dijo: “Sígueme”.
\vs p138 3:4 \pc Todos volvieron a la casa de Mateo, donde hablaron mucho sobre política y religión hasta la hora de la cena. La familia Leví llevaba mucho tiempo dedicada a los negocios y a la recaudación de impuestos; así pues, a muchos de los que Mateo había invitado a este banquete los fariseos les habrían denominado “publicanos y pecadores”.
\vs p138 3:5 En aquellos tiempos, cuando se le ofrecía a alguien prominente un banquete de recepción de este tipo, se acostumbraba que todas las personas interesadas permaneciesen en el salón del banquete para observar a los invitados y escuchar la conversación y los discursos de los hombres honorables. En consecuencia, en esta ocasión había muchos fariseos de Cafarnaúm presentes para observar la forma de proceder de Jesús en esta insólita reunión social.
\vs p138 3:6 Conforme avanzaba la cena, la alegría de los comensales escaló en animación y, estando todos disfrutando de un rato tan agradable, los expectantes fariseos comenzaron secretamente a criticar a Jesús por tomar parte en algo tan liviano y desenfadado. Más tarde en la noche, cuando se pronunciaban los discursos, uno de los fariseos más maliciosos llegó hasta el extremo de criticar ante Pedro el comportamiento de Jesús, diciendo: “¿Cómo te atreves a predicar que este hombre es justo cuando come con publicanos y pecadores y se presta a estas escenas de desenfado y alegres placeres?”. Pedro le susurró esta crítica a Jesús antes de que él impartiera la bendición de despedida a todos los congregados. Cuando comenzó a hablar, Jesús dijo: “Al venir aquí esta noche para dar la bienvenida a Mateo y a Simón a nuestra fraternidad, me complace ser testigo de vuestra alegría, sociabilidad y buen ánimo, pero deberíais regocijaos aún más porque muchos de vosotros entrará en el reino del espíritu que está por venir, en el que gozaréis con mayor abundancia de las cosas buenas del reino de los cielos. Y a los que estáis ahí criticándome en vuestro interior porque estoy aquí para celebrar con mis amigos, sabed que he venido para proclamar gozo a los oprimidos sociales y libertad espiritual a los cautivos morales. ¿Necesito recordaros que los sanos no tienen necesidad de médico, sino los enfermos? No he venido a llamar a los justos sino a los pecadores”.
\vs p138 3:7 Y verdaderamente aquello resultaba una escena extraña para todo el judaísmo: ver a un hombre de carácter recto y nobles sentimientos mezclándose con soltura y gustosamente con la gente común, incluso con una multitud irreligiosa y hedonista de publicanos y supuestos pecadores. Simón el Zelote quiso dar un discurso ante los reunidos en casa de Mateo, pero Andrés, consciente de que Jesús no quería que se confundiera el reino venidero con el movimiento de los zelotes, lo persuadió para que evitara hacer declaraciones públicas.
\vs p138 3:8 Jesús y los apóstoles pasaron la noche en casa de Mateo y, la gente, al marcharse a sus casas, únicamente hablaba de la bondad y la afabilidad de Jesús.
\usection{4. EL LLAMAMIENTO DE LOS GEMELOS}
\vs p138 4:1 La mañana siguiente, los nueve se desplazaron por barco hasta Queresa, para llamar como correspondía a los siguientes dos apóstoles, a Santiago y a Judas, los hijos gemelos de Alfeo, los candidatos elegidos por Santiago y Juan Zebedeo. Ambos pescadores estaban a la expectativa de la llegada de Jesús y sus apóstoles y los aguardaban, pues, en la orilla. Santiago Zebedeo presentó al Maestro a los pescadores de Queresa, y Jesús, poniendo su mirada en ellos, asintió y dijo: “Seguidme”.
\vs p138 4:2 \pc Esa tarde, en la que estuvieron todos juntos, Jesús les instruyó convenientemente sobre la asistencia a las reuniones festivas, y concluyó sus observaciones diciendo: “Todos los hombres son mis hermanos. Mi Padre celestial no desprecia a ninguna de las criaturas de nuestra creación. El reino de los cielos está abierto para todos los hombres y mujeres. Nadie puede cerrar la puerta de la misericordia a un alma hambrienta que trate de entrar en él. Nos sentaremos a comer con todos aquellos que deseen oír del reino. Desde lo alto, en la mirada del Padre, todos los hombres son iguales. No os neguéis a romper el pan con un fariseo o un pecador, con un saduceo o un publicano, con un romano o un judío, con alguien rico o pobre, libre o esclavo. La puerta del reino está abierta de par en par para todos aquellos que deseen conocer la verdad y encontrar a Dios”.
\vs p138 4:3 \pc Aquella noche en una sencilla cena en casa de Alfeo, se recibió a los gemelos en la familia apostólica. Más tarde en la noche, Jesús dio a sus apóstoles su primera lección sobre el origen, la naturaleza y el destino de los espíritus impuros, pero no pudieron comprender el alcance de lo que les decía. Encontraban muy fácil amar y admirar a Jesús, pero muy difícil entender muchas de sus enseñanzas.
\vs p138 4:4 Tras una noche de descanso, todo el grupo, que sumaban ya once, se dirigió en barco a Tariquea.
\usection{5. EL LLAMAMIENTO DE TOMÁS Y JUDAS}
\vs p138 5:1 Tomás, el pescador, y Judas, el vagabundo, se encontraron con Jesús y los apóstoles en el desembarcadero de pesqueros de Tariquea, y Tomás llevó al grupo a su casa, que se encontraba cerca. Entonces, Felipe presentó a Tomás, el candidato que él había propuesto para el apostolado, y Natanael hizo lo mismo con Judas Iscariote, su candidato de Judea. Jesús miró a Tomás y le dijo: “Tomás, te falta fe; sin embargo, yo te recibo. Sígueme”. A Judas Iscariote, el Maestro le dijo: “Judas, somos todos de la misma carne y, al recibirte entre nosotros, ruego que seas siempre leal a tus hermanos galileos. Sígueme”.
\vs p138 5:2 \pc Cuando habían descansado, Jesús llevó a los doce aparte durante un tiempo para orar con ellos e instruirles en la naturaleza y obra del espíritu santo, pero de nuevo, y en gran medida, fueron incapaces de comprender el significado de esas magníficas verdades que él trataba de impartirles. Había quien comprendía un concepto y quien comprendía otro, pero ninguno de ellos, alcanzaba a enterarse de la totalidad de sus enseñanzas. Siempre cometían el error de intentar adaptar el nuevo evangelio de Jesús a sus viejos sistemas de creencias religiosas. No podían captar la idea de que Jesús había venido para proclamar un nuevo evangelio de salvación y establecer un nuevo camino de encontrar a Dios; no percibían que él \bibemph{era} una nueva revelación del Padre celestial.
\vs p138 5:3 Al día siguiente, Jesús dejó completamente solos a sus doce apóstoles; quería que se conocieran y que estuvieran sin su compañía para que comentaran entre ellos lo que les había enseñado. El Maestro volvió para la cena y, durante las horas de la sobremesa, les habló sobre el ministerio de los serafines, y algunos de los apóstoles comprendieron sus enseñanzas. Descansaron durante una noche y, al día siguiente, se dirigieron en barco hacia Cafarnaúm.
\vs p138 5:4 Zebedeo y Salomé se habían ido a vivir con su hijo David, para poder ceder su amplia casa a Jesús y a sus doce apóstoles. Aquí Jesús pasó un \bibemph{sabbat} tranquilo con sus mensajeros, ya elegidos; esbozó detenidamente los planes para proclamar el reino y explicó convenientemente la importancia de evitar cualquier conflicto con las autoridades civiles, diciendo: “Si los gobernantes civiles han de ser reprendidos, dejadme a mí esa tarea. Cuidaos de no hacer ninguna denuncia del césar o de sus sirvientes”. Fue esta misma noche cuando Judas Iscariote llevó a Jesús a un lado para preguntarle por qué no se hacía nada para sacar a Juan de la prisión. Y Judas no quedó suficientemente satisfecho con la actitud de Jesús.
\usection{6. LA SEMANA DE FORMACIÓN INTENSIVA}
\vs p138 6:1 La semana siguiente fue un período de formación intensiva. Todos los días, cada uno de los seis apóstoles nuevos se ponía en manos del respectivo apóstol que lo había propuesto y realizaba un estudio a fondo de todo lo que los primeros discípulos habían aprendido y experimentado en su preparación para la labor del reino. Los apóstoles más antiguos repasaban detenidamente, en beneficio de los seis más nuevos, las enseñanzas de Jesús impartidas hasta ese momento. Por las noches, todos ellos se reunían en el jardín de Zebedeo para recibir instrucción de Jesús.
\vs p138 6:2 Fue entonces cuando Jesús estableció un día de descanso y esparcimiento a mediados de la semana. Y continuaron con este plan semanal de distensión durante el resto de la vida material del Maestro. Por regla general, los miércoles nunca llevaban a cabo sus actividades regulares. En este día festivo de la semana, Jesús, normalmente, se alejaba de ellos diciendo: “Hijos míos, tomaros una jornada de ocio. Descansad de la ardua labor del reino y disfrutad del estimulante efecto de volver a nuestras antiguas vocaciones o de descubrir actividades lúdicas nuevas”. Aunque Jesús, en este período de su vida en la tierra, no precisaba realmente este día de descanso, se ajustó a este plan porque sabía que era lo mejor para sus compañeros humanos. Jesús era el preceptor ---el Maestro---; sus acompañantes eran sus alumnos ---los discípulos---.
\vs p138 6:3 \pc Jesús procuró dejar claro a sus apóstoles la diferencia entre sus enseñanzas y su \bibemph{vida entre ellos} y las enseñanzas que más tarde podrían aparecer \bibemph{sobre} él. Jesús dijo: “Mi reino y el evangelio vinculado a él deben ser el objeto de vuestro mensaje. No os desviéis predicando \bibemph{sobre} mí y \bibemph{sobre} mis enseñanzas. Proclamad el evangelio del reino y describid mi revelación del Padre de los cielos, pero no os descaminéis creando leyendas y desarrollando un sistema de culto relacionados con creencias y enseñanzas \bibemph{sobre} mis creencias y enseñanzas”. Pero, una vez más, ellos no entendían por qué les hablaba de este modo, y ninguno de ellos se atrevió a preguntarle por qué les enseñaba así.
\vs p138 6:4 En estas tempranas enseñanzas, Jesús intentó, en todo lo posible, evitar controversias con sus apóstoles, salvo aquellas pertinentes a conceptos erróneos sobre su Padre celestial. En todas esas cuestiones, nunca titubeó en corregir sus creencias equivocadas. Con posterioridad a su bautismo, a Jesús, durante su vida en Urantia, le movía \bibemph{un único} motivo: realizar una revelación mejor y más verdadera de su Padre del Paraíso; él fue el pionero del nuevo y mejor camino de llegar a Dios: la vía de la fe y el amor. Siempre, exhortaba así a los apóstoles: “Buscad a los pecadores; encontrad al afligido y consolad al angustiado”.
\vs p138 6:5 Jesús comprendía su situación a la perfección; poseía un ilimitado poder, que podría haber usado para hacer avanzar su misión; pero estaba totalmente satisfecho con unos medios y personas que la mayoría de la gente hubiese considerado como insuficientes e irrelevantes. Había emprendido una misión de enormes y profundas posibilidades, pero insistió en dedicarse a los asuntos de su Padre de la forma más silenciosa y menos impresionante; evitó deliberadamente cualquier manifestación de poder. Y se propuso entonces actuar discretamente, al menos durante algunos meses, con sus doce apóstoles en torno al mar de Galilea.
\usection{7. UNA NUEVA DECEPCIÓN}
\vs p138 7:1 Jesús tenía previsto una apacible campaña misionera de labor personal durante cinco meses. No había comentado a los apóstoles cuánto tiempo duraría; trabajaban de semana en semana. Y temprano ese primer día de la semana, justo cuando estaba a punto de comunicarles este plan a sus doce apóstoles, Simón Pedro, Santiago Zebedeo y Judas Iscariote vinieron a verlo para conversar privadamente con él. Llevando a Jesús aparte, Pedro se atrevió a decir: “Maestro, venimos a petición de nuestros compañeros para preguntar si no es ya el momento propicio de entrar en el reino. ¿Y proclamarás el reino en Cafarnaúm o hemos de seguir hasta Jerusalén? ¿Y cuándo sabremos, cada uno de nosotros, el puesto que ocuparemos contigo cuando el reino se instaure\ldots ” y Pedro hubiera continuado haciendo más preguntas, pero Jesús levantó una mano en un gesto de reproche y lo interrumpió. Y haciendo señas a los otros apóstoles, que estaban cerca, para que se unieran a ellos, Jesús dijo: “Hijos míos, ¿hasta cuándo os tendré que soportar? ¿No os he dejado claro que mi reino no es de este mundo? Os he dicho muchas veces que no he venido para sentarme sobre el trono de David, y entonces, ¿cómo es que estáis indagando qué lugar ocupará cada uno de vosotros en el reino del Padre? ¿No percibís que os he llamado como embajadores de un reino espiritual? ¿Es que no os dais cuenta de que pronto, muy pronto, me representaréis en el mundo y en la proclamación del reino tal como yo represento ahora a mi Padre que está en los cielos? ¿Es posible que yo os haya elegido e instruido como mensajeros del reino y, aun así, no os percatéis de la naturaleza y la trascendencia de este reino venidero que tendrá preeminencia divina en el corazón de los hombres? Amigos míos, oídme una vez más. Desterrad de vuestras mentes la idea de que mi reino es un gobierno de poder o un reino de gloria. Efectivamente, pronto todo el poder en el cielo y en la tierra serán puestos en mis manos, pero no es la voluntad del Padre que empleemos esta dote divina para glorificarnos durante esta era. En otra era, en verdad os sentaréis conmigo en poder y gloria, pero nos corresponde ahora someternos a la voluntad del Padre y seguir adelante en humilde obediencia para llevar a cabo su mandato en la tierra”.
\vs p138 7:2 Una vez más sus acompañantes se quedaron impresionados y perplejos. Jesús los envió de dos en dos para que orasen, pidiéndoles que volvieran a él al mediodía. Durante esta decisiva mañana, cada cual intentó encontrar a Dios, y cada uno de ellos procuró animar y dar fortaleza al otro, y regresaron a Jesús tal como les había solicitado.
\vs p138 7:3 Entonces, Jesús les relató la venida de Juan, el bautismo en el Jordán, la fiesta de bodas de Caná, la reciente elección de los seis y cómo había retirado de ellos a sus propios hermanos en la carne, y les advirtió que el enemigo del reino procuraría asimismo alejarlos. Tras esta charla breve, pero franca, los apóstoles se levantaron, bajo el liderazgo de Pedro, para declarar su inquebrantable devoción a su Maestro y prometer, como Tomás lo expresó, su profunda lealtad al reino, “a este reino venidero, sea tal cual sea, aun cuando no lo entienda por completo”. En verdad, todos \bibemph{creían en Jesús,} aunque no comprendieran por completo sus enseñanzas.
\vs p138 7:4 Entonces, Jesús les preguntó cuánto dinero tenían entre todos; también quiso saber qué habían previsto hacer por sus familias. Cuando resultó que escasamente disponían de recursos suficientes para mantenerse dos semanas, dijo: “No es la voluntad de mi Padre que comencemos nuestra obra de esta manera. Nos quedaremos aquí, junto al mar, durante dos semanas para pescar o hacer todo lo que tengamos a nuestro alcance; entretanto, bajo la guía de Andrés, el primer apóstol elegido, os organizaréis con el fin de proveeros de todo lo necesario para vuestra futura labor, tanto en el actual ministerio personal como cuando yo os mande después a predicar el evangelio e instruir a los creyentes”. Todos se regocijaron profundamente con estas palabras; se trataba del primer indicio claro y seguro de que Jesús planeaba más adelante tomar iniciativas de carácter público de mayor impacto y trascendencia.
\vs p138 7:5 Los apóstoles pasaron el resto del día organizándose de la mejor manera posible y ultimando los arreglos necesarios para las barcas y las redes con objeto de salir a pescar al día siguiente. Todos habían tomado la decisión de dedicarse a la pesca; la mayoría de ellos habían sido pescadores, incluso Jesús era un experto barquero y pescador. Él, con sus propias manos, había construido muchas de las barcas que ellos usarían en esos pocos años siguientes. Y eran buenas y fiables.
\vs p138 7:6 Jesús les instó a que pescaran durante dos semanas, añadiendo: “Y luego saldréis para convertiros en pescadores de hombres”. Pescaban en tres grupos: cada noche, Jesús iba con uno diferente. ¡Y cuánto disfrutaron de su presencia! Era buen pescador, compañero animoso y alentador amigo; cuanto más trabajaban con él, más lo amaban. Mateo dijo un día: “Cuánto más entiendes a algunas personas, menos las admiras, pero a este hombre, incluso conociéndole cada vez menos, más lo amo”.
\vs p138 7:7 Este plan de pescar dos semanas y salir dos semanas para llevar a cabo su labor personal en nombre del reino se prolongó durante cinco meses, e incluso hasta finales de este año, el 26 d.C., hasta después de que se pusiera fin a esas persecuciones dirigidas en particular contra los discípulos de Juan tras su encarcelamiento.
\usection{8. EL PRIMER TRABAJO DE LOS DOCE}
\vs p138 8:1 Tras vender el pescado capturado en las dos semanas, Judas Iscariote, el escogido para actuar en calidad de tesorero de los doce, dividía los fondos apostólicos en seis partes iguales; ya se habían aprovisionado de recursos para el cuidado de las familias de ellos que los necesitaran. Y, entonces, cerca de mediados de agosto del año 26 d. C., se marcharon de dos en dos para actuar en los lugares asignados por Andrés. Las primeras dos semanas, Jesús salió con Andrés y Pedro; las dos segundas semanas, con Santiago y Juan; y, así sucesivamente, con las otras parejas en el orden de su elección. De esta manera, él pudo ir, al menos una vez, con cada una de estas antes de convocarlas para dar comienzo a su ministerio público.
\vs p138 8:2 Jesús les enseñó a predicar el perdón de los pecados mediante la \bibemph{fe en Dios} sin penitencia o sacrificios, y que el Padre que está en los cielos ama a todos sus hijos con el mismo amor eterno. Ordenó a sus apóstoles que se abstuvieran de comentar:
\vs p138 8:3 \li{1.}La labor y el encarcelamiento de Juan el Bautista.
\vs p138 8:4 \li{2.}La voz oída en su bautismo. Jesús dijo: “Solamente aquellos que oyeron la voz pueden referirse a ella. Hablad únicamente de las cosas que habéis oído de mí; no habléis de rumores”.
\vs p138 8:5 \li{3.}La conversión del agua en vino en Caná. Jesús les encargó seriamente, diciendo: “No digáis a nadie lo del agua y el vino”.
\vs p138 8:6 \pc Pasaron momentos magníficos durante esos cinco o seis meses, en los que trabajaban de pescadores cada dos semanas alternas, ganando el dinero suficiente para su sustento, mientras viajaban durante las dos semanas siguientes en su labor misionera por el reino.
\vs p138 8:7 La gente común se quedaba maravillada ante las enseñanzas y el ministerio de Jesús y sus apóstoles. Desde hacía mucho tiempo, los rabinos habían enseñado a los judíos que el ignorante no podía ser ni piadoso ni recto. Pero los apóstoles de Jesús eran a la vez piadosos y rectos; si bien, eran felizmente ignorantes de mucho del conocimiento de los rabinos y de la sabiduría del mundo.
\vs p138 8:8 \pc Jesús indicó claramente a sus apóstoles la diferencia entre el arrepentimiento mediante la realización de las supuestas buenas obras que enseñaban los judíos y la transformación de la mente mediante la fe ---el nuevo nacimiento---, que él exigía como precio para ser admitido en el reino. Enseñó a sus apóstoles que la \bibemph{fe} era el único requisito que se precisaba para entrar en el reino del Padre. Juan les había enseñado: “Arrepentíos ---huid de la ira venidera---”. Jesús impartía la enseñanza de que “La fe es la puerta abierta para tener acceso al amor presente, perfecto y eterno de Dios”. Jesús no hablaba como profeta, como alguien que viene a decir la palabra de Dios. Parecía hablar de sí mismo como alguien que ostentaba autoridad. Jesús procuraba desviar la atención de sus mentes de la búsqueda del milagro hacia el descubrimiento de una vivencia verdadera y personal, ante la satisfacción y la certeza de la inhabitación en ellos del espíritu de Dios, de amor y gracia salvadora.
\vs p138 8:9 Los discípulos aprendieron muy pronto que el Maestro tenía un profundo respeto y una misericordiosa consideración por \bibemph{cualquier} ser humano con el que se encontrara, y estaban extraordinariamente impresionados ante esta atención que persistente e invariablemente brindaba a toda clase de hombres, mujeres y niños. A veces, hacía una pausa en medio de algún profundo discurso para salir al camino y hablarle palabras de ánimo a una mujer que pasaba por allí sobrecargada con el peso de su cuerpo y de su alma. Interrumpía una importante conversación con sus apóstoles para fraternizar con un niño algo inoportuno. Para Jesús no parecía haber nada más importante que ese \bibemph{ser humano} que por casualidad se hallaba en su inmediata presencia. Él era preceptor y maestro, pero incluso algo más ---era también amigo y vecino, un compañero comprensivo---.
\vs p138 8:10 \pc Aunque la enseñanza pública de Jesús consistía principalmente en parábolas y discursos breves, instruía sistemáticamente a sus apóstoles mediante preguntas y respuestas. Durante sus posteriores discursos públicos, siempre haría una pausa para responder a las preguntas que sinceramente le hacían.
\vs p138 8:11 En un principio, los apóstoles se quedaron impactados por el modo en el que Jesús trataba a las mujeres, pero pronto se acostumbraron; les dejó muy claro que, en el reino, se debería otorgar a las mujeres los mismos derechos que a los hombres.
\usection{9. CINCO MESES DE EXTENUANTE PRUEBA}
\vs p138 9:1 Este período, hasta cierto punto monótono, en el que alternaban la pesca con la labor personal, fue para los doce una experiencia agotadora, una prueba que lograron superar con éxito. Es cierto que se quejaron, que tuvieron dudas, que sintieron insatisfacciones transitorias, pero permanecieron fieles a sus votos de devoción y lealtad hacia el Maestro. Durante estos meses, fue su relación personal con Jesús la que hizo que todos (excepto Judas Iscariote) lo amaran y se mantuviesen con esa lealtad y fidelidad ante incluso las sombrías horas del juicio y la crucifixión. En realidad, unos verdaderos hombres, como ellos eran, no podían dejar abandonado al Maestro que tanto veneraban, que tan próximo a ellos había vivido y que tanta dedicación les había mostrado, como había hecho Jesús. Durante esas horas aciagas de la muerte del Maestro, se alejó del corazón de los apóstoles cualquier razonamiento, juicio y lógica en deferencia a una única y extraordinaria emoción humana ---el supremo sentimiento de amistad\hyp{}lealtad---. Estos cinco meses de trabajo con Jesús llevaron a estos apóstoles, a cada uno de ellos, a apreciarle como el mejor \bibemph{amigo} que tenían en el mundo. Y fue este sentimiento humano, más que sus formidables enseñanzas o sus obras maravillosas, el que los mantuvo unidos hasta después de la resurrección y de la reanudación de la proclamación del evangelio del reino.
\vs p138 9:2 Estos meses de labor silenciosa y de inactividad pública no solo significaron una gran prueba para los apóstoles, que supieron sobrellevar, sino también para la familia de Jesús. En el momento en el que Jesús se preparaba para emprender su ministerio público, toda su familia (excepto Ruth) había llegado prácticamente a abandonarlo. Solo en pocas ocasiones tratarían en lo sucesivo de ponerse en contacto con él, y fue para intentar persuadirlo de que regresara a casa con ellos; estaban casi convencidos de que estaba fuera de sí. Sencillamente, no podían alcanzar a comprender su filosofía ni sus enseñanzas; aquello era demasiado para los de su propia carne y sangre.
\vs p138 9:3 \pc Los apóstoles llevaron a cabo su labor personal en Cafarnaúm, Betsaida\hyp{}Julias, Corazín, Gérasa, Hipos, Magdala, Caná, Belén de Galilea, Jotapata, Ramá, Safed, Giscala, Gadara y Abila. Aparte de estas ciudades, también trabajaron en muchas aldeas al igual que en las zonas rurales. Al término de este período, los doce habían hecho planes, con resultados bastante satisfactorios, para el cuidado de sus respectivas familias. La mayoría de los apóstoles eran casados, algunos tenían varios hijos, pero habían tomado tales medidas para el sustento de los miembros de su familia que, con una pequeña ayuda de los fondos apostólicos, podían dedicar sus energías por entero a la obra del Maestro, sin tener que preocuparse por su bienestar económico.
\usection{10. ORGANIZACIÓN DE LOS DOCE}
\vs p138 10:1 Pronto, los apóstoles se organizaron de la siguiente manera:
\vs p138 10:2 \li{1.}A Andrés, el primer apóstol en ser elegido, se le nombró presidente de las comisiones y director general de los doce.
\vs p138 10:3 \li{2.}A Pedro, Santiago y Juan se les nombraron acompañantes personales de Jesús. Tendrían que asistirlo día y noche, ocuparse de sus diversas necesidades materiales y estar con él en esas vigilias nocturnas de oración y misteriosa comunión con el Padre celestial.
\vs p138 10:4 \li{3.}A Felipe se le designó como encargado de abastecimientos del grupo. Su deber era proporcionar los alimentos y comprobar que los visitantes, e incluso a veces el gran número de personas que asistían, tuvieran algo para comer.
\vs p138 10:5 \li{4.}Natanael velaba por las necesidades de las familias de los doce apóstoles. Regularmente, recibía notificación sobre las condiciones en las que se encontraban estas familias y, solicitando los convenientes fondos a Judas, el tesorero, se los enviaba a aquellas que los necesitaran.
\vs p138 10:6 \li{5.}Mateo era el gestor de la economía del cuerpo apostólico. Su deber era asegurarse de que hubiese un equilibrio presupuestario y de que las arcas estuviesen repletas. Si no se conseguían fondos para su propio sustento, si no se recibían donativos suficientes para que el grupo pudiese mantenerse, Mateo estaba autorizado para ordenar a los doce a que regresaran a sus redes de pescadores durante algún tiempo. Pero esto no fue nunca necesario desde que comenzaron su trabajo público; él siempre disponía en las arcas de fondos suficientes para sufragar sus actividades.
\vs p138 10:7 \li{6.}Tomás se hizo cargo de los desplazamientos. Recayó sobre él la responsabilidad de organizar el hospedaje y, de forma general, elegir los lugares en los que se impartirían las enseñanzas y las predicaciones, garantizando de este modo un itinerario sin dificultades ni demoras.
\vs p138 10:8 \li{7.}A Santiago y Judas, los hijos gemelos de Alfeo, se les asignó la gestión de las multitudes. Su tarea era reclutar a un número suficiente de ujieres que ayudaran, durante la predicación, a mantener el orden en las muchedumbres.
\vs p138 10:9 \li{8.}A Simón el Zelote se le encargó de las actividades recreativas y lúdicas. Gestionaba el programa de los miércoles y procuraba asimismo proporcionar algunas pocas horas al día de esparcimiento y distracción.
\vs p138 10:10 \li{9.}A Judas Iscariote se le nombró tesorero. Llevaba la bolsa del dinero. Pagaba todos los gastos y llevaba las cuentas. Hacía las previsiones presupuestarias de semana en semana para Mateo e igualmente le presentaba a Andrés los informes semanales. Judas abonaba los fondos con la autorización de Andrés.
\vs p138 10:11 \pc Esta era la manera en la que los doce actuaron desde su temprana organización hasta el momento en el que fue necesario realizar una reorganización debido a la deserción de Judas, el traidor. El Maestro y sus discípulos\hyp{}apóstoles continuaron viviendo de este modo sencillo hasta el domingo 12 de enero del año 27 d. C., momento en el que él los convocó y les ordenó debidamente como embajadores del reino y predicadores de su buena nueva. Y poco tiempo después se dispusieron a salir para Jerusalén y Judea en su primer viaje de predicación pública.
