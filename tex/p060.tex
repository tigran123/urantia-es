\upaper{60}{Urantia durante la era de la vida terrestre primitiva}
\author{Portador de vida}
\vs p060 0:1 La era de la vida exclusivamente marina ha concluido. La elevación de suelo, el enfriamiento de la corteza terrestre y de los océanos, la delimitación de los mares, junto con el gran aumento del suelo en latitudes septentrionales, fueron los grandes causantes del cambio climático del mundo en todas las regiones alejadas de la zona ecuatorial.
\vs p060 0:2 La última época de la era anterior fue, efectivamente, la era de las ranas, pero estos ancestros de los vertebrados terrestres ya no predominaban; habían sobrevivido en un número muy reducido. Muy pocas de ellas pudieron superar las rígidas pruebas a las que estuvieron expuestas durante el período anterior de hecatombe biológica. Incluso las esporofitas estuvieron a punto de extinguirse.
\usection{1. LA ERA TEMPRANA DE LOS REPTILES}
\vs p060 1:1 Los depósitos de erosión de este período fueron mayormente conglomerados, esquisto, y arenisca. Tanto en América como en Europa, el yeso y las capas rojas de todas estas sedimentaciones muestran que el clima de estos continentes era árido. Estas zonas áridas estuvieron sometidas a una gran erosión a causa de los aguaceros violentos y periódicos que cayeron sobre las altiplanicies circundantes.
\vs p060 1:2 Pocos fósiles se pueden encontrar en estas capas, pero en las piedras de arenisca es factible observar un gran número de huellas de reptiles terrestres. En muchas regiones, el depósito de piedra de arenisca roja de trescientos metros, correspondiente a este periodo, no contiene fósil alguno. Los animales terrestres solo vivieron de manera permanente en algunas partes de África.
\vs p060 1:3 El grosor de estos depósitos oscila entre unos 900 y 3000 metros, y alcanza incluso 5500 metros en la costa del Pacífico. Más tarde, la lava se introdujo con fuerza entre muchas de estas capas. Los acantilados del Río Hudson se formaron por la expulsión de lavas basálticas entre estos estratos triásicos. La actividad volcánica era de gran magnitud en diferentes partes del mundo.
\vs p060 1:4 Se pueden encontrar depósitos de este período en Europa, especialmente en Alemania y en Rusia. En Inglaterra, la nueva piedra arenisca roja pertenece a esta época. La piedra caliza se depositó en los Alpes meridionales como consecuencia de la irrupción del mar y, en la actualidad, se puede observar, bajo la forma peculiar de muros, picos y pilares de piedra caliza dolomítica de esas regiones. Esta capa se encuentra por toda África y Australia. El mármol de Carrara procede de dicha caliza modificada. No se encontrará nada de este período en las regiones meridionales de América del Sur; esa parte del continente permaneció sumergida y, por consiguiente, solo presenta un depósito acuático o marino ininterrumpido con las épocas precedentes y posteriores.
\vs p060 1:5 \pc Hace \bibemph{150\,000\,000} de años comenzaron los tempranos períodos de vida terrestre de la historia del mundo. La vida en general no experimentaba grandes progresos, pero lo hacía más que durante el arduo y adverso cierre de la era de la vida marina.
\vs p060 1:6 Al inaugurarse esta era, las partes orientales y centrales de América del Norte, la mitad septentrional de América del Sur, la mayoría de Europa y toda Asia están muy por encima del agua. Por primera vez, América del Norte se halla geográficamente aislada, pero no por mucho tiempo, puesto que el puente terrestre del Estrecho de Bering emerge pronto de nuevo comunicando el continente con Asia.
\vs p060 1:7 En América del Norte, se desarrollaron grandes depresiones, en paralelo con las costas del Atlántico y del Pacífico. En Connecticut apareció la gran falla oriental, uno de cuyos lados acabó por hundirse más de tres kilómetros. Luego, muchas de estas depresiones norteamericanas al igual que muchas de las cuencas de los lagos de agua dulce y salada de las regiones montañosas se llenaron de depósitos de la erosión. Más adelante, estas depresiones del suelo terrestre, repletas, se elevaron sobremanera debido a los flujos de lava que se producían bajo tierra. Los bosques petrificados de muchas regiones pertenecen a esta época.
\vs p060 1:8 La costa del Pacífico, situada normalmente por encima del agua durante los sumergimientos continentales, se hundió, exceptuando la parte sur de California y una gran isla que existía entonces en lo que hoy es el Océano Pacífico. Este ancestral mar de California, que poseía una rica vida marina, se extendía hacia el este hasta comunicarse con la antigua cuenca marina de la región del oeste medio.
\vs p060 1:9 \pc Hace \bibemph{140\,000\,000} de años, \bibemph{de repente} y con el único vestigio de los dos prerreptiles antecesores que se habían desarrollado en África durante la época anterior, aparecieron los reptiles como criaturas completas. Se desarrollaron con rapidez, dando pronto origen a los cocodrilos, a los reptiles con escamas y, finalmente, a las serpientes marinas y a los reptiles voladores. Sus ascendentes, que fueron animales de transición, desaparecieron rápidamente.
\vs p060 1:10 Estos dinosaurios reptiles, que evolucionaban velozmente, pronto se convirtieron en los reyes de esta era. Ponían huevos y se distinguían de todos los demás animales por sus pequeños cerebros que pesaban menos de medio kilo, y que usaban para controlar un cuerpo que luego llegaría a pesar cuarenta toneladas. Pero los primeros reptiles eran más pequeños y carnívoros, y caminaban sobre sus patas traseras al igual que los canguros. Sus huesos eran huecos como los de las aves y, posteriormente, desarrollaron solamente tres dedos en sus patas traseras; muchas de sus huellas fosilizadas se han confundido con las de las aves gigantes. Más tarde, evolucionaron los dinosaurios herbívoros, que caminaban sobre sus cuatro patas; una rama de este grupo desarrolló una coraza protectora.
\vs p060 1:11 Algunos millones de años más tarde, aparecieron los primeros mamíferos. Eran no placentarios, y pronto resultaron ser un fracaso; ninguno sobrevivió. Se trataba de una iniciativa de tipo experimental para mejorar los tipos de mamíferos, pero no prosperó en Urantia.
\vs p060 1:12 La vida marina de este período era escasa, pero se incrementó rápidamente con la nueva invasión del mar, que una vez más originó amplios litorales de aguas poco profundas. Puesto que había más aguas de poca profundidad alrededor de Europa y Asia, es en estos continentes donde se encuentran los lechos más ricos en fósiles. Hoy día, si queréis estudiar la vida de esa era, analizad las regiones del Himalaya, de Siberia y del Mediterráneo, así como la India y las islas de la cuenca del Pacífico sur. Un rasgo relevante de la vida marina era la presencia de grandes cantidades de hermosos amonites, cuyos restos fósiles se encuentran por todo el mundo.
\vs p060 1:13 \pc Hace \bibemph{130\,000\,000} de años, los mares habían cambiado muy poco. Siberia y América del Norte estaban comunicadas por el puente terrestre del Estrecho de Bering. Una vida marina, abundante y única, apareció en la costa californiana del Pacífico, donde se desarrollaron más de mil especies de amonites a partir de los tipos superiores de cefalópodos. De hecho, los cambios de vida en este período fueron innovadores, a pesar de ser transitorios y graduales.
\vs p060 1:14 \pc Este período engloba veinticinco millones de años y se le conoce como \bibemph{Triásico}.
\usection{2. LA POSTRERA ERA DE LOS REPTILES}
\vs p060 2:1 Hace \bibemph{120\,000\,000} de años, comenzó una nueva fase en la era de los reptiles. El gran acontecimiento de este período fue la evolución y el declive de los dinosaurios. En cuanto a tamaño, la vida animal terrestre había llegado a su desarrollo máximo y, hacia el final de esta era, había prácticamente desaparecido de la faz de la tierra. Los dinosaurios evolucionaron en todos los tamaños, desde una especie de unos sesenta centímetros de largo hasta los enormes dinosaurios no carnívoros de casi veintitrés metros de largo, una corpulencia jamás igualada desde entonces por criatura viva alguna.
\vs p060 2:2 El más grande de los dinosaurios se originó en el oeste de América del Norte. Estos gigantescos reptiles están enterrados en todas las regiones de las Montañas Rocosas, a lo largo de toda la costa atlántica de América del Norte, en Europa occidental, África del Sur y la India, aunque no en Australia.
\vs p060 2:3 Estas imponentes criaturas se volvieron menos activas y fuertes a medida que fueron creciendo en tamaño; pero necesitaban tal enorme cantidad de comida, y el suelo terrestre estaba tan repleto de ellos, que literalmente murieron por inanición y se extinguieron: Les faltó la inteligencia necesaria para hacer frente a esta situación.
\vs p060 2:4 Hacia ese tiempo, la mayoría de la parte oriental de América del Norte, que había estado elevada durante mucho tiempo, había bajado de nivel y había sido arrastrada por el agua hasta el Océano Atlántico, de manera que la costa se extendió varios cientos de kilómetros más lejos de lo que hoy está. La parte occidental del continente estaba todavía elevada, pero, más tarde, tanto el mar del norte como el Pacífico irrumpieron incluso en estas regiones, las cuales se extendieron hacia el este, hasta la región de Black Hills, en Dakota.
\vs p060 2:5 Se trataba de una era de agua dulce caracterizada por muchos lagos interiores, tal como se demuestra por los abundantes fósiles de agua dulce de los llamados lechos de Morrison en Colorado, Montana y Wyoming. El grosor de estos depósitos combinados de agua dulce y salada oscila entre 600 y 1500 metros; aunque hay muy poca piedra caliza presente en estas capas.
\vs p060 2:6 El mismo mar polar que avanzó tanto en dirección sur sobre América del Norte cubrió igualmente toda América del Sur, salvo la cordillera de los Andes que pronto haría su aparición. La mayoría de China y Rusia estaba inundada, pero la irrupción del agua fue más importante en Europa. Durante este sumergimiento, se sedimentó la hermosa piedra litográfica de Alemania meridional; se trata de unos estratos en los que fósiles, como las alas más delicadas de los antiguos insectos, se han preservado como si fuesen de ayer mismo.
\vs p060 2:7 La flora de esta era tenía muchas similitudes con la de la era anterior. Los helechos perduraban, en tanto que las coníferas y los pinos se parecían cada vez más a las variedades de hoy en día. Incluso se estaba formando algo de carbón a lo largo de las costas septentrionales del Mediterráneo.
\vs p060 2:8 La vuelta de los mares mejoró el clima. Los corales se extendieron por las aguas europeas, demostrando que el clima era todavía suave y uniforme, pero nunca volvieron a aparecer en los mares polares, que iban pausadamente enfriándose. La vida marina de estos tiempos incrementó y se desarrolló de forma considerable, especialmente en las aguas europeas. Tanto los corales como los crinoideos aparecieron temporalmente en un número mayor al de antes, pero los amonites dominaban la vida invertebrada de los océanos; su tamaño medio variaba entre más de siete centímetros y medio y diez centímetros, aunque una especie alcanzó un diámetro de casi dos metros y medio. Las esponjas estaban por todos lados, y tanto las sepias como las ostras continuaron con su evolución.
\vs p060 2:9 \pc Hace \bibemph{110\,000\,000} de años, el potencial de la vida marina continuaba desplegándose. El erizo de mar fue una de las excepcionales mutaciones de esta época. Los cangrejos, las langostas y otros tipos modernos de crustáceos se desarrollaron por completo. Se produjeron cambios relevantes en la familia de los peces; apareció por primera vez un tipo de esturión, pero las feroces serpientes de mar, descendientes de los reptiles terrestres, infestaban todos los mares todavía, amenazando con extinguir a toda la familia de los peces.
\vs p060 2:10 Aquella continuaba siendo, fundamentalmente, la era de los dinosaurios. Tomaron posesión del suelo terrestre de tal manera que, durante el anterior periodo de invasión marina, dos especies habían tenido que adaptarse al agua para subsistir. Estas serpientes de mar constituyen un paso atrás en la evolución. Mientras que algunas especies nuevas van progresando, determinadas variedades permanecen estacionarias y otras vuelven atrás, revertiendo a un estado anterior. Y esto es lo que sucedió cuando estos dos tipos de reptiles abandonaron el suelo terrestre.
\vs p060 2:11 A medida que transcurría el tiempo, las serpientes de mar alcanzaron un tamaño tal que se hicieron muy lentas, acabaron por perecer por no tener el cerebro lo suficientemente grande como para proteger sus enormes cuerpos. Su cerebro pesaba menos de cincuenta y seis gramos, a pesar de que estos enormes ictiosauros alcanzaban a veces quince metros de largo, y la mayoría sobrepasaba los diez metros y medio. Los cocodriloideos marinos también significaron una reversión del tipo de reptil terrestre; pero, a diferencia de las serpientes de mar, estos animales siempre volvían al suelo para poner sus huevos.
\vs p060 2:12 Poco después de que dos especies de dinosaurios emigraran al agua en un inútil intento de autopreservación, otros dos tipos se vieron movidos a buscar el aire debido a la implacable competencia por la vida existente en el suelo terrestre. Pero estos pterosaurios no fueron los predecesores de las verdaderas aves de las eras por llegar. Evolucionaron a partir de los dinosaurios saltadores de huesos huecos, y sus alas, semejantes a la de los murciélagos, alcanzaban una envergadura que oscilaba entre seis y siete metros y medio. Estos ancestrales reptiles voladores crecieron hasta tener tres metros de largo; sus mandíbulas eran movibles, muy parecidas a las de las culebras modernas. Durante algún tiempo, parece que tuvieron éxito, pero no lograron evolucionar de manera que pudieran sobrevivir como navegantes aéreos. Representan las variedades no supervivientes de los antecesores de las aves.
\vs p060 2:13 Las tortugas aumentaron su número durante este período, apareciendo en América del Norte por primera vez. Sus ancestros llegaron de Asia a través del puente terrestre septentrional.
\vs p060 2:14 \pc Hace cien millones de años, la era de los reptiles tocaba a su fin. En relación a su enorme masa, los dinosaurios eran animales prácticamente sin cerebro; les faltaba la inteligencia que los ayudara a conseguir la alimentación suficiente para cuerpos tan descomunales. Y de este modo, cada vez en mayor número, se fueron extinguiendo estos lentos reptiles de suelo terrestre. En adelante, la evolución emprendería el crecimiento de los cerebros y no el de la masa física, y el desarrollo de estos será un rasgo característico de la evolución animal y del progreso planetario de cada una de las épocas que seguirían.
\vs p060 2:15 \pc Este período, que abarcó el apogeo y el principio de la decadencia de los reptiles, se prolongó por casi veinticinco millones de años y se conoce como \bibemph{Jurásico}.
\usection{3. LA ETAPA CRETÁCICA. EL PERÍODO DE LAS PLANTAS FLORÍFERAS. LA ERA DE LAS AVES}
\vs p060 3:1 El gran período Cretácico deriva su nombre del predominio en los mares de los prolíficos foraminíferos, productores de creta. Este período aproxima a Urantia al final de la larga primacía de los reptiles y es testigo de la aparición en el suelo terrestre de las plantas floríferas y de la avifauna. Estos también son los tiempos en los que finaliza el desplazamiento de los continentes hacia el oeste y hacia el sur, que viene acompañado de formidables deformaciones de la corteza terrestre junto a erupciones generalizadas de lava y una gran actividad volcánica.
\vs p060 3:2 Cerca del final del previo período geológico, una gran parte del suelo continental estaba elevada por encima del agua, aunque, hasta ese momento, no había picos montañosos. Si bien, a medida que continuaba la deriva continental, esta se encontró, en el fondo profundo del Océano Pacífico, con su primer gran obstáculo. Esta oposición de fuerzas geológicas dio impulso a la formación de la totalidad de la inmensa cordillera septentrional y meridional, que se extiende desde Alaska, a través de México, hasta el Cabo de Hornos.
\vs p060 3:3 Por tanto, este período se convierte en la \bibemph{etapa de la formación de las montañas modernas} de la historia geológica. Con anterioridad a este tiempo, existían pocos picos; había simplemente cerros elevados de gran anchura. La cordillera de la costa del Pacífico empezó entonces a elevarse, pero estaba situada a 1100 kilómetros al oeste del actual litoral. Las Sierras empezaban a formarse; sus estratos de cuarzo, portadores de oro, son productos del flujo de lava de esta época. En la parte oriental de América del Norte, la presión del mar Atlántico operaba igualmente para provocar la elevación del suelo.
\vs p060 3:4 \pc Hace \bibemph{100\,000\,000} de años, el continente norteamericano y una parte de Europa se encontraban muy por encima del agua. La deformación de los continentes americanos continuaba, resultando en la metamorfosis de los Andes de América del Sur y en la gradual elevación de las planicies orientales de América del Norte. La mayoría de México se hundió bajo el mar, y el Atlántico meridional invadió la costa oriental de América del Sur, acabando por alcanzar al actual litoral. Los océanos Atlántico e Índico eran entonces prácticamente como lo son hoy.
\vs p060 3:5 \pc Hace \bibemph{95\,000\,000} de años, las masas de suelo de América y Europa empezaron a hundirse de nuevo. Los mares meridionales comenzaron a invadir América del Norte y, poco a poco, se extendieron hacia el norte hasta comunicarse con el Océano Ártico, lo que significó el segundo y gran sumergimiento del continente. Cuando este mar finalmente retrocedió, dejó el continente casi como actualmente es. Antes de iniciarse este gran sumergimiento, las altiplanicies de los Apalaches orientales se habían desgastado casi por completo hasta bajar al nivel del mar. Las capas policromas de arcilla pura usadas hoy en la fabricación de utensilios de barro se depositaron durante esta era en las regiones costeras del Atlántico; estas capas tienen un grosor de unos 600 metros.
\vs p060 3:6 Hubo una gran actividad volcánica en el sur de los Alpes y a lo largo de la actual cordillera costera de California. En México, ocurrieron las mayores deformaciones de la corteza terrestre habidas en muchos millones de años. Se produjeron igualmente grandes cambios en Europa, Rusia, Japón y en la parte meridional de América del Sur. El clima se volvió cada vez más diverso.
\vs p060 3:7 \pc Hace \bibemph{90\,000\,000} de años, las angiospermas surgieron de estos tempranos mares cretáceos y pronto irrumpieron en los continentes. Estas plantas terrestres aparecieron \bibemph{de repente} junto con las higueras, las magnolias y los tulipaneros. Poco tiempo después, las higueras, los árboles de pan y las palmeras se extendieron por Europa y por las planicies del oeste de América del Norte. Ningún nuevo animal terrestre hizo su aparición.
\vs p060 3:8 \pc Hace \bibemph{85\,000\,000} de años, el estrecho de Bering se cerró, aislando las aguas de los mares del norte en proceso de enfriamiento. Hasta ese momento, había habido una gran diferencia entre la vida marina de las aguas del Atlántico y del Golfo y las del Océano Pacífico, debido a las variaciones de temperatura de estas dos masas de agua, que se han vuelto ahora uniformes.
\vs p060 3:9 Los depósitos de creta y de marga de arenisca verde dan su nombre a este periodo. Las sedimentaciones de estos tiempos son jaspeadas; constan de creta, esquisto, arenisca y pequeñas cantidades de caliza, junto con carbón inferior o lignito y, en muchas regiones, de petróleo. Estas capas tienen un grosor que oscila entre 61 y 3000 metros en algunos lugares del oeste de América del Norte y en un buen número de localidades europeas. Estos depósitos caben observarse en las estribaciones inclinadas, a lo largo de los bordes orientales de las Montañas Rocosas.
\vs p060 3:10 En todo el mundo, estos estratos están cargados de creta, y estas capas de semirroca porosa recogen agua en los afloramientos inclinados y la transportan hacia abajo para abastecer de agua a una gran parte de las actuales regiones áridas de la Tierra.
\vs p060 3:11 \pc Hace \bibemph{80\,000\,000} de años ocurrieron grandes convulsiones en la corteza de la Tierra. El avance hacia el oeste de la deriva continental estaba alcanzando su punto muerto, y la inmensa energía contenida en el lento impulso de la masa continental hizo que se desplomara la costa del Pacífico de América del Norte y de América del Sur e inició las consecuentes modificaciones profundas a lo largo de las costas asiáticas del Pacífico. Esta elevación del suelo que circunda al Pacífico, que culminó en las cordilleras existentes hoy en día, tiene una longitud de más de cuarenta mil kilómetros. Y los levantamientos que acompañaron a su nacimiento resultaron ser las mayores deformaciones de la superficie jamás antes producidas desde que apareció la vida en Urantia. Los flujos de lava, tanto por encima de la tierra como por debajo, eran generalizados y de gran magnitud
\vs p060 3:12 \pc Hace \bibemph{75\,000\,000} de años se señala el fin de la deriva continental. Desde Alaska hasta el Cabo de Hornos, las cordilleras de la costa del Pacífico habían terminado de formarse, aunque hasta ese momento existían pocos picos de montaña.
\vs p060 3:13 El empuje de retroceso por la parada de la deriva continental continuó elevando las planicies occidentales de América del Norte, mientras que, en el este, los desgastados Montes Apalaches de la región costera atlántica se proyectaban verticalmente, con poco o ningún declive.
\vs p060 3:14 \pc Hace \bibemph{70\,000\,000} de años, se produjeron las deformaciones de la corteza terrestre relacionadas con la máxima elevación ocurrida en la región de las Montañas Rocosas. Un gran segmento de roca se solevantó veinticuatro kilómetros sobre la superficie de la Columbia Británica; aquí las rocas cámbricas avanzaron oblicuamente sobre las capas cretácicas. En la ladera oriental de las Montañas Rocosas, cercana a la frontera con Canadá, se dio otro espectacular solevantamiento; aquí se pueden encontrar las capas rocosas previas a la vida que se habían desplazado por encima de los entonces recientes depósitos cretácicos.
\vs p060 3:15 Aquella era una era de actividad volcánica generalizada en todo el mundo y que produjo un gran número de pequeños conos volcánicos aislados. En la región sumergida del Himalaya, estallaron volcanes submarinos. Una gran parte del resto de Asia, incluida Siberia, estaba también todavía por debajo del agua.
\vs p060 3:16 \pc Hace \bibemph{65\,000\,000} de años ocurrió uno de los más grandes flujos de lava de todos los tiempos. Las capas que estos y otros flujos de lava anteriores dejaron depositadas se pueden encontrar por todas las Américas, África del Norte y del Sur, Australia y partes de Europa.
\vs p060 3:17 Los animales terrestres habían cambiado poco, pero, debido a una mayor emersión continental, especialmente en América del Norte, se multiplicaron rápidamente. Estando la mayoría de Europa sumergida, América del Norte fue el gran entorno evolutivo de los animales terrestres de estos tiempos.
\vs p060 3:18 El clima seguía siendo aún cálido y uniforme. Las regiones árticas gozaban de un tiempo muy parecido al que hoy reina en América Central, Meridional y del Norte.
\vs p060 3:19 La vida vegetal estaba evolucionando a grandes pasos. Entre las plantas terrestres, predominaban las angiospermas y, por primera vez, aparecieron muchos árboles de los tiempos actuales, incluyendo hayas, abedules, robles, nogales, sicomoros, arces y palmeras modernas. Abundaban las frutas, las hierbas y los cereales, y estas hierbas y árboles con semillas fueron para el mundo vegetal lo que los ancestros del hombre fueron para el mundo animal ---su relevancia evolutiva quedó subordinada a la aparición del hombre mismo---. \bibemph{De repente} y sin gradación previa, la gran familia de plantas con flores apareció por mutación. Pronto, esta nueva flora se propagaría por todo el mundo.
\vs p060 3:20 \pc Hace \bibemph{60\,000\,000} de años, aunque los reptiles terrestres estaban en decadencia, los dinosaurios continuaron siendo los reyes del suelo terrestre. Pero pasaron a tomar la iniciativa los tipos más ágiles y activos de dinosaurios carnívoros, correspondientes a las variedades más pequeñas saltadoras como los canguros. Pero, previamente, en algún momento, habían aparecido nuevos tipos de dinosaurios herbívoros, cuyo rápido aumento se debió a la aparición de la familia herbácea de las plantas terrestres. Uno de estos nuevos dinosaurios herbívoros era un verdadero cuadrúpedo, con dos cuernos y un reborde en forma de capa sobre los hombros. Apareció un tipo de tortuga de suelo, de seis metros de ancho, al igual que un cocodrilo moderno y verdaderas serpientes del tipo moderno. Ocurrieron además grandes cambios entre los peces y entre otras formas de vida marina.
\vs p060 3:21 Las preaves zancudas y nadadoras de las eras anteriores no habían tenido éxito en el aire, como tampoco los dinosaurios voladores. Fueron especies de vida corta, que pronto se extinguieron. Sufrieron la misma fatalidad adversa de los dinosaurios ---su erradicación---, por tener demasiada poca sustancia cerebral en comparación con el tamaño de sus cuerpos. Este segundo intento de producir animales que pudieran surcar la atmósfera fracasó, al igual que la malograda tentativa de crear mamíferos durante esta era y la precedente.
\vs p060 3:22 \pc Hace 55\,000\,000 de años, la marcha evolutiva se vio marcada por la \bibemph{repentina} aparición de la primera de las aves auténticas, una pequeña criatura parecida a la paloma, que fue el ancestro de toda la avifauna. Se trataba del tercer tipo de criatura voladora que aparecía sobre la tierra, y surgió directamente del grupo de los reptiles, no del de sus coetáneos, los dinosaurios voladores, ni de los tipos tempranos de aves terrestres dentadas. Así pues, esta época llega a conocerse como la \bibemph{era de las aves} al igual que como la era del declive de los reptiles.
\usection{4. EL FIN DEL PERÍODO CRETÁCICO}
\vs p060 4:1 El gran período Cretácico iba llegando a su fin, lo que significaba la terminación de las grandes invasiones marinas de los continentes. Esto es particularmente cierto respecto a América del Norte, donde se habían producido exactamente veinticuatro grandes inundaciones. Y aunque se dieron con posterioridad sumergimientos menos importantes, ninguno de ellos era equiparable a las inmensas y prolongadas invasiones marinas de esta era y de otras anteriores. Estos períodos, en los que se alternaban respectivamente el dominio del mar y el del suelo terrestre, tuvieron lugar en ciclos de un millón de años cada uno. Ha existido un ritmo largamente asociado a esa emersión y sumergimiento del fondo oceánico y de los niveles del suelo continental. Y estos mismos movimientos rítmicos de la corteza terrestre continuarán desde este momento en adelante durante toda la historia de la Tierra, pero con menor frecuencia y magnitud.
\vs p060 4:2 En este periodo, se presencia también el final de la deriva continental y la formación de las montañas modernas de Urantia. Pero la presión de las masas continentales y el impulso que contrarresta su multisecular deriva no son los únicos elementos que influyen en la formación de las montañas. El factor principal y subyacente que determina la ubicación de una cordillera es la preexistencia de una tierra baja, o depresión, que se ha rellenado con los depósitos, relativamente más ligeros, de la erosión del suelo y del desplazamiento de los mares durante eras previas. Estas zonas más ligeras de suelo tienen a veces un grosor que oscila entre 4500 y 6000 metros; por consiguiente, cuando la corteza terrestre está sometida a una presión, sea cual fuese su causa, estas zonas más ligeras son las primeras en contraerse, plegarse y levantarse favoreciendo una adaptación compensatoria de las fuerzas y presiones en conflicto y contrapuestas que actúan en la corteza terrestre o bajo ella. En ocasiones, estos solevantamientos del suelo ocurren sin plegamientos. Si bien, en relación a la elevación de las Montañas Rocosas, se produjeron grandes plegamientos y declives, acompañados de enormes solevantamientos de las distintas capas, tanto subterráneas como en la superficie.
\vs p060 4:3 \pc Las montañas más antiguas del mundo están situadas en Asia, Groenlandia y Europa septentrional entre las de los más antiguos sistemas este\hyp{}oeste. Las montañas de edad intermedia se encuentran en el grupo que circunda el Pacífico y en el segundo sistema este\hyp{}oeste europeo, que nació prácticamente al mismo tiempo. Este gigantesco levantamiento tiene casi dieciséis mil kilómetros de largo, y se extiende desde Europa hasta las elevaciones de suelo de las Antillas. Las montañas más jóvenes se encuentran en el sistema de las Montañas Rocosas donde, durante eras, tales elevaciones se produjeron solo para ser cubiertas sucesivamente por el mar, aunque algunas de las tierras más altas permanecieron como islas. Tras la formación de las montañas de edad intermedia, se elevaron unas altiplanicies realmente montañosas, cuyo destino fue ser talladas por el arte en combinación de los elementos de la naturaleza, que le dieron forma como las Montañas Rocosas actuales.
\vs p060 4:4 La presente región de las Montañas Rocosas de América del Norte no es la elevación original del suelo; la erosión había allanado hacía mucho tiempo aquella elevación, para luego volver a elevarse. Hoy día, la cordillera frontal de las montañas es lo que queda de los restos de la cordillera original que resurgió. Los picos Pikes y Longs son excepcionales ejemplos de la actividad de estas montañas, que se han prolongado durante dos o más generaciones de las vidas de estas. Estos dos picos mantuvieron sus cumbres por encima del agua durante algunas de las anteriores inundaciones.
\vs p060 4:5 Biológica y geológicamente esta fue una era agitada y activa, sobre el suelo terrestre y bajo el agua. Los erizos de mar aumentaron, mientras que los corales y crinoideos disminuyeron. Los amonites, cuya influencia había sido predominante durante una era anterior, también declinaron de forma rápida. Sobre el suelo terrestre, los pinos y otros árboles modernos, incluyendo a las gigantescas secuoyas, reemplazaron en gran medida a los bosques de helechos. Hacia el fin de este período, aunque los mamíferos placentarios aún no han evolucionado, el escenario biológico está completamente listo para la aparición, en una era posterior, de los tempranos ancestros de los futuros tipos de mamíferos.
\vs p060 4:6 \pc Y de este modo termina una larga era en la evolución del mundo, que se extiende desde la temprana aparición de la vida terrestre hasta los tiempos más recientes de los ancestros inmediatos de la especie humana y sus ramas colaterales. Esta era, la \bibemph{Cretácica,} abarca cincuenta millones de años y pone fin a la era de vida terrestre previa a la de los mamíferos, que se prolonga durante un período de cien millones de años y se conoce como \bibemph{Mesozoica}.
\vsetoff
\vs p060 4:7 [Exposición de un portador de vida de Nebadón asignado a Satania, y que ejerce sus funciones actualmente en Urantia.]
