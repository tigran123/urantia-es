\upaper{99}{Los problemas sociales de la religión}
\author{Melquisedec}
\vs p099 0:1 La religión realiza su más alto ministerio social cuanto menor sea su relación con las instituciones laicas de la sociedad. En eras pasadas, puesto que las reformas sociales estaban en gran medida limitadas a los ámbitos morales, la religión no se veía obligada a adaptar su posición a los grandes cambios de los sistemas económicos y políticos. El problema principal de la religión radicaba en la tarea de reemplazar el mal con el bien dentro del orden social existente de cultura política y económica. Así pues, la religión tendía de forma indirecta a perpetuar el orden establecido de la sociedad, a promover el mantenimiento del tipo de civilización vigente.
\vs p099 0:2 Pero la religión no debe participar directamente en la creación de nuevos órdenes sociales ni en la preservación de los antiguos. La verdadera religión sí se opone a la violencia como modo de evolución social, pero no a las sesudas iniciativas de la sociedad por adaptar sus usos y adecuar sus instituciones a las nuevas condiciones económicas y a las exigencias culturales.
\vs p099 0:3 La religión aprobó las ocasionales reformas sociales de siglos pasados, pero en el siglo XX resulta imprescindible que haga frente y se adapte a una profunda y permanente reconstrucción social. Las condiciones de vida cambian con tanta rapidez que hay que acelerar considerablemente las modificaciones institucionales y la religión, en consecuencia, debe agilizar su adaptación a este orden social nuevo y siempre cambiante.
\usection{1. RELIGIÓN Y RECONSTRUCCIÓN SOCIAL}
\vs p099 1:1 Las invenciones mecánicas y la difusión del conocimiento están modificando la civilización; si se quiere evitar el desastre cultural, es imperativo realizar ciertos cambios sociales y adaptaciones económicas. Este nuevo orden social venidero no se establecerá complacientemente en un milenio. La raza humana debe acomodarse a una secuencia de cambios, adaptaciones y readaptaciones. La humanidad está marchando hacia un nuevo destino planetario no revelado.
\vs p099 1:2 \pc La religión debe convertirse en una poderosa influencia para la consecución de la estabilidad moral y el avance espiritual, obrando con dinamismo en medio de estas condiciones siempre cambiantes y de estas inacabables modificaciones económicas.
\vs p099 1:3 \pc La sociedad de Urantia nunca podrá aspirar a acomodarse como lo ha hecho en eras pasadas. El navío social ha zarpado de las resguardadas bahías de la tradición establecida y ha emprendido su crucero en los altos mares del destino evolutivo; y el alma del hombre, como nunca antes en la historia del mundo, necesita escudriñar con detenimiento sus mapas de moralidad y observar minuciosamente la brújula de la guía religiosa. Como elemento de influencia social, la misión primordial de la religión es estabilizar los ideales de la humanidad durante estos tiempos peligrosos de transición entre un estadio de civilización a otro, entre un nivel de cultura a otro.
\vs p099 1:4 La religión no tiene nuevos deberes que cumplir, pero está llamada de modo urgente a ser una guía sensata y una experta consejera ante todas estas situaciones nuevas y rápidamente cambiantes de la humanidad. La sociedad se está volviendo cada vez más mecánica, más compacta, más compleja y más críticamente interdependiente. La religión debe obrar para evitar que estas nuevas y estrechas correlaciones se conviertan en mutuamente regresivas o incluso destructivas. La religión debe actuar como sal cósmica que impida que los fermentos del progreso destruyan el sabor cultural de la civilización. Solo por medio del ministerio de la religión pueden estas nuevas relaciones sociales y convulsiones económicas dar lugar a una perdurable confraternidad.
\vs p099 1:5 Humanamente hablando, un humanitarismo sin Dios es un noble gesto, pero la verdadera religión es la única fuerza que puede incrementar perdurablemente la capacidad de respuesta de un grupo social a las necesidades y sufrimientos de otros grupos. En el pasado, la religión institucional podía permanecer pasiva mientras que los estratos superiores de la sociedad hacían caso omiso de los sufrimientos y de la opresión de los indefensos estratos inferiores, pero en los tiempos modernos estos órdenes sociales más humildes ya no son tan deplorablemente ignorantes ni tan políticamente indefensos.
\vs p099 1:6 La religión no debe involucrarse orgánicamente en la tarea laica de la reconstrucción social ni de la reorganización económica. Pero debe activamente mantenerse a la altura de estos avances de la civilización mediante la reafirmación inequívoca y enérgica de sus mandatos morales y preceptos espirituales, de su filosofía progresiva de la vida humana y de su suprema supervivencia. El espíritu de la religión es eterno, pero su forma de expresión debe reformularse cada vez que se revisa el diccionario del lenguaje humano.
\usection{2. DEBILIDAD DE LA RELIGIÓN INSTITUCIONAL}
\vs p099 2:1 La religión institucional no puede proveer inspiración ni ofrecer liderazgo en esta construcción social y reorganización económica inminente que se dará a nivel mundial, porque, lamentablemente, se ha convertido, en mayor o menor grado, en parte orgánica del orden social y del sistema económico destinados a ser reconstruidos. Solo la verdadera religión de la experiencia espiritual personal puede actuar de manera provechosa y creativa en la actual crisis de la civilización.
\vs p099 2:2 La religión institucional está atrapada en el estancamiento de un círculo vicioso. No puede reconstruir la sociedad si primero no se reconstruye a sí misma; y, siendo en tan gran medida una parte integral del orden establecido, no puede llevar esto a cabo hasta que la sociedad no sea objeto de una profunda reconstrucción.
\vs p099 2:3 \pc Los devotos religiosos deben actuar en la sociedad, en la industria y en la política de forma individual, no como grupos, partidos o instituciones. Una agrupación religiosa que pretenda obrar como tal, más allá de la actividad religiosa, se convierte de inmediato en un partido político, en una organización económica o en una institución social. El colectivismo religioso debe limitar sus iniciativas a la promoción de las causas religiosas.
\vs p099 2:4 Los creyentes no son de mayor valor en las labores de la reconstrucción social que los que no lo son, salvo en la medida en la que su religión les haya conferido una mayor percepción cósmica y los haya dotado de esa sabiduría social superior, que nace del deseo sincero de amar a Dios de forma suprema y de amar a cada hombre como a un hermano en el reino celestial. Un orden social ideal es aquel en el que cada hombre ama a su prójimo como se ama a sí mismo.
\vs p099 2:5 \pc En el pasado, la Iglesia institucionalizada ha podido dar la impresión de haber servido a la sociedad, glorificando los órdenes políticos y económicos establecidos, pero, si ha de sobrevivir, debe poner fin con celeridad a tal forma de proceder. Su única y adecuada actitud ha de consistir en impartir la no violencia, la doctrina de la evolución pacífica en lugar de la revolución violenta ---paz en la tierra y buena voluntad entre todos los hombres---.
\vs p099 2:6 A la religión moderna le resulta difícil adaptar su actitud respecto a la rápida evolución de los cambios sociales únicamente porque se ha permitido a sí misma volverse profundamente tradicionalista, dogmática e institucionalizada. Para la religión de la experiencia viva no existen inconvenientes en mantenerse en primera línea de todos los desarrollos sociales y convulsiones económicas, en medio de los que obra siempre como estabilizadora moral, guía social y piloto espiritual. La religión verdadera transfiere de una era a la otra esa cultura meritoria y aquella sabiduría que nacen de la experiencia de conocer a Dios y de querer ser como él.
\usection{3. LA RELIGIÓN Y EL DEVOTO RELIGIOSO}
\vs p099 3:1 El cristianismo primitivo estaba enteramente exento de cualquier implicación civil, compromiso social y alianza económica. Solo el posterior cristianismo institucionalizado se convirtió en parte orgánica del entramado político y social de la civilización occidental.
\vs p099 3:2 \pc El reino de los cielos no es un orden social ni económico; es una confraternidad exclusivamente espiritual de personas conocedoras de Dios. Es cierto, tal confraternidad es en sí misma un nuevo y asombroso fenómeno social que conlleva unas consecuencias políticas y económicas sorprendentes.
\vs p099 3:3 La persona religiosa no es indiferente al sufrimiento social ni desconsiderado hacia la injusticia civil ni está aislado del pensamiento económico, como tampoco es insensible a la tiranía política. La religión influye directamente sobre la reconstrucción social porque espiritualiza y enaltece al ciudadano individual. Indirectamente, la civilización cultural está influenciada por la actitud de estos creyentes individuales a medida que se convierten en miembros activos y prominentes de los distintos grupos sociales, morales, económicos y políticos.
\vs p099 3:4 \pc Lograr una civilización cultural elevada exige, primeramente, el tipo ideal de ciudadano y, luego, unos mecanismos sociales idóneos y adecuados con los que dicha ciudadanía pueda regir las instituciones económicas y políticas de una sociedad humana de tal avance.
\vs p099 3:5 La Iglesia, debido a un sentimiento extremadamente equivocado, ha asistido durante mucho tiempo a los desfavorecidos y a los desafortunados, y todo eso ha estado muy bien, pero este mismo sentimiento ha llevado a la insensata perpetuación de unos linajes racialmente en declive degenerativo, que han retrasado enormemente el progreso de la civilización.
\vs p099 3:6 De forma individual, muchos reconstruccionistas sociales, aunque rechacen enérgicamente la religión institucionalizada, son, al fin y al cabo, celosamente religiosos en lo que respecta a la propagación de sus reformas sociales. Y, así, la motivación religiosa, personal y más o menos no reconocida, está desempeñando un papel importante en la agenda actual de la reconstrucción social.
\vs p099 3:7 \pc El gran defecto de todo este tipo de actividad religiosa no reconocida e inconsciente estriba en que no es capaz de beneficiarse de la crítica religiosa honesta con la que obtendría niveles provechosos de autocorrección. Es un hecho que la religión no crece a menos que esté regulada por la crítica constructiva, reforzada por la filosofía, purificada por la ciencia y nutrida por una fiel comunidad.
\vs p099 3:8 Siempre existe el gran peligro de que la religión se desvirtúe y envilezca al buscar metas erróneas, como cuando, en tiempos de guerra, cada nación enfrentada degrada su religión y la pone al servicio de la propaganda militar. El fervor falto de amor es siempre dañino para la religión, mientras que la persecución hace desviar la actividad de la religión, impulsándola hacia la consecución de algún objetivo sociológico o teológico.
\vs p099 3:9 \pc La religión puede mantenerse libre de alianzas laicas y profanas únicamente mediante:
\vs p099 3:10 \li{1.}Una filosofía críticamente correctiva.
\vs p099 3:11 \li{2.}La libertad de cualquier alianza social, económica y política.
\vs p099 3:12 \li{3.}Unas comunidades creativas, que traigan consuelo y engrandezcan y hagan crecer el amor.
\vs p099 3:13 \li{4.}La mejora progresiva de la percepción espiritual y de la apreciación de los valores cósmicos.
\vs p099 3:14 \li{5.}La prevención del fanatismo por las aportaciones de una actitud mental científica.
\vs p099 3:15 \pc Los devotos religiosos, como grupo, no deben implicarse en nada que no sea la \bibemph{religión,} pese a que cualquier creyente, como ciudadano individual, pueda convertirse en el destacado líder de algún movimiento de reconstrucción social, económico o político.
\vs p099 3:16 La religión tiene la misión de crear, sostener e inspirar en el ciudadano individual una lealtad cósmica que lo guíe al logro del éxito en el avance de todos estos servicios sociales difíciles, aunque deseables.
\usection{4. DIFICULTADES DE LA TRANSICIÓN}
\vs p099 4:1 La auténtica religión hace que el devoto religioso sea socialmente fragante y cree en él una mejor consciencia de la fraternidad del hombre. Pero, en numerosas ocasiones, la oficialización de los grupos religiosos acaba con esos mismos valores para cuyo fomento se organizaron. La amistad humana y la religión divina se ayudan mutuamente y son significativamente reveladoras si el crecimiento de cada cual es equiparado y está armonizado. La religión da nuevo sentido a todas las relaciones de grupo: familias, escuelas y clubes. Imparte nuevos valores a lo lúdico y exalta el verdadero humor.
\vs p099 4:2 La percepción espiritual transforma el liderazgo social; la religión impide que cualquier movimiento colectivo pierda de vista sus verdaderos objetivos. Junto con los niños, la religión es la gran unificadora de la vida familiar, a condición de que sea una fe viva y creciente. La vida familiar no puede vivirse sin niños; sí se puede vivir sin religión, pero tal desventaja multiplica enormemente las dificultades de esta íntima relación humana. Durante las primeras décadas del siglo XX, la vida familiar, después de la experiencia religiosa personal, es la que más se ve afectada por la decadencia resultante de la transición entre las antiguas lealtades religiosas y los incipientes nuevos contenidos y valores.
\vs p099 4:3 \pc La verdadera religión es una forma significativa de vivir, en la que se afrontan con dinamismo las realidades habituales de la vida diaria. Pero si la religión ha de estimular el desarrollo individual del carácter e incrementar la integración de la persona, no debe ser estandarizada. Si ha de estimular la evaluación de las vivencias y servir como un valor\hyp{}aliciente, no debe ser estereotipada. Si la religión ha de promover lealtades supremas, no debe ser formalizada.
\vs p099 4:4 Sin importar las convulsiones que puedan estar presentes en el crecimiento económico de la civilización, la religión es genuina y válida si fomenta en las personas unas vivencias en las que predomine la soberanía de la verdad, la belleza y la bondad, ya que ello es el verdadero concepto espiritual de la realidad suprema. Y, a través del amor y de la adoración, esto cobra significado bajo la forma de fraternidad con el hombre y filiación con Dios.
\vs p099 4:5 Al fin y al cabo, se trata de lo que uno cree en lugar de lo que uno sabe lo que determina la conducta y rige la actuación de la persona. El conocimiento netamente fáctico ejerce muy poca influencia sobre el hombre corriente, a menos que este llegue a activarse emocionalmente. Pero la activación de la religión es supraemocional, unificando por completo las vivencias humanas en unos niveles supremos mediante el contacto con las energías espirituales y la liberación de estas en la vida del mortal.
\vs p099 4:6 \pc Durante los tiempos psicológicamente inestables del siglo XX, en medio de las convulsiones económicas, las contracorrientes morales y las marejadas sociológicas por las transiciones tempestuosas de la era científica, miles y miles de hombres y mujeres se han sentido humanamente desplazados; están ansiosos, agitados, temerosos, inciertos e inquietos; como nunca antes en la historia del mundo, necesitan el consuelo y la estabilización de una religión sólida. En esta situación de logros científicos y desarrollos mecánicos sin precedentes, hay estancamiento espiritual y caos filosófico.
\vs p099 4:7 \pc No existe peligro alguno en que la religión se convierta cada vez más en un asunto privado ---una experiencia personal--- siempre que esta no pierda su motivación hacia el servicio social altruista y amoroso. La religión se ha visto afectada por muchas influencias secundarias: una mezcla repentina de culturas, la amalgama de credos, la disminución de la autoridad eclesiástica, el cambio de la vida familiar, junto con la urbanización y la mecanización.
\vs p099 4:8 El mayor peligro espiritual del hombre consiste en el avance parcial, la difícil situación de un crecimiento inacabado: renunciar a las religiones evolutivas del temor sin abrazar de inmediato la religión revelada del amor. La ciencia moderna, en particular la psicología, solo ha debilitado aquellas religiones que tanto dependen del temor, la superstición y la emoción.
\vs p099 4:9 La transición siempre va acompañada de confusión, y existirá poca tranquilidad en el mundo religioso hasta que no concluya la gran lucha entre las tres filosofías de la religión enfrentadas:
\vs p099 4:10 \li{1.}La creencia espiritualista (en una Deidad providencial) de numerosas religiones.
\vs p099 4:11 \li{2.}La creencia humanista e idealista de numerosas filosofías.
\vs p099 4:12 \li{3.}Las concepciones mecanicistas y naturalistas de numerosas ciencias.
\vs p099 4:13 \pc Y estos tres planteamientos parciales respecto a la realidad del cosmos deben llegar, con el tiempo, a armonizarse mediante la exposición revelada de la religión, la filosofía y la cosmología, que muestran la existencia trina del espíritu, la mente y la energía provenientes de la Trinidad del Paraíso y que logran su unificación espacio\hyp{}temporal en la Deidad del Supremo.
\usection{5. ASPECTOS SOCIALES DE LA RELIGIÓN}
\vs p099 5:1 Aunque la religión sea exclusivamente una experiencia espiritual y personal ---conocer a Dios como Padre---, el resultado de esta experiencia ---conocer al hombre como hermano--- entraña la adaptación del yo a otros yos, y en eso estriba el aspecto social o grupal de la vida religiosa. La religión es primero una adaptación interior o personal, y luego se convierte en una cuestión de servicio social o de adaptación al grupo. El hecho del gregarismo del hombre determina forzosamente que empezarán a existir grupos religiosos. Lo que le suceda a tales grupos depende, en gran medida, de la inteligencia de sus líderes. En la sociedad primitiva, el grupo religioso no es siempre muy diferente de los económicos o políticos. La religión siempre ha preservado la moral y ha estabilizado la sociedad. Y esto sigue siendo cierto, a pesar de las enseñanzas contrarias de muchos socialistas y humanistas modernos.
\vs p099 5:2 Tened siempre presente: la verdadera religión consiste en conocer a Dios como Padre y al hombre como hermano. La religión no consiste en creer sumisamente en las amenazas del castigo ni en las mágicas promesas de recompensas místicas futuras.
\vs p099 5:3 \pc La religión de Jesús constituye la influencia más dinámica que haya movilizado jamás a la raza humana. Jesús hizo trizas las tradiciones, derribó los dogmas y llamó a la humanidad a lograr sus ideales más elevados en el tiempo y en la eternidad ---ser perfectos, tal como el Padre de los cielos es perfecto---.
\vs p099 5:4 \pc La religión tiene pocas posibilidades de obrar hasta que el grupo religioso no se separe de todos los otros grupos y forme la unión social de la membrecía espiritual del reino de los cielos.
\vs p099 5:5 La doctrina de la depravación total del hombre destruyó gran parte del potencial de la religión para producir unas repercusiones sociales de carácter edificante, cuyos valores sirvieran de inspiración. Jesús trató de restablecer la dignidad del hombre cuando declaró que todos los hombres son los hijos de Dios.
\vs p099 5:6 Cualquier convicción religiosa que sea eficaz en cuanto a la espiritualización del devoto religioso tendrá sin duda un impacto poderoso en la vida social de este. La experiencia religiosa produce invariablemente los “frutos del espíritu” en la vida diaria de los mortales a quienes el espíritu guía.
\vs p099 5:7 De la misma manera que es cierto que los hombres comparten sus creencias religiosas, lo es igualmente que constituyen algún tipo de grupo religioso que con el tiempo establece objetivos comunes. Algún día los creyentes se reunirán y llevarán a cabo una auténtica colaboración sobre la base de la unidad de ideales y objetivos, en lugar de tratar de hacerlo en función de opiniones psicológicas y convicciones teológicas. Más que los credos, son los objetivos los que deberían unificar a los creyentes. Ya que la verdadera religión es una cuestión de experiencia personal y espiritual, es inevitable que cada creyente tenga de forma individual su propia interpretación sobre la forma de llevar a efecto su experiencia espiritual. Dejad que el término “fe” represente la relación de la persona con Dios más que la elaboración de un credo sobre lo que un grupo de mortales haya podido convenir como actitud religiosa común. “¿Tienes tú fe? Tenla entonces para ti mismo”.
\vs p099 5:8 Que la fe tenga únicamente por objeto alcanzar unos valores ideales queda demostrado en la definición del Nuevo Testamento, que proclama que la fe es la certeza de lo que se espera y la convicción de lo que no se ve.
\vs p099 5:9 El hombre primitivo se esforzó poco por articular con palabras sus creencias religiosas. En su religión se danzaba más que se pensaba. El hombre moderno ha elaborado muchos credos y ha elaborado numerosas pruebas de su fe religiosa. El futuro creyente religioso ha de vivir su religión, dedicarse al servicio incondicional de la hermandad del hombre. Ya es hora de que el hombre viva la religión de forma tan personal y sublime que solo se pueda hacer realidad y expresar mediante “sentimientos que yacen demasiado profundos como para decirse con palabras”.
\vs p099 5:10 Jesús no exigió a sus seguidores que se congregasen periódicamente para recitar una serie de palabras que mostraran unas creencias comunes. Él solo mandó que se reunieran para \bibemph{hacer algo} en concreto: participar todos de la cena comunitaria en memoria de su vida de gracia en Urantia.
\vs p099 5:11 \pc ¡Qué error cometen los cristianos cuando, al presentar a Cristo como el supremo ideal de liderazgo espiritual, se atreven a exigir a hombres y mujeres con conciencia de Dios que rechacen el liderazgo histórico de personas conocedoras de Dios, que han contribuido a su particular enaltecimiento nacional o racial durante épocas pasadas!
\usection{6. LA RELIGIÓN INSTITUCIONAL}
\vs p099 6:1 El sectarismo es una enfermedad de la religión institucional, el dogmatismo inhibe la naturaleza espiritual. Es mucho mejor tener una religión sin Iglesia que una Iglesia sin religión. La agitación religiosa del siglo XX no entraña, en sí misma y por sí misma, decadencia espiritual. La confusión precede al crecimiento al igual que a la destrucción.
\vs p099 6:2 Existe un genuino propósito en la socialización de la religión. Las actividades religiosas de carácter grupal tienen por objeto escenificar las lealtades de la religión; intensificar el aliciente por la verdad, la belleza y la bondad; fomentar el atractivo de los valores supremos; enaltecer el servicio de una fraternidad solidaria; glorificar los potenciales de vida de la familia; promover la educación religiosa; ofrecer sabios consejos y guía espiritual; y alentar a la adoración en grupo. Y todas las religiones vivas estimulan la amistad humana, conservan la moral, favorecen el bienestar de su prójimo y facilitan la diseminación del evangelio fundamental de sus respectivos mensajes de salvación eterna.
\vs p099 6:3 Pero cuando la religión se institucionaliza, queda restringido su poder para hacer el bien, mientras que se multiplican enormemente sus posibilidades de hacer el mal. Los peligros de una religión formalizada son: la fijación de las creencias y la cristalización de los sentimientos; la acumulación de intereses creados y el aumento de la laicidad; la tendencia a normalizar y fosilizar la verdad; la desviación de la religión del servicio a Dios al servicio a la Iglesia; la inclinación de los líderes a volverse administradores en lugar de servidores; la tendencia a formar confesiones y escisiones discrepantes; el establecimiento de una autoridad eclesiástica opresora; la creación de una actitud aristocrática de “pueblo elegido”; el fomento de ideas falsas y exageradas de lo sagrado; la transformación de la religión en rutina y la petrificación de la adoración; la propensión a venerar el pasado, mientras se ignoran las necesidades del presente; la incapacidad de hacer interpretaciones actualizadas de la religión; el conflicto con las funciones de las instituciones laicas. La religión institucionalizada crea la funesta discriminación de las castas religiosas; se convierte en juez intolerante de la ortodoxia; no consigue mantener el interés de jóvenes intrépidos y pierde paulatinamente el mensaje salvador del evangelio de la salvación eterna.
\vs p099 6:4 La religión formal limita a los hombres en su actividad espiritual personal en lugar de liberarlos para que realicen un mejor servicio como constructores del reino.
\usection{7. CONTRIBUCIÓN DE LA RELIGIÓN}
\vs p099 7:1 Aunque las iglesias y otros grupos religiosos deberían mantenerse al margen de cualquier actividad laica, al mismo tiempo, la religión no debe hacer nada que dificulte o demore la coordinación social de las instituciones humanas. La vida debe seguir creciendo con nuevos contenidos; el hombre ha de continuar reformando la filosofía y clarificando la religión.
\vs p099 7:2 La ciencia política debe realizar la reconstrucción de la economía y de la industria mediante los métodos que aprende de las ciencias sociales y con la ayuda de las ideas y las motivaciones que la vida religiosa facilita. En cualquier reconstrucción social, la religión proporciona una lealtad estabilizante hacia un fin supremo, una meta estable por encima y más allá de objetivos inmediatos y temporales. En medio de las confusiones de un entorno rápidamente cambiante, el hombre mortal necesita el sostén de una extensa percepción cósmica.
\vs p099 7:3 La religión inspira al hombre a vivir con valentía y deleite sobre la faz de la tierra; enlaza la paciencia con la pasión, la percepción con el entusiasmo, la comprensión con el poder y los ideales con la actividad.
\vs p099 7:4 El hombre nunca puede decidir certeramente sobre cuestiones temporales ni trascender el egoísmo de sus intereses personales a no ser que medite en presencia de la soberanía de Dios y reconozca las realidades de los contenidos divinos y de los valores espirituales.
\vs p099 7:5 En última instancia, la interdependencia económica y la fraternidad social conducirán a la hermandad entre todos. El hombre es por naturaleza un soñador, pero la ciencia lo está moderando de manera que la religión pueda pronto estimularlo con mucho menos riesgo de provocar en él reacciones extremadamente entusiastas. Las necesidades económicas hacen que el hombre se conecte con la realidad, y la experiencia religiosa personal lleva a ese mismo hombre a afrontar las realidades eternas de una ciudadanía cósmica siempre en continua expansión y progreso.
\vsetoff
\vs p099 7:6 [Exposición de un melquisedec de Nebadón.]
