\upaper{179}{La última cena}
\author{Comisión de seres intermedios}
\vs p179 0:1 Durante la tarde de ese jueves, cuando Felipe recordó al Maestro la proximidad de la Pascua y quiso saber sobre sus planes para celebrarla, él tenía en mente la cena de Pascua, que debía comerse la noche del día siguiente, viernes. Era costumbre comenzar los preparativos para esta celebración no más tarde del mediodía del día anterior. Y puesto que los judíos contaban el comienzo del día con la puesta de sol, esto significaba que la cena de Pascua del sábado se haría el viernes, en algún momento antes de la medianoche.
\vs p179 0:2 En este sentido, los apóstoles estaban completamente perplejos; no podían entender el anuncio del Maestro de que celebrarían la Pascua un día antes. Al menos, algunos de ellos pensaban que Jesús era consciente de que lo arrestarían antes de la cena pascual del viernes por la noche y que, en consecuencia, los reunía para tener una cena especial todos juntos la noche de aquel jueves. Otros creían que se trataba simplemente de una singular ocasión de estar juntos, con anterioridad a la celebración formal de la Pascua.
\vs p179 0:3 Los apóstoles sabían que Jesús había celebrado otras Pascuas sin cordero; sabían que no participaba personalmente en ningún rito sacrificial del sistema judío. Muchas veces, había tomado parte como invitado en una comida en la que se servía el cordero pascual, pero, cuando él era el anfitrión, nunca era así. Para los apóstoles, no habría significado una gran sorpresa que no hubiera cordero ni incluso la noche de Pascua y, dado que aquella cena tenía lugar un día antes, no les preocupó aquello.
\vs p179 0:4 Después de los saludos de bienvenida por parte del padre y de la madre de Juan Marcos, los apóstoles subieron de inmediato al aposento alto, mientras Jesús se quedaba atrás para charlar con la familia de Marcos.
\vs p179 0:5 Previamente, se había acordado que el Maestro conmemoraría aquella ocasión a solas con sus doce apóstoles; así pues, no estaba previsto que ningún sirviente los atendiera.
\usection{1. EL AFÁN DE SER EL FAVORITO}
\vs p179 1:1 Cuando Juan Marcos los llevó escaleras arriba, contemplaron un aposento amplio y cómodo, totalmente listo para la cena, y observaron que el pan, el vino, el agua y las hierbas estaban dispuestos para su consumo en un extremo de la mesa. Excepto por el lado en el que estaban colocados el pan y el vino, esta larga mesa estaba rodeada por trece divanes reclinables, los mismos que se empleaban para la celebración de la Pascua en las casas judías acomodadas.
\vs p179 1:2 Al entrar en la sala de arriba, los doce notaron, justamente al lado de la puerta, los cántaros de agua, las vasijas y las toallas para el lavado de sus pies polvorientos; y, puesto que no tenían sirvientes que los ayudaran en esto, en cuanto Juan Marcos los dejó, los apóstoles empezaron a mirase entre ellos, cada cual pensando para sus adentros, ¿quién lavará nuestros pies? Y, de igual manera, cada uno pensó que no sería él quien hiciera de sirviente de los demás.
\vs p179 1:3 Mientras estaban allí, de pie, debatiéndose en sus corazones, observaron cómo estaban organizados los asientos en la mesa, advirtiendo que el diván de más estatus, reservado al anfitrión, tenía un lecho a la derecha, mientras que los otros once se disponían alrededor de la mesa llegando hasta el frente de este segundo asiento de honor situado a la derecha del anfitrión.
\vs p179 1:4 Esperaban que el Maestro viniese en cualquier momento, pero tenían ante sí el dilema de si debían sentarse o esperar su llegada y que fuera él quien les asignara sus lugares. Mientras estaban en aquella duda, Judas se dirigió al asiento de honor, a la izquierda del anfitrión, e indicó que tenía la intención de reclinarse allí como invitado favorito. Aquel acto de Judas provocó enseguida una acalorada disputa entre los demás apóstoles. Y, no había acabado Judas de tomar el asiento de honor, cuando Juan Zebedeo reclamó para sí el siguiente asiento preferente, el que estaba a la derecha del anfitrión. A Simón Pedro le airó tanto que Judas y Juan se atreviesen a ocupar aquellas posiciones privilegiadas que, ante la mirada enojada de los demás apóstoles, caminó con decisión alrededor de la mesa y se situó en el diván de menor estatus, al final de la fila de asientos y justo frente al elegido por Juan Zebedeo. Puesto que otros habían ocupado los asientos de mayor estatus, Pedro pensó en escoger el de menor rango, e hizo aquello, no meramente como protesta contra el indecoroso orgullo de sus hermanos, sino con la esperanza de que Jesús, cuando llegara y lo viera allí, lo llamara a uno de mayor rango, moviendo así de su sitio a quien se había atribuido a sí mismo tal honor.
\vs p179 1:5 Con las posiciones de mayor y menor estatus ya ocupadas, el resto de los apóstoles seleccionaron sus lugares, algunos se situaron cerca de Judas y, otros, de Pedro, hasta que todos estaban colocados en sus respectivos asientos. Se sentaron alrededor de la mesa en forma de U, en estos divanes, en el siguiente orden: a la derecha del Maestro estaba Juan; a la izquierda estaba Judas, Simón Zelotes, Mateo, Santiago Zebedeo, Andrés, los gemelos Alfeo, Felipe, Natanael, Tomás y Simón Pedro.
\vs p179 1:6 \pc Allí están, todos reunidos para celebrar, al menos en espíritu, una institución que antecedía incluso a Moisés y que remontaba a los tiempos en las que sus padres eran esclavos en Egipto. Esta cena es su último encuentro con Jesús, y hasta en dicha solemne ocasión, los apóstoles, liderados por Judas, se dejan llevar una vez más por su antigua propensión hacia el honor, el favoritismo y el enaltecimiento personal.
\vs p179 1:7 \pc Aún se hacían furiosos reproches entre ellos, cuando el Maestro apareció en la entrada de la sala y se quedó vacilante un momento, mientras que, lentamente, se fue apoderándose de su rostro un gesto de decepción. Sin comentarios, se dirigió a su sitio, y no alteró la disposición de sus asientos.
\vs p179 1:8 Ya estaban entonces listos para empezar la cena, salvo que sus pies estaban aún sin lavar, y se encontraban en un estado mental muy poco placentero. Cuando el Maestro llegó, continuaban todavía con sus comentarios despreciativos unos a otros, por no hablar de los pensamientos de algunos de ellos que tenían el suficiente dominio emocional como para refrenarse de expresar públicamente sus sentimientos.
\usection{2. COMIENZA LA CENA}
\vs p179 2:1 No se habló ni una sola palabra hasta trascurridos unos momentos desde que el Maestro ocupara su sitio. Jesús los miró a todos y, aliviando la tensión con una sonrisa, dijo: “Sentía grandes deseos de comer esta Pascua con vosotros. Quería tener una comida todos juntos una vez más antes de mi padecimiento. Sabiendo que mi hora ha llegado, lo preparé todo para cenar con vosotros esta noche, ya que, en cuanto al mañana, todos estamos en las manos del Padre, cuya voluntad he venido a cumplir. No volveré a comer con vosotros hasta que no os sentéis conmigo en el reino que mi Padre me dará cuando haya terminado aquello para lo que él me envió a este mundo”.
\vs p179 2:2 Tras mezclar el vino y el agua, le trajeron la copa a Jesús, quien, cuando la recibió de manos de Tadeo, la sostuvo dando gracias. Y cuando terminó de dar gracias, dijo: “Tomad esta copa y repartidla entre vosotros, y cuando seáis partícipes de ella, sabed que no beberé más con vosotros del fruto de la vid, que esta es nuestra última cena. Cuando nuevamente nos sentemos así, será en el reino que ha de venir”.
\vs p179 2:3 Jesús comenzó a hablar de estas cosas con sus apóstoles porque se daba cuenta de que su hora se acercaba. Comprendía que había llegado el momento de regresar al Padre, y que su labor en la tierra había casi concluido. El Maestro sabía que había revelado el amor del Padre en la tierra y había mostrado su misericordia a la humanidad, y que había finalizado aquello para lo que había venido al mundo, incluso hasta recibir todo poder y autoridad en el cielo y en la tierra. Igualmente conocía el hecho de que Judas Iscariote ya había tomado por completo la decisión de entregarlo aquella noche a manos de sus enemigos. Era totalmente consciente de que aquella desleal traición era un acto de Judas, pero que también complacía a Lucifer, a Satanás y a Caligastia, el príncipe de las tinieblas. Pero no temía a ninguno de aquellos que buscaban su derrumbe espiritual ni a quienes procuraban su muerte física. El Maestro solo tenía una preocupación, y era la seguridad y la salvación de los seguidores que él había elegido. Y así pues, con el pleno conocimiento de que el Padre había puesto todas las cosas bajo su autoridad, el Maestro se preparaba ahora para llevar a cabo la parábola del amor fraternal.
\usection{3. EL LAVADO DE LOS PIES DE LOS APÓSTOLES}
\vs p179 3:1 Después de beber la primera copa de la Pascua, la costumbre judía era que el anfitrión se levantara de la mesa y se lavara las manos. Más tarde, en la comida, y tras la segunda copa, todos los invitados se levantaban también y se lavaban las manos. Dado que los apóstoles sabían que su Maestro nunca observaba estos ritos del lavado ceremonial de las manos, tenían mucha curiosidad por saber qué haría cuando, tras haber todos compartido esta primera copa, se levantó de la mesa y en silencio se encaminó hasta cerca de la puerta, donde se habían colocado los cántaros de agua, las vasijas y las toallas. Y su curiosidad se transformó en asombro cuando vieron que el Maestro se quitó su manto, se ciñó una toalla y comenzaba a echar agua en uno de los cuencos para los pies. Imaginad la sorpresa de estos doce hombres, que no hacía mucho se habían negado a lavarse los pies unos a otros, y que se habían ensalzado en disputas tan poco decorosas sobre los sitios de honor en la mesa, cuando lo vieron dirigirse, rodeando el lado desocupado de la mesa, hasta el asiento de menos estatus de la cena, en el que se reclinaba Simón Pedro y, arrodillándose como si fuese un sirviente, se preparó para lavarle a él los pies. Cuando el Maestro hizo aquello, los doce se pusieron de pie como un solo hombre; incluso el traidor Judas se olvidó por un momento tanto de su infamia que se levantó junto con sus compañeros apóstoles en esta manifestación de sorpresa, respeto y completo asombro.
\vs p179 3:2 Allí estaba Simón Pedro de pie, mirando hacia abajo, al rostro de su Maestro, que alzaba su mirada para mirarlo a él. Jesús no dijo nada; no era necesario que lo hiciera. Su actitud revelaba claramente que tenía la intención de lavarle los pies a Simón Pedro. Pese a sus fragilidades humanas, Pedro amaba al Maestro. Este pescador galileo fue el primer ser humano en creer incondicionalmente en la divinidad de Jesús \bibemph{y} en hacer una confesión plena y pública de esa fe. Y, desde ese momento, Pedro nunca había dudado realmente de la naturaleza divina del Maestro. Puesto que Pedro veneraba y honraba tanto a Jesús en su corazón, no era extraño que le afectase en su alma el pensamiento de ver a Jesús arrodillado allí ante él, con la actitud de un sirviente ordinario, disponiéndose a lavarle los pies como lo haría un esclavo. Cuando Pedro se recompuso lo suficiente como para dirigirse al Maestro, habló, expresando las profundas emociones que todos sus compañeros apóstoles sentían.
\vs p179 3:3 Tras algunos momentos en los que se sintieron muy turbados, Pedro dijo: “Maestro, ¿realmente vas a lavarme los pies?”. Y, entonces, levantando la mirada al rostro de Pedro, Jesús dijo: “Quizás no entiendas por completo lo que voy a hacer ahora; pero conocerás después el significado de todas estas cosas”. Entonces Simón Pedro, respirando hondo, dijo: “Maestro, ¡no me lavarás los pies jamás!”. Y cada uno de los apóstoles asintió dando su aprobación a las firmes palabras de Pedro en su negativa a permitir que Jesús se humillara ante ellos, de esa manera.
\vs p179 3:4 El corazón de Judas Iscariote se vio en un principio movido por la emoción de esta insólita escena; pero, cuando, en su jactancioso intelecto, reflexionó sobre aquel incidente, llegó a la conclusión que este gesto de humildad no era sino un hecho más que probaba definitivamente que a Jesús nunca se le podría considerar como el libertador de Israel, y que él no había cometido ninguna equivocación cuando decidió desertar de la causa del Maestro.
\vs p179 3:5 Al quedarse todos mudos de asombro, Jesús dijo: “Pedro, te digo que si no te lavo los pies, no tendrás parte conmigo en lo que estoy a punto de llevar a cabo”. Cuando Pedro oyó estas palabras, sumadas al hecho de que Jesús seguía allí de rodillas a sus pies, tomó una de esas decisiones que se adoptan sin recapacitar para contentar el deseo de aquel a quien él tanto respetaba y amaba. Cuando Simón Pedro empezó a caer en la cuenta de que este previsto acto de servicio por parte de Jesús conllevaba alguna futura relación personal con la obra del Maestro, no solo se resignó a la idea de permitirle que le lavara los pies sino que, con su manera característica e impetuosa, dijo: “Entonces, Maestro, lava no solo mis pies, sino también las manos y la cabeza”.
\vs p179 3:6 Conforme el Maestro se preparaba para lavarle los pies a Pedro, dijo: “El que ya está lavado no necesita sino lavarse los pies. Vosotros que os sentáis conmigo esta noche limpios estáis, aunque no todos. Pero deberíais haber lavado el polvo de vuestros pies antes de sentaros a comer conmigo. Y, además, haré este servicio por vosotros como una parábola para ilustrar el significado de un nuevo mandamiento que en breve os daré”.
\vs p179 3:7 De igual modo, el Maestro se movió alrededor de la mesa, en silencio, lavando los pies de sus doce apóstoles, sin ni siquiera pasar por alto a Judas. Así que, una vez que terminó, se puso el manto, volvió a su asiento de anfitrión y, después de echar una mirada a sus desconcertados apóstoles, dijo:
\vs p179 3:8 \pc “¿En verdad entendéis lo que os he hecho? Vosotros me llamáis Maestro, y decís bien, porque eso soy. Pues si yo, el Maestro, he lavado vuestros pies, ¿por qué no estabais dispuestos a lavaros unos a otros los pies? ¿Qué lección debéis aprender de esta parábola en la que el Maestro, de tan buena voluntad, sirve a sus hermanos en algo que ellos no querían hacer el uno por el otro? De cierto, de cierto os digo: el siervo no es mayor que su señor, ni el enviado es mayor que el que lo envió. Habéis visto la forma en la que he servido en mi vida junto a vosotros, y bienaventurados sois vosotros que tendréis la benevolente determinación de servir de este modo. Pero, ¿por qué sois tan tardos para aprender que el secreto de la grandeza en el reino espiritual no se parece en nada a los métodos de poder que se emplean en el mundo material?
\vs p179 3:9 “Cuando esta noche entré a este aposento, en vuestra arrogancia, no os contentabais con negaros a lavaros los pies unos a otros, sino que también os enredasteis en disputas sobre quienes deberían ocupar los lugares de honor en mi mesa. Los fariseos y los hijos de este mundo son los que buscan estos honores, pero no debe ser así entre los embajadores del reino celestial. ¿Es que no sabéis que no puede haber ningún lugar preferente en mi mesa? ¿Es que no entendéis que os amo a cada uno de vosotros como amo a los demás? ¿Es que no sabéis que el lugar más próximo a mí, que los hombres consideran como un honor, no significa nada en relación a vuestro estatus en el reino de los cielos? Sabéis que los reyes de los gentiles se enseñorean sobre sus súbditos, mientras que los que ejercen sobre ellos dicha autoridad son a veces llamados bienhechores. Pero no así vosotros, sino que, el que mayor entre vosotros, sea como el más joven; y el que dirige sea como el que sirve, pues, ¿cuál es el mayor el que se sienta a la mesa, o el que sirve? ¿No se reconoce por lo general que el que se sienta a la mesa es el mayor? Pero observaréis que estoy entre vosotros como aquel que sirve. Si estáis dispuestos a ser compañeros míos en el servicio a la voluntad del Padre, os sentaréis conmigo en poder, y seguiréis cumpliendo la voluntad del Padre en la gloria futura”.
\vs p179 3:10 Cuando Jesús acabó de hablar, los gemelos Alfeo trajeron el siguiente plato de la última cena: el pan y el vino, con las hierbas amargas y la pasta de frutas secas.
\usection{4. ÚLTIMAS PALABRAS AL TRAIDOR}
\vs p179 4:1 Durante algunos minutos, los apóstoles comieron en silencio pero, contagiados por el comportamiento alegre del Maestro, pronto se sintieron empujados a charlar entre ellos, y, en poco tiempo, la cena discurrió como si nada fuera de lo habitual hubiera ocurrido que pudiese interferir con el buen ánimo y la camaradería de aquella extraordinaria ocasión. Tras cierto tiempo, a mitad de este segundo plato de la cena, Jesús los miró diciendo: “Os he dicho lo mucho que deseaba tener esta cena con vosotros, y sabiendo de qué modo se han confabulado las fuerzas maléficas de las tinieblas para propiciar la muerte del Hijo del Hombre, me determiné a organizarla secretamente en este aposento y un día antes de la Pascua, ya que, mañana por la noche, a esta misma hora, ya no estaré con vosotros. Os he dicho una y otra vez que debo volver al Padre. Ahora ha llegado mi hora, pero era preciso que uno de vosotros me traicionara y me entregara en manos de mis enemigos”.
\vs p179 4:2 Cuando los doce oyeron estas palabras, faltos ya de gran parte de su arrogancia y prepotencia gracias a la parábola del lavado de los pies y a la charla que seguidamente les dio el Maestro, se miraron unos a otros mientras que, con tono de desconcierto y titubeantes, inquirieron: “¿Soy yo?”. Y cuando todos acabaron de hacer esta pregunta, Jesús dijo: “Aunque es necesario que yo vaya al Padre, no lo era que uno de vosotros se convirtiera en un traidor para cumplir con la voluntad del Padre. No es sino el afloramiento del mal escondido en el corazón de quien no supo amar la verdad con toda su alma. ¡Cuán engañosa es la soberbia intelectual que antecede a la caída espiritual! Mi amigo de muchos años, el que a pesar de todo come mi pan, está dispuesto a traicionarme, tal como ahora mete sus manos conmigo en el plato”.
\vs p179 4:3 Cuando Jesús acabó de decir estas cosas, todos ellos comenzaron de nuevo a preguntar: “¿Soy yo?”. Y cuando Judas, que se sentaba a la izquierda de su Maestro, preguntó nuevamente: “¿Soy yo?”. Jesús, mojando el pan en el plato de las hierbas, se lo dio a Judas diciendo: “Tú lo has dicho”. Pero los demás no oyeron estas palabras de Jesús a Judas. Juan, que se reclinaba a la derecha de Jesús, se inclinó y preguntó al Maestro: “¿Quién es? Deberíamos saber quién ha sido desleal a su cometido”. Jesús contestó: “Ya os lo he dicho, el mismo al que le he dado el pan mojado”. Pero era tan normal que el anfitrión hiciera esto con el que se sentaba a su izquierda que ninguno se había dado cuenta de aquello, a pesar de la claridad con la que se había expresado el maestro. Pero Judas era terriblemente consciente del significado de las palabras del Maestro, referidas a sus actos, y sintió temor de que sus hermanos pudiesen también, entonces, enterarse de que él era el traidor.
\vs p179 4:4 Pedro estaba bastante agitado por lo que se había dicho e, inclinándose hacia adelante sobre la mesa, se dirigió a Juan: “Pregúntale quién es o, si te lo ha contado, dime quién es el traidor”.
\vs p179 4:5 Jesús puso fin a sus murmuraciones, diciendo: “Me aflige que este acto de maldad haya tenido que darse y, hasta incluso, en este mismo momento, había tenido la esperanza de que el poder de la verdad pudiera triunfar sobre los engaños del mal, pero estas victorias no se ganan sin una fe fundamentada en un amor leal por la verdad. Me gustaría no haber tenido que deciros estas cosas en esta, nuestra última cena, pero deseo advertiros de estos pesares y prepararos, pues, para lo que se cierne sobre nosotros. Os he informado de esto porque deseo que recordéis, una vez que me haya ido, que yo conocía todas estas malvadas maquinaciones, y que os previne de que sería traicionado. Y lo hago todo solo para que os sintáis fortalecidos ante las tentaciones y las pruebas que están al llegar”.
\vs p179 4:6 Cuando acabó de decir estas cosas, Jesús se inclinó hacia Judas, y dijo: “Lo que vayas a hacer, hazlo pronto”. Y, cuando Judas oyó estas palabras, se levantó de la mesa y se fue a toda prisa de la habitación, adentrándose en la noche para llevar a cabo lo que ya tenía en mente hacer. Cuando los otros apóstoles vieron que Judas salía apresuradamente después de que Jesús le hablara, presuponiendo que todavía llevaba la bolsa, pensaron que había ido por alguna cosa más para la cena o a hacer algún otro recado para el Maestro.
\vs p179 4:7 \pc Entonces, Jesús supo que ya nada podía impedir que Judas se convirtiera en un traidor. Había empezado con doce; ahora, tenía once. Él había escogido a seis de estos apóstoles y, aunque Judas estaba entre aquellos que estos apóstoles, primeramente elegidos, habían nombrado, el Maestro, no obstante, lo había admitido y había creído en él y, hasta aquella misma hora, había hecho todo lo posible por guiarlo al Padre y salvarlo, tal como había hecho para lograr la paz espiritual y la salvación de los demás.
\vs p179 4:8 Esta cena, con su ternura y afectividad, significó el último intento de Jesús para evitar que Judas, como ya lo había decidido, desertara, pero no sirvió de nada. Como norma general, las advertencias, incluso si se hacen de la manera más amable y se expresan con el espíritu más piadoso, cuando realmente ha fenecido el amor, solo logran intensificar el odio y encender la determinación, perversa, de llevar plenamente a cabo unos planes trazados con egoísmo.
\usection{5. INSTITUCIÓN DE LA CENA DEL MEMORIAL}
\vs p179 5:1 Cuando le trajeron la tercera copa de vino, la “copa de bendición”, Jesús se levantó del diván y, tomándola en sus manos, la bendijo, diciendo: “Tomad esta copa, todos vosotros, y bebed de ella. Será la copa de mi memorial. Es la copa de la bendición de una nueva dispensación de gracia y verdad. Será para vosotros el emblema de la dádiva y el ministerio del espíritu divino de la verdad. Y no volveré a beber de nuevo esta copa con vosotros hasta que yo beba en una nueva forma junto a vosotros en el reino eterno del Padre.
\vs p179 5:2 Cuando bebían de esta copa de bendición, con profunda veneración y en absoluto silencio, los apóstoles sintieron que algo fuera de lo ordinario estaba ocurriendo. La antigua Pascua conmemoraba el momento en el que sus padres emergían del estado de esclavitud racial a otro de libertad personal; ahora, el Maestro instituía una nueva cena en memoria de él, que simbolizaba la nueva dispensación en la que la persona en estado de esclavitud emerge individualmente de las cadenas del ritualismo y del egoísmo para surgir al gozo espiritual de la hermandad y el compañerismo de los hijos de la fe liberados del Dios vivo.
\vs p179 5:3 Cuando terminaron de beber esta nueva copa del memorial, el Maestro tomó el pan y, después de dar gracias, lo partió en trozos y, pidiéndoles que lo pasaran, dijo: “Tomad este pan del memorial y comedlo. Os he dicho que yo soy el pan de vida. Y este pan de vida es la vida unida del Padre y del Hijo en un solo don. La palabra del Padre, revelada en el Hijo, es en verdad el pan de vida”. Cuando habían compartido el pan del memorial, símbolo de la palabra viva de la verdad encarnada con la semejanza de un hombre mortal, se sentaron.
\vs p179 5:4 \pc Al instituir esta cena del memorial, el Maestro, como era su costumbre, recurría a parábolas y a símbolos. Usó símbolos porque quería enseñar ciertas grandes verdades espirituales de modo que a sus sucesores les resultara difícil atribuir a sus palabras interpretaciones precisas y significados concretos. De esta forma, procuraba evitar que las siguientes generaciones cristalizaran sus enseñanzas y ataran sus significados espirituales con las inertes ligaduras de la tradición y el dogma. Al instituir la única ceremonia o sacramento relacionado con su misión completa de vida, Jesús se esforzó mucho por \bibemph{sugerir} sus significados más que por aportar \bibemph{definiciones precisas}. No deseaba acabar con la idea de la comunión personal divina de la que cada persona es poseedora, instaurando un determinado sistema; tampoco deseaba limitar la imaginación espiritual del creyente, dificultándola con convencionalismos. Trataba más bien de liberar al alma renacida, dándole las gozosas alas de una libertad espiritual nueva y viva.
\vs p179 5:5 A pesar del esfuerzo realizado por el Maestro para instaurar este nuevo sacramento memorial de la manera que lo hizo, con el paso de los siglos quienes vinieron tras él se cuidaron de que su expreso deseo se viera ciertamente obstaculizado, haciendo que el sencillo simbolismo espiritual de aquella última noche en la carne se degradara a interpretaciones precisas y se sometiera a la exactitud casi matemática de una fórmula hecha. De todas las enseñanzas de Jesús, ninguna se ha vuelto más ritualista y rutinaria.
\vs p179 5:6 Cuando la cena del memorial se comparte por aquellos que creen en el Hijo y conocen a Dios, no se necesita que su simbolismo se asocie a ninguna de las erróneas y pueriles interpretaciones del hombre acerca del significado de la presencia divina, porque, en todas tales ocasiones, el Maestro está \bibemph{realmente presente}. La cena del memorial es el encuentro simbólico del creyente con Miguel. Cuando os hacéis pues conscientes del espíritu, el Hijo está verdaderamente presente, y su espíritu confraterniza con esa fracción de su Padre que habita en vosotros.
\vs p179 5:7 \pc Tras haber estado todos meditando durante unos momentos, Jesús continuó hablando: “Cuando hagáis estas cosas, recordad la vida que he vivido en la tierra entre vosotros y gozaos de que seguiré viviendo en la tierra con vosotros y sirviendo a través de vosotros. Personalmente, no contendáis sobre quién será el mayor. Sed todos como hermanos. Cuando el reino crezca hasta incluir grandes grupos de creyentes, debéis asimismo refrenaos de contender por ser los más grandes ni tratar de tener lugares de preferencia entre ellos”.
\vs p179 5:8 Este notable acontecimiento tuvo lugar en el aposento alto de un amigo. No hubo nada de rito sagrado ni de consagración ceremonial ni en la cena ni en el edificio. La cena del memorial se instituyó sin ninguna aprobación eclesiástica.
\vs p179 5:9 Cuando instituyó la cena del memorial en estos términos, Jesús dijo a los once: “Y cuántas veces hagáis esto, hacedlo en memoria de mí. Y cuando me recordéis, mirad primero hacia atrás, hacia lo que fue mi vida en la carne, recordad que estuve una vez con vosotros y, entonces, por la fe, anticipad que todos vendréis en algún momento a cenar conmigo en el reino eterno del Padre. Esta es la nueva Pascua que yo os dejo, esto es la memoria de mi vida de gracia, de la palabra de la verdad eterna; y de mi amor por vosotros, el derramamiento de mi espíritu de la verdad sobre toda carne”.
\vs p179 5:10 Y terminaron esta celebración de la antigua pero incruenta Pascua con motivo de la institución de la nueva cena del memorial, cantando, todos juntos, el salmo ciento dieciocho.
