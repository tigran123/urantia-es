\upaper{32}{Evolución de los universos locales}
\author{Mensajero poderoso}
\vs p032 0:1 Cada universo local es obra de uno de los hijos creadores del Paraíso del orden de Miguel. Cada cual consta de cien constelaciones, que están a su vez compuestas de mil sistemas de mundos habitados. Con el tiempo, cada sistema terminará por tener aproximadamente mil esferas habitadas.
\vs p032 0:2 Todos estos universos del tiempo y del espacio son evolutivos. El plan creativo de los migueles del Paraíso sigue siempre la senda del crecimiento gradual y el desarrollo progresivo de la naturaleza y de la capacidad física, intelectual y espiritual de las múltiples criaturas que habitan los diferentes tipos de esferas que integran cualquier universo local.
\vs p032 0:3 Urantia pertenece a un universo local cuyo soberano es el Dios\hyp{}hombre de Nebadón, Jesús de Nazaret, Miguel de Lugar de Salvación. Y la Trinidad del Paraíso dio su plena aprobación a todo el plan de Miguel previsto para este universo local antes de que él emprendiera la suprema aventura del espacio.
\vs p032 0:4 Los Hijos de Dios pueden elegir las áreas de su actividad creadora, pero son los arquitectos paradisíacos del universo matriz los que originariamente conciben y diseñan estas creaciones materiales.
\usection{1. LA APARICIÓN FÍSICA DE LOS UNIVERSOS}
\vs p032 1:1 La actuación sobre el espacio\hyp{}fuerza y las energías primordiales que precede a la creación de los universos constituye el trabajo de los organizadores paradisíacos mayores de la fuerza; pero en los ámbitos del suprauniverso, cuando la energía emergente se hace sensible a la gravedad local o lineal, estos se retiran en favor de los directores de la potencia del suprauniverso correspondiente.
\vs p032 1:2 En cuanto a la creación de los universos locales, los directores de la potencia actúan solos durante las etapas premateriales y posteriores a la fuerza. Los hijos creadores no tienen posibilidad alguna de comenzar la organización del universo hasta que los directores de la potencia no hayan llevado a efecto la suficiente movilización del espacio\hyp{}energías como para proporcionar una base material ---soles verdaderos y esferas materiales--- al universo emergente.
\vs p032 1:3 \pc Aunque difieran de forma considerable en sus dimensiones físicas y puedan variar cada cierto tiempo en su contenido de materia visible, todos los universos locales tienen aproximadamente un mismo potencial energético. La carga de la potencia y la dotación del potencial de la materia de un universo local se establecen por medio de la acción directiva de los directores de la potencia y de sus predecesores, al igual que por la actividad del hijo creador y por la inherente capacidad de su colaboradora creativa para el control físico.
\vs p032 1:4 La carga energética de un universo local es aproximadamente una cienmilésima parte de la dotación de fuerza del suprauniverso al que pertenece. En el caso de Nebadón, vuestro universo local, la materialización de la masa es algo menor. En un sentido físico, Nebadón posee toda la dotación física de la energía y la materia que se pueda encontrar en cualquiera de las creaciones locales de Orvontón. La única limitación física a la expansión evolutiva del universo de Nebadón radica en la carga cuantitativa del espacio\hyp{}energía sujeta al control gravitatorio de la conjunción de las potencias y seres personales del mecanismo combinado del universo.
\vs p032 1:5 \pc Cuando la energía\hyp{}materia ha alcanzado un cierto grado de materialización de su masa, aparece en escena un hijo creador del Paraíso, acompañado por una hija creativa del Espíritu Infinito. En simultaneidad con la llegada del hijo creador, da comienzo la labor de construir la esfera arquitectónica, que se convertirá en el mundo sede central del universo local en proyecto. Durante largas eras, esa creación local evoluciona, los soles se estabilizan, los planetas se forman y giran alrededor de sus órbitas, mientras que continúa la tarea de creación de los mundos arquitectónicos, destinados a ser las sedes centrales de las constelaciones y las capitales de los sistemas.
\usection{2. LA ORGANIZACIÓN DEL UNIVERSO}
\vs p032 2:1 En la organización del universo, a los hijos creadores les preceden los directores de la potencia al igual que otros seres de origen en la Tercera Fuente y Centro. A partir de las energías del espacio, previamente organizadas de ese modo, Miguel, vuestro hijo creador, estableció las regiones habitadas del universo de Nebadón y desde entonces siempre se ha dedicado con esmero al gobierno de las mismas. A partir de la energía preexistente, estos hijos divinos materializan la materia visible, diseñan a las criaturas vivas y, con la cooperación de la presencia en el universo del Espíritu Infinito, crean una variada comitiva de seres personales espirituales.
\vs p032 2:2 Estos directores de la potencia y controladores de la energía, que precedieron con tanta antelación al hijo creador en la tarea física preliminar de organizar el universo, realizan posteriormente un espléndido servicio en coordinación con este hijo del universo, conservando para siempre el control de aquellas energías que organizaron y pusieron en circulación. En Lugar de Salvación, todavía están en funcionamiento estos mismos cien centros de la potencia que cooperaron con vuestro hijo creador en la formación inicial de este universo local.
\vs p032 2:3 \pc El primer acto completo de creación física ocurrido en Nebadón consistió en la organización del mundo sede central o esfera arquitectónica de Lugar de Salvación, con sus satélites. Desde el momento de la actuación inicial de los centros de la potencia y de los controladores físicos hasta la llegada del equipo de ayuda de existencia viva a las esferas ya completadas de Lugar de Salvación, transcurrieron algo más de mil millones de años de vuestro presente tiempo planetario. A la construcción de Lugar de Salvación, le siguió inmediatamente la creación de cien mundos sedes de las constelaciones en proyecto y diez mil esferas sedes de los sistemas locales igualmente en proyecto, planificados para la dirección y la administración planetarias, junto con sus satélites arquitectónicos. Estos mundos arquitectónicos se diseñan para albergar tanto a seres personales físicos como a seres espirituales, al igual que a seres en estado morontial intermedio o de transición.
\vs p032 2:4 Lugar de Salvación, la sede central de Nebadón, está situada en el centro exacto de la energía\hyp{}masa del universo local. Pero vuestro universo local no es un sistema astronómico simple, a pesar del hecho de que existe un sistema de gran tamaño en su centro físico.
\vs p032 2:5 Lugar de Salvación es la sede personal de Miguel de Nebadón, pero no siempre se encuentra allí. Para que vuestro universo local funcione armoniosamente no es preciso que el hijo creador esté presente en la esfera capital permanentemente; esto, sin embargo, no era cierto en las primeras épocas de la organización física del universo. Un hijo creador no puede dejar su mundo sede hasta que no se instaura la gravedad de la zona espacial, que se realiza por medio de una materialización de energía suficiente como para permitir que las distintas vías y sistemas se equilibren entre sí mediante la atracción mutua de la materia.
\vs p032 2:6 \pc En este momento, llega a su fin el proyecto físico del universo, y el hijo creador junto con el espíritu creativo diseñan su plan para la creación de la vida; tras ello, esta representante del Espíritu Infinito inicia su actividad en el universo como ser personal creativo diferenciado. Cuando se lleva a efecto y se articula esta primera acción creativa, surge a la existencia la brillante estrella de la mañana, la personificación de este concepto creativo inicial de identidad e ideal de la divinidad. Se trata del mandatario en jefe del universo, el colaborador personal del hijo creador, alguien semejante a él en todos los aspectos de su carácter, aunque notablemente limitado en cuanto a atributos divinos.
\vs p032 2:7 Y, entonces, a la aparición del brazo derecho y mandatario en jefe del hijo creador le sigue la creación de un inmenso y maravilloso grupo de criaturas de diversa índole. Empezaron a aparecer hijos e hijas del universo local y, poco después, se establece el gobierno de ese universo, desde los consejos supremos del universo hasta los padres de las constelaciones y los soberanos de los sistemas locales ---integrados por esos mundos que están destinados a convertirse más tarde en la morada de las distintas razas mortales de criaturas de voluntad---. Un príncipe planetario presidirá cada uno de estos mundos.
\vs p032 2:8 Y entonces, cuando dicho universo se ha organizado por completo y se ha provisto de abundante personal, el hijo creador acomete el plan del Padre de crear al hombre mortal a su imagen divina.
\vs p032 2:9 \pc En Nebadón, la organización de las moradas planetarias sigue su curso. Este universo constituye, de hecho, un aglomerado joven entre las regiones estelares y planetarias de Orvontón. De acuerdo con el último registro, había 3\,840\,101 planetas habitados en Nebadón, y Satania, que es el sistema local al que pertenece vuestro mundo, es bastante común en relación a otros sistemas.
\vs p032 2:10 Satania no es un sistema físico uniforme, esto es, una unidad o estructura astronómica simple. Los 619 mundos habitados de que consta se localizan en más de quinientos sistemas físicos diferentes. Únicamente cinco de estos sistemas tienen más de dos mundos habitados, y de ellos solamente uno tiene cuatro planetas poblados; mientras que hay cuarenta y seis sistemas con dos mundos habitados.
\vs p032 2:11 El sistema de mundos habitados de Satania está a una gran distancia de Uversa y de ese gran aglomerado de soles que funciona como centro físico o astronómico del séptimo suprauniverso. Desde Jerusem, la sede central de Satania, hay más de doscientos mil años luz hasta el centro físico del suprauniverso de Orvontón, situado lejos, muy lejos en el denso diámetro de la Vía Láctea. Satania está en la periferia del universo local, y Nebadón se encuentra ahora muy afuera hacia el borde de Orvontón. Desde el sistema ultraperiférico de mundos habitados hasta el centro del suprauniverso hay algo menos de doscientos cincuenta mil años luz.
\vs p032 2:12 En este momento, el universo de Nebadón gira lejos hacia el sureste en la vía circulatoria del suprauniverso de Orvontón. Los universos limítrofes más cercanos son: Avalón, Henselón, Sanselón, Portalón, Wolverín, Fanovín y Alvorín.
\vs p032 2:13 \pc Pero la evolución de un universo local es un largo relato. En los escritos dedicados al estudio del suprauniverso se introduce este tema. En los de esta sección, que tratan de las creaciones locales, se continúa, mientras que en los siguientes, que abordan la historia y el destino de Urantia, se completa dicho relato. No obstante, solo podréis comprender suficientemente el destino de los mortales de una creación local como esta si examináis el relato de la vida y enseñanzas de vuestro hijo creador, cuando vivió la vida de un hombre mortal de vuestro propio mundo evolutivo.
\usection{3. LA IDEA EVOLUTIVA}
\vs p032 3:1 La única creación perfectamente estable es Havona, el universo central, que se creó directamente por el pensamiento del Padre Universal y la palabra del Hijo Eterno. Havona es un universo existencial, perfecto y completo, que rodea la morada de las Deidades eternas, el centro de todas las cosas. Los siete suprauniversos son creaciones finitas, evolutivas y en avance continuo y sistemático.
\vs p032 3:2 Los sistemas físicos del tiempo y del espacio tienen un origen evolutivo. Ni siquiera son estables físicamente hasta que no entran en las vías circulatorias establecidas de sus suprauniversos. Tampoco se asienta en luz y vida un universo local hasta que no se hayan agotado sus posibilidades físicas de expansión y desarrollo, y hasta que no se haya asentado y estabilizado para siempre el estatus espiritual de todos sus mundos habitados.
\vs p032 3:3 Con la excepción del universo central, la perfección se consigue de forma progresiva. En la creación central encontramos un modelo de perfección, pero todas las otras regiones espaciales deben alcanzar esa perfección siguiendo los métodos establecidos para el avance de esos mundos o universos concretos. Y los planes de los hijos creadores en relación con la organización, evolución, entrenamiento sistemático y establecimiento de sus respectivos universos locales se caracterizan por una casi infinita variedad de métodos.
\vs p032 3:4 \pc Exceptuando la presencia en cuanto deidad del Padre, cada universo local es, en cierto sentido, una copia de la organización administrativa de la creación central o modelo. Aunque el Padre Universal está personalmente presente en el universo en el que reside, no habita en las mentes de los seres originarios de dicho universo tal como lo hace literalmente en las almas de los mortales del tiempo y del espacio. Allí parece darse una omnisapiente compensación en cuanto al ajuste y regulación de los acontecimientos espirituales de la extensa creación. En el universo central, el Padre está personalmente presente como tal, pero está ausente de la mente de los hijos de esa creación perfecta. En los universos del espacio, el Padre está ausente en persona, siendo los hijos soberanos quienes lo representan, aunque está íntimamente presente en las mentes de sus hijos mortales, representado espiritualmente por la presencia prepersonal de los mentores misteriosos, que habitan en las mentes de estas criaturas de voluntad.
\vs p032 3:5 En la sede de un universo local residen todos aquellos seres personales creadores y creativos que ostentan autoridad de forma independiente y autonomía de orden administrativo, salvo la presencia personal del Padre Universal. En el universo local, salvo al Padre universal, es posible encontrar algo de todas y de cada una de casi todas las clases de seres inteligentes existentes en el universo central. A pesar de que el Padre Universal no está personalmente presente en un universo local, está personalmente representado por su hijo creador, primeramente actuando como vicerregente de Dios y, después, como gobernante soberano y supremo por derecho propio.
\vs p032 3:6 Cuanto más descendemos en la escala de la vida, más arduo es encontrar, con los ojos de la fe, al Padre invisible. A las criaturas más modestas ---y, a veces, incluso a los seres personales más elevados---, les resulta siempre difícil concebir al Padre Universal en sus hijos creadores. Así pues, en espera de su exaltación espiritual, cuando su perfección alcance un desarrollo que les permita percibir a Dios en persona, se sienten cansados de este proceso, albergan dudas espirituales, caen en la confusión y, como consecuencia, se apartan del propósito de progreso espiritual de su tiempo y universo. De esta manera, pierden la capacidad de ver al Padre cuando contemplan al hijo creador. Durante el prolongado esfuerzo por alcanzar al Padre, en el período en que las condiciones que esto conlleva hacen imposible tal logro, la más segura salvaguarda para la criatura consiste en aferrarse con tenacidad a la verdad\hyp{}hecho de la presencia del Padre en sus hijos del Paraíso. De forma literal y figurativa, espiritual y personal, el Padre y sus hijos son uno solo. Es un hecho: el que ha visto a un hijo creador, ha visto al Padre.
\vs p032 3:7 \pc Los seres personales de un determinado universo son solamente estables y dignos de confianza en el comienzo, según su grado de confluencia con la Deidad. Cuando el origen de la criatura está demasiado lejos de las fuentes originales y divinas, ya sea en referencia a los Hijos de Dios o a las criaturas servidoras pertenecientes al Espíritu Infinito, se puede dar la posibilidad de que aumente la falta de armonía, la confusión y a veces la rebelión ---esto es, el pecado---.
\vs p032 3:8 \pc Con la excepción de los seres perfectos que se originan en la Deidad, todas las criaturas de voluntad de los suprauniversos son de naturaleza evolutiva; comienzan desde una condición humilde y se elevan siempre hacia arriba, en realidad hacia dentro. Incluso los seres personales sumamente espirituales ascienden continuamente en la escala de la vida mediante traslaciones progresivas de vida en vida y de esfera en esfera. Y en el caso de aquellos que albergan a mentores misteriosos, no existe, de hecho, límite alguno en cuanto a las alturas posibles a las que pueden llegar en su ascenso espiritual y en sus logros en el universo.
\vs p032 3:9 Esta perfección de las criaturas del tiempo, cuando finalmente se consigue, es enteramente posesión genuina de su ser personal. Aunque los elementos de la gracia se añadan profusamente, los logros de la criatura son, sin embargo, el resultado de su esfuerzo individual y de verdaderas experiencias de vida, esto es, de la respuesta de su persona al entorno existente.
\vs p032 3:10 Desde la visión del universo, el origen evolutivo animal no supone un estigma para ningún ser personal dado que este es el método exclusivo de creación de uno de los dos tipos básicos de criaturas de voluntad finitas e inteligentes. Cuando se alcancen las cimas de la perfección y de la eternidad, se tendrá en gran consideración a aquellos que comenzaron desde abajo y ascendieron con regocijo en la escala de la vida, peldaño tras peldaño, y que, cuando verdaderamente lleguen a las alturas gloriosas, habrán adquirido una experiencia personal que conlleva un conocimiento real de cada una de las etapas de la vida, desde abajo hacia arriba.
\vs p032 3:11 En todo esto se ve la sabiduría de los creadores. Sería igualmente fácil para el Padre Universal hacer de todos los mortales seres perfectos, impartir perfección con su palabra divina. Pero eso los privaría de la maravillosa vivencia de la aventura y de la formación en su largo y gradual ascenso hacia el interior, una experiencia reservada solo para aquellos que tienen la buena fortuna de comenzar su existencia en el escalón más bajo de la vida.
\vs p032 3:12 En los universos que rodean a Havona, solamente se dispone de un número de criaturas perfectas suficiente para atender la necesidad de guías instructores que sirvan de modelo a los órdenes de seres que ascienden en la escala evolutiva de la vida. La naturaleza experiencial del ser personal de tipo evolutivo constituye el complemento cósmico natural de la naturaleza perfecta de las criaturas del Paraíso\hyp{}Havona. En realidad, tanto las criaturas perfectas como las perfeccionadas son incompletas en lo que respecta a la totalidad finita. Si bien, en la relación de complementariedad de las criaturas existencialmente perfectas del sistema del Paraíso\hyp{}Havona con los finalizadores experiencialmente perfeccionados que ascienden desde los universos evolutivos, ambos tipos de seres logran liberarse de sus propias limitaciones y, de este modo, pueden conjuntamente intentar alcanzar las sublimes alturas de la ultimidad de la condición creatural.
\vs p032 3:13 Esta interacción entre criaturas es la consecuencia de acciones y reacciones que se dan en el universo dentro de la Deidad Séptupla, en la cual la divinidad eterna de la Trinidad del Paraíso se une con la divinidad evolutiva de los creadores supremos de los universos espacio\hyp{}temporales en y por medio de la actualización\hyp{}potenciación de la Deidad del Ser Supremo.
\vs p032 3:14 La criatura divinamente perfecta y la criatura evolutiva perfeccionada tienen un mismo grado de potencial de divinidad aunque de diferente clase. Cada cual tiene que depender de la otra para alcanzar la supremacía de servicio. Los suprauniversos evolutivos dependen del perfecto Havona para proporcionar la formación final de sus ciudadanos ascendentes, pero al universo central perfecto también le hace falta la existencia de los suprauniversos en proceso de perfección para proporcionar a sus habitantes descendentes un desarrollo pleno.
\vs p032 3:15 Las dos manifestaciones fundamentales de la realidad finita, la perfección innata y la perfección evolucionada, ya se trate de seres personales o de universos, son iguales en importancia, dependientes e integradas. Cada cual precisa que la otra logre completar su cometido, servicio y destino.
\usection{4. RELACIÓN DE DIOS CON LOS UNIVERSOS LOCALES}
\vs p032 4:1 No alberguéis la idea de que, puesto que el Padre Universal ha delegado tanto de sí mismo y de su poder en otros, sea un miembro silente o inactivo de entre sus compañeros de la Deidad. Aparte de su área de acción respecto al ser personal y de su dádiva de los modeladores, es al parecer el menos activo de las Deidades del Paraíso, ya que permite a sus coiguales en la Deidad, a sus hijos creadores y a muchas inteligencias creadas desempeñar un papel tan relevante en la realización de su propósito eterno. Es, en efecto, el miembro silente del trío creativo únicamente por el hecho de que nunca hace nada que cualquiera de sus colaboradores de igual o de menor rango pueda hacer.
\vs p032 4:2 Dios comprende a la perfección la necesidad de todas las criaturas inteligentes de actuar y experimentar y, por consiguiente, en cualquier situación, ya se trate del destino de un universo o del bien de la más humilde de sus criaturas, se aparta en favor de la constelación de seres personales creados y creadores que de forma natural median entre él mismo y cualquier situación dada del universo o acontecimiento creativo. No obstante, pese a este retiro, a esta manifestación de coordinación infinita, hay por parte de Dios una participación real, literal y personal en estos acontecimientos, directa e indirectamente, por medio de dichas instancias intermedias y seres personales designados. El Padre obra en y por medio de estos canales por el bien de toda su extensa creación.
\vs p032 4:3 \pc Por lo que se refiere a normas, al curso de acción y a la administración de un universo local, el Padre Universal actúa en la persona de su hijo creador. El Padre Universal nunca interviene en lo que respecta a la interrelación entre los Hijos de Dios, a las relaciones grupales de seres personales originados en la Tercera Fuente y Centro o a las relaciones que se dan entre otros tipos de criaturas, tales como los seres humanos. La ley del hijo creador, las directrices de los padres de las constelaciones, de los soberanos de los sistemas y de los príncipes planetarios ---las normas y procedimientos prescritos para tal universo--- siempre prevalecen. No hay división de autoridad, nunca hay discordia entre poder y propósito divino. Las Deidades obran con unanimidad perfecta y eterna.
\vs p032 4:4 El hijo creador gobierna supremo en todas las cuestiones de tipo ético, esto es, en las relaciones que se dan entre distintos grupos y clases de criaturas o entre dos o más seres individuales dentro de un grupo dado, pero este plan no significa que el Padre Universal no pueda a su manera intervenir y hacer todo lo que complazca a su mente divina con cualquier \bibemph{criatura determinada} por doquier de la creación, en cuanto al estatus presente de ese ser o a sus perspectivas futuras, y en lo pertinente al plan eterno del Padre y a su propósito infinito.
\vs p032 4:5 \pc El Padre está realmente presente en las criaturas mortales de voluntad mediante el modelador interior, una fracción de su espíritu prepersonal; y el Padre es también la fuente del ser personal de estas criaturas mortales.
\vs p032 4:6 \pc Estos modeladores del pensamiento, que otorga el Padre Universal, se encuentran relativamente aislados; moran en las mentes humanas pero no tienen relación perceptible con las cuestiones éticas de una creación local. No se coordinan directamente con el servicio seráfico ni con la administración de los sistemas, constelaciones o universos locales, ni siquiera con el gobierno de los hijos creadores, cuya voluntad es la ley suprema de su universo.
\vs p032 4:7 Los modeladores interiores constituyen uno de los modos de contacto particular pero unificado de Dios con las criaturas de su creación casi infinita. Así pues, él, que es invisible para el hombre mortal, manifiesta su presencia y, si pudiera hacerlo, se nos mostraría de otras maneras, pero otro tipo de revelación no es divinamente posible.
\vs p032 4:8 Podemos observar y entender el mecanismo por el que los hijos creadores disfrutan de un conocimiento directo y completo de los universos bajo su jurisdicción. Pero no podemos entender del todo los modos por los que Dios conoce de forma tan completa y personal los detalles del universo de los universos, aunque seamos capaces al menos de reconocer los cauces por los que el Padre Universal puede recibir información acerca de los seres de su inmensa creación, al igual que manifestarles su presencia. A través de la vía circulatoria del ser personal el Padre es consciente ---tiene conocimiento personal--- de todos los pensamientos y actos de todos los seres de todos los sistemas de todos los universos de toda la creación. Aunque no podemos alcanzar a comprender por completo la forma de comunión de Dios con sus hijos, podemos sentirnos fortalecidos en la seguridad de que “el Señor conoce a sus hijos” y de que, de cada uno de nosotros, “toma nota de dónde hemos nacido”.
\vs p032 4:9 \pc Espiritualmente hablando, el Padre Universal está presente en vuestro universo y en vuestros corazones mediante uno de los siete espíritus mayores de la morada central y, particularmente, por medio del modelador divino que vive y obra y aguarda en las profundidades de la mente mortal.
\vs p032 4:10 \pc Dios no es un ser personal egocéntrico. El Padre se distribuye profusamente en su creación y en sus criaturas. Vive y obra no solo en las Deidades, sino también en sus hijos, a quienes encomienda que hagan todo aquello que les es divinamente posible hacer. El Padre Universal se ha despojado verdaderamente de toda aquella labor que cualquier otro ser pueda llevar a cabo. Y esto es cierto tanto del hombre mortal como del hijo creador, que gobierna en lugar de Dios en la sede de un universo local. Así es como contemplamos el resultado del amor ideal e infinito del Padre Universal.
\vs p032 4:11 En este darse a sí mismo tenemos una prueba suficiente tanto de la magnitud como de la magnanimidad de la naturaleza divina del Padre. Si Dios ha retenido algo de la creación universal para sí mismo, entonces, de ese resto, otorga, con prodigalidad y generosidad, los modeladores del pensamiento a los mortales de las regiones espaciales, los mentores misteriosos del tiempo que con tanta paciencia habitan en los aspirantes mortales a la vida eterna.
\vs p032 4:12 El Padre Universal se ha derramado, por decirlo de alguna manera, para que toda la creación se beneficie con la posesión del ser personal y el potencial para alcanzar el logro espiritual. Dios nos ha dado de sí mismo para que nosotros podamos ser como él, y de poder y gloria se ha reservado tan solo lo necesario para mantener aquellas cosas por cuyo amor se ha despojado a sí mismo de todo lo demás.
\usection{5. EL PROPÓSITO ETERNO Y DIVINO}
\vs p032 5:1 Existe un propósito grande y glorioso en la marcha de los universos por el espacio. Todas vuestras luchas humanas no son en vano. Todos somos parte de un plan inmenso, de un gigantesco emprendimiento, y es la inmensidad de esta labor la que imposibilita que, en un momento determinado o durante una vida, veamos gran parte de ella. Todos formamos parte de un proyecto eterno que los Dioses dirigen y llevan a cabo. La espléndida totalidad del mecanismo universal sigue su marcha a través del espacio, con majestuosidad, al compás de la música del pensamiento infinito y del propósito eterno de la Primera Gran Fuente y Centro.
\vs p032 5:2 El propósito eterno del Dios eterno es un elevado ideal espiritual. Los acontecimientos del tiempo y los afanes de la existencia material no son otra cosa que el andamiaje transitorio que tiende un puente hacia el otro lado, hacia la tierra prometida de la realidad espiritual y de la existencia celestial. Es natural que a vosotros, los mortales, os resulte difícil captar la idea de este propósito eterno; sois prácticamente incapaces de comprender el concepto de eternidad, de algo que no tiene ni principio ni fin. Todo lo que os resulta familiar tiene un final.
\vs p032 5:3 \pc Si nos fijamos en una vida particular, en la duración de un mundo o en la cronología de una serie interconectada de sucesos, parece que observamos un trecho aislado de tiempo y nos da la impresión de que todo tiene comienzo y fin. Y podría parecer que un conjunto de tales experiencias, vidas, eras o épocas, dispuesto de forma sucesiva, constituyera una línea recta, un acontecimiento aislado del tiempo que destellara momentáneamente cruzando la faz infinita de la eternidad. Pero cuando contemplamos todo esto entre bastidores, surge una visión de conjunto y un entendimiento más completo que sugieren que esto es un pensamiento inadecuado, desconectado y totalmente inconsistente para describir con propiedad y para poder, además, correlacionar los acontecimientos temporales con el propósito subyacente y la respuesta fundamental de la eternidad.
\vs p032 5:4 Con el fin de ofrecer una explicación asequible a la mente mortal, me parece más conveniente concebir la eternidad como un ciclo y el propósito eterno como un círculo interminable, un ciclo de eternidad de algún modo sincronizado con los ciclos materiales transitorios del tiempo. En lo referente a los sectores de tiempo conectados con el ciclo de la eternidad, y del que forman parte, nos vemos obligados a reconocer que, estas épocas temporales del tiempo se originan, tienen su existencia y se extinguen precisamente al igual que nacen, viven y mueren los seres transitorios. La mayoría de los seres humanos mueren porque, al no lograr alcanzar el nivel espiritual que conlleva la fusión con el modelador, la metamorfosis de la muerte constituye el único modo posible por el que pueden escapar de las cadenas del tiempo y de las ataduras de la creación material, pudiendo así marchar al paso espiritual de la procesión progresiva de la eternidad. Habiendo sobrevivido a la prueba de la vida temporal y de la existencia material, os será posible continuar en contacto con la eternidad, incluso como parte de ella, girando para siempre con los mundos del espacio alrededor del círculo de las eras eternas.
\vs p032 5:5 Los sectores del tiempo son como los rápidos destellos del ser personal en forma temporal; aparecen durante un tiempo y luego se pierden de la visión humana, solo para reaparecer como nuevos actores y factores persistentes en la vida superior que se desplaza sin fin alrededor del círculo eterno. Difícilmente se puede concebir la eternidad como una línea recta a la vista de nuestra percepción de un universo delimitado, que se mueve en un círculo inmenso y alargado, alrededor de la morada central del Padre Universal.
\vs p032 5:6 Con franqueza, la eternidad es incomprensible para la mente temporal y finita. Sencillamente no podéis concebirla; no podéis llegar a comprenderla. Yo no la visualizo del todo, e incluso si pudiese hacerlo, me sería imposible transmitir a la mente humana la noción que tengo de ella. No obstante, he hecho todo lo posible para exponer algo de nuestro punto de vista, y hablaros de cómo entendemos nosotros las cosas eternas. Mi afán es ayudaros a formar vuestros pensamientos en relación a estos valores que son de naturaleza infinita y de eterna relevancia.
\vs p032 5:7 \pc Hay en la mente de Dios un plan que incluye a todas las criaturas de todos sus inmensos dominios, y este plan consiste en un propósito eterno de oportunidades sin límites, de progreso ilimitado y de vida sin fin. ¡Y los tesoros infinitos de una andadura tan inigualable son vuestros si os esforzáis por conseguirlos!
\vs p032 5:8 ¡La meta de la eternidad os aguarda! ¡La aventura de lograr la divinidad se encuentra frente a vosotros! ¡La carrera por la perfección está en marcha! Quien lo desee puede correr, y la victoria de cierto coronará los esfuerzos de todo ser humano que participe en la carrera de la fe y de la esperanza, dependiendo a cada paso de la dirección del modelador interior y de la guía de ese buen espíritu del Hijo del Universo, que tan generosamente se ha derramado sobre toda carne.
\vsetoff
\vs p032 5:9 [Exposición de un mensajero poderoso adscrito temporalmente al Consejo Supremo de Nebadón y asignado a esta misión por Gabriel de Lugar de Salvación.]
