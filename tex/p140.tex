\upaper{140}{La ordenación de los doce}
\author{Comisión de seres intermedios}
\vs p140 0:1 Poco antes del mediodía del domingo 12 de enero del año 27 d. C., Jesús congregó a los apóstoles para ordenarlos como predicadores públicos del evangelio del reino. Los doce esperaban que se les llamase casi en cualquier momento; así que esa mañana no se habían alejado mucho de la costa para pescar. Algunos de ellos se habían quedado cerca de la orilla, reparando las redes y arreglando sus aparejos de pesca.
\vs p140 0:2 Conforme Jesús recorría la playa, empezó a llamar a los apóstoles. Llamó primeramente a Andrés y Pedro, que estaban pescando próximos a la costa; luego, hizo un gesto para que se acercaran Santiago y Juan, que se encontraban en una barca cercana charlando con su padre Zebedeo y remendando sus redes. A los demás apóstoles, los reunió de dos en dos y, cuando había congregado a los doce, viajó con ellos a las mesetas del norte de Cafarnaúm, donde comenzó a instruirlos en preparación a su ordenación ceremonial.
\vs p140 0:3 Por primera vez, los doce apóstoles permanecieron en silencio; hasta Pedro estaba en una actitud reflexiva. ¡Por fin había llegado la tan esperada hora! Iban con el Maestro a un lugar apartado para participar en algún tipo de ceremonia solemne, que significaría su consagración personal y su dedicación como grupo a la labor sagrada de representar a su Maestro en la proclamación de la llegada del reino de su Padre.
\usection{1. INSTRUCCIÓN PREVIA}
\vs p140 1:1 Antes del ceremonial servicio de ordenación, Jesús habló a los doce, que se encontraban sentados a su alrededor: “Hermanos míos, ha llegado esta hora del reino. Os he traído conmigo aquí, aparte, para presentaros ante el Padre como embajadores de este reino. En la sinagoga, algunos de vosotros me oísteis hablar de dicho reino cuando en un principio se os llamó. Y, cada uno de vosotros ha aprendido más sobre el reino del Padre trabajando conmigo en las ciudades próximas al mar de Galilea. Pero, en este momento, tengo algo más que deciros sobre este reino.
\vs p140 1:2 “El nuevo reino que mi Padre está a punto de establecer en el corazón de sus hijos de la tierra será en perpetuidad. No habrá final para este gobierno de mi Padre en los corazones de quienes desean hacer su voluntad divina. Os declaro que mi Padre no es el Dios de judíos ni de gentiles. Vendrán muchos de Oriente y de Occidente para sentarse con nosotros en el reino del Padre, mientras que muchos de los hijos de Abraham rehusarán entrar en esta nueva hermandad en la que el espíritu de Padre reinará en el corazón de los hijos de los hombres.
\vs p140 1:3 “El poder de este reino consistirá, no en la fuerza de los ejércitos ni en el poderío de las riquezas, sino más bien en la gloria del espíritu divino que vendrá a enseñar a las mentes y a regir en el corazón de los ciudadanos renacidos de este reino celestial, de los hijos de Dios. Esta es la hermandad de amor donde impera la rectitud, y cuya consigna será: en la tierra paz y buena voluntad para con los hombres. Este reino, que muy pronto saldréis a proclamar, es el deseo de los hombres de buena voluntad de todos las eras, la esperanza de toda la tierra y el cumplimiento de las sabias promesas de todos los profetas.
\vs p140 1:4 “Pero para vosotros, hijos míos, y para todos aquellos que os seguirán en este reino, habrá una dura prueba. Solo la fe os hará cruzar sus puertas, pero vosotros debéis rendir los frutos del espíritu de mi Padre si queréis continuar ascendiendo y avanzando en la vida de la fraternidad divina. De cierto os digo que no todo el que me dice: ‘¡Señor, Señor!’, entrará en el reino de los cielos, sino aquel que hace la voluntad de mi Padre que está en los cielos.
\vs p140 1:5 “Vuestro mensaje para el mundo será: buscad primeramente el reino de Dios y su justicia y, cuando lo encontréis, todas las otras cosas precisas para vuestra supervivencia eterna os serán añadidas. Y desearía pues que os quedase claro que este reino de mi Padre no vendrá con muestras externas de poder ni con demostraciones fatuas. No deberéis proclamar el reino diciendo, ‘helo aquí’ o ‘helo allí’, porque este reino que predicaréis es Dios en vosotros.
\vs p140 1:6 “El que quiera hacerse grande en el reino de mi Padre será vuestro servidor; y el que quiera ser el primero entre vosotros, será el siervo de sus hermanos. Pero cuando en verdad se os haya recibido como ciudadanos en el reino celestial, ya no seréis sirvientes sino hijos, hijos del Dios vivo. Y así avanzará este reino en el mundo, hasta romper cualquier barrera y llevar a todos los hombres al conocimiento de mi Padre y a creer en la verdad salvadora que yo he venido a proclamar. Ahora mismo el reino se ha acercado, y algunos entre vosotros no moriréis hasta no haber visto venir el reino de Dios con gran poder.
\vs p140 1:7 “Y esto que vuestros ojos contemplan ahora, este pequeño principio de doce hombres corrientes, se multiplicará y crecerá hasta que toda la tierra acabe por llenarse de alabanzas a mi Padre. Y no será tanto por las palabras que habléis sino por la vida que viváis que los hombres conocerán que habéis estado conmigo y que habéis aprendido las realidades del reino. Y aunque no es mi deseo colocar cargas gravosas sobre vuestras mentes, estoy a punto de depositar sobre vuestras almas la solemne responsabilidad de representarme en el mundo, cuando en breve yo os deje, de la misma manera que yo ahora represento a mi Padre en esta vida que estoy viviendo en la carne”. Y, al acabar de hablar, se levantó.
\usection{2. LA ORDENACIÓN}
\vs p140 2:1 Jesús pidió ahora a los doce mortales que acababan de escuchar su proclamación del reino que se arrodillaran formando un círculo a su alrededor. Entonces, el Maestro colocó las manos sobre la cabeza de cada uno de los apóstoles, comenzando con Judas Iscariote y terminando con Andrés. Cuando los bendijo, extendió sus manos y oró:
\vs p140 2:2 “Padre mío, traigo ahora ante ti a estos hombres, mis mensajeros. Entre nuestros hijos de la tierra, he elegido a estos doce para que salgan al mundo y me representen al igual que yo vine para representarte a ti. Ámalos y está tú con ellos como me has amado y has estado conmigo. Y ahora, Padre mío, concede sabiduría a estos hombres, mientras yo confío todos los asuntos del reino venidero en sus manos. Y quisiera, si es tu voluntad, quedarme algún tiempo en la tierra para poder ayudarlos en su labor por el bien del reino. Y, de nuevo, Padre mío, te doy las gracias por estos hombres, y los encomiendo a tu cuidado, a la vez que yo sigo adelante para acabar la obra que me has encargado”.
\vs p140 2:3 \pc Cuando Jesús terminó de orar, cada uno de los apóstoles permaneció en su sitio con la cabeza inclinada. Y pasaron muchos minutos antes de que incluso Pedro se atreviese a alzar la mirada para mirar al Maestro. Uno a uno abrazaron a Jesús, pero nadie dijo nada. Un gran silencio invadió aquel lugar mientras que una multitud de seres celestiales contemplaban desde lo alto esta solemne y sagrada escena ---el creador de un universo que ponía los asuntos de la hermandad divina del hombre bajo la dirección de unas mentes humanas---.
\usection{3. EL SERMÓN DE ORDENACIÓN}
\vs p140 3:1 Entonces Jesús habló y dijo: “Ahora, que sois embajadores del reino de mi Padre, os habéis convertido, pues, en una clase de hombres separada y distinta de todos los demás hombres de la tierra. Ya no sois hombres entre los hombres, sino que seréis, entre las criaturas ignorantes de este oscuro mundo, ciudadanos iluminados de otro país celestial. Ahora no basta con que viváis como habéis vivido antes de esta hora, sino que, en adelante, debéis vivir como quienes han degustado la gloria de una vida mejor y a quienes se les ha enviado de vuelta a la tierra como embajadores del Soberano de ese mundo nuevo y mejor. Se espera más del maestro que del alumno; se exige más del amo que del sirviente. De los ciudadanos del reino celestial se demanda más que de los ciudadanos del gobierno terrenal. Algunas de las cosas que estoy a punto de deciros os parecerán duras, pero habéis elegido representarme en el mundo, al igual que yo represento ahora al Padre; y, como mis emisarios en el mundo, os atendréis a esas enseñanzas y prácticas que reflejan mi ideal de la vida mortal en los mundos del espacio y que yo, en mi vida en la tierra, ejemplifico revelando al Padre que está en los cielos.
\vs p140 3:2 “Os envío para proclamar libertad a los cautivos espirituales, alegría a los que están atenazados por el miedo y sanación a los enfermos conforme a la voluntad de mi Padre de los cielos. Cuando halléis a mis hijos en la aflicción, habladles palabras de aliento, diciendo:
\vs p140 3:3 Dichosos los pobres de espíritu, los humildes, porque suyos son los tesoros del reino de los cielos.
\vs p140 3:4 “Dichosos los que tienen hambre y sed de rectitud, porque serán saciados.
\vs p140 3:5 “Dichosos los mansos, porque recibirán la tierra por heredad.
\vs p140 3:6 “Dichosos los puros de corazón, porque verán a Dios.
\vs p140 3:7 “Y hablad además a mis hijos estas otras palabras de consuelo y de promesa espiritual:
\vs p140 3:8 “Dichosos los afligidos, porque serán consolados. Dichosos los que lloran, porque recibirán el espíritu de la alegría.
\vs p140 3:9 “Dichosos los misericordiosos, porque alcanzarán misericordia.
\vs p140 3:10 “Dichosos los pacificadores, porque serán llamados hijos de Dios.
\vs p140 3:11 “Dichosos los que padecen persecución por causa de la rectitud, porque de ellos es el reino de los cielos. Dichosos seréis cuando os insulten, os persigan y digan toda clase de mal contra vosotros, mintiendo. Gozaos y llenaos de alegría, porque vuestra recompensa es grande en los cielos.
\vs p140 3:12 “Hermanos míos, al enviaros al mundo, vosotros sois la sal de la tierra, una sal con sabor de salvación. Pero si la sal pierde su sabor, ¿con qué será salada? No sirve más para nada, sino para ser echada fuera y pisoteada por los hombres.
\vs p140 3:13 “Vosotros sois la luz del mundo. Una ciudad asentada sobre un monte no se puede esconder. Ni tampoco se enciende una luz para ponerla debajo del celemín, sino sobre el candelero para que alumbre a todos los que están en la casa. Así alumbre vuestra luz delante de los hombres, para que vean vuestras buenas obras y glorifiquen a vuestro Padre que está en los cielos.
\vs p140 3:14 “Os envío al mundo para que me representéis y actuéis como embajadores del reino de mi Padre, y cuando salgáis para proclamar la buena nueva, depositad vuestra confianza en el Padre, de quien sois sus mensajeros. No confrontéis por la fuerza si os tratan con injusticia; no pongáis vuestra confianza en el brazo de la carne. Si vuestro vecino os hiere en la mejilla derecha, volvedle también la otra. Estad dispuestos a sufrir injusticia en lugar de poner pleito entre vosotros. En bondad y con misericordia atended a todos los angustiados y necesitados.
\vs p140 3:15 Yo os digo: amad a vuestros enemigos, haced bien a los que os odian, bendecid a los que os maldicen y orad por los que os ultrajan. Y como creéis que yo haría a los hombres, hacedlo también vosotros.
\vs p140 3:16 “Vuestro Padre en los cielos hace salir su sol sobre malos y buenos y llover sobre justos e injustos. Vosotros sois los hijos de Dios; aún más, ahora sois los embajadores del reino de mi Padre. Sed misericordiosos, como Dios es misericordioso, y, en el futuro eterno del reino, seréis perfectos, como vuestro Padre celestial es perfecto.
\vs p140 3:17 “Se os ha encargado que salvéis a los hombres, no que los juzguéis. Al final de vuestra vida en la tierra, todos vosotros esperaréis misericordia; así pues, exijo de vosotros que, durante vuestra vida mortal, mostréis misericordia a todos vuestros hermanos en la carne. No cometáis el error de intentar quitar la mota del ojo de vuestro hermano cuando hay una viga en vuestros propios ojos. Y habiendo primeramente quitado la viga de los vuestros, podréis ver mejor cómo quitar la mota del ojo de vuestro hermano.
\vs p140 3:18 “Percibid la verdad claramente, vivid sin temor la vida recta y así seréis mis apóstoles y los embajadores de mi Padre. Habéis oído decir: ‘Si el ciego guía al ciego, ambos caerán en el hoyo’. Si queréis guiar otros al reino, vosotros mismos debéis caminar en la clara luz de la verdad viva. En todos los asuntos del reino, os exhorto a que mostréis juicio justo e intensa sabiduría. No deis lo santo a los perros, ni echéis vuestras perlas delante de los cerdos, no sea que las pisoteen y se vuelvan y os despedacen.
\vs p140 3:19 “Guardaos de los falsos profetas, que vienen a vosotros vestidos de ovejas, y son por dentro lobos rapaces. Por sus frutos los conoceréis. ¿Acaso se recogen uvas de los espinos o higos de los abrojos? Así, todo buen árbol da buenos frutos, pero el árbol malo da frutos malos. No puede el buen árbol dar malos frutos ni el árbol malo dar frutos buenos. Todo árbol que no da buen fruto es pronto cortado y echado en el fuego. Para entrar en el reino de los cielos, lo que cuenta es el motivo. Mi Padre conoce el corazón de los hombres y los juzga por sus anhelos internos y sus sinceras intenciones.
\vs p140 3:20 “Muchos me dirán en aquel gran día del juicio del reino: ‘¿No profetizamos en tu nombre, y en tu nombre echamos fuera demonios y en tu nombre hicimos obras portentosas?’ Pero les tendré que decir, ‘Nunca os conocí. ¡Apartaos de mí, falsos maestros!’. Pero a cualquiera que oye estas palabras y cumpla con sinceridad el cometido de representarme ante los hombres como yo he representado a mi Padre ante vosotros, hallará amplia y generosa entrada en mi servicio y en el reino del Padre celestial”.
\vs p140 3:21 \pc Los apóstoles nunca antes habían oído hablar a Jesús de este modo, pues lo hacía como quien tiene autoridad suprema. Bajaron de la montaña sobre el atardecer, pero nadie hizo ninguna pregunta a Jesús.
\usection{4. VOSOTROS SOIS LA SAL DE LA TIERRA}
\vs p140 4:1 El llamado “sermón de la montaña” no es el evangelio de Jesús. Es verdad que contiene muchas provechosas enseñanzas, pero consistía más bien en el cometido dado por Jesús a los apóstoles en su ordenación. Fue el encargo personal del Maestro a quienes seguirían predicando el evangelio y aspiraban a representarlo en el mundo de los hombres tal como él, que de forma tan elocuente y perfectamente representaba a su Padre.
\vs p140 4:2 \pc “\bibemph{Vosotros sois la sal de la tierra, una sal con sabor de salvación. Pero si la sal pierde su sabor, ¿con qué será salada? No sirve más para nada, sino para ser echada fuera y pisoteada por los hombres”.}
\vs p140 4:3 En los tiempos de Jesús la sal era un bien valioso. Incluso se usaba como moneda. La palabra moderna “salario” procede de la palabra “sal”. La sal no solo da sabor a los alimentos, sino que es también un conservante. Hace que otras cosas sean más sabrosas y, por tanto, sirve al darle uso.
\vs p140 4:4 \pc “\bibemph{Vosotros sois la luz del mundo. Una ciudad asentada sobre un monte no se puede esconder. Ni se enciende una luz y se pone debajo de un celemín, sino sobre el candelero para que alumbre a todos los que están en la casa. Así alumbre vuestra luz delante de los hombres, para que vean vuestras buenas obras y glorifiquen a vuestro Padre que está en los cielos”.}
\vs p140 4:5 Aunque la luz despeja la oscuridad, también puede resultar tan “cegadora” como para confundir y obstaculizar. Se nos exhorta a que permitamos que nuestra luz brille \bibemph{tanto} que nuestros hermanos se vean guiados hacia los nuevos caminos divinos de una vida enaltecida. Nuestra luz debe brillar de modo que no atraiga la atención hacia el yo. Hasta el propio oficio que se elija puede emplearse como un eficiente “reflector” que disemine esta luz de vida.
\vs p140 4:6 La solidez de carácter no depende de que \bibemph{no} hagamos el mal, sino de que hagamos el bien. El altruismo es el distintivo de la grandeza humana. Los más altos niveles de realización de uno mismo se alcanzan mediante la adoración y el servicio. La persona feliz y eficiente no está motivada por el miedo a hacer el mal, sino por el amor a hacer el bien.
\vs p140 4:7 \pc “\bibemph{Por sus frutos los conoceréis}”. El ser personal es esencialmente inmutable; lo que cambia ---lo que crece--- es el carácter moral. El mayor error de las religiones modernas es el negativismo. El árbol que no da fruto es “cortado y echado en el fuego”. El valor moral no puede basarse en la mera represión ---la obediencia al mandamiento “no harás”---. El miedo y la vergüenza son motivos indignos de la vida religiosa. La religión es únicamente válida cuando revela la paternidad de Dios y enaltece la hermandad de los hombres.
\vs p140 4:8 \pc Una eficiente filosofía se vida se forma mediante la combinación de la percepción cósmica y la suma de las propias reacciones emocionales hacia al medio social y económico. Recordad: aunque esencialmente no se puedan modificar los impulsos heredados, sí se pueden cambiar las respuestas emocionales a estos impulsos; así pues, la naturaleza moral es modificable, el carácter es mejorable. En un carácter sólido, las respuestas emocionales están integradas y armonizadas, dando, pues, como resultado un ser personal unificado. Si tal unificación es deficiente, se debilita la naturaleza moral y se engendra infelicidad
\vs p140 4:9 Sin un objetivo meritorio, la vida pierde su rumbo y se convierte en improductiva, y se produce mucha desdicha. El discurso de Jesús en la ordenación de los doce constituye una filosofía magistral de la vida. Jesús exhortó a sus seguidores a que ejercitaran una fe vivencial. Les advirtió que no estuvieran supeditados al mero asentimiento intelectual, a la credulidad y a la autoridad establecida.
\vs p140 4:10 La educación debe ser un modo de aprender (de descubrir) el mejor modo de satisfacer los impulsos naturales y heredados, y la felicidad es el total resultante de mejores prácticas de satisfacción emocional. La felicidad depende poco del entorno, aunque un ambiente placentero puede contribuir en gran medida a ella.
\vs p140 4:11 \pc Realmente, el anhelo de cualquier mortal es ser una persona completa, ser perfecto como el Padre de los cielos es perfecto, y dicho logro es posible porque, en definitiva, el “universo es verdaderamente paternal”.
\usection{5. AMOR PATERNAL Y AMOR FRATERNAL}
\vs p140 5:1 Desde el sermón de la montaña hasta el discurso de la última cena, Jesús enseñó a sus seguidores a profesar amor \bibemph{paternal} en lugar de amor \bibemph{fraternal}. El amor fraternal conllevaría amar al prójimo como a uno mismo, lo que significaría cumplir debidamente la “regla de oro”. Pero el afecto paternal requiere que améis a vuestros semejantes mortales como Jesús os ama a vosotros.
\vs p140 5:2 Jesús ama a la humanidad con un doble afecto. Vivió en la tierra como una persona de doble naturaleza: humana y divina. Como Hijo de Dios, ama al hombre con amor paternal ---es el creador del hombre, su Padre en el universo---. Como Hijo del Hombre, Jesús ama a los mortales como un hermano ---verdaderamente fue un hombre entre los hombres---.
\vs p140 5:3 Jesús no esperaba que sus seguidores manifestaran un amor fraternal que fuese imposible de poner en práctica, pero sí esperaba que se esforzaran por ser como Dios ---perfectos como el Padre de los cielos es perfecto---, que mirasen al hombre como Dios mira a sus criaturas y pudieran, por tanto, comenzar a amar a los hombres como Dios los ama ---empezar a expresar un afecto paternal---. Durante sus exhortaciones a los doce apóstoles, Jesús pretendió revelar este nuevo concepto del \bibemph{amor paternal} recurriendo a determinadas actitudes emocionales que permiten a una persona realizar muchas adaptaciones sociales al entorno.
\vs p140 5:4 \pc El Maestro presentó este memorable discurso llamando la atención hacia cuatro actitudes de \bibemph{fe,} como preludio de la posterior descripción de sus cuatro reacciones trascendentales y supremas de amor paternal en comparación con las limitaciones del mero amor fraternal.
\vs p140 5:5 Habló primeramente de quienes eran pobres de espíritu, tenían sed de justicia, permanecen en mansedumbre y eran de corazón limpio. De estos mortales que perciben el espíritu cabría esperar que lograsen tales niveles de abnegación divina como para tratar de llevar a cabo el magnífico ejercicio del \bibemph{afecto paternal;} que, incluso en su aflicción, estarían facultados para mostrar misericordia, fomentar la paz y padecer persecuciones y, en todas estas situaciones difíciles, amar con un amor paternal incluso a una humanidad poco hermosa. El afecto de un padre puede alcanzar unos niveles de devoción que trasciendan inmensurablemente el afecto de un hermano.
\vs p140 5:6 La fe y el amor de estas bienaventuranzas refuerzan el carácter moral y generan felicidad. El miedo y la ira debilitan el carácter y destruyen la felicidad. Este memorable sermón se inició en un tono de dicha.
\vs p140 5:7 \li{1.}\bibemph{“Dichosos los pobres de espíritu ---los humildes---”}. Para un niño, la felicidad significa satisfacer un deseo placentero inmediato. El adulto está dispuesto a sembrar las semillas de la abnegación a fin de poder recolectar más tarde una felicidad mayor. En los tiempos de Jesús y, desde entonces, la felicidad se ha relacionado demasiado frecuentemente con la idea de la posesión de las riquezas. En la historia del fariseo y del publicano que oraban en el templo, uno de ellos se sentía rico en espíritu ---era egocéntrico---; el otro se sentía “pobre de espíritu” ---era humilde---. Uno era autosuficiente; el otro estaba abierto a aprender y buscaba la verdad. Los pobres de espíritu tienen como meta la riqueza espiritual ---Dios---. Y estos buscadores de la verdad no tienen que esperar sus recompensas en un futuro distante; se les recompensa \bibemph{ahora}. Encuentran el reino de los cielos en sus propios corazones, y vivencian \bibemph{ahora} esa felicidad.
\vs p140 5:8 \li{2.}“\bibemph{Dichosos los que tienen hambre y sed de rectitud, porque serán saciados”}. Solo los que se sienten pobres de espíritu tienen hambre de rectitud. Solo los humildes buscan la fuerza divina y anhelan el poder espiritual. Si bien, es muy peligroso practicar intencionadamente el ayuno espiritual con el objeto de aumentar el apetito por los dones espirituales. El ayuno físico se vuelve peligroso a los cuatro o cinco días, porque existe propensión a perder el deseo por los alimentos. Un largo período de ayuno, ya sea físico o espiritual, hace que el hambre desaparezca.
\vs p140 5:9 Experimentar rectitud es un placer, no un deber. La rectitud de Jesús se basa en la dinámica del amor ---en el afecto paterno\hyp{}fraternal---. No es el tipo de rectitud negativa o no lo hagas. ¿Cómo se puede sentir hambre de algo negativo ---de algo que “no se hace”?---.
\vs p140 5:10 \pc No es fácil enseñar estas dos primeras bienaventuranzas a una mente infantil, pero la mente madura debería ser capaz de captar su trascendencia.
\vs p140 5:11 \li{3.}\bibemph{“Dichosos los mansos, porque recibirán la tierra por heredad”}. La genuina mansedumbre no guarda relación con el miedo. Es más bien la actitud del hombre que colabora con Dios ---“hágase tu voluntad”---. Incluye la paciencia y la indulgencia y está motivada por la fe inquebrantable en un universo ordenado y afable. La mansedumbre ejerce dominio sobre cualquier tentación de rebelarse contra el gobierno divino. En Urantia, la mansedumbre de Jesús fue la idónea, y heredó un universo inmenso.
\vs p140 5:12 \li{4.}“\bibemph{Dichosos los puro de corazón, porque ellos verán a Dios}”. La pureza espiritual no es una cualidad negativa, excepto que está exenta de desconfianza y agravio. Al aludir a la pureza, Jesús no intentaba abordar exclusivamente las actitudes sexuales del hombre. Se refería más a esa fe que el hombre debe tener en su prójimo; esa fe que un padre deposita en su hijo, y que le permite amar a sus semejantes tal como un padre los amaría. El amor de un padre no tiene necesidad de consentir, y no tolera el mal, no obstante, es siempre confiado. El amor paternal tiene un único propósito, y siempre busca lo mejor del hombre; esa es la actitud de un auténtico padre.
\vs p140 5:13 Ver a Dios ---por la fe--- significa adquirir una verdadera percepción espiritual. Y la percepción espiritual favorece la guía del modelador y ambos, en último término, aumentan la conciencia de Dios. Cuando conoces al Padre, se ratifica tu filiación divina, y puedes amar más y más a cada uno de tus hermanos en la carne, no solo como un hermano ---con amor fraternal--- sino también como un padre ---con afecto paternal---.
\vs p140 5:14 Es fácil impartir este consejo incluso a un niño. Los niños son confiados de forma natural, y los padres deberían velar para que no pierdan esa sencilla fe. En vuestro trato con los niños, evitad todo engaño y absteneros de suscitar en ellos la desconfianza. Sensatamente, ayudadlos a escoger sus héroes y a seleccionar su labor de vida.
\vs p140 5:15 \pc Y, entonces, Jesús siguió instruyendo a sus seguidores en el logro del principal propósito de cualquier empeño humano, la perfección, e incluso de la divina. Siempre los exhortaba: “Sed vosotros perfectos, como vuestro Padre que está en los cielos es perfecto”. No urgía a los doce a que amaran al prójimo como a sí mismos, algo que habría significado un logro encomiable y habría puesto de manifiesto la realización del amor fraternal. Los exhortaba en cambio a que amaran a su prójimo como él los había amado a ellos ---con un afecto \bibemph{paternal} al igual que fraternal---. Y lo ilustró destacando cuatro resultados supremos del amor paternal:
\vs p140 5:16 \li{1.}\bibemph{“Dichosos los afligidos, porque serán consolados”.} El llamado sentido común o la lógica más plausible jamás plantearían que la felicidad pudiese provenir de la aflicción. Pero Jesús no se refería a la aflicción externa u ostentosa, sino que aludía a un sentimiento emotivo de ternura. Es un grave error enseñar a los niños y a los jóvenes que es impropio de los hombres mostrar ternura o exteriorizar, de alguna manera, emociones o sufrimientos físicos. La compasión es una cualidad valiosa tanto del hombre como de la mujer. No es necesario ser insensible para ser varonil. Se trata de la forma equivocada de crear hombres valerosos. Los grandes hombres del mundo no han tenido miedo de manifestar su aflicción. Moisés, el doliente, fue un hombre más grande que Sansón o Goliat. Fue un formidable líder, pero también un hombre dotado de mansedumbre. Ser sensible y receptivo a las necesidades humanas genera una felicidad genuina y perdurable, a la vez que estas actitudes amables salvaguardan al alma de la acción destructiva de la ira, el odio y la desconfianza.
\vs p140 5:17 \li{2.}\bibemph{“Dichosos los misericordiosos, porque alcanzarán misericordia”.} Aquí la misericordia designa la altura y la profundidad y la anchura del grado de amistad más verdadero ---la benevolencia---. Algunas veces, la misericordia puede ser pasiva, pero aquí es activa y dinámica ---el amor supremo de un padre---. A un padre amoroso no le resulta difícil perdonar a su hijo, incluso muy repetidas veces. Y, en un niño no consentido, el impulso a aliviar el sufrimiento de los demás es natural. Los niños, cuando tienen la edad suficiente, son generalmente bondadosos y compasivos como para apreciar las circunstancias reales de la vida.
\vs p140 5:18 \li{3.}\bibemph{“Dichosos los pacificadores, porque serán llamados hijos de Dios”}. Quienes oían a Jesús anhelaban tener líderes que los liberaran militarmente, no pacificadores. Pero la paz de Jesús no es un tipo de paz pacífico ni negativo. Ante las pruebas y las persecuciones, él dijo: “Mi paz os dejo”. “No se turbe vuestro corazón ni tenga miedo”. Esta es la paz que evita conflictos devastadores. La paz personal que integra el ser personal. La paz social previene el miedo, la codicia y la ira. La paz política impide los antagonismos raciales, el recelo entre las naciones y la guerra. La construcción de la paz sana la desconfianza y el recelo.
\vs p140 5:19 Resulta fácil enseñar a los niños a obrar como pacificadores. Disfrutan con el trabajo en equipo; les gusta jugar juntos. El Maestro dijo en otro momento: “Todo el que quiera salvar su vida, la perderá; y todo el que pierda su vida la hallará”.
\vs p140 5:20 \li{4.}\bibemph{“Dichosos los que padecen persecución por causa de la rectitud, porque de ellos es el reino de los cielos}. Dichosos seréis cuando os insulten, os persigan y digan toda clase de mal contra vosotros, mintiendo. Gozaos y llenaos de alegría, porque vuestra recompensa es grande en los cielos”.
\vs p140 5:21 Muy a menudo la persecución sigue a la paz. Pero la gente joven y los adultos valerosos jamás huyen de las dificultades o del peligro. “Nadie tiene mayor amor que este, que uno ponga su vida por sus amigos”. Y el amor paternal puede hacer todo esto ampliamente ---algo que apenas puede asumir el amor fraternal---. Y siempre ha sido el progreso la cosecha final de la persecución.
\vs p140 5:22 Los niños siempre responden a cualquier cosa que desafíe su valentía. La juventud está siempre dispuesta a “aceptar un reto”. Y cualquier niño debe aprender pronto a sacrificarse.
\vs p140 5:23 \pc Y, así, se pone de manifiesto que las bienaventuranzas del Sermón de la montaña se basan en la fe y en el amor y no en la ley ---o en la ética y el deber---.
\vs p140 5:24 \pc El amor paternal se complace en devolver bien por mal ---en hacer el bien en represalia por la injusticia---.
\usection{6. LA NOCHE DE LA ORDENACIÓN}
\vs p140 6:1 El domingo por la noche, al llegar a la casa de Zebedeo desde las mesetas del norte de Cafarnaúm, Jesús y los doce tomaron una sencilla cena. Más tarde, mientras Jesús fue a dar un paseo por la playa, los doce hablaron entre ellos. Tras una breve charla, mientras los gemelos encendían un pequeño fuego que los calentase y les diese más luz, Andrés fue a buscar a Jesús, y cuando llegó a él, le dijo: “Maestro, mis hermanos son incapaces de comprender lo que has dicho sobre el reino. No nos sentimos capaces de comenzar esta labor hasta que no nos des a conocerlo algo más. He venido para pedirte que te unas a nosotros en el jardín y nos ayudes a entender el significado de tus palabras”. Y Jesús fue con Andrés para reunirse con los apóstoles.
\vs p140 6:2 Cuando entró al jardín, Jesús congregó a los apóstoles a su alrededor y les impartió otras enseñanzas, diciéndoles: “Os resulta difícil comprender lo que os explico porque queréis edificar las nuevas enseñanzas directamente sobre las antiguas, pero yo os declaro que debéis renacer. Debéis recomenzar como niños pequeños y estar dispuestos a confiar en mis enseñanzas y a creer en Dios. El nuevo evangelio del reino no se pude hacer conformar a aquello que ya es. Albergáis ideas equivocadas sobre el Hijo del Hombre y su misión en la tierra. Pero cometéis el error de pensar que he venido a desechar la Ley y los Profetas; no he venido a abolir, sino a cumplir, a engrandecer y a iluminar. No he venido para transgredir la Ley sino para grabar estos nuevos mandamientos en las tablas de vuestro corazón.
\vs p140 6:3 “Demando de vosotros una rectitud que exceda la de aquellos que procuran obtener el favor del Padre con la limosna, la oración y el ayuno. Si queréis entrar al reino, debéis caracterizaros por una rectitud que conste de amor, misericordia y verdad ---el deseo sincero de hacer la voluntad de mi Padre del cielo” ---.
\vs p140 6:4 Entonces, Simón Pedro dijo: “Maestro, si tienes un nuevo mandamiento, nos gustaría oírlo. Revélanos el nuevo camino”. Jesús respondió a Pedro: “Habéis oído decir de quienes enseñan la Ley: ‘No matarás; que cualquiera que mate se someterá a juicio’. Pero yo miro más allá del acto, para desvelar el motivo. Os declaro que cualquiera que se enoje con su hermano tiene peligro de condena. Aquel que alimenta el odio en su corazón y planea en su mente la venganza corre el riesgo de ser culpable de juicio. Vosotros habéis de juzgar a vuestros semejantes por sus acciones; el Padre celestial juzga por las intenciones.
\vs p140 6:5 “Habéis oído decir a los maestros de la Ley: ‘No cometerás adulterio’. Pero yo os digo que cualquiera que mira a una mujer para codiciarla, ya adulteró con ella en su corazón. Vosotros solamente podéis juzgar a los hombres por sus actos, pero mi Padre mira el corazón de sus hijos y los juzga con misericordia conforme a sus intenciones y deseos reales”.
\vs p140 6:6 Jesús estaba resuelto a seguir analizando los otros mandamientos, cuando Santiago Zebedeo le interrumpió, preguntándole: “Maestro, ¿qué enseñaremos a la gente sobre el divorcio? ¿Permitiremos que un hombre divorcie a su mujer tal como Moisés dispuso?”. Cuando Jesús oyó esta pregunta, dijo: “No he venido para legislar sino para iluminar. No he venido para reformar los reinos de este mundo sino más bien para instaurar el reino de los cielos. No es la voluntad de mi Padre que yo ceda ante la tentación de enseñaros reglas de gobierno, comercio o comportamiento social que, aunque puedan ser buenas para el día de hoy, distarían de ser apropiadas para la sociedad en otras eras. Estoy en la tierra solamente para confortar las mentes, liberar los espíritus y salvar las almas de los hombres. Pero, sobre esta cuestión del divorcio, diré que aunque Moisés mostrara su favor hacia tales cosas, no era así en los tiempos de Adán y en el Jardín”.
\vs p140 6:7 Después de que los apóstoles hablaran brevemente entre ellos, Jesús añadió: “Siempre debéis reconocer las dos perspectivas de cualquier conducta mortal ---la humana y la divina; los caminos de la carne y la vía del espíritu; la estimación del tiempo y el punto de vista de la eternidad---”. Y aunque los doce no podían comprender todo lo que se les enseñaba, esta clarificación les sirvió verdaderamente de ayuda.
\vs p140 6:8 Y entonces Jesús dijo: “Pero tropezaréis con mis enseñanzas porque queréis interpretar mi mensaje de forma literal; sois lentos en percibir el espíritu de mi enseñanza. Nuevamente, debéis recordar que sois mis mensajeros; habréis de vivir vuestra vida como yo he vivido la mía en espíritu. Sois mis representantes personales; pero no os equivoquéis y esperéis que todos los hombres vivan como vosotros en todos los aspectos. Debéis igualmente recordar que tengo otras ovejas que no son de este redil, y a las que también estoy obligado a proporcionarles la pauta a seguir para hacer la voluntad de Dios, mientras vivo la vida de la naturaleza mortal”.
\vs p140 6:9 Entonces preguntó Natanael: “Maestro, ¿no daremos cabida a la justicia? La Ley de Moisés dice, ‘ojo por ojo y diente por diente’. ¿Qué diremos nosotros?”. Y Jesús contestó: “Vosotros devolveréis bien por mal. Mis mensajeros no deben pelear con los hombres, sino mostrarle afabilidad. Vuestra regla no será con la medida con que medís se os medirá. Los gobernantes de los hombres pueden tener estas leyes, pero en el reino no es así; la misericordia siempre dictará vuestro juicio y el amor vuestro proceder. Si lo que digo os parece duro, todavía podéis volveros atrás. Si encontráis que los requisitos del apostolado son demasiado exigentes, podéis regresar al sendero menos estricto del discipulado”.
\vs p140 6:10 Al escuchar estas sorprendentes palabras, los apóstoles se apartaron a un lado durante un rato, pero regresaron pronto, y Pedro dijo: “Maestro, queremos continuar contigo; ninguno de nosotros dará marcha atrás. Estamos totalmente preparados para pagar el sobreprecio; beberemos de la copa. Queremos ser apóstoles y no meramente discípulos”.
\vs p140 6:11 Cuando Jesús oyó esto, dijo: “Estad dispuestos, pues, a asumir vuestra responsabilidad y a seguirme. Haced vuestras buenas obras en secreto; cuando deis limosna, que no sepa vuestra mano izquierda lo que hace vuestra derecha. Cuando oréis, iros a solas y no empleéis repeticiones vanas ni frases sin sentido. Recordad siempre que el Padre sabe de vuestras necesidades antes de que se las pidáis. Y, cuando ayunéis, no pongáis cara triste para que los hombres la vean. Como mis apóstoles elegidos, apartados ahora para servir al reino, no os hagáis tesoros en la tierra, sino que, mediante vuestro abnegado servicio, haceos tesoros en el cielo, porque, donde vuestros tesoros estén, allí estarán también vuestros corazones.
\vs p140 6:12 “La lámpara del cuerpo es el ojo; así que, si tu ojo es bueno, todo tu cuerpo estará lleno de luz; pero si tu ojo es maligno, todo tu cuerpo estará en tinieblas. Así que, si la luz que hay en ti es tinieblas, ¿cuántas no serán las mismas tinieblas?”.
\vs p140 6:13 Y, entonces, Tomás preguntó a Jesús si debían “continuar teniéndolo todo en común”. Dijo el Maestro: “Sí, hermanos míos; quisiera que viviéramos todos juntos como una familia bien allegada. Se os encomienda una gran tarea, y deseo vuestro pleno servicio. Sabéis que es verdad lo que se ha dicho de que ‘ningún hombre podrá servir a dos señores’. No podéis adorar sinceramente a Dios y al mismo tiempo servir incondicionalmente a las riquezas. Habiéndoos incorporado ahora, sin reservas, a la tarea del reino, no os angustiéis por vuestra vida ni os preocupéis mucho menos por lo que habéis de comer o beber; ni por vuestros cuerpos, qué habéis de vestir. Ya habéis aprendido que unas manos dispuestas y unos corazones fervientes no padecerán hambre. Ahora, cuando os preparáis para dedicar todas vuestras fuerzas al trabajo del reino, estad seguros de que el Padre será consciente de vuestras necesidades. Buscad primero el reino de Dios y, cuando hayáis hallado la puerta de entrada, todas las cosas necesarias os serán dadas. Así que no os angustiéis por el día de mañana. Baste a cada día su propio afán.
\vs p140 6:14 Cuando vio que estaban todos dispuestos a pasar la noche en vela haciendo preguntas, Jesús les dijo: “Hermanos míos, sois vasijas de barro; es mejor que os vayáis a descansar y así estaréis listos para el trabajo de mañana”. Pero el sueño se había ido de sus ojos. Pedro se aventuró a pedir a su Maestro “solo una breve charla privada contigo. No es que quiera ocultar nada de mis hermanos, pero mi espíritu está turbado y si, tal vez, merezco una reprimenda de mi Maestro, podría soportarla mejor a solas contigo”. Y Jesús le dijo: “Ven conmigo, Pedro” ---mostrándole el camino hacia la casa. Cuando Pedro regresó de estar en presencia de su Maestro muy alegre y bastante animado, Santiago decidió entrar para hablar con Jesús. Y, así sucesivamente, hasta tempranas horas de la mañana, los demás apóstoles entraron uno a uno para hablar con el Maestro. Cuando todos habían mantenido conversaciones privadas con él salvo los gemelos, que se habían quedado dormidos, Andrés entró a ver a Jesús y le dijo: “Maestro, los gemelos se han quedado dormidos en el jardín junto al fuego; ¿los debo despertar para ver si quieren también hablar contigo?”. Y Jesús, sonriente, le dijo a Andrés: “Hacen bien ---no los molestes”. Y así la noche fue pasando; alboreaba la luz de un nuevo día.
\usection{7. LA SEMANA SIGUIENTE A LA ORDENACIÓN}
\vs p140 7:1 Después de dormir solo unas horas, cuando los doce estaban reunidos con Jesús desayunando tarde, él dijo: “Ahora debéis comenzar vuestra labor de predicar la buena nueva y enseñar a los creyentes. Estad listos para ir a Jerusalén”. Tras haber hablado Jesús, Tomás hizo acopio de valor para decir: “Maestro, sé que debemos estar preparados para comenzar el trabajo, pero me temo que todavía no nos sintamos capacitados para llevar a cabo tal ingente obra. ¿Darías tu consentimiento para que nos quedáramos por aquí algunos días más antes de comenzar la tarea del reino?”. Y, cuando Jesús vio que esa misma aprehensión se había adueñado de todos los apóstoles, dijo: “Será como habéis pedido; estaremos aquí hasta el día del \bibemph{sabbat}”.
\vs p140 7:2 \pc Semanas tras semanas llegaban a Betsaida pequeños grupos de fervorosos buscadores de la verdad junto otros simplemente curiosos para ver a Jesús. Las noticias sobre él ya se habían extendido por toda la región; llegaban grupos de personas deseosas de saber desde ciudades tan lejanas como Tiro, Sidón, Damasco, Cesarea y Jerusalén. Hasta ese momento, Jesús había recibido a esas personas y les había enseñando acerca del reino, pero el Maestro puso este cometido en manos de los doce. Andrés elegía a uno de los apóstoles y lo asignaba a un grupo de visitantes; a veces los doce al completo estaban ocupados en dicha tarea.
\vs p140 7:3 Trabajaron durante dos días, enseñando de día y manteniendo charlas privadas hasta altas horas de la noche. Al tercer día, Jesús se quedó conversando con Zebedeo y Salomé una vez que se despidió de sus apóstoles diciéndoles: “Salid a pescar, tomaos para variar algún tiempo de esparcimiento o id quizás a visitar a vuestras familias”. El jueves regresaron para recibir tres días más de instrucción.
\vs p140 7:4 Durante esta semana de práctica, Jesús reiteró muchas veces a sus apóstoles los dos grandes motivos de su misión posbautismal en la tierra:
\vs p140 7:5 \li{1.}Revelar el Padre al hombre.
\vs p140 7:6 \li{2.}Guiar a los hombres a que tomen conciencia de su filiación ---a que se den cuenta por medio de la fe de que son los hijos del Altísimo---.
\vs p140 7:7 \pc Una semana plena en distintas experiencias sirvió de bastante ayuda a los doce; algunos incluso comenzaron a sentirse demasiado seguros de sí mismos. En la última charla que mantuvieron, la noche después del \bibemph{sabbat,} Pedro y Santiago se acercaron a Jesús diciendo: “Estamos listos; salgamos ahora a conquistar el reino”. A lo que Jesús contestó: “Que vuestra sabiduría sea igual a vuestro entusiasmo y vuestro valor compense vuestra ignorancia”.
\vs p140 7:8 Aunque los apóstoles no lograban entender muchas de sus enseñanzas, no por ello dejaron de apreciar la trascendencia de la magníficamente hermosa vida que él vivió con ellos.
\usection{8. EL JUEVES POR LA TARDE EN EL LAGO}
\vs p140 8:1 Jesús sabía muy bien que sus apóstoles no estaban asimilando por completo sus enseñanzas. Decidió dar a Pedro, Santiago y Juan orientaciones especiales, esperando que ellos pudiesen aclarar las ideas de sus compañeros. Percibía que, aunque los doce captaban algunos aspectos de la idea de un reino espiritual, persistían obcecadamente en asociar estas nuevas enseñanzas espirituales directamente con sus antiguos y firmemente enraizados conceptos terrenales del reino de los cielos. Para ellos, el reino de los cielos significaba la restauración del trono de David y el restablecimiento de Israel en la tierra como poder temporal. Así pues, el jueves por la tarde, Jesús, con Pedro, Santiago y Juan, se alejó en barca de la costa para hablarles de los asuntos del reino. La instructiva charla, que duró cuatro horas, consistió en decenas de preguntas y respuestas y, para facilitar la explicación de su contenido, se podría reorganizar el resumen de esta crucial tarde siguiendo el relato de los hechos tal como Simón Pedro se los expuso a su hermano Andrés a la mañana siguiente:
\vs p140 8:2 \li{1.}\bibemph{Hacer la voluntad del Padre.} Las enseñanzas de Jesús de confiar en los atentos cuidados del Padre celestial no conllevaban un fatalismo ciego y pasivo. Esa tarde, citó, dándole su aprobación, un antiguo dicho hebreo: “Si alguno no quiere trabajar, tampoco coma”. Se refirió a su propia experiencia para ejemplificar suficientemente sus enseñanzas. Sus preceptos sobre la confianza en el Padre no deben considerarse sobre la base de las condiciones sociales o económicas de los tiempos modernos o de ninguna otra era. Sus enseñanzas constituyen los principios ideales para vivir cerca de Dios en todas las eras y en todos los mundos.
\vs p140 8:3 Jesús explicó claramente a los tres la diferencia entre los requisitos exigidos del apostolado y del discipulado. E, incluso entonces, no prohibió a los doce que fuesen prudentes y previsores. No se oponía a la previsión sino a la ansiedad, a la angustia. Instruía en la sumisión activa y alerta a la voluntad de Dios. En respuesta a muchas de las preguntas de los tres apóstoles sobre la frugalidad y el ahorro, se limitó a ponerles de relieve su vida como carpintero, fabricante de embarcaciones y pescador, y su minuciosa organización de los doce. Procuró dejarles claro que el mundo no debía considerarse un enemigo; que las circunstancias de la vida son parte de un plan divino que obra en conjunción con los hijos de Dios.
\vs p140 8:4 A Jesús le fue muy difícil hacerles comprender su práctica personal de la no confrontación. Se oponía tajantemente a defenderse a sí mismo, y a los apóstoles les dio la impresión de que le hubiese gustado que ellos siguieran esa misma norma. Les enseñó a no confrontar el mal, a no combatir la injusticia o la injuria, pero no enseñó tolerancia pasiva a la maleficencia. Y esa tarde les puso de manifiesto que aprobaba el castigo social de los malvados y los delincuentes, y que el gobierno civil debía emplear algunas veces la fuerza para mantener el orden social y ejercer la justicia.
\vs p140 8:5 Nunca dejó de alertar a sus discípulos contra la perniciosa práctica de la \bibemph{represalia;} no consentía la venganza, la idea del ajuste de cuentas. Reprobaba que se guardase rencor. Desautorizaba la idea de ojo por ojo y diente por diente. Desaprobaba cualquier noción de venganza privada y personal, dejando estas cuestiones en manos del gobierno civil, por una parte, y al juicio de Dios, por otra. Indicó claramente a los tres que sus enseñanzas se aplicaban a los seres humanos \bibemph{a nivel individual,} y no al Estado. Resumió las enseñanzas impartidas hasta ese momento sobre estas cuestiones de la siguiente manera:
\vs p140 8:6 Amad a vuestros enemigos ---recordad las exigencias morales de la hermandad de los hombres---.
\vs p140 8:7 La inutilidad del mal: una ofensa no se enmienda con la venganza. No cometáis el error de luchar contra el mal con sus propias armas.
\vs p140 8:8 Tened fe ---confianza en el triunfo futuro de la justicia divina y de la bondad eterna---.
\vs p140 8:9 \li{2.}\bibemph{Actitud política}. Advirtió a sus apóstoles que fuesen discretos en sus comentarios respecto a las tensas relaciones entonces existentes entre el pueblo judío y el gobierno romano; les prohibió que se vieran envueltos de manera alguna en estas complicaciones. Tuvo siempre cuidado de evitar las trampas políticas que sus enemigos le tendían, constantemente respondiendo: “Dad al césar lo que es del césar y a Dios lo que es de Dios”. Se negaba a que se desviara su atención de la misión de establecer un camino de salvación nuevo; no se permitía a sí mismo preocuparse por cualquier otra cosa. En su vida personal, siempre observó debidamente todas las leyes y normas civiles; en todas sus enseñanzas públicas hizo caso omiso de los ámbitos cívicos, sociales y económicos. Les dijo a los tres apóstoles que a él solamente le concernían los valores de la vida espiritual interior y personal del hombre.
\vs p140 8:10 Jesús no fue, por lo tanto, un reformador político. No vino para reorganizar el mundo; incluso si lo hubiese hecho, las reformas políticas únicamente podrían haberse aplicado a ese día y generación. No obstante, mostró al hombre la mejor forma de vivir, y ninguna generación está exenta de la labor de descubrir el mejor modo de adaptar la vida de Jesús a sus propios problemas. Pero nunca cometáis el error de identificar las enseñanzas de Jesús con alguna teoría política o económica, con algún sistema social o industrial.
\vs p140 8:11 \li{3.}\bibemph{Actitud social}. Desde hacía mucho tiempo, los rabinos judíos habían tratado de dar respuesta a la pregunta: ¿Quién es mi prójimo? Jesús vino a exponer la idea de una benevolencia activa y espontánea, de un amor por el prójimo tan genuino que ampliaba el concepto de aquellos cercanos hasta abarcar todo el mundo, convirtiendo a todos los hombres en prójimo. Pero pese a todo esto, Jesús estaba interesado solamente en el ser humano \bibemph{a nivel individual,} no en la muchedumbre. Jesús no era sociólogo, pero se ocupó de acabar con todas las formas de aislamiento insolidario. Enseñó comprensión pura, compasión. Miguel de Nebadón era un Hijo misericordioso en esencia; su naturaleza misma era la compasión.
\vs p140 8:12 El Maestro nunca dijo que los hombres no convidasen a sus amigos a comer, pero sí dijo que sus discípulos debían dar banquetes y llamar a los pobres y a las personas desafortunadas. Jesús tenía un fuerte sentido de la justicia, pero se trataba de una justicia siempre templada con la misericordia. No enseñó a sus apóstoles a que se sometieran a los dictados de los parásitos sociales o a los mendigos profesionales. Lo más próximo que estuvo de hacer un pronunciamiento sociológico fue: “No juzguéis, para que no seáis juzgados”.
\vs p140 8:13 Claramente manifestó que muchos males sociales pueden deberse a la benevolencia ejercida de forma indiscriminada. Al día siguiente, Jesús ordenó explícitamente a Judas que no se repartieran los fondos apostólicos como limosnas salvo a petición suya o previa solicitud conjunta de dos de los apóstoles. En todos estos asuntos, Jesús acostumbraba a decir: “Sed astutos como serpientes pero inocentes como palomas”. En cualquier situación social su propósito era enseñar paciencia, tolerancia y perdón.
\vs p140 8:14 La familia ocupaba el centro mismo de la filosofía de la vida de Jesús ---aquí y en el más allá---. Basaba sus enseñanzas acerca de Dios en la familia, procurando, al mismo tiempo, corregir la tendencia judía a honrar en exceso a los ancestros. Enalteció la vida familiar como el más elevado deber de la humanidad, pero dejó claro que las relaciones familiares no deben interferir con las obligaciones religiosas. Llamó la atención al hecho de que la familia es una institución temporal; que no sobrevive a la muerte. Jesús no dudó en renunciar a su familia cuando esta era contraria a la voluntad del Padre. Enseñó la nueva y más grande hermandad del hombre ---la de los hijos de Dios---. En los tiempos de Jesús, las costumbres sobre el divorcio tanto en Palestina como en todo el Imperio romano eran laxas. Y repetidamente se negó a dictar leyes respecto al matrimonio y al divorcio, pero muchos de los primeros seguidores de Jesús tenían fuertes convicciones sobre el divorcio y no vacilaron en atribuírselas a él. Todos los escritores del Nuevo Testamento, excepto Juan Marcos, se aferraron a estas ideas más rígidas y avanzadas sobre el divorcio.
\vs p140 8:15 \li{4.}\bibemph{Actitud hacia la economía}. Jesús trabajó, vivió y fue comerciante en el mundo con el que se encontró. No era un reformador económico, a pesar de que, con frecuencia, hacía hincapié en la injusticia del reparto desigual de la riqueza. Pero no aportó ninguna medida correctiva. Manifestó con claridad a los tres que, aunque sus apóstoles no debían poseer bienes, no se oponía en sus predicaciones a la riqueza ni la propiedad, sino exclusivamente a su distribución no equitativa e injusta. Reconocía la necesidad de la justicia social y de las prácticas laborales equitativas, pero no propuso ninguna regla para llevarlas a cabo.
\vs p140 8:16 Nunca enseñó a sus seguidores a abstenerse de las posesiones terrenales, sino solo a sus doce apóstoles. Lucas, el médico, fue un firme creyente en la equidad social, e hizo mucho por interpretar los dichos de Jesús en sintonía con sus creencias personales. Jesús jamás mandó personalmente a sus discípulos a que siguieran un modo de vida comunitario; no se pronunció de manera alguna sobre este tipo de cuestiones.
\vs p140 8:17 A menudo, Jesús advertía, a quienes le seguían, contra la avaricia, afirmando que “la felicidad de un hombre no consiste en la abundancia de sus posesiones materiales”. Reiteraba continuamente: “¿Qué aprovecha al hombre si gana todo el mundo y pierde su propia alma?”. No atacó directamente a la propiedad privada, pero de cierto insistió en que era eternamente esencial que los valores espirituales fuesen prioritarios. Con posterioridad, en sus enseñanzas, procuró corregir muchas de las nociones equivocadas que se albergaban en Urantia sobre la vida contando, en el trascurso de su ministerio público, numerosas parábolas. Jesús nunca tuvo la intención de formular teorías económicas; sabía bien que en cada época se debía encontrar el remedio a los problemas presentes. Y si, hoy en día, Jesús estuviera en la tierra, viviendo su vida en la carne, constituiría una gran decepción para una mayoría de los hombres y mujeres buenos, por la sencilla razón de que no se posicionaría en las disputas políticas, sociales o económicas de los tiempos presentes. Se mantendría majestuosamente al margen, mientras os enseñaba cómo perfeccionar vuestra vida espiritual interior a fin de capacitaros mucho más para hallar la solución a vuestros problemas puramente humanos.
\vs p140 8:18 \pc Jesús querría que todos los hombre se semejasen a Dios y, entonces, benévolamente, aguardaría cerca de ellos mientras estos hijos de Dios resolviesen sus propios problemas sociales, políticos y económicos. No denunciaba la riqueza, sino lo que la riqueza ocasiona a la mayor parte de sus adeptos. El jueves por la tarde, y por primera vez, Jesús dijo a sus acompañantes que “más bienaventurado es dar que recibir”.
\vs p140 8:19 \li{5.}\bibemph{Religión personal}. Vosotros, al igual que le sucedió a sus apóstoles, deberíais poder entender mejor las enseñanzas de Jesús a través de su propia vida. En Urantia, Jesús logró vivir una vida perfecta, y sus excepcionales enseñanzas solo pueden entenderse cuando se concibe esa vida en su inmediato contexto. Es su vida, y no su instrucción a los doce ni los sermones a las multitudes, la que más contribuirá a desvelar el carácter divino y la persona amorosa del Padre.
\vs p140 8:20 Jesús no arremetió contra las enseñanzas de los profetas hebreos ni las de los moralistas griegos. El Maestro reconocía las muchas cosas buenas que estos grandes pensadores propugnaban, pero había venido a la tierra para enseñar algo \bibemph{más,} “a que la voluntad del hombre se conformara voluntariamente con la voluntad de Dios”. Jesús no quería simplemente dar origen a un \bibemph{hombre religioso,} a un mortal que se ocupara por completo de los sentimientos religiosos y que estuviese motivado solamente por los impulsos espirituales. Si hubieseis puesto vuestra mirada en él una sola vez, habríais sabido que Jesús era un hombre real con gran experiencia en las cosas de este mundo. Las enseñanzas de Jesús en este sentido han sido flagrantemente distorsionadas y muy tergiversadas a lo largo de los siglos de la era cristiana; también se han desvirtuado las ideas sobre la mansedumbre y la humildad del Maestro. Su vida estaba encaminada a tener un \bibemph{formidable respeto por sí mismo}. Solo aconsejaba al hombre que se humillara a sí mismo para poder ser verdaderamente enaltecido; lo que realmente pretendía era la genuina humildad del hombre ante Dios. Le concedía un gran valor a la sinceridad ---a la pureza de corazón--- tener un corazón puro. La fidelidad era una virtud cardinal en su valoración del carácter, mientras que la \bibemph{valentía} constituía el núcleo de sus enseñanzas. “No temáis” era su lema y, la persistencia paciente, su ideal de la fuerza del carácter. Las enseñanzas de Jesús constituyen una religión de valor, arrojo y heroísmo. Y fue por ello por lo que eligió, para que lo representaran personalmente, a doce hombres comunes, mayoritariamente pescadores rudos, varoniles y pujantes.
\vs p140 8:21 Jesús aludió poco a los vicios sociales de su era; raras veces se refirió a la delincuencia moral. Era un maestro positivo de la verdadera virtud. Evitó intencionadamente el método negativo de impartir instrucción; se negaba a divulgar el mal. Ni siquiera fue un reformador moral. Sabía bien, y así lo enseñó a sus apóstoles, que los impulsos sensuales de la humanidad no se reprimen mediante la censura religiosa ni las prohibiciones legales. Las pocas denuncias que hizo iban dirigidas principalmente contra la soberbia, la crueldad, la opresión y la hipocresía.
\vs p140 8:22 Jesús ni siquiera denunció con rotundidad a los fariseos, como Juan lo hizo. Sabía que muchos de los escribas y fariseos eran honestos de corazón; entendía su esclavizante servidumbre a las tradiciones religiosas. Jesús insistía en “hacer primeramente bueno al árbol”. A los tres les impresionó que valorase la vida en su totalidad y no solamente algunas de sus especiales virtudes.
\vs p140 8:23 \pc Lo que Juan aprendió de este día de enseñanza fue que el núcleo de la religión de Jesús consistía en la adquisición de un carácter compasivo junto a un ser personal motivado a hacer la voluntad del Padre de los cielos.
\vs p140 8:24 Pedro logró entender la idea de que el evangelio que se disponían a proclamar era realmente un nuevo inicio para toda la raza humana. Con posterioridad, transmitiría esta opinión a Pablo, que, a partir de esta, definiría su doctrina de Cristo como “el segundo Adán”.
\vs p140 8:25 Santiago llegó a entender la emocionante verdad de que Jesús quería que sus hijos de la tierra vivieran como si fuesen, pues, ciudadanos del reino celestial, tal como llegaría a realizarse por completo en la tierra.
\vs p140 8:26 \pc Jesús sabía que cada hombre era diferente, y así se lo enseñó a sus apóstoles. Constantemente los exhortaba a que se abstuvieran de intentar adaptar a los discípulos y creyentes a un patrón fijo. Procuraba que cada alma se desarrollase a su propio modo, que fuese ante Dios un ser diferenciado que caminaba hacia la perfección. En respuesta a una de las numerosas preguntas que Pedro le formulaba, el Maestro dijo: “Quiero liberar a los hombres para que puedan recomenzar como niños una vida nueva y mejor”. Jesús siempre insistía que la verdadera bondad no debía ser intencionada, que, al hacer caridad, la mano izquierda no supiera lo que hacía la derecha.
\vs p140 8:27 Aquella tarde, los tres apóstoles se sorprendieron al percatarse de que la religión de su Maestro no contemplaba el autoexamen espiritual. Todas las religiones anteriores y posteriores a los tiempos de Jesús, incluso el cristianismo, establecen un escrupuloso examen de conciencia. Pero no es así en el caso de la religión de Jesús de Nazaret. La filosofía de vida de Jesús está desprovista de introspección religiosa. El hijo del carpintero nunca enseñó la \bibemph{edificación} del carácter, sino su \bibemph{crecimiento,} afirmando que el reino de los cielos es como un grano de mostaza. Pero Jesús no dijo nada que censurara el autoanálisis como forma de prevenir un egocentrismo presuntuoso.
\vs p140 8:28 La fe, la creencia personal, es la condición exigida para entrar en el reino. Continuar el progresivo ascenso en el reino tiene un coste; es la perla preciosa y, para poseerla, el hombre vende todo lo que tiene.
\vs p140 8:29 Las enseñanzas de Jesús constituyen una religión para todos, no solo para el débil o el esclavo. Su religión nunca llegó a cristalizarse (durante sus días) en credos y leyes teológicas; no dejó tras él ni una sola línea escrita. Su vida y sus enseñanzas fueron un legado al universo como un patrimonio, inspirador e ideal, apto para la guía espiritual y la instrucción moral de todas las eras en todos los mundos. E, incluso hoy día, las enseñanzas de Jesús se diferencian de todas las religiones, por sí misma, aunque representa la esperanza viva de cada una de ellas.
\vs p140 8:30 Jesús no impartió a sus apóstoles la enseñanza de que la religión debía ser el único afán del hombre en la tierra; esa era la idea judía de servir a Dios. Pero sí se empeñó en que la religión fuese la dedicación exclusiva de los doce. Jesús no enseñó nada que disuadiera a sus creyentes de la búsqueda de una cultura auténtica; tan solo desacreditó las escuelas religiosas de Jerusalén, aferradas a la tradición. Era de mente abierta, de gran corazón, culto y tolerante. La piedad pretenciosa no tenía cabida en su filosofía de la vida en rectitud.
\vs p140 8:31 El Maestro no aportó ninguna solución a los problemas no religiosos de su propia era ni a los de las eras posteriores. Jesús deseaba desarrollar la percepción espiritual de las realidades eternas y estimular la propia originalidad de vida; se ciñó exclusivamente a las necesidades espirituales, esenciales y permanentes, de la raza humana. Reveló una bondad igual a la de Dios. Enalteció el amor ---la verdad, la belleza y la bondad--- como el ideal divino y la realidad eterna.
\vs p140 8:32 El Maestro llegó para crear en el hombre un nuevo espíritu, una nueva voluntad ---para concederle la capacidad nueva de conocer la verdad, vivenciar la compasión y elegir la bondad---, la voluntad de estar en armonía con la voluntad de Dios, acompañada del impulso eterno de hacerse perfecto, tal como el Padre de los cielos es perfecto.
\usection{9. EL DÍA DE LA CONSAGRACIÓN}
\vs p140 9:1 Jesús dedicó el siguiente \bibemph{sabbat} a sus apóstoles, regresando a las mesetas donde los había ordenado; y, allí, tras un mensaje personal de aliento, extenso y bellamente emotivo, dio inicio al acto solemne de consagración de los doce. Esa tarde del \bibemph{sabbat,} en la ladera de la colina, Jesús reunió a los apóstoles a su alrededor y los puso en las manos de su Padre celestial, como preparación para el día en el que se viese avocado a dejarlos solos en el mundo. En esta ocasión, no hubo enseñanzas nuevas, simplemente conversación y comunión.
\vs p140 9:2 Jesús reexaminó muchos aspectos del sermón de ordenación, impartido en ese mismo lugar y, entonces, llamándolos uno a uno ante él, les encargó que salieran al mundo para representarle. El mandato de consagración dado por el Maestro fue: “Id a todo el mundo y predicad la buena nueva del reino. Liberad a los cautivos espirituales, consolad a los oprimidos y auxiliad a los afligidos. De gracia recibisteis, dad de gracia”.
\vs p140 9:3 Jesús les recomendó que no llevaran ni dinero ni ropa de sobra, diciendo: “El obrero es digno de su salario”. Y finalmente dijo: “He aquí que yo os envío como corderos en medio de lobos; sed, pues, astutos como serpientes pero inocentes como palomas. Pero guardaos, porque os entregarán a los consejos, y en las sinagogas os castigarán. Os llevarán delante de gobernadores y de reyes por creer en este evangelio, y esto os será ocasión para dar testimonio de mí ante ellos. Y cuando os presenten a juicio, no os preocupéis por lo que habéis de decir, porque el espíritu de mi Padre habita en vosotros y hablará en ese momento a través de vosotros. A algunos os darán muerte y, antes de que establezcáis el reino en la tierra, seréis odiados por muchos pueblos por causa de este evangelio; pero no temáis; yo estaré con vosotros, y mi espíritu os antecederá en todo el mundo. Y la presencia de mi Padre morará en vosotros cuando vayáis primeramente a los judíos y después a los gentiles”.
\vs p140 9:4 \pc Y cuando bajaron de la montaña, volvieron a casa de Zebedeo, en la que se estaban quedando.
\usection{10. LA NOCHE DESPUÉS DE LA CONSAGRACIÓN}
\vs p140 10:1 Esa noche, mientras impartía sus enseñanzas dentro de la casa porque había comenzado a llover, Jesús habló muy extensamente, intentando mostrar a los doce cómo debían \bibemph{ser,} no lo que debían \bibemph{hacer}. Ellos solo conocían una religión que les imponía \bibemph{hacer} determinadas cosas como medio de lograr la rectitud ---la salvación---. Pero Jesús reiteraba: “En el reino, debéis \bibemph{ser} rectos para hacer el trabajo”. Repitió muchas veces: “\bibemph{Sed} vosotros perfectos, como vuestro Padre que está en los cielos es perfecto”. Todo el tiempo, el Maestro explicaba a sus desconcertados apóstoles que la salvación que había venido a traer al mundo se conseguía solamente \bibemph{creyendo,} mediante la fe sencilla y sincera. Jesús decía: “Juan predicó un bautismo de arrepentimiento, de aflicción por la antigua forma de vivir. Vosotros debéis proclamar el bautismo de la fraternidad con Dios. Predicad arrepentimiento a los que lo necesiten, pero a quienes están ya sinceramente buscando entrar al reino, abridles las puertas de par en par e invitadles a entrar en la gozosa hermandad de los hijos de Dios”. Pero era una tarea difícil persuadir a estos pescadores galileos de que, en el reino, \bibemph{ser} rectos, a través de la fe, debía tener preferencia al hecho de \bibemph{actuar} rectamente en la vida diaria de los mortales de la tierra.
\vs p140 10:2 \pc Otro gran obstáculo en esta labor de enseñar a los doce era la tendencia que tenían a tomar los principios altamente idealistas y espirituales de la verdad religiosa y rehacerlos en reglas concretas de conducta personal. Jesús les exponía el bello espíritu de la actitud del alma, pero ellos se empeñaban en transformar tales enseñanzas en normas de comportamiento personal. Muchas veces, cuando estaban seguros de recordar lo que el Maestro decía, casi con toda certeza se olvidaban de lo que \bibemph{no} decía. Pero, lentamente, ellos asimilaban sus enseñanzas, porque Jesús \bibemph{era} todo lo que enseñaba. Lo que no podían conseguir de su instrucción oral, lo adquirían, poco a poco, en su vida con él.
\vs p140 10:3 No le quedaba claro a los apóstoles que su Maestro estaba viviendo una vida que inspirase espiritualmente a todas las personas de todas las eras de todos los mundos de un vasto universo. A pesar de que Jesús se lo mencionaba en ocasiones, los apóstoles no llegaban a entender la idea de que estaba realizando una labor \bibemph{en} este mundo pero \bibemph{para} todos los otros mundos de su inmensa creación. Jesús vivió su vida terrenal en Urantia, no para sentar un ejemplo personal de vida humana para los hombres y mujeres de este mundo, sino más bien para establecer un \bibemph{ideal altamente espiritual que sirviera de inspiración} a todos los seres mortales de todos los mundos.
\vs p140 10:4 \pc Esta misma noche, Tomás le preguntó a Jesús: “Maestro, dices que debemos volvernos como niños antes de poder entrar al reino del Padre y, sin embargo, nos has advertido de que no nos dejemos engañar por los falsos profetas ni que seamos culpables de echar nuestras perlas delante de los cerdos. Así que estoy francamente confundido. No puedo entender tus enseñanzas”. Jesús contestó a Tomás: “¿Hasta cuándo habré de soportaros? Persistís en ver todo lo que yo enseño en un sentido literal. Cuando os pedí que os hicierais como niños como precio para entrar al reino, no me refería a que cayerais fácilmente en el engaño, a estar meramente dispuestos a creer, ni tampoco a la rapidez en dar vuestra confianza a complacientes desconocidos. Lo que yo quería que dedujerais con esta ilustración era la relación entre hijo y padre. Tú eres el hijo, y es en el reino de \bibemph{tu} padre donde tratas de entrar. Existe ese afecto natural entre todo niño ordinario y su padre que asegura una relación basada en la comprensión y el amor, y que impide para siempre cualquier inclinación al regateo para lograr el amor y la misericordia del Padre. Y el evangelio que vais a salir a predicar está relacionado con la salvación que crece a partir de la toma de conciencia, mediante la fe, de esta misma y eterna relación entre un niño y su padre”.
\vs p140 10:5 \pc La excepcional característica de las enseñanzas de Jesús era que la \bibemph{moral} de su filosofía se fundamentaba en la relación personal de cada criatura con Dios ---esta misma relación entre el niño y su padre---. Jesús hacía hincapié en el ser humano \bibemph{a nivel individual,} no en la raza o en la nación. Mientras tomaban la cena, Jesús mantuvo una charla con Mateo en la que le explicó que la moral de cualquier acto está determinada por motivaciones individuales. La moral de Jesús era siempre positiva. La regla de oro, tal como Jesús la reelaboró, exige un contacto social activo; la regla negativa, más antigua, podía obedecerse en soledad. Jesús despojó de todas las reglas y ceremonias la moral y la elevó a niveles majestuosos de pensamiento espiritual y de vida verdaderamente en rectitud.
\vs p140 10:6 Esta nueva religión de Jesús no carecía de sus implicaciones prácticas, pero todo lo que sea de valor práctico a niveles político, social o económico que se pueda encontrar en sus enseñanzas es el resultado natural de esta experiencia interior del alma; y esta vivencia religiosa personal y genuina se manifiesta como los frutos del espíritu, espontáneamente, en el ministerio diario.
\vs p140 10:7 Cuando Jesús y Mateo acabaron de hablar, Simón el Zelote preguntó: “Pero, Maestro, ¿son \bibemph{todos} los hombres hijos de Dios?”. Y Jesús respondió: “Sí, Simón, todos los hombres son hijos de Dios, y esa es la buena nueva que proclamaréis”. Pero los apóstoles no lograban entender aquella doctrina; era una declaración nueva, insólita y sorprendente. Y fue su deseo de hacerles comprender esta verdad lo que llevó a Jesús a enseñar a sus discípulos a tratar a todos los hombres como a sus hermanos.
\vs p140 10:8 Respondiendo a una pregunta de Andrés, el Maestro manifestó con claridad que la moral de sus enseñanzas era inseparable de la religión de su propia vida. Enseñaba la moral, no derivada de la \bibemph{naturaleza} del hombre, sino de la \bibemph{relación} del hombre con Dios.
\vs p140 10:9 \pc Juan le preguntó a Jesús: “Maestro, ¿qué es el reino de los cielos?”. Y Jesús contestó: “El reino de los cielos está compuesto por tres elementos esenciales: primero, el reconocimiento del hecho de la soberanía de Dios; segundo, la creencia en la verdad de la filiación con Dios; y, tercero, la fe en la eficacia del supremo deseo humano de hacer la voluntad de Dios ---de semejarse a Dios---. Y esta es la buena nueva del evangelio: que por medio de la fe, cualquier mortal puede poseer estos tres elementos necesarios para la salvación”.
\vs p140 10:10 \pc Y ya la semana de espera había acabado, y se prepararon para partir hacia Jerusalén a la mañana siguiente.
