\upaper{158}{El monte de la transfiguración}
\author{Comisión de seres intermedios}
\vs p158 0:1 El viernes 12 de agosto del año 29 d. C., casi al atardecer, Jesús y sus acompañantes llegaron al pie del monte Hermón, cerca del mismo lugar en donde el joven Tiglat lo había esperado alguna vez mientras el Maestro subía por sí solo la montaña para decidir los destinos espirituales de Urantia y poner oficialmente fin a la rebelión de Lucifer. Y allí se quedaron durante dos días preparándose espiritualmente para los sucesos que no tardarían en producirse.
\vs p158 0:2 En términos generales, Jesús conocía de antemano lo que iba a suceder en la montaña, y estaba deseoso de que todos sus apóstoles pudieran ser partícipes de aquello con él. Con el fin de prepararlos para esta revelación de sí mismo, permaneció con ellos al pie de la montaña. Pero no podían alcanzar esos niveles espirituales de entendimiento que justificara el poder revelarles por completo la visión de la visita de los seres espirituales, que tan pronto aparecerían en la tierra. Y puesto que no podía llevar a todos sus acompañantes con él, decidió que solo lo hicieran los tres apóstoles que normalmente lo acompañaban en dichas vigilias especiales. Por ello, únicamente Pedro, Santiago y Juan pudieron al menos compartir con el Maestro alguna parte de esta extraordinaria experiencia.
\usection{1. LA TRANSFIGURACIÓN}
\vs p158 1:1 El lunes 15 de agosto, temprano por la mañana, Jesús y estos tres apóstoles comenzaron la subida al monte Hermón, algo que sucedió seis días después de la memorable confesión de Pedro ocurrida un mediodía, al borde del camino, bajo los árboles de morera.
\vs p158 1:2 Jesús había sido llamado a ascender a lo alto de la montaña, él solo, para abordar importantes cuestiones, que tenían que ver con el progreso de su ministerio de gracia en la carne al estar dicho ministerio relacionado con el universo creado por él. Es significativo que este extraordinario acontecimiento se iba a celebrar estando Jesús y sus apóstoles en las tierras de los gentiles y que, de hecho, ocurrió en una montaña de estas tierras.
\vs p158 1:3 Alcanzaron su destino casi a medio camino de ascenso por la montaña, algo antes del mediodía y, mientras almorzaban, Jesús contó a los tres apóstoles algo de sus experiencias en las colinas al este del Jordán, poco tiempo después de su bautismo, y también algo más sobre lo que le sucedió en el monte Hermón con motivo de su anterior visita a este lugar, apartado y solitario.
\vs p158 1:4 Jesús, cuando era niño, solía subir a la colina cercana de su casa y soñar con las batallas que se habían librado en la llanura de Esdraelón entre los ejércitos de distintos imperios; ahora, ascendía el monte Hermón para recibir las atribuciones que lo prepararían para descender a las llanuras del Jordán y acometer los actos finales de su ministerio de gracia en Urantia. Aquel día, en el Monte Hermón, el Maestro podría haber cejado en su lucha y haber vuelto a gobernar sobre los dominios de su universo, pero no solo optó por cumplir los requerimientos de su orden de filiación divina, incluidos en el mandato del Hijo Eterno del Paraíso, sino que también eligió dar respuesta a la voluntad final y plena de su Padre del Paraíso en ese momento. En aquel día de agosto, tres de sus apóstoles vieron cómo rechazó ser investido de la total autoridad sobre el universo. Con gran asombro, observaron cómo los mensajeros celestiales partían, dejándolo solo para que pudiera dar término a su vida en la tierra como el Hijo del Hombre y como el Hijo de Dios.
\vs p158 1:5 En el momento de la alimentación de los cinco mil, la fe de los apóstoles había llegado a su punto álgido, para luego caer rápidamente hasta casi su punto mínimo. Ahora, una vez que el Maestro había admitido su divinidad, la debilitada fe de los doce se elevó, durante las pocas semanas que siguieron, hasta su grado más alto, solo para después declinar paulatinamente. El tercer resurgimiento de su fe no ocurriría hasta después de la resurrección del Maestro.
\vs p158 1:6 Sobre las tres de aquella hermosa tarde, Jesús se despidió de los tres apóstoles, diciendo: “Me voy solo, durante un tiempo, para estar en comunión con el Padre y sus mensajeros; quedaos aquí y, mientras aguardáis mi regreso, orad para que se haga la voluntad del Padre en vosotros en lo que resta de la misión de gracia del Hijo del Hombre”. Y tras decirles esto, Jesús se retiró y mantuvo un largo encuentro con Gabriel y el Padre Melquisedec; eran sobre las seis de la tarde cuando volvió. Al notar la ansiedad de sus apóstoles, provocada por su prolongada ausencia, les dijo: “¿Por qué teníais miedo? Sabéis bien que debo ocuparme de los asuntos de mi Padre; ¿por qué dudáis cuando yo no estoy con vosotros? En este momento, os manifiesto que el Hijo del Hombre ha elegido seguir plenamente con su vida en medio de vosotros y como uno de vosotros. Tened buen ánimo; no os abandonaré hasta que no haya concluido mi trabajo”.
\vs p158 1:7 Mientras tomaban la exigua cena, Pedro preguntó al Maestro: “¿Cuánto tiempo estaremos en esta montaña alejados de nuestros hermanos?”. Y Jesús le contestó: “Hasta que veáis la gloria del Hijo del Hombre y sepáis que cuanto os he declarado es verdad”. Y comentaron los asuntos de la rebelión de Lucifer sentados alrededor de las brasas resplandecientes del fuego que habían encendido hasta que la oscuridad se les vino encima; a los apóstoles se les cerraban ya los ojos por el cansancio. Para llegar hasta allí, habían tenido que partir muy temprano aquella misma mañana.
\vs p158 1:8 Cuando los tres llevaban una media hora profundamente dormidos, un sonido cercano crepitante los despertó de súbito y, para su gran asombro y consternación, al mirar a su alrededor, vieron a Jesús conversando privadamente con dos seres resplandecientes, revestidos de luz del mundo celestial. Y el rostro y la figura de Jesús brillaban con la luminosidad de la luz celestial. Los tres se comunicaban en una lengua extraña, pero, por ciertas cosas que se dijeron, Pedro supuso, equivocadamente, que los seres que estaban con Jesús eran Moisés y Elías; en realidad, se trataba de Gabriel y del Padre Melquisedec. Por petición de Jesús, los controladores físicos habían hecho los ajustes necesarios para que los apóstoles pudieran ser testigos de aquella escena.
\vs p158 1:9 Los tres apóstoles estaban tan asustados que tardaron en volver a pensar con claridad, pero Pedro, que fue el primero en recobrar el sentido, dijo conforme la deslumbrante visión se desvanecía delante de ellos y viendo a Jesús solo: “Jesús, Maestro, es bueno haber estado aquí. Nos regocijamos de ver esta gloria. Somos reacios a volver abajo, al deplorable mundo. Si es tu voluntad, permanezcamos aquí, y erijamos tres tiendas, una para ti, una para Moisés y una para Elías”. Pedro dijo esto debido a su confusión y, porque en aquel preciso momento, no se le vino a la mente ninguna otra cosa.
\vs p158 1:10 Mientras Pedro decía esto, vino una nube plateada que hizo sombra a los cuatro. Los apóstoles entonces se asustaron mucho y, al postrarse sobre sus rostros para adorar, oyeron una voz, la misma que había hablado con motivo del bautismo de Jesús, que decía: “Este es mi Hijo amado; a él oíd”. Y cuando la nube se desvaneció, Jesús estuvo de nuevo solo con los tres e inclinándose los tocó, diciendo: “Levantaos y no temáis; veréis cosas más grandes que esta”. Pero los apóstoles estaban realmente asustados; estaban silenciosos y meditabundos cuando, poco antes de la media noche, se dispusieron a descender de la montaña.
\usection{2. EL DESCENSO DE LA MONTAÑA}
\vs p158 2:1 Durante casi la primera mitad de la bajada de la montaña, no dijeron ni una sola palabra. Jesús, entonces, inició la conversación: “Aseguraos de no decir a nadie, ni siquiera a vuestros hermanos, lo que habéis visto y oído en esta montaña hasta que el Hijo del Hombre no haya resucitado de los muertos”. Los tres apóstoles se quedaron conmocionados y perplejos por las palabras del Maestro “hasta que el Hijo del Hombre no haya resucitado de los muertos”. Hacía muy poco tiempo que habían reafirmado su fe en él como Libertador, como el Hijo de Dios, y lo acababan de contemplar transfigurado en gloria ante sus mismos ojos, ¡y ahora comenzaba a hablarles de “resucitar de los muertos!”.
\vs p158 2:2 Pedro se estremeció al pensar en la muerte del Maestro ---era una idea que le acongojaba--- y temiendo que Santiago o Juan pudieran hacer alguna pregunta sobre esta afirmación, creyó que sería mejor desviar la conversación; si bien, no sabiendo qué más decir, expresó el primer pensamiento que le vino a la mente, y que fue: “Maestro, ¿por qué dicen los escribas que es necesario que Elías venga primero antes de que lo haga el Mesías?”. Y Jesús, dándose cuenta de que Pedro trataba de evitar cualquier alusión a su muerte y resurrección, respondió: “A la verdad, Elías viene primero para preparar el camino del Hijo del Hombre, que habrá de padecer mucho y ser despreciado. Pero yo os digo que Elías ya vino, y no lo conocieron, sino que hicieron con él todo lo que quisieron”. Entonces los tres apóstoles comprendieron que les había hablado de Juan el Bautista como si fuera Elías. Jesús sabía que, si persistían en considerarlo a él como el Mesías, entonces Juan debía ser el Elías de la profecía.
\vs p158 2:3 Jesús les mandó que no comentaran el vislumbre de gloria que habían observado y que él recibiría tras su resurrección, porque, siendo ahora considerado como el Mesías, no quería dar a pensar que secundaba de ningún modo su equivocado concepto de un libertador hacedor de prodigios. Aunque Pedro, Santiago y Juan sopesaron todo aquello en sus mentes, no dijeron nada a nadie hasta después de la resurrección del Maestro.
\vs p158 2:4 Conforme continuaban descendiendo la montaña, Jesús les dijo: “No quisisteis reconocerme como el Hijo del Hombre; por eso, he accedido a que me aceptéis de acuerdo con vuestra firme convicción, pero, no os equivoquéis, la voluntad de mi Padre debe imperar. Si optáis pues por seguir la propensión de vuestra propia voluntad, debéis estar listos para sufrir muchas decepciones y pasar muchas pruebas, pero la formación que os he dado debería bastar para poder sobreponeros a las aflicciones que vosotros mismos habéis elegido.
\vs p158 2:5 Jesús llevó a Pedro, a Santiago y a Juan con él a la montaña de la transfiguración no porque estuvieran en sentido alguno mejor preparados que los otros apóstoles para ser testigos de lo que sucedió ni porque fuesen espiritualmente más aptos para gozar de tal excepcional privilegio. De ninguna manera. Sabía bien que ninguno de los doce era espiritualmente idóneo para este acontecimiento; así pues, se llevó con él solo a los tres apóstoles asignados a acompañarlo en esos momentos en los que deseaba estar solo y en solitaria comunión.
\usection{3. EL SIGNIFICADO DE LA TRANSFIGURACIÓN}
\vs p158 3:1 Lo que Pedro, Santiago y Juan presenciaron en la montaña de la transfiguración fue una visión fugaz de la celebración celestial ocurrida aquel día memorable en el monte Hermón. El motivo de la transfiguración fue:
\vs p158 3:2 \li{1.}El reconocimiento, por parte de la Madre\hyp{}Hijo Eterna del Paraíso, de la consumación del ministerio de gracia de la vida encarnada de Miguel en Urantia. En lo que respecta a los requisitos establecidos por el Hijo Eterno, Jesús recibió en ese momento la garantía de haberlos cumplido. Y fue Gabriel quien trajo a Jesús esa confirmación.
\vs p158 3:3 \li{2.}El testimonio de la satisfacción del Espíritu Infinito en cuanto a la consumación del ministerio de gracia en Urantia en semejanza de un hombre mortal. La representante en el universo del Espíritu Infinito, la compañera directa de Miguel en Lugar de Salvación y su colaboradora, siempre presente, habló, en esta ocasión, a través del Padre Melquisedec.
\vs p158 3:4 \pc Jesús acogió con beneplácito esta declaración del éxito de su misión en la tierra presentada a él por los mensajeros del Hijo Eterno y del Espíritu Infinito, pero observó que su Padre no reveló que su ministerio de gracia en Urantia había finalizado; la presencia invisible del Padre dio testimonio mediante el modelador personificado de Jesús, diciendo: “Este es mi Hijo amado; a él oíd”. Y esto se habló con palabras para que también las oyeran los tres apóstoles.
\vs p158 3:5 Tras esta visita celestial, Jesús buscó conocer la voluntad de su Padre y optó por proseguir su ministerio de gracia como mortal hasta su fin natural. Esto fue lo que la transfiguración implicó para Jesús. Para los tres apóstoles, aquel suceso significó la entrada del Maestro en la etapa final de su andadura en la tierra como el Hijo de Dios y el Hijo del Hombre.
\vs p158 3:6 Después de la visita a efectos formales de Gabriel y del Padre Melquisedec, Jesús departió de forma casual con ellos, sus Hijos en el ministerio, y conversó con ellos sobre los asuntos del universo.
\usection{4. EL MUCHACHO EPILÉPTICO}
\vs p158 4:1 Poco antes del desayuno, en la mañana de aquel martes, Jesús y sus acompañantes llegaron al campamento apostólico. A medida que se acercaban, vieron a una gran multitud alrededor de los apóstoles y empezaron pronto a oír voces de discusiones y disputas que venían de un grupo de unas cincuenta personas, entre las que estaban los nueve apóstoles y un igual número de escribas de Jerusalén y de discípulos creyentes, que habían seguido a Jesús y a sus apóstoles en su viaje desde Magadán.
\vs p158 4:2 Aunque la multitud se enzarzaba en numerosas polémicas, la principal era en relación a un ciudadano de Tiberias que había llegado buscando a Jesús el día antes. Este hombre, Santiago de Safed, tenía un único hijo de unos catorce años de edad, que estaba gravemente aquejado de epilepsia. Además de esta enfermedad nerviosa, el muchacho estaba poseído por uno de esos seres intermedios errantes, maliciosos y rebeldes, presentes en la tierra y sin control, por lo que el joven, además de ser epiléptico, estaba poseído por un demonio.
\vs p158 4:3 Durante casi dos semanas, este angustiado padre, oficial de menor rango de Herodes Antipas, había deambulado por las fronteras occidentales de los dominios de Felipe tratando de hallar a Jesús para rogarle que curara a su hijo enfermo. Y no alcanzó al grupo apostólico hasta alrededor del mediodía de ese día, cuando Jesús estaba en la montaña con los tres apóstoles.
\vs p158 4:4 Los nueve apóstoles se sorprendieron e inquietaron bastante cuando de repente se les acercó un hombre acompañado de casi cuarenta personas que también buscaban a Jesús. En el momento de la llegada de este grupo, los nueve apóstoles, al menos la mayoría de ellos, habían sucumbido a su antigua tentación de debatir sobre quién sería el mayor en el reino venidero y qué puesto ocuparía cada cual en él. Simplemente, no eran capaces de erradicar de ellos del todo la idea, por tanto tiempo anhelada, de la misión material del Mesías. Y ahora que Jesús mismo había reconocido la confesión realizada por ellos de que él era realmente el Libertador ---había llegado a admitir al menos el hecho de su divinidad--- no había nada más natural que, durante este período de separación del Maestro, se pusieran a hablar de esas esperanzas y aspiraciones que tan prioritarias eran en sus corazones. Y estaban enredados en estas conversaciones, cuando Santiago de Safed y los demás, que venían a ver a Jesús se encontraron con ellos.
\vs p158 4:5 Andrés se adelantó para saludar a este padre y a su hijo, preguntándoles: “¿A quién buscáis?”. Santiago respondió: “Mi buen hombre, busco a vuestro Maestro para que él cure a mi hijo enfermo. Deseo que Jesús eche fuera a este diablo que lo posee”. Y entonces el padre procedió a relatar a los apóstoles que era tanta la gravedad del padecimiento de su hijo que muchas veces casi había perdido la vida por estas terribles convulsiones.
\vs p158 4:6 Mientras los apóstoles escuchaban, Simón Zelotes y Judas Iscariote se aproximaron al padre, diciendo: “Nosotros podemos sanarlo; no necesitas esperar a que regrese el Maestro. Somos los embajadores del reino; ya no guardamos estas cosas en secreto. Jesús es el Libertador, y se nos han entregado las llaves del reino”. Andrés y Tomás, aparte, se consultaron entre sí. Natanael y los demás contemplaban la escena, estupefactos; todos ellos estaban horrorizados por el súbito atrevimiento, e incluso arrogancia, de Simón y Judas. Entonces dijo el padre: “Si se os ha dado el hacer estas obras, os pido que digáis las palabras que liberen a mi hijo de su cautiverio”. Entonces Simón se adelantó y, colocando su mano sobre la cabeza del niño, lo miró directamente a los ojos y ordenó: “Sal de él, espíritu impuro; en el nombre de Jesús, obedéceme”. Pero el muchacho experimentó una sacudida incluso más violenta, en tanto que los escribas se mofaban con desprecio de los apóstoles y los creyentes, decepcionados, sufrían las burlas de estos detractores, nada amigables.
\vs p158 4:7 Andrés se sentía profundamente desconcertado ante este desacertado acto y su desolador fracaso. Llamó a los apóstoles a un lado para conversar y orar. Después de un tiempo de meditación, sintiendo el agudo aguijón de la derrota y todo el peso de la humillación sobre ellos, Andrés hizo un segundo intento por echar fuera al demonio, pero solamente el fracaso coronó sus esfuerzos. Andrés confesó honestamente su derrota y le pidió al padre que se quedara con ellos por la noche o hasta que Jesús volviera, diciendo: “Tal vez esta índole de demonio no desaparece, a no ser que el Maestro lo ordene personalmente”.
\vs p158 4:8 Y, así, mientras que Jesús bajaba de la montaña con los exultantes y eufóricos Pedro, Santiago y Juan, sus nueve hermanos estaban asimismo inquietos por el sentimiento de confusión y de pesarosa humillación que los embargaba. El grupo se sentía descorazonado y escarmentado. Pero Santiago de Safed no desistió. Aunque no le podían indicar cuándo volvería Jesús, él decidió permanecer allí hasta que el Maestro regresara.
\usection{5. JESÚS SANA AL MUCHACHO}
\vs p158 5:1 Conforme Jesús se acercaba, los nueve apóstoles se sintieron más que aliviados de recibirlo de vuelta, y se animaron bastante al ver los semblantes de Pedro, Santiago y Juan, de tan buen ánimo y con un inusitado entusiasmo. Todos se apresuraron a saludar a Jesús y a sus tres hermanos. Mientras intercambiaban saludos, la multitud se fue hacia ellos, y Jesús preguntó: “¿Qué estabais discutiendo al llegar nosotros?”. Pero antes de que los apóstoles, desconcertados y humillados, pudieran responder a la pregunta del Maestro, el padre del afligido joven se adelantó y, arrodillándose a los pies de Jesús, le dijo angustiado: “Maestro, tengo un único hijo y está poseído por un espíritu maligno. Cuando tiene algún ataque, no solo da gritos de terror, echa espumarajos por la boca y cae como muerto, sino que muchas veces este espíritu inmundo que lo posee lo sacude con convulsiones y a veces lo arroja al agua e incluso al fuego. Con tanto crujir de dientes y a causa de tantas contusiones, mi hijo se está consumiendo. Su vida es peor que la muerte; tenemos, su madre y yo, el corazón triste y el espíritu destrozado. Ayer, sobre el mediodía, buscándote a ti, encontré a tus discípulos y, mientras esperábamos, tus apóstoles quisieron expulsar a este demonio, pero no lo lograron. Y, ahora, Maestro, ¿harás tú esto por nosotros?, ¿sanarás a mi hijo?”.
\vs p158 5:2 Cuando Jesús escuchó aquello, tocó al padre arrodillado y le mandó a que se incorporara mientras miraba fijamente a los apóstoles, que se encontraban próximos. Luego dijo Jesús a todos los que estaban delante de él: “¡Oh, generación incrédula y perversa! ¿Hasta cuándo os he de soportar? ¿Hasta cuándo he de estar con vosotros? ¿Cuánto tiempo habrá de pasar para que aprendáis que las obras de la fe no vienen a instancias de la increencia o de la duda?”.  Entonces, señalando al desconcertado padre, Jesús dijo: “Trae acá a tu hijo”. Y cuando Santiago había llevado al joven ante Jesús, él preguntó: “¿Cuánto tiempo hace que le sucede esto?”. El padre contestó: “Desde muy niño”. Y mientras hablaban, el joven sufrió una fuerte sacudida y cayó entre ellos. Le rechinaban los dientes y echaba espumarajos por la boca. Tras sucesivas convulsiones violentas, quedó tendido allí como muerto. El padre de nuevo se arrodilló a los pies de Jesús mientras imploraba al Maestro: “Si puedes curarlo, te suplico que tengas compasión de nosotros y nos libres de esta aflicción”. Y cuando Jesús oyó estas palabras, bajó la mirada al angustiado rostro del padre, diciendo: “No cuestiones el poder del amor de mi Padre, sino solo la sinceridad y la medida de tu fe. Todas las cosas son posibles para quien cree realmente”. Y entonces Santiago de Safed pronunció esas palabras, mezcla de fe y duda, por mucho tiempo recordadas: “Señor, creo. Te ruego que ayudes mi incredulidad”.
\vs p158 5:3 Cuando Jesús oyó estas palabras, dio un paso adelante y, tomando al joven de la mano, dijo: “Haré esto de conformidad con la voluntad de mi Padre y en aras de la fe viva. Hijo mío, ¡levántate! Sal de él, espíritu desobediente, y no entres más en él”. Y poniendo la mano del muchacho en la de su padre, Jesús dijo: “Iros por vuestro camino. El Padre ha concedido el deseo de vuestra alma”. Y todos los que estaban presentes, incluso los enemigos de Jesús, estaban asombrados de lo que veían.
\vs p158 5:4 Para los tres apóstoles, que tan recientemente habían gozado del éxtasis espiritual de las escenas y experiencias de la transfiguración, fue realmente decepcionante volver tan pronto a esta escena de derrota y turbación por parte de sus compañeros apóstoles. Pero siempre fue así con estos doce embajadores del reino. Nunca cesaron de alternar en sus experiencias de vida entre la exaltación y la humillación.
\vs p158 5:5 Aquello fue una auténtica sanación de una doble afección: una dolencia física y una enfermedad del espíritu. Y, desde aquella hora, el joven quedó curado permanentemente. Cuando Santiago había partido con su hijo ya recuperado, Jesús dijo: “Nos vamos ahora a Cesarea de Filipo; preparaos para partir de inmediato”. Y el grupo se mantuvo en silencio a medida que se dirigían hacia el sur con la multitud siguiéndole los pasos.
\usection{6. EN EL JARDÍN DE CELSO}
\vs p158 6:1 Pasaron la noche con Celso y, a última hora de esa tarde en el jardín, tras haber comido y descansado, se reunieron los doce alrededor de Jesús, y Tomás dijo: “Maestro, los que nos quedamos atrás desconocemos todavía lo que sucedió en la montaña, y que tanta alegría trajo a los hermanos nuestros que fueron contigo, pero viendo que lo que sucedió en la montaña no puede desvelarse en este momento, desearíamos que nos hables de nuestro fracaso y que nos instruyas en estos temas”.
\vs p158 6:2 Jesús respondió a Tomás, diciendo: “Todo lo que tus hermanos oyeron en la montaña se os revelará a su debido tiempo. Pero, os mostraré ahora la causa de vuestro fracaso en lo que tan insensatamente intentasteis hacer. Mientras vuestro Maestro y sus acompañantes, vuestros hermanos, ascendían la montaña en busca de un mayor conocimiento de la voluntad del Padre y pedir una mayor abundancia de sabiduría para lograr llevarla a cabo, vosotros, que permanecisteis aquí vigilantes, con instrucciones para que os esforzarais por abrir vuestra mente a la percepción espiritual y orarais con nosotros para que la voluntad del Padre se revelara con más plenitud, errasteis al ejercer la fe que tenéis a vuestra disposición, y, en su lugar, os rendisteis a la tentación de caer en vuestras antiguas y nefastas tendencias de procuraros un lugar preferente en el reino de los cielos ---el reino material y temporal sobre el que continuáis pensando---. Y os aferráis a estos conceptos erróneos, a pesar de que os he expresado reiteradamente que mi reino no es de este mundo.
\vs p158 6:3 “Tan pronto como vuestra fe capta la identidad del Hijo del Hombre, os dejáis, entonces, arrastrar de nuevo por el deseo egoísta de tener preferencias mundanas, y discutís entre vosotros sobre quién será el más grande en el reino de los cielos, un reino que, tal como vosotros insistís en concebir, no existe, ni existirá jamás. ¿Es que no os he dicho que quien quiera hacerse grande en el reino de la hermandad espiritual de mi Padre, debe hacerse pequeño en sus propios ojos y ser pues el servidor de sus hermanos? La grandeza espiritual consiste en un amor y un entendimiento que se asemejan al de Dios y no en disfrutar del poder material para la exaltación del yo. En lo que vosotros intentasteis, y en lo que fracasasteis tan completamente, vuestro propósito no era puro. Vuestro motivo no era divino. Vuestro ideal no era espiritual. Vuestra aspiración no era altruista. Vuestra forma de actuar no se basaba en el amor, y vuestro objetivo no era hacer la voluntad del Padre de los cielos.
\vs p158 6:4 “¿Cuánto tiempo os llevará aprender que no podéis acortar temporalmente el curso de los fenómenos naturales establecidos, excepto cuando tales cosas estén en conformidad con la voluntad del Padre? Ni tampoco podéis llevar a cabo trabajo espiritual en ausencia de poder espiritual. Y no podéis hacer nada de esto, estando incluso su potencial presente, sin la existencia de ese esencial tercer factor humano, la experiencia personal de poseer la fe viva. ¿Es que siempre debéis tener manifestaciones materiales para sentiros atraídos hacia las realidades espirituales del reino? ¿Es que no sois capaces de captar el significado espiritual de mi misión sin la manifestación vivible de obras extraordinarias? ¿Cuándo podré tener la confianza de que os ceñiréis a las realidades espirituales superiores, al margen de la apariencia exterior de las manifestaciones materiales?”.
\vs p158 6:5 Una vez que se había dirigido a los doce con aquellas palabras, Jesús añadió: “Y ahora, id a descansar porque mañana regresamos a Magadán y allí deliberaremos sobre nuestra misión en las ciudades y aldeas de la Decápolis. Y como conclusión a la experiencia vivida hoy, permitidme que os explique, a cada uno de vosotros, lo que hablé con vuestros hermanos en la montaña, y dejad que estas palabras se graben profundamente en vuestros corazones: el Hijo del Hombre emprende ahora la última etapa de su ministerio de gracia. Estamos a punto de comenzar esa labor que llevará pronto a la gran prueba final de vuestra fe y devoción cuando sea entregado en manos de los hombres que buscan mi destrucción. Y recordad lo que os estoy diciendo: se dará muerte al Hijo del Hombre, pero resucitará”.
\vs p158 6:6 Entristecidos, se retiraron a dormir. Estaban desconcertados; no podían entender estas palabras y, aunque tenían miedo de preguntarle sobre lo que había dicho, sí las recordarían todas después de su resurrección.
\usection{7. LA PROTESTA DE PEDRO}
\vs p158 7:1 Ese miércoles por la mañana temprano, Jesús y los doce salieron de Cesarea de Filipo en dirección al parque de Magadán, situado cerca de Betsaida\hyp{}Julias. Los apóstoles habían dormido muy poco aquella noche, por lo que se levantaron temprano, y estaban pronto dispuestos para partir. Incluso los impasibles gemelos Alfeo estaban consternados por esta charla sobre la muerte de Jesús. Conforme viajaban al sur, justo más allá de las Aguas de Merom, llegaron a la carretera de Damasco y, queriendo evitar a los escribas y otras personas que, como Jesús sabía, irían pronto tras ellos, mandó que continuaran hasta Cafarnaúm por la carretera de Damasco que pasaba por Galilea. Y lo hizo así porque era consciente de que quienes los seguían tomarían el camino al este del Jordán, creyendo que Jesús y los apóstoles tendrían miedo de cruzar el territorio de Herodes Antipas. Jesús intentaba eludir a sus detractores y a la multitud que iba en pos de ellos para poder así estar a solas con sus apóstoles aquel día.
\vs p158 7:2 Continuaron con su viaje a través de Galilea hasta bien pasada la hora del almuerzo, luego pararon bajo una sombra para descansar. Tras la comida, Andrés, dirigiéndose a Jesús, dijo: “Maestro, mis hermanos no comprenden tus profundos dichos. Hemos llegado a creer por completo que tú eres el Hijo de Dios y, ahora, oímos estas insólitas palabras de dejarnos, de tu muerte. No entendemos lo que dices. ¿Es que nos estás hablando en parábolas? Te rogamos que nos hables directa y llanamente”.
\vs p158 7:3 Respondiendo a Andrés, Jesús dijo: “Hermanos míos, es precisamente porque habéis confesado que yo soy el Hijo de Dios, por lo que me siento en la obligación de desvelaros la verdad sobre el fin del ministerio de gracia del Hijo del Hombre en la tierra. Insistís en aferraros a la creencia de que soy el Mesías, y no renunciáis a la idea de que el Mesías debe sentarse en un trono en Jerusalén; es por lo que persisto yo en deciros que el Hijo del Hombre debe ir en breve a Jerusalén, padecer mucho, ser rechazado por los escribas, los ancianos y los principales sacerdotes y, tras todo ello, ser llevado a la muerte y resucitar de los muertos. Y no os hablo en parábolas; os hablo la verdad con el fin de que estéis preparados para estos hechos cuando, de repente, nos sobrevengan a nosotros”. Y mientras estaba aún hablando, Simón Pedro, yendo impetuosamente hacia él, puso su mano sobre el hombro del Maestro y dijo: “Maestro, nunca nos atreveríamos a contradecirte, pero yo declaro que en ninguna manera esto te acontecerá”.
\vs p158 7:4 Pedro se expresó así porque amaba a Jesús; pero la naturaleza humana del Maestro reconoció, en sus palabras de cariño bienintencionado, una sutil y tentadora sugerencia para que cambiara su plan de terminar su ministerio de gracia en la tierra conforme era la voluntad de su Padre del Paraíso. Y fue precisamente porque percibió el peligro de que las sugerencias de incluso sus afectuosos y leales amigos, lo disuadieran, por lo que se volvió a Pedro y a los otros apóstoles, diciendo: “¡Quítate de delante de mí, Satanás! Suenas como el espíritu del adversario, el tentador. Cuando habláis de esta manera no estáis de mi lado, sino de lado de nuestro enemigo. De esta manera, convertís vuestro amor por mí en tropiezo para que haga la voluntad del Padre. No pongáis la mira en los caminos de los hombres, sino en la voluntad de Dios”.
\vs p158 7:5 Cuando se recuperaron de la primera impresión que les causó el incisivo reproche de Jesús, y antes de reanudar su viaje, el Maestro les dijo además: “Si alguien quiere venir en pos de mí, niéguese a sí mismo, asuma sus responsabilidades diarias y sígame, porque todo el que quiera salvar su vida egoístamente la perderá, pero todo el que pierda su vida por causa de mí y del evangelio, la hallará. ¿De qué le servirá al hombre ganar el mundo entero, si pierde su alma? ¿Qué recompensa dará el hombre por la vida eterna? No os avergoncéis de mí ni de mis palabras en esta generación pecaminosa e hipócrita, tal como yo no me avergonzaré de reconoceros cuando aparezca en gloria ante mi Padre en la presencia de todas las multitudes celestiales. No obstante, muchos de los que estáis ahora ante mí no gustaréis la muerte hasta que hayáis visto que el reino de Dios ha venido con poder”.
\vs p158 7:6 Y así, Jesús dejó claro a los doce la dolorosa y adversa senda por la que deberían caminar si querían ir en pos de él. ¡Qué consternación provocaron aquellas palabras en estos pescadores de Galilea, obstinados en soñar con un reino terrenal en el que ocuparían puestos de honor! Pero sus corazones, leales, se conmovieron ante este valiente llamamiento, y ni uno solo de ellos se resolvió a abandonarlo. Jesús no los enviaba solos a la contienda; él los conduciría. Solo les pedía que tuvieran valor para seguirlo.
\vs p158 7:7 Lentamente, los doce fueron captando la idea de que Jesús les decía algo sobre su posible muerte. Solo lo entendían de manera imprecisa, pero la afirmación de que resucitaría de los muertos no se les quedó de manera alguna impresa en sus mentes. Conforme pasaban los días, Pedro, Santiago y Juan, recordando las experiencias vividas en el monte de la transfiguración, llegarían a un mayor entendimiento de algunos de estos asuntos.
\vs p158 7:8 En toda la relación que los doce habían mantenido con el Maestro, muy pocas veces habían visto ese destello en su mirada ni oído, como en aquella ocasión, esas repentinas palabras de reproche dirigidas a Pedro y al resto de ellos. Jesús siempre había sido paciente con sus fragilidades humanas, pero no cuando tenía que enfrentarse a una inminente amenaza que ponía en riesgo su plan de llevar a cabo, de manera implícita, la voluntad de su Padre, durante el resto de su andadura terrenal. Literalmente, los apóstoles se quedaron atónitos; estaban impresionados y horrorizados. No encontraban palabras capaz de expresar su dolor. Paulatinamente, comenzaron a darse cuenta de lo que el Maestro tenía que sufrir y de que ellos debían acompañarlo en aquellas adversidades, pero no despertaron a la realidad de lo que estaba por acontecer sino hasta mucho después de observar estos primeros indicios, que apuntaban a la inminente tragedia por la que el Maestro atravesaría en sus últimos días.
\vs p158 7:9 En silencio, Jesús y los doce emprendieron el viaje hacia su campamento del parque de Magadán, por el camino de Cafarnaúm. A medida que la tarde avanzaba con lentitud, los apóstoles, aunque no conversaban con Jesús, sí lo hacían bastante entre ellos; Andrés, entretanto, charlaba con el Maestro.
\usection{8. EN LA CASA DE PEDRO}
\vs p158 8:1 Al anochecer, tras entrar en Cafarnaúm, se dirigieron directamente, por vías poco frecuentadas, hasta la casa de Simón Pedro donde cenaron. Mientras David Zebedeo se preparaba para llevarlos al otro lado del lago, se quedaron en la casa de Simón y, Jesús, alzando la mirada a Pedro y los demás apóstoles, preguntó: “Cuando caminabais juntos esta tarde, ¿de qué hablabais entre vosotros tan seriamente?”. Los apóstoles permanecieron en silencio; muchos de ellos habían continuado la discusión ya iniciada en el monte Hermón sobre los puestos que tendrían en el reino venidero; quién sería el mayor entre ellos y otras cosas. Jesús, sabiendo qué era lo que ocupaba sus pensamientos aquel día, hizo señas a uno de los hijos pequeños de Pedro, lo puso en medio de ellos y dijo: “En verdad, en verdad os digo que si no os volvéis y os hacéis más como este niño, progresaréis poco en el reino de los cielos. Así que cualquiera que se humille a sí mismo y sea como este niño, ese será el mayor en el reino de los cielos. Y cualquiera que reciba a un niño como este, a mí me recibe. Y los que me reciban, también reciben a Aquel que me envió. Si queréis ser los primeros en el reino, procurad impartir estas buenas verdades a vuestros hermanos en la carne. Pero a cualquiera que haga tropezar a uno de estos pequeños, mejor le sería que le ataran una piedra de molino al cuello y se le arrojara al mar. Si lo que hacéis con vuestras manos o lo que veis con vuestros ojos son ocasión de transgresión en vuestro avance en el reino, sacrificad esos acariciados ídolos, porque os es mejor entrar al reino sin muchas de las cosas queridas de la vida que aferrarse a estos ídolos y quedar fuera del reino. Pero, sobre todo, mirad que no despreciéis a uno de estos pequeños, porque sus ángeles ven siempre los rostros de las multitudes celestiales”.
\vs p158 8:2 158:8.2 (1761.3) Cuando Jesús terminó de hablar, subieron a la barca y cruzaron navegando al otro lado, en dirección a Magadán.
