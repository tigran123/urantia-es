\upaper{118}{El Supremo y el Último ---el tiempo y el espacio---}
\author{Mensajero poderoso}
\vs p118 0:1 Con respecto a las diferentes naturalezas de la Deidad, podría afirmarse que:
\vs p118 0:2 \li{1.}El Padre es un yo que existe por sí mismo.
\vs p118 0:3 \li{2.}El Hijo es un yo coexistente.
\vs p118 0:4 \li{3.}El Espíritu es un yo existente en conjunción.
\vs p118 0:5 \li{4.}El Supremo es un yo evolutivo\hyp{}experiencial.
\vs p118 0:6 \li{5.}El Séptuplo es una divinidad que se distribuye a sí misma.
\vs p118 0:7 \li{6.}El Último es un yo trascendental\hyp{}experiencial.
\vs p118 0:8 \li{7.}El Absoluto es un yo existencial\hyp{}experiencial.
\vs p118 0:9 \pc Aunque el Dios Séptuplo es indispensable para el éxito evolutivo del Supremo, el Supremo lo es de la misma manera para la aparición futura del Último. Y la doble presencia del Supremo y del Último constituye la relación esencial de la Deidad subabsoluta y derivada, porque son complementarias de forma interdependiente en la consecución de su destino. Juntas constituyen el puente experiencial que enlaza los comienzos y compleción de todo el crecimiento creativo en el universo matriz.
\vs p118 0:10 \pc El crecimiento creativo es inacabable, aunque siempre satisfactorio; interminable en su extensión, aunque siempre marcado por esos momentos de satisfacción del ser personal en su alcance de un objetivo transitorio, que con tanta efectividad sirve como preludio de su movilización hacia nuevas aventuras en el crecimiento cósmico, la exploración del universo y el logro de la Deidad.
\vs p118 0:11 Aunque el ámbito de las matemáticas conlleva limitaciones cualitativas, ciertamente proporciona a la mente finita una base conceptual para contemplar la infinitud. No existen limitaciones cuantitativas en los números, ni siquiera en la comprensión de la mente finita. Con independencia de la magnitud del número que se conciba, siempre podéis plantearos añadir uno más. E igualmente podéis entender que el resultado será menor que la infinitud, porque, no importa que sigáis sumando números, siempre podríais añadir alguno más.
\vs p118 0:12 Asimismo, en algún momento determinado, la serie infinita puede sumarse y este total (más acertadamente, subtotal) brindará, a cualquier persona dada, en algún momento específico y de un estatus concreto, la dulzura de haber alcanzado la meta. Pero tarde o temprano, esta misma persona comenzará a sentir la sed y el anhelo de alcanzar objetivos nuevos y más grandiosos, y estos apasionantes episodios de crecimiento seguirán permanentemente haciendo su aparición en la plenitud de los tiempos y en los ciclos de la eternidad.
\vs p118 0:13 Cada era consecutiva del universo en cuanto al crecimiento cósmico es la antesala de la siguiente, y cada época del universo aporta un destino inmediato para todas las etapas precedentes. Havona es, en sí misma y por sí misma, una creación perfecta, pero de limitada perfección; su perfección, que se expande hacia afuera, hacia los suprauniversos evolutivos, encuentra no solamente su destino cósmico en esta expansión sino también la liberación de las limitaciones de su existencia preevolutiva.
\usection{1. EL TIEMPO Y LA ETERNIDAD}
\vs p118 1:1 Resulta de utilidad, para su orientación cósmica, que el hombre consiga el mayor entendimiento posible de la relación de la Deidad con el cosmos. Aunque la Deidad absoluta es eterna en cuanto a su naturaleza, los Dioses se relacionan con el tiempo como experiencia en la eternidad. En los universos evolutivos la eternidad es perpetuidad temporal ---el \bibemph{ahora} sempiterno---.
\vs p118 1:2 \pc El ser personal de la criatura mortal puede eternizarse mediante su propia identificación con el espíritu interior, al optar por elegir hacer la voluntad del Padre. Una consagración así de la voluntad equivale a la realización de la eternidad\hyp{}realidad de propósito. Esto significa que el propósito de la criatura ha quedado fijado en lo que se refiere a la sucesión de los momentos; dicho de otro modo, que tal sucesión de momentos no verá cambio alguno en su intencionalidad. Un millón o mil millones de momentos no suponen ninguna diferencia. El número ha dejado de tener significado con respecto a la intención de la criatura. Por ello, su elección más la elección de Dios devienen en las realidades eternas de la unión interminable del espíritu de Dios y de la naturaleza del hombre, en el perdurable servicio de los Hijos de Dios y de su Padre del Paraíso.
\vs p118 1:3 Existe una relación directa entre la madurez y la unidad de la conciencia del tiempo en cualquier intelecto concreto. La unidad de tiempo puede ser un día, un año o un período más largo, pero inevitablemente es el criterio por el que el yo consciente evalúa las circunstancias de la vida, y mediante el cual el intelecto, al conceptualizar, mide y evalúa los hechos de la existencia temporal.
\vs p118 1:4 La experiencia, la sabiduría y el juicio son elementos inherentes a la prolongación de las unidades del tiempo en la experiencia del mortal. Cuando la mente humana retrocede al pasado, está evaluando la experiencia pasada con el objetivo de que influya en la situación presente. Cuando una mente se extiende al futuro, intenta evaluar el significado futuro de una posible acción. Y habiendo, pues, contado a la vez con la experiencia y la sabiduría, la voluntad humana ejerce su decisión\hyp{}juicio en el presente, y el plan de acción nacido, pues, del pasado y del futuro llega a existir.
\vs p118 1:5 En la madurez del yo en desarrollo, el pasado y el futuro se unifican para iluminar el verdadero significado del presente. A medida que el yo madura, se adentra más y más en el pasado en busca de la experiencia, mientras que sus predicciones, basadas en la sabiduría, tratan de penetrar, cada vez con mayor profundidad, en el futuro desconocido. Y, a medida que el yo pensante se extiende cada vez más lejos tanto en el pasado como en el futuro, del mismo modo su juicio se vuelve cada vez menos dependiente del momentáneo presente. De este modo, la toma de decisiones comienza a escaparse de las ligaduras del presente en movimiento, en tanto que comienza a asumir aspectos significativos del pasado y del futuro.
\vs p118 1:6 \pc La paciencia es propia de esos mortales que viven en una reducida dimensión temporal; la verdadera madurez trasciende la paciencia mediante la indulgencia nacida de la auténtica comprensión.
\vs p118 1:7 \pc Madurar conlleva vivir más intensamente en el presente, escapando, al mismo tiempo, de sus limitaciones. Los planes realizados en la madurez, fundamentados en experiencias pasadas, se concretizan en el presente y potencian, de este modo, los valores del futuro.
\vs p118 1:8 En la dimensión temporal de la inmadurez, el significado\hyp{}valor se concentra en el momento presente de tal forma que el presente se separa de su verdadera relación con el no presente ---con el pasado\hyp{}futuro---. La dimensión temporal de la madurez se amplía con el fin de revelar la relación armonizada del pasado\hyp{}presente\hyp{}futuro, de tal forma que el yo comienza a conseguir una apreciación de la totalidad de los acontecimientos, comienza a contemplar el entorno del tiempo desde una perspectiva panorámica con más amplios horizontes, comienza quizás a considerar la existencia de un continuo eterno sin principio ni fin, a cuyos fragmentos se les denomina tiempo.
\vs p118 1:9 En los niveles de lo infinito y de lo absoluto, el momento presente contiene todo el pasado, al igual que todo el futuro. YO SOY significa también YO FUI y YO SERÉ. Y esto constituye nuestro más óptimo concepto de la eternidad y de lo eterno.
\vs p118 1:10 En el nivel absoluto y eterno, la realidad potencial resulta tan significativa como la realidad actual. Solo en el nivel finito, y para las criaturas sujetas al tiempo, parece existir una diferencia tan inmensa entre ambas realidades. Para Dios, como absoluto, un mortal ascendente que haya tomado la decisión eterna es ya un finalizador del Paraíso. Pero el Padre Universal, mediante el modelador del pensamiento interior, no está limitado por tanto en su conocimiento, sino que también puede saber, y ser partícipe, de toda la lucha temporal con las dificultades a las que se enfrenta la criatura en su ascensión, desde unos niveles de existencia semejantes a los de los animales hasta los semejantes a Dios.
\usection{2. OMNIPRESENCIA Y UBICUIDAD}
\vs p118 2:1 No debe confundirse la ubicuidad de la Deidad con la ultimidad de la omnipresencia divina. Es voluntad del Padre Universal que el Supremo, el Último y el Absoluto compensen, coordinen y unifiquen su ubicuidad espacio\hyp{}temporal y su omnipresencia, trascendente del tiempo y el espacio, con su presencia universal y absoluta sin tiempo ni espacio. Y debéis recordar que, aunque la ubicuidad de la Deidad pueda estar tan a menudo relacionada con el espacio, no está condicionada necesariamente por el tiempo.
\vs p118 2:2 \pc Como ascendentes mortales y morontiales, percibís progresivamente a Dios por medio del ministerio del Dios Séptuplo. Mediante Havona descubrís al Dios Supremo. En el Paraíso, lo encontráis como persona y, después, como finalizadores, intentaréis, en poco tiempo, conocerlo como Último. Siendo finalizadores, parece que haya un único camino a seguir tras haber alcanzado al Último, que sería iniciar la búsqueda del Absoluto. Ningún finalizador se verá inquietado por la incertidumbre de lograr el Absoluto de la Deidad, puesto que, al final de sus ascensiones supremas y últimas, habrá encontrado al Dios Padre. No cabe duda de que estos finalizadores creen que, incluso si tuvieran éxito en encontrar al Dios Absoluto, estarían únicamente descubriendo a Dios mismo, al Padre del Paraíso, que se manifiesta en niveles infinitos y universales más cercanos. Sin duda, la consecución de Dios en la absolutidad revelaría al Ancestro Primigenio de los universos, así como también al Padre Final de los seres personales.
\vs p118 2:3 El Dios Supremo puede que no sea una expresión de la omnipresencia espacio\hyp{}temporal de la Deidad, pero es, literalmente, una manifestación de la ubicuidad divina. Entre la presencia espiritual del Creador y las manifestaciones materiales de la creación existe una inmensa área en la que la ubicuidad \bibemph{acontece} ---la aparición en el universo de la Deidad evolutiva---.
\vs p118 2:4 Si alguna vez el Dios Supremo asume el control directo sobre los universos del tiempo y del espacio, tenemos la seguridad de que dicha administración por parte de la Deidad obrará bajo la acción directiva del Último. En cuyo caso, el Dios Último comenzaría a hacerse manifiesto para los universos del tiempo como el Todopoderoso trascendental (el Omnipotente), ejerciendo su acción directiva sobre el supratiempo y el espacio trascendido, en lo que respecta a la actividad administrativa del Todopoderoso Supremo.
\vs p118 2:5 La mente mortal puede preguntarse, tal como nosotros lo hacemos: si la evolución del Dios Supremo hacia su asunción de autoridad administrativa en el gran universo viene acompañada de un aumento de las manifestaciones del Dios Último, ¿vendrá acompañada la correspondiente gradual aparición del Dios Último en los teoréticos universos del espacio exterior de revelaciones similares y ampliadas del Dios Absoluto? Realmente, no lo sabemos.
\usection{3. RELACIONES ESPACIO\hyp{}TEMPORALES}
\vs p118 3:1 Solo mediante la ubicuidad podría la Deidad unificar las manifestaciones espacio\hyp{}temporales para la comprensión finita, puesto que el tiempo es una sucesión de instantes, mientras que el espacio es un sistema de puntos vinculados. Vosotros, al fin y al cabo, percibís el tiempo a través del análisis y el espacio a través de la síntesis. Coordináis y vinculáis estos dos conceptos diferentes a través de la percepción integrada del ser personal. De todo el mundo animal, únicamente el hombre posee esta perceptibilidad espacio\hyp{}temporal. Para un animal, el movimiento tiene cierto significado, pero solo posee valor para una criatura con estatus de persona.
\vs p118 3:2 \pc La cosas están condicionadas por el tiempo, pero la verdad es atemporal. Cuánta más verdad conozcáis, más verdad \bibemph{seréis,} más entenderéis el pasado y el futuro.
\vs p118 3:3 La verdad es estable ---está exenta por siempre de cualquier vicisitud transitoria, aunque nunca es inerte ni ceremoniosa, sino siempre vibrante y adaptable--- radiantemente viva. Pero cuando la verdad se vincula al hecho, entonces, el tiempo y el espacio condicionan sus contenidos y correlacionan sus valores. Estas realidades de la verdad agregadas al hecho se convierten en conceptos y, por consiguiente, se les relega al ámbito de las realidades cósmicas relativas.
\vs p118 3:4 La vinculación de la verdad absoluta y eterna del Creador con la experiencia efectiva de la criatura finita y temporal deviene en un nuevo y emergente valor del Supremo. El concepto del Supremo es esencial para la coordinación del mundo de arriba, divino e invariable, con el mundo de abajo, finito y en continuo cambio.
\vs p118 3:5 \pc De todas las cosas no absolutas, el espacio es el que más se acerca a la absolutidad. El espacio es, al parecer, absolutamente último. La verdadera dificultad que tenemos para comprender el espacio en un nivel material se debe al hecho de que, aunque los cuerpos materiales existen en el espacio, el espacio existe igualmente en estos mismos cuerpos materiales. Aunque el espacio tiene bastante de absoluto, eso no significa que lo sea.
\vs p118 3:6 Para comprender las relaciones espaciales, quizás pueda ser de utilidad suponer que, relativamente hablando, el espacio es, en definitiva, una propiedad característica de todos los cuerpos materiales. Así pues, cuando un cuerpo se mueve a través del espacio, lleva también consigo todas sus propiedades, incluso el espacio que está en dicho cuerpo en movimiento y forma parte de él.
\vs p118 3:7 Todos los modelos de la realidad ocupan espacio en los niveles materiales, pero los modelos espirituales solo existen en relación con el espacio; ni ocupan ni desplazan espacio, tampoco lo contienen. Pero, para nosotros, el principal enigma del espacio corresponde al modelo de una idea. Cuando penetramos en el entorno de la mente, encontramos muchos misterios. ¿Acaso el modelo ---la realidad--- de una idea ocupa espacio? Realmente no lo sabemos, a pesar de estar seguros de que el modelo de una idea no contiene espacio. Si bien, no sería muy adecuado teorizar que lo inmaterial es siempre no espacial.
\usection{4. CAUSALIDAD PRIMARIA Y SECUNDARIA}
\vs p118 4:1 Muchas de las dificultades teológicas y de los dilemas metafísicos del hombre mortal se deben al equivocado desplazamiento de la persona de la Deidad y la consiguiente asignación de los atributos infinitos y absolutos a la Divinidad de menor rango y a la Deidad evolutiva. No debéis olvidar que, aunque, en efecto, existe una verdadera Primera Causa, también hay multitud de causas coiguales y menores, causas tanto similares como secundarias.
\vs p118 4:2 La distinción fundamental entre primeras y segundas causas consiste en que las primeras producen efectos primigenios, exentos de la herencia de cualquier factor derivado de alguna causalidad antecedente. Las causas secundarias originan efectos que, invariablemente, muestran la herencia de otra causalidad precedente.
\vs p118 4:3 \pc Los potenciales puramente estáticos intrínsecos al Absoluto Indeterminado son reactivos a esas causalidades del Absoluto de la Deidad, producidas por la acción de la Trinidad del Paraíso. En presencia del Absoluto Universal, estos potenciales estáticos, permeados de causalidad, se vuelven de inmediato activos y susceptibles a la influencia de ciertas instancias intermedias trascendentales, cuyos actos resultan en la transmutación de estos potenciales activados a un estatus con auténticas posibilidades de desarrollarse en el universo, con capacidades actualizadas para crecer. Sobre dichos potenciales madurados, los creadores y los rectores del gran universo emprenden la inacabable historia de la evolución cósmica.
\vs p118 4:4 La causalidad, al margen de la de los existenciales, es triple en su constitución básica. Tal como opera en esta era del universo y con respecto al nivel finito de los siete suprauniversos, puede concebírsele como sigue:
\vs p118 4:5 \li{1.}\bibemph{La activación de los potenciales estáticos}. El establecimiento del destino en el Absoluto Universal por la acción del Absoluto de la Deidad, que opera en y sobre el Absoluto Indeterminado y a consecuencia de los mandatos volitivos de la Trinidad del Paraíso.
\vs p118 4:6 \li{2.}\bibemph{El devenir de las capacidades del universo}. Aquí se incluye la transformación de los potenciales indiferenciados en planificaciones separadas y definidas. Se trata del acto de la Ultimidad de la Deidad y de las múltiples instancias intermedias del nivel trascendental. Tales actos se realizan en perfecta anticipación a las necesidades futuras de todo el universo matriz. En relación con la separación de los potenciales, existen los arquitectos del universo matriz como auténtica expresión del concepto de la Deidad de los universos. Sus proyectos parecen estar, en último término, limitados en su extensión por el espacio, por la periferia conceptual del universo matriz, pero, \bibemph{como proyectos} no están sujetos en ningún caso al tiempo ni al espacio.
\vs p118 4:7 \li{3.}\bibemph{La creación y evolución de los actuales en los universos}. Es sobre un cosmos permeado por la presencia generadora de capacidad de la Ultimidad de la Deidad donde los creadores supremos actúan para efectuar las transmutaciones temporales de los potenciales madurados en actuales experienciales. En el seno del universo matriz, cualquier actualización de la realidad potencial está limitada por su capacidad última para desarrollarse y está condicionada espacio\hyp{}temporalmente en las etapas finales de su aparición. Los hijos creadores que parten del Paraíso son, en realidad, creadores \bibemph{transformadores} en el sentido cósmico. Pero esto no anula, en modo alguno, el concepto que el hombre tiene de ellos como creadores; desde la perspectiva finita, ellos ciertamente tienen capacidad para crear y, de hecho, lo hacen.
\usection{5. OMNIPOTENCIA Y COMPOSIBILIDAD}
\vs p118 5:1 La omnipotencia de la Deidad no supone ostentar el poder de realizar lo irrealizable. Dentro del marco del espacio\hyp{}tiempo y desde el punto de referencia intelectual de la comprensión humana, ni incluso el Dios infinito puede crear círculos cuadrados ni causar un mal inherentemente bueno. Dios no puede hacer algo que no se asemeje a él. En términos filosóficos, tal contradicción es el equivalente a la no existencia y significa que nada se crea de este modo. Un rasgo del ser personal no puede ser al mismo tiempo como Dios o disimilar a él. La composibilidad es innata en el poder divino. Y todo esto se deriva del hecho de que la omnipotencia no solo crea cosas con una naturaleza, sino que también da origen a la naturaleza de todas las cosas y todos los seres.
\vs p118 5:2 \pc En un principio, el Padre lo hace todo, pero a medida que el panorama de la eternidad se despliega en respuesta a la voluntad y a los mandatos del Infinito, resulta más patente que las criaturas, inclusive los hombres, han de volverse compañeros de Dios en la realización de la completud del destino. Y esto es cierto incluso en la vida en la carne; cuando el hombre y Dios entablan una alianza, no se puede imponer limitaciones a las posibilidades futuras de tal cooperación. Cuando el hombre se percata de que el Padre Universal es compañero de él en su eterno avance, cuando se fusiona con la presencia del Padre que mora en su interior, ha roto, en espíritu, los grilletes del tiempo y ha emprendido los caminos de progreso de la eternidad en la búsqueda del Padre Universal.
\vs p118 5:3 La conciencia de los mortales avanza del hecho al significado y, luego, al valor; la conciencia del Creador lo hace del pensamiento\hyp{}valor al hecho de la acción, pasando por la palabra\hyp{}significado. Dios debe siempre actuar para acabar con el estancamiento de la unidad incondicionada inherente a la infinitud existencial. La Deidad debe siempre proporcionar el universo modelo; los seres personales perfectos; la verdad, la belleza y la bondad primigenias, por las que todas las creaciones por debajo de la deidad pugnan. Dios debe siempre encontrar primero al hombre para que el hombre pueda después encontrar a Dios. Siempre debe haber un Padre Universal, antes de que pueda haber alguna vez una filiación universal y la consiguiente hermandad universal.
\usection{6. OMNIPOTENTE, PERO NO OMNIFICIENTE}
\vs p118 6:1 Dios es verdaderamente omnipotente, pero no es todo hacedor ---no hace personalmente todo lo que es realizable---. La omnipotencia engloba la potencia\hyp{}potencial del Todopoderoso Supremo y del Ser Supremo, pero los actos volitivos del Dios Supremo no son las acciones personales del Dios Infinito.
\vs p118 6:2 Propugnar que la Deidad primigenia es todo hacedora sería equivalente a denegar sus derechos a prácticamente un millón de hijos creadores del Paraíso, sin mencionar a las innumerables multitudes de otros distintos órdenes de asistentes creativos que cooperan con ellos. No hay sino una Causa Incausada en todo el universo. Todas las demás causas se derivan de esta única Primera Gran Fuente y Centro. Y en esta filosofía no hay nada que atente contra la libre voluntad de las miríadas de hijos de la Deidad repartidos por el inmenso universo.
\vs p118 6:3 \pc Dentro de un marco local, la volición puede que parezca obrar como causa incausada, pero muestra, indefectiblemente, unos componentes hereditarios que establecen su relación con las Causas Primeras, únicas, primigenias y absolutas.
\vs p118 6:4 Toda volición es relativa. En un sentido primordial, únicamente el Padre\hyp{}YO SOY posee completud de volición; en un sentido absoluto, únicamente el Padre, el Hijo y el Espíritu manifiestan atributos de poseer una volición incondicionada por el tiempo e ilimitada por el espacio. El hombre mortal está dotado de libre voluntad, de la facultad de elección y, aunque esta elección no es absoluta es, no obstante, relativamente final en el nivel finito y en lo que respecta al destino de la persona que la realiza.
\vs p118 6:5 En cualquier nivel menor al absoluto, la volición se encuentra con limitaciones constitutivas de la persona misma que ejerce la facultad de elegir. El hombre no puede elegir más allá del rango de lo que es elegible. No puede, por ejemplo, optar por ser algo distinto de un ser humano, excepto que puede escoger convertirse en más que un hombre; puede optar por emprender el viaje de ascenso en los universos, pero esto es así porque resulta que la elección humana y la voluntad divina son coincidentes en este punto. Y lo que el hijo desea y el Padre quiere, ciertamente acontecerá.
\vs p118 6:6 En la vida mortal, se abren y se cierran continuamente vías conductuales diferenciadas y, durante el tiempo en el que la elección es factible, el ser personal humano está constantemente decidiendo entre estas múltiples líneas de acción. La volición temporal está ligada al tiempo y debe esperar el paso de este para poder hallar la oportunidad de expresarse. La volición espiritual ha empezado a probar la liberación de los grilletes del tiempo al haber logrado escapar parcialmente del orden temporal, y esto es debido a que la volición espiritual se va identificando a sí misma con la voluntad de Dios.
\vs p118 6:7 La volición, el acto de elegir, debe obrar dentro del marco conceptual del universo que se ha actualizado en respuesta a una elección anterior y más elevada. El rango completo de la voluntad humana está estrictamente limitado por lo finito, salvo en algo concreto: cuando el hombre opta por encontrar a Dios y ser como él, tal opción es suprafinita; solo la eternidad podrá desvelar si esta elección es también supraabsonita.
\vs p118 6:8 \pc Reconocer la omnipotencia de la Deidad es gozar de la seguridad de vuestra experiencia como ciudadanos cósmicos, poseer la garantía de la seguridad en el largo viaje al Paraíso. Pero aceptar la falacia del todo hacedor es abrazar el colosal error del panteísmo.
\usection{7. OMNISCIENCIA Y PREDESTINACIÓN}
\vs p118 7:1 En el gran universo, la labor de la voluntad del Creador y de la voluntad de la criatura opera dentro de las limitaciones, y en consonancia con las posibilidades, establecidas por los arquitectos mayores. Sin embargo, la predeterminación de unos límites máximos no coarta, en lo más mínimo, la soberanía de la voluntad de la criatura cuando se ejerce dentro de ellos. El preconocimiento último ---la completa consideración de cualquier elección--- tampoco supone la invalidación de la volición finita. Un ser humano maduro y con visión de futuro podría ser capaz de pronosticar, con gran precisión, la decisión de algún compañero de tarea más joven, pero este preconocimiento no resta nada a la libertad ni al carácter genuino de la propia decisión. Con su sabiduría, los Dioses han delimitado el rango de acción de la voluntad inmadura, pero se trata de una voluntad auténtica dentro de ciertos límites definidos.
\vs p118 7:2 Incluso la correlación suprema de cualquier elección pasada, presente y futura no invalida la autenticidad de tales decisiones. Indica, más bien, la tendencia predeterminada del cosmos y da a entender el preconocimiento de aquellos seres volitivos que, pueden o no, optar por convertirse en contribuidores de la actualización experiencial de toda la realidad.
\vs p118 7:3 \pc El error en la elección finita está sujeto al tiempo y delimitado por este. Puede existir solamente en el tiempo y \bibemph{dentro} de la presencia evolutiva del Ser Supremo. Tal opción equivocada es factible en el ámbito del tiempo e indica (además de la incompletitud del Supremo) ese determinado rango de elección con el que las criaturas inmaduras deben estar dotadas a fin de gozar de su progreso en el universo, al entablar un contacto con la realidad mediante su libre voluntad.
\vs p118 7:4 En el espacio sujeto al tiempo, el pecado demuestra claramente la libertad temporal ---incluso el libertinaje--- de la voluntad finita. El pecado describe la inmadurez encandilada por la libertad de la voluntad, relativamente soberana, de la persona que, al mismo tiempo, no consigue percibir las obligaciones y deberes supremos de la ciudadanía cósmica.
\vs p118 7:5 En el ámbito finito, la iniquidad revela la realidad transitoria de cualquier yo que no está identificado con Dios. Solamente cuando una criatura se identifica con Dios se hace verdaderamente real en los universos. El ser personal finito no se crea a sí mismo, pero en el escenario de la elección, en el entorno del suprauniverso, determina por sí mismo su destino.
\vs p118 7:6 \pc La dádiva de la vida capacita a los sistemas de la energía material poder perpetuarse, propagarse y adaptarse por sí mismos. La dádiva del ser personal imparte, además, a los organismos vivos, las prerrogativas de la propia determinación, evolución e identificación con un espíritu de fusión de la Deidad.
\vs p118 7:7 Los seres vivos subpersonales muestran que la mente puede activar la energía\hyp{}materia, en primer lugar, bajo la forma de los controladores físicos y, luego, como espíritus asistentes de la mente. La dotación del ser personal procede del Padre y confiere prerrogativas únicas al sistema vivo en el entorno de la elección. Pero si el ser personal tiene la facultad de ejercer volitivamente la opción de identificarse con la realidad, y si resulta ser una decisión verdadera y libre, entonces, el ser personal evolutivo ha de poseer también la opción de confundirse, trastornarse y destruirse a sí mismo. La posibilidad de la autodestrucción cósmica es inevitable si el ser personal evolutivo ha de ser realmente libre en el ejercicio de su voluntad finita.
\vs p118 7:8 Por consiguiente, existe una mayor seguridad al reducirse los límites de la elección del ser personal a lo largo de todos los niveles más modestos de la existencia. La elección se hace crecientemente más libre a medida que se asciende en los universos; se aproxima en algún momento a la libertad divina cuando el ser personal ascendente consigue estatus de divinidad, la consagración suprema a los propósitos del universo, la compleción de la sabiduría cósmica y la completitud de la identificación creatural con la voluntad y la manera de Dios.
\usection{8. CONTROL Y ACCIÓN DIRECTIVA}
\vs p118 8:1 En las creaciones del espacio\hyp{}tiempo, la libre voluntad está circundada por restricciones, por limitaciones. La evolución de la vida material es, en un principio mecánica, luego se activa por la mente y (tras la dádiva del ser personal) puede llegar a estar regida por el espíritu. Los potenciales de las implantaciones primigenias de vida física, por parte de los portadores de vida, limitan físicamente la evolución orgánica en los mundos habitados.
\vs p118 8:2 El hombre mortal es una máquina, un mecanismo vivo; sus raíces están realmente insertas en el mundo físico de la energía. Muchas reacciones humanas son de naturaleza mecánica; gran parte de su vida se asemeja a la de una máquina. Pero, el hombre, un mecanismo destinado a un fin, es mucho más que una máquina; está dotado de mente y habitado por el espíritu; y, a pesar de que durante toda su vida material jamás pueda eludir el mecanicismo químico y eléctrico de su existencia, sí puede aprender a subordinar, cada vez más, el aspecto mecánico de su vida física a la conveniente dirección de la experiencia, por medio de la consagración de su mente humana a la realización de los impulsos espirituales del modelador del pensamiento interior.
\vs p118 8:3 \pc El espíritu libera, y los mecanismos limitan, el ejercicio de la voluntad. La elección imperfecta, sin estar regida por unos mecanismos reguladores, ni identificada con el espíritu, es peligrosa e inestable. El dominio por parte de estos mecanismos asegura la estabilidad a expensas del progreso; la alianza con el espíritu libera la elección del nivel físico y, al mismo tiempo, asegura la estabilidad divina originada por una más amplia percepción del universo y una mayor comprensión cósmica.
\vs p118 8:4 El gran peligro que acecha a la criatura es que, al conseguir su liberación de las cadenas del mecanismo de la vida, pueda fracasar al compensar esta pérdida de estabilidad con el logro de una armoniosa cooperación con el espíritu. En su elección, la criatura, cuando se libera relativamente de la estabilidad mecánica, puede intentar asimismo liberarse a sí misma, independizándose de una mayor identificación con el espíritu.
\vs p118 8:5 En su totalidad, el principio de la evolución biológica imposibilita que el hombre primitivo haga su aparición en los mundos habitados con grandes dotes de autocontención. Por ello, este mismo diseño creativo, establecido por la evolución, proporcionó igualmente esas restricciones externas del tiempo y el espacio, el hambre y el miedo, que delimitaron eficazmente el rango de la elección subespiritual de estas rudas criaturas. A medida que la mente del hombre vence satisfactoriamente unas barreras crecientemente más difíciles, este mismo diseño creativo aporta, de igual manera, la lenta acumulación de una herencia racial consistente en una sabiduría experiencial dolorosamente adquirida ---en otras palabras, facilita el mantenimiento de un equilibrio entre la disminución de las restricciones externas y el aumento de las restricciones internas---.
\vs p118 8:6 La lentitud de la evolución, del progreso cultural humano, da fe de la eficacia de ese freno ---de esa inercia material--- que tan eficientemente opera para retrasar las peligrosas velocidades del progreso. De esta forma, el tiempo mismo amortigua y distribuye los resultados, por otra parte letales, del escape prematuro de las próximas barreras que se avecinan a la acción humana. Porque, cuando la cultura avanza con demasiada rapidez, cuando el logro material aventaja a la evolución de la adoración\hyp{}sabiduría, entonces, la civilización contiene en sí misma las semillas de la regresión; y, a menos que se refuercen por un rápido aumento de la sabiduría experiencial, dichas sociedades humanas retrocederán desde sus elevados logros, conseguidos prematuramente, y las “eras de las tinieblas” del interregno de la sabiduría darán testimonio del inexorable restablecimiento del desequilibrio entre la libertad y el dominio de sí mismo.
\vs p118 8:7 La iniquidad de Caligastia consistió en prescindir del control del tiempo en la liberación progresiva del hombre ---la destrucción innecesaria de las barreras restrictivas, unas barreras que las mentes de los mortales de aquellos tiempos aún no habían llegado a salvar de forma experiencial---.
\vs p118 8:8 La mente que puede realizar una reducción parcial del tiempo y del espacio da pruebas, por este mismo acto, que es poseedora de las semillas de una sabiduría que efectivamente puede servir en lugar de la anterior barrera de restricción que había previamente trascendido.
\vs p118 8:9 Igualmente, Lucifer intentó alterar el componente tiempo, que operaba restringiendo el logro prematuro de determinadas libertades en el sistema local. Un sistema local asentado en luz y vida ha logrado de forma experiencial esas perspectivas y percepciones que hacen viable la aplicación de numerosos métodos de acción, que serían perjudiciales y destructivos en ese mismo mundo en las eras previas a tal asentamiento.
\vs p118 8:10 A medida que se deshace de sus ataduras al miedo, a medida que conecta continentes y océanos con sus máquinas y generaciones y siglos con sus registros, el hombre debe sustituir cada restricción que haya vencido con una nueva, asumida voluntariamente, conforme a los dictados morales de su sabiduría humana en expansión. Estas restricciones autoimpuestas son, a la vez, los elementos más poderosos y frágiles de todos los que componen la civilización humana ---los conceptos de la justicia y los ideales de la fraternidad---. El hombre incluso queda habilitado para recibir las vestimentas restrictivas de la misericordia cuando ama con valentía a sus semejantes, mientras que consigue dar inicio a la hermandad espiritual cuando opta por tratarlos como le gustaría que lo trataran a él, incluso con el tratamiento que él cree que Dios le daría.
\vs p118 8:11 En el universo, una reacción automática es estable y, de algún modo, tiene su continuidad en el cosmos. Un ser personal que conoce a Dios y que desea hacer su voluntad, que tiene percepción espiritual, es divinamente estable y eternamente existente. La gran aventura del hombre en el universo consiste en el tránsito de su mente mortal desde la estabilidad de la estática mecánica a la dinámica espiritual de la divinidad, y consigue dicha transformación por la fuerza y la constancia de sus propias decisiones personales, declarando, en cada una de las situaciones de su vida: “Es mi voluntad que se haga tu voluntad”.
\usection{9. LOS MECANISMOS DEL UNIVERSO}
\vs p118 9:1 El tiempo y el espacio conforman en su conjunto un mecanismo del universo matriz. Son instrumentos que posibilitan a las criaturas finitas coexistir en el cosmos con el Infinito. Las criaturas finitas están adecuadamente aisladas de los niveles absolutos por el tiempo y el espacio. Pero estos medios de aislamiento, sin los que ningún mortal podría existir, operan directamente para delimitar el rango de la acción finita. Sin ellos, ninguna criatura podría actuar, si bien, gracias a su presencia, los actos de todas las criaturas están claramente limitados.
\vs p118 9:2 Los mecanismos originados por mentes de orden superior obran para liberar sus fuentes creativas pero, en cierta medida, delimitan invariablemente la acción de todas las inteligencias subordinadas. Para las criaturas de los universos, esta limitación se hace patente en el mecanismo de los universos. El hombre no tiene una irrestricta libre voluntad; hay límites a su rango de elección, pero, dentro del radio de dicha elección, su voluntad es relativamente soberana.
\vs p118 9:3 El mecanismo de vida del ser personal mortal, el cuerpo humano, es consecuencia de un diseño creativo de carácter supramortal; por ello, el hombre mismo no puede dominarlo de modo perfecto. Solo cuando el hombre ascendente, en coordinación con su modelador fusionado, crea por sí el mecanismo para la expresión del ser personal, conseguirá tener un dominio perfeccionado del mismo.
\vs p118 9:4 El gran universo es un mecanismo al igual que un organismo, mecánico y vivo ---un mecanismo vivo activado por una mente suprema, un mecanismo que se coordina con un espíritu supremo, y que encuentra su expresión en los niveles máximos de la unificación de la potencia y del ser personal como el Ser Supremo---. Negar el mecanismo de la creación finita es negar un hecho e ignorar la realidad.
\vs p118 9:5 Los mecanismos son creaciones de la mente, de una mente creativa que actúa sobre y en los potenciales cósmicos. Los mecanismos son cristalizaciones fijas del pensamiento del Creador, y por siempre obran conforme al concepto volitivo que les dio origen. Pero el propósito de la existencia de cada uno de ellos está en su origen, no en su acción.
\vs p118 9:6 No debería pensarse que estos mecanismos son limitadores de la acción de la Deidad; sino más bien que, en estos mismos mecanismos, la Deidad ha alcanzado una faceta de su expresión eterna. Los mecanismos fundamentales del universo se han originado en respuesta a la voluntad absoluta de la Primera Fuente y Centro y, por consiguiente, obrarán eternamente en perfecta armonía con el plan del Infinito; son, de hecho, los modelos no volitivos de ese mismo plan.
\vs p118 9:7 Entendemos algo de cómo el mecanismo del Paraíso se correlaciona con el ser personal del Hijo Eterno; se trata de la función del Actor Conjunto. Y albergamos teorías sobre las operaciones del Absoluto Universal en relación a los mecanismos teóricos del Absoluto Indeterminado y a la persona potencial del Absoluto de la Deidad. Pero, en las Deidades evolutivas del Supremo y del Último, observamos que ciertas facetas impersonales se están uniendo de hecho con sus equivalentes volitivos y, de este modo, está evolucionando una nueva relación entre el modelo y la persona.
\vs p118 9:8 En la eternidad del pasado, el Padre y el Hijo encontraron una unión que se expresó en el Espíritu Infinito. Si, en la eternidad del futuro, los hijos creadores y los espíritus creativos de los universos locales del tiempo y del espacio lograran su unión creativa en los ámbitos del espacio exterior, ¿qué sería lo que crearía su unidad como consecuencia de la expresión combinada de sus naturalezas divinas? Bien podría ser que seamos testigos de una manifestación, hasta este momento sin revelar, de la Deidad Última, un nuevo orden de supradministrador. Estos seres estarían dotados de prerrogativas únicas relativas a su ser personal, al ser la unión de la experiencia de un creador personal, de un espíritu creativo impersonal, de la experiencia de la criatura mortal y de la adquisición progresiva del estado personal por parte de la benefactora divina. Dichos seres podrían ser últimos en el sentido que incluirían la realidad personal e impersonal, a la vez que combinarían las experiencias del creador y de la criatura. Cualesquiera que sean sus atributos, dichas terceras personas de estas teoréticas trinidades operativas en las creaciones del espacio exterior sostendrán una relación con sus padres creadores y sus madres creativas algo similar a la que sostiene el Espíritu Infinito con el Padre Universal y el Hijo Eterno.
\vs p118 9:9 \pc El Dios Supremo es la manifestación personal de toda la experiencia del universo, el punto de convergencia de toda la evolución finita, la maximización de toda la realidad de la criatura, la consumación de la sabiduría cósmica, la personificación de las armoniosas bellezas de las galaxias del tiempo, la verdad de los contenidos mentales cósmicos y la bondad de los supremos valores espirituales. Y el Dios Supremo, en el futuro eterno, sintetizará estas múltiples diversidades finitas en un todo experiencialmente significativo, al igual que se encuentran ahora unidas de forma existencial en los niveles absolutos en la Trinidad del Paraíso.
\usection{10. FUNCIONES DE LA PROVIDENCIA}
\vs p118 10:1 La providencia no significa que Dios haya decidido todas las cosas por nosotros y con antelación. Dios nos ama demasiado como para hacer eso; no sería sino una tiranía cósmica. El hombre tiene ciertamente facultades de elección relativas. El amor divino tampoco es un cariño corto de miras que consentiría y malcriaría a los hijos de los hombres.
\vs p118 10:2 \pc El Padre, el Hijo y el Espíritu ---como Trinidad--- no son el Todopoderoso Supremo, pero la supremacía del Todopoderoso jamás puede manifestarse sin ellos. El \bibemph{crecimiento} del Todopoderoso se centra en los Absolutos de actualidad y se basa en los Absolutos de potencialidad. Si bien, las \bibemph{funciones} del Todopoderoso Supremo están relacionadas con las de la Trinidad del Paraíso.
\vs p118 10:3 Todo parece indicar que, en el Ser Supremo, todas las facetas de la actividad del universo se están reuniendo parcialmente por el ser personal de esta Deidad experiencial. Cuando, por consiguiente, deseamos visualizar la Trinidad como un Dios, y si limitamos este concepto al gran universo, actualmente conocido y organizado, descubrimos que el Ser Supremo en evolución es la representación parcial de la Trinidad del Paraíso. Y, además, descubrimos que esta Deidad Suprema está evolucionando como la síntesis\hyp{}ser personal de la materia, la mente y el espíritu finitos del gran universo.
\vs p118 10:4 Los Dioses poseen atributos, pero la Trinidad posee funciones y, como la Trinidad, la providencia \bibemph{constituye} una función compuesta por una acción directiva del universo de los universos distinta a la personal, que se extiende desde los niveles evolutivos del Séptuplo y se sintetiza en la potencia del Todopoderoso, que se eleva cruzando los ámbitos trascendentales de la Ultimidad de la Deidad.
\vs p118 10:5 \pc Dios ama a cada una de las criaturas como a un hijo, y ese amor da cobijo a cada una de ellas a lo largo del tiempo y la eternidad. La providencia obra respecto al total y trata de la actuación de cualquier criatura en la medida en la que esta actuación guarda relación con la totalidad. La intervención providencial en lo que concierne a cualquier ser indica la importancia de la \bibemph{actuación} de ese ser, en cuanto al crecimiento evolutivo de algún total; este total puede ser la raza total, la nación total, el planeta total o incluso un total de orden superior. Es la importancia de la actuación de la criatura la que da lugar a esta intervención, no la importancia de la criatura como persona.
\vs p118 10:6 No obstante, el Padre, como persona, puede, en cualquier momento, colocar su mano paterna en la corriente de los acontecimientos cósmicos conforme a la voluntad de Dios y en consonancia con la sabiduría de Dios al estar tal acto motivado por su amor.
\vs p118 10:7 Pero lo que el hombre llama providencia es demasiado a menudo consecuencia de su propia imaginación, una yuxtaposición fortuita de las circunstancias del azar. Existe, no obstante, una providencia real y emergente en el entorno finito de la existencia en el universo, una correlación verdadera en proceso de actualización de las energías del espacio, de los movimientos del tiempo, de los pensamientos del intelecto, de los ideales del carácter, de los deseos de las naturalezas espirituales y de las acciones volitivas resolutivas de los seres personales evolutivos. Las circunstancias que tienen lugar en los mundos materiales encuentran su integración finita final en las presencias interconectadas del Supremo y del Último.
\vs p118 10:8 Conforme los mecanismos del gran universo se perfeccionan hasta el punto de lograr su precisión final mediante la acción directiva de la mente, y conforme la mente creatural asciende a la perfección en búsqueda de la divinidad mediante su integración perfeccionada con el espíritu, y conforme el Supremo, por consiguiente, emerge como unificador \bibemph{real} de todos estos fenómenos que se producen en el universo, de igual manera la providencia se va haciendo cada vez más perceptible.
\vs p118 10:9 Algunas de las condiciones sorprendentemente fortuitas que en ocasiones imperan en los mundos evolutivos pueden deberse a la presencia gradualmente emergente del Supremo, la anticipación de su actividad futura en el universo. La mayor parte de lo que un mortal llamaría providencial no lo es; su juicio en estos asuntos está muy dificultado por la ausencia de visión de futuro en cuanto a los verdaderos contenidos de las circunstancias de la vida. La mayor parte de lo que un mortal llamaría buena suerte puede resultar en realidad mala suerte; la sonrisa de la fortuna, que concede un tiempo libre no ganado y una riqueza no merecida, puede ser la mayor de las aflicciones humanas; la aparente crueldad de un destino adverso, que colma de tribulaciones al sufriente mortal, puede ser, en realidad, un fuego que temple y que transmute el hierro dulce de la persona inmadura en el acero templado del verdadero carácter.
\vs p118 10:10 Existe una providencia en los universos evolutivos, y las criaturas pueden descubrirla justo en la medida en la que estas hayan logrado la capacidad de percibir la finalidad de dichos universos. La facultad plena para reconocer los propósitos del universo equivale a la compleción evolutiva de la criatura y, expresándolo de otra forma, a la consecución del Supremo dentro de los límites del estatus actual de incompletitud de los universos.
\vs p118 10:11 El amor del Padre obra directamente en el corazón de la persona, con independencia de las acciones o reacciones de todas las demás; la relación es personal ---hombre y Dios---. La presencia impersonal de la Deidad (el Todopoderoso Supremo y la Trinidad del Paraíso) muestra su consideración al todo y no a la parte. La providencia, percibida desde el control directivo de la Supremacía, se hace cada vez más patente conforme las partes del universo progresan sucesivamente en la consecución de sus destinos finitos. Conforme los sistemas, constelaciones, universos y suprauniversos se asientan en luz y vida, el Supremo emerge, cada vez más, correlacionando significativamente todo lo que está sucediendo, mientras que el Último emerge gradualmente unificando trascendentalmente todas las cosas.
\vs p118 10:12 \pc En los inicios de un mundo evolutivo, los acontecimientos naturales de orden material y los deseos personales de los seres humanos parecen ser a menudo antagónicos. Al hombre mortal le resulta bastante difícil entender satisfactoriamente todo lo que sucede en un mundo en evolución ---con bastante frecuencia, la ley natural parece ser cruel, despiadada e indiferente a todo lo que es verdadero, bello y bueno para la comprensión humana---. Pero, a medida que la humanidad avanza en su desarrollo planetario, observamos que esta perspectiva se modifica debido a los siguientes factores:
\vs p118 10:13 \li{1.}\bibemph{La visión en incremento del hombre:} su entendimiento creciente del mundo en el que vive; la ampliación de su capacidad para comprender los hechos materiales del tiempo, las ideas significativas del pensamiento y los ideales valiosos de la percepción espiritual. Mientras que los hombres evalúen solo con el criterio de las cosas de naturaleza física, jamás podrán tener la esperanza de hallar la unidad en el tiempo y en el espacio.
\vs p118 10:14 \li{2.}\bibemph{El dominio creciente del hombre:} la acumulación paulatina de su conocimiento de las leyes del mundo material, de la finalidad de la existencia espiritual y de las posibilidades de coordinar filosóficamente estas dos realidades. El hombre salvaje estaba indefenso ante los embates de las fuerzas naturales, era un esclavo ante el cruel control de sus propios miedos internos. El hombre semicivilizado comienza a liberar el repositorio de los secretos de los reinos naturales y su ciencia va destruyendo, lenta pero con eficacia, sus supersticiones, mientras que, al mismo tiempo, aporta una base efectiva, nueva y engrandecida, para comprender los significados de la filosofía y los valores de la verdadera experiencia espiritual. Algún día, el hombre civilizado poseerá un relativo dominio de las fuerzas físicas de su planeta; el amor de Dios de su corazón se derramará realmente como amor hacia sus semejantes, mientras que los valores de la existencia mortal se estarán acercando a los límites de la capacidad humana.
\vs p118 10:15 \li{3.}\bibemph{La integración del hombre en el universo:} El incremento de su percepción humana más el incremento de su logro humano de carácter experiencial conducen al hombre hacia una armonía más íntima con las presencias unificadoras de la Supremacía ---la Trinidad del Paraíso y el Ser Supremo---. Y esto es lo que establece la soberanía del Supremo en los mundos asentados en luz y vida por mucho tiempo. Estos planetas avanzados son, de hecho, poemas de armonía, imágenes de la belleza de la bondad lograda, adquirida a través de la búsqueda de la verdad cósmica. Y si tales cosas pueden acontecer en un planeta, entonces cosas aún más grandes pueden acontecer en un sistema y en las unidades mayores del gran universo, conforme estos también consigan su equilibrio, indicativo del agotamiento de los potenciales para el crecimiento finito.
\vs p118 10:16 \pc En un planeta de tal avanzado orden, la providencia se ha convertido en una actualidad, las circunstancias de la vida están correlacionadas, pero esto no es solo porque el hombre haya llegado a dominar los problemas materiales de su mundo; es también porque ha comenzado a vivir de acuerdo a la tendencia de los universos; está siguiendo el sendero de la Supremacía que lo lleva a la consecución del Padre Universal.
\vs p118 10:17 \pc El reino de Dios está en el corazón de los hombres; y cuando este reino se convierte en actual en el corazón de cada una de las personas de algún mundo, el gobierno de Dios se convertirá en actual en ese planeta; y esto significará que el Ser Supremo habrá logrado allí su soberanía.
\vs p118 10:18 Para hacer realidad la providencia en el tiempo, el hombre debe cumplir con la labor de alcanzar la perfección. Pero él, incluso ahora, puede contemplar anticipadamente esta providencia en sus contenidos eternos al reflexionar sobre el hecho, propio del universo, de que todas las cosas, sean buenas o malas, colaboran juntas para el avance de los mortales conocedores de Dios en su búsqueda del Padre de todo.
\vs p118 10:19 \pc La providencia se vuelve cada vez más perceptible a medida que los hombres se elevan desde lo material a lo espiritual. Lograr una completa percepción espiritual hace que la persona ascendente pueda descubrir armonía en lo que hasta entonces era caos. Incluso la mota morontial representa un avance real en esta dirección.
\vs p118 10:20 La providencia es, en parte, el control directivo del Supremo incompleto, manifestado en los universos incompletos, y, por lo tanto, siempre ha de ser:
\vs p118 10:21 \li{1.}\bibemph{Parcial:} debido a la incompletitud de la actualización del Ser Supremo, y
\vs p118 10:22 \li{2.}\bibemph{Imprevisible:} debido a las fluctuaciones en la actitud de la criatura, que constantemente varía de un nivel a otro, ocasionando, por ello, en apariencias, una reciprocidad de respuesta variable en el Supremo.
\vs p118 10:23 \pc Cuando los hombres oran para que haya una intervención providencial en las circunstancias de su vida, en muchas ocasiones la respuesta a sus oraciones es su propia actitud de cambio hacia la vida. Pero la providencia no es caprichosa, como tampoco es mágica ni una fantasía. Es la manifestación lenta pero segura del poderoso soberano de los universos finitos, cuya presencia majestuosa descubren ocasionalmente las criaturas evolutivas en su progreso en el universo. La providencia es la marcha cierta y segura de las galaxias del espacio y de los seres personales del tiempo hacia las metas de la eternidad, primeramente en el Supremo, después en el Último y quizás en el Absoluto. Y en la infinitud creemos que existe la misma providencia y que se trata de la voluntad, las acciones, el propósito de la Trinidad del Paraíso, que estimula, de este modo, el panorama cósmico de un universo tras otro.
\vsetoff
\vs p118 10:24 [Auspiciado por un mensajero poderoso con residencia temporal en Urantia.]
