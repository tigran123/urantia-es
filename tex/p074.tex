\upaper{74}{Adán y Eva}
\author{Solonia}
\vs p074 0:1 Si contamos desde el año 1934 d. C., Adán y Eva llegaron a Urantia hace 37\,848 años. Lo hicieron a mitad de temporada cuando el Jardín estaba en plena floración. Justo al mediodía y sin previo aviso, los dos transportes seráficos, acompañados del personal de Jerusem encargado del traslado de estos mejoradores biológicos a Urantia, se posó lentamente sobre la superficie del planeta rotatorio en las proximidades del templo del Padre Universal. Toda la labor de rematerialización de los cuerpos de Adán y de Eva se llevó a cabo dentro de los recintos de este santuario recién creado. Y, desde su llegada hasta que se rehicieron bajo doble forma humana a fin de presentarse como los nuevos soberanos del mundo, trascurrieron diez días. Recobraron la consciencia al mismo tiempo. Los hijos e hijas materiales siempre prestan juntos sus servicios; de hecho, la esencia de su servicio es no estar separados en ningún momento ni en ningún lugar. Están concebidos para trabajar en parejas; rara vez actúan solos.
\usection{1. ADÁN Y EVA EN JERUSEM}
\vs p074 1:1 El adán y la eva planetarios de Urantia eran miembros del colectivo de mayor rango de hijos materiales de Jerusem; ambos tenían conjuntamente el número 14311. Pertenecían a la tercera serie física y medían unos dos metros y medio de altura.
\vs p074 1:2 En el momento de ser elegido para venir a Urantia, Adán trabajaba, con su compañera, en los laboratorios físicos de prueba y ensayo de Jerusem. Llevaban más de quince mil años en calidad de directores de la sección de energía experimental aplicada a la modificación de las formas vivas. Mucho tiempo antes, habían sido maestros en las escuelas de ciudadanía para los recién llegados a Jerusem. Y conviene tener todo esto en cuenta a propósito de la narración de su futura forma de proceder en Urantia.
\vs p074 1:3 Cuando se expidió el anuncio pidiendo voluntarios para la misión y aventura adánica en Urantia, todo el colectivo de mayor rango de hijos e hijas materiales se ofreció como voluntario. Los examinadores melquisedecs, con la aprobación de Lanaforge y la de los altísimos de Edentia, seleccionaron finalmente a quienes más adelante ejercerían en Urantia la labor de mejoradores biológicos.
\vs p074 1:4 Adán y Eva habían permanecido leales a Miguel durante la rebelión de Lucifer; no obstante, se les convocó ante el soberano del sistema y todos los miembros de su gobierno para ser entrevistados e instruidos. Se les expuso en detalle todos los asuntos concernientes a Urantia; se les instruyó exhaustivamente en relación a los planes a seguir al aceptar las responsabilidades del gobierno en un mundo tan asolado por los conflictos. De forma conjunta, se les tomó respectivo juramento de lealtad hacia los altísimos de Edentia y hacia Miguel de Lugar de Salvación. Y se les informó debidamente que se consideraran a sí mismos bajo la autoridad del colectivo de síndicos melquisedecs de Urantia, hasta que dicho órgano de gobierno no estimara conveniente renunciar al mando del mundo al que se les había asignado.
\vs p074 1:5 \pc Esta pareja de Jerusem dejaba atrás, en la capital de Satania y en otros lugares, a cien vástagos ---cincuenta hijos y cincuenta hijas---, a magníficas criaturas que habían salvado los escollos del camino de progreso y que, en el momento de la partida de sus padres para Urantia, estaban en servicio como depositarios de la confianza del universo. Y todos ellos estaban presentes en el hermoso templo de los hijos materiales asistiendo a los actos de despedida vinculados a las últimas ceremonias de aceptación de su misión de gracia. Estos hijos acompañaron a sus padres a la central de desmaterialización de su orden y fueron los últimos en despedirse de ellos y en desearles buena suerte, a medida que se quedaban dormidos en el lapso de conciencia del ser personal que precede a la preparación para el transporte seráfico. Estuvieron como familia algún tiempo juntos llenos de júbilo porque sus padres se convertirían pronto en las cabezas visibles, verdaderamente en los únicos gobernantes, del planeta número 606 del sistema de Satania.
\vs p074 1:6 Y así fue como Adán y Eva dejaron Jerusem en medio del reconocimiento y los buenos deseos de sus ciudadanos. Iban de camino a sus nuevas responsabilidades debidamente preparados y estaban bien informados acerca de los deberes que asumirían y de los peligros que encontrarían en Urantia.
\usection{2. LA LLEGADA DE ADÁN Y EVA}
\vs p074 2:1 Adán y Eva se quedaron dormidos en Jerusem y, cuando despertaron en el templo del Padre en Urantia, ante la presencia de la imponente multitud que se había congregado para recibirles, se encontraron en persona con dos seres de los que habían oído hablar mucho, a Van y su fiel compañero Amadón. Estos dos héroes de la secesión de Caligastia fueron los primeros en darles la bienvenida a su nuevo hogar ajardinado.
\vs p074 2:2 El idioma de Edén era el dialecto andónico tal como lo hablaba Amadón. Van y Amadón lo habían mejorado notablemente, elaborando un nuevo alfabeto de veinticuatro letras, y esperaban que llegase a convertirse en la lengua de Urantia, a medida que la cultura de Edén se extendiese por el mundo. Adán y Eva habían conseguido tener un perfecto dominio de este dialecto humano antes de partir de Jerusem, de manera que este hijo de Andón oyó al excelso gobernante de su mundo dirigirse a él en su propia lengua.
\vs p074 2:3 Y aquel día hubo mucha emoción y júbilo en todo Edén; los corredores fueron, a toda prisa, por las palomas mensajeras traídas de lugares cercanos y lejanos, gritando: “Soltad las aves; que lleven la nueva de que ha venido el hijo prometido”. Año tras año, en cientos de asentamientos de creyentes, se había mantenido fielmente una provisión de estas palomas domésticas precisamente para tal ocasión.
\vs p074 2:4 \pc Conforme se difundía en el exterior la noticia de la llegada de Adán, miles de miembros de las tribus cercanas acogieron las enseñanzas de Van y Amadón y, durante meses y meses, los peregrinos continuaron entrando en gran número en Edén para recibir a Adán y Eva y para darles la bienvenida y rendir homenaje a su Padre invisible.
\vs p074 2:5 \pc Poco después de despertarse, se escoltó a Adán y a Eva hasta en el gran montículo situado al norte del templo, donde se les daría una recepción formal. Se había agrandado y preparado esta colina natural para la investidura de los nuevos gobernantes del mundo. Aquí, a mediodía, el comité de recepción de Urantia acogió con beneplácito a este hijo y a esta hija del sistema de Satania. Amadón era el presidente de este comité, que incluía a los siguientes doce miembros: un representante de cada una de las seis razas sangiks; el jefe en funciones de los seres intermedios; Annán, hija leal y portavoz de los noditas; Noé, el hijo del arquitecto y constructor del Jardín y ejecutor de los proyectos de su difunto padre; y los dos portadores de vida con residencia en el planeta.
\vs p074 2:6 En el siguiente acto, el melquisedec de mayor rango, jefe del consejo de los síndicos de Urantia, transfirió el cargo de la custodia planetaria a Adán y a Eva. El hijo y la hija materiales prestaron juramento de lealtad a los altísimos de Norlatiadec y a Miguel de Nebadón, y Van los proclamó gobernantes de Urantia, renunciando así a la autoridad nominal que él ostentaba desde hacía más de ciento cincuenta mil años por actuación de los síndicos melquisedecs.
\vs p074 2:7 Y se presentó a Adán y a Eva con vestiduras reales con motivo de su toma de posesión oficial del gobierno del mundo. No todas las artes de Dalamatia se habían perdido para el mundo; aún se practicaba la tejeduría en los días de Edén.
\vs p074 2:8 Entonces se oyó la proclamación de los arcángeles y se trasmitió la voz de Gabriel, que decretó el llamamiento nominal al segundo juicio de Urantia y la resurrección de los supervivientes dormidos de la segunda dispensación de gracia y misericordia del planeta 606 de Satania. La dispensación del príncipe ha pasado; la era de Adán, la tercera época planetaria, se inaugura entre escenas de sencilla grandeza; y los nuevos gobernantes de Urantia comienzan su reinado bajo condiciones aparentemente favorables, pese a la confusión mundial provocada por la falta de cooperación de su antecesor respecto a la autoridad del planeta.
\usection{3. ADÁN Y EVA CONOCEN EL PLANETA}
\vs p074 3:1 Y ahora, tras su investidura formal, Adán y Eva tomaron penosamente conciencia de su aislamiento planetario. Las transmisiones que les eran familiares estaban en silencio y todas las vías de comunicación extraplanetarias cortadas. Sus compañeros de Jerusem habían ido a mundos que marchaban sin contratiempos, con un príncipe planetario bien asentado y una experimentada comitiva preparada para recibirlos y capacitada para cooperar con ellos durante su temprana experiencia en tales mundos. Pero en Urantia la rebelión lo había cambiado todo. Aquí el príncipe planetario estaba muy presente y, aunque se le había despojado de la mayor parte de su poder para obrar el mal, aún le era posible dificultar la labor de Adán y Eva y, hasta cierto punto, hacerla arriesgada de llevar a cabo. Aquella noche, este hijo e hija de Jerusem pasearon, serios y desencantados, por el Jardín bajo el brillo de la luna llena, mientras comentaban los planes para el siguiente día.
\vs p074 3:2 De este modo acabó el primer día de Adán y Eva en la aislada Urantia, aquel confuso planeta lugar de la traición de Caligastia; caminaron y hablaron hasta muy entrada la noche, su primera noche en la tierra. Se sintieron muy solos.
\vs p074 3:3 \pc Adán pasó su segundo día en la tierra en reunión con los síndicos planetarios y el consejo consultivo. De los melquisedecs, y sus colaboradores, Adán y Eva tuvieron una información más detallada sobre la rebelión de Caligastia y las consecuencias de aquel levantamiento para el progreso del mundo. Y el largo relato sobre la mala gestión de los asuntos del mundo resultaba, en líneas generales, descorazonador. Conocieron todos los hechos relativos al total derrumbe de la trama urdida por Caligastia para acelerar el proceso de la evolución social. También se dieron perfecta cuenta de la insensatez de intentar lograr el avance planetario con independencia del plan divino de progreso. Y así terminó un triste pero instructivo día ---su segundo en Urantia---.
\vs p074 3:4 \pc Dedicaron su tercer día a inspeccionar el Jardín. Trasportados por los fándores, unas enormes aves de pasajeros, sobrevolaron el Jardín, pudiendo contemplar las inmensas extensiones del más bello paraje de la tierra. Aquel día de inspección culminó con un gran banquete en honor a todos los que habían trabajado para crear este jardín de una belleza y una grandiosidad paradisíacas. Y, de nuevo, en su tercer día, este hijo material y su compañera pasearon por el Jardín hasta altas horas de la noche, mientras conversaban sobre la enormidad de sus problemas.
\vs p074 3:5 \pc Al cuarto día, Adán y Eva pronunciaron un discurso ante la asamblea del Jardín. Desde el montículo inaugural, se dirigieron a la gente para hablarles de sus proyectos de rehabilitación del mundo y esbozaron los métodos por los que tratarían de rescatar la cultura social de Urantia de los bajos niveles en los que había caído como resultado del pecado y la rebelión. Aquel fue un gran día, y concluyó con una celebración en consideración del consejo de hombres y mujeres que habían sido elegidos para asumir responsabilidades en la nueva administración de los asuntos del mundo. ¡Tomad nota! En este grupo había tanto mujeres como hombres, y era la primera vez que ocurría algo así en la tierra desde los días de Dalamatia. Era sorprendentemente novedoso observar a Eva, una mujer, compartir los honores y las responsabilidades de los asuntos mundiales con un hombre. Y así acabó su cuarto día en Urantia.
\vs p074 3:6 \pc Ocuparon su quinto día en la organización del gobierno provisional, que ejercería su actividad hasta que los síndicos melquisedecs se marcharan de Urantia.
\vs p074 3:7 \pc El sexto día lo dedicaron a inspeccionar las numerosas clases de hombres y animales de allí. Por las murallas orientales de Edén, Adán y Eva, escoltados todo el día, observaron la vida animal del planeta, llegando a una mejor comprensión de lo que se debía hacer para poner orden en la confusión de un mundo habitado por tanta variedad de criaturas vivas.
\vs p074 3:8 En esta excursión, los que acompañaban a Adán se quedaron muy sorprendidos al observar su gran conocimiento de la naturaleza y del comportamiento de los miles y miles de animales que se le mostraban. Tan solo con mirar un instante al animal sabía especificar sus características. De inmediato podía asignar nombres que detallaban el origen, la naturaleza y la actividad de todas las criaturas materiales. Quienes le servían de guía en esta gira de inspección no sabían que el nuevo gobernante del mundo era el anatomista más experimentado de toda Satania; y Eva estaba igualmente cualificada. Adán maravilló a sus acompañantes al describir multitudes de seres vivos demasiado pequeños como para poder percibirse a simple vista.
\vs p074 3:9 Cuando terminó su sexto día de estancia en la tierra, Adán y Eva descansaron por primera vez en su nuevo hogar del “oriente de Edén”. Los primeros seis días de su aventura urantiana habían sido muy ajetreados y ansiaban pasar todo un día de forma placentera, libres de cualquier actividad.
\vs p074 3:10 Pero las circunstancias determinaron algo diferente. La experiencia del día anterior en la que Adán había analizado con tanta inteligencia y exhaustividad la vida animal de Urantia, junto a su magistral discurso inaugural y sus encantadores modos, se había ganado los corazones de los moradores del Jardín y había sobrecogido sus intelectos, de tal manera que no solo estaban sinceramente decididos a aceptar como gobernantes al hijo y a la hija materiales recién llegados de Jerusem, sino que la mayoría estaba prácticamente dispuesta a postrarse ante ellos y adorarles como dioses.
\usection{4. LA PRIMERA REVUELTA}
\vs p074 4:1 Aquella noche, la siguiente al sexto día, mientras Adán y Eva dormían, ocurrieron cosas extrañas en las inmediaciones del templo del Padre, en el sector central de Edén. Allí, bajo los rayos de la tenue luna, cientos de hombres y mujeres escucharon durante horas, con entusiasmo y emoción, el apasionado alegato de sus líderes. Tenían buenas intenciones, pero no alcanzaban a comprender la sencillez de la actitud fraternal y democrática de los nuevos gobernantes. Y, mucho antes del amanecer, los nuevos miembros temporales de la administración de los asuntos del mundo llegaron a la conclusión, prácticamente unánime, de que Adán y su compañera eran demasiado modestos y discretos. Convinieron que la Divinidad había descendido a la tierra en forma corporal, que Adán y Eva eran en realidad dioses o bien estaban tan cerca de tal condición que eran dignos de reverencia y culto.
\vs p074 4:2 Los sorprendentes acontecimientos de los seis primeros días de Adán y Eva en la tierra excedían por completo la capacidad de las mentes, poco preparadas, incluso de los mejores hombres del mundo; sus cabezas eran un torbellino; se dejaron arrastrar por la idea de llevar a la noble pareja, al mediodía, al templo del Padre para rendirles adoración y respeto y postrarse humilde y sumisamente ante ellos. Los habitantes del Jardín actuaban con total sinceridad.
\vs p074 4:3 Van protestó. Amadón se encontraba ausente por estar a cargo de la guardia de honor que durante la noche se había quedado con Adán y Eva. Pero se hizo caso omiso a su protesta. Le dijeron que, igualmente, él era demasiado modesto, demasiado discreto; que él tampoco estaba muy lejos de ser un dios, si no, ¿cómo podía haber vivido tanto tiempo en la tierra, y cómo había propiciado tan magno evento como la venida de Adán? Y cuando los agitados edenitas estaban a punto de atraparlo y llevarlo al montículo para adorarle, Van se abrió paso entre la muchedumbre y, como podía comunicarse con los seres intermedios, envió a toda prisa a su líder a Adán.
\vs p074 4:4 Casi al alba de su séptimo día en la tierra, Adán y Eva oyeron la alarmante noticia sobre el propósito de estos errados, aunque bienintencionados mortales; y, entonces, mientras que las aves de pasajeros volaban con celeridad para recogerlos y llevarlos al templo, los seres intermedios, capaces de llevar a cabo estas cosas, transportaron a Adán y Eva al templo del Padre. Temprano, en la mañana de este séptimo día, Adán, desde el montículo donde tan recientemente se les había recibido, ofreció una explicación acerca de los órdenes divinos de filiación, dejando claro a estas mentes terrenales que solo el Padre y aquellos a los que él designara podían ser objeto de adoración. Adán puso de manifiesto que aceptaría cualquier honor y muestra de respeto, pero la adoración, ¡nunca!
\vs p074 4:5 Aquel fue un día memorable y, justo antes del mediodía, acercándose el momento de la llegada del mensajero seráfico que portaba el reconocimiento, de parte de Jerusem, de la instauración de los gobernantes del mundo, Adán y Eva, apartándose de la muchedumbre, señalaron al templo del Padre y dijeron: “Id ahora al símbolo material de la presencia invisible del Padre e inclinaos para adorar a quien nos hizo a todos y nos mantiene vivos. Y que este acto sea expresión de vuestra promesa sincera de que nunca más estaréis tentados de adorar a nadie que no sea Dios”. Todos hicieron lo que Adán les indicó. El hijo y la hija materiales se quedaron solos en el montículo con las cabezas inclinadas mientras que la gente se postraba en torno al templo.
\vs p074 4:6 \pc Y así fue como se originó la tradición del día del \bibemph{sabbat}. Siempre en Edén se consagró el séptimo día para congregarse en el templo al mediodía; durante mucho tiempo, fue costumbre emplear este día para la formación personal. La mañana se dedicaba al mejoramiento físico; el mediodía, al culto espiritual; las primeras horas de tarde, al cultivo de la mente; y, las últimas, al disfrute social. Esto nunca fue la ley en Edén, pero se hizo habitual mientras que el gobierno adánico predominó en la tierra.
\usection{5. EL GOBIERNO DE ADÁN}
\vs p074 5:1 Durante casi siete años después de la llegada de Adán, los síndicos melquisedecs siguieron activos, pero había llegado finalmente el momento de entregar la administración de los asuntos mundiales a Adán y regresar a Jerusem.
\vs p074 5:2 La despedida de los síndicos llevó el día entero y, avanzada la tarde, cada uno de ellos, al partir, ofreció a Adán y a Eva sus recomendaciones y sus mejores deseos. Varias veces Adán había solicitado a sus asesores que permaneciesen con él en la tierra, pero siempre se le denegaron sus peticiones. Había llegado la hora de que los hijos materiales asumieran plena responsabilidad en la dirección de los asuntos mundiales. Y así, a medianoche, los transportes seráficos de Satania dejaron el planeta rumbo a Jerusem con catorce seres, entre los que se encontraban los doce melquisedecs y Van y Amadón, que lo hacían al mismo tiempo.
\vs p074 5:3 \pc Durante un tiempo, todo marchó bastante bien en Urantia, y pareció que Adán, en algún momento, sería capaz de llevar a cabo algún plan para promover la expansión paulatina de la civilización edénica. Atendiendo a las recomendaciones de los melquisedecs, comenzó a fomentar las artes de la manufactura con la idea de desarrollar relaciones comerciales con el mundo exterior. Cuando Edén convulsionó, había más de cien plantas manufactureras operando, y se habían establecido amplias relaciones comerciales con las tribus cercanas.
\vs p074 5:4 Durante muchas eras, se había formado expresamente a Adán y a Eva para mejorar el mundo con vistas a contribuir de forma particular al avance de la civilización evolutiva; pero ahora tenían que enfrentarse a problemas acuciantes, tales como el establecimiento de la ley y el orden en un mundo de seres humanos salvajes, primitivos y semicivilizados. Aparte de lo más selecto de la población de la tierra, que se hallaba reunida en el Jardín, solo algunos pocos grupos dispersos estaban de alguna manera listos para ser receptores de la cultura adánica.
\vs p074 5:5 Adán hizo un esfuerzo heroico y decidido para establecer un gobierno mundial, pero se encontraba a cada paso con una tenaz resistencia. Adán ya había puesto en funcionamiento por todo Edén un sistema de control de los distintos grupos existentes y había federado a estas compañías en la liga edénica. Si bien, se produjeron problemas muy graves cuando salió del Jardín y trató de aplicar estas ideas a las tribus de la periferia. En el momento en el que los colaboradores de Adán comenzaron a operar fuera del Jardín, se toparon con la resistencia directa y bien planificada de Caligastia y Daligastia. El príncipe caído había sido depuesto como gobernante del mundo, pero no se le había apartado del planeta. Seguía presente en él y capaz, al menos hasta cierto punto, de oponer resistencia a todos los planes de Adán para rehabilitar la sociedad humana. Adán quiso prevenir a las razas contra Caligastia, pero la tarea se vio muy dificultada porque su archienemigo era invisible a los ojos de los mortales.
\vs p074 5:6 Incluso entre los edenitas había mentes confundidas que apoyaban las enseñanzas de Caligastia sobre la desmedida libertad personal; y causaron a Adán un sinfín de problemas; siempre tenían que estar alterando el trazado de sus mejores planes para el progreso metódico y el relevante desarrollo del mundo. Finalmente, Adán se vio forzado a retirar su programa de inmediata socialización; volvió al método de organización de Van, dividiendo a los edenitas en compañías de cien miembros con capitanes al mando de cada una de ellas y tenientes a cargo de grupos de diez miembros.
\vs p074 5:7 Adán y Eva habían venido a instituir un gobierno representativo en lugar de monárquico, pero no encontraron ningún gobierno que mereciera ese nombre sobre la faz de la tierra. De momento, Adán cesó en su empeño por instaurar un gobierno representativo y, antes del colapso del régimen edénico, logró establecer casi cien centros de comercio y sociales en zonas periféricas, donde personas de fortaleza gobernaban en su nombre. La mayoría de estos centros habían sido organizados anteriormente por Van y Amadón.
\vs p074 5:8 El envío de embajadores de una tribu a otra data de los tiempos de Adán; significó un gran paso hacia adelante en la evolución del gobierno.
\usection{6. VIDA FAMILIAR DE ADÁN Y EVA}
\vs p074 6:1 Los terrenos de la familia adánica abarcaban algo más de trece kilómetros cuadrados. En sus inmediaciones, se habían tomado medidas para atender a más de trescientos mil descendientes por línea directa. No obstante, solo se construyó la primera parte de las edificaciones planeadas. Antes de que el número de miembros de la familia adánica sobrepasara esta previsión inicial, todo el plan edénico se había trastocado y el Jardín se había desalojado.
\vs p074 6:2 \pc Adánez fue el primogénito de la raza violeta; le siguió su hermana, y Evánez, el segundo hijo de Adán y Eva. Eva era madre de cinco hijos antes de que se marcharan los melquisedecs ---tres hijos y dos hijas---. Los dos siguientes fueron gemelos. Antes de cometerse la transgresión, Eva había dado a luz a sesenta y tres vástagos, treinta y dos hijas y treinta y un hijos. Cuando Adán y Eva abandonaron el Jardín, su familia constaba de cuatro generaciones con un total de 1647 descendientes por línea directa. Tuvieron cuarenta y dos hijos después de dejar el Jardín, además de los dos descendientes tenidos al emparejarse respectivamente con el linaje de los mortales de la tierra. Y esto no incluye los ancestros adánicos entre los noditas y las razas evolutivas.
\vs p074 6:3 Cuando a la edad de un año los hijos de Adán dejaban de ser amamantados por su madre, no tomaban leche de animales. Eva tenía a su disposición la leche de una gran variedad de frutos secos y zumos de muchas frutas y, conociendo perfectamente la composición química y el valor energético de estos alimentos, los combinaba convenientemente para alimentar a sus hijos hasta que les aparecieran los dientes.
\vs p074 6:4 Aunque la cocción se empleaba generalmente en las inmediaciones de la zona adánica de Edén, en el hogar de Adán no se cocinaba. Sus alimentos ---frutas, frutos secos y cereales--- ya estaban listos para consumirse cuando maduraban. Comían una vez al día, poco después del mediodía. Adán y Eva también absorbían “luz y energía” directamente de ciertas emanaciones espaciales al mismo tiempo asistidos por el árbol de la vida.
\vs p074 6:5 \pc Los cuerpos de Adán y de Eva desprendían una luz resplandeciente, pero siempre se vestían con arreglo a la costumbre de sus acompañantes. Aunque llevaban poca ropa durante el día, a la caída de la tarde usaban túnicas como vestimenta nocturna. El origen de la tradicional aureola que rodea las cabezas de hombres supuestamente píos y sagrados se remonta a los días de Adán y Eva. Puesto que sus vestiduras oscurecían gran parte de las emanaciones lumínicas de sus cuerpos, solo se percibía la radiante luminosidad que irradiaba de sus cabezas. Los descendientes de Adánez siempre representaban así su concepto de las personas que se tenían como extraordinarias en cuanto a su desarrollo espiritual.
\vs p074 6:6 Adán y Eva podían comunicarse entre sí y con su inmediata progenie hasta una distancia de unos ochenta kilómetros. Este intercambio de pensamientos se llevaba a efecto por medio de unas sensibles cavidades de gas localizadas muy cerca de sus estructuras cerebrales. Mediante este mecanismo, podían enviar y recibir las vibraciones del pensamiento. Pero esta capacidad quedó instantáneamente interrumpida en cuanto su mente se rindió a la discordia y a la perturbación del mal.
\vs p074 6:7 \pc Los hijos de Adán asistían a sus propias escuelas hasta que cumplían los dieciséis años de edad; los mayores daban clases a los más jóvenes. Los pequeños cambiaban de actividad cada treinta minutos; los mayores, cada hora. Y, resultaba algo realmente novedoso en Urantia, poder observar a estos hijos de Adán y de Eva en sus juegos, realizando actividades alegres y estimulantes por pura diversión. El juego y el humor de las razas de hoy en día proceden, en gran medida, del linaje adánico. Todos los adanitas tenían un gran aprecio por la música, al igual que un buen sentido del humor.
\vs p074 6:8 La edad media para prometerse en matrimonio era de dieciocho años; entonces, estos jóvenes iniciaban un curso de dos años de instrucción con miras a asumir sus responsabilidades conyugales. A los veinte años podían contraer matrimonio y, tras este, comenzaban su vida laboral o se preparaban especialmente para ella.
\vs p074 6:9 La costumbre de algunas naciones que surgirían más tarde de permitir que en las familias reales, supuestamente descendientes de los dioses, se diese el matrimonio entre hermanos y hermanas se remonta a la práctica que se estableció por necesidad entre la progenie adánica. Adán y Eva siempre oficiaron las ceremonias matrimoniales de la primera y segunda generación del Jardín.
\usection{7. LA VIDA EN EL JARDÍN}
\vs p074 7:1 Salvo por los cuatro años que asistían a las escuelas del oeste, los hijos de Adán vivían y trabajaban al “oriente de Edén”. Hasta los dieciséis años de edad recibían una formación intelectual, siguiendo la metodología de las escuelas de Jerusem. Desde los dieciséis hasta los veinte, se les instruía en las escuelas de Urantia del otro extremo del Jardín, en donde también servían como maestros de cursos inferiores.
\vs p074 7:2 El único propósito del sistema escolar de las escuelas del oeste era la \bibemph{socialización}. Los períodos de receso, previos al mediodía, se dedicaban a la práctica de la horticultura y de la agricultura. Los mediodías se empleaban en los juegos competitivos. Los períodos vespertinos se destinaban a las relaciones sociales y al cultivo de las amistades personales. La formación religiosa y sexual era competencia del entorno familiar, esto es, era deber de los padres.
\vs p074 7:3 En estas escuelas, se impartía enseñanza acerca de:
\vs p074 7:4 \li{1.}La salud y el cuidado del cuerpo.
\vs p074 7:5 \li{2.}La regla de oro: las normas para las relaciones sociales.
\vs p074 7:6 \li{3.}La relación de los derechos individuales con los del grupo y las obligaciones comunitarias.
\vs p074 7:7 \li{4.}La historia y la cultura de las distintas razas de la tierra.
\vs p074 7:8 \li{5.}Los métodos para el avance y la mejora del comercio mundial.
\vs p074 7:9 \li{6.}La coordinación de deberes y emociones en conflicto.
\vs p074 7:10 \li{7.}El cultivo del juego, el humor y las alternativas en el terreno de la competición a la lucha física.
\vs p074 7:11 \pc Las escuelas, de hecho cualquier actividad que se realizara en el Jardín, eran siempre accesibles a los visitantes. Se admitían libremente a observadores no armados durante breves visitas. Para residir en el Jardín, los urantianos tenían que ser “adoptados”. Se les instruía sobre el plan y el objetivo de la misión de gracia adánica, expresaban su intención de adherirse a ella y luego realizaban una declaración de lealtad a las reglas sociales de Adán y a la soberanía espiritual del Padre Universal.
\vs p074 7:12 \pc Las leyes del Jardín se basaban en los antiguos códigos de Dalamatia y se promulgaron con arreglo a siete apartados:
\vs p074 7:13 \li{1.}Leyes de la salud y el saneamiento.
\vs p074 7:14 \li{2.}Reglamentación social del Jardín.
\vs p074 7:15 \li{3.}Código de intercambio y comercio.
\vs p074 7:16 \li{4.}Leyes de juego limpio y competición.
\vs p074 7:17 \li{5.}Leyes de la vida familiar.
\vs p074 7:18 \li{6.}Códigos civiles de la regla de oro.
\vs p074 7:19 \li{7.}Los siete mandamientos de la suprema regla moral.
\vs p074 7:20 \pc La ley moral de Edén difería poco de los siete mandamientos de Dalamatia. Pero los adanitas impartían muchas otras razones para su cumplimiento; por ejemplo, en el caso del mandato de no matar, la presencia interior del modelador del pensamiento se ofrecía como otro motivo más para no destruir la vida humana. Enseñaban que “el que derramase la sangre del hombre, por el hombre su sangre será derramada, porque a imagen de Dios él hizo al hombre”.
\vs p074 7:21 El mediodía era la hora del culto público en Edén; el atardecer, la del culto familiar. Adán hizo todo lo posible para desincentivar el uso de oraciones preestablecidas, enseñando que para que una oración fuese eficaz tenía que ser enteramente personal, que debía ser “el anhelo del alma”; pero los edenitas continuaron empleando las oraciones y fórmulas heredadas de los tiempos de Dalamatia. Adán también procuró reemplazar los sacrificios de sangre en las ceremonias religiosas por las ofrendas del fruto de la tierra, pero había hecho pocos progresos en este sentido antes de la caída del Jardín.
\vs p074 7:22 \pc Adán trató de instruir a las razas en la igualdad entre los sexos. La manera en la que Eva trabajaba al lado de su marido dejó una profunda impresión en todos los habitantes del Jardín. Adán claramente les impartía la enseñanza de que la mujer, al igual que el hombre, contribuía a aquellos elementos vitales que se unían para formar un nuevo ser. Hasta entonces, la humanidad había supuesto que la procreación residía en las “entrañas del padre”. Habían considerado a la madre simplemente como alguien que nutría al no nato y amamantaba al neonato.
\vs p074 7:23 Adán enseñó a sus contemporáneos todo cuanto podían comprender, pero, comparativamente hablando, resultó no ser demasiado. No obstante, las razas más inteligentes de la tierra aguardaban ansiosamente el momento en el que se les permitiría contraer matrimonio con los hijos superiores de la raza violeta. ¡Y qué mundo tan diferente habría sido Urantia si se hubiese llevado a cabo este gran plan de elevación de las razas! Incluso así, la pequeña cantidad de sangre que los pueblos evolutivos casualmente obtuvieron de esta raza importada trajo consigo enormes beneficios.
\vs p074 7:24 Y así fue como Adán trabajó por el bienestar y la elevación del mundo en el que residía. Pero liderar a estos pueblos mezclados y mestizos para que siguiesen el mejor camino resultó ser una difícil tarea.
\usection{8. LA LEYENDA DE LA CREACIÓN}
\vs p074 8:1 El relato de la creación de Urantia en seis días está basado en la tradición de que Adán y Eva habían empleado justamente seis días en su primer reconocimiento del Jardín. Esta circunstancia dio pie a la casi sagrada afirmación del periodo de tiempo de la semana, que había sido originariamente introducido por los dalamatianos. El hecho de que Adán estuviese seis días inspeccionando el Jardín y haciendo planes previos para su organización no se debió a un acuerdo prefijado; se decidió día tras día. La elección del séptimo día para el culto fue algo enteramente fortuito surgido a partir de los hechos que aquí se narran.
\vs p074 8:2 La leyenda de la creación del mundo en seis días se fraguó posteriormente, en realidad, más de treinta mil años después. Un rasgo de esta narrativa, la repentina aparición del sol y la luna, tiene posiblemente su origen en las tradiciones que se hicieron eco de la aparición repentina, en otro tiempo, del mundo a partir de una densa nube espacial de materia minúscula, que durante mucho tiempo había ocultado al sol y a la luna.
\vs p074 8:3 La historia de la creación de Eva a partir de la costilla de Adán es una confusa sinopsis de la llegada de Adán y de la cirugía celestial relacionada con el intercambio de sustancias vivas a la venida del personal corpóreo del príncipe planetario más de cuatrocientos cincuenta mil años antes.
\vs p074 8:4 \pc La mayoría de los pueblos del mundo se ha visto influenciada por la tradición de que Adán y Eva poseían formas físicas creadas para ellos al llegar a Urantia. La opinión de que el hombre procedía del barro estaba bastante generalizada en el hemisferio oriental; esta tradición se puede rastrear por todo el mundo, desde las islas Filipinas hasta África. Y muchos colectivos aceptaron la historia de que el hombre venía del barro, basándose en alguna forma de creación especial, en lugar de acudir a sus anteriores creencias sobre la creación progresiva: la evolución.
\vs p074 8:5 Alejada de las influencias de Dalamatia y Edén, la humanidad se inclinaba a creer en el ascenso gradual de la raza humana. El hecho evolutivo no es un descubrimiento moderno; los antiguos entendían el carácter lento y evolutivo del progreso humano. Los primeros griegos tenían las ideas claras en este sentido al margen de su proximidad con Mesopotamia. Aunque las distintas razas de la tierra llegarían lamentablemente a confundirse en cuanto a sus nociones sobre la evolución, muchas tribus primitivas creían y enseñaban que eran descendientes de diferentes animales. Los pueblos primitivos adoptaron la costumbre de elegir para sus “tótems” a animales que suponían eran ancestros suyos. Algunas tribus de indios norteamericanos creían que habían tenido su origen en los castores y en los coyotes. Otras tribus de África enseñan que descienden de la hiena; una tribu malaya, del lémur; un grupo de Nueva Guinea, del loro.
\vs p074 8:6 Debido a su contacto directo con los vestigios de la civilización adanita, los babilonios magnificaron y embellecieron la historia de la creación del hombre; enseñaron que había descendido directamente de los dioses. Adoptaron la idea de un origen aristocrático de la raza, que era incompatible incluso con la doctrina de la creación a partir del barro.
\vs p074 8:7 \pc El relato de la creación del Antiguo Testamento data de mucho tiempo después de Moisés; él nunca enseñó a los hebreos tal distorsionada historia. Aunque sí presentó a los israelitas una narración sencilla y condensada de la creación, esperando así reforzar su llamamiento a la adoración del Creador, del Padre Universal, que él llamaba el Señor Dios de Israel.
\vs p074 8:8 En sus primeras enseñanzas, Moisés, actuando con mucha sensatez, no quiso retroceder hasta antes de los tiempos de Adán, y puesto que él era el maestro supremo de los hebreos, la historia de Adán llegó a asociarse estrechamente con la de la creación. Las más antiguas tradiciones reconocían la existencia de una civilización preadánica, algo claramente demostrado por el intento de redactores posteriores de eliminar, sin conseguirlo, toda referencia a asuntos humanos previos a los tiempos de Adán. Se olvidaron de borrar la reveladora alusión a la emigración de Caín a la tierra de “Nod”, donde tomó esposa.
\vs p074 8:9 Durante mucho tiempo después de su llegada a Palestina, los hebreos no poseían ningún lenguaje escrito de uso generalizado. Aprendieron a utilizar el alfabeto de sus vecinos los filisteos, que eran refugiados políticos de la civilización superior de Creta. Los hebreos escribieron poco hasta alrededor del año 900 a. C., y como no dispusieron de escritura hasta esta fecha tan tardía, circularon entre ellos distintos relatos de la creación; aunque, tras el cautiverio en Babilonia, tendieron más a aceptar una versión mesopotámica modificada de este relato.
\vs p074 8:10 La tradición judía se cristalizó en torno a Moisés y, debido a su esfuerzo por remontar el linaje de Abraham hasta Adán, los judíos asumieron que Adán fue el primer hombre de la raza humana. Yahvé fue el creador y, puesto que se suponía que Adán era el primer hombre, él debía haber hecho el mundo justo antes de crear a Adán. Y, entonces, la tradición de los seis días de Adán se entrelazó con la historia, con el resultado de que, casi mil años después de la estancia de Moisés en la tierra, el relato de la creación en seis días tomó forma escrita, atribuyéndosele a él, posteriormente, la autoría.
\vs p074 8:11 Cuando los sacerdotes judíos volvieron a Jerusalén, ya habían terminado de escribir su narrativa sobre el principio de las cosas, y pronto reivindicaron que se trataba de una historia recién descubierta de la creación, escrita por Moisés. Pero los hebreos contemporáneos de alrededor del año 500 a. C. no consideraban que estos escritos fuesen revelaciones divinas; los veían de la misma forma que los pueblos posteriores verían las narraciones mitológicas.
\vs p074 8:12 Este espurio documento, que se pensaba contenía las enseñanzas de Moisés, se puso en conocimiento de Ptolomeo, el rey griego de Egipto, que lo hizo traducir al griego, para su nueva biblioteca en Alejandría, acudiendo a una comisión de setenta eruditos. Y así fue como este relato se incorporó a los escritos, que formarían luego parte de las posteriores recopilaciones de las “escrituras sagradas” de las religiones hebrea y cristiana. Al identificarse con estos sistemas teológicos, tales ideas tuvieron, durante mucho tiempo, una profunda influencia sobre la filosofía de muchos pueblos occidentales.
\vs p074 8:13 Los maestros cristianos perpetuaron la creencia de que la raza humana se había creado con un “hágase”, y esto llevó directamente a la formación de la hipótesis de una antigua edad de oro de un utópico éxtasis y a la teoría de la caída del hombre o del suprahombre que diera explicación a la condición, no utópica, de la sociedad. Estas perspectivas de la vida y del lugar del hombre en el universo fueron, en el mejor de los casos, desalentadoras, puesto que se basaban en la creencia de un retroceso en lugar de un avance, además de suponer la existencia de una Deidad vengativa, que había hecho descender su ira sobre la raza humana como castigo por los errores de ciertos gobernantes planetarios del pasado.
\vs p074 8:14 \pc La “edad de oro” es un mito, pero Edén fue una realidad, y ciertamente se produjo la caída del Jardín. Adán y Eva llevaban ciento diecisiete años en él cuando, por la impaciencia de ella y los errores de juicio de él, osaron desviarse del camino prescrito, provocándoles pronto a ellos la desgracia y a toda Urantia un pernicioso retraso de su desarrollo evolutivo.
\vsetoff
\vs p074 8:15 [Narrado por Solonia, la “voz” seráfica “del Jardín”.]
