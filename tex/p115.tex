\upaper{115}{El Ser Supremo}
\author{Mensajero poderoso}
\vs p115 0:1 Con Dios Padre, la filiación es una magnífica forma de relacionarse. Con el Dios Supremo, el logro es un prerrequisito para la obtención de estatus ---se debe hacer algo al igual que ser algo---.
\usection{1. LA RELATIVIDAD DE LOS MARCOS CONCEPTUALES}
\vs p115 1:1 Los intelectos de carácter parcial, incompleto y evolutivo se sentirían impotentes, serían incapaces incluso de precisar un primer patrón de pensamiento racional si no fuese por la habilidad consustancial a cualquier mente, mejor o peor dotada, de articular un \bibemph{marco conceptual del universo} en el que pensar. Si la mente no puede extraer conclusiones, si no puede adentrarse hasta los verdaderos orígenes, dicha mente, indefectiblemente, postulará conclusiones e inventará orígenes para obtener un instrumento de pensamiento lógico dentro del marco de estos postulados creados por la mente. Aunque tales marcos conceptuales, que sirven de base para el pensamiento de las criaturas, resultan indispensables para las operaciones intelectuales racionales, son, sin excepción, erróneos en mayor o menor grado.
\vs p115 1:2 Los marcos conceptuales del universo solo son relativamente verdaderos; representan un útil andamiaje que acabará por ceder el paso a la expansión de una comprensión cósmica ampliada. Las concepciones de la verdad, la belleza y la bondad, la moral, la ética, el deber, el amor, la divinidad, el origen, la existencia, el propósito, el destino, el tiempo, el espacio, e incluso de la Deidad, son solamente ciertas de forma relativa. Dios es mucho, mucho más que un Padre, pero el concepto de Padre que tiene el hombre de Dios es el más elevado que puede albergar; no obstante, la representación Padre\hyp{}Hijo de la relación Creador\hyp{}criatura se engrandecerá gracias a esa conceptualización sobrehumana de la Deidad, que se adquirirá en Orvontón, Havona y en el Paraíso. El hombre debe pensar en el contexto del marco conceptual humano del universo, pero eso no significa que no pueda contemplar otros marcos, y de mayor elevación, en los que el pensamiento pueda efectuarse.
\vs p115 1:3 Con el fin de facilitar la comprensión humana del universo de los universos, los distintos niveles de la realidad cósmica se han designado como finito, absonito y absoluto. De ellos, únicamente el absoluto es incondicionalmente eterno, genuinamente existencial. Los absonitos y los finitos son derivaciones, modificaciones, condicionamientos y atenuaciones de la realidad absoluta, primigenia y primordial, de la infinitud.
\vs p115 1:4 Los ámbitos de lo finito existen en virtud del propósito eterno de Dios. Las criaturas finitas, elevadas y humildes, pueden proponer teorías, y así lo han hecho, respecto a la necesidad de lo finito en la eficiente organización cósmica, pero, en última instancia, existe porque Dios lo quiso. No hay explicación para el universo, como tampoco puede una criatura finita dar una razón lógica de su propia existencia individual sin apelar a actos anteriores y a la voluntad preexistente de seres ancestrales, creadores o procreadores.
\usection{2. LA BASE ABSOLUTA PARA LA SUPREMACÍA}
\vs p115 2:1 Desde el punto de vista existencial, no puede suceder nada nuevo en ninguna galaxia, porque la completitud de la infinitud intrínseca al YO SOY está eternamente presente en los siete Absolutos, está vinculada operativamente en las triunidades y, es parte de las triodidades mediante transmisión. Pero el hecho de que la infinitud esté, por lo tanto, existencialmente presente en estas agrupaciones absolutas no imposibilita, de ningún modo, la realización de nuevos experienciales cósmicos. Desde el punto de vista de la criatura finita, la infinitud contiene mucho de potencial, más del orden de una posibilidad futura que de una realidad presente en actualidad.
\vs p115 2:2 El valor es un elemento único en la realidad del universo. No comprendemos cómo es posible incrementar el valor de lo que es infinito y divino. Pero nos damos cuenta de que los \bibemph{contenidos} se pueden modificar si no aumentar, incluso en la relaciones de la Deidad infinita. Para los universos experienciales, hasta los valores divinos se modifican e incrementan mediante la comprensión ampliada de los contenidos de la realidad.
\vs p115 2:3 La totalidad del plan de la creación y de la evolución del universo en todos su niveles experimentables es al parecer una cuestión de convertir las potencialidades en actualidades; y esta transmutación guarda igualmente relación con los ámbitos de la potencialidad espacial, mental y espiritual.
\vs p115 2:4 Aparentemente, el método por el que las posibilidades del cosmos adquieren existencia real difiere de nivel en nivel; en lo finito resulta de la evolución experiencial y, en lo absonito, del devenir experiencial. La infinitud existencial es además incondicionada en cuanto a su total inclusividad y esta misma inclusividad plena debe, ineludiblemente, abarcar la posibilidad de la evolución experiencial de lo finito. Y la posibilidad de tal crecimiento experiencial llega a manifestarse en el universo por medio de las relaciones existentes en la triodidad, las cuales inciden sobre y en el Supremo.
\usection{3. LO PRIMIGENIO, LO ACTUAL Y LO POTENCIAL}
\vs p115 3:1 Conceptualmente, el cosmos absoluto no tiene límites; definir la magnitud y la naturaleza de esta realidad primordial es poner restricciones a la infinitud y atenuar el concepto puro de la eternidad. La idea de lo infinito\hyp{}eterno, lo eterno\hyp{}infinito, es incondicionada en su magnitud y absoluta de hecho. En Urantia, no hay, ni en el pasado ni en el presente ni en el futuro, términos adecuados para expresar la realidad de la infinitud o la infinitud de la realidad. El hombre, una criatura finita en un cosmos infinito, debe contentarse con reflexiones distorsionadas y concepciones atenuadas de esa existencia ilimitada, sin limitaciones, sin principio ni fin, que está realmente más allá de su capacidad de comprensión.
\vs p115 3:2 La mente no puede jamás aspirar a aprehender la noción de algún Absoluto sin pretender primero romper la unidad de dicha realidad. La mente unifica las divergencias, pero, en la ausencia misma de estas, la mente no encuentra ninguna base sobre la que fundamentar comprensiblemente sus conceptos.
\vs p115 3:3 Se precisa realizar una segmentación de la estasis primordial de la infinitud antes de cualquier intento humano por comprenderla. Hay una unidad en la infinitud que se expresa en estos escritos como el YO SOY ---el postulado fundamental de la mente creatural---. Pero la criatura nunca entenderá cómo esta unidad se convierte en dualidad y en triunidad y se diversifica, continuando, no obstante siendo una unidad incondicionada. El hombre afronta una dificultad semejante cuando se detiene a contemplar la Deidad Trinitaria indivisa junto con la manifestación personal plural de Dios.
\vs p115 3:4 Es únicamente la distancia del hombre de la infinitud la que ocasiona que este concepto se exprese con una sola palabra. Aunque la infinitud es, por un lado, UNIDAD, por otro, es DIVERSIDAD sin fin ni límite. La infinitud, tal como las inteligencias finitas la aprecian, constituye la mayor paradoja de la filosofía de las criaturas y de la metafísica finita. Aunque la naturaleza espiritual del hombre se eleva, en su experiencia de adoración, hacia el Padre, que es infinito, el concepto máximo abarcable del Ser Supremo agota la capacidad de comprensión. Más allá de Supremo, los conceptos son crecientemente nombres; son cada vez menos las verdaderas designaciones de la realidad; cada vez más se convierten en la proyección de la comprensión finita de la criatura hacia lo suprafinito.
\vs p115 3:5 \pc La concepción básica del nivel absoluto entraña enunciar un postulado de tres fases, que incluyen:
\vs p115 3:6 \li{1.}\bibemph{Lo Primigenio}. El concepto incondicionado de la Primera Fuente y Centro, esa manifestación del YO SOY que es fuente y origen de toda la realidad.
\vs p115 3:7 \li{2.}\bibemph{Lo Actual}. La unión de los tres Absolutos de actualidad: las Fuentes y Centros Segunda, Tercera y Paradisíaca. Esta triodidad del Hijo Eterno, el Espíritu Infinito y la Isla del Paraíso constituye la revelación en actualidad de la primigeneidad de la Primera Fuente y Centro.
\vs p115 3:8 \li{3.}\bibemph{Lo Potencial}. La unión de los tres Absolutos de potencialidad: el Absoluto de la Deidad, el Absoluto Indeterminado y el Absoluto Universal. Esta triodidad de potencialidad existencial constituye la revelación potencial de la primigeneidad de la Primera Fuente y Centro.
\vs p115 3:9 \pc La correlación de lo Primigenio, lo Actual y lo Potencial produce tensiones dentro de la infinitud que dan como resultado la posibilidad de crecimiento de todo el universo; y el crecimiento es la naturaleza del Séptuplo, el Supremo y el Último.
\vs p115 3:10 En la vinculación de los Absolutos de la Deidad, Universal e Indeterminado, la potencialidad es absoluta mientras que la actualidad es emergente; en la vinculación de las Fuentes y Centros Segunda, Tercera y Paradisíaca, la actualidad es absoluta mientras que la potencialidad es emergente; en la primigeneidad de la Primera Fuente y Centro, no podemos decir que la actualidad o la potencialidad sean existentes o emergentes ---\bibemph{el Padre es}---.
\vs p115 3:11 Desde la perspectiva del tiempo, lo Actual es aquello que fue y es; lo Potencial es lo que llegará a ser y será; lo Primigenio es lo que es. Desde la perspectiva de la eternidad, las diferencias entre lo Primigenio, lo Actual y lo Potencial no son tan patentes. Estas cualidades trinas no se distinguen, por tanto, en los niveles Paraíso\hyp{}eternidad. En la eternidad todo es ---únicamente que aún no todo se ha revelado en el tiempo y en el espacio---.
\vs p115 3:12 Desde la perspectiva de la criatura, la actualidad es sustancia; la potencialidad, capacidad. La actualidad existe en el centro y se expande desde ahí hacia la infinitud periférica; la potencialidad viene en dirección al interior desde la periferia de la infinitud y converge en el centro de todas las cosas. La primigeneidad es aquello que primeramente causa y luego equilibra los movimientos dobles del ciclo de la metamorfosis de la realidad desde los potenciales hasta los actuales y la potencialización de los actuales existentes.
\vs p115 3:13 Los tres Absolutos de la potencialidad son operativos en un nivel puramente eterno del cosmos, de ahí que, como tales, no desempeñen su labor en niveles subabsolutos. En los niveles descendentes de la realidad, la triodidad de potencialidad se manifiesta con el Último y sobre el Supremo. Lo potencial puede no lograr actualizarse en el tiempo con respecto a una parte de algún nivel subabsoluto, pero esto no ocurre nunca en cuanto a la colectividad. La voluntad de Dios, en última instancia, no siempre prima en relación a lo individual, aunque lo hace indefectiblemente en cuanto a la totalidad.
\vs p115 3:14 Es en la triodidad de actualidad donde los existentes del cosmos tienen su centro; sea este espíritu, mente o energía, todos convergen en esta conjunción del Hijo, el Espíritu y el Paraíso. El ser personal del Hijo, que es espíritu, es el modelo matriz para cualquier ser personal a lo largo de todos los universos. La sustancia de la Isla del Paraíso es el modelo matriz del que Havona es una revelación perfecta y, los suprauniversos, una revelación en perfección. El Actor Conjunto es, a la vez, la activación mental de la energía cósmica, la conceptualización del propósito espiritual y la integración de las causas y efectos matemáticos de los niveles materiales con los propósitos y motivos volitivos del nivel espiritual. En un universo finito y, en relación a él, el Hijo, el Espíritu y el Paraíso obran en el Último y sobre él, tal como este se condiciona y limita en el Supremo.
\vs p115 3:15 La actualidad (de la Deidad) constituye lo que el hombre busca al ascender al Paraíso. La potencialidad (de la divinidad humana) constituye la evolución del hombre en esa búsqueda. Lo Primigenio es lo que posibilita la coexistencia y la integración del hombre en actualidad, del hombre en potencialidad y del hombre en eternidad.
\vs p115 3:16 \pc La dinámica final del cosmos está relacionada con la constante transferencia de la realidad desde la potencialidad a la actualidad. En teoría, esta metamorfosis puede tener fin, pero, en la práctica, resulta imposible ya que tanto lo Potencial como lo Actual están encauzados en lo Primigenio (el YO SOY), y esta identificación imposibilita para siempre establecer un límite en la progresión evolutiva del universo. Todo lo que se identifique con el YO SOY jamás podrá cesar en su progreso puesto que la actualidad de los potenciales del YO SOY es absoluta, y la potencialidad de los actuales del YO SOY también lo es. Los actuales siempre abrirán nuevas vías para que se realicen unos potenciales imposibles hasta ese momento ---cada decisión humana no solamente actualiza una nueva realidad en la experiencia humana, sino que también abre una nueva posibilidad para el crecimiento humano---. El hombre vive en cada niño, y el progresador morontial reside en el hombre maduro conocedor de Dios.
\vs p115 3:17 La estática en el crecimiento jamás puede aparecer en el cosmos total, dado que la base del crecimiento ---los actuales absolutos--- es incondicionada, y que las posibilidades del crecimiento ---los potenciales absolutos--- son ilimitadas. Desde un punto de vista práctico, los filósofos del universo han llegado a la conclusión de que no hay nada que se asemeje a un \bibemph{fin}.
\vs p115 3:18 Desde una perspectiva delimitada, hay, en efecto, muchos finales, muchos acabamientos de la actividad, pero desde un punto de vista más amplio, en un nivel superior del universo, no hay finalizaciones, sino meramente transiciones de una faceta de desarrollo a otra. La cronicidad fundamental del universo matriz se centra en las varias eras del universo: la de Havona, la del suprauniverso y la de los universos exteriores. Pero incluso estas divisiones básicas de las relaciones secuenciales no pueden ser más que puntos de referencias relativos en el interminable sendero a la eternidad.
\vs p115 3:19 La comprensión final de la verdad, la belleza y la bondad del Ser Supremo revelará a las criaturas, en su avance, esas cualidades absonitas de la divinidad última que están más allá de los niveles conceptuales de la verdad, la belleza y la bondad.
\usection{4. FUENTES DE LA REALIDAD SUPREMA}
\vs p115 4:1 Cualquier consideración sobre los \bibemph{orígenes} del Dios Supremo debe empezar con la Trinidad del Paraíso, ya que la Trinidad es la Deidad primigenia mientras que el Supremo es una Deidad derivada. Cualquier examen del \bibemph{crecimiento} del Supremo debe tener en cuenta las triodidades existenciales, ya que engloban toda la actualidad absoluta y toda la potencialidad infinita (en conjunción con la Primera Fuente y Centro). Y el Supremo evolutivo es el punto central, culminante y personalmente volitivo, de la transmutación ---la transformación--- de los potenciales en actuales en y sobre el nivel finito de la existencia. Las dos triodidades, actual y potencial, incluyen la totalidad de las interrelaciones en lo que respecta al crecimiento que se produce en los universos.
\vs p115 4:2 La fuente del Supremo está en la Trinidad del Paraíso ---Deidad eterna, actual e indivisa---. El Supremo es ante todo un espíritu\hyp{}persona, que procede de la Trinidad. Pero el Supremo es, en segundo lugar, una Deidad en crecimiento ---crecimiento evolutivo--- y este crecimiento se deriva de las dos triodidades: la actual y la potencial.
\vs p115 4:3 Si es difícil comprender que las triodidades infinitas puedan obrar en el nivel finito, deteneos a considerar que su propia infinitud debe, en sí misma, contener la potencialidad de lo finito; la infinitud abarca todas las cosas que van desde la existencia finita más modesta y circunscrita hasta las realidades de orden superior e incondicionadamente absolutas.
\vs p115 4:4 No es tan difícil comprender que lo infinito contiene de cierto lo finito, como lo es entender cómo se manifiesta realmente este infinito con respecto a lo finito. Pero los modeladores del pensamiento, que moran en los hombres mortales, son una de las pruebas eternas de que incluso el Dios absoluto (como absoluto) puede ponerse directamente en contacto, como de hecho hace, incluso con las criaturas de voluntad más modestas del universo.
\vs p115 4:5 Las triodidades que engloban de forma colectiva lo actual y lo potencial se manifiestan en el nivel finito en conjunción con el Ser Supremo. El método usado para dicha manifestación es a la vez directo como indirecto: directo en tanto que las relaciones de la triodidad tienen una repercusión directa en el Supremo e, indirecto, en tanto que se derivan del nivel que deviene de lo absonito.
\vs p115 4:6 La realidad Suprema, que es realidad finita total, está en proceso de crecimiento dinámico entre los potenciales incondicionados del espacio exterior y los actuales incondicionados del centro de todas las cosas. El ámbito finito se hace efectivo de este modo mediante la cooperación de instancias absonitas del Paraíso y de seres personales como los creadores supremos del tiempo. El acto de desarrollar las posibilidades condicionadas de los tres grandes Absolutos potenciales es la labor absonita de los arquitectos del universo matriz y de sus colaboradores trascendentales. Cuando estos devenires han alcanzado un determinado punto de desarrollo, dichos creadores supremos emergen del Paraíso para dedicarse a la tarea de los tiempos de traer a los universos evolutivos a una existencia fehaciente.
\vs p115 4:7 El crecimiento de la Supremacía se deriva de las triodidades; el espíritu\hyp{}persona del Supremo, de la Trinidad; pero las prerrogativas sobre la potencia del Todopoderoso se basan en los logros divinos del Dios Séptuplo, mientras que la conjunción de las prerrogativas sobre la potencia del Todopoderoso Supremo con la persona\hyp{}espíritu del Dios Supremo tiene lugar en virtud del ministerio del Actor Conjunto, que otorgó la mente del Supremo para que sirviera de elemento de tal conjunción en esta Deidad evolutiva.
\usection{5. RELACIÓN DEL SUPREMO CON LA TRINIDAD DEL PARAÍSO}
\vs p115 5:1 El Ser Supremo depende absolutamente de la existencia y acción de la Trinidad del Paraíso para conformar la realidad de su naturaleza personal y espiritual. Aunque el crecimiento del Supremo guarde relación con las triodidades, el ser personal espiritual del Dios Supremo depende, y se deriva, de la Trinidad del Paraíso, que por siempre permanece como centro\hyp{}fuente absoluto de la estabilidad perfecta e infinita en torno al cual se despliega progresivamente el crecimiento evolutivo del Supremo.
\vs p115 5:2 La labor de la Trinidad está relacionada con la del Supremo, porque la Trinidad opera en todos los niveles (totales), incluyendo el nivel de actuación de la Supremacía. Pero a medida que la era de Havona da lugar a la era de los suprauniversos, de igual manera, la acción perceptible de la Trinidad, como creadora inmediata, da lugar a los actos creativos de los hijos de las Deidades del Paraíso.
\usection{6. RELACIÓN DEL SUPREMO CON LAS TRIODIDADES}
\vs p115 6:1 La triodidad de actualidad continúa obrando directamente en las épocas posteriores a Havona; la gravedad del Paraíso sujeta firmemente las unidades básicas de la existencia material, la gravedad espiritual del Hijo Eterno opera directamente sobre los valores fundamentales de la existencia espiritual y la gravedad mental del Actor Conjunto abarca infaliblemente todos los contenidos vitales de la existencia intelectual.
\vs p115 6:2 Pero a medida que cada etapa de la actividad creativa prosigue su avance a través del espacio inexplorado, esta actúa y existe cada vez más lejos de la acción directa de las fuerzas creativas y de los seres personales con emplazamiento en el centro ---la Isla del Paraíso absoluta y las Deidades infinitas allí residentes---. Estos sucesivos niveles de la existencia cósmica se hacen, por lo tanto, crecientemente más dependientes de los desarrollos que se producen en el ámbito de las tres potencialidades de los Absolutos de la infinitud.
\vs p115 6:3 El Ser Supremo abarca las posibilidades del ministerio cósmico que al parecer no se manifiestan en el Hijo Eterno, en el Espíritu Infinito o en las realidades no personales de la Isla del Paraíso. Esta afirmación se hace teniendo debidamente en cuenta la absolutidad de estas tres actualidades fundamentales, pero el crecimiento del Supremo no se basa solamente en estas actualidades de la Deidad y del Paraíso, sino que también participa de los desarrollos que se dan en el seno del Absoluto de la Deidad, el Absoluto Universal y el Absoluto Indeterminado.
\vs p115 6:4 \pc El Supremo no solo crece a medida que los creadores y las criaturas de los universos evolutivos logran semejarse a Dios; esta Deidad finita crece también a raíz del dominio de criaturas y creadores sobre las posibilidades finitas del gran universo. La moción del Supremo es doble: se extiende hacia el Paraíso y la Deidad y se mueve hacia los ilimitados Absolutos de lo potencial ampliándose.
\vs p115 6:5 En la presente era del universo, dicha doble moción se revela en los seres personales descendentes y ascendentes del gran universo. Los seres personales creadores supremos y todos sus colaboradores divinos reflejan esta moción hacia fuera y divergente del Supremo, mientras que los peregrinos ascendentes de los siete suprauniversos son indicativos de la tendencia hacia dentro y convergente de la Supremacía.
\vs p115 6:6 La Deidad finita está siempre buscando la correlación doble, hacia dentro, hacia el Paraíso y sus Deidades, y, hacia fuera, hacia la infinitud y sus Absolutos. La poderosa erupción de la divinidad creativa de las Deidades del Paraíso que se hace personal en los hijos creadores y se potencializa en los controladores de la potencia supone la inmensa emersión de la Supremacía de los ámbitos de lo potencial, mientras que la incesante procesión de las criaturas ascendentes del gran universo da testimonio de la poderosa inmersión de la Supremacía en la unidad con la Deidad del Paraíso.
\vs p115 6:7 Los seres humanos han aprendido que, en ocasiones, se puede percibir la moción de lo invisible al observar sus efectos en lo visible; y nosotros, en los universos, hace mucho tiempo que hemos aprendido a detectar los movimientos y tendencias de la Supremacía al observar las repercusiones de tales evoluciones en los seres personales y en los patrones del gran universo.
\vs p115 6:8 Aunque no estamos seguros, creemos que el Supremo, como reflejo finito de la Deidad del Paraíso, está envuelto en un eterno avance hacia el espacio exterior; pero en su delimitación de los tres potenciales Absolutos del espacio exterior, este Ser Supremo está por siempre buscando coherencia con el Paraíso. Y estos movimientos dobles parecen dar explicación a la mayoría de la actividad fundamental que se desarrolla en los universos organizados actualmente.
\usection{7. LA NATURALEZA DEL SUPREMO}
\vs p115 7:1 En la Deidad del Supremo, el Padre\hyp{}YO SOY ha logrado liberarse, prácticamente por completo, de las limitaciones consustanciales a su estatus de infinitud, a la eternidad de su ser y a la absolutidad de su naturaleza. Pero el Dios Supremo se ha liberado de todas las limitaciones existenciales solo habiendo quedado sujeto en su actividad universal a delimitaciones experienciales. Al conseguir la facultad de adquirir experiencias, el Dios finito también queda sujeto a la necesidad de ellas; al alcanzar la liberación de la eternidad, el Todopoderoso se topa con la barrera del tiempo; y el Supremo solo podía conocer el crecimiento y el desarrollo como resultado de la parcialidad de su existencia y la incompletitud de su naturaleza, de la falta de absolutidad de su ser.
\vs p115 7:2 Todo esto debe realizarse conforme al plan del Padre, que ha basado el progreso finito en el esfuerzo, el logro de la perseverante criatura y el desarrollo del ser personal en la fe. De este modo, al ordenar la experiencia\hyp{}evolución del Supremo, el Padre ha posibilitado que las criaturas finitas existan en los universos y, mediante su progreso experiencial, alcancen, en algún momento, la divinidad de la Supremacía.
\vs p115 7:3 \pc Cualquier realidad, incluyendo el Supremo e incluso el Último, es relativa, con excepción de los valores incondicionados de los siete Absolutos. El hecho de la Supremacía se basa en la potencia del Paraíso, en el ser personal del Hijo Eterno y en la acción del Actor Conjunto, pero el crecimiento del Supremo está inmerso en el Absoluto de la Deidad, el Absoluto Indeterminado y el Absoluto Universal. Y esta Deidad en vías de síntesis y unificación ---el Dios Supremo--- es la personificación de la sombra finita proyectada sobre el gran universo por la unidad infinita de la naturaleza inescrutable del Padre del Paraíso, la Primera Fuente y Centro.
\vs p115 7:4 Las triodidades, en la medida en la que operan en el nivel finito, inciden sobre el Supremo, que es donde converge la Deidad y representa la suma cósmica de las delimitaciones finitas de las naturalezas de lo Actual Absoluto y de lo Potencial Absoluto.
\vs p115 7:5 \pc La Trinidad del Paraíso se considera como la inevitabilidad absoluta; los siete espíritus mayores son, según parece, inevitabilidades de la Trinidad; la actualización de la potencia\hyp{}mente\hyp{}espíritu\hyp{}ser personal del Supremo debe ser una inevitabilidad evolutiva.
\vs p115 7:6 El Dios Supremo no parece haber sido inevitable en la infinitud incondicionada, pero parece serlo en todos los niveles de la relatividad. Indispensablemente, él hace converger, sintetiza y abarca la experiencia evolutiva, unificando eficientemente los resultados de este modo de percibir la realidad en su naturaleza en cuanto deidad. Y parece hacer todo esto con el propósito de contribuir a la aparición del \bibemph{devenir inevitable,} de la supraexperiencia y la manifestación suprafinita del Dios el Último.
\vs p115 7:7 No se puede apreciar al Supremo en su plenitud sin tomar en consideración su fuente, actividad y destino: su relación con la Trinidad que le dio origen, con el universo en el que actúa y con la Trinidad Última que constituye su destino inmediato.
\vs p115 7:8 Debido al proceso de suma de la experiencia evolutiva, el Supremo conecta lo finito con lo absonito, así como la mente del Actor Conjunto integra la espiritualidad divina del Hijo personal con las energías inmutables del modelo del Paraíso, y al igual que la presencia del Absoluto Universal unifica la activación de la Deidad con la reactividad del Absoluto Indeterminado. Y esta unidad debe ser una revelación del funcionamiento no detectado de la unidad primigenia de la Primera Causa\hyp{}Padre y Fuente\hyp{}Modelo de todas las cosas y de todos los seres.
\vsetoff
\vs p115 7:9 [Auspiciado por un mensajero poderoso con residencia temporal en Urantia.]
