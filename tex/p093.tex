\upaper{93}{Maquiventa Melquisedec}
\author{Melquisedec}
\vs p093 0:1 Los melquisedecs son generalmente conocidos como hijos de emergencia, porque se dedican a una sorprendente gama de tareas en los mundos de un universo local. Cuando surge algún problema extraordinario o cuando se ha de intentar algo poco común, muy a menudo es un melquisedec quien acepta tal misión. La capacidad de los hijos melquisedecs para actuar en casos de emergencias y en niveles muy divergentes del universo, incluso manifestándose físicamente como personas, es característica de este orden. Solo los portadores de vida comparten en algún grado esta diversidad metamórfica de funciones como seres personales.
\vs p093 0:2 \pc El orden de filiación del universo de los melquisedecs ha sido sumamente activo en Urantia. Un colectivo de doce de ellos sirvió en Urantia en colaboración con los portadores de vida. Posteriormente, otro colectivo de doce melquisedecs se convirtió en los síndicos de vuestro mundo, poco después de la secesión de Caligastia, y continuó ejerciendo su autoridad hasta los tiempos de Adán y Eva. Estos doce melquisedecs volvieron a Urantia tras la transgresión de Adán y Eva, y continuaron a partir de entonces como síndicos planetarios hasta el día en el que Jesús de Nazaret, como Hijo del Hombre, se erigió como príncipe planetario titular de Urantia.
\usection{1. LA ENCARNACIÓN DE MAQUIVENTA}
\vs p093 1:1 La verdad revelada estaba en peligro de desaparecer durante los milenios que siguieron al malogro de la misión adánica en Urantia. Las razas humanas, aunque avanzaban intelectualmente, iban paulatinamente perdiendo terreno en el ámbito espiritual. Alrededor del año 3000 a. C., el concepto de Dios se había vuelto muy confuso en la mente de los hombres.
\vs p093 1:2 Los doce síndicos melquisedecs sabían del inminente ministerio de gracia de Miguel en el planeta, pero no sabían cuándo ocurriría exactamente. En consecuencia, se celebró un solemne consejo y se presentó una solicitud a los altísimos de Edentia para que se adoptara alguna disposición a fin de mantener la luz de la verdad en Urantia. Esta petición se desestimó con la directiva de que: “la gestión de los asuntos del 606 de Satania está plenamente en manos de sus custodios melquisedecs”. Los síndicos hicieron un llamamiento de ayuda al Padre Melquisedec, pero solo recibieron el mensaje de que debían continuar manteniendo la verdad en la forma en la que eligieran “hasta la llegada de un hijo de gracia” que “rescataría los títulos planetarios de pérdida e incertidumbre”.
\vs p093 1:3 Y como consecuencia de haber quedado dependiendo tan enteramente de sus propios recursos, Maquiventa Melquisedec, uno de los doce síndicos planetarios, se ofreció voluntario para hacer lo que tan solo se había realizado seis veces en toda la historia de Nebadón: personificarse en la tierra como hombre temporal del mundo, darse a sí mismo como hijo de emergencia para la realización de un ministerio a escala mundial. Las autoridades de Lugar de Salvación concedieron el permiso para esta aventura y la encarnación de Maquiventa Melquisedec se efectuó cerca de lo que acabaría siendo la ciudad de Salem, en Palestina. Todo el proceso de materialización de este hijo melquisedec se llevó a cabo por los síndicos planetarios con la cooperación de los portadores de vida, algunos de los controladores físicos mayores y otros seres personales celestiales residentes en Urantia.
\usection{2. EL SABIO DE SALEM}
\vs p093 2:1 Maquiventa se dio de gracia a las razas humanas de Urantia mil novecientos setenta y tres años antes del nacimiento de Jesús. Su venida no fue nada espectacular; su materialización no se presenció por ojos humanos. La primera vez que algún mortal lo vio fue ese día memorable en el que entró en la tienda de Amdón, un pastor caldeo de origen sumerio. Y la proclamación de su misión quedó plasmada en la sencilla declaración que hizo a este pastor: “Soy Melquisedec, sacerdote de El Elyón, el Altísimo, el solo y único Dios”.
\vs p093 2:2 Cuando el pastor se había recuperado de su asombro y, tras asediar a este extraño con numerosas preguntas, pidió a Melquisedec que cenara con él, y aquella fue la primera vez en su larga andadura en el universo que Maquiventa tomaba alguna comida material, el alimento que habría de sustentarle a lo largo de sus noventa y cuatro años de vida como ser físico.
\vs p093 2:3 Y aquella noche, mientras conversaban bajo las estrellas, Melquisedec comenzó su misión de revelar la verdad de la realidad de Dios cuando, con un amplio movimiento del brazo, se volvió a Amdón y le dijo: “El Elyón, el Altísimo, es el creador divino de las estrellas del firmamento e incluso de esta misma tierra en la que vivimos, y también el Dios supremo del cielo”.
\vs p093 2:4 \pc En pocos años, Melquisedec había reunido en torno a él a un grupo de pupilos, discípulos y creyentes que formaron el núcleo de la posterior comunidad de Salem. Pronto se le conoció en toda Palestina como el sacerdote de El Elyón, el Altísimo, y como el sabio de Salem. En algunas de las tribus de los alrededores se referían a él como el jeque, o rey, de Salem. Salem era el emplazamiento que tras la desaparición de Melquisedec se convertiría en la ciudad de Jebús, la cual se llamó más tarde Jerusalén.
\vs p093 2:5 \pc En cuanto a su aspecto personal, Melquisedec se parecía a los nativos de los pueblos nodita y sumerio, entonces mezclados; tenía alrededor de un metro ochenta de estatura y poseía una presencia imponente. Hablaba caldeo y media docena de otras lenguas. Vestía de forma parecida a la de los sacerdotes cananeos, con la excepción de que en su pecho llevaba un emblema con tres círculos concéntricos, el símbolo de la Trinidad del Paraíso en Satania. En el trascurso de su ministerio, sus seguidores consideraron esta insignia de los tres círculos concéntricos tan sagrada que jamás se atreverían a utilizarla, por lo que en pocas generaciones quedó pronto en el olvido.
\vs p093 2:6 Aunque Maquiventa vivió tal como lo hacían los hombres de su entorno, nunca se casó, ni podía haber dejado descendencia en la tierra. Su cuerpo físico, aunque similar al de un varón humano, era realmente del tipo de aquellos especialmente creados que utilizaron los cien miembros materializados de la comitiva del príncipe Caligastia, excepto que no llevaba el plasma de vida de ninguna raza humana. Tampoco estaba disponible el árbol de la vida en Urantia. Si Maquiventa hubiese permanecido sobre la tierra por un largo período de tiempo, su mecanismo físico se habría deteriorado de forma paulatina; de todos modos, completó su misión de gracia en noventa y cuatro años, mucho antes de que su cuerpo material hubiese empezado a decaer.
\vs p093 2:7 \pc Este melquisedec encarnado recibió un modelador del pensamiento, que habitó en su persona sobrehumana como supervisor del tiempo y mentor de la carne, adquiriendo así experiencia e introducción práctica a los problemas de Urantia y al método de residir en un hijo encarnado; esto permitió a este espíritu del Padre obrar con tanta valentía en la mente humana de Miguel, el hijo de Dios, que aparecería más tarde en la tierra con la semejanza de un hombre mortal. Y este fue el único modelador del pensamiento que obró en dos mentes en Urantia, pero ambas mentes eran divinas a la vez que humanas.
\vs p093 2:8 Durante su encarnación, Maquiventa se mantuvo en contacto permanente con sus once compañeros del colectivo de custodios planetarios, pero no podía comunicarse con otros órdenes de seres personales celestiales. Aparte de los síndicos melquisedecs, no tuvo más contacto con inteligencias sobrehumanas que el que tendría un ser humano.
\usection{3. LAS ENSEÑANZAS DE MELQUISEDEC}
\vs p093 3:1 En el trascurso de una década, Melquisedec organizó sus escuelas en Salem, basándose en el ancestral sistema que los primeros sacerdotes setitas habían desarrollado en el segundo Edén. Incluso la idea de un régimen de diezmos, que introdujo Abraham, más tarde convertido en seguidor suyo, también provenía de las persistentes tradiciones de los métodos de los antiguos setitas.
\vs p093 3:2 Melquisedec instruyó en el concepto de un solo Dios, de una Deidad universal, pero permitió que se relacionara esta enseñanza con el Padre de la Constelación de Norlatiadec, a quien llamó El Elyón ---el Altísimo---. Melquisedec permaneció prácticamente silencioso en cuanto a la situación de Lucifer y cuestiones relacionadas con Jerusem. Lanaforge, el Soberano del Sistema, tuvo poco que ver con Urantia hasta después de la finalización del ministerio de gracia de Miguel. Para la mayoría de los alumnos de Salem, Edentia era el cielo y, el Altísimo, Dios.
\vs p093 3:3 El símbolo de los tres círculos concéntricos, que Melquisedec adoptó como insignia de su ministerio de gracia, se interpretó por una mayoría de personas como distintivo de los tres reinos: el de los hombres, el de los ángeles y el de Dios. Y se les permitió que continuasen con esa creencia; muy pocos de sus seguidores supieron jamás que esos tres círculos ilustraban la infinitud, la eternidad y la universalidad de la Trinidad del Paraíso, sustentadora y guía divina; incluso Abraham consideró este símbolo más bien como representación de los tres altísimos de Edentia, puesto que se le había instruido que ellos obraban como uno solo. En este sentido enseñó Melquisedec el concepto de la Trinidad simbolizado en su insignia, asociándolo generalmente con los tres gobernantes vorondadecs de la constelación de Norlatiadec.
\vs p093 3:4 Melquisedec no realizó el esfuerzo de presentar a las bases de sus seguidores conocimientos más allá del hecho del gobierno de los altísimos de Edentia: los Dioses de Urantia. Pero a algunos de estos seguidores, sí les enseñó verdades avanzadas, incluyendo la gestión y la organización del universo local, mientras que a su brillante discípulo Nordán el Ceneo y a su grupo de dedicados discípulos les impartió las verdades del suprauniverso e incluso de Havona.
\vs p093 3:5 Los miembros de la familia de Katro, con quienes Melquisedec vivió más de treinta años, conocían muchas de estas verdades superiores y las perpetuaron durante mucho tiempo en su familia, incluso hasta los días de su ilustre descendiente Moisés, que dispuso de una poderosa tradición de los días de Melquisedec, transmitida hasta él por parte de su padre, además por otras fuentes a través de su madre.
\vs p093 3:6 Melquisedec enseñó a sus seguidores todo lo que tenían capacidad para recibir y asimilar. Incluso muchas ideas religiosas modernas sobre el cielo y la tierra, el hombre, Dios y los ángeles no difieren demasiado de estas enseñanzas de Melquisedec. Si bien, este gran maestro lo supeditó todo a la doctrina de un solo Dios, una Deidad universal, un Creador celestial, un Padre divino. Hizo hincapié en esta enseñanza con el propósito de apelar a la adoración del hombre y de preparar el camino para la aparición venidera de Miguel como hijo de este mismo Padre Universal.
\vs p093 3:7 Melquisedec enseñó que, en algún tiempo futuro, otro hijo de Dios vendría en la carne tal como él había venido, pero que nacería de una mujer; y esta es la razón por la que, más tarde, numerosos maestros futuros sostendrían que Jesús era sacerdote, o pastor, “para siempre según el orden de Melquisedec”.
\vs p093 3:8 Y, así, Melquisedec preparó el camino y dispuso el escenario monoteísta sentando esta tendencia a escala mundial en aras del ministerio de gracia de un verdadero hijo del Paraíso del Dios único, a quien él tan vívidamente describió como el Padre de todos, y a quien él presentó a Abraham como el Dios que acepta al hombre bajo la simple condición de la fe personal. Y Miguel, cuando apareció en la tierra, confirmó todo lo que Melquisedec había enseñado sobre el Padre del Paraíso.
\usection{4. LA RELIGIÓN DE SALEM}
\vs p093 4:1 Las ceremonias de adoración de Salem eran muy sencillas. Cualquier persona que firmaba o marcaba en las listas de las tablas de arcilla de la Iglesia de Melquisedec se comprometía a memorizar y a adherirse a las siguientes creencias:
\vs p093 4:2 \li{1.}Creo en El Elyón, el Dios altísimo, el único Padre Universal y Creador de todas las cosas.
\vs p093 4:3 \li{2.}Acepto el pacto de Melquisedec con el Altísimo, por el que se me otorga el favor de Dios por mi fe, y no por sacrificios ni holocaustos.
\vs p093 4:4 \li{3.}Prometo obedecer los siete mandamientos de Melquisedec y dar a conocer a todos los hombres la buena nueva de este pacto con el Altísimo.
\vs p093 4:5 \pc Y ese era todo el credo de la comunidad de Salem. Pero incluso una declaración de fe tan breve y simple resultaba del todo excesiva y demasiado avanzada para los hombres de aquellos días. Sencillamente no podían comprender la idea de lograr el favor divino a cambio de nada ---solo por la fe---. Estaba demasiado arraigada en ellos la creencia de que el hombre había nacido en deuda con los dioses. Habían ofrecido sacrificios y obsequiado a los sacerdotes durante tanto tiempo y con tan extremado fervor como para poder ser capaces de comprender la buena nueva de que la salvación, el favor divino, fuese un don gratis para todos aquellos que quisieran creer en el pacto de Melquisedec. Abraham lo creyó aunque sin entusiasmo, e incluso eso “le fue contado por justicia”.
\vs p093 4:6 \pc Los siete mandamientos promulgados por Melquisedec se basaban en la suprema ley antigua de Dalamatia y se parecían en gran medida a los siete mandamientos impartidos en el primer Edén y en el segundo. Estos mandamientos de la religión de Salem eran:
\vs p093 4:7 \li{1.}No servirás a ningún Dios salvo al Creador Altísimo del cielo y de la tierra.
\vs p093 4:8 \li{2.}No dudarás de que la fe es el único requisito para la salvación eterna.
\vs p093 4:9 \li{3.}No darás falso testimonio.
\vs p093 4:10 \li{4.}No matarás.
\vs p093 4:11 \li{5.}No robarás.
\vs p093 4:12 \li{6.}No cometerás adulterio.
\vs p093 4:13 \li{7.}No mostrarás falta de respeto por tus padres ni por los ancianos.
\vs p093 4:14 \pc Aunque no se permitían sacrificios dentro de la comunidad, Melquisedec sabía lo difícil que era eliminar de repente costumbres establecidas desde hacía tanto tiempo y, por consiguiente, decidió acertadamente ofrecer a estas personas sustituir el antiguo sacrificio de carne y sangre por un sacramento de pan y vino. Está escrito: “Melquisedec, rey de Salem, sacó pan y vino”. Pero ni siquiera este cauteloso cambio tuvo todo el éxito esperado. Las distintas tribus mantenían núcleos poblacionales alternativos en las afueras de Salem en donde ofrecían sacrificios y holocaustos. Incluso Abraham recurrió a esta salvaje práctica tras su victoria sobre Quedorlaómer; sencillamente, no se sentía tranquilo del todo hasta no haber hecho el sacrificio habitual. En realidad, Melquisedec nunca lograría erradicar por completo esta proclividad al sacrificio de las prácticas religiosas de sus seguidores, ni siquiera de Abraham.
\vs p093 4:15 Como Jesús, Melquisedec se ciñó estrictamente al cumplimiento de la misión de su ministerio de gracia. No intentó reformar las costumbres, cambiar los hábitos del mundo, ni siquiera dar a conocer prácticas sanitarias avanzadas o verdades científicas. Llegó para llevar a cabo dos tareas: mantener viva en la tierra la verdad del Dios único y preparar el camino para el venidero ministerio de gracia como humano de un hijo del Paraíso de ese Padre Universal.
\vs p093 4:16 \pc A lo largo de noventa y cuatro años, Melquisedec impartió en Salem una sencilla verdad revelada, y, durante este período, Abraham asistió a la escuela de esta comunidad en tres ocasiones diferentes. Finalmente se convirtió a las enseñanzas de Salem, llegando a ser uno de los más brillante pupilos de Melquisedec y en uno de sus principales defensores.
\usection{5. LA ELECCIÓN DE ABRAHAM}
\vs p093 5:1 Aunque pueda ser un error hablar de “pueblo elegido”, no se trata de una equivocación referirse a Abraham como a alguien elegido. Melquisedec hizo de hecho recaer sobre Abraham la responsabilidad de mantener viva la verdad de un solo Dios a diferencia de la creencia prevalente sobre unas deidades plurales.
\vs p093 5:2 La elección de Palestina como sede de la actuación de Maquiventa se debió en parte al deseo de establecer contacto con una familia humana que personificara unos potenciales de liderazgo. En el momento de la encarnación de Melquisedec, había numerosas familias en la tierra tan bien preparadas para recibir las doctrinas de Salem como lo estaba la de Abraham. Al oeste y al norte, existían familias igualmente dotadas entre los hombres rojos, los hombres amarillos y los descendientes de los anditas. Sin bien, por otro lado, ninguna de estas poblaciones estaba tan favorablemente situada para la posterior aparición en la tierra de Miguel como la costa oriental del Mar Mediterráneo. La misión de Melquisedec en Palestina y la ulterior aparición de Miguel en el seno del pueblo hebreo se debieron en buena parte a la geografía, al hecho de que Palestina estaba situada en un punto central respecto al comercio, a las vías de comunicación y a la civilización del mundo, entonces existentes.
\vs p093 5:3 Durante algún tiempo los síndicos melquisedecs habían estado observando a los ancestros de Abraham, y aguardaban esperanzados el nacimiento en alguna generación de descendientes que se caracterizaran por su inteligencia, iniciativa, sabiduría y honestidad. En todos los sentidos, los hijos de Taré, el padre de Abraham, cumplían estas expectativas. La posibilidad de establecer contacto con los polifacéticos hijos de Taré tuvo bastante que ver con la aparición de Maquiventa en Salem, en lugar de Egipto, China, la India o las tribus del norte.
\vs p093 5:4 Taré y la totalidad de su familia eran conversos poco convencidos de la religión de Salem, predicada en Caldea; supieron de Melquisedec a través de los sermones de Ovid, maestro fenicio que proclamó la doctrina de Salem en Ur. Dejaron Ur con la idea de dirigirse directamente a Salem, pero Nacor, el hermano de Abraham, al no conocer a Melquisedec, era poco entusiasta y los persuadió para que se quedaran en Harán. Y pasó mucho tiempo, tras su llegada a Palestina, antes de que estuviesen dispuestos a destruir \bibemph{todos} los dioses domésticos que habían traído con ellos; tardaron en renunciar a los numerosos dioses de Mesopotamia en favor del Dios único de Salem.
\vs p093 5:5 Pocas semanas después de la muerte de Taré, el padre de Abraham, Melquisedec envió a uno de sus discípulos, Jarán el Hitita, para hacer la siguiente invitación a Abraham y a Nacor: “Venid a Salem, donde oiréis nuestras enseñanzas de la verdad del Creador eterno, y en la progenie iluminada de vosotros, ambos hermanos, será el mundo bendecido”. Si bien, Nacor no había aceptado por completo el evangelio de Melquisedec; se quedó atrás y construyó una poderosa ciudad\hyp{}estado que llevó su nombre; pero Lot, el sobrino de Abraham, decidió ir con su tío a Salem.
\vs p093 5:6 Al llegar a Salem, Abraham y Lot eligieron un lugar seguro entre las colinas, cerca de la ciudad, desde donde poder defenderse de los muchos ataques por sorpresa de los asaltantes del norte. En esta época, los hititas, asirios, filisteos y otros grupos saqueaban constantemente las tribus de Palestina central y meridional. Desde esta fortaleza, Abraham y Lot hacían peregrinajes frecuentes a Salem.
\vs p093 5:7 \pc No mucho después de haberse establecido cerca de Salem, Abraham y Lot viajaron al valle del Nilo para conseguir víveres, ya que Palestina sufría entonces una sequía. Durante su breve residencia en Egipto, Abraham encontró a un pariente lejano en el trono egipcio, y sirvió como comandante de dos expediciones militares de gran éxito para este rey. En la última parte de su estancia en el Nilo, él y su esposa, Sara, vivieron en la corte y, al abandonar Egipto, recibió una parte del botín de sus campañas militares.
\vs p093 5:8 Se precisó de una gran determinación por parte de Abraham para renunciar a los honores de la corte egipcia y retornar a la labor más espiritual auspiciada por Maquiventa. Pero Melquisedec era venerado incluso en Egipto y, cuando se explicó al faraón toda la historia, él instó encarecidamente a Abraham a volver al cumplimiento de sus promesas a favor de la causa de Salem.
\vs p093 5:9 \pc Abraham aspiraba a ser rey y, en el camino de regreso de Egipto, planteó a Lot su plan de someter a todo Canaán y poner a su pueblo bajo el mandato de Salem. Lot se inclinaba más por los negocios; así pues, tras una desavenencia surgida con posterioridad, Lot se dirigió a Sodoma para dedicarse al comercio y a la ganadería. Lot no sentía aprecio ni por la vida militar ni por la de pastor.
\vs p093 5:10 Al volver a Salem con su familia, Abraham comenzó a darle forma a sus proyectos militares. Pronto se le reconoció como gobernante civil del territorio de Salem y había confederado bajo su dirección a siete tribus cercanas. De hecho, Melquisedec tuvo grandes dificultades para refrenar a Abraham, que estaba inflamado con el ardor de salir a congregar a las tribus vecinas, usando la fuerza de las armas para así ponerles más rápidamente en conocimiento de las verdades de Salem.
\vs p093 5:11 Melquisedec mantenía relaciones pacíficas con todas las tribus de los alrededores; no era militarista y jamás ninguno de los ejércitos, en sus desplazamientos de un lado para otro, lo atacó. Estaba totalmente dispuesto a que Abraham elaborara una política de defensa para Salem, tal como la que llegaría a ponerse en práctica después, pero no aprobaba los ambiciosos proyectos de conquista de su pupilo; por lo que se produjo una ruptura amistosa de relaciones, y Abraham se trasladó a Hebrón para establecer allí su capital militar.
\vs p093 5:12 Debido a su estrecha relación con el ilustre Melquisedec, Abraham poseía una gran ventaja sobre los reyezuelos de los alrededores; todos ellos reverenciaban a Melquisedec y temían excesivamente a Abraham. Abraham conocía este temor y tan solo aguardaba la ocasión oportuna para atacar a sus vecinos, y tal pretexto se presentó cuando algunos de estos dirigentes se atrevieron a asaltar la propiedad de su sobrino Lot, que habitaba en Sodoma. Al oír esto, Abraham, a la cabeza de sus siete tribus confederadas, avanzó sobre el enemigo. Su propia guardia personal de trescientos dieciocho hombres dirigió un ejército con más de cuatro mil soldados, que atacó de momento.
\vs p093 5:13 Cuando Melquisedec se enteró de la declaración de guerra de Abraham, salió para disuadirle, pero solo lo alcanzó cuando su antiguo discípulo volvía victorioso de la batalla. Abraham sostenía que el Dios de Salem le había otorgado la victoria sobre sus enemigos y se mostró decidido a entregar una décima parte del botín al erario público de Salem. El noventa por ciento que restaba se lo llevó a Hebrón, su capital.
\vs p093 5:14 Tras esta batalla de Sidim, Abraham se convirtió en el líder de una segunda confederación de once tribus y no solo pagaba el diezmo a Melquisedec sino que se aseguró de que todos los demás poblados de aquellos territorios hicieran lo mismo. Sus relaciones diplomáticas con el rey de Sodoma, junto con el temor que generalmente le tenían, dieron como resultado que el rey de Sodoma y otros se unieran a la confederación militar de Hebrón; Abraham iba bien encaminado para establecer un estado poderoso en Palestina.
\usection{6. EL PACTO DE MELQUISEDEC CON ABRAHAM}
\vs p093 6:1 Abraham tenía previsto conquistar todo Canaán. Su determinación se veía tan solo menoscabada por el hecho de que Melquisedec no quería dar su aprobación a estos planes. Pero Abraham estaba a punto de emprender esta conquista cuando le empezó a preocupar la idea de que no tenía un hijo que le sucediese como gobernante de este reino previsto. Concertó otra reunión con Melquisedec; y fue en el transcurso de esta entrevista cuando el sacerdote de Salem, el hijo visible de Dios, persuadió a Abraham para que renunciara a su plan de conquista material y del gobierno temporal, en favor de un acercamiento espiritual al reino de los cielos.
\vs p093 6:2 Melquisedec explicó a Abraham la inutilidad de enfrentarse a la confederación amorita, pero le dejó igualmente claro que estos atrasados clanes, por sus insensatas prácticas, estaban sin duda cometiendo suicidio, por lo que, en unas pocas generaciones, quedarían tan debilitados que los descendientes de Abraham, cuyo número se incrementaría considerablemente, podrían derrotarlos fácilmente.
\vs p093 6:3 Y Melquisedec hizo un pacto formal con Abraham en Salem. Le dijo a Abraham: “Mira ahora los cielos y cuenta las estrellas si es que las puedes contar; así de numerosa será tu simiente”. Abraham creyó lo que Melquisedec decía, “y eso le fue contado por justicia”. Y entonces Melquisedec le contó a Abraham el relato de la ocupación futura de Canaán por sus descendientes después de su estancia en Egipto.
\vs p093 6:4 \pc Este pacto de Melquisedec con Abraham representa el gran acuerdo alcanzado en Urantia entre la divinidad y la humanidad por el que Dios se compromete a hacerlo \bibemph{todo;} el hombre solo se compromete a \bibemph{creer} en las promesas de Dios y seguir sus instrucciones. Anteriormente, se había creído que la salvación podía únicamente obtenerse por medio de las obras ---sacrificios y ofrendas---; en ese momento, Melquisedec trajo de nuevo a Urantia la buena nueva de que la salvación, el favor de Dios, ha de lograrse por la \bibemph{fe}. Pero este evangelio de sencillez de fe en Dios era demasiado avanzado; los hombres de las tribus semíticas optaron más adelante por volver a los sacrificios más antiguos y a la expiación de los pecados mediante el derramamiento de sangre.
\vs p093 6:5 Al poco tiempo de alcanzarse este pacto, nació Isaac, el hijo de Abraham, conforme a la promesa de Melquisedec. Tras el nacimiento de Isaac, Abraham adoptó una actitud de gran solemnidad hacia su pacto con Melquisedec, desplazándose a Salem para subscribirlo por escrito. Fue en este reconocimiento público y formal del pacto cuando cambió su nombre de Abram a Abraham.
\vs p093 6:6 La mayor parte de los creyentes de Salem habían practicado la circuncisión, aunque Melquisedec nunca la había establecido como obligatoria. Ahora pues, Abraham que siempre se había opuesto a ella, decidió, con tal motivo, formalizar el acto aceptando formalmente este rito como símbolo de la ratificación del pacto de Salem.
\vs p093 6:7 Como consecuencia de esta renuncia pública y sincera de sus ambiciones personales en nombre de los planes más importantes de Melquisedec, tres seres celestiales se le aparecieron en las llanuras de Mamre. Esta aparición sucedió de hecho, al margen de su asociación con las narrativas posteriormente inventadas referentes a la destrucción natural de Sodoma y Gomorra. Y estas leyendas de sucesos acaecidos en aquellos días indican el retraso que sufrían la moral y la ética incluso en un tiempo tan reciente.
\vs p093 6:8 Con la realización de este solemne pacto, se consumó la reconciliación entre Abraham y Melquisedec. Abraham retomó el liderazgo civil y militar de la comunidad de Salem, que en su momento álgido contaba con más de cien mil contribuyentes regulares al diezmo en las listas de la hermandad de Melquisedec. Abraham mejoró notablemente el templo de Salem y proporcionó nuevas tiendas para el conjunto de la escuela. No solo amplió el sistema de diezmo sino que instituyó también numerosos y mejores métodos de llevar a cabo la actividad de esta escuela, aparte de contribuir considerablemente a la más conveniente gestión del área de expansión misionera. También hizo mucho por mejorar los rebaños y reorganizar la producción lechera de Salem. Abraham era un hombre de negocios astuto y eficiente, una persona rica para su época; no era excesivamente piadoso, pero totalmente sincero y creía verdaderamente en Maquiventa Melquisedec.
\usection{7. LOS MISIONEROS DE MELQUISEDEC}
\vs p093 7:1 Durante varios años, Melquisedec continuó instruyendo a sus alumnos y formando a los misioneros, que se adentraron en todas las tribus de los alrededores, especialmente en Egipto, Mesopotamia y Asia Menor. Y, a medida que trascurrían las décadas, estos maestros se alejaron cada vez más de Salem, llevando con ellos el evangelio de Maquiventa sobre la creencia y la fe en Dios.
\vs p093 7:2 Los descendientes de Adánez, concentrados en las orillas del lago de Van, escucharon entusiasmados a los maestros hititas del sistema de culto de Salem. Desde este antiguo centro andita, se enviaron maestros a remotas regiones de Europa y Asia. Los misioneros de Salem penetraron en toda Europa, e incluso en Islas Británicas. Un grupo se dirigió vía Feroe hasta los andonitas de Islandia, mientras que otro atravesó China y llegó hasta los japoneses de las islas orientales. La vida y experiencias de los hombres y mujeres que se aventuraron desde Salem, Mesopotamia y el lago de Van para iluminar a las tribus del hemisferio oriental constituyen un episodio heroico en los anales de la raza humana.
\vs p093 7:3 Pero la tarea era tan enorme y las tribus tan atrasadas que los resultados fueron vagos e indeterminados. De generación a generación se abrazó el evangelio de Salem de forma ocasional. Salvo en Palestina, la idea de un solo Dios nunca logró la adhesión continuada de toda una tribu o raza. Mucho tiempo antes de la llegada de Jesús, las enseñanzas de los primeros misioneros de Salem habían quedado generalmente inmersas en supersticiones y creencias universales más antiguas. Las doctrinas sobre la Gran Madre, el Sol y otros antiguos sistemas de culto habían absorbido casi por completo al evangelio original de Melquisedec.
\vs p093 7:4 \pc Vosotros que disfrutáis hoy en día de las ventajas de la tipografía poco podéis comprender lo difícil que era perpetuar la verdad durante estas tempranas épocas; lo fácil que era olvidarse de una nueva doctrina con el paso de las generaciones. Siempre primaba la tendencia a que esta quedase absorbida en el conjunto más antiguo de enseñanzas religiosas y prácticas mágicas. Una nueva revelación siempre se contamina con las creencias evolutivas más antiguas.
\usection{8. LA PARTIDA DE MELQUISEDEC}
\vs p093 8:1 Poco después de la destrucción de Sodoma y Gomorra, Maquiventa decidió dar por terminado su ministerio de gracia de carácter urgente en Urantia. Hubo numerosos factores que influyeron en esta toma de decisión de terminar su estancia en la carne; la más importante de ellas fue la creciente tendencia entre las tribus de los alrededores, e incluso entre sus allegados inmediatos, de considerarlo como un semidiós, de verle como un ser sobrenatural, algo que de hecho era; pero estaban empezando a venerarlo excesivamente y con un temor sumamente supersticioso. Además de estas razones, Melquisedec quería dejar el escenario de sus actividades terrenales con suficiente antelación a la muerte de Abraham como para asegurarse de que la verdad de un solo y único Dios se asentase firmemente en las mentes de sus seguidores. En consecuencia, Maquiventa se retiró una noche a su tienda de Salem tras haber deseado las buenas noches a sus compañeros humanos y, cuando estos fueron a llamarlo por la mañana, ya no estaba allí, porque semejantes suyos se lo habían llevado.
\usection{9. TRAS LA PARTIDA DE MELQUISEDEC}
\vs p093 9:1 La desaparición tan repentina de Melquisedec fue una dura prueba para Abraham. Aunque él había advertido convenientemente a sus seguidores de que alguna vez se iría tal como había venido, no se resignaban a la pérdida de su magnífico líder. La gran comunidad creada en Salem casi llegó a desaparecer, aunque las tradiciones de estos días sirvieron a Moisés de base cuando guió a los hebreos esclavos fuera de Egipto.
\vs p093 9:2 \pc La pérdida de Melquisedec produjo en el corazón de Abraham una tristeza que nunca llegó a superar del todo. Había abandonado Hebrón cuando renunció a su ambición de construir un reino material; y ahora, con la pérdida de su acompañante en la construcción del reino espiritual, salió de Salem y se dirigió al sur para vivir en Gerar, más próximo a asuntos de su interés.
\vs p093 9:3 Abraham se volvió temeroso e inseguro justo después de la desaparición de Melquisedec. A su llegada a Gerar mantuvo en reserva su identidad, por lo que Abimelec se apropió de su esposa. (Poco después de su matrimonio con Sara, una noche Abraham se había enterado de un complot para asesinarlo y arrebatarle a su brillante esposa. El temor que sentía este líder, por otra parte valiente y audaz, se volvió muy intenso; toda su vida sintió miedo de que alguien lo matara a escondidas para llevarse a Sara. Esto explica por qué, en tres ocasiones distintas, este hombre valeroso dio muestras de verdadera cobardía).
\vs p093 9:4 Pero Abraham no desistió por mucho tiempo de su misión como sucesor de Melquisedec. Pronto hizo muchas conversiones entre los filisteos y el pueblo de Abimelec, hizo tratados con ellos y, a su vez, se contaminó de muchas de sus supersticiones, en particular de la costumbre de sacrificar a los primogénitos. De ese modo, Abraham se convirtió de nuevo en un gran líder de Palestina. Todos los grupos le tenían respeto y los reyes lo honraban. Era el gran líder espiritual de todas las tribus de los alrededores, y su influencia continuó por algún tiempo tras su muerte. Durante los últimos años de su vida, regresó a Hebrón, el escenario de sus primeras actividades y el lugar en el que había trabajado en colaboración con Melquisedec. El último acto de Abraham fue enviar a siervos suyos de confianza a la ciudad de su hermano Nacor, en la frontera de Mesopotamia, para buscar a una mujer de su propio pueblo como esposa para su hijo Isaac. Desde hacía mucho tiempo, entre la gente de Abraham, existía la costumbre de casarse con sus primos. Y Abraham murió confiando en esa fe en Dios que había aprendido de Melquisedec en las desaparecidas escuelas de Salem.
\vs p093 9:5 \pc A la siguiente generación le resultó difícil comprender la historia de Melquisedec; en el plazo de quinientos años, hubo muchos que consideraron este relato como un mito. Isaac conservó bastante bien las enseñanzas de su padre y fomentó el evangelio de la comunidad de Salem, pero Jacob tuvo más dificultades para entender el significado de estas tradiciones. José era un gran creyente en Melquisedec y, en gran parte a causa de esto, sus hermanos lo veían como un soñador. Los honores que se le rindieron a José en Egipto se debieron principalmente a la memoria de su bisabuelo Abraham. A José se le ofreció el mando militar de los ejércitos egipcios, pero, al creer tan firmemente en las tradiciones de Melquisedec y en las enseñanzas que Abraham e Isaac impartieron después, optó por servir como gobernador civil, estimando que haría así una mejor labor por el avance del reino de los cielos.
\vs p093 9:6 Las enseñanzas de Melquisedec eran completas y plenas, si bien, a los sacerdotes hebreos posteriores las crónicas de esos días les parecieron imposibles de creer y fantásticas; no obstante, muchos de ellos tuvieron cierta comprensión de estos hechos, al menos hasta los tiempos en los que se produjo en Babilonia la revisión masiva del Antiguo Testamento.
\vs p093 9:7 Los escritos del Antiguo Testamento describen como conversaciones entre Abraham y Dios lo que en realidad fueron reuniones mantenidas entre Abraham y Melquisedec. Con posterioridad, los escribas consideraron el término “Melquisedec” como sinónimo de Dios. En la crónica de los muchos encuentros habidos entre Abraham y Sara y “el ángel del Señor” se hace referencia a sus numerosas charlas con Melquisedec.
\vs p093 9:8 Las narrativas hebreas sobre Isaac, Jacob y José son mucho más fidedignas que las referidas a Abraham, aunque también contengan muchas distorsiones de los hechos; son alteraciones intencionadas y no intencionadas realizadas en el momento de la compilación de estas crónicas por parte de los sacerdotes hebreos durante el cautiverio de Babilonia. Cetura no era esposa de Abraham; como Agar, era simplemente una concubina. Isaac, el hijo de Sara, su esposa legal, heredó todas las propiedades de Abraham. Por otro lado, Abraham no era de tan avanzada edad como se ha hecho constar, y su esposa era mucho más joven. Sus edades se modificaron de forma deliberada para justificar el posterior y supuesto nacimiento milagroso de Isaac.
\vs p093 9:9 \pc El ego nacional de los judíos estaba tremendamente abatido por el cautiverio de Babilonia. En reacción contra la inferioridad nacional se fueron al otro extremo de egocentrismo nacional y racial, distorsionando y desvirtuando sus tradiciones con miras a exaltarse a sí mismos por encima de todas las razas como pueblo elegido de Dios; y, así pues, corrigieron cuidadosamente todas sus crónicas con el fin de exaltar a Abraham y a sus otros líderes nacionales sobre todas las demás personas, sin exceptuar ni siquiera al mismo Melquisedec. Por lo tanto, los escribas hebreos destruyeron toda constancia que pudieron encontrar de estos tiempos memorables, preservando tan solo el relato del encuentro de Abraham y Melquisedec tras la batalla de Sidim, que, según pensaban, confería a Abraham un gran honor.
\vs p093 9:10 Y, por consiguiente, al pasar por alto a Melquisedec, también dejaron de lado las enseñanzas de este hijo de emergencia en lo que concierne a la misión espiritual del hijo de gracia prometido; ignoraron tan absolutamente la naturaleza de esta misión que muy pocos de sus descendientes tuvieron la capacidad o la voluntad de reconocer y recibir a Miguel, cuando este apareció en la tierra y en la carne tal como Maquiventa había predicho.
\vs p093 9:11 Pero uno de los escritores del Libro de los Hebreos comprendió la misión de Melquisedec, porque está escrito: “Este Melquisedec, sacerdote del Altísimo, fue también un rey de paz; sin padre ni madre ni genealogía, ni tuvo principio de días ni fin de vida, sino que fue hecho a semejanza del Hijo de Dios, y permanece sacerdote para siempre”. Este escritor concibió a Melquisedec como de la misma clase que Miguel en su posterior ministerio de gracia cuando afirmó que Jesús era “sacerdote para siempre, según el orden de Melquisedec”. Aunque esta comparación no fue del todo afortunada, era literalmente verdad que Cristo realmente recibió la titularidad provisional de Urantia “por orden de los doce síndicos melquisedecs”, que prestaban servicio en el momento de su ministerio de gracia en el mundo.
\usection{10. ESTADO ACTUAL DE MAQUIVENTA MELQUISEDEC}
\vs p093 10:1 Durante los años de la encarnación de Maquiventa, los síndicos melquisedecs que actuaban en Urantia eran once. Cuando Maquiventa consideró que su misión como hijo de emergencia había terminado, indicó esta circunstancia a sus once compañeros, y estos de inmediato dispusieron el medio por el que sería liberado de la carne y restablecido de forma segura a su originario estado como melquisedec. Y, al tercer día de haber desaparecido de Salem, apareció entre sus once semejantes asignados a Urantia, retomando su andadura interrumpida como síndico planetario del planeta 606 de Satania.
\vs p093 10:2 Maquiventa terminó su ministerio de gracia como criatura de carne y hueso de forma tan repentina y exenta de formalismos como lo había comenzado. Ni su aparición ni su partida estuvieron acompañadas de anuncios o manifestaciones extraordinarias; su aparición en Urantia no se caracterizó ni por un llamamiento nominal a la resurrección ni por el final de una dispensación planetaria; su ministerio de gracia tuvo carácter de urgencia. Si bien, Maquiventa no puso fin a su encarnación como ser humano hasta que el Padre Melquisedec no lo liberó de sus obligaciones y se le informó de que el cumplimiento de este ministerio había recibido la aprobación del mandatario en jefe de Nebadón, Gabriel de Lugar de Salvación.
\vs p093 10:3 \pc Maquiventa Melquisedec siguió mostrándose muy interesado en los asuntos de los descendientes de aquellos hombres que habían creído en sus enseñanzas cuando estaba en la carne. Pero la progenie de Abraham a través de Isaac al entremezclarse con los ceneos fue el único linaje que continuó durante mucho tiempo propiciando un concepto claro de las enseñanzas de Salem.
\vs p093 10:4 A lo largo de los diecinueve siglos siguientes, este melquisedec continuó colaborando con numerosos profetas y videntes, esforzándose así por mantener vivas las verdades de Salem hasta el tiempo debido en el que Miguel hiciera su aparición en la tierra.
\vs p093 10:5 Maquiventa continuó como síndico planetario hasta los tiempos del triunfo de Miguel en Urantia. Con posterioridad, se le adscribió al servicio de Urantia en Jerusem como uno de sus veinticuatro directores y, apenas recientemente, se le elevó a la posición de embajador personal en Jerusem del hijo creador, con el título de Vicerregente del Príncipe Planetario de Urantia. Creemos que mientras Urantia siga siendo un planeta habitado, Maquiventa Melquisedec no volverá a desempeñar plenamente los deberes de su orden de filiación, sino que permanecerá por siempre, hablando en términos temporales, ejerciendo su ministerio planetario como representante de Cristo Miguel.
\vs p093 10:6 Como su ministerio de gracia tuvo ese carácter de urgencia, no hay constancia de cuál será el futuro de Maquiventa. Es posible que el colectivo de los melquisedecs de Nebadón haya sufrido la pérdida permanente de uno de sus miembros. Algunas resoluciones recientes transmitidas por los altísimos de Edentia, y más adelante confirmadas por los ancianos de días de Uversa, sugieren que este melquisedec de gracia está destinado a reemplazar a Caligastia, el príncipe planetario caído. Si nuestras suposiciones en este sentido son correctas, es del todo posible que Maquiventa Melquisedec aparezca nuevamente en persona en Urantia y, de alguna forma modificada, asuma el papel del príncipe planetario destronado, o bien que aparezca en la tierra en calidad de vicerregente del príncipe planetario representando a Cristo Miguel, que ostenta en la actualidad el título de príncipe planetario de Urantia. Aunque dista mucho de estar claro para nosotros cuál será el destino de Maquiventa, no obstante, los acontecimientos ocurridos tan recientemente indican inequívocamente que, con toda probabilidad, las consideraciones antes expuestas no estén demasiado lejos de la verdad.
\vs p093 10:7 Comprendemos bien cómo Miguel, por su triunfo en Urantia, se convirtió en el sucesor tanto de Caligastia como de Adán; cómo llegó a ser Príncipe Planetario de la Paz y el segundo Adán. Y está ahora en nuestras consideraciones la concesión del título de Vicerregente del Príncipe Planetario de Urantia a este melquisedec. ¿Se erigirá también en Hijo Material Vicerregente de Urantia? O ¿existe la posibilidad de que se produzca un hecho inesperado y sin precedentes como el regreso al planeta en algún momento de Adán y de Eva o de algunos de su progenie como representantes de Miguel con los títulos de vicerregentes del segundo Adán de Urantia?
\vs p093 10:8 Todas estas conjeturas unidas a la certidumbre de las apariciones futuras tanto de los hijos magistrados como de los hijos preceptores de la Trinidad, junto con la promesa explícita del hijo creador de regresar algún día, hacen que Urantia sea un planeta de incertidumbre ante el futuro y una de las esferas más interesantes y fascinantes del universo de Nebadón. Es perfectamente posible que, en algún tiempo venidero, cuando Urantia se esté aproximando a la era de luz y vida, después de que se haya finalmente dictado un dictamen sobre los asuntos de la rebelión de Lucifer y de la secesión de Caligastia, podamos ser testigos de la presencia en Urantia, de manera simultánea, de Maquiventa, Adán, Eva y Cristo Miguel, así como también de un hijo magistrado o incluso de hijos preceptores de la Trinidad.
\vs p093 10:9 Durante mucho tiempo, nuestro orden ha tenido la opinión de que la presencia de Maquiventa en el colectivo de los directores de Urantia de Jerusem, con sus veinticuatro consejeros, es prueba suficiente para justificar la creencia de que está destinado a seguir a los mortales de Urantia a través del plan de progreso y ascenso hasta el colectivo final del Paraíso. Sabemos de igual modo que el destino de Adán y de Eva es acompañar a sus semejantes de la tierra en su aventura al Paraíso cuando Urantia se haya establecido en luz y vida.
\vs p093 10:10 Hace menos de mil años, este mismo Maquiventa Melquisedec, el que una vez fue el sabio de Salem, estuvo presente de forma invisible en Urantia, durante un período de cien años, en calidad de gobernador general residente del planeta; y si continúa el actual sistema de dirección de los asuntos planetarios, deberá regresar con el mismo cargo en algo más de mil años.
\vs p093 10:11 \pc Esta es la historia de Maquiventa Melquisedec, una de las figuras más extraordinarias jamás antes vinculadas a la historia de Urantia y una persona que puede estar destinada a desempeñar un importante papel en los acontecimientos futuros de vuestro mundo irregular e insólito.
\vsetoff
\vs p093 10:12 [Exposición de un melquisedec de Nebadón.]
