\upaper{180}{El discurso de despedida}
\author{Comisión de seres intermedios}
\vs p180 0:1 Una vez que concluyó la última cena y, tras cantar el salmo, los apóstoles pensaron que Jesús querría volver de inmediato al campamento, pero él les pidió que se sentaran. El Maestro dijo:
\vs p180 0:2 “Recordáis bien cuando os envié sin bolsa ni alforja e incluso os aconsejé que no llevarais ropa de más. Y todos os acordaréis de que nada os faltó. Pero ahora vienen tiempos turbulentos sobre vosotros. Nunca más podréis depender de la buena voluntad de las multitudes. En lo sucesivo, el que tenga una bolsa, que la lleve con él. Cuando salgáis al mundo para proclamar este evangelio, velad convenientemente por vuestro mantenimiento. He venido para traer la paz, pero esta no se manifestará en algún tiempo.
\vs p180 0:3 “Ha llegado la hora en la que el Hijo del Hombre sea glorificado, y el Padre sea glorificado en mí. Amigos míos, aún estaré con vosotros un poco más. Pero pronto me buscaréis y no me encontraréis, pues voy a un lugar al que vosotros no podéis venir en este momento. No obstante, cuando hayáis acabado vuestra tarea en la tierra, tal como yo he acabado la mía, vendréis a mí de la misma manera que yo me preparo ya para ir al Padre. En poco tiempo os dejaré, ya no me veréis más en la tierra, pero todos me veréis en la era próxima, cuando ascendáis al reino que mi Padre me ha dado”.
\usection{1. EL NUEVO MANDAMIENTO}
\vs p180 1:1 Tras unos momentos de conversación informal, Jesús se puso de pie y dijo: “Cuando representé para vosotros la parábola en la que os mostraba de qué manera debéis estar dispuestos a serviros unos a otros, dije que deseaba daros un nuevo mandamiento; y eso es lo que haré ahora que estoy a punto de dejaros. Conocéis bien el mandamiento que manda que os améis unos a otros; que ames a tu prójimo como a ti mismo. Pero ni incluso esa fervorosa dedicación por parte de mis hijos llega a satisfacerme plenamente. Quiero que realicéis actos de amor aún más grandes en el reino de la hermandad de los creyentes. Y, por ello, os doy este nuevo mandamiento: que os améis unos a otros como yo os he amado. En esto conocerán todos que sois mis discípulos, si tenéis amor unos por otros.
\vs p180 1:2 “Al daros este nuevo mandamiento, no os impongo un nuevo peso sobre vuestras almas; os traigo, en cambio, un gozo nuevo y hago posible que os sintáis una vez más complacidos, conociendo el disfrute de entregar el cariño de vuestro corazón a vuestros semejantes. A punto estoy de sentir el supremo gozo, a pesar de soportar un gran pesar ante los demás, de daros mi amor a vosotros y a vuestros semejantes.
\vs p180 1:3 “Cuando os pido que os améis unos a otros, tal como yo os he amado, presento ante vosotros la medida suprema del verdadero cariño, porque nadie tiene mayor amor que este, que uno dé su vida por sus amigos. Y vosotros sois mis amigos; seguiréis siendo mis amigos con que estéis dispuestos a hacer lo que yo os he enseñado. Me habéis llamado Maestro, pero yo no os llamo siervos. Si os amáis unos a otros como yo os amo, seréis mis amigos y, por siempre, os hablaré de lo que el Padre me revela.
\vs p180 1:4 “No es que vosotros simplemente me hayáis elegido, sino que también yo os he elegido a vosotros, y os he ordenado para que vayáis al mundo y rindáis vuestro servicio amoroso, fruto del espíritu, a vuestros semejantes, al igual que yo he vivido entre vosotros y os he revelado al Padre. El Padre y yo trabajaremos con vosotros, y vosotros experimentaréis el gozo, en su plenitud divina, si obedecéis mi mandamiento de amaros unos a otros, tal como yo os he amado a vosotros”.
\vs p180 1:5 \pc Si queréis compartir el gozo del Maestro, debéis compartir su amor. Y compartir su amor significa que os habéis unido a él en su servicio. Experimentar amor, de manera semejante, no os libra de las dificultades de este mundo; no crea un mundo nuevo, pero, sin lugar a dudas, hace nuevo al viejo mundo.
\vs p180 1:6 Tened esto presente: lo que Jesús pide de vosotros no es sacrificio, sino lealtad. La noción del sacrificio supone la falta de ese cariño incondicional que hubiera convertido dicho servicio amoroso en supremo gozo. La idea del \bibemph{deber} significa que tenéis mentalidad de sirviente y, con ello, dejáis de lado la gran emoción de prestar vuestro servicio como amigo y para un amigo. El impulso a la amistad trasciende cualquier concepto de deber, y el servicio de un amigo a un amigo nunca puede llamarse sacrificio. El Maestro enseñó a los apóstoles que ellos eran hijos de Dios. Los llama hermanos, y ahora, antes de irse, los llama sus amigos.
\usection{2. LA VID Y LOS PÁMPANOS}
\vs p180 2:1 Entonces, Jesús se puso en pie de nuevo y continuó enseñando a sus apóstoles: “Yo soy la vid verdadera, y mi Padre es el labrador. Yo soy la vid, vosotros sois los pámpanos. Y el Padre solo pide de mí que llevéis mucho fruto. La vid se poda únicamente para aumentar el fruto de sus pámpanos. Todo pámpano que en mí no dé fruto, el Padre lo quitará; y todo aquel que dé fruto, el Padre lo limpiará, para que dé más. Ya vosotros estáis limpios por la palabra que os he hablado, pero debéis continuar estando limpios. Debéis permanecer en mí, y yo en vosotros; el pámpano morirá si se le separa de la vid. Como el pámpano no puede dar fruto a no ser que permanezca en la vid, así tampoco podéis vosotros dar el fruto del servicio amoroso a no ser que permanezcáis en mí. Recordad: yo soy la vid verdadera, vosotros los pámpanos vivos. El que vive en mí y yo en él dará mucho fruto del espíritu y experimentará la felicidad suprema de rendir esta cosecha espiritual. Si mantenéis este vínculo vivo y espiritual conmigo, rendiréis abundante fruto. Si permanecéis en mí y mis palabras viven en vosotros, podréis estar en comunión conmigo sin límite, y entonces mi espíritu vivo se infundirá en vosotros de tal manera que podéis pedir lo que mi espíritu desee, y hacerlo todo en la seguridad de que el Padre concederá nuestra petición. En esto es glorificado el Padre: que la vid tenga muchos pámpanos vivos, y que cada pámpano lleve mucho fruto. Y cuando el mundo vea estos fructíferos pámpanos ---mis amigos que se aman unos a otros, tal como yo los he amado a ellos--- todos sabrán que sois realmente mis discípulos.
\vs p180 2:2 “Como el Padre me ha amado, así también yo os he amado. Vivid en mi amor así como yo vivo en el amor del Padre. Si hacéis tal como yo os he enseñado, permaneceréis en mi amor al igual que yo he guardado la palabra del Padre y permanezco para siempre en su amor”.
\vs p180 2:3 Desde hacía mucho tiempo, los judíos habían enseñado que el Mesías sería “un vástago que retoñará de la vid” de los ancestros de David y, en conmemoración de esta antigua enseñanza, un gran emblema de la uva adosada a su parra decoraba la entrada del templo de Herodes. Todos los apóstoles recordaron estas cosas mientras el Maestro les hablaba aquella noche en el aposento alto.
\vs p180 2:4 Pero fue una gran lástima que se malinterpretaran más tarde los comentarios del Maestro sobre la oración. Habrían surgido menos problemas en cuanto a estas enseñanzas si se hubiese recordado sus palabras con exactitud y luego transcrito verazmente. Pero, por la manera en la que se copiaron, los creyentes acabaron por considerar la oración que se hace en nombre de Jesús como una especie de magia suprema, creyeron que recibirían del Padre todo lo que pidieran. Durante siglos, la fe de muchas almas sinceras continúa quebrándose por este escollo. ¿Cuánto tiempo le llevará al mundo de los creyentes entender que la oración no es un método de conseguir nuestros propósitos, sino, más bien, un procedimiento para aceptar el camino de Dios, para tener la experiencia de aprender a reconocer y dar cumplimiento a la voluntad del Padre? Es totalmente cierto que, cuando vuestra voluntad está verdaderamente en consonancia con la suya, podréis pedir cualquier cosa concebible, que os será otorgada por esa misma unión de voluntades. Y tal unión de voluntades se efectúa por medio de Jesús y a través de él, al igual como la vida de la vid fluye por y a través de los pámpanos vivos.
\vs p180 2:5 Cuando existe este vínculo vivo entre divinidad y humanidad, si el hombre, irreflexiva e ignorantemente, ora por su propia confortabilidad y arrogantes logros, solo puede existir una respuesta divina: los vástagos de los pámpanos vivos han de dar más y más frutos del espíritu. Cuando el pámpano de la vid está vivo, solo puede haber una respuesta a todas las oraciones: llevar cada vez más uvas. De hecho, el pámpano solo existe para llevar, dar uvas, y no puede hacer otra cosa excepto esa. De igual manera, el verdadero creyente solo existe con el propósito de dar los frutos del espíritu: amar a los hombres como Dios lo ha amado a él mismo ---que nos amemos unos a otros, tal como Jesús nos ha amado---.
\vs p180 2:6 Y cuando el Padre pone sobre la vid su mano correctora, lo hace con amor para que los pámpanos puedan llevar mucho fruto. El agricultor sensato tan solo corta los pámpanos muertos e infructíferos.
\vs p180 2:7 A Jesús le resultó muy difícil hacer que incluso sus mismos apóstoles reconocieran que la oración era una necesidad para los creyentes nacidos del espíritu en el reino regido por el espíritu.
\usection{3. LA ENEMISTAD DEL MUNDO}
\vs p180 3:1 Apenas terminaron los once de hacer sus comentarios sobre la charla del Maestro en torno a la vid y los pámpanos, Jesús, indicándoles que deseaba decirles algunas cosas más y sabiendo que su tiempo se acababa, dijo: “Cuando os haya dejado, no os desalentéis por la enemistad del mundo hacia vosotros. No os sintáis abatidos cuando incluso algunos tímidos creyentes se vuelvan contra vosotros y alíen sus esfuerzos con los enemigos del reino. Si el mundo os odia, recordad que antes que a vosotros me odió a mí. Si fuerais de este mundo, entonces el mundo os amaría porque seríais de los suyos, pero porque no lo sois, el mundo se niega a amaros. Vosotros sois de este mundo, pero vuestra vida no debe semejarse a la del mundo. Os he escogido para liberaros de las influencias negativas de este mundo y representar el espíritu de otro mundo incluso en este mundo del que habéis sido elegidos. Pero recordad siempre las palabras que os he hablado: el siervo no es mayor que su amo. Si se atreven a perseguirme a mí, también os perseguirán a vosotros. Si mis palabras ofenden a los incrédulos, también las vuestras ofenderán a los impíos. Y os harán a vosotros todo esto porque no creen en mí ni en Aquel que me envió; padeceréis mucho por mi evangelio; pero cuando soportéis estas tribulaciones, recordad que yo también, antes que vosotros, sufrí en nombre de este evangelio del reino celestial.
\vs p180 3:2 “Muchos de los que os atacarán desconocen la luz del cielo, pero esto no es cierto de algunos que ahora nos persiguen. Si no les hubiéramos enseñado la verdad, podrían hacer muchas cosas reprensibles sin ser condenados, pero ahora, puesto que han conocido la luz y han tenido la osadía de rechazarla, no hay excusa para su actitud. Aquel que me odia a mí, odia a mi Padre. No puede ser de otro modo; la luz que os salvaría al aceptarla, tan solo puede condenaros si la rechazáis intencionadamente. ¿Qué les he hecho yo a estos hombres para que sientan por mí tan terrible odio? Nada, excepto ofrecerles fraternidad en la tierra y salvación en el cielo. Pero, ¿es que no habéis leído lo que dicen las Escrituras: ‘Y me odiaron sin causa’?
\vs p180 3:3 “Si bien, no os dejaré solos en el mundo. Muy pronto, una vez que me haya ido, os enviaré a un ayudante espiritual. Tendréis a alguien que tomará mi lugar entre vosotros, a alguien que continuará enseñando el camino de la verdad, y que también os consolará.
\vs p180 3:4 “Que no se turbe vuestro corazón. Creéis en Dios; continuad creyendo también en mí. Y aunque os tengo que dejar, no estaré lejos de vosotros. Ya os he dicho que en el universo de mi Padre hay muchos lugares de estancia. Si así no fuera, no os habría hablado tantas veces de ellos. Yo volveré a estos mundos de luz, estaciones de tránsito del cielo del Padre a los que vosotros alguna vez ascenderéis. Desde estos lugares he venido a este mundo, y ahora ha llegado el momento en que debo regresar a la obra de mi Padre en las esferas de lo alto.
\vs p180 3:5 “Si yo voy, pues, antes que vosotros al reino celestial de mi Padre, no dudéis de que enviaré a buscaros para que podáis estar conmigo en los lugares ya preparados para los hijos mortales de Dios antes de que este mundo existiera. Aunque debo dejaros, estaré presente con vosotros en espíritu y, con el tiempo, estaréis conmigo en persona cuando hayáis ascendido hasta mí en mi universo, tal como yo estoy a punto de ascender a mi Padre en su universo más grande. Y lo que os he dicho es por siempre verdad, aunque no podáis entenderlo del todo. Yo voy al Padre, y aunque vosotros no podéis seguirme ahora, lo haréis en las eras por venir”.
\vs p180 3:6 Cuando Jesús se sentó, Tomás se puso de pie y dijo: “Maestro, no sabemos adónde vas; por lo que evidentemente no conocemos el camino. Pero te seguiremos esta misma noche si nos lo muestras”.
\vs p180 3:7 Cuando Jesús escuchó a Tomás, contestó: “Tomás, yo soy el camino, la verdad y la vida; nadie va al Padre sino por mí. Todos los que encuentran al Padre, me encuentran a mí primero. Si me conocéis a mí, conocéis el camino al Padre. Y en verdad me conocéis, pues habéis vivido conmigo y ahora me veis”.
\vs p180 3:8 Pero estas enseñanzas eran demasiado profundas para muchos de los apóstoles, en especial para Felipe que, después de comentar unas palabras con Natanael, se levantó y dijo: “Maestro, muéstranos al Padre, y todo lo que has dicho quedará claro”.
\vs p180 3:9 Y cuando Felipe terminó de hablar, Jesús le dijo: “Felipe, tanto tiempo hace que estoy contigo, ¿y a pesar de todo no me conoces aún? De nuevo os digo: quien me ha visto a mí, ha visto al Padre; ¿cómo, pues, dices: “Muéstranos al Padre”? ¿No crees que yo soy en el Padre y el Padre en mí? ¿Es que no os he enseñado que las palabras que os hablo no son mis palabras sino las palabras del Padre? Yo hablo por el Padre y no de mí mismo. Estoy en este mundo para hacer la voluntad del Padre, y así he hecho. Mi Padre mora en mí y obra a través de mí. Creedme cuando os digo que el Padre es en mí, y que yo soy en el Padre, o bien, creedme por causa de la vida que he vivido ---por causa de mi obra---”.
\vs p180 3:10 Al irse el Maestro a un lado para refrescarse con agua, los once se enzarzaron en una animada reflexión sobre estas enseñanzas y, cuando se disponía Pedro a dar una de sus largas charlas, él volvió y les hizo un gesto para que se sentaran.
\usection{4. EL ESPÍRITU AYUDANTE PROMETIDO}
\vs p180 4:1 Jesús continuó sus enseñanzas, añadiendo: “Cuando yo haya ido al Padre, y una vez que él haya aceptado por completo la labor que he hecho por vosotros en la tierra, y después de que haya recibido la soberanía definitiva de mis propios dominios, le diré a mi Padre: habiendo dejado solos a mis hijos en la tierra, he prometido que les enviaría a otro maestro. Y cuando el Padre dé su aprobación, derramaré el espíritu de la verdad sobre toda carne. El espíritu de mi Padre ya está en vuestros corazones, y cuando llegue ese día, también me tendréis a mí con vosotros tal como tenéis ahora al Padre. Este nuevo don es el espíritu de la verdad viva. Al principio, los incrédulos no escucharán las enseñanzas de este espíritu, pero los hijos de la luz lo acogerán con gozo y de todo corazón. Y conoceréis a este espíritu cuando llegue al igual que me habéis conocido a mí, y recibiréis este don en vuestros corazones, y él vivirá con vosotros. Veis, por tanto, que no os dejaré sin ayuda ni guía. No os dejaré huérfanos. Hoy solo puedo estar con vosotros en persona. En los tiempos que vienen estaré con vosotros y con todos los demás hombres que deseen mi presencia, dondequiera que estéis y con cada uno de vosotros al mismo tiempo. ¿No percibís que es preferible que me vaya; que os deje de forma corporal para poder estar con vosotros, mejor y con mayor plenitud, en espíritu?
\vs p180 4:2 “En muy pocas horas, el mundo no volverá a verme; pero vosotros seguiréis conociéndome en vuestros corazones incluso hasta que yo envíe a este nuevo maestro, el espíritu de la verdad. Así como he vivido con vosotros en persona, viviré entonces en vosotros; seré uno con vosotros en vuestra experiencia personal del reino del espíritu. Y cuando esto ocurra, ciertamente conoceréis que yo soy en el Padre, y que, aunque vuestra vida está sustentada por el Padre, a través de mí, yo también soy en vosotros. Yo he amado al Padre y he guardado su palabra; vosotros me habéis amado, y guardaréis mi palabra. Tal como mi Padre me ha dado de su espíritu, así os daré yo del mío. Este espíritu de la verdad que os concederé os guiará y os consolará y terminará por llevaros a toda la verdad.
\vs p180 4:3 “Os digo estas cosas mientras estoy aún con vosotros con el fin de que estéis mejor preparados y podáis soportar las pruebas que se ciernen sobre nosotros. Y cuando llegue ese nuevo día, tanto el Hijo como el Padre harán morada en vosotros. Y estos dones del cielo obrarán por siempre el uno con el otro tal como el Padre y yo hemos actuado en la tierra y ante vuestros mismos ojos como una sola persona, como el Hijo del Hombre. Y este amigo espiritual os recordará todo lo que yo os he enseñado”.
\vs p180 4:4 Cuando el Maestro hizo por un momento una pausa, Judas Alfeo se atrevió a hacer una de las pocas preguntas que él o su hermano hicieran jamás a Jesús en público. Judas dijo: “Maestro, has vivido siempre entre nosotros como un amigo; ¿cómo te conoceremos cuando ya no te manifiestes ante nosotros salvo por este espíritu? Si el mundo no te ve, ¿cómo podremos tener la certeza de que eres tú? ¿Cómo te mostrarás a nosotros?”.
\vs p180 4:5 Jesús los miró a todos, sonrió y dijo: “Pequeños míos, yo me voy, regreso al Padre. Dentro de poco, ya no me veréis como me veis aquí, en carne y hueso. En un breve espacio de tiempo, enviaré mi espíritu, que es como yo, salvo por este cuerpo material. Este nuevo maestro es el espíritu de la verdad que vivirá con cada uno de vosotros, en vuestros corazones, y así todos los hijos de la luz serán como uno solo y se atraerán unos a otros. Y, precisamente de esta manera, mi Padre y yo podremos vivir en las almas de cada uno de vosotros y también en el corazón de todos los demás hombres que nos aman y hacen realidad ese amor en sus experiencias al amarse unos a otros, tal como yo os amo ahora a vosotros”.
\vs p180 4:6 Judas Alfeo no entendió del todo lo que el Maestro le dijo, pero llegó a comprender la promesa de un nuevo maestro y, por la expresión en el rostro de Andrés, se dio cuenta de que su pregunta había recibido una satisfactoria respuesta.
\usection{5. EL ESPÍRITU DE LA VERDAD}
\vs p180 5:1 El \bibemph{espíritu de la verdad} es el nuevo espíritu ayudante que Jesús prometió mandar a los corazones de los creyentes, derramándolo sobre toda carne. Este don divino no es la letra ni la ley de la verdad, tampoco actúa como el patrón o la expresión de la verdad. El nuevo maestro es la \bibemph{convicción de la verdad,} la conciencia y la certeza de los verdaderos significados que se experimentan en los niveles reales espirituales. Y este nuevo maestro es el espíritu de la verdad viva y creciente, de la verdad que se expande, desvela y adapta.
\vs p180 5:2 La verdad divina es una realidad viva que el espíritu percibe. La verdad solo existe en los niveles espirituales elevados, y solo cuando se reconoce la divinidad y se tiene conciencia de la comunión con Dios. Podéis conocer la verdad, podéis vivir la verdad; podéis experimentar el crecimiento de la verdad en vuestras almas y disfrutar de la libertad de percibirla espiritualmente en la mente, pero no podéis confinarla en fórmulas, códigos, credos o patrones intelectuales de la conducta humana. Cuando la verdad divina se formula a la manera humana, esta muere con rapidez. Rescatar la verdad aprisionada, ya fenecida, incluso en el mejor de los casos, solo puede llevarse a cabo cuando se toma conciencia de una forma peculiar de sabiduría intelectualizada y glorificada. La verdad estática es verdad muerta, y únicamente la verdad muerta puede contenerse en una teoría. La verdad viva es dinámica y solo puede existir en la mente humana como experiencia personal.
\vs p180 5:3 La inteligencia surge de una vida material que la presencia de la mente cósmica ilumina. La sabiduría consiste en la conciencia que eleva el conocimiento a nuevos niveles de significados y que se activa mediante el don del asistente de la sabiduría, dotación del universo. La verdad es un valor característico de la realidad espiritual solo experimentada por los seres dotados del espíritu, que actúan, en el seno del universo, en los niveles supramateriales de la conciencia y quienes, al reconocer la verdad, permiten al espíritu que la activó que viva y reine en sus almas.
\vs p180 5:4 El verdadero hijo, en su percepción del universo, busca al espíritu vivo de la verdad en cualquier expresión de sabiduría. Individualmente, el conocedor de Dios eleva constantemente la sabiduría a los niveles de la verdad viva ---o logros divinos---; el alma que no progresa espiritualmente está permanentemente arrastrando a la verdad viva hasta los niveles inertes de la sabiduría y al ámbito del mero enaltecimiento del conocimiento.
\vs p180 5:5 Cuando la regla de oro está desprovista de la percepción suprahumana del espíritu de la verdad, se convierte nada más que en una norma de conducta éticamente elevada. Cuando se interpreta literalmente, la regla de oro puede convertirse en un medio de infligir una gran ofensa a vuestros semejantes. Sin una percepción espiritual de la regla de oro de la sabiduría, podríais tener el razonamiento de que, puesto que deseáis que todos los hombres os hablen con completa franqueza toda la verdad que albergan en sus mentes, deberíais, por lo tanto, expresarles a ellos con completa franqueza cualquier pensamiento que alberguéis en las vuestras. Semejante interpretación de la regla de oro, tan poco espiritual, podría resultar en una indecible infelicidad y en un pesar sin límite.
\vs p180 5:6 Algunas personas perciben e interpretan la regla de oro como una reafirmación puramente intelectual de la fraternidad humana. Otras ven esta expresión de las relaciones humanas como una forma de satisfacer emocionalmente los tiernos sentimientos de la persona humana. Otros mortales reconocen esta misma regla de oro como el rasero con el que medir las relaciones sociales, el patrón de la conducta social. E incluso otros la consideran como el mandato positivo de un gran maestro de la conducta moral, que englobó en este enunciado la más alta noción de obligación moral en cuanto a todas las relaciones fraternales. En la vida de estos seres morales, la regla de oro se convierte en el juicioso centro y circunferencia de toda su filosofía.
\vs p180 5:7 En el reino de la hermandad de los creyentes, amantes de la verdad y conocedores de Dios, esta regla de oro, en sus niveles superiores de interpretación y de entendimiento espiritual, adquiere unas cualidades vivas, que hacen que los hijos mortales de Dios vean en este mandato del Maestro la necesidad de relacionarse con sus semejantes de manera que reciban, en tal contacto con el creyente, el mayor bien posible. Esta es la esencia de la verdadera religión: que améis a vuestro prójimo como a vosotros mismos.
\vs p180 5:8 Pero el mayor entendimiento y la interpretación de mayor verdad de la regla de oro consisten en la toma de conciencia del espíritu de la verdad, que es la realidad, perdurable y viva, de esta proclamación divina. El verdadero significado cósmico de esta regla de las relaciones universales se revela solo en su entendimiento espiritual, en la interpretación de la ley de la conducta que hace el espíritu del Hijo para el espíritu del Padre que mora en el alma del hombre mortal. Y cuando estos mortales a quienes el espíritu guía se dan cuenta del verdadero significado de esta regla de oro, rebosan plenos con la certeza de su ciudadanía en un universo amigable, y sus ideales de la realidad espiritual se satisfacen únicamente cuando aman a sus semejantes tal como Jesús nos amó a todos nosotros, y esa es la realidad del entendimiento del amor de Dios.
\vs p180 5:9 Esta misma filosofía de la flexibilidad viva y de la adaptabilidad cósmica de la verdad divina a las necesidades individuales y a la capacidad de cada hijo de Dios debe reconocerse, antes de que podáis tener la esperanza de entender suficientemente la enseñanza y práctica del Maestro de la no confrontación con el mal. La enseñanza del Maestro es, fundamentalmente, un pronunciamiento espiritual. Incluso las consecuencias materiales de su filosofía no pueden considerarse de utilidad al margen de su relevancia espiritual. El espíritu del mandamiento del Maestro consiste en la no confrontación con todas las reacciones egoístas hacia el universo, sumada a la consecución, dinámica y en incremento, de los niveles de rectitud de los verdaderos valores espirituales: la belleza divina, la bondad infinita y la verdad eterna ---conocer a Dios y ser cada vez más como él---.
\vs p180 5:10 El amor, el altruismo, debe estar sujeto a una interpretación de las relaciones personales y a su continua readaptación conforme a la guía del espíritu de la verdad. Por tanto, en el amor, se debe tener en cuenta el hecho, siempre en continuo cambio y ampliación, del más alto bien cósmico de la persona a la que se ama. Y, entonces, el amor continúa creando esta misma actitud hacia todas las demás personas que puedan posiblemente verse influidas por la relación creciente y viva del amor de un mortal, a quien guía el espíritu, hacia otros ciudadanos del universo. Y toda esta adaptación viva del amor debe llevarse a efecto a la luz tanto del entorno del mal presente como de la meta eterna de la perfección: nuestro destino divino.
\vs p180 5:11 Y así, debemos reconocer con claridad que ni la regla de oro ni la enseñanza de la no resistencia jamás podrán entenderse correctamente si se les percibe como dogmas o preceptos. Tan solo se pueden comprender viviéndolas, descubriendo sus significados a través de la interpretación viva del espíritu de la verdad, que dirige la relación de amor entre los seres humanos.
\vs p180 5:12 Y todo esto pone claramente de manifiesto la diferencia entre la vieja religión y la nueva. La vieja religión enseñaba el sacrificio de uno mismo; la nueva religión solo enseña el olvido de sí, el incremento del desarrollo personal mediante el servicio social en conjunción con la comprensión del universo. La vieja religión se movía por la aprehensión del temor; el nuevo evangelio del reino se rige por la convicción de la verdad, por el espíritu de la verdad eterna y universal. Y ninguna cantidad de piedad ni de lealtad a un credo pueden compensar la ausencia, en la experiencia de vida de los creyentes del reino, de esa amigabilidad espontánea, generosa y sincera que caracteriza a los hijos del Dios vivo que han nacido del espíritu. Ni la tradición ni los sistemas ritualistas de la adoración establecida pueden enmendar la falta de genuina compasión hacia los semejantes.
\usection{6. ES CONVENIENTE QUE ME VAYA}
\vs p180 6:1 Tras las numerosas preguntas que Pedro, Santiago, Juan y Mateo le hicieron al Maestro, él continuó su discurso de despedida, diciendo: “Y os digo esto antes de dejaros con el fin de que estéis preparados para lo que se cierne sobre vosotros y no tengáis graves tropiezos. Las autoridades no se contentarán simplemente con expulsaros de las sinagogas; os advierto que viene la hora cuando cualquiera que os mate pensará que rinde servicio a Dios. Y os harán estas cosas a vosotros y a los que guiéis al reino de los cielos, porque no conocen al Padre. Se han negado a conocerlo a él al negarse a recibirme a mí; y se niegan a recibirme a mí cuando os rechazan a vosotros, siempre que hayáis guardado mi nuevo mandamiento, el de amaros unos a otros como yo os he amado. Os digo estas cosas con antelación para que, cuando llegue vuestra hora, como ha llegado la mía ahora, podáis sentiros fortalecidos con el conocimiento de que yo sabía todo esto, y de que mi espíritu estará con vosotros en todos vuestros sufrimientos por mi causa y por la del evangelio. Por esta razón, os he hablado con tanta claridad desde un principio. Incluso os he avisado que los enemigos de un hombre pueden ser aquellos de su misma casa. Aunque este evangelio del reino siempre traerá una gran paz al alma de cada uno de los creyentes, no la traerá a la tierra hasta que el hombre no esté dispuesto a creer firmemente en mis enseñanzas y establezca, como práctica y propósito principal de su vida mortal, hacer la voluntad del Padre.
\vs p180 6:2 “Ahora que os dejo, viendo que ha llegado la hora en la que estoy a punto de partir hacia el Padre, me sorprende que ninguno de vosotros me haya preguntado: ¿Por qué nos dejas? No obstante, sé que os hacéis estas preguntas en vuestros corazones. Hablaré claramente, de un amigo a otro. Os es muy conveniente que así lo haga. Si no me voy, el nuevo maestro no vendrá a vuestros corazones. Debo despojarme de este cuerpo mortal y ser restablecido en mi lugar en lo alto, antes de poder enviaros a este maestro espiritual para que viva en vuestras almas y lleve a vuestros espíritus a la verdad. Y cuando mi espíritu venga para habitar en vosotros, iluminará la diferencia entre el pecado y la rectitud, y os permitirá juzgar con sabiduría, en vuestros corazones en cuanto a estos.
\vs p180 6:3 “Aún tengo muchas cosas que deciros, pero ahora no las podréis sobrellevar. Aunque, cuando él, el espíritu de la verdad, venga, os guiará finalmente a toda la verdad, conforme paséis por las muchas moradas del universo de mi Padre.
\vs p180 6:4 “Este espíritu no hablará por su propia cuenta, sino que proclamará lo que el Padre ha revelado al Hijo, e incluso os hará saber las cosas que habrán de venir; me glorificará como yo he glorificado a mi Padre. Este espíritu surge de mí, y os revelará mi verdad a vosotros. Todo lo que tiene el Padre en este universo es ahora mío; por eso dije que este nuevo maestro tomará de lo mío y os lo revelará.
\vs p180 6:5 “Dentro de muy poco os dejaré por un breve período de tiempo. Después, cuando me veáis de nuevo, estaré ya de camino al Padre, por lo que, incluso entonces, no me veréis por mucho tiempo.
\vs p180 6:6 Cuando hizo una pausa momentánea, los apóstoles comenzaron a hablar entre ellos: “¿Qué es esto que nos dice? ‘Dentro de muy poco os dejaré’, y ‘cuando me veáis de nuevo no será por mucho tiempo, porque estaré de camino al Padre’. ¿Qué quiere decir con ‘dentro de muy poco’ y ‘no por mucho tiempo’? No entendemos lo que dice”.
\vs p180 6:7 Sabiendo Jesús que se hacían estas preguntas, les dijo: “¿Os preguntáis entre vosotros a qué me refería cuando dije que en poco tiempo ya no estaría con vosotros y que, cuando me vierais de nuevo, estaría de camino al Padre? Os he dicho expresamente que el Hijo del Hombre debe morir, pero que se levantará de nuevo. ¿Es que no podéis comprender el sentido de mis palabras? Primero os lamentaréis, pero luego os alegraréis junto a otros muchos que entenderán estas cosas después de que hayan pasado. De cierto, la mujer cuando da a luz tiene dolor porque ha llegado su hora, pero una vez que ha dado a luz a un niño, ya no se acuerda de su angustia, por el gozo de conocer que haya nacido un hombre en el mundo. Y vosotros también tenéis tristeza por mi partida, pero yo os volveré a ver muy pronto y, entonces, vuestra aflicción se convertirá en gozo, y vendrá a vosotros una nueva revelación de la salvación de Dios, que nadie jamás os quitará. Y todos los mundos serán bendecidos con esta revelación misma de la vida que trae el derribo de la muerte. Hasta ahora habéis pedido en nombre de mi Padre. Después de que me veáis de nuevo, también podréis pedir en mi nombre, y yo os oiré.
\vs p180 6:8 “Aquí abajo os he enseñado con proverbios y os he hablado en parábolas. Así lo hice porque erais solo niños en el espíritu; pero la hora viene cuando os hablaré claramente acerca del Padre y de su reino. Y así lo haré, pues el Padre mismo os ama y desea revelarse con mayor plenitud en vosotros. El hombre mortal no puede ver al Padre espiritual; por eso vine yo al mundo para mostrar al Padre ante vuestros ojos creaturales. Pero cuando lleguéis a ser perfectos en vuestro crecimiento espiritual, veréis entonces al Padre mismo”.
\vs p180 6:9 Cuando los once lo oyeron hablar así, se dijeron entre sí: “Mirad, de cierto nos habla claramente. Sin duda, el Maestro ha salido de Dios. Pero, ¿por qué dice que debe regresar al Padre?”. Y Jesús vio que incluso aún en aquel momento no le comprendían. Estos once hombres no podían desprenderse de las ideas, por tanto tiempo gestadas, sobre el concepto judío del Mesías. Cuanta mayor era su creencia en Jesús como el Mesías, más problemáticas se volvían estas nociones, profundamente arraigadas, del glorioso triunfo material del reino en la tierra.
