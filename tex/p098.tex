\upaper{98}{Las enseñanzas de Melquisedec en occidente}
\author{Melquisedec}
\vs p098 0:1 Las enseñanzas de Melquisedec penetraron en Europa por distintas rutas, pero principalmente lo hicieron a través de Egipto y se incorporaron en la filosofía occidental tras haber sido completamente helenizadas y, a continuación, cristianizadas. Los ideales del mundo occidental eran principalmente socráticos y, con posterioridad, su filosofía religiosa pasó a ser la de Jesús, aunque modificada e influenciada por el contacto con la filosofía y la religión occidentales que allí se desarrollaban, todo lo cual culminó en la Iglesia cristiana.
\vs p098 0:2 Los misioneros de Salem llevaron a cabo su labor durante mucho tiempo en Europa y, paulatinamente, se fueron asimilando en los numerosos sistemas de culto y grupos rituales que iban surgiendo periódicamente. Entre quienes mantuvieron las enseñanzas de Salem en su forma más pura cabe mencionar a los cínicos. Estos predicadores de fe y confianza en Dios continuaban todavía desempeñando su actividad en la Europa romana del siglo I d. C., y se incorporaron más tarde en la nueva religión cristiana que estaba formándose.
\vs p098 0:3 Una buena parte de la doctrina salemita se difundió en Europa a través de los soldados mercenarios judíos que lucharon en tantas contiendas militares occidentales. En los tiempos antiguos, los judíos eran célebres tanto por su valor militar como por sus particularidades teológicas.
\vs p098 0:4 En su esencia, las doctrinas de la filosofía griega, la teología judía y la ética cristiana fueron, fundamentalmente, consecuencias de las tempranas enseñanzas de Melquisedec.
\usection{1. LA RELIGIÓN DE SALEM ENTRE LOS GRIEGOS}
\vs p098 1:1 Los misioneros salemitas podrían haber creado una gran estructura religiosa entre los griegos si no hubiese sido por su estricta interpretación del juramento de ordenación, una promesa, impuesta por Maquiventa que prohibía organizar congregaciones exclusivas para el culto de adoración y que exigía el compromiso de cada uno de los maestros de no actuar jamás como sacerdote, de no recibir tasas por sus servicios religiosos, sino solamente alimentos, vestido y cobijo. Cuando los maestros de Melquisedec se adentraron en la Grecia prehelénica, se encontraron con un pueblo que aún propiciaba las tradiciones de Adánez y las de los tiempos de los anditas, pero estas enseñanzas se habían adulterado sobremanera con las nociones y creencias de la avalancha de esclavos, de inferior condición, que, en número creciente, se habían traído a las orillas griegas. Tal deterioro de las enseñanzas produjo un retorno a un tosco animismo plagado de ritos sangrientos, hasta el punto en el que las clases menos favorecidas convertían en actos ceremoniales la ejecución de los criminales condenados a muerte.
\vs p098 1:2 La influencia previa de los maestros salemitas casi desapareció debido a la conocida como invasión aria venida desde el sur de Europa y del este. Estos invasores helénicos trajeron con ellos ideas antropomórficas de Dios, similares a los que sus semejantes arios habían llevado a la India. Tal contacto dio inicio a la proliferación de la familia griega de dioses y diosas. Esta nueva religión se basaba parcialmente en los sistemas de culto de los incivilizados helenos que habían llegado, aunque también participaba de los mitos de los antiguos habitantes de Grecia.
\vs p098 1:3 En el mundo mediterráneo, los griegos helenos descubrieron que el sistema de culto imperante era el dedicado a la madre e impusieron a estos pueblos su dios hombre, Dyaus\hyp{}Zeus, que ya se había convertido, al igual que Yahvé entre los henoteístas semitas, en cabeza de todo el panteón griego de dioses menores. Y los griegos habrían adquirido finalmente un auténtico monoteísmo a partir del concepto de Zeus si no hubiesen conservado la idea del total dominio del Hado. Un dios de valor último debe ser, en sí mismo, el árbitro del azar y el creador del destino.
\vs p098 1:4 Como consecuencia de estos factores involucrados en la evolución religiosa, pronto se desarrolló una creencia popular en los dioses felices e indiferentes del Monte Olimpo, dioses más humanos que divinos, y dioses que los griegos inteligentes nunca tomaron muy en serio. Estas divinidades, de su propia creación, no eran ni muy queridas ni muy temidas. Los griegos poseían sentimientos patrióticos y raciales hacia Zeus y su familia de semihombres y semidioses, pero apenas los reverenciaban o adoraban.
\vs p098 1:5 Los helenos se dejaron imbuir de tal manera por las doctrinas anticlericales de los primeros maestros salemitas que jamás surgió en Grecia ningún destacado sacerdocio. Incluso la talla de las imágenes de los dioses se convirtió más en una obra de arte que en una forma de adoración.
\vs p098 1:6 Los dioses del Olimpo ejemplifican el antropomorfismo característico del hombre, pero la mitología griega era más estética que ética. La religión griega fue útil por cuanto que describía un universo gobernado por un grupo de deidades. Pero la moralidad griega, la ética y la filosofía avanzaron en poco tiempo mucho más allá de su concepto de dios, y este desequilibrio entre el crecimiento intelectual y el espiritual resultó tan perjudicial para Grecia como lo había sido para la India.
\usection{2. EL PENSAMIENTO FILOSÓFICO GRIEGO}
\vs p098 2:1 Pero una religión poco considerada y superficial no puede perdurar, especialmente si no cuenta con un sacerdocio que impulse sus ceremonias y llene de temor reverencial el corazón de los devotos. La religión del Olimpo no prometía salvación, ni saciaba la sed espiritual de sus creyentes; así pues, estaba condenada a perecer. En el trascurso de un milenio de su creación, había prácticamente desaparecido, y los griegos se quedaron sin una religión nacional; los dioses del Olimpo habían perdido su influencia sobre las mejores mentes.
\vs p098 2:2 Esa era la situación cuando, durante el siglo VI a. C., el Oriente y el Levante experimentaron un renacimiento de la conciencia espiritual y un nuevo despertar hacia el reconocimiento del monoteísmo. Pero el oeste no gozó de este nuevo desarrollo; ni Europa ni el norte de África participaron extensivamente en este renacimiento religioso. Sin embargo, los griegos sí llevaron a cabo un magnífico avance intelectual. Habían comenzado a dominar su miedo y ya no intentaban encontrar en la religión un antídoto contra este, pero tampoco percibían la verdadera religión como una cura para la sed del alma, la zozobra espiritual y la desesperación moral. Buscaban el consuelo del alma en el pensamiento profundo ---en la filosofía y la metafísica---. Pasaron de la reflexión acerca de la autopreservación ---la salvación--- a la realización personal y al conocimiento de sí mismos.
\vs p098 2:3 Mediante un pensamiento riguroso, los griegos trataron de lograr un sentido de la seguridad que les sirviera como substituto de la creencia en la supervivencia, pero fracasaron rotundamente. De entre las clases más altas de los pueblos helénicos, solo los más inteligentes lograron llegar a comprender esta enseñanza distinta; el grueso de la progenie de los esclavos de generaciones anteriores no tenía capacidad alguna para asimilar este nuevo sustituto de la religión.
\vs p098 2:4 \pc Los filósofos despreciaban cualquier forma de adoración, a pesar de que todos ellos se aferraban en parte a creencias que subyacían en la doctrina de Salem como “la Inteligencia del universo”, “la idea de Dios” y “la Gran Fuente”. Los filósofos griegos eran abiertamente monoteístas, en cuanto que reconocían lo divino y lo suprafinito; daban un reconocimiento escaso a toda una constelación de dioses y diosas del Olimpo.
\vs p098 2:5 Los poetas griegos del siglo V y VI, Píndaro en particular, intentaron reformar la religión griega. Elevaron sus ideales, pero eran más artistas que devotos religiosos. No lograron desarrollar la forma de fomentar y conservar los valores supremos.
\vs p098 2:6 Jenófanes impartió la idea de un solo Dios, pero su concepto de la deidad era demasiado panteísta para erigirse como Padre personal del hombre mortal. Anaxágoras fue un mecanicista, aunque sí llegó a reconocer una Causa Primera, una Mente Originaria. Sócrates y sus sucesores, Platón y Aristóteles, enseñaron que la virtud es conocimiento y, la bondad, salud del alma; que es mejor sufrir injusticias que ser culpable de ellas, que es equivocado devolver mal por mal, y que los dioses son sabios y buenos. Sus virtudes cardinales eran la sabiduría, el valor, la templanza y la justicia.
\vs p098 2:7 \pc La evolución de la filosofía religiosa de los pueblos helénico y hebreo ejemplifica dos visiones diferentes sobre la labor de la Iglesia como institución en la formación del progreso cultural. En Palestina, el pensamiento humano estaba tan dominado por el sacerdocio y tan dirigido por las escrituras que la filosofía y la estética yacían hundidas bajo las aguas de la religión y de la moralidad. En Grecia, la falta casi total de sacerdotes y de “escrituras sagradas” dejó a la mente humana libre y sin restricciones, dando lugar a un desarrollo sorprendentemente profundo del pensamiento. Si bien, la religión como experiencia personal no logró mantenerse a la altura de su exploración intelectual sobre la naturaleza y la realidad del cosmos.
\vs p098 2:8 En Grecia, las creencias se subordinaban al pensamiento; en Palestina, el pensamiento estaba sujeto a las creencias. Gran parte de la fortaleza del cristianismo se debe al hecho de haber tomado prestado considerablemente tanto de la moralidad hebrea como del pensamiento griego.
\vs p098 2:9 En Palestina, el dogma religioso llegó a estar tan cristalizado que puso en peligro cualquier crecimiento añadido; en Grecia, el pensamiento humano se volvió tan abstracto que el concepto de Dios quedó envuelto en el vapor nebuloso de la especulación panteísta, no muy distinta a la Infinitud impersonal de los filósofos brahmánicos.
\vs p098 2:10 \pc Pero el hombre común de esos tiempos no podía comprender, ni estaba demasiado interesado, en la filosofía griega de la autorrealización de la Deidad abstracta; deseaba más bien promesas de salvación, aunadas a un Dios personal capaz de escuchar sus oraciones. Desterraron a los filósofos, persiguieron a los seguidores que quedaban del sistema de culto de Salem, ambas doctrinas habían pasado a estar muy mezcladas, y se prepararon para la terrible zambullida de naturaleza lasciva en las locuras de los cultos de misterio que, por aquel entonces, se estaban difundiendo por las tierras mediterráneas. Los misterios eleusinos, una versión griega de la adoración de la fertilidad, crecieron dentro del panteón del Olimpo; floreció el sistema de culto dionisiaco de la naturaleza; el mejor de los sistemas de culto fue el de la fraternidad órfica, cuyas predicaciones morales y promesas de salvación resultaron muy atractivas para muchas personas.
\vs p098 2:11 Toda Grecia se vio involucrada en estos nuevos métodos de conseguir salvación, en estos ceremoniales vibrantes y apasionados. Jamás nación alguna había alcanzado cotas tan altas de filosofía artística en tan breve plazo de tiempo; jamás se había creado un sistema tan avanzado de ética prácticamente sin Deidad y enteramente desprovisto de las promesas de salvación humana. Jamás nación alguna había caído de forma tan rápida, profunda y violenta en tales profundidades de estancamiento intelectual, depravación moral y pobreza espiritual como estos mismos pueblos griegos, cuando se arrojaron al torbellino de locura de los cultos de misterio.
\vs p098 2:12 \pc Las religiones han perdurado bastante tiempo sin un respaldo filosófico, pero pocas filosofías, propiamente dichas, han persistido mucho tiempo sin algún tipo de identificación con una religión. La filosofía es para la religión lo que el concepto es para la acción. Pero el estado ideal humano es aquel en el que la filosofía, la religión y la ciencia se inscriben en una unidad significativa mediante la actuación conjunta de la sabiduría, la fe y la experiencia.
\usection{3. LAS ENSEÑANZAS DE MELQUISEDEC EN ROMA}
\vs p098 3:1 Era natural que la religión de los latinos, tras originarse en las tempranas ceremonias religiosas de adoración de los dioses familiares y llegar hasta la reverencia tribal de Marte, el dios de la guerra, fuese más una celebración de orden político de lo que eran los sistema intelectuales de los griegos y los brahmanes, o las religiones más espirituales de distintos otros pueblos.
\vs p098 3:2 Durante el gran resurgir monoteísta del evangelio de Melquisedec, en el siglo VI a. C., muy pocos misioneros de Salem se adentraron en Italia, y los que lo hicieron fueron incapaces de vencer la influencia del sacerdocio etrusco, que se expandía rápidamente, con su nueva constelación de dioses y templos, todos los cuales tomaron forma en la religión estatal romana. Esta religión de las tribus latinas no era trivial y corruptible como la de los griegos ni tampoco austera y tiránica como la de los hebreos; consistía mayoritariamente en el cumplimiento de simples ceremonias, votos y tabúes.
\vs p098 3:3 En gran medida, la religión romana estaba influida por importantes aportaciones culturales llegadas de Grecia. Finalmente, una gran parte de los dioses del Olimpo se trasladó e integró en el panteón latino. Durante mucho tiempo, los griegos adoraron el fuego del fogón familiar ---Hestia era la diosa virgen del fogón familiar; Vesta, la diosa romana del hogar. Zeus se convirtió en Júpiter; Afrodita, en Venus; y así sucedió con las múltiples deidades del Olimpo---.
\vs p098 3:4 La iniciación religiosa de los jóvenes romanos era motivo de su consagración solemne al servicio del Estado. Los juramentos y la admisión a la categoría de ciudadanos eran en realidad ceremonias religiosas. Los pueblos latinos mantenían templos, altares y santuarios y, en caso de crisis, consultaban a los oráculos. Primeramente, conservaban los huesos de los héroes y, más tarde, los de los santos cristianos.
\vs p098 3:5 Esta ceremonia formal y desapasionada de patriotismo pseudorreligioso estaba condenada al fracaso, al igual que la adoración sumamente intelectual y artística de los griegos se había reducido ante la adoración fervorosa y profundamente emocional de los cultos de misterio. El mayor exponente de estos nefastos sistemas del culto fue la religión de misterio de la secta de la Madre de Dios, que en aquellos días tuvo su sede en Roma, en el sitio exacto de la actual Iglesia de San Pedro.
\vs p098 3:6 \pc El nuevo estado romano conquistaba políticamente, pero fue a su vez conquistado por los sistemas de culto, ritos, misterios y conceptos de dios de Egipto, Grecia y el Levante. Estos sistemas de culto importados siguieron floreciendo por todo el estado romano hasta los tiempos de Augusto, que, únicamente por razones políticas y cívicas, realizó un esfuerzo heroico y con cierto grado de éxito por hacer desaparecer los misterios y revivir la antigua religión política.
\vs p098 3:7 Uno de los sacerdotes de la religión del Estado contó a Augusto sobre los primeros intentos de los maestros de Salem por difundir la doctrina de un solo Dios, una Deidad última, que presidía sobre todos los seres sobrenaturales; y esta idea arraigó con tanta firmeza en la mente del emperador que construyó numerosos templos, los abasteció bien de bellas imágenes, reorganizó el sacerdocio estatal, restableció la religión del Estado y se nombró a sí mismo como el actual sumo sacerdote sobre todos y, como emperador, no vaciló en proclamarse dios supremo.
\vs p098 3:8 Esta nueva religión de adoración a Augusto se extendió y se siguió por la totalidad del imperio a lo largo de su vida, salvo en Palestina, el hogar de los judíos. Y esta época de dioses humanos continuó hasta que el sistema de culto oficial romano llegó a contar con una lista de más de cuarenta deidades humanas autoenaltecidas, que reivindicaban nacimientos milagrosos y otros atributos sobrehumanos.
\vs p098 3:9 \pc El último determinado esfuerzo del colectivo, cada vez más escaso, de creyentes de Salem lo realizó un honesto grupo de predicadores, los cínicos, que exhortaron a los romanos a que abandonaran sus ritos religiosos desenfrenados e insensatos y volvieran a una forma de adoración que incluyera el evangelio de Melquisedec, una vez modificado y contaminado por su contacto con la filosofía de los griegos. Pero, en su conjunto, el pueblo rechazó a los cínicos; prefirieron sumirse en los ritos de misterios, que no solo ofrecían esperanzas de salvación personal, sino que también satisfacían el deseo de distracción, diversión y esparcimiento.
\usection{4. LOS CULTOS DE MISTERIO}
\vs p098 4:1 La mayoría de los integrantes del mundo grecorromano, al haber perdido sus primitivas religiones familiares y estatales, y siendo incapaces o no estando dispuestos a comprender el significado de la filosofía griega, centraron su atención en los cultos de misterio, espectaculares y apasionados, procedentes de Egipto y del Levante. La gente común anhelaba promesas de salvación ---el consuelo religioso para hoy y la seguridad de una esperanza de inmortalidad tras la muerte---.
\vs p098 4:2 Los tres cultos de misterio que se hicieron más populares eran:
\vs p098 4:3 \li{1.}El culto frigio de Cibeles y su hijo Atis.
\vs p098 4:4 \li{2.}El culto egipcio de Osiris y su madre Isis.
\vs p098 4:5 \li{3.}El culto iraní de adoración de Mitras como salvador y redentor de la humanidad pecadora.
\vs p098 4:6 \pc Los misterios frigio y egipcio enseñaban que el hijo divino (Atis y Osiris respectivamente) había experimentado la muerte y había resucitado por poder divino y, además, que todos los que se iniciaban debidamente en el misterio, y que celebraban reverentemente el aniversario de la muerte y resurrección del dios, serían por ello partícipes de su naturaleza divina y de su inmortalidad.
\vs p098 4:7 \pc Las ceremonias frigias eran impresionantes pero denigrantes; sus festivales sangrientos indican cuán degradados y primitivos se habían vuelto estos misterios levantinos. El día más sagrado era el Viernes Negro, el “día de la sangre”, en el que se conmemoraba la muerte autoinfligida de Atis. Después de tres días de celebración del sacrificio y la muerte de Atis, el festival se convertía en alegría en homenaje a su resurrección.
\vs p098 4:8 Los rituales de adoración de Isis y Osiris eran más refinados y espectaculares que los del culto frigio. Este rito egipcio se articulaba en torno a la leyenda del dios Nilo de la antigüedad, un dios que murió y fue resucitado, derivado conceptualmente de la observación de la recurrente interrupción anual en el crecimiento de la vegetación, seguido por la renovación primaveral de las plantas vivas. El frenesí del cumplimiento de estos cultos de misterio y las orgías de sus ceremonias, que conducían a un presunto “entusiasmo” por la conciencia de su divinidad, eran a veces de lo más repulsivas.
\usection{5. EL CULTO DE MITRAS}
\vs p098 5:1 Los misterios frigio y egipcio terminaron por ceder ante el más notable de todos los cultos de misterio: la adoración de Mitras. El sistema de culto mitraico atrajo a un amplio espectro de la condición humana y, gradualmente, reemplazó a sus dos predecesores. El mitraísmo se extendió por el Imperio romano por influjo de las legiones romanas reclutadas en el Levante donde esta religión estaba en auge, ya que llevaban con ellos tal creencia adondequiera que fuesen. Y este nuevo rito religioso supuso una gran mejora respecto a los anteriores cultos de misterio.
\vs p098 5:2 El sistema de culto de Mitras surgió en Irán y persistió por largo tiempo en su territorio patrio a pesar de la militante oposición de los seguidores de Zoroastro. Pero para cuando llegó a Roma, el mitraísmo había mejorado considerablemente por la asimilación de las enseñanzas de Zoroastro. Fue sobre todo a través del sistema de culto mitraico cuando la religión de Zoroastro ejerció su influencia sobre el cristianismo, que aparecería más tarde.
\vs p098 5:3 \pc En el sistema de culto mitraico se presentaba a un dios combativo originado en una gran roca, que realizaba valientes hazañas y que hacía que el agua brotase de las rocas al golpearlas con sus flechas. Se produjo una inundación de la cual había escapado un hombre en una barca especialmente construida y una última cena en la que Mitras festejaba con el dios\hyp{}sol antes de ascender al cielo. Este dios, o Sol Invictus, era una degradación del concepto de la deidad Aura\hyp{}Mazda del zoroastrismo. Mitras se percibía como ese campeón que había sobrevivido a la lucha entre el dios\hyp{}sol y el dios de la oscuridad. Y, en reconocimiento por haber matado al mítico toro sagrado, a Mitras se le hizo en inmortal y se le exaltó al rango de intercesor de la raza humana ante los dioses de lo alto.
\vs p098 5:4 Los partidarios de este sistema de culto adoraban en cuevas y otros lugares secretos, cantaban himnos, susurraban palabras mágicas, comían la carne de los animales sacrificados y bebían su sangre. Tenían cultos de adoración tres veces al día, con ceremonias especiales cada semana en el día del dios\hyp{}sol; la celebración más elaborada de todas se realizaba en el festival anual de Mitras, el veinticinco de diciembre. Se creía que participar del sacramento garantizaba la vida eterna, el paso inmediato, tras la muerte, al seno de Mitras, para quedarse allí dichosos hasta el día del juicio. En tal día, las llaves mitríacas del cielo abrirían las puertas del Paraíso para recibir a los fieles; tras lo cual, todos los no bautizados, vivos o muertos, serían aniquilados cuando Mitras regresara a la tierra. Se enseñaba que, cuando un hombre moría, iba ante Mitras para ser juzgado y que, en el fin del mundo, Mitras haría un llamamiento a todos los muertos de sus tumbas para afrontar el juicio final. Los malvados serían exterminados por el fuego y los justos reinarían con Mitras para siempre.
\vs p098 5:5 Al principio era una religión solo para los hombres, y había siete órdenes diferentes en las que los creyentes se iniciaban de forma sucesiva. Más tarde, las esposas e hijas de estos se admitían a los templos de la Gran Madre, que estaban en las proximidades de los templos mitraicos. El sistema de culto de las mujeres era una mezcla del rito mitraico y de las ceremonias del culto frigio de Cibeles, la madre de Atis.
\usection{6. MITRAÍSMO Y CRISTIANISMO}
\vs p098 6:1 Antes de la llegada de los cultos de misterio y del cristianismo, la religión personal apenas se había desarrollado como institución independiente en las tierras civilizadas de África del norte y de Europa; era más bien una cuestión de la familia, de la ciudad\hyp{}Estado, de la política y del imperio. Los griegos helénicos no desarrollaron jamás un sistema de adoración centralizado; el rito era local; no tenían sacerdocio ni “libros sagrados”. Tal como los romanos, sus instituciones religiosas carecían de una poderosa fuerza impulsora que preservara sus más elevados valores morales y espirituales. Aunque es cierto que la institucionalización de la religión le ha restado, por lo general, calidad espiritual, es también un hecho que ninguna religión hasta ahora ha conseguido sobrevivir sin la ayuda, en mayor o menor grado, de alguna organización institucional.
\vs p098 6:2 Así pues, la religión occidental permaneció estancada hasta los días de los escépticos, los cínicos, los epicúreos y los estoicos, pero muy destacablemente, hasta los tiempos de la gran contienda entre el mitraísmo y el cristianismo, la nueva religión de Pablo.
\vs p098 6:3 \pc Durante el siglo III d. C., las iglesias mitraicas y cristianas eran muy similares tanto en apariencia como en el carácter de sus rituales. La mayoría de los lugares de culto estaban bajo tierra y ambas contenían altares, en cuyos ámbitos, de diversa forma, se representaban los sufrimientos del salvador que había traído salvación a una raza humana marcada por el pecado.
\vs p098 6:4 Al entrar en el templo, la práctica de los devotos mitraicos siempre había sido la de sumergir los dedos en agua bendita. Y puesto que en algunas zonas existían personas que en determinado momento habían pertenecido a ambas religiones, se introdujo esta costumbre en la mayoría de las iglesias cristianas de las inmediaciones de Roma. Ambas religiones hacían uso del bautismo y compartían el sacramento del pan y del vino. La gran diferencia entre el mitraísmo y el cristianismo, aparte de la individualidad de Mitras y Jesús, era que mientras que el primero alentaba el militarismo, el otro era sumamente pacífico. La tolerancia del mitraísmo hacia otras religiones (salvo más tarde hacia el cristianismo) lo llevó a su perdición final. Pero el factor decisivo en la lucha entre las dos religiones fue la admisión de las mujeres en la plena fraternidad de la fe cristiana.
\vs p098 6:5 \pc En definitiva, la llamada fe cristiana, imperó en Occidente. La filosofía griega proporcionó los conceptos de valor ético; el mitraísmo, el ritual de la celebración del culto de adoración; y el cristianismo, propiamente dicho, la forma de preservar los valores morales y sociales.
\usection{7. LA RELIGIÓN CRISTIANA}
\vs p098 7:1 El hijo creador no se encarnó en semejanza de un hombre mortal ni se dio de gracia a la humanidad de Urantia para reconciliarla con un Dios airado, sino más bien para lograr que toda la humanidad reconociera el amor del Padre y tomara conciencia de su filiación con Dios. En efecto, incluso el gran defensor de la doctrina de la expiación se percató de parte de esta verdad, pues declaró que “Dios estaba en Cristo reconciliando al mundo consigo mismo”.
\vs p098 7:2 No es competencia de este escrito tratar del origen y diseminación de la religión cristiana. Baste señalar que se articula en torno a la persona de Jesús de Nazaret, el Hijo Miguel de Nebadón encarnado como humano, conocido en Urantia como Cristo, el ungido. El cristianismo se difundió por el Levante y el Occidente por medio de los seguidores de este galileo, y su celo misionero igualó al de sus ilustres predecesores, los setitas y salemitas, así como al de sus fervientes contemporáneos asiáticos, los maestros budistas.
\vs p098 7:3 La religión cristiana, como sistema de fe urantiano, surgió de la combinación de las siguientes enseñanzas, influencias, creencias, cultos y actitudes personales e individuales:
\vs p098 7:4 \li{1.}Las enseñanzas de Melquisedec, que constituyen un factor esencial en todas las religiones de Occidente y de Oriente surgidas en los últimos cuatro mil años.
\vs p098 7:5 \li{2.}El sistema hebraico de moralidad, ética, teología y creencia tanto en la Providencia como en el Yahvé supremo.
\vs p098 7:6 \li{3.}La concepción zoroástrica de la lucha entre el bien y el mal cósmicos, que ya había dejado su impronta tanto en el judaísmo como en el mitraísmo. Mediante el prolongado contacto que se produjo a raíz de la pugna entre el mitraísmo y el cristianismo, las doctrinas del profeta iraní se convirtieron en un factor importante en el establecimiento de la forma y la estructura teológicas y filosóficas de los dogmas, postulados y cosmología de las versiones helenizada y latinizada de las enseñanzas de Jesús.
\vs p098 7:7 \li{4.}Los cultos de misterio, en particular el mitraísmo, aunque también la adoración a la Gran Madre del sistema de culto frigio. Incluso las leyendas sobre el nacimiento de Jesús en Urantia llegaron a estar contaminadas por la versión romana del nacimiento milagroso del salvador\hyp{}héroe iraní Mitras, cuya llegada a la tierra fue supuestamente presenciada solo por un puñado de pastores cargados de presentes, a quienes los ángeles habían informado de este inminente acontecimiento.
\vs p098 7:8 \li{5.}El hecho histórico de la vida humana de Josué ben José; la existencia de Jesús de Nazaret, el Hijo de Dios, como Cristo glorificado.
\vs p098 7:9 \li{6.}El punto de vista personal de Pablo de Tarso. Y es preciso dejar constancia que el mitraísmo era la religión predominante en Tarso durante su adolescencia. Pablo poco imaginaba que sus bienintencionadas cartas dirigidas a sus conversos algún día se considerarían por los cristianos que vendrían después “como palabra de Dios”. Estos maestros, de buenos propósitos, no se deben considerar responsables del uso que se ha hecho de sus escritos por parte de sus sucesores posteriores.
\vs p098 7:10 \li{7.}El pensamiento filosófico de los pueblos helénicos, desde Alejandría hasta Antioquía y, atravesando Grecia, hasta Siracusa y Roma. La filosofía de los griegos estaba más en armonía con la versión paulina del cristianismo que con cualquier otro sistema religioso imperante, y se convirtió en un factor determinante en el éxito del cristianismo en Occidente. La filosofía griega, junto con la teología de Pablo, todavía constituye la base de la ética europea.
\vs p098 7:11 \pc A medida que las enseñanzas primigenias de Jesús penetraron en Occidente, se occidentalizaron y, a medida que sucedió así, comenzaron a perder su atractivo, universal en potencia, para todas las razas y clases de hombres. Hoy en día, el cristianismo se ha convertido en una religión adaptada a las costumbres sociales, económicas y políticas de las razas blancas. Hace mucho tiempo que dejó de ser la religión de Jesús, a pesar de que aún se describe en ella, valerosamente, una hermosa religión sobre Jesús para aquellos que buscan con honestidad seguir el camino de sus enseñanzas. Ha glorificado a Jesús como el Cristo, el ungido mesiánico de Dios, pero se ha olvidado en gran parte del evangelio personal del Maestro: la Paternidad de Dios y la hermandad universal de todos los hombres.
\vs p098 7:12 \pc Y esta es la larga historia de las enseñanzas de Maquiventa Melquisedec en Urantia. Han pasado casi cuatro mil años desde que este hijo de urgencia de Nebadón se dio de gracia en Urantia y, en ese tiempo, las enseñanzas del “sacerdote de El Elyón, el Dios Altísimo” se han dado a conocer a todas las razas y pueblos. Y Maquiventa consiguió el objetivo de su extraordinario ministerio de gracia; cuando Miguel se dispuso a aparecer en Urantia, el concepto de Dios estaba presente en el corazón de hombres y mujeres, el mismo concepto de Dios que se irradia nuevamente en la experiencia espiritual viva de los numerosos hijos del Padre Universal, conforme viven sus fascinantes vidas temporales en los planetas rotatorios del espacio.
\vsetoff
\vs p098 7:13 [Exposición de un melquisedec de Nebadón.]
