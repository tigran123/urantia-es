\upaper{114}{El gobierno planetario seráfico}
\author{Jefe de los serafines}
\vs p114 0:1 Los altísimos gobiernan en los reinos de los hombres a través de muchas fuerzas e instancias intermedias celestiales pero, primordialmente, mediante el ministerio de los serafines.
\vs p114 0:2 Hoy al mediodía, en el llamamiento nominal efectuado en Urantia a los ángeles planetarios, guardianes y a otros ángeles, había 501\,234\,619 pares de serafines. Asignados a mi mando, había doscientos cuerpos de ejército ---597\,196\,800 pares de serafines o 1\,194\,393\,600 ángeles individuales---. Sin embargo, en el registro consta un total de 1\,002\,469\,238 ángeles tomados de forma individual. Se deduce, pues, que 191\,294\,362 ángeles se encontraban ausentes de este mundo, en cumplimiento de sus deberes de transporte, mensajería y fallecimientos. (En Urantia existe aproximadamente el mismo número de querubines que de serafines, y están organizados de forma similar).
\vs p114 0:3 Los serafines y sus querubines adjuntos tienen mucho que ver con los pormenores del gobierno sobrehumano de un planeta, especialmente de los mundos que quedaron aislados por la rebelión. Los ángeles, hábilmente asistidos por los seres intermedios, actúan en Urantia como verdaderos servidores supramateriales, ejecutando los mandatos del gobernador general residente y de todos sus colaboradores y subalternos. Los serafines como clase se ocupan de muchas tareas al margen de aquellas referidas a la custodia personal y grupal.
\vs p114 0:4 Urantia no carece de una supervisión adecuada y eficaz proveniente del sistema, de la constelación y de los gobernantes del universo. Pero el gobierno planetario es diferente al de cualquier otro mundo del sistema de Satania, e incluso de todo Nebadón. El carácter singular de vuestro plan de supervisión se debe a una serie de inusuales circunstancias:
\vs p114 0:5 \li{1.}El estatus de la modificación de vida realizada en Urantia.
\vs p114 0:6 \li{2.}Las exigencias a las que obliga la rebelión de Lucifer.
\vs p114 0:7 \li{3.}Los trastornos ocasionados por la transgresión adánica.
\vs p114 0:8 \li{4.}Las irregularidades surgidas del hecho de que Urantia fuese uno de los mundos en el que el soberano del Universo se dio de gracia. Miguel de Nebadón es el príncipe planetario del planeta.
\vs p114 0:9 \li{5.}La especial actividad de los veinticuatro directores planetarios.
\vs p114 0:10 \li{6.}El emplazamiento en el planeta de una vía circulatoria de los arcángeles.
\vs p114 0:11 \li{7.}El nombramiento más reciente de Maquiventa Melquisedec, en antaño encarnado, como príncipe planetario vicerregente.
\usection{1. LA SOBERANÍA DE URANTIA}
\vs p114 1:1 Inicialmente, la soberanía de Urantia estaba encomendada al soberano del sistema de Satania. Él la delegó primeramente a una comisión conjunta de melquisedecs y de portadores de vida, y este grupo desempeñó esta función en Urantia hasta la llegada de un príncipe planetario legítimamente nombrado. Tras la caída del príncipe Caligastia, en los tiempos de la rebelión de Lucifer, Urantia no mantuvo una relación segura y estable con el universo local y sus divisiones administrativas hasta la finalización del ministerio de gracia de Miguel en la carne, momento en el que el unión de días lo proclamó príncipe planetario de Urantia. Dicha proclamación estableció para siempre, con garantía y por ley, el estatus de vuestro mundo; si bien, en la práctica, el hijo creador soberano no ha adoptado medidas para administrar personalmente el planeta, aparte de la constitución de la comisión de Jerusem de veinticuatro antiguos urantianos con autoridad para representarlo en el gobierno de Urantia y en todos los otros planetas en cuarentena del sistema. En la actualidad, uno de los miembros de este consejo siempre reside en Urantia en calidad de gobernador general residente.
\vs p114 1:2 Recientemente se ha investido a Maquiventa Melquisedec con autoridad como vicerregente y con el poder para actuar como príncipe planetario en nombre de Miguel, pero este hijo del universo local no ha hecho el más mínimo movimiento para modificar el actual régimen planetario de los sucesivos gobiernos de los gobernadores generales residentes.
\vs p114 1:3 Es poco probable de que se produzca un marcado cambio en el gobierno de Urantia durante la presente dispensación, a menos que el vicerregente del príncipe planetario llegue para asumir las responsabilidades de su cargo nominal. Algunos de nuestros compañeros tienen la impresión de que, en algún momento del futuro próximo, el plan de enviar a uno de los veinticuatro consejeros a Urantia para actuar como gobernador general se sustituirá por la llegada oficial de Maquiventa Melquisedec con el mandato de vicerregente de la soberanía de Urantia. Como príncipe planetario en funciones, él, sin duda, continuaría al frente del planeta hasta el dictamen final sobre la rebelión de Lucifer y, probablemente, hasta el futuro distante en el que el planeta se asiente en luz y vida.
\vs p114 1:4 Algunos creen que Maquiventa no vendrá para tomar la dirección de los asuntos de Urantia de forma personal hasta el fin de la dispensación vigente. Otros mantienen que tal vez el príncipe vicerregente puede que no venga, como tal, hasta que Miguel no regrese en algún momento a Urantia, tal como prometió estando aún en la carne. Y hay quienes, entre los que se incluye este narrador, que aguardan la aparición de Melquisedec en cualquier día u hora.
\usection{2. LA JUNTA DE LOS SUPERVISORES PLANETARIOS}
\vs p114 2:1 Desde los tiempos del ministerio de gracia de Miguel en vuestro mundo, la gestión general de Urantia está encomendada, en Jerusem, a un grupo de veinticuatro antiguos urantianos. Desconocemos los requisitos exigidos para formar parte de esta comisión, pero hemos observado que todos aquellos designados para este puesto contribuyeron al engrandecimiento de la soberanía del Supremo en el sistema de Satania. Por su propia naturaleza, todos eran auténticos líderes cuando actuaron en Urantia, y (exceptuando a Maquiventa Melquisedec) estas cualidades de liderazgo se han incrementado aún más gracias a su experiencia en los mundos de las moradas y se han visto complementadas por la formación recibida durante su ciudadanía en Jerusem. Es el gabinete de Lanaforge quien nombra a los miembros integrantes de los veinticuatro, que son, a su vez, confirmados oficialmente por los altísimos de Edentia, aprobados por el centinela con destino de Jerusem y nombrados por Gabriel de Lugar de Salvación, en conformidad con el mandato de Miguel. Aquellos a los que se les designa de forma temporal desempeñan sus funciones tan plenamente como los miembros permanentes de esta comisión de supervisores especiales.
\vs p114 2:2 Esta junta de directores planetarios se ocupa particularmente de supervisar la actividad de este mundo que se desprende del hecho de que Miguel experimentó aquí su ministerio de gracia final. Mantienen un contacto, estrecho e inmediato, con Miguel por medio de la labor de enlace de determinada brillante estrella vespertina, la misma que acompañó a Jesús durante su ministerio de gracia como mortal.
\vs p114 2:3 En la actualidad cierto Juan, conocido por vosotros como “el Bautista”, ejerce la presidencia de este consejo en los periodos de sesiones que se celebran en Jerusem. Pero el jefe de oficio de este consejo es el centinela con destino de Satania, representante directo y personal del inspector adjunto de Lugar de Salvación y del mandatario supremo de Orvontón.
\vs p114 2:4 Los miembros de esta misma comisión de antiguos urantianos obran asimismo como supervisores con carácter consultivo de los treinta y seis otros mundos del sistema aislados por la rebelión; prestan un servicio muy valioso al mantener a Lanaforge, el soberano del sistema, en contacto, estrecho y comprensivo, con los asuntos de estos planetas, que aún continúan en cierto modo bajo la acción directiva de los padres de la constelación de Norlatiadek. Estos veinticuatro consejeros realizan con frecuencia, de forma individual, viajes a cada uno de los planetas en cuarentena, particularmente a Urantia.
\vs p114 2:5 Cada uno de los demás mundos aislados está asesorado por comisiones similares, compuestas por un número variable de sus antiguos habitantes, pero estas otras comisiones son dependientes del grupo urantiano de los veinticuatro. Aunque los miembros de esta anterior comisión tengan, por tanto, un interés activo por cualquier faceta del progreso humano de los mundos en cuarentena de Satania, se ocupan sobre todo y en particular, por el bienestar y el avance de las razas mortales de Urantia, ya que no supervisan de manera inmediata y directa los asuntos de ningún otro planeta que no sea Urantia e, incluso aquí, su autoridad no es completa, salvo en ciertos ámbitos relacionados con la supervivencia de los mortales.
\vs p114 2:6 Nadie sabe cuánto tiempo continuarán estos veinticuatro consejeros de Urantia en su presente estatus, apartados del programa regular de actividad del universo. No cabe duda de que continuarán sirviendo en sus competencias actuales hasta que se produzca algún cambio en su situación planetaria, tal como el fin de una dispensación, la asunción de la plena autoridad por parte de Maquiventa Melquisedec, el dictamen final acerca de la rebelión de Lucifer o la reaparición de Miguel en el mundo de su último ministerio de gracia. El actual gobernador general residente de Urantia parece ser propenso a opinar que, excepto Maquiventa, quizás todos estén exentos de ascender al Paraíso en el momento en que el sistema de Satania sea restituido a las vías circulatorias de la constelación. Pero hay también otras opiniones vigentes.
\usection{3. EL GOBERNADOR GENERAL RESIDENTE}
\vs p114 3:1 Cada cien años del tiempo de Urantia, el colectivo de Jerusem de veinticuatro supervisores planetarios nombra a uno de sus miembros para que resida en vuestro mundo y actúe como representante suyo con poder ejecutivo, en calidad de gobernador general residente. En los tiempos de la preparación de estas narrativas se sustituyó a este jefe ejecutivo, con lo que el gobernador vigésimo reemplazaba al décimo noveno. Se os oculta el nombre del actual supervisor planetario porque el hombre mortal es proclive a venerar, incluso deificar, a sus compatriotas de dotes extraordinarias y a superiores suyos de naturaleza sobrehumana.
\vs p114 3:2 El gobernador general residente no tiene autoridad personal real en la gestión de los asuntos del mundo excepto como representante de los veinticuatro consejeros de Jerusem. Actúa como coordinador del gobierno sobrehumano y es el jefe respetado y el líder universalmente reconocido de los seres celestiales con funciones en Urantia. Todos los órdenes de las multitudes angélicas lo ven como su director de coordinación, mientras que los seres intermedios unidos, desde la partida de 1\hyp{}2\hyp{}3 el primero para integrarse en el grupo de los veinticuatro consejeros, consideran verdaderamente a los sucesivos gobernadores generales como sus padres planetarios.
\vs p114 3:3 Aunque el gobernador general no posea autoridad real y personal sobre el planeta, cada día expide decenas de resoluciones y decisiones que se aceptan como definitivas por todos los seres personales a quienes estas les conciernen. Es mucho más un consejero paternal que estrictamente un gobernante. En ciertos sentidos, actúa como lo haría un príncipe planetario, pero su gobierno tiene más parecido al de los hijos materiales.
\vs p114 3:4 \pc El gobierno de Urantia está representado en los consejos de Jerusem con arreglo a un acuerdo por el que el gobernador general saliente se convierte en miembro temporal del gabinete de príncipes planetarios del soberano del sistema. Cuando se designó a Maquiventa príncipe vicerregente, se esperaba que asumiese de inmediato su puesto en el consejo de los príncipes planetarios de Satania, pero hasta ahora no ha hecho gesto alguno que lo indique.
\vs p114 3:5 El gobierno supramaterial de Urantia no mantiene una relación orgánica muy estrecha con las unidades de orden superior del universo local. De alguna manera, el gobernador general residente representa a Lugar de Salvación al igual que a Jerusem dado que actúa en nombre de los veinticuatro consejeros, que son a su vez representantes directos de Miguel y de Gabriel. Y al ser ciudadano de Jerusem, el gobernador planetario puede desempeñar su actividad como portavoz del soberano del sistema. Las autoridades de la constelación están representadas directamente por un hijo vorondadec, el observador de Edentia.
\usection{4. EL OBSERVADOR ALTÍSIMO}
\vs p114 4:1 La soberanía de Urantia se vio complicada incluso más por la confiscación discrecional en el pasado de la autoridad planetaria por parte del gobierno de Norlatiadek, poco tiempo después de la rebelión planetaria. Aún reside en Urantia un hijo vorondadec, un observador en nombre de los altísimos de Edentia y depositario de la soberanía planetaria en ausencia de una acción directa de Miguel. El presente observador altísimo (antiguo regente) es el vigésimo tercero que presta sus servicios en Urantia.
\vs p114 4:2 Existe un cierto número de asuntos planetarios que continúan bajo la dirección de los altísimos de Edentia, al asumir la competencia sobre ellos en tiempos de la rebelión de Lucifer. Ejerce la autoridad en estas cuestiones un hijo vorondadec, el observador de Norlatiadek, que mantiene relaciones de asesoramiento muy estrechas con los supervisores planetarios. Los comisionados para las razas son muy activos en Urantia, y los jefes de sus distintos grupos están adscritos de manera no oficial al observador vorondadec residente, que obra en calidad de su director consultivo.
\vs p114 4:3 En el caso de alguna crisis, el jefe legítimo y soberano del gobierno, salvo en algunas materias puramente espirituales, sería este hijo vorondadec de Edentia, ahora en cumplimiento de su deber de observación. (En estas cuestiones exclusivamente espirituales y en determinados asuntos puramente personales, la autoridad suprema parece recaer en el arcángel al mando, adscrito a la sede local de ese orden, que se estableció recientemente en Urantia).
\vs p114 4:4 \pc Un observador altísimo está facultado, a discreción suya, para hacerse con el gobierno planetario en momentos de crisis planetarias graves, y hay constancia de que esto ha sucedido treinta y tres veces en la historia de Urantia. En dichas ocasiones, este observador actúa como regente altísimo, ejerciendo una incuestionable autoridad sobre todos los ministros y administradores residentes en el planeta, con excepción de la organización local de los arcángeles.
\vs p114 4:5 Las regencias de los vorondadecs no son exclusivas de los planetas aislados por la rebelión, puesto que los altísimos pueden intervenir, en cualquier momento, en los asuntos de los mundos habitados, al interponer la sabiduría superior de los gobernantes de la constelación en las cuestiones de los reinos de los hombres.
\usection{5. EL GOBIERNO PLANETARIO}
\vs p114 5:1 El gobierno actual de Urantia es ciertamente difícil de describir. No existe un gobierno formal en el sentido en el que está organizado el universo, con la separación de los órganos legislativo, ejecutivo y judicial. Los veinticuatro consejeros son los que más próximos están de constituir la rama legislativa del gobierno planetario. El gobernador general es el mandatario en jefe consultivo con carácter provisional, pero es en el observador altísimo en quien recae el poder de veto. Y no hay en el planeta poderes judiciales que operen con autoridad absoluta ---solamente comisiones conciliadoras---.
\vs p114 5:2 La mayoría de los problemas que afectan a los serafines y a las criaturas intermedias se resuelven, de común acuerdo, por el gobernador general. Si bien, salvo cuando se hacen eco de los mandatos de los veinticuatro consejeros, sus pronunciamientos son susceptibles de apelación ante las comisiones conciliadoras, ante las autoridades locales constituidas para el buen funcionamiento del planeta o incluso ante el soberano del sistema de Satania.
\vs p114 5:3 La ausencia de la comitiva corpórea de un príncipe planetario y del régimen material de un hijo y de una hija adánicos se compensa en parte por el ministerio especial de los serafines y por los extraordinarios servicios de los seres intermedios. La ausencia del príncipe planetario se compensa, a su vez, con eficacia por la presencia trina de los arcángeles, el observador altísimo y el gobernador general.
\vs p114 5:4 Este gobierno, más bien flexiblemente organizado y administrado hasta cierto punto de manera personal, es más eficiente de lo que cabría esperar debido a la intervención, y al ahorro de tiempo que ello supone, de los arcángeles y de su vía circulatoria, siempre disponible, y de la que se hace uso con tanta frecuencia en emergencias planetarias y dificultades de tipo administrativo. Estrictamente hablando, el planeta está aún espiritualmente aislado de las vías circulatorias de Norlatiadek, pero ahora, en una emergencia, este impedimento se puede eludir gracias a la utilización de la vía de los arcángeles. El aislamiento planetario es, desde luego, poco preocupante para los mortales a nivel individual desde el derramamiento, hace mil novecientos años, del espíritu de la verdad sobre toda carne.
\vs p114 5:5 \pc En Urantia, cada día de carácter administrativo comienza con una conferencia consultiva, que cuenta con la asistencia del gobernador general, del jefe planetario de los arcángeles, del observador altísimo, de los supernafines supervisores, del jefe de los portadores de vida residentes y de invitados de entre los elevados hijos del universo o de entre algunos de los visitantes estudiantiles que se encuentren circunstancialmente residiendo en el planeta.
\vs p114 5:6 El gabinete administrativo directo del gobernador general se compone de doce serafines, jefes en funciones de los doce grupos de ángeles especiales que actúan en calidad de directores sobrehumanos inmediatos a cargo del progreso y de la estabilidad planetarios.
\usection{6. LOS SERAFINES MAYORES A CARGO DE LA SUPERVISIÓN PLANETARIA}
\vs p114 6:1 Cuando el primer gobernador general llegó a Urantia, en coincidencia con el derramamiento del espíritu de la verdad, venía acompañado de doce colectivos de serafines especiales, graduados de Lugar de Serafines, que se asignaron de inmediato a determinados servicios planetarios especiales. Estos excelsos ángeles se conocen como serafines mayores a cargo de la supervisión planetaria y están, al margen de la acción directiva que ejerce el observador planetario altísimo, bajo la inmediata dirección del gobernador general residente.
\vs p114 6:2 Estos doce grupos de ángeles, aunque ejercen su actividad bajo la supervisión del gobernador general residente, se dirigen por el consejo seráfico de los doce, que son los jefes en funciones de cada grupo. Tal consejo sirve igualmente como gabinete con carácter voluntario de dicho gobernador general.
\vs p114 6:3 Como jefe planetario de los serafines, yo presido este consejo de jefes seráficos, y soy un supernafín voluntario del orden primario, prestando mi servicios en Urantia como sucesor del antiguo jefe de las multitudes angélicas planetarias que transgredió su deber en el momento de la secesión de Caligastia.
\vs p114 6:4 Los doce colectivos de estos serafines mayores operan en Urantia de la siguiente manera:
\vs p114 6:5 \li{1.}\bibemph{Los ángeles de época}. Son los ángeles de la era imperante, el grupo de esta dispensación. Estos servidores celestiales se encargan de la supervisión y dirección de los asuntos de cada una de las generaciones, puesto que están concebidos para encajar en el mosaico de la era en la que estos ocurren. El colectivo actual de ángeles de época que presta sus servicios en Urantia es el tercer grupo destinado al planeta durante la dispensación vigente.
\vs p114 6:6 \li{2.}\bibemph{Los ángeles del progreso}. A estos serafines se les ha confiado la tarea de iniciar el progreso evolutivo de las sucesivas eras sociales. Propician el desarrollo de la tendencia progresiva intrínseca a las criaturas evolutivas; trabajan de forma incesante para hacer que las cosas sean como deben ser. El grupo que presta sus servicios en este momento es el segundo que se destinó al planeta.
\vs p114 6:7 \li{3.}\bibemph{Los guardianes religiosos}. Son los “ángeles de las iglesias”, los fervientes batalladores por lo que es y lo que ha sido. Se esfuerzan por mantener los ideales de lo que ha sobrevivido en aras de la transmisión segura de los valores morales de una época a otra. Contrabalancean a los ángeles del progreso, buscando todo el tiempo trasladar constantemente, de una generación a la otra, los valores imperecederos de las formas antiguas y pasadas a los patrones de pensamiento y conducta más nuevos y, por ello, menos estables. Estos ángeles contienden por las formas espirituales, pero no dan lugar al ultrasectarismo ni a las controvertidas escisiones sin sentido de creyentes declarados. En la actualidad, el colectivo operativo en Urantia es el quinto que presta de esta manera sus servicios.
\vs p114 6:8 \li{4.}\bibemph{Los ángeles de la vida nacional}. Son los “ángeles de las trompetas”, directores de las actuaciones políticas de la vida nacional en Urantia. En este momento, el grupo que ejerce una acción directiva sobre las relaciones internacionales es el cuarto colectivo que presta sus servicios en el planeta. Es particularmente a través del ministerio de esta división seráfica como “los altísimos gobiernan en los reinos de los hombres”.
\vs p114 6:9 \li{5.}\bibemph{Los ángeles de las razas.} Trabajan por la conservación de las razas evolutivas del tiempo, con independencia de sus implicaciones políticas y de agrupaciones religiosas. En Urantia existen remanentes de las nueve razas humanas que se han mezclado e integrado en los pueblos de los tiempos modernos. Estos serafines están estrechamente relacionados con el ministerio de los comisionados de las razas, y el grupo que está ahora en Urantia es el colectivo primigenio destinado al planeta poco después del día de Pentecostés.
\vs p114 6:10 \li{6.}\bibemph{Los ángeles del futuro}. Son los ángeles planificadores, diseñan la era futura y planifican el logro de las mejores cosas de una dispensación nueva y en avance; son los arquitectos de las eras sucesivas. El grupo que está actualmente en el planeta ha desempeñado esta función desde el comienzo de la dispensación imperante.
\vs p114 6:11 \li{7.}\bibemph{Los ángeles del conocimiento}. Urantia recibe en el presente la ayuda del tercer colectivo de serafines dedicados a promocionar la educación planetaria. Estos ángeles se ocupan de la formación mental y moral en lo referido a personas individuales, familias, grupos, escuelas, comunidades, naciones y razas completas.
\vs p114 6:12 \li{8.}\bibemph{Los ángeles de la salud.} Son los servidores seráficos destinados a la asistencia de aquellas instancias humanas dedicadas al fomento de la salud y a la prevención de las enfermedades. El colectivo actual es el sexto grupo que presta sus servicios durante esta dispensación.
\vs p114 6:13 \li{9.}\bibemph{Los serafines del hogar}. Urantia cuenta en este momento con los servicios del quinto grupo de servidores angélicos, que se dedica a preservar y a hacer avanzar el hogar, la institución básica de la civilización humana.
\vs p114 6:14 \li{10.}\bibemph{Los ángeles de la industria.} Este grupo seráfico se preocupa de propiciar el desarrollo industrial y de mejorar las condiciones económicas de los pueblos de Urantia. Desde el ministerio de gracia de Miguel, este colectivo se ha sustituido siete veces.
\vs p114 6:15 \li{11.}\bibemph{Los ángeles de la diversión}. Son los serafines que fomentan los valores del esparcimiento, el humor y el descanso. Tratan siempre de enaltecer las distracciones recreativas del hombre y de promocionar, de este modo, el uso más provechoso del tiempo libre humano. El presente colectivo es el tercero de este orden que realiza su ministerio en Urantia.
\vs p114 6:16 \li{12.}\bibemph{Los ángeles del ministerio sobrehumano}. Son los ángeles de los ángeles, esos serafines destinados a ejercer su ministerio en favor de toda la otra vida sobrehumana existente en el planeta, de forma temporal o permanente. Este colectivo ha prestado sus servicios desde el comienzo de la dispensación presente.
\vs p114 6:17 \pc Cuando estos grupos de serafines mayores están en desacuerdo respecto a cuestiones relativas a la política o a los procedimientos planetarios, es normalmente el gobernador general quien solventa sus discrepancias, pero todos sus dictámenes son objeto de recurso, en conformidad con la naturaleza y gravedad de los asuntos planteados en tal desacuerdo.
\vs p114 6:18 Ninguno de estos grupos angélicos ejerce un control directo o subjetivo sobre las áreas en las que se desarrollan sus destinos. No pueden regir totalmente los asuntos de sus respectivos ámbitos de acción, pero pueden actuar, y de hecho lo hacen, sobre las condiciones planetarias y relacionar de tal manera las circunstancias que pueden influir favorablemente sobre las esferas de la actividad humana a las que están adscritos.
\vs p114 6:19 Los serafines mayores a cargo de la supervisión planetaria hacen uso de numerosas instancias intermedias para llevar a cabo sus misiones. Recopilan y distribuyen las ideas, centran la atención de la mente y propician proyectos. Aunque no pueden introducir conceptos nuevos ni de mayor elevación en las mentes humanas, actúan a menudo intensificando algún ideal supremo que ya haya aparecido en el intelecto humano.
\vs p114 6:20 Pero, al margen de estas múltiples formas de acción positiva, los serafines mayores aseguran el progreso planetario contra graves amenazas mediante la movilización, la formación y mantenimiento del colectivo de reserva de destino. La función principal de estos reservistas consiste en salvaguardar el progreso evolutivo del colapso; constituyen la dotación de las fuerzas celestiales para contrarrestar las sorpresas; son la garantía contra los desastres.
\usection{7. EL COLECTIVO DE RESERVA DE DESTINO}
\vs p114 7:1 El colectivo de reserva de destino está formado por hombres y mujeres vivos admitidos al servicio especial de la administración sobrehumana de los asuntos del mundo. Este colectivo está integrado por hombres y mujeres de cada generación que los directores espirituales del planeta eligen para contribuir a la impartición del ministerio de la misericordia y de la sabiduría a los hijos del tiempo de los mundos evolutivos. En la gestión de las cuestiones relacionadas con los planes de ascensión, es una práctica generalizada comenzar a utilizar este vínculo de criaturas volitivas mortales en cuanto estas son competentes y dignas de asumir tales responsabilidades. Por consiguiente, en el momento en el que aparecen hombres y mujeres en el escenario de la acción temporal con suficiente capacidad mental, adecuado estatus moral y la espiritualidad exigida, se les asigna rápidamente al apropiado grupo celestial de seres personales planetarios como enlaces humanos, como ayudantes mortales.
\vs p114 7:2 Cuando se elige a seres humanos como protectores del destino planetario, cuando se convierten en personas determinantes en los planes que los administradores del mundo están ejecutando, en ese momento, el jefe planetario de los serafines confirma la adscripción temporal de estos al colectivo seráfico y designa guardianes personales del destino al servicio de dichos reservistas mortales. Todos ellos tienen modeladores autoconscientes y, en su mayoría, actúan en los círculos cósmicos superiores representativos de su nivel de logro intelectual y espiritual.
\vs p114 7:3 A los mortales de los planetas se les elige para servir en el colectivo de reservistas del destino de los mundos habitados en razón de:
\vs p114 7:4 \li{1.}Su capacidad especial para preparárseles en secreto con el fin de realizar las numerosas posibles misiones de emergencia en su gestión de las actividades diversas de los asuntos del mundo.
\vs p114 7:5 \li{2.}Su total dedicación a alguna causa social, económica, política, espiritual u otra, de especial relevancia, sumada a su buena disposición de servir sin reconocimientos ni recompensas humanos.
\vs p114 7:6 \li{3.}Estar en posesión de un modelador del pensamiento de extraordinaria versatilidad y probable experiencia anterior a la urantiana para abordar las dificultades planetarias y hacer frente a situaciones inminentes de emergencia mundial.
\vs p114 7:7 \pc Cada división del servicio celestial planetario tiene derecho a un colectivo de enlace formado por estos mortales con condición de destino. Como promedio, un mundo habitado recurre a setenta colectivos del destino distintos, que están estrechamente relacionados con la actual gestión de los asuntos de ese mundo. En Urantia, hay doce colectivos de reservistas del destino, uno para cada uno de los colectivos planetarios bajo supervisión seráfica.
\vs p114 7:8 En Urantia, los doce grupos de reservistas del destino están integrados por habitantes mortales de la esfera, a los que se les ha preparado para ocupar numerosos puestos cruciales en la tierra y se les mantiene listos para actuar en posibles emergencias planetarias. Combinado, este colectivo consta ahora de 962 personas. El colectivo menos numeroso tiene 41 y, el más numeroso, 172. Salvo menos de una veintena de seres personales de contacto, los miembros de este grupo único, son completamente inconscientes de su preparación para una posible actuación en determinadas crisis planetarias. Estos reservistas mortales se eligen por el colectivo al que se les adscribe respectivamente y se les entrena y prepara en su mente profunda mediante una metodología conjunta que parte del ministerio del modelador del pensamiento y del guardián seráfico. Muchas veces, numerosos otros seres personales celestiales participan en este entrenamiento inconsciente y, en toda esta preparación especial, los seres intermedios realizan unos servicios valiosos e indispensables.
\vs p114 7:9 En muchos mundos, las criaturas intermedias secundarias mejor adaptadas están facultadas para establecer distintos grados de contacto con los modeladores del pensamiento de determinados mortales, favorablemente constituidos. Lo hacen accediendo hábilmente a las mentes en las que los modeladores de estos últimos mora. (Y, justamente, mediante esta combinación fortuita de ajustes cósmicos se materializaron en Urantia, en lengua inglesa, estas revelaciones). Dichos mortales de los mundos evolutivos, con tal potencial de contacto, se movilizan en los numerosos colectivos de reserva, y, en cierta medida, a través de estos pequeños grupos de personas con amplias miras, avanza la civilización espiritual y los altísimos pueden gobernar en los reinos de los hombres. Los hombres y mujeres de estos colectivos de reservistas gozan, pues, de distintos grados de contacto con sus modeladores gracias a la mediación del ministerio de las criaturas intermedias; pero sus semejantes conocen poco a estos mismos mortales, excepto en esas raras emergencias sociales y dificultades espirituales en las que estas personas actúan para evitar el colapso de la cultura evolutiva o la extinción de la luz de la verdad viva. En Urantia, a estos reservistas del destino raras veces se les ha elogiado en las páginas de la historia humana.
\vs p114 7:10 Los reservistas conservan, de manera inconsciente, la información planetaria básica. Numerosas veces, al morir un reservista, se realiza, mediante la conjunción de dos modeladores del pensamiento, una transferencia de ciertos datos fundamentales desde la mente del reservista que va a fallecer a la de un sucesor más joven. Sin lugar a dudas, los modeladores desempeñan su actividad, en relación a estos colectivos de reserva, de otras muchas formas desconocidas para nosotros.
\vs p114 7:11 En Urantia, el colectivo de reserva, aunque no tiene un jefe permanente, sí posee sus propios consejos permanentes, que representan su organización gobernativa. Estos incluyen el consejo judicial, el consejo de la historicidad, el consejo de la soberanía política y otros muchos. De vez en cuando, conforme a la organización del colectivo, se ha asignado a los jefes titulares (mortales) de todo el colectivo de reserva a estos consejos permanentes para llevar a cabo algún cometido específico. La duración del mandato de estos jefes reservistas es generalmente cuestión de pocas horas, limitándose a la realización de la tarea concreta encomendada.
\vs p114 7:12 El colectivo de reserva de Urantia tuvo su más amplia membrecía en los días de los adanitas y los anditas, pero declinó de forma constante a raíz de la disolución de la sangre violeta, alcanzando su mínimo cerca de la época de Pentecostés, desde cuyo momento el número de miembros del colectivo de reserva fue aumentando regularmente.
\vs p114 7:13 \pc (En Urantia, el colectivo cósmico de reservistas, conscientes de ser ciudadanos del universo, cuenta en estos momentos con más de mil mortales cuya percepción de la ciudadanía cósmica transciende ampliamente los dominios de su morada terrestre, pero se me ha prohibido revelar la verdadera naturaleza de la actividad de este excepcional grupo de seres humanos vivos.)
\vs p114 7:14 \pc Los mortales de Urantia no deberían permitir que el relativo aislamiento espiritual de su mundo de algunas vías circulatorias del universo local les suscite un sentimiento de abandono cósmico o de orfandad planetaria. En el planeta, hay en operación una supervisión sobrehumana, muy definida y eficiente, de los asuntos del mundo y de los destinos humanos.
\vs p114 7:15 Si bien es verdad que vosotros tenéis, a lo sumo, solo una exigua idea del gobierno planetario ideal. Desde los tiempos primitivos del príncipe planetario, Urantia ha padecido el malogro del plan divino en cuanto al progreso del mundo y al desarrollo racial. No se gobiernan los mundos habitados leales de Satania como se gobierna Urantia. No obstante, en comparación con los otros mundos aislados, vuestros gobiernos planetarios no resultan de tan inferior orden; podría decirse que uno o dos mundos se pueden considerar peores y que, algunos pocos, ligeramente mejores, pero la mayoría se halla en un plano de igualdad con vosotros.
\vs p114 7:16 En el universo local, nadie parece saber cuándo acabará el incierto estatus del gobierno planetario. Los melquisedecs de Nebadón son propensos a opinar que se producirán pocos cambios en el gobierno y en la administración planetarios hasta que Miguel no venga personalmente por segunda vez a Urantia. Sin duda, en ese momento, si no antes, se llevarán a cabo profundos cambios en la gestión del planeta. Pero, en cuanto al carácter de tales modificaciones en la gobernación del mundo, nadie parece ser capaz ni incluso de hacer conjeturas. En toda la historia de los mundos habitados del universo de Nebadón, no hay precedente alguno de tales circunstancias. Entre las muchas cosas difíciles de entender respecto al futuro gobierno de Urantia, una de las más prominentes es el emplazamiento en el planeta de una vía circulatoria y de una sede local de los arcángeles.
\vs p114 7:17 En los consejos del universo, no se olvida a vuestro aislado mundo. Urantia no es un huérfano cósmico estigmatizado por el pecado ni un planeta apartado de la custodia divina a causa de la rebelión. Desde Uversa hasta Lugar de Salvación y continuando hasta Jerusem, incluso en Havona y en el Paraíso, todos saben que estamos aquí; y por vosotros, los mortales que ahora habitáis en Urantia, se siente tal cariñoso afecto y se vela con la misma fidelidad como si esta esfera jamás hubiese sufrido la traición del desleal príncipe planetario, e incluso más. Es eternamente verdad que “el Padre mismo os ama”.
\vsetoff
\vs p114 7:18 [Exposición del jefe de los serafines emplazados en Urantia.]
