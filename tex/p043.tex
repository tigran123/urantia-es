\upaper{43}{Las constelaciones}
\author{Malavatia Melquisedec}
\vs p043 0:1 Por lo común, se hace referencia a Urantia con el número 606 de Satania, de Norlatiadec de Nebadón, lo que quiere decir: mundo habitado seiscientos seis del sistema local de Satania, situado en la constelación de Norlatiadec, una de las cien constelaciones del universo local de Nebadón. Al ser las constelaciones las principales divisiones de la que constan los universos locales, sus gobernantes vinculan los sistemas locales de mundos habitados a la administración central del universo local con sede en Lugar de Salvación y, mediante la reflectividad, a la administración de los suprauniversos, competencia de los ancianos de días en Uversa.
\vs p043 0:2 \pc El gobierno de vuestra constelación está situado en un conjunto de 771 esferas arquitectónicas, de las cuales, la más grande y más central es Edentia, la sede administrativa de los padres de la constelación o altísimos de Norlatiadec. Edentia es aproximadamente cien veces mayor que vuestro mundo. Las setenta esferas principales que rodean Edentia son unas diez veces el tamaño de Urantia, mientras que los diez satélites que giran alrededor de cada uno de estos setenta mundos son más o menos de las dimensiones de Urantia. Estas 771 esferas arquitectónicas tienen un tamaño prácticamente comparable al de esas otras constelaciones.
\vs p043 0:3 \pc El cálculo del tiempo y la distancia en Edentia es el usado en Lugar de Salvación y, como las esferas de la capital del universo, los mundos sedes de la constelación están completamente provistos de todos los órdenes de inteligencias celestiales. En general, estos seres personales no son muy diferentes de los que se describen en relación con la administración del universo.
\vs p043 0:4 Los serafines supervisores, el tercer orden de ángeles del universo local, están asignados al servicio de las constelaciones. Instauran su sede en las esferas capitales y sirven ampliamente en los mundos de formación morontial que las circundan. En Norlatiadec, las setenta esferas principales, junto con los setecientos satélites menores, están habitadas por los univitatias, los ciudadanos permanentes de la constelación. Todos estos mundos arquitectónicos están enteramente regidos por diferentes grupos de vida nativa, en su mayor parte no revelados, pero en los que están incluidos los eficientes espirongas y los hermosos espornagias. Al constituir el punto medio del régimen de formación morontial, la vida morontial de las constelaciones, como es de esperar, es tanto típica como ideal.
\usection{1. LA SEDE CENTRAL DE LA CONSTELACIÓN}
\vs p043 1:1 Edentia tiene abundantes y fascinantes altiplanicies, de extensas elevaciones de materia física coronada de vida morontial e impregnada de gloria espiritual, pero carece de cadenas montañosas escarpadas tal como las que se encuentran en Urantia. Hay decenas de miles de brillantes lagos conectados entre sí por miles y miles de arroyos, pero no hay grandes océanos ni ríos torrenciales. Solo las altiplanicies carecen de estos arroyos en superficie.
\vs p043 1:2 El agua en Edentia y en esferas arquitectónicas similares no difiere de la existente en los planetas evolutivos. Los sistemas hidráulicos de tales esferas son tanto de superficie como subterráneos, y el aire húmedo circula de forma constante. Se puede circunnavegar Edentia siguiendo diferentes rutas acuáticas, aunque el medio principal de transporte es la atmósfera. Los seres espirituales viajan de forma natural por encima de la superficie de la esfera, mientras que los seres morontiales y materiales hacen uso de medios materiales y semimateriales para cruzar la atmósfera.
\vs p043 1:3 Edentia y los mundos vinculados a esta sede central poseen una auténtica atmósfera; se trata de una mezcla habitual de tres gases, característica de tales creaciones arquitectónicas, que incluye los dos elementos de la atmósfera de Urantia además del gas morontial, conveniente para la respiración de las criaturas morontiales. Si bien, aunque dicha atmósfera es tanto material como morontial, no hay tormentas ni huracanes; tampoco hay verano ni invierno. Esta ausencia de perturbaciones atmosféricas y de variaciones estacionales faculta el embellecimiento exterior de estos mundos específicamente creados.
\vs p043 1:4 Las altiplanicies de Edentia gozan de un espléndido relieve físico y de una belleza enaltecida por la inagotable profusión de vida que abunda a todo su largo y ancho. Salvo algunas construcciones de carácter más bien aislado, estas altiplanicies no han sido tocadas por las manos de criatura alguna. La ornamentación material y morontial se circunscribe a las zonas habitadas. En las elevaciones menores, bellamente engalanadas tanto con arte biológico como con arte morontial, se sitúan una excepcionales residencias.
\vs p043 1:5 \pc En la cumbre de la séptima cadena de altiplanicies, se encuentran las salas de resurrección de Edentia, en las que despiertan los mortales ascendentes de los órdenes modificados secundarios de ascensión. Estas cámaras de reconstitución de la criatura están bajo la supervisión de los melquisedecs. La primera de las esferas receptoras de Edentia (al igual que el planeta Melquisedec cercano a Lugar de Salvación) tiene también salas especiales de resurrección, en las que se reconstituye al mortal de dichos órdenes modificados.
\vs p043 1:6 Los melquisedecs también mantienen dos facultades especiales en Edentia. Por un lado, está la escuela de emergencias, que se dedica al estudio de los problemas surgidos por la rebelión en Satania y, por otro, la escuela del ministerio de gracia, que se dedica a la enseñanza de los nuevos problemas que resultan del hecho de que Miguel realizó su último ministerio de gracia en uno de los mundos de Norlatiadec. Esta última facultad se estableció hace casi cuarenta mil años, inmediatamente después de que Miguel anunciara que había elegido Urantia como mundo para dicho último ministerio.
\vs p043 1:7 \pc El mar de cristal, el área de recepción de Edentia, está cerca del centro administrativo y está rodeado por el anfiteatro de esta sede central. Circundando esta área, se hallan los centros de gobierno para las setentas divisiones de gestión de los asuntos de la constelación. La mitad de Edentia está dividida en setenta secciones triangulares, cuyas lindes convergen en los edificios que conforman la sede de sus sectores respectivos. El resto de la esfera es un inmenso parque natural: los jardines de Dios.
\vs p043 1:8 Durante vuestras visitas periódicas a Edentia, aunque podréis perfectamente examinar todo el planeta, pasaréis la mayoría de vuestro tiempo en ese triángulo de gestión administrativa, cuyo número corresponde al de vuestro mundo de residencia actual. Siempre seréis bien recibidos en calidad de observadores en las asambleas legislativas.
\vs p043 1:9 El área morontial asignada a los mortales ascendentes con residencia en Edentia está situada en la zona intermedia del triángulo número treinta y cinco colindante con la sede de los finalizadores, que se halla en el triángulo treinta y seis. La sede de los univitatias ocupa una enorme área de la región intermedia del triángulo treinta y cuatro justo al lado de la zona residencial reservada a los ciudadanos morontiales. A partir de esta configuración, es posible observar que se han adoptado disposiciones para dar cabida al menos a setenta grupos mayores de vida celestial e, igualmente, que cada uno de estos setenta sectores triangulares se correlaciona con alguna de las setenta esferas principales de formación morontial.
\vs p043 1:10 El mar de cristal de Edentia es una inmensa superficie circular de vidrio de alrededor de ciento sesenta kilómetros de circunferencia y de unos cuarenta y ocho kilómetros de profundidad. Este espléndido vidrio sirve de campo de recepción para todos los serafines de transporte y para algunos otros seres que llegan de lugares externos a la esfera; este mar de cristal facilita enormemente el aterrizaje de dichos serafines.
\vs p043 1:11 En casi todos los mundos arquitectónicos, hay este tipo de área de vidrio; y, aparte de su valor decorativo, responde a muchos propósitos; se usa para describir el fenómeno de la reflectividad del suprauniverso a grupos allí reunidos y como factor determinante en la transformación de la energía para modificar las corrientes del espacio y para adaptar otras corrientes de energía física entrantes.
\usection{2. EL GOBIERNO DE LAS CONSTELACIONES}
\vs p043 2:1 Las constelaciones constituyen las unidades autónomas de los universos locales; cada una de ellas se rige según sus propias disposiciones legislativas. Cuando los tribunales de Nebadón juzgan los asuntos del universo, todas las cuestiones internas se deciden conforme a las leyes imperantes en la constelación correspondiente. Son los administradores de los sistemas locales los que se encargan de la ejecución de los decretos judiciales de Lugar de Salvación en conjunción con las promulgaciones de las constelaciones.
\vs p043 2:2 Así pues, las constelaciones funcionan como unidades legislativas o elaboradoras de leyes, mientras que los sistemas locales sirven como unidades de ejecución o de aplicación de las leyes. El gobierno de Lugar de Salvación es el organismo supremo de coordinación judicial.
\vs p043 2:3 \pc Aunque el poder judicial supremo corresponde a la administración central del universo local, en la sede central de cada constelación existen dos tribunales secundarios pero de importancia mayor: el consejo Melquisedec y el tribunal del altísimo.
\vs p043 2:4 Los melquisedecs primeramente realizan una revisión judicial de los casos problemáticos. Se faculta a doce miembros de este orden, con cierta necesaria experiencia en los planetas evolutivos y en los mundos de gobierno del sistema, para revisar pruebas, recopilar alegaciones y formular veredictos provisionales, los cuales se comunican al tribunal del altísimo, o padre reinante de la constelación. El componente humano del mencionado tribunal está integrado por siete jueces, todos ellos mortales ascendentes. Cuanto más ascendáis en el universo, más ciertos estaréis de ser juzgados por seres de vuestra misma clase.
\vs p043 2:5 \pc El órgano legislativo de la constelación está dividido en tres grupos. Su organización comienza con la cámara baja de los ascendentes, integrada por un grupo de mil representantes mortales bajo la autoridad de un finalizador. Cada sistema nombra a diez miembros para que participen en esta asamblea deliberante. En la actualidad, en Edentia, este organismo no está al completo.
\vs p043 2:6 La cámara media de los legisladores está compuesta por las multitudes seráficas y sus colaboradores, otros hijos del espíritu materno del universo local. A este grupo, que cuenta con cien miembros, lo nombran los seres personales supervisores que dirigen a su vez la diversa actividad de dichos seres cuando desempeñan su labor en la constelación.
\vs p043 2:7 El órgano consultivo u órgano legislador de más alto rango de la constelación está compuesto por la cámara de los iguales ---la cámara de los hijos divinos---. Los padres altísimos son los que eligen a este colectivo de diez miembros. Solo aquellos hijos divinos que hayan adquirido una especial experiencia pueden servir en esta cámara alta. Se trata de un grupo que determina hechos y ahorra tiempo, y sirve de manera eficaz a las dos divisiones de menor rango de la asamblea legislativa.
\vs p043 2:8 El consejo conjunto de legisladores se compone de tres miembros de cada una de estas ramas separadas de la asamblea deliberante de la constelación y está presidido por el altísimo reinante de menor rango. Este grupo sanciona la forma final de todas las disposiciones y autoriza su promulgación por medio de los informadores. Con su aprobación, esta comisión suprema convierte en ley de la constelación las disposiciones legislativas; sus medidas son definitivas. Los pronunciamientos legislativos de Edentia constituyen la ley fundamental de todo Norlatiadec.
\usection{3. LOS ALTÍSIMOS DE NORLATIADEC}
\vs p043 3:1 Los gobernantes de las constelaciones pertenecen al orden vorondadec de filiación del universo local. Cuando se les destina al servicio activo en calidad de gobernantes de las constelaciones o en alguna otra función, a estos hijos se les conoce como los \bibemph{altísimos} por personificar la más alta sabiduría en asuntos de gobierno emparejada con la lealtad más juiciosa e inteligente de todos los órdenes de los Hijos de Dios del universo local. Su integridad personal y su lealtad al grupo nunca se han puesto en cuestión; nunca se ha dado en Nebadón deslealtad alguna de parte de los hijos vorondadecs.
\vs p043 3:2 \pc Gabriel nombra al menos a tres hijos vorondadecs como los altísimos de cada una de las constelaciones de Nebadón. Al miembro que preside este trío se le conoce como el \bibemph{Padre de la Constelación} y a sus dos colaboradores como el \bibemph{Altísimo de Rango Mayor} y el \bibemph{Altísimo de Rango Menor}. Los padres de las constelaciones reinan por diez mil años de tiempo regular (alrededor de 50\,000 años de Urantia), habiendo previamente servido como colaborador de menor y de mayor rango respectivamente durante períodos iguales.
\vs p043 3:3 El salmista sabía que Edentia estaba gobernada por tres padres de la constelación y, por consiguiente, se refirió a su morada en plural: “Del río sus corrientes alegran la ciudad de Dios, el santuario de los tabernáculos de los altísimos”.
\vs p043 3:4 \pc A lo largo de los siglos ha existido una gran confusión en Urantia en relación a los distintos gobernantes del universo. Muchos maestros posteriores llegaron a confundir sus imprecisas y ambiguas deidades tribales con los padres altísimos. Incluso más tarde, los hebreos fusionaron todos estos gobernantes celestiales en una deidad compuesta. Uno de sus maestros comprendió que los altísimos no eran los gobernantes supremos, puesto que dijo: “El que habita al abrigo del altísimo morará bajo la sombra del Omnipotente”. En los textos que se conservan en Urantia es muy difícil a veces saber con exactitud qué quieren decir con el término “altísimo”. Pero Daniel lo comprendió plenamente cuando dijo “El Altísimo gobierna en el reino de los hombres y lo da a quien él quiere”.
\vs p043 3:5 \pc Los padres de las constelaciones se implican poco con los seres individuales de los planetas habitados, pero están estrechamente relacionados con las funciones legislativa y elaboradora de leyes de las constelaciones, que tanto incumben a cada una de las \bibemph{razas} mortales y al \bibemph{grupo} nacional de los mundos habitados.
\vs p043 3:6 Aunque el régimen de la constelación se sitúa entre vosotros y la administración del universo, por lo común, de manera individual, no prestaríais mucha atención al gobierno de la constelación, sino que normalmente os centraríais, en su mayor parte, en el sistema local de Satania; pero Urantia está, de forma temporal, íntimamente vinculada a los gobernantes de la constelación debido a ciertas circunstancias planetarias y en relación al sistema local que resultaron de la rebelión de Lucifer.
\vs p043 3:7 En el momento de la secesión de Lucifer, los altísimos de Edentia asumieron ciertos niveles de autoridad planetaria en los mundos rebeldes, y continúan haciendo uso de esta atribución. Hace mucho tiempo que los ancianos de días dieron su confirmación a tal asunción de poder sobre estos mundos insubordinados. Sin duda, continuarán ejerciendo esa jurisdicción que asumieron mientras Lucifer esté vivo. En un sistema leal, una gran parte de esta autoridad se confiaría generalmente al soberano del sistema.
\vs p043 3:8 Pero todavía existe otro modo en el que Urantia mantuvo una singular relación con los altísimos. Cuando Miguel, el hijo creador, realizaba su última misión de gracia, toda vez que el sucesor de Lucifer no poseía plenos poderes en el sistema local, los altísimos de Norlatiadec se encargaron directamente de supervisar todos los asuntos relativos a este ministerio de Miguel en Urantia.
\usection{4. EL MONTE DE LA ASAMBLEA: EL FIEL DE DÍAS}
\vs p043 4:1 El monte santísimo de la asamblea es la morada del fiel de días, el representante de la Trinidad del Paraíso que opera en Edentia.
\vs p043 4:2 Este fiel de días es hijo de la Trinidad del Paraíso y está presente en Edentia en calidad de representante personal de Emanuel desde la creación de dicho mundo sede. Siempre está a la diestra de los padres de la constelación para prestarles asesoramiento, pero jamás brinda sus consejos a menos que se le pida. Los elevados hijos del Paraíso nunca toman parte en la dirección de los asuntos de los universos locales salvo cuando los actuales gobernantes de tales dominios se lo soliciten. Pero todo lo que el unión de días es para el hijo creador, lo es el fiel de días para los altísimos de una constelación.
\vs p043 4:3 La residencia del fiel de días de Edentia constituye, para la constelación, el centro del sistema paradisíaco de comunicación e información externo al universo. Estos hijos de la Trinidad, con sus asistentes personales de Havona y del Paraíso, en conjunción con el unión de días encargado de la supervisión, se mantienen en contacto directo y permanente con los miembros de su orden en todos los universos, incluyendo Havona y el Paraíso.
\vs p043 4:4 El monte santísimo es de una espléndida belleza y está magníficamente decorado, pero la morada del hijo del Paraíso es modesta si se la compara con la morada central de los altísimos y las setenta construcciones que la rodean, y que constituyen el complejo residencial de los hijos vorondadecs. Estas instalaciones son exclusivamente residenciales; están completamente retiradas de los grandes edificios que conforman la sede administrativa en donde se tramitan los asuntos de la constelación.
\vs p043 4:5 La residencia del fiel de días de Edentia está situada al norte de las residencias de los altísimos y se le conoce como “el monte de la asamblea del Paraíso”. En esta montaña consagrada, los mortales ascendentes se reúnen periódicamente para oír narrar a este hijo del Paraíso el largo y fascinante viaje de los mortales a su paso por los mil millones de mundos de perfección de Havona para continuar hasta el indescriptible gozo del Paraíso. Aquí, en estas congregaciones de carácter especial, que tienen lugar en el Monte de la Asamblea, los mortales morontiales llegan a tener un más profundo conocimiento de los distintos grupos de seres personales originarios del universo central.
\vs p043 4:6 Cuando el traidor Lucifer, antiguo soberano de Satania, dio noticia de su reivindicación de una mayor área de autoridad en el esquema de gobierno del universo local, buscaba reemplazar a todos los órdenes superiores de filiación. Se lo propuso en su corazón, diciendo: “En lo alto, levantaré mi trono sobre los Hijos de Dios; y en el monte de la asamblea me sentaré, en los extremos del norte; y seré semejante al altísimo”.
\vs p043 4:7 \pc Los cien soberanos de los sistemas acuden regularmente a los cónclaves de Edentia que deliberan sobre el bien común de la constelación. Tras la rebelión de Satania, los archirrebeldes de Jerusem solían presentarse en estos consejos tal como lo habían hecho en ocasiones anteriores. Y no se halló el modo de detener esta arrogancia e insolencia hasta que Miguel completó su ministerio de gracia en Urantia y asumió la soberanía ilimitada sobre todo Nebadón. Desde ese día, en Edentia, nunca se ha permitido a estos instigadores del pecado asistir a los consejos de los soberanos leales de los sistemas.
\vs p043 4:8 El hecho de que los maestros de otro tiempo tenían conocimiento de estas cuestiones se demuestra por este texto: “Y hubo un día en el que los Hijos de Dios se presentaron delante de los altísimos, y entre ellos vino también Satanás para presentarse delante de ellos”. Esta es la constatación de un hecho, con independencia del contexto en el que pueda aparecer.
\vs p043 4:9 \pc Desde el triunfo de Cristo, todo Norlatiadec se purifica de pecados y de rebeldes. En algún momento, antes de la muerte de Miguel en la carne, el colaborador del caído Lucifer, Satanás, trató de asistir a uno de los cónclaves de Edentia, pero el sentimiento de unión contra los archirrebeldes había llegado al punto en el que las puertas de la conmiseración estaban tan absolutamente cerradas que no se encontró apoyo alguno para los adversarios de Satania. Cuando no se abren las puertas al mal, no se da pie al pecado. Las puertas de los corazones de toda Edentia se cerraron para Satanás; el rechazo de los soberanos de los sistemas allí congregados fue unánime, y fue en este momento cuando el Hijo del Hombre “vio a Satanás caer de los cielos como un rayo”.
\vs p043 4:10 Desde la rebelión de Lucifer se dispone de una nueva construcción cerca de la residencia del fiel de días. Este edificio de carácter transitorio constituye la sede del altísimo de enlace, que actúa en estrecho contacto con el hijo del Paraíso, en calidad de asesor para el gobierno de la constelación en todas las cuestiones relacionadas con la política y la actitud del orden de días hacia el pecado y la rebelión.
\usection{5. LOS PADRES DE EDENTIA DESDE LA REBELIÓN DE LUCIFER}
\vs p043 5:1 La rotación en cargos de los altísimos de Edentia quedó suspendida en el momento de la rebelión de Lucifer. En la actualidad, contamos con los mismos gobernantes que estaban de turno en aquel momento. Deducimos que estos gobernantes no serán reemplazados hasta que Lucifer y sus colaboradores sean definitivamente destruidos.
\vs p043 5:2 No obstante, el gobierno actual de la constelación, se ha ampliado hasta constar de doce hijos del orden de los vorondadecs. Estos son los siguientes:
\vs p043 5:3 \li{1.}El Padre de la Constelación. El actual gobernante altísimo de Norlatiadec es el número de secuencia 617\,318 de los vorondadecs de Nebadón. Prestó servicios en muchas de las constelaciones de todo nuestro universo local antes de asumir sus responsabilidades con respecto a Edentia.
\vs p043 5:4 \li{2.}El colaborador altísimo de rango mayor.
\vs p043 5:5 \li{3.}El colaborador altísimo de rango menor.
\vs p043 5:6 \li{4.}El asesor altísimo: el representante personal de Miguel desde que este alcanzó su estatus de hijo mayor.
\vs p043 5:7 \li{5.}El mandatario altísimo: el representante personal de Gabriel emplazado en Edentia desde la rebelión de Lucifer.
\vs p043 5:8 \li{6.}El jefe altísimo de los observadores planetarios: el director de los observadores Vorondadec emplazados en los mundos de Satania en aislamiento.
\vs p043 5:9 \li{7.}El árbitro altísimo: el hijo vorondadec a quien se le confió el cometido de corregir todos los problemas resultantes de la rebelión surgida en la constelación.
\vs p043 5:10 \li{8.}El gestor de emergencias altísimo: el hijo vorondadec encargado de la labor de adaptar los estatutos de emergencia del poder legislativo de Norlatiadec a los mundos de Satania aislados por la rebelión.
\vs p043 5:11 \li{9.}El mediador altísimo: el hijo vorondadec con la misión de armonizar los cambios surgidos por el ministerio de gracia habidos en Urantia con la administración rutinaria de la constelación. La existencia de cierta actividad de parte de algunos arcángeles y de numerosos otros servicios atípicos en Urantia, junto con la actividad especial de las brillantes estrellas vespertinas en Jerusem, requiere la intervención de este hijo.
\vs p043 5:12 \li{10.}El juez auditor altísimo: el jefe del tribunal de emergencia dedicado a la resolución de los problemas especiales surgidos en Norlatiadec por la confusión derivada de la rebelión en Satania.
\vs p043 5:13 \li{11.}El altísimo de enlace: el hijo vorondadec adscrito a los gobernantes de Edentia pero con nombramiento en calidad de asesor especial del fiel de días con respecto al mejor camino a seguir en la gestión de los problemas relativos a la rebelión y a la deslealtad de las criaturas.
\vs p043 5:14 \li{12.}El director altísimo: el presidente del consejo de emergencia de Edentia. Todos los seres personales asignados a Norlatiadec como consecuencia de la sublevación ocurrida en Satania constituyen el consejo de emergencia que está presidido por un hijo vorondadec de una extraordinaria experiencia.
\vs p043 5:15 Lo anterior no tiene en cuenta a los numerosos vorondadecs, enviados de las constelaciones de Nebadón, y a algunos otros que residen igualmente en Edentia.
\vs p043 5:16 \pc Desde la rebelión de Lucifer, los padres de Edentia han prestado una especial atención a Urantia y otros mundos aislados de Satania. Hace mucho tiempo que el profeta percibió la mano rectora de los padres de la constelación en los asuntos de las naciones: “Cuando el altísimo hizo heredar a las naciones, cuando hizo dividir a los hijos de Adán, estableció los límites de los pueblos”.
\vs p043 5:17 En cualquier mundo en cuarentena o en aislamiento hay un hijo vorondadec que actúa en calidad de observador. No toma parte en la administración del planeta salvo cuando el padre altísimo de la constelación lo manda intervenir en los asuntos de las naciones. En realidad, es este observador altísimo el que “tiene dominio sobre el reino de los hombres”. Urantia es uno de los mundos aislados de Norlatiadec y, desde la traición de Caligastia, hay un observador vorondadec emplazado en el planeta. Cuando Maquiventa Melquisedec desempeñó su ministerio en Urantia con una forma semimaterial, rindió, tal como está escrito, un respetuoso homenaje al observador altísimo en aquel momento allí destinado: “Y Melquisedec rey de Salem y sacerdote del Altísimo”. Melquisedec reveló la relación de este observador altísimo con Abraham cuando dijo, “Bendito sea el Altísimo, que entregó a tus enemigos en tus manos”.
\usection{6. LOS JARDINES DE DIOS}
\vs p043 6:1 Las capitales de los sistemas están singularmente embellecidas por construcciones de orden material y mineral, mientras que la sede del universo refleja más la gloria espiritual, pero las capitales de las constelaciones constituyen la cúspide de la actividad morontial y de la ornamentación viva. En los mundos sedes de las constelaciones, este tipo de ornamentación viva se utiliza de forma más generalizada, y es esta preponderancia de la vida ---el arte botánico--- la que hace que a estos mundos se les llame “los jardines de Dios”.
\vs p043 6:2 \pc Alrededor de la mitad de Edentia está destinada a los excelentes jardines de los altísimos, que se encuentran entre las creaciones morontiales más fascinantes del universo local. Esto explica por qué tan a menudo haya, en los mundos habitados de Norlatiadec, sitios de extraordinaria belleza con el apelativo de “jardines del Edén”.
\vs p043 6:3 En un lugar central de estos magníficos jardines, se encuentra el santuario de culto de los altísimos. El salmista debe haber tenido algún conocimiento de estas cosas cuando escribió: “¿Quién subirá las colinas de los Altísimos? ¿Quién estará en su lugar sagrado? El limpio de manos y puro de corazón, el que no ha elevado su alma a cosas vanas ni ha jurado con engaño”. En este templo, los altísimos, cada décimo día de ocio, guían a toda Edentia en adoración contemplativa al Dios Supremo.
\vs p043 6:4 \pc Los mundos arquitectónicos gozan de diez formas de vida de tipo material. En Urantia hay vida vegetal y vida animal, pero en un mundo como Edentia existen diez grupos de órdenes materiales de vida. Si pudieseis contemplarlos, rápidamente clasificaríais a las tres primeras de vegetales y a las tres últimas de animales, pero seríais manifiestamente incapaces de comprender la naturaleza de los cuatro grupos intermedios de fértiles y fascinantes formas de vida.
\vs p043 6:5 Incluso la vida ostensiblemente animal es muy distinta a la de los mundos evolutivos, tan distinta que resulta del todo imposible que la mente mortal pueda percibir el carácter único y el temperamento cariñoso de estas criaturas sin habla. Vuestra imaginación sería incapaz de concebir las miles y miles de criaturas vivas que allí existen. Toda la creación animal es de un orden enteramente diferente al de las toscas especies animales de los planetas evolutivos. Toda esta vida animal es, por otro lado, muy inteligente y delicadamente servicial, y todas las diversas especies son sorprendentemente dóciles y conmovedoramente amigables. En estos mundos arquitectónicos no hay criaturas carnívoras; en toda Edentia no existe nada que pueda atemorizar a un ser vivo.
\vs p043 6:6 La vida vegetal es también muy distinta a la de Urantia; se compone de variedades tanto de tipo material como morontial. La vegetación de tipo material tiene un colorido verde que la caracteriza, mientras que el equivalente morontial de vida de carácter vegetal tiene una tonalidad del color de las violetas o de las orquídeas con diversos matices y reflejos. Esta vegetación morontial es simplemente un brote de energía; cuando se consume, no deja residuo alguno.
\vs p043 6:7 Al estar dotados de diez grupos de vida de orden físico, por no mencionar las variantes de tipo morontial que poseen, estos mundos arquitectónicos ofrecen enormes posibilidades para el embellecimiento biológico del paisaje y de las construcciones materiales y morontiales. Los artesanos celestiales dirigen a los nativos espornagias en su gran labor de llevar a cabo la decoración vegetal y el engalanamiento biológico. Mientras que vuestros artistas tienen que recurrir a la inerte pintura y al inanimado mármol para plasmar sus conceptos, los artesanos celestiales y los univitatias hacen un uso más frecuente de materiales vivos para representar sus ideas y captar sus ideales.
\vs p043 6:8 Si disfrutáis de las flores, los arbustos y los árboles de Urantia, vuestros ojos se sentirán agasajados al contemplar la belleza de las plantas y la magnificencia floral de los supremos jardines de Edentia. Pero sobrepasa mi capacidad de descripción poder transmitir a la mente mortal una idea adecuada de la belleza de los mundos celestiales. En verdad, el ojo no ha visto glorias como las que os aguardan a vuestra llegada a estos mundos en vuestra aventura de ascensión como mortales.
\usection{7. LOS UNIVITATIAS}
\vs p043 7:1 Los univitatias son los ciudadanos permanentes de Edentia y de los mundos vinculados a esta sede de la constelación; la totalidad de los setecientos setenta mundos que rodean tal sede están bajo su supervisión. Esta progenie del hijo creador y del espíritu creativo tiene su existencia en un plano que media entre lo material y lo espiritual, pero no son seres morontiales. Los nativos de cada una de las setenta esferas principales de Edentia poseen diferentes formas visibles, y a los mortales morontiales se les adapta su forma morontial para que se corresponda con la escala ascendente de los univitatias cada vez que dichos mortales trasladan su residencia de una esfera a otra de Edentia, en su paso consecutivo desde el mundo número uno al setenta.
\vs p043 7:2 Espiritualmente, los univitatias son similares; intelectualmente, varían al igual que lo hacen los mortales; en su forma, guardan un gran parecido con la del estado morontial de la existencia; y se crean para obrar en calidad de setenta órdenes diferentes de seres personales. Cada uno de estos órdenes de univitatias presenta diez variaciones principales de actividad intelectual, y cada uno de estos tipos de variantes intelectuales preside las escuelas de formación especial y culturales, dedicadas a la socialización progresiva de tipo ocupacional o práctica de alguno de los diez satélites que giran alrededor de cada uno de los mundos principales de Edentia.
\vs p043 7:3 Estos setecientos mundos menores son esferas técnicas donde se imparte una enseñanza de carácter práctico sobre el funcionamiento de todo el universo local, y están disponibles para todas las clases de seres inteligentes. Estas escuelas, en las que se instruye en destrezas especiales y en conocimiento técnico, no se dirigen exclusivamente a los mortales ascendentes, aunque sean estos estudiantes morontiales los que constituyen con diferencia el grupo más grande de todos aquellos que asisten a estos cursos de formación. Cuando se os admita en cualquiera de los setenta mundos principales de cultura social, de inmediato se os concederá autorización para cada uno de los diez satélites que lo rodean.
\vs p043 7:4 En las diversas colonias de cortesía, los mortales morontiales ascendentes predominan entre los directores de reversión, si bien los univitatias representan el grupo más numeroso vinculado al colectivo de Nebadón de artesanos celestiales. En todo Orvontón, ningún ser de fuera de Havona, salvo los abandontes de Uversa, puede igualar a los univitatias en cuanto a habilidad artística, adaptabilidad social y lucidez para la coordinación.
\vs p043 7:5 En realidad, estos ciudadanos de la constelación no son miembros del colectivo de artesanos, pero trabajan libremente con todos sus grupos y contribuyen de forma considerable a hacer de los mundos de la constelación las esferas principales para el desarrollo de las espléndidas posibilidades artísticas de esta cultura de transición. No desempeñan funciones más allá de los confines de los mundos sedes de la constelación
\usection{8. LOS MUNDOS FORMATIVOS DE EDENTIA}
\vs p043 8:1 La dotación física de Edentia y de las esferas que la rodean es casi perfecta; difícilmente podrían igualar la grandeza espiritual de las esferas de Lugar de Salvación, pero superan con creces la magnificencia de los mundos formativos de Jerusem. Todas estas esferas de Edentia están energizadas directamente por las corrientes universales del espacio, y sus imponentes sistemas de potencia, tanto material como morontial, se supervisan y distribuyen con toda pericia por los centros de la constelación, asistidos por un capacitado colectivo de controladores físicos mayores y de supervisores de la potencia morontial.
\vs p043 8:2 El tiempo transcurrido en los setenta mundos formativos de transición y cultura morontial, en conjunción con todo el periodo de dedicación de los mortales ascendentes en Edentia, es el intervalo de tiempo más estable de la andadura de estos ascendentes hasta que alcanzan el estatus de finalizadores; esta es sin duda la vida morontial característica. Aunque se os vuelve a re\hyp{}afinar cada vez que pasáis de un preeminente mundo cultural a otro, continuáis conservando el mismo cuerpo morontial; no hay, además, períodos de inconsciencia del ser personal.
\vs p043 8:3 Ocuparéis vuestra estancia en Edentia y en sus esferas vinculadas principalmente en alcanzar un dominio de la ética de grupo, la clave de una interrelación placentera y provechosa entre los diferentes órdenes de seres personales inteligentes del universo y del suprauniverso.
\vs p043 8:4 En los mundos de las moradas, completasteis la unificación del ser personal en desarrollo como mortal ascendente; en la capital del sistema, conseguisteis la ciudadanía en Jerusem y llevasteis a cabo la disposición de someter vuestro yo al ejercicio de las actividades de grupo y de otras tareas relacionadas; pero ahora, en los mundos formativos de la constelación, llegaréis a lograr la verdadera socialización en el desarrollo de vuestro ser personal como ser morontial. Este supremo logro de carácter cultural consiste en aprender cómo:
\vs p043 8:5 \li{1.}Vivir felizmente y trabajar con eficacia con diez semejantes morontiales distintos, mientras que diez de estos grupos se unen para formar compañías de cien miembros y se coaligan después en colectivos de mil.
\vs p043 8:6 \li{2.}Convivir con alegría y cooperar encarecidamente con diez univitatias, los cuales, aunque intelectualmente similares a los seres morontiales, son muy diferentes en todos los demás aspectos. Y, además, tendréis que interactuar con este grupo de diez en la medida en que este se coordina con otras diez agrupaciones, coaligadas a su vez en un colectivo de mil univitatias.
\vs p043 8:7 \li{3.}Conseguir una adaptación de forma simultánea tanto con vuestros iguales morontiales como con estos anfitriones univitatias. Adquirir la habilidad de cooperar con voluntariedad y eficacia con vuestro propio orden de seres y en colaboración estrecha y activa con un grupo algo diferente de criaturas inteligentes.
\vs p043 8:8 \li{4.}Así, al operar socialmente con seres similares y diferentes a vosotros, lograr la armonía intelectual y llevar a efecto una adaptación de orden ocupacional, con ambos grupos de compañeros.
\vs p043 8:9 \li{5.}Mientras conseguís una satisfactoria socialización de vuestro ser personal tanto a nivel intelectual como ocupacional, lograr una mayor perfección en la habilidad de vivir en estrecho contacto con seres similares y algo diferentes con cada vez menor irritabilidad y enojo. Los directores de reversión contribuyen considerablemente a esto último mediante sus actividades lúdicas de grupo.
\vs p043 8:10 \li{6.}Conformar todos estos distintos modos de socialización a fin de que favorezcan la coordinación progresiva de vuestra andadura de ascenso al Paraíso; aumentar vuestra percepción del universo al acrecentarse vuestra capacidad para comprender los objetivos y los contenidos eternos subyacentes en este tipo de actividades espacio\hyp{}temporales aparentemente sin importancia.
\vs p043 8:11 \li{7.}Y, además, alcanzar el punto culminante de todo este sistema de multisocialización en simultaneidad con el incremento de vuestra percepción espiritual en su relación con el engrandecimiento de todas las facetas de vuestras dotes personales, por medio de la vinculación espiritual y la coordinación morontial con el grupo. Intelectual, social y espiritualmente, dos criaturas con capacidad moral, al cooperar entre ellas, no duplican simplemente sus potenciales personales de éxito en el universo sino que multiplican casi por cuatro sus posibilidades de logro y cumplimiento.
\vs p043 8:12 \pc Hemos descrito el sistema de socialización de Edentia como la cooperación de un mortal morontial con una agrupación de univitatias de diez individuos, intelectualmente diferentes, en actuación conjunta de orden similar con diez de sus semejantes morontiales. Si bien, en los primeros siete mundos principales, solo un mortal ascendente convive con diez univitatias. En el segundo grupo de siete mundos principales, dos mortales moran con cada uno de los grupos nativos de diez miembros y así sucesivamente hasta que, en el último grupo de siete esferas principales, diez seres morontiales cohabitan con diez univitatias. A medida que aprendéis la mejor manera de socializar con los univitatias, más factible os será aplicar dicha ética perfeccionada a vuestra relación con esos progresadores morontiales compañeros vuestros.
\vs p043 8:13 Como mortales ascendentes disfrutaréis de vuestra estancia en los mundos de perfeccionamiento de Edentia, pero no experimentaréis esa sensación de satisfacción personal que caracteriza vuestro primer contacto con los asuntos del universo en la sede del sistema o vuestra despedida final de tales realidades en los últimos mundos de la capital del universo.
\usection{9. CIUDADANÍA EN EDENTIA}
\vs p043 9:1 Tras completar su formación en el mundo número setenta, los mortales ascendentes fijan su residencia en Edentia. Estos ascendentes asisten ahora, por primera vez, a las “asambleas del Paraíso” y oyen la historia de sus dilatadas andaduras narrada por el fiel de días, el primero de los seres personales supremos con origen en la Trinidad que han conocido.
\vs p043 9:2 \pc La totalidad de esta estancia en los mundos formativos de la constelación, que culmina con la ciudadanía en Edentia, constituye un período de verdadera dicha celestial para los progresadores morontiales. A lo largo de vuestra estancia en los mundos del sistema, evolucionasteis desde criaturas casi animales a criaturas morontiales; erais más materiales que espirituales. En las esferas de Lugar de Salvación, evolucionaréis desde seres morontiales al estatus de verdaderos espíritus; seréis más espirituales que materiales. Pero en Edentia, los ascendentes están a medio camino entre su estatus anterior y la postrera, a medio camino en su travesía desde el animal evolutivo al espíritu ascendente. Durante toda vuestra estancia en Edentia y en sus mundos, sois “como los ángeles”; avanzáis continuamente, aunque siempre manteniendo un estatus morontial general característico.
\vs p043 9:3 Esta estancia del mortal ascendente en la constelación es la época de mayor uniformidad y estabilidad de toda su andadura progresiva como ser morontial. Tal experiencia abarca el aprendizaje de socialización de los ascendentes previa a su estatus de espíritus. Es análoga a la experiencia espiritual de los pre\hyp{}finalizadores de Havona y a la formación de los seres pre\hyp{}absonitos en el Paraíso.
\vs p043 9:4 \pc En Edentia, los mortales ascendentes atienden principalmente a las tareas que se les asigna en los setenta mundos de perfeccionamiento de los univitatias. También sirven en distintos puestos en Edentia misma, mayormente en conjunción con el programa de la constelación relativo al bien común del grupo, la raza, la nación y el planeta. Los altísimos no se implican sobremanera en fomentar el avance individual de los habitantes de los mundos; gobiernan más bien en los reinos de los hombres que en el corazón de las criaturas individuales.
\vs p043 9:5 El día en que estéis preparados para partir de Edentia y llevar a cabo vuestra andadura en Lugar de Salvación, os detendréis y rememoraréis una de vuestras épocas más hermosas y reconfortantes de formación a este lado del Paraíso. Pero toda esta gloria se engrandecerá, conforme ascendáis hacia el interior y alcancéis una capacidad en aumento para apreciar de forma creciente los contenidos divinos y los valores espirituales.
\vsetoff
\vs p043 9:6 [Auspiciado por Malavatia Melquisedec.]
