\upaper{39}{Las multitudes seráficas}
\author{Melquisedec}
\vs p039 0:1 Por lo que sabemos, el Espíritu Infinito, tal como se manifiesta personalmente en la sede del universo local, se propone crear serafines uniformemente perfectos, pero, por alguna razón desconocida, estos vástagos seráficos resultan ser muy diversos. Esta diversidad puede ser el resultado de una interposición desconocida de parte de la Deidad experiencial en evolución; si es así, no podemos demostrarlo. Sin embargo, observamos que, cuando los serafines se someten a pruebas educativas y a una disciplina formativa, se clasifican inequívoca y claramente en los siguientes siete grupos:
\vs p039 0:2 \li{1.}Los serafines supremos.
\vs p039 0:3 \li{2.}Los serafines superiores.
\vs p039 0:4 \li{3.}Los serafines supervisores.
\vs p039 0:5 \li{4.}Los serafines gestores.
\vs p039 0:6 \li{5.}Los ayudantes planetarios.
\vs p039 0:7 \li{6.}Los servidores de las criaturas en transición.
\vs p039 0:8 \li{7.}Los serafines del futuro.
\vs p039 0:9 \pc Decir que un serafín de uno de estos grupos es inferior al de cualquier otro grupo no sería decir la verdad. No obstante, cada ángel está solo en un principio, limitado, en su prestación de servicio, al grupo al que pertenece intrínseca y originariamente. Manotia, mi colaborador seráfico en la elaboración de esta exposición, es un serafín supremo que, en otro tiempo, actuaba solamente como tal. Por su perseverancia y dedicación al servicio, ha llevado a cabo, uno a uno, todos los siete tipos de cometidos seráficos, habiendo desempeñado su actividad prácticamente en todos los cauces al alcance de un serafín, y actualmente está destinado como jefe adjunto de serafines en Urantia.
\vs p039 0:10 A los seres humanos les resulta a veces difícil comprender que la facultad con la que se ha creado un ser para ejercer un ministerio de orden superior no implique necesariamente la capacidad de poder ejercerla en niveles relativamente inferiores. El hombre empieza la vida como un pequeño indefenso; de ahí que cualquier logro humano deba abarcar todas las condiciones experienciales previas para conseguirlo; los serafines no tienen existencia previa a su estado de madurez ---no tienen infancia---. Sin embargo, son criaturas experienciales y, a través de la experiencia y mediante una formación complementaria, pueden engrandecer sus facultades innatas y divinas y adquirir vivencialmente la destreza que les capacite para realizar uno o más de los servicios seráficos.
\vs p039 0:11 Tras su nombramiento, se les destina a las reservas de su propio grupo. Aquellos serafines con cometidos planetarios y de gestión de la administración con frecuencia sirven durante largos períodos de acuerdo a su clasificación inicial, pero cuanto más alto sea el nivel de servicio en el que están inherentemente inscritos, con mayor persistencia tratan estos servidores angélicos de que se les asigne a los más modestos tipos de servicio del universo. Desean especialmente que se les destine a las reservas de los ayudantes planetarios y, si tienen éxito, se incorporan a las escuelas celestiales adscritas a la sede del príncipe planetario de algún mundo evolutivo. Aquí comienzan el estudio de los idiomas, la historia y las costumbres locales de las razas de la humanidad. Los serafines han de adquirir conocimiento y experiencia tal como lo hacen los seres humanos. No están muy lejos de vosotros en cuanto a ciertos atributos personales. Todos ellos anhelan comenzar desde abajo en el nivel más inferior posible de su ministerio; de este modo, pueden esperar alcanzar el nivel más elevado posible de destino como seres experienciales.
\usection{1. LOS SERAFINES SUPREMOS}
\vs p039 1:1 De los siete órdenes revelados de ángeles del universo local, este es el de mayor elevación. Estos serafines obran en siete grupos, cada uno de los cuales está estrechamente relacionado con los servidores angélicos del colectivo seráfico de la consumación.
\vs p039 1:2 \li{1.}\bibemph{Los servidores de hijos\hyp{}espíritus}. El primer grupo de serafines supremos está asignado al servicio de los elevados hijos y de seres de origen en el espíritu que residen y realizan su actividad en el universo local. Este grupo de servidores angélicos también sirve al hijo del universo y al espíritu del universo, y se encuentra estrechamente unido al cuerpo de información de la brillante estrella de la mañana, el mandatario en jefe del universo de las voluntades unidas del hijo creador y del espíritu creativo.
\vs p039 1:3 Al estar asignados a los elevados hijos y a los espíritus, estos serafines están de forma natural vinculados a los dilatados servicios que desempeñan los avonales del Paraíso, vástagos divinos del Hijo Eterno y del Espíritu Infinito. En todas sus misiones en rango de magistrado y de gracia, los avonales están siempre asistidos por este orden elevado y experto de serafines, que, en tales ocasiones, se dedica a organizar y dirigir la tarea especial relacionada con la terminación de una dispensación planetaria y la inauguración de una nueva era. Si bien, no se implican en el proceso judicial, que pueda estar relacionado con tal cambio de dispensaciones.
\vs p039 1:4 \pc \bibemph{Los auxiliares de las misiones de gracia}. Cuando los avonales del Paraíso, no los hijos creadores, realizan una misión de gracia siempre están acompañados de un colectivo de 144 de estos auxiliares. Estos 144 ángeles son los jefes de todos los otros servidores de los hijos y de los espíritus que puedan estar relacionados con este tipo de misión. Puede que haya legiones de ángeles bajo el mando de un hijo de Dios encarnado en su ministerio planetario, pero todos estos serafines estarán organizados y dirigidos por los 144 auxiliares de las misiones de gracia. Los órdenes superiores de ángeles, los supernafines y los seconafines, pueden también formar parte de este grupo de acompañantes y, aunque sus cometidos sean distintos a los de los serafines, toda su actividad estará bajo la coordinación de estos auxiliares.
\vs p039 1:5 Los auxiliares de las misiones de gracia son serafines consumados; todos ellos han pasado por los círculos de Lugar de Serafines y han alcanzado el colectivo seráfico de la consumación. Y están especialmente entrenados, además, para hacer frente a las dificultades y abordar las situaciones de emergencia que puedan surgir en relación a los ministerios de gracia de los Hijos de Dios, llevados a efecto para el avance de los hijos del tiempo. Todos estos serafines han alcanzado el Paraíso y logrado el acogimiento personal de la Segunda Fuente y Centro: el Hijo Eterno.
\vs p039 1:6 Los serafines anhelan por igual que se les asigne a las misiones de los hijos encarnados y estar vinculados como guardianes del destino a los mortales de los mundos; esto último constituye su salvoconducto más seguro para el Paraíso, aunque, de entre los consumados serafines destinados a alcanzar el Paraíso, son los auxiliares de las misiones de gracia los que prestan un servicio de mayor relevancia al universo local.
\vs p039 1:7 \li{2.}\bibemph{Los asesores de los tribunales}. Son los asesores y ayudantes seráficos adscritos a todos los órdenes de órganos judiciales desde los conciliadores hasta los más altos tribunales de los mundos. No es el propósito de estos tribunales adoptar sentencias punitivas sino más bien resolver las legítimas diferencias de opinión que se suscitan y decretar la supervivencia eterna de los mortales ascendentes. Aquí radica el deber de los asesores de los tribunales: asegurarse de que todas las imputaciones contra las criaturas mortales se consideren con justicia y se dictaminen con misericordia. En esta tarea están en estrecha colaboración con los altos comisionados, los mortales ascendentes fusionados con el espíritu que prestan sus servicios en el universo local.
\vs p039 1:8 Los asesores seráficos realizan un amplio servicio como defensores de los mortales. No es que haya propensión alguna a tratar de forma injusta a las humildes criaturas de los mundos, pero mientras que la justicia exige el juicio de todas las faltas cometidas en el ascenso hacia la perfección divina, la misericordia dicta que cada uno de estos fallos se considere adecuadamente conforme a la naturaleza de la criatura y al propósito divino. Estos ángeles son los exponentes y la ejemplificación de una misericordia consustancial a la justicia divina ---de una ecuanimidad basada en el conocimiento de los hechos que subyacen en las motivaciones personales y en las tendencias raciales---.
\vs p039 1:9 Este orden de ángeles presta sus servicios desde los consejos de los príncipes planetarios hasta los más altos tribunales del universo local, mientras que sus colaboradores del colectivo seráfico de la consumación obran en los dominios superiores de Orvontón, e incluso en los tribunales de los ancianos de días de Uversa.
\vs p039 1:10 \li{3.}\bibemph{Los orientadores del universo}. Estos son los auténticos amigos y consejeros de los posgraduados mortales, de todas aquellas criaturas ascendentes que se detienen por última vez en Lugar de Salvación, en su universo de origen, y están a punto de emprender la aventura espiritual que se extiende ante ellos en el enorme suprauniverso de Orvontón. En ese momento, muchos de los ascendentes experimentan un sentimiento que los mortales podrían solamente comprender si se comparara con la emoción humana de la nostalgia. Atrás quedan los dominios alcanzados, esos dominios que se han vuelto familiares gracias al prolongado servicio y al logro en el camino morontial; queda por delante el estimulante misterio de un universo más grande y más inmenso.
\vs p039 1:11 Es la tarea de los orientadores del universo facilitar el paso de los peregrinos ascendentes desde el grado de servicio alcanzado en el universo hasta el no alcanzado, ayudar a estos peregrinos a efectuar esas modificaciones, como si de un caleidoscopio se tratase, en la comprensión de los contenidos y de los valores, tomando fundamentalmente conciencia de que un espíritu de la primera etapa no se encuentra ni al final ni en el apogeo de su ascensión morontial en el universo local, sino más bien en la parte más baja de la larga escalera de ascensión espiritual que lleva al Padre Universal del Paraíso.
\vs p039 1:12 Muchos graduados de Lugar de Serafines, miembros del colectivo seráfico de la consumación que están relacionados con estos serafines, se ocupan de amplios programas de enseñanza en ciertas escuelas de Lugar de Salvación, involucradas en la preparación de las criaturas de Nebadón para las interacciones que han de tener lugar en la próxima era del universo.
\vs p039 1:13 \li{4.}\bibemph{Los consejeros de enseñanza}. Estos ángeles son los inestimables asistentes del colectivo de enseñanza espiritual del universo local. Los consejeros de enseñanza son secretarios de todos los órdenes de enseñantes, desde los melquisedecs y los hijos preceptores de la Trinidad hasta los mortales morontiales destinados como ayudantes de aquellos de su especie que se hallan justo detrás de ellos en la escala de la vida ascendente. \bibemph{Veréis} por primera vez a estos serafines, docentes adjuntos, en alguno de los siete mundos de morada que rodean Jerusem.
\vs p039 1:14 Estos serafines se convierten en los colaboradores de los jefes de división de las numerosas instituciones educativas y formativas de los universos locales, y están adscritos en grandes números al profesorado de los siete mundos de formación de los sistemas locales y de las setenta esferas educativas de las constelaciones. Su ministerio se extiende hacia abajo hasta abarcar a los distintos mundos. Incluso los verdaderos y devotos maestros del tiempo reciben la asistencia, y a menudo el acompañamiento, de los consejeros de estos serafines supremos.
\vs p039 1:15 El cuarto ministerio de gracia del hijo creador como criatura fue a semejanza de un consejero de enseñanza de los serafines supremos de Nebadón.
\vs p039 1:16 \li{5.}\bibemph{Los directores por designación}. Periódicamente, los ángeles que prestan su servicio en las esferas evolutivas y arquitectónicas habitadas por las criaturas eligen un cuerpo de 144 serafines supremos que constituye el más alto consejo de cualquier otra esfera. Este consejo coordina las facetas autónomas del servicio y nombramiento seráficos. Estos ángeles presiden todas las asambleas de serafines relacionadas con el cumplimiento del deber o el llamamiento a la adoración.
\vs p039 1:17 \li{6.}\bibemph{Los archivistas}. Estos son los archivistas oficiales de los serafines supremos. Muchos de estos elevados ángeles nacieron con sus facultades plenamente desarrolladas; otros se han capacitado para sus puestos de confianza y responsabilidad mediante su diligente aplicación al estudio y el fiel cumplimiento de deberes similares mientras estaban asignados a órdenes más humildes o menos responsables.
\vs p039 1:18 \li{7.}\bibemph{Los servidores sin adscripción}. Hay un gran número de serafines, pertenecientes a este orden supremo, que carece de adscripción y que presta su servicio de manera autónoma en las esferas arquitectónicas y en los planetas habitados. Estos servidores satisfacen voluntariamente los diferentes requerimientos para la prestación de servicio de los serafines supremos, constituyendo así la reserva general de este orden.
\usection{2. LOS SERAFINES SUPERIORES}
\vs p039 2:1 Los serafines superiores no reciben este nombre porque sean en sentido alguno cualitativamente superiores a los demás órdenes de ángeles, sino porque se encargan de la actividad de mayor prestancia del universo local. Muchos integrantes de los primeros dos grupos de este colectivo seráfico han llegado a él por sus propios logros; son ángeles que han servido en todas las etapas de formación y han regresado para llevar a cabo una glorificada labor como directores de sus congéneres en las esferas donde previamente realizaron sus propias actividades. Siendo un universo joven, Nebadón no tiene muchos miembros de este orden.
\vs p039 2:2 Los serafines superiores desempeñan su actividad en los siguientes siete grupos:
\vs p039 2:3 \li{1.}\bibemph{El cuerpo de información}. Estos serafines pertenecen a los asistentes personales de Gabriel, la brillante estrella de la mañana. Recorren el universo local reuniendo información de los mundos para facilitar la dirección de Gabriel en los consejos de Nebadón. Constituyen el cuerpo de información de las poderosas multitudes que Gabriel preside como vicerregente del hijo mayor. Estos serafines no están directamente vinculados ni con a sistemas ni a las constelaciones, y su información fluye de forma directa a Lugar de Salvación por una vía circulatoria continua, directa e independiente.
\vs p039 2:4 Los cuerpos de información de los distintos universos locales pueden comunicarse entre sí, y de hecho lo hacen, pero solo dentro de un determinado suprauniverso. Existe un diferencial de energía que eficientemente separa los asuntos y actuaciones de los distintos gobiernos de los suprauniversos. Generalmente, un suprauniverso puede comunicarse con otro suprauniverso únicamente a través de las disposiciones e instalaciones del centro de intercambio de información del Paraíso.
\vs p039 2:5 \li{2.}\bibemph{La voz de la misericordia}. La misericordia es la clave del servicio seráfico y del ministerio angélico. Resulta, por tanto, apropiado que haya un colectivo de ángeles que, de forma especial, represente la misericordia. En los universos locales, estos serafines son los que verdaderamente ejercen el ministerio de la misericordia. Son guías inspirados que alientan los deseos superiores y las emociones más sagradas de hombres y ángeles. En la actualidad, los directores de estas legiones de seres son siempre serafines consumados y, a su vez, guardianes graduados de destino de los mortales; o sea, que cada par angélico ha servido de guía por lo menos a un alma de origen animal durante su vida en la carne y que, posteriormente, ha pasado por las vías circulatorias de Lugar de Serafines y se ha incorporado al colectivo seráfico de la consumación.
\vs p039 2:6 \li{3.}\bibemph{Los coordinadores espirituales}. El tercer grupo de serafines superiores tiene su sede en Lugar de Salvación, pero estos actúan en el universo local, en cualquier sitio en el que puedan prestar algún fructífero servicio. A pesar de que sus tareas sean esencialmente espirituales y, por lo tanto, realmente inalcanzables para la mente humana, tal vez podáis comprender algo del ministerio que imparten a los mortales, si se explica que a estos ángeles se les confía la labor de preparar a los ascendentes que residen en Lugar de Salvación para su última transición en el universo local: desde el nivel morontial de mayor grado hasta el estatus de seres espirituales recién nacidos. De igual manera que los planificadores de la mente ayudan en los mundos de las moradas a la criatura superviviente a que se adapten a los potenciales de la mente morontial y los sepa utilizar con eficacia, estos serafines instruyen a los graduados morontiales de Lugar de Salvación acerca de las facultades recién adquiridas de la mente espiritual. También sirven de muchas otras formas a los mortales ascendentes.
\vs p039 2:7 \li{4.}\bibemph{Los maestros auxiliares}. Los maestros auxiliares son ayudantes y colaboradores de sus compañeros serafines, los consejeros de enseñanza. De forma individual, están igualmente implicados en los grandes proyectos educativos del universo local, especialmente en el programa séptuplo de capacitación en vigor en los mundos de las moradas de los sistemas locales. Un magnífico grupo de este orden de serafines presta servicio en Urantia con el propósito de fomentar y hacer avanzar la causa de la verdad y de la rectitud.
\vs p039 2:8 \li{5.}\bibemph{Los transportadores}. Todos los grupos de espíritus servidores tienen sus colectivos de transporte u órdenes angélicas dedicadas a la tarea de transportar de una esfera a otra a aquellos seres personales que no pueden hacerlo por sí mismos. Este quinto grupo de serafines superiores tiene su sede en Lugar de Salvación y desempeña su labor en calidad de surcadores del espacio, yendo y viniendo desde la sede del universo local. Como en el caso de otras categorías de serafines superiores, algunos se crearon así mientras que otros han alcanzado esta condición partiendo de grupos de menor rango o menos dotados.
\vs p039 2:9 \pc La “gama energética” de los serafines es enteramente adecuada para las necesidades del universo local e incluso del suprauniverso, pero no podrían soportar jamás las exigencias energéticas que entraña un viaje tan largo como el de Uversa a Havona. Este viaje tan agotador requiere las facultades especiales de un seconafín primario dotado para este fin. Los transportadores repostan energía para el vuelo mientras están en tránsito y recuperan su energía personal al fin del viaje.
\vs p039 2:10 \pc Los mortales ascendentes no tienen sus propios medios de desplazamiento ni siquiera cuando están en Lugar de Salvación; deben depender del transporte seráfico para avanzar de un mundo a otro hasta después de su última dormición de descanso en el círculo interior de Havona y el despertar eterno en el Paraíso. Luego, ya no dependeréis de los ángeles para trasladaros de un universo a otro.
\vs p039 2:11 El traslado envuelto en un serafín no es muy diferente de la experiencia de la muerte o del sueño, salvo que en la dormición de tránsito hay un elemento temporal necesario. Estáis conscientemente inconscientes durante el descanso seráfico. Pero el modelador del pensamiento está entera y plenamente consciente; de hecho, es especialmente eficiente puesto que no podéis oponeros, resistiros o, de manera alguna, dificultar su labor creativa y transformadora.
\vs p039 2:12 Cuando viajáis en un serafín, os dormís por un período determinado de tiempo y despertáis en el momento indicado. Durante el sueño de tránsito, la duración del trayecto es irrelevante. No sois directamente conscientes del paso del tiempo. Es como si os durmierais en un vehículo de transporte en una ciudad y, tras disfrutar de un sueño tranquilo toda la noche, os despertarais en otra metrópolis lejana. Habéis viajado mientras dormíais. Así pues voláis a través del espacio, envuelto en un serafín, mientras descansáis: dormís. La acción conjunta de los modeladores y los transportadores seráficos es la que induce a este sueño.
\vs p039 2:13 \pc Los ángeles no pueden transportar cuerpos combustibles ---carne y huesos--- como los que tenéis ahora, pero sí pueden transportar todas las otras formas de seres, desde la morontial de menor rango hasta la espiritual de rango mayor. No realizan este servicio en el caso de muerte corporal. Cuando finalizas tu andadura terrenal, tu cuerpo se queda en este planeta. Tu modelador del pensamiento se dirige al seno del Padre, y estos ángeles no se implican directamente en la posterior reconstrucción de tu ser personal en el mundo de morada en el que se procederá a tu identificación. Allí tu nuevo cuerpo es una forma morontial, una forma que sí puede viajar en estos serafines. “Se siembra un cuerpo mortal” en la tumba; y “se cosecha una forma morontial” en los mundos de las moradas.
\vs p039 2:14 \li{6.}\bibemph{Los archivistas}. Estos seres personales se encargan específicamente de la recepción, archivo y reenvío de los registros de Lugar de Salvación y de los mundos vinculados a esta capital. También sirven como archivistas especiales para grupos residentes del suprauniverso y para seres personales de rango superior, y como oficiales de los tribunales de Lugar de Salvación y secretarios de los dirigentes de este.
\vs p039 2:15 \pc \bibemph{Los informadores} ---receptores y emisores--- constituyen un subgrupo especializado dentro de los archivistas seráficos implicados en el envío de documentación y en la diseminación de información de gran importancia. Su labor es de tan elevado orden que procesan multicircuitos que hacen que 144\,000 mensajes puedan atravesar de forma simultánea las mismas líneas de energía. Adaptan las técnicas ideográficas superiores de los archivistas jefes superáficos y, con estos símbolos comunes, mantienen un contacto mutuo tanto con los coordinadores de la información de los supernafines terciarios como con los coordinadores glorificados de la información del colectivo seráfico de la consumación.
\vs p039 2:16 Los archivistas seráficos del orden superior logran, de esta manera, una acción conjunta con los cuerpos de información de su propia clase y con todos los archivistas de menor rango, mientras que las transmisiones les permiten mantener una comunicación constante con los archivistas superiores del suprauniverso y, mediante este canal, con los archivistas de Havona y con los custodios del conocimiento del Paraíso. Muchos de los archivistas de este orden superior son serafines ascendidos que han realizado tareas similares en sectores inferiores del universo.
\vs p039 2:17 \li{7.}\bibemph{Las reservas}. En Lugar de Salvación se mantiene un gran número de reservas de todas las clases de serafines superiores, disponibles de forma instantánea para ser enviados a los mundos más distantes de Nebadón, por requerimiento de los directores por designación o petición de los administradores del universo. Estas reservas también facilitan ayudantes mensajeros por solicitud del jefe de las brillantes estrellas vespertinas, encargado de la custodia y el envío de todas las comunicaciones personales. El universo local dispone, en todos los aspectos, de medios adecuados de intercomunicación, pero siempre queda un resto de mensajes que necesita de mensajeros personales para su envío.
\vs p039 2:18 \pc Las reservas centrales de las que se abastece todo el universo local se localizan en los mundos seráficos de Lugar de Salvación. En este colectivo se incluyen todas las clases de todos los grupos de ángeles.
\usection{3. LOS SERAFINES SUPERVISORES}
\vs p039 3:1 Este versátil orden de ángeles del universo está destinado exclusivamente al servicio de las constelaciones. Estos avezados servidores tienen sus sedes centrales en las capitales de las constelaciones, pero desempeñan su actividad en todo Nebadón, en beneficio de los mundos a los que están destinados.
\vs p039 3:2 \li{1.}\bibemph{Los asistentes supervisores}. El primer orden de serafines supervisores está asignado a la labor colectiva de los padres de las constelaciones, y siempre representan una eficaz ayuda para los altísimos. Estos serafines están mayormente implicados en la unificación y estabilización de toda la constelación.
\vs p039 3:3 \li{2.}\bibemph{Los predictores de la ley}. El fundamento intelectual de la justicia es la ley y, en el universo local, la ley tiene su origen en las asambleas legislativas de las constelaciones. Estos órganos deliberantes codifican y promulgan oficialmente las leyes fundamentales de Nebadón, leyes diseñadas para facilitar la máxima coordinación de toda una constelación en coherencia con la política establecida de no intrusión de la libre voluntad moral de las criaturas personales. El cometido de este segundo orden de serafines supervisores es presentar, ante los legisladores de la constelación, un pronóstico de cómo la adopción de una ley podría afectar a la vida de las criaturas de libre voluntad. Están bien cualificados para desempeñar esta tarea en virtud de su larga experiencia en los sistemas locales y en los mundos habitados. Estos serafines no buscan favorecer de manera especial a uno u otro grupo, pero comparecen ante los legisladores celestiales para hablar por aquellos que no pueden estar presentes para hacerlo por sí mismos. Incluso el hombre mortal es capaz de contribuir al desarrollo de la ley del universo, puesto que estos mismos serafines representan fiel y enteramente, no necesariamente los deseos transitorios y conscientes del hombre, sino más bien los verdaderos anhelos del hombre interior, del alma morontial evolutiva del mortal material de los mundos del espacio.
\vs p039 3:4 \li{3.}\bibemph{Los arquitectos sociales}. Estos serafines obran desde los distintos planetas hasta los mundos morontiales de formación con el fin de mejorar todos los contactos sociales sinceros y de fomentar la evolución social de las criaturas del universo. Son ángeles que tratan de despojar a las relaciones entre los seres inteligentes de cualquier artificialidad, procurando, al mismo tiempo, favorecer la interrelación entre las criaturas de voluntad sobre la base de una comprensión verdadera de sí mismos y de un auténtico aprecio mutuo.
\vs p039 3:5 Los arquitectos sociales hacen todo lo posible, dentro de su competencia y capacidad, para reunir a seres idóneos y constituir, en la tierra, grupos de trabajo eficientes y compatibles; a veces, estos mismos grupos se vuelven a formar en los mundos de las moradas para continuar allí con su fructífero servicio. Si bien, estos serafines no siempre consiguen sus objetivos; no siempre logran reunir a aquellos que podrían formar ese grupo perfecto para conseguir algún determinado propósito o para realizar alguna tarea específica; en estas condiciones, han de hacer uso de las mejores herramientas y medios que tengan a su alcance.
\vs p039 3:6 Estos ángeles continúan su ministerio en los mundos de las moradas y en los mundos morontiales de orden superior. Se implican en toda tarea que tenga que ver con el progreso en los mundos morontiales y que concierna a tres o más personas. Cuando dos seres operan juntos se considera que lo hacen sobre la base de la pareja, el complemento mutuo o la asociación; pero cuando tres o más se agrupan para realizar algún servicio, constituye un reto social y, por lo tanto, es competencia de la jurisdicción de los arquitectos sociales. En Edentia, estos eficaces serafines se dividen en setenta unidades, que prestan sus servicios en los setenta mundos de perfeccionamiento morontial que circundan la esfera sede.
\vs p039 3:7 \li{4.}\bibemph{Los sensibilizadores éticos}. Estos serafines tienen la misión de impulsar y promover en las criaturas el desarrollo de su valoración de la moralidad en las relaciones interpersonales, pues este es el germen y la clave del crecimiento continuado e intencional de la sociedad y del gobierno, tanto humano como sobrehumano. Estos ángeles, facultados para mejorar la apreciación ética de las criaturas, obran en todo lugar en el que puedan desempeñar su labor como consejeros voluntarios de los gobernantes planetarios y como maestros de intercambio en los mundos de formación del sistema. Sin embargo, vosotros no estaréis bajo su total guía hasta que no lleguéis a las escuelas de hermandad de Edentia, donde se avivará vuestro reconocimiento de aquellas mismas verdades sobre la fraternidad, que en ese momento estaréis explorando tan fervorosamente mediante vuestra experiencia real de vivir con los univitatias en los laboratorios sociales de Edentia, los setenta satélites de la capital de Norlatiadec.
\vs p039 3:8 \li{5.}\bibemph{Los transportadores}. El quinto grupo de serafines supervisores opera como transportadores de seres personales, trasladándolos desde y hasta las sedes de las constelaciones. Dichos serafines, mientras vuelan de una esfera a otra, son totalmente conscientes de su velocidad, dirección y paradero astronómico. No cruzan el espacio como lo haría un proyectil inanimado. Pueden pasar uno al lado del otro durante el vuelo espacial sin el menor riesgo de colisión. Son perfectamente capaces de variar la velocidad de marcha y de alterar la dirección del vuelo, e incluso de cambiar el destino, si sus directores así se lo requieren en cualquiera de las intersecciones espaciales de las vías de información del universo.
\vs p039 3:9 Estos seres personales de transporte están organizados de tal manera que pueden utilizar simultáneamente las tres líneas de energía distribuidas por el universo, cada una de las cuales tiene una velocidad espacial neta de 299.788,60 kilómetros por segundo. Estos seres son, por tanto, capaces de superponer la velocidad de la energía sobre la velocidad de la potencia hasta alcanzar, en sus largos desplazamientos, una velocidad media que oscila entre 893.185,92 y 899.623,29 de vuestros kilómetros por segundo de vuestro tiempo. La velocidad se ve afectada por la masa y la proximidad de la materia y por la intensidad y dirección de las principales vías circulatorias cercanas de la potencia del universo. Existen numerosos tipos de seres, similares a los serafines, que son capaces de surcar el espacio e, igualmente, de trasladar a otros seres que hayan sido debidamente preparados.
\vs p039 3:10 \li{6.}\bibemph{Los archivistas}. El sexto orden de serafines supervisores actúa como archivista especial de los asuntos de la constelación. Hay un amplio y eficiente colectivo de estos serafines que desempeña su función en Edentia, la sede de la constelación de Norlatiadec, a la cual pertenece vuestro sistema y vuestro planeta.
\vs p039 3:11 \li{7.}\bibemph{Las reservas}. En las sedes de las constelaciones se mantienen reservas de serafines supervisores. Estos reservistas angélicos no están, en modo alguno, inactivos; muchos de ellos sirven como ayudantes mensajeros para los gobernantes de la constelación. Otros se asignan a las reservas de los vorondadecs sin destino de Lugar de Salvación; e incluso otros se pueden adscribir a los hijos vorondadecs en alguna misión especial, tal como el de observador vorondadec y, a veces, regente altísimo de Urantia.
\usection{4. LOS SERAFINES GESTORES}
\vs p039 4:1 El cuarto orden de serafines está asignado a la gestión de los sistemas locales. Son oriundos de las capitales del sistema pero, en un gran número, están emplazados en las esferas de morada y morontiales y en los mundos habitados. Los serafines de este orden están dotados por naturaleza de una capacidad de gestión poco común. Asisten eficazmente a los directores de las divisiones menores del gobierno del universo de un hijo creador y se ocupan principalmente de los asuntos de los universos locales y de los mundos que los integran. Están organizados para el servicio de la forma siguiente:
\vs p039 4:2 \li{1.}\bibemph{Los asistentes gestores}. Estos hábiles serafines son los asistentes directos del soberano del sistema, de un hijo lanonandec primario. Representan una inapreciable ayuda en la realización de los intricados detalles de la labor ejecutiva de la sede del sistema. También sirven como agentes personales de los gobernantes de este y, en un gran número, van y vienen a los distintos mundos de transición y a los planetas habitados, llevando a cabo numerosos cometidos para el bienestar del sistema y en el interés físico y biológico de los mundos habitados.
\vs p039 4:3 Estos mismos gestores seráficos están igualmente adscritos a los gobiernos de los gobernantes de los mundos o príncipes planetarios. La mayoría de los planetas de cualquier universo está bajo la jurisdicción de un hijo lanonandec secundario, pero en ciertos mundos, como por ejemplo Urantia, el plan divino se malogró por la deserción de su príncipe planetario. En este caso, estos serafines se adscriben a los síndicos melquisedecs y a sus sucesores en la autoridad planetaria. Un colectivo de mil seres de este versátil orden presta asistencia al gobernante actual en funciones de Urantia.
\vs p039 4:4 \li{2.}\bibemph{Los guías de la justicia}. Estos ángeles aportan el sumario de las pruebas pertinentes al eterno bienestar de los hombres y de los ángeles cuando estos asuntos se presentan a juicio en los tribunales de un sistema o de un planeta. Desarrollan los informes para todas las audiencias previas relativas a la supervivencia de los mortales, informes que se presentan con posterioridad, junto con las actas de dichos casos, ante los tribunales superiores del universo y del suprauniverso. Estos serafines preparan la defensa de todos los casos de incierta supervivencia; comprenden perfectamente todos los detalles y características de todos los cargos de las acusaciones emitidas por los administradores de la justicia del universo.
\vs p039 4:5 No es el objetivo de estos ángeles hacer fracasar o demorar la justicia sino más bien asegurar que se aplique, con abundante misericordia y ecuanimidad a todas las criaturas, una justicia sin errores. Estos serafines a menudo desempeñan su labor en los mundos locales, compareciendo normalmente ante los tríos arbitrales de las comisiones conciliadoras ---los tribunales a cargo de desacuerdos de poca importancia---. Muchos de los ángeles que alguna vez sirvieron como guías de la justicia en los mundos más modestos aparecen más adelante como voces de la misericordia en las esferas superiores y en Lugar de Salvación.
\vs p039 4:6 Durante la rebelión de Lucifer en Satania, se perdieron muy pocos guías de la justicia, pero más de un cuarto de los otros serafines gestores y de órdenes de menor rango de servidores seráficos cayeron en el error y en el engaño a causa de los sofismas de una libertad personal desenfrenada.
\vs p039 4:7 \li{3.}\bibemph{Los intérpretes de la ciudadanía cósmica}. Cuando los mortales ascendentes han completado su formación en los mundos de las moradas, su primer aprendizaje como estudiantes de su andadura en el universo, se les permite gozar de las satisfacciones pasajeras de una lograda madurez relativa: su ciudadanía en la capital del sistema. Aunque la consecución de cada una de las metas trazadas para los ascendentes sea una realidad, en un sentido más amplio dichas metas son sencillamente hitos en el largo sendero que el ascendente ha de recorrer hasta llegar al Paraíso. Pero, por muy relativos que sean estos éxitos, a ninguna criatura evolutiva se le niega la plena satisfacción, aunque pasajera, de haber alcanzado tal objetivo. En ocasiones, se hace una pausa en el ascenso al Paraíso, un breve respiro, durante la cual los horizontes del universo permanecen inmóviles, la condición de la criatura está estacionaria y el ser personal saborea la dulzura de haber cumplido una meta.
\vs p039 4:8 En la andadura del ascendente mortal, el primero de estos períodos tiene lugar en la capital de uno de los sistemas locales. Durante esta pausa, vosotros, como ciudadanos de Jerusem, intentaréis expresar en vuestra vida lo conseguido durante esas ocho experiencias vitales que os precedieron, y que corresponden a Urantia y a los siete mundos de las moradas.
\vs p039 4:9 Los intérpretes seráficos de la ciudadanía cósmica guían a los nuevos ciudadanos de las capitales de los sistemas y estimulan su valoración de las responsabilidades del gobierno del universo. Estos serafines también están estrechamente relacionados con los hijos materiales en la administración de los sistemas, mientras representan la responsabilidad y la moralidad de la ciudadanía cósmica para los mortales materiales de los mundos habitados.
\vs p039 4:10 \li{4.}\bibemph{Los acrecentadores de la moral}. En los mundos de las moradas comenzáis a aprender el dominio de vosotros mismos por el bien de todos. Vuestra mente aprende a cooperar, aprende a establecer planes con otros seres de más sabiduría. En la sede del sistema, los maestros seráficos acrecentarán además vuestra valoración de la moral cósmica ---de las interacciones entre la libertad y la lealtad---.
\vs p039 4:11 ¿Qué es la lealtad? Es el fruto de la valoración inteligente de la hermandad del universo; no se puede recibir mucho y dar nada a cambio. A medida que ascendéis en la escala del ser personal, primero aprendéis a ser leales, luego a amar, después a ser filiales y, más adelante, podréis ser libres; pero hasta que no seáis finalizadores, hasta que no hayáis alcanzado perfección de lealtad, no podréis conseguir por vosotros mismos completud de libertad.
\vs p039 4:12 \pc Estos serafines imparten en sus enseñanzas el fruto de la paciencia: que estancarse es la muerte segura, pero que crecer con una rapidez excesiva es igualmente suicida; que al igual que una gota de agua cae desde un nivel superior a otro inferior, fluye adelante, y desciende continuamente por medio de una sucesión de cortas caídas, así es el progresivo ascenso en los mundos morontiales y espirituales ---igual de lento y por medio de similares etapas graduales---.
\vs p039 4:13 Los acrecentadores de la moral representan la vida mortal para los mundos habitados como una cadena ininterrumpida de múltiples eslabones. Vuestra corta estancia en Urantia, en esta esfera lugar de infancia de los mortales, es tan solo uno de estos eslabones, el primero de la larga cadena que ha de extenderse por los universos y a lo largo de las eras eternas. No es tanto lo que aprendéis en esta primera vida; es la experiencia de vivirla lo importante. Incluso la \bibemph{labor} que realizáis en este mundo, aunque primordial, no es ni mucho menos tan importante como la \bibemph{manera} en la que la lleváis a cabo. No existe recompensa material para una vida en rectitud, pero hay una profunda satisfacción ---una conciencia de logro--- y esta trasciende toda recompensa material imaginable.
\vs p039 4:14 Las llaves del reino de los cielos son sinceridad, más sinceridad y más sinceridad. Todos los hombres poseen estas llaves. Los hombres las usan ---avanzan en condición espiritual--- mediante decisiones, más decisiones y más decisiones. La elección moral más elevada es optar por el valor más preeminente posible, y siempre ---en cualquier esfera y en todas ellas--- esto conlleva elegir hacer la voluntad de Dios. Si el hombre lo hace así, \bibemph{es} grande, aunque sea el ciudadano más humilde de Jerusem o incluso el más insignificante de los mortales de Urantia.
\vs p039 4:15 \li{5.}\bibemph{Los transportadores}. Estos son los serafines de transporte que desempeñan su actividad en los sistemas locales. En Satania, vuestro sistema, trasladan a los pasajeros de ida y vuelta a Jerusem y sirven, además, como transportadores interplanetarios. Rara vez pasa un día en el que algún serafín de transporte de Satania no deposite en las orillas de Urantia a un visitante estudiantil o algún otro viajero de naturaleza espiritual o semiespiritual. Ellos mismos, como surcadores del espacio, os llevarán y traerán algún día a los distintos mundos del grupo planetario de la sede del sistema y, cuando hayáis concluido vuestro destino en Jerusem, os trasladarán más allá, a Edentia. Pero bajo ninguna circunstancia os llevarán de vuelta a vuestro mundo de origen. Un mortal jamás regresa a su planeta nativo durante la dispensación en la que tuvo su existencia temporal y, si lo hiciese durante una dispensación posterior, lo haría acompañado de un serafín de transporte de servicio en el grupo planetario de la sede del universo.
\vs p039 4:16 \li{6.}\bibemph{Los archivistas}. Estos serafines son los custodios de los registros de los sistemas locales, los cuales se conservan por triplicado. El templo de los archivos está situado en la capital del sistema; se trata de una singular estructura: un tercio es material y está construida por metales y cristales luminosos; un tercio es morontial y está fabricada conjuntamente de energía espiritual y material, pero fuera del rango de la visión humana; y un tercio es espiritual. Los archivistas de este orden dirigen y mantienen este sistema triple de archivos. Los mortales ascendentes, al principio, realizan consultas de los archivos materiales; los hijos materiales y los seres de transición de orden superior lo hacen de aquellos archivos de las salas morontiales; mientras que los serafines y los seres personales espirituales superiores de los mundos examinan los archivos de la sección espiritual.
\vs p039 4:17 \li{7.}\bibemph{Las reservas}. El colectivo de reserva de los serafines gestores de Jerusem pasa una gran parte de su tiempo de espera relacionándose, en camaradería espiritual, con los mortales ascendentes recién llegados desde los distintos mundos del sistema ---graduados acreditados de los mundos de las moradas---. Uno de los placeres de vuestra estancia en Jerusem consistirá en conversar y relacionaros, durante los períodos de receso, con estos serafines del colectivo de reserva en espera, que tanto han viajado y tanta experiencia han adquirido.
\vs p039 4:18 Son concretamente estas relaciones de amistad las que tanto hacen a los ascendentes mortales encariñarse por la capital de un sistema. En Jerusem, veréis como, por vez primera, los hijos materiales, ángeles y peregrinos ascendentes se interrelacionan. Aquí fraternizan seres enteramente espirituales y semiespirituales y seres individuales que acaban de emerger de la existencia material. Las formas mortales se han modificado tanto y el campo de la reacción humana hacia la luz se ha engrandecido tanto, que todos podrán disfrutar del reconocimiento mutuo y del entendimiento y comprensión de la persona del otro.
\usection{5. LOS AYUDANTES PLANETARIOS}
\vs p039 5:1 Estos serafines mantienen sus sedes en las capitales de los sistemas y, aunque estrechamente relacionados con los ciudadanos adánicos que allí residen, están asignados principalmente al servicio de los adanes planetarios, los mejoradores biológicos o físicos de las razas materiales de los mundos evolutivos. La labor servicial de los ángeles adquiere un mayor interés conforme se extiende a los mundos habitados, conforme se acerca a los problemas reales que afrontan los hombres y mujeres del tiempo, que se preparan para intentar alcanzar la meta de la eternidad.
\vs p039 5:2 En Urantia, la mayoría de los ayudantes planetarios se retiraron tras la caída del régimen adánico, y la supervisión seráfica de vuestro mundo recayó en mayor medida sobre los gestores, los servidores de las criaturas en transición y los guardianes del destino. Si bien, los siguientes grupos de asistentes seráficos de vuestros hijos materiales transgresores aún prestan servicio a Urantia. Estos son:
\vs p039 5:3 \li{1.}\bibemph{Las voces del jardín}. Cuando el curso planetario de la evolución humana está alcanzando su nivel biológico más elevado, siempre aparecen los hijos e hijas materiales, los adanes y las evas, para fomentar el avance de la evolución de las razas contribuyendo materialmente con su plasma vital de orden superior. Por lo general, a la sede planetaria de dichos adanes y evas se la denomina el Jardín del Edén, y a sus serafines personales a menudo se les conoce como las “voces del jardín”. Estos serafines prestan un inestimable servicio a los adanes planetarios en todos sus proyectos destinados a elevar física e intelectualmente a las razas evolutivas. Después de la transgresión adánica en Urantia, algunos de estos serafines permanecieron en el planeta y se asignaron a los sucesores en autoridad de Adán.
\vs p039 5:4 \li{2.}\bibemph{Los espíritus de la fraternidad}. Es evidente que, cuando un adán y una eva llegan a un mundo evolutivo, la tarea de conseguir la armonía racial y la cooperación social entre sus diversas razas es de considerables proporciones. Es raro que estas razas de diferentes colores y de naturalezas distintas acepten con facilidad un plan que busque la fraternidad humana. Estos hombres primitivos solo llegan a percatarse de la sensatez de la interrelación pacífica al madurar con la experiencia humana y por medio del fiel ministerio de los espíritus seráficos de la fraternidad. Sin la labor de estos serafines, la tarea de los hijos materiales de armonizar y hacer avanzar a las razas de un mundo en evolución se dilataría enormemente. Si vuestro Adán se hubiese adherido al plan original trazado para el avance de Urantia, estos espíritus de la fraternidad, llegado este punto, habrían realizado transformaciones increíbles en la raza humana. Teniendo en cuenta la trasgresión adánica, es, de hecho, extraordinario que dichos órdenes seráficos hayan sido capaces de fomentar y propiciar incluso el grado de fraternidad del que contáis ahora en Urantia.
\vs p039 5:5 \li{3.}\bibemph{Las almas de la paz}. Muchas son las luchas que marcan el afán por superarse del hombre evolutivo durante los primeros milenios. La paz no es el estado natural de los mundos materiales. Los mundos llegan a darse cuenta de “la paz en la tierra y buena voluntad entre los hombres” mediante el ministerio de las almas seráficas de la paz. Aunque en Urantia los primeros esfuerzos de estos ángeles se vieron en gran parte frustrados, Vevona, jefe de las almas de la paz en los días de Adán, permaneció en Urantia y, en este momento, está adscrito al grupo de asistentes del gobernador general residente. Fue el mismo Vevona quien, como dirigente de las multitudes angélicas, cuando nació Miguel, anunció a los mundos: “Gloria a Dios en Havona y en la tierra paz y buena voluntad entre los hombres”.
\vs p039 5:6 En las épocas de mayor avance de la evolución planetaria, estos serafines contribuyen decisivamente a reemplazar la idea de la expiación por el concepto de sintonía con lo divino como filosofía de la supervivencia de los mortales.
\vs p039 5:7 \li{4.}\bibemph{Los espíritus de la confianza}. La desconfianza es una reacción innata en el hombre primitivo; las luchas por la supervivencia de las primeras eras no traen consigo la confianza de forma natural. La confianza es algo nuevo que adquieren los seres humanos gracias al ministerio de estos serafines planetarios del régimen adánico. La misión de estos ángeles consiste en infundir confianza en la mente del hombre evolutivo. Los Dioses son muy confiados; el Padre Universal está dispuesto a confiarse a sí mismo sin reservas ---el modelador--- al vincularse con el hombre.
\vs p039 5:8 Tras el malogro adánico, se destinó a todo este grupo de serafines al nuevo régimen y, desde entonces, continúan con su labor en Urantia. Y sus esfuerzos no están siendo del todo infructuosos, puesto que se está desarrollando en este momento una civilización que recoge muchos de sus ideales de una verdadera confianza.
\vs p039 5:9 En las eras planetarias más avanzadas, estos serafines amplían la comprensión humana de la verdad de que la incertidumbre es fuente de continua satisfacción. Ayudan a los filósofos mortales a percatarse de que, cuando la ignorancia es esencial para tener éxito, sería un error descomunal para la criatura conocer el futuro. Estos serafines realzan la inclinación del hombre hacia la dulzura de la incertidumbre, hacia el romanticismo y el encanto de un futuro indeterminado y desconocido.
\vs p039 5:10 \li{5.}\bibemph{Los transportadores}. Los transportadores planetarios prestan sus servicios en los distintos planetas. La mayoría de los seres que viajan en serafín y llegan a este planeta están en tránsito; hacen simplemente una escala; están bajo la custodia de sus propios transportadores seráficos personales; si bien, existe un gran número de estos serafines emplazados en Urantia. Estos son los seres personales de transporte que operan desde los planetas locales, como por ejemplo desde Urantia hasta Jerusem.
\vs p039 5:11 \pc La idea tradicional que tenéis de los ángeles se ha formado de la siguiente manera: durante momentos, justo antes de la muerte física, a veces se produce en la mente humana un fenómeno reflectante, y esta conciencia que se va debilitando cree visualizar algo con la figura del ángel que nos custodia, y esto se traduce, de inmediato, en términos de la imagen habitual que la mente de esa persona tiene formada de los ángeles.
\vs p039 5:12 La idea equivocada de que los ángeles poseen alas no proviene totalmente de nociones antiguas de que debían tener alas para poder volar por el aire. A los seres humanos algunas veces se les ha permitido observar a los serafines cuando se les está preparando para realizar su servicio de transporte, y los relatos de estas experiencias han dado pie en gran parte a la idea que se tiene en Urantia de los ángeles. Al observarlos así, en preparación para acoger a un pasajero para su traslado interplanetario, es posible que se vea lo que son aparentemente conjuntos dobles de alas que se extienden desde la cabeza hasta los pies del ángel. En realidad estas alas son aislantes de la energía: escudos contra la fricción.
\vs p039 5:13 \pc Cuando los seres celestiales se trasladan de un mundo a otro en un serafín, se les lleva a la sede de la esfera y, tras identificarse de forma conveniente, se les induce al sueño de tránsito. Entretanto, el serafín de transporte se coloca en posición horizontal justo por encima del polo energético del universo relativo al planeta. Mientras los escudos de energía están completamente abiertos, los asistentes seráficos, en el ejercicio de sus funciones, colocan hábilmente a la persona dormida directamente encima del ángel transportador. Entonces, los dos pares de escudos, tanto los superiores como los inferiores, se cierran y se ajustan cuidadosamente.
\vs p039 5:14 Tras ello, bajo la acción de los transformadores y de los transmisores, comienza una extraña metamorfosis a medida que el serafín se dispone a desplazarse hacia las corrientes energéticas de las vías circulatorias del universo. En su apariencia exterior, el serafín se alarga en ambos extremos y se cubre en tal grado de una rara luz de tonalidad ámbar que muy pronto se hace imposible percibir a la persona que viaja en él. Cuando todo está preparado para la salida, el jefe de transportes realiza la apropiada inspección de este medio de conducción de la vida, lleva a cabo las pruebas rutinarias para determinar si el ángel está adecuadamente conectado a las vías circulatorias y, a continuación, anuncia que el viajero está debidamente envuelto en el serafín, que las energías están reguladas, que el ángel está aislado de la fricción y que todo está listo para su destellante partida. Dos de los controladores mecánicos ocupan luego sus puestos. Llegado este momento, el serafín de transporte ha adquirido una silueta casi transparente, vibrante, como la forma de un torpedo de refulgente luminosidad. En este momento, el expedidor de transportes del mundo convoca a los grupos auxiliares de transmisores de energía viva, generalmente mil de ellos; al anunciar el destino del transporte, toca el punto más cercano del vehículo seráfico, que sale disparado veloz como relámpago, dejando una estela de luminosidad celestial hasta donde se extiende la capa atmosférica planetaria. En menos de diez minutos, este maravilloso espectáculo se desvanece incluso ante la grandiosa visión de los serafines.
\vs p039 5:15 \pc Aunque los informes planetarios espaciales se reciben al mediodía en el meridiano de la sede espiritual indicada, los transportadores se envían a medianoche, desde este mismo lugar. Ese es el momento más favorable para partir y es la hora común de hacerlo siempre que no se especifique lo contrario.
\vs p039 5:16 \li{6.}\bibemph{Los archivistas}. Estos son los custodios de los asuntos principales del planeta en su función como parte del sistema y en relación y correspondencia al gobierno del universo. Registran los asuntos planetarios, pero no se ocupan de los asuntos de la vida y la existencia de los seres de forma individual.
\vs p039 5:17 \li{7.}\bibemph{Las reservas}. El colectivo de reserva de serafines planetarios de Satania se mantiene en Jerusem en estrecha relación con las reservas de los hijos materiales. Estas cuantiosas reservas aseguran plenamente la realización de las múltiples facetas de este orden seráfico. Estos ángeles son también los portadores de los mensajes personales provenientes de los sistemas locales. Sirven a los mortales en transición, a los ángeles y a los hijos materiales al igual que a otros seres con domicilio en la sede del sistema. Aunque Urantia está actualmente fuera de las vías circulatorias espirituales de Satania y de Norlatiadec, vosotros sois conocedores directos de los asuntos interplanetarios, puesto que estos mensajeros de Jerusem acuden con frecuencia a este mundo a la vez que a todas las demás esferas del sistema.
\usection{6. LOS SERVIDORES DE LAS CRIATURAS EN TRANSICIÓN}
\vs p039 6:1 Como su nombre indica, los serafines que desempeñan este ministerio sirven dondequiera que puedan contribuir a la transición de la criatura desde el estado material al espiritual. Estos ángeles realizan su labor desde los mundos habitados hasta las capitales de los sistemas, pero los que lo hacen actualmente en Satania orientan sus mayores esfuerzos hacia la educación de los mortales supervivientes de los siete mundos de las moradas. Este ministerio varía conforme a las siete clases de tareas a las que están asignados:
\vs p039 6:2 \li{1.}Los evangelistas seráficos.
\vs p039 6:3 \li{2.}Los intérpretes de las razas.
\vs p039 6:4 \li{3.}Los planificadores de la mente.
\vs p039 6:5 \li{4.}Los asesores morontiales.
\vs p039 6:6 \li{5.}Los técnicos.
\vs p039 6:7 \li{6.}Los archivistas\hyp{}maestros.
\vs p039 6:8 \li{7.}El colectivo de reserva de los servidores.
\vs p039 6:9 \pc Aprenderéis más acerca de estos servidores seráficos de los ascendentes en transición cuando se haga el relato de los mundos de las moradas y de la vida morontial.
\usection{7. LOS SERAFINES DEL FUTURO}
\vs p039 7:1 Estos ángeles no ejercen su amplio ministerio salvo en los mundos de mayor antigüedad y en los planetas más avanzados de Nebadón. Un gran número de ellos se mantiene en reserva en los mundos seráficos cercanos a Lugar de Salvación, donde desarrollan actividades pertinentes al nacimiento de la era de luz y vida que ha de tener lugar en Nebadón. Estos serafines se ocupan de hecho de la andadura de los mortales ascendentes, pero atienden casi exclusivamente a aquellos mortales que sobreviven siguiendo alguno de los órdenes modificados de ascensión.
\vs p039 7:2 Puesto que estos ángeles no están directamente relacionados ni con Urantia ni con los urantianos, consideramos preferible poner límite a la descripción de sus fascinantes actividades.
\usection{8. EL DESTINO DE LOS SERAFINES}
\vs p039 8:1 Los serafines tienen su origen en los universos locales y, en estos mismos dominios en los que nacieron, algunos alcanzan el servicio futuro al que están destinados. Con la ayuda y el consejo de los arcángeles de mayor rango y experiencia, algunos serafines pueden ascender en su labor y realizar el elevado cometido de las brillantes estrellas vespertinas, mientras que otros consiguen el estatus y ejercen el servicio de los coiguales no revelados de las estrellas vespertinas. Hay también otros trepidantes destinos a su alcance en el universo local, pero Lugar de Serafines constituye siempre la eterna meta de todos los ángeles. Lugar de Serafines significa para ellos llegar al umbral del Paraíso y alcanzar la Deidad; es la esfera que marca la transición entre el ministerio del tiempo y el glorioso servicio de la eternidad.
\vs p039 8:2 \pc Los serafines pueden alcanzar el Paraíso en numerosas ---cientos--- de maneras, pero las más importantes son las que se narran a continuación:
\vs p039 8:3 \li{1.}Obtener la admisión en la morada seráfica del Paraíso a título personal por su perfección de servicio, al haberlo prestado de forma especializada como artesano celestial, asesor técnico o archivista celestial. Convertirse en acompañante del Paraíso y habiendo llegado, por consiguiente, al centro de todas las cosas, convertirse tal vez entonces en eterno servidor y asesor de los órdenes seráficos y al igual que de otros órdenes de seres.
\vs p039 8:4 \li{2.}Ser convocado a Lugar de Serafines. Bajo ciertas condiciones se emplaza a los serafines a comparecer en las alturas; en otras circunstancias, los ángeles a veces logran el Paraíso en un período de tiempo mucho más corto que los mortales. Pero por muy capacitada que esté la pareja seráfica, no puede dar comienzo a su partida hacia Lugar de Serafines ni hacia ningún otro lugar. Únicamente los guardianes del destino que han tenido éxito tienen la seguridad de continuar hacia el Paraíso, siguiendo el camino de progreso que conlleva el ascenso evolutivo. Todos los demás deben esperar con paciencia la llegada de los mensajeros del Paraíso de los supernafines terciarios que traen la convocatoria emplazándolos a hacer acto de presencia en las alturas.
\vs p039 8:5 \li{3.}Alcanzar el Paraíso por el método evolutivo de los mortales. En su trayectoria en el tiempo, los serafines pueden optar por la decisión suprema de tomar el puesto de ángel guardián con el fin de completar su andadura final y cualificarse para su destino como servidores seráficos en las esferas eternas. Estos guías personales de los hijos del tiempo se llaman guardianes del destino, lo que significa que guardan a las criaturas mortales en su camino hacia el destino divino y, con ello, establecen su propio elevado destino.
\vs p039 8:6 Los guardianes del destino se seleccionan de entre los seres personales angélicos más experimentados de todos los órdenes de serafines que se han capacitado para este servicio. Todos los mortales supervivientes destinados a fusionarse con el modelador tienen asignados guardianes temporales y pueden llegar a vincularse a ellos de forma permanente cuando dichos mortales alcanzan el necesario desarrollo intelectual y espiritual. Antes de que los ascendentes mortales dejen los mundos de las moradas, todos tienen sus acompañantes seráficos permanentes. Nos aproximaremos a este grupo de espíritus servidores cuando lleguemos a las narraciones que tratan de Urantia.
\vs p039 8:7 \pc No les es posible a los ángeles alcanzar a Dios desde el nivel que parten los humanos, porque son creados “algo superiores a vosotros”; pero con sabiduría se ha dispuesto que, aunque no puedan comenzar desde el mismo fondo, desde las modestas tierras espirituales de la existencia mortal, pueden descender hasta aquellos que ciertamente comienzan allí su andadura y guiar a estas criaturas, paso a paso, mundo a mundo, hasta las puertas de Havona. Cuando los mortales ascendentes salen de Uversa para empezar en los círculos de Havona, aquellos guardianes, que habían estado unidos a ellos tras su vida en la carne, se despiden temporalmente de sus acompañantes peregrinos y viajan a Lugar de Serafines, el destino de los ángeles en el gran universo. Estos guardianes intentarán allí, e indudablemente lograrán, alcanzar los siete círculos de la luz seráfica.
\vs p039 8:8 Aunque no todos, hay muchos de estos serafines designados como guardianes del destino durante la vida material que acompañan a sus allegados mortales a través de los círculos de Havona, y hay algunos otros que pasan por los círculos del universo central de un modo completamente diferente al que siguen los ascendentes mortales. Pero con independencia de la ruta de ascensión que se tome, todos los serafines evolutivos atraviesan Lugar de Serafines, y la mayoría pasa por esa experiencia en lugar de recorrer las vías circulatorias de Havona.
\vs p039 8:9 \pc Para los ángeles, Lugar de Serafines es la esfera a la que están destinados; si bien, la experiencia que tienen al lograr ese mundo difiere de la de los peregrinos mortales cuando llegan a Lugar de la Ascensión. Los ángeles no están completamente seguros de su futuro eterno hasta que no han llegado a Lugar de Serafines. No se sabe de ningún ángel que haya alcanzado Lugar de Serafines y se haya descarriado; el pecado nunca hallará respuesta en el corazón de un serafín consumado.
\vs p039 8:10 Los graduados de Lugar de Serafines asumen distintos tipos de misiones: los guardianes del destino con experiencia en los círculos de Havona normalmente entran a formar parte del colectivo de los finalizadores mortales. Otros guardianes, habiendo pasado sus pruebas de separación de Havona, con frecuencia se reencuentran en el Paraíso con sus allegados mortales, y algunos se convierten en los acompañantes sempiternos de los finalizadores mortales, mientras que otros ingresan en los distintos colectivos de finalizadores no mortales y muchos se incorporan al colectivo seráfico de la consumación.
\usection{9. EL COLECTIVO SERÁFICO DE LA CONSUMACIÓN}
\vs p039 9:1 Tras conseguir llegar al Padre de los espíritus y ser admitidos en el servicio seráfico de la consumación, a los ángeles, a veces, se les encomienda el ministerio de los mundos que se han asentado en luz y vida. Logran adscribirse a los elevados seres trinitizados de los universos y al glorioso servicio del Paraíso y de Havona. Experiencialmente, estos serafines de los universos locales han compensado la diferencia de potencial divino que anteriormente les separaba de los espíritus servidores del universo central y de los suprauniversos. Los ángeles del colectivo seráfico de la consumación sirven como colaboradores de los seconafines del suprauniverso y como asistentes de los altos órdenes de supernafines del Paraíso\hyp{}Havona. Para estos ángeles, la andadura del tiempo ha terminado; de ahí en adelante, y para siempre, son los siervos de Dios, los consortes de los seres personales divinos y los compañeros de los finalizadores del Paraíso.
\vs p039 9:2 Un gran número de serafines consumados regresa a sus universos nativos para complementar el ministerio de dotación divina con el de perfección experiencial. Nebadón es, en términos comparativos, uno de los universos más jóvenes y, por lo tanto, no tiene tantos graduados retornados de Lugar de Serafines como sucede en otros universos de mayor antigüedad; no obstante, nuestro universo local dispone de un apropiado número de serafines consumados y es, en efecto, significativo que los dominios evolutivos muestren una creciente necesidad de esta asistencia a medida que se van aproximando a la condición de luz y vida. Actualmente, estos serafines sirven más ampliamente al lado de los órdenes supremos de serafines, pero algunos lo hacen al lado de cada uno de los demás órdenes angélicos. Hasta vuestro mundo disfruta del amplio ministerio de doce grupos especializados de dicho colectivo seráfico de la consumación; estos serafines mayores que atienden la supervisión planetaria acompañan a cada uno de los príncipes planetarios destinado a los mundos habitados.
\vs p039 9:3 A los serafines consumados se les abren muchas fascinantes vías de acción, pero al igual que todos ellos anhelaban ser designados como guardianes del destino en los días anteriores a su consecución del Paraíso, en su experiencia posterior a dicho logro, desean fervientemente servir como acompañantes, en sus misiones de gracia, a los hijos encarnados del Paraíso. Siguen estando dedicados supremamente a ese plan universal de poner en marcha a las criaturas mortales de los mundos evolutivos en el largo y tentador viaje hacia el Paraíso, meta divina y eterna. A lo largo de toda la aventura de los mortales por encontrar a Dios y alcanzar la perfección divina, estos servidores espirituales consumados, junto con los fieles espíritus servidores del tiempo, son y serán para siempre, unos auténticos amigos, cuya ayuda nunca os faltará.
\vsetoff
\vs p039 9:4 [Exposición de un melquisedec que actúa a instancias del jefe de las multitudes seráficas de Nebadón.]
