\upaper{36}{Los portadores de vida}
\author{Hijo vorondadec}
\vs p036 0:1 La vida no se origina espontáneamente. La vida se forma según los planes formulados por los (no revelados) arquitectos del ser y aparece en los planetas habitados por importación directa o como consecuencia de las intervenciones de los portadores de vida de los universos locales. Estos portadores de vida se cuentan entre los seres más interesantes y versátiles de la diversa familia de hijos del universo. A ellos se les ha confiado diseñar y llevar la vida a las esferas planetarias. Y una vez que han implantado la vida en estos mundos nuevos, permanecen allí durante largos períodos de tiempo para impulsar su desarrollo.
\usection{1. ORIGEN Y NATURALEZA DE LOS PORTADORES DE VIDA}
\vs p036 1:1 Aunque los portadores de vida pertenecen a la familia de filiación divina, constituyen un tipo de hijos del universo peculiar y diferenciado; es el único grupo de vida inteligente del universo local en cuya creación participan los gobernantes del suprauniverso. Los portadores de vida son los vástagos de tres seres personales preexistentes: el hijo creador, el espíritu materno del universo y, por designación, uno de los tres ancianos de días que presiden los destinos del suprauniverso correspondiente. Estos ancianos de días, que son los únicos que pueden decretar la extinción de la vida inteligente, participan en la creación de los portadores de vida, a quienes se les confía el establecimiento de la vida física en los mundos evolutivos.
\vs p036 1:2 En el universo de Nebadón tenemos constancia de la creación de cien millones de portadores de vida. Este eficiente colectivo encargado de diseminar la vida no es un grupo que verdaderamente se gobierne a sí mismo. Los dirige el trío que determina la vida, integrado por Gabriel, el padre Melquisedec y Nambia, el portador de vida primigenio y primogénito de Nebadón; si bien, son autónomos en todas las facetas de sus divisiones administrativas.
\vs p036 1:3 Los portadores de vida se clasifican en tres grandes categorías: la primera corresponde a los portadores de mayor rango; la segunda, a los asistentes; y, la tercera, a los custodios. La primera categoría se subdivide a su vez en doce grupos de portadores que son especialistas en las distintas formas de manifestación de la vida. La separación de estas tres categorías se llevó a cabo por parte de los melquisedecs, que para tal propósito realizaron pruebas en la esfera sede de los portadores de vida. Los melquisedecs están, desde entonces, estrechamente vinculados a estos seres y siempre los acompañan cuando salen para establecer la vida en un planeta nuevo.
\vs p036 1:4 Cuando un planeta evolutivo finalmente se asienta en luz y vida, los portadores se organizan en órganos deliberantes supremos con capacidad consultiva a fin de ayudar a fomentar la administración y el desarrollo del mundo y de sus seres glorificados. En esas eras futuras de estabilización de un universo evolutivo a estos portadores de vida se les confían muchos cometidos nuevos.
\usection{2. LOS MUNDOS DE LOS PORTADORES DE VIDA}
\vs p036 2:1 Los melquisedecs ejercen la supervisión general del cuarto grupo de siete esferas primarias de la vía circulatoria de Lugar de Salvación. A estos mundos de los portadores de vida se les denomina de la siguiente manera:
\vs p036 2:2 \li{1.}La sede de los portadores de vida.
\vs p036 2:3 \li{2.}La esfera de planificación de la vida.
\vs p036 2:4 \li{3.}La esfera de conservación de la vida.
\vs p036 2:5 \li{4.}La esfera de la evolución de la vida.
\vs p036 2:6 \li{5.}La esfera de la vida vinculada con la mente.
\vs p036 2:7 \li{6.}La esfera de la mente y del espíritu en los seres vivos.
\vs p036 2:8 \li{7.}La esfera de la vida no revelada.
\vs p036 2:9 \pc Cada una de estas esferas primarias está rodeada por seis satélites, en los que se centran las facetas especiales de la actividad que los portadores de vida realizan en el universo.
\vs p036 2:10 \pc \bibemph{El mundo número uno,} la esfera sede, junto con sus seis satélites dependientes, está dedicado al estudio de la vida universal, esto es, de la vida en todas sus manifestaciones conocidas. Aquí se sitúa la facultad de planificación de la vida, en la que maestros y asesores de Uversa y Havona, e incluso del Paraíso, ejercen su labor. Me está permitido revelar que en este mundo de los portadores de vida se hallan los siete emplazamientos centrales de los espíritus asistentes de la mente.
\vs p036 2:11 El número diez ---el sistema decimal--- es consustancial al universo físico, aunque no al espiritual. El ámbito de la vida se caracteriza por tres, siete y doce o por múltiplos y combinaciones de estos números básicos. Existen tres planificaciones de la vida que son primordiales y esencialmente diferentes, según el orden de las tres Fuentes y Centros del Paraíso, y, en el universo de Nebadón, estas tres formas elementales de la vida se dividen en tres tipos diferentes de planetas. Había, originariamente, doce conceptos divinos distintos de la vida transmisible. Este número doce, con sus divisores y múltiplos, está presente en todos los modelos básicos de la vida de la totalidad de los siete suprauniversos. También hay siete tipos arquitecturales en relación al diseño de la vida, o distribuciones fundamentales de la configuración reproductiva de la materia viva. Los modelos de la vida de Orvontón se configuran en doce portadores de la herencia. Los distintos órdenes de criaturas volitivas se configuran según los números 12, 24, 48, 96, 192, 384 y 768. En Urantia existen cuarenta y ocho unidades rectoras del modelo de la vida ---o rasgos determinantes--- en las células sexuales de la reproducción humana.
\vs p036 2:12 \pc \bibemph{El segundo mundo} es la esfera donde se diseña la vida. Aquí se determinan todas sus formas nuevas de organización. Aunque los diseños originales de la vida parten del hijo creador, son los portadores de vida y sus colaboradores quienes llevan realmente a efecto estos planes. Una vez que se establece el plan general de vida para un nuevo mundo, este se transmite a la esfera sede, donde el consejo supremo de los portadores de vida de mayor rango, en colaboración con un colectivo de asesores melquisedecs, lo analiza minuciosamente. Si este plan se desvía de las especificaciones previamente aceptadas, debe someterse al estudio y aprobación del hijo creador. Con frecuencia, el jefe de los melquisedecs representa al hijo creador en estas deliberaciones.
\vs p036 2:13 Por tanto, aunque la vida planetaria sea similar en ciertos aspectos, difiere de muchas maneras en cada uno de los mundos evolutivos. Incluso en una simple familia de mundos que comparten sucesivamente uniformidad de vida, esta no es exactamente idéntica en dos planetas; existe siempre algún tipo de vida planetario que los distingue. Esto es así porque los portadores de vida laboran constantemente para mejorar las fórmulas vitales asignadas a su cuidado.
\vs p036 2:14 Hay más de un millón de fórmulas químicas fundamentales o cósmicas que componen los modelos parentales y las numerosas variaciones funcionales básicas de las manifestaciones de la vida. El satélite número uno de la esfera de la planificación de la vida es el campo de acción de los físicos y de los electroquímicos del universo, que sirven como asistentes técnicos de los portadores de vida en la labor de capturar, organizar y actuar sobre las unidades esenciales de energía empleadas para construir los vehículos materiales de transmisión de la vida, el llamado plasma germinal.
\vs p036 2:15 Los laboratorios de planificación de la vida planetaria están situados en el segundo satélite de este mundo número dos. En dichos laboratorios, los portadores de vida y todos sus colaboradores cooperan con los melquisedecs a fin de modificar y posiblemente mejorar la vida concebida para su implantación en los planetas decimales de Nebadón. La vida que evoluciona actualmente en Urantia se planificó y se desarrolló en parte en este mismo mundo, ya que Urantia es un \bibemph{planeta decimal,} un mundo de experimentación de la vida. En un mundo de cada diez se permite que los diseños normales de la vida sufran una variación mayor que en los otros mundos (no experimentales).
\vs p036 2:16 \pc \bibemph{El mundo número tres} está dedicado a la conservación de la vida. Aquí los asistentes y los custodios del colectivo de portadores de vida estudian y desarrollan distintos modos de protegerla y preservarla. En la planificación de la vida para cualquier mundo nuevo, siempre se prevé el rápido establecimiento de la comisión de conservación de la vida, integrada por custodios especialistas que saben manipular con destreza los modelos básicos de la vida. En Urantia había veinticuatro de estos custodios, miembros de la comisión, dos por cada modelo fundamental o parental de la organización arquitectural de la vida material. En planetas como el vuestro, la forma más elevada de vida se reproduce mediante un conjunto portador de la vida que consta de veinticuatro unidades modelo. (Y puesto que la vida intelectual se deriva de la vida física, y se funda en ella, de ahí tienen su existencia los veinticuatro órdenes elementales característicos de la organización psíquica).
\vs p036 2:17 \pc \bibemph{La esfera número cuatro} y sus satélites dependientes están dedicados al estudio de la evolución de la vida creatural en general y de los antecedentes evolutivos de cada uno de los niveles de vida en particular. El plasma de vida original de un mundo evolutivo debe contener el pleno potencial de todas las posibles variaciones que se puedan desarrollar y de todos los cambios evolutivos y modificaciones que se puedan producir con posterioridad. Los requisitos de estos proyectos de tan largo alcance en relación a la metamorfosis de la vida pueden exigir la aparición de muchas formas aparentemente inútiles de vida animal y vegetal. Estos subproductos de la evolución planetaria, previstos o imprevistos, aparecen en el campo de acción solamente para desaparecer; si bien, en todo este largo proceso subyacen las acertadas e inteligentes formulaciones de los diseñadores primigenios que planificaron la vida planetaria y el esquema de las especies. Los múltiples subproductos de la evolución biológica son en su totalidad esenciales para el funcionamiento último e íntegro de las formas superiores de vida inteligente, a pesar de que periódicamente pueda predominar una gran disonancia externa en la larga lucha ascendente de las criaturas superiores por lograr el dominio sobre las formas de vida de inferior rango, muchas de las cuales resultan a veces tan hostiles para la paz y la complacencia de las criaturas volitivas evolutivas.
\vs p036 2:18 \pc \bibemph{El mundo número cinco} se ocupa en su totalidad de la vida en vinculación con la mente. Cada uno de sus satélites está dedicado al estudio de una única faceta de la mente de las criaturas en su correlación con la vida creatural. La mente, tal como el hombre la comprende, es un don de los siete espíritus asistentes de la mente que las instancias intermedias del Espíritu Infinito superponen a los niveles mecánicos o no educables de la mente. Los modelos de vida responden de distintas maneras a estos asistentes y a los distintos ministerios espirituales que operan en todos los universos del tiempo y del espacio. La capacidad de respuesta espiritual de las criaturas materiales depende enteramente de la dote de la mente a ellas vinculada, algo que, a su vez, ha determinado el curso de la evolución biológica de estas mismas criaturas mortales.
\vs p036 2:19 \pc \bibemph{El mundo número seis} está dedicado a la correlación de la mente con el espíritu tal como se vinculan con las formas y los organismos vivos. Este mundo y sus seis mundos dependientes contienen las escuelas de coordinación creatural, donde los maestros del universo central y del suprauniverso colaboran con los instructores de Nebadón en la exposición de los más elevados niveles de logro alcanzables por las criaturas en el tiempo y el espacio.
\vs p036 2:20 \pc \bibemph{La séptima esfera} de los portadores de vida está dedicada a los ámbitos no revelados de la vida de las criaturas evolutivas en relación con la filosofía cósmica de la creciente efectuación del Ser Supremo.
\usection{3. TRASPLANTE DE LA VIDA}
\vs p036 3:1 La vida no aparece de forma espontánea en los universos; los portadores de vida han de iniciarla en planetas estériles. Ellos son los portadores, diseminadores y custodios de la vida tal como aparece en los mundos evolutivos del espacio. Toda vida conocida en Urantia, cualquiera que sea su orden o forma, hace su aparición con estos hijos, aunque no todas las formas de vida planetaria existen en Urantia.
\vs p036 3:2 El colectivo de portadores encargado de implantar la vida en un nuevo mundo consta generalmente de cien portadores de mayor rango, cien asistentes y mil custodios. Con frecuencia, aunque no siempre, los portadores traen específicamente el plasma de vida a un nuevo mundo. A veces, organizan los modelos de la vida tras llegar al planeta de destino, siguiendo directrices previamente aprobadas, a fin de dar comienzo a una nueva aventura en el establecimiento de la misma. Así fue el origen de la vida planetaria en Urantia.
\vs p036 3:3 Una vez que se conforman los modelos físicos según dichas directrices, los portadores de vida catalizan este material inanimado e imparten a través de sus personas la chispa vital del espíritu e, inmediatamente, los modelos inertes se convierten en materia viva.
\vs p036 3:4 \pc La chispa vital ---el misterio de la vida--- se otorga a través de los portadores de vida, no por ellos. Estos ciertamente supervisan esos procesos y formulan el plasma de la vida misma, pero es el espíritu materno del universo quien proporciona el factor esencial del plasma vivo. De la hija creativa del Espíritu Infinito emerge esa chispa de energía que vivifica el cuerpo y augura la llegada de la mente.
\vs p036 3:5 \pc En la dádiva de la vida, los portadores no transmiten nada de su naturaleza personal, ni siquiera en aquellas esferas en las que se proyectan nuevos órdenes de vida. En estas ocasiones, sencillamente inician y transmiten la chispa de la vida, dando comienzo a las mociones de rotación necesarias de la materia según las especificaciones físicas, químicas y eléctricas de las planificaciones y modelos previstos. Los portadores de vida son presencias catalíticas vivas que agitan, organizan y vitalizan los elementos de orden material que en otras circunstancias estarían inertes.
\vs p036 3:6 \pc Para establecer la vida en un nuevo mundo, el colectivo planetario de portadores dispone de un plazo determinado de aproximadamente medio millón de años del tiempo de ese planeta. Al final de este período, indicado por la consecución de ciertos logros en el desarrollo de la vida planetaria, dan por concluida esta tarea y, de ahí en adelante, ya no les es posible añadir nada nuevo o suplementario a la vida de ese planeta.
\vs p036 3:7 Durante las épocas que median entre el establecimiento de la vida y la gradual aparición de criaturas humanas de estatus moral, a los portadores se les permite actuar sobre el entorno de la vida además de dirigir favorablemente el curso de la evolución biológica. Así lo hacen durante largos períodos de tiempo.
\vs p036 3:8 La labor de los portadores de vida que operan en un nuevo mundo acaba en el momento en el que consiguen dar nacimiento a un ser de voluntad, con capacidad de decisión moral y de elección espiritual. Ya han terminado. Ya no pueden actuar más sobre la vida en evolución. De ahí en adelante, el desarrollo de los seres vivos debe continuar conforme a los recursos propios de la naturaleza y las tendencias impartidas, y establecidas, en las fórmulas y modelos de la vida planetaria. A los portadores de vida no se les permite ni experimentar ni interferir con la voluntad. No pueden a su arbitrio ejercer ningún dominio ni influencia sobre las criaturas morales.
\vs p036 3:9 A la llegada del príncipe planetario se preparan para partir, aunque dos de los portadores de mayor rango y experiencia y doce custodios pueden ofrecerse como voluntarios, haciendo votos temporales de renuncia, para permanecer indefinidamente en el planeta como asesores en lo referente al posterior desarrollo y conservación del plasma de vida. Dos de esos hijos y sus doce colaboradores prestan en la actualidad sus servicios en Urantia.
\usection{4. LOS PORTADORES DE VIDA MELQUISEDECS}
\vs p036 4:1 En cada uno de los sistemas locales de mundos habitados de todo Nebadón, hay una única esfera en la que los melquisedecs desempeñan funciones como portadores de vida. Estas moradas se conocen como los mundos \bibemph{midsonitas} de los sistemas y, en cada uno de ellos, un hijo melquisedec, materialmente modificado, elige y se une a una de las hijas pertenecientes al orden material de filiación. Las madres evas de estos mundos midsonitas acuden procedentes de la sede del sistema en jurisdicción una vez que el portador de vida melquisedec, designado para esta labor, las escoge de entre las numerosas voluntarias que responden a la llamada que el soberano del sistema ha realizado a las hijas materiales de su ámbito.
\vs p036 4:2 A la progenie de un portador de vida melquisedec y de una hija material se la denomina \bibemph{midsonitas}. El melquisedec, padre de esta raza de excelsas criaturas, termina por dejar el planeta en el que realiza esta excepcional función de vida y, Eva, madre de este orden especial de seres del universo, se marcha igualmente de allí cuando la séptima generación de sus vástagos planetarios hace su aparición. La dirección de dicho mundo se delega entonces en su hijo primogénito.
\vs p036 4:3 Las criaturas midsonitas viven y actúan como seres reproductivos en sus magníficos mundos hasta que cumplen mil años de edad de tiempo estándar; después de lo cual se les traslada mediante transporte seráfico. A partir de entonces los midsonitas pierden su capacidad de reproducción, porque la técnica de desmaterialización por la que pasan en preparación para tal tipo de transporte en el que viajan envueltos en un serafín los priva para siempre de esta prerrogativa.
\vs p036 4:4 En su estatus actual, estos seres no pueden considerarse ni mortales ni inmortales ni se les puede clasificar de forma categórica como humanos o divinos. Estas criaturas no tienen modelador interior, por lo tanto, es difícil suponer que sean inmortales. Pero tampoco parecen ser mortales; ningún midsonita ha experimentado la muerte. Todos los midsonitas nacidos en Nebadón siguen vivos hoy en día, ejercen su actividad en sus mundos nativos, en alguna esfera intermedia o en la esfera midsonita de Lugar de Salvación, en el grupo de mundos de los finalizadores.
\vs p036 4:5 \pc \bibemph{Los mundos de los finalizadores de Lugar de Salvación}. Los portadores de vida melquisedecs, así como las madres evas a ellos vinculadas, van desde las esferas midsonitas del sistema a los mundos de los finalizadores de la vía circulatoria de Lugar de Salvación, donde su vástagos están igualmente destinados a reunirse.
\vs p036 4:6 Se debe explicar a este respecto que el quinto grupo de siete mundos primarios de la vía circulatoria de Lugar de Salvación corresponde a los mundos de los finalizadores de Nebadón. Los hijos de los portadores de vida melquisedecs y las hijas materiales tienen su residencia en el séptimo mundo de los finalizadores, la esfera midsonita de Lugar de Salvación.
\vs p036 4:7 Los satélites de los siete mundos primarios de los finalizadores son el lugar de reunión de los seres personales de los suprauniversos y del universo central que puedan estar prestando sus servicios en Nebadón. Aunque los mortales ascendentes circulan libremente por todos los mundos culturales y esferas de formación de los 490 mundos que componen la Universidad Melquisedec, hay ciertas escuelas especiales y numerosas zonas restringidas a las que no se les permite el acceso. Esto es particularmente cierto de las cuarenta y nueve esferas bajo la jurisdicción de los finalizadores.
\vs p036 4:8 \pc Actualmente no se conoce el propósito que mueve a estas criaturas midsonitas, pero parece que estos seres personales se están reuniendo en el séptimo mundo de los finalizadores en preparación para cualquier eventualidad futura respecto a la evolución del universo. Siempre dirigimos nuestras cuestiones sobre las razas midsonitas a los finalizadores, y siempre estos declinan hablar del destino de sus pupilos. A pesar de nuestra incertidumbre con respecto al futuro de estas criaturas, sí sabemos que cada universo local en Orvontón alberga un colectivo en aumento de estos misteriosos seres. Los portadores de vida melquisedecs creen que, a sus hijos midsonitas, el Dios Último los dotará algún día del espíritu trascendental y eterno de la absonitidad.
\usection{5. LOS SIETE ESPÍRITUS ASISTENTES DE LA MENTE}
\vs p036 5:1 Es la presencia de los siete espíritus asistentes de la mente en los mundos primitivos la que determina el curso de la evolución orgánica; eso explica por qué la evolución tiene un propósito definido y no casual. Estos asistentes representan ese ministerio de la mente del Espíritu Infinito, que se extiende a los órdenes más modestos de vida inteligente mediante la actuación del espíritu materno del universo local. Los asistentes son los hijos del espíritu materno del universo y constituyen el ministerio personal de este espíritu materno a las mentes materiales de los mundos. Estos espíritus obran de diversas maneras dependiendo del momento o lugar en el que se manifieste este tipo de mente.
\vs p036 5:2 Los siete espíritus asistentes de la mente responden a nombres que equivalen a los siguientes apelativos: intuición, entendimiento, valentía, conocimiento, consejo, adoración y sabiduría. Estos espíritus de la mente expanden su influencia a todos los mundos habitados siguiendo un impulso diferenciado, cada cual buscando capacidad receptiva para manifestarse, con independencia del grado en el que sus compañeros puedan encontrar receptividad y oportunidad para operar.
\vs p036 5:3 Los alojamientos centrales de los espíritus asistentes, situados en el mundo sede de los portadores de vida, indican a los supervisores de estos portadores el alcance y la calidad de la actividad mental de los asistentes en cualquier mundo y en cualquier organismo determinado dotado de intelecto. Estos emplazamientos de vida y mente son indicadores perfectos de la actividad de la mente viva de los primeros cinco asistentes. No obstante, con respecto a los espíritus asistentes sexto y séptimo ---adoración y sabiduría---, estos alojamientos solo muestran el aspecto cualitativo de su actividad. El aspecto cuantitativo de la actividad del asistente de la adoración y del de la sabiduría se registra en la inmediata presencia de la benefactora divina en Lugar de Salvación; es una experiencia personal de este espíritu materno del universo.
\vs p036 5:4 \pc Los siete espíritus asistentes de la mente siempre acompañan a los portadores de vida a un nuevo planeta, pero no deben considerarse como entidades; son más bien vías circulatorias. Los espíritus de los siete asistentes del universo no actúan como seres personales a no ser en la presencia de la benefactora divina en el universo; son, de hecho, un nivel de la conciencia de la benefactora divina y están siempre subordinados a la acción y presencia de su madre creativa.
\vs p036 5:5 No encontramos las palabras adecuadas para designar a estos siete espíritus asistentes de la mente. Realizan su servicio en los niveles inferiores de la mente experiencial, y se les puede describir, siguiendo la secuencia de sus logros evolutivos, como sigue:
\vs p036 5:6 \li{1.}\bibemph{El espíritu de intuición:} percepción rápida, inherentes instintos primitivos físicos y reflejos, dotación del sentido de la dirección y de otros sentidos de preservación de todas las creaciones dotadas de mente; el único de los asistentes que obra, en gran medida, en los órdenes inferiores de la vida animal y el único que tiene un amplio contacto de carácter operativo con los niveles no educables de la mente maquinal.
\vs p036 5:7 \li{2.}\bibemph{El espíritu de entendimiento:} impulso de coordinación, asociación de ideas espontánea y aparentemente automática. Se trata del don de coordinación del conocimiento adquirido, del fenómeno de razonamiento rápido, juicio rápido y prontitud de decisión.
\vs p036 5:8 \li{3.}\bibemph{El espíritu de valentía,} o don de la fidelidad, en los seres personales, la base de la adquisición del carácter y la raíz intelectual del vigor moral y de la valentía espiritual. Cuando está iluminado por los hechos, e inspirado por la verdad, este don se convierte en el secreto del impulso de la ascensión evolutiva por la vía de la autodirección inteligente y concienzuda.
\vs p036 5:9 \li{4.}\bibemph{El espíritu de conocimiento:} la curiosidad, madre de la aventura y del descubrimiento, el espíritu científico; guía y fiel colaborador de los espíritus de valentía y de consejo; el impulso para dirigir los dones de la valentía por las sendas del crecimiento útil y progresivo.
\vs p036 5:10 \li{5.}\bibemph{El espíritu de consejo:} el impulso hacia lo social, el sentido de cooperación del que están dotadas las especies; la facultad de las criaturas volitivas para relacionarse con los demás, el origen del instinto gregario entre las humildes criaturas.
\vs p036 5:11 \li{6.}\bibemph{El espíritu de adoración:} el afán por lo religioso, el primer impulso diferenciador que separa a las criaturas de mente en dos clases fundamentales de mortales. El espíritu de adoración marca para siempre la distinción entre el animal al que está vinculado y aquellas criaturas sin alma, dotadas de mente. La adoración es el distintivo que nos permite aspirar a la ascensión espiritual.
\vs p036 5:12 \li{7.}\bibemph{El espíritu de sabiduría:} la inherente tendencia de todas las criaturas morales hacia el avance evolutivo metódico y progresivo. Este es el más elevado de los asistentes, el espíritu que coordina y articula la labor de todos los demás. Este espíritu es el secreto de ese impulso innato de las criaturas dotadas de mente, que inicia y mantiene el plan práctico y efectivo de la escala ascendente de la existencia; ese don de los seres vivos que da razón de su inexplicable capacidad para sobrevivir y ejercitar, en la supervivencia, la coordinación de todas sus experiencias pasadas y oportunidades presentes a fin de adquirir, en su totalidad, lo que los otros seis servidores de la mente pueden movilizar en la mente del organismo correspondiente. La sabiduría es la cima de la realización intelectual; la sabiduría es el objetivo de una existencia puramente mental y moral.
\vs p036 5:13 \pc Los espíritus asistentes de la mente crecen en experiencia, pero nunca se vuelven personales. Evolucionan en cuanto a su función y, la función de los primeros cinco en los órdenes animales es, hasta cierto punto, esencial, para la de los siete como intelecto humano. Esta relación con el orden animal hace que los asistentes sean, desde el punto de vista práctico, más efectivos en su función como mente humana; así pues, los animales son, en cierto modo, indispensables tanto para la evolución intelectual como física del hombre.
\vs p036 5:14 Estos asistentes de la mente del espíritu materno del universo local guardan relación con la vida de las criaturas inteligentes de una forma similar a la relación de los centros de la potencia y los controladores físicos con las fuerzas no vivas del universo. En los mundos habitados, los asistentes realizan un servicio inestimable en las vías circulatorias de la mente y colaboran eficazmente con los controladores físicos mayores, que también sirven como rectores y directores de los niveles mentales que preceden a los asistentes, esto es, los niveles no educables o mecánicos de la mente.
\vs p036 5:15 La mente viva, antes de la aparición de la capacidad de aprender de la experiencia, es competencia del servicio que realizan los controladores físicos mayores. La mente creada, antes de adquirir la facultad de reconocer la divinidad y adorar a la Deidad, es competencia exclusiva de los espíritus asistentes. Con la aparición de la respuesta espiritual del intelecto de la criatura, estas mentes creadas se tornan de inmediato en supramentes y se encauzan en el acto en la vía circulatoria de los ciclos espirituales del espíritu materno del universo local.
\vs p036 5:16 Los espíritus asistentes de la mente no están directamente relacionados de manera alguna con la labor diversa y sumamente espiritual del espíritu santo de los mundos habitados, que constituye la presencia personal de la benefactora divina; pero son operativamente precursores, y sirven de preparación, a la aparición de este mismo espíritu en el hombre evolutivo. Los asistentes proporcionan al espíritu materno del universo un contacto variado con las criaturas materiales vivas del universo local al igual que un influjo directivo sobre las mismas, pero cuando actúan en niveles prepersonales no tienen repercusión en el Ser Supremo.
\vs p036 5:17 \pc Una mente no espiritual es o bien una manifestación de espíritu\hyp{}energía o bien un fenómeno de energía física. Incluso la mente humana, la mente personal, no puede poseer cualidades de supervivencia a no ser que se identifique con el espíritu. La mente es un don de naturaleza divina, pero no es inmortal cuando actúa sin percepción espiritual y cuando está desprovista de la capacidad para adorar y para ansiar la supervivencia.
\usection{6. LAS FUERZAS VIVAS}
\vs p036 6:1 La vida es a la vez mecanicista y vitalista ---material y espiritual---. Los físicos y los químicos de Urantia siempre podrán progresar en su comprensión de las formas protoplásmicas de la vida vegetal y animal, pero nunca podrán dar origen a organismos vivos. La vida es algo que difiere de todas las manifestaciones de energía; incluso la vida material de las criaturas físicas no es algo inherente a la materia.
\vs p036 6:2 Las cosas materiales pueden tener una existencia independiente, pero la vida surge tan solo de la vida. La mente solamente puede derivarse de una mente preexistente. El espíritu solo puede proceder de antecedentes espirituales. La criatura puede dar origen a las formas de la vida, pero solamente un ser personal creador o una fuerza creativa pueden proporcionar la chispa activadora de la vida.
\vs p036 6:3 Los portadores de vida pueden organizar las formas materiales, o modelos físicos, de los seres vivos, pero el espíritu proporciona la chispa inicial de la vida y otorga el don de la mente. Hasta incluso las formas vivas de vida experimental que los portadores organizan, en sus mundos de Lugar de Salvación, están siempre desprovistas de poderes reproductivos. Cuando las fórmulas de la vida y los modelos vitales están correctamente ensamblados y adecuadamente organizados, la presencia de un portador es suficiente para iniciar la vida, pero todos estos organismos vivos que resultan carecen de dos atributos esenciales: del don de la mente y de poderes reproductivos. La mente animal y la mente humana son dones del espíritu materno del universo local, que obra por mediación de los siete espíritus asistentes de la mente, mientras que la capacidad de reproducción de la criatura es concesión expresa y personal del espíritu del universo al plasma ancestral de la vida establecido por los portadores.
\vs p036 6:4 \pc Una vez que los portadores han diseñado los modelos de la vida, tras haber organizado los sistemas de energía, se ha de producir un nuevo fenómeno; se ha de impartir “el aliento de vida” a estas formas sin vida. Los Hijos de Dios construyen las formas de la vida, pero es el Espíritu de Dios quien realmente contribuye con la chispa vital. Y cuando la vida que se ha conferido se extingue, entonces el cuerpo material que queda se vuelve materia no viva. Cuando se agota esa vida otorgada, el cuerpo retorna al seno del universo material de donde los portadores de vida lo tomaron prestado, a fin de servir como vehículo transitorio para ese don de vida que transmitieron a esa visible conjunción de energía\hyp{}materia.
\vs p036 6:5 La vida que los portadores otorgan a las plantas y a los animales no regresa a los portadores tras la muerte de la planta o del animal. La vida que sale de estos seres vivos no posee ni identidad ni ser personal; no sobrevive individualmente a la muerte. Durante su existencia y el tiempo de estancia en el cuerpo material, ha sufrido un cambio; ha experimentado la evolución de la energía y sobrevive solamente como parte de las fuerzas cósmicas del universo; no sobrevive como vida individual. La supervivencia de las criaturas mortales está basada enteramente en la evolución de un alma inmortal dentro de la mente mortal.
\vs p036 6:6 \pc Hablamos de la vida como “energía” y como “fuerza”, pero en realidad no es ninguna de las dos. La fuerza\hyp{}energía es sensible a la gravedad de distintas formas; la vida no lo es. El modelo tampoco es sensible a la gravedad, al ser una configuración de energías que ya ha pagado su tributo respecto a su respuesta a la gravedad. La vida, como tal, constituye la animación de un sistema de energía ---material, mental o espiritual--- ya sea configurado en un modelo o separado de alguna manera.
\vs p036 6:7 \pc Hay algunas cosas relacionadas con la elaboración de la vida en los planetas evolutivos que no nos resultan del todo claras. Comprendemos plenamente la organización física de las fórmulas electroquímicas de los portadores de vida, pero no entendemos por completo la naturaleza y fuente de la \bibemph{chispa activadora de la vida}. Sabemos que la vida fluye del Padre a través del Hijo y \bibemph{mediante} el Espíritu. Es más que posible que los espíritus mayores sean el canal séptuplo del río de vida que se derrama sobre toda la creación. Pero no comprendemos el modo en el que el espíritu mayor supervisor participa en el episodio inicial de la concesión de la vida en un nuevo planeta. Estamos seguros de que los ancianos de días participan también en esta inauguración de la vida en un nuevo mundo, pero ignoramos por completo la naturaleza de esta participación. Sabemos de cierto que el espíritu materno del universo realmente vitaliza los modelos sin vida e imparte en ese plasma activado las prerrogativas de la reproducción orgánica. Observamos que ellos tres, a veces denominados creadores supremos del tiempo y el espacio, constituyen los niveles del Dios Séptuplo; pero, por lo demás, sabemos poco más al respecto de lo que saben los mortales de Urantia ---simplemente que el concepto es inherente al Padre, expresión en el Hijo y realización de vida en el Espíritu---.
\vsetoff
\vs p036 6:8 [Redactado por un hijo vorondadec emplazado en Urantia como observador, que desempeña esta función a petición del jefe melquisedec del colectivo de reveladores encargado de la supervisión.]
