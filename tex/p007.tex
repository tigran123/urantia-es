\upaper{7}{La relación del Hijo Eterno con el universo}
\author{Consejero divino}
\vs p007 0:1 El Hijo Primigenio siempre se ocupa de dar cumplimiento a los aspectos espirituales relacionados con el eterno propósito del Padre, a medida que este se desarrolla progresivamente en los fenómenos que tienen lugar en los universos evolutivos con su multitud de grupos de seres vivos. No comprendemos del todo este plan eterno, pero el Hijo del Paraíso sin duda lo comprende.
\vs p007 0:2 El Hijo es semejante al Padre en el hecho de que procura dar todo lo posible de sí mismo a sus hijos, iguales en rango entre sí, y a los hijos subordinados a estos. El Hijo también participa de la naturaleza autodistributiva del Padre al otorgarse sin límites al Espíritu Infinito, mandatario conjunto a ellos.
\vs p007 0:3 \pc Como sostenedora de las realidades espirituales, la Segunda Fuente y Centro es el eterno contrapeso de la Isla del Paraíso, que tan magníficamente sostiene todas las cosas materiales. De este modo, la Primera Fuente y Centro se revela eternamente en la belleza material de los espléndidos modelos de la Isla central y en los valores espirituales del excelso ser personal del Hijo Eterno.
\vs p007 0:4 \pc El Hijo Eterno es el auténtico sostenedor de la inmensa creación de realidades espirituales y de seres espirituales. El mundo espiritual constituye la manera de conducirse, el ámbito de la actuación personal, del Hijo, y las realidades impersonales de naturaleza espiritual siempre responden a la voluntad y al propósito del Hijo Absoluto, del ser personal perfecto.
\vs p007 0:5 Sin embargo, el Hijo no es personalmente responsable del proceder de todos los seres personales espirituales. La voluntad de una criatura personal es relativamente libre y, por consiguiente, determina las acciones de los seres volitivos. Así pues, el mundo espiritual, en su libre voluntad, no siempre representa verdaderamente el carácter del Hijo Eterno, al igual que la naturaleza en Urantia no revela en verdad la perfección e inmutabilidad del Paraíso y de la Deidad. Si bien, con independencia de lo que pueda caracterizar la libre actuación de un hombre o de un ángel, el dominio que mantiene el Hijo sobre la gravedad universal de todas las realidades espirituales continúa siendo absoluto.
\usection{1. LA VÍA CIRCULATORIA DE LA GRAVEDAD ESPIRITUAL}
\vs p007 1:1 Todo lo impartido concerniente a la inmanencia de Dios, a su omnipresencia, omnipotencia y omnisciencia, es igualmente cierto del Hijo en los ámbitos espirituales. La gravedad espiritual pura y universal de toda la creación, esa vía por donde circula exclusivamente lo espiritual, conduce directamente de vuelta a la persona de la Segunda Fuente y Centro en el Paraíso. Él, en su autoridad, rige y efectúa ese dominio espiritual, infalible y siempre presente, de todos los verdaderos valores espirituales. De este modo, el Hijo Eterno ejerce una soberanía espiritual absoluta. Literalmente, él sostiene todas las realidades espirituales y todos los valores espiritualizados, por así decirlo, en el hueco de su mano. La potestad sobre la gravedad espiritual universal \bibemph{constituye} la soberanía espiritual universal.
\vs p007 1:2 Esta potestad sobre la gravedad de las cosas espirituales actúa con independencia del tiempo y del espacio; por consiguiente, la energía espiritual no disminuye al transmitirse. La gravedad espiritual nunca experimenta dilaciones de tiempo ni disminuciones de espacio. No decrece de acuerdo con el cuadrado de la distancia de su transmisión; la masa de la creación material no demora las vías circulatorias de la potencia espiritual pura. Y esta trascendencia sobre el tiempo y el espacio de las energías del espíritu puro es consustancial a la absolutidad del Hijo; no se debe a la interposición de las fuerzas de la antigravedad de la Tercera Fuente y Centro.
\vs p007 1:3 Las realidades espirituales responden al poder de atracción del centro de gravedad espiritual de acuerdo a su valor cualitativo, a su grado real de naturaleza espiritual. La sustancia espiritual (cualidad) responde a la gravedad espiritual como la energía sistematizada de la materia física (cantidad) responde a la gravedad física. Los valores espirituales y las fuerzas espirituales \bibemph{son} reales. Desde el punto de vista del ser personal, el espíritu es el alma de la creación; la materia, el cuerpo físico, es la sombra.
\vs p007 1:4 Las reacciones y fluctuaciones de la gravedad espiritual están siempre en consonancia con el contenido de los valores espirituales, con el estatus espiritual cualitativo de un ser individual o de un mundo. Este poder de atracción responde, de forma instantánea, a los valores interespirituales e intraespirituales de cualquier situación en el universo o condición planetaria. Cada vez que una realidad espiritual se actualiza en los universos, su cambio necesita del inmediato e instantáneo reajuste de la gravedad espiritual. Ese nuevo espíritu forma realmente parte de la Segunda Fuente y Centro y, ciertamente, el hombre mortal llega al Hijo espiritual, a la fuente y centro de la gravedad espiritual, a medida que se espiritualiza.
\vs p007 1:5 \pc El poder de atracción espiritual del Hijo es consustancial, en menor grado, a muchos órdenes de filiación del Paraíso. Porque de hecho existen, dentro de la vía absoluta de circulación de la gravedad espiritual, unos sistemas locales de atracción espiritual que obran en unidades menores de la creación. Estas convergencias subabsolutas de dicha gravedad espiritual forman parte en el tiempo y el espacio de la divinidad de los seres personales creadores, y se correlacionan con la acción directiva experiencial emergente del Ser Supremo.
\vs p007 1:6 La fuerza de gravedad espiritual y la respuesta a ella actúan no solo en el universo de manera global, sino incluso entre seres y grupos de seres. Existe una cohesión espiritual entre los seres personales espirituales y espiritualizados de cualquier mundo, raza, nación o grupo de creyentes. Hay una atracción directa de naturaleza espiritual entre personas de mentalidad espiritual con los mismos gustos y deseos. El término \bibemph{espíritus afines} no es una figura totalmente retórica.
\vs p007 1:7 \pc Al igual que la gravedad material del Paraíso, la gravedad espiritual del Hijo Eterno es absoluta. El pecado y la rebelión pueden interferir en la utilización de las vías circulatorias de los universos locales, pero nada puede hacer cesar la gravedad espiritual del Hijo Eterno. La rebelión de Lucifer produjo muchos cambios en Urantia y en vuestro sistema de mundos habitados, si bien, no hemos notado que la cuarentena espiritual que se produjo como resultado en vuestro planeta afectara, en lo más mínimo, a la presencia y labor ni del espíritu omnipresente del Hijo Eterno ni de la vía gravitatoria espiritual a él vinculada.
\vs p007 1:8 \pc Se pueden predecir todas las reacciones de la vía circulatoria de la gravedad espiritual del gran universo. Conocemos todas las acciones y reacciones del espíritu omnipresente del Hijo Eterno y las encontramos constantes. Podemos medir y medimos, conforme a leyes bien conocidas, la gravedad espiritual de la misma manera que el hombre realiza el cálculo del efecto de la gravedad física finita. El espíritu del Hijo responde de forma invariable a todas las personas, seres y cosas espirituales, y esta respuesta está siempre de acuerdo con el grado de actualidad (el grado cualitativo de realidad) de todos esos valores espirituales.
\vs p007 1:9 Pero precisamente junto a esta labor tan segura y predecible de la presencia espiritual del Hijo Eterno, se dan fenómenos cuyas reacciones no son tan predecibles. Estos fenómenos probablemente indican la acción correlacionada del Absoluto de la Deidad en los ámbitos de los potenciales espirituales emergentes. Sabemos que la presencia espiritual del Hijo Eterno constituye la influencia de un ser personal majestuoso e infinito, pero consideramos que no son del todo personales las reacciones relacionadas con la presumible actuación del Absoluto de la Deidad.
\vs p007 1:10 \pc Si se mira desde la perspectiva del ser personal y por las personas, el Hijo Eterno y el Absoluto de la Deidad parecen estar relacionados de la siguiente manera: el Hijo Eterno domina en el reino de los valores espirituales actuales, mientras que el Absoluto de la Deidad parece difundirse por el inmenso ámbito de los valores espirituales potenciales. Todo valor actual de naturaleza espiritual encuentra cabida en el dominio gravitatorio del Hijo Eterno, pero, si es potencial, entonces aparentemente la encuentra en la presencia del Absoluto de la Deidad.
\vs p007 1:11 El espíritu parece emerger de los potenciales del Absoluto de la Deidad; el espíritu evolutivo encuentra su correlación en los dominios experienciales e incompletos del Supremo y del Último; el espíritu halla eventualmente su destino final en el dominio absoluto de la gravedad espiritual del Hijo Eterno. Este parece ser el ciclo del espíritu experiencial, pero el espíritu existencial es inherente a la infinitud de la Segunda Fuente y Centro.
\usection{2. EL GOBIERNO DEL HIJO ETERNO}
\vs p007 2:1 En el Paraíso, la presencia y la actividad personal del Hijo Primigenio es profunda, es absoluta en un sentido espiritual. Cuando salimos del Paraíso, a través de Havona, y nos adentramos en el ámbito de los siete suprauniversos, percibimos cada vez menos la actividad personal del Hijo Eterno. En los universos posteriores a Havona, la presencia del Hijo Eterno se hace personal en los hijos del Paraíso, está condicionada por las realidades experienciales del Supremo y del Último y se coordina con el ilimitado potencial espiritual del Absoluto de la Deidad.
\vs p007 2:2 En el universo central, la actividad personal del Hijo Primigenio se discierne en la excelente armonía espiritual de la creación eterna. Havona es tan maravillosamente perfecta que el estatus espiritual y los estados energéticos de este universo modelo están en perfecto y perpetuo equilibrio.
\vs p007 2:3 El Hijo no reside ni está personalmente presente en los suprauniversos; en estas creaciones solo tiene representación suprapersonal. Estas manifestaciones espirituales del Hijo no son personales; no están en la vía circulatoria del ser personal del Padre Universal. No conocemos ningún término mejor para designarlos que el de \bibemph{seres suprapersonales;} son seres finitos; no son ni absonitos ni absolutos.
\vs p007 2:4 Los seres personales creaturales no pueden percibir el gobierno del Hijo Eterno en los suprauniversos por ser este exclusivamente espiritual y suprapersonal. No obstante, el impulso espiritual y ubicuo, relativo a la influencia personal del Hijo, se encuentra en cada una de las facetas de la actividad de todos los sectores pertenecientes a los ámbitos de los ancianos de días. En los universos locales, sin embargo, observamos que el Hijo Eterno está personalmente presente en las personas de los hijos del Paraíso. Aquí el Hijo Infinito obra de manera espiritual y creativa en las personas del majestuoso colectivo de los hijos creadores iguales en rango.
\usection{3. LA RELACIÓN DEL HIJO ETERNO CON EL SER HUMANO}
\vs p007 3:1 Al ascender en el universo local, los mortales del tiempo ven en el hijo creador al representante personal del Hijo Eterno, pero cuando comienzan a ascender siguiendo el régimen de formación del suprauniverso, los peregrinos del tiempo perciben cada vez más la presencia excelsa e inspiradora del espíritu del Hijo Eterno y pueden beneficiarse al recibir este ministerio espiritualmente energizante. En Havona, los ascendentes se hacen todavía más conscientes del amoroso acogimiento del espíritu ubicuo del Hijo Primigenio. En ninguna etapa de la ascensión del mortal el espíritu del Hijo Eterno mora en la mente o en el alma del peregrino del tiempo, pero su acción benéfica está por siempre cercana y siempre atenta al bien y a la seguridad espiritual conforme los hijos del tiempo avanzan.
\vs p007 3:2 La atracción de la gravedad espiritual del Hijo Eterno constituye el secreto consustancial a la ascensión al Paraíso de las almas humanas que sobreviven. Todos los genuinos valores espirituales y todos los seres auténticamente espiritualizados se mantienen dentro de la infalible atracción de la gravedad espiritual del Hijo Eterno. La mente mortal, por ejemplo, inicia su andadura como sistema material y termina por incorporarse en el colectivo de los finalizadores, en una existencia espiritual prácticamente perfeccionada, estando cada vez menos sujeta a la gravedad material y, en consecuencia, cada vez más sensible, durante toda esta experiencia, al impulso de la atracción hacia el interior de la gravedad espiritual. La vía circulatoria de la gravedad espiritual atrae literalmente al alma del hombre en dirección al Paraíso.
\vs p007 3:3 \pc Esta vía circulatoria es el cauce fundamental para transmitir las oraciones genuinas que salen del corazón del ser humano creyente desde el nivel de la conciencia humana hasta la propia conciencia de la Deidad. Lo que contenga un verdadero valor espiritual de vuestras peticiones se recogerá en dicha vía universal y pasará de forma inmediata y simultánea a todos los seres personales correspondientes, cada uno de los cuales se ocupará de aquellas pertenecientes a su ámbito personal. Por lo tanto, en la práctica de vuestra experiencia religiosa, es irrelevante si, al dirigir vuestras súplicas, pensáis en el hijo creador de vuestro universo local o en el Hijo Eterno, en el centro de todas las cosas.
\vs p007 3:4 \pc La acción discriminadora de la vía circulatoria de la gravedad espiritual quizás se pudiera comparar con las funciones de las vías neuronales del cuerpo humano material: los estímulos recorren interiormente estas vías neuronales; algunos se detienen y encuentran respuesta por parte de los centros espinales reflejos inferiores; otros continúan hasta los centros menos reflejos pero rutinarios del cerebro inferior, mientras que los impulsos más importantes y vitales, al llegar, se transmiten rápidamente por estos centros secundarios y se registran de forma inmediata en los niveles más elevados de la conciencia humana.
\vs p007 3:5 ¡Cuánta mayor perfección existe no obstante en el magnífico proceder del mundo espiritual! Si aquello que se origina en vuestra conciencia se llena de supremo valor espiritual, una vez que le hayáis dado expresión no habrá en el universo ningún poder que impida su transmisión rápida y directa al Ser Personal Absoluto y Espiritual de toda la creación.
\vs p007 3:6 Por el contrario, si vuestras súplicas son enteramente materiales y totalmente egocéntricas, no existe plan alguno por el que dichas oraciones merezcan encontrar cabida en la vía espiritual del Hijo Eterno. El contenido de todo ruego que no “rebose de espíritu” no puede encontrar lugar en dicha vía universal. Esas peticiones, enteramente egoístas y materiales, no progresan, no ascienden a las vías en donde se encauzan los verdaderos valores espirituales. Tales palabras son como “metal que resuena y címbalo que retiñe”.
\vs p007 3:7 Es el pensamiento motivado, el contenido espiritual, el que valida la súplica del mortal. Las palabras carecen de valor.
\usection{4. LOS PLANES DE PERFECCIÓN DIVINA}
\vs p007 4:1 El Hijo Eterno tiene un nexo perpetuo con el Padre en cuanto al seguimiento satisfactorio del \bibemph{plan divino de progreso:} el plan universal para la creación, evolución, ascensión y perfección de las criaturas de voluntad. Y, en cuanto a su dedicación divina, el Hijo es el eterno igual al Padre.
\vs p007 4:2 El Padre junto con su Hijo son como uno solo en el establecimiento y consumación de este colosal plan establecido para que los seres materiales del tiempo logren avanzar hacia la perfección eterna. Este designio para la elevación espiritual de las almas ascendentes del espacio es una creación conjunta del Padre y del Hijo, los cuales, con la cooperación del Espíritu Infinito, se dedican unidos a llevar a cabo tal propósito divino.
\vs p007 4:3 \pc Este plan divino para lograr la perfección abarca tres cometidos únicos, aunque maravillosamente correlacionados, de la aventura universal:
\vs p007 4:4 \li{1.}\bibemph{El plan de logro progresivo}. Este es el plan de ascensión evolutiva establecido por el Padre Universal, cuyas instrucciones fueron sin reservas asumidas por el Hijo Eterno cuando asintió a lo que el Padre proponía: “Hagamos a las criaturas mortales a nuestra propia imagen”. Esta disposición para la elevación de las criaturas del tiempo incluye la dádiva de los modeladores del pensamiento de parte del Padre y la dotación de las criaturas materiales con las prerrogativas del ser personal.
\vs p007 4:5 \li{2.}\bibemph{El plan de gracia}. El siguiente plan universal consiste en el sublime empeño del Hijo Eterno y de sus hijos homólogos entre sí de revelar al Padre. Esto constituye la propuesta del Hijo Eterno y consiste en su dádiva de los Hijos de Dios a las creaciones evolutivas para allí hacer personal y llevar a efecto ---encarnar y hacer real--- el amor del Padre y la misericordia del Hijo para las criaturas de todos los universos. Propio del plan de gracia, y como rasgo preparatorio de este servicio de amor, los hijos del Paraíso sirven de rehabilitadores de aquello que la equivocada voluntad de las criaturas ha hecho peligrar espiritualmente. En cualquier momento y lugar donde ocurra una demora en la consecución del plan de logro, si alguna rebelión acaso afectara o complicara este objetivo, las disposiciones previstas para situaciones de emergencia entrarían, entonces, en acción inmediata. Los hijos del Paraíso se han comprometido a actuar con prontitud como rescatadores, a entrar en los ámbitos mismos de la rebelión y restaurar allí la condición espiritual de las esferas. Y uno de estos hijos creadores iguales en rango realizó en Urantia esta heroica contribución, en relación con la experiencia de su andadura de gracia para el logro de su soberanía.
\vs p007 4:6 \li{3.}\bibemph{El plan del ministerio de la misericordia}. Una vez que se hubo establecido y proclamado el plan de logro y el plan de gracia, el Espíritu Infinito, solo y a partir de sí mismo, concibió y puso en acción el cometido formidable y universal del ministerio de la misericordia. Esta contribución resulta sumamente esencial para llevar a efecto, de forma práctica y con eficacia, tanto la labor de logro progresivo como la de gracia, y todos los seres personales espirituales de la Tercera Fuente y Centro participan del espíritu del ministerio de la misericordia, que de tal manera forma parte de la naturaleza de la Tercera Persona de la Deidad. No solo en la creación sino también en la administración de los universos, el Espíritu Infinito obra verdadera y literalmente como mandatario conjunto con el Padre y el Hijo.
\vs p007 4:7 \pc El Hijo Eterno es el fiduciario personal, el custodio divino, del plan universal del Padre para la ascensión de las criaturas. Habiendo promulgado el mandato universal, “Sed vosotros perfectos, como yo soy perfecto”, el Padre confió la realización de esta extraordinaria empresa al Hijo Eterno; y el Hijo Eterno comparte con su divino igual en rango, el Espíritu Infinito, su apoyo a este excelso cometido. Así pues, las Deidades cooperan de forma efectiva en la tarea de creación, dirección, evolución, revelación y servicio y, si es necesario, de restablecimiento y rehabilitación.
\usection{5. EL ESPÍRITU DE GRACIA}
\vs p007 5:1 El Hijo Eterno se unió sin reservas al Padre Universal en la transmisión de aquel formidable dictado para toda la creación: “Sed vosotros perfectos, como vuestro Padre en Havona es perfecto”. Y desde entonces, esa invitación\hyp{}mandato ha alentado todos los planes de supervivencia y los designios de gracia del Hijo Eterno y de su inmensa familia de hijos homólogos y adjuntos. Y en estos mismos ministerios de gracia los Hijos de Dios se han erigido como “el camino, la verdad y la vida” para todas las criaturas evolutivas.
\vs p007 5:2 \pc El Hijo Eterno no puede entrar en contacto directo con los seres humanos como lo hace el Padre, a través del don de los prepersonales modeladores del pensamiento, pero el Hijo Eterno sí se acerca a los seres personales creados mediante una serie de disminuciones escalonadas de filiación divina hasta que le es posible presentarse al hombre y, a veces, hacerlo como hombre mismo.
\vs p007 5:3 La naturaleza enteramente personal del Hijo Eterno no se puede fraccionar. El Hijo Eterno realiza su ministerio o bien como persona o bien ejerciendo una influencia espiritual, nunca de otro modo. Al Hijo le resulta imposible hacerse parte de la experiencia de la criatura del modo en la que participa en ella el Padre\hyp{}modelador, pero el Hijo Eterno mediante el ministerio de gracia compensa esta limitación. Lo que la experiencia de las entidades repartidas significa para el Padre Universal, las experiencias de encarnación de los hijos del Paraíso significan para el Hijo Eterno.
\vs p007 5:4 El Hijo Eterno no viene al hombre mortal como voluntad divina, como modelador del pensamiento que mora en la mente humana; en cambio, el Hijo Eterno ciertamente vino al hombre mortal de Urantia cuando el \bibemph{ser personal} divino de su hijo, Miguel de Nebadón, se encarnó en la naturaleza humana de Jesús de Nazaret. Para compartir la experiencia de los seres personales creados, los Hijos de Dios del Paraíso deben asumir la naturaleza misma de estas criaturas y encarnar sus personas divinas como criaturas mismas. La encarnación, el secreto de Lugar del Hijo, es el método utilizado por el Hijo para liberarse de lo que sería su atadura completa al absolutismo del ser personal.
\vs p007 5:5 \pc Hace muchísimo tiempo el Hijo Eterno se dio de gracia en cada una de las vías circulatorias de la creación central para lucidez y avance de todos los habitantes y peregrinos de Havona, incluyendo a los peregrinos ascendentes del tiempo. En ninguno de estos siete ministerios de gracia obró ni como ascendente ni como originario de Havona. Él existía como él mismo. Su experiencia fue única; no fue \bibemph{con} ni \bibemph{como} un humano u otro peregrino sino de alguna manera vinculado a estos en un sentido suprapersonal.
\vs p007 5:6 Tampoco pasó por el descanso que transcurre entre la vía interior orbital de Havona y las orillas del Paraíso. No es posible para él, un ser absoluto, suspender su conciencia personal, porque en él se centran todas las líneas de la gravedad espiritual. Y durante los períodos de estos ministerios de gracia, no se atenuó la luminosidad espiritual de su aposentamiento en el Paraíso central ni disminuyó el dominio de la gravedad espiritual universal que ejerce el Hijo.
\vs p007 5:7 \pc Los ministerios de gracia del Hijo Eterno en Havona no están al alcance de la imaginación humana: fueron trascendentales. Él amplió entonces y posteriormente la experiencia de todo Havona, pero no sabemos si amplió la supuesta capacidad experiencial de su propia naturaleza existencial. Eso formaría parte del misterio de gracia de los hijos del Paraíso. Sí creemos, no obstante, que todo lo que el Hijo adquirió en estos ministerios, lo ha retenido desde entonces, aunque no sabemos qué es.
\vs p007 5:8 \pc Sean cuales fueren nuestras dificultades para comprender los ministerios de gracia de la segunda persona de la Deidad, sí comprendemos bien el efectuado en Havona por un hijo del Hijo Eterno, que de cierto recorrió las vías circulatorias del universo central y que realmente compartió las experiencias que forman a los ascendentes para poder alcanzar la Deidad. Este fue el miguel primigenio, el hijo creador primogénito, que pasó por la experiencia de vida de los peregrinos ascendentes, circundando vía tras vía, recorriendo personalmente con ellos las etapas de cada círculo en los tiempos de Granfanda, el primero de los mortales en llegar a Havona.
\vs p007 5:9 Sea lo que además revelara este miguel primigenio, él hizo realidad para las criaturas de Havona el supremo ministerio de gracia de la Madre Hijo Primigenia. Tan real fue, que por siempre todo peregrino del tiempo, en su venturoso afán por alcanzar las vías circulatorias de Havona, se siente alentado y fortalecido por la certeza de que el Hijo Eterno de Dios renunció siete veces al poder y a la gloria del Paraíso para ser partícipe, en las siete vías de realización continua de Havona, de las experiencias de los peregrinos del tiempo y del espacio.
\vs p007 5:10 \pc El Hijo Eterno sirve como modelo de inspiración para todos los Hijos de Dios en los ministerios de gracia que realizan en todos los universos del tiempo y del espacio. Los hijos creadores de igual rango y los hijos magistrados adjuntos a ellos, junto a otros órdenes de filiación no revelados, comparten esta extraordinaria buena disposición de darse de gracia a todos los diversos órdenes de vida creatural y como criaturas mismas. Por tanto, en espíritu y debido a la afinidad de naturaleza así como al hecho de su origen, se hace verdad que en los ministerios de gracia de todo hijo de Dios a los mundos del espacio mediante tales ministerios, el Hijo Eterno se da a las criaturas de los universos con inteligencia y voluntad.
\vs p007 5:11 En espíritu y en naturaleza, por no decir en todos los atributos, todo hijo del Paraíso es un retrato divinamente perfecto del Hijo Primigenio. Es literalmente cierto que quien ha visto a un hijo del Paraíso, ha visto al Hijo Eterno de Dios.
\usection{6. LOS HIJOS DE DIOS DEL PARAÍSO}
\vs p007 6:1 La falta de conocimiento respecto a los múltiples Hijos de Dios origina en Urantia una gran confusión. Y este desconocimiento persiste a pesar de las afirmaciones que constan en un cónclave de estos seres personales divinos: “Cuando se regocijaban los Hijos de Dios, y todas las estrellas de la mañana alababan juntas”. Cada milenio en el tiempo estándar del sector, los diferentes órdenes de hijos divinos se congregan para sus cónclaves periódicos.
\vs p007 6:2 El Hijo Eterno constituye la fuente personal de los admirables atributos de misericordia y servicio que con tanta plenitud caracterizan a todos los órdenes descendentes de Hijos de Dios, según obran por toda la creación. El Hijo Eterno transmite de forma indefectible toda la naturaleza divina, por no decir todos los infinitos atributos, a los hijos del Paraíso que salen de la Isla eterna para revelar al universo de los universos su carácter divino.
\vs p007 6:3 \pc El Hijo Primigenio y Eterno es el vástago\hyp{}persona del “primer” pensamiento completo e infinito del Padre Universal. Cada vez que el Padre Universal y el Hijo Eterno conciben de forma conjunta un pensamiento personal nuevo, primigenio, idéntico, único y absoluto, en ese mismo instante esa idea creativa se hace personal de forma perfecta y, finalmente, constituye el ser y el ser personal de un nuevo y primigenio \bibemph{hijo creador}. En naturaleza espiritual, sabiduría divina y mismo rango de poder creativo, estos hijos creadores son potencialmente coiguales al Dios Padre y al Dios Hijo.
\vs p007 6:4 Los hijos creadores salen desde el Paraíso a los universos del tiempo y, con la cooperación de las instancias intermedias controladoras y creativas de la Tercera Fuente y Centro, completan la organización de los universos locales en evolución. Estos hijos no tienen a su cargo ni se ocupan de los dictados centrales y universales sobre la materia, la mente y el espíritu. Por consiguiente, están limitados en sus actos creativos por la preexistencia, prioridad y primacía de la Primera Fuente y Centro y de sus Absolutos de igual rango. Estos hijos solamente pueden regir a lo que dan nacimiento. La administración absoluta de lo creado es consustancial a la prioridad de la existencia y es inseparable de la eternidad de la presencia. El Padre permanece primordial en los universos.
\vs p007 6:5 \pc Al igual que los hijos creadores se hacen personales mediante el Padre y el Hijo, los \bibemph{hijos magistrados} se hacen personales mediante el Hijo y el Espíritu. Estos son los hijos que, en sus experiencias de encarnación como criaturas, ganan el derecho a servir en las creaciones del tiempo y del espacio como jueces dictaminadores de la supervivencia.
\vs p007 6:6 \pc El Padre, el Hijo y el Espíritu también se unen para hacer personales a los versátiles \bibemph{hijos preceptores de la Trinidad,} que recorren el gran universo obrando como los excelsos maestros de todos los seres personales, tanto humanos como divinos. Y hay muchos otros órdenes de filiación del Paraíso que no se han desvelado a los mortales de Urantia.
\vs p007 6:7 \pc Entre la Madre Hijo Primigenia y estas multitudes de hijos del Paraíso dispersos por toda la creación hay un canal de comunicación directo y exclusivo, un canal cuyo cometido es inherente a la cualidad de la afinidad espiritual que los une, correlacionándolos de una manera casi absolutamente espiritual. Esta vía interfilial es enteramente diferente de la vía universal de circulación de la gravedad espiritual, que también se centra en la persona de la Segunda Fuente y Centro. Todos los Hijos de Dios que se originan en las personas de las Deidades del Paraíso están en comunicación constante y directa con la Madre Hijo Eterna. Y esta comunicación se realiza de forma instantánea; es independiente del tiempo, aunque a veces el espacio la condiciona.
\vs p007 6:8 El Hijo Eterno no solo tiene en todo momento un conocimiento perfecto en lo que concierne al estatus, pensamientos y múltiple actividad de todos los órdenes de filiación del Paraíso, sino que también tiene, en todo momento, un conocimiento perfecto respecto a todo lo que exista de valor espiritual en los corazones de todas las criaturas de la creación primaria y central de la eternidad y de las creaciones secundarias temporales de los hijos creadores iguales en rango.
\usection{7. LA REVELACIÓN SUPREMA DEL PADRE}
\vs p007 7:1 El Hijo Eterno es una revelación completa, exclusiva, universal y final del espíritu y el ser personal del Padre Universal. Todo conocimiento y toda información concerniente al Padre deben provenir del Hijo Eterno y de sus hijos del Paraíso. El Hijo Eterno procede de la eternidad y es, totalmente y sin condicionamiento espiritual alguno, uno con el Padre. En ser personal divino son equiparables; en naturaleza espiritual, iguales; en divinidad, idénticos.
\vs p007 7:2 El carácter de Dios no podría intrínsecamente mejorarse de modo alguno en la persona del Hijo porque el Padre divino es infinitamente perfecto, pero ese carácter y ese ser personal se magnifican, despojándose de lo no personal y no espiritual, para revelarse a los seres creados. La Primera Fuente y Centro es mucho más que un ser personal, pero todas las cualidades espirituales del ser personal paternal de la Primera Fuente y Centro están espiritualmente presentes en el ser personal absoluto del Hijo Eterno.
\vs p007 7:3 El Hijo primordial y sus hijos están dedicados a la revelación universal de la naturaleza espiritual y personal del Padre para toda la creación. En el universo central, en los suprauniversos, en los universos locales o en los planetas habitados, es un hijo del Paraíso quien revela al Padre Universal a los hombres y a los ángeles. El Hijo Eterno y sus hijos revelan el cauce por el que la criatura accede al Padre Universal. E incluso nosotros, los de elevado origen, entendemos al Padre con mayor plenitud cuando examinamos la revelación de su carácter y de su ser personal en el Hijo Eterno y en los hijos del Hijo Eterno.
\vs p007 7:4 El Padre desciende a vosotros como ser personal solamente a través de los hijos divinos del Hijo Eterno. Y vosotros llegáis al Padre por ese mismo camino vivo; ascendéis al Padre mediante la guía de este grupo de hijos divinos. Y esto es verdad a pesar de que vuestro mismo ser personal sea la dádiva directa del Padre Universal.
\vs p007 7:5 \pc En toda esta extensa actividad de la dilatada administración espiritual del Hijo Eterno, no olvidéis que el Hijo es una persona tan verdadera y real como lo es el Padre. En efecto, a los seres que alguna vez pertenecieron al orden humano, el Hijo Eterno les resultará de más fácil acceso que el Padre Universal. En el progreso de los peregrinos del tiempo a través de las vías circulatorias de Havona, vosotros podréis llegar al Hijo mucho antes de estar preparados para percibir al Padre.
\vs p007 7:6 Comprenderéis más acerca del carácter y de la naturaleza misericordiosa del Hijo Eterno según meditéis sobre la revelación de estos atributos divinos que en un acto de amor realizó vuestro propio hijo creador, el que fuera Hijo del Hombre en la tierra, ahora excelso soberano de vuestro universo local: el Hijo del Hombre y el Hijo de Dios.
\vsetoff
\vs p007 7:7 [Redactado por un consejero divino designado para describir al Hijo Eterno del Paraíso.]
