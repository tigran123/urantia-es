\upaper{41}{Aspectos físicos del universo local}
\author{Arcángel}
\vs p041 0:1 La presencia del espíritu creativo es el factor espacial diferenciador que distingue a un universo local de todos los demás. Todo Nebadón está ciertamente infundido de la presencia espacial de la benefactora divina de Lugar de Salvación, y dicha presencia termina, ciertamente igual, en las fronteras exteriores de nuestro universo local. Lo que nuestro espíritu materno del universo local infunde \bibemph{es} Nebadón; aquello que se extiende más allá de su presencia espacial está fuera de Nebadón; se trata de las regiones espaciales del suprauniverso de Orvontón, externas a Nebadón ---otros universos locales---.
\vs p041 0:2 \pc Aunque en la organización administrativa del gran universo se percibe una clara división entre los gobiernos del universo central, de los suprauniversos y de los universos locales, y aunque estas divisiones tengan su paralelo astronómico en la separación espacial entre Havona y los siete suprauniversos, no existen unas líneas divisorias físicas tan perceptibles que delimiten las creaciones locales. Aunque los sectores mayores y menores de Orvontón (para nosotros) se puedan distinguir con claridad, no resulta tan fácil identificar los límites físicos de los universos locales. Esto se debe a que estas creaciones locales se organizan administrativamente en conformidad con ciertos principios \bibemph{creativos} que rigen la segmentación de la carga total de energía de un suprauniverso, mientras que sus componentes físicos, las esferas del espacio ---los soles, las islas oscuras, los planetas, etc.--- se originan principalmente en las nebulosas, y estas hacen su aparición astronómica de acuerdo con ciertas planificaciones \bibemph{precreativas} (trascendentales) de los arquitectos del universo matriz.
\vs p041 0:3 Una o más de estas nebulosas ---incluso muchas de ellas--- se pueden circunscribir dentro del ámbito de un solo universo local, tal como Nebadón se formó físicamente a partir de la progenie estelar y planetaria de Andrónover y de otras nebulosas. Las esferas de Nebadón proceden de distintas nebulosas, pero todas tenían un cierto grado de movimiento mínimo espacial que se ajustó de tal manera mediante la actividad inteligente de los directores de la potencia, dando origen a nuestro conjunto actual de cuerpos espaciales, que viajan juntos, uno al lado del otro, como una unidad, por las órbitas del suprauniverso.
\vs p041 0:4 Esta es la constitución de la nube estelar local de Nebadón, que en la actualidad gira en una órbita cada vez más estable alrededor del centro, en Sagitario, del sector menor de Orvontón al que pertenece nuestra creación local.
\usection{1. LOS CENTROS DE LA POTENCIA DE NEBADÓN}
\vs p041 1:1 Los organizadores de la fuerza del Paraíso dan inicio a las nebulosas espirales y a otras formas de nebulosas, las ruedas matrices de las esferas del espacio, y, consiguiente a la evolución de la nebulosa en su respuesta a la gravedad, se reemplaza a estos organizadores de la fuerza de su función en el suprauniverso por los centros de la potencia y por los controladores físicos que, acto seguido, asumen toda la responsabilidad de dirigir la evolución física de generaciones futuras de descendientes estelares y planetarios. Con la llegada de nuestro hijo creador y, con él, de su plan de organización del universo, se llevó a cabo de inmediato la coordinación de esta supervisión física del preuniverso de Nebadón. Dentro del ámbito de este hijo de Dios del Paraíso, los centros supremos de la potencia y los controladores físicos mayores colaboraron con los supervisores de la potencia morontial, que harían su aparición con posterioridad, y con otros seres para producir el inmenso complejo de líneas de comunicación, vías circulatorias de la energía y rutas de potencia, que enlazan firmemente los múltiples cuerpos espaciales de Nebadón en una unidad administrativa conjunta.
\vs p041 1:2 Hay cien centros supremos de la potencia del cuarto orden permanentemente asignados a nuestro universo local. Estos seres reciben las líneas entrantes de la potencia procedentes de los centros del tercer orden de Uversa y redirigen las vías circulatorias reducidas y modificadas a los centros de la potencia de nuestras constelaciones y sistemas. Estos centros de la potencia actúan en conjunción y producen el sistema vivo de control e igualación que opera para mantener el equilibrio y la distribución de las energías que, de otra manera, serían fluctuantes y variables. Los centros de la potencia no intervienen, sin embargo, en las convulsiones transitorias y locales de la energía, tales como las manchas solares y las perturbaciones eléctricas del sistema; la luz y la electricidad no son las energías básicas del espacio; son manifestaciones secundarias y subsidiarias.
\vs p041 1:3 Los cien centros del universo local están emplazados en Lugar de Salvación y operan en el centro energético exacto de esta esfera. Las esferas arquitectónicas, tales como Lugar de Salvación, Edentia, y Jerusem, están iluminadas, calentadas y energizadas mediante métodos que las hacen totalmente independientes de los soles del espacio. Los centros de la potencia y los controladores físicos construyeron ---hicieron por encargo--- estas esferas que se diseñaron para ejercer un poderoso efecto sobre la distribución de la energía. Basando su actividad en tales puntos de actividad y control de la energía, los centros de la potencia, por su presencia viva, dirigen y canalizan las energías físicas del espacio. Y estas vías circulatorias de la energía son fundamentales para todos los fenómenos físico\hyp{}materiales y morontio\hyp{}espirituales.
\vs p041 1:4 Hay diez centros supremos de la potencia del quinto orden asignados a cada una de las subdivisiones principales de Nebadón, a sus cien constelaciones. En Norlatiadec, vuestra constelación, estos centros no están emplazados en su esfera sede sino que se localizan en el centro del enorme sistema estelar que constituye el núcleo físico de la constelación. En Edentia hay diez controladores mecánicos obrando en vinculación y diez frandalancs que actúan en perfecta y constante coordinación con los centros de la potencia cercanos.
\vs p041 1:5 Hay un centro supremo de la potencia del sexto orden emplazado en el punto exacto de actividad de la gravedad de cada uno de los sistemas locales. En el sistema de Satania, el centro de la potencia allí asignado ocupa una isla oscura del espacio que se sitúa en el centro astronómico del sistema. Muchas de estas islas oscuras son inmensas dinamos que movilizan y orientan ciertas energías espaciales, y estas circunstancias naturales se utilizan eficientemente por el centro de la potencia de Satania, cuya masa viva actúa en conjunción con los centros superiores, dirigiendo las corrientes de la potencia más materializada a los controladores físicos mayores de los planetas evolutivos del espacio.
\usection{2. LOS CONTROLADORES FÍSICOS DE SATANIA}
\vs p041 2:1 Aunque los controladores físicos prestan sus servicios con los centros de la potencia en todo el gran universo, sus funciones, en un sistema local como Satania, resultan más fáciles de comprender. Satania es uno de los cien sistemas locales que integran la organización administrativa de la constelación de Norlatiadec; en su entorno más inmediato, están los sistemas de Sandmatia, Assuntia, Porogia, Sortoria, Rantulia y Glantonia. Los sistemas de Norlatiadec difieren en muchos aspectos, pero todos ellos son evolutivos y progresivos, muy parecidos a Satania.
\vs p041 2:2 Satania misma se compone de más de siete mil grupos astronómicos, o sistemas físicos, pocos de los cuales tuvieron un origen similar al de vuestro sistema solar. El centro astronómico de Satania es una enorme isla oscura del espacio que, con sus esferas acompañantes, se sitúa no lejos de la sede del gobierno del sistema.
\vs p041 2:3 \pc Salvo por la presencia del centro de la potencia destinado en el sistema, la supervisión de todo el sistema de energía física de Satania se centra en Jerusem. Uno de los controladores físicos mayores, emplazado en esta esfera sede, opera en coordinación con el centro de la potencia del sistema, prestando sus servicios como jefe de enlace de los inspectores de la potencia con sede en Jerusem y ejerciendo sus funciones por todo el sistema local.
\vs p041 2:4 Distribuidos por todo Satania, hay quinientos mil operadores de la energía; son seres vivos e inteligentes que se encargan del establecimiento de los ciclos de la energía y de su canalización. Gracias a la acción de estos regidores de la energía física, los centros de la potencia en supervisión ejercen un perfecto y completo control sobre la mayoría de las energías básicas del espacio, incluyendo las emanaciones de órbitas altamente recalentadas y de las esferas oscuras cargadas de energía. Este grupo de entidades vivas es capaz de movilizar, transformar, transmutar, manejar y transmitir casi todas las energías físicas del espacio organizado.
\vs p041 2:5 La vida tiene una capacidad intrínseca para movilizar y transmutar la energía universal. Estáis familiarizados con la acción de la vida vegetal y su transformación de la energía material de la luz en diversas manifestaciones del reino vegetal. También conocéis algo del método por el que esta energía vegetal se puede convertir en los fenómenos del funcionamiento de vida animal, pero no sabéis prácticamente nada de los métodos de los directores de la potencia y de los controladores físicos, que están dotados de la capacidad de movilizar, transformar, orientar y concentrar las múltiples energías del espacio.
\vs p041 2:6 \pc A estos seres de los reinos energéticos no les concierne de forma directa la energía como factor componente de las criaturas vivas, ni siquiera el ámbito de la química fisiológica. Algunas veces les competen los preliminares físicos de la vida, la elaboración de aquellos sistemas de energía que pueden servir como vehículos físicos para las energías vivas de los organismos materiales elementales. De alguna manera, los controladores físicos están relacionados con las manifestaciones anteriores a la vida de la energía material, al igual que los espíritus asistentes de la mente lo están respecto a las funciones pre\hyp{}espirituales de la mente material.
\vs p041 2:7 \pc Estas criaturas inteligentes que controlan la potencia y dirigen la energía deben acomodar su acción a la constitución y a la arquitectura física del planeta implicado. Utilizan de forma infalible los cálculos y conclusiones de sus respectivos asistentes físicos personales y de otros asesores técnicos con respecto a la influencia local de los soles altamente recalentados y de otros tipos de estrellas sobrecargadas. Se debe contar incluso con los enormes gigantes fríos y oscuros del espacio y las nubes plagadas de polvo estelar; hay que tener en cuenta todas estas cosas materiales al abordar las cuestiones prácticas del manejo de la energía.
\vs p041 2:8 La supervisión de la energía\hyp{}potencia de los mundos habitados evolutivos es cometido de los controladores físicos mayores, pero estos seres no son responsables de todos los desajustes energéticos que se producen en Urantia. Hay varias razones que justifican estas perturbaciones, algunas de las cuales están más allá del ámbito y del control de los custodios físicos. Urantia se sitúa en el paso de energías formidables; es un pequeño planeta en medio de una vía de masas enormes y, a veces, los controladores locales hacen uso de un gran número de miembros de su orden para intentar equilibrar estas líneas de energía. En las vías circulatorias de Satania tienen bastante éxito, pero encuentran dificultades para aislar al planeta de las poderosas corrientes de Norlatiadec.
\usection{3. NUESTROS ACOMPAÑANTES ESTELARES}
\vs p041 3:1 Hay más de dos mil soles brillantes que derraman luz y energía en Satania, y vuestro Sol es un llameante globo de tipo medio. De los treinta soles más cercanos al vuestro, solo tres son más luminosos. Los directores de la potencia del universo dan origen a las corrientes de energía, creadas con un propósito específico, que entran en juego entre cada una de las estrellas y sus sistemas respectivos. Estos hornos solares, junto con los gigantes oscuros del espacio, sirven a los centros de la potencia y a los controladores físicos como estaciones de tránsito para concentrar y orientar eficazmente las vías circulatorias de la energía de las creaciones materiales.
\vs p041 3:2 Los soles de Nebadón no difieren de los soles de los otros universos. La composición material de todos los soles; islas oscuras, planetas y satélites, e incluso meteoros, es muy similar. Estos soles tienen un diámetro medio de alrededor de un millón setecientos mil kilómetros; vuestro propio globo solar es ligeramente menor. La estrella más grande del universo, la nube estelar de Antares, tiene un diámetro que es cuatrocientas cincuenta veces mayor que el de vuestro Sol y un volumen sesenta millones mayor. Pero hay espacio abundante para dar cabida a todos estos enormes soles. En comparación, tienen tanto espacio disponible como el que tendría una docena de naranjas circulando por el interior de Urantia, si el planeta fuese una esfera hueca.
\vs p041 3:3 \pc Cuando una rueda matriz nebular libera soles demasiado grandes, estos pronto se fragmentan o forman estrellas dobles. Todos los soles son, en un principio, realmente gaseosos, aunque más tarde puedan existir, de forma transitoria, en estado semilíquido. Cuando vuestro Sol alcanzó este estado cuasi líquido sometido a la presión de un extraordinario gas, no era lo suficientemente grande para dividirse por su ecuador, que es la manera en la que se forman las estrellas dobles.
\vs p041 3:4 Cuando son menores de una décima parte del tamaño de vuestro sol, estas abrasadoras esferas se contraen, se condensan y se enfrían rápidamente. Cuando tienen más de treinta veces su tamaño ---más bien treinta veces el contenido bruto de su materia real---, los soles se dividen rápidamente en dos cuerpos separados o se convierten en los centros de sistemas nuevos o permanece cada cual dentro de la atracción de la gravedad del otro, girando alrededor de un centro común como un tipo de estrella doble.
\vs p041 3:5 \pc La más reciente de todas las grandes erupciones cósmicas ocurridas en Orvontón fue la extraordinaria explosión de una estrella doble, cuya luz llegó a Urantia en el año 1572. La llamarada fue tan intensa que la explosión se divisó con claridad a plena luz del día.
\vs p041 3:6 \pc No todas las estrellas son sólidas, aunque muchas de las más antiguas sí lo son. Algunas de las estrellas rojizas, que brillan con una débil luz trémula, han adquirido una densidad en el centro de sus enormes masas que se podría explicar diciendo que un centímetro cúbico de dicha estrella, si estuviese en Urantia, pesaría 166 kilos. La enorme presión, acompañada de la pérdida de calor y de energía circulante, ha dado lugar a que las órbitas de sus unidades materiales elementales se aproximen cada vez más hasta llegar acercarse en este momento a un estado próximo al de la condensación electrónica. Este proceso de enfriamiento y contracción puede continuar hasta el punto límite y crítico de explosión en masa de la condensación ultimatónica.
\vs p041 3:7 La mayoría de los soles gigantes son relativamente jóvenes; la mayor parte de las estrellas enanas, aunque no todas, son viejas. Las estrellas enanas formadas por colisión pueden ser muy jóvenes y pueden brillar con una intensa luz blanca; nunca han pasado por la etapa inicial de color rojo brillante de las estrellas jóvenes. Por lo general, tanto los soles muy jóvenes como los más longevos despiden un brillo de color rojizo. El tono amarillento es característico de las estrellas moderadamente jóvenes o ya cercanas a la longevidad, si bien, la luz blanca brillante es indicativa de estrellas resistentes y duraderas que han llegado a su pleno desarrollo.
\vs p041 3:8 \pc Aunque no todos los soles más jóvenes pasan por una etapa de pulsación, al menos no de forma visible, cuando oteéis el espacio quizás podáis observar muchas de estas estrellas, cuyos gigantescos vaivenes respiratorios necesitan de dos a siete días para completar un ciclo. Vuestro propio Sol todavía lleva consigo un remanente en disminución de las grandes oscilaciones de sus días más jóvenes, pero la periodicidad se ha prolongado desde las anteriores pulsaciones de tres días y medio hasta los actuales ciclos de manchas solares de once años y medio.
\vs p041 3:9 Las variables estelares tienen un buen número de causas. En algunas estrellas dobles, las corrientes de marea ocasionadas por la rapidez de los cambios de distancia entre los dos cuerpos al girar alrededor de sus órbitas también provocan fluctuaciones periódicas de luz. Estas variaciones de la gravedad producen llamaradas regulares y reiteradas, al igual que el aumento de energía\hyp{}materia en la superficie por la colisión de meteoros podría resultar en un destello de luz relativamente repentino, que con toda rapidez se reduciría hasta el brillo normal de ese sol. A veces un Sol puede atraer a una sucesión de meteoros en un punto que oponga poca resistencia a la gravedad, y a veces las colisiones producen llamaradas estelares, pero la mayoría de dichos fenómenos se debe por completo a fluctuaciones internas.
\vs p041 3:10 En un grupo variable de estrellas, el período de fluctuación de la luz depende directamente de su luminosidad, y el conocimiento de este hecho permite a los astrónomos utilizar dichos soles como si fuesen faros en el universo o puntos de medición precisos para la exploración más detallada de aglomeraciones distantes de estrellas. Mediante este procedimiento, es posible calcular las distancias estelares con mayor exactitud de hasta más de un millón de años luz. Algún día, existirán mejores métodos de medición del espacio y técnicas telescópicas más perfectas que os mostrarán, de forma más completa, las diez grandes divisiones del suprauniverso de Orvontón; vosotros llegaréis a distinguir al menos ocho de estos inmensos sectores como aglomeraciones estelares enormes y bastante simétricas.
\usection{4. LA DENSIDAD DE LOS SOLES}
\vs p041 4:1 La masa de vuestro Sol es ligeramente mayor de la estimada por vuestros físicos, que han calculado que es unos dos mil cuatrillones (2 x 1027) de toneladas. En este momento, está aproximadamente a medio camino entre las estrellas de mayor densidad y las más difusas; tiene alrededor de una vez y media la densidad del agua. Pero vuestro Sol no es ni líquido ni sólido ---es gaseoso--- y esto es cierto a pesar de la dificultad de explicar cómo puede alcanzar la materia gaseosa esta densidad e incluso densidades mucho mayores.
\vs p041 4:2 \pc Los estados gaseoso, líquido y sólido hacen referencia a relaciones atómico\hyp{}moleculares, pero la densidad alude a la relación entre espacio y masa. La densidad de un cuerpo es directamente proporcional a la cantidad de masa que tiene en relación al espacio que ocupa e inversamente proporcional al volumen de espacio que contiene en relación a su masa; se trata del espacio existente entre los núcleos centrales de la materia y las partículas que giran alrededor de estos centros, al igual que el que existe dentro de dichas partículas materiales.
\vs p041 4:3 \pc Las estrellas en estado de enfriamiento pueden ser físicamente gaseosas y enormemente densas al mismo tiempo. No estáis familiarizados con los \bibemph{extraordinarios gases} solares, pero estas y otras formas poco comunes de la materia explican cómo incluso los soles no sólidos pueden alcanzar una densidad igual a la del hierro ---más o menos la misma que tiene Urantia--- y, sin embargo, encontrarse en un estado gaseoso altamente recalentado y seguir funcionando como soles. Los átomos de estos densos gases son excepcionalmente pequeños; contienen pocos electrones. Dichos soles también han perdido, en gran medida, sus reservas energéticas de ultimatones libres.
\vs p041 4:4 Uno de los soles cercanos a vosotros, que empezó su vida con una masa prácticamente igual a la del vuestro, se ha contraído actualmente a un tamaño como el de Urantia y se ha vuelto cuarenta mil veces más denso que vuestro Sol. El peso de este sólido\hyp{}gaseoso, a su vez caliente\hyp{}frío, es de unos sesenta y un kilogramos por centímetro cúbico. Y este Sol todavía brilla con un débil resplandor de color rojizo, el trémulo y senil destello de un monarca de luz que se extingue.
\vs p041 4:5 La mayor parte de los soles, sin embargo, no son tan densos. Uno de vuestros vecinos más cercanos tiene una densidad exactamente igual a la de vuestra atmósfera a nivel del mar. Si estuvieseis en el interior de este sol, no podríais distinguir nada. Y si la temperatura lo permitiese, podrías penetrar en la mayoría de los soles que destellan en el cielo nocturno, pero no notaríais más materia que la que podéis percibir en el aire de vuestra sala de estar terrestre.
\vs p041 4:6 El masivo Sol de Veluntia, uno de los soles más grandes de Orvontón, tiene una densidad que es solamente una milésima parte de la atmósfera de Urantia. Si su composición fuese similar a la de vuestra atmósfera y no estuviese sobrecalentado, habría tal vacío que los seres humanos se asfixiarían rápidamente si estuviesen sobre él o dentro de él.
\vs p041 4:7 Otro de los gigantes de Orvontón tiene en este momento una temperatura en su superficie de algo menos de mil seiscientos cuarenta y nueve grados (C). Su diámetro es de más de cuatrocientos ochenta y tres millones de kilómetros ---suficiente espacio como para dar cabida a vuestro Sol y a la órbita actual de la tierra---. No obstante, a pesar de su enorme tamaño, más de cuarenta millones de veces el de vuestro sol, su masa es únicamente unas treinta veces mayor. Estos enormes soles tienen una periferia que se expande hasta casi alcanzarse el uno al otro.
\usection{5. LA RADIACIÓN SOLAR}
\vs p041 5:1 Las mismas corrientes constantes de energía luminosa que despiden los soles del espacio demuestran que estos no son muy densos. Una densidad excesivamente grande retendría la luz por opacidad hasta que la presión de la energía luminosa alcanzara el punto de explosión. Hay una enorme presión de luz o gas dentro de un Sol que le hace lanzar tal corriente de energía que es capaz de traspasar el espacio durante millones y millones de kilómetros para energizar, iluminar y calentar planetas distantes. Unos cuatro metros y medio de superficie con la densidad de Urantia evitarían de hecho el escape de todos los rayos X y todas las energías lumínicas de un sol, hasta que la creciente presión interna de las energías acumuladas resultantes de la desmembración atómica vencieran a la gravedad con una colosal explosión.
\vs p041 5:2 La luz, en presencia de gases propulsores, es sumamente explosiva cuando se le constriñe a temperaturas altas mediante paredes opacas de contención. La luz es real. Según valoráis la energía y la potencia en vuestro mundo, la luz del Sol resultaría económica aunque medio kilo costase un millón de dólares.
\vs p041 5:3 El interior de vuestro Sol es un inmenso generador de rayos X. Los soles se sostienen desde dentro mediante un incesante bombardeo de estas potentes emanaciones.
\vs p041 5:4 Se necesita más de medio millón de años para que un electrón potenciado por rayos X se abra camino desde el centro mismo de un Sol de grado medio hasta la superficie solar, desde donde parte hacia su aventura espacial, quizás para calentar un planeta habitado, ser atrapado por un meteoro, participar en el nacimiento de un átomo, ser atraído por una isla oscura del espacio altamente cargada o finalizar su vuelo espacial, precipitándose sobre la superficie de un Sol de características similares al que le dio origen.
\vs p041 5:5 Los rayos X del interior de un Sol cargan a los electrones altamente recalentados y agitados de una energía suficiente como para impulsarlos por el espacio hasta las distantes esferas de los sistemas remotos, pasando por un considerable número de influencias de orden material que los frenan y obstruyen, y a pesar de las distintas fuerzas de atracción de la gravedad. La gran cantidad de energía de la velocidad necesaria para escapar del agarre de la gravedad de un Sol es suficiente para garantizar que el rayo de Sol viaje con velocidad incesante hasta encontrar amplias masas de materia; después de lo cual se transforma rápidamente en calor con la liberación de otras energías.
\vs p041 5:6 \pc En su vuelo por el espacio, la energía, ya sea de luz o tenga otras formas, avanza de forma directa. Las partículas de orden material cruzan el espacio como si fuesen una descarga cerrada de fusilería. Se desplazan en línea o secuencia recta e ininterrumpida salvo cuando actúan bajo el impulso de fuerzas superiores, y exceptuando que siempre han de obedecer a la fuerza de atracción de la gravedad lineal, inherente a la masa material, y a la presencia de la gravedad circular de la Isla del Paraíso.
\vs p041 5:7 \pc Podría parecer que la energía solar se propaga en ondas, pero esta impresión se debe a la acción y coexistencia de diversos tipos de influencias. Si se observa detenidamente una determinada forma de energía organizada, se percibirá que no se mueve en ondas sino en línea recta. Es la presencia de una segunda o tercera forma de fuerza\hyp{}energía la que puede dar lugar a que la corriente energética \bibemph{parezca} viajar en forma de ondas. Esto mismo le sucede al agua de lluvia que cae en una tormenta cerrada cuando viene acompañada de fuertes vientos. La lluvia al caer parece a menudo formar una cortina de agua u ondearse por el mismo viento. No obstante, las gotas de lluvia caen verticalmente de forma ininterrumpida; es la acción del viento la que produce ese visible efecto de cortina de agua y de ondas.
\vs p041 5:8 Es tal la acción de ciertas energías de orden secundario y de otras no descubiertas, presentes en las regiones espaciales de vuestro universo local, que en las emanaciones de luz solar parecen producirse ciertos fenómenos de ondulación al igual que una fragmentación en porciones infinitesimales de longitud y peso definidas. Y, en un sentido práctico, eso es exactamente lo que sucede. Es difícil que podáis alcanzar a entender mejor el comportamiento de la luz hasta ese momento en el que lleguéis a tener una más clara noción de la interacción e interrelación de las distintas fuerzas del espacio y energías solares que operan en las regiones espaciales de Nebadón. Vuestra confusión actual también se debe a un incompleto conocimiento de esta cuestión en lo que respecta a la actividad correlacionada que se establece en la dirección personal y no personal del universo matriz ---presencias, actuaciones y coordinación del Actor Conjunto y del Absoluto Indeterminado---.
\usection{6. EL CALCIO: EL ERRANTE DEL ESPACIO}
\vs p041 6:1 Al interpretar los fenómenos espectrales, debe recordarse que el espacio no está vacío; que la luz, al atravesar el espacio organizado, sufre una ligera modificación como producto de las distintas formas de energía y materia que circulan por todo él. Algunas de las líneas, indicativas de materia desconocida, que aparecen en el espectro de vuestro Sol se deben a modificaciones de elementos bien conocidos que flotan reducidos a añicos por todo el espacio; se trata de remanentes fragmentarios de las violentas conflagraciones solares que ocurren de forma natural. El espacio está surcado de estos residuos errantes, especialmente de sodio y calcio.
\vs p041 6:2 El calcio es, de hecho, el elemento principal de la materia que impregna el espacio de todo Orvontón. Todo nuestro suprauniverso está salpicado de piedra sumamente pulverizada. La piedra es literalmente el material básico usado para la construcción de los planetas y de las esferas del espacio. La nube cósmica, el gran manto espacial, está compuesta, en su mayor parte, de átomos modificados de calcio. El átomo de piedra es uno de los elementos de la materia más extendidos y persistentes. No solo resiste la ionización solar ---desintegración parcial--- sino que mantiene su identidad asociativa incluso tras haber sido sacudido por los destructivos rayos X y fragmentado por las altas temperaturas solares. El calcio posee una individualidad y una longevidad sin parangón entre las otras formas más comunes de la materia.
\vs p041 6:3 \pc Tal como vuestros físicos suponen, estos mermados restos de calcio solar literalmente viajan en los rayos de luz recorriendo distancias diferentes, lo que facilita enormemente su amplia diseminación por todo el espacio. El átomo de sodio, con ciertas modificaciones, es también capaz de trasladarse de un lugar a otro en la luz y en la energía. La proeza del calcio es incluso más extraordinaria ya que su masa duplica a la del sodio. La impregnación de parte del calcio del espacio local se debe al hecho de que, bajo una forma modificada, se escapa de la fotosfera solar viajando, en el más estricto sentido de la palabra, en los rayos solares salientes. De todos los elementos solares, el calcio, a pesar de su masa relativa ---puesto que contiene veinte electrones giratorios---, es el que más éxito tiene en lograr escapar del interior del Sol en dirección hacia los reinos del espacio. Esto explica por qué existe una capa de calcio, una superficie de piedra gaseosa, en el sol, de casi diez mil kilómetros de espesor; y esto a pesar del hecho de que hay diecinueve elementos más ligeros, y numerosos otros más pesados, por debajo de esta capa.
\vs p041 6:4 El calcio es un elemento activo y versátil cuando se somete a las temperaturas solares. El átomo de piedra tiene dos electrones ágiles y levemente enlazados en sus órbitas externas, que están muy próximos el uno del otro. En la reacción atómica, este átomo pierde pronto su electrón externo; tras lo cual, sucede en él un imponente proceso de moción continua del electrón número diecinueve de adelante y atrás entre la órbita electrónica diecinueve y la órbita veinte en su movimiento alrededor del núcleo. Este electrón, al oscilar con dicho movimiento más de veinticinco mil veces por segundo entre su propia órbita y la del otro electrón, y acompañante perdido, hace que el mermado átomo de piedra sea capaz de vencer en parte a la gravedad y emprender así con éxito un viaje en las corrientes emergentes de luz y energía, entre los rayos del sol, hacia la libertad y la aventura. Este átomo de calcio se mueve hacia fuera impulsándose mediante sacudidas alternas, asiendo y soltando el rayo de Sol unas veinticinco mil veces por segundo. Y esta es la razón por la que la piedra es el principal componente de los mundos del espacio. La piedra tiene una sobresaliente capacidad para escapar de su prisión solar.
\vs p041 6:5 La adaptabilidad de este diestro electrón de calcio se refleja en el hecho de que, cuando las fuerzas solares de los rayos X y de las temperaturas lo lanzan al círculo de la órbita superior, solo permanece en esta una millonésima de segundo, pero antes de que la potencia de la gravedad eléctrica del núcleo atómico lo haga retroceder a su antigua órbita, es capaz de completar un millón de rotaciones alrededor del centro del átomo.
\vs p041 6:6 \pc Vuestro Sol se ha desprendido de un volumen enorme de su calcio; perdió cantidades inmensas en las épocas de incontrolables erupciones durante la formación del sistema solar. Una gran parte del calcio solar se halla ahora en la corteza exterior del sol.
\vs p041 6:7 \pc Se debe recordar que los análisis del espectro solar solamente muestran la composición de la superficie del sol. Por ejemplo: estos espectros presentan muchas líneas de hierro, pero el hierro no es el elemento principal del sol. Este fenómeno se debe casi exclusivamente a la temperatura actual de la superficie del sol, algo menos de 3315 grados (C), la cual es muy favorable para el registro del espectro del hierro.
\usection{7. FUENTES DE LA ENERGÍA SOLAR}
\vs p041 7:1 La temperatura interna de muchos soles, incluido el vuestro, es mucho más alta de lo que comúnmente se cree. En el interior de los soles no existen prácticamente átomos enteros; todos están más o menos desintegrados por el intenso bombardeo de los rayos X, algo propio de temperaturas tan altas. Sean cuales fueran los elementos materiales que puedan aparecer en las capas exteriores de los soles, los que están en el interior siguen un proceso muy similar por la acción disociativa de los disruptivos rayos X. Estos rayos son los grandes niveladores de la estructura del átomo.
\vs p041 7:2 La temperatura de la superficie de vuestro Sol es de casi 3315 grados (C), pero a medida que se penetra en el interior esta se incrementa rápidamente hasta alcanzar la increíble cifra de aproximadamente 19\,440\,000 grados (C), en las regiones centrales. (Todas estas temperaturas están expresadas en grados Celsius).
\vs p041 7:3 \pc Todos estos fenómenos son indicios de un ingente consumo de energía, y las fuentes de energía solar, por orden de importancia, son:
\vs p041 7:4 \li{1.}La completa destrucción de los átomos y, finalmente, de los electrones.
\vs p041 7:5 \li{2.}La transmutación de los elementos, incluyendo el grupo radioactivo de energías liberadas de este modo.
\vs p041 7:6 \li{3.}La acumulación y transmisión de ciertas energías espaciales universales.
\vs p041 7:7 \li{4.}La materia espacial y los meteoros que incesantemente se precipitan sobre los abrasadores soles.
\vs p041 7:8 \li{5.}La contracción solar; el enfriamiento y la consiguiente contracción de los soles producen energía y calor a veces en una cantidad superior a la que proporciona la materia espacial.
\vs p041 7:9 \li{6.}La acción de la gravedad a altas temperaturas transforma ciertas vías circulatorias de la potencia en energías radiantes.
\vs p041 7:10 \li{7.}La luz y alguna otra materia recapturadas que son atraídas de nuevo al Sol tras haber salido de él, junto con otras energías que tienen un origen extrasolar.
\vs p041 7:11 \pc Existe una capa reguladoras de gases calientes (a veces con millones de grados de temperatura) que envuelve a los soles, y que actúa para estabilizar la pérdida de calor, aparte de prevenir las impredecibles fluctuaciones producidas por la disipación del calor. Durante la vida activa de los soles, la temperatura interna, de 19\,440\,000 grados (C), continúa siendo más o menos la misma a pesar del progresivo descenso de la temperatura externa.
\vs p041 7:12 \pc Tratad de imaginaros 19\,440\,000 grados (C) de calor, conjuntamente con ciertas presiones de la gravedad, como el punto de ebullición electrónico. Bajo dicha presión y a esa temperatura todos los átomos se degradan y desintegran en sus componentes electrónicos y en otros componentes ancestrales; incluso los electrones y otras agrupaciones de ultimatones pueden desintegrarse, pero la acción de los soles no trae consigo la degradación de los ultimatones.
\vs p041 7:13 Estas temperaturas solares actúan acelerando enormemente a los ultimatones y a los electrones, al menos a aquellos de estos últimos que continúan existiendo en estas condiciones. Os daréis cuenta de lo que las altas temperaturas traen consigo en cuanto a la aceleración de la actividad de ultimatones y electrones cuando os detengáis a considerar que una gota de agua común contiene más de mil trillones de átomos. Esto equivale a la energía que producirían más de cien caballos de vapor durante dos años seguidos. El calor total que en la actualidad emite el Sol de vuestro sistema solar cada segundo es suficiente como para hacer hervir toda el agua de todos los océanos de Urantia en tan solo un segundo de tiempo.
\vs p041 7:14 \pc Solamente los soles que están en operación en los canales directos de las corrientes principales de la energía del universo pueden brillar para siempre. Estos hornos solares arden de forma indefinida porque tienen la capacidad de reponer su pérdida de materia tomando energía de la fuerza espacial y de energías análogas en circulación. Pero las estrellas que están muy apartadas de estos canales principales de recarga están llamadas a sufrir el agotamiento de su energía ---esto es, a enfriarse de forma gradual y a acabar por apagarse---.
\vs p041 7:15 Estos soles muertos o moribundos pueden vigorizarse por el impacto de alguna colisión o pueden recargarse debido a la acción de ciertas islas del espacio de energía no luminosa o apropiándose, por medio de la gravedad, de soles o sistemas cercanos más pequeños. La mayoría de los soles muertos siguen este u otros medios evolutivos para reactivarse. Aquellos soles que finalmente no consiguen recargarse de este modo están llamados a desintegrarse por explosión de su masa, cuando la condensación gravitatoria alcance el nivel crítico de la condensación ultimatónica por presión de la energía. Estos soles desaparecidos se convierten, de este modo, en una de las formas más raras de energía, magníficamente adaptada para energizar a otros soles mejor situados.
\usection{8. REACCIONES DE LA ENERGÍA SOLAR}
\vs p041 8:1 En aquellos soles que están dentro de la vía circulatoria de los canales de energía espacial, la energía solar se libera por medio de varias cadenas complejas de reacción nuclear, la más común de las cuales es la reacción de hidrógeno\hyp{}carbono\hyp{}helio. En esta metamorfosis, el carbono actúa como catalizador de la energía, puesto que, en realidad, no sufre cambio alguno durante ese proceso de conversión del hidrógeno en helio. En ciertas condiciones de altas temperaturas, el hidrógeno penetra en los núcleos del carbono; y puesto que el carbono no puede contener más de cuatro de dichos protones, cuando se alcanza este estado de saturación, comienza a emitir protones con tanta rapidez como llegan los nuevos. En esta reacción las partículas entrantes de hidrógeno salen como átomos de helio.
\vs p041 8:2 \pc La reducción del contenido de hidrógeno aumenta la luminosidad de los soles. En esos soles llamados a apagarse, el punto álgido de su luminosidad se alcanza en el momento del agotamiento del hidrógeno. Con posterioridad a esta circunstancia, el brillo se mantiene como resultado del proceso de contracción gravitatoria. Tal estrella acabará por convertirse en lo que se denomina una estrella enana blanca, esto es, en una esfera extremadamente condensada.
\vs p041 8:3 \pc En grandes soles ---pequeñas nebulosas circulares--- cuando se agota el hidrógeno y comienza a tener efecto la contracción gravitatoria, si dicho cuerpo no es lo suficientemente opaco como para retener la presión interna que da sostén a las regiones gaseosas exteriores, se produce un colapso repentino. Los cambios electro\hyp{}gravitatorios dan origen a grandes cantidades de pequeñas partículas desprovistas de potencial eléctrico, y tales partículas se escapan rápidamente del interior del sol, ocasionando así, en pocos días, el desplome de un Sol gigantesco. Fue la emigración de estas “partículas fugitivas” la que ocasionó el desmoronamiento de la nova gigante de la nebulosa Andrómeda hace unos cincuenta años. Este enorme cuerpo estelar colapsó en cuarenta minutos del tiempo de Urantia.
\vs p041 8:4 Por regla general, en los soles residuales en enfriamiento sigue existiendo una inmensa eyección de materia bajo la forma de extensas nubes de gases nebulares. Y todo esto explica el origen de muchos tipos de nebulosas irregulares, como la nebulosa del Cangrejo que se originó hace aproximadamente novecientos años, y en la que todavía se hace patente su esfera matriz, como una estrella solitaria cerca del centro de esta irregular masa nebular.
\usection{9. LA ESTABILIDAD DE LOS SOLES}
\vs p041 9:1 Los soles más grandes ejercen tal control gravitatorio sobre sus electrones que la luz solamente puede escaparse gracias a la ayuda de los poderosos rayos X. Dichos rayos penetran por todo el espacio y están implicados en el mantenimiento de las agrupaciones ultimatónicas básicas de la energía. Las grandes pérdidas de energía del Sol en sus primeros días, tras alcanzar su máxima temperatura ---más de 19\,440\,000 grados (C)---, no se deben tanto al escape de luz como a la fuga de ultimatones. Para emprender la aventura de su conjunción electrónica y materialización de la energía, estas energías ultimatónicas se escapan hacia el espacio causando, en los soles más jóvenes, una auténtica explosión de energía.
\vs p041 9:2 \pc Los átomos y los electrones están sujetos a la gravedad. Los ultimatones \bibemph{no} están sujetos a la gravedad local, a la interacción de la atracción material, pero si obedecen enteramente a la gravedad absoluta o gravedad del Paraíso, a la dirección, a la moción en arco, del círculo universal y eterno del universo de los universos. La energía ultimatónica no responde a la atracción lineal o directa de la gravedad de las masas materiales cercanas o lejanas, sino que siempre se impulsa siguiendo la vía circular de la gran elipse de la extensa creación.
\vs p041 9:3 \pc Vuestro propio centro solar irradia casi cien mil millones de toneladas de materia real cada año, mientras que los soles gigantescos pierden materia a un enorme ritmo durante su etapa de crecimiento inicial, durante sus primeros mil millones de años. La vida de un Sol se estabiliza una vez que su temperatura alcanza el grado máximo y se empiezan a liberar las energías subatómicas. Es precisamente en este punto crítico cuando los soles más grandes experimentan sus incontrolables y violentas pulsaciones.
\vs p041 9:4 La estabilidad del Sol depende por completo del equilibrio entre dos fuerzas oponentes como son la gravedad y el calor ---unas formidables presiones contrabalanceadas por temperaturas inimaginables---. La elasticidad del gas interior de los soles da sostén a las capas superpuestas de diversos materiales, y cuando la gravedad y el calor están en equilibrio, el peso de los materiales exteriores iguala exactamente la presión de la temperatura de los gases que subyacen en el interior. En muchas de las estrellas más jóvenes, la persistente condensación de la gravedad produce temperaturas internas en constante aumento y, a medida que aumenta el calor interior, la presión interior que producen los rayos X en conjunto con los vientos de extraordinarios gases, llega a ser de tal magnitud que, en conjunción con el movimiento centrífugo, los soles comienzan a arrojar sus capas externas hacia el espacio, compensando así la falta de equilibrio entre la gravedad y el calor.
\vs p041 9:5 Hace mucho tiempo que vuestro Sol alcanzó un equilibrio relativo entre sus ciclos de expansión y contracción, esas perturbaciones que producen las gigantescas pulsaciones de muchas de las estrellas más jóvenes. Vuestro Sol está completando ahora sus seis mil millones de años de existencia. En este momento, pasa por su periodo de mayor eficacia en el uso de la energía. Durante más de veinticinco mil millones de años, brillará con la eficiencia que le caracteriza ahora. Es probable que experimente un período de declive parcial de esta eficiencia tan largo como la combinación de los dos períodos, el de su juventud y el de su funcionamiento estabilizado.
\usection{10. ORIGEN DE LOS MUNDOS HABITADOS}
\vs p041 10:1 Algunas de las estrellas variables, en estado de máxima pulsación o cercanas a dicho estado, están dando origen a sistemas secundarios, muchos de los cuales acabarán siendo como vuestro propio Sol y sus planetas circundantes. Vuestro Sol se encontraba en ese preciso estado de potente pulsación cuando el masivo sistema de Angona se colocó muy cerca, y la superficie externa del Sol comenzó a arrojar auténticas corrientes ---capas continuas--- de materia. Esto prosiguió con una violencia en aumento hasta que se produjo una mayor cercanía de Angona, momento en el que el Sol alcanzó su límite de cohesión y un enorme pináculo de materia, el ancestro de vuestro sistema solar, salió despedido. En circunstancias similares, la máxima aproximación de la atrayente masa provoca que se desprendan planetas enteros, incluso una cuarta parte o un tercio del sol. Estas grandes eyecciones forman ciertos tipos peculiares de mundos rodeados de nubes, de esferas muy parecidas a Júpiter y Saturno.
\vs p041 10:2 La mayoría de los sistemas solares, sin embargo, tuvieron un origen enteramente diferente al vuestro, y esto es también cierto para aquellos que se crearon mediante las mareas gravitatorias. Pero cualquiera que sea el modo en el que se generen los mundos, la gravedad es siempre causa de la creación de algún tipo de sistema solar; o sea, un Sol central o isla oscura con planetas, satélites, subsatélites y meteoros.
\vs p041 10:3 \pc El aspecto físico de los diferentes mundos está, en buena parte, determinado por su modo de origen, su situación astronómica y su entorno físico. Otros factores determinantes son la edad, el tamaño, el índice de rotación y la velocidad a la que se desplaza por el espacio. Tanto los mundos que se originaron por contracción gaseosa como aquellos que lo hicieron por un gradual incremento de materia sólida se caracterizan por tener montañas y, durante su vida primitiva, si no son demasiado pequeños, por tener agua y aire. Los mundos que resultan de derretimiento\hyp{}división o de colisiones carecen a veces de extensas cadenas montañosas.
\vs p041 10:4 Durante las primeras eras de todos estos nuevos mundos, los terremotos son frecuentes, y todos ellos se distinguen por grandes perturbaciones físicas. Esto ocurre de manera especial en esas esferas que se formaron por contracción de gases; se trata de mundos nacidos de los inmensos anillos nebulares que quedan a raíz de la condensación y contracción tempranas de algunos soles determinados. Los planetas que tienen un doble origen, como Urantia, pasan por unas etapas de juventud menos violentas y tempestuosas. No obstante, vuestro mundo experimentó una fase primitiva de grandes cataclismos, caracterizada por erupciones volcánicas, terremotos, inundaciones y formidables tormentas.
\vs p041 10:5 \pc Urantia está relativamente aislada en las lindes de Satania, y vuestro sistema solar es, con una excepción, el que se encuentra más alejado de Jerusem, mientras que el mismo sistema local de Satania está próximo al sistema más exterior de la constelación Norlatiadec, que a su vez se desplaza por el borde externo de Nebadón. Verdaderamente, estabais entre las criaturas más humildes de toda la creación hasta que Miguel, al darse de gracia, elevó a vuestro planeta a una posición de honor que atrajo poderosamente el interés de todo el universo. A veces el último es el primero, al igual que el más humilde será en verdad el más grande.
\vsetoff
\vs p041 10:6 [Exposición de un arcángel que colabora con el jefe de los centros de la potencia de Nebadón.]
