\upaper{78}{La raza violeta después de los días de Adán}
\author{Arcángel}
\vs p078 0:1 El segundo Edén fue la cuna de la civilización durante casi treinta mil años. Los pueblos adánicos continuaron aquí en Mesopotamia, enviando a sus descendientes hasta los confines de la tierra; los mismos que, más tarde, al mezclarse con las tribus noditas y sangiks se llegarían a conocer como “anditas”. De esta región, salieron esos hombres y mujeres que propiciaron los logros de los tiempos históricos y que aceleraron tan enormemente el progreso cultural de Urantia.
\vs p078 0:2 Este escrito narra la historia planetaria de la raza violeta, comenzando poco después de la transgresión de Adán, hacia el año 35\,000 a. C., pasando por su unión con las razas noditas y sangiks, sobre el año 15\,000 a. C., para formar los pueblos anditas, hasta su desaparición final de las tierras natales mesopotámicas, alrededor del año 2000 a. C.
\usection{1. DISTRIBUCIÓN RACIAL Y CULTURAL}
\vs p078 1:1 Aunque las mentes y la moral de las razas estaban en un deficiente nivel en el momento de la llegada de Adán, la evolución física no se había visto demasiado afectada por las vicisitudes de la rebelión de Caligastia. Y la contribución de Adán a las condiciones biológicas de las razas, a pesar del fracaso parcial de su misión, mejoró enormemente a los pueblos de Urantia.
\vs p078 1:2 Igualmente, Adán y Eva contribuyeron de forma notable a todo aquello de valor para el progreso social, moral e intelectual de la humanidad; la presencia de su progenie aceleró sumamente la civilización. Si bien, hace treinta y cinco mil años, el mundo en su conjunto estaba poco culturizado. Había aquí y allí algunos centros de civilización, pero la mayor parte de Urantia languidecía en el salvajismo. La distribución racial y cultural era la siguiente:
\vs p078 1:3 \li{1.}\bibemph{La raza violeta: adanitas y adanecitas}. El centro principal de la cultura adanita estaba en el segundo jardín, localizado en el triángulo de los ríos Tigris y Éufrates, de hecho la cuna de las civilizaciones occidental e india. El centro secundario o septentrional de la raza violeta era la sede adanecita, situada al este de la costa meridional del Mar Caspio, cerca de los montes Kopet. De estos dos centros, se expandieron, a las tierras circundantes, la cultura y el plasma vital que, de forma tan inmediata, vivificaron a todas las razas.
\vs p078 1:4 \li{2.}\bibemph{Los presumerios y otros noditas}. También estaban presentes en Mesopotamia, cerca de la desembocadura de los ríos, los restos de la ancestral cultura de los días de Dalamatia. Con el paso de los milenios, este grupo se llegaría a mezclar por completo con los adanitas del norte, pero nunca perdieron del todo sus tradiciones noditas. Varios otros grupos noditas, que se habían establecido en el Levante, fueron, por lo general, absorbidos racialmente por la raza violeta al expandirse después.
\vs p078 1:5 \li{3.}\bibemph{Los andonitas} mantenían cinco o seis asentamientos bastante representativos al norte y al este de la sede de Adánez. También estaban dispersos por todo el Turquestán, en tanto que persistían comunidades aisladas por toda Eurasia, especialmente en las regiones montañosas. Estos aborígenes seguían ocupando las tierras del norte del continente eurasiático, junto con Islandia y Groenlandia, pero hacía mucho tiempo que habían sido expulsados de las llanuras europeas por el hombre azul y de los valles fluviales de la más distante Asia por la raza amarilla en su expansión.
\vs p078 1:6 \li{4.}\bibemph{El hombre rojo} ocupaba las Américas, al haber sido expulsado de Asia más de cincuenta mil años antes de la llegada de Adán.
\vs p078 1:7 \li{5.}\bibemph{La raza amarilla}. Los pueblos chinos habían consolidado bien sus dominios en Asia oriental. Sus asentamientos más avanzados estaban situados al noroeste de la China moderna en regiones fronterizas con el Tíbet.
\vs p078 1:8 \li{6.}\bibemph{La raza azul}. Los hombres azules estaban dispersos por toda Europa, pero sus mejores centros de cultura se localizaban en los entonces fértiles valles de la cuenca mediterránea y en el noroeste de Europa. La absorción racial de los neandertales había retrasado notablemente su cultura; pero, con esta salvedad, los hombres azules eran los más combativos, aventureros y exploradores de todos los pueblos evolutivos de Eurasia.
\vs p078 1:9 \li{7.}\bibemph{La India pre\hyp{}dravídica}. La compleja mezcla de razas de la India ---que englobaba a todas las razas de la tierra, pero en especial a la verde, la naranja y la negra--- mantenía una cultura ligeramente superior a la de las regiones periféricas.
\vs p078 1:10 \li{8.}\bibemph{La civilización sahariana}. Los componentes mejor dotados de la raza índigo tenían sus asentamientos más avanzados en lo que hoy día es el gran desierto del Sahara. Este grupo índigo\hyp{}negro portaba un considerable linaje de las asimiladas razas naranja y verde.
\vs p078 1:11 \li{9.}\bibemph{La cuenca del Mediterráneo}. La raza más mezclada, dejando la India al margen, ocupaba lo que actualmente es la cuenca mediterránea. Aquí los hombres azules del norte y los saharianos del sur se encontraron y se mezclaron con los noditas y adanitas del este.
\vs p078 1:12 \pc Este era el panorama del mundo con anterioridad al comienzo de las grandes expansiones de la raza violeta, hace unos veinticinco mil años. La esperanza de una civilización futura estaba en el segundo jardín, entre los ríos de Mesopotamia. Aquí, en el suroeste de Asia, existía el potencial de una gran civilización, la posibilidad de que se extendiesen por el mundo las ideas y los ideales rescatados de los días de Dalamatia y de los tiempos de Edén.
\vs p078 1:13 Adán y Eva habían dejado atrás una limitada, pero poderosa progenie, y los observadores celestiales apostados en Urantia esperaban con ansiedad conocer el comportamiento de los descendientes de los descarriados hijo e hija materiales.
\usection{2. LOS ADANITAS EN EL SEGUNDO JARDÍN}
\vs p078 2:1 Durante miles de años, los hijos de Adán trabajaron a lo largo de los ríos de la Mesopotamia, solucionando sus problemas de riego y de control de inundaciones en el sur, perfeccionando sus defensas en el norte e intentando conservar sus tradiciones de la gloria del primer Edén.
\vs p078 2:2 El heroísmo mostrado en el liderazgo del segundo jardín constituye una de las epopeyas más asombrosas e inspiradora de la historia de Urantia. Estas espléndidas almas nunca perdieron enteramente de vista el propósito de la misión adánica y, por lo tanto, con valentía se defendieron de las influencias de las tribus peor dotadas de los alrededores, a la vez que enviaban voluntariamente un flujo constante de sus más selectos hijos e hijas como emisarios a las razas de la tierra. A veces, esta expansión disminuía su misma cultura natal, pero siempre estos pueblos mejor dotados podían rehabilitarse a sí mismos.
\vs p078 2:3 El estatus de la civilización, sociedad y cultura de los adanitas era muy superior al nivel general que el de las razas evolutivas de Urantia. Solo entre los antiguos asentamientos de Van y Amadón y los adanecitas había una civilización de alguna manera equiparable a la suya. Pero su civilización tenía un armazón artificial ---\bibemph{no se había desarrollado}--- y, por lo tanto, estaba condenada a sucumbir si no alcanzaba un nivel evolutivo natural.
\vs p078 2:4 Adán dejó tras él una gran cultura intelectual y espiritual, pero no era avanzada en cuanto a utensilios mecánicos, puesto que cada civilización está limitada por los recursos naturales disponibles, el ingenio inherente y el ocio suficiente como para garantizar que la inventiva dé sus frutos. La civilización de la raza violeta se basaba en la presencia de Adán y en las tradiciones del primer Edén. Tras su muerte y, a medida que las tradiciones se iban difuminando con el paso de los milenios, el nivel cultural de los adanitas se fue deteriorando continuamente, hasta darse un mutuo equilibrio entre las condiciones en las que estaban los pueblos de alrededor y las habilidades culturales de la raza violeta, que evolucionaban de forma natural.
\vs p078 2:5 No obstante, hacia el año 19\,000 a. C., los adanitas eran una verdadera nación. Su número ascendía a cuatro millones y medio de personas, y ya habían enviado a millones de sus vástagos a los pueblos del entorno.
\usection{3. LAS PRIMERAS EXPANSIONES DE LOS ADANITAS}
\vs p078 3:1 La raza violeta conservó la tradición pacifista de Edén durante muchos milenios, lo que explica su gran demora en emprender conquistas territoriales. Cuando sufrían la presión demográfica, en lugar de hacer la guerra y conseguir más territorio, enviaban el excedente de sus habitantes como maestros a las otras razas. El efecto cultural de estas primeras expansiones no era permanente, pero la absorción racial de los maestros, comerciantes y exploradores adanitas fue biológicamente vigorizante para los pueblos de los alrededores.
\vs p078 3:2 Algunos de los adanitas viajaron tempranamente rumbo oeste al valle del Nilo; otros, una minoría, lo hicieron en dirección este, penetrando en Asia. En días posteriores, hubo un amplio traslado masivo hacia el norte y, desde allí, se encaminaron al oeste. Por lo general, consistía en un avance paulatino pero incesante hacia el norte; la mayoría se abrió camino en esa dirección y, luego, bordeando por el oeste el Mar Caspio, se adentraron en Europa.
\vs p078 3:3 Hace unos veinticinco mil años, muchos de los adanitas de más puro linaje habían recorrido una buena distancia en su viaje hacia el norte y, a medida que se adentraban siguiendo este rumbo, se volvían cada vez menos adánicos, hasta que, en los tiempos de su ocupación de Turquestán, se habían mezclado totalmente con las otras razas, particularmente con los noditas. Fueron pocos los pueblos de pura estirpe violeta que llegarían a penetrar profundamente en Europa o Asia.
\vs p078 3:4 Desde cerca del año 30\,000 hasta el 10\,000 a. C., estaban teniendo lugar, en todo el suroeste de Asia, unas trascendentales mezclas raciales. Los habitantes de las altiplanicies de Turquestán eran un pueblo poderoso y enérgico. Al noroeste de la India perduraba gran parte de la cultura de los días de Van. Más al norte de estos asentamientos se había conservado lo mejor de los primitivos andonitas. Y estas dos razas, de superior cultura y carácter fueron absorbidas racialmente por los adanitas en su desplazamiento al norte. Este mestizaje resultó en la adopción de muchas ideas nuevas; propició el progreso de la civilización e hizo avanzar notablemente el arte, las ciencias y la cultura social en todas sus facetas.
\vs p078 3:5 \pc Al acabar el período de las primeras migraciones adánicas, alrededor del año 15\,000 a. C., había ya más descendientes de Adán en Europa y Asia central que en cualquier otro lugar del mundo, incluso que en Mesopotamia. Las razas azules europeas se habían cruzado en gran parte con los adanitas. Las áreas meridionales de las tierras hoy en día conocidas como Rusia y Turquestán estaban ocupadas por una gran acumulación de adanitas mezclados con noditas, andonitas y sangiks rojos y amarillos. Europa meridional y la franja Mediterránea estaban bajo la dominación de una raza mixta de pueblos andonitas y sangiks ---naranjas, verdes e índigos--- con una pequeña proporción de la estirpe adanita. Asia Menor y las tierras de Europa central y oriental estaban en poder de tribus predominantemente andonitas.
\vs p078 3:6 Una raza de color, mixta, fortalecida considerablemente sobre esta época por los viajeros llegados de Mesopotamia, se estableció en Egipto y se preparó para hacerse con el control de la cultura en proceso de desaparición del valle del Éufrates. Los pueblos negros se adentraron cada vez más en el sur de África y, tal como la raza roja, quedaron prácticamente aislados.
\vs p078 3:7 La civilización sahariana se había visto perjudicada por las sequías y, la de la cuenca del Mediterráneo, por inundaciones. Las razas azules, hasta ese momento, no habían conseguido desarrollar una cultura avanzada. Los andonitas estaban todavía dispersos por la región ártica y de Asia central. Las razas verde y naranja habían sido exterminadas como tales. La raza índigo se fue trasladando al sur de África, en donde comenzaría su lento, pero largo y continuado deterioro racial.
\vs p078 3:8 Los pueblos de la India permanecían estancados, con una civilización que no progresaba; los hombres amarillos iban consolidando sus posesiones en Asia central; los hombres cobrizos aún no habían comenzado su civilización en las cercanas islas del Pacífico.
\vs p078 3:9 \pc Estas distribuciones raciales, en conjunción con los generalizados cambios climáticos, prepararon el escenario mundial para la inauguración de la era andita de la civilización urantiana. Estas tempranas migraciones se extendieron durante un período de diez mil años, desde el año 25\,000 hasta el 15\,000 a. C. Las migraciones posteriores, la de los anditas, se prolongaron desde alrededor del año 15\,000 hasta el 6000 a. C.
\vs p078 3:10 Las primeras oleadas de adanitas tardaron tanto tiempo en atravesar Eurasia que, en gran medida, su cultura se perdió en el trayecto. Solo los posteriores anditas se desplazaron con la velocidad suficiente como para conservar la cultura edénica a gran distancia de Mesopotamia.
\usection{4. LOS ANDITAS}
\vs p078 4:1 Las razas anditas eran primordialmente una mezcla de la raza violeta de pura estirpe con los noditas, pero también incluía una mezcla con los pueblos evolutivos. En general, se debe tener en consideración que los anditas tenían una proporción de sangre adánica mucho mayor que la de las razas modernas. Esencialmente se emplea el término “andita” para designar a aquellos pueblos cuya herencia racial era entre una sexta y una octava parte violeta. Los urantianos modernos, incluso las razas blancas del norte, contienen una proporción de sangre de Adán mucho menor que la de ellos.
\vs p078 4:2 Los primeros pueblos anditas tuvieron su origen en las regiones colindantes con Mesopotamia, hace más de veinticinco mil años; eran una mezcla de adanitas y noditas. El segundo jardín estaba rodeado por zonas circundantes de cada vez menos sangre violeta y, en la periferia de este crisol racial, nació la raza andita. Más tarde, cuando, en sus movimientos migratorios, los adanitas y noditas entraron en las entonces fértiles regiones de Turquestán, no tardaron en mezclarse con sus habitantes mejor dotados y la mezcla racial resultante extendió el linaje andita en dirección norte.
\vs p078 4:3 Los anditas constituían la mejor y más completa estirpe humana que había aparecido desde los tiempos de los pueblos de linaje violeta puro. Abarcaban la mayoría de los tipos superiores de los restos supervivientes de las razas adanita y nodita y, más adelante, algunos de los mejores linajes de los hombres amarillos, azules y verdes.
\vs p078 4:4 \pc Estos anditas primitivos no eran arios; sino prearios. No eran blancos; sino preblancos. Como pueblo, no eran ni occidentales ni orientales. Pero, es la herencia andita la que confiere al cruce multirracial de las llamadas razas blancas esa generalizada homogeneidad que se ha denominado caucasoide.
\vs p078 4:5 \pc Los linajes más puros de la raza violeta habían conservado la tradición adánica de la búsqueda de la paz, lo que explica por qué sus primeras migraciones habían tenido más bien un carácter pacífico. Pero a medida que los adanitas se unían con las estirpes noditas, que para entonces era una raza beligerante, sus descendientes anditas se convirtieron, para su época, en los militaristas más diestros y sagaces jamás vistos en Urantia. A partir de entonces, los desplazamientos de los mesopotámicos eran, cada vez, más de carácter militar y más semejantes a verdaderas conquistas.
\vs p078 4:6 Estos anditas eran aventureros; tenían predisposición a la itinerancia. Un incremento de linaje sangik o andonita tendió a estabilizarlos. Pero incluso así, sus descendientes venideros no cesaron hasta haber circunnavegado el globo y descubrir el último continente remoto.
\usection{5. LAS MIGRACIONES ANDITAS}
\vs p078 5:1 La cultura del segundo jardín subsistió durante veinte mil años, pero sufrió un continuo declive hasta cerca del año 15\,000 a. C., momento en el que el resurgimiento del sacerdocio setita y el liderazgo de Amosad inauguraron una formidable era. Grandes oleadas civilizadoras, que luego se extenderían por Eurasia, siguieron de inmediato al gran renacimiento ocurrido en el Jardín, resultado de la copiosa unión de los adanitas con los mezclados noditas de los alrededores, que daría origen a los anditas.
\vs p078 5:2 Estos anditas introdujeron nuevos avances por toda Eurasia y África del norte. Desde Mesopotamia hasta Sinkiang, la cultura andita dominó, y sus constantes migraciones hacia Europa eran continuamente compensadas por los últimos inmigrantes llegados de Mesopotamia. Pero no es realmente correcto aludir a los anditas como una raza propiamente mesopotámica hasta casi al principio de las últimas migraciones de los descendientes de Adán. Para entonces, incluso las razas que se encontraban en el segundo jardín se habían mezclado de tal manera que ya no se podían considerar como adanitas.
\vs p078 5:3 La civilización del Turquestán se reavivaba y renovada constantemente por los recién llegados de Mesopotamia, en especial de los últimos jinetes anditas. La llamada lengua materna aria estaba en vías de formación en las altiplanicies del Turquestán; era una mezcla del dialecto andónico de esa región con el lenguaje de los adanecitas y de los posteriores anditas. Muchos idiomas modernos provienen del habla primitiva de estas tribus de Asia central que conquistaron Europa, la India y las franjas altas de las llanuras mesopotámicas. Esta ancestral lengua dio a los idiomas occidentales todas esas similitudes que las designan como “arias”.
\vs p078 5:4 \pc Hacia el año 12\,000 a. C., las tres cuartas partes de los descendientes andonitas del mundo residían en el norte y el este de Europa y, cuando se produjo el éxodo final desde Mesopotamia, entraron en Europa el sesenta y cinco por ciento de estas últimas oleadas migratorias.
\vs p078 5:5 \pc Los anditas no solo emigraron a Europa, sino también al norte de China y a la India, al tiempo que muchos de ellos se desplazaron hasta los confines de la tierra como misioneros, maestros y comerciantes. Hicieron una gran contribución a los grupos norteños de los pueblos sangiks del Sahara. Si bien, únicamente algunos pocos maestros y comerciantes se adentraron en África más al sur de las cabeceras del Nilo. Con posterioridad, anditas mestizos y egipcios siguieron las costas orientales y occidentales de África muy por debajo del ecuador, pero no llegaron hasta Madagascar.
\vs p078 5:6 Estos anditas fueron los conquistadores llamados dravidianos, y después arios, de la India; y su presencia en Asia central mejoró notablemente a los ancestros de los turanianos. Gran parte de esta raza viajó a China a través de Xinjiang y del Tíbet y añadió cualidades beneficiosas a las posteriores estirpes chinas. En ocasiones, pequeños grupos se abrieron camino hasta Japón, Formosa, las Indias Occidentales y el sur de China, aunque muy pocos entraron en el sur de China por la ruta costera.
\vs p078 5:7 Ciento treinta y dos miembros de esta raza, que se embarcaron en una flotilla de pequeñas embarcaciones desde el Japón, alcanzaron finalmente América del Sur y, mediante matrimonios mixtos con los nativos de los Andes, establecieron la ascendencia de los futuros gobernantes de los Incas. Cruzaron el Pacífico en pequeñas etapas, deteniéndose en las muchas islas que encontraban en el camino. Las islas del archipiélago polinesio eran más numerosas y más grandes entonces que ahora, y estos marineros anditas, junto con algunos seguidores, modificaron biológicamente, durante su trayecto, a las comunidades nativas. Como resultado del avance de los marineros anditas, crecieron muchos florecientes centros de civilización en estas tierras, ahora sumergidas. Durante mucho tiempo, la Isla de Pascua fue el centro religioso y administrativo de una de estas comunidades perdidas. Pero, de los anditas que navegaron el Pacífico de los pasados tiempos, únicamente esos ciento treinta y dos lograron alcanzar la parte continental de las Américas.
\vs p078 5:8 \pc Las conquistas migratorias de los anditas continuaron hasta sus últimas dispersiones, desde el año 8000 al 6000 a. C. A medida que salían de Mesopotamia, agotaban continuamente las reservas biológicas de sus tierras natales a la vez que fortalecían notablemente a los pueblos de los alrededores. Y a todas las naciones a las que viajaban, aportaban humor, arte, aventura, música y manufactura. Eran hábiles en la domesticación de animales y expertos agricultores. En este momento, al menos, su presencia habitualmente mejoraba las creencias religiosas y las prácticas morales de las razas más antiguas. Y así se difundió tranquilamente la cultura mesopotámica por Europa, la India, China, África del norte y las Islas del Pacífico.
\usection{6. LAS ÚLTIMAS DISPERSIONES ANDITAS}
\vs p078 6:1 Las tres últimas oleadas de los anditas salieron de Mesopotamia entre el año 8000 y el 6000 a. C. Las presiones de las tribus de las colinas del este y el hostigamiento de los hombres de las llanuras del oeste provocaron estas grandes oleadas culturales. Los habitantes del valle del Éufrates y del territorio contiguo acometieron su éxodo final siguiendo varias direcciones:
\vs p078 6:2 El sesenta y cinco por ciento entró en Europa por la ruta del Mar Caspio para conquistar y mezclarse con las recién aparecidas razas blancas ---el cruce de los hombres azules con los primeros anditas---.
\vs p078 6:3 El diez por ciento, incluyendo un numeroso grupo de sacerdotes setitas, se dirigió al este por las altiplanicies elamitas hasta la meseta iraní y el Turquestán. Muchos de sus descendientes fueron más tarde empujados a la India con sus hermanos arios de las regiones del norte.
\vs p078 6:4 El diez por ciento de los mesopotámicos, en su camino al norte, se desviaron al este y penetraron en Xinjiang, donde se mezclaron con los habitantes amarillos\hyp{}anditas. La mayoría de la capaz descendencia de esta unión racial entró posteriormente en China y contribuyó mucho al mejoramiento inmediato de la rama septentrional de la raza amarilla.
\vs p078 6:5 El diez por ciento de estos anditas que huían se abrieron paso a través de Arabia y entraron en Egipto.
\vs p078 6:6 \pc El cinco por ciento de los anditas, componentes de una cultura muy superior de la zona costera cercana a la desembocadura del Tigris y el Éufrates que se había guardado de matrimonios mixtos con miembros de las tribus inferiores, se negaron a abandonar sus hogares. Este grupo representaba la supervivencia de muchos linajes superiores de noditas y adanitas.
\vs p078 6:7 \pc Hacia el año 6000 a. C., los anditas habían evacuado casi enteramente esta región, aunque sus descendientes, mezclados en gran medida con las razas sangiks circundantes y con los andonitas de Asia Menor, estaban allí para luchar contra los invasores del norte y del este en una fecha muy posterior.
\vs p078 6:8 El creciente cruce con linajes menos dotados puso fin a la era cultural del segundo jardín. La civilización se desplazó hacia el oeste al Nilo y a las islas del Mediterráneo, donde continuó prosperando y avanzando mucho tiempo después de que su origen y fuente mesopotámicos hubiesen declinado. Y este flujo descontrolado de pueblos mal dotados preparó el terreno para la posterior conquista de toda Mesopotamia por parte de los incivilizados habitantes del norte, que expulsaron a las aptas estirpes que quedaban. Incluso años después, los pobladores remanentes cultos se vieron afectados por la presencia de estos invasores toscos e ignorantes.
\usection{7. LAS INUNDACIONES DE MESOPOTAMIA}
\vs p078 7:1 Los pobladores ribereños estaban acostumbrados al desbordamiento de los cauces fluviales en ciertas estaciones; estas inundaciones periódicas repercutían anualmente en sus vidas. Pero nuevos peligros amenazaban al valle de Mesopotamia como consecuencia de los paulatinos cambios geológicos que se daban en el norte.
\vs p078 7:2 Durante miles de años después del sumergimiento del primer Edén, las montañas de la costa oriental del Mediterráneo y las del noroeste y noreste de Mesopotamia continuaron elevándose. Dicho levantamiento de las altiplanicies se aceleró de forma considerable hacia el año 5000 a. C., algo que, junto con el notable aumento de las nevadas en las cordilleras del norte, provocaba que cada primavera se produjesen inundaciones sin precedentes por todo el valle del Éufrates. Estas inundaciones primaverales fueron paulatinamente a mayor, de manera que los habitantes de las regiones ribereñas se vieron finalmente empujados a las altiplanicies del este. Durante casi mil años, decenas de ciudades quedaron prácticamente desiertas a causa de estos copiosos diluvios.
\vs p078 7:3 \pc Casi cinco mil años después, cuando los sacerdotes hebreos cautivos en Babilonia trataron de rastrear el origen del pueblo judío hasta Adán, tuvieron grandes dificultades a la hora de encajar las partes de la historia; a uno de ellos se le ocurrió desistir de este empeño, permitir que todo el mundo se ahogara en su maldad en la época del diluvio de Noé, y estar así en una mejor posición para trazar el origen de Abraham hasta uno de los tres hijos supervivientes de Noé.
\vs p078 7:4 Son universales las leyendas que aluden a un tiempo en el que el agua cubría toda la superficie de la tierra. Muchas razas preservan el relato de un diluvio a escala mundial ocurrido durante ciertas épocas pasadas. La historia bíblica de Noé, el arca y el diluvio es un invento del sacerdocio hebreo durante su cautividad en Babilonia. No ha habido nunca tal inundación universal desde que se estableció la vida en Urantia. El único momento en el que la superficie de la tierra estuvo completamente cubierta de agua fue durante esas eras arqueozoicas antes de que las tierras empezaran a emerger.
\vs p078 7:5 Pero Noé vivió realmente; era un vinicultor de Aram, un asentamiento ribereño cerca de Erec. Llevaba un registro en el que anotaba de año en año los días de crecida del río. Fue muy ridiculizado porque iba rio arriba y río abajo recorriendo el valle fluvial, recomendando que todas las casas se construyeran de madera, en forma de barco, y que, al aproximarse la estación de las inundaciones, se subiera cada noche a ella a los animales domésticos. Cada año se desplazaba a los asentamientos ribereños colindantes y les advertía de que, en un número de días, vendrían las inundaciones. Finalmente, llegó un año en el que las inundaciones anuales aumentaron enormemente debido a inusuales lluvias torrenciales, de forma que la repentina crecida de las agua barrió la aldea por completo; solo Noé y su familia más directa, gracias a su casa flotante, lograron salvarse.
\vs p078 7:6 \pc Estas inundaciones terminaron de desvertebrar a la civilización andita. Al acabar este periodo de diluvios, el segundo jardín había dejado de existir. Solamente en el sur y entre los sumerios quedaba algún rastro de la antigua gloria.
\vs p078 7:7 Los restos de esta civilización, una de las más antiguas, se hallan en estas regiones de Mesopotamia al igual que al noreste y noroeste de ellas. Si bien, hay vestigios aún más antiguos de los días de Dalamatia bajo las aguas del Golfo Pérsico, y el primer Edén yace sumergido bajo el extremo oriental del Mar Mediterráneo.
\usection{8. LOS SUMERIOS: LOS ÚLTIMOS DE LOS ANDITAS}
\vs p078 8:1 Cuando la última dispersión de los anditas quebró el eje biológico principal de la civilización mesopotámica, una pequeña minoría de esta raza mejor dotada permaneció en su tierra natal cerca de las desembocaduras de los ríos. Se trataba de los sumerios que, hacia el año 6000 a. C., habían llegado a ser en gran parte de origen andita, aunque su cultura era más exclusivamente de carácter nodita, y se aferraban a las ancestrales tradiciones de Dalamatia. No obstante, estos sumerios de las regiones costeras eran los últimos anditas mesopotámicos. Si bien, las razas de Mesopotamia, para esta fecha tardía, estaban completamente mezcladas, como se desprende de los tipos de cráneos encontrados en las tumbas de ese tiempo.
\vs p078 8:2 Durante la época de las inundaciones, Susa prosperó notablemente. La primera ciudad, la más baja, se inundó, por lo que la segunda ciudad, y la más alta, tomó su lugar en cuanto a sede de las artesanías características de esos días. Cuando más tarde estas inundaciones disminuyeron, Ur se convertiría en el centro de la industria alfarera. Hace unos siete mil años, Ur estaba en la orilla del Golfo Pérsico, en un momento en que los depósitos fluviales aún no habían elevado el suelo hasta su altura actual. Estos asentamientos padecieron menos las inundaciones gracias a la mejora de las obras de contención y al ensanchamiento de las desembocaduras de los ríos.
\vs p078 8:3 \pc Los pacíficos cultivadores de granos de los valles de los ríos Éufrates y Tigris venían siendo hostigados, desde hacía mucho tiempo, por las redadas de los incivilizados pobladores del Turquestán y de la meseta iraní. Pero sucedió que la creciente sequía de los pastizales del altiplano trajo consigo una invasión conjunta del valle del Éufrates. Y esta invasión fue todavía más grave porque estos pastores y cazadores de los alrededores poseían un gran número de caballos domados, lo que les dio una ingente ventaja militar sobre sus ricos vecinos del sur. En poco tiempo, invadieron toda Mesopotamia, expulsando a los vencidos y propiciando con esto las últimas oleadas de cultura que se extenderían por toda Europa, Asia occidental y África del norte.
\vs p078 8:4 Estos conquistadores de Mesopotamia tenían entre sus filas a muchos de los mejores linajes anditas de las razas cruzadas del norte del Turquestán, incluyendo algunos de la estirpe adanecita. Dichas tribus norteñas, menos avanzadas pero más vigorosas, enseguida se mezclaron por voluntad propia con los remanentes de la civilización mesopotámica y pronto se convertirían en esos pueblos mestizos hallados en el valle del Éufrates, al comienzo de los anales de la historia. Rápidamente reavivaron muchos aspectos de la civilización mesopotámica ya en declive, adoptando las artes de las tribus del valle y una parte considerable de la cultura de los sumerios. Trataron incluso de construir una tercera torre de Babel y, más tarde, elegirían este término como nombre de su nación.
\vs p078 8:5 Cuando estos incivilizados jinetes del noreste invadieron por completo el valle del Éufrates, no lograron conquistar al resto de los anditas que habitaban en el Golfo Pérsico, en la desembocadura del río. Estos sumerios pudieron defenderse debido a su inteligencia superior, mejores armas y a un amplio sistema de zanjas militares, un complemento a su plan de regadío que interconectaba los estanques de agua. Formaban un pueblo unido porque tenían una religión colectiva y uniforme. Pudieron así mantener su integridad racial y nacional mucho tiempo después de que sus vecinos del noroeste se fragmentaran en ciudades estado. Ninguna de estas agrupaciones urbanas fue capaz de vencer a los sumerios unidos.
\vs p078 8:6 Y los invasores del norte aprendieron pronto a confiar y a valorar a estos sumerios amantes de la paz como capaces maestros y administradores. Fueron muy respetados y solicitados como maestros de las artes y de la manufacturación, como responsables del comercio y como gobernantes civiles por todos los pueblos del norte y desde Egipto en el oeste hasta India en el este.
\vs p078 8:7 Tras la disolución de la primera confederación sumeria, las siguientes ciudades estado se gobernaron por los descendientes apóstatas de los sacerdotes setitas. Estos sacerdotes solo se llamaban a sí mismos reyes cuando conquistaban las ciudades vecinas. Los reyes posteriores de estas ciudades no lograron formar confederaciones poderosas antes de los días de Sargón porque eran muy celosos de sus deidades. Cada ciudad creía que su dios comunal era superior al de los demás dioses y, por lo tanto, se negaban a someterse a un líder común.
\vs p078 8:8 Sargón, el sacerdote de Kish, que se autoproclamó rey y emprendió la conquista de toda Mesopotamia y de sus tierras limítrofes, puso fin a este largo período de gobiernos débiles de los sacerdotes metropolitanos. Y, por un tiempo, esto acabó con las ciudades estado, regidas y oprimidas por los sacerdotes, cada cual con su propio dios comunal y sus respectivas prácticas ceremoniales.
\vs p078 8:9 A la terminación de esta confederación de Kish le siguió un largo período de continuas guerras entre estas ciudades del valle por la supremacía. Y la soberanía alternó indistintamente entre Sumer, Akkad, Kish, Erec, Ur y Susa.
\vs p078 8:10 Sobre el año 2500 a. C., los sumerios sufrieron fuertes reveses a manos de los suitas y de los guitas del norte. Lagash, la capital sumeria construida en elevaciones aluviales de terreno, cayó. Erec resistió durante treinta años tras la caída de Akkad. En el momento de la instauración del reinado de Hamurabi, los semitas del norte habían absorbido racialmente a los sumerios, y los anditas mesopotámicos desaparecieron de las páginas de la historia.
\vs p078 8:11 Desde el año 2500 hasta el 2000 a. C., los nómadas arrasaron todo lo que encontraron a su paso desde el Atlántico hasta el Pacífico. Los neritas constituían el brote final del grupo del Mar Capio, descendientes mesopotámicos cruzados de las razas andita y andonita. Lo que a estos incivilizados seres les faltó por hacer para llevar a cabo la caída de Mesopotamia, lo harían los cambios climáticos
\vs p078 8:12 \pc Y esta es la historia de la raza violeta después de los días de Adán y de la suerte corrida por la tierra natal de esta raza, situada entre el Tigris y Éufrates. Su ancestral civilización acabó por desmoronarse debido a la emigración de los pueblos mejor dotados y a la inmigración de vecinos peor dotados. Pero mucho antes de que unos salvajes jinetes conquistaran el valle, gran parte de la cultura del jardín se había expandido a Asia, África y Europa, para aportar allí el fermento que daría como resultado la civilización del siglo XX de Urantia.
\vsetoff
\vs p078 8:13 [Exposición de un Arcángel de Nebadón.]
