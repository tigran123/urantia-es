\upaper{30}{Los seres personales del gran universo}
\author{Mensajero poderoso}
\vs p030 0:1 Los seres personales y las entidades distintas a las personales que en este momento desempeñan su labor en el Paraíso y en el gran universo suman un número casi ilimitado de seres vivos. Incluso el número de sus principales órdenes y tipos haría vacilar la imaginación humana, sin considerar los incontables subtipos y variaciones. Sin embargo, se hace deseable proporcionar alguna información sobre dos grupos principales de seres vivos y ofrecer, por un lado, una idea de la clasificación de los seres relacionados con el Paraíso y, por otro, un listado de los que constan en el registro de seres personales de Uversa.
\vs p030 0:2 No es posible hacer una clasificación completa y totalmente uniforme de los seres personales del gran universo porque no se han revelado \bibemph{todos} los grupos. Para clasificarlos de forma sistemática, haría falta añadir material revelado y aumentar el número de escritos. De todos modos, una ampliación de conceptos de este orden sería poco aconsejable porque, durante los próximos mil años, despojaría a los pensantes mortales del estímulo para la reflexión creativa que dichos conceptos, parcialmente revelados, les pueden proporcionar. Para no coartar la imaginación del hombre, es mejor no hacerle partícipe de demasiados elementos revelados.
\usection{1. CLASIFICACIÓN DE LOS SERES VIVOS RELACIONADOS CON EL PARAÍSO}
\vs p030 1:1 En el Paraíso se clasifican los seres vivos de acuerdo con su relación connatural y adquirida con las Deidades del Paraíso. Durante las magníficas asambleas que tienen lugar en el universo central y en los suprauniversos, los presentes se clasifican con frecuencia de acuerdo con su origen: los de origen trino, o que han alcanzado la Trinidad; los de origen doble; y los de origen único. Se hace difícil interpretar para la mente mortal esta clasificación de seres vivos del Paraíso, pero se nos ha autorizado para dar la siguiente información:
\vs p030 1:2 \pc I. \bibemph{LOS SERES DE ORIGEN TRINO}. Lo componen los seres creados por las tres Deidades del Paraíso, ya sea individualmente o como Trinidad, junto con el colectivo trinitizado, que abarca a todos los grupos de seres trinitizados, revelados y no revelados.
\vs p030 1:3 \pc A. \bibemph{Los espíritus supremos.}
\vs p030 1:4 \li{1.}Los siete espíritus mayores.
\vs p030 1:5 \li{2.}Los siete mandatarios supremos.
\vs p030 1:6 \li{3.}Los siete órdenes de espíritus reflectores.
\vs p030 1:7 \pc B. \bibemph{Los hijos estacionarios de la Trinidad.}
\vs p030 1:8 \li{1.}Los secretos trinitizados de supremacía.
\vs p030 1:9 \li{2.}Los eternos de días.
\vs p030 1:10 \li{3.}Los ancianos de días.
\vs p030 1:11 \li{4.}Los perfecciones de días.
\vs p030 1:12 \li{5.}Los recientes de días.
\vs p030 1:13 \li{6.}Los uniones de días.
\vs p030 1:14 \li{7.}Los fieles de días.
\vs p030 1:15 \li{8.}Los perfeccionadores de la sabiduría.
\vs p030 1:16 \li{9.}Los consejeros divinos.
\vs p030 1:17 \li{10.}Los censores universales.
\vs p030 1:18 \pc \bibemph{C. Los seres de origen en la Trinidad y trinitizados.}
\vs p030 1:19 \li{1.}Los hijos preceptores de la Trinidad.
\vs p030 1:20 \li{2.}Los espíritus inspirados de la Trinidad.
\vs p030 1:21 \li{3.}Los nativos de Havona.
\vs p030 1:22 \li{4.}Los ciudadanos del Paraíso.
\vs p030 1:23 \li{5.}Los seres no revelados de origen en la Trinidad.
\vs p030 1:24 \li{6.}Los seres no revelados trinitizados por la Deidad.
\vs p030 1:25 \li{7.}Los hijos trinitizados de logro.
\vs p030 1:26 \li{8.}Los hijos trinitizados de selección.
\vs p030 1:27 \li{9.}Los hijos trinitizados de perfección.
\vs p030 1:28 \li{10.}Los hijos trinitizados por criaturas.
\vs p030 1:29 \pc II. \bibemph{LOS SERES DE ORIGEN DOBLE}. Son aquellos que se originan a partir de cualesquiera dos de las Deidades del Paraíso o se han creado de otra manera por dos seres descendientes directa o indirectamente de las Deidades del Paraíso.
\vs p030 1:30 \pc A. \bibemph{Los órdenes descendentes.}
\vs p030 1:31 \li{1.}Los hijos creadores.
\vs p030 1:32 \li{2.}Los hijos magistrados.
\vs p030 1:33 \li{3.}Las estrellas brillantes de la mañana.
\vs p030 1:34 \li{4.}Los padres Melquisedec.
\vs p030 1:35 \li{5.}Los melquisedecs.
\vs p030 1:36 \li{6.}Los vorondadecs.
\vs p030 1:37 \li{7.}Los lanonandecs.
\vs p030 1:38 \li{8.}Las estrellas brillantes vespertinas.
\vs p030 1:39 \li{9.}Los arcángeles.
\vs p030 1:40 \li{10.}Los portadores de vida.
\vs p030 1:41 \li{11.}Los auxiliares del universo no revelados.
\vs p030 1:42 \li{12.}Los hijos de Dios no revelados.
\vs p030 1:43 \pc B. \bibemph{Los órdenes estacionarios.}
\vs p030 1:44 \li{1.}Los abandontes.
\vs p030 1:45 \li{2.}Los susatias.
\vs p030 1:46 \li{3.}Los univitatias.
\vs p030 1:47 \li{4.}Los espirongas.
\vs p030 1:48 \li{5.}Los seres de origen doble no revelados.
\vs p030 1:49 \pc \bibemph{C. Los órdenes ascendentes.}
\vs p030 1:50 \li{1.}Los mortales fusionados con el modelador.
\vs p030 1:51 \li{2.}Los mortales fusionados con el hijo.
\vs p030 1:52 \li{3.}Los mortales fusionados con el espíritu.
\vs p030 1:53 \li{4.}Los seres intermedios trasladados.
\vs p030 1:54 \li{5.}Los ascendentes no revelados.
\vs p030 1:55 \pc III. \bibemph{LOS SERES DE ORIGEN ÚNICO}. Son aquellos que se originan a partir de una de las Deidades del Paraíso o se han creado de otra manera por algún ser descendiente directa o indirectamente de las Deidades del Paraíso.
\vs p030 1:56 \pc A. \bibemph{Los espíritus supremos.}
\vs p030 1:57 \li{1.}Los mensajeros de gravedad.
\vs p030 1:58 \li{2.}Los siete espíritus de las vías de Havona.
\vs p030 1:59 \li{3.}Los duodécuplos ayudantes de las vías de Havona.
\vs p030 1:60 \li{4.}Los auxiliares reflectores de imagen.
\vs p030 1:61 \li{5.}Los espíritus maternos de los universos.
\vs p030 1:62 \li{6.}Los séptuplos espíritus asistentes de la mente.
\vs p030 1:63 \li{7.}Los seres no revelados de origen en la Deidad.
\vs p030 1:64 \pc B. \bibemph{Los órdenes ascendentes.}
\vs p030 1:65 \li{1.}Los modeladores personificados.
\vs p030 1:66 \li{2.}Los hijos materiales ascendentes.
\vs p030 1:67 \li{3.}Los serafines evolutivos.
\vs p030 1:68 \li{4.}Los querubines evolutivos.
\vs p030 1:69 \li{5.}Los ascendentes no revelados.
\vs p030 1:70 \pc C. \bibemph{La familia del Espíritu Infinito.}
\vs p030 1:71 \li{1.}Los mensajeros solitarios.
\vs p030 1:72 \li{2.}Los supervisores de las vías circulatorias del universo.
\vs p030 1:73 \li{3.}Los directores del censo.
\vs p030 1:74 \li{4.}Los auxiliares personales del Espíritu Infinito.
\vs p030 1:75 \li{5.}Los inspectores adjuntos.
\vs p030 1:76 \li{6.}Los centinelas con destino.
\vs p030 1:77 \li{7.}Los guías de los graduados.
\vs p030 1:78 \li{8.}Los servitales de Havona.
\vs p030 1:79 \li{9.}Los conciliadores universales.
\vs p030 1:80 \li{10.}Los acompañantes morontiales.
\vs p030 1:81 \li{11.}Los supernafines.
\vs p030 1:82 \li{12.}Los seconafines.
\vs p030 1:83 \li{13.}Los terciafines.
\vs p030 1:84 \li{14.}Los omniafines.
\vs p030 1:85 \li{15.}Los serafines.
\vs p030 1:86 \li{16.}Los querubines y sanobines.
\vs p030 1:87 \li{17.}Los seres de origen espiritual no revelado.
\vs p030 1:88 \li{18.}Los siete directores supremos de la potencia.
\vs p030 1:89 \li{19.}Los centros supremos de la potencia.
\vs p030 1:90 \li{20.}Los controladores físicos mayores.
\vs p030 1:91 \li{21.}Los supervisores de la potencia morontial.
\vs p030 1:92 \pc IV. \bibemph{LOS SERES TRASCENDENTALES DEVENIDOS}. Se puede encontrar en el Paraíso una inmensa multitud de seres trascendentales cuyo origen no se desvela de ordinario a los universos del tiempo y del espacio hasta que estos no se asientan en luz y vida. Estos seres no son creadores ni criaturas; son hijos \bibemph{devenidos} de la divinidad, de la ultimidad y de la eternidad. Tales “seres devenidos” no son finitos ni infinitos, son \bibemph{absonitos;} y la absonitidad no es infinitud ni absolutidad.
\vs p030 1:93 Estos seres no creadores y no creados son siempre leales a la Trinidad del Paraíso y obedientes al Último. Existen en cuatro niveles últimos de actividad en cuanto ser personal y desarrollan su labor en los siete niveles de lo absonito, en doce grandes divisiones consistentes en mil grupos principales de trabajo de siete clases cada uno. Dichos seres devenidos abarcan los siguientes órdenes:
\vs p030 1:94 \li{1.}Los arquitectos del universo matriz.
\vs p030 1:95 \li{2.}Los archivistas trascendentales.
\vs p030 1:96 \li{3.}Otros trascendentales.
\vs p030 1:97 \li{4.}Los organizadores mayores de la fuerza y devenidos primarios.
\vs p030 1:98 \li{5.}Los organizadores mayores de la fuerza y trascendentales adjuntos.
\vs p030 1:99 \pc Dios, como suprapersona, deviene; Dios, como persona, crea; Dios, como prepersona, fracciona; y esta fracción de sí mismo, el modelador, hace evolucionar el alma espiritual en la mente material y mortal, en conformidad con la libre voluntad de elección del ser personal que se ha otorgado a dicha criatura mortal por el acto paterno de Dios como Padre.
\vs p030 1:100 \pc V. \bibemph{LAS ENTIDADES FRACCIONADAS DE LA DEIDAD}. Este orden de existencia vivo, que tiene su origen en el Padre Universal, se ilustra de la mejor manera en los modeladores del pensamiento, aunque estas entidades no son de modo alguno las únicas fracciones de la realidad prepersonal de la Primera Fuente y Centro. Dichas fracciones prepersonales aparte de los modeladores desempeñan funciones numerosas y poco conocidas. La fusión con un modelador o con otra fracción de este otro tipo hace de la criatura un \bibemph{ser fusionado con el Padre}.
\vs p030 1:101 Aunque difícilmente comparables con las fracciones del Padre, se debe mencionar aquí que las fracciones del espíritu premente de la Tercera Fuente y Centro difieren, en buena parte, de los modeladores. No moran como tales en Lugar del Espíritu ni recorren las vías circulatorias de la gravedad\hyp{}mente; tampoco habitan en las criaturas mortales durante la vida de estas en la carne. No son prepersonales en el mismo sentido que lo son los modeladores, si bien, dichas fracciones de espíritu premente se otorgan a algunos de los mortales supervivientes, y la fusión con estos hace de ellos \bibemph{mortales fusionados con el espíritu} en contraste con los mortales fusionados con el modelador.
\vs p030 1:102 Es incluso más difícil de describir el espíritu individualizado de un hijo creador, la unión con el cual hace de la criatura un \bibemph{mortal fusionado con el hijo}. Existen, no obstante, otras fracciones de la Deidad.
\vs p030 1:103 \pc VI. \bibemph{LOS SERES SUPRAPERSONALES}. Hay una inmensa multitud de seres de origen divino aparte de los personales que prestan numerosos servicios en el universo de los universos. Algunos de ellos residen en los mundos del Hijo que circundan el Paraíso; otros, como los representantes suprapersonales del Hijo Eterno, se hallan en otros lugares. En su mayor parte, no se les menciona en estas narraciones, y resultaría realmente inútil describirlos para las criaturas \bibemph{personales}.
\vs p030 1:104 \pc VII. \bibemph{LOS ÓRDENES NO CLASIFICADOS Y NO REVELADOS}. Durante la presente era del universo no sería posible incluir a todos los seres, personales o no, dentro de las clasificaciones pertenecientes a esta era; tampoco se han revelado todas sus categorías en estas narraciones; es por ello por lo que muchos órdenes se han omitido de estas listas. Tomad en consideración los siguientes:
\vs p030 1:105 El consumador del destino del universo.
\vs p030 1:106 Los vicerregentes condicionados del Último.
\vs p030 1:107 Los supervisores incondicionados del Supremo.
\vs p030 1:108 Las instancias intermedias creativas no reveladas de los ancianos de días.
\vs p030 1:109 Majestón del Paraíso.
\vs p030 1:110 Los enlaces reflectantes innominados de Majestón.
\vs p030 1:111 Los órdenes midsonitas de los universos locales.
\vs p030 1:112 \pc No es necesario conceder especial importancia al hecho de que estos órdenes se enumeren de forma conjunta dejando a salvo que ninguno de ellos aparece en la clasificación de los seres relacionados con el Paraíso, tal como aquí se dan a conocer. Estos son unos pocos de los no clasificados, pero todavía os falta por conocer a los muchos no revelados.
\vs p030 1:113 Existen espíritus: entidades espirituales, presencias espirituales, espíritus personales, espíritus prepersonales, espíritus suprapersonales, existencias espirituales, seres personales espirituales ---pero ni el lenguaje de los mortales ni su intelecto tienen las cualidades necesarias para su reconocimiento---. Podemos afirmar, sin embargo, que no existen seres personales de “mente pura”; ninguna entidad posee ser personal a menos que Dios, que es espíritu, se lo conceda. Cualquier entidad mental que no esté vinculada a la energía espiritual o física no es un ser personal. Si bien, en el mismo sentido en que hay seres personales espirituales que poseen mente, hay seres personales mentales que poseen espíritu. Majestón y sus colaboradores son ejemplos bastantes ilustrativos de estos seres dominados por la mente, pero hay ejemplos ilustrativos todavía mejores de este tipo de ser personal. Existen incluso órdenes completos no revelados de tales \bibemph{seres personales mentales,} pero están siempre vinculados al espíritu. Algunas otras criaturas no reveladas son las que podrían designarse \bibemph{seres personales de energía mental y física}. Este tipo de ser no es sensible a la gravedad espiritual, pero es, no obstante, un verdadero ser personal: está dentro de la vía del Padre.
\vs p030 1:114 \pc 30:1.114 (334.8) Estos escritos ni siquiera empiezan a agotar, ni pueden, la historia de las criaturas vivas, de los creadores, de los seres eventuados, ni incluso de seres que tienen otro modo de existencia, que viven y adoran y sirven en los universos pululantes del tiempo y en el universo central de la eternidad. Vosotros, los mortales, sois personas y, por consiguiente, podemos describir seres \bibemph{que poseen forma personal,} pero ¿cómo se os podría dar una explicación de un ser \bibemph{absonitizado?}
\usection{2. EL REGISTRO DE LOS SERES PERSONALES DE UVERSA}
\vs p030 2:1 La familia divina de seres vivos con registro en Uversa está dividida en siete grandes grupos:
\vs p030 2:2 \li{1.}Las Deidades del Paraíso.
\vs p030 2:3 \li{2.}Los espíritus supremos.
\vs p030 2:4 \li{3.}Los seres de origen en la Trinidad.
\vs p030 2:5 \li{4.}Los Hijos de Dios.
\vs p030 2:6 \li{5.}Los seres personales del Espíritu Infinito.
\vs p030 2:7 \li{6.}Los directores de la potencia del universo.
\vs p030 2:8 \li{7.}El colectivo de ciudadanos permanentes.
\vs p030 2:9 \pc Estos grupos de criaturas volitivas se dividen en numerosas clases y subdivisiones menores. Sin embargo, esta clasificación de los seres personales del gran universo trata principalmente de dar cabida a aquellos órdenes de seres inteligentes revelados en estas narraciones, con la mayoría de los que los mortales del tiempo se encontrarán conforme experimenten su paulatino ascenso hacia el Paraíso. En la siguiente lista no se hace mención del inmenso número de órdenes de seres del universo que desempeñan su labor aparte del plan de ascenso de los mortales.
\vs p030 2:10 \pc I. \bibemph{LAS DEIDADES DEL PARAÍSO.}
\vs p030 2:11 \li{1.}El Padre Universal.
\vs p030 2:12 \li{2.}El Hijo Eterno.
\vs p030 2:13 \li{3.}El Espíritu Infinito.
\vs p030 2:14 \pc II. \bibemph{LOS ESPÍRITUS SUPREMOS.}
\vs p030 2:15 \li{1.}Los siete espíritus mayores.
\vs p030 2:16 \li{2.}Los siete mandatarios supremos.
\vs p030 2:17 \li{3.}Los siete grupos de espíritus reflectores.
\vs p030 2:18 \li{4.}Los auxiliares reflectores de imagen.
\vs p030 2:19 \li{5.}Los siete espíritus de las vías.
\vs p030 2:20 \li{6.}Los espíritus creativos de los universos locales.
\vs p030 2:21 \li{7.}Los espíritus asistentes de la mente.
\vs p030 2:22 \pc III. \bibemph{LOS SERES DE ORIGEN EN LA TRINIDAD.}
\vs p030 2:23 \li{1.}Los secretos trinitizados de la supremacía.
\vs p030 2:24 \li{2.}Los eternos de días.
\vs p030 2:25 \li{3.}Los ancianos de días.
\vs p030 2:26 \li{4.}Los perfecciones de días.
\vs p030 2:27 \li{5.}Los recientes de días.
\vs p030 2:28 \li{6.}Los uniones de días.
\vs p030 2:29 \li{7.}Los fieles de días.
\vs p030 2:30 \li{8.}Los hijos preceptores de la Trinidad.
\vs p030 2:31 \li{9.}Los perfeccionadores de la sabiduría.
\vs p030 2:32 \li{10.}Los consejeros divinos.
\vs p030 2:33 \li{11.}Los censores universales.
\vs p030 2:34 \li{12.}Los espíritus inspirados de la Trinidad.
\vs p030 2:35 \li{13.}Los nativos de Havona.
\vs p030 2:36 \li{14.}Los ciudadanos del Paraíso.
\vs p030 2:37 \pc IV. \bibemph{LOS HIJOS DE DIOS.}
\vs p030 2:38 \pc A. \bibemph{Los hijos descendentes.}
\vs p030 2:39 \li{1.}Los hijos creadores migueles.
\vs p030 2:40 \li{2.}Los hijos magistrados avonales.
\vs p030 2:41 \li{3.}Los hijos preceptores dainales de la Trinidad.
\vs p030 2:42 \li{4.}Los hijos melquisedecs.
\vs p030 2:43 \li{5.}Los hijos vorondadecs.
\vs p030 2:44 \li{6.}Los hijos lanonandecs.
\vs p030 2:45 \li{7.}Los hijos portadores de vida.
\vs p030 2:46 \pc B. \bibemph{Los hijos ascendentes.}
\vs p030 2:47 \li{1.}Los mortales fusionados con el Padre.
\vs p030 2:48 \li{2.}Los mortales fusionados con el hijo.
\vs p030 2:49 \li{3.}Los mortales fusionados con el espíritu.
\vs p030 2:50 \li{4.}Los serafines evolutivos.
\vs p030 2:51 \li{5.}Los hijos materiales ascendentes.
\vs p030 2:52 \li{6.}Los seres intermedios trasladados.
\vs p030 2:53 \li{7.}Los modeladores personificados.
\vs p030 2:54 \pc C. \bibemph{Los hijos trinitizados.}
\vs p030 2:55 \li{1.}Los mensajeros poderosos.
\vs p030 2:56 \li{2.}Aquellos elevados en autoridad.
\vs p030 2:57 \li{3.}Los sin nombre ni número.
\vs p030 2:58 \li{4.}Los custodios trinitizados.
\vs p030 2:59 \li{5.}Los embajadores trinitizados.
\vs p030 2:60 \li{6.}Los guardianes celestiales.
\vs p030 2:61 \li{7.}Los asistentes de los hijos elevados.
\vs p030 2:62 \li{8.}Los hijos trinitizados de ascendentes.
\vs p030 2:63 \li{9.}Los hijos trinitizados por criaturas del Paraíso\hyp{}Havona
\vs p030 2:64 \li{10.}Los hijos trinitizados de destino.
\vs p030 2:65 \pc V. \bibemph{LOS SERES PERSONALES DEL ESPÍRITU INFINITO.}
\vs p030 2:66 \pc A. \bibemph{Los seres personales superiores del Espíritu Infinito.}
\vs p030 2:67 \li{1.}Los mensajeros solitarios.
\vs p030 2:68 \li{2.}Los supervisores de las vías circulatorias del universo.
\vs p030 2:69 \li{3.}Los directores del censo.
\vs p030 2:70 \li{4.}Los auxiliares personales del Espíritu Infinito.
\vs p030 2:71 \li{5.}Los inspectores adjuntos.
\vs p030 2:72 \li{6.}Los centinelas asignados a los sistemas locales.
\vs p030 2:73 \li{7.}Los guías de los graduados.
\vs p030 2:74 \pc B. \bibemph{Las multitudes de mensajeros del espacio.}
\vs p030 2:75 \li{1.}Los servitales de Havona.
\vs p030 2:76 \li{2.}Los conciliadores universales.
\vs p030 2:77 \li{3.}Los consejeros técnicos.
\vs p030 2:78 \li{4.}Los custodios de los archivos del Paraíso.
\vs p030 2:79 \li{5.}Los archivistas celestiales.
\vs p030 2:80 \li{6.}Los acompañantes morontiales.
\vs p030 2:81 \li{7.}Los acompañantes del Paraíso.
\vs p030 2:82 \pc C. \bibemph{Los espíritus servidores.}
\vs p030 2:83 \li{1.}Los supernafines.
\vs p030 2:84 \li{2.}Los seconafines.
\vs p030 2:85 \li{3.}Los terciafines.
\vs p030 2:86 \li{4.}Los omniafines.
\vs p030 2:87 \li{5.}Los serafines.
\vs p030 2:88 \li{6.}Los querubines y sanobines.
\vs p030 2:89 \li{7.}Los seres intermedios.
\vs p030 2:90 \pc VI. \bibemph{LOS DIRECTORES DE LA POTENCIA DEL UNIVERSO.}
\vs p030 2:91 \pc A. \bibemph{Los siete directores supremos de la potencia.}
\vs p030 2:92 \pc B. \bibemph{Los centros supremos de la potencia.}
\vs p030 2:93 \li{1.}Los supervisores supremos de los centros.
\vs p030 2:94 \li{2.}Los centros de Havona.
\vs p030 2:95 \li{3.}Los centros de los suprauniversos.
\vs p030 2:96 \li{4.}Los centros de los universos locales.
\vs p030 2:97 \li{5.}Los centros de las constelaciones.
\vs p030 2:98 \li{6.}Los centros de los sistemas.
\vs p030 2:99 \li{7.}Los centros no clasificados.
\vs p030 2:100 \pc C. \bibemph{Los controladores físicos mayores.}
\vs p030 2:101 \li{1.}Los directores adjuntos de la potencia.
\vs p030 2:102 \li{2.}Los controladores mecánicos.
\vs p030 2:103 \li{3.}Los transformadores de la energía.
\vs p030 2:104 \li{4.}Los transmisores de la energía.
\vs p030 2:105 \li{5.}Los asociadores primarios.
\vs p030 2:106 \li{6.}Los disociadores secundarios.
\vs p030 2:107 \li{7.}Los frandalancs y los cronoldecs.
\vs p030 2:108 \pc D. \bibemph{Los supervisores de la potencia morontial.}
\vs p030 2:109 \li{1.}Los reguladores de las vías circulatorias.
\vs p030 2:110 \li{2.}Los coordinadores de los sistemas.
\vs p030 2:111 \li{3.}Los custodios planetarios.
\vs p030 2:112 \li{4.}Los controladores combinados.
\vs p030 2:113 \li{5.}Los estabilizadores de enlace.
\vs p030 2:114 \li{6.}Los resintonizadores selectivos.
\vs p030 2:115 \li{7.}Los registradores adjuntos.
\vs p030 2:116 \pc VII. \bibemph{EL COLECTIVO DE CIUDADANOS PERMANENTES.}
\vs p030 2:117 \li{1.}Los seres intermedios planetarios.
\vs p030 2:118 \li{2.}Los hijos adánicos de los sistemas.
\vs p030 2:119 \li{3.}Los univitatias de las constelaciones.
\vs p030 2:120 \li{4.}Los susatias de los universos locales.
\vs p030 2:121 \li{5.}Los mortales de los universos locales fusionados con el espíritu.
\vs p030 2:122 \li{6.}Los abandontes del suprauniverso.
\vs p030 2:123 \li{7.}Los mortales de los suprauniversos fusionados con el hijo.
\vs p030 2:124 \li{8.}Los nativos de Havona.
\vs p030 2:125 \li{9.}Los nativos de las esferas del Espíritu que orbitan el Paraíso.
\vs p030 2:126 \li{10.}Los nativos de las esferas del Padre que orbitan el Paraíso.
\vs p030 2:127 \li{11.}Los ciudadanos creados del Paraíso.
\vs p030 2:128 \li{12.}Los mortales ciudadanos del Paraíso fusionados con el modelador.
\vs p030 2:129 \pc Esta es una clasificación operativa de los seres personales de los universos tal como están registrados en la sede central de Uversa.
\vs p030 2:130 \pc \bibemph{GRUPOS DE SERES PERSONALES COMPUESTOS}. Hay registrados en Uversa numerosos otros grupos de seres inteligentes, seres que también están estrechamente relacionados con la organización y la administración del gran universo. Entre tales órdenes se hallan los tres grupos siguientes de seres personales compuestos:
\vs p030 2:131 \pc A. \bibemph{El colectivo final del Paraíso.}
\vs p030 2:132 \li{1.}El colectivo de finalizadores mortales.
\vs p030 2:133 \li{2.}El colectivo de finalizadores del Paraíso.
\vs p030 2:134 \li{3.}El colectivo de finalizadores trinitizados.
\vs p030 2:135 \li{4.}El colectivo de finalizadores trinitizados conjuntos.
\vs p030 2:136 \li{5.}El colectivo de finalizadores de Havona.
\vs p030 2:137 \li{6.}El colectivo de finalizadores trascendentales.
\vs p030 2:138 \li{7.}El colectivo de hijos de destino no revelados.
\vs p030 2:139 \pc El colectivo final de los mortales se trata en la siguiente y última narración de esta serie.
\vs p030 2:140 \pc B. \bibemph{Los auxiliares del universo.}
\vs p030 2:141 \li{1.}Las estrellas brillantes de la mañana.
\vs p030 2:142 \li{2.}Las estrellas brillantes vespertinas.
\vs p030 2:143 \li{3.}Los arcángeles.
\vs p030 2:144 \li{4.}Los asistentes altísimos.
\vs p030 2:145 \li{5.}Los altos comisionados.
\vs p030 2:146 \li{6.}Los supervisores celestiales.
\vs p030 2:147 \li{7.}Los maestros de los mundos de las moradas.
\vs p030 2:148 \pc En todos los mundos sedes tanto de los universos locales como de los suprauniversos se hace previsión de estos seres que se ocupan de misiones específicas para los hijos creadores, los gobernantes de los universos locales. En Uversa, damos la bienvenida a estos \bibemph{auxiliares del universo,} sobre los que no tenemos jurisdicción. Estos emisarios desempeñan su labor y llevan a cabo sus observaciones bajo la autoridad de los hijos creadores. Su actividad se describe con más detenimiento en el relato de vuestro universo local.
\vs p030 2:149 \pc C. \bibemph{Las siete colonias de cortesía, compuestas respectivamente de:}
\vs p030 2:150 \li{1.}Los estudiosos de las estrellas.
\vs p030 2:151 \li{2.}Los artesanos celestiales.
\vs p030 2:152 \li{3.}Los directores de reversión.
\vs p030 2:153 \li{4.}Los instructores de las facultades de extensión.
\vs p030 2:154 \li{5.}Los distintos colectivos de reserva.
\vs p030 2:155 \li{6.}Los visitantes estudiantiles.
\vs p030 2:156 \li{7.}Los peregrinos ascendentes.
\vs p030 2:157 \pc Estos siete grupos de seres se organizan y gobiernan de ese modo en todos los mundos sede, desde los sistemas locales hasta las capitales de los suprauniversos, en particular en estos últimos. Las capitales de los siete suprauniversos son los lugares de encuentro de casi todas las clases y órdenes de seres inteligentes. Exceptuando a numerosos grupos del Paraíso\hyp{}Havona, aquí se puede observar y estudiar a las criaturas volitivas en cualquier faceta de su existencia.
\usection{3. LAS COLONIAS DE CORTESÍA}
\vs p030 3:1 Las siete colonias de cortesía se alojan temporalmente en las esferas arquitectónicas durante periodos más o menos prolongados mientras se ocupan de desarrollar sus misiones y de llevar a cabo sus cometidos especiales. Su función se puede describir de la manera siguiente:
\vs p030 3:2 \li{1.}\bibemph{Los estudiosos de las estrellas,} o los astrónomos celestiales, optan por realizar su labor en esferas como Uversa porque estos mundos, especialmente construidos, resultan particularmente favorables para sus observaciones y cálculos. Uversa está propiciamente situada para el trabajo de esta colonia no solamente debido a su ubicación central, sino también porque no hay soles gigantescos cercanos ni vivos ni inertes que podrían perturbar las corrientes de energía. Estos estudiosos no están, de manera alguna, relacionados orgánicamente con los asuntos del suprauniverso; son meramente invitados.
\vs p030 3:3 En la colonia astronómica de Uversa, hay seres procedentes de muchas regiones espaciales cercanas, del universo central e incluso de Norlatiadec. Todo ser de cualquier mundo de cualquier sistema de cualquier universo puede convertirse en un estudioso de las estrellas, puede aspirar a unirse a algún colectivo de astrónomos celestiales. Los únicos requisitos son una vida continuada allí y unos conocimientos suficientes de los mundos del espacio, especialmente de sus leyes físicas de evolución y control. No se exige de estos estudiosos que sirvan eternamente en este colectivo, pero nadie que haya sido admitido a este grupo puede dejarlo en menos de un milenio del tiempo de Uversa.
\vs p030 3:4 La colonia de observadores de las estrellas de Uversa cuenta en este momento con más de un millón de seres. Estos astrónomos van y vienen, aunque algunos se quedan durante períodos relativamente largos. Llevan a cabo su labor ayudados de una multitud de instrumentos mecánicos y de dispositivos físicos. Reciben también una gran ayuda de parte de los mensajeros solitarios y de otros exploradores espirituales. En su estudio de las estrellas y en su investigación espacial, estos astrónomos celestiales hacen un uso constante de los transformadores y transmisores vivos de la energía, al igual que de los seres personales reflectantes. Analizan todas las formas y facetas de la materia del espacio y de las manifestaciones de la energía, y están tan interesados en la función de la fuerza como en los fenómenos estelares; no hay nada en todo el espacio que escape a su atenta observación.
\vs p030 3:5 Se pueden encontrar colonias similares de astrónomos en los mundos sedes de los sectores del suprauniverso así como en las capitales arquitectónicas de los universos locales y en sus subdivisiones administrativas. Salvo en el Paraíso, el conocimiento no es connatural; entender el universo físico depende en gran manera de la observación y de la investigación.
\vs p030 3:6 \li{2.}\bibemph{Los artesanos celestiales} prestan sus servicios en todas las partes de los siete suprauniversos. Los mortales ascendentes tienen un contacto inicial con estos grupos durante su andadura morontial en el universo local. Más adelante analizaremos detenidamente la relación de los mortales con estos artesanos.
\vs p030 3:7 \li{3.}\bibemph{Los directores de reversión} incentivan el ocio y el humor ---se revierte a memorias del pasado---. Su aportación es de gran utilidad para el aspecto práctico del plan de ascenso progresivo diseñado para los mortales, especialmente durante las primeras fases de su transición morontial y de su experiencia espiritual. La historia de estos seres se verá en la siguiente narración que trata de la andadura de los mortales en el universo local.
\vs p030 3:8 \li{4.}\bibemph{Los instructores de las escuelas de extensión.} En la andadura del ascendente, el mundo de residencia inmediatamente superior siempre cuenta con un sólido colectivo de maestros en el mundo justo por debajo. Constituyen una especie de centros de preparación para aquellos residentes que hacen sus progresos en dicha esfera. Se trata de una fase del esquema ascendente para el avance de los peregrinos del tiempo. Los métodos de instrucción y exámenes de estas facultades son completamente diferentes a los que ponéis en práctica en Urantia.
\vs p030 3:9 Todo plan de ascenso progresivo de los mortales se caracteriza por la práctica de transmitir a otros seres las nuevas verdades y experiencias tan pronto como se adquieren. Os abrís camino a través de la permanente escuela que os lleva a alcanzar el Paraíso, sirviendo como maestros de aquellos pupilos que están inmediatamente por debajo de vosotros en su grado de progreso.
\vs p030 3:10 \li{5.}\bibemph{Los diversos colectivos de reserva}. En Uversa se movilizan inmensas reservas de seres que no están bajo nuestra directa supervisión, formando una colonia del colectivo de reserva. Esta colonia de Uversa está dividida principalmente en setenta grupos; permitirle a alguien pasar una temporada con estos extraordinarios seres personales constituye una verdadera formación integral. En Lugar de Salvación y en otras capitales de los universos, se mantienen reservas generales similares que se envían al servicio activo a solicitud de sus respectivos directores de grupo.
\vs p030 3:11 \li{6.}\bibemph{Los visitantes estudiantiles}. Procedentes de todas partes del universo, un torrente constante de visitantes va y viene a los distintos mundos sede. Tanto de forma individual al igual que como clase estos diversos tipos de seres acuden a nosotros en gran cantidad como observadores, estudiantes de intercambio y ayudantes de los estudiantes. En esta colonia de cortesía de Uversa, en el momento presente, hay más de mil millones de personas. Algunos de estos visitantes pueden permanecer un día otros, un año, dependiendo de la naturaleza de su misión. Esta colonia contiene prácticamente todas las clases de seres del universo salvo seres personales creadores y mortales morontiales.
\vs p030 3:12 Los mortales morontiales son visitantes estudiantiles únicamente dentro de los confines del universo local del que son originarios. Están facultados para ser visitantes del suprauniverso solamente tras haber logrado el estatus de espíritu. Al menos una mitad de nuestra colonia de visitantes consiste en seres que están “en escala”, que se encaminan hacia otros lugares y hacen una pausa para visitar la capital de Orvontón. Estos seres personales pueden estar llevando a cabo alguna misión en el universo o estar disfrutando de un periodo de esparcimiento ---libres de cometidos---. Tener el privilegio del viaje y la observación dentro de las lindes del universo forma parte de la andadura de todos los seres ascendentes. Se dará toda complacencia al deseo humano de viajar y de observar nuevos pueblos y mundos durante la larga y memorable subida al Paraíso a través del universo local, de los suprauniversos y del universo central.
\vs p030 3:13 \li{7.}\bibemph{Los peregrinos ascendentes}. Cuando se asigna a los peregrinos ascendentes a diferentes servicios como parte de su progreso y ascenso al Paraíso, se les domicilia como colonia de cortesía en las diversas esferas sede. Mientras realizan su labor por doquier en todo un suprauniverso, estos grupos son en gran parte autónomos. Conforman una colonia en constante cambio que abarca todos los órdenes de mortales evolutivos y sus colaboradores ascendentes.
\usection{4. LOS MORTALES ASCENDENTES}
\vs p030 4:1 Aun cuando a los mortales supervivientes del tiempo y del espacio se les denomina \bibemph{peregrinos ascendentes} cuando se les autoriza para su ascenso progresivo al Paraíso, estas criaturas evolutivas ocupan un lugar tan importante en estas narraciones que es nuestro deseo ofrecer a continuación una sinopsis de las siete etapas en las que consiste la andadura del ascendente en el universo:
\vs p030 4:2 \li{1.}Los mortales planetarios.
\vs p030 4:3 \li{2.}Los supervivientes dormidos.
\vs p030 4:4 \li{3.}Los estudiantes de los mundos de las moradas.
\vs p030 4:5 \li{4.}Los progresadores morontiales.
\vs p030 4:6 \li{5.}Los pupilos de los suprauniversos.
\vs p030 4:7 \li{6.}Los peregrinos de Havona.
\vs p030 4:8 \li{7.}Los que arriban al Paraíso.
\vs p030 4:9 \pc El siguiente relato describe la andadura en el universo del mortal que ha sido habitado por el modelador. Los mortales fusionados con el hijo o con el espíritu tienen en común ciertas partes de esta andadura, pero hemos decidido contar la historia centrándonos en los mortales fusionados con el modelador, el destino previsto para todas las razas humanas de Urantia.
\vs p030 4:10 \li{1.}Los \bibemph{mortales planetarios}. Los mortales son seres evolutivos de origen animal con potencial para convertirse en ascendentes. En origen, naturaleza y destino, estos distintos grupos de seres humanos no son muy diferentes de los pueblos de Urantia. En cada mundo, las razas humanas reciben el mismo ministerio por parte de los Hijos de Dios y gozan de la presencia de los espíritus servidores del tiempo. Tras la muerte física, en los mundos de las moradas, todas las diferentes clases de ascendentes fraternizan como una única familia morontial.
\vs p030 4:11 \li{2.}\bibemph{Los supervivientes dormidos}. Todos los mortales en su condición de supervivientes, bajo el cuidado de los guardianes personales del destino, cruzan las puertas de la muerte física y, al tercer periodo, retoman su ser personal en los mundos de las moradas. Aquellos seres autorizados que, por alguna razón, hayan sido incapaces de alcanzar ese nivel de dominio de la inteligencia y de dotación de espiritualidad que les posibilitaría poder contar con estos custodios personales, no pueden, por consiguiente, dirigirse de forma directa e inmediata a los mundos de las moradas. Dichas almas supervivientes han de reposar en un sueño inconsciente hasta el día del juicio que inaugurará una nueva época, una nueva dispensación, la llegada de un hijo de Dios para hacer el llamamiento nominal de los tiempos y el dictamen del mundo. Tal es la costumbre general en todo Nebadón. Se ha dicho de Cristo Miguel que, cuando ascendió a las alturas al concluir su misión en la tierra, “llevó a una gran multitud de cautivos”. Estos cautivos eran los supervivientes dormidos desde los días de Adán hasta el día de la resurrección del Maestro en Urantia.
\vs p030 4:12 El paso del tiempo no tiene consecuencias para los mortales dormidos; están completamente inconscientes y ajenos a la duración de su reposo. En el momento de reconstitución de su ser personal al fin de una era, aquellos que hayan dormido cinco mil años no reaccionarán de forma diferente a los que solamente lo hayan hecho cinco días. Aparte de la demora de tiempo, estos supervivientes pasan por el régimen de ascensión de modo idéntico al de aquellos que eluden el sueño más largo o más corto de la muerte.
\vs p030 4:13 Se acude a estas clases de peregrinos de los mundos con este tipo de dispensación para la actividad morontial de grupo en las tareas de los universos locales. Hay una gran ventaja en la movilización de esos grupos tan enormes porque, gracias a este eficaz servicio, se les mantiene unidos durante largos periodos.
\vs p030 4:14 \li{3.}\bibemph{Los estudiantes de los mundos de las moradas}. A este grupo pertenecen todos los mortales supervivientes que hacen su nuevo despertar en los mundos de las moradas.
\vs p030 4:15 El cuerpo físico de carne mortal no es parte de la reconstitución del superviviente dormido; el cuerpo físico vuelve al polvo. El serafín asignado a este auspicia el nuevo cuerpo, la forma morontial, como el nuevo vehículo de vida para el alma inmortal y para la inhabitación del modelador que ha retornado. El modelador es el cuidador de las transcripciones espirituales de la mente del superviviente dormido. El serafín en cumplimiento de su labor es quien mantiene la identidad que sobrevive ---el alma inmortal--- hasta donde haya evolucionado. Y cuando los dos, el modelador y el serafín, aúnan los factores del ser personal que se les ha encomendado, el nuevo ser se erige como la resurrección del antiguo ser personal, conllevando la supervivencia de la identidad evolutiva morontial del alma. A esta revinculación de alma y del modelador se la denomina muy apropiadamente “resurrección”, una reconstitución de dichos factores del ser personal. No obstante, incluso esto no explicaría del todo la reaparición del \bibemph{ser personal} superviviente. Y, aunque nunca podréis comprender las circunstancias de tal inexplicable proceso, si no rechazáis el plan de supervivencia diseñado para los mortales, en algún momento conoceréis experiencialmente su verdad.
\vs p030 4:16 \pc El plan de detenimiento inicial de los mortales en los siete mundos de formación continuada es casi global en Orvontón. En cada sistema local de aproximadamente mil planetas habitados, hay siete mundos de morada, por lo general satélites o subsatélites de la capital del sistema. En ellos se recibe a la mayoría de los mortales ascendentes.
\vs p030 4:17 A veces se denominan “moradas” del universo a todos los mundos de formación en los que los mortales tienen su residencia, y fue a estas esferas a las que Jesús se refirió cuando dijo: “En la casa de mi Padre muchas moradas hay”. De aquí en adelante, dentro de un grupo determinado de esferas tales como los mundos de las moradas, los ascendentes progresarán de forma individual de una esfera a la otra y de una fase de vida a otra, pero siempre avanzarán de una etapa de estudio del universo a otra formándose en grupos de características comunes.
\vs p030 4:18 \li{4.}\bibemph{Los progresadores morontiales}. Desde los mundos de las moradas hacia arriba, a través de las esferas del sistema, de la constelación y del universo, se categoriza a los mortales como progresadores morontiales. Son mortales que, al ascender, atraviesan las esferas de transición y, a medida que progresan desde los mundos morontiales inferiores hasta los superiores, sirven, en incontables tareas, en colaboración con sus maestros y en compañía de sus hermanos más avanzados y de mayor rango.
\vs p030 4:19 El progreso en la etapa morontial se corresponde con un avance continuado de la forma del intelecto, del espíritu y del ser personal. Los supervivientes continúan siendo seres de naturaleza triple. A todo lo largo de su experiencia morontial, ellos son los pupilos del universo local. El régimen del suprauniverso no surte efecto hasta que no comienza la andadura espiritual.
\vs p030 4:20 Los mortales adquieren su verdadera identidad espiritual justo antes de dejar la sede del universo local para dirigirse a los mundos receptores de los sectores menores del suprauniverso. El transcurso desde la última etapa morontial al estatus espiritual inicial o menor no conlleva sino una leve transición. La mente, el ser personal y el carácter permanecen inalterados como producto de tal avance; solamente la forma sufre modificación. Pero la forma del espíritu es tan real como el cuerpo morontial e igualmente perceptible.
\vs p030 4:21 Antes de partir de los universos locales de los que son originarios hacia los mundos receptores del suprauniverso, los mortales del tiempo reciben la confirmación espiritual del hijo creador y del espíritu materno del universo local. Desde aquí en adelante, se asienta para siempre el estatus del mortal ascendente. Nunca se ha conocido que los pupilos del suprauniverso se hayan descarriado. A los serafines ascendentes también se les eleva en su estatus angélico en el momento de su partida de los universos locales.
\vs p030 4:22 \li{5.}\bibemph{Los pupilos del suprauniverso}. Todos los ascendentes que llegan a los mundos de formación de los suprauniversos se convierten en los pupilos de los ancianos de días. Han pasado por la vida morontial del universo local y ahora son espíritus reconocidos. Como jóvenes espíritus, comienzan su ascensión siguiendo el sistema cultural y de formación del suprauniverso, que se extiende desde las esferas receptoras de su sector menor, a través de los mundos de estudio de los diez sectores mayores, hasta las esferas culturales superiores de la sede del suprauniverso.
\vs p030 4:23 Existen tres órdenes de espíritus estudiantes, de acuerdo con su estancia en el sector menor, en los sectores mayores y en el mundo sede del suprauniverso, en los que progresan espiritualmente. Al igual que los seres ascendentes morontiales estudiaron y laboraron en los mundos del universo local, los seres ascendentes espirituales continúan superando nuevos mundos, mientras siguen impartiendo a otros aquello que han bebido de las fuentes experienciales de la sabiduría. Pero el proceso de aprendizaje del ser espiritual en su andadura en el suprauniverso es muy diferente a lo que la mente material del hombre puede llegar jamás a imaginar.
\vs p030 4:24 Antes de dejar el suprauniverso con destino a Havona, estos espíritus ascendentes reciben el mismo exhaustivo curso sobre la dirección del suprauniverso que han recibido sobre la supervisión del universo local durante su experiencia morontial. Con anterioridad a que los mortales espirituales alcancen Havona, su estudio principal, aunque no su ocupación exclusiva, se centra en el conocimiento de la administración del universo local y del suprauniverso. La razón de toda esta trayectoria no está en este momento del todo clara, pero no hay duda de que tal formación es acertada y necesaria a la vista de su posible destino futuro como miembros del colectivo de finalizadores.
\vs p030 4:25 El régimen que se sigue en el suprauniverso no es el mismo para todos los mortales ascendentes. Reciben la misma enseñanza de tipo general, pero hay grupos y clases particulares que han de superar cursos especiales de instrucción, y se les hace pasar por cursos específicos de capacitación.
\vs p030 4:26 \li{6.}\bibemph{Los peregrinos de Havona}. Cuando su evolución espiritual es completa, aunque no plena, el mortal superviviente se prepara para el largo vuelo a Havona, el abrigo natural de los espíritus evolutivos. En la tierra, eras una criatura de carne y hueso; a lo largo del universo local, un ser morontial; a través del suprauniverso, un espíritu en desarrollo; con tu llegada a los mundos receptores de Havona, comienza real y seriamente tu educación espiritual; en el Paraíso, aparecerás finalmente como espíritu perfeccionado.
\vs p030 4:27 El viaje desde la sede del suprauniverso hasta las esferas receptoras de Havona siempre se hace a solas. De aquí en adelante ya no se impartirá más instrucción en clases o grupos. Habéis acabado la formación técnica y administrativa de los mundos evolutivos del tiempo y del espacio. Ahora comienza vuestra \bibemph{educación personal,} vuestra formación espiritual individual. De principio a fin, a lo largo de todo Havona, la instrucción es personal y de naturaleza triple: intelectual, espiritual y experiencial.
\vs p030 4:28 El primer acto de vuestra andadura en Havona será apreciar y agradecer a vuestro seconafín de transporte el viaje largo y seguro que os ha facilitado. Luego se os presentará a aquellos seres que auspiciarán vuestras primeras tareas en Havona. Seguidamente, iréis a inscribir vuestra llegada y a preparar vuestro mensaje de acción de gracias y adoración para que sea enviado al hijo creador de vuestro universo local, el Padre de vuestro universo, que hizo posible vuestra andadura de filiación en Dios. Con esto concluye el protocolo de vuestra llegada a Havona. Tras ella, se os concederá un largo periodo de ocio para la libre observación. Esto os permitirá buscar a amigos, compañeros y colaboradores con los que habéis compartido vuestra larga trayectoria como ascendentes. También podréis consultar las transmisiones existentes para determinar quiénes son los otros peregrinos que han partido hacia Havona desde el momento en que abandonasteis Uversa.
\vs p030 4:29 El hecho de vuestra llegada a los mundos receptores de Havona se transmitirá cumplidamente a la sede de vuestro universo local y se le comunicará personalmente a vuestro guardián seráfico dondequiera que se encuentre.
\vs p030 4:30 Los mortales ascendentes ya han adquirido una concienzuda formación en los asuntos de los mundos evolutivos del espacio; ahora comienzan su largo y provechoso contacto con las esferas creadas en perfección. ¡Qué gran preparación para alguna futura labor se les proporciona mediante esta experiencia combinada, única y extraordinaria! Pero no puedo deciros nada sobre Havona; debéis ver estos mundos para apreciar su gloria o comprender su grandiosidad.
\vs p030 4:31 \li{7.}\bibemph{Los que arriban al Paraíso}. Al alcanzar el Paraíso con el estatus de residente, comenzáis un curso avanzado en divinidad y absonitidad. Tener la residencia en el Paraíso significa que habéis encontrado a Dios y que se os incorporará al colectivo final de los mortales. De todas las criaturas del gran universo, solo los que están fusionados con el Padre se suman a dicho colectivo, y solo estos prestan el juramento del finalizador. Otros seres que han alcanzado la perfección del Paraíso pueden adscribirse de forma temporal a este colectivo final, pero no se les asigna eternamente a la misión desconocida y no revelada de esta creciente multitud de veteranos evolutivos y perfeccionados del tiempo y del espacio.
\vs p030 4:32 A estos mortales que arriban al Paraíso se les otorga un periodo de libertad, tras el que comienzan a entablar relaciones con los siete grupos de los supernafines primarios. Una vez que han acabado su curso con los conductores de la adoración, se les designa “graduados del Paraíso” y, posteriormente, como finalizadores, se les asigna al servicio de observación y de cooperación que se extiende hasta los confines de la inmensa creación. Hasta el momento, no parece haber un destino determinado o establecido para el colectivo final de los mortales, aunque llevan a cabo muchos cometidos en los mundos asentados en luz y vida.
\vs p030 4:33 Si no existiese un destino futuro o no revelado para el colectivo final de los mortales finalizadores, la labor que estos seres ascendentes realizan en la actualidad ya sería de por sí enteramente satisfactoria y gloriosa. Su destino presente justifica ya, de hecho y por completo, el plan universal de la ascensión evolutiva. No obstante, en épocas futuras, con la evolución de las esferas del espacio exterior, no hay duda de que se expandirán significativamente, y se iluminarán de forma divina, con mayor plenitud, la sabiduría y la amorosa benevolencia de los Dioses en la consumación de su plan divino para la supraexperiencia humana y la ascensión de los mortales.
\vs p030 4:34 \pc En esta narración, junto con lo que se os ha revelado y con lo que podáis adquirir en conjunción con la información sobre vuestro propio mundo, se hace un bosquejo de la andadura de los mortales ascendentes. La historia varía de forma considerable según se trate de uno u otro suprauniverso; si bien, en este relato se ofrece un atisbo del plan ordinario respecto al progreso de los mortales, tal como se lleva a cabo en el universo local de Nebadón y en el séptimo segmento del gran universo, en el suprauniverso de Orvontón.
\vsetoff
\vs p030 4:35 [Auspiciado por un mensajero poderoso procedente de Uversa.]
