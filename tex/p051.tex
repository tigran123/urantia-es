\upaper{51}{Los adanes planetarios}
\author{Lanonandec secundario}
\vs p051 0:1 Durante la dispensación de un príncipe planetario, el hombre primitivo alcanza el límite de su desarrollo evolutivo natural, y este logro biológico indica al soberano del sistema el momento de enviar a ese mundo al segundo orden de filiación: a los mejoradores biológicos. Estos hijos, dado que hay dos de ellos ---el hijo material y la hija material---, normalmente se conocen en los planetas como Adán y Eva. Adán es el hijo material primigenio de Satania, y aquellos que acuden a los mundos del sistema como mejoradores biológicos llevan siempre el nombre de este hijo primero y primigenio del singular orden al que pertenecen.
\vs p051 0:2 Estos hijos constituyen el don material del hijo creador a los mundos habitados. Junto con el príncipe planetario, permanecen en el planeta que se les ha asignado durante todo el curso evolutivo de dicha esfera. Esta aventura no conlleva mucho riesgo en un mundo con un príncipe planetario, pero en un planeta apóstata, en un lugar sin gobernante espiritual y privado de comunicación interplanetaria, tal misión entraña un grave peligro.
\vs p051 0:3 Aunque no podéis albergar la esperanza de conocer todo en torno a la labor que realizan estos hijos en todos los mundos de Satania y en otros sistemas, en otros escritos se describe más plenamente la vida y experiencias de esta fascinante pareja, Adán y Eva, miembro del colectivo de mejoradores biológicos de Jerusem, que acudió a Urantia para la elevación de sus razas. Aunque no se cumplieron por completo los planes para este mejoramiento de vuestras razas autóctonas, la misión de Adán no fue del todo en vano; Urantia se ha beneficiado de forma inconmensurable del don otorgado al mundo por Adán y Eva; entre sus semejantes y en los consejos en las alturas, su labor no se considera una total pérdida.
\usection{1. ORIGEN Y NATURALEZA DE LOS HIJOS MATERIALES DE DIOS}
\vs p051 1:1 Los hijos e hijas materiales o sexuados son vástagos del hijo creador; el espíritu materno del universo no participa en la creación de estos seres destinados a realizar su cometido en los mundos evolutivos en calidad de mejoradores físicos.
\vs p051 1:2 Este orden material de filiación no es homogéneo en todo el universo local. En cada sistema local, el hijo creador da origen solamente a una pareja de estos seres; estas parejas primigenias, al tener que adaptarse al patrón de vida de sus respectivos sistemas, son de diversa naturaleza. Esto es una condición necesaria puesto que, de otra manera, el potencial reproductor de los adanes no sería de utilidad respecto al de los seres mortales evolutivos de los mundos de cualquier sistema concreto. El adán y eva que vinieron a Urantia provenían de la pareja primigenia de hijos materiales de Satania.
\vs p051 1:3 \pc La altura de los hijos materiales varía entre los dos metros y medio y los tres metros, y sus cuerpos resplandecen con un halo de luz radiante de tonalidad violeta. Aunque circula sangre material por sus cuerpos materiales, están igualmente repletos de energía divina y saturados con luz celestial. Estos hijos materiales (los adanes) e hijas materiales (las evas) son iguales entre sí; difieren tan solo en su naturaleza reproductora y en ciertas dotes químicas. Son iguales pero diferenciados, masculino y femenino ---complementarios por tanto--- y están concebidos para servir en pareja en casi todas sus misiones.
\vs p051 1:4 Los hijos materiales disfrutan de una doble nutrición; realmente tienen doble naturaleza y constitución, al participar de la energía materializada tal como lo hacen los seres físicos de los mundos, a la vez que mantienen plenamente su existencia inmortal mediante la ingestión directa y natural de ciertas energías cósmicas. Si fracasan en alguna misión o incluso si se rebelan de forma consciente y deliberada, este orden de hijos queda aislado, desconectado de la fuente de luz y vida del universo. En ese momento, se vuelven prácticamente seres materiales, destinándoseles a seguir el curso de la vida material en el mundo que se les ha asignado y obligándoseles a contar con el fallo de los magistrados de los universos. La muerte física finalmente acabará con la andadura planetaria de unos hijos o hijas materiales tan desacertados e insensatos.
\vs p051 1:5 Los adanes y las evas primigenios, o creados de forma directa, están inherentemente dotados de inmortalidad, al igual que todos los demás órdenes de filiación del universo local, pero sus hijos e hijas se caracterizan por la disminución del potencial de inmortalidad. Esta pareja primigenia no puede transmitir inmortalidad incondicionada a los hijos e hijas que han procreado. Su progenie depende, para continuar con la vida, de que estén sincronizados intelectualmente y de forma ininterrumpida con la vía circulatoria de la gravedad mental del Espíritu. Desde el comienzo del sistema de Satania, se han perdido trece adanes planetarios por rebelión y transgresión, y 681\,204 en puestos subordinados de confianza. La mayoría de estas deserciones se produjeron en la época de la rebelión de Lucifer.
\vs p051 1:6 \pc Mientras viven como ciudadanos permanentes en las capitales de los sistemas, incluso cuando actúan en misiones en las que descienden a los planetas evolutivos, los hijos materiales no poseen modelador del pensamiento, pero, a través de estos mismos servicios que prestan, adquieren la capacidad experiencial para ser morada de un modelador y partícipes de la andadura de ascensión al Paraíso. Estos seres tan extraordinarios y tan magníficamente útiles constituyen el eslabón que vincula el mundo espiritual con el mundo físico. Se encuentran en las sedes de los sistemas, donde se reproducen y se comportan como ciudadanos materiales de su entorno y, desde aquí, se envían a los mundos evolutivos.
\vs p051 1:7 A diferencia de otros hijos creados para el servicio planetario, este orden material de filiación no es, por naturaleza, invisible para criaturas materiales como los habitantes de Urantia. Estos Hijos de Dios se pueden ver y entender, y ellos pueden, a su vez, mezclarse de hecho con las criaturas del tiempo, incluso podrían procrear con ellas, aunque esta tarea de mejoramiento biológico recae, por lo general, en la progenie de los adanes planetarios.
\vs p051 1:8 \pc En Jerusem, los fieles hijos de cualquier adán y eva son inmortales, pero los vástagos de un hijo e hija materiales procreados con posterioridad a su llegada a un planeta evolutivo no son, de la misma manera, inmunes a la muerte natural. Cuando se rematerializa a estos hijos en un mundo evolutivo con un fin reproductor, ocurre un cambio en el mecanismo trasmisor de la vida. Los portadores de vida deliberadamente privan a los adanes y las evas planetarios de la facultad de engendrar hijos e hijas inmortales. Si no se rebelan, un adán y una eva en misión planetaria pueden vivir de forma indefinida, pero sus hijos, dentro de ciertos límites, experimentan una longevidad que disminuye al paso de cada generación.
\usection{2. EL TRASLADO DE LOS ADANES PLANETARIOS}
\vs p051 2:1 Al recibir la noticia de que otro mundo habitado ha alcanzado la cumbre de su evolución física, el soberano del sistema reúne en la capital de dicho sistema al colectivo de hijos e hijas materiales; y tras analizar las necesidades de este mundo evolutivo, se selecciona, de entre el grupo de voluntarios, a dos miembros ---un adán y una eva del colectivo de mayor rango y experiencia de los hijos materiales--- para emprender tal aventura, someterse al profundo sueño preparatorio a fin de ser envueltos en un serafín y ser transportados desde su hogar, en donde sirven de forma conjunta, a un mundo nuevo, con nuevas oportunidades y nuevos peligros.
\vs p051 2:2 Los adanes y las evas son criaturas semimateriales y, como tales, los serafines no los pueden transportar. Antes de poder viajar en un serafín para su traslado al mundo de destino, deben someterse en la capital del sistema a la desmaterialización. Los serafines transportadores son capaces de efectuar en los hijos materiales y en otros seres semimateriales unos cambios que les permitan ser envueltos en el serafín y ser, por tanto, trasladados, a través del espacio, desde un mundo o un sistema a otro. Se necesitan unos tres días de tiempo regular para preparar este traslado, y se requiere la cooperación de un portador de vida para restablecer a esta criatura desmaterializada a su existencia normal, una vez llega a término su viaje en trasporte seráfico.
\vs p051 2:3 \pc A pesar de haber un procedimiento de desmaterialización para preparar a los adanes en su viaje desde Jerusem a los mundos evolutivos, no hay un método equivalente para sacarlos de dichos mundos, a menos que todo el planeta tenga que ser desalojado, en cuyo caso, se instauraría, de urgencia, este procedimiento de desmaterialización para toda la población salvable. Si alguna catástrofe de tipo físico acechara la morada planetaria de una raza en evolución, los melquisedecs y los portadores de vida seguirían dicho procedimiento para todos los supervivientes, y el transporte seráfico llevaría a estos seres al nuevo mundo dispuesto para que continuaran su existencia. La evolución de la raza humana, una vez iniciada en un mundo del espacio, debe continuar independiente por completo de la supervivencia física de ese planeta, pero durante las eras evolutivas no se concibe otra manera de que el adán y la eva planetarios abandonen el mundo que han elegido.
\vs p051 2:4 \pc Al llegar a su destino planetario, se rematerializa al hijo y a la hija material, bajo la dirección de los portadores de vida. Todo este proceso lleva de diez a veintiocho días de tiempo de Urantia. Durante todo el período de reconstrucción, continúan en la inconsciencia de la dormición seráfica. Cuando se ha completado la reconstitución del organismo físico, estos hijos e hijas materiales permanecen en sus nuevos hogares y en sus nuevos mundos prácticamente tal como estaban antes de someterse en Jerusem al proceso de desmaterialización.
\usection{3. LAS MISIONES ADÁNICAS}
\vs p051 3:1 En los mundos habitados, los hijos e hijas materiales construyen sus propios hogares ajardinados, siendo pronto ayudados por sus propios hijos. Por lo general, el príncipe planetario escoge el emplazamiento y su comitiva corpórea realiza, con la ayuda de muchos de los órdenes mejor dotados de las razas nativas, gran parte de los preparativos iniciales.
\vs p051 3:2 Se llaman Jardines del Edén en honor de Edentia, la capital de la constelación, y debido a que se diseñan siguiendo la grandeza botánica del mundo sede de los padres altísimos. Estos hogares ajardinados se emplazan, por lo general, en sectores apartados, en una zona cerca de los trópicos. Son creaciones extraordinarias en mundos ordinarios. No os podéis haceros una idea de estos bellos centros de cultura partiendo del incompleto relato del malogrado desarrollo de tal proyecto en Urantia.
\vs p051 3:3 \pc Los adanes y las evas planetarios significan, en potencia, la plena dádiva de gracia física a las razas mortales. La actividad principal de esta pareja, venida de fuera, consiste en multiplicarse y elevar a los hijos del tiempo. Pero no hay, de forma inmediata, cruzamiento entre los seres del jardín y los seres del mundo; durante muchas generaciones Adán y Eva permanecen biológicamente separados de los mortales evolutivos, mientras desarrollan una raza fuerte a partir de su orden. En los mundos habitados, este es el origen de la raza violeta.
\vs p051 3:4 El príncipe planetario y su comitiva realizan los planes destinados al avance de la raza y Adán y Eva los llevan a cabo. Y fue aquí donde vuestro hijo material y su compañera se encontraron con un gran inconveniente al llegar a Urantia. Caligastia se opuso astuta y contundentemente a la misión adánica; y, a pesar de que los síndicos melquisedecs de Urantia habían advertido oportunamente tanto a Adán como a Eva de los peligros planetarios consecuentes a la presencia del príncipe planetario rebelde. Este archirrebelde, mediante una astuta estratagema, se mostró más hábil que la pareja edénica y los incitó a la violación del pacto de responsabilidad que habían contraído como gobernantes visibles de vuestro mundo. El traidor príncipe planetario consiguió poner a vuestro adán y eva en una situación comprometida, pero fracasó en su intento de implicarlos en la rebelión de Lucifer.
\vs p051 3:5 \pc El quinto orden de ángeles, o ayudantes planetarios, está adscrito a la misión adánica y siempre acompaña a los adanes planetarios en sus aventuras en los mundos. El colectivo que, por lo general, se asigna inicialmente a esta misión suma unos cien mil. Cuando se precipitó de forma prematura la tarea de Adán y de Eva en Urantia, cuando se desviaron del plan establecido, fue una de las voces seráficas del Jardín la que les amonestó por su censurable conducta. Y vuestra narración de este suceso pone bien de manifiesto la tendencia de vuestras tradiciones planetarias a atribuir a Dios nuestro Señor todo lo sobrenatural. Debido a esto, los urantianos, a menudo se sienten confundidos acerca de la naturaleza del Padre Universal por habérsele atribuido a él, de forma tan generalizada, las palabras y actos de todos sus colaboradores y subordinados. En el caso de Adán y Eva, el ángel del Jardín no era otro que el jefe de los ayudantes planetarios entonces en servicio. Este serafín, Solonia, dio a conocer el malogro del plan divino y solicitó el regreso a Urantia de los síndicos melquisedecs.
\vs p051 3:6 \pc Las criaturas intermedias secundarias son connaturales a las misiones adánicas. Como ocurre con la comitiva corpórea del príncipe planetario, los descendientes de los hijos e hijas materiales son de dos clases: sus hijos físicos y el orden secundario de criaturas intermedias. Estos servidores planetarios materiales, aunque generalmente invisibles, contribuyen notablemente al avance de la civilización e incluso al sometimiento de minorías insubordinadas que traten de socavar el desarrollo social y el progreso espiritual.
\vs p051 3:7 No se debe confundir a los seres intermedios secundarios con el orden primario de estos seres, que se remonta a los tiempos cercanos a la llegada del príncipe planetario. En Urantia, la mayoría de estas criaturas intermedias anteriores se unieron a la rebelión de Caligastia y, desde Pentecostés, han estado internados. Muchos seres del grupo adánico, que no permanecieron leales al gobierno planetario están igualmente internados.
\vs p051 3:8 El día de Pentecostés, los seres intermedios primarios y secundarios leales se unieron de forma voluntaria y, desde entonces, han obrado en los asuntos del mundo como una unidad. Sirven bajo el liderazgo de seres intermedios leales elegidos alternativamente de ambos grupos.
\vs p051 3:9 \pc Cuatro órdenes filiales han visitado vuestro mundo: Caligastia, el príncipe planetario; Adán y Eva, hijos materiales de Dios; Maquiventa Melquisedec, el “sabio de Salem” en los días de Abraham; y Cristo Miguel, que vino como el hijo del Paraíso en su ministerio de gracia. ¡Nada hubiese sido más hermoso y válido si a Miguel, el gobernante supremo del universo de Nebadón, lo hubieran recibido en vuestro mundo un príncipe planetario leal y eficiente y un hijo material devoto y triunfante! ¡Ambos podrían haber hecho tanto para potenciar la labor de vida y la misión del hijo de gracia! Pero no todos los mundos han sido tan desafortunados como Urantia, ni las misiones de los adanes planetarios han sido siempre tan difíciles y tan arriesgadas. Cuando alcanzan el éxito, contribuyen al desarrollo de un gran pueblo, continuando como cabezas visibles de los asuntos planetarios incluso más allá de la era del asentamiento de ese mundo en luz y vida.
\usection{4. LAS SEIS RAZAS EVOLUTIVAS}
\vs p051 4:1 Durante las primeras eras de los mundos habitados, la raza dominante es la del hombre rojo, habitualmente la primera en alcanzar niveles humanos de desarrollo. Si bien, aunque el hombre rojo constituya la raza más antigua de los planetas, los pueblos de color que le siguen comienzan a surgir muy temprano en la era de la aparición de los mortales.
\vs p051 4:2 Las primeras razas están, de alguna manera, mejor dotadas que las posteriores; el hombre rojo se sitúa muy por encima de la raza índigo o negra. Los portadores de vida confieren la dotación plena de las energías vivas a la raza inicial, o raza roja, y cada manifestación evolutiva sucesiva de un grupo distinto de mortales representa una variación a expensas de la dotación primigenia. Incluso la estatura de los mortales tiende a disminuir desde el hombre rojo hasta la raza índigo, aunque en Urantia, entre los pueblos verde y naranja, aparecieron estirpes inesperadas de gigantismo.
\vs p051 4:3 En esos mundos que tienen las seis razas evolutivas, los pueblos mejor dotados son las razas primera, tercera y quinta ---la roja, la amarilla y la azul---. Las razas evolutivas, por tanto, alternan en su capacidad para el crecimiento intelectual y el desarrollo espiritual, siendo la segunda, la cuarta y la sexta, en cierto modo, las menos dotadas. Estas razas secundarias son los pueblos que faltan en determinados mundos y aquellos que han sido exterminados en muchos otros. En Urantia, es de lamentar que hayáis perdido una buena parte de vuestros mejor dotados hombres azules, excepto por su persistencia en vuestra mezclada “raza blanca”. La pérdida de vuestros linajes naranja y verde no reviste gran importancia.
\vs p051 4:4 La evolución de seis ---o de tres--- razas de color, aunque parezca empeorar la dotación primigenia del hombre rojo, proporciona ciertas variaciones muy convenientes en los grupos de mortales y permite la manifestación, de otra manera inalcanzable, de distintos potenciales humanos. Estas modificaciones son beneficiosas para el progreso de la humanidad en su conjunto, siempre que sean, con posterioridad, mejoradas por la introducción de la raza adánica o raza violeta. En Urantia no se llevó a cabo con amplitud este plan habitual de cruzamiento, y el fracaso en el cumplimiento del plan de evolución racial imposibilita que, partiendo de la observación de los restos de estas primeras razas en vuestro mundo, entendáis, en buena medida, la condición de estos pueblos en un típico planeta habitado.
\vs p051 4:5 \pc En los primeros días del desarrollo racial, existe una ligera tendencia entre los hombres rojos, los amarillos y los azules al cruzamiento; se da una inclinación similar a entremezclarse en las razas naranja, verde e índigo.
\vs p051 4:6 Normalmente, las razas más adelantadas utilizan a los humanos más atrasados como obreros. Aquí radica el origen de la esclavitud en los planetas durante las primeras eras. Por lo general, los hombres rojos someten a los hombres naranjas y los reducen a la condición de sirvientes ---a veces los exterminan---. Los hombres amarillos y los hombres rojos generalmente confraternizan, pero no siempre. La raza amarilla normalmente esclaviza a la verde, mientras que el hombre azul somete al índigo. Estas razas de hombres primitivos no les dan mayor importancia al hecho de utilizar los servicios de sus congéneres más atrasados en trabajos forzosos que la que le darían los urantianos al hecho de comprar y vender caballos y ganado.
\vs p051 4:7 En la mayoría de los mundos, la servidumbre involuntaria no pervive a la dispensación del príncipe planetario, aunque sea todavía frecuente obligar a los deficientes mentales y a los delincuentes sociales a realizar trabajos no voluntarios. Pero, en todas las esferas normales, esta clase de esclavitud primitiva queda abolida poco después de la importada raza violeta o raza adánica.
\vs p051 4:8 Estas seis razas evolutivas están destinadas a mezclarse y enaltecerse mediante el cruzamiento con la progenie de los mejoradores adánicos. Si bien, antes de que se mezclen estos pueblos, los menos dotados e inaptos quedan, en gran medida, excluidos. El príncipe planetario y el hijo material, junto con otras autoridades planetarias pertinentes, deciden sobre la aptitud adaptativa de los linajes reproductores. La dificultad de llevar a cabo, en Urantia, un programa tan radical se debe a la ausencia de expertos competentes que decidan sobre la adaptación o inadaptación biológica de los miembros de las razas de vuestro mundo. A pesar de este obstáculo, parece que deberíais ser capaces de estar de acuerdo con el apartamiento biológico de vuestros linajes más acentuadamente inaptos, deficientes, en declive degenerativo y antisociales.
\usection{5. CRUZAMIENTO RACIAL: LA DOTACIÓN DE LA SANGRE ADÁNICA}
\vs p051 5:1 Cuando un Adán y una Eva Planetarios llegan a un mundo habitado, sus superiores los han instruido plenamente sobre la mejor manera de llevar a efecto el mejoramiento de las razas de seres inteligentes que allí existen. El plan a seguir no es inflexible, se deja mucho del ministerio de esta pareja a su propio criterio, y los errores no son infrecuentes, especialmente en mundos como Urantia, con desórdenes e insurrección.
\vs p051 5:2 Por lo general, los pueblos violetas no comienzan a mezclarse con los nativos del planeta hasta que su propio grupo no suma más de un millón de seres. Pero, entretanto, la comitiva del príncipe planetario proclama que los hijos de los Dioses han descendido, por así decirlo, para efectuar su unión con las razas de los hombres; y la gente espera anhelante la llegada de ese día en el que se les anuncie que aquellos cualificados como pertenecientes a estirpes raciales mejor dotadas pueden dirigirse al Jardín del Edén para ser elegidos por los hijos e hijas de Adán, como padres y madres evolutivos de un nuevo orden humano que surge de la mezcla de razas.
\vs p051 5:3 En mundos normales, el adán y la eva planetarios nunca se emparejan con las razas evolutivas. Esta labor de mejoramiento biológico es obra de la progenie adánica. Pero estos adanitas no salen a encontrarse con las razas; la comitiva del Príncipe trae al Jardín del Edén a los hombres y mujeres mejor dotados para que, de forma voluntaria, se emparejen con los vástagos adánicos. Y en la mayoría de los mundos, ser elegido aspirante para emparejarse con los hijos e hijas del jardín representa un gran honor.
\vs p051 5:4 Por primera vez, aminoran las guerras raciales y las otras luchas tribales, al mismo tiempo que las razas del mundo pugnan cada vez más por alcanzar reconocimiento y admisión al jardín. En el mejor de los casos, solo podéis llegar a tener una mínima idea de cómo esta lucha competitiva pasa a ocupar en un planeta normal el centro de toda su actividad. En Urantia, todo este plan de mejoramiento de las razas se quebró prematuramente.
\vs p051 5:5 \pc La raza violeta es un pueblo monógamo, y todo hombre o mujer promete, al unirse con los hijos e hijas adánicos, no tomar ninguna otra pareja e instruir a sus hijos o hijas en el emparejamiento único. Los hijos de cada una de estas uniones se educan y forman en las escuelas del príncipe planetario y, posteriormente, se les permite marchar a la raza de sus progenitores evolutivos, para desposarse allí con miembros de los grupos elegidos de mortales mejor dotados.
\vs p051 5:6 Cuando este linaje de los hijos materiales se agrega a las razas en evolución de los mundos, se inicia una nueva era, una gran era en el progreso evolutivo. Tras este derramamiento procreador de capacidad y de rasgos supraevolutivos importados, sobrevienen una serie de rápidos avances en la civilización y en el desarrollo racial; en cien mil años se hacen más avances que en un millón de años de enfrentamientos previos. En vuestro mundo, a pesar de que se malograron los planes dispuestos, se ha efectuado un gran progreso desde el momento en que se produjo el don a vuestros pueblos del plasma vital de Adán.
\vs p051 5:7 Pero aunque los hijos por línea pura del Jardín del Edén planetario puedan darse a los miembros mejor dotados de las razas evolutivas y, por tanto, elevar el nivel biológico de la humanidad, no resultaría beneficioso que estas estirpes de mortales de Urantia se emparejaran con las razas menos dotadas; proceder de forma tan insensata pondría en peligro toda la civilización de vuestro mundo. Al no haber logrado armonizar las razas según el método adánico, debéis ahora resolver vuestros problemas planetarios de mejoramiento racial mediante otros métodos, en gran parte humanos, de adaptación y de control.
\usection{6. EL RÉGIMEN EDÉNICO}
\vs p051 6:1 En la mayoría de los mundos habitados, los Jardines del Edén permanecen como magníficos centros culturales y continúan operativos era tras era como modelos sociales de proceder y uso planetarios. Incluso en los primeros tiempos, cuando los pueblos violetas están relativamente apartados, sus escuelas acogen a aspirantes idóneos provenientes de las razas del mundo; entretanto, con el desarrollo industrial del jardín se abren nuevas vías de intercambio comercial. Los adanes y las evas y su progenie contribuyen así a la repentina expansión de la cultura y al rápido mejoramiento de las razas evolutivas de sus mundos. Y el cruzamiento de las razas evolutivas y los hijos de Adán sirve de refuerzo y sella todas estas relaciones, teniendo como resultado la inmediata elevación del estatus biológico, el aumento del potencial intelectual y el realce de la receptividad espiritual.
\vs p051 6:2 En los mundos normales, la sede jardín de la raza violeta se erige como segundo centro de la cultura mundial y, junto con la ciudad sede del príncipe planetario, marca la pauta del desarrollo de la civilización. A lo largo de los siglos, las escuelas de dicha ciudad sede y las escuelas jardín de Adán y de Eva son coetáneas. Por lo general, no están muy distantes entre sí, y cooperan armoniosamente en labores conjuntas.
\vs p051 6:3 Pensad lo que significaría para vuestro mundo si en algún lugar del Levante hubiese un centro mundial de civilización, una gran universidad planetaria de la cultura, que hubiese estado operando de forma ininterrumpida durante 37\,000 años. Y de nuevo, deteneos a pensar cómo se reforzaría la autoridad moral de un centro tan antiguo gracias a la proximidad de otra sede más antigua dedicada al ministerio celestial, cuyas tradiciones ejercerían una importancia creciente de 500\,000 años de armoniosa influencia evolutiva. Es la costumbre la que acaba por difundir a todo el mundo los ideales del Edén.
\vs p051 6:4 Las escuelas de los príncipes planetarios se ocupan principalmente de la filosofía, de la religión, de la moral y de los logros intelectuales y artísticos de orden superior. Las escuelas jardín de Adán y Eva están normalmente dedicadas a las artes prácticas, al entrenamiento intelectual básico, a la cultura social, al desarrollo económico, a las relaciones comerciales, a la preparación física y al gobierno civil. Estos centros mundiales acaban por fusionarse, pero la verdadera vinculación a veces no ocurre hasta los tiempos del primer hijo magistrado.
\vs p051 6:5 \pc La existencia continuada del adán y eva planetarios, junto con el núcleo de la línea pura de la raza violeta, confiere un crecimiento estable a la cultura edénica en virtud de que llega a actuar, con el convincente valor de la tradición, sobre la civilización de un mundo. En estos hijos e hijas materiales inmortales, encontramos el último e indispensable eslabón que enlaza a Dios con el hombre, salvando el abismo casi infinito entre el Creador eterno y los más modestos seres personales finitos del tiempo. He aquí a un ser de elevado origen que es físico, material, incluso sexual como los mortales de Urantia, que puede ver y comprender al invisible príncipe planetario y desvelarlo a las criaturas mortales del mundo, porque los hijos e hijas materiales son capaces de ver a todos los órdenes menores de seres espirituales; visualizan al príncipe planetario y a toda su comitiva, a los visibles y a los invisibles.
\vs p051 6:6 Con el paso de los siglos, mediante el cruzamiento de su progenie con las razas de los hombres, estos mismos hijos e hijas materiales se llegan a aceptar como los ancestros comunes de la humanidad, como los padres que comparten los ahora mezclados descendientes de las razas evolutivas. Se pretende que los mortales que parten de un mundo habitado tengan la experiencia de reconocer a siete padres:
\vs p051 6:7 \li{1.}El padre biológico: el padre en la carne.
\vs p051 6:8 \li{2.}El padre del mundo: el adán planetario.
\vs p051 6:9 \li{3.}El padre de las esferas; el soberano del sistema.
\vs p051 6:10 \li{4.}El padre altísimo: el padre de la constelación.
\vs p051 6:11 \li{5.}El padre del universo: el hijo creador y gobernante supremo de las creaciones locales.
\vs p051 6:12 \li{6.}Los suprapadres: los ancianos de días que gobiernan el suprauniverso.
\vs p051 6:13 \li{7.}El Padre Espiritual o Padre de Havona: el Padre Universal, que mora en el Paraíso y da su espíritu como dádiva para que viva y obre en las mentes de las modestas criaturas que habitan el universo de los universos.
\usection{7. ADMINISTRACIÓN CONJUNTA}
\vs p051 7:1 En ocasiones, los hijos avonales del Paraíso acuden a los mundos habitados para llevar a cabo actuaciones judiciales, pero el primer avonal en llegar en calidad de magistrado inaugura la cuarta dispensación del mundo evolutivo del tiempo y del espacio en el que hace aparición. En planetas donde se acepta de manera generalizada a dicho hijo magistrado, este permanece por una era; y, por tanto, el planeta prospera bajo el gobierno conjunto de tres hijos: el príncipe planetario, el hijo material y el hijo magistrado; los dos últimos son visibles para todos los habitantes del mundo.
\vs p051 7:2 Antes de que el primer hijo magistrado concluya su misión en un mundo evolutivo normal, ya se ha efectuado la unión de la labor educativa y administrativa del príncipe planetario y del hijo material. Esta fusión de la doble dirección de un planeta da origen a una administración del mundo de un orden nuevo y eficaz. Cuando el hijo magistrado parte, el adán planetario asume la dirección externa de la esfera. El hijo y la hija materiales actúan, pues, de forma conjunta en calidad de regidores del planeta hasta el asentamiento del mundo en la era de luz y vida; con lo cual, el príncipe planetario es elevado a la condición de soberano planetario. Durante esta era de progreso evolutivo, Adán y Eva se convierten en lo que se podría llamar presidentes conjuntos del mundo glorificado.
\vs p051 7:3 En cuanto la nueva y consolidada capital del mundo evolutivo está bien establecida, y con la celeridad con la que se pueda formar adecuadamente a capaces regidores de menor rango, se fundan unas capitales secundarias en tierras distantes y entre los distintos pueblos. Antes de la llegada de otro hijo para la inauguración de una dispensación, se habrán creado entre cincuenta y cien de estos centros secundarios.
\vs p051 7:4 El príncipe planetario y su comitiva continúan fomentando los ámbitos de actividad espirituales y filosóficos. Adán y Eva prestan una atención especial a la condición física, científica y económica del mundo. Los dos grupos dedican por igual sus energías a promover las artes, las relaciones sociales y los logros intelectuales.
\vs p051 7:5 En el momento de inaugurarse la quinta dispensación de los asuntos del mundo, ya se ha conseguido instaurar una magnífica administración de la actividad planetaria. La existencia mortal en una esfera tan bien regentada es, en efecto, estimulante y fructífera. Y ojalá pudiesen los urantianos observar la vida en un planeta así, pues de inmediato apreciarían el valor de aquellas cosas que su mundo ha perdido al abrazar el mal y participar en la rebelión.
\vsetoff
\vs p051 7:6 [Exposición de un hijo lanonandec secundario del colectivo de reserva.]
