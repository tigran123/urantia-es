\upaper{133}{Regreso de Roma}
\author{Comisión de seres intermedios}
\vs p133 0:1 Cuando se preparaba para dejar Roma, Jesús no se despidió de ninguno de sus amigos. El Escriba de Damasco había aparecido en Roma sin anunciarse y desapareció de igual manera. Transcurriría un año completo antes de que aquellos que lo conocieron y amaron perdieran cualquier esperanza de volver a verlo de nuevo. Antes del final del segundo año, unidos por un interés común en sus enseñanzas y por el recuerdo mutuo de los buenos momentos vividos con él, se formaron pequeños grupos de los que lo habían conocido. Y estos pequeños grupos de estoicos, cínicos y miembros de los cultos mistéricos continuaron celebrando estas reuniones ocasionales e informales hasta el momento en el que los primeros predicadores de la religión cristiana hicieron su aparición en Roma.
\vs p133 0:2 \pc Gonod y Ganid habían adquirido tantas cosas en Alejandría y en Roma que enviaron con antelación todas sus pertenencias por transporte de caravana a Tarento, mientras que los tres viajeros caminaban placenteramente a través de Italia por la gran vía Apia. En el trayecto, se encontraron con toda clase de seres humanos. Muchos nobles ciudadanos romanos y colonos griegos vivían a lo largo de esta vía, si bien, la progenie de un gran número de esclavos pobremente dotados estaba ya comenzando a hacer su aparición.
\vs p133 0:3 Cierto día, mientras descansaban durante el almuerzo, aproximadamente a medio camino de Tarento, Ganid le preguntó abiertamente a Jesús qué pensaba sobre el sistema de castas de la India. Jesús le contestó: “Aunque los seres humanos difieren unos de otros de muchas formas, ante Dios y en el mundo espiritual, todos los mortales están en términos de igualdad. Solo existen dos grupos de mortales a los ojos de Dios: aquellos que desean hacer su voluntad y aquellos que no lo desean. Cuando desde el universo se contempla un mundo habitado, se perciben igualmente dos grandes clases: aquellos que conocen a Dios y aquellos que no lo conocen. Los que no pueden conocerlo se cuentan entre los animales de un determinado planeta. La humanidad se puede clasificar convenientemente en muchas clases de acuerdo a diferentes cualidades, ya sea desde el punto de vista físico, mental, social, vocacional o moral, pero cuando estas diferentes clases de mortales comparecen ante el tribunal de juicio de Dios, lo hacen en igualdad de condiciones. Verdaderamente Dios no hace acepción de personas. Aunque no se puede eludir el reconocimiento de las distintas capacidades y dotes humanas en cuestiones intelectuales, sociales y morales, no se debería hacer tales distinciones en lo que respecta a la fraternidad espiritual de los hombres cuando se congregan para adorar en la presencia de Dios”.
\usection{1. LA MISERICORDIA Y LA JUSTICIA}
\vs p133 1:1 Una tarde, cuando se acercaban a Tarento, se produjo un hecho muy llamativo al borde del camino. Observaron que un joven rudo y bravucón atacaba brutalmente a un muchacho más pequeño. Jesús se apresuró a asistir al que estaba siendo agredido y, cuando lo había rescatado, sujetó firmemente al agresor hasta que el más joven logró escapar. En el momento en que Jesús soltó al acosador, Ganid se abalanzó sobre él y empezó a golpearle de forma contundente, si bien, ante su estupor, Jesús intervino rápidamente. Una vez que sujetó a Ganid y permitió que el asustado muchacho escapara, el joven, tan pronto como recobró el aliento, exclamó con agitación: “Maestro, no te comprendo. Si la misericordia requiere que se rescate al muchacho más pequeño, ¿no exige la justicia el castigo del más grande, del agresor?”. Jesús le dijo:
\vs p133 1:2 “Ganid, es verdad que no lo entiendes. El ministerio de la misericordia es siempre la obra del individuo, pero el castigo de la justicia es la labor de los grupos sociales, gubernamentales o administrativos del universo. Como persona, estoy obligado a mostrar misericordia; debo ir al rescate del muchacho agredido y, como corresponde, podré emplear la fuerza necesaria para contener al agresor. Y eso es precisamente lo que hice. Conseguí liberar al muchacho agredido; hasta aquí el fin del ministerio de la misericordia. Entonces, retuve por la fuerza al agresor el tiempo necesario como para permitir la huida de la persona más débil de la disputa, tras lo cual me retiré del asunto. No procedí a juzgar al agresor, determinando pues el móvil ---arbitrando todo aquello que había contribuido al ataque contra su semejante--- y ni administré entonces el castigo que mi mente pudiera imponer como reparación justa por su mala acción. Ganid, la misericordia puede ser pródiga, pero la justicia es precisa. ¿No comprendes que es poco probable que dos personas pudieran ponerse de acuerdo con respecto a un castigo que satisficiera las exigencias de la justicia? Uno impondría cuarenta latigazos; otro, veinte; mientras que habría quien recomendara el aislamiento como el único castigo justo. ¿No ves que en este mundo sería mejor que tales responsabilidades recayeran sobre el grupo o estuviesen a cargo de representantes elegidos de ese grupo? En el universo, el juicio es competencia de aquellos que conocen plenamente los antecedentes de toda maleficencia al igual que sus motivaciones. En una sociedad civilizada y en un universo organizado, la administración de la justicia implica el dictado de una sentencia justa impartida en consonancia a un juicio justo, y estas prerrogativas recaen en los grupos jurídicos de los mundos y en los omniscientes administradores de los universos más elevados de toda la creación”.
\vs p133 1:3 Durante varios días hablaron sobre este problema de manifestar misericordia y administrar justicia. Y Ganid llegó a comprender, al menos hasta cierto punto, por qué Jesús no quería participar en una pelea personal. Pero Ganid le hizo una última pregunta, a la que nunca recibió una respuesta totalmente satisfactoria; y esa pregunta fue: “Pero, Maestro, si una criatura más fuerte y enfurecida te atacara y amenazara con matarte, ¿qué harías? ¿No tratarías de defenderte?”. Si bien, Jesús no podía responder plena y satisfactoriamente a la pregunta del muchacho, debido a que no deseaba revelarle que él (Jesús) estaba viviendo en la tierra como representación del amor del Padre del Paraíso para todo un expectante universo, sí le dijo esto:
\vs p133 1:4 “Ganid, entiendo bien cómo estos problemas te pueden desconcertar, e intentaré dar respuesta a tu pregunta. Primero, en cualquier ataque que pudiera hacerse contra mi persona, determinaría si el agresor es o no un hijo de Dios ---mi hermano en la carne---, y si pensara que esa criatura no posee juicio moral ni discernimiento espiritual, me defendería sin vacilaciones hasta el máximo de la capacidad de mi fuerza de resistencia, con independencia de las consecuencias para el atacante. Pero no agrediría así a un semejante con rango de filiación, ni siquiera en defensa propia. Esto es, no le castigaría anticipadamente y sin juicio por agredirme. Intentaría cualquier estratagema posible para prevenir que llevase a cabo su ataque, disuadiéndole para que no lo hiciera, pero, si no consiguiera evitarlo, trataría de mitigar su intensidad. Ganid, tengo confianza absoluta en los atentos cuidados de mi Padre celestial. Estoy consagrado a hacer la voluntad de mi Padre de los cielos. No creo que me pueda ocurrir ningún daño \bibemph{real;} no creo que las intenciones de mis enemigos respecto a mí puedan poner en peligro mi labor en la vida, y de seguro que no hemos de temer violencia alguna de parte de nuestros amigos. Estoy absolutamente convencido de que el universo entero me es amigable; estoy empeñado en creer fervorosamente esta verdad todopoderosa, a pesar de que las apariencias puedan indicar lo contrario”.
\vs p133 1:5 Pero Ganid no quedó satisfecho del todo. Hablaron repetidas veces sobre estos temas, y Jesús le contó algunas de sus experiencias juveniles y también lo sucedido con Jacob, el hijo del albañil. Al conocer cómo Jacob se había erigido defensor de Jesús, Ganid dijo: “¡Oh, ahora empiezo a entender! En primer lugar, muy raramente se le ocurriría a un ser humano normal atacar a una persona tan bondadosa como tú, pero incluso si alguien fuese tan irreflexivo como para hacer tal cosa, habría con toda seguridad algún otro mortal cerca que se apresurase a ayudarte, de la misma manera que tú siempre acudes a rescatar a cualquier persona que ves en apuros. En mi corazón, Maestro, estoy de acuerdo contigo, pero en mi cabeza, todavía pienso que de haber sido yo Jacob, me habría gustado castigar a esos brutos que se atrevían a agredirte a sabiendas de que tú no te defenderías. Supongo que te encontrarás bastante seguro en tu viaje por la vida, ya que dedicas mucho tiempo a ayudar a los demás y a dar asistencia a los afligidos. Bueno, lo más probable es que siempre habrá alguien dispuesto a defenderte”. Y Jesús le contestó: “Esa prueba aún no ha llegado, Ganid, y cuando llegue, debemos cumplir la voluntad del Padre”. Y fue eso casi todo lo que el muchacho pudo conseguir que su maestro le dijera sobre el difícil tema de la defensa propia y la no confrontación. En otra ocasión, consiguió recabar de Jesús la opinión de que la sociedad organizada tenía todo el derecho de emplear la fuerza en la ejecución de sus justos mandatos.
\usection{2. EMBARQUE EN TARENTO}
\vs p133 2:1 Mientras hacían tiempo en el embarcadero, esperando a que el barco descargara sus mercancías, los viajeros vieron a un hombre maltratando a su esposa. Como era su costumbre, Jesús intervino en defensa de la persona que estaba siendo agredida. Avanzó por detrás del airado marido y, golpeándole ligeramente en el hombro, le dijo: “Amigo mío, ¿me permites que hable en privado contigo un momento?”. El furioso marido, perplejo por la forma en la que se había visto abordado, tras un momento de embarazosa indecisión, balbuceó: “¡Eh! ¿Por qué? Sí, ¿qué quieres de mí?”. Cuando Jesús se lo llevó a un lado, le dijo: “Amigo mío, veo que te debe haber ocurrido algo terrible; deseo encarecidamente que me expliques qué ha podido suceder que lleve a un hombre fuerte como tú a atacar a su esposa, a la madre de sus hijos, y justo aquí a la vista de todos. Estoy seguro de que debes pensar que tienes alguna buena razón para esta agresión ¿Qué hizo esta mujer para merecer semejante trato de su marido? Al mirarte, creo apreciar en tu rostro el amor por la justicia e incluso el deseo de ser misericordioso. Aseguraría que si me encontraras a un lado del camino siendo atacado por unos ladrones, te apresurarías a rescatarme sin titubear. Me atrevería a decir que has hecho muchos de estos valientes actos en el transcurso de tu vida. Ahora bien, amigo mío, dime ¿qué pasa? ¿Hizo tu mujer algo malo, o es que perdiste neciamente la cabeza y la agrediste sin pensar?”. Lo que conmovió el corazón de aquel hombre no fue tanto lo que Jesús le dijo sino la mirada bondadosa y la sonrisa compasiva que le brindó al concluir sus comentarios. El hombre dijo: “Entiendo que eres un sacerdote de los cínicos, y te estoy agradecido por haberme contenido. Mi mujer no ha cometido ningún gran mal; es una buena mujer, pero me irrita porque se mete conmigo en público, y pierdo la cabeza. Lamento mi falta de autodominio, y me comprometo a tratar de vivir de acuerdo a la promesa que le hice a uno de tus hermanos que, hace muchos años, me enseñó el mejor camino. Te lo prometo”.
\vs p133 2:2 Y luego, al despedirse de él, Jesús le dijo: “Hermano mío, recuerda siempre que el hombre no tiene ninguna autoridad legítima sobre la mujer a menos que ella se la haya concedido de buen grado y voluntariamente. Tu esposa se ha comprometido a pasar contigo la vida, a ayudarte a librar sus batallas y a asumir con diferencia la mayor parte de la responsabilidad al tener que dar a luz y criar a tus hijos; y, a cambio de esta excepcional labor, lo justo es que reciba de ti esa protección especial que el hombre puede dar a la mujer, a la compañera que ha de gestar, dar a luz y alimentar a los hijos. El amoroso cuidado y la consideración que un hombre esté dispuesto a dar a su esposa y a sus hijos es lo que mide el logro de ese hombre de los niveles superiores de autoconciencia creativa y espiritual. ¿Acaso no sabes que los hombres y las mujeres son copartícipes con Dios en el sentido de que cooperan en la creación de crear seres que crecen para ser poseedores del potencial de un alma inmortal? El Padre de los cielos trata al Espíritu Madre de sus hijos del universo como su igual. Tiene carácter divino compartir tu vida y todo lo que esta implica en igualdad de condiciones con la madre y compañera que tan plenamente participa contigo en esa experiencia divina de reproduciros en las vidas de vuestros hijos. Si llegaras a amar a tus hijos como Dios te ama a ti, amarías y valorarías a tu esposa como el Padre celestial honra y ensalza al Espíritu Infinito, la madre de todos los hijos espirituales de un inmenso universo”.
\vs p133 2:3 Al subir a bordo de la embarcación, se volvieron para contemplar la escena de la pareja unida en un abrazo silencioso y con lágrimas en los ojos. Al haber oído la última parte del mensaje de Jesús al hombre, Gonod pasó todo el día ocupado reflexionando al respecto, y dispuso que reorganizaría su hogar cuando volviese a la India.
\vs p133 2:4 El trayecto a Nicópolis fue placentero, pero lento porque el viento no era favorable. Los tres pasaron muchas horas comentando las experiencias vividas en Roma y rememorando todo lo que les había sucedido desde que se conocieron por primera vez en Jerusalén. Ganid se había impregnado del espíritu del ministerio personal. Empezó su labor con el mayordomo del barco, pero, al segundo día, al verse en las difíciles aguas de la religión, solicitó la ayuda de Josué.
\vs p133 2:5 Pasaron algunos días en Nicópolis, la ciudad que Augusto había fundado unos cincuenta años atrás como “la ciudad de la victoria”, en conmemoración de la batalla de Actium; lugar en el que había acampado con su ejército antes de la batalla. Se hospedaron en la casa de un tal Jerami, un prosélito griego de la fe judía, a quien habían conocido a bordo del barco. El apóstol Pablo pasó todo el invierno con el hijo de Jerami en la misma casa durante el transcurso de su tercer viaje misionero. Desde Nicópolis, partieron en el mismo barco para Corinto, la capital de la provincia romana de Acaya.
\usection{3. EN CORINTO}
\vs p133 3:1 Cuando llegaron a Corinto, Ganid empezaba a mostrar un gran interés en la religión judía, y no resultó extraño pues que, cierto día, al pasar por la sinagoga y ver a la gente entrar, Ganid le pidiera a Jesús que lo llevara a los servicios de culto. Ese día oyeron el discurso de un erudito rabino sobre el “Destino de Israel” y, tras los oficios, conocieron a un tal Crispo, el jefe principal de la sinagoga. Asistieron a los servicios de la sinagoga numerosas veces, pero volvían principalmente para encontrarse con Crispo. Ganid llegó a tomarle mucho cariño a Crispo, a su esposa y a su familia de cinco hijos. Disfrutaba enormemente observando cómo era la vida en familia de un judío.
\vs p133 3:2 Mientras Ganid examinaba la vida familiar, Jesús le enseñaba a Crispo los mejores caminos de la vida religiosa. Jesús mantuvo más de veinte reuniones con este judío de amplias miras; no es de extrañar, pues, que años más tarde, cuando Pablo predicó en esta misma sinagoga, y los judíos rechazaron su mensaje y votaron en contra de que siguiera predicando allí, haciendo que Pablo fuese entonces a los gentiles, que ese mismo Crispo, junto con toda su familia, abrazara la nueva religión, convirtiéndose en uno de los pilares principales de la Iglesia cristiana que Pablo organizó posteriormente en Corinto.
\vs p133 3:3 Durante los dieciocho meses que pasó Pablo predicando en Corinto ---Silas y Timoteo se le unirían más tarde---, conoció a muchos otros a quienes “el Tutor Judío del hijo de un mercader indio” les había impartido sus enseñanzas.
\vs p133 3:4 En Corinto conocieron a personas de todas las razas llegadas de tres continentes. Tras Alejandría y Roma, esta era la ciudad más cosmopolita del imperio mediterráneo. Había mucho que ver en esa ciudad, y Ganid nunca se cansaba de visitar la ciudadela, que se alzaba a más de seiscientos metros por encima del nivel del mar. También pasaba buena parte de su tiempo libre entre la sinagoga y la casa de Crispo. En un principio le chocó, pero después quedó encantado con el estatus de la mujer en el hogar judío; aquello fue todo un descubrimiento para este joven indio.
\vs p133 3:5 A menudo, Jesús y Ganid eran invitados en otro hogar judío, el de Justo, un mercader devoto que vivía junto a la sinagoga. Y muchas veces, tiempo después, cuando residía en esta misma casa, el Apóstol Pablo escuchó el relato de estas visitas del muchacho indio y de su tutor judío, y tanto Pablo como Justo se preguntaban qué habría sido de aquel maestro hebreo tan sabio y brillante.
\vs p133 3:6 Cuando estaban en Roma, Ganid notó que Jesús se negaba a acompañarlos a los baños públicos. Luego, en distintas ocasiones, el joven había tratado de persuadirlo para que se manifestara respecto a las relaciones entre los sexos. Aunque Jesús solía contestar a las preguntas del joven, nunca parecía estar dispuesto a hablar de estos temas en profundidad. Un día, a la caída de la tarde, mientras caminaban por Corinto, cerca de donde la muralla de la ciudadela se asomaba al mar, fueron abordados por dos mujeres de vida pública. Ganid había asimilado la idea, con toda razón, de que Jesús era hombre de altos ideales, que aborrecía todo lo que participara de la impureza y se deleitase en el mal; por tanto, habló a las mujeres con dureza y, abruptamente, les hizo señas para que se fueran. Cuando Jesús vio aquello, le dijo a Ganid: “Tienes buenas intenciones, pero no puedes permitirte hablar de ese modo a las hijas de Dios, aunque sean sus hijas descarriadas. ¿Quiénes somos nosotros para juzgar a estas mujeres? ¿Acaso conoces las circunstancias que las empujaron a recurrir a este modo de ganarse la vida? Detengámonos aquí y hablemos de estas cosas”. Las cortesanas estaban más atónitas con las palabras de Jesús que incluso el mismo Ganid.
\vs p133 3:7 Mientras estaban allí de pie a la luz de la luna, Jesús continuó: “En la mente de todo ser humano vive un espíritu divino, dádiva del Padre celestial. Este buen espíritu pugna constantemente para guiarnos hacia Dios, para ayudarnos a encontrar y conocer a Dios; pero en los mortales se dan también muchas tendencias físicas naturales que el Creador colocó ahí para que estuviesen al servicio del bienestar individual y de la raza humana. Ahora bien, a menudo, los hombres y las mujeres se confunden en sus intentos por comprenderse a sí mismos y hacer frente a las múltiples dificultades con las que se encuentran al tener que ganarse la vida en un mundo tan mayoritariamente dominado por el egoísmo y el pecado. He percibido, Ganid, que ninguna de estas dos mujeres lleva una mala vida a propósito. Por sus semblantes me doy cuenta de que han padecido muchos pesares, de que han sufrido mucho a manos de un destino aparentemente cruel, que no han optado intencionalmente por esta clase de vida; en su desaliento que raya en la desesperación, se han rendido a la presión del momento y han aceptado este reprobable medio de subsistir, considerándolo el mejor modo de salir de una situación que les parecía desesperanzadora. Ganid, hay gente realmente malvada de corazón y eligen deliberadamente hacer cosas mezquinas, pero, dime, cuando miras sus rostros llorosos, ¿ves en ellas algo malo o perverso?”. Y, cuando Jesús hizo una pausa para que el joven respondiera, Ganid, con la voz ahogada, balbuceó: “No, Maestro, no lo veo. Y pido disculpas por la rudeza con la que las traté. Les ruego que me perdonen”. Jesús dijo entonces: “Y afirmo en su nombre que te han perdonado tal como les digo a ellas en nombre de mi Padre del cielo que él las ha perdonado. Ahora venid todos conmigo a la casa de un amigo mío; allí comeremos algo y haremos planes para una vida por llegar nueva y mejor”. Hasta este momento las maravilladas mujeres no habían pronunciado una sola palabra; se miraron entre sí y siguieron en silencio a los hombres en su camino.
\vs p133 3:8 Imaginad la sorpresa de la mujer de Justo cuando, a esta avanzada hora, apareció Jesús con Ganid y estas dos desconocidas, diciendo: “Por favor, perdónanos que vengamos a esta hora, pero Ganid y yo tenemos necesidad de algo de comer, y querríamos compartirlo con estas nuevas amigas, que también necesitan alimentarse; y acudimos a ti, además, pensando que te interesará consultar con nosotros sobre el mejor modo de ayudar a estas mujeres a emprender una vida nueva. Pueden contarte su historia, pero deduzco que han pasado por muchas dificultades, y su presencia misma aquí en tu casa da testimonio de cuán encarecidamente anhelan conocer a personas buenas y con cuánto agrado aprovecharán la oportunidad de demostrarle a todo el mundo ---e incluso a los ángeles del cielo--- lo valientes y nobles que pueden llegar a ser”.
\vs p133 3:9 Cuando Marta, la esposa de Justo, había servido la comida en la mesa, Jesús, despidiéndose de ellos de forma inesperada, dijo: “Como se hace tarde y el padre del joven estará esperándonos, os rogamos que nos disculpéis; os dejamos aquí juntas ---las tres mujeres---, las hijas amadas del Altísimo. Y yo oraré por vuestra guía espiritual mientras hacéis planes para una vida nueva y mejor en la tierra y para la vida eterna en el más allá”.
\vs p133 3:10 Así se despidieron de las mujeres Jesús y Ganid. Hasta ese momento, las dos cortesanas no habían dicho nada; Ganid se había quedado también sin palabras. E igual le sucedió a Marta por unos instantes, pero pronto estuvo a la altura de las circunstancias e hizo por estas desconocidas todo lo que Jesús esperaba que se hiciera. La mayor de ellas murió poco tiempo después, con vivas esperanzas en la supervivencia eterna; la más joven trabajó en el lugar de negocios de Justo y, más tarde, se convertiría en miembro de por vida de la primera Iglesia cristiana de Corinto.
\vs p133 3:11 En distintas ocasiones, en casa de Crispo, Jesús y Ganid se habían encontrado con un tal Gayo, que posteriormente se convirtió en un leal seguidor de Pablo. Durante esos dos meses pasados en Corinto, ellos sostuvieron conversaciones privadas con veintenas de personas de valía y, como resultado de estos encuentros aparentemente casuales, más de la mitad de estas personas se sintieron movidas y llegaron a ser miembros de la posterior comunidad cristiana.
\vs p133 3:12 Cuando Pablo fue por primera vez a Corinto, no tenía previsto prolongar su visita por mucho tiempo. Pero no era consciente de lo bien que el Tutor Judío había preparado el camino para sus enseñanzas. Y descubrió además ese gran interés que ya se había suscitado en ellas gracias a Aquila y Priscila, al ser Aquila uno de los cínicos con los que Jesús había entrado en contacto cuando estaba en Roma. Se trataba de una pareja de refugiados judíos llegados de Roma, que abrazaron rápidamente las enseñanzas de Pablo. Él vivió y trabajó con ellos, porque eran también fabricantes de tiendas. Fue debido a esas circunstancias por lo que Pablo dilató su estancia en Corinto.
\usection{4. LABOR PERSONAL EN CORINTO}
\vs p133 4:1 Jesús y Ganid tuvieron otras muchas interesantes experiencias en Corinto. Conversaron privadamente con un gran número de personas que se beneficiaron enormemente de las enseñanzas de Jesús.
\vs p133 4:2 \pc Enseñó al molinero cómo moler el grano de la verdad en el molino de la experiencia viva para hacer que las cosas difíciles de la vida divina fuesen prontamente aceptables incluso entre los más débiles y frágiles de sus semejantes mortales. Jesús dijo: “Da la leche de la verdad a los que son niños en percepción espiritual. En tu ministerio vivo y amoroso, sirve el alimento espiritual de manera atractiva y adaptada a la capacidad receptiva de todo aquel que te lo solicite”.
\vs p133 4:3 \pc Al centurión romano le dijo: “Da al césar lo que es del césar y a Dios lo que es de Dios. El servicio sincero de Dios y el servicio leal del césar no están reñidos a menos que el césar pretenda abrogarse ese honor que solo la Deidad puede requerir. La lealtad a Dios, si llegaras a conocerlo, te haría todavía más leal y fiel en tu entrega a un emperador digno”.
\vs p133 4:4 \pc Al honesto líder del culto mitraico le dijo: “Haces bien en buscar una religión de salvación eterna, pero te equivocas al ir en busca de esa gloriosa verdad entre los misterios creados por el hombre y las filosofías humanas. ¿Acaso no sabes que el misterio de la salvación eterna habita en tu propia alma? ¿Acaso no sabes que el Dios del cielo ha enviado su espíritu para que viva en ti, y que este espíritu guiará a todos los mortales, amantes de la verdad y servidores de Dios, más allá de esta vida y a través de los portales de la muerte hasta las alturas eternas de la luz donde Dios aguarda para recibir a sus hijos? Y nunca te olvides: vosotros los que conocéis a Dios sois los hijos de Dios si verdaderamente anheláis ser como él”.
\vs p133 4:5 \pc Al maestro epicúreo le dijo: “Haces bien en escoger lo mejor y en apreciar lo bueno, pero ¿eres sabio cuando dejas de percibir las grandes cosas de la vida mortal que se manifiestan en los ámbitos espirituales, procedentes de la conciencia de la presencia de Dios en el corazón humano? En toda experiencia humana, lo mejor de todo es la conciencia de conocer a Dios, cuyo espíritu vive en ti y trata de guiarte en ese largo y casi interminable viaje para alcanzar la presencia personal de nuestro Padre común, Dios de toda la creación, el Señor de los universos”.
\vs p133 4:6 \pc Al contratista y constructor griego le dijo: “Amigo mío, mientras edificas las construcciones materiales de los hombres, desarrolla un carácter espiritual a semejanza del espíritu divino que reside en tu alma. No dejes que tus éxitos como constructor temporal superen tu logro como hijo espiritual del reino del cielo. Al edificar las mansiones temporales para los demás, no te olvides de asegurar para ti mismo el derecho como titular a las mansiones de la eternidad. Recuerda siempre: existe una ciudad cuyos fundamentos son la rectitud y la verdad, y cuyo constructor y hacedor es Dios”.
\vs p133 4:7 \pc Al juez romano le dijo: “Cuando juzgues a los hombres, recuerda que algún día tú mismo serás también juzgado ante el tribunal de los gobernantes de un universo. Juzga con justicia, al igual que con misericordia, de la misma forma que algún día desearás con anhelo la consideración misericordiosa de las manos del Arbitro Supremo. Juzga como querrías que te juzgaran en similares circunstancias, siendo pues guiado por el espíritu de la ley al igual que por su letra. Y tal como impartes justicia con ecuanimidad, teniendo en cuenta presente las necesidades de los que ante ti comparecen, así tendrás derecho a aspirar a una justicia atemperada por la misericordia cuando en algún momento te presentes ante el Juez de toda la tierra”.
\vs p133 4:8 \pc A la dueña de la posada griega le dijo: “Da tu hospitalidad como quien da albergue a los hijos del Altísimo. Eleva el arduo trabajo de tu quehacer diario a los altos niveles de un bello arte mediante el reconocimiento creciente de que sirves a Dios en las personas en las que él habita a través de su espíritu que ha descendido para morar en el corazón de los hombres, buscando, de este modo, transformar sus mentes y guiar sus almas al conocimiento del Padre paradisíaco de todos estos dones que el espíritu divino otorga”.
\vs p133 4:9 \pc Jesús mantuvo numerosas conversaciones con un mercader chino. Al despedirse de él, le exhortó diciendo: “Adora solo a Dios, que es tu verdadero ancestro espiritual. Recuerda que el espíritu del Padre vive en ti en todo momento y siempre orienta tu alma hacia el cielo. Si sigues la guía imperceptible de este espíritu inmortal, de cierto continuarás en el camino que te elevará hasta encontrar a Dios. Y cuando logres alcanzar al Padre celestial, será porque al buscarlo te habrás hecho cada vez más como él. Así pues, adiós, Chang, pero solo por un tiempo, porque nos encontraremos de nuevo en los mundos de luz, donde el Padre de las almas espirituales brinda muchos placenteros lugares de parada para aquellos cuyo destino es el Paraíso”.
\vs p133 4:10 \pc Al viajero de Bretaña le dijo: “Hermano mío, he percibido que buscas la verdad, y te indico que el espíritu del Padre de toda verdad quizás habita en ti. ¿Has tratado alguna vez de hablar, con franqueza, con el espíritu de tu propia alma? Es algo ciertamente difícil y rara vez se tiene conciencia de haberlo logrado; pero todo intento honesto de la mente material por comunicarse con su espíritu morador tiene cierto éxito, a pesar de que la mayoría de todas estas gloriosas experiencias humanas han de permanecer por largo tiempo inscritas de forma supraconsciente en el alma de esos mortales conocedores de Dios”.
\vs p133 4:11 \pc Al joven huido Jesús le dijo: “Recuerda, hay dos cosas de las que no puedes escapar: de Dios y de ti mismo. Dondequiera que vayas, contigo va tu yo y el espíritu del Padre celestial que vive dentro de tu corazón. Hijo mío, deja de intentar engañarte a ti mismo; toma medidas para enfrentarte con valor a las circunstancias de la vida; mantente firme en la seguridad de tu filiación con Dios y en la certeza de la vida eterna, como te he enseñado. De ahora en adelante, ten el propósito de ser un hombre verdadero, un hombre resuelto a hacer frente a la vida con valentía e inteligencia”.
\vs p133 4:12 \pc Al criminal condenado le dijo en su última hora: “Hermano mío, son tiempos aciagos para ti. Perdiste el rumbo; te has enmarañado en las redes de la delincuencia. Por lo que he hablado contigo, sé bien que no tenías la intención de hacer lo que ahora va a costarte la vida temporal. Pero, efectivamente, cometiste este mal, y tus semejantes te han encontrado culpable; y han decidido que debes morir. Ni tú ni yo podemos negarle al Estado este derecho a defenderse a sí mismo en el modo que ha elegido. No parece haber forma alguna de que humanamente eludas al castigo por tu delito. Tus semejantes deben juzgarte por lo que has hecho, pero hay un Juez a quien puedes solicitar el perdón, y quien te juzgará por tus verdaderos motivos y tus mejores intenciones. No temas acudir al juicio de Dios si tu arrepentimiento es auténtico y tu fe sincera. El hecho de que tu error traiga consigo la pena de muerte impuesta por el hombre no menoscaba la oportunidad que tiene tu alma de obtener justicia y gozar de la misericordia ante los tribunales celestiales”.
\vs p133 4:13 \pc Jesús disfrutó de muchas charlas privadas con un gran número de almas hambrientas, demasiadas para poder incluirlas en este documento. Los tres viajeros gozaron de su estancia en Corinto. Con excepción de Atenas, que era más conocida como centro educativo, Corinto era la ciudad más importante de Grecia en esta época romana, y su permanencia durante dos meses en este floreciente centro comercial les ofreció a los tres la oportunidad de adquirir una experiencia muy valiosa. La estancia en esta ciudad constituyó una de las paradas más interesantes en su camino de vuelta de Roma.
\vs p133 4:14 Gonod tenía muchos intereses en Corinto, pero finalmente concluyó sus negocios, y se dispusieron a zarpar para Atenas. Viajaron en un pequeño barco que podía transportarse por tierra sobre un carril desde uno de los puertos de Corinto al otro, a una distancia de unos dieciséis kilómetros.
\usection{5. EN ATENAS: PLÁTICA SOBRE LA CIENCIA}
\vs p133 5:1 Pronto llegaron al antiguo centro de la ciencia y del saber griegos, y Ganid estaba emocionado de saberse en Atenas, en Grecia, en el centro cultural de lo que en otro tiempo fue el imperio de Alejandro, que había extendido sus fronteras hasta incluso la India, su propio país. Allí había pocos negocios que realizar; así que Gonod pasó la mayor parte de su tiempo con Jesús y Ganid, visitando los muchos sitios de interés y escuchando las interesantes conversaciones entre el muchacho y su versátil maestro.
\vs p133 5:2 En Atenas aún florecía una gran universidad, y los tres hicieron frecuentes visitas a sus salas de clase. Jesús y Ganid habían analizado a fondo las enseñanzas de Platón cuando asistieron a las conferencias en el museo de Alejandría. A todos les encantaba el arte de Grecia, ejemplos del cual todavía se podían encontrar por doquier en la ciudad.
\vs p133 5:3 Padre e hijo disfrutaron enormemente del debate sobre la ciencia que Jesús mantuvo con un filósofo griego en la posada a últimas horas de la tarde. Una vez que aquel pedante había hablado durante casi tres horas, y cuando había terminado su plática, Jesús, en términos modernos, dijo:
\vs p133 5:4 \pc Algún día, los científicos podrán medir la energía, o manifestaciones de fuerza, de la gravedad, de la luz y de la electricidad, pero estos mismos científicos nunca podrán (científicamente) decir qué \bibemph{son} estos fenómenos del universo. La ciencia trata de la actividad de la energía física; la religión, de los valores eternos. La verdadera filosofía surge de la sabiduría, que correlaciona, de la mejor manera, estas observaciones cuantitativas y cualitativas. Siempre existe el peligro de que el científico puramente físico llegue a dejarse influenciar por el placer del orgullo matemático y el egocentrismo estadístico, por no mencionar la ceguera espiritual.
\vs p133 5:5 La lógica es válida en el mundo material, y las matemáticas son fiables cuando se limita su aplicación a las cosas físicas; pero ninguna de las dos puede llegar a considerarse totalmente dignas de confianza o infalibles cuando se aplican a los problemas de la vida. En la vida se dan fenómenos que no son enteramente materiales. La aritmética dice que si un hombre puede esquilar una oveja en diez minutos, diez hombres la podrían esquilar en un minuto. Se trata de una verdad matemática, pero no es exacta, porque los diez hombres no podrían hacerlo de esta manera; se entorpecerían unos a otros hasta tal punto que el trabajo se demoraría excesivamente.
\vs p133 5:6 Las matemáticas sostienen que, si una persona representa una determinada unidad de valor intelectual y moral, diez personas representarían diez veces este valor. Pero cuando se trata de seres personales humanos, sería más acertado decir que tal agrupación de personas constituye una suma igual al cuadrado del número de personas referidas en la ecuación, en lugar de la simple suma aritmética. Un grupo social de seres humanos trabajando armónicamente y de forma coordinada conlleva una fuerza mucho mayor que la simple suma de sus partes.
\vs p133 5:7 La cantidad se puede identificar como un \bibemph{hecho,} convirtiéndose de este modo en una uniformidad científica. La cualidad, siendo una cuestión de interpretación de la mente, constituye una estimación aproximada de los \bibemph{valores} y debe, por consiguiente, permanecer como una experiencia de la persona. Cuando la ciencia y la religión se tornen menos dogmáticas y más tolerantes frente a la crítica, la filosofía comenzará entonces a lograr la \bibemph{unidad} en la comprensión inteligente del universo.
\vs p133 5:8 Si pudierais percibir su genuino funcionamiento, observaríais que en el universo cósmico existe unidad. El universo real es amigable para todos los hijos del Dios eterno. El auténtico problema es: ¿Cómo puede la mente finita del hombre alcanzar una unidad de pensamiento lógica, verdadera y en consonancia? Este estado mental de conocimiento del universo solo puede lograrse si se concibe la idea de que el hecho cuantitativo y el valor cualitativo tienen una causalidad común en el Padre del Paraíso. Esta concepción de la realidad proporciona una percepción más amplia de la unidad de propósito de los fenómenos del universo; e incluso revela una meta espiritual de logro progresivo para el ser personal, y constituye un concepto de unidad capaz de apreciar los antecedentes invariables de un universo vivo de relaciones impersonales en continuo cambio y de relaciones personales en constante desarrollo.
\vs p133 5:9 La materia y el espíritu y el estado intermedio entre ellos constituyen tres niveles, interrelacionados e interconectados, de la verdadera unidad del universo real. Al margen de cuán divergentes parezcan ser los fenómenos del universo de los hechos y de los valores, estos están, finalmente, unificados en el Supremo.
\vs p133 5:10 Reconocemos que la materia visible tiene existencia real, pero también la tiene la energía no reconocida. Cuando las energías del universo se ralentizan tanto que adquieren el grado necesario de movimiento, entonces, en condiciones favorables, estas mismas energías se convierten en masa. Y no olvidéis que la mente, que es la única que puede percibir la presencia de las realidades fehacientes, es en sí misma igualmente real. Y la causa fundamental de este universo de energía\hyp{}masa, mente y espíritu es eterna ---existe y consiste en la naturaleza y en la reacción del Padre Universal y sus coiguales absolutos---.
\vs p133 5:11 \pc Quedaron todos más que maravillados por las palabras de Jesús, y cuando el griego se despidió de ellos les dijo: “Por fin han visto mis ojos a un judío que piensa en algo más que en la superioridad racial y que habla de algo más que de religión”. Y se retiraron para pasar la noche.
\vs p133 5:12 La estancia en Atenas fue placentera y provechosa, pero no especialmente fructífera en cuanto a contactos humanos. Demasiados atenienses de ese tiempo estaban intelectualmente orgullosos de su notoriedad de otros días o eran mentalmente faltos de inteligencia e ignorantes, al ser los vástagos de esclavos pobremente dotados de esas épocas anteriores cuando existía gloria en Grecia y sabiduría en las mentes de su gente. No obstante, entre los ciudadanos de Atenas, aún era posible encontrar muchas mentes perspicaces.
\usection{6. EN ÉFESO: PLÁTICA SOBRE EL ALMA}
\vs p133 6:1 Al dejar Atenas, los viajeros se dirigieron por el camino de Troas hacia Éfeso, la capital de la provincia romana de Asia. Desde allí, se desplazaron muchas veces al famoso templo de Artemisa de los efesios, a más de tres kilómetros de la ciudad. Artemisa era la diosa más famosa de toda Asia Menor, perpetuación de la diosa madre aún más antigua de los tiempos de la ancestral Anatolia. Se decía que aquel tosco ídolo que se exhibía en el enorme templo dedicado a su culto había caído del cielo. Al no haber erradicado toda su anterior formación en cuanto al respeto de las imágenes como símbolos de la divinidad, Ganid pensó que sería lo mejor comprar un pequeño santuario de plata en honor de esta diosa de la fertilidad del Asia Menor. Aquella noche hablaron largamente sobre la adoración de las cosas hechas por manos humanas.
\vs p133 6:2 Al tercer día de estancia, caminaron junto al río para observar el dragado de la boca del puerto. Al mediodía charlaron con un joven fenicio que estaba nostálgico por su tierra natal y bastante desanimado; pero sobre todo estaba envidioso de otro joven a quien habían ascendido en su lugar. Jesús le dirigió palabras de consuelo y citó el antiguo proverbio hebreo: “El don del hombre le ensancha el camino y lo lleva delante de los grandes”.
\vs p133 6:3 De todas las grandes ciudades que visitaron en su viaje por el Mediterráneo, fue en este lugar donde menos pudieron hacer para favorecer la labor de los futuros misioneros cristianos. El cristianismo tuvo sus comienzos en Éfeso mayormente gracias a las iniciativas de Pablo, que residió aquí durante más de dos años, ganándose la vida con la fabricación de tiendas y dando conferencias todas las noches sobre religión y filosofía en el auditorio principal de la escuela de Tirano.
\vs p133 6:4 Había un escritor progresista vinculado a esta escuela local de filosofía, con quien Jesús tuvo varias provechosas reuniones. En el transcurso de las charlas que mantuvo con él, Jesús usó repetidas veces la palabra “alma”. Finalmente, este griego erudito le preguntó qué quería decir con “alma”. Y Jesús le respondió:
\vs p133 6:5 \pc “El alma es esa parte del hombre autorreflexiva, discernidora de la verdad y perceptiva del espíritu que eleva para siempre al ser humano por encima del nivel del mundo animal. La autoconciencia, por sí misma, no es el alma. La autoconciencia moral es la verdadera autorrealización humana y constituye el fundamento del alma humana, y el alma es esa parte del hombre que representa el valor de la supervivencia potencial de la experiencia humana. La elección moral y el logro espiritual, la capacidad para conocer a Dios y el impulso de ser como él, son atributos del alma. El alma del hombre no puede existir al margen del pensamiento moral y de la actividad espiritual. Un alma estancada es un alma agonizante. Pero el alma del hombre es diferente del espíritu divino que habita en su mente. El espíritu divino llega de forma simultánea al primer acto moral de la mente humana, y ese es el momento en el que nace el alma.
\vs p133 6:6 “La salvación o la pérdida de un alma guarda relación con el hecho de si la conciencia moral logra alcanzar o no el estado de supervivencia mediante la alianza eterna con su correspondiente don espiritual. La salvación es la espiritualización de la autorrealización de la conciencia moral, la cual, por consiguiente, adquiere valor de supervivencia. Todas las formas de conflictos del alma radican en la falta de armonía entre la autoconciencia moral, o espiritual, y la autoconciencia puramente intelectual.
\vs p133 6:7 “Cuando el alma humana madura, ennoblecida y espiritualizada, se aproxima al estado celestial en el sentido de que está cercana de ser una entidad intermedia entre lo material y lo espiritual, entre el yo material y el espíritu divino. Resulta difícil describir el alma evolutiva de un ser humano y mucho más difícil demostrarla, porque no es observable ni por los métodos de investigación material ni de verificación espiritual. La ciencia material no puede demostrar la existencia de un alma como tampoco puede hacerlo una evaluación puramente espiritual. Sin embargo, a pesar de la insuficiencia de la ciencia material y de los criterios espirituales para descubrir la existencia del alma humana, todo mortal moralmente consciente \bibemph{conoce} la existencia de \bibemph{su} alma como una experiencia personal real y auténtica”.
\usection{7. LA ESTANCIA EN CHIPRE: PLÁTICA SOBRE LA MENTE}
\vs p133 7:1 Poco después, los viajeros zarparon para Chipre, con parada en Rodas. Disfrutaron de la larga travesía marítima y llegaron a la isla de destino muy descansados de cuerpo y renovados de espíritu.
\vs p133 7:2 Su plan era sentir el placer de un período de auténtico descanso y ocio durante su visita a Chipre, cuando su recorrido por el Mediterráneo tocaba su fin. Desembarcaron en Pafos y, de inmediato, comenzaron a acumular provisiones para su estancia de varias semanas en las montañas cercanas. Tres días después de su llegada, partieron hacia las colinas con sus animales bien cargados.
\vs p133 7:3 Durante dos semanas, los tres viajeros disfrutaron bastante, pero entonces, sin previo aviso, el joven Ganid cayó de repente gravemente enfermo. Estuvo dos semanas sufriendo una virulenta fiebre que a menudo le hacía entrar en delirio; tanto Jesús como Gonod se ocuparon de lleno en atenderle. Jesús cuidó del joven con habilidad y ternura, ante un padre asombrado por la delicadeza y la destreza que Jesús demostraba en todas sus atenciones hacia el doliente joven. Estaban alejados de cualquier presencia humana, y el muchacho se encontraba demasiado enfermo como para ser trasladado; se dispusieron, pues, allí mismo en las montañas, a hacer todo lo que les fuera posible por cuidarle y devolverle la salud.
\vs p133 7:4 Durante las tres semanas de la convalecencia de Ganid, Jesús le contó muchas cosas interesantes sobre la naturaleza y sus diversos estados. Y cuánto se divirtieron paseando por las montañas, mientras que el muchacho hacía preguntas, Jesús se las contestaba, y el padre se maravillaba ante toda aquella escena.
\vs p133 7:5 La última semana de su estancia en las montañas, Jesús y Ganid tuvieron una larga charla sobre las funciones de la mente humana. Tras varias horas de análisis del tema, el muchacho le hizo la siguiente pregunta: “Pero, Maestro, ¿qué quieres decir cuando afirmas que el hombre experimenta una forma más elevada de autoconciencia que la de los animales superiores?”. Y reformulado en términos modernos, Jesús le respondió así:
\vs p133 7:6 \pc Hijo mío, ya te he hablado bastante de la mente del hombre y del espíritu divino que vive en ella, pero ahora quisiera hacer hincapié en el hecho de que la autoconciencia es una \bibemph{realidad}. Cuando un animal llega a ser consciente de sí mismo, se convierte en un hombre primitivo. Tal logro resulta de la coordinación de funciones entre la energía impersonal y la mente perceptiva del espíritu, y es este fenómeno el que asegura la dádiva al ser personal humano de un punto central absoluto: el espíritu del Padre de los cielos.
\vs p133 7:7 Las ideas no son simplemente una transcripción de las sensaciones. Las ideas son sensaciones más las interpretaciones reflexivas del yo personal; y el yo es más que la suma de nuestras sensaciones. En un yo en evolución comienza a existir algo así como un camino hacia la unidad, y esa unidad es fruto de la presencia interior de una parte de la unidad absoluta que activa, espiritualmente, a esa mente de origen animal consciente de sí misma.
\vs p133 7:8 Un mero animal no puede poseer autoconciencia del tiempo. Los animales poseen una coordinación fisiológica de la conjunción sensación\hyp{}reconocimiento y su consiguiente memoria, pero ningún animal experimenta un reconocimiento significativo de las sensaciones ni evidencia una conexión intencionada de estas experiencias físicas combinadas, tal como se manifiesta en las conclusiones que se generan a partir de interpretaciones humanas inteligentes y reflexivas. Y este hecho de una existencia autoconsciente, unido a la realidad de su consecuente experiencia espiritual, constituye al hombre como hijo potencial del universo y prefigura su ulterior logro de la Suprema Unidad del universo.
\vs p133 7:9 Tampoco es el yo humano la mera suma de sucesivos estados de conciencia. Sin la eficiente actuación de un organizador y asociador de la conciencia, no existiría la unidad suficiente para poder asegurar la conformación de un yo. Dicha mente, no unificada, difícilmente podría lograr los niveles de conciencia característicos de la condición humana. Si las asociaciones que se dan en la conciencia fuesen aleatorias, la mente de todos los hombres mostraría asociaciones incontroladas y fortuitas propias de determinadas facetas de la locura mental.
\vs p133 7:10 Una mente humana, forjada solamente sobre la base de la conciencia de las sensaciones físicas, nunca podría alcanzar niveles espirituales; esta clase de mente material carecería totalmente del sentido de los valores morales y le faltaría la guía rectora del espíritu, que es tan esencial para lograr una unidad armoniosa del ser personal en el tiempo, y que es inseparable de la supervivencia del ser personal en la eternidad.
\vs p133 7:11 La mente humana comienza pronto a manifestar cualidades que son supramateriales; el intelecto humano verdaderamente reflexivo no está atado del todo a las restricciones del tiempo. El hecho de que las personas difieran tanto en cuanto a su comportamiento en la vida indica no solo la diversidad de dotes hereditarias y las distintas influencias del entorno, sino también el grado de unificación con el espíritu morador del Padre alcanzado por el yo, esto es, la magnitud en la que se identifican el uno con el otro.
\vs p133 7:12 La mente humana no resiste bien el conflicto que conlleva experimentar una doble lealtad. Constituye un gran peso para el alma tratar de servir tanto al bien como al mal. La mente supremamente satisfecha y eficientemente unificada es aquella que está dedicada totalmente a hacer la voluntad del Padre de los cielos. Los conflictos no resueltos destruyen la unidad y pueden terminar en la perturbación de la mente. Pero el carácter de supervivencia del alma no se incentiva intentando asegurar la paz mental a cualquier precio, renunciando a nobles aspiraciones o poniendo en peligro los ideales espirituales; dicha paz se consigue más bien reafirmándose incondicionalmente en el triunfo de lo que es verdadero, y esta victoria se logra venciendo al mal con la poderosa fuerza del bien.
\vs p133 7:13 \pc Al siguiente día partieron hacia Salamina, desde donde embarcaron rumbo a Antioquía, en la costa de Siria.
\usection{8. EN ANTIOQUÍA}
\vs p133 8:1 Antioquía era la capital de la provincia romana de Siria, y aquí tenía su residencia el gobernador imperial. Antioquía tenía medio millón de habitantes; era la tercera ciudad del imperio en tamaño y la primera en iniquidad y flagrante inmoralidad. Gonod tenía bastantes asuntos de negocio que tratar, de manera que Jesús y Ganid estuvieron por sí mismos bastante tiempo. Vieron todo lo visitable de esta ciudad de múltiples culturas, salvo el bosque de Dafne. Gonod y Ganid sí visitaron este notorio templo de la vergüenza, pero Jesús no quiso acompañarlos. Estas escenas no resultaban tan impactantes para los indios, pero eran repelentes para un hebreo idealista.
\vs p133 8:2 A medida que se aproximaba a Palestina y el viaje tocaba a su fin, Jesús se iba poniendo más serio y pensativo. En Antioquía conversó con pocas personas; rara vez transitaba por la ciudad. Tras mucho preguntarle a su maestro por qué manifestaba tan poco interés en Antioquía, Ganid logró que Jesús finalmente le dijera: “Esta ciudad no está lejos de Palestina; tal vez regrese aquí algún día”.
\vs p133 8:3 \pc Ganid tuvo una experiencia bastante interesante en Antioquía. Este joven había demostrado ser un buen pupilo y ya había comenzado a hacer un uso práctico de algunas de las enseñanzas de Jesús. Había cierto indio relacionado con los negocios de su padre en Antioquía que se había vuelto tan desagradable y malhumorado que se estaban planteando despedirle. Cuando Ganid lo supo, se dirigió al establecimiento de su padre y tuvo una larga reunión con su compatriota. Este hombre creía que le habían puesto en el trabajo equivocado. Ganid le habló del Padre de los cielos y amplió, de muchas formas, sus puntos de vista sobre la religión. Pero de todo lo que Ganid le dijo, la cita de un proverbio hebreo fue lo que mayor bien le hizo; y esas sabias palabras fueron: “Todo lo que te venga a mano para hacer, hazlo según tus fuerzas”.
\vs p133 8:4 Tras preparar su equipaje para la caravana de camellos, bajaron a Sidón y, desde allí, se dirigieron a Damasco; tres días después estaban listos para la larga travesía por las arenas del desierto.
\usection{9. EN MESOPOTAMIA}
\vs p133 9:1 El trayecto en caravana por el desierto no era una nueva experiencia para aquellos hombres tan experimentados en viajes. Al notar Ganid cómo su maestro ayudaba a cargar sus veinte camellos y, al observar que se había ofrecido voluntariamente a conducir el propio animal de ambos, exclamó: “Maestro, ¿hay algo que no sepas hacer?”. Jesús solamente sonrió, diciendo: “Ciertamente no hay maestro sin mérito a los ojos de un pupilo diligente”. Y entonces se pusieron en camino para la antigua ciudad de Ur.
\vs p133 9:2 Jesús estaba muy interesado en la temprana historia de Ur, lugar de nacimiento de Abraham, y estaba igualmente fascinado con las ruinas y tradiciones de Susa, tanto era así que Gonod y Ganid prolongaron tres semanas su estancia en estos lugares para que tuviera más tiempo de hacer sus pesquisas y, al mismo tiempo, aprovechar la ocasión para convencerle de que se fuese con ellos cuando volvieran a la India.
\vs p133 9:3 Fue en Ur donde Ganid mantuvo una larga charla con Jesús respecto a la diferencia entre conocimiento, sabiduría y verdad. Y quedó encantando con el dicho del sabio hebreo: “Sabiduría ante todo; adquiere pues sabiduría. En tu búsqueda del conocimiento, adquiere entendimiento. Exalta la sabiduría y ella te engrandecerá. Te honrará, si tú la abrazas”.
\vs p133 9:4 \pc El día de la separación acabó por llegar. Todos mantuvieron su fortaleza de ánimo, especialmente el joven, pero aquello era una dura prueba. Tenían todos los ojos llorosos, pero había entereza en sus corazones. Al despedirse de su maestro, Ganid le dijo: “Adiós, Maestro, aunque no para siempre. Cuando vuelva a Damasco, iré a buscarte. Te amo, porque pienso que el Padre de los cielos debe ser alguien como tú; al menos yo sé que eres muy parecido a lo que me has contado de él. Recordaré tus enseñanzas, pero, sobre todo, jamás me olvidaré de ti”. El padre dijo: “Adiós a un gran maestro, a alguien que nos ha hecho mejores y que nos ha ayudado a conocer a Dios”. Y Jesús les respondió: “La paz esté con vosotros, y que la bendición del Padre de los cielos esté siempre en vosotros”. Y Jesús permaneció en la orilla, contemplando cómo la pequeña barca los llevaba hasta el barco anclado. Así fue como el maestro se separó de sus amigos de la India en Charax, para no volver a verlos en este mundo; ni ellos sabrían jamás, en este mundo, que el hombre que aparecería después como Jesús de Nazaret era este mismo amigo del que acababan de despedirse: Josué su maestro.
\vs p133 9:5 En la India, Ganid creció hasta convertirse en un hombre influyente, un digno sucesor de su eminente padre, llegando a difundir en el extranjero muchas de las nobles verdades que había aprendido de Jesús, su amado maestro. Más adelante en su vida, cuando oyó hablar del extraño maestro de Palestina que había terminado su andadura en una cruz, y aunque reconoció la semejanza entre el evangelio de este Hijo del Hombre y las enseñanzas de su tutor judío, jamás se imaginó que aquellos dos hombres fuesen en realidad la misma persona.
\vs p133 9:6 \pc Así concluyó ese episodio de la vida del Hijo del Hombre que podría denominarse: \bibemph{La misión de Josué el Maestro}.
