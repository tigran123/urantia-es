\upaper{59}{La era de la vida marina en Urantia}
\author{Portador de vida}
\vs p059 0:1 Según nuestros cálculos, la historia de Urantia comenzó hace unos mil millones de años y se prolongó durante las principales cinco eras siguientes:
\vs p059 0:2 \li{1.}\bibemph{La era previa a la vida} engloba los primeros cuatrocientos cincuenta millones de años, desde alrededor del momento en el que el planeta alcanzó su tamaño actual hasta el momento de la implantación de la vida. Vuestros estudiosos lo llaman periodo \bibemph{Arqueozoico}.
\vs p059 0:3 \li{2.}\bibemph{La era de los albores de la vida} engloba los siguientes ciento cincuenta millones de años. Esta época media entre la era previa a la vida anterior, o era de cataclismos, y el siguiente período de vida marina altamente desarrollada. Vuestros investigadores la conocen como era \bibemph{Proterozoica}.
\vs p059 0:4 \li{3.}\bibemph{La era de la vida marina} abarca los siguientes doscientos cincuenta millones de años y se conoce mejor con el nombre de \bibemph{Paleozoica}.
\vs p059 0:5 \li{4.}\bibemph{La era de la vida terrestre primitiva} engloba los siguientes cien millones de años y se la conoce como era \bibemph{Mesozoica}.
\vs p059 0:6 \li{5.}\bibemph{La era de los mamíferos} abarca los últimos cincuenta millones de años. Se conoce a esta era de tiempos recientes como \bibemph{Cenozoica}.
\vs p059 0:7 \pc Así pues, la era de la vida marina abarca alrededor de una cuarta parte de la historia de vuestro planeta. Se puede subdividir en seis largos períodos, cada cual caracterizado por ciertos desarrollos, bien definidos, tanto en el terreno geológico como en el biológico.
\vs p059 0:8 Al comienzo de esta era, los fondos marinos, las grandes plataformas continentales y las numerosas cuencas de poca profundidad próximas a la costa están cubiertas de una prolífera vegetación. Las formas más simples y más primitivas de la vida animal ya se han desarrollado a partir de anteriores organismos vegetales, y los primeros organismos animales se han abierto camino, de forma gradual, a lo largo de los extensos litorales de las distintas masas terrestres hasta los múltiples mares interiores repletos de vida marina primitiva. Al tener pocos de estos tempranos organismos conchas, no hay muchos que se hayan preservado como fósiles. No obstante, el escenario está dispuesto para los capítulos iniciales de ese gran “libro de piedra” dedicado a la conservación de los registros de la vida, que tan sistemáticamente se fueron asentando durante las eras sucesivas.
\vs p059 0:9 El continente de América del Norte posee magníficos depósitos de fósiles de toda la era de la vida marina. Sus capas primeras y más antiguas están separadas de los estratos tardíos del período precedente por grandes depósitos de erosión, que claramente separan estas dos etapas del desarrollo planetario.
\usection{1. LA VIDA MARINA PRIMITIVA DE LOS MARES POCO PROFUNDOS. LA ERA DE LOS TRILOBITES}
\vs p059 1:1 Hacia los albores de este período de relativa tranquilidad en la superficie de la Tierra, la vida está confinada en los diferentes mares interiores y en el litoral oceánico; todavía no ha evolucionado ninguna forma de organismo terrestre. Los animales marinos primitivos están bien asentados y listos para el siguiente desarrollo evolutivo. Las amebas, que habían aparecido hacia el cierre del precedente período de transición, son las habituales supervivientes de esta etapa inicial de la vida animal.
\vs p059 1:2 \pc Hace \bibemph{400\,000\,000} de años, la vida marina, tanto la vegetal como la animal, está bastante bien repartida por el mundo entero. El clima mundial se hace ligeramente más cálido y se vuelve más estable. Se produce una inundación general de las costas de los diversos continentes, particularmente de América del Norte y del Sur. Aparecen nuevos océanos, y las masas de agua más antiguas aumentan de forma considerable.
\vs p059 1:3 Por primera vez, la vegetación trepa ahora hacia el suelo terrestre y rápidamente hace grandes avances en su adaptación a un hábitat no marino.
\vs p059 1:4 \bibemph{De repente,} y sin línea ascendente de cambios graduales, hacen su aparición los primeros animales multicelulares. Los trilobites evolucionaron y durante millones de años dominan los mares. Desde la perspectiva de la vida marina, esta es la era de los trilobites.
\vs p059 1:5 En la última fracción de este segmento de tiempo, una gran parte de América del Norte y de Europa emergió del mar. La corteza terrestre se había estabilizado temporalmente; las montañas, o más bien unas altas elevaciones de suelo, se levantaban a lo largo de las costas del Atlántico y del Pacífico, en las Indias Occidentales y en la Europa meridional. Toda la región del Caribe tenía una gran elevación.
\vs p059 1:6 \pc Hace \bibemph{390\,000\,000} de años el suelo continuaba elevado. En sectores de América oriental y occidental y de Europa occidental se pueden hallar los estratos rocosos que quedaron depositados durante estos tiempos, y que constituyen las rocas más antiguas de fósiles trilobites. Había numerosos y largos golfos que, como brazos de mar, se adentraban en las masas de suelo, depositando allí estas rocas portadoras de fósiles.
\vs p059 1:7 En unos pocos millones de años, el Océano Pacífico comenzó a invadir los continentes americanos. El hundimiento del suelo se debió principalmente al ajuste de la corteza terrestre, aunque la expansión lateral de suelo, o desplazamiento continental, fue también un factor a tener en cuenta.
\vs p059 1:8 \pc Hace \bibemph{380\,000\,000} de años, Asia se estaba hundiendo, y todos los otros continentes experimentaban un surgimiento transitorio. Pero, a medida que transcurría esta época, el Océano Atlántico, que acababa de hacer su aparición, hizo importantes avances en todos los litorales cercanos. Los mares Atlántico o Árticos del norte se comunicaban entonces con las aguas meridionales del Golfo. Cuando este mar del sur penetró en la depresión de los Apalaches, sus olas rompían en el este contra montañas tan altas como los Alpes; no obstante, en general, los continentes eran tierras bajas anodinas, totalmente desprovistas de bellezas paisajísticas.
\vs p059 1:9 \pc En estas eras, hay cuatro tipos de depósitos sedimentarios:
\vs p059 1:10 \li{1.}Conglomerados: materias depositadas cerca de los litorales.
\vs p059 1:11 \li{2.}Areniscas: depósitos formados en aguas poco profundas pero en las que las olas resultaron suficientes para impedir que el lodo se asentara.
\vs p059 1:12 \li{3.}Esquistos: depósitos formados en aguas más tranquilas y de mayor profundidad.
\vs p059 1:13 \li{4.}Piedras calizas, que contienen los depósitos de conchas de los trilobites en aguas profundas.
\vs p059 1:14 \pc Los fósiles de trilobites de estos tiempos presentan determinadas homogeneidades fundamentales en conjunción con algunas variaciones bien definidas. Los primeros animales que se desarrollaron a partir de las tres implantaciones de vida originarias eran singulares; los que aparecieron en el hemisferio occidental eran algo diferentes a los del grupo eurasiático y a los del tipo australasiático o antártico australiano.
\vs p059 1:15 \pc Hace \bibemph{370\,000\,000} de años se produjo el gran y casi total sumergimiento de América del Norte y del Sur, seguido por el hundimiento de África y Australia. Solo algunas partes de América del Norte permanecieron por encima de estos mares cámbricos poco profundos. Cinco millones de años después, los mares retrocedieron ante la elevación del suelo terrestre. Todos estos fenómenos de hundimiento y elevación del suelo fueron poco espectaculares; ocurrieron lentamente, a lo largo de millones de años.
\vs p059 1:16 Los estratos con fósiles de trilobites de esta época afloran por doquier en todos los continentes salvo en Asia central. En muchas regiones, estas rocas son horizontales, pero en las montañas están inclinadas y deformadas a causa de la presión y el plegamiento. Y tal presión, en muchos lugares, ha hecho cambiar el carácter original de estos depósitos. La arenisca se ha transformado en cuarzo, el esquisto en pizarra, mientras que la caliza se ha convertido en mármol.
\vs p059 1:17 \pc Hace \bibemph{360\,000\,000} de años, el suelo terrestre seguía ascendiendo. América del Norte y del Sur estaban bien elevadas. Europa occidental y las Islas Británicas estaban emergiendo, exceptuando algunas partes de Gales, que estaban profundamente sumergidas. Durante estas eras, no había grandes capas de hielo. Los supuestos depósitos glaciales que aparecen en relación con estos estratos en Europa, África, China y Australia se deben a glaciares de montaña aislados o al desplazamiento de detritos glaciales de un origen posterior. El clima mundial era oceánico, no era continental. En aquel entonces, los mares del sur eran más cálidos que en la actualidad, y se extendían en dirección norte sobre América del Norte hasta las regiones polares. La corriente del Golfo recorría la parte central de América del Norte y se desviaba hacia el este para bañar y calentar las costas de Groenlandia, convirtiendo a ese continente, que ahora está cubierto de un manto de hielo, en un auténtico paraíso tropical.
\vs p059 1:18 \pc En todo el mundo la vida marina era bastante similar; consistía en algas marinas, organismos unicelulares, esponjas simples, trilobites y otros crustáceos ---camarones, cangrejos y langostas---. Tres mil variedades de braquiópodos aparecieron al final de este período, de las que solo doscientas han sobrevivido. Estos animales componen un tipo de vida temprana que ha llegado hasta los tiempos presentes prácticamente inalterada.
\vs p059 1:19 Pero los trilobites eran las criaturas vivas dominantes. Eran animales sexuados que habían adoptado diversas formas; al ser nadadores deficientes, flotaban perezosamente en el agua o se arrastraban por los fondos marinos, enroscándose en defensa propia como protección contra los ataques de sus enemigos, que aparecerían más tarde. Alcanzaban una longitud que variaba entre cinco y treinta centímetros y evolucionaron en cuatro grupos distintos: carnívoros, herbívoros, omnívoros y “comedores de fango”. La capacidad del último grupo de subsistir principalmente a base de materia inorgánica ---fueron estos los últimos animales multicelulares que lo pudieron hacer--- explica su aumento considerable y su larga supervivencia.
\vs p059 1:20 Este era el marco bio\hyp{}geológico de Urantia al final de aquel largo período de la historia del mundo, que abarcó cincuenta millones de años, y que vuestros geólogos han denominado \bibemph{Cámbrico}.
\usection{2. LA PRIMERA ETAPA DE LAS INUNDACIONES CONTINENTALES. LA ERA DE LOS ANIMALES INVERTEBRADOS}
\vs p059 2:1 Los fenómenos periódicos de elevación y hundimiento del suelo, característicos de estos tiempos, se produjeron de forma gradual y sin espectacularidad; iban acompañados de poca o ninguna actividad volcánica. En todas estas elevaciones y sucesivas depresiones del suelo terrestre, el continente asiático madre no participó del todo en la crónica de los acontecimientos de las otras masas de tierra. Experimentó muchas inundaciones, sumergiéndose primero en una dirección y luego en otra, más particularmente en su crónica temprana, pero no presenta los depósitos rocosos uniformes que se pueden descubrir en los demás continentes. En las eras recientes, Asia ha sido la más estable de todas las masas de tierra.
\vs p059 2:2 \pc Hace \bibemph{350\,000\,000} de años dio comienzo el período de las grandes inundaciones de todos los continentes excepto Asia central. Las masas de tierra se cubrieron repetidamente de agua; solo las regiones costeras altas permanecieron sobre el nivel de estos mares interiores, oscilantes y poco profundos, aunque extendidos. Este periodo se caracterizó por tres inundaciones principales, pero antes de que finalizara, los continentes subieron de nuevo; el suelo emergió un total de quince por ciento más que el nivel existente en la actualidad. La región del Caribe se elevó bastante. Este período no se encuentra bien demarcado en Europa debido a que las fluctuaciones del suelo fueron menores, mientras que la actividad volcánica fue más constante.
\vs p059 2:3 \pc Hace \bibemph{340\,000\,000} de años se produjo otro gran hundimiento del suelo terrestre, excepto en Asia y Australia. Las aguas de los océanos del mundo estaban en general entremezcladas. Esta fue la gran era de la piedra caliza; las algas secretoras de cal cementaron gran parte de su roca.
\vs p059 2:4 Algunos millones de años después, grandes áreas de los continentes americanos y de Europa empezaron a emerger del agua. En el hemisferio occidental solo un brazo del Océano Pacífico permanecía sobre México y sobre las actuales regiones de las Montañas Rocosas, pero, al acercarse el fin de esta época, las costas del Atlántico y del Pacífico comenzaron de nuevo a hundirse.
\vs p059 2:5 \pc Hace \bibemph{330\,000\,000} de años se señala el comienzo de un segmento temporal de relativa calma en todo el mundo, con gran parte del suelo terrestre nuevamente sobre el nivel del agua. La única excepción a este entorno de tranquilidad terrestre fue la erupción del gran volcán norteamericano de Kentucky oriental, una de las actividades volcánicas más grandes de carácter aislado jamás antes conocida en el mundo. Las cenizas de este volcán cubrieron casi mil trescientos kilómetros cuadrados con una profundidad de entre cinco y seis metros.
\vs p059 2:6 \pc Hace \bibemph{320\,000\,000} de años se produjo la tercera gran inundación de este período. Las aguas cubrieron todo el suelo terrestre que el diluvio anterior había sumergido, extendiéndose a mayor distancia en muchas direcciones por las Américas y Europa. América del Norte oriental y Europa occidental estaban entre 3000 y 4500 metros bajo el agua.
\vs p059 2:7 \pc Hace \bibemph{310\,000\,000} de años, las masas de tierra del mundo nuevamente estaban bien elevadas, exceptuando las partes meridionales de América del Norte. México emergió, creando así el Mar del Golfo, que desde entonces ha mantenido sus rasgos propios.
\vs p059 2:8 La vida de este período continúa su evolución. El mundo está de nuevo en calma y relativamente tranquilo; el clima sigue templado y estable; las plantas terrestres emigran cada vez más lejos de las costas. Los modelos de vida están bien desarrollados, aunque son pocos los fósiles vegetales de estos tiempos factibles de encontrar.
\vs p059 2:9 \pc Esta fue la gran era de la evolución de los organismos animales individuales, aunque muchos de los cambios fundamentales, como la transición de planta a animal, se habían producido con anterioridad. La fauna marina se desarrolló hasta el punto de que cualquier tipo de vida, por debajo de la escala de los vertebrados, estaba representado en los fósiles de las rocas que se depositaron durante estos tiempos. Pero todos estos animales eran organismos marinos. Ningún animal terrestre había aparecido todavía, exceptuando algunos tipos de lombrices que cavaban a lo largo de la costa, ni tampoco las plantas se habían extendido por los continentes; seguía habiendo aún demasiado dióxido de carbono en la atmósfera para permitir la existencia de los respiradores de aire. Mayormente, todos los animales, salvo algunos de los más primitivos, dependen directa o indirectamente para su existencia de la vida vegetal.
\vs p059 2:10 Los trilobites continuaban con su predominio. Había decenas de miles de especies de estos pequeños animales, predecesores de los crustáceos modernos. Algunos trilobites tenían entre veinticinco y cuatro mil minúsculos ojos; otros tenían ojos sin desarrollar. Al finalizar este período, los trilobites compartían el dominio de los mares con algunas otras formas de vida invertebrada, pero perecieron por completo al comienzo del siguiente período.
\vs p059 2:11 Las algas secretoras de cal estaban muy extendidas. Existían miles de especies de los ancestros tempranos de los corales. Abundaban los gusanos de mar, y había muchas variedades de medusas que desde entonces están extintas. Los corales y los tipos más recientes de esponjas evolucionaron. Los cefalópodos estaban bien desarrollados, y han sobrevivido en los modernos nautilos perlados, pulpos, sepias y calamares.
\vs p059 2:12 Había muchas variedades de animales con concha, pero entonces no les eran estas tan necesarias como defensa como en eras posteriores. Había gasterópodos en las aguas de los mares ancestrales, incluyendo horadadores de concha simple, bígaros y caracoles. Los gasterópodos bivalvos han llegado a nuestros días, tras los millones de años transcurridos, tal como existían entonces; se incluyen en ellos a los mejillones, a las almejas, a las ostras y a las vieiras. También tuvieron su evolución los organismos de conchas compuestas de valvas, y estos braquiópodos vivían en aquellas aguas ancestrales de forma muy similar a la que existen hoy día; sus valvas disponían de pliegues, ranuras y de otros tipos de mecanismos de protección.
\vs p059 2:13 \pc Así finaliza la historia evolutiva del segundo gran período de vida marina, que vuestros geólogos conocen como el \bibemph{Ordovícico}.
\usection{3. LA SEGUNDA ETAPA DE LAS GRANDES INUNDACIONES. EL PERÍODO DEL CORAL: LA ERA DE LOS BRAQUIÓPODOS}
\vs p059 3:1 Hace \bibemph{300\,000\,000} de años comenzó otro gran período en el sumergimiento del suelo terrestre. La invasión de los ancestrales mares silúricos en dirección al sur y al norte hizo que la mayor parte de Europa y América del Norte quedara inundada. El suelo no estaba muy elevado sobre el nivel del mar, por lo que no se produjo mucha sedimentación en los litorales. Los mares estaban repletos de organismos vivos de concha caliza, y la caída de estas conchas al fondo del mar hizo que gradualmente se formaran gruesas capas de piedra caliza. Este fue el primer depósito generalizado de caliza, y cubrió prácticamente toda Europa y América del Norte, pero hay solo algunos sitios de la superficie de la Tierra en los que aparece. El grosor medio de esta capa rocosa ancestral es de unos trescientos metros, aunque, desde entonces, muchos de estos depósitos se han deformado considerablemente por inclinaciones, levantamientos y fallas, y muchos se han convertido en cuarzo, esquisto y mármol.
\vs p059 3:2 En las capas rocosas de este periodo no se encuentran rocas ígneas ni lava, exceptuando las de los grandes volcanes de Europa meridional y de Maine oriental y los flujos de lava de Quebec. La actividad volcánica había mayormente pasado. Este fue el culmen de las grandes sedimentaciones marinas; poca o ninguna montaña se llegaría a formar.
\vs p059 3:3 \pc Hace \bibemph{290\,000\,000} de años, el mar en gran medida se había retirado de los continentes, y los fondos de los océanos circundantes se hundían. Las masas de tierra habían cambiado poco hasta que volvieron de nuevo a sumergirse. En todos los continentes, comenzaban los tempranos movimientos de las montañas; los más importantes de estos levantamientos de la corteza terrestre fueron los Himalayas de Asia y las grandes montañas de la Caledonia, que se extienden desde Irlanda, a través de Escocia, hasta llegar a Spitzbergen.
\vs p059 3:4 Es en los sedimentos de esta era donde se encuentra gran parte de gas, petróleo, zinc y plomo; el gas y el petróleo se derivan de las enormes acumulaciones de materia vegetal y animal que quedaron depositadas durante el anterior sumergimiento del suelo terrestre, mientras que los depósitos minerales provienen de la sedimentación de masas de aguas mansas. La mayoría de los depósitos de sal de roca corresponden a este período.
\vs p059 3:5 Los trilobites experimentaron un rápido descenso, y los moluscos mayores, o cefalópodos, se colocaron en el centro de la escena. Estos animales llegaron a alcanzar unas dimensiones de más de cuatro metros y medio de largo por treinta centímetros de diámetro, y se hicieron los dueños de los mares. Esta especie apareció \bibemph{de repente} y se hizo con la supremacía de la vida marina.
\vs p059 3:6 La gran actividad volcánica de esta era tuvo lugar en el área europea. Desde hacía muchos millones de años no se habían producido unas erupciones volcánicas tan virulentas y de tanta magnitud como las ocurridas en este momento en torno a la depresión del Mediterráneo y, especialmente, en las inmediaciones de las Islas Británicas. Este flujo de lava que se había derramado sobre la región de las Islas Británicas está presente hoy en día bajo la forma de capas alternas de lava y roca de más de 7500 metros de grosor. Estas rocas se depositaron por los flujos intermitentes de lava que se extendieron sobre un lecho de mar de poca profundidad, esparciendo así los depósitos de roca, con la consiguiente elevación de todo ello a gran altura sobre el nivel del mar. En el norte de Europa, particularmente en Escocia, se produjeron violentos terremotos.
\vs p059 3:7 El clima oceánico continuaba siendo moderado y constante, y los mares cálidos bañaban las costas de las tierras polares. Se pueden encontrar braquiópodos y otros fósiles de vida marina en estos depósitos hasta en el Polo Norte. Los gasterópodos, braquiópodos, esponjas y los corales formadores de arrecifes seguían aumentando.
\vs p059 3:8 Al final de esta época se evidencia el segundo avance de los mares silúricos con otra mezcla de las aguas de los océanos meridionales y septentrionales. Los cefalópodos dominan la vida marina, en tanto que formas relacionadas de vida se van desarrollando y diferenciándose de manera progresiva.
\vs p059 3:9 \pc Hace \bibemph{280\,000\,000} de años, los continentes habían emergido, en gran parte, tras la segunda inundación silúrica. Los depósitos de roca de este sumergimiento se conocen en América del Norte con el nombre de piedra caliza del Niágara, porque este es el estrato rocoso sobre el que hoy día fluyen las cataratas de Niágara. Esta capa rocosa se extiende desde las montañas orientales hasta la región del valle del Misisipí, pero no hacia el oeste sino hacia el sur. Algunas capas se extienden sobre Canadá, partes de América del Sur, Australia y la mayoría de Europa; el grosor medio de esta serie de estratos del Niágara es de unos mil ochocientos metros. En muchas regiones, directamente superpuesto al depósito del Niágara, se puede encontrar un conjunto de conglomerado, esquisto y sal de roca por la acumulación de hundimientos de carácter secundario. Esta sal se asentó en grandes lagunas, que alternativamente se abrían al mar y se cerraban después, de manera que se produjo la evaporación con la sedimentación de sal junto con otras materias contenidas en la disolución. En algunas regiones estos lechos de sal de roca son de un espesor que sobrepasa los veinte metros.
\vs p059 3:10 El clima es uniforme y moderado, y los fósiles marinos se depositan en las regiones árticas. Pero, para el fin de esta época, los mares son tan excesivamente salados que poca vida es capaz de sobrevivir en ellos.
\vs p059 3:11 Al acabar el último sumergimiento silúrico, hay un gran aumento de equinodermos ---o lirios de mar--- tal como se percibe en los depósitos calcáreos de crinoideos. Los trilobites casi han desaparecido, y los moluscos continúan siendo los reyes de los mares; la formación de arrecifes coralinos se incrementa de manera considerable. Durante esta era, en los lugares más favorables, evolucionan, por primera vez, los escorpiones acuáticos primitivos. Poco después, y \bibemph{de repente,} los auténticos escorpiones ---los verdaderos respiradores de aire--- hacen su aparición.
\vs p059 3:12 Con estos avances, se termina el tercer período de la vida marina, que se extiende durante veinticinco millones de años y que vuestros investigadores conocen con el nombre de \bibemph{Silúrico}.
\usection{4. LA ETAPA DE LA GRAN EMERSIÓN DEL SUELO TERRESTRE. EL PERÍODO DE LA VIDA VEGETAL TERRESTRE. LA ERA DE LOS PECES}
\vs p059 4:1 En la multisecular pugna entre el suelo y el agua, durante largos períodos, el mar ha salido relativamente victorioso, pero los tiempos del triunfo del suelo están por venir. Y las derivas continentales no han llegado tan lejos, sino que, a veces, prácticamente todo el suelo del mundo está comunicado por angostos istmos y estrechos y puentes terrestres.
\vs p059 4:2 Cuando el suelo emerge tras la última inundación silúrica, llega a su fin un importante período en el desarrollo del mundo y en la evolución de la vida. Es el comienzo de una nueva era en la Tierra. El paisaje desnudo y sin atractivo de tiempos precedentes se va vistiendo de un verdor exuberante, y pronto aparecerán los primeros y magníficos bosques.
\vs p059 4:3 La vida marina de esta era fue muy diversa debido a la separación de las primeras especies, aunque más tarde todas estas clases diferentes de especies se mezclaron y asociaron sin restricción. Los braquiópodos alcanzaron temprano su punto culminante; les sucedieron los artrópodos y los percebes que aparecían por primera vez. Pero el más grande de todos los acontecimientos fue la aparición repentina de la familia de los peces. Aquella época se convirtió en la era de los peces, en ese período de la historia del mundo caracterizado por los animales del orden de los \bibemph{vertebrados.}
\vs p059 4:4 \pc Hace \bibemph{270\,000\,000} de años, todos los continentes estaban sobre el nivel del agua. Desde hacía muchos millones de años, no había habido tanto suelo por encima del agua en un mismo momento; fue aquella una de las épocas de más grande emersión del suelo de toda la historia del mundo.
\vs p059 4:5 Cinco millones de años más tarde, las áreas de suelo de América del Norte y del Sur, Europa, África, Asia septentrional y Australia se inundaron durante un breve periodo de tiempo; en algún momento, el sumergimiento de América del Norte fue casi total, y las capas de piedra caliza resultantes oscilan entre unos 150 y 1500 metros de grosor. Estos diferentes mares devónicos se extendían primeramente en una dirección y después en otra, de forma que el inmenso mar interior ártico de América del Norte encontró su salida al Océano Pacífico a través del norte de California.
\vs p059 4:6 \pc Hace \bibemph{260\,000\,000} de años, hacia el final de esta época de depresión del suelo, América del Norte estaba parcialmente cubierta por unos mares que se comunicaban simultáneamente con las aguas del Pacífico, del Atlántico, del Ártico y del Golfo. Los depósitos de estas etapas tardías de la primera inundación devónica tienen un grosor medio de unos trescientos metros. Los arrecifes coralinos característicos de estos tiempos muestran que los mares interiores eran transparentes y poco profundos. Dichos depósitos de coral están al descubierto en las márgenes del río Ohio cerca de Louisville, Kentucky; tienen unos treinta metros de grosor y abarcan más de doscientas variedades. Estas formaciones de coral se extienden a través de Canadá y del norte de Europa hasta las regiones árticas.
\vs p059 4:7 Tras estos sumergimientos, muchos litorales se elevaron de forma considerable de modo que los depósitos anteriores quedaron cubiertos de lodo o esquisto. Existe también un estrato de piedra arenisca roja que caracteriza una de las sedimentaciones devónicas, y esta capa roja se extiende por gran parte de la superficie de la Tierra, encontrándose en América del Norte y del Sur, Europa, Rusia, China, África y Australia. Aunque el clima de esta época era todavía templado y estable, estos depósitos rojos muestran la presencia de un entorno árido o semiárido.
\vs p059 4:8 A lo largo de todo este período, el suelo al sureste de la Isla de Cincinnati continuó muy elevado sobre el nivel del agua. Si bien, una gran parte de Europa occidental, incluyendo las Islas Británicas, estaba sumergida. En Gales, Alemania y otros lugares de Europa, las rocas devónicas tienen un grosor de 6000 metros.
\vs p059 4:9 \pc Hace \bibemph{250\,000\,000} de años, se observó la aparición de la familia de los peces, de los vertebrados, que constituyó una de las fases más importantes de toda la evolución prehumana.
\vs p059 4:10 Los artrópodos, o crustáceos, fueron los ancestros de los primeros vertebrados. Los antecesores de la familia de los peces fueron dos artrópodos modificados; uno poseía un cuerpo largo que servía de unión entre la cola y la cabeza, mientras que el otro pre\hyp{}pez carecía de espina dorsal y de mandíbula. Sin embargo, estos predecesores de los peces se extinguieron pronto cuando los peces, los primeros vertebrados del mundo animal, hicieron \bibemph{de repente} su aparición desde el norte.
\vs p059 4:11 Una gran parte de los auténticos y más grandes peces pertenecen a esta era; algunas de las variedades dentadas tenían entre siete metros y medio y nueve metros de largo; los actuales tiburones son los supervivientes de estos peces ancestrales. Los peces con pulmón y coraza llegaron a alcanzar su apogeo evolutivo, y antes de que esta época hubiere terminado, los peces se habían adaptado tanto al agua dulce como a la salada.
\vs p059 4:12 Se pueden encontrar verdaderos lechos óseos de dientes y esqueletos de peces en los depósitos que se crearon hacia el final de este período, y existen lechos ricos en fósiles situados a lo largo de la costa de California, puesto que muchas bahías resguardadas del Océano Pacífico se expandían en el suelo de esa región.
\vs p059 4:13 La Tierra se invadió rápidamente de nuevos órdenes de vegetación terrestre. Hasta ese momento crecían pocas plantas en el suelo terrestre, excepto junto a los bordes del agua. Entonces, y de repente, la prolífica \bibemph{familia de los helechos} apareció y se propagó velozmente por la superficie terrestre, que de forma rápida se elevaba en todas las partes del mundo. Pronto se desarrollaron algunos tipos de árbol de un grosor de sesenta centímetros y de una altura de doce metros; después evolucionaron las hojas, aunque estas variedades tempranas solo poseían un follaje simple. Había muchas plantas más pequeñas, pero no es posible encontrar sus fósiles; las bacterias, que habían aparecido con anterioridad, solían destruirlas.
\vs p059 4:14 Al elevarse el suelo, América del Norte se unió a Europa por medio de puentes de tierra que se extendían hasta Groenlandia. Y, Groenlandia, hoy en día, conserva los restos de estas tempranas plantas terrestres bajo su manto de hielo.
\vs p059 4:15 \pc Hace \bibemph{240\,000\,000} de años, el suelo de algunas partes de Europa, de América del Norte y del Sur empezó a hundirse. Este hundimiento significó la aparición de la última y menos extensa de las inundaciones devónicas. Los mares árticos de nuevo se desplazaron hacia el sur sobre una extensa zona de América del Norte, el Atlántico inundó una gran parte de Europa y de Asia occidental, en tanto que el Pacífico meridional cubrió la mayoría de la India. Esta inundación fue lenta en aparecer e igualmente lenta en retirarse. Las montañas Catskill, situadas a lo largo de la margen occidental del río Hudson, constituyen uno de los más grandes monumentos geológicos de esta época factibles de encontrar sobre la superficie de América del Norte.
\vs p059 4:16 \pc Hace \bibemph{230\,000\,000} de años, los mares continuaban retrocediendo. Una gran parte de América del Norte estaba sobre el nivel del agua, y, en la región de San Lorenzo, se produjo una gran actividad volcánica. El monte Real, en Montreal, es la chimenea erosionada de uno de estos volcanes. Los depósitos de toda esta época se perciben claramente en los montes Apalaches de América del Norte, en donde el río Susquehanna ha tallado un valle y ha dejado al descubierto estas capas sucesivas, que alcanzaron un grosor de unos 4000 metros.
\vs p059 4:17 \pc La elevación de los continentes prosiguió, y la atmósfera se fue enriqueciendo en oxígeno. La Tierra estaba cubierta de inmensos bosques de helechos de una altura de unos treinta metros y de árboles característicos de aquellos tiempos. Eran bosques silenciosos; no se escuchaba ni un ruido, ni siquiera el crujido de una hoja, porque aquellos árboles no tenían hojas.
\vs p059 4:18 \pc Y, de este modo, llegó a su fin uno de los períodos más largos de la evolución de la vida marina, la era de los peces. Esta etapa de la historia del mundo duró casi cincuenta millones de años; vuestros investigadores la conocen como el período \bibemph{Devónico}.
\usection{5. LA ETAPA DEL DESPLAZAMIENTO DE LA CORTEZA TERRESTRE. EL PERÍODO CARBONÍFERO DE LOS BOSQUES DE HELECHOS. LA ERA DE LAS RANAS}
\vs p059 5:1 La aparición de los peces durante el anterior período señala el punto álgido de la evolución de la vida marina. A partir de este momento, la evolución de la vida terrestre se vuelve cada vez más importante. Y este periodo se inaugura en un marco casi idóneo para la aparición de los primeros animales terrestres.
\vs p059 5:2 \pc Hace \bibemph{220\,000\,000} de años, muchas áreas de suelo continental, incluyendo la mayoría de América del Norte, estaban sobre el nivel del agua. Una exuberante vegetación se había adueñado del suelo terrestre; se trataba, en efecto, de la era de los helechos. El dióxido de carbono seguía presente en la atmósfera, pero cada vez en menor grado.
\vs p059 5:3 Poco después se inundó la región central de América del Norte, creando dos grandes mares interiores. Tanto las zonas altas costeras del Pacífico como las del Atlántico estaban situadas algo más allá de los litorales actuales. Estos dos mares se unieron pronto, mezclándose sus distintas formas de vida. La unión de esta fauna marina marcó el comienzo del rápido declive global de la vida marina y el principio del siguiente período de vida terrestre.
\vs p059 5:4 \pc Hace \bibemph{210\,000\,000} de años, las cálidas aguas de los mares árticos cubrían la mayor parte de América del Norte y Europa. Las aguas polares del sur anegaban América del Sur y Australia, mientras que África y Asia estaban bastante elevadas.
\vs p059 5:5 Cuando los mares estaban en su punto álgido, ocurrió \bibemph{de repente} un nuevo desarrollo evolutivo. De súbito aparecieron los primeros animales terrestres. Había numerosas especies de estos capaces de vivir tanto en el suelo como en el agua. Estos anfibios que respiraban aire se habían desarrollado a partir de los artrópodos, cuyas vejigas natatorias se habían convertido en pulmones.
\vs p059 5:6 De las salobres aguas de los mares, salieron caracoles, escorpiones y ranas que se deslizaban por el suelo terrestre. Hoy en día, las ranas siguen poniendo sus huevos en el agua, y sus crías poseen, al comienzo de su existencia, la forma de pequeños peces, o renacuajos. Este período puede conocerse propiamente como la \bibemph{era de las ranas}.
\vs p059 5:7 Muy poco después, y por primera vez, aparecieron los insectos y, junto con las arañas, los escorpiones, las cucarachas, los grillos y las cigarras, pronto se extendieron por los continentes del mundo. Las libélulas medían más de setenta y cinco centímetros de ancho. Se desarrollaron mil especies de cucarachas, llegando algunas a medir diez centímetros de largo.
\vs p059 5:8 Dos grupos de equinodermos adquirieron especialmente un buen desarrollo, y son en realidad los fósiles guía de esta época. Los grandes tiburones, comedores de animales con concha, estaban igualmente bien desarrollados, y durante más de cinco millones de años dominaron los océanos. El clima era todavía templado y estable; la vida marina cambió poco. Los peces de agua dulce se iban desarrollando y los trilobites estaban cerca de la extinción. Los corales eran escasos y los crinoideos eran los productores de gran parte de la piedra caliza. Durante esta época se depositaron las piedras calizas más finas para la construcción.
\vs p059 5:9 Las aguas de muchos mares interiores estaban tan sobrecargadas de cal y de otros minerales que interferían enormemente en el progreso y en el desarrollo de muchas especies marinas. Los mares acabaron por limpiarse como resultado de un gran depósito de roca que, en algunas partes, contenía zinc y plomo.
\vs p059 5:10 Los depósitos de esta temprana era Carbonífera oscilan entre 150 y 600 metros de grosor, y están compuestos de arenisca, esquisto y piedra caliza. Los estratos más antiguos contienen los fósiles de animales y plantas tanto terrestres como marinos, junto con mucha graba y sedimentos de las cuencas. En estos estratos más antiguos se encuentra poco carbón explotable. Estos sedimentos localizados por toda Europa son muy similares a los que se asentaron en América del Norte.
\vs p059 5:11 Hacia el final de esta época, el suelo de América del Norte comenzó a elevarse. Hubo una breve interrupción, y el mar volvió a cubrir aproximadamente la mitad de sus lechos anteriores. Aquella fue una breve inundación y, pronto, la mayoría del suelo se encontró sobre las aguas. América del Sur, a través de África, se comunicaba todavía con Europa.
\vs p059 5:12 En esta época se presencia el comienzo de los Vosgos, la Selva Negra y los Montes Urales. Por toda Gran Bretaña y Europa se encuentran los restos de otras montañas más antiguas.
\vs p059 5:13 \pc Hace \bibemph{200\,000\,000} de años comenzaron las etapas realmente activas del período carbonífero. Durante los veinte millones de años que precedieron a este tiempo, se iban asentando los más tempranos depósitos de carbón, pero ahora la actividad de formación de carbón se puso en marcha en mayor extensión. La duración de la época de la auténtica sedimentación del carbón fue de poco más de veinticinco millones de años.
\vs p059 5:14 Periódicamente, el suelo terrestre subía y bajaba debido a la variabilidad del nivel del mar, a su vez causada por la actividad que tenía lugar en los fondos oceánicos. Esta inestabilidad de la corteza terrestre ---el asentamiento y la elevación del suelo---, en conjunción con la prolífica vegetación de los pantanos costeros, contribuyó a la creación de grandes depósitos de carbón, razón por la que se conoce a este período como \bibemph{Carbonífero}. Y el clima seguía siendo templado en todo el mundo.
\vs p059 5:15 Las capas de carbón alternan con esquisto, piedra y conglomerado. Estos lechos de carbón del centro y del este de los Estados Unidos tienen un grosor que oscila entre doce y quince metros. Pero muchos de estos depósitos fueron arrastrados por el agua durante las posteriores elevaciones de suelo. En algunas partes de América del Norte y Europa, los estratos carboníferos tienen un grosor de unos 5500 metros.
\vs p059 5:16 La presencia de las raíces de los árboles creciendo en la arcilla bajo los actuales lechos de carbón demuestra que el carbón se formó exactamente donde se encuentra hoy. El carbón es un resto, conservado por el agua y modificado por la presión, de la exuberante vegetación que se desarrolló en las ciénagas y en las márgenes de los pantanos de esta remota era. Las capas de carbón contienen a menudo gas al igual que petróleo. Los lechos de turba, restos de una antigua vegetación, se convertirían en un tipo de carbón si se las sometiera a una presión y calor adecuados. La antracita ha estado expuesta a más presión y calor que otros tipos de carbón.
\vs p059 5:17 En América del Norte, las capas carboníferas de los distintos lechos, cuyo número muestra la cantidad de veces que el suelo se hundió y se elevó, varían desde las diez de Illinois, las veinte de Pensilvania y las treinta y cinco de Alabama hasta las setenta y cinco de Canadá. En los lechos de carbón, se encuentran fósiles tanto de agua dulce como de agua salada.
\vs p059 5:18 A lo largo de toda esta época, las montañas de América del Norte y del Sur estaban activas; tanto los Andes como las ancestrales Montañas Rocosas del sur se elevaron. Las grandes zonas costeras altas del Atlántico y del Pacífico empezaron a hundirse, llegando a estar finalmente tan erosionadas y sumergidas que los litorales de ambos océanos retrocedieron hasta aproximadamente sus posiciones actuales. Los depósitos de esta inundación tienen un grosor medio de unos trescientos metros.
\vs p059 5:19 \pc Hace \bibemph{190\,000\,000} de años se evidenció una expansión hacia el oeste del mar carbonífero de América del Norte sobre la presente región de las Montañas Rocosas, con salida al Océano Pacífico a través del norte de California. El carbón continuó asentándose, capa sobre capa, por todas las Américas y Europa, a medida que las regiones costeras se elevaban y descendían durante estas eras de oscilación de las orillas de los mares.
\vs p059 5:20 \pc Hace \bibemph{180\,000\,000} de años concluyó el período carbonífero, durante el que se había formado carbón por todo el mundo ---en Europa, India, China, África del Norte y las Américas---. Al final de este período, el este del valle del Misisipí, en América del Norte, se elevó, y la mayoría de esta zona ha permanecido, desde entonces, sobre el nivel del mar. Esta época de elevación del suelo terrestre señala el comienzo de las montañas modernas de América del Norte, tanto en las regiones de los Apalaches como en el oeste. Los volcanes estaban activos en Alaska y California y en las zonas de Europa y Asia donde se estaban formando las montañas. América oriental y Europa occidental se comunicaban por el continente de Groenlandia.
\vs p059 5:21 La elevación del suelo empezó a modificar el clima marino de las eras anteriores y a sustituirlo por los comienzos del clima continental, menos templado y más variable.
\vs p059 5:22 Las plantas de estos tiempos eran esporíferas, y el viento podía diseminar ampliamente sus esporas. Los troncos de los árboles carboníferos solían tener un diámetro de más de dos metros y, con frecuencia, una altura de treinta y ocho metros. Los helechos modernos son verdaderas reliquias de estas eras pasadas.
\vs p059 5:23 En general, estas fueron las épocas en las que los organismos de agua dulce alcanzaron su desarrollo; se produjo poco cambio en la vida marina previa. Pero la característica importante de este período fue la aparición \bibemph{repentina} de las ranas y de sus numerosos primos. La vida en la era carbonífera se caracterizó por la presencia de los \bibemph{helechos} y de las \bibemph{ranas}.
\usection{6. LA ETAPA DE TRANSICIÓN CLIMÁTICA. EL PERÍODO DE LAS PLANTAS CON SEMILLAS. LA ERA DE LA HECATOMBE BIOLÓGICA}
\vs p059 6:1 Este período señala el fin del decisivo desarrollo evolutivo de la vida marina y el comienzo del período de transición que condujo a las siguientes eras de los animales terrestres.
\vs p059 6:2 Aquella fue una era de gran empobrecimiento de la vida. Perecieron miles de especies marinas, y la vida terrestre apenas estaba aún asentada. Fue un periodo de hecatombe biológica, una era en la que la vida estuvo a punto de extinguirse de la faz de la Tierra y de las profundidades de los océanos. Hacia el final de la larga era de vida marina, existían más de cien mil especies de seres vivos en la Tierra. Al concluirse este período de transición, menos de quinientas de ellas habían sobrevivido.
\vs p059 6:3 Las peculiaridades de este nuevo período no se debieron tanto al enfriamiento de la corteza terrestre ni a la larga ausencia de actividad volcánica, sino a una excepcional combinación de influencias habituales existentes con antelación: la delimitación de los mares y la creciente elevación de enormes masas de tierra. El clima marino templado de tiempos pasados iba desapareciendo, y el tipo de clima continental más severo se expandía rápidamente.
\vs p059 6:4 \pc Hace \bibemph{170\,000\,000} de años ocurrieron sobre toda la faz de la Tierra grandes cambios y adaptaciones evolutivas. El suelo terrestre se elevaba en todo el mundo conforme se hundían los lechos oceánicos. Aparecieron cadenas montañosas aisladas. La parte oriental de América del Norte estaba muy por encima del nivel del mar; la parte occidental se elevaba lentamente. Los continentes estaban cubiertos de lagos salados, grandes y pequeños, al igual que de numerosos mares interiores que se comunicaban con los océanos por angostos estrechos. Los estratos de este período de transición oscilan entre 300 y 2100 metros de grosor.
\vs p059 6:5 La corteza terrestre se plegó de manera considerable durante las elevaciones del suelo. Aquel fue el periodo de la emersión continental excepto por la desaparición de algunos puentes terrestres, incluidos los continentes que habían comunicado por tan largo tiempo América del Sur con África y América del Norte con Europa.
\vs p059 6:6 Por todo el mundo, los lagos y los mares interiores se fueron secando progresivamente. Empezaron a aparecer glaciares de montaña aislados y regionales, especialmente en el hemisferio sur; y en muchas regiones es posible encontrar el depósito glacial de estas formaciones locales de hielo incluso entre los estratos superiores y posteriores de algunos depósitos carboníferos. Dos nuevos factores climáticos aparecieron: la congelación y la aridez. Muchas de las regiones más elevadas de la Tierra se habían vuelto áridas y estériles.
\vs p059 6:7 \pc Durante todos estos tiempos de cambio climático, se produjeron igualmente grandes variaciones en las plantas terrestres. Por primera vez, aparecieron las plantas con \bibemph{semilla,} que proporcionaron un mejor suministro de alimentos para la vida animal terrestre en su posterior incremento. Los insectos sufrieron un cambio radical. Se desarrollaron las \bibemph{etapas de reposo} para adecuarse a las necesidades de la animación suspendida durante el invierno y las sequías.
\vs p059 6:8 \pc Entre los animales terrestres, las ranas alcanzaron su punto álgido en la era previa y se redujeron con celeridad, pero sobrevivieron porque podían vivir largamente incluso en los charcos y estanques en desecación de estos tiempos remotos y extremadamente difíciles. Durante esta era de declive de la rana, se produjo en África su primer paso evolutivo al reptil. Y puesto que las masas de tierra estaban todavía comunicadas entre sí, esta criatura pre\hyp{}reptil, respiradora de aire, se propagó por todo el mundo. Para entonces la atmósfera había cambiado tanto que servía admirablemente como sostén de la respiración animal. Poco tiempo tras la llegada de estas ranas pre\hyp{}reptiles, América del Norte quedó temporalmente aislada, apartada de Europa, Asia y América del Sur.
\vs p059 6:9 El enfriamiento gradual de las aguas oceánicas contribuyó sobremanera a la erradicación de la vida oceánica. Los animales marinos de aquellas eras encontraron refugio temporal en tres lugares cuyas condiciones les eran favorables: la región actual del Golfo de México, la bahía del Ganges, en la India, y la bahía siciliana, en la cuenca mediterránea. Y desde estas tres regiones, las nuevas especies marinas, nacidas en la adversidad, se pusieron más tarde en marcha para repoblar los mares.
\vs p059 6:10 \pc Hace \bibemph{160\,000\,000} de años, el suelo estaba, en gran medida, recubierto de una vegetación adaptada para dar soporte a la vida animal terrestre, y la atmósfera era ya óptima para la respiración animal. Terminaba así el período de decadencia de la vida marina y de aquellos duros tiempos de adversidad biológica que provocaron la extinción de todas las formas de vida excepto aquellas aptas para sobrevivir y, que, de este modo, estaban legitimadas para constituirse en los ancestros de la vida de más rápido desarrollo, y sumamente diferenciada, de las eras de evolución planetaria que estaban al llegar.
\vs p059 6:11 La finalización de este período de hecatombe biológica, que vuestros estudiosos conocen como Pérmico, señala además el fin de la larga era \bibemph{Paleozoica,} que abarca una cuarta parte de la historia planetaria, esto es, doscientos cincuenta millones de años.
\vs p059 6:12 El inmenso vivero oceánico de vida que ha sido Urantia ha cumplido su propósito. Durante las largas eras en las que la Tierra no era idónea para el sostenimiento de la vida, antes de que la atmósfera tuviese suficiente oxígeno para mantener a los animales terrestres superiores, el mar cuidaba y nutría la vida temprana del planeta. Ahora, la relevancia biológica del mar disminuye paulatinamente a medida que la segunda etapa evolutiva comienza su despliegue en el suelo terrestre.
\vsetoff
\vs p059 6:13 [Exposición de un portador de vida de Nebadón, miembro del colectivo original asignado a Urantia.]
