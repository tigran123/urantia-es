\upaper{14}{El universo central y divino}
\author{Perfeccionador de la sabiduría}
\vs p014 0:1 El universo perfecto y divino ocupa el centro de toda la creación. Es el núcleo eterno a cuyo alrededor giran las inmensas creaciones del tiempo y del espacio. El Paraíso es una gigantesca isla nuclear absolutamente estable, que reposa inmóvil en el corazón mismo del magnífico universo eterno. Esta agrupación planetaria central se llama Havona y está muy distante del universo local de Nebadón. Es de dimensiones enormes y de una masa casi increíble. Consta de mil millones de esferas de belleza inimaginable y de grandeza majestuosa, pero la verdadera magnitud de esta inmensa creación excede totalmente la comprensión de la mente humana.
\vs p014 0:2 Es el único conjunto de mundos perfectos y estables. Es un universo totalmente creado y perfecto. No proviene de un desarrollo evolutivo. Es el núcleo eterno de perfección en torno al cual gira esa procesión infinita de universos que constituyen el formidable experimento evolutivo, la audaz aventura de los hijos creadores de Dios que aspiran a duplicar en el tiempo y a reproducir en el espacio este modelo del universo, el ideal de la completitud divina, de la completud suprema, de la realidad última y de la perfección eterna.
\usection{1. EL SISTEMA PARAÍSO\hyp{}HAVONA}
\vs p014 1:1 Desde la periferia del Paraíso hasta los límites interiores de los siete suprauniversos, se dan en el espacio las siete condiciones y movimientos siguientes:
\vs p014 1:2 \li{1.}Las zonas quiescentes del espacio medio que lindan con el Paraíso.
\vs p014 1:3 \li{2.}El movimiento procesional en el sentido de las manecillas del reloj de las tres vías circulatorias en torno al Paraíso y las siete de Havona.
\vs p014 1:4 \li{3.}La zona espacial semiquieta que separa las vías circulatorias de Havona de los cuerpos oscuros de gravedad del universo central.
\vs p014 1:5 \li{4.}El cinturón interior de los cuerpos oscuros de gravedad que se mueven en sentido opuesto al de las manecillas del reloj.
\vs p014 1:6 \li{5.}La segunda zona espacial única que divide las dos rutas espaciales de los cuerpos oscuros de gravedad.
\vs p014 1:7 \li{6.}El cinturón exterior de los cuerpos oscuros de gravedad que giran en el sentido de las manecillas del reloj alrededor del Paraíso.
\vs p014 1:8 \li{7.}Una tercera zona espacial ---una zona semiquieta--- que separa el cinturón exterior de los cuerpos oscuros de gravedad de las vías circulatorias más interiores de los siete suprauniversos.
\vs p014 1:9 \pc Los mil millones de mundos de Havona están dispuestos en siete vías circulatorias concéntricas que rodean directamente a las tres vías circulatorias de satélites del Paraíso. Hay más de treinta y cinco millones de mundos en la vía más interior de Havona y más de doscientos cuarenta y cinco millones en la más exterior, con cantidades proporcionales entre ambos. Cada vía difiere, pero todas están perfectamente equilibradas y excelentemente organizadas, y cada una está infundida de una representación distintiva del Espíritu Infinito, de uno de los siete espíritus de las vías. Además de otros cometidos, este espíritu impersonal coordina la realización de los asuntos celestiales en todas y cada una de estas vías.
\vs p014 1:10 Las vías planetarias de Havona no están superpuestas, sino que sus mundos se suceden en una ordenada procesión lineal. El universo central gira alrededor de la Isla estacionaria del Paraíso en un inmenso plano que consta de diez unidades concéntricas estabilizadas: las tres vías circulatorias formadas por las esferas del Paraíso y las siete formadas por los mundos de Havona. Desde el punto de vista físico, las vías de Havona y las del Paraíso constituyen un mismo sistema; su separación responde a una división de carácter operativo y administrativo.
\vs p014 1:11 \pc En el Paraíso no se computa el tiempo. Los oriundos de la Isla central tienen, de forma innata, una noción secuencial ininterrumpida de los acontecimientos. Pero el tiempo es propio de las vías planetarias de Havona y de numerosos seres de origen tanto celestial como terrestres que allí habitan. Cada mundo de Havona tiene su propio tiempo local, determinado por la vía circulatoria en la que se encuentran. En todos los mundos de una determinada vía, los años tienen la misma duración puesto que giran uniformemente alrededor del Paraíso, y la duración de estos años planetarios decrece según se va de la vía circulatoria más exterior a la más interior.
\vs p014 1:12 Aparte del tiempo propio de las vías de Havona, existe en el conjunto Paraíso\hyp{}Havona un día regular además de otras divisiones temporales que se determinan a partir de los siete satélites del Espíritu Infinito del Paraíso. El día regular del Paraíso\hyp{}Havona se computa sobre la base del tiempo que las moradas planetarias de la primera vía circulatoria de Havona, la más interior, requieren para completar una rotación alrededor de la Isla del Paraíso; y aunque su velocidad es enorme, debido a su situación entre los cuerpos oscuros de gravedad y el gigantesco Paraíso, lleva a estas esferas casi mil años completar su órbita. Sin apercibiros leísteis la verdad cuando vuestros ojos se posaron sobre la frase “un día es como mil años para con Dios, no más que una de las vigilias de la noche”. Un día del Paraíso\hyp{}Havona dura mil años del presente calendario bisiesto de Urantia menos solamente siete minutos, tres segundos y un octavo de segundo.
\vs p014 1:13 Este día del Paraíso\hyp{}Havona constituye la medida de tiempo estándar en los siete suprauniversos, aunque cada uno mantiene sus propios patrones temporales internos.
\vs p014 1:14 \pc En los alrededores de este inmenso universo central, en la lejanía, más allá del séptimo cinturón de los mundos de Havona, gira un increíble número de enormes cuerpos oscuros de gravedad. Estas innumerables masas oscuras son bastante diferentes a los otros cuerpos espaciales en muchos aspectos; incluso en su forma son muy diferentes. Estos cuerpos oscuros de gravedad no reflejan ni absorben la luz; no son reactivos a la luz de la energía física, y rodean y envuelven tan completamente a Havona que la ocultan a la visión de incluso los universos habitados cercanos del tiempo y del espacio.
\vs p014 1:15 Una singular intrusión espacial divide a este gran cinturón de cuerpos oscuros en dos vías circulatorias elípticas iguales. El cinturón interior gira en sentido contrario a las manecillas del reloj; el exterior, en el sentido de las manecillas. Estas direcciones alternadas de movimiento, combinadas con la extraordinaria masa de los cuerpos oscuros, equilibran tan eficazmente las líneas de la gravedad de Havona que convierten al universo central en una creación físicamente equilibrada y perfectamente estabilizada.
\vs p014 1:16 La procesión interior de cuerpos oscuros de gravedad tiene una disposición tubular consistente en tres agrupaciones circulares. Una sección transversal de esta vía mostraría tres círculos concéntricos de densidad aproximadamente igual. La vía exterior de cuerpos oscuros de gravedad está dispuesta de forma perpendicular, siendo diez mil veces más alta que la vía interior. El diámetro vertical de la vía exterior es cincuenta mil veces el del diámetro transversal.
\vs p014 1:17 El espacio intermedio que existe entre estas dos vías circulatorias de cuerpos de gravedad es \bibemph{único} puesto que no se encuentra nada que se le asemeje en ninguna otra parte del inmenso universo. En esta zona, caracterizada por enormes movimientos en onda de naturaleza vertical, se da una enorme actividad energética de orden desconocido.
\vs p014 1:18 En nuestra opinión, no habrá nada de características similares a los cuerpos oscuros de gravedad del universo central en la evolución futura de los niveles del espacio exterior; consideramos estas procesiones alternadas de formidables cuerpos equilibradores de la gravedad como únicos en el universo matriz.
\usection{2. CONSTITUCIÓN DE HAVONA}
\vs p014 2:1 Los seres espirituales no moran en un espacio nebuloso; no habitan mundos etéreos sino que residen en genuinas esferas de naturaleza material, en mundos tan reales como aquellos en los que viven los mortales. Los mundos de Havona son reales y auténticos, aunque su masa real difiere del patrón material de los planetas de los siete suprauniversos.
\vs p014 2:2 Las realidades físicas de Havona conforman un tipo de organización energética radicalmente diferente al de cualquier otro imperante en los universos evolutivos del espacio. Las energías de Havona son triples. Las unidades de energía\hyp{}materia del suprauniverso contienen una doble carga de energía, aunque existe una forma de energía en fases negativa y positiva. La creación del universo central es triple (la Trinidad); la creación de un universo local (de forma directa) es doble, por un hijo creador y un espíritu creativo.
\vs p014 2:3 La materia de que está hecha Havona consiste en un sistema de exactamente mil elementos químicos básicos y conlleva la actuación equilibrada de siete formas de energía. Cada una de estas energías básicas manifiesta siete facetas de sensaciones, de manera que los originarios de Havona responden a cuarenta y nueve estímulos sensoriales diferentes. En otras palabras, visto desde un punto de vista puramente físico, los originarios del universo central poseen cuarenta y nueve formas específicas de sensaciones. Los sentidos morontiales son setenta, y los más elevados órdenes espirituales de reacción varían, según las diferentes clases de seres, de setenta a doscientos diez.
\vs p014 2:4 Ninguno de los seres físicos del universo central sería visible para los urantianos. Tampoco suscitaría ninguno de los estímulos físicos de esos mundos remotos reacción alguna en vuestros elementales órganos sensitivos. Si uno de los mortales de Urantia pudiese ser trasladado a Havona, sería allí sordo, ciego y carente por completo de toda otra reacción sensorial. Solamente podría obrar como un ser de limitada conciencia de sí mismo privado de todo estímulo ambiental y de toda reacción al entorno.
\vs p014 2:5 \pc Hay numerosos fenómenos físicos y reacciones espirituales que ocurren en la creación central que resultan desconocidos en mundos como los de Urantia. La organización fundamental de una creación triple es completamente distinta de la constitución doble de los universos creados del tiempo y del espacio.
\vs p014 2:6 Toda ley natural se coordina sobre una base que difiere enteramente de la de los sistemas duales de energía de las creaciones en evolución. El universo central está organizado de acuerdo con un sistema triple de supervisión perfecta y simétrica. En todo el sistema Paraíso\hyp{}Havona, se mantiene un perfecto equilibrio entre todas las realidades cósmicas y todas las fuerzas espirituales. El Paraíso, en su potestad absoluta sobre la creación material, regula y mantiene perfectamente las energías físicas de este universo central. El Hijo Eterno, como parte de esta potestad espiritual que todo lo abarca, sostiene, con la mayor perfección, el estatus espiritual de todos aquellos que habitan en Havona. En el Paraíso nada es experimental, y el sistema Paraíso\hyp{}Havona constituye una unidad de perfección creativa.
\vs p014 2:7 La gravedad universal espiritual del Hijo Eterno es sorprendentemente activa en todo el universo central. Todos los valores del espíritu y todos los seres personales espirituales son de forma incesante atraídos hacia adentro, hacia la morada de los Dioses. Este impulso hacia Dios es intenso e ineludible. El anhelo de alcanzar a Dios es más intenso en el universo central, no porque la gravedad espiritual sea más intensa que en los universos más distantes, sino porque esos seres que han llegado a Havona están más plenamente espiritualizados y, como consecuencia, responden más a la acción, siempre presente, de la atracción universal de la gravedad espiritual del Hijo Eterno.
\vs p014 2:8 De igual manera, el Espíritu Infinito atrae hacia el Paraíso todos los valores intelectuales. En todo el universo central, la gravedad mental del Espíritu Infinito obra en conjunción con la gravedad espiritual del Hijo Eterno y, juntos, constituyen el impulso combinado de las almas que ascienden para encontrar a Dios, alcanzar la Deidad, lograr el Paraíso y conocer al Padre.
\vs p014 2:9 \pc Havona es un universo espiritualmente perfecto y físicamente estable. La supervisión y la estabilidad equilibrada del universo central parecen ser perfectos. Todo lo físico o lo espiritual es perfectamente previsible, pero los fenómenos de la mente y la volición personal no lo son. De esto deducimos que el pecado es imposible que ocurra allí, pero lo hacemos partiendo de la base de que a las criaturas de libre voluntad originarias de Havona nunca se les ha culpado de transgredir la voluntad de la Deidad. Durante toda la eternidad, estos seres excelsos han sido invariablemente leales a los eternos de días. Tampoco ha aparecido el pecado en ninguna criatura que haya llegado a Havona como peregrino. No ha habido nunca ningún caso de conducta indebida de parte de ninguna criatura de ningún grupo de seres personales creados en el universo central de Havona, o admitidos a este. Tan perfectos y tan divinos son los métodos y medios de selección en los universos del tiempo que no hay constancia en Havona de error alguno; no se han cometido jamás equivocaciones; jamás se ha admitido prematuramente en el universo central a ninguna de las almas ascendentes.
\usection{3. LOS MUNDOS DE HAVONA}
\vs p014 3:1 No existe gobierno alguno en el universo central. Havona es tan excelentemente perfecto que no necesita ningún sistema intelectual de gobierno. No existen tribunales que se constituyan regularmente ni hay asambleas legislativas. Havona requiere tan solo instrucciones administrativas. Aquí puede observarse la altura de los ideales del verdadero \bibemph{auto} gobierno.
\vs p014 3:2 No hace falta gobierno entre estas inteligencias de tal perfección y casi perfectas. No tienen ninguna necesidad de regulación, porque son seres de perfección innata entremezclados con criaturas evolutivas que han estado mucho antes bajo la observación de los tribunales supremos de los suprauniversos.
\vs p014 3:3 La administración de Havona no es espontánea, pero es maravillosamente perfecta y divinamente eficiente. Es mayormente planetaria y pertenece al eterno de días que allí reside. Cada una de las esferas de Havona está dirigida por uno de estos seres personales de origen en la Trinidad. Los eternos de días no son creadores, sino administradores perfectos. Imparten sus enseñanzas con suprema pericia y dirigen a sus hijos planetarios con una sabiduría tan perfecta que linda en lo absoluto.
\vs p014 3:4 Los mil millones de esferas del universo central constituyen los mundos de formación de los elevados seres personales originarios del Paraíso y de Havona y, además, sirven de prueba final para las criaturas que ascienden desde los mundos evolutivos del tiempo. En la ejecución del magno plan del Padre Universal para la ascensión de la criatura, los peregrinos del tiempo arriban a los mundos receptores de la vía exterior o séptima y, tras formarse de manera progresiva y engrandecerse experiencialmente, avanzan continuamente hacia adentro, de planeta en planeta y de círculo en círculo, hasta que acaban por alcanzar las Deidades y lograr su residencia en el Paraíso.
\vs p014 3:5 Actualmente, aunque las esferas que forman las siete vías circulatorias se mantienen en toda su excelsa gloria, tan solo se utiliza aproximadamente el uno por ciento de toda la capacidad planetaria en la labor de promover el plan universal del Padre relativo a la ascensión de los mortales. Alrededor de un décimo del uno por ciento del área de estos enormes mundos está dedicado a la vida y actividad de los colectivos finales, de seres eternamente asentados en luz y vida que con frecuencia permanecen y sirven en los mundos de Havona. Estos excelsos seres tienen su residencia personal en el Paraíso.
\vs p014 3:6 La estructura planetaria de las esferas de Havona es enteramente diferente a la de los mundos y sistemas evolutivos del espacio. En ninguna otra parte de todo el gran universo es conveniente utilizar esferas tan enormes como mundos habitados. El elemento físico de la triata, combinado con el efecto equilibrador de los inmensos cuerpos oscuros de gravedad, posibilita igualar de manera perfecta las fuerzas físicas y equilibrar de forma excelente la diversa atracción de esta formidable creación. La antigravedad también se emplea en la organización de las tareas materiales y de la actividad espiritual de estos enormes mundos.
\vs p014 3:7 La arquitectura, iluminación y calentamiento, así como el embellecimiento biológico y artístico de las esferas de Havona están mucho más allá del mayor alcance posible de la imaginación humana. No es mucho lo que os puedo decir acerca de Havona; para comprender su belleza y grandeza, debéis verla. Pero hay ríos y lagos verdaderos en estos mundos perfectos.
\vs p014 3:8 Espiritualmente, estos mundos están inmejorablemente equipados y se adaptan perfectamente a su propósito de alojar a numerosos y diferenciados órdenes de seres que obran en el universo central. En estos mundos hermosos tiene lugar mucha actividad cuya comprensión escapa a los humanos.
\usection{4. LAS CRIATURAS DEL UNIVERSO CENTRAL}
\vs p014 4:1 Hay siete formas esenciales de organismos vivos y de seres en los mundos de Havona, y cada una de ellas se compone de tres modalidades distintas. Cada una de estas tres modalidades tiene a su vez setenta divisiones principales, y cada una de estas divisiones principales se compone de mil divisiones menores, que a su vez se subdividen, y así sucesivamente. Estos grupos esenciales de vida pueden clasificarse como:
\vs p014 4:2 \li{1.}Material.
\vs p014 4:3 \li{2.}Morontial.
\vs p014 4:4 \li{3.}Espiritual.
\vs p014 4:5 \li{4.}Absonito.
\vs p014 4:6 \li{5.}Último.
\vs p014 4:7 \li{6.}Coabsoluto.
\vs p014 4:8 \li{7.}Absoluto.
\vs p014 4:9 \pc La descomposición y la muerte no son parte del ciclo de vida en los mundos de Havona. En el universo central, los organismos vivos de menor rango sufren la transmutación de su materialidad. Cambian de forma y manifestación, pero no se transforman siguiendo un proceso de descomposición y muerte celular.
\vs p014 4:10 \pc Los originarios de Havona son todos vástagos de la Trinidad del Paraíso. No tienen progenitores y no se reproducen. No podemos describir la creación de estos ciudadanos del universo central porque son seres que no fueron creados. La historia completa de la creación de Havona responde a un intento de poner en términos espacio\hyp{}temporales un hecho de la eternidad, que no tiene ninguna relación con el tiempo ni con el espacio, tal como el hombre mortal los entiende. Pero debemos conceder a la filosofía humana un punto de origen ya que incluso los seres personales que están muy por encima del nivel humano requieren un concepto de “principio”. Sin embargo, el sistema Paraíso\hyp{}Havona es eterno.
\vs p014 4:11 Los originarios de Havona viven en los mil millones de esferas del universo central de la misma manera en la que otros órdenes de ciudadanía habitan de forma permanente en sus respectivas esferas de origen. Tal como el orden de hijos materiales se ocupa de los recursos materiales, intelectuales y espirituales de mil millones de sistemas locales dentro de un suprauniverso, del mismo modo, en un sentido más amplio, los originarios de Havona viven y obran en los mil millones de mundos del universo central. Quizás pudierais considerar a estos habitantes de Havona como criaturas materiales si la palabra “material” pudiera ampliar su significado para poder describir las realidades físicas del universo divino.
\vs p014 4:12 Existe una vida que es oriunda de Havona y que posee significación en sí y a partir de sí misma. Los originarios de Havona sirven de muchas maneras a los que descienden del Paraíso y a los que ascienden de los suprauniversos, pero también viven una vida única en el universo central que, con independencia del Paraíso o de los suprauniversos, tiene un sentido propio.
\vs p014 4:13 Así como la adoración de los hijos de fe en los mundos evolutivos sirve para satisfacción del amor del Padre Universal, del mismo modo la sublime adoración de las criaturas de Havona contenta los perfectos ideales de belleza y de verdad divinas. Así como el hombre mortal se afana por hacer la voluntad de Dios, estos seres del universo central viven para complacer los ideales de la Trinidad del Paraíso. En su naturaleza misma, \bibemph{son} la voluntad de Dios. El hombre se regocija en la bondad de Dios, los habitantes de Havona gozan de la belleza divina, mientras que ambos disfrutáis del ministerio liberador de la verdad viva.
\vs p014 4:14 Los originarios de Havona tienen destinos no revelados, tanto presentes como futuros, a los que pueden optar. Y existe entre estas criaturas una forma de desarrollo peculiar al universo central, un desarrollo que no incluye ni el ascenso al Paraíso ni la incursión en los suprauniversos. Este desarrollo a un estatus más elevado en Havona puede indicarse de la siguiente manera:
\vs p014 4:15 \li{1.}Desarrollo experiencial hacia fuera, desde la primera hasta la última vía circulatoria.
\vs p014 4:16 \li{2.}Desarrollo hacia dentro desde la séptima hasta la primera vía circulatoria.
\vs p014 4:17 \li{3.}Desarrollo entre las vías orbitales: desarrollo dentro de los mundos de una vía circulatoria determinada.
\vs p014 4:18 \pc Además de los originarios de Havona, entre los habitantes del universo central hay numerosas clases de seres que conforman los modelos para diversos grupos del universo ---consejeros, directores y maestros de su tipo y para su tipo en toda la creación---. Todos los seres en todos los universos se modelan según algún orden de modelo creatural que vive en uno de los mil millones de mundos de Havona. Incluso los mortales del tiempo tienen su meta e ideales en cuanto a su existencia creatural en las vías planetarias exteriores de estas esferas modelos de las alturas.
\vs p014 4:19 Luego existen esos seres que han alcanzado al Padre Universal, y que tienen derecho a ir y venir, que están asignados por doquier en los universos, en misiones de servicio especial. Y en cada mundo de Havona se hallan quienes pretenden llegar hasta el final, quienes físicamente han llegado al universo central, pero no han logrado aún ese desarrollo espiritual que les permitirá solicitar residencia en el Paraíso.
\vs p014 4:20 Hay un gran número de seres personales que representan al Espíritu Infinito en los mundos de Havona; son seres de gracia y gloria que rigen los intrincados asuntos intelectuales y espirituales del universo central. En estos mundos de perfección divina, realizan una labor connatural a la conducta normal de esta inmensa creación y, además, llevan a cabo múltiples tareas de enseñanza, formación y servicio en favor del enorme número de criaturas ascendentes que han escalado a la gloria desde los oscuros mundos del espacio.
\vs p014 4:21 Existen numerosos grupos de seres originarios del sistema Paraíso\hyp{}Havona que no están de modo alguno vinculado directamente con el plan de ascensión de las criaturas en su logro de la perfección; así pues, se han omitido de la clasificación de los seres personales dada a conocer a las razas mortales. Aquí se exponen solamente los grupos principales de seres sobrenaturales y aquellos órdenes de seres directamente relacionados con vuestra experiencia de supervivencia.
\vs p014 4:22 Havona está repleta de la vida de todos las variedades de seres inteligentes, que buscan allí avanzar desde las vías circulatorias inferiores hasta las superiores y se esfuerzan por alcanzar niveles cada vez más elevados de cognición de la divinidad y una percepción más amplia de los contenidos supremos, de los valores últimos y de la realidad absoluta.
\usection{5. LA VIDA EN HAVONA}
\vs p014 5:1 En Urantia pasáis por una prueba breve e intensa durante vuestra vida inicial en la existencia material. En los mundos de las moradas y a través de vuestro sistema, constelación y universo local, ascendéis a través de las fases morontiales. En los mundos de formación del suprauniverso pasáis por las verdaderas etapas de desarrollo espiritual y se os prepara para el tránsito final a Havona. En las siete vías de Havona, vuestra realización es intelectual, espiritual y experiencial. Y en cada uno de los mundos de cada una de estas vías ha de realizarse una tarea específica.
\vs p014 5:2 La vida en los mundos divinos del universo central es tan rica y plena, tan completa y pletórica, que trasciende del todo lo que un ser creado pudiera imaginar sentir. Las actividades sociales y económicas de esta creación eterna son enteramente distintas a las ocupaciones de las criaturas materiales que viven en mundos evolutivos como Urantia. Incluso el método de pensamiento en Havona difiere del proceso de pensamiento en Urantia.
\vs p014 5:3 Las reglas en el universo central son adecuadas e intrínsecamente naturales; las normas de conducta no son arbitrarias. Tras cada imperativo exigido en Havona se desvela la razón de la rectitud y el reinado de la justicia. Y estos dos factores, combinados, equivalen a lo que en Urantia se denomina \bibemph{ecuanimidad}. Cuando lleguéis a Havona, disfrutaréis haciendo de manera natural las cosas del modo en el que deben hacerse.
\vs p014 5:4 \pc Cuando los seres inteligentes alcanzan por primera vez el universo central, se les recibe y alberga en el mundo piloto de la séptima vía circulatoria de Havona. Cuando los recién llegados progresan espiritualmente y consiguen comprender la identidad del espíritu mayor de su suprauniverso, se les transfiere al sexto círculo (a partir de esta distribución en el universo central se han asignado los círculos de progreso de la mente humana). Una vez que los seres ascendentes han logrado reconocer la presencia de la Supremacía y están por tanto preparados para la aventura de la Deidad, se trasladan a la quinta vía; y, después de reconocer la presencia del Espíritu Infinito, se les transfiere a la cuarta. Tras alcanzar el Hijo Eterno, se las traslada a la tercera; y, cuando han reconocido la presencia del Padre Universal, proceden a la segunda vía circulatoria de los mundos donde se familiarizan más con las multitudes de seres del Paraíso. La llegada de los candidatos del tiempo a la primera vía de Havona significa su aceptación para servir en el Paraíso. Según la duración y naturaleza de su ascensión como criaturas, los candidatos permanecen en el círculo interior continuando, de forma indefinida, con su progreso espiritual. Desde este círculo interior, los peregrinos ascendentes proseguirán hacia dentro para obtener la residencia en el Paraíso y su admisión en el colectivo final.
\vs p014 5:5 Durante tu estancia en Havona como peregrino en ascenso, se te permitirá visitar con libertad los mundos de la vía planetaria que se te han asignado. También se te permitirá regresar a los planetas de aquellas vías que ya hayas atravesado. Todo esto es posible para los que residen en los círculos de Havona sin tener que ser transportados por un supernafín. Los peregrinos del tiempo pueden equiparse para cruzar el espacio “alcanzado”, pero han de depender del procedimiento establecido para franquear el espacio “no alcanzado”. Un peregrino no puede salir de Havona ni ir más allá de la vía asignada sin la ayuda de un supernafín transportador.
\vs p014 5:6 \pc Existe en esta inmensa creación central una singularidad que resulta reconfortante. Aparte de la organización física de la materia y de la constitución innata de los órdenes esenciales de seres inteligentes y de otros seres vivos, los mundos de Havona no tienen nada en común. Cada uno de estos planetas es una creación singular, única y exclusiva; cada planeta ha sido formado de una manera inigualable, espléndida y perfecta. Y esta diversidad individual, que caracteriza a estos planetas, se extiende a todos los aspectos físicos, intelectuales y espirituales. Cada uno de estos mil millones de esferas perfectas se ha desarrollado y embellecido conforme a los planes del eterno de días que allí reside. Y es precisamente por esto por lo que no hay dos esferas iguales.
\vs p014 5:7 Hasta que no hayas recorrido la última de las vías circulatorias de Havona y visitado el último de los mundos de Havona no desaparecerán de tu camino el incentivo hacia la aventura y el estímulo hacia la curiosidad. Y entonces, el impulso, el ímpetu de seguir adelante hacia la eternidad, reemplazará a lo anterior, al aliciente hacia la aventura del tiempo.
\vs p014 5:8 La monotonía indica la inmadurez de la imaginación creativa y la inactividad en cuanto a la coordinación intelectual con la dotación espiritual. Cuando un mortal que asciende comienza la exploración de estos mundos celestiales, ya ha alcanzado la madurez emocional, intelectual y social, e incluso la espiritual.
\vs p014 5:9 No solo te enfrentarás a cambios inimaginables a medida que avances en Havona de vía en vía, sino que tu asombro será indescriptible conforme progreses de planeta en planeta dentro de cada vía. Cada uno de estos mil millones de mundos de aprendizaje es una verdadera fábrica de sorpresas. El continuo asombro, la interminable maravilla, constituye la experiencia de los que recorren estas vías y visitan estas gigantescas esferas. La monotonía no forma parte de tu camino en Havona.
\vs p014 5:10 El amor por la aventura, la curiosidad y el temor a la monotonía ---esas características innatas en la naturaleza evolutiva del hombre--- no están ahí tan solo para exasperarte y perturbarte durante tu breve estancia en la tierra, sino más bien para sugerirte que la muerte es tan solo el comienzo de un interminable camino en la aventura, de una vida perpetua en la expectación, de un viaje eterno en el descubrimiento.
\vs p014 5:11 La curiosidad ---la tendencia hacia el análisis, el impulso hacia el descubrimiento, el estímulo hacia la exploración--- forma parte de la dotación innata y divina de las criaturas evolutivas del espacio. No se te dieron estos deseos naturales simplemente para que los coartaras y reprimieras. Es verdad que este impulso hacia lo ideal debe contenerse durante tu corta vida en la tierra y, a veces, se debe experimentar la decepción, pero se realizarán en plenitud y se satisfarán con gloria durante las largas eras por venir.
\usection{6. EL PROPÓSITO DEL UNIVERSO CENTRAL}
\vs p014 6:1 En las siete vías planetarias de Havona se desarrolla un enorme ámbito de actividad. De manera general, este se puede describir como:
\vs p014 6:2 \li{1.}Havonal.
\vs p014 6:3 \li{2.}Paradisíaca.
\vs p014 6:4 \li{3.}Ascendente\hyp{}finita ---evolutiva suprema\hyp{}última---.
\vs p014 6:5 \pc En la presente era del universo, tiene lugar en Havona bastante actividad suprafinita, incluyendo una inenarrable diversidad de etapas absonitas y de otras facetas de las actuaciones de la mente y del espíritu. Es posible que el universo central sirva muchos propósitos que no me han sido revelados, ya que realiza numerosas funciones que están más allá de la comprensión de la mente creada. No obstante, intentaré describir de qué manera se ocupa esta creación perfecta de las necesidades de los siete órdenes de inteligencia del universo y contribuye a su satisfacción.
\vs p014 6:6 \li{1.}\bibemph{El Padre Universal:} la Primera Fuente y Centro. La perfección de la creación central proporciona al Dios Padre una suprema satisfacción paterna. Goza con la experiencia de la plenitud del amor en términos cercanos a la correspondencia. Al Creador perfecto le complace divinamente la adoración de la criatura perfecta.
\vs p014 6:7 Havona permite al Padre conseguir una gratificación suprema. La realización de la perfección en Havona compensa la demora espacio\hyp{}temporal en la expansión infinita del impulso hacia la eternidad.
\vs p014 6:8 El Padre disfruta de la respuesta de Havona a la belleza divina. Le es de satisfacción a la mente divina ofrecer a todos los universos en evolución un modelo de perfección de excelente armonía.
\vs p014 6:9 Nuestro Padre contempla el universo central con plenitud de gozo porque es una meritoria revelación de la realidad espiritual para todos los seres personales del universo de los universos.
\vs p014 6:10 El Dios de los universos tiene en favorable consideración a Havona y al Paraíso como el núcleo eterno de la potencia para toda la posterior expansión universal en el tiempo y en el espacio.
\vs p014 6:11 El Padre eterno contempla con ilimitada satisfacción la creación de Havona como meta digna y fascinante para los candidatos que ascienden del tiempo, sus nietos mortales del espacio en su logro de la residencia eterna de su Padre\hyp{}Creador. Y Dios se complace en el universo Paraíso\hyp{}Havona como la eterna residencia de la Deidad y de la familia divina.
\vs p014 6:12 \li{2.}\bibemph{El Hijo Eterno:} la Segunda Fuente y Centro. Para el Hijo Eterno, la espléndida creación central ofrece la eterna evidencia de los fuertes vínculos existentes en la familia divina: el Padre, el Hijo y el Espíritu. Constituye el fundamento espiritual y material para la confianza absoluta en el Padre Universal.
\vs p014 6:13 Havona proporciona al Hijo Eterno una base casi ilimitada para llevar a cabo la constante expansión de la potencia espiritual. El universo central proporcionó al Hijo Eterno el escenario en el que pudo demostrar de manera segura y cierta la esencia y el proceder respecto a la instrucción de sus hijos del Paraíso, a él vinculados, en su ministerio de gracia.
\vs p014 6:14 Havona constituye el fundamento de la realidad para el Hijo Eterno, en cuanto al dominio de la gravedad espiritual que ejerce en el universo de los universos. Este universo proporciona al Hijo la gratificación de su anhelo paterno, de la reproducción espiritual.
\vs p014 6:15 Los mundos de Havona y sus habitantes perfectos constituyen la demostración primera y concluyente en la eternidad de que el Hijo es el Verbo del Padre. De este modo, se gratifica de forma perfecta la conciencia del Hijo como el infinito complementario del Padre.
\vs p014 6:16 Y este universo ofrece la oportunidad de realizar de forma recíproca la fraternidad en igualdad entre el Padre Universal y el Hijo Eterno, y esto constituye la prueba perpetua de la infinitud del ser personal de cada uno de ellos.
\vs p014 6:17 \li{3.}\bibemph{El Espíritu Infinito:} la Tercera Fuente y Centro. El universo de Havona proporciona al Espíritu Infinito la prueba de ser el Actor Conjunto, la infinita representación de la unificación del Padre\hyp{}Hijo. En Havona, el Espíritu Infinito siente la satisfacción conjunta de llevar a cabo su actividad creativa mientras goza de la satisfacción de la coexistencia absoluta con este logro divino.
\vs p014 6:18 En Havona, el Espíritu Infinito encontró el escenario para demostrar su capacidad y deseo de obrar como benefactor potencial de la misericordia. En esta creación perfecta, el Espíritu se preparó para su trepidante ministerio en los universos evolutivos.
\vs p014 6:19 Esta creación perfecta proporcionó al Espíritu Infinito la oportunidad de participar en la administración del universo con sus dos progenitores divinos ---para regir un universo como vástago y creador\hyp{}compañero, preparándose así para la administración conjunta de los universos locales bajo la forma de los espíritus creativos colaboradores de los hijos creadores---.
\vs p014 6:20 Los mundos de Havona constituyen un laboratorio en el terreno de la mente para los creadores de la mente cósmica y los asesores de esta para todas las criaturas existentes. La mente es diferente en cada mundo de Havona y sirve de modelo para todos los intelectos de las criaturas espirituales y materiales.
\vs p014 6:21 Estos mundos perfectos son las escuelas graduadas de la mente para todos los seres destinados a ser parte de la sociedad del Paraíso. Proporcionaron al Espíritu una gran oportunidad para probar su método de impartir el ministerio de la mente en seres personales seguros y dignos de confianza.
\vs p014 6:22 Havona compensa al Espíritu Infinito por su inmensa y desinteresada labor en los universos del espacio. Havona es el perfecto lugar de residencia y retiro para el infatigable benefactor de la mente del tiempo y del espacio.
\vs p014 6:23 \li{4.}\bibemph{El Ser Supremo:} la unificación evolutiva de la Deidad experiencial. La creación de Havona representa la prueba eterna y perfecta de la realidad espiritual del Ser Supremo. Esta creación perfecta es una revelación de la naturaleza espiritual perfecta y proporcionada del Dios Supremo antes de emprender la síntesis de la potencia\hyp{}ser personal entre los reflejos finitos de las Deidades del Paraíso y los universos experienciales del tiempo y del espacio.
\vs p014 6:24 En Havona los potenciales de fuerza del Todopoderoso se unifican con la naturaleza espiritual del Supremo. Esta creación central constituye un ejemplo de la futura unidad eterna del Supremo.
\vs p014 6:25 Havona constituye el modelo perfecto del potencial y universalidad del Supremo. Este universo representa de forma final la perfección futura del Supremo e indica el potencial del Último.
\vs p014 6:26 Havona pone de manifiesto la completud de los valores espirituales existentes en las criaturas vivas de voluntad con un supremo y perfecto dominio de sí mismas, en una mente que existe como equivalente último del espíritu con una realidad y una unidad intelectual potencialmente ilimitadas.
\vs p014 6:27 \li{5.}\bibemph{Los hijos creadores de igual rango}. Havona es el campo de entrenamiento en el que los migueles del Paraíso se preparan para sus posteriores aventuras en la creación de los universos. Esta creación perfecta y divina constituye el modelo para cada hijo creador, que se esfuerza por lograr que su propio universo acabe por alcanzar esos niveles de perfección del Paraíso\hyp{}Havona.
\vs p014 6:28 Los hijos creadores se benefician de las criaturas de Havona por su capacidad de ser modelos del ser personal para sus propios hijos mortales y seres espirituales. Los migueles, junto con otros hijos del Paraíso, consideran el Paraíso y Havona como el destino divino de los hijos del tiempo.
\vs p014 6:29 Los hijos creadores saben que la creación central es la verdadera fuente de esa indispensable acción directiva sobre el universo que estabiliza y unifica sus universos locales. Saben que en Havona está la presencia personal de la siempre presente influencia del Supremo y del Último.
\vs p014 6:30 Havona y el Paraíso constituyen la fuente del poder creativo de un hijo miguel. Aquí moran los seres que cooperan con él en la creación de los universos. Del Paraíso proceden los espíritus maternos de los Universos, los cocreadores de los universos locales.
\vs p014 6:31 Los hijos del Paraíso consideran la creación central como el hogar de sus divinos padres, como su propio hogar. Es el sitio al que les gusta regresar periódicamente.
\vs p014 6:32 \li{6.}\bibemph{Las hijas servidoras de igual rango}. Los espíritus maternos de los universos, cocreadores de los universos locales, obtienen su formación de carácter prepersonal en los mundos de Havona en estrecha vinculación con los espíritus de las vías. En el universo central, las hijas espirituales de los universos locales fueron debidamente instruidas para poder cooperar con los hijos del Paraíso, siempre sujetos a la voluntad del Padre.
\vs p014 6:33 En los mundos de Havona, el Espíritu y las hijas del Espíritu encuentran los modelos de mentes para todos sus grupos de inteligencias espirituales y materiales, y este universo central es el destino futuro de esas criaturas que un espíritu materno del universo auspicia en conjunto con su compañero, un hijo creador.
\vs p014 6:34 La creadora materna del universo recuerda el Paraíso y Havona como el lugar de su origen y el lugar de residencia del Espíritu Materno Infinito, morada de la presencia personal de la Mente Infinita.
\vs p014 6:35 De este universo central también procede la concesión de las prerrogativas personales de creación que una benefactora divina del universo emplea para complementar la labor de un hijo creador en su creación de criaturas vivas de voluntad.
\vs p014 6:36 Y finalmente, al ser probable que estos espíritus hijas del Espíritu Materno Infinito no regresen jamás a su lugar de residencia en el Paraíso, encuentran una gran satisfacción en el fenómeno universal de la reflectividad vinculado con el Ser Supremo en Havona y manifestado como ser personal en Majestón en el Paraíso.
\vs p014 6:37 \li{7.}\bibemph{Los mortales evolutivos en su camino de ascensión}. Havona es el lugar de residencia del modelo del ser personal para todas las clases de mortales y el lugar de residencia de todos los seres personales sobrenaturales vinculados a los mortales que no son oriundos de las creaciones del tiempo.
\vs p014 6:38 Estos mundos son un estímulo para el impulso que incita al ser humano a lograr los verdaderos valores espirituales en los más altos niveles concebibles de la realidad. Havona constituye la antesala del Paraíso en la formación de cada uno de los mortales ascendentes. Aquí los mortales encuentran la Deidad preparadisíaca ---el Ser Supremo---. Havona se erige ante todas las criaturas de voluntad como la vía de acceso al Paraíso y a la consecución de Dios.
\vs p014 6:39 El Paraíso es el lugar de residencia y Havona el taller y lugar de recreo para los que llegan hasta el final. Y todo mortal que conozca a Dios anhela ser un finalizador.
\vs p014 6:40 El universo central no es solamente el destino establecido del hombre, sino que es también el punto de partida en la eterna andadura de los finalizadores hasta que alguna vez emprendan la aventura universal no revelada: la experiencia de explorar la infinitud del Padre Universal.
\vs p014 6:41 \pc Es incuestionable que Havona seguirá en su función y significación absonita incluso en las eras futuras del universo, que quizás sean testigos del intento de los peregrinos del espacio por encontrar a Dios en niveles suprafinitos. Havona tiene la capacidad de servir como universo de formación para seres absonitos. Probablemente, será el centro de instrucción final al igual que los siete suprauniversos sirven de centros de instrucción de nivel medio para los graduados de los centros de educación primaria del espacio exterior. Y tendemos a opinar que la eterna Havona tiene un potencial realmente ilimitado, que el universo central está capacitado eternamente para servir como universo de formación experiencial para todas las clases de seres creados, tanto del pasado y del presente como del futuro.
\vsetoff
\vs p014 6:42 [Exposición de un perfeccionador de la sabiduría, a quien los ancianos de días de Uversa le han encomendado esta labor.]
