\upaper{142}{En la Pascua de Jerusalén}
\author{Comisión de seres intermedios}
\vs p142 0:1 Durante el mes de abril, Jesús y los apóstoles realizaron su labor en Jerusalén, aunque dejaban la ciudad al caer la tarde para pernoctar en Betania. Jesús, por su parte, pasaba una o dos noches semanales en Jerusalén en la casa de Flavio, un judío griego. Allí acudían en secreto muchos judíos prominentes para entrevistarse con él.
\vs p142 0:2 \pc El primer día en Jerusalén, Jesús fue a ver a Anás, un amigo de años pasados, antiguo sumo sacerdote y pariente de Salomé, la esposa de Zebedeo. Anás había oído hablar de Jesús y de sus enseñanzas, y cuando Jesús llegó a la casa del sumo sacerdote, se le recibió con mucha cautela. Cuando Jesús notó la frialdad de Anás, se despidió de él de inmediato, diciéndole al partir: “El miedo es el mayor tirano del hombre y, el orgullo, su gran debilidad; ¿te traicionarás a ti mismo entregándote a la esclavitud de estos dos asoladores de la felicidad y de la libertad?”. Pero Anás no contestó. El Maestro no volvió a verlo hasta el momento en el que él y su yerno hicieron juicio contra el Hijo del Hombre.
\usection{1. ENSEÑANZA EN EL TEMPLO}
\vs p142 1:1 Durante todo este mes, Jesús o alguno de los apóstoles enseñaban diariamente en el templo. Cuando las multitudes que asistían a la Pascua eran demasiado grandes para entrar en el templo y oír las enseñanzas, los apóstoles se llevaban a muchos de estos grupos fuera de los recintos sagrados. El núcleo de su mensaje era:
\vs p142 1:2 \li{1.}El reino de los cielos se ha acercado.
\vs p142 1:3 \li{2.}Por medio de la fe en la paternidad de Dios, podéis entrar en el reino de los cielos y convertiros en los hijos de Dios.
\vs p142 1:4 \li{3.}El amor es la regla de la vida en el seno del reino ---devoción suprema a Dios a la vez que amáis a vuestro prójimo como a vosotros mismos---.
\vs p142 1:5 \li{4.}La obediencia a la voluntad del Padre, que rinde los frutos del espíritu en la vida personal, es la ley del reino.
\vs p142 1:6 \pc Las multitudes que venían a celebrar la Pascua oyeron estas enseñanzas de Jesús, y centenares de ellos se regocijaron de la buena nueva. Ante esto, los sumos sacerdotes y los dignatarios de los judíos se inquietaron bastante por Jesús y sus apóstoles y comenzaron a debatir entre ellos qué se podría hacer al respecto.
\vs p142 1:7 Además de enseñar dentro y en las inmediaciones del templo, los apóstoles y otros creyentes realizaban una gran labor personal entre las multitudes llegadas para la Pascua. Estos hombres y mujeres llevaron con interés la nueva del mensaje de Jesús desde este lugar de celebración pascual hasta las zonas más distantes del Imperio romano y también al este, lo que significó el comienzo de la difusión del evangelio del reino al mundo exterior. La labor de Jesús dejaría de circunscribirse solamente a Palestina.
\usection{2. LA IRA DE DIOS}
\vs p142 2:1 Había un tal Santiago que se encontraba en Jerusalén para asistir a las festividades de la Pascua. Santiago era un rico comerciante judío procedente de Creta, que había acudido a ver a Andrés para solicitarle una reunión con Jesús en privado. Andrés organizó ese encuentro secreto con Jesús en la casa de Flavio para que tuviese lugar al día siguiente por la noche. Este hombre no podía comprender las enseñanzas del Maestro, y había venido a indagar más sobre el reino de Dios. Santiago le dijo a Jesús: “Pero, Rabí, Moisés y los profetas de ataño nos dicen que Yahvé es un Dios celoso, un Dios de gran ira y furor. Los profetas dicen que odia a los malhechores y cobra venganza de quienes no obedecen su ley. Tú y tus discípulos nos enseñáis que Dios es un Padre generoso y compasivo, que ama tanto a todos los hombres que los acogería complacientemente en este nuevo reino de los cielos, el cual, como proclamas, está tan cerca”.
\vs p142 2:2 \pc Cuando Santiago acabó de hablar, Jesús le respondió: “Santiago, has expuesto con claridad las enseñanzas de los antiguos profetas que instruían a los hijos de su generación de acuerdo a la luz de su día. Nuestro Padre del Paraíso es invariable. Pero el concepto de su naturaleza se ha ampliado y ha madurado desde la época de Moisés, pasando por el tiempo de Amós, hasta llegar incluso a la generación del profeta Isaías. Y, ahora, yo he venido en la carne para revelar al Padre en nueva gloria y manifestar su amor y misericordia hacia todos los hombres de todos los mundos. Conforme el evangelio de este reino se disemine por el mundo con su mensaje de gozo espiritual y buena voluntad para todos los hombres, las relaciones entre las familias de todas las naciones se harán más perfectas y mejores. Conforme pase el tiempo, los padres y sus hijos se amarán más unos a otros y se propiciará, pues, un mayor entendimiento del amor del Padre de los cielos por sus hijos en la tierra. Recuerda, Santiago, que un padre, verdadero y bueno, no solo ama a su familia como un todo ---como una familia--- sino que también ama verdaderamente y se preocupa con cariño de \bibemph{cada uno} de los miembros de esta”.
\vs p142 2:3 Tras una prolongada conversación sobre el carácter del Padre celestial, Jesús se detuvo para decir: “Tú, Santiago, siendo padre de muchos, conoces bien la verdad de mis palabras”. Y Santiago dijo: “Pero Maestro, ¿quién te dijo que soy padre de seis hijos? ¿Cómo sabías esto de mí?”. Y el Maestro le respondió: “Baste decir que el Padre y el Hijo conocen todas las cosas, porque de cierto lo ven todo. Como padre, al amar en la tierra a tus hijos, debes aceptar ahora como una realidad el amor del Padre celestial hacia \bibemph{ti} ---no solamente hacia todos los hijos de Abraham, sino hacia ti, a tu alma particular”---.
\vs p142 2:4 \pc Jesús continuó diciendo: “Cuando tus hijos son muy jóvenes e inmaduros, y cuando debes castigarlos, quizás imaginen que su padre está furioso y lleno de rencor e ira. Su inmadurez no les permite ver más allá del castigo y reconocer el cariño previsor y correctivo del padre. Pero cuando estos mismos hijos se convierten en hombres y mujeres adultos, ¿no sería necio que se aferrasen a estas nociones erróneas que tenían de él con anterioridad? Como hombres y mujeres, deben saber ahora percibir el amor de su padre en estas primeras restricciones. ¿Y no debería la humanidad, conforme trascurren los siglos, llegar a un mejor entendimiento de la verdadera naturaleza y del carácter amoroso del Padre de los cielos? ¿Qué beneficios obtenéis de generaciones seguidas aportando su luz espiritual si insistís en ver a Dios tal como Moisés y los profetas lo veían? Te digo, Santiago, que, a la luz brillante de esta hora, deberías ver al Padre como nunca nadie antes de ti lo contempló. Y, al verlo de este modo, deberías regocijarte de entrar en el reino en donde tal misericordioso Padre gobierna y procurar que su voluntad amorosa domine tu vida a partir de ahora”.
\vs p142 2:5 Y Santiago contestó: “Rabí, yo creo; deseo que me lleves al reino del Padre”.
\usection{3. CONCEPTO DE DIOS}
\vs p142 3:1 Esa noche, los doce apóstoles, la mayoría de los cuales había escuchado este comentario acerca del carácter de Dios, le hicieron a Jesús muchas preguntas sobre el Padre de los cielos. Las respuestas del Maestro a estas se pueden exponer mejor realizando, en términos modernos, el siguiente resumen:
\vs p142 3:2 Jesús reprendió levemente a los doce, diciéndoles esencialmente: ¿Es que no conocéis las tradiciones de Israel respecto al desarrollo de la idea de Yahvé e ignoráis además las enseñanzas de las Escrituras sobre la doctrina de Dios? Y, entonces, el Maestro procedió a instruir a los apóstoles sobre la evolución del concepto de la Deidad a lo largo de toda la historia del pueblo judío. Destacó las siguientes facetas del crecimiento de la idea de Dios:
\vs p142 3:3 \li{1.}\bibemph{Yahvé:} el dios de los clanes del Sinaí. Se trataba del concepto primitivo de la Deidad, que Moisés enalteció hasta el nivel superior de Señor Dios de Israel. El Padre de los cielos jamás deja de aceptar la adoración sincera de sus hijos de la tierra por tosco que sea su concepto de la Deidad, y sin importarle el nombre con el que simbolicen la naturaleza divina.
\vs p142 3:4 \li{2.}\bibemph{El Altísimo}. Melquisedec dio a conocer a Abraham este nombre del padre de los cielos, que se extendió lejos de Salem gracias a quienes después creyeron en esta idea elevada y ampliada de la Deidad. Abraham y su hermano se fueron de Ur por la instauración allí de la adoración al sol, y se convirtieron en creyentes de las enseñanzas de Melquisedec sobre El Elyón ---el Dios Altísimo---. Tenían un concepto compuesto de Dios, que consistía en una mezcla de sus antiguas ideas mesopotámicas y de la doctrina del Altísimo.
\vs p142 3:5 \li{3.}\bibemph{El Shaddai}. Durante estos tiempos primitivos muchos de los hebreos adoraban a El Shaddai, el concepto egipcio del Dios de los cielos, adquirido durante su cautiverio en la tierra del Nilo. Mucho después de los tiempos de Melquisedec, estos tres conceptos de Dios se convirtieron en uno, conformándose así la doctrina de la Deidad creadora, el Señor Dios de Israel.
\vs p142 3:6 \li{4.}\bibemph{Elohim}. Desde los tiempos de Adán subsiste la enseñanza de la Trinidad del Paraíso. ¿Recordáis cómo las Escrituras comienzan declarando que “en el principio crearon los Dioses los cielos y la tierra”? Esto indica que cuando quedó constancia de esta afirmación, el concepto de la Trinidad, de tres Dioses en uno, había hallado cabida en la religión de nuestros ancestros.
\vs p142 3:7 \li{5.}\bibemph{El supremo Yahvé}. En los tiempos de Isaías, estas creencias sobre Dios se habían expandido hasta convertirse en el concepto de un Creador Universal que era a la vez todopoderoso y todo misericordioso. Y este concepto de Dios, que se fue desarrollando y engrandeciendo, reemplazó prácticamente todas las ideas anteriores sobre la Deidad de la religión de nuestros padres.
\vs p142 3:8 \li{6.}\bibemph{El Padre de los cielos}. Y ahora conocemos a Dios como nuestro Padre de los cielos. Nuestras enseñanzas conforman una religión en la que el creyente \bibemph{es} hijo de Dios. En esto consiste la buena nueva del evangelio del reino de los cielos. En coexistencia con el Padre están el Hijo y el Espíritu, y la revelación de la naturaleza y el ministerio de estas Deidades del Paraíso continuará magnificándose e iluminándose, conforme los hijos ascendentes de Dios progresan espiritualmente a través de las eras de la eternidad. En todos los tiempos y durante todas las eras, el espíritu interior reconoce la genuina adoración de cualquier ser humano ---en lo concerniente al progreso espiritual individual--- como un homenaje que se tributa al Padre de los cielos.
\vs p142 3:9 \pc Nunca antes había impresionado nada tanto a los apóstoles como este relato sobre el desarrollo del concepto de Dios en las mentes judías de generaciones anteriores; se quedaron demasiado desconcertados como para hacer preguntas. Mientras estaban sentados ante Jesús en silencio, el Maestro continuó: “Y habríais conocido estas verdades si hubieseis leído las Escrituras. ¿No habéis leído en Samuel el pasaje que dice: ‘Y se encendió la ira del Señor contra Israel, e incitó a David contra ellos, diciéndole: ve y haz un censo de Israel y Judá’? Y aquello no resultaba extraño porque en los días de Samuel los hijos de Abraham creían ciertamente que Yahvé creaba tanto el bien como el mal. Pero cuando, más tarde, un escritor narró estos acontecimientos, tras la ampliación del concepto judío sobre la naturaleza de Dios, no se atrevió atribuir el mal a Yahvé, sino que dijo entonces: ‘Y se levantó Satanás contra Israel e incitó a David a que hiciera censo de Israel’ ¿Es que no sois capaces de percibir que tales pasajes de las Escrituras muestran con claridad cómo el concepto de la naturaleza de Dios continuó evolucionando de una generación a la otra?
\vs p142 3:10 “Asimismo, deberíais haber observado el desarrollo del entendimiento de la ley divina en consonancia perfecta con estos conceptos magnificados de la divinidad. Cuando los hijos de Israel salieron de Egipto, en esos días anteriores a esta ampliación de la revelación de Yahvé, tenían diez mandamientos que les sirvieron de ley hasta el momento en el que acamparon ante el Sinaí. Estos eran:
\vs p142 3:11 “1. No adorarás a ningún otro dios, pues el Señor es un Dios celoso.
\vs p142 3:12 “2. No te harás dioses de fundición.
\vs p142 3:13 “3. No dejarás de guardar la fiesta de los Panes sin Levadura.
\vs p142 3:14 “4. Todos los primogénitos varones tanto de hombres como de animales míos serán, dice el Señor.
\vs p142 3:15 “5. Seis días trabajarás, pero en el séptimo día descansarás.
\vs p142 3:16 “6. No dejarás de celebrar la fiesta de los Primeros Frutos y la fiesta de la Cosecha a la salida del año.
\vs p142 3:17 “7. No ofrecerás con pan leudado la sangre de ningún sacrificio.
\vs p142 3:18 “8. No dejarás hasta la mañana el sacrificio de la fiesta de Pascua.
\vs p142 3:19 “9. Traerás a la casa del Señor tu Dios las primicias de los primeros frutos de la tierra.
\vs p142 3:20 “10. No guisarás el cabrito en la leche de su madre.
\vs p142 3:21 \pc “Y luego, en medio de los truenos y relámpagos del Sinaí, Moisés les entregó los nuevos diez mandamientos, que, como todos admitiréis, son expresiones más meritorias para servir de acompañamiento a los conceptos yahvistas en su enaltecimiento de la Deidad. ¿Y es que nunca prestasteis atención en estos mandamientos, tal como se enuncia doblemente en las Escrituras, el hecho de que primeramente se atribuye a la liberación de Egipto la razón para guardar el \bibemph{sabbat,} mientras que luego se hace constar que el avance de las creencias religiosas de nuestros ancestros hizo necesario que esto variara y se reconociera la creación como la razón para observar el \bibemph{sabbat?}
\vs p142 3:22 “Y luego, recordaréis que de nuevo ---en la época de mayor lucidez espiritual de los días de Isaías--- estos diez mandamientos negativos se modificaron para expresar la ley, positiva y grande, del amor, por el mandato de amar a Dios sobre todas las cosas y a vuestro prójimo como a vosotros mismos. Y es esta ley suprema del amor por Dios y por el hombre la que yo os proclamo que constituye el deber total del hombre”.
\vs p142 3:23 \pc Y cuando acabó de hablar, nadie le preguntó nada. Cada cual se retiró a descansar.
\usection{4. FLAVIO Y LA CULTURA GRIEGA}
\vs p142 4:1 Flavio, el judío griego, era un prosélito de la puerta, que no estaba circuncidado ni bautizado; y, puesto que era un gran amante de la belleza en el arte y en la escultura, la vivienda que ocupaba durante su estancia en Jerusalén era un hermoso edificio. La casa estaba exquisitamente adornada con invaluables tesoros, que él había acumulado de sus viajes por el mundo. Cuando pensó primeramente en invitar a Jesús a su casa, temía que el Maestro se sintiese ofendido al ver tantas de las llamadas imágenes. Pero Flavio se sorprendió gratamente cuando Jesús entró en ella y, en lugar de reprenderle por poseer estos objetos supuestamente idólatras distribuidos por la casa, manifestó gran interés por toda la colección e hizo muchas preguntas elogiosas sobre cada objeto, a medida que Flavio lo acompañaba de habitación en habitación, mostrándole sus estatuas favoritas.
\vs p142 4:2 El Maestro notó que su anfitrión estaba perplejo por su actitud positiva hacia el arte; y, por lo tanto, cuando habían acabado de contemplar la colección completa, Jesús dijo: “Si aprecias la belleza de las cosas creadas por mi Padre y modeladas por las manos artísticas del hombre, ¿por qué has de esperar mi reproche? Porque Moisés tratara en su tiempo de combatir la idolatría y la adoración de los dioses falsos, ¿deben todos los hombres desaprobar la reproducción de la gracia y la belleza? Te digo, Flavio, que los hijos de Moisés lo malinterpretaron y ahora crean falsos dioses incluso de sus prohibiciones de las imágenes y de la semejanza de lo que está en el cielo y en la tierra. Pero, aunque Moisés instruyera en esas restricciones a las mentes oscuras de aquella época, ¿qué relación tiene eso con estos días en los que el Padre de los cielos se revela como el Soberano Espiritual de todos? Y, Flavio, declaro ante ti que en el reino venidero no se continuará preceptuando, ‘No adoréis esto; no adoréis aquello’; ya no se ocuparán de ordenar que os abstengáis de esto o cuidado de no hacer aquello, sino que todos se preocuparán de un supremo deber, de un deber del hombre que se manifiesta en dos grandes privilegios: adorar sinceramente al Creador infinito, al Padre del Paraíso, y servir con amor a los semejantes. Si amas a tu prójimo como te amas a ti mismo, sabes realmente que eres hijo de Dios.
\vs p142 4:3 “En una era en la que no se comprendía bien a mi Padre, los esfuerzos de Moisés por hacer frente a la idolatría estaban justificados, pero, en la era por llegar, el Padre se habrá revelado en la vida de su Hijo; y esta nueva revelación de Dios hará innecesario para siempre confundir al Padre Creador con ídolos de piedra o imágenes de oro y plata. En adelante, los hombres de inteligencia podrán disfrutar de los tesoros del arte sin confundir el aprecio material de la belleza con la adoración y el servicio al Padre del Paraíso, al Dios de todas las cosas y de todos los seres”.
\vs p142 4:4 \pc Flavio creyó todo lo que Jesús le enseñó. Al día siguiente, fue a Betania, al otro lado del Jordán, y recibió el bautizo de manos de los discípulos de Juan. En aquel momento, los apóstoles de Jesús aún no bautizaban a los creyentes. Cuando Flavio regresó a Jerusalén, organizó una gran fiesta para Jesús e invitó a sesenta amigos suyos. Y muchos de ellos se convirtieron igualmente en creyentes del mensaje del reino venidero.
\usection{5. DISCURSO SOBRE LA CERTEZA ESPIRITUAL}
\vs p142 5:1 Durante esta semana de Pascua, uno de los grandes sermones que Jesús predicó en el templo fue en respuesta a una pregunta que uno de los asistentes, un hombre de Damasco, le hizo: “Pero, Rabí, ¿cómo sabremos con certeza que Dios te ha enviado, y que nosotros podremos verdaderamente entrar en un reino, cuya proximidad tú y tus discípulos anunciáis?”. Y Jesús respondió:
\vs p142 5:2 \pc “En cuanto a mi mensaje y a las enseñanzas de mis discípulos, debéis juzgarlos por sus frutos. Si os proclamamos las verdades del espíritu, el espíritu mismo dará testimonio en vuestros corazones de que nuestro mensaje es auténtico. En cuanto al reino y a la certeza de vuestra aceptación por el Padre celestial, os pregunto ¿qué padre entre vosotros, que sea digno y bondadoso, mantendría a su hijo en la ansiedad e inseguridad respecto a su estatus en la familia o en cuanto a tener un sitio seguro en el corazón amoroso de su padre? ¿Es que vosotros, padres terrenales, disfrutáis torturando a vuestros hijos dejándolos en la incertidumbre sobre su lugar en el continuo amor de vuestros corazones humanos? Tampoco vuestro Padre de los cielos abandona a sus hijos del espíritu por la fe en la dudosa incertidumbre de su posición en el reino. Si recibís a Dios como vuestro Padre, entonces, sois en efecto y de cierto, hijos de Dios. Y si sois sus hijos, entonces, estaréis seguros de vuestra posición y condición respecto a vuestra filiación eterna y divina. Si creéis mis palabras, creéis, pues, en Aquel que me envió; y creyendo, por tanto, en el Padre, os habéis asegurado vuestro estatus en la ciudadanía celestial. Si hacéis la voluntad del Padre de los cielos, no erraréis en el logro de la vida eterna, que es progreso en el reino divino.
\vs p142 5:3 “El Espíritu Supremo dará testimonio a vuestros espíritus de que verdaderamente sois hijos de Dios. Y si sois hijos de Dios, entonces, habéis nacido del espíritu de Dios; y el que haya nacido del espíritu, tiene en sí mismo el poder para superar cualquier duda, y esta es la victoria que vence cualquier incertidumbre, vuestra fe.
\vs p142 5:4 “Dijo el profeta Isaías, hablando de estos tiempos: ‘cuando el espíritu se derrame sobre nosotros de lo alto, entonces, la labor de la justicia será paz, reposo y seguridad para siempre’. Para todos aquellos que crean verdaderamente en este evangelio, yo me convertiré en la garantía de que serán acogidos en la misericordia eterna y en la imperecedera vida del reino de mi Padre. Por consiguiente, vosotros que oís este mensaje y creéis en este evangelio del reino sois hijos de Dios y tenéis vida imperecedera; y la evidencia para todo el mundo de que habéis nacido del espíritu es que sinceramente os amáis unos a otros”.
\vs p142 5:5 \pc La multitud que oía a Jesús estuvo durante horas haciéndole preguntas y escuchando con atención sus reconfortantes respuestas. Gracias a estas enseñanzas, incluso los apóstoles se sintieron alentados a predicar el evangelio del reino con más fuerza y seguridad. Estas experiencias vividas en Jerusalén fueron de gran inspiración para los doce. Significaron su primer contacto con muchedumbres tan enormes, y aprendieron muchas lecciones útiles que les resultarían muy beneficiosas para su futura labor.
\usection{6. CONVERSACIÓN CON NICODEMO}
\vs p142 6:1 Una noche, estando Jesús en la casa de Flavio, vino a verle un tal Nicodemo, un hombre rico, miembro de los ancianos del sanedrín judío. Había oído hablar mucho de las enseñanzas del galileo y, por esa razón, había acudido por la tarde a escucharle cuando enseñaba en los patios del templo. Hubiese deseado asistir más a menudo a las predicaciones de Jesús, pero temía que quienes presenciaban estas enseñanzas le viesen. En ese momento, los dirigentes judíos estaban en tal desavenencia con Jesús que ningún miembro del sanedrín quería ser identificado abiertamente con él. Así pues, Nicodemo había acordado con Andrés en verse con Jesús en privado, y tras el anochecer, de ese día en particular. Pedro, Santiago y Juan se hallaban en el jardín de Flavio cuando comenzó el encuentro, pero más tarde entraron a la casa en donde continuó la charla.
\vs p142 6:2 Cuando recibió a Nicodemo, Jesús no mostró ninguna deferencia especial hacia él y, al hablar con él, no hubo transigencia ni ninguna injustificada pretensión de resultar convincente. El Maestro no hizo ningún intento de rechazar a la persona que clandestinamente le requería ni empleó el sarcasmo. En todo su trato con el distinguido visitante, Jesús se mostró calmado, serio y digno. Nicodemo no venía a Jesús como representante oficial del sanedrín, sino enteramente debido a su interés personal y sincero en las enseñanzas del Maestro.
\vs p142 6:3 Al presentarlo Flavio, Nicodemo dijo: “Rabí, sabemos que has venido de Dios como maestro, porque nadie podría enseñar como tú lo haces si no está Dios con él. Y yo estoy deseoso de saber más de tus enseñanzas sobre el reino venidero”.
\vs p142 6:4 Jesús le respondió: “De cierto, de cierto, te digo, Nicodemo, que si un hombre no nace de lo alto, no puede ver el reino de Dios”. Entonces, Nicodemo contestó: “Pero, ¿cómo puede un hombre nacer de nuevo siendo viejo? No puede entrar por segunda vez en el vientre de su madre y nacer”.
\vs p142 6:5 Jesús dijo: “No obstante, declaro ante ti que el que no nace del espíritu, no puede entrar en el reino de Dios. Lo que nace de la carne, carne es, y lo que nace del espíritu, espíritu es. No te maravilles que te dijera que debes nacer de lo alto. Cuando el viento sopla, oyes el susurro de las hojas, pero no ves el viento ---de dónde viene ni adónde va---, y así es todo aquel que nace del Espíritu. Con los ojos de la carne puedes contemplar las manifestaciones del espíritu, pero ciertamente no puedes percibirlo”.
\vs p142 6:6 Nicodemo contestó: “Pero no comprendo; ¿cómo puede ser eso así?”. Jesús le dijo: “¿Puede ser que seas un maestro de Israel y que, sin embargo, ignores todo esto? Es pues responsabilidad de los que conocen las realidades del espíritu revelar estas cosas a los que solamente perciben las manifestaciones del mundo material. Pero ¿nos creerás si te hablamos de las verdades celestiales? ¿Tienes la valentía, Nicodemo, de creer en quien ha descendido del cielo, que es el mismo Hijo del Hombre?”.
\vs p142 6:7 Y Nicodemo preguntó: “Pero ¿cómo puedo empezar a estar en contacto con este espíritu que ha de rehacerme para poder entrar al reino?”. Jesús le respondió: “El espíritu del Padre de los cielos ya habita en ti. Si te dejas llevar por este espíritu que viene de lo alto, comenzarás muy pronto a ver con sus mismos ojos. Y, entonces, mediante tu decisión incondicional de dejar que el espíritu te guíe, nacerás del espíritu ya que el único propósito de tu vida será hacer la voluntad de tu Padre celestial. Y, así, al descubrir que has nacido del espíritu y que estás gozosamente en el reino de Dios, empezarás a rendir en tu vida diaria los abundantes frutos del espíritu”.
\vs p142 6:8 Nicodemo era totalmente sincero. Estaba profundamente impresionado, pero se alejó perplejo. Nicodemo había logrado un buen desarrollo personal, un buen dominio de sí mismo e incluso poseía elevadas cualidades morales. Era refinado, seguro de sí mismo y altruista; pero no sabía cómo \bibemph{someter} su voluntad a la voluntad del Padre divino, tal como un niño pequeño está dispuesto a someterse a la guía y a la dirección de un padre terrenal sensato y amoroso y convertirse, pues, en un hijo de Dios, en un heredero que avanza en el reino eterno.
\vs p142 6:9 Pero Nicodemo ciertamente hizo acopio de fe suficiente como para entrar en el reino. Protestó levemente cuando sus compañeros del sanedrín trataron de condenar a Jesús sin un juicio justo; y, con José de Arimatea, reconoció, luego, su fe con valentía, llegando a reclamar el cuerpo de Jesús, incluso cuando la mayoría de los discípulos habían huido aterrados del escenario en el que tenía lugar el sufrimiento final y la muerte del Maestro.
\usection{7. LECCIÓN SOBRE LA FAMILIA}
\vs p142 7:1 Tras el período de intensa actividad docente y de labor personal realizado en Jerusalén durante la semana de Pascua, Jesús pasó el siguiente miércoles en Betania, descansando con sus apóstoles. Esa tarde, Tomás hizo una pregunta que suscitó una respuesta larga y esclarecedora. Tomás dijo: “Maestro, el día que nos seleccionaste como embajadores del reino, nos dijiste muchas cosas, nos instruiste sobre la forma de vida que deberíamos llevar a nivel personal, pero, ¿qué le enseñaremos a la multitud? ¿Cómo deben vivir estas personas una vez que el reino haya llegado en su mayor plenitud? ¿Poseerán esclavos tus discípulos? ¿Buscarán tus creyentes la pobreza y evitarán la riqueza? ¿Prevalecerá por sí misma la misericordia sin que haya más necesidad de ley y justicia?”. Jesús y los doce pasaron toda la tarde y esa noche, después de la cena, comentando las preguntas de Tomás. A efectos de este relato, exponemos el siguiente resumen de la instrucción impartida por el maestro.
\vs p142 7:2 Jesús quiso primeramente dejar claro a sus apóstoles que él estaba en la tierra viviendo una excepcional vida en la carne y que ellos, los doce, habían sido llamados para participar en este ministerio de gracia del Hijo del Hombre; y que, como tales compañeros en esta labor, debían asimismo compartir muchas de las restricciones y obligaciones que dicho ministerio conllevaba. Hubo una indicación velada de que el Hijo del Hombre era la única persona que jamás había vivido sobre la tierra, que pudiera, de forma simultánea, ver dentro del corazón mismo de Dios y en las profundidades del alma del hombre.
\vs p142 7:3 Jesús explicó con mucha claridad que el reino de los cielos era una experiencia evolutiva, que comenzaba aquí en la tierra y progresaba a través de sucesivas estaciones de vida hasta llegar al Paraíso. En el transcurso de la noche, señaló categóricamente que en algún estadio del desarrollo del reino, visitaría nuevamente este mundo en poder espiritual y gloria divina.
\vs p142 7:4 A continuación explicó que la “idea del reino” no era la mejor manera de ilustrar la relación del hombre con Dios; que empleaba este lenguaje figurado porque el pueblo judío estaba a la espera del reino y porque Juan había predicado aludiendo al reino venidero. Jesús dijo: “La gente de otra era comprenderá más fácilmente el evangelio del reino cuando se presente en términos que expresen relación familiar ---cuando el hombre comprenda que la religión consiste en la enseñanza de la paternidad de Dios y la hermandad de los hombres, en la filiación con Dios”---. Entonces, el Maestro disertó extensamente sobre la familia terrenal como ilustración de la familia celestial, reiteró las dos leyes fundamentales de la vida: el primer mandamiento de amor por el padre, por el jefe de la familia, y, el segundo mandamiento de amor mutuo entre los hijos, el precepto de amar a tu hermano como a ti mismo. Y entonces precisó que tan alto nivel de afecto fraternal se manifestaría de forma invariable como servicio social, desinteresado y amoroso.
\vs p142 7:5 Tras esto, vino la charla memorable sobre los rasgos esenciales de la vida familiar y su aplicación a la relación existente entre Dios y el hombre. Jesús afirmó que una verdadera familia se fundamenta en los siete hechos siguientes:
\vs p142 7:6 \li{1.}\bibemph{El hecho de la existencia}. Hay una relación natural y unos parecidos físicos que vinculan entre sí a los miembros de una familia: los niños heredan determinados caracteres de sus padres. Los hijos tienen origen en sus padres. La existencia de la persona depende del acto de los padres. La relación entre padres e hijos es innata en toda la naturaleza e impregna todas las existencias vivas.
\vs p142 7:7 \li{2.}\bibemph{La seguridad y la satisfacción.} A los verdaderos padres les complace sumamente atender las necesidades de sus hijos. Muchos padres no se contentan con remediar las simples carencias de sus hijos, sino que disfrutan haciendo previsiones para que ellos tengan igualmente sus propias satisfacciones.
\vs p142 7:8 \li{3.}\bibemph{La educación y la formación.} Los padres sensatos planean detenidamente proporcionar a sus hijos e hijas una educación y una adecuada formación. Siendo jóvenes se les prepara para las mayores responsabilidades de su vida futura.
\vs p142 7:9 \li{4.}\bibemph{La disciplina y las restricciones}. Los padres con visión de futuro hacen también previsiones para la necesaria disciplina, guía, corrección y, algunas veces, restricción de sus vástagos jóvenes e inmaduros.
\vs p142 7:10 \li{5.}\bibemph{La compañía y la lealtad.} El padre cariñoso mantiene una relación cercana y amorosa con sus hijos. Sus oídos están siempre abiertos a oír sus peticiones; está siempre listo para compartir sus tribulaciones y asistirlos en sus dificultades. Un padre se interesa supremamente por el creciente bienestar de su progenie.
\vs p142 7:11 \li{6.}\bibemph{El amor y la misericordia.} Un padre compasivo sabe perdonar sin límites; los padres no albergan ideas vengativas contra sus hijos. Los padres no son jueces, ni enemigos ni acreedores. Las verdaderas familias se edifican sobre las bases de la tolerancia, la paciencia y el perdón.
\vs p142 7:12 \li{7.}\bibemph{Previsiones para el futuro.} A los padres temporales les complace dejarles una herencia a sus hijos. La familia continúa de una generación a la otra. La muerte solo acaba con una generación para señalar el comienzo de la siguiente. La muerte pone fin a una vida individual, pero no acaba necesariamente con la familia.
\vs p142 7:13 \pc Durante horas, el Maestro explicó de cómo estos rasgos de la vida familiar se aplicaban a las relaciones del hombre, el hijo de la tierra, con Dios, el Padre del Paraíso. Y estas fueron sus conclusiones: “Conozco en perfección, y enteramente, la relación de un hijo con el Padre, porque todo lo que habréis de conseguir respecto a vuestra filiación con él en el futuro eterno, yo ya lo he conseguido. El Hijo del Hombre está preparado para ascender a la derecha del Padre, por lo que en mí está el camino, ahora aún más amplio, para que todos vosotros veáis a Dios y, antes de que hayáis terminado vuestra glorioso camino de avance, lleguéis a ser perfectos al igual que el Padre de los cielos es perfecto”.
\vs p142 7:14 Cuando los apóstoles oyeron estas sorprendentes palabras, recordaron la proclamación realizada por Juan en el momento del bautismo de Jesús, y rememorarían asimismo, vívidamente, este hecho en sus predicaciones y enseñanzas tras la muerte y resurrección del Maestro,
\vs p142 7:15 Jesús es un hijo divino, alguien que tiene la total confianza del Padre Universal. Había estado con el Padre y lo comprendía plenamente. Ahora, había vivido su vida en la tierra conforme a la total satisfacción del Padre, y esta encarnación en la carne le había permitido entender por completo al hombre. Jesús significaba la perfección del hombre; había logrado la misma perfección que todos los verdaderos creyentes están destinados a lograr en él y por medio de él. Jesús reveló al hombre un Dios de perfección y presentó en sí mismo a Dios al hijo en perfección de los mundos.
\vs p142 7:16 Aunque Jesús habló durante varias horas, Tomás no estaba aún satisfecho, puesto que dijo: “Pero, Maestro, no nos parece que el Padre de los cielos nos trate siempre con benignidad y misericordia. Muchas veces sufrimos penosamente en la tierra, y no siempre nuestras oraciones obtienen respuesta. ¿En qué fallamos para no llegar a comprender el significado de tus enseñanzas?”.
\vs p142 7:17 Jesús le respondió: “Tomás, Tomás, ¿cuánto tiempo ha de pasar hasta que seas capaz de oír con los oídos del espíritu? ¿Cuánto tiempo te llevará hasta que percibas que este reino es un reino espiritual, y que mi Padre es también un ser espiritual? ¿Es que no entiendes que os estoy instruyendo como hijos espirituales de la familia espiritual del cielo, cuyo jefe y padre es un espíritu infinito y eterno? ¿Es que no vais a permitirme que emplee la idea de la familia terrenal para ilustrar las relaciones divinas sin que apliquéis mis enseñanzas de forma tan literal a las cuestiones materiales? ¿Es que en vuestras mentes no podéis separar las realidades espirituales del reino de los problemas materiales, sociales, económicos y políticos de esta época? Cuando hablo la lengua del espíritu, ¿por qué insistís en traducir mis significados a la lengua de la carne precisamente porque yo pretenda hacer uso de relaciones comunes y concretas a efectos ilustrativos? Hijos míos, os ruego que dejéis de aplicar las enseñanzas del reino del espíritu a los innobles asuntos de la esclavitud, la pobreza, la vivienda y las tierras, y a los problemas materiales de la equidad y la justicia humanas. Estas cuestiones temporales forman parte de la preocupación de los hombres de este mundo y aunque, de algún modo, atañen a todos los hombres, vosotros habéis sido llamados para representarme a mí en el mundo, tal como yo represento a mi Padre. Sois los embajadores espirituales de un reino espiritual, los representantes especiales del Padre espiritual. Para este momento, yo ya debería poder instruiros como hombres maduros del reino espiritual. ¿Es que tengo que dirigirme siempre a vosotros como si fuerais unos niños? ¿Es que no vais a crecer nunca en percepción espiritual? No obstante, os amo y seré tolerante con vosotros hasta el fin de nuestra relación en la carne. E incluso entonces mi espíritu irá por delante de vosotros en todo el mundo”.
\usection{8. EN JUDEA DEL SUR}
\vs p142 8:1 Hacia finales de abril, la hostilidad hacia Jesús entre los fariseos y los saduceos se había vuelto tan intensa que el Maestro y sus apóstoles decidieron dejar Jerusalén durante algún tiempo y se dirigieron al sur para realizar su labor en Belén y Hebrón. Todo el mes de mayo lo dedicaron a tareas de tipo personal en estas ciudades y entre la gente de las aldeas vecinas. No hubo predicación pública en este viaje; solo iban de casa en casa. Una parte de este tiempo, mientras que los apóstoles enseñaban el evangelio y atendían a los enfermos, Jesús y Abner lo pasaban en En\hyp{}gadi, visitando el asentamiento nazareo. De aquel lugar había salido Juan el Bautista, y Abner había sido jefe de este grupo. Muchos de los miembros de la hermandad nazarea se convirtieron en creyentes de Jesús, pero la mayoría de estos hombres ascéticos y excéntricos se negaron a aceptarlo como un maestro enviado del cielo porque no enseñaba a ayunar ni ninguna otra forma de abnegación.
\vs p142 8:2 La gente que vivía en esta región no sabía que Jesús había nacido en Belén. Como la inmensa mayoría de sus discípulos, siempre pensaron que el Maestro había nacido en Nazaret, aunque los doce sí sabían la verdad.
\vs p142 8:3 Esta estancia en el sur de Judea significó para ellos un tiempo tranquilo y fructífero; gracias a su labor se añadieron muchas almas al reino. En los primeros días de junio, los disturbios contra Jesús en Jerusalén se habían apaciguado lo suficiente como para que el Maestro y los apóstoles volvieran allí para instruir y confortar a los creyentes.
\vs p142 8:4 Aunque Jesús y los apóstoles estuvieron todo el mes de junio en Jerusalén o en sus alrededores, no impartieron enseñanza pública durante este tiempo. Vivían mayormente en tiendas, que montaban en un parque, o jardín, conocido en esa época como Getsemaní. Este parque, de abundante sombra, estaba situado en la ladera occidental del Monte de los Olivos, no lejos del torrente Cedrón. Normalmente, pasaban los fines de semana del \bibemph{sabbat} con Lázaro y sus hermanas en Betania. Jesús cruzó los muros para entrar en Jerusalén solo unas pocas veces, pero un gran número de personas, interesadas en sus enseñanzas, iban a Getsemaní para hablar con él. Un viernes por la noche, Nicodemo y un tal José de Arimatea se aventuraron a salir para ir a ver a Jesús, pero por temor, estando ya ante la tienda de Jesús, se volvieron atrás. Y, obviamente, no se percataron de que Jesús conocía estos hechos.
\vs p142 8:5 Cuando los dignatarios de los judíos tuvieron conocimiento de que Jesús había regresado a Jerusalén, se dispusieron a arrestarlo; pero, al observar que no daba predicaciones públicas, llegaron a la conclusión de que se había asustado por los disturbios anteriores y decidieron permitirle que siguiera con sus enseñanzas de modo privado, sin volver a molestarlo. Y, así, las cosas siguieron su curso tranquilo hasta los últimos días de junio, cuando un tal Simón, miembro del sanedrín, se adhirió públicamente a las enseñanzas de Jesús tras declararlo él mismo ante estos dignatarios. De inmediato, se organizó un nuevo revuelo para la detención de Jesús, creciendo con tanta intensidad que el Maestro decidió retirarse a las ciudades de Samaria y la Decápolis.
