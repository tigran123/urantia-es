\upaper{46}{La sede del sistema local}
\author{Arcángel}
\vs p046 0:1 Jerusem, la sede de Satania, es una capital de tipo medio de un sistema local y, aparte de las numerosas circunstancias excepcionales ocasionadas por la rebelión de Lucifer y el ministerio de gracia de Miguel en Urantia, es una esfera convencional del orden de otras similares. Vuestro sistema local ha pasado por algunas experiencias tormentosas, pero en la actualidad se gobierna con una gran eficacia y, conforme transcurren las eras, los efectos de la discordia se van erradicando de forma lenta pero segura. Se están restableciendo el orden y la buena voluntad, y las condiciones de Jerusem se acercan, cada vez más, al estatus celestial que describen vuestras tradiciones, porque esta sede del sistema es, en verdad, el cielo esperado por la mayoría de los creyentes religiosos del siglo XX.
\usection{1. ASPECTOS FÍSICOS DE JERUSEM}
\vs p046 1:1 Jerusem se divide en mil sectores latitudinales y diez mil zonas longitudinales. Esta esfera tiene siete capitales principales y setenta centros administrativos menores. En las siete capitales jurisdiccionales, se lleva a cabo una actividad variada; el soberano del sistema, al menos una vez al año, hace acto de presencia en cada una de dichas capitales.
\vs p046 1:2 \pc La milla regular de Jerusem equivale a unas siete millas de Urantia. El peso estándar, el “gradant”, se ha estimado en el sistema decimal partiendo del ultimatón en su mayor grado de desarrollo y se corresponde a unos doscientos ochenta gramos de vuestro peso. El día de Satania es igual a tres días del tiempo de Urantia, menos una hora, cuatro minutos y quince segundos, que corresponden al tiempo de duración de la rotación axial de Jerusem. El año del sistema se compone de cien días de tiempo de Jerusem. Las horas del sistema las emiten los cronoldecs mayores.
\vs p046 1:3 \pc La energía de Jerusem está magníficamente regida y circula por toda la esfera en canales segmentados que se alimentan directamente de las cargas energéticas del espacio y están bajo la experta dirección de los controladores físicos mayores. La resistencia natural al paso de estas energías por los canales físicos de conducción produce el calor necesario para crear la temperatura estable de esta esfera. La temperatura a plena luz del día se mantiene alrededor de veintiún grados centígrados, mientras que durante el período de recesión de la luz desciende a algo menos de diez grados.
\vs p046 1:4 \pc El sistema de iluminación de Jerusem no os debería resultar difícil de comprender. No hay ni días ni noches, ni estaciones de calor y frío. Los transformadores de la potencia se encargan del mantenimiento de cien mil puntos centrales desde los que proyectan las energías enrarecidas hacia arriba a través de la atmósfera planetaria, experimentando ciertos cambios, hasta que alcanzan la capa límite de aire eléctrico de la esfera; entonces estas energías se reflejan hacia abajo como luz suave, tamizada y uniforme, de una intensidad parecida a la de la luz del sol que brilla sobre Urantia a las diez de la mañana.
\vs p046 1:5 Con una iluminación de tales características, los rayos de luz no parecen provenir de un solo lugar, sino que simplemente se filtran y dispersan desde el cielo, emanando por igual desde todas las direcciones del espacio. Esta luz es muy similar a la luz natural del sol, excepto que irradia mucho menos calor. De ello se puede deducir que los mundos sedes son cuerpos no luminosos del espacio; si Jerusem estuviese muy cerca de Urantia, no sería visible.
\vs p046 1:6 Los gases que reflejan esta luz\hyp{}energía desde la parte superior de la ionosfera de Jerusem hacia el suelo son muy similares a los que se encuentran en las capas superiores de aire de Urantia y que dan lugar a los fenómenos aurorales de vuestras llamadas luces del norte, aunque estas se producen por causas diferentes. En Urantia, es este mismo escudo gaseoso el que impide que las ondas de transmisión terrestre se escapen porque, al elevarse directamente hacia el exterior, chocan con dicho cinturón que hace que reboten y que retornen a la tierra. De este modo, las transmisiones, cuando se desplazan por el aire alrededor de vuestro mundo, se mantienen cerca de la superficie.
\vs p046 1:7 La esfera se mantiene así iluminada de forma uniforme durante el setenta y cinco por ciento del día de Jerusem; luego hay una recesión gradual de la luz hasta que, en el momento de mínima iluminación, la intensidad de esta se aproxima a la de vuestra luna llena en una noche clara. Es la hora calma para todo Jerusem. Tan solo las estaciones de transmisión y recepción permanecen operativas durante este período de descanso y recuperación.
\vs p046 1:8 \pc Jerusem recibe la tenue luz de algunos soles cercanos ---una especie de luz brillante de las estrellas---, pero no depende de ellos; los mundos como Jerusem no están expuestos a las vicisitudes de las perturbaciones solares ni han de afrontar el problema de un sol en período de enfriamiento o moribundo.
\vs p046 1:9 Los siete mundos de estudio y transición y sus cuarenta y nueve satélites se calientan, iluminan, energizan y proveen de agua por el mismo procedimiento de Jerusem.
\usection{2. CARACTERÍSTICAS FÍSICAS DE JERUSEM}
\vs p046 2:1 En Jerusem echaréis de menos las cadenas de montañas escarpadas de Urantia y de otros mundos evolutivos. Tampoco hay terremotos ni precipitaciones de lluvia; no obstante, disfrutaréis de las bellas altiplanicies y de otras singulares variaciones de su topografía y paisaje. Jerusem tiene áreas inmensas que se conservan en su “estado natural”. La magnificencia de estas regiones excede en mucho la capacidad imaginativa humana.
\vs p046 2:2 Hay miles y miles de lagos pequeños, pero no hay ríos embravecidos ni extensos océanos. No hay lluvias ni tormentas ni ventiscas en ninguno de los mundos arquitectónicos, pero, cuando la temperatura desciende a su máximo, algo que se produce en paralelo a la recesión de la luz, la humedad se condensa en forma de precipitación diaria. (El punto de rocío es más elevado en un mundo de tres gases que en un planeta de dos gases como Urantia). La humedad es necesaria tanto para la vida física de las plantas como para los seres vivos del mundo morontial, pero dicha humedad se origina, mayormente, gracias al sistema de circulación del subsuelo que se extiende por toda la esfera, y que llega incluso a las mismas cimas de las altiplanicies. Este sistema hidráulico no es totalmente subterráneo, ya que existen muchos canales que comunican entre sí a los destellantes lagos de Jerusem.
\vs p046 2:3 La atmósfera de Jerusem está constituida por una mezcla de tres gases. Este aire es muy similar al de Urantia aunque tiene un gas adicional, propicio para la respiración del orden de vida morontial. Este tercer gas de ninguna manera hace al aire irrespirable para los animales o plantas de los órdenes materiales.
\vs p046 2:4 El sistema de transporte está vinculado a las corrientes circulatorias implicadas en el movimiento de la energía, las cuales están situadas a intervalos de dieciséis kilómetros. Adaptando sus mecanismos físicos, los seres materiales del planeta pueden trasladarse a una velocidad que varía entre trescientos y ochocientos kilómetros por hora. Las aves transportadoras vuelan a una velocidad aproximada de ciento sesenta kilómetros por hora. Los artefactos aéreos usados por los hijos materiales viajan a unos ochocientos kilómetros por hora. Los seres materiales y los incipientes seres morontiales han de hacer uso de estos medios mecánicos de transporte, si bien, los seres personales espirituales se desplazan enlazando con las fuerzas superiores y las fuentes espirituales de la energía.
\vs p046 2:5 \pc En Jerusem y en los mundos vinculados a esta sede se dan las típicas diez categorías de la vida física, características de las esferas arquitectónicas de Nebadón. Y puesto que no hay evolución orgánica en Jerusem, no existen formas de vida antagonistas ni pugna por la existencia ni la ley del más fuerte. Hay más bien una adaptación creativa que prefigura la belleza, la armonía y la perfección de los mundos eternos del universo central y divino. Y, en toda esta perfección creativa, existe el más asombroso encaje entre la vida física y la morontial, que los artesanos celestiales y sus colaboradores combinan de manera artística.
\vs p046 2:6 De hecho, Jerusem es un preludio de la gloria y el esplendor paradisíacos. Y por mucho que se intente describirlos, no abriguéis la esperanza de llegar a tener una idea adecuada de estos gloriosos mundos arquitectónicos. Hay poco que sea comparable a vuestro mundo y, si este fuese el caso, las cosas de Jerusem trascienden tanto a las de Urantia que cualquier comparación parecería prácticamente incongruente y distorsionada. En realidad, hasta que no lleguéis a Jerusem es difícil que podáis albergar noción alguna que se acerque a un verdadero concepto de estos mundos celestiales; sin embargo, en un futuro no muy lejano se comparará vuestra experiencia venidera en la capital del sistema con vuestra llegada postrera a las esferas más remotas de formación del universo, del suprauniverso y de Havona.
\vs p046 2:7 \pc El sector de fabricación y de laboratorios de Jerusem ocupa una amplia área, la cual, por su carencia de chimeneas humeantes, sería difícil de reconocer por los habitantes de Urantia; no obstante, en conexión con estos mundos especiales, hay todo un intrincado y eficaz sistema organizado de orden material al igual que unos procedimientos mecánicos perfectos y unos logros físicos que asombrarían e incluso sobrecogerían a vuestros químicos e inventores más experimentados. Considerad que este primer mundo en el que os detenéis en vuestro viaje al Paraíso es mucho más material que espiritual. Durante toda vuestra estancia en Jerusem y en sus mundos de transición, estaréis mucho más cerca de las cosas materiales de vuestra vida terrestre de lo que estaréis en vuestra ulterior y progresiva existencia espiritual.
\vs p046 2:8 \pc Monte Serafín, con sus casi cuatro mil seiscientos metros, constituye la elevación de mayor altitud de Jerusem y el punto de salida de todos los serafines transportadores. Se utilizan numerosos desarrollos de tipo mecánico a fin de suministrar la energía inicial necesaria para escapar de la gravedad planetaria y vencer la resistencia del aire. Durante todo el periodo de luz y, a veces, hasta bien entrada su recesión, cada tres segundos del tiempo de Urantia parte uno de los transportes seráficos. Los transportadores toman vuelo a una velocidad aproximada de veinticinco millas regulares por segundo del tiempo de Urantia y no alcanzan su velocidad estándar hasta que no se sitúan a más de dos mil millas de Jerusem.
\vs p046 2:9 Los transportes llegan al campo de vidrio, al llamado mar de cristal. Alrededor de esta área se hallan las estaciones de recepción de los distintos órdenes de seres que surcan el espacio utilizando el transporte seráfico. Cerca de la estación receptora del mar polar de los visitantes estudiantiles, puedes ascender al observatorio nacarado y contemplar el inmenso mapa en relieve de todo el planeta sede.
\usection{3. LAS ESTACIONES DE RECEPCIÓN Y TRANSMISIÓN DE JERUSEM}
\vs p046 3:1 Las transmisiones procedentes del suprauniverso y del Paraíso\hyp{}Havona se reciben en Jerusem en coordinación con Lugar de Salvación y mediante un procedimiento relacionado con el vidrio polar, o mar de cristal. Además de su capacidad para la recepción de estos comunicados de más allá de Nebadón, existen tres grupos diferentes de estaciones receptoras. Dichos grupos, distintos pero tricirculares, están configurados para la recepción de estas emisiones que se realizan desde los mundos locales, desde la sede de la constelación y desde la capital del universo local. Todas ellas automáticamente se manifiestan de modo que puedan resultar perceptibles para todas las clases de seres presentes en el anfiteatro receptor central; de todos los atractivos de los que goza el mortal ascendente en Jerusem, ninguno es más fascinante ni capta más su atención como el de escuchar un inagotable flujo de informes espaciales provenientes del universo.
\vs p046 3:2 Rodeando la estación de recepción de transmisiones de Jerusem, hay un enorme anfiteatro construido con materiales resplandecientes, en su mayor parte desconocidos en Urantia, que es capaz de albergar a cinco mil millones de seres ---materiales y morontiales--- además de dar cabida a incontables seres personales espirituales. El entretenimiento favorito de todo Jerusem es pasar su tiempo de ocio en esta estación receptora a fin de conocer el bienestar y el estado de las cosas del universo. Es la única actividad planetaria que no decae durante la recesión de la luz.
\vs p046 3:3 A este anfiteatro, llegan continuamente los mensajes de Lugar de Salvación. Cerca de aquí, al menos una vez al día, se reciben desde Edentia las palabras de los padres altísimos de la constelación. Periódicamente, a través de Lugar de Salvación, se transmiten las emisiones regulares y especiales procedentes de Uversa y, cuando los mensajes proceden del Paraíso, la población al completo se reúne en torno al mar de cristal, y los colaboradores de Uversa añaden al sistema de transmisión del Paraíso el fenómeno de la reflectividad para que se haga visible todo lo que se oiga. De esta manera, conforme viajan hacia el interior en su eterna aventura, se ofrece a los supervivientes mortales un anticipo continuo del progresivo aumento de lo bello y lo grandioso.
\vs p046 3:4 \pc La estación emisora de Jerusem está situada en el polo opuesto de la esfera. Todas las transmisiones con destino a los distintos mundos se realizan desde las capitales de los sistemas, salvo los mensajes de Miguel, que suelen alcanzar su destino directamente a través de los canales de transmisión de los arcángeles.
\usection{4. LAS ÁREAS RESIDENCIALES Y ADMINISTRATIVAS}
\vs p046 4:1 Jerusem tiene grandes extensiones que se dedican a áreas residenciales, mientras que tiene otras que se destinan a la realización de las necesarias funciones administrativas relacionadas con la supervisión de los asuntos de 619 esferas habitadas, de 56 mundos de cultura y transición y de la capital misma del sistema. En Jerusem y en Nebadón esta distribución se configura de la siguiente manera:
\vs p046 4:2 \li{1.}\bibemph{Los círculos:} las áreas residenciales para los no nativos.
\vs p046 4:3 \li{2.}\bibemph{Los cuadrados:} las áreas ejecutivas y administrativas del sistema.
\vs p046 4:4 \li{3.}\bibemph{Los rectángulos:} el punto de encuentro de la vida nativa de inferior rango.
\vs p046 4:5 \li{4.}\bibemph{Los triángulos:} las áreas locales o administrativas de Jerusem.
\vs p046 4:6 \pc Tal distribución de la actividad del sistema en círculos, cuadrados, rectángulos y triángulos es común a todas las capitales de los sistemas de Nebadón. Es posible que en otro universo predomine una disposición totalmente diferente. Son los hijos creadores los que, de acuerdo con sus distintos planes, deciden estas cuestiones.
\vs p046 4:7 \pc Nuestro relato de estas áreas residenciales y administrativas no tiene en cuenta las bellas e inmensas heredades de los hijos materiales de Dios ---los ciudadanos permanentes de Jerusem--- ni tampoco hace alusión a numerosos otros órdenes de fascinantes criaturas espirituales y semiespirituales. Por ejemplo: Jerusem goza del eficiente servicio de los espirongas, creados para desempeñar sus funciones en el sistema. Estos seres se dedican al ministerio espiritual de los residentes y visitantes supramateriales. Constituyen un extraordinario grupo de admirables seres inteligentes que son servidores de transición de las criaturas morontiales de mayor rango y de los asistentes morontiales, cuya labor consiste en mantener y embellecer todas las creaciones morontiales. Son para Jerusem lo que las criaturas intermedias son para Urantia: ayudantes mediadores que operan entre lo material y lo espiritual.
\vs p046 4:8 Las capitales de los sistemas son singulares en el sentido de que son los únicos mundos en los que se muestran, de modo casi perfecto, las tres fases de la existencia universal: la material, la morontial y la espiritual. Si eres un ser personal material, morontial o espiritual, te sentirás en casa en Jerusem; así se sienten también los seres de naturaleza combinada como las criaturas intermedias y los hijos materiales.
\vs p046 4:9 Jerusem posee magníficos edificios tanto de tipo material como morontial, aunque la magnificencia de las zonas puramente espirituales no es menos excelente ni plena. ¡Ojalá pudiese encontrar palabras que me sirvieran para hablaros de los equivalentes morontiales del excelente equipamiento físico de Jerusem! ¡Ojalá pudiese describir la sublime grandeza y la magnífica perfección de las instalaciones de orden espiritual de este mundo sede! Por mucho que vuestra imaginación pueda llegar a concebir la perfección de la belleza y la plenitud de medios empleados, sería difícil que se pudiesen aproximar a estas grandiosidades. Y Jerusem no es sino el primer paso en el camino hacia la suprema perfección que caracteriza la belleza del Paraíso.
\usection{5. LOS CÍRCULOS DE JERUSEM}
\vs p046 5:1 Se les da el nombre de “círculos de Jerusem” a las zonas residenciales reservadas para los principales grupos de vida del universo. En esta narración se mencionan los siguientes conjuntos de círculos:
\vs p046 5:2 \li{1.}Los círculos de los Hijos de Dios.
\vs p046 5:3 \li{2.}Los círculos de los ángeles y de los espíritus de superior rango.
\vs p046 5:4 \li{3.}Los círculos de los auxiliares del universo, incluyendo a los hijos trinitizados por criaturas no asignados a los hijos preceptores de la Trinidad.
\vs p046 5:5 \li{4.}Los círculos de los controladores físicos mayores.
\vs p046 5:6 \li{5.}Los círculos de los mortales ascendentes seleccionados, incluyendo a las criaturas intermedias.
\vs p046 5:7 \li{6.}Los círculos de las colonias de cortesía.
\vs p046 5:8 \li{7.}Los círculos del colectivo final.
\vs p046 5:9 \pc Cada uno de estos agrupamientos residenciales consiste en siete círculos concéntricos que se elevan sucesivamente. Todos ellos están construidos conforme a unas mismas directrices pero tienen tamaños diferentes y están fabricados con materiales distintos. Todos están rodeados por recintos de grandes proporciones, que se erigen formando espaciosos paseos envolviendo por completo a cada uno de los siete conjuntos de círculos concéntricos.
\vs p046 5:10 \li{1.}\bibemph{Los círculos de los Hijos de Dios.} Aunque los Hijos de Dios poseen su propio planeta de socialización ---uno de los mundos de cultura y transición---, también ocupan en Jerusem estas extensas áreas. En dicho mundo, los ascendentes mortales se relacionan libremente con todos los órdenes de filiación divina. Allí llegaréis a conocerlos personalmente y a amarlos, si bien su vida social se ciñe, considerablemente, a ese mundo especial y a sus satélites. Sin embargo, en los círculos de Jerusem, se puede observar a estos distintos grupos de filiación en sus menesteres. Y puesto que la visión morontial tiene un inmenso alcance, podréis caminar por los paseos de los Hijos de Dios y contemplar la fascinante labor de sus numerosos órdenes.
\vs p046 5:11 Los siete círculos de los Hijos de Dios son concéntricos y se elevan de forma consecutiva, de tal manera que cada uno de los círculos exteriores, de mayor tamaño, tiene vistas a los círculos interiores, de menor tamaño. Cada cual está rodeado de un muro construido con resplandecientes gemas de cristal que sirve de paseo público. Estos muros son tan elevados que es factible visualizar desde ellos la totalidad de sus respectivos círculos residenciales. Las numerosas puertas ---entre cincuenta y ciento cincuenta mil---, que dan acceso a estos muros, están hechas de cristales nacarados individuales.
\vs p046 5:12 El primero de los círculos de esta área reservada a los hijos divinos está ocupado por los hijos magistrados y sus asistentes personales. Aquí se centran todos los planes y la actividad directa pertinentes a los servicios de gracia y judiciales de estos juristas. También, a través de este centro de actividad, los avonales del sistema se mantienen en contacto con el universo.
\vs p046 5:13 El segundo círculo está ocupado por los hijos preceptores de la Trinidad. En esta área sagrada, los dainales y sus colaboradores llevan a cabo la formación de los hijos preceptores primarios recién incorporados. Para esta tarea cuentan con la eficiente ayuda de una división de ciertos coiguales de las brillantes estrellas vespertinas. Los hijos trinitizados por criaturas ocupan un sector dentro del círculo de los dainales o preceptores de la Trinidad que, al tener, al menos, origen en la Trinidad, son los que más próximos están de representar al Padre Universal en un sistema local. Este segundo círculo constituye un área de extraordinario interés para todos los pobladores de Jerusem.
\vs p046 5:14 El tercer círculo está asignado a los melquisedecs. Aquí residen los jefes del sistema, encargados de supervisar la casi interminable actividad de estos versátiles hijos. Los melquisedecs son los padres adoptivos y asesores permanentes de los mortales ascendentes durante toda su andadura en Jerusem, desde que llegan al primero de los mundos de las moradas. Dejando al margen la labor siempre presente que realizan los hijos e hijas materiales, no sería equivocado afirmar que la que llevan a cabo los melquisedecs ejerce una predominante influencia en Jerusem.
\vs p046 5:15 El cuarto círculo es el hogar de los vorondadecs y de todos los otros órdenes de Hijos de Dios visitantes y observadores que no tienen otro acomodo. Los padres altísimos de la constelación establecen su domicilio en este círculo cuando se hallan en visitas de inspección en el sistema local. Los perfeccionadores de la sabiduría, los consejeros divinos y los censores universales residen en él cuando están de servicio en el sistema.
\vs p046 5:16 El quinto círculo es la morada de los lanonandecs, el orden de filiación de los soberanos de los sistemas y de los príncipes planetarios. Estos tres grupos de lanonandecs interactúan entre sí como uno solo cuando residen en esta área. En dicho círculo se encuentran las reservas de lanonandecs del sistema; el soberano del sistema dispone, en la colina de la administración, de un templo situado en el centro del grupo de edificios de gobernación.
\vs p046 5:17 El sexto círculo es el lugar de estancia de los portadores de vida del sistema. Aquí se congregan todas las categorías de estos hijos divinos, y desde aquí salen hacia sus mundos de destino.
\vs p046 5:18 El séptimo círculo es el punto de encuentro de los hijos ascendentes, esto es, de aquellos mortales destinados, junto con sus acompañantes seráficos, a desempeñar temporalmente actos de servicio en la sede del sistema. Todos aquellos que una vez fueron mortales y que gozan de una condición superior a la de ciudadanos de Jerusem e inferior a la de finalizadores se consideran como pertenecientes al grupo que tiene su sede en este círculo.
\vs p046 5:19 Los círculos reservados a los Hijos de Dios ocupan una enorme extensión y tenían, en su centro, hace mil novecientos años, un gran espacio abierto. Sin embargo, esta zona central está ahora ocupada por el monumento en recuerdo de Miguel, terminado hace quinientos años. Hace cuatrocientos noventa y cinco años, cuando se le dedicó este templo, Miguel acudió en persona, y todo Jerusem oyó el emocionante relato del ministerio de gracia de un hijo mayor en Urantia, el más pequeño de los planetas de Satania. En la actualidad, este monumento constituye el centro de todos los menesteres relacionados con la gestión del sistema, modificada como consecuencia de este ministerio de gracia, sin exceptuar la mayor parte de la actividad recientemente transferida desde Lugar de Salvación. La dotación de este monumento suma más de un millón de seres personales.
\vs p046 5:20 \li{2.}\bibemph{Los círculos de los ángeles}. Al igual que el área residencial de los Hijos de Dios, los círculos de los ángeles constan de siete círculos concéntricos que se elevan de forma sucesiva, cada cual con vistas a las zonas interiores.
\vs p046 5:21 \pc El primer círculo de los ángeles está ocupado por los seres personales superiores del Espíritu Infinito, tales como los mensajeros solitarios y sus colaboradores, que puedan estar emplazados en el mundo sede. El segundo círculo está dedicado a las multitudes de mensajeros, a los asesores técnicos, a los acompañantes, a los inspectores y a los archivistas que puedan estar realizando actos de servicio periódicos en Jerusem. El tercer círculo pertenece a los espíritus servidores de los órdenes y agrupaciones de seres de índole superior.
\vs p046 5:22 El cuarto círculo está asignado a los serafines gestores, y los serafines que sirven en un sistema local como Satania forman una “innumerable multitud de ángeles”. El quinto círculo lo ocupan los serafines planetarios, mientras que el sexto es el hogar de los servidores de las criaturas en transición. El séptimo círculo es el área de estancia de ciertos órdenes de serafines no revelados. Los archivistas de todos estos grupos de ángeles no residen con sus compañeros, sino que lo hacen en el templo de los archivos de Jerusem. En este triple edificio, todos los archivos se conservan por triplicado. En la sede de un sistema, los registros siempre se guardan en forma material, morontial y espiritual.
\vs p046 5:23 Estos siete círculos están rodeados por las exposiciones panorámicas de Jerusem, de cinco mil millas regulares de circunferencia, dedicadas a la presentación en su estatus evolutivo de los mundos poblados de Satania. Las exposiciones sufren revisiones constantes con el fin de poder verdaderamente representar las condiciones actualizadas de los distintos planetas. Sin duda, este inmenso lugar de paseo con vistas a los círculos de los ángeles será la primera visión de Jerusem que captará vuestra atención cuando, en vuestras primeras visitas, se os permita disponer de abundante tiempo libre.
\vs p046 5:24 Las exposiciones están a cargo de los nativos de Jerusem, pero cuentan con la ayuda de los ascendentes de los distintos mundos de Satania que, en su camino a Edentia, hacen estancia en Jerusem. Para la descripción de las condiciones planetarias y del progreso de los mundos se utilizan numerosas técnicas. Aunque conocéis algunas, muchas de ellas son desconocidas en Urantia. Estas exposiciones ocupan la margen externa de este inmenso muro; el resto del paseo está casi completamente vacío y extraordinaria y magníficamente embellecido.
\vs p046 5:25 \li{3.}\bibemph{Los círculos de los auxiliares del universo,} en su enorme espacio central, albergan la sede de las estrellas vespertinas. Aquí se encuentra la sede de Galantia en el sistema, el jefe adjunto de este poderoso grupo de superángeles y el primero en alcanzar dicho nombramiento entre todas las estrellas vespertinas ascendentes. Aunque sea una de las construcciones más recientes, se trata de uno de los sectores administrativos de mayor magnificencia de Jerusem. Este punto central tiene cincuenta kilómetros de diámetro. La sede de Galantia es un cristal fundido de una pieza, totalmente transparente. Estos cristales de naturaleza morontial y material cuentan con un gran aprecio entre los seres morontiales y los materiales. Las estrellas vespertinas creadas ejercen su influencia sobre todo Jerusem, siendo poseedores de tales atributos extrapersonales. Todo este mundo está envuelto en fragancia espiritual desde que mucha de su actividad se transfirió aquí desde Lugar de Salvación.
\vs p046 5:26 \li{4.}\bibemph{Los círculos de los controladores físicos mayores}. Los distintos órdenes de estos controladores físicos se distribuyen concéntricamente alrededor del inmenso templo de la potencia, que preside el jefe de la potencia del sistema en colaboración con el jefe de los supervisores de la potencia morontial. Este templo es uno de los dos sectores de Jerusem donde no se permite el acceso de los mortales ascendentes ni de las criaturas intermedias; el otro es el sector de desmaterialización, en el área de los hijos materiales, compuesto de una serie de laboratorios en los que los serafines de transporte transforman a los seres materiales en un estado bastante parecido al del orden morontial de existencia.
\vs p046 5:27 \li{5.}\bibemph{Los círculos de los mortales ascendentes}. El área central de estos círculos lo ocupa un grupo de 619 zonas monumentales planetarias que representan a los mundos habitados del sistema. Periódicamente, estas construcciones experimentan grandes cambios. Los mortales de cada uno de los mundos gozan del privilegio de poder convenir ocasionalmente ciertas modificaciones o incorporaciones a sus monumentos planetarios. Incluso, en este momento, se están efectuando muchos cambios en las construcciones dedicadas a Urantia. En el centro de estos 619 templos, hay un modelo prototipo de Edentia y de sus numerosos mundos de culturización de los ascendentes. Este modelo tiene un diámetro de cuarenta millas y es una reproducción exacta de la organización de Edentia, fiel al original en cada uno de sus detalles.
\vs p046 5:28 Para los ascendentes, es un placer servir en Jerusem y contemplar el quehacer de otros grupos. Todo lo que se hace en estos distintos círculos está completamente abierto a la observación de la totalidad de Jerusem.
\vs p046 5:29 En dicho mundo se desarrollan tres tipos diferentes de actividad: labor, progreso y esparcimiento. Esto es, servicio, estudio y ocio. Dicho de otra manera, son: interacción social, esparcimiento en grupos y adoración divina. La interrelación con diversos grupos de seres personales, con órdenes de seres muy distintos a los de uno mismo, es de un gran valor educativo.
\vs p046 5:30 \li{6.}\bibemph{Los círculos de las colonias de cortesía}. Los siete círculos de las colonias de cortesía están embellecidos por tres construcciones de grandes proporciones: el inmenso observatorio astronómico de Jerusem, la gigantesca galería de arte de Satania y el enorme salón de actos de los directores de reversión, que conforman el escenario de la actividad morontial dedicada al reposo y al esparcimiento.
\vs p046 5:31 Los artesanos celestiales dirigen la labor de los espornagias y proveen esa gran cantidad de ornamentos creativos y de monumentos conmemorativos que abundan en cualquier espacio de reunión público. Los talleres de los artesanos se encuentran entre los más grandes y bellos de todas las inigualables construcciones de este extraordinario mundo. Las otras colonias de cortesía mantienen unas sedes amplias y excelentes. Muchos de estos edificios están construidos en su totalidad de gemas de cristal. Todos los mundos arquitectónicos están repletos de cristal y de los llamados metales preciosos.
\vs p046 5:32 \li{7.}\bibemph{Los círculos de los finalizadores} contienen, en su centro, una singular construcción. Y este mismo templo vacío se encuentra por todo Nebadón, en cada uno de los mundos sedes de sus sistemas. En Jerusem, dicho edificio está sellado con la insignia de Miguel y porta la siguiente inscripción: “Por dedicar a la séptima etapa espiritual ---el servicio eterno---”. Gabriel selló este misterioso templo y nadie, excepto Miguel, podrá romper el sello de su soberanía colocado por la brillante estrella de la mañana. Algún día contemplaréis este templo silencioso, aunque no podáis comprender su misterio.
\vs p046 5:33 \pc \bibemph{Otros círculos de Jerusem:} Además de estos círculos residenciales, existen en Jerusem numerosos otros igualmente destinados a viviendas.
\usection{6. LOS CUADRADOS EJECUTIVOS\hyp{}ADMINISTRATIVOS}
\vs p046 6:1 Las divisiones ejecutivas\hyp{}administrativas del sistema se encuentran situadas en los inmensos cuadrados departamentales, cuyo número asciende a mil. Cada unidad de gobernación consta de cien subdivisiones de diez subgrupos cada una. Estos mil cuadrados se distribuyen en diez grandes sectores, conformándose así los diez departamentos administrativos siguientes:
\vs p046 6:2 \li{1.}Mantenimiento físico y mejoramiento material, los ámbitos de la potencia y de la energía físicas.
\vs p046 6:3 \li{2.}Arbitraje, ética y juicio administrativo.
\vs p046 6:4 \li{3.}Asuntos planetarios y locales.
\vs p046 6:5 \li{4.}Asuntos de la constelación y del universo.
\vs p046 6:6 \li{5.}Educación y otra actividad de los melquisedecs.
\vs p046 6:7 \li{6.}Progreso físico planetario y del sistema, los ámbitos científicos de la actividad que se desarrolla en Satania.
\vs p046 6:8 \li{7.}Asuntos morontiales.
\vs p046 6:9 \li{8.}Actividad y ética puramente espirituales.
\vs p046 6:10 \li{9.}Ministerio a los ascendentes.
\vs p046 6:11 \li{10.}Filosofía del gran universo.
\vs p046 6:12 \pc Estas construcciones son transparentes; de ahí que incluso los visitantes estudiantiles puedan ver toda la actividad que tiene lugar en relación al sistema.
\usection{7. LOS RECTÁNGULOS: LOS ESPORNAGIAS}
\vs p046 7:1 Los mil \bibemph{rectángulos} de Jerusem dan cabida a los seres nativos de menor rango de este planeta sede; en el centro de estos se encuentra la inmensa sede circular de los espornagias.
\vs p046 7:2 En Jerusem los logros agrícolas de los maravillosos espornagias os llenarán de asombro. Allí la tierra se cultiva, en gran parte, con un propósito estético y de ornamentación. Los espornagias son los jardineros paisajistas de los mundos sedes y, en su tratamiento de los espacios abiertos de Jerusem, son a la vez originales y artísticos. En el cultivo de la tierra, los espornagias utilizan tanto a animales como a un gran número de dispositivos mecánicos; hacen un uso experto e inteligente de los mecanismos de la potencia del entorno a su cargo al igual que de numerosos órdenes de hermanos suyos menores pertenecientes a creaciones animales de inferior rango, muchos de los cuales se les facilitan en estos mundos especiales. En la actualidad, este orden de vida animal está en gran parte dirigido por criaturas intermedias ascendentes procedentes de las esferas evolutivas.
\vs p046 7:3 En los espornagias no residen los modeladores. No poseen almas que sobrevivan, pero gozan de una larga vida, que puede alcanzar a veces hasta cuarenta o cincuenta mil años regulares. Su número es ingente, y proporcionan su ministerio físico a todos los órdenes de seres personales del universo que necesiten este servicio de índole material.
\vs p046 7:4 \pc Aunque los espornagias no poseen ni desarrollan almas que puedan sobrevivir, aunque no tengan ser personal, sí desarrollan, no obstante, una individualidad que los capacita para la reencarnación. Cuando, al pasar el tiempo, los cuerpos físicos de estas criaturas singulares se deterioran por el uso y la edad, sus creadores, en colaboración con los portadores de vida, dan origen a nuevos cuerpos en los que los viejos espornagias vuelven a establecer su morada.
\vs p046 7:5 De todo el universo de Nebadón, los espornagias son las únicas criaturas que experimentan esta clase o cualquiera otra clase de reencarnación. Estos seres solo son receptivos a los primeros cinco espíritus asistentes de la mente; no responden a los espíritus de adoración ni al de sabiduría. Pero una mente sensible a estos cinco asistentes equivale a una totalidad o nivel sexto de la realidad, y este es el factor que continúa como identidad experiencial.
\vs p046 7:6 \pc Cuando intento describir a estas criaturas, útiles y extraordinarias, me faltan elementos con los que poder establecer comparaciones, ya que no existen animales en los mundos evolutivos con los que se puedan comparar. No son seres evolutivos puesto que los portadores de vida los diseñaron en su forma y condición actuales. Son bisexuales y procrean en la medida en que se les necesita para responder a las necesidades de una población creciente.
\vs p046 7:7 Quizás la mejor manera de evocar en las mentes de Urantia algo de la naturaleza de estas criaturas, bellas y serviciales, es diciendo que representan una combinación de los rasgos de un caballo fiel y de un perro cariñoso, y manifiestan una inteligencia que excede a la de los tipos superiores de chimpancé. Y, juzgándoles según los criterios de la belleza física de Urantia, son seres muy hermosos. Aprecian en mucho las atenciones que los residentes materiales y semimateriales de estos mundos arquitectónicos les dispensan. Tienen una visión que les permite reconocer ---además de a los seres materiales--- a las creaciones de carácter morontial, a los órdenes angélicos menores, a las criaturas intermedias y a algunos de los órdenes de seres personales espirituales de menor rango. No comprenden la adoración del Infinito ni captan la trascendencia del Eterno, pero, por el afecto que sienten hacia sus superiores, sí participan en las devociones espirituales externas de sus mundos.
\vs p046 7:8 \pc Hay quienes creen que, en una futura era del universo, estos fieles espornagias conseguirán librarse de su estado animal y alcanzar un valioso destino evolutivo de crecimiento intelectual progresivo e incluso de logro espiritual.
\usection{8. LOS TRIÁNGULOS DE JERUSEM}
\vs p046 8:1 Los asuntos puramente locales y rutinarios de Jerusem se dirigen desde cien \bibemph{triángulos}. Estas unidades se agrupan en torno a diez magníficas construcciones, sedes de la administración local de Jerusem. Alrededor de los triángulos hay una representación panorámica de la historia de la sede del sistema. Actualmente, se han prescindido de más de dos millas regulares de esta historia circular. Este sector se restablecerá en el momento en el que Satania sea readmitida como parte de la familia de la constelación. Siguiendo los decretos de Miguel, todo está previsto para tal acontecimiento, si bien, el tribunal de los ancianos de días aún no ha concluido su dictamen sobre los asuntos de la rebelión de Lucifer. Satania no puede volver a una hermandad plena con Norlatiadec mientras dé cabida a archirrebeldes, seres de alto origen que cayeron de la luz a las tinieblas.
\vs p046 8:2 Cuando Satania pueda volver al seno de la constelación, entonces se considerará la readmisión de los mundos aislados en la familia de planetas habitados del sistema, que viene acompañada de su restablecimiento a la comunión espiritual de los mundos. Pero incluso si Urantia quedara restablecida en las vías circulatorias del sistema, vosotros seguiríais en una situación difícil por el hecho de que todo vuestro sistema permanece en cuarentena por disposición de Norlatiadec, que la separaba parcialmente de todos los demás sistemas.
\vs p046 8:3 \pc Pero en poco tiempo, el veredicto sobre Lucifer y de sus colaboradores restituirá el sistema de Satania a la constelación de Norlatiadec y, posteriormente, Urantia y las demás esferas aisladas regresarán a las vías de Satania y, una vez más, estos mundos podrán gozar del privilegio de la comunicación interplanetaria y de la comunión entre sistemas.
\vs p046 8:4 \pc Llegará el fin para los rebeldes y la rebelión. Los gobernantes supremos son misericordiosos y pacientes, pero la ley en relación al mal deliberado se ejecuta de modo universal e infalible. “La paga del pecado es muerte” ---la obliteración eterna---.
\vsetoff
\vs p046 8:5 [Exposición de un arcángel de Nebadón.]
