\upaper{90}{Chamanismo: Curanderos y sacerdotes}
\author{Melquisedec}
\vs p090 0:1 Los ceremoniales religiosos fueron evolucionando desde el apaciguamiento, la evitación, el exorcismo, la coacción, la conciliación y la propiciación hasta el sacrificio, la expiación y la redención. El modo de ritualización pasó desde las prácticas del sistema de culto primitivo, a través de los fetiches, a la magia y a los milagros; y, conforme el ritual se hizo más complejo como respuesta al concepto cada vez más intrincado del hombre sobre los ámbitos supramateriales, este se vio inevitablemente dominado por los curanderos, los chamanes y los sacerdotes.
\vs p090 0:2 Al avanzar el hombre primitivo en estos conceptos, se llegó a creer que el mundo de los espíritus era insensible a los mortales comunes. Solo algunos seres humanos excepcionales podían captar la atención de los dioses y solo ciertos hombres o mujeres extraordinarios podían ser oídos por los espíritus. En consecuencia, la religión entra en una nueva fase, en una etapa en la que, de forma gradual, se convierte en algo ajeno; siempre ha de mediar un curandero, un chamán o un sacerdote entre el devoto religioso y el objeto de adoración. Y, hoy en día, en Urantia, la mayoría de los sistemas de creencias religiosas organizadas están pasando por este nivel de desarrollo evolutivo.
\vs p090 0:3 La religión evolutiva nace de un temor simple y todopoderoso, de un temor que surge de la mente humana cuando se enfrenta a lo desconocido, a lo inexplicable y a lo incomprensible. Con el tiempo, la religión llegará a tomar conciencia, de forma profunda y sencilla, de un amor todopoderoso, de un amor que recorre el alma humana cuando esta despierta a la apreciación del afecto ilimitado del Padre Universal por sus hijos del universo. Pero, entre el comienzo y la culminación de la evolución religiosa, median largas eras de chamanes, que se atreven a interponerse entre el hombre y Dios como intermediarios, intérpretes e intercesores.
\usection{1. LOS PRIMEROS CHAMANES: LOS CURANDEROS}
\vs p090 1:1 El chamán era el curandero de mayor rango, el hombre fetiche de las ceremonias y la persona central en todas las prácticas de la religión evolutiva. En muchos grupos, el chamán estaba por encima del jefe de guerra, dándose así el comienzo del dominio de la Iglesia sobre el Estado. A veces, el chamán actuaba como sacerdote e incluso como sacerdote\hyp{}rey. Más adelante, en algunas tribus, existían primitivos chamanes\hyp{}curanderos (videntes) al igual que chamanes\hyp{}sacerdotes, de más tarde aparición. Y, en muchos casos, el cargo de chamán se convirtió en hereditario.
\vs p090 1:2 Desde los tiempos antiguos, cualquier anormalidad se atribuía a la posesión de los espíritus; cualquier deficiencia mental o física impactante cualificaba para ser curandero. Muchos de estos hombres eran epilépticos y muchas de las mujeres estaban afectadas de histeria; y estos dos tipos son el origen de la inspiración antigua y de la posesión de espíritus y demonios. Muchos de estos primeros sacerdotes pertenecían a una clase que desde entonces se ha denominado como paranoica.
\vs p090 1:3 Aunque pueden haber actuado con engaño en cuestiones menores, la gran mayoría de los chamanes creían en el hecho de estar poseídos por los espíritus. Las mujeres que eran capaces de entrar en trance o en estado cataléptico se volvieron poderosas mujeres chamanes; más tarde, se convirtieron en profetas y médiums espiritualistas. Por lo general, sus trances catalépticos entrañaban presuntas comunicaciones con los espectros de los muertos. Multitud de chamanes femeninas eran también bailarinas profesionales.
\vs p090 1:4 Pero no todos los chamanes se engañaban a sí mismos; muchos eran astutos y hábiles embaucadores. A medida que la profesión se desarrolló, se exigió a los aprendices un periodo de formación de diez años de penurias y abnegación para capacitarse como curanderos. Los chamanes elaboraron una vestimenta profesional y mostraban un comportamiento misterioso. Con frecuencia, empleaban drogas para provocarse ciertos estados físicos que impresionasen y desconcertaran a sus compañeros de tribu. La gente común consideraba sobrenaturales los trucos de prestidigitación, y la ventriloquía se usó primeramente por algunos taimados sacerdotes. Muchos de los chamanes antiguos descubrieron involuntariamente el hipnotismo; otros se inducían a sí mismos a la hipnosis mirándose fijamente el ombligo durante largo rato.
\vs p090 1:5 Aunque muchos recurrieron a estas artimañas y engaños, su reputación, como clase, después de todo, se basaba en logros aparentes. Si un chamán fracasaba en sus iniciativas y era incapaz de ofrecer un pretexto convincente, se le degradaba o mataba. Por tanto, los chamanes honrados perecían pronto; solo aquellos que actuaban con astucia sobrevivían.
\vs p090 1:6 El chamanismo quitó la dirección exclusiva de los asuntos de la tribu de las manos de los ancianos y de los fuertes y los depositó en las de los astutos, los listos y los clarividentes.
\usection{2. PRÁCTICAS CHAMÁNICAS}
\vs p090 2:1 La invocación de los espíritus era un procedimiento muy preciso y sumamente complicado, comparable a los rituales de la Iglesia de hoy en día llevados a cabo en una lengua ancestral. Muy pronto, la raza humana buscó la ayuda sobrehumana, la \bibemph{revelación,} y los hombres creían que el chamán realmente era depositario de tales revelaciones. Aunque los chamanes utilizaban en su trabajo el gran poder de la sugestión, era casi invariablemente una sugestión de tipo negativo; solo en tiempos muy recientes se ha empleado el método de la sugestión positiva. En el desarrollo temprano de su profesión, los chamanes comenzaron a especializarse en tales ocupaciones como la propiciación de lluvia, la sanación de enfermedades y la detección de delitos. No obstante, la sanación de las enfermedades no era la labor principal de los curanderos chamánicos, sino más bien la de conocer y combatir las contingencias de la vida.
\vs p090 2:2 La antigua magia negra, tanto religiosa como laica, se denominaba magia blanca cuando la practicaban los sacerdotes, videntes, chamanes o curanderos. A los que practicaban la magia negra se les llamaba hechiceros, magos, agoreros, brujos, encantadores, nigromantes, exorcistas y adivinos. Con el paso del tiempo, todos estos pretendidos contactos con lo sobrenatural se clasificaron como brujería o chamanismo.
\vs p090 2:3 La brujería comprendía la \bibemph{magia,} que se llevaba a cabo por espíritus primitivos, erráticos y no reconocidos; el chamanismo guardaba relación con los \bibemph{milagros} realizados por espíritus ordinarios y por los dioses reconocidos de la tribu. En tiempos posteriores, el brujo se vinculó al diablo y, de este modo, se prepararía el terreno para las numerosas manifestaciones, relativamente recientes, de intolerancia religiosa. Para muchas tribus primitivas, la brujería era una religión.
\vs p090 2:4 Los chamanes creían profundamente en la misión del azar como elemento revelador de la voluntad de los espíritus; con frecuencia echaban suertes para adoptar decisiones. Los vestigios modernos de esta propensión a echar suertes quedan ilustrados no solo en sus muchos juegos de azar, sino también en las conocidas rimas de “descarte”. En otro tiempo, la persona descartada debía morir; ahora es tan solo un juego infantil en el que queda \bibemph{fuera} el niño que se señala al acabar la rima. Lo que había sido un asunto serio para el hombre primitivo ha sobrevivido como una diversión del niño moderno.
\vs p090 2:5 Los curanderos confiaban sobremanera en signos y presagios, tales como, “cuando oigas el sonido de un susurro por las copas de las moreras, sal entonces”. Muy pronto en la historia de la raza, los chamanes centraron su atención en las estrellas. La astrología primitiva era una creencia que se practicaba en todo el mundo; la interpretación de los sueños también se convirtió en una práctica generalizada. Todo esto fue seguido pronto por la aparición de las temperamentales mujeres chamanes que manifestaban poder comunicarse con los espíritus de los muertos.
\vs p090 2:6 Aunque su origen es ancestral, los hacedores de lluvia, o chamanes del tiempo, han persistido a través de la historia. Para los agricultores primitivos, una sequía grave significaba la muerte; gran parte de la magia antigua tenía como objetivo controlar el tiempo atmosférico. El hombre civilizado aún hace del tiempo un tema común de conversación. Los pueblos antiguos creían en el poder del chamán como artífice de la lluvia, pero era costumbre matarlo cuando fracasaba, a menos que pudiera ofrecer un pretexto verosímil para justificar su fracaso.
\vs p090 2:7 Los césares desterraron repetidamente a los astrólogos, pero ellos invariablemente regresaban debido a la creencia popular en sus poderes. No podían expulsarlos e, incluso en el siglo XVI d. C., los dirigentes de la Iglesia Occidental y del Estado eran los mecenas de la astrología. Miles de personas supuestamente inteligentes aún creen que alguien puede nacer bajo el influjo afortunado o desafortunado de las estrellas; que la yuxtaposición de los cuerpos celestes determina el resultado de los distintos sucesos terrenales. Los adivinos tienen todavía la condescendencia de los crédulos.
\vs p090 2:8 Los griegos creían en la eficacia del consejo del oráculo, los chinos utilizaban la magia como protección contra los demonios, el chamanismo floreció en India y aún continúa abiertamente en Asia central. Es una práctica que solo recientemente se ha abandonado en gran parte del mundo.
\vs p090 2:9 Ocasionalmente, surgieron verdaderos profetas y maestros que denunciaron y desenmascararon el chamanismo. Incluso el hombre rojo, en vías de desaparición, tuvo tal profeta en los últimos cien años, el shawnee Tenskwatawa, que predijo el eclipse de sol de 1808 y denunció los vicios del hombre blanco. Durante las largas eras de la historia evolutiva, han surgido muchos auténticos maestros entre las distintas tribus y razas. Y siempre continuarán apareciendo para retar a los chamanes o sacerdotes de cualquier época que se opongan a la educación general y traten de obstaculizar el progreso científico.
\vs p090 2:10 De muchas maneras y mediante retorcidos métodos, los antiguos chamanes establecieron su reputación como voceros de Dios y custodios de la providencia. Rociaban con agua al recién nacido y le otorgaban un nombre; circuncidaban a los varones. Presidían todas las ceremonias fúnebres y anunciaban la llegada segura del muerto a la tierra de los espíritus.
\vs p090 2:11 A menudo, los sacerdotes chamánicos y los curanderos se volvían muy ricos mediante el acopio de diferentes tasas, que supuestamente se ofrendaban a los espíritus. No era raro que un chamán acumulase prácticamente toda la riqueza material de su tribu. Cuando moría un hombre rico, se acostumbraba a dividir su patrimonio en partes iguales entre el chamán y alguna iniciativa pública o caritativa. Esta práctica todavía continúa en algunas partes del Tíbet, donde la mitad de la población masculina pertenece a esta improductiva clase.
\vs p090 2:12 Los chamanes vestían bien y solían tener varias esposas; fueron la primera aristocracia y estaban exentos de todas las restricciones tribales. En muchos casos, su nivel mental y su sentido moral eran ínfimos. Reprimían a sus rivales acusándolos de brujería o hechicería y, con bastante frecuencia, alcanzaban puestos de tal influencia y poder que podían dominar a jefes o reyes.
\vs p090 2:13 El hombre primitivo consideraba al chamán como un mal necesario; le temía pero no lo estimaba. El hombre primitivo respetaba el conocimiento; honraba y premiaba la sabiduría. El chamán era principalmente un farsante, pero la veneración por el chamanismo pone de manifiesto la importancia que se le dio a la sabiduría en la evolución de la raza.
\usection{3. TEORÍA CHAMÁNICA DE LA ENFERMEDAD Y DE LA MUERTE}
\vs p090 3:1 Puesto que el hombre de la antigüedad se consideraba a sí mismo y a su entorno material como directamente receptivos a las veleidades de los espectros y a los antojos de los espíritus, no es de extrañar que su religión se ocupara tan particularmente de cuestiones materiales. El hombre moderno interviene directamente en sus problemas materiales; reconoce que la materia responde a la dirección inteligente de la mente. De igual manera, el hombre primitivo deseaba modificar e incluso controlar la vida y las energías de su entorno físico; y dado que su limitada comprensión del cosmos lo llevó a la idea de que los espectros, los espíritus y los dioses estaban personal y directamente preocupados por el control minucioso de la vida y de la materia destinó sus esfuerzos a lograr el favor y el apoyo de estas instancias sobrehumanas.
\vs p090 3:2 Desde esta perspectiva, gran parte de lo inexplicable e irracional de los sistemas de culto ancestrales resulta comprensible. Sus ceremonias significaban un intento del hombre primitivo por controlar el mundo material en el que se hallaba. Y muchos de sus esfuerzos se encaminaban a prolongar la vida y a asegurar la salud. Dado que todas las enfermedades y la muerte misma se consideraban inicialmente como fenómenos espirituales, era inevitable que los chamanes, al actuar como curanderos y sacerdotes, también lo hicieran como médicos y cirujanos.
\vs p090 3:3 La mente primitiva puede verse obstaculizada por falta de información, pero, a pesar de todo, es lógica. Cuando los hombres reflexivos observan la enfermedad y la muerte, proceden a determinar las causas de su aparición y, de acuerdo con su comprensión, los chamanes y los científicos han propuesto las siguientes teorías de la aflicción:
\vs p090 3:4 \li{1.}\bibemph{Los espectros: influencias directas de los espíritus}. La hipótesis más temprana que se formulaba para explicar la enfermedad y la muerte consistía en que los espíritus producían las enfermedades al seducir al alma a que saliese fuera del cuerpo; si esta no lograba regresar, se originaba la muerte. Los antiguos temían tanto la acción malévola de los espectros causantes de las enfermedades que a menudo abandonaban a las personas enfermas sin ni siquiera alimentos ni agua. A pesar del carácter erróneo de estas creencias, se aislaba efectivamente a las personas afectadas para evitar la diseminación de enfermedades contagiosas.
\vs p090 3:5 \li{2.}\bibemph{La violencia: causas obvias}. Las causas de algunos accidentes y muertes eran tan fácil de identificar que pronto se eliminaron de aquellas producidas por la acción de los espectros. Los casos de muerte y de heridas derivados de la guerra, del enfrentamiento con los animales y de otras circunstancias fácilmente identificables se consideraban como sucesos naturales. Pero, durante mucho tiempo, se creyó que los espíritus eran responsables por el retraso de la curación o por la infección de las heridas producidas incluso por causa “natural”. Si no se podía descubrir ningún agente natural observable, se continuaba culpando a los espíritus y a los espectros de la enfermedad y la muerte.
\vs p090 3:6 Hoy día, en África y en otros lugares, es posible encontrar pueblos primitivos que matan a alguien cada vez que ocurre una muerte no violenta. Los curanderos indican quiénes son los culpables. Si una madre muere de parto, se estrangula al niño de inmediato: una vida por otra.
\vs p090 3:7 \li{3.}\bibemph{La magia: influencia de los enemigos}. Se pensaba que los hechizos, el mal de ojo y el arco mágico señalador causaban un gran número de enfermedades. En cierto momento, era realmente peligroso señalar con el dedo a alguien; incluso hoy en día, se considera de mala educación. En casos de enfermedad y muerte poco claras, los antiguos realizaban una investigación formal, diseccionaban el cuerpo y establecían en base a algún hallazgo la causa de la muerte; de lo contrario, se atribuía esta a la brujería y se requería entonces la ejecución del brujo responsable. Estas antiguas pesquisas forenses salvaron la vida a muchos supuestos brujos. Había quienes creían que cualquier compañero de tribu podía morir como resultado de su propia brujería, en cuyo caso no se acusaba a nadie.
\vs p090 3:8 \li{4.}\bibemph{El pecado: castigo por violación del tabú}. En tiempos relativamente recientes, se ha creído que la enfermedad era un castigo por el pecado, personal o racial. Entre los pueblos que están atravesando este nivel de evolución, la teoría imperante es que no se puede estar aquejado a no ser que haya violado algún tabú. Es típico de tales creencias considerar la enfermedad y el sufrimiento como “flechas del Todopoderoso que se han clavado”. Durante mucho tiempo, los chinos y los mesopotámicos veían la enfermedad como resultado de la acción de demonios malignos, aunque los caldeos también suponían que las estrellas eran las causantes del sufrimiento. Esta teoría de las dolencias como consecuencia de la ira divina persiste aún entre muchos grupos de urantianos presumiblemente civilizados.
\vs p090 3:9 \li{5.}\bibemph{Las causas naturales}. La humanidad ha aprendido muy lentamente las claves materiales de la interrelación entre causa y efecto en los ámbitos físicos de la energía, la materia y la vida. Los antiguos griegos, al haber conservado las tradiciones de las enseñanzas de Adánez, figuran entre los primeros en reconocer que cualquier enfermedad se debía a causas naturales. Paulatina y ciertamente, el despliegue de la era científica está aboliendo las viejas teorías del hombre sobre la enfermedad y la muerte. La fiebre fue una de las primeras afecciones humanas que se excluyeron de aquellas atribuidas a desórdenes sobrenaturales y, progresivamente, la era de la ciencia ha roto las cadenas de la ignorancia que, por tanto tiempo, mantuvieron a la mente humana prisionera. El entendimiento de la vejez y del contagio está, de forma gradual, erradicando el temor del hombre a los espectros, espíritus y dioses como autores personales de las desgracias humanas y del padecimiento.
\vs p090 3:10 \pc Inequívocamente, la evolución consigue sus fines: infunde en el hombre ese temor supersticioso a lo desconocido y ese miedo a lo oculto, que es el andamiaje para llegar a la noción de Dios. Y habiendo presenciado el nacimiento de una comprensión avanzada de la Deidad, mediante la acción correlacionada de la revelación, este mismo método evolutivo pone en movimiento, de modo infalible, aquellas fuerzas del pensamiento que borran inexorablemente tal andamiaje, al haber servido su propósito.
\usection{4. LA MEDICINA BAJO LOS CHAMANES}
\vs p090 4:1 Toda la vida del hombre de la antigüedad era una cuestión de profilaxis; su religión era, en no poca medida, una forma de prevenir las enfermedades. Y con independencia de sus equivocadas teorías, los hombres las ponían en práctica de forma incondicional; tenían una fe ciega en sus métodos de tratamiento y eso, de por sí, es un poderoso remedio.
\vs p090 4:2 \pc La fe precisa para ponerse bien bajo los insensatos cuidados de uno de estos antiguos chamanes no era, después de todo, sustancialmente diferente de la que se requiere para sanar a manos de algunos de sus sucesores más recientes que tratan las enfermedades de forma no científica.
\vs p090 4:3 \pc Las tribus más primitivas tenían un gran temor a los enfermos y, durante mucho tiempo, se les evitaba cautelosamente, se les desatendía de forma vergonzosa. Se dio un gran avance en el humanitarismo cuando el desarrollo del chamanismo trajo consigo la aparición de sacerdotes y curanderos, que consintieron en tratar las enfermedades. Entonces se convirtió en costumbre que todo el clan se reuniese en la habitación del enfermo para ayudar al chamán a espantar con alaridos a los espectros que provocaban la enfermedad. No era infrecuente que fuese una mujer la chamán que hiciera el diagnóstico, mientras que el hombre administraba el tratamiento. El método habitual de diagnosticar la dolencia era examinar las entrañas de un animal.
\vs p090 4:4 La enfermedad se trataba con cantos, alaridos, imposiciones de manos, soplidos sobre el paciente y muchas otras prácticas. En tiempos más recientes, se generalizó la dormición en el templo, durante la cual supuestamente se llevaba a efecto la sanación. Finalmente, los curanderos ensayaron una verdadera cirugía con motivo de tal dormición; una de las primeras operaciones fue la de trepanar el cráneo para que el espíritu del dolor de cabeza pudiese escapar. Los chamanes aprendieron a tratar fracturas y dislocaciones, a abrir forúnculos y abscesos; las mujeres chamanes se convirtieron en expertas parteras.
\vs p090 4:5 Un habitual método de tratamiento consistía en frotar algo mágico sobre la parte infectada o dañada del cuerpo, deshacerse del amuleto y experimentar presuntamente la curación. Si alguien por casualidad levantaba el amuleto desechado, se creía que la persona de inmediato quedaría infectada o dañada. Pasó mucho tiempo antes de que se usaran hierbas y otros auténticos medicamentos. El masaje se desarrolló en conexión con el encantamiento, con la idea de frotar para expulsar al espíritu, y este estuvo precedido por la medida de introducir el medicamento en el cuerpo mediante fricciones, tal como la práctica moderna de untar linimentos. La aplicación de ventosas para la absorción de las partes afectadas, junto con la sangría, se creía de utilidad para librarse del espíritu causante de la enfermedad.
\vs p090 4:6 Debido a que el agua era un poderoso fetiche, se empleaba en el tratamiento de muchas dolencias. Durante bastante tiempo, se creyó que el espíritu que provocaba la enfermedad se podía eliminar a través del sudor. Los baños de vapor eran muy apreciados; los manantiales naturales de aguas termales se convirtieron pronto en sanatorios primitivos. El hombre primitivo descubrió que el calor aliviaba el dolor; usó la luz del sol, órganos frescos de animales, arcilla y piedras calientes, y muchos de estos métodos continúan utilizándose. Se practicaba el ritmo en un intento por influir en los espíritus; el tamtam estaba generalizado.
\vs p090 4:7 En algunos pueblos se pensaba que la maligna confabulación entre los espíritus y los animales provocaba la enfermedad. Esto dio pie a la creencia de que existía un remedio beneficioso de origen vegetal para cada una de las enfermedades causadas por los animales. Los hombres rojos estaban particularmente comprometidos con la teoría del uso de las plantas como remedio para todo tipo de dolencias; siempre ponían una gota de sangre en el agujero que dejaba la raíz cuando se arrancaba la planta.
\vs p090 4:8 A menudo se empleaban el ayuno, la dieta y los contrairritantes como medidas curativas. Las secreciones humanas, al ser claramente mágicas, gozaban de alta consideración; la sangre y la orina estaban entre los primeros remedios y se complementaron pronto con raíces y sales diversas. Los chamanes creían que las medicinas mal olientes y de mal sabor podían expulsar del cuerpo a los espíritus causantes de la enfermedad. Bien temprano, los purgantes se convirtieron en tratamientos rutinarios, y las propiedades del cacao y de la quinina crudos se cuentan entre los más tempranos descubrimientos farmacéuticos.
\vs p090 4:9 Los griegos fueron los primeros en desarrollar métodos verdaderamente racionales de tratar a los enfermos. Tanto ellos como los egipcios recibieron sus conocimientos médicos del valle del Éufrates. El aceite y el vino se emplearon muy tempranamente para sanar las heridas; los sumerios usaban el aceite de castor y el opio. Muchos de estos antiguos y eficaces remedios secretos perdían su efectividad cuando se hacían conocidos; el secretismo siempre ha sido esencial para la provechosa práctica del fraude y la superstición. Solo los hechos y la verdad cultivan la clara luz de la comprensión y se regocijan en la clarificación y elucidación de la investigación científica.
\usection{5. SACERDOTES Y RITUALES}
\vs p090 5:1 La esencia del ritual es la perfección de su consecución; entre los salvajes debe practicarse con precisión exacta. Solamente cuando el ritual se ha desempeñado con corrección, la ceremonia posee un poder apremiante sobre los espíritus. Si el ritual es defectuoso, solo despierta la ira y el resentimiento de los dioses. Por lo tanto, puesto que la mente del hombre, en su lenta evolución, concibió que el \bibemph{método del ritual} constituía el factor determinante de su eficacia, era inevitable que los primeros chamanes, tarde o temprano, evolucionaran hacia un sacerdocio capacitado para dirigir la práctica meticulosa del ritual. Y, así, durante miles y miles de años un sinfín de rituales ha alterado la sociedad y traído calamidad a la civilización, convirtiéndose en una carga insoportable para cualquier acto de la vida, para cualquier iniciativa de la raza humana.
\vs p090 5:2 El ritual es un modo de santificar la costumbre; crea y perpetúa mitos, al mismo tiempo que contribuye a la conservación de las costumbres sociales y religiosas. Por otra parte, los mitos generan rituales. Con frecuencia, los rituales son primeramente sociales, luego se vuelven lucrativos y, finalmente, adquieren la santidad y la dignidad de un ceremonial religioso. En su práctica, el ritual puede ser personal o colectivo ---o ambos---, tal como se muestra en la oración, en la danza y en las representaciones teatrales.
\vs p090 5:3 Las palabras se convierten en parte del ritual, como se observa en el uso de los términos amén y \bibemph{selah}. El hábito de maldecir, de usar un lenguaje profano, representa un envilecimiento de la antigua reiteración ritualista de nombres sagrados. Hacer peregrinajes a los santuarios sagrados es un ritual muy antiguo. Seguidamente, el ritual se transformó en elaboradas ceremonias de purificación, limpieza y santificación. Las ceremonias de iniciación de las primitivas sociedades secretas tribales eran realmente un burdo rito religioso. El modo de adoración de los antiguos cultos de misterio no eran sino una larga consecución de rituales religiosos acumulados. Por último, el ritual se convirtió en los tipos modernos de ceremoniales sociales y de adoración religiosa, unos servicios que incluyen oraciones, cánticos, lecturas corales y otras devociones espirituales individuales y colectivas.
\vs p090 5:4 \pc Los sacerdotes evolucionaron a partir de los chamanes, pasando por oráculos, adivinadores, cantores, bailarines, hacedores de lluvia, guardianes de reliquias religiosas, custodios del templo y profetizadores, hasta la condición de verdaderos directores de la adoración religiosa. Con el tiempo, el cargo se volvió hereditario, surgiendo una continuada casta sacerdotal.
\vs p090 5:5 A medida que se desarrollaba la religión, los sacerdotes comenzaron a especializarse conforme a sus talentos innatos o preferencias particulares. Algunos se convirtieron en cantores, otros en orantes e, incluso otros, en sacrificadores; más tarde, aparecieron los oradores ---los predicadores---. Cuando la religión se institucionalizó, estos sacerdotes alegaron que “tenían las llaves del cielo”.
\vs p090 5:6 Los sacerdotes siempre han tratado de impresionar e intimidar a la gente común dirigiendo el ritual religioso en una lengua antigua y mediante diversas posturas y movimientos mágicos, así desconciertan a los fieles y ensalzan su propia piedad y autoridad. El gran peligro de todo esto es que el ritual tiende a convertirse en el sustituto de la religión.
\vs p090 5:7 El sacerdocio ha hecho mucho por retrasar el desarrollo científico y dificultar el progreso espiritual, pero ha contribuido a estabilizar la civilización y a enaltecer determinadas clases de cultura. Si bien, muchos sacerdotes modernos han cesado de desempeñar su función como directores del ritual de adoración a Dios y han centrado su atención en la teología: el intento de definir a Dios.
\vs p090 5:8 No se puede negar que los sacerdotes han sido una piedra de molino atada al cuello de las razas, pero los auténticos líderes religiosos han sido de valor inestimable señalando el camino hacia realidades superiores y mejores.
\vsetoff
\vs p090 5:9 [Exposición de un melquisedec de Nebadón.]
