\upaper{72}{El gobierno de un planeta vecino}
\author{Melquisedec}
\vs p072 0:1 Con el permiso de Lanaforge y con la aprobación de los altísimos de Edentia, dispongo de autorización para narrar algunos aspectos de la vida política, moral y social de la más avanzada raza humana que habita en un planeta no muy lejano, perteneciente al sistema de Satania.
\vs p072 0:2 De todos los mundos de Satania que quedaron aislados por su participación en la rebelión de Lucifer, este planeta es el que presenta una historia más similar a la de Urantia. El parecido entre ambas esferas es lo que, sin duda, explica la razón por la que se concedió el permiso para llevar a cabo esta extraordinaria exposición, ya que resulta, de lo más inusual, que los gobernantes del sistema den su consentimiento a que se narren en un planeta los asuntos de otro.
\vs p072 0:3 Este planeta, al igual que Urantia, se descarrió a consecuencia de la deslealtad de su príncipe planetario con motivo de la rebelión de Lucifer. Recibió a un hijo material poco tiempo después de que Adán llegara a Urantia, y este hijo también incumplió su deber, dejando la esfera aislada, al no haberse nunca otorgado a sus razas mortales un hijo magistrado.
\usection{1. LA NACIÓN CONTINENTAL}
\vs p072 1:1 Al margen de estos obstáculos planetarios, en un continente aislado, aproximadamente del tamaño de Australia, está evolucionando una civilización muy superior. Esta nación cuenta con unos 140 millones de habitantes. Su población es de una raza mixta, predominantemente azul y amarilla; tienen una proporción algo mayor de violeta que la llamada raza blanca de Urantia. Estas diferentes razas aún no se han mezclado por completo, pero confraternizan y se relacionan entre sí de forma muy satisfactoria. La duración media de vida en este continente es ahora de noventa años, un quince por ciento más elevada que la de cualquier otra población del planeta.
\vs p072 1:2 El entramado industrial de esta nación goza ciertamente de grandes ventajas gracias a las peculiaridades de la topografía del continente. Las altas montañas, sobre las que se precipitan intensas lluvias durante ocho meses del año, están situadas en el centro mismo del país. Esta disposición natural favorece el aprovechamiento de la energía hidráulica y facilita en gran medida el regadío de la cuarta parte occidental del continente, de mayor aridez.
\vs p072 1:3 Esta población es autosuficiente, es decir, puede vivir por tiempo indefinido sin importar nada de las naciones que la rodean. Está bien provista de recursos naturales y, mediante métodos científicos, han aprendido cómo suplir sus carencias en cuanto a elementos esenciales para la vida. Gozan de un activo comercio interno, pero tienen poco comercio exterior a causa de la generalizada hostilidad de sus vecinos menos avanzados.
\vs p072 1:4 \pc En líneas generales, esta nación continental ha seguido la tendencia evolutiva planetaria: la evolución desde la etapa tribal hasta la aparición de mandatarios y reyes poderosos duró miles de años. A los monarcas absolutos le sucedieron muchos tipos distintos de gobierno: repúblicas fallidas, Estados comunales y dictadores que iban y venían de forma frecuente e interminable. Este desarrollo continuó hasta hace unos quinientos años cuando, durante un período políticamente agitado, uno de los poderosos dictadores\hyp{}triunviros de la nación cambió de parecer. Se ofreció a abdicar de forma voluntaria bajo la condición de que uno de los otros dos soberanos, el más abyecto de ellos, renunciara también a su dictadura. Así fue como la soberanía del continente quedó en manos de un solo dirigente. El Estado, unificado, progresó bajo un fuerte gobierno monárquico durante más de cien años, periodo en el que se elaboró un magistral estatuto de libertades.
\vs p072 1:5 La posterior transición de la monarquía a la forma representativa de gobierno se produjo de forma gradual; los reyes permanecieron como meras figuras sociales o sentimentales, para luego desaparecer al extinguirse la línea masculina de descendencia. La actual república lleva vigente solo doscientos años, tiempo durante el que ha habido un avance continuo hacia métodos de gobierno que se describen a continuación; ha sido durante la década pasada cuando han ocurrido los últimos desarrollos en los ámbitos político e industrial.
\usection{2. ORGANIZACIÓN POLÍTICA}
\vs p072 2:1 Actualmente, esta nación continental tiene un gobierno representativo cuya capital nacional está centralmente localizada. El gobierno central está formado por una gran federación de cien estados relativamente libres. Dichos estados eligen a sus gobernadores y legisladores por un período de diez años, y ninguno se puede considerar para la reelección. Los gobernadores nombran a los jueces estatales con carácter vitalicio, y son sus asambleas legislativas, que constan de un representante por cada cien mil ciudadanos, los que los confirman en sus cargos.
\vs p072 2:2 Existen cinco tipos diferentes de gobiernos metropolitanos, dependiendo del tamaño de la ciudad, pero a ninguna ciudad se le permite tener más de un millón de habitantes. En su conjunto, estos sistemas de gobiernos municipales son muy sencillos, directos y económicos. Los pocos puestos existentes en la gestión administrativa urbana son muy codiciados por los ciudadanos de mayor preeminencia.
\vs p072 2:3 El gobierno federal consta de tres divisiones del mismo rango: ejecutiva, legislativa y judicial. El mandatario federal en jefe se elige cada seis años por sufragio universal territorial, y no puede acogerse a la reelección salvo por petición de setenta y cinco asambleas legislativas estatales con la aceptación de sus respectivos gobernadores estatales; y, en tal caso, solo por un mandato más. Tiene el asesoramiento de un gabinete de elevado rango compuesto por todos los ex mandatarios en jefe existentes.
\vs p072 2:4 \pc La división legislativa consta de tres cámaras:
\vs p072 2:5 \li{1.}\bibemph{La cámara alta:} elegida por grupos de trabajadores industriales, trabajadores profesionales cualificados, trabajadores agrícolas o de cualquier otro tipo; la votación se efectúa con arreglo a su actividad económica.
\vs p072 2:6 \li{2.}\bibemph{La cámara baja:} elegida por determinadas organizaciones de la sociedad que abarcan grupos sociales, políticos y filosóficos, no incluidos los industriales ni los profesionales cualificados. Todos los ciudadanos de buena reputación participan en la elección de ambas clases de representantes, pero se agrupan de forma diferente según corresponda la elección a la cámara alta o a la baja.
\vs p072 2:7 \li{3.}\bibemph{La tercera cámara} ---los ancianos estadistas---: abarca a los veteranos del servicio cívico e incluye a muchas personas ilustres nombradas por el jefe del ejecutivo, por los mandatarios regionales (subfederales), por el jefe del tribunal supremo y por los líderes de una u otra cámara legislativa. El límite de integrantes de este grupo es de cien, los cuales se eligen mediante la decisión mayoritaria de los mismos ancianos estadistas. El nombramiento es vitalicio y, cuando se generan vacantes, está previsto que se elija, de entre los candidatos, al que más votos reciba. La competencia de este órgano es de carácter puramente consultivo, si bien, es un poderoso regulador de la opinión pública y ejerce una gran influencia sobre todas las ramas del gobierno.
\vs p072 2:8 \pc Una gran parte del trabajo de la administración federal lo llevan a cabo los diez organismos regionales (subfederales), que están compuestos, cada cual, de la asociación de diez estados. Estas divisiones regionales tienen carácter enteramente ejecutivo y administrativo; no tienen funciones ni legislativas ni judiciales. El jefe federal del ejecutivo nombra personalmente a los diez mandatarios regionales, cuya duración en el cargo ---seis años--- coincide con la suya. El tribunal supremo federal aprueba el nombramiento de estos diez mandatarios regionales y, aunque no se pueden volver a nombrar, el mandatario saliente se convierte automáticamente en compañero y asesor de su sucesor. Por otro lado, estos jefes regionales escogen sus propios gabinetes de funcionarios de la administración.
\vs p072 2:9 \pc En esta nación, la justicia se administra por medio de dos sistemas principales de tribunales: los tribunales de justicia y los tribunales socioeconómicos. Los tribunales de justicia operan en los tres siguientes niveles:
\vs p072 2:10 \li{1.}\bibemph{Los tribunales menores} de jurisdicción local y municipal, cuyas decisiones pueden recurrirse ante los altos tribunales estatales.
\vs p072 2:11 \li{2.}\bibemph{Los tribunales supremos} estatales cuyas resoluciones son definitivas en todas las cuestiones que no conciernan al gobierno federal o que pongan en peligro los derechos y libertades de la ciudadanía. Los mandatarios regionales están facultados para llevar de inmediato cualquier caso ante el tribunal supremo federal.
\vs p072 2:12 \li{3.}\bibemph{El tribunal supremo federal} ---el alto tribunal para la resolución de litigios nacionales y los casos de apelación procedentes de los tribunales estatales---. Este tribunal supremo está formado por doce hombres mayores de cuarenta años y menores de setenta y cinco, que hayan servido durante dos o más años en un tribunal estatal y que hayan sido nombrados para este alto cargo por el mandatario en jefe, con la aprobación mayoritaria del gabinete de elevado rango y de la tercera cámara de la asamblea legislativa. Todas las decisiones que este órgano judicial supremo toma precisan al menos de dos tercios de los votos.
\vs p072 2:13 \pc Los tribunales socioeconómicos operan en las siguientes tres secciones:
\vs p072 2:14 \li{1.}\bibemph{Los tribunales parentales,} relacionados con las divisiones legislativa y ejecutiva del sistema familiar y social.
\vs p072 2:15 \li{2.}\bibemph{Los tribunales educacionales:} organismos jurídicos vinculados con los sistemas escolares estatales y regionales, y relacionados con las ramas ejecutiva y legislativa del régimen administrativo de la educación.
\vs p072 2:16 \li{3.}\bibemph{Los tribunales industriales:} tribunales jurisdiccionales revestidos de plena autoridad para la resolución de todos los desencuentros de carácter económico.
\vs p072 2:17 \pc El tribunal supremo federal no dicta resolución sobre los casos socioeconómicos, salvo con el pronunciamiento de las tres cuartas partes de los votos de la tercera rama legislativa del gobierno nacional, la cámara de los ancianos estadistas. De otro modo, todas las decisiones de los tribunales superiores parentales, educacionales e industriales son definitivas.
\usection{3. LA VIDA FAMILIAR}
\vs p072 3:1 En este continente, va contra la ley que dos familias vivan bajo el mismo techo. Y puesto que las viviendas comunales se han declarado ilegales, la mayoría de los edificios de viviendas se han demolido, aunque los solteros siguen viviendo en clubes, hoteles y otras residencias comunales. La parcela más pequeña de suelo que se permite para el emplazamiento de los hogares ha de tener unos cuatro mil seiscientos metros cuadrados. Todo el terreno y otras propiedades usados para fines domésticos están exentos de impuestos hasta diez veces por encima de la asignación mínima de suelo previsto para dicho emplazamiento.
\vs p072 3:2 Durante el último siglo, la vida familiar de la población ha mejorado notablemente. Los padres de familia, tanto el padre como la madre, tienen obligatoriamente que asistir a las escuelas de puericultura para padres. Incluso los agricultores que residen en pequeños núcleos rurales llevan a cabo esta tarea por correspondencia, desplazándose a los centros cercanos para recibir instrucción oral una vez cada diez días; esto es, cada dos semanas, puesto que su semana es de cinco días.
\vs p072 3:3 Cada familia tiene un promedio de cinco hijos, que están bajo la plena autoridad de los padres o, en caso del fallecimiento de uno de ellos o de ambos, bajo la de los tutores designados por los tribunales parentales. Para cualquier familia representa un gran honor conseguir la tutela de un huérfano de padre y madre. Los padres han de competir entre ellos, y se otorga el huérfano al hogar de quienes demuestren mejores aptitudes como padres.
\vs p072 3:4 \pc Estas personas consideran el hogar como la institución esencial de su civilización. Se exige que los hijos reciban en el hogar de sus padres la parte más valiosa de su educación y la formación de su carácter; y los padres dedican casi tanta atención a la crianza del hijo como las madres.
\vs p072 3:5 Son los padres o los tutores legales los que imparten en el hogar la educación sexual. Los maestros ofrecen la formación moral durante los períodos de descanso en los talleres escolares, pero no ocurre así con la enseñanza religiosa, que se considera un privilegio exclusivo de los padres; la religión se percibe como parte integrante de la vida familiar. La instrucción puramente religiosa solo se imparte públicamente en los templos de la filosofía; en este país, no se han desarrollado instituciones exclusivamente religiosas como las iglesias de Urantia. En su filosofía, la religión es el afán de conocer a Dios y de manifestar, mediante el servicio, el amor a sus semejantes, pero esto no es lo habitual del carácter religioso de las otras naciones de este planeta. En esta sociedad, la religión es una cuestión tan enteramente familiar que no existen lugares públicos dedicados de modo exclusivo a las reuniones religiosas. Políticamente, la Iglesia y el Estado, como los urantianos suelen decir, están totalmente separados, pero existe un extraño solapamiento entre la religión y la filosofía.
\vs p072 3:6 Hasta hace veinte años, los maestros espirituales (equiparables a los pastores religiosos de Urantia) que visitan a cada familia de forma periódica para valorar a los niños y comprobar si sus padres les han dado la debida formación, estaban bajo la supervisión del gobierno. Estos asesores y evaluadores espirituales están ahora bajo la dirección de la recién creada Fundación del Progreso Espiritual, una institución apoyada por contribuciones voluntarias. Posiblemente, esta institución no continúe desarrollándose después de la llegada de un hijo magistrado del Paraíso.
\vs p072 3:7 \pc Los niños permanecen legalmente sujetos a sus padres hasta que tienen quince años de edad, momento en el que tiene lugar la primera iniciación a las responsabilidades cívicas. A continuación, cada cinco años y durante cinco periodos consecutivos, se llevan a cabo ejercicios públicos similares para estos grupos de edades en los que se reducen sus obligaciones hacia los padres, a la par que asumen nuevas responsabilidades cívicas y sociales para con el Estado. El derecho al voto se concede a los veinte años de edad, el derecho a contraer matrimonio sin consentimiento de los padres no se otorga hasta los veinticinco años de edad y los hijos tienen que abandonar el hogar paterno al cumplir los treinta.
\vs p072 3:8 En toda la nación existe una legislación común sobre el matrimonio y el divorcio. El matrimonio no está permitido antes de cumplir los veinte años ---la edad de la emancipación civil---. El permiso para casarse se concede solamente tras haber notificado la intención de hacerlo con un año de antelación, y después de que tanto el novio como la novia hayan presentado las certificaciones acreditativas de que han recibido la debida instrucción en las escuelas de padres sobre las responsabilidades de la vida matrimonial.
\vs p072 3:9 Las estipulaciones sobre el divorcio son algo flexibles, pero las sentencias de separación, que dictan los tribunales parentales, no se pueden obtener hasta un año después de haberse presentado la solicitud, y el año de este planeta es mucho más largo que el de Urantia. No obstante, a pesar de que las leyes que regulan los divorcios son permisivas, la tasa actual de estos es únicamente la décima parte de la de las razas civilizadas de Urantia.
\usection{4. EL SISTEMA EDUCATIVO}
\vs p072 4:1 El sistema educativo de esta nación es obligatorio y mixto en las escuelas preuniversitarias, a las que los estudiantes asisten desde los cinco hasta los dieciocho años de edad. Estas escuelas difieren enormemente de las de Urantia. No hay aulas, se cursa una sola materia cada vez y, tras los primeros tres años, todos los alumnos se convierten en maestros auxiliares, instruyendo a aquellos que están por debajo de ellos. Solo se usan los libros para obtener información que ayude a resolver los problemas que se planteen en los talleres y en las granjas escolares. En estos talleres escolares se fabrica una gran parte del mobiliario utilizado en el continente al igual que muchos artefactos mecánicos ---se está en la gran era de la invención y de la mecanización---. Junto a cada taller, hay una biblioteca laboral donde el alumno puede encontrar los libros de consulta que necesite. Se imparte también Agricultura y Horticultura durante todo el período educativo en unas amplias granjas colindantes con cada una de las escuelas locales.
\vs p072 4:2 \pc A las personas con deficiencia mental se les forma solamente en la agricultura y en la ganadería, y se les confinan de por vida en colonias especiales custodiadas en las que se les separa por sexo para evitar la procreación, que se niega a aquellos de capacidad intelectual inferior a la normal. Estas medidas restrictivas llevan en funcionamiento setenta y cinco años; son los tribunales parentales los que dictan los mandatos de confinamiento.
\vs p072 4:3 \pc Todos disfrutan de un mes de vacaciones al año. Las escuelas preuniversitarias operan nueve de los diez meses de los que se compone el año; las vacaciones se utilizan para viajar con los padres o con los amigos. Estos viajes forman parte del programa de educación de adultos y continúan a lo largo de toda la vida; los fondos para sufragar este tipo de gastos se reúnen del mismo modo que los que se emplean para el seguro de vejez.
\vs p072 4:4 Una cuarta parte del periodo escolar se dedica al deporte ---al atletismo competitivo---; los estudiantes compiten avanzando, en pruebas de destreza y valor, desde el nivel local hasta el nacional, pasando por el regional y el estatal. Asimismo, los certámenes de oratoria y música, al igual que los de ciencia y filosofía, ocupan la atención de los estudiantes desde los niveles inferiores hasta aquellos que conllevan honores a nivel nacional.
\vs p072 4:5 El gobierno de la escuela es una réplica del gobierno nacional con sus tres ramas correlacionadas; al personal docente le corresponde la división tercera o legislativa y consultiva. En este continente, el objetivo fundamental de la educación es hacer de cada uno de los alumnos un ciudadano autosuficiente.
\vs p072 4:6 Todos los jóvenes que se gradúan del sistema escolar preuniversitario a los dieciocho años de edad son expertos artesanos. Comienza entonces el estudio de los libros y la adquisición de conocimientos especiales, ya sea en las escuelas de adultos o en las universidades. Cuando un alumno brillante completa sus estudios antes de lo programado, se le premia con tiempo y medios para que pueda elaborar su propio proyecto personal. Todo el sistema educativo está diseñado para formar adecuadamente al estudiante.
\usection{5. ORGANIZACIÓN INDUSTRIAL}
\vs p072 5:1 La situación industrial de este país dista mucho de sus propios ideales; el capital y la mano de obra tienen todavía sus conflictos, pero ambos se están adecuando al plan de sincera cooperación. En este singular continente, cada vez más, los trabajadores van convirtiéndose en accionistas de todas las empresas industriales; y, paulatinamente, todo trabajador inteligente se va volviendo un pequeño capitalista.
\vs p072 5:2 Los antagonismos sociales están disminuyendo y la buena voluntad está aumentando rápidamente. No ha sobrevenido ningún problema económico grave como consecuencia de la abolición de la esclavitud (ocurrida hace más de cien años), ya que esta modificación se efectuó de modo gradual mediante la liberación, cada año, del dos por ciento de los esclavos. A aquellos que pasaron satisfactoriamente las pruebas físicas, mentales y morales se les otorgó la ciudadanía; una gran parte de estos esclavos más dotados eran prisioneros de guerra o hijos de ellos. Hace unos cincuenta años que deportaron al último de sus esclavos menos dotados e, incluso, en fechas más recientes, están acometiendo la tarea de reducir el número de las clases en declive degenerativo y depravadas.
\vs p072 5:3 \pc Estas personas han desarrollado recientemente nuevas prácticas para afrontar las desavenencias en el ámbito industrial y para corregir los abusos económicos, que representan una notable mejora frente a los antiguos métodos de resolución de dichos problemas. Se ha prohibido la violencia como procedimiento para solucionar las discrepancias personales o industriales. Los salarios, los beneficios y otras cuestiones económicas no están estrictamente regulados, pero son los órganos legislativos de asuntos industriales los que los rigen generalmente, mientras que todas las disputas que surgen de la industria se resuelven en los tribunales industriales.
\vs p072 5:4 Los tribunales industriales llevan existiendo desde hace solamente treinta años, pero operan de manera muy satisfactoria. En su regulación más reciente se dispone que, en lo sucesivo, dichos tribunales deberán reconocer que las remuneraciones legales se contemplen en tres apartados:
\vs p072 5:5 \li{1.}Tipos legales de interés sobre el capital invertido.
\vs p072 5:6 \li{2.}Salarios razonables por la destreza empleada en las operaciones industriales.
\vs p072 5:7 \li{3.}Remuneraciones justas y equitativas por el trabajo.
\vs p072 5:8 \pc Estas remuneraciones se satisfarán primeramente de acuerdo a un contrato, o ante una disminución de los beneficios compartirán proporcionalmente una reducción transitoria. Y, a partir de entonces, todos los beneficios que excedan estos gastos fijos se considerarán como dividendos y se prorratearán entre estos tres apartados: capital, destreza y trabajo.
\vs p072 5:9 \pc Cada diez años, los mandatarios regionales ajustan y establecen las horas legales diarias de trabajo remunerado. En este momento, la industria opera con arreglo a una semana de cinco días, cuatro de trabajo y uno de esparcimiento. Estas personas tienen jornadas laborales de seis horas al día y, como los estudiantes, durante los nueve meses de los diez que tiene el año. Habitualmente, disfrutan sus vacaciones viajando y, como muy recientemente se han desarrollado nuevos métodos de transporte, la nación entera es propensa a viajar. El clima favorece los viajes unos ocho meses al año, y sacan el máximo de provecho de sus posibilidades.
\vs p072 5:10 \pc Hace doscientos años la industria estaba completamente dominada por el afán de lucro, pero hoy en día este está siendo rápidamente desplazado por otras fuerzas impulsoras de orden superior. En este continente existe una viva competitividad, pero se ha transferido gran parte de ella de la industria al deporte, a la especialización, al logro científico y al desarrollo intelectual. Sigue bastante activa en los servicios sociales y en la lealtad al gobierno. En este país, el servicio público se está convirtiendo con celeridad en el objetivo fundamental de sus aspiraciones. El hombre más rico del continente trabaja seis horas al día en el despacho de su taller mecánico y luego se apresura hacia la sede local de la escuela de estadistas, en la que intenta cumplir los requisitos exigidos para el servicio público.
\vs p072 5:11 El trabajo se está volviendo cada vez más respetable en este continente, y todos los ciudadanos sanos de más de dieciocho años de edad trabajan o bien en casa y en granjas, en alguna empresa reconocida, en las obras públicas que dan ocupación a los desempleados temporales o en el colectivo de obreros obligatorios de las minas.
\vs p072 5:12 Estas personas están también comenzando a albergar una nueva forma de repulsión social ---repulsión hacia la ociosidad al igual que hacia la riqueza inmerecida---. De forma lenta pero decidida, están conquistando a sus máquinas. Hace tiempo, ellos también lucharon por la libertad política y, posteriormente, por la libertad económica. Están ahora comenzando a disfrutar de ambas a la par que aprecian su bien merecido ocio, que se puede dedicar al desarrollo de su realización personal.
\usection{6. EL SEGURO DE VEJEZ}
\vs p072 6:1 Esta nación está haciendo un decidido esfuerzo por reemplazar el tipo de caridad que destruye la autoestima con una cobertura gubernamental digna que garantice la seguridad en la vejez. Aquí se proporciona educación a todos los niños y trabajo a todos los hombres, por lo que se puede llevar a buen término tal sistema de seguros para proteger a los enfermos y a los ancianos.
\vs p072 6:2 Todas las personas de esta nación deben jubilarse y cesar cualquier actividad remunerada a los sesenta y cinco años de edad, a menos que obtengan un permiso del comisario estatal de trabajo que las autorice a seguir en activo hasta los setenta años. Este límite de edad no afecta a los funcionarios públicos ni a los filósofos. Los discapacitados físicos o con invalidez permanente pueden acceder a la jubilación a cualquier edad, mediante un mandamiento judicial refrendado por el comisario de pensiones del gobierno regional.
\vs p072 6:3 \pc Los fondos para las pensiones de jubilación proceden de cuatro recursos económicos:
\vs p072 6:4 \li{1.}De las remuneraciones de un día al mes, que el gobierno federal deduce para estos fines, en un país donde todos trabajan.
\vs p072 6:5 \li{2.}De los legados: muchos ciudadanos adinerados dejan fondos para este propósito.
\vs p072 6:6 \li{3.}De las ganancias de los trabajos obligatorios en las minas estatales. Una vez que los trabajadores que han sido reclutados forzosamente tienen para su propio sustento y han reservado la cuota para su jubilación, entregan todo el beneficio excedente que resulta de su trabajo a este fondo de pensión.
\vs p072 6:7 \li{4.}De los ingresos por los recursos naturales. El gobierno federal es el depositario social de todas las riquezas naturales del continente, y los ingresos que se derivan se utilizan para fines sociales, tales como la prevención de enfermedades, la educación de los superdotados y los gastos de personas especialmente prometedoras que atienden las escuelas de estadistas. La mitad de los ingresos provenientes de los recursos naturales se destinan al fondo de pensiones de jubilación.
\vs p072 6:8 \pc Aunque las fundaciones de actuarios estatales y regionales proporcionan muchas formas de seguros de protección, es el gobierno federal, a través de sus diez departamentos regionales, el que gestiona las pensiones de jubilación.
\vs p072 6:9 Desde hace mucho tiempo se han gestionado estos fondos públicos con honradez. Después de la traición y el asesinato, los castigos más severos que imponen los tribunales están relacionados con la traición a las responsabilidades públicas. Hoy en día, se considera la deslealtad social y política como el más atroz de todos los delitos.
\usection{7. EL SISTEMA TRIBUTARIO}
\vs p072 7:1 El gobierno federal es paternalista únicamente en la gestión de las pensiones de jubilación y el fomento del talento y la originalidad creativa; los gobiernos estatales se preocupan algo más del ciudadano individual, mientras que los gobiernos locales son mucho más paternalistas o socialistas. El gobierno metropolitano (o alguna subdivisión de este) se ocupa de asuntos como la salud, el saneamiento, la normativa en materia de construcción, el embellecimiento urbano, el suministro de agua, el alumbrado, la calefacción, el ocio, la música y la comunicación.
\vs p072 7:2 En todo el ámbito industrial se presta una atención primordial a la salud; ciertas facetas del bienestar físico se consideran como prerrogativas industriales y comunitarias, pero los problemas de salud a nivel individual y familiar son asuntos de interés estrictamente personal. En la medicina, como en todos las demás cuestiones exclusivamente personales, el plan del gobierno es abstenerse, cada vez más, de intervenir.
\vs p072 7:3 \pc Las ciudades carecen de poder tributario, ni tampoco pueden endeudarse. Perciben prestaciones per cápita de la tesorería del estado y han de complementar estos ingresos con las ganancias de sus empresas de dominio público y con el cobro de licencias para la realización de diferentes actividades comerciales.
\vs p072 7:4 Es el municipio el que gestiona las instalaciones de transporte rápido, que facilitan enormemente la ampliación de los límites de la ciudad. Las fundaciones de prevención de incendios y de seguros sufragan los departamentos de bomberos de la ciudad, y todos los edificios, ya sean de la ciudad o del entorno rural, están a prueba de incendios ---lo han estado desde hace más de setenta y cinco años---.
\vs p072 7:5 Los municipios no nombran a los agentes del orden público; son los gobiernos estatales los que gestionan las fuerzas policiales. Este cuerpo se recluta casi por entero de entre los varones no casados de veinticinco a cincuenta años de edad. La mayoría de los estados imponen a los solteros una fuerte presión tributaria, que se condona a todos los hombres que ingresan en la policía estatal. En el estado medio, la fuerza policial es solo una décima parte de lo que era hace cincuenta años.
\vs p072 7:6 \pc Existe muy poca o ninguna homogeneidad entre los sistemas tributarios de los cien estados relativamente libres y soberanos, ya que las condiciones económicas y de otro tipo varían considerablemente en los diferentes sectores del continente. Cada estado tiene diez disposiciones constitucionales fundamentales que no son alterables salvo mediante la autorización del tribunal supremo federal, y uno de sus artículos impide aplicar un gravamen que exceda el uno por ciento anual del valor de cualquier propiedad, quedando exenta aquella destinada al emplazamiento del hogar, sea urbano o rural.
\vs p072 7:7 El gobierno federal no puede endeudarse, y se requiere un referéndum con una mayoría de las tres cuartas partes de los votos para que un estado pueda conseguir un préstamo, exceptuando que sea para fines bélicos. Puesto que el gobierno federal no puede contraer deudas, en caso de guerra el Consejo Nacional de Defensa está facultado para exigir un impuesto a los estados, al igual que hombres y material, según se precisen. Pero ninguna puede dejarse sin saldar durante más de veinticinco años.
\vs p072 7:8 \pc Los ingresos para sufragar al gobierno federal proceden de los cinco recursos económicos siguientes:
\vs p072 7:9 \li{1.}\bibemph{De los derechos de importación}. Todas las importaciones están sujetas a un arancel destinado a velar por el nivel de vida de este continente, que está muy por encima del de cualquier otra nación del planeta. El más alto tribunal industrial establece estos aranceles una vez que las dos cámaras del congreso industrial han ratificado las recomendaciones del jefe ejecutivo de asuntos económicos, que es la persona designada conjuntamente por estos dos órganos legislativos. Los trabajadores eligen la cámara industrial superior y, el capital, la baja.
\vs p072 7:10 \li{2.}\bibemph{De los derechos de autor}. El gobierno federal apoya la invención y las creaciones originales en diez laboratorios regionales, ayudando a todo tipo de genios ---artistas, autores y científicos--- y protegiendo sus patentes. A cambio, el gobierno retiene la mitad de los beneficios obtenidos de todos estos inventos y creaciones, ya se refieran a maquinas, libros, arte, plantas o animales.
\vs p072 7:11 \li{3.}\bibemph{Del impuesto de sucesiones}. El gobierno federal grava un impuesto escalonado sobre la herencia que oscila entre el uno y el cincuenta por ciento, dependiendo de la cuantía del patrimonio así como de otras condiciones.
\vs p072 7:12 \li{4.}\bibemph{Del equipamiento militar}. El gobierno obtiene una importante suma de dinero del alquiler del equipamiento militar y naval para usos comerciales y recreativos.
\vs p072 7:13 \li{5.}\bibemph{De los recursos naturales.} Los ingresos procedentes de los recursos, cuando no se precisan en su totalidad para los fines específicos señalados en los estatutos del Estado federal, se aportan a la tesorería nacional.
\vs p072 7:14 \pc Las asignaciones federales, exceptuando los fondos militares que deduce el Consejo Nacional de Defensa, se originan en la cámara legislativa superior, se respaldan por la cámara baja, se aprueban por el jefe del ejecutivo y, finalmente, se validan por la comisión federal de presupuestos de los cien. Los gobernadores estatales son quienes proponen a los miembros de esta comisión y las asambleas legislativas estatales los eligen. Prestan sus servicios durante veinticuatro años, eligiéndose a una cuarta parte de ellos cada seis años. Cada seis años, este órgano, con la mayoría de las tres cuartas partes de los votos, escoge a uno de sus integrantes como presidente, con lo cual se convierte en director\hyp{}administrador de la tesorería federal.
\usection{8. ESCUELAS ESPECIALES}
\vs p072 8:1 Además del programa obligatorio de enseñanza elemental que se prolonga desde los cinco hasta los dieciocho años de edad, hay escuelas especiales que se organizan de la siguiente manera:
\vs p072 8:2 \li{1.}\bibemph{Las escuelas de estadistas}. Estas escuelas son de tres clases: nacional, regional y estatal. Los cargos públicos de la nación se agrupan en cuatro categorías. La primera categoría de puestos de confianza pública atañe principalmente a la administración nacional, y todos los cargos públicos de este grupo tienen que ser graduados de las escuelas regionales y nacionales de estadistas. En la segunda categoría, se pueden ocupar cargos políticos, electos o por nombramiento al graduarse de cualquiera de las diez escuelas regionales de estadistas; su cometido entraña responsabilidades en la administración regional y en los gobiernos estatales. En la tercera categoría, se incluyen las responsabilidades estatales y, para ella, solo se le requiere a los cargos estar en posesión de títulos estatales de estadista. Para la cuarta y última categoría de cargos públicos, no se necesita estar en posesión del título de estadista, ya que son puestos que se otorgan enteramente por nombramiento. Son puestos menores de ayudantía, secretaría y responsabilidades técnicas que distintos profesionales altamente cualificados desempeñan en funciones gubernamentales de carácter administrativo.
\vs p072 8:3 Los jueces de los tribunales menores y estatales ostentan títulos de las escuelas estatales de estadistas. Los jueces de los tribunales jurisdiccionales de asuntos sociales, educacionales e industriales poseen títulos de las escuelas regionales. Los jueces del tribunal supremo federal han de tener títulos de todas estas escuelas de estadistas.
\vs p072 8:4 \li{2.}\bibemph{Las escuelas de filosofía}. Estas escuelas están afiliadas a los templos de filosofía y están más o menos relacionadas con la religión como labor pública.
\vs p072 8:5 \li{3.}\bibemph{Las instituciones científicas}. Estas escuelas técnicas están más en coordinación con la industria que con los sistemas educativos y se rigen con arreglo a quince divisiones departamentales.
\vs p072 8:6 \li{4.}\bibemph{Escuelas de formación profesional}. Estas instituciones especiales proporcionan la formación técnica de las diferentes profesiones altamente cualificadas, que suman doce.
\vs p072 8:7 \li{5.}\bibemph{Escuelas militares y navales}. Cerca de los cuarteles generales nacionales y en los veinticinco centros militares costeros están operativas estas instituciones dedicadas a la formación militar de ciudadanos voluntarios de dieciocho a treinta años de edad. Para poder ingresar en estas escuelas, los menores de veinticinco años necesitan el consentimiento paterno.
\usection{9. EL SISTEMA DE SUFRAGIO UNIVERSAL}
\vs p072 9:1 Aunque los candidatos a todos los cargos públicos se limitan a los graduados de las escuelas de estadistas estatales, regionales o federales, los líderes progresistas de esta nación descubrieron una grave deficiencia en su sistema de sufragio universal y, hace unos cincuenta años, desarrollaron una disposición constitucional modificando su método de votación, que incluye las siguientes características:
\vs p072 9:2 \li{1.}Todo hombre y mujer de veinte o más años de edad tiene un voto. Al llegar a esta edad, todos los ciudadanos deben pertenecer a dos grupos de votantes: se incorporarán al primero de ellos con arreglo a su actividad económica ---industrial, profesional, agrícola o comercial---; ingresarán en el segundo grupo de acuerdo con sus inclinaciones políticas, filosóficas y sociales. Todos los trabajadores pertenecen, por tanto, a algún grupo de sufragio de orden económico, y estos gremios, al igual que las asociaciones no económicas, se regulan de forma muy similar a la del gobierno nacional con su triple división de poderes. Durante doce años, no se puede cambiar la inscripción en estos grupos.
\vs p072 9:3 \li{2.}Tras su designación por los gobernadores estatales o por los mandatarios regionales y por decreto de los consejos regionales supremos, a quienes han prestado gran servicio a la sociedad, o han demostrado una sabiduría extraordinaria al servicio del gobierno, se les puede conferir votos adicionales, solamente una vez cada cinco años y sin que tal sufragio de excelencia exceda las nueve veces. El máximo número de votos que cualquier votante múltiple puede emitir es de diez. También, y de igual manera, se reconoce la labor de los científicos, inventores, maestros, filósofos y líderes espirituales, honrándoles con un aumento de sus atribuciones políticas. Estas prerrogativas cívicas de rango superior se otorgan por los consejos supremos estatales y regionales de manera muy similar a como se conceden los títulos por parte de las escuelas superiores especiales, y sus beneficiarios se enorgullecen de añadir los símbolos de tal reconocimiento cívico, junto con sus otros títulos, a sus listas de logros personales.
\vs p072 9:4 \li{3.}Todos los individuos condenados a trabajo forzoso en las minas y todos los funcionarios públicos que perciben aportación económica de los impuestos están privados de su derecho al voto durante los períodos de tales servicios. Esto no se aplica a los pensionistas, que se han jubilado a los sesenta y cinco años de edad.
\vs p072 9:5 \li{4.}Existen cinco tramos de sufragio que reflejan los impuestos anuales medios tributados cada lustro. A los grandes contribuyentes se les permite votos adicionales hasta un máximo de cinco. Tal concesión es independiente de todos los demás reconocimientos, pero en ningún caso puede una persona emitir más de diez votos.
\vs p072 9:6 \li{5.}Cuando se adoptó este régimen electoral, se abandonó el método territorial de sufragio en favor del sistema económico u ocupacional. En la actualidad, todos los ciudadanos votan como miembros de grupos industriales, sociales o profesionales, con independencia de su lugar de residencia. De este modo, el electorado está integrado por grupos consolidados, unificados e inteligentes, que eligen solamente a los mejores miembros de sus respectivos grupos para ocupar puestos de confianza y responsabilidad en el gobierno. Existe una excepción a este método de sufragio ocupacional o grupal: la elección de un mandatario federal en jefe se efectúa cada seis años, mediante una votación a nivel nacional en la que ningún ciudadano puede emitir más de un voto.
\vs p072 9:7 \pc Por consiguiente, exceptuando la elección del mandatario en jefe, el sufragio se ejerce por agrupaciones económicas, profesionales, intelectuales y sociales de la ciudadanía. El estado ideal es orgánico, y cada grupo libre e inteligente de ciudadanos representa un órgano vital y de carácter operativo dentro del más amplio organismo gubernamental.
\vs p072 9:8 Las escuelas de estadistas tienen competencia para iniciar actuaciones en los tribunales estatales dirigidas a privar del derecho de voto a personas con deficiencias mentales, ociosas, desinteresadas o delictivas. Se piensa que cuando el cincuenta por ciento de los ciudadanos de una nación está peor dotado o es deficiente y está en posesión del voto, tal nación está destinada al fracaso. Creen que la supremacía de la mediocridad significaría la perdición de cualquier nación. Votar es obligatorio, y se imponen cuantiosas multas a todo el que deje de hacerlo.
\usection{10. TRATAMIENTO DE LA DELINCUENCIA}
\vs p072 10:1 Los métodos de esta nación para hacer frente a la delincuencia, a la demencia y al declive degenerativo aunque, en algunos aspectos satisfactorios, en otros resultan, sin duda, impactantes para la mayoría de los urantianos. A los delincuentes ordinarios y a los deficientes se les internan, por sexo, en diferentes colonias agrícolas y se mantienen sobradamente por sí mismos. A los delincuentes reincidentes más graves y a los dementes irremediables los tribunales los condenan a morir en las cámaras de gas letal. Numerosos actos criminales, además del asesinato, incluyendo la traición a las responsabilidades gubernamentales, también conllevan la pena capital; y la aplicación de la justicia es segura y pronta.
\vs p072 10:2 Este país está saliendo de la era negativa de la ley para incorporarse a la positiva. Últimamente han llegado al extremo de intentar prevenir la criminalidad, sentenciando a aquellos supuestos asesinos y grandes delincuentes potenciales a servir de por vida en las colonias de reclusión. Si estos presos demuestran posteriormente que se han normalizado, se les da la libertad condicional o se les indulta. El índice de homicidios en este continente es solamente el uno por ciento del de las otras naciones.
\vs p072 10:3 Hace más de cien años se pusieron en marcha medidas para prevenir la procreación de los delincuentes y de los deficientes, y ya están arrojando resultados alentadores. No existen presidios ni hospitales para enfermos mentales. Por un motivo: de estos grupos solo hay aproximadamente el diez por ciento de los que existen en Urantia.
\usection{11. PREPARACIÓN MILITAR}
\vs p072 11:1 El presidente del Consejo Nacional de Defensa puede designar, con arreglo a siete rangos, a los graduados de las escuelas militares federales, como “guardianes de la civilización”, según sea su capacidad y experiencia. Este consejo consta de veinticinco miembros, que se nombran por los más altos tribunales parentales, educacionales e industriales, se ratifican por el tribunal supremo federal y se preside, de oficio, por el jefe del estado mayor de asuntos militares conjuntos. Estos miembros desempeñan su cargo hasta la edad de setenta años.
\vs p072 11:2 Los cursos que estos oficiales comisionados llevan a cabo tienen una duración de cuatro años y siempre se correlacionan con el dominio de algún oficio o profesión. Nunca se imparte la formación militar sin este correspondiente aprendizaje industrial, científico o profesional. Al completar tal formación, la persona ha recibido, durante estos cuatro años, la mitad de la educación que se imparte en cualquiera de las escuelas especiales, cuyos cursos duran igualmente cuatro años. De esta manera, se evita la creación de una clase militar profesional, ya que se proporciona a un gran número de hombres la oportunidad de tener su propio sustento mientras adquieren la primera mitad de una formación técnica o profesional.
\vs p072 11:3 En tiempos de paz, el servicio militar es totalmente voluntario, y el alistamiento en cualquiera de sus ramas de servicio es de cuatro años; durante estos, todo hombre realiza algún área especial de estudio, además de obtener pericia en tácticas militares. La formación musical es una de las principales áreas de las escuelas militares centrales y de los veinticinco campos de entrenamiento repartidos por la periferia del continente. En los períodos de estancamiento de la industria, se recurre de inmediato a muchos miles de desempleados para fortalecer las defensas militares del continente en tierra, mar y aire.
\vs p072 11:4 \pc Aunque este país mantiene un poderoso sistema bélico como defensa contra la invasión de los pueblos hostiles que lo rodean, se puede hacer constar como mérito que desde hace más de cien años no ha empleado estos recursos militares en una ofensiva de guerra. Se han vuelto tan civilizados que pueden defender enérgicamente su civilización sin caer en la tentación de utilizar su poderío militar en actos de agresión. Desde la instauración del Estado continental unido, no se han producido guerras civiles, pero durante los dos últimos siglos, se han visto abocados a librar nueve duros conflictos de tipo defensivo; tres de ellos contra poderosas confederaciones de potencias mundiales. Aunque esta nación mantiene una adecuada defensa contra el ataque de vecinos hostiles, presta mucha más atención a la formación de estadistas, científicos y filósofos.
\vs p072 11:5 Cuando la nación está en paz con el mundo, todos los mecanismos móviles de defensa se emplean casi por completo en el intercambio, el comercio y el esparcimiento. Cuando se declara la guerra, la nación entera se moviliza. Durante el transcurso de las hostilidades, impera el sueldo militar en las industrias, y los jefes de todos los cuerpos militares se convierten en miembros del gabinete del jefe del ejecutivo.
\usection{12. LAS OTRAS NACIONES}
\vs p072 12:1 Aunque la sociedad y el gobierno de este singular país son superiores en muchos aspectos a los de las naciones de Urantia, cabe señalar que en los otros continentes (hay once en este planeta) los gobiernos son claramente de orden inferior a los de las naciones más avanzadas de Urantia.
\vs p072 12:2 Justo ahora este gobierno superior tiene previsto establecer relaciones diplomáticas con los pueblos menos avanzados y, por primera vez, ha surgido un gran líder religioso que aboga por el envío de misioneros a estas naciones circundantes. Nos tememos que estén a punto de cometer el mismo error que otros muchos cometieron cuando se propusieron imponer una cultura y religión de índole superior a otras razas. ¡Qué cosas tan extraordinarias se podrían hacer en este mundo si esta nación continental de avanzada cultura tan solo saliese a los pueblos vecinos y trajese a sus mejores especímenes y luego, tras haberlos instruidos, enviarlos de vuelta como emisarios de la cultura a sus ignorantes hermanos! Por supuesto que si un hijo magistrado llegase pronto a esta adelantada nación, cosas maravillosas podrían acontecer rápidamente en este mundo.
\vs p072 12:3 \pc Este relato de los asuntos de un planeta vecino se lleva a efecto gracias a la concesión de un permiso especial y con el propósito de hacer avanzar la civilización y mejorar el desarrollo gubernamental en Urantia. Se podrían añadir muchas cosas a esta narración que sin duda serían del interés de los urantianos y suscitarían su curiosidad, pero la información aquí revelada respeta los límites que nuestro mandato permite.
\vs p072 12:4 \pc Los urantianos, sin embargo, deben prestar atención al hecho de que su esfera hermana de la familia de Satania no se ha beneficiado de la misión de magistrado ni de la de gracia de los hijos del Paraíso. Tampoco hay tanta divergencia cultural entre los distintos pueblos de Urantia como la que existe entre esta nación continental y sus semejantes planetarios.
\vs p072 12:5 El derramamiento del espíritu de la verdad proporciona las bases espirituales para la realización de grandes logros en aras de la raza humana de los mundos de gracia. Urantia está, pues, mucho mejor preparada para la más inmediata consecución de un gobierno planetario con sus leyes, mecanismos, símbolos, convenciones e idioma ---todo lo cual podría contribuir poderosamente al establecimiento de la paz mundial bajo la ley y podría dar lugar, en algún momento, al amanecer de una verdadera era de conquista espiritual, que constituye el umbral planetario de las eras ideales de luz y vida---.
\vsetoff
\vs p072 12:6 [Exposición de un melquisedec de Nebadón.]
