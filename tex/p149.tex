\upaper{149}{El segundo viaje de predicación}
\author{Comisión de seres intermedios}
\vs p149 0:1 El segundo viaje de predicación pública realizado por Galilea comenzó el domingo, 3 de octubre del año 28 d. C., y continuó durante casi tres meses, concluyendo el 30 de diciembre. En este participaron Jesús y sus doce apóstoles, con el apoyo del colectivo, recientemente reclutado, de ciento diecisiete evangelistas y de numerosas otras personas interesadas. En este recorrido, visitaron Gadara, Tolemaida, Jafia, Dabarita, Megido, Jezreel, Escitópolis, Tariquea, Hipos, Gamala, Betsaida\hyp{}Julias y muchas otras ciudades y aldeas.
\vs p149 0:2 Antes de partir, ese domingo por la mañana, Andrés y Pedro pidieron a Jesús que les diera las últimas instrucciones a los nuevos evangelistas, pero el Maestro se negó a ello, diciendo que no era su responsabilidad hacer algo que otros podían llevar a cabo perfectamente. Tras su debido debate, se decidió que sería Santiago Zebedeo quien las diese. Al finalizar Santiago su charla, Jesús se dirigió a los evangelistas con estas palabras: “Salid ahora a hacer la labor que os ha sido encomendada y, más tarde, cuando hayáis demostrado que sois aptos y leales, yo os ordenaré para que prediquéis el evangelio del reino”.
\vs p149 0:3 En este viaje, solo Santiago y Juan iban con Jesús; Pedro y los demás apóstoles fueron, cada uno, con una docena de evangelistas y se mantenían en estrecho contacto con ellos, mientras predicaban y enseñaban. En cuanto los creyentes estaban preparados para entrar en el reino, los apóstoles les administraban el bautismo. Jesús y sus dos compañeros recorrieron un extenso trayecto en estos tres meses; visitaban con frecuencia dos ciudades en un solo día para observar el trabajo de los evangelistas y estimularlos en su misión de instaurar el reino. Este segundo viaje de predicación se llevó a cabo principalmente para que este grupo de ciento diecisiete evangelistas, recientemente formados, adquiriera experiencia práctica.
\vs p149 0:4 \pc Durante todo este período y, posteriormente, hasta el momento en el que Jesús y los doce partieron en su último viaje a Jerusalén, David Zebedeo mantuvo una sede permanente en Betsaida, en la casa de su padre, para secundar el trabajo del reino. Era un centro de intercambio de información respecto a la labor de Jesús en la tierra y una estación de relevo para el servicio de mensajería, dirigida por él mismo, para distribuir cualquier dato pertinente entre los trabajadores del reino presentes en las distintas partes de Palestina y de las regiones colindantes. David hizo todo por propia iniciativa, aunque con la aprobación de Andrés. Utilizó entre cuarenta y cincuenta mensajeros en esta actividad de recogida y difusión de la información sobre el trabajo del reino, que se ampliaba y extendía con celeridad. Mientras se ocupaba de esta tarea, David se ganaba parcialmente la vida dedicando algo de su tiempo a su antiguo quehacer de pescador.
\usection{1. NOTORIEDAD DE JESÚS}
\vs p149 1:1 En el momento del desmantelamiento del campamento de Betsaida, la fama de Jesús, en especial como sanador, ya se había difundido por todas partes de Palestina y por toda Siria y los países del entorno. Durante semanas, tras partir de Betsaida, continuaron llegando enfermos, que, al no encontrar al Maestro y enterarse por David donde se encontraba, iban en su búsqueda. En este viaje, Jesús no obró, intencionadamente, ninguno de los denominados milagros de curación. No obstante, un gran número de personas afligidas volvieron a recobrar la salud y la alegría gracias al poder recuperador de la intensa fe que los impulsaba a buscar la curación.
\vs p149 1:2 Sobre la época de esta misión, comenzaron a manifestarse ---y continuó así durante el resto de la vida de Jesús en la tierra--- una serie de fenómenos de sanación, singulares e inexplicables. En el transcurso de este viaje, de tres meses de duración, más de cien hombres, mujeres y niños de Judea, Idumea, Galilea, Siria, Tiro y Sidón, y del otro lado del Jordán se beneficiaron de estas curaciones inconscientes por parte de Jesús, y, que, de regreso a sus casas, contribuyeron a extender la fama de Jesús. Y esto lo hicieron a pesar de que Jesús, cada vez que observaba uno de estos casos de curación espontánea, advertía directamente a los beneficiados: “Mirad que nadie lo sepa”.
\vs p149 1:3 \pc Jamás se nos ha revelado qué ocurría en estos casos de sanación espontánea o inconsciente. El Maestro nunca llegó a explicar a sus apóstoles cómo se producían, más allá de simplemente comentar en algunos momentos: “Siento que el poder ha salido”. En una ocasión, cuando tocó a un niño enfermo, expresó: “Siento que la vida ha salido de mí”.
\vs p149 1:4 Al no disponer de la confirmación directa del Maestro en cuanto a la naturaleza de estos casos de curación espontánea, sería un atrevimiento, por nuestra parte, tratar de explicar de qué forma ocurrían, pero es permisible que dejemos constancia de nuestra opinión sobre estos fenómenos curativos. Creemos que muchos de estos supuestos milagros, tal como se dieron en el transcurso del ministerio de Jesús, se debieron a la coexistencia de los siguientes tres factores, eficaces, potentes y conexos:
\vs p149 1:5 \li{1.}La presencia, en el corazón del ser humano, de una fe fuerte, dominante y viva que ha buscado la curación con perseverancia, junto con el hecho de que esta se deseaba más por sus beneficios espirituales que por mera recuperación física.
\vs p149 1:6 \li{2.}La existencia, coincidente con dicha fe humana, de la gran empatía, compasión y preponderante misericordia del hijo creador de Dios encarnado, que realmente poseía, en su persona, poderes creativos y prerrogativas de sanación casi ilimitados y atemporales.
\vs p149 1:7 \li{3.}Junto con la fe de la criatura y la vida del creador, cabe destacar asimismo que este Dios\hyp{}hombre era la expresión personificada de la voluntad del Padre. Si, en el contacto entre la necesidad humana y la voluntad divina que la satisfaría, el Padre no deseaba lo contrario, las dos se convertían en una sola, y la curación se producía de forma inconsciente para el Jesús humano, pero era de inmediato reconocida por la naturaleza divina de Jesús. La explicación, pues, de muchos de estos casos de sanación hay que hallarla en una gran ley desde hace mucho conocida por nosotros, que consiste en: Lo que el hijo creador desea y el Padre eterno quiere ES.
\vs p149 1:8 \pc Somos, entonces, de la opinión de que, en la presencia personal de Jesús, ciertas expresiones de profunda fe humana \bibemph{impelían,} literal y verdaderamente, la manifestación de la curación por medio de determinadas fuerzas y seres personales creativos del universo que, en ese momento, estaban muy estrechamente relacionados con el Hijo del Hombre. Queda constancia, por lo tanto, del hecho de que Jesús frecuentemente permitía que los hombres se curaran a sí mismos en su presencia gracias a la poderosa fe personal presente en ellos.
\vs p149 1:9 Otros muchos buscaban sanarse por motivos enteramente egoístas. Una rica viuda de Tiro llegó con su comitiva pretendiendo que se la curara de sus enfermedades, que eran muchas; y siguió a Jesús por Galilea, ofreciéndole cada vez más dinero, como si el poder de Dios pudiese venderse al mejor postor. Pero ella nunca llegaría a interesarse en el evangelio del reino; solo procuraba la curación de sus dolencias físicas.
\usection{2. LA ACTITUD DE LA GENTE}
\vs p149 2:1 Jesús entendía las mentes de los hombres. Sabía lo que había en sus corazones, y si sus enseñanzas se hubiesen preservado tal como él las presentó, y el único comentario hubiese sido una interpretación inspirada de su vida en la tierra, entonces, todas las religiones del mundo habrían abrazado rápidamente el evangelio del reino. La tarea, bien intencionada, de los primeros seguidores de Jesús por replantear sus enseñanzas para hacerlas más aceptables a ciertas naciones, razas y religiones, solo ocasionaron que fueran menos propicias para todas las demás naciones, razas y religiones.
\vs p149 2:2 El apóstol Pablo, en su afán por hacer que determinados grupos de su época acogieran favorablemente las enseñanzas de Jesús, escribió muchas cartas instructivas y con recomendaciones. Otros maestros del evangelio de Jesús actuaron de igual manera, pero ninguno de ellos pensó que algunos de estos escritos serían posteriormente recopilados y presentados como si fuesen la expresión de las enseñanzas de Jesús. Así pues, aunque el denominado cristianismo contiene, de hecho, más componentes del evangelio del Maestro que cualquier otra religión, posee, además, muchas enseñanzas no impartidas por él. Al margen de la incorporación de muchas doctrinas de los misterios persas y de la filosofía griega en el cristianismo primitivo, se cometieron dos grandes errores:
\vs p149 2:3 \li{1.}El empeño por entroncar las enseñanzas del evangelio directamente con la teología judía, como se ilustra en las doctrinas cristianas de la expiación ---la idea de que Jesús, como Hijo, se sacrificó para satisfacer la justicia severa del Padre y apaciguar la ira divina---. Esta doctrina tuvo su origen en el encomiable afán por hacer que el evangelio del reino fuese más aceptable para los descreídos judíos. Y, aunque tal empeño fracasó y no logró ganarse a los judíos para el reino, sí contribuyó a confundir y a enajenar a numerosas almas honestas de todas las generaciones posteriores.
\vs p149 2:4 \li{2.}El segundo gran desacierto de los primeros seguidores del Maestro, y que todas las generaciones venideras han seguido perpetuando, fue el de organizar las enseñanzas cristianas mayormente sobre la \bibemph{persona} de Jesús, y este excesivo énfasis de la teología del cristianismo ha contribuido a oscurecer sus enseñanzas. Y todo esto ha hecho crecientemente más difícil que judíos, mahometanos, hindúes y otros devotos orientales aceptaran las enseñanzas de Jesús. No queremos restarle importancia a la persona de Jesús en una religión que lleva su nombre, pero no podemos permitir que dicha consideración eclipse su inspiradora vida ni suplante su mensaje salvífico: la paternidad de Dios y la hermandad de los hombres.
\vs p149 2:5 \pc Los maestros de la religión de Jesús deberían aproximarse a otras religiones reconociendo las verdades que tienen en común (la mayoría de las cuales provienen de forma directa o indirecta del mensaje de Jesús), y abstenerse, al mismo tiempo, de hacer tanto hincapié en las diferencias.
\vs p149 2:6 \pc Aunque, en aquel momento dado, la notoriedad de Jesús residía principalmente en su reputación como sanador, de esto no se desprende que continuara siempre así. Conforme pasaba el tiempo, cada vez se esperaba más de él su asistencia espiritual. Pero, para la gente común, fueron sus curaciones físicas las que, de forma más directa e inmediata, mayor interés suscitaron. Si bien, crecientemente, buscaban a Jesús las víctimas de la esclavitud moral y de las vejaciones mentales, y él, invariablemente les mostraba cómo liberarse de ellas. Los padres pedían su consejo sobre la forma de tratar a sus hijos y las madres venían en busca de guía para sus hijas. Quienes habitaban en tinieblas iban a él, y él les revelaba la luz de la vida. Sus oídos estaban continuamente atentos al dolor de la humanidad, y prestaba siempre su ayuda a los que precisaban de su ministerio.
\vs p149 2:7 Cuando el creador mismo estuvo en la tierra, encarnado con la semejanza de un hombre mortal, era inevitable que sucedieran acontecimientos extraordinarios. Pero que nunca sean estos llamados fenómenos milagrosos los que os acerquen a Jesús. Aprended a aproximaros a los milagros teniendo presente a Jesús, pero no cometáis la equivocación de aproximaros a Jesús teniendo presente sus milagros. Y este consejo está justificado, pese a que Jesús de Nazaret es el único fundador de una religión en la que se realizaron en la tierra actos sobrenaturales.
\vs p149 2:8 \pc El elemento más sorprendente y revolucionario de la misión de Miguel en la tierra fue su actitud hacia las mujeres. En un día y generación en los que se suponía que un hombre no debía saludar ni siquiera a su propia esposa en una plaza pública, Jesús se atrevió a llevar con él a mujeres como maestras del evangelio durante su tercer viaje por Galilea. Y tuvo el sumo arrojo de hacerlo frente a las enseñanzas rabínicas que decían: “Quema las palabras de la ley antes que enseñárselas a las mujeres”.
\vs p149 2:9 En una generación, Jesús sacó a las mujeres del irrespetuoso olvido y de la esclavizante servidumbre de los tiempos. Y es un hecho vergonzoso de la religión, que tuvo la presunción de adoptar el nombre de Jesús, que le haya faltado valentía moral para seguir este noble ejemplo en su posterior actitud hacia la mujer.
\vs p149 2:10 \pc Cuando Jesús se mezclaba con las personas, estas descubrían que estaba enteramente libre de las supersticiones de esos días. Estaba exento de prejuicios religiosos; nunca fue intolerante. En su corazón no existía nada parecido a hostilidad en contra de la sociedad. Aunque cumplía con lo que la religión de sus padres tenía de positivo, no dudaba en ignorar las tradiciones basadas en la superstición y tiranizantes de fabricación humana. Se atrevió a instruir en la idea de que las catástrofes de la naturaleza, los accidentes del tiempo y otras calamidades no son castigos por decreto divino ni actos misteriosos de la Providencia. Denunció la devoción servil a vanos ceremoniales y expuso la falacia de la adoración materialista. Proclamó con valentía la libertad espiritual del hombre y tuvo el coraje de enseñar que incluso los mortales son, de hecho y en verdad, hijos del Dios vivo.
\vs p149 2:11 Jesús transcendió en su totalidad las enseñanzas de sus ancestros cuando, valientemente, sustituyó tener las manos limpias por tener los corazones limpios como signo de la auténtica religión. Puso la verdadera realidad en el lugar de la tradición y eliminó por completo todo acto de vanidad e hipocresía. Sin embargo, este valeroso hombre de Dios no dio rienda suelta a la crítica destructiva ni llegó a desdeñar los usos religiosos, sociales, económicos y políticos de su época. No era un revolucionario militante; permitía un desarrollo lento pero progresivo de la humanidad. Se preocupaba por desmantelar lo que \bibemph{era} solo cuando ofrecía a sus semejantes simultáneamente una noción superior de lo que \bibemph{debía ser}.
\vs p149 2:12 \pc Sin exigírsela, los seguidores de Jesús le rendían obediencia. De quienes recibieron su llamamiento personal, solo tres se negaron a aceptar su invitación al discipulado. Ejercía un singular poder de atracción sobre el hombre, pero sin ser imperioso. Transmitía confianza, y nadie se sintió jamás molesto por alguna orden suya. Asumió total potestad sobre sus discípulos, y nadie le puso objeción alguna. Permitió a sus seguidores que lo llamaran Maestro.
\vs p149 2:13 Todos los que conocieron al Maestro lo admiraban, salvo aquellos que tenían prejuicios religiosos profundamente arraigados o que creían percibir algún peligro político en sus ideas. Los hombres se asombraban ante la originalidad y la autoridad con la que impartía sus enseñanzas. Se maravillaban de la paciencia que demostraba al tratar las preguntas de personas poco desarrolladas y problemáticas. Inspiraba esperanza y confianza en los corazones de quienes se acogían a su ministerio. Solo le temían los que no lo habían conocido, y lo odiaban únicamente quienes lo consideraban como el guardián de esa verdad, destinada a derribar el mal y el error, que ellos habían decidido albergar en sus corazones a cualquier precio.
\vs p149 2:14 Ejercía una influencia poderosa y peculiarmente cautivadora sobre amigos y enemigos a la vez. Las multitudes lo seguían durante semanas solo para oír sus afables palabras y contemplar su vida sencilla. Hombres y mujeres leales a Jesús sentían por él un afecto casi sobrehumano. Y cuanto mejor lo conocían, más lo amaban. Y esto es todavía verdad; incluso hoy en día y en todas las eras futuras, cuanto más conozca el hombre a este Dios\hyp{}hombre, más lo amará e irá en pos de él.
\usection{3. LA HOSTILIDAD DE LOS LÍDERES RELIGIOSOS}
\vs p149 3:1 Pese a la favorable acogida, por parte de la gente corriente, de las enseñanzas de Jesús, los líderes religiosos de Jerusalén empezaron cada vez más a alarmarse y a mostrar hostilidad. Los fariseos habían elaborado una teología sistematizada y dogmática. Jesús, como maestro, instruía según se presentaba la ocasión; no era metódico. Lo hacía apoyándose no tanto en la ley como en la vida misma, con parábolas. (Y cuando empleaba alguna de estas para ilustrar su mensaje, su pretensión era hacer uso solamente de \bibemph{un solo} elemento de la historia para ese propósito. Se puede llegar a muchas ideas equivocadas sobre las enseñanzas impartidas por Jesús si se intentan hacer alegorías de sus parábolas.)
\vs p149 3:2 Los líderes religiosos de Jerusalén estaban casi frenéticos ante la reciente conversión del joven Abraham y la deserción de tres de los espías, a los que Pedro había bautizado, y que ahora estaban con los evangelistas en este segundo viaje de predicación por Galilea. El temor y el prejuicio cegaban cada vez más a los líderes judíos, a la vez que sus corazones se endurecían por su continuo rechazo a las atrayentes verdades del evangelio del reino. Cuando los hombres se cierran a la llamada del espíritu que habita en ellos, es poco lo que puede hacerse para modificar su actitud.
\vs p149 3:3 Al reunirse Jesús por primera vez con los evangelistas en el campamento de Betsaida, al concluir su discurso, dijo: “Debéis recordar que en cuerpo y mente ---emocionalmente--- los hombres responden de manera individual. Lo único que hay de \bibemph{uniformidad} en los hombres es el espíritu morador. Aunque los espíritus divinos puedan de alguna manera variar en cuanto a su naturaleza y grado de experiencia, responden por igual a cualquier llamamiento espiritual. Solo mediante este espíritu, y la llamada que se le hace, podrá la humanidad alguna vez lograr la unidad y la hermandad”. Pero muchos de los líderes de los judíos habían cerrado las puertas de sus corazones a la llamada espiritual del evangelio. Desde aquel día, no cejaron en su empeño de planear y tramar la muerte del Maestro. Estaban convencidos de que Jesús debía ser capturado, condenado y ejecutado como infractor religioso, por violar las enseñanzas cardinales de la sagrada ley judía.
\usection{4. PROGRESOS EN EL VIAJE DE PREDICACIÓN}
\vs p149 4:1 Jesús, mientras viajaba con Santiago y Juan, hizo muy poca labor pública en este viaje de predicación, pero, a última hora de la tarde, impartió muchas clases a los creyentes de la mayoría de las ciudades y aldeas donde tuvo la oportunidad de quedarse. En una de estas sesiones, uno de los evangelistas más jóvenes hizo a Jesús una pregunta sobre la ira, y el Maestro, entre otras cosas, le respondió:
\vs p149 4:2 \pc “La ira es una manifestación física que representa, en términos generales, el grado de fracaso de la naturaleza espiritual para asumir el control sobre el conjunto de las naturalezas intelectual y física. La ira señala vuestra falta de amor fraternal y de tolerancia sumada a vuestra carencia de respeto por vosotros mismos y de autocontrol. La ira merma la salud, degrada la mente y obstaculiza la labor del maestro espiritual del alma del hombre. ¿Es que no habéis leído en las Escrituras que ‘la ira mata al necio”, y que el hombre ‘se destroza en su furor’? ¿Que ‘el que tarda en airarse es grande de entendimiento’, mientras que ‘el impaciente de espíritu pone de manifiesto su necedad’? Todos sabéis que ‘la respuesta suave aplaca la ira, y cómo ‘la palabra áspera hace subir el furor’. ‘La cordura aplaca el furor’, mientras que ‘como ciudad destruida y sin murallas es aquel que no pone freno a sí mismo’. ‘Cruel es la ira e impetuoso el furor’. ‘El hombre iracundo provoca contiendas, mientras que el furioso multiplica sus transgresiones’. ‘No te apresures en tu espíritu a enojarte, porque el enojo reposa en el seno de los necios’”. Antes de terminar de hablar, Jesús dijo además: “Que el amor rija tan intensamente vuestros corazones, que vuestro espíritu guía encuentre poca dificultad para libraros de la tendencia a dar rienda suelta a esos arrebatos animales de ira que tan incompatibles son con el estatus de filiación divina”.
\vs p149 4:3 \pc En esta misma ocasión, el Maestro comentó al grupo la conveniencia de poseer un carácter bien equilibrado. Reconocía que era necesario que la mayoría de los hombres se dedicase a dominar un oficio, pero se lamentaba de cualquier tendencia hacia la sobre especialización, a convertirse en estrecho de mente y delimitarse en cuanto a las actividades de la vida. Llamó la atención sobre el hecho de que cualquier virtud, llevada a extremos, se puede volver un mal hábito. Jesús siempre predicó la templanza y enseñó constancia ---adaptación proporcional a los problemas de la vida---. Destacó que un exceso de comprensión y piedad puede desembocar en una seria inestabilidad emocional; que el entusiasmo puede conducir al fanatismo. Les refirió el caso de uno de sus antiguos compañeros, cuya imaginación lo había llevado a iniciativas visionarias e imprácticas. Al mismo tiempo, les advirtió de los peligros intrínsecos a la necedad de la mediocridad ultraconservadora.
\vs p149 4:4 Y, entonces, Jesús abordó los peligros del arrojo y de la fe, de cómo estos, algunas veces, llevan a las almas irreflexivas a la temeridad y a la arrogancia. También mostró cómo la prudencia y la discreción, llevadas demasiado lejos, resultan en cobardía y fracaso. Exhortó a sus oyentes a que se afanaran por ser originales, pero evitando la tendencia a la excentricidad. Abogó por una comprensión sin sentimentalismo, por una piedad sin santurronería. Enseño una veneración libre de miedo y superstición.
\vs p149 4:5 No fueron tanto las enseñanzas de Jesús sobre el carácter equilibrado las que impresionaron a sus acompañantes, sino el hecho de que su propia vida era un ejemplo elocuente de sus enseñanzas. Jesús vivía en medio de tensiones y agitación, pero jamás flaqueó. Continuamente, sus enemigos le tendían trampas, pero nunca cayó en ellas. Los sabios y eruditos se esforzaron por hacerlo tropezar, pero nunca se tambaleó. Procuraron enredarlo en disputas, pero sus respuestas eran siempre edificantes, respetables y terminantes. Cuando se le interrumpía en sus discursos con multitudinarias preguntas, sus respuestas eran siempre significativas y contundentes. Nunca recurrió a tácticas innobles para hacer frente a la continua presión de sus enemigos, que no vacilaban en emplear contra él cualquier modo de ataque, ya fuese deshonesto, injusto o inicuo.
\vs p149 4:6 Aunque es verdad que muchos hombres y mujeres deben emplearse diligentemente al logro de algún oficio que le sirva de sustento, es, no obstante, del todo conveniente que los seres humanos adquieran un amplio rango de conocimiento de la cultura de la vida tal como se vive en la tierra. Las personas verdaderamente formadas no se contentan con permanecer ignorantes de las vidas y actos de sus semejantes.
\usection{5. LECCIÓN SOBRE EL CONTENTAMIENTO}
\vs p149 5:1 Estando Jesús visitando a un grupo de evangelistas que trabajaban bajo la supervisión de Simón Zelotes, este, durante sus charlas nocturnas, preguntó al Maestro: “¿Cómo es que hay personas que se sienten mucho más felices y contentas que otras? ¿Es el contentamiento una cuestión de vivencia religiosa?”. Jesús respondió a Simón diciendo, entre otras cosas, lo siguiente:
\vs p149 5:2 \pc “Simón, algunas personas son más felices que otras por naturaleza. Depende en gran medida de la predisposición del hombre a dejarse guiar y dirigir por el espíritu del Padre que mora en él. ¿Es que no has leído en las Escrituras las palabras del sabio: ‘Lámpara del Señor es el espíritu del hombre, la cual escudriña lo más profundo del corazón’? Y también lo que dicen estos hombres que se dejan regir por el espíritu: ‘Las cuerdas me cayeron en lugares deleitosos; sí, es hermosa la heredad que me ha tocado. ‘Mejor es lo poco del justo que las riquezas de muchos impíos’, porque ‘el hombre se satisfará desde dentro’. ‘El corazón alegre hermosea el rostro y es banquete constante. Mejor es lo poco venerando al Señor, que un gran tesoro donde hay turbación. Mejor es la comida de legumbres donde hay amor, que de buey engordado donde hay odio. Mejor es lo poco con justicia que la muchedumbre de frutos sin rectitud’. ‘El corazón alegre es una buena medicina’. ‘Más vale un puño lleno de descanso, que ambos puños llenos de pesar y vejación de espíritu’.
\vs p149 5:3 “Gran parte del sufrimiento del hombre proviene del fracaso de sus anhelos y de las heridas de su orgullo. Aunque los hombres tienen el deber consigo mismos de hacer lo mejor de sus vidas en la tierra, habiéndose esforzado en esto con sinceridad, deberían aceptar su destino con alegría y aplicar su ingenio para sacar el mayor provecho de lo que tienen entre sus manos. Muchísimos de los problemas de los hombres se originan en el sustrato de temor que anida en su propio corazón. ‘Huye el malvado sin que nadie lo persiga’. ‘Los impíos son como el mar en tempestad, que no puede estarse quieto y sus aguas arrojan cieno y lodo; no hay paz para los impíos’, dice Dios’.
\vs p149 5:4 “No procuréis, pues, la paz falsa y el gozo transitorio, sino la seguridad de la fe y las promesas de la filiación divina que proporciona calma, contentamiento y gozo supremo en el espíritu”.
\vs p149 5:5 \pc Ciertamente, Jesús no consideraba este mundo como “valle de lágrimas”. Más bien lo contemplaba como la esfera en la que nacen los espíritus eternos e inmortales que comienzan su ascenso al Paraíso, “el valle en el que se hacen las almas”.
\usection{6. EL “TEMOR AL SEÑOR”}
\vs p149 6:1 En Gamala, durante la charla nocturna, Felipe le preguntó a Jesús: “¿Maestro, por qué nos enseñan las Escrituras que ‘temamos al Señor’ mientras tú quieres que contemplemos al Padre de los cielos sin ningún temor? ¿Cómo podemos reconciliar estas enseñanzas?”. Y Jesús respondió a Felipe, diciéndole:
\vs p149 6:2 \pc “Hijos míos, no me extraña que me hagáis tales preguntas. Al principio, solo a través del temor pudo el hombre aprender a ser reverente, pero yo he venido para revelar el amor del Padre y para que os sintáis movidos a la adoración del Eterno, al ser llevados como un hijo a reconocer cariñosamente el amor profundo y perfecto del Padre y corresponderle con vuestro amor. Yo os libraré de la servidumbre que os lleva por medio del esclavizante temor al tedioso servicio de un Rey\hyp{}Dios celoso y colérico. Deseo instruiros en la relación Padre\hyp{}hijo que existe entre Dios y el hombre, para que podáis ser guiados, gozosamente y con libertad, hacia esa adoración sublime y suprema de un Padre\hyp{}Dios amoroso, justo y misericordioso.
\vs p149 6:3 “A través de las eras sucesivas, el ‘temor al Señor’ ha adquirido significados diferentes, partiendo desde el miedo, pasando por la angustia y el sentimiento de aprehensión, hasta el sobrecogimiento y la veneración. Pues bien, desde la veneración, os llevaré, mediante vuestro reconocimiento, conciencia interior y gratitud, al \bibemph{amor}. Cuando el hombre reconoce solo las obras de Dios, se ve llevado a temer al Supremo; pero cuando el hombre comienza a entender y a estar en comunión con la persona y el carácter del Dios vivo, se siente crecientemente movido a amar a un Padre tan bueno y perfecto, tan universal y eterno. Y es precisamente este cambio de la relación entre el hombre y Dios la clave de la misión del Hijo del Hombre en la tierra.
\vs p149 6:4 “Un hijo sensato no teme a su padre con miras a recibir buenos obsequios de su mano; sino que, habiendo ya recibido una abundancia de cosas buenas de su parte, al haber seguido los dictados del afecto que siente por sus hijos e hijas, este hijo, tan amado, llega a amarlo a su vez en reconocimiento y agradecimiento por tan bondadosa benevolencia. La bondad de Dios lleva al arrepentimiento; la benevolencia de Dios, al servicio; la misericordia de Dios, a la salvación; mientras que el amor de Dios, a la adoración inteligente y gozosamente espontánea.
\vs p149 6:5 “Vuestros ancestros temían a Dios porque era poderoso y misterioso. Vosotros debéis adorarlo porque es esplendoroso en amor, abundante en misericordia y glorioso en verdad. El poder de Dios engendra temor en el corazón del hombre, pero la nobleza y la rectitud de su persona originan veneración, amor y adoración voluntaria. Un hijo cumplidor y cariñoso no teme ni siente aprehensión hacia un padre, incluso si es poderoso y venerable. He venido al mundo para cambiar el temor en amor, el pesar en gozo, el pavor en confianza, el sometimiento servil y las vanas ceremonias en servicio amoroso y adoración consciente. Pero es todavía verdad para aquellos que habitan en las tinieblas que ‘el temor del Señor es el principio de la sabiduría’. Sin embargo, cuando la luz haya llegado en mayor plenitud, los hijos de Dios se sentirán guiados a alabar al Infinito por lo que \bibemph{es} en lugar de temerlo por lo que \bibemph{hace}.
\vs p149 6:6 “Cuando los hijos son jóvenes e irreflexivos, se les insta necesariamente a honrar a sus padres; pero, cuando se hacen mayores y se vuelven más agradecidos de los beneficios obtenidos por el cuidado y la protección paterna, se ven llevados, mediante su entendimiento respetuoso y un creciente afecto, a ese grado de vivencia en el que realmente aman a sus padres por lo que son, más que por lo que han hecho. Por naturaleza, el padre ama a su hijo, pero el hijo debe desarrollar su amor por el padre partiendo desde el temor de lo que el padre pueda hacer, pasando por el sobrecogimiento, el sentimiento de aprehensión, la dependencia y la veneración, hasta llegar a considerar este amor con agradecimiento y cariño.
\vs p149 6:7 “Se os ha enseñado que debéis ‘temer a Dios y guardar sus mandamientos, porque esto es el todo del hombre’. Pero yo he venido a daros un mandamiento nuevo y mayor. Os quiero enseñar a ‘amar a Dios y a aprender a hacer su voluntad, porque ese es el más alto privilegio de los hijos liberados de Dios’. A vuestros padres se les enseñó a ‘temer a Dios ---al Rey Todopoderoso---’.Yo os enseño: ‘Amad a Dios ---al Padre todo misericordioso---’.
\vs p149 6:8 En el reino de los cielos, que he venido para proclamar, no hay reyes grandes y poderosos; este reino es una familia divina. Mi Padre, y el vuestro, es el centro y rector, universalmente reconocido e incondicionalmente adorado, de esta inmensa hermandad de seres inteligentes. Yo soy su Hijo, y vosotros sois también sus hijos. Por tanto, es cierto, eternamente, que vosotros y yo somos hermanos en la heredad celestial y, con mayor razón, desde que nos hemos hecho hermanos en la carne, en la vida terrenal. Dejad, pues, de temer a Dios como a un rey o de servirle como a un amo; aprended a venerarlo como Creador; a honrarlo como al Padre de vuestra juventud espiritual; a amarlo como vuestro defensor misericordioso; y, en último término, a adorarlo como a un Padre amoroso y omnisapiente gracias a vuestra mayor madurez de conciencia y gratitud espiritual.
\vs p149 6:9 “A partir de vuestros equivocados conceptos sobre el Padre de los cielos, se desarrollan falsas ideas de la humildad y surgen mucha de vuestra hipocresía. El hombre puede ser un gusano de la tierra en su naturaleza y origen, pero cuando el espíritu de mi Padre lo habita, ese hombre se vuelve divino en cuanto a su destino. El espíritu que mi Padre da de gracia regresará con toda seguridad a la fuente divina y a su plano de origen en el universo y el alma humana del hombre mortal, que se convertirá en el hijo renacido de este espíritu morador, ascenderá de cierto con el espíritu divino hasta la presencia misma del Padre eterno.
\vs p149 6:10 “La humildad debe ser, de hecho, la conveniente actitud del hombre mortal que recibe todos estos obsequios del Padre de los cielos, pese a que cada aspirante por la fe al ascenso eterno del reino de los cielos posee dignidad divina. El hábito, vano e irrelevante, de hacer ostentación de una falsa humildad resulta incompatible cuando se aprecia la fuente de vuestra salvación y se reconoce el destino de vuestras almas, nacidas del espíritu. Es totalmente apropiado sentir humildad ante Dios en la profundidad de vuestros corazones; la mansedumbre ante los hombres es encomiable; pero la hipocresía de una humildad conscientemente afectada y deseosa de atención es pueril e indigna de los hijos iluminados del reino.
\vs p149 6:11 “Hacéis bien en ser mansos ante Dios y en tener dominio de vosotros mismos ante los hombres, pero que vuestra mansedumbre sea de origen espiritual y no la manifestación falaz de una consciente arrogancia y superioridad. El profeta habló con sensatez cuando dijo, ‘caminad humildemente con Dios’ porque, aunque el Padre de los cielos es el Infinito y el Eterno, él también habita ‘con el que es contrito de mente y humilde de espíritu’. Mi Padre desprecia el orgullo, detesta la hipocresía y aborrece la iniquidad. Y es para recalcar el valor de la sinceridad y la perfecta confianza en el apoyo amoroso y la guía fiel de nuestro Padre celestial por lo que yo me he referido, tan a menudo, a los niños como ejemplos de actitud de mente del hombre y de la respuesta del espíritu, tan esenciales para que el hombre mortal entre en las realidades espirituales del reino de los cielos.
\vs p149 6:12 “El profeta Jeremías hizo una buena descripción de muchos mortales cuando dijo: ‘Cercanos estáis de Dios en vuestra boca, pero lejos de él en vuestro corazón.’ Y es que no habéis leído también esa terrible advertencia del profeta, que dijo: ‘sus sacerdotes enseñan por precio, sus profetas adivinan por dinero y al mismo tiempo profesan piedad y proclaman que el Señor está con ellos’. ¿Es que no se os ha prevenido contra quienes ‘hablan paz con sus prójimos, pero la maldad está en su corazón’, contra quienes adulan con los labios, pero con doblez de corazón’? De todos los pesares de un hombre confiado, ninguno es tan terrible como el de ser ‘herido en casa de un amigo de confianza’”.
\usection{7. REGRESO A BETSAIDA}
\vs p149 7:1 Andrés, previa consulta con Simón Pedro y con la aprobación de Jesús, había dado instrucciones a David en Betsaida para que enviara mensajeros a los distintos grupos de predicadores, indicándoles que finalizaran su recorrido y regresaran a Betsaida en algún momento del jueves, 30 de diciembre. Hacia la hora de la cena, de aquel lluvioso día, todos los miembros del cuerpo apostólico y de los maestros evangélicos habían llegado a la casa de Zebedeo.
\vs p149 7:2 El grupo se mantuvo unido hasta el día del \bibemph{sabbat,} alojándose en casas de Betsaida y de la cercana Cafarnaúm, tras lo cual, se les dio a todos ellos dos semanas de descanso para ir a ver a sus familias, visitar a sus amigos o ir de pesca. Los dos o tres días que estuvieron juntos en Betsaida fueron ciertamente emotivos e inspiradores; incluso los maestros de más antigüedad se beneficiaron de las experiencias que los jóvenes predicadores les relataban.
\vs p149 7:3 De los ciento diecisiete evangelistas que participaron en este segundo viaje de predicación por Galilea, solo unos setenta y cinco superaron la prueba práctica y estarían disponibles para que se les asignara a alguna misión al concluir las dos semanas de receso. Jesús, con Andrés, Pedro, Santiago y Juan, permaneció en la casa de Zebedeo y pasaron mucho tiempo en reuniones dedicadas al bien del reino y a su expansión.
