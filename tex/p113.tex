\upaper{113}{Los guardianes seráficos del destino}
\author{Jefe de los serafines}
\vs p113 0:1 Habiendo expuesto las narrativas sobre los espíritus servidores del tiempo y las multitudes de mensajeros del espacio, nos proponemos hacer el examen de los ángeles guardianes, los serafines dedicados al ministerio de los mortales individuales, para cuya elevación y perfeccionamiento se ha dispuesto todo un inmenso plan de supervivencia y de avance espiritual. En Urantia, en eras pasadas, estos guardianes del destino constituían casi el único grupo de ángeles conocidos. Los serafines planetarios son, de hecho, espíritus servidores enviados para ejercer su ministerio con quienes sobrevivirán. Estos serafines acompañantes han obrado como ayudantes espirituales del hombre mortal en todos los grandes acontecimientos del pasado y del presente. En muchas revelaciones, “la palabra se dijo por medio de los ángeles”; muchos de los mandatos del cielo se han “recibido mediante el ministerio de los ángeles”.
\vs p113 0:2 Los serafines son tradicionalmente los ángeles del cielo; son los espíritus servidores que tan cerca viven de vosotros y tanto hacen por vosotros. Han realizado su ministerio en Urantia desde los tiempos más tempranos de la inteligencia humana.
\usection{1. LOS ÁNGELES GUARDIANES}
\vs p113 1:1 La enseñanza acerca de los ángeles guardianes no es un mito; determinados grupos de seres humanos tienen de hecho ángeles personales. En reconocimiento de esto, Jesús, al hablar de los niños del reino celestial, dijo: “Mirad que no menospreciéis a uno de estos pequeños, porque os digo que sus ángeles siempre contemplan la presencia del espíritu de mi Padre”.
\vs p113 1:2 Originariamente, los serafines se asignaron expresamente a las distintas razas de Urantia por separado. Pero desde el ministerio de gracia de Miguel, se han asignado conforme a la inteligencia, la espiritualidad y el destino humanos. Intelectualmente, la humanidad se divide en tres clases:
\vs p113 1:3 \li{1.}Quienes tienen una mente por debajo de la normalidad ---aquellos que no ejercen el poder de su voluntad de forma normal; que no toman decisiones corrientes---. Esta clase abarca a los que no pueden comprender a Dios, a quienes carecen de la capacidad de adorar con inteligencia a la Deidad. Los seres de Urantia con este orden de mente tienen, asignados a ellos, un colectivo de serafines, una compañía, con un batallón de querubines, para asistirles y dar fe de que se les hace extensivas la justicia y la misericordia ante las dificultades de su vida en la esfera.
\vs p113 1:4 \li{2.}El tipo de mente humana, promedio y normal. Desde el punto de vista del ministerio seráfico, la mayor parte de los hombres y mujeres se agrupan en siete clases según el estatus conseguido al ir superando los círculos de progreso humano y de desarrollo espiritual.
\vs p113 1:5 \li{3.}Quienes tienen una mente de orden superior a la normal ---aquellos que adoptan grandes decisiones y gozan de un indudable potencial para realizar logros espirituales; hombres y mujeres que disfrutan de un mayor o menor contacto con sus modeladores interiores; miembros de los distintos colectivos de reserva de destino---. Al margen del círculo en el que se encuentre, si ese ser llega a ingresar en cualquiera de estos diferentes colectivos, justo en ese instante, se le asignan serafines personales y, desde ese momento hasta el fin de su andadura terrenal, ese mortal se beneficiará del ministerio continuo y del permanente cuidado de un ángel guardián. También, cuando un ser humano toma \bibemph{la} decisión suprema, cuando existe un verdadero compromiso con el modelador, se asigna de inmediato, a esa alma, un guardián personal.
\vs p113 1:6 \pc En el ministerio de los denominados seres normales, la designación de los serafines se produce según el logro humano de los círculos de intelectualidad y espiritualidad. Empezáis con vuestra mente investida de mortalidad en el séptimo círculo y viajáis hacia el interior con la tarea del entendimiento, la conquista y el dominio de vosotros mismos; y de círculo en círculo avanzáis hasta alcanzar (si la muerte natural no pone fin a vuestra andadura transfiriendo vuestras luchas a los mundos de las moradas) el primer círculo o círculo interior en el que se establece un relativo contacto y comunión con el modelador interior.
\vs p113 1:7 En el círculo inicial o séptimo, los seres humanos tienen un ángel guardián, con una compañía de querubines que los asisten, asignados al cuidado y a la custodia de mil mortales. En el sexto círculo, se nombra a un par seráfico, con una compañía de querubines, para guiar a estos mortales ascendentes en grupos de quinientos. Cuando se alcanza el quinto círculo, los seres humanos se organizan en grupos de aproximadamente cien, y un par de serafines guardianes, con un grupo de querubines, se hacen cargo de ellos. Al lograr el cuarto círculo, se reúnen a los seres mortales en grupos de diez y, de nuevo, un par de serafines, asistidos por una compañía de querubines, se encargan de ellos.
\vs p113 1:8 Cuando una mente mortal rompe la inercia de su legado animal y consigue la capacidad intelectual humana y la adquisición espiritual del tercer círculo, un ángel personal (en realidad dos) se dedicará en lo sucesivo, de forma completa y exclusiva, a dicho mortal ascendente. Y, por consiguiente, estas almas humanas, además de la presencia constante y crecientemente eficaz de los modeladores del pensamiento interiores, reciben la ayuda conjunta de estos guardianes personales del destino en todos sus esfuerzos para finalizar el tercer círculo, pasar por el segundo y lograr el primero.
\usection{2. LOS GUARDIANES DEL DESTINO}
\vs p113 2:1 A los serafines no se les conoce como guardianes del destino hasta el momento en el que se les destina al acompañamiento de un alma humana que ha conseguido uno o más de tres logros: ha tomado la suprema decisión de ser semejante a Dios, ha ingresado en el tercer círculo o se ha incorporado a uno de los colectivos de reserva de destino.
\vs p113 2:2 En la evolución de las razas, se asigna un guardián del destino al primer ser humano que logra conquistar el círculo exigido. En Urantia, el primer mortal en conseguir un guardián personal fue Rantowoc, un hombre sabio de la raza roja de mucho tiempo atrás.
\vs p113 2:3 Todas las misiones angélicas se realizan por un grupo de serafines voluntarios, y estos nombramientos se hacen siempre conforme a las necesidades humanas y según el estatus del par angélico ---en función de la experiencia, destreza y sabiduría seráficas---. Solamente se nombran guardianes del destino a aquellos serafines que hayan prestado sus servicios durante mucho tiempo, a los más experimentados y probados. Muchos guardianes han hecho acopio de una experiencia de gran valor en esos mundos del grupo en los se produce la fusión con el modelador. Al igual que los modeladores, los serafines atienden a estos seres durante el espacio de una sola vida y se les libera entonces para llevar a cabo una nueva misión. Muchos guardianes en Urantia han tenido esta anterior experiencia práctica en otros mundos.
\vs p113 2:4 \pc Cuando los seres humanos no logran sobrevivir, sus guardianes personales o grupales pueden servir repetidas veces en funciones similares en el mismo planeta. Los serafines desarrollan una afectuosa estima hacia cada uno de los mundos y dispensan un cariño especial por determinadas razas y órdenes de criaturas mortales con los que han estado tan estrecha e íntimamente relacionados.
\vs p113 2:5 Los ángeles desarrollan un perdurable afecto por sus acompañantes humanos; y, si pudierais visualizar a los serafines, vosotros también llegaríais a sentir un cálido afecto por ellos. Despojados de cuerpos materiales, poseyendo formas espirituales, estaríais muy cerca de los ángeles en muchos de sus atributos personales. Comparten la mayoría de vuestras emociones y experimentan algunas otras. El único sentimiento que os impulsa a actuar y que para ellos es, de alguna manera, difícil de comprender es el legado del miedo animal, que tanta preponderancia tiene en la vida mental del habitante ordinario de Urantia. A los ángeles les cuesta realmente entender por qué vosotros permitís tan insistentemente que vuestras más elevadas facultades intelectuales, incluso vuestra fe religiosa, se vean tan dominadas por el temor, tan profundamente desmoralizadas por el irreflexivo pánico del miedo y de la ansiedad.
\vs p113 2:6 \pc Todos los serafines tienen nombres individuales, pero en los expedientes sobre su servicio en el mundo de destino se les designa frecuentemente por sus números planetarios. En la sede del universo, están registrados por nombre y número. El guardián del destino de la persona humana a través de la que se realiza esta comunicación de contacto es el número 3 del grupo 17, de la compañía 126, del batallón 4, de la unidad 384, de la legión 6, del cuerpo de ejército 37, del ejército seráfico 182\,314 de Nebadón. El número actual del nombramiento planetario de este serafín en Urantia y en relación a esta persona es el 3\,641\,852.
\vs p113 2:7 \pc En el ministerio de la custodia personal, o nombramiento de los ángeles como guardianes del destino, los serafines siempre ofrecen sus servicios de forma voluntaria. En la ciudad en la que se realiza esta visita, se admitió recientemente en el colectivo de reserva de destino a un determinado mortal y, como el ángel guardián atiende personalmente a los humanos en estas condiciones, más de cien serafines cualificados solicitaron tal tarea. El director planetario eligió a doce de los más experimentados y, posteriormente, nombró al serafín que ellos seleccionaron como el mejor adaptado para guiar a este ser humano en su viaje por la vida. Esto es, optaron por un par de serafines igualmente cualificados; de este par, uno de ellos estará siempre de servicio.
\vs p113 2:8 La labor seráfica puede ser incesante, pero cualquiera de los dos del par angélico puede desempeñar todas las responsabilidades de su ministerio. Como los querubines, los serafines normalmente prestan sus servicios en parejas, pero a diferencia de sus compañeros menos avanzados, los serafines a veces lo hacen de forma individual. En prácticamente todos sus contactos con los seres humanos, pueden actuar de esta manera. Ambos ángeles son necesarios solamente para la comunicación y el servicio en las vías circulatorias más elevadas de los universos.
\vs p113 2:9 Cuando un par seráfico acepta la tarea de custodia, sirve al ser humano por el resto de su vida. El compañero del ser (uno de los dos ángeles) dejará constancia de la actividad humana. Estos serafines, complementarios en su labor, son los ángeles archivistas de los mortales de los mundos evolutivos. Un par de querubines (un querubín y un sanobín), vinculado a los guardianes seráficos, guardan estos expedientes, aunque siempre están bajo la responsabilidad de uno de los serafines.
\vs p113 2:10 Con el propósito de descansar y de recargarse con la energía vital de las vías circulatorias del universo, el guardián queda relevado periódicamente por su compañero y, durante su ausencia, el querubín acompañante actúa como archivista, tal como sucede también cuando el serafín que lo complementa se encuentra asimismo ausente.
\usection{3. RELACIÓN CON OTRAS INFLUENCIAS ESPIRITUALES}
\vs p113 3:1 Una de las cosas más importantes que el guardián del destino hace por su tutelado mortal es llevar a cabo la coordinación personal de las numerosas influencias espirituales impersonales que moran, circundan e inciden en la mente y en el alma de la criatura material evolutiva. Los seres humanos son seres personales, y es extremadamente difícil para los espíritus no personales y las entidades prepersonales efectuar un contacto directo con estas mentes tan sumamente materiales y específicamente personales. A través del ministerio del ángel guardián, estas influencias espirituales, en su totalidad, se unifican en mayor o menor medida y se hacen prácticamente más perceptibles mediante la naturaleza moral en expansión del ser personal humano evolutivo.
\vs p113 3:2 Más concretamente, el guardián seráfico puede correlacionar, tal como en efecto hace, las múltiples instancias intermedias e influencias del Espíritu Infinito, que abarcan desde los ámbitos de los controladores físicos y los espíritus asistentes de la mente hasta el espíritu santo, de la benefactora divina, y la presencia del Espíritu Omnipresente, de la Tercera Fuente y Centro del Paraíso. En consecuencia, habiendo combinado y hecho más personales estos inmensos ministerios del Espíritu Infinito, el serafín asume el compromiso de correlacionar esta influencia unificada del Actor Conjunto con la presencia espiritual del Padre y del Hijo.
\vs p113 3:3 El modelador es la presencia del Padre; el espíritu de la verdad es la presencia de los hijos del Paraíso. Estos dones divinos se unifican y coordinan, en los niveles de menor rango de la experiencia espiritual humana, gracias al ministerio de los serafines guardianes. Los servidores angélicos poseen el don de poder combinar el amor del Padre y la misericordia del Hijo en su ministerio a las criaturas mortales.
\vs p113 3:4 Y aquí es donde se desvela la razón por la que el guardián seráfico acaba por convertirse en el custodio personal de los patrones mentales, de las fórmulas de la memoria y de las realidades del alma del mortal superviviente durante ese intervalo entre la muerte física y la resurrección morontial. Nadie, excepto los hijos servidores del Espíritu Infinito, podría actuar así, en beneficio de la criatura humana, durante esta etapa de transición de un nivel del universo a otro nivel superior. Incluso cuando estéis inmersos en el último sueño de transición, cuando pasáis del tiempo a la eternidad, un alto supernafín comparte asimismo el tránsito contigo como custodio de la identidad de la criatura y garantía de la integridad personal.
\vs p113 3:5 En el nivel espiritual, los serafines hacen personales muchos ministerios del universo, por lo demás impersonales y prepersonales; son coordinadores. En el nivel intelectual, correlacionan la mente y la morontia; son intérpretes. Y en el nivel físico, actúan sobre el entorno terrestre a través de su enlace con los controladores físicos mayores y mediante el ministerio cooperativo de las criaturas intermedias.
\vs p113 3:6 Este es un relato de la actividad, múltiple e intrincada, de un serafín en su labor de acompañante; pero, ¿de qué manera puede este ser personal angélico de menor rango, creado tan solo algo por encima del nivel de la humanidad en el entorno del universo, hacer cosas tan difíciles y complejas? En realidad no lo sabemos, pero suponemos que su excepcional ministerio se ve facilitado, de una manera no desvelada, por el trabajo desconocido del Ser Supremo, la Deidad en vías de actualización de los universos evolutivos del tiempo y del espacio. En la totalidad del ámbito de la supervivencia en su avance en el Ser Supremo, y por medio de él, los serafines son una parte esencial del avance permanente de los mortales.
\usection{4. ÁREAS DE ACCIÓN SERÁFICAS}
\vs p113 4:1 Los serafines guardianes no son mente, aunque ciertamente nacen de la misma fuente que da también origen a la mente mortal, esto es, del espíritu creativo. Los serafines estimulan la mente; procuran continuamente propiciar en la mente humana las decisiones que llevan a superar los círculos. Lo llevan a efecto no como lo hace el modelador, operando desde dentro y a través del alma, sino, más bien, desde fuera hacia dentro, obrando a través del entorno social, ético y moral de los seres humanos. Los serafines no atraen a los mortales hacia el Padre como hacen los divinos modeladores, pero actúan como intermediarios personales del ministerio impartido por el Espíritu Infinito.
\vs p113 4:2 El hombre mortal, sujeto a la dirección del modelador, es también susceptible a la guía seráfica. El modelador es la esencia de la eterna naturaleza del hombre; el serafín es el maestro de la naturaleza evolutiva del hombre ---en esta vida, la mente mortal; en la próxima, el alma morontial---. En los mundos de las moradas seréis conscientes y conocedores de los instructores seráficos; pero, en la primera vida, los hombres por lo general los desconocen.
\vs p113 4:3 Los serafines actúan en calidad de maestros de los hombres, orientando los pasos del ser personal humano por los caminos de experiencias nuevas y en avance. Aceptar la guía de un serafín en raras ocasiones significa llevar una vida fácil. Al seguirla, encontraréis con toda seguridad las escarpadas colinas de la elección moral y del progreso espiritual y, si tenéis el coraje, las llegaréis a cruzar.
\vs p113 4:4 En gran medida, el estímulo a la adoración se origina en los impulsos espirituales de los asistentes superiores de la mente, reforzados por la guía del modelador. Pero el deseo de orar, que tan a menudo, experimentan los mortales conscientes de Dios, se presenta, con gran frecuencia, como resultado de la acción seráfica. El serafín guardián está constantemente actuando sobre el entorno del mortal a fin de aumentar la percepción cósmica del ascendente humano, con el objeto de que dicho aspirante a la supervivencia pueda adquirir una mejor toma de conciencia de la presencia del modelador interior y, de este modo, esté en condiciones de ofrecer una creciente cooperación con la misión espiritual de la presencia divina.
\vs p113 4:5 Aunque al parecer no existe comunicación alguna entre los modeladores interiores y los serafines que circundan al hombre, parecen siempre trabajar en perfecta armonía y excelente consonancia. Los guardianes son más activos en esos momentos en los que los modeladores lo son menos, pero su ministerio está, de alguna manera, extrañamente correlacionado. Resulta difícil que una cooperación tan espléndida pueda ser casual o circunstancial.
\vs p113 4:6 El ser personal del serafín guardián que imparte su ministerio, la presencia divina del modelador interior, la acción encauzada del espíritu santo y la conciencia del hijo del Paraíso que posee el espíritu de la verdad se correlacionan de forma divina conformando una unidad significativa de ministerio espiritual en y para el ser personal. Aunque proceden de diferentes fuentes y niveles, todas estas influencias celestiales se integran en la presencia envolvente y en evolución del Ser Supremo.
\usection{5. EL MINISTERIO SERÁFICO IMPARTIDO A LOS MORTALES}
\vs p113 5:1 Los ángeles no invaden la santidad de la mente humana; no actúan sobre la voluntad de los mortales; tampoco se ponen en contacto directo con los modeladores interiores. El guardián del destino os influye de todas las formas posibles que sean compatibles con la dignidad de vuestro ser personal; bajo ningún concepto interfieren estos ángeles con la libre acción de la voluntad humana. Ni los ángeles ni ningún otro orden de ser personal del universo tienen poder o autoridad para restringir o coartar las prerrogativas de la elección humana.
\vs p113 5:2 Los ángeles están tan cerca de vosotros y os cuidan con tan honda emoción que, figurativamente, “lloran por vuestra deliberada intolerancia y terquedad”. Los serafines no derraman lágrimas físicas; carecen de cuerpos físicos; tampoco poseen alas. Pero, ciertamente, tienen emociones espirituales y experimentan sensaciones y sentimientos de naturaleza espiritual que son, de determinadas maneras, equiparables a las emociones humanas.
\vs p113 5:3 Los serafines obran en vuestro favor, totalmente al margen de que se lo pidáis directamente; ejecutan los mandatos de sus superiores y, por lo tanto, desempeñan sus funciones, a pesar de vuestras conveniencias circunstanciales o de vuestros mudables estados de ánimo. Esto no significa que no podáis facilitar o dificultar sus tareas, sino más bien que a los ángeles no les concierne, de forma directa, vuestras peticiones ni vuestras oraciones.
\vs p113 5:4 En la vida en la carne, la inteligencia de los ángeles no es directamente accesible a los hombres mortales. No imperan sobre vosotros ni os dirigen; son simplemente guardianes. Los serafines os \bibemph{guardan;} no buscan influir sobre vosotros de un modo directo; debéis trazar vuestro propio camino y, es entonces, cuando los ángeles hacen el mejor uso posible de la ruta por la que habéis optado. No intervienen (habitualmente) con arbitrariedad en las cuestiones rutinarias de la vida humana. Si bien, cuando reciben instrucciones de sus superiores para realizar alguna gesta extraordinaria, podéis tener la seguridad de que estos guardianes hallarán la forma de llevar a cabo tales mandatos. Por consiguiente, no se inmiscuyen en el escenario de la actuación humana salvo en casos de urgencia y suelen hacerlo, además, cumpliendo las órdenes directas de sus superiores. Son los seres que os seguirán durante muchas eras, y están recibiendo, de esta manera, una introducción acerca de su labor futura y de las relaciones con otros seres personales.
\vs p113 5:5 \pc En circunstancias determinadas, los serafines son capaces de actuar en calidad de servidores materiales para los seres humanos, pero su labor en esta función es muy poco común. Con la asistencia de las criaturas intermedias y de los controladores físicos, pueden desarrollar una amplia variedad de actividades en beneficio de los mortales, incluso establecer un contacto real con la humanidad, pero estos acontecimientos son muy poco comunes. En la mayoría de los casos, las circunstancias del ámbito material transcurren sin alteraciones gracias a la acción seráfica, aunque se han suscitado ocasiones en las que existía riesgo para algún eslabón vital de la cadena de la evolución humana, en ella, los guardianes seráficos han obrado, y debidamente, por iniciativa propia.
\usection{6. LOS ÁNGELES GUARDIANES TRAS LA MUERTE}
\vs p113 6:1 Habiéndoos dicho algo sobre el ministerio de los serafines durante la vida natural, trataré de informaros acerca de la conducta de los guardianes del destino en el momento de la disolución material de sus acompañantes humanos. Tras la muerte, vuestro historial, las especificaciones de vuestra identidad y la entidad morontial del alma humana ---evolucionada de forma conjunta mediante el ministerio de la mente mortal y del modelador divino--- se conservan lealmente por el guardián del destino junto con todos los demás valores relacionados con vuestra existencia futura, todo lo que constituye vuestro yo, vuestro yo real, salvo la identidad de la existencia continuada, representada por el modelador que parte y el ser personal existente.
\vs p113 6:2 En el instante en el que desaparece la luz piloto presente en la mente humana, la luminosidad espiritual que los serafines relacionan con la presencia del modelador, el ángel acompañante presenta en persona un informe ante los ángeles al mando, sucesivamente, del grupo, compañía, batallón, unidad, legión y cuerpo de ejército; y tras registrarse debidamente para la aventura final del tiempo y del espacio, dicho ángel recibe la acreditación por parte del jefe planetario de serafines para presentarse ante la estrella vespertina (o algún otro lugarteniente de Gabriel) al mando del ejército seráfico de este aspirante a ascender en el universo. Y cuando el comandante de esta unidad organizativa suprema le concede el permiso, dicho guardián del destino avanza al primer mundo de las moradas y espera allí la recuperación de la conciencia de su antiguo pupilo en la carne.
\vs p113 6:3 \pc En caso de que el alma humana no logre sobrevivir tras habérsele asignado un ángel personal, el serafín que lo acompañaba debe continuar hasta la sede del universo local para dar testimonio de los expedientes completos aportados con anterioridad por su ser complementario. A continuación, comparece ante los tribunales de los arcángeles para ser eximido de toda culpa respecto al fracaso de la supervivencia de su tutelado y, luego, regresa a los mundos, para ser destinado a otro mortal potencialmente ascendente o a alguna otra división del ministerio seráfico.
\vs p113 6:4 \pc Pero los ángeles imparten su ministerio a las criaturas evolutivas de muchas maneras, aparte de los servicios que prestan de custodia personal y grupal. Los guardianes personales cuyos tutelados no van inmediatamente a los mundos de las moradas, no permanecen ociosos esperando el llamamiento nominal para el juicio de una dispensación; se les vuelve a destinar a numerosas misiones para prestar su servicio en todo el universo.
\vs p113 6:5 El guardián serafín es el depositario de la custodia de los valores de supervivencia del alma dormida del hombre mortal, al igual que el modelador ausente \bibemph{es} la identidad de ese ser inmortal del universo. Cuando ambos colaboran en las salas de resurrección de los mundos de las moradas en conjunción con la forma morontial recién creada, se produce la reconstrucción de las partes constituyentes del ser personal del ascendente mortal.
\vs p113 6:6 El modelador os identificará; el serafín guardián reconstituirá vuestro ser personal y seguidamente os presentará de nuevo al mentor fiel de vuestros días en la tierra.
\vs p113 6:7 E incluso así, cuando termina una era planetaria, cuando se congregan aquellos que están superando los círculos inferiores, son sus guardianes grupales los que los reúnen en las salas de resurrección de las esferas de las moradas, tal como consta en vuestros archivos: “Y enviará a sus ángeles con gran voz y juntarán a sus escogidos desde un extremo del mundo hasta el otro”.
\vs p113 6:8 \pc El método de la justicia precisa que los guardianes personales o grupales respondan, al fin de una dispensación, al llamamiento nominal en representación de todos los seres personales no supervivientes. Los modeladores de esos seres que no han sobrevivido no regresan y, cuando se pasa lista, los serafines responden, pero los modeladores no. Esto constituye la “resurrección de los injustos”, que se trata en realidad del reconocimiento formal del cese de la existencia de dichas criaturas. Dicho llamamiento nominal de la justicia siempre se produce de forma inmediata al de la misericordia, o resurrección de los supervivientes dormidos. Pero estas son cuestiones que no incumben a nadie más que a los supremos y omnisapientes jueces de los valores de supervivencia. No nos conciernen realmente estos problemas en los que hay resoluciones judiciales implicadas.
\vs p113 6:9 \pc Los guardianes grupales pueden servir en un planeta era tras era y, en algún momento, convertirse en custodios de las almas dormidas de miles y miles de supervivientes dormidos. Pueden prestar sus servicios en numerosos mundos diferentes de un sistema determinado, puesto que su respuesta a la resurrección ocurre en los mundos de las moradas.
\vs p113 6:10 Todos los guardianes personales y grupales del sistema de Satania que llegaron a descarriarse en la rebelión de Lucifer, a pesar de que muchos se arrepintieron sinceramente de su locura, deben permanecer recluidos en Jerusem hasta el dictamen final de la rebelión. Ya los censores universales han desprovisto, de forma discrecional, a estos guardianes desobedientes e infieles de todos los aspectos de los que eran depositarios como responsables de las almas, y estas realidades morontiales se han depositado para su salvaguarda bajo la custodia de seconafines voluntarios.
\usection{7. LOS SERAFINES Y LA ANDADURA ASCENDENTE}
\vs p113 7:1 En la andadura del mortal ascendente, este primer despertar en las orillas del mundo de las moradas significa, de hecho, un momento trascendental; allí, por primera vez, veréis realmente a vuestros compañeros angélicos, por tanto tiempo amados y siempre presentes con vosotros durante los días en la tierra; allí también seréis verdaderamente conscientes de la identidad y de la presencia del mentor divino que, por tanto tiempo, habitó en vuestra mente en la tierra. Esta experiencia representa un despertar glorioso, una verdadera resurrección.
\vs p113 7:2 En las esferas morontiales, los serafines que os asisten (hay dos de ellos) son abiertamente compañeros vuestros. Estos ángeles no solo os acompañan a medida que progresáis en vuestra andadura por los mundos de transición, ayudándoos de todas las formas posibles a adquirir el estado morontial y el espiritual, sino que también aprovechan esa oportunidad para avanzar, estudiando en las escuelas de extensión formativa destinadas para los serafines evolutivos que continúan en los mundos de las moradas.
\vs p113 7:3 La raza humana se creó solo algo por debajo de los tipos más sencillos de órdenes angélicos. Por consiguiente, una vez que lográis la conciencia de vuestro ser personal, con posterioridad a la liberación de vuestras ataduras de la carne, la primera tarea que os aguarda de forma inmediata será la de ser asistentes del serafín.
\vs p113 7:4 Antes de dejar los mundos de las moradas, todos los mortales tendrán unos compañeros o guardianes seráficos de forma permanente. Y, a medida que ascendáis las esferas morontiales, serán los guardianes seráficos, en última instancia, los que testimoniarán y certificarán los decretos de vuestra unión eterna con los modeladores del pensamiento. Juntos han constituido vuestras identidades personales como hijos de la carne de los mundos del tiempo. Luego, una vez alcanzado los ámbitos morontiales más avanzados, os acompañan en vuestro paso por Jerusem y sus mundos vinculados, dedicados a la formación sobre el progreso y la cultura del sistema. Tras eso, van con vosotros a Edentia y a sus setenta esferas que se especializan en la socialización avanzada y, posteriormente, os guiarán hasta los mundos melquisedecs y os seguirán a lo largo de la formidable andadura en los mundos sedes del universo. Y cuando hayáis aprendido la sabiduría y cultura de este orden de seres, os conducirán hasta Lugar de Salvación, en donde os hallaréis frente al Soberano de todo Nebadón. Estos guías seráficos incluso os seguirán en vuestro recorrido por los sectores menor y mayor del suprauniverso y hasta los mundos receptores de Uversa, permaneciendo con vosotros hasta que finalmente os envolváis en un seconafín para el largo viaje a Havona.
\vs p113 7:5 Algunos de los guardianes del destino asignados a la andadura de los mortales siguen la ruta de los peregrinos a su paso por Havona. Los otros se despiden temporalmente de los que han sido por tanto tiempo sus compañeros mortales y, entonces, mientras estos hacen su travesía por los círculos del universo central, dichos guardianes del destino llegan a los círculos de Lugar de Serafines. Y estarán esperando en las orillas del Paraíso cuando sus compañeros humanos despierten del último sueño de tránsito del tiempo a las nuevas experiencias de la eternidad. Estos serafines ascendentes entran luego, prestando servicios diferentes, en el colectivo de finalizadores y en el colectivo seráfico de la consumación.
\vs p113 7:6 El hombre y el ángel pueden o no reunirse para emprender el servicio eterno, pero donde quiera que el nombramiento seráfico los pueda llevar, los serafines están siempre en comunicación con sus antiguos pupilos de los mundos evolutivos, con los mortales ascendentes del tiempo. Las relaciones estrechas y los vínculos afectivos de los mundos de origen humano no se olvidan jamás ni se rompen por completo. En eras eternas, hombres y ángeles cooperarán en el servicio divino así como lo hicieron en la andadura del tiempo.
\vs p113 7:7 \pc Para los serafines, la forma más segura de llegar hasta las Deidades del Paraíso es poder guiar con éxito un alma de origen evolutivo hasta las puertas del Paraíso. Por este motivo, el nombramiento de guardián del destino es para ellos el cometido seráfico más altamente apreciado.
\vs p113 7:8 Solo los guardianes del destino se incorporan al colectivo primario o final de los mortales, y estas parejas han emprendido la suprema aventura de su identidad en unicidad; los dos seres han alcanzado la biunificación espiritual en Lugar de Serafines antes de su ingreso en el colectivo de los finalizadores. En este caso, las dos naturalezas angélicas, tan complementarias en el ejercicio de toda su actividad en el universo, logran la condición última de dos espíritus en unicidad, lo que se refleja en una nueva capacidad para la recepción de una fracción del Padre del Paraíso diferente a la del modelador y su fusión con ella. Así, algunos de vuestros afectuosos compañeros seráficos del tiempo se convierten también en vuestros compañeros finalizadores de la eternidad ---hijos del Supremo e hijos perfeccionados del Padre del Paraíso---.
\vsetoff
\vs p113 7:9 [Exposición del jefe de los serafines emplazados en Urantia.]
