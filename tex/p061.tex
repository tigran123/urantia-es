\upaper{61}{La era de los mamíferos en Urantia}
\author{Portador de vida}
\vs p061 0:1 La era de los mamíferos se extiende desde el origen de los mamíferos placentarios hasta el final de la edad del hielo y abarca algo menos de cincuenta millones de años.
\vs p061 0:2 Durante esta era Cenozoica, el paisaje del mundo tenía un aspecto agradable: colinas onduladas, extensos valles, amplios ríos y grandes bosques. En este espacio de tiempo, el istmo de Panamá se elevó y hundió dos veces; al puente terrestre del Estrecho de Bering le sucedió lo mismo tres veces. Los tipos de animales existentes eran muchos y distintos. Los árboles estaban repletos de aves, y el mundo al completo era un paraíso para los animales, a pesar de la incesante pugna por la supremacía que libraban las especies animales en evolución.
\vs p061 0:3 Los depósitos acumulados de los cinco períodos de esta era, de cincuenta millones de años de duración, contienen los registros fosilizados del dominio sucesivo de los mamíferos, y conducen directamente hasta los tiempos de la aparición misma del hombre.
\usection{1. LA NUEVA ETAPA DE SUELO CONTINENTAL. LA ERA DE LOS PRIMEROS MAMÍFEROS}
\vs p061 1:1 Hace \bibemph{50\,000\,000}de años, las áreas de suelo terrestres del mundo se hallaban en general por encima del agua o solo ligeramente sumergidas. Las formaciones y los depósitos de este período son, a la vez, de tierra y de mar, aunque principalmente de la primera. Durante bastante tiempo, el suelo se elevaba de forma gradual pero, simultáneamente, era arrastrado por las aguas a niveles inferiores y hacia los mares.
\vs p061 1:2 En los inicios de este período, apareció de repente, en América del Norte, el tipo de mamífero placentario, constituyéndose en el desarrollo evolutivo más importante hasta este momento. Previamente, había habido órdenes de mamíferos no placentarios, pero esta nueva clase de mamíferos surgió directa y \bibemph{de repente} de un ancestro reptil ya existente, cuyos descendientes habían persistido a través de los tiempos del declive de los dinosaurios. El padre de los mamíferos placentarios fue un tipo de dinosaurio pequeño, bastante activo, carnívoro y saltador.
\vs p061 1:3 Los instintos básicos de los mamíferos empezaron a manifestarse ya en estos primitivos tipos. En cuanto a su supervivencia, los mamíferos poseen una gran ventaja sobre todas las demás formas de vida animal, puesto que pueden:
\vs p061 1:4 \li{1.}Dar nacimiento a una descendencia relativamente madura y bien desarrollada.
\vs p061 1:5 \li{2.}Alimentar, cuidar y proteger a su descendencia con atención y afecto.
\vs p061 1:6 \li{3.}Emplear su capacidad cerebral superior para perpetuarse a sí mismos.
\vs p061 1:7 \li{4.}Utilizar una mayor destreza para escapar de sus enemigos.
\vs p061 1:8 \li{5.}Aplicar su inteligencia superior para adaptarse al cambio medioambiental.
\vs p061 1:9 \pc Hace \bibemph{45\,000\,000} de años, las crestas continentales se elevaron en conjunción con un hundimiento muy generalizado de las costas. Los mamíferos evolucionaban con rapidez. Prosperó un tipo de reptil mamífero pequeño, ponedor de huevos, y los ancestros de los futuros canguros recorrían Australia. Pronto hubo pequeños caballos, rinocerontes veloces, tapires con trompa, cerdos primitivos, ardillas, lémures, zarigüeyas y algunas tribus de animales parecidos a los monos. Todos eran pequeños, primitivos y mejor adaptados para vivir en los bosques de las regiones montañosas. Un ave terrestre grande, similar al avestruz, llegó a alcanzar una altura de tres metros, y ponía huevos de unos veintitrés por treinta y tres centímetros. Estas serían las antecesoras de las futuras aves gigantes de pasaje que tan inteligentes eran y que, en otro tiempo, transportaron a seres humanos por los aires.
\vs p061 1:10 Los mamíferos del primer periodo del Cenozoico vivían en el suelo terrestre, bajo el agua, en el aire y en las copas de los árboles. Tenían de uno a once pares de glándulas mamarias y todos estaban cubiertos de abundante pelo. En común con los órdenes que aparecerían después, desarrollaron dos conjuntos sucesivos de dientes y poseían cerebros grandes en proporción a su tamaño corporal. Pero ninguna de las especies modernas se contaba entre ellos.
\vs p061 1:11 \pc Hace \bibemph{40\,000\,000} de años, las áreas de suelo terrestre del hemisferio norte comenzaron a elevarse, y a esto le siguieron unos nuevos y extensos yacimientos de suelo junto a otros tipos de actividad terrestre, en los que se incluyen flujos de lava, deformaciones, formaciones de lagos y erosiones.
\vs p061 1:12 Durante la última parte de esta época, la mayoría de Europa estaba sumergida. Tras una ligera elevación del suelo, el continente se cubrió de lagos y bahías. El Océano Ártico, a través de la depresión de los Urales, se desplazó hacia el sur hasta quedar comunicado con el Mar Mediterráneo, que entonces se extendía hacia el norte, con las altiplanicies de los Alpes, con los Cárpatos, con los Apeninos y con los Pirineos, que estaban por encima del agua como islas marinas. El istmo de Panamá había emergido; los Océanos Atlántico y Pacífico estaban separados. América del Norte se comunicaba con Asia mediante el puente terrestre del Estrecho de Bering y con Europa mediante Groenlandia e Islandia. El trazado del suelo de la Tierra en las latitudes septentrionales estaba solamente interrumpido por los Estrechos de los Urales, que unían los mares árticos con el expandido Mediterráneo.
\vs p061 1:13 En las aguas europeas se depositó una gran cantidad de piedra caliza foraminífera. Hoy día, esta misma piedra se halla a una altura de 3000 metros en los Alpes, a casi 4900 metros en el Himalaya y a 6000 metros en el Tíbet. Los depósitos de creta de este período se encuentran a lo largo de las costas de África y de Australia, en la costa occidental de América del Sur y alrededor de las Antillas.
\vs p061 1:14 \pc A lo largo de todo este periodo, llamado \bibemph{Eoceno,} la evolución de los mamíferos y de otras formas de vida afines continuó con escasa o ninguna interrupción. América del Norte se comunicaba entonces por tierra con todos los continentes excepto con Australia, y el mundo, gradualmente, se fue llenando de una fauna con distintas clases de mamíferos primitivos.
\usection{2. LA ÚLTIMA ETAPA DE INUNDACIONES. LA ERA DE LOS MAMÍFEROS AVANZADOS.}
\vs p061 2:1 Este período se caracterizó por una nueva y rápida evolución de los mamíferos placentarios; durante estos tiempos, se desarrollaron las formas más avanzadas de los mamíferos.
\vs p061 2:2 Aunque los ancestros de los primeros placentarios eran carnívoros, muy pronto se desarrollaron grupos de mamíferos herbívoros y, antes de que transcurriera mucho tiempo, también surgieron familias de mamíferos omnívoros. Las angiospermas, la flora terrestre moderna, en las que se incluyen la mayoría de las plantas y árboles actuales, y que habían aparecido durante períodos anteriores, constituían la principal fuente alimenticia de estos mamíferos, que rápidamente aumentaron su número.
\vs p061 2:3 \pc Hace \bibemph{35\,000\,000} de años comienza la era del predominio de los mamíferos placentarios a nivel mundial. El puente terrestre meridional tenía una gran extensión y de nuevo comunicó al entonces enorme continente antártico con América del Sur, África del Sur y Australia. A pesar de la concentración del suelo terrestre en altas latitudes, el clima global permaneció relativamente suave, debido al enorme incremento del tamaño de los mares tropicales; tampoco se elevó el suelo lo suficiente como para formarse glaciares. Se produjeron grandes flujos de lava en Groenlandia e Islandia, depositándose alguna cantidad de carbón entre estas capas.
\vs p061 2:4 En la fauna del planeta se estaban dando cambios notables. La vida marina estaba experimentando una gran modificación; la mayoría de los órdenes de la vida marina, presentes hoy día, ya existían, y los foraminíferos continuaban ejerciendo una importante función. Los insectos eran muy parecidos a los de la era previa. Los lechos de fósiles de Florissant, en Colorado, pertenecen a los postreros años de estos tiempos tan distantes. La mayoría de las familias de los insectos vivos se remontan a este período, pero muchos de los que existían entonces están ahora extintos, aunque permanecen sus fósiles.
\vs p061 2:5 En el suelo terrestre, esta era fue fundamentalmente la de la renovación y expansión de los mamíferos. Antes de finalizar este periodo de los mamíferos primeros y más primitivos, más de cien especies se habían extinguido. Incluso los mamíferos de gran tamaño y cerebro pequeño perecieron pronto. La inteligencia y la agilidad habían reemplazado a las corazas y al tamaño en el progreso de la supervivencia animal. Y con el declive de la familia de los dinosaurios, los mamíferos lentamente dominaron la Tierra, acabando de forma rápida y por completo con el resto de sus ancestros reptiles.
\vs p061 2:6 Junto con la desaparición de los dinosaurios, ocurrieron otros grandes cambios en las distintas ramas de la familia de los saurios. Los miembros supervivientes de las tempranas familias de los reptiles son las tortugas, las serpientes y los cocodrilos, a la vez que la venerable rana, el único grupo representativo restante de los primeros antecesores del hombre.
\vs p061 2:7 Varios grupos de mamíferos se originaron a partir de un animal excepcional, ahora extinto. Esta criatura, carnívora, era una especie de cruce entre gato y foca; podía vivir en el suelo o en el agua y era extraordinariamente inteligente y muy activa. En Europa, este antecesor de la familia canina evolucionó, dando pronto origen a muchas especies de perros pequeños. Casi al mismo tiempo, aparecieron los roedores, incluyendo los castores, las ardillas, los topos, los ratones y los conejos, que pronto llegaron a destacar como forma de vida. Pocos cambios se han producido en esta familia desde entonces. Los depósitos tardíos de este período contienen los restos fósiles de perros, gatos, mapaches y comadrejas en su forma ancestral.
\vs p061 2:8 \pc Hace \bibemph{30\,000\,000} de años, los tipos modernos de mamíferos comenzaron a hacer su aparición. Con anterioridad, los mamíferos habían vivido, en su mayor parte, en los montes; eran de tipo montaraz; \bibemph{de repente,} empezó la evolución de los mamíferos de planicie o ungulados, esto es, las especies pasteadoras a diferencia de las carnívoras con garras. Las pasteadoras provenían de un ancestro indiferenciado de cinco dedos y cuarenta y cuatro dientes, que desapareció antes del final de esta era. Durante este periodo, no se evolucionó más allá de los tres dedos.
\vs p061 2:9 El caballo, un excelente ejemplo de la evolución, vivió, durante estos tiempos en América del Norte al igual que en Europa, aunque su desarrollo no se llegó a completar del todo hasta la edad del hielo tardía. Aunque la familia de los rinocerontes apareció al final de este período, sería después cuando experimentaría su mayor expansión. Además, evolucionó una pequeña criatura parecida al cerdo, que llegó a ser el antecesor de las numerosas especies de cerdos, pecaríes e hipopótamos. Los camellos y las llamas tuvieron su origen en América del Norte, hacia mediados de este período, y se expandieron por las llanuras occidentales. Posteriormente, las llamas emigraron hacia América del Sur, los camellos hacia Europa y pronto ambos se extinguieron en América del Norte, aunque algunos camellos sobrevivieron hasta la edad del hielo.
\vs p061 2:10 Por estos tiempos, sucedió algo relevante en el oeste de América del Norte: los primeros antecesores de los ancestrales lémures aparecieron por primera vez. Aunque no se puede considerar a esta familia como auténticos lémures, su llegada significó el establecimiento de la línea de la que surgirían después los verdaderos lémures.
\vs p061 2:11 Al igual que las serpientes terrestres de una era previa que se habían adaptado a los mares, ahora una tribu entera de mamíferos placentarios dejó la tierra para habitar los océanos. Desde entonces, han permanecido en el mar y han dado origen a las ballenas, delfines, marsopas, focas y lobos marinos modernos.
\vs p061 2:12 La avifauna del planeta continuó desarrollándose, pero con pocos cambios evolutivos notables. La mayoría de las aves modernas ya existían, incluyendo a las gaviotas, las garzas, los flamencos, los buitres, los halcones, las águilas, los búhos, las codornices y los avestruces.
\vs p061 2:13 \pc Hacia el final de este período \bibemph{Oligoceno,} que abarcó diez millones de años, la vida vegetal, junto con la vida marina y los animales terrestres, habían mayormente evolucionado y estaban presentes en la Tierra tal como hoy en día. Luego se dio un elevado grado de especialización adaptativa, pero las formas ancestrales de la mayoría de los seres vivos existían ya en aquel momento.
\usection{3. LA ETAPA DE LAS MONTAÑAS MODERNAS. LA ERA DEL ELEFANTE Y DEL CABALLO}
\vs p061 3:1 La elevación del suelo y la separación de los mares estaban lentamente cambiando el tiempo atmosférico del mundo, enfriándolo de forma gradual; sin embargo, el clima seguía siendo suave. Las secuoyas y las magnolias crecían en Groenlandia, pero las plantas subtropicales comenzaron a emigrar en dirección sur. Hacia el final del período, estas plantas y árboles de clima caluroso habían desaparecido en gran parte de las latitudes septentrionales; siendo reemplazados por plantas más resistentes y árboles caducifolios.
\vs p061 3:2 Hubo gran incremento de las variedades de hierbas y, gradualmente, se transformaron los dientes de muchas especies de mamíferos hasta que se llegaron a corresponder con el tipo actual de pasteadores.
\vs p061 3:3 \pc Hace \bibemph{25\,000\,000} de años, tras la larga época de elevación del suelo, se produjo un ligero sumergimiento. La región de las Montañas Rocosas permaneció muy elevada por lo que la sedimentación del material de erosión continuó depositándose en todas las tierras bajas del este. Las Sierras se volvieron a elevar bastante; de hecho, siguen haciéndolo desde entonces. La gran falla vertical de seis kilómetros y medio de la región de California data de este tiempo.
\vs p061 3:4 \pc Hace \bibemph{20\,000\,000} de años se dio de hecho la edad de oro de los mamíferos. El puente terrestre del Estrecho de Bering se elevó, y muchos grupos de animales emigraron a América del Norte desde Asia, incluyendo a los mastodontes de cuatro colmillos, a los rinocerontes de patas cortas y a muchas variedades de la familia felina.
\vs p061 3:5 Aparecieron los primeros ciervos, y pronto América del Norte se llenó de rumiantes: ciervos, bueyes, camellos, bisontes y algunas especies de rinocerontes, pero los cerdos gigantes, cuya altura era de casi dos metros, se extinguieron.
\vs p061 3:6 Los enormes elefantes de este periodo y de los siguientes tenían un cerebro grande al igual que un cuerpo grande, y pronto se expandieron por todo el mundo, exceptuando Australia. Por una vez, el mundo estaba dominado por un animal de tan grandes dimensiones con un cerebro lo suficientemente grande como para posibilitarle perpetuarse. Frente a la vida altamente inteligente de estas eras, ningún animal del tamaño de un elefante podría haber sobrevivido a no ser que poseyese un cerebro de mayor tamaño y atributos superiores. En relación a inteligencia y adaptación, solo el caballo le va a la zaga y solo el hombre lo supera. Incluso así, de las cincuenta especies de elefantes que existían al comienzo de este período, solo dos han sobrevivido.
\vs p061 3:7 \pc Hace \bibemph{15\,000\,000} de años, las regiones montañosas de Eurasia se estaban elevando y se produjo alguna actividad volcánica por todas estas regiones, pero nada que se pudiera comparar con los flujos de lava del hemisferio occidental. En todo el mundo predominaban estas inestables condiciones.
\vs p061 3:8 El Estrecho de Gibraltar se cerró, y España quedó comunicada con África por el viejo puente terrestre, pero el Mediterráneo desembocaba en el Atlántico mediante un estrecho canal que se extendía a través de Francia; las cumbres montañosas y las zonas altas asomaban por encima de este ancestral mar como si se tratasen de islas. Luego, estos mares europeos comenzaron a retroceder. Aún más tarde, el Mediterráneo se unió con el Océano Índico, mientras que, al final de este período, la región de Suez se elevaba de tal forma que el Mediterráneo llegó a ser, durante un tiempo, un mar interior de agua salada.
\vs p061 3:9 El puente terrestre de Islandia se sumergió y las aguas árticas se mezclaron con las del Océano Atlántico. La costa atlántica de América del Norte se enfrió rápidamente, pero la costa del Pacífico permaneció más cálida de lo que lo está en el presente. Las grandes corrientes oceánicas se pusieron en funcionamiento y afectaron al clima tanto como lo hacen hoy.
\vs p061 3:10 Los mamíferos continuaron evolucionando. Ingentes manadas de caballos se unieron a los camellos en las llanuras occidentales de América del Norte; aquella fue realmente la era de los caballos al igual que la de los elefantes. En cuanto a sus atributos animales, el cerebro del caballo está próximo al del elefante, pero en un aspecto es indudablemente inferior: los caballos nunca llegaron a vencer del todo su propensión, profundamente asentada, de huir cuando se asustan. El caballo carece del dominio emocional del elefante, mientras que el elefante está muy limitado por su tamaño y falta de agilidad. Durante este período, evolucionó un animal que se parecía en cierta manera tanto al elefante como al caballo, pero fue pronto exterminado por la familia felina que aumentaba de forma acelerada.
\vs p061 3:11 \pc A medida que Urantia entra en la llamada “era sin caballos”, deberíais deteneros a reflexionar sobre lo que este animal significó para vuestros ancestros. El hombre usó primero al caballo para alimentarse, luego para transportarse y después para la agricultura y la guerra. Desde hace mucho tiempo, el caballo ha servido a la humanidad y ha desempeñado una importante función en el desarrollo de la civilización humana.
\vs p061 3:12 \pc Los desarrollos biológicos de este período contribuyeron sobremanera a preparar el escenario para la posterior aparición del hombre. En Asia central evolucionaron los verdaderos tipos de monos y gorilas primitivos, los cuales tenían un antecesor común, ahora extinto. Pero ninguna de estas especies está relacionada con la línea de seres vivos que, posteriormente, se convertirían en los ancestros de la raza humana.
\vs p061 3:13 La familia canina estaba representada por varios grupos, en especial por los lobos y por los zorros; la tribu felina, por las panteras y por los grandes tigres de dientes de sable; estos últimos se desarrollaron primero en América del Norte. Las familias felina y canina modernas proliferaron en todo el mundo. Las comadrejas, martas, nutrias y mapaches proliferaron y se desarrollaron en todas las latitudes septentrionales.
\vs p061 3:14 Las aves continuaron evolucionando, aunque se produjeron pocos cambios de consideración. Los reptiles eran similares a los tipos modernos: serpientes, cocodrilos y tortugas.
\vs p061 3:15 \pc De este modo, un período crucial e interesante de la historia del mundo llegaba a su fin. Esta era del elefante y del caballo se conoce como el \bibemph{Mioceno}.
\usection{4. ÚLTIMA ETAPA DE LA ELEVACIÓN CONTINENTAL. LA ÚLTIMA GRAN EMIGRACIÓN DE LOS MAMÍFEROS}
\vs p061 4:1 Este es el período de la elevación preglacial del suelo en América del Norte, Europa y Asia. La topografía del suelo varió de forma considerable. Nacieron cordilleras, las corrientes de agua cambiaron su curso y por todo el mundo aparecieron volcanes aislados.
\vs p061 4:2 \pc Hace \bibemph{10\,000\,000} de años comenzó una era de amplios depósitos locales de suelo en las tierras bajas de los continentes, pero, más tarde, la mayoría de estas sedimentaciones quedaron eliminadas. En este momento, una gran parte de Europa estaba todavía bajo el agua, incluyendo partes de Inglaterra, Bélgica y Francia, y el Mar Mediterráneo cubría una gran extensión del norte de África. En América del Norte, se formaron importantes cantidades de depósitos en las bases de las montañas, en los lagos y en las grandes cuencas terrestres. Estos depósitos tienen un grosor medio de unos sesenta metros solamente, son más o menos coloreados y los fósiles son escasos. Existían dos grandes lagos de agua dulce en el oeste de América del Norte. Las Sierras se estaban elevando; los Montes Shasta, Hood y Rainier empezaban a convertirse en montañas. Pero no fue hasta la edad del hielo, que seguiría después, cuando América del Norte empezó su lento desplazamiento hacia la depresión atlántica.
\vs p061 4:3 Durante un breve periodo de tiempo, todo el suelo del mundo estaba unido de nuevo, salvo Australia, y se produjo la última gran emigración mundial de animales. América del Norte se comunicaba tanto con América del Sur como con Asia, y se dio un libre intercambio de animales. Los perezosos, los armadillos, los antílopes y los osos asiáticos entraron en América del Norte, mientras que los camellos norteamericanos se fueron a China. Los rinocerontes emigraron por el mundo entero, a excepción de Australia y América del Sur, pero, al finalizar este periodo, se extinguieron en el hemisferio occidental.
\vs p061 4:4 En general, la vida del período anterior continuó evolucionando y propagándose. La familia felina dominaba la vida animal, y la vida marina estaba casi estancada. Muchos de los caballos tenían aún tres dedos, pero iban llegando los tipos modernos; las llamas y los camellos parecidos a las jirafas pastaban en las llanuras mezclados con los caballos. La jirafa apareció en África, con un cuello tan largo entonces como el que tiene hoy en día. En América del Sur evolucionaron los perezosos, los armadillos, los osos hormigueros y el tipo sudamericano de mono primitivo. Antes de que los continentes quedaran definitivamente aislados, esos animales de gran envergadura, los mastodontes, emigraron a todas partes, salvo a Australia.
\vs p061 4:5 \pc Hace \bibemph{5\,000\,000} de años, evolucionó el caballo adquiriendo el estado que tiene ahora y, desde América del Norte, emigró a todo el mundo. Pero el caballo ya se había extinguido en su continente originario mucho antes de que llegara el hombre rojo.
\vs p061 4:6 De forma paulatina, el clima se fue enfriando; las plantas terrestres se iban desplazando con lentitud hacia el Sur. En un principio, fue el creciente frío del norte el que detuvo las emigraciones animales por los istmos septentrionales; posteriormente, estos puentes terrestres norteamericanos se hundieron. Poco después, se acabó por sumergir la vía de comunicación terrestre entre África y América del Sur, y el hemisferio occidental quedó aislado de forma similar a la de hoy. Desde ese momento, empezaron a desarrollarse tipos diferentes de vida en los hemisferios oriental y occidental.
\vs p061 4:7 \pc Y, de este modo, llega a su fin este período de casi diez millones de años, y aún no ha aparecido el ancestro del hombre. Se suele designar a este tiempo como el \bibemph{Plioceno}.
\usection{5. LA TEMPRANA EDAD DEL HIELO}
\vs p061 5:1 Hacia el término del período anterior, el suelo de la parte nordeste de América del Norte y de Europa septentrional se elevó bastante y a gran escala; en América del Norte se alzaron inmensas áreas hasta una altura de 9000 metros y más. Antes había predominado un clima templado en estas regiones septentrionales, y las aguas árticas estaban al completo expuestas a la evaporación, y continuaron estando libres de hielo casi hasta el final del período glacial.
\vs p061 5:2 Al mismo tiempo que se produjeron estas elevaciones del suelo, las corrientes oceánicas se desviaron y los vientos estacionales cambiaron de dirección. Estas condiciones acabaron por producir una casi constante precipitación de humedad debido al movimiento de la atmósfera, fuertemente saturada, sobre las altiplanicies septentrionales. La nieve empezó a caer sobre estas regiones elevadas y, por consiguiente, frías, y continuó cayendo hasta alcanzar una profundidad de 6000 metros. Las áreas de mayor profundidad de la nieve, a la vez que de altura, determinaron los puntos centrales de los siguientes flujos de presión glacial. Y la edad del hielo perduró hasta que esta excesiva precipitación continuó cubriendo las altiplanicies septentrionales con este enorme manto de nieve, que pronto se transformó en hielo sólido, pero movedizo.
\vs p061 5:3 Las grandes capas de hielo de este período no estaban en las regiones montañosas como lo están hoy en día, sino en las altiplanicies. La mitad del hielo glacial se hallaba en América del Norte; una cuarta parte de este, en Eurasia; y la otra cuarta parte, en otros lugares, fundamentalmente en la Antártida. El hielo afectó poco a África, pero Australia sí estuvo casi totalmente cubierta con el manto de hielo antártico.
\vs p061 5:4 Las regiones septentrionales de este mundo han pasado por seis invasiones de hielo, separadas y diferenciadas, aunque hubo decenas de avances y retrocesos vinculados a la acción de cada una de las capas que se formaron. En América del Norte, el hielo se acumuló en dos y, más tarde, en tres centros; Groenlandia se cubrió de hielo e Islandia quedó sepultada completamente bajo el brote de hielo; en Europa, el hielo, en distintas ocasiones, cubrió las Islas Británicas, exceptuando la costa meridional de Inglaterra, y se expandió por Europa occidental llegando hasta Francia.
\vs p061 5:5 \pc Hace \bibemph{2\,000\,000} de años, el primer glaciar norteamericano empezó su avance en dirección sur. Comenzaba la edad del hielo, y este glaciar tardó cerca de un millón de años en avanzar desde los centros de presión septentrionales y retroceder hacia ellos. La capa central de hielo se extendía hacia el sur, hasta llegar a Kansas; los centros de hielo orientales y occidentales no eran tan extensos en ese momento.
\vs p061 5:6 \pc Hace \bibemph{1\,500\,000} años, el primer gran glaciar estaba retrocediendo hacia el norte. Entretanto, enormes cantidades de nieve habían caído sobre Groenlandia y la parte nordeste de América del Norte y, trascurrido poco tiempo, esta masa oriental de hielo empezó a deslizarse hacia el sur. Esta fue la segunda invasión del hielo.
\vs p061 5:7 En Eurasia, estas dos primeras invasiones de hielo no fueron generalizadas. Durante estas tempranas épocas de la edad del hielo, América del Norte se llenó de mastodontes, mamuts lanudos, caballos, camellos, ciervos, bueyes almizcleros, bisontes, perezosos terrestres, castores gigantes, tigres de dientes de sable, perezosos tan grandes como elefantes y muchos grupos de las familias felina y canina. Si bien, a partir de este momento, se fueron reduciendo rápidamente debido al incremento del frío del período glacial. Hacia el término de la edad del hielo, la mayoría de estas especies animales ya se habían extinguido en América del Norte.
\vs p061 5:8 Alejada del hielo, la vida terrestre y acuática del mundo cambió poco. Entre las invasiones glaciales, el clima era casi tan templado como lo es en el presente, quizás algo más cálido. Al fin y al cabo, los glaciales eran fenómenos locales, aunque se expandieron hasta cubrir inmensas áreas de suelo. El clima costero varió de forma considerable entre los tiempos de inactividad glacial y aquellos otros en los que los enormes icebergs se deslizaban desde la costa de Maine hasta el Atlántico, resbalaban por Puget Sound hasta llegar al Pacífico y retumbaban en los fiordos noruegos hasta alcanzar el Mar del Norte.
\usection{6. EL HOMBRE PRIMITIVO EN LA EDAD DEL HIELO}
\vs p061 6:1 El gran acontecimiento de este período glacial fue la evolución del hombre primitivo. Ligeramente hacia el oeste de la India, sobre suelo terrestre, ahora bajo el agua, y entre los vástagos de los emigrantes asiáticos de los tipos norteamericanos de lémures, aparecieron \bibemph{de repente} los mamíferos primigenios. Estos pequeños animales caminaban mayormente sobre sus patas traseras, y poseían un cerebro de grandes proporciones en relación a su tamaño y en comparación con el cerebro de otros animales. En la septuagésima generación de este orden de vida, un nuevo grupo de animales superiores se diferenció de repente de los demás. Estos mamíferos intermedios, nuevos, ---casi doblaban el tamaño de sus ancestros y tenían una capacidad cerebral que había aumentado de forma proporcional--- acababan apenas de consolidarse cuando \bibemph{de repente} aparecieron los primates, la tercera mutación vital. (Al mismo tiempo, un desarrollo retrógrado en el linaje de los mamíferos intermedios dio origen a los ascendentes simios; y, desde ese día hasta hoy, la rama humana ha avanzado mediante evolución progresiva, mientras que las tribus simias se han estacionado o incluso han retrocedido.)
\vs p061 6:2 \pc Hace \bibemph{1\,000\,000} de años, Urantia quedó registrada como \bibemph{mundo habitado}. Una mutación en el linaje de los primates en evolución dio origen \bibemph{de repente} a dos seres humanos primitivos: los verdaderos ancestros de la humanidad.
\vs p061 6:3 Este acontecimiento coincidió aproximadamente con el comienzo del tercer avance glacial; por consiguiente, puede observarse que vuestros primeros ancestros nacieron y se criaron en un entorno difícil al igual que vigorizante y estimulador. Y los únicos supervivientes de estos aborígenes de Urantia, los esquimales, prefieren, incluso en la actualidad, habitar en los gélidos climas septentrionales.
\vs p061 6:4 \pc Hasta cerca de final de la edad del hielo, los seres humanos no estaban presentes en el hemisferio occidental. Si bien, durante las épocas interglaciares, rodeando el Mediterráneo, se habían dirigido hacia el oeste, y pronto dominaron el continente europeo. En las grutas de Europa occidental, se pueden encontrar huesos humanos mezclados con los restos de animales árticos y tropicales, testimoniando que el hombre vivió en estas regiones durante las épocas tardías de unos glaciales que avanzaban y retrocedían.
\usection{7. CONTINUACIÓN DE LA ERA GLACIAL}
\vs p061 7:1 Durante todo el período glacial, otros hechos acontecían, pero la acción del hielo en las latitudes septentrionales eclipsa todos los demás fenómenos. Ningún otro acontecimiento terrestre deja unas muestras tan peculiares en la topografía. Los característicos cantos rodados y las grietas de la superficie, tales como marmitas de gigante, lagos, piedras desplazadas y harina de roca, no existen en relación con ningún otro fenómeno de la naturaleza. El hielo es también el responsable de esas elevaciones suaves, u ondulaciones de la superficie, conocidas como “colinas redondeadas”. Y un glacial, en su avance, desplaza ríos y modifica toda la faz de la Tierra. Únicamente los glaciales dejan tras de sí esos reveladores derrubios: morrenas de fondo, laterales y terminales. Estos derrubios, en particular las morrenas de fondo, se extienden desde el litoral oriental en dirección norte y oeste en América del Norte y se encuentran en Europa y Siberia.
\vs p061 7:2 \pc Hace \bibemph{750\,000} años, la cuarta capa de hielo, unión de los campos de hielo central y oriental de América del Norte, estaba claramente en camino al sur; en su momento álgido, alcanzó el sur de Illinois, desplazando el río Misisipi más de 80 kilómetros hacia el oeste, y, en el este, se extendió hasta el sur, llegando al río Ohio y a la región central de Pensilvania.
\vs p061 7:3 En Asia, la capa de hielo siberiana se expandió hasta llegar a su punto más meridional, mientras que, en Europa, el hielo se detuvo muy próximo a la barrera montañosa de los Alpes.
\vs p061 7:4 \pc Hace \bibemph{500\,000} años, durante el quinto avance del hielo, un nuevo desarrollo aceleró el curso de la evolución humana. \bibemph{De repente} y en una sola generación, las seis razas de color aparecieron por mutación del linaje humano aborigen. Esta fecha es doblemente significativa porque también señala la llegada del príncipe planetario.
\vs p061 7:5 En América del Norte, el invasivo quinto glacial estaba formado por un conjunto de tres centros de hielo. El lóbulo oriental, sin embargo, se expandió solamente hasta corta distancia por debajo del valle de San Lorenzo, y la capa de hielo occidental avanzó poco en su camino hacia el sur. Pero el lóbulo central alcanzó el sur y cubrió la mayor parte del estado de Iowa. En Europa, esta invasión de hielo no fue tan extensa como la anterior.
\vs p061 7:6 \pc Hace \bibemph{250\,000} años comenzó la sexta y última glaciación. Y pese al hecho de que las altiplanicies del norte habían empezado a hundirse ligeramente, este fue el periodo en el que el mayor depósito de nieve se acumuló sobre los campos de hielo septentrionales.
\vs p061 7:7 En esta invasión del hielo, las tres grandes capas confluyeron en una inmensa masa de hielo que alcanzó a todas las montañas del oeste. Aquella fue la mayor de todas las invasiones de hielo ocurridas en América del Norte; el hielo se desplazó hacia el sur hasta unos dos mil quinientos kilómetros de sus centros de presión; en América del Norte se experimentaron las más bajas temperaturas.
\vs p061 7:8 \pc Hace \bibemph{200\,000} años, durante el avance del último glacial, ocurrió un suceso que tuvo mucho que ver con el rumbo de acontecimientos en Urantia: La rebelión de Lucifer.
\vs p061 7:9 \pc Hace \bibemph{150\,000} años, el sexto y último glacial alcanzó los puntos más extremos en su expansión meridional: la capa de hielo occidental cruzó escasamente la frontera con Canadá; la central bajó hasta Kansas, Missouri e Illinois; y la capa oriental avanzó hacia el sur, cubriendo la mayor parte de Pensilvania y Ohio.
\vs p061 7:10 Este es el glacial que lanzó las múltiples lenguas, o lóbulos de hielo, que tallaron los lagos grandes y pequeños de los tiempos actuales. Durante su retroceso se formó el sistema norteamericano de los Grandes Lagos. Los geólogos urantianos han averiguado con gran precisión las distintas etapas de este desarrollo y han supuesto correctamente que, en distintos momentos, estas extensiones de agua desembocaron primeramente en el valle del Misisipi, luego hacia el este, en el valle del Hudson y, finalmente, siguiendo una ruta septentrional, en el rio San Lorenzo. Hace treinta y siete mil años que el interconectado sistema de los Grandes Lagos empezó a vaciar sus aguas en la vía actual del Niágara.
\vs p061 7:11 \pc Hace \bibemph{100\,000} años, durante el retroceso del último glacial, empezaron a formarse las inmensas capas de hielo polares, y el centro del cúmulo de hielo se desplazó de manera considerable hacia el norte. Y siempre que las regiones polares continúen cubiertas de hielo, es prácticamente imposible que haya otra edad glacial, con independencia de que en el futuro se den elevaciones del suelo o modificaciones de las corrientes oceánicas.
\vs p061 7:12 Este último glacial continuó avanzando durante cien mil años, y precisó de igual lapso de tiempo para replegarse completamente hacia el norte. Las regiones templadas llevan algo más de cincuenta mil años libres de hielo.
\vs p061 7:13 El severo período glacial erradicó numerosas especies y modificó muchas otras de forma radical. Muchas especies fueron seriamente cribadas, algo inevitable por el vaivén migratorio que el avance y el retroceso del hielo provocó. Los animales terrestres que siguieron a los glaciales de un lado para otro fueron el oso, el bisonte, el reno, el buey almizclero, el mamut y el mastodonte.
\vs p061 7:14 El mamut buscó las praderas abiertas, pero el mastodonte prefirió las márgenes protegidas de las regiones forestales. El mamut, hasta más tarde, deambuló desde México a Canadá; la variedad siberiana se cubrió de lana. El mastodonte perseveró en América del Norte hasta ser exterminado por el hombre rojo, tal como, más tarde, el hombre blanco haría lo mismo con el bisonte.
\vs p061 7:15 En América del Norte, durante la última glaciación, el caballo, el tapir, la llama y el tigre de dientes de sable se extinguieron. Sus lugares los ocuparon los perezosos, los armadillos y los cerdos de agua, que llegaron desde América del Sur.
\vs p061 7:16 La emigración forzosa de la vida ante el avance del hielo resultó en una mezcla extraordinaria de plantas y animales, y, con el retroceso de la última ola de hielo, numerosas especies árticas tanto vegetales como animales quedaron atrapadas en lo alto de algunos picos montañosos, lugares adonde habían emigrado para escapar de la extinción por el glaciar. Y por ello, hoy día, se pueden encontrar estas plantas y animales, que tuvieron que desplazarse, en lo alto de los Alpes de Europa e incluso en los Montes Apalaches de América del Norte.
\vs p061 7:17 \pc La edad del hielo es el último período geológico completo, el llamado \bibemph{Pleistoceno,} que tuvo una duración de más de dos millones de años.
\vs p061 7:18 \pc Hace \bibemph{35\,000} años que concluyó la gran edad del hielo, excepto en las regiones polares del planeta. Esta fecha es también relevante ya que se aproxima la llegada del hijo y la hija materiales y el comienzo de la dispensación de Adán, correspondiente, a grandes rasgos, con el principio del período \bibemph{Holoceno} o posglacial.
\vs p061 7:19 \pc Esta narración, que se extiende desde la aparición de los mamíferos hasta el retroceso del hielo y los tiempos históricos, engloba un periodo de tiempo de casi cincuenta millones de años. Este es el último período geológico ---el actual--- y vuestros investigadores lo conocen como el \bibemph{Cenozoico} o la era de los tiempos recientes.
\vsetoff
\vs p061 7:20 [Auspiciado por un portador de vida residente.]
