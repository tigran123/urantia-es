\upaper{189}{La resurrección}
\author{Comisión de seres intermedios}
\vs p189 0:1 EL viernes por la tarde, poco después del entierro de Jesús, el jefe de los arcángeles de Nebadón, presente en aquel momento en Urantia, reunió a su consejo de la resurrección de las criaturas volitivas durmientes y empezaron a concebir una posible forma de reconstituir a Jesús. Estos hijos del universo local, estas criaturas de Miguel, allí congregadas, hicieron aquello bajo su propia iniciativa; Gabriel no los había convocado. A medianoche, habían llegado a la conclusión de que las criaturas no podían hacer nada para facilitar la resurrección del creador. Estaban en disposición de aceptar el consejo de Gabriel, que les había instruido que, puesto que Miguel “dio su vida por propia y libre voluntad, también tiene el poder de volver a tomarla conforme a su propia decisión”. Poco después de que se levantara esta sesión del consejo de arcángeles, portadores de vida y de varios de sus acompañantes en la labor de rehabilitar las criaturas y de creación de orden morontial, el modelador personificado de Jesús, que comandaba personalmente a las multitudes celestiales reunidas en ese momento en Urantia, dijo estas palabras a los ansiosos y expectantes observadores:
\vs p189 0:2 “Ninguno de vosotros puede hacer nada para ayudar a vuestro padre\hyp{}creador a volver a la vida. Como mortal del mundo, él ha experimentado la muerte humana; como soberano de todo un universo, él vive. Lo que observáis es el tránsito como mortal de Jesús de Nazaret desde su vida en la carne a la vida en la morontia. El tránsito espiritual de este Jesús se completó en el momento en que me separé de su persona y me convertí temporalmente en vuestro director. Vuestro padre\hyp{}creador ha optado por pasar por la experiencia completa por la que pasan sus criaturas mortales, desde su nacimiento en los mundos materiales, pasando por la muerte natural y la resurrección morontial, hasta llegar al estatus de una verdadera existencia espiritual. Estáis a punto de contemplar cierta faceta de esta experiencia, pero no podéis participar en ella. Lo que ordinariamente hacéis por las criaturas, no podéis hacerlo por el creador. Un hijo creador tiene en sí mismo el poder de darse de gracia en semejanza de cualquiera de sus hijos creados; tiene en sí mismo el poder de dar su vida material y de volver a tomarla; y él tiene este poder por la autoridad directamente recibida del Padre del Paraíso, y sé de lo que hablo”.
\vs p189 0:3 Cuando oyeron hablar estas cosas al modelador personificado, todos ellos, desde Gabriel hasta el querubín más humilde, asumieron una actitud de tensa espera. Veían el cuerpo mortal de Jesús en el sepulcro; detectaban evidencias de la actividad de su amado soberano en el universo; y, como no entendían dichos fenómenos, aguardaron pacientemente el desarrollo de los acontecimientos.
\usection{1. EL TRÁNSITO MORONTIAL}
\vs p189 1:1 El domingo, a las dos cuarenta y cinco de la mañana, la comisión del Paraíso de la encarnación, constituida por siete seres personales, llegó al lugar y, de inmediato, se desplegaron en torno al sepulcro. A las tres menos diez, este sepulcro nuevo, propiedad de José, empezó a emitir intensas vibraciones a causa de la actividad material y morontial que se estaba realizando, y, a las tres y dos minutos de este domingo por la mañana, 9 de abril del año 30 d. C., la forma y el ser personal morontial resucitados de Jesús de Nazaret salieron de la tumba.
\vs p189 1:2 Cuando Jesús resucitado emergió de su tumba, el cuerpo carnal en el que había vivido y obrado en la tierra durante casi treinta y seis años aún yacía allí, en el nicho del sepulcro, inalterado y envuelto en la sábana de lino, tal como lo habían colocado José y sus acompañantes el viernes por la tarde. Tampoco, en modo alguno, estaba la piedra de la entrada desplazada; el sello de Pilato seguía intacto; los soldados aún estaban de guardia. Los guardias del templo habían continuado de servicio; durante la medianoche, se había reemplazado a la guardia romana. Ninguno de estos guardianes sospechó que quien era objeto de su vigilia había resucitado a una forma nueva y más elevada de existencia, y que el cuerpo que ellos estaban vigilando ya no era sino una envoltura de la que se había prescindido, y que ya no guardaba ninguna relación con la persona morontial transferida y resucitada de Jesús.
\vs p189 1:3 \pc La humanidad es lenta en percibir que, en todo lo que es personal, la materia es el esqueleto de la morontia, y que ambos constituyen la sombra del reflejo de la perdurable realidad espiritual. ¿Cuánto tiempo ha de pasar hasta que contempléis el tiempo como la imagen móvil de la eternidad y el espacio como la sombra fugaz de las realidades del Paraíso?
\vs p189 1:4 A nuestro juicio, ninguna criatura de este universo ni ningún ser personal de otros universos tuvieron nada que ver con esta resurrección morontial de Jesús de Nazaret. El viernes, dio su vida como un mortal del mundo; el domingo por la mañana, la volvió a tomar como un ser morontial del sistema de Satania en Norlatiadek. Hay muchas cosas acerca de la resurrección de Jesús que no entendemos. No obstante, sabemos que ocurrió tal como hemos contado y sobre la hora indicada. También podemos constatar que todos los fenómenos conocidos, vinculados con este tránsito como mortal o resurrección morontial, tuvieron lugar allí mismo, en la nueva tumba de José, donde yacían los restos mortales de Jesús envueltos en paños mortuorios.
\vs p189 1:5 Sabemos que ninguna criatura del universo local participó en este despertar morontial. Pudimos percibir a los sietes seres personales del Paraíso en torno a la tumba, pero no los vimos hacer nada respecto al despertar del Maestro. En el momento en el que Jesús apareció al lado de Gabriel, justo por encima del sepulcro, estos seres personales mostraron su intención de partir de inmediato hacia Uversa.
\vs p189 1:6 Dejemos claro para siempre el concepto de la resurrección de Jesús con las siguientes afirmaciones:
\vs p189 1:7 \li{1.}Su cuerpo material o físico no formaba parte del ser personal resucitado. Cuando Jesús salió de la tumba, su cuerpo carnal permaneció inalterado en el sepulcro. Emergió de la sepultura, sin mover las piedras que cerraban la entrada y sin romper los sellos de Pilato.
\vs p189 1:8 \li{2.}No emergió de la tumba como un espíritu ni como Miguel de Nebadón; no apareció con la forma del Soberano Creador, que había tenido antes de encarnarse semejando las criaturas mortales de Urantia.
\vs p189 1:9 \li{3.}Ciertamente salió de esta tumba de José con la misma apariencia de los seres personales morontiales de aquellos que emergen, como seres ascendentes morontiales resucitados, de las salas de resurrección del primer mundo de estancia de este sistema local de Satania. Y la presencia del monumento en conmemoración de Miguel, en el centro del inmenso patio de las salas de resurrección del primer mundo de estancia, nos lleva a suponer que la resurrección del Maestro en Urantia se impulsó en cierto modo en este mundo, el primero de los mundos de estancia del sistema.
\vs p189 1:10 \pc La primera acción de Jesús, al resucitar de la tumba, fue saludar a Gabriel y darle instrucciones para que siguiera en su cargo de mandatario de los asuntos del universo bajo Emanuel; y entonces pidió al jefe de los melquisedecs que transmitiera a Emanuel sus saludos fraternos. Acto seguido, solicitó del altísimo de Edentia la certificación de los ancianos de días relativa a la culminación de su tránsito como mortal; y dirigiéndose a los grupos morontiales de los siete mundos de estancia, allí reunidos para saludar y dar la bienvenida a su creador, que semejaba una criatura de su propio orden, Jesús pronunció las primeras palabras de su andadura posmortal. El Jesús morontial dijo: “Habiendo terminado mi vida en la carne, quiero quedarme aquí durante un corto tiempo mientras estoy en esta forma morontial de transición para conocer más profundamente la vida de mis criaturas ascendentes y continuar revelando la voluntad de mi Padre del Paraíso”.
\vs p189 1:11 Tras haber dicho estas cosas, Jesús hizo un gesto al modelador personificado, y todas las inteligencias de este universo, que estaban congregadas en Urantia para presenciar la resurrección, se retiraron de inmediato a ocupar sus respectivos puestos en el universo.
\vs p189 1:12 Luego, comenzó su toma de contacto con el nivel morontial, iniciándosele como criatura a las exigencias de la vida que había escogido vivir por breve tiempo en Urantia. Esta iniciación al mundo morontial tomó más de una hora de tiempo de la tierra y quedó interrumpida dos veces por su deseo de comunicarse con sus antiguas acompañantes en la carne cuando salieron de Jerusalén y se asomaron maravilladas a la tumba vacía, descubriendo lo que a su juicio era un indicio de su resurrección.
\vs p189 1:13 El tránsito como mortal de Jesús ---la resurrección morontial del Hijo del Hombre--- ha llegado ya a su fin. Ha comenzado la experiencia transitoria del Maestro como ser personal a medio camino entre lo material y lo espiritual. Y lo ha hecho todo por medio de un poder inherente en él mismo; ningún ser personal le ha prestado su asistencia. Ahora vive como Jesús de morontia y, conforme comienza esta vida morontial, su cuerpo material carnal yace inalterado en la tumba. Los soldados siguen de guardia, y aún no está roto el sello del gobernador que se había colocado en torno a las piedras.
\usection{2. EL CUERPO MATERIAL DE JESÚS}
\vs p189 2:1 A las tres y diez, mientras que el Jesús resucitado confraternizaba con los seres personales morontiales, allí congregados, de los siete mundos de estancia de Satania, el jefe de los arcángeles ---los ángeles de la resurrección--- se acercó a Gabriel y le solicitaron el cuerpo mortal de Jesús. El jefe de los arcángeles dijo: “No nos es posible participar en la resurrección morontial de Miguel, nuestro soberano, tras su experiencia de darse de gracia; pero quisiéramos tener sus restos mortales en nuestra custodia para su inminente disolución. No nos proponemos usar nuestro método de desmaterialización; queremos simplemente aplicar el proceso de aceleración del tiempo. Resulta suficiente con que hayamos visto al Soberano vivir y morir en Urantia; las multitudes de los cielos desearían evitar tener en la memoria permanentemente la visión de la forma humana del creador y sostenedor de un universo descomponiéndose lentamente. En nombre de las inteligencias celestiales de todo Nebadón, expídenos un mandato por el que se me entregue pues el cuerpo de Jesús de Nazaret y se nos autorice a llevar a cabo su disolución inmediata”.
\vs p189 2:2 Y una vez que Gabriel habló de este asunto con el altísimo de mayor rango de Edentia, se le concedió al arcángel portavoz de las multitudes celestiales el permiso para disponer de los restos físicos de Jesús de la forma que creyera necesaria.
\vs p189 2:3 Después de conseguir este permiso, el jefe de los arcángeles convocó a muchos de sus propios compañeros, a la vez que a un gran grupo de representantes de todos los órdenes de seres personales celestiales para que le prestaran ayuda, y, entonces, con la cooperación de los seres intermedios de Urantia, logró hacerse con el cuerpo físico de Jesús. Este cuerpo de muerte era una creación puramente material; era físico y tangible; no se podía retirar de la tumba sellada de la misma manera que la forma morontial resucitada había podido escapar. Con la asistencia de determinados seres personales auxiliares morontiales, se puede lograr que la forma morontial sea, en algún momento, como la espiritual, sin verse obstaculizada por la materia ordinaria, mientras que, en otro momento, puede llegar a ser perceptible y contactable por seres materiales como los mortales del mundo.
\vs p189 2:4 En cuanto se disponían a sacar el cuerpo de Jesús del sepulcro en preparación para su disolución casi instantánea, confiriéndole pues dignidad y reverencia, se asignó a los seres intermedios secundarios de Urantia que apartaran rodando las piedras que tapaban la entrada a la tumba. La más grande de estas dos piedras era de gran tamaño y redonda, muy parecida a una rueda de molino, y se desplazaba por una ranura cincelada en la roca, de modo que se podía hacer rodar hacia adelante o hacia atrás para abrir o cerrar la tumba. Cuando los guardias judíos y los soldados romanos que estaban de guardia, vieron, en la tenue luz de la madrugada, cómo esta enorme piedra comenzaba a rodar, apartándose de la entrada de la tumba, aparentemente por sí misma ---sin ningún medio visible que justificara tal movimiento--- se vieron atenazados por el miedo y el pánico, y huyeron del lugar a toda prisa. Los judíos se escaparon a sus casas, regresando más tarde al templo para contar estos hechos a su capitán. Los romanos fueron al fuerte de Antonia e informaron al centurión de lo que habían visto en cuanto entró de turno.
\vs p189 2:5 Ofreciéndoles sobornos a Judas, los líderes judíos habían comenzado con su sórdido intento de querer, deshacerse supuestamente de Jesús, y, ahora, cuando se enfrentaban a aquella situación embarazosa, en lugar de pensar en castigar a los guardias por haber desertado de sus puestos, recurrieron al soborno de estos guardias y de los soldados romanos. Pagaron a cada uno de estos veinte hombres una suma de dinero y les dieron instrucciones para que dijeran a todos: “Por la noche, mientras dormíamos, los discípulos cayeron sobre nosotros y se llevaron el cuerpo”. Y los líderes judíos prometieron encarecidamente a los soldados que los defenderían ante Pilato, llegado el caso de que el gobernador conociera que habían aceptado un soborno.
\vs p189 2:6 \pc La creencia cristiana en la resurrección de Jesús se basa en el hecho de la “tumba vacía”. En verdad, fue un \bibemph{hecho} que la tumba estuviera vacía, pero esa no fue la \bibemph{verdad} de la resurrección. La tumba estaba realmente vacía cuando llegaron los primeros creyentes, y esto, sumado a la indudable resurrección del Maestro, llevó a elaborar una creencia que no era cierta: la doctrina de que el cuerpo material y mortal de Jesús había resucitado de su tumba. No siempre se puede construir una verdad que guarde relación con las realidades espirituales y los valores eternos sobre una combinación de hechos aparentes. Aunque haya hechos aislados que puedan resultar materialmente ciertos, esto no implica que la asociación de un conjunto de circunstancias deba necesariamente llevar a conclusiones espirituales veraces.
\vs p189 2:7 La tumba de José estaba vacía, no porque el cuerpo de Jesús hubiera vuelto a la vida o resucitado, sino porque las multitudes celestiales tenían permiso para proporcionarle una disolución de índole especial y singular, un regreso del “polvo al polvo”, sin la intervención del pausado recorrer del tiempo y sin la actuación de los procesos ordinarios y visibles que ocasionan la descomposición humana y la corrupción de la materia.
\vs p189 2:8 Los restos mortales de Jesús soportaron el mismo proceso natural de desintegración material, característico de todos los cuerpos humanos de la tierra, excepto que, en un momento dado, este modo natural de disolución se aceleró extremadamente, intensificándose hasta el punto de volverse casi instantáneo.
\vs p189 2:9 Las verdaderas evidencias de la resurrección de Miguel son de naturaleza espiritual, aunque dicha creencia se haya corroborado por el testimonio de muchos mortales del mundo, que se encontraron con el Maestro morontial resucitado, lo reconocieron y se comunicaron con él. Él llegó a formar parte de la experiencia personal de casi mil seres humanos antes de dejar finalmente Urantia.
\usection{3. LA RESURRECCIÓN DISPENSACIONAL}
\vs p189 3:1 Poco después de las cuatro y media de la madrugada de ese domingo, Gabriel llamó a su lado a los arcángeles y se preparó para inaugurar la resurrección general consiguiente al fin de la dispensación adánica en Urantia. Cuando las inmensas multitudes de serafines y de querubines implicados en este gran acontecimiento se habían posicionado en la formación adecuada, apareció, ante Gabriel, el Miguel morontial diciendo: “Al igual que mi Padre tiene vida en sí mismo, él ha dado también al Hijo tener vida en sí mismo. Aunque aún no he reanudado por completo el ejercicio de mi jurisdicción sobre el universo, esta limitación autoimpuesta no restringe de ninguna manera que dé la vida a mis hijos durmientes; que comience el llamamiento nominal a la resurrección planetaria”.
\vs p189 3:2 La vía circulatoria de los arcángeles operó, entonces, por primera vez desde Urantia. Gabriel y las multitudes de arcángeles se trasladaron al lugar de la polaridad espiritual del planeta; y cuando Gabriel dio la señal, su voz se transmitió al momento al primero de los mundos de estancia del sistema diciendo: “Por mandato de Miguel, ¡que resuciten los muertos de una dispensación de Urantia!”. Entonces, todos los supervivientes de las razas humanas de Urantia que habían estado dormidos desde los días de Adán, y que aún no habían sido juzgados, aparecieron en las salas de resurrección de los mundos de estancia, preparados para recibir su vestidura morontial. En un instante de tiempo, los serafines y sus acompañantes se dispusieron a partir hacia los mundos de estancia. Ordinariamente, estos guardianes seráficos, anteriormente asignados a la custodia grupal de estos mortales supervivientes, habrían estado presentes cuando despertaban en las salas de resurrección de los mundos de estancia, pero, en tal momento, se encontraban en este mismo mundo, ya que la presencia de Gabriel era necesaria aquí con motivo de la resurrección morontial de Jesús.
\vs p189 3:3 Pese a que incontables personas que tenían guardianes seráficos personales y otras que habían alcanzado el exigido progreso espiritual habían ido a los mundos de estancia en eras posteriores a los tiempos de Adán y Eva, y aunque había habido muchas resurrecciones especiales y milenarias de los hijos de Urantia, se trataba del tercero de los llamamientos nominales planetarios, o resurrecciones dispensacionales completas. El primero había tenido lugar en el momento de la llegada del príncipe planetario; el segundo ocurrió durante los tiempos de Adán; y este, el tercero, señalizaba la resurrección morontial, el tránsito como mortal, de Jesús de Nazaret.
\vs p189 3:4 \pc Cuando el jefe de los arcángeles recibió la señal para iniciar la resurrección planetaria, el modelador personificado del Hijo del Hombre renunció a su autoridad sobre las multitudes celestiales congregadas en Urantia, devolviendo a todos estos hijos del universo local a la jurisdicción de sus mandos respectivos. Y, una vez hecho esto, partió hacia Lugar de Salvación para dar constancia ante Emanuel de la finalización del tránsito como mortal de Miguel. Y a él le siguieron de inmediato toda la multitud celestial cuyo servicio no era necesario en Urantia. Si bien, Gabriel permaneció en Urantia con el Jesús morontial.
\vs p189 3:5 \pc Y este es el relato de los acontecimientos que se produjeron en la resurrección de Jesús según los percibieron quienes fueron testigos de lo que verdaderamente ocurrió, libres de las limitaciones a las que está sujeta la visión humana, parcial y restringida.
\usection{4. EL DESCUBRIMIENTO DE LA TUMBA VACÍA}
\vs p189 4:1 Al aproximarnos al momento de la resurrección de Jesús, en aquella madrugada del domingo, cabe recordar que los diez apóstoles se alojaban en la casa de Elías y María Marcos, en cuyo aposento alto estaban durmiendo, justo sobre los mismos divanes en los que se habían reclinado durante la última cena con su Maestro. En esas horas tempranas del domingo, estaban todos congregados allí salvo Tomás. Tomás había estado con ellos durante unos minutos, ya avanzada la noche del sábado, cuando primeramente se reunieron, pero la visión de los apóstoles, sumada al pensamiento de lo que le había sucedido a Jesús, resultó ser demasiado para él. Dio una mirada a sus compañeros y abandonó de inmediato la habitación, dirigiéndose a la casa de Simón en Betfagé, donde pensaba lamentarse de sus aflicciones en soledad. Todos los apóstoles sufrían, no tanto por las dudas y la desesperación, sino por el miedo, el pesar y la vergüenza.
\vs p189 4:2 \pc En la casa de Nicodemo se encontraban reunidos, junto a David Zebedeo y José de Arimatea, unos doce o quince de los discípulos de Jesús, los más prominentes de Jerusalén. En la de José de Arimatea, había unas quince o veinte de las principales mujeres creyentes. Solo estas mujeres se quedaban en la casa de José y, como habían estado aisladas durante las horas del \bibemph{sabbat} y la noche después, desconocían que una guardia militar estaba de vigilancia en la tumba; tampoco sabían que se había rodado una segunda piedra delante de esta, y que ambas piedras habían sido precintadas con el sello de Pilato.
\vs p189 4:3 Algo antes de las tres de la mañana de este domingo, cuando el día empezaba a mostrarse por el este, cinco de estas mujeres se encaminaron hacia el sepulcro de Jesús. Habían preparado una gran cantidad de lociones aromatizantes especiales, y llevaban con ellas muchas vendas de lino. Tenían la intención de ungir más concienzudamente el cuerpo de Jesús con bálsamos fúnebres y envolverlo con mayor cuidado con los vendajes nuevos.
\vs p189 4:4 Las mujeres que llevaron a cabo esta labor de ungir el cuerpo de Jesús fueron: María Magdalena; María, la madre de los gemelos Alfeo; Salomé, la madre de los hermanos Zebedeo; Juana, la mujer de Chuza; y Susana, la hija de Esdras de Alejandría.
\vs p189 4:5 Eran alrededor de las tres y media cuando las cinco mujeres, llevando sus ungüentos, llegaron a la tumba vacía. Conforme salían por la puerta de Damasco, se encontraron con un cierto número de soldados que huían de alguna manera presa del pánico hacia la ciudad, algo que hizo que se detuvieran por unos minutos; pero como no sucedió nada más, retomaron su camino.
\vs p189 4:6 Se sorprendieron mucho al ver la piedra que tapaba la entrada al sepulcro rodada hacia fuera, ya que al salir se habían dicho entre ellas: “¿Quién nos ayudará a rodar y apartar la piedra?”. Pusieron sus fardos en el suelo y empezaron a mirarse unas a otras temerosas y sumamente asombradas. Mientras estaban allí, María Magdalena, temblando de miedo, le dio la vuelta a la piedra más pequeña y se atrevió a entrar en el sepulcro abierto. Esta tumba de José estaba en su jardín, en la ladera de la colina situada al este de la carretera, mirando también al este. En aquella hora, el nuevo día había clareado ya lo suficiente como para permitir que María mirara al fondo, hacia el lugar donde el cuerpo del maestro había yacido, y percibiera que no estaba allí. En la oquedad de piedra donde habían colocado a Jesús, María solo vio el sudario, sobre el que se había apoyado su cabeza, doblado, y los vendajes con los que se le había envuelto, intactos sobre la piedra, tal como antes de que las multitudes celestiales sacaran el cuerpo. La sábana que lo cubría estaba tendida a los pies del nicho fúnebre.
\vs p189 4:7 Tras detenerse en la entrada del sepulcro durante unos momentos (cuando primeramente entró en él no podía ver con claridad el interior), María observó que ya no estaba el cuerpo de Jesús y que en su lugar tan solo quedaban estos paños mortuorios. Dio entonces un grito de alarma y de angustia. Las mujeres estaban extremadamente nerviosas; se sentían en vilo desde que se encontraron con los despavoridos soldados en la puerta de la ciudad y, cuando María lanzó este angustioso grito, todas huyeron a toda prisa sobrecogidas por el terror. Y no pararon de correr hasta que no llegaron a la puerta de Damasco. En ese momento, Juana sintió remordimiento de conciencia por haber dejado abandonada a María, por lo que movilizó a sus compañeras y volvieron al sepulcro.
\vs p189 4:8 Conforme se acercaban al sepulcro, la asustada Magdalena, que se había sentido incluso más aterrada cuando vio, al salir de la tumba, que no estaban allí sus hermanas esperándola, corrió precipitadamente hacia ellas, exclamando agitada: “No está ahí! ¡Se lo han llevado!”. Y las condujo de vuelta a la tumba, y todas entraron y vieron que estaba vacía.
\vs p189 4:9 Las cinco mujeres se sentaron todas, entonces, sobre la piedra cercana a la entrada y conversaron acerca de la situación. Aún no habían caído en la cuenta de que Jesús había resucitado. Habían estado durante el \bibemph{sabbat} a solas, y supusieron que habían trasladado el cuerpo a otro sepulcro. Pero cuando consideraron aquella posibilidad como solución a su dilema, vieron que no podían dar razón del hecho de que los paños mortuorios estuvieran dispuestos de forma ordenada; ¿cómo podían haber sacado el cuerpo, si los mismos vendajes en los que había estado envuelto estaban en el mismo lugar y aparentemente intactos sobre el anaquel de la tumba?
\vs p189 4:10 \pc Estando allí sentadas en las tempranas horas del alba de este nuevo día, miraron a un lado y observaron que había alguien extraño que se mantenía en silencio y quieto. En ese momento, volvieron a asustarse, pero María Magdalena, corriendo hacia él y, suponiendo que se trataba del jardinero, dijo: “¿Dónde os habéis llevado al Maestro? ¿Dónde lo han puesto? Dínoslo para poder ir a buscarlo”. Como el desconocido no le contestó, ella comenzó a llorar. Entonces Jesús les habló, diciendo: “¿A quién buscáis?”. María dijo: “Buscamos a Jesús que fue enterrado en la tumba de José, pero ya no está allí. ¿Sabes tú adónde lo han llevado?”. Entonces Jesús les dijo: “¿Es que no os dijo este Jesús, cuando aún se hallaba en Galilea, que moriría, pero que resucitaría?”. Estas palabras sobresaltaron a las mujeres, pero el Maestro estaba tan cambiado que aún no le reconocieron al estar dándole la espalda a la tenue luz. Y mientras sopesaban sus palabras, él se dirigió a Magdalena con una voz que le resultaba familiar, diciendo: “María”. Cuando ella oyó esa palabra de comprensión, bien conocida para ella, y de cariñoso saludo, supo que era la voz del Maestro, y corrió a arrojarse de rodillas a sus pies mientras exclamaba: “¡Mi Señor, y mi Maestro!”. Y todas las demás mujeres se dieron cuenta de que era el Maestro quien estaba ante ellas con una forma glorificada, y enseguida se arrodillaron ante él.
\vs p189 4:11 A estos ojos humanos les fue posible ver la forma morontial de Jesús gracias al ministerio especial de los transformadores de la energía y de los seres intermedios, en conjunción con algunos de los seres personales morontiales que en aquel momento acompañaban a Jesús.
\vs p189 4:12 \pc Al querer María abrazar sus pies, Jesús dijo: “No me toques, María, porque no soy como me conociste en la carne. En esta forma me quedaré durante un tiempo con vosotros antes de ascender al Padre. Pero, id, todas vosotras ahora, y decid a mis apóstoles ---y a Pedro--- que he resucitado, y que habéis hablado conmigo”.
\vs p189 4:13 Una vez que estas asombradas mujeres se recobraron de su consternación, se apresuraron a volver a la ciudad, a la casa de Elías Marcos, donde contaron a los diez apóstoles todo lo que les había sucedido; pero los apóstoles no estaban muy decididos a creerlas. En un principio, pensaban que las mujeres habían visto una visión, pero cuando María Magdalena repitió las palabras que Jesús les había dicho y, cuando Pedro oyó su nombre, salió a toda prisa del aposento alto, seguido de cerca por Juan, en dirección a la tumba para poder ver estas cosas por sí mismo.
\vs p189 4:14 Las mujeres repitieron a los otros apóstoles la noticia de su conversación con Jesús, pero ellos no las creyeron; ni tampoco quisieron ir a cerciorarse por sí mismos tal como Pedro y Juan habían hecho.
\usection{5. PEDRO Y JUAN EN LA TUMBA}
\vs p189 5:1 Mientras los dos apóstoles corrían al Gólgota y a la tumba de José, los pensamientos de Pedro discurrían entre el temor y la esperanza; temía encontrarse con el Maestro, pero la noticia de que Jesús le había enviado un mensaje particular había suscitado la esperanza en él. Tenía la convicción a medias de que Jesús estaba realmente vivo; recordó la promesa de que resucitaría al tercer día. Aunque resulte extraño narrar esto, a Juan no se le había pasado por la mente esta promesa hasta el momento en el que se dirigía al norte cruzando apresuradamente Jerusalén. A medida que salía deprisa de la ciudad, Juan sintió cómo le brotaba del alma un extraño arrebatamiento de gozo y esperanza. Estaba casi convencido de que las mujeres habían visto verdaderamente al Maestro resucitado.
\vs p189 5:2 Al ser más joven, Juan dejó atrás a Pedro y llegó primero al sepulcro. Juan se quedó en la entrada, observando desde allí, y era tal como María lo había descrito. Poco después Simón Pedro llegó allí apresurado y, al entrar, vio lo mismo ---la tumba vacía con los paños mortuorios tan peculiarmente dispuestos---. Y, cuando Pedro salió, también entró Juan, viéndolo todo por él mismo; luego se sentaron en la piedra para reflexionar sobre el significado de lo que habían visto y oído. Y, mientras estaban allí sentados, empezó a rondarles por la cabeza todas las cosas que les habían dicho acerca de Jesús, pero no podían entender con claridad lo qué había sucedido.
\vs p189 5:3 En un principio, Pedro apuntó que habían saqueado la tumba, que los enemigos habían robado el cuerpo, quizás sobornando a los centinelas. Pero Juan expuso que, si hubieran robado el cuerpo, difícilmente la habrían dejado tan ordenada, y también planteó la cuestión de cómo podía ser que hubieran dejado atrás los vendajes, que aparentemente estaban tan intactos. Y ambos volvieron de nuevo a entrar en el sepulcro para examinar más detenidamente los paños mortuorios. Al salir del sepulcro por segunda vez, encontraron a María Magdalena que había regresado y estaba llorando delante de la entrada. María había ido a buscar a los apóstoles creyendo que Jesús había resucitado, pero cuando todos ellos se habían negado a creer en su palabra, se sintió abatida y desesperada. Anhelaba volver a estar cerca de la tumba, donde le pareció haber oído la voz tan familiar de Jesús.
\vs p189 5:4 Mientras María se quedaba allí, una vez que Pedro y Juan se marcharon, el Maestro se apareció a ella de nuevo, diciendo: “No tengas dudas; ten la valentía de creer en lo que has visto y oído. Vuelve con mis apóstoles y diles otra vez que yo he resucitado, que me apareceré a ellos y que en breve iré delante de ellos a Galilea como les prometí”.
\vs p189 5:5 María se volvió con celeridad a la casa de Marcos y les dijo a los apóstoles que había hablado nuevamente con Jesús, pero ellos no quisieron tomarla en cuenta. Si bien, cuando Pedro y Juan regresaron, dejaron sus mofas y se llenaron de temor y aprensión.
