\upaper{181}{Consejos y advertencias finales}
\author{Comisión de seres intermedios}
\vs p181 0:1 Una vez terminado su discurso de despedida, Jesús conversó informalmente con los once y recordó muchas de las experiencias vividas con ellos tanto en grupo como de manera individual. Por fin, estos galileos empezaban a darse cuenta de que su amigo y maestro iba a dejarlos y, esperanzados, se aferraban a la promesa de que, al cabo de poco tiempo, estaría una vez más con ellos; pero se olvidaban de que este regreso sería igualmente por poco tiempo. Muchos de los apóstoles y los discípulos más destacados pensaban que esta promesa de volver y de estar brevemente con ellos (el corto intervalo entre la resurrección y la ascensión) indicaba que Jesús se iba tan solo para hacer una corta visita a su Padre, tras la que volvería para instaurar el reino. Interpretaban las enseñanzas del Maestro a la vez conforme a sus creencias preconcebidas y a sus más fervientes esperanzas. Puesto que estas creencias, de toda una vida, y las esperanzas de satisfacer sus deseos eran coincidentes no les resultaba muy difícil percibir en las palabras del Maestro una forma de justificar sus profundas aspiraciones.
\vs p181 0:2 Después de haber comentado y asentado en sus mentes el discurso de despedida, Jesús llamó nuevamente la atención de los apóstoles y empezó a impartirles sus consejos y advertencias.
\usection{1. ÚLTIMAS PALABRAS DE CONSUELO}
\vs p181 1:1 Cuando los once ocuparon sus asientos, Jesús se levantó y les habló, diciéndoles: “Mientras yo continúe con vosotros en la carne, no seré sino alguien que esté en medio de vosotros o en el mundo entero. Pero cuando me haya liberado de esta investidura carnal, podré regresar como espíritu para habitar en cada uno de vosotros y de todos los demás creyentes de este evangelio del reino. De este modo, el Hijo del Hombre se encarnará espiritualmente en las almas de todos los fieles creyentes.
\vs p181 1:2 “Cuando haya vuelto para vivir en vosotros y obrar mediante vosotros, podré llevaros mejor por esta vida y guiaros en la vida futura a través de las muchas moradas del cielo de los cielos. La vida en la creación eterna del Padre no es un descanso ocioso sin fin, lleno de facilidades e intereses egoístas, sino más bien un incesante camino de progreso en la gracia, en la verdad y en la gloria. Cada una de estas estaciones de tránsito de la casa de mi Padre es un lugar de parada, una vida destinada a prepararos para la que os espera en adelante. Y, así, los hijos de la luz irán de gloria en gloria hasta alcanzar la condición divina en la que se perfeccionarán espiritualmente, tal como el Padre es perfecto en todas las cosas.
\vs p181 1:3 “Si queréis seguirme cuando os deje, dedicad vuestros más denodados esfuerzos a vivir de acuerdo con el espíritu de mis enseñanzas y el ideal de mi vida ---cumplir la voluntad de mi Padre---. Haced esto en lugar de tratar de imitar la vida que viví de forma natural en la carne tal como a mí se me ha requerido necesariamente que la viva en este mundo.
\vs p181 1:4 “El Padre me envió a este mundo, pero solo unos pocos de vosotros decidisteis recibirme del todo. Yo derramaré mi espíritu sobre toda carne, pero no todos los hombres optarán por acoger a este nuevo maestro como guía y consejero de su alma. Si bien, todos los que lo hagan serán iluminados, limpiados y consolados. Y este espíritu de la verdad se convertirá en ellos en una fuente de agua viva que brote para la vida eterna.
\vs p181 1:5 “Y ahora, estando ya a punto de dejaros, quisiera deciros unas palabras de consuelo. La paz os dejo; mi paz os doy. Yo no doy estos dones como el mundo los da ---con limitaciones--- sino que os daré a cada uno de vosotros todo lo que queráis recibir. No se turbe vuestro corazón ni tenga miedo. Yo he vencido al mundo, y todos triunfaréis en mí por la fe. Os he avisado de que matarán al Hijo del Hombre, pero os aseguro que volveré antes de ir al Padre, aunque solo sea por un breve tiempo. Y una vez que yo haya ascendido al Padre, ciertamente os enviaré al nuevo maestro para que esté con vosotros y viva en vuestros propios corazones. Y cuando veáis que todo esto llegue a pasar, no desmayéis, sino, más bien, creed, puesto que lo sabíais todo de antemano. Os amo con un gran afecto y no quisiera dejaros, pero es la voluntad del Padre. Ha llegado mi hora.
\vs p181 1:6 “No dudéis de ninguna de estas verdades, incluso si estáis dispersos a causa de las persecuciones y afligidos por muchos pesares. Cuando sintáis que estáis solos en el mundo, yo sabré de vuestra soledad, al igual que, cuando estéis vosotros esparcidos cada uno por su lado, dejando al Hijo del Hombre en manos de sus enemigos, vosotros sabréis de la mía. Pero yo nunca estoy solo; el Padre está siempre conmigo. Incluso en esos momentos, oraré por vosotros. Y estas cosas os he hablado para que en mí tengáis paz, y la tengáis más abundantemente. Tendréis aflicción en este mundo, pero estad de buen ánimo; yo he triunfado en el mundo y os he mostrado el camino al gozo eterno y al servicio para siempre”.
\vs p181 1:7 \pc Jesús da la paz a los hacedores con él de la voluntad de Dios, pero no es el tipo de paz que traiga los goces y las satisfacciones de este mundo material. En su incredulidad, el materialista y el fatalista solo pueden aspirar a disfrutar de dos clases de paz y de consuelo para su alma: o bien deben ser estoicos, estar firme y determinadamente resueltos a afrontar lo inevitable y a soportar lo peor; o bien deben ser optimistas, complaciéndose siempre en esa esperanza que emana eternamente del interior del ser humano, en un vano anhelo por una paz que nunca llega realmente.
\vs p181 1:8 Cuando se vive la vida en la tierra, puede ser provechoso disponer de cierta cantidad de estoicismo y de optimismo, pero ni el uno ni el otro guardan relación con esa formidable paz que ofrece el Hijo de Dios a sus hermanos en la carne. La paz que Miguel da a sus hijos en la tierra es esa misma paz que colmaba su propia alma cuando él mismo vivía su vida mortal en la carne y en este mismo mundo. La paz de Jesús constituye el gozo y la satisfacción de alguien que conoce a Dios y que ha logrado aprender a cumplir en plenitud la voluntad de Dios mientras ha vivido la vida mortal en la carne. La paz de mente de Jesús estaba fundamentada en una fe humana, absoluta, de la realidad de los sensatos y comprensivos cuidados del Padre divino. Jesús tuvo dificultades en la tierra, incluso se le ha llamado injustamente “varón de dolores”, pero, en medio de todos estos momentos, disfrutó del consuelo de esa confianza que siempre lo fortaleció para continuar adelante con su objetivo de vida, en la total seguridad de que llevaba a cabo la voluntad del Padre.
\vs p181 1:9 Jesús era resuelto, perseverante y totalmente entregado a la realización de su misión, pero no era un estoico insensible e indiferente. Siempre buscó los aspectos alegres de sus experiencias de vida, pero no era un optimista ciego que se diera a engaños. El Maestro sabía todo lo que le acontecería, y no sentía miedo. Después de ofrecer esta paz a cada uno de sus seguidores, podía verdaderamente decir: “No se turbe vuestro corazón ni tenga miedo”.
\vs p181 1:10 La paz de Jesús es, entonces, la paz y la certidumbre del hijo que cree plenamente que su andadura en el tiempo y en la eternidad está totalmente segura bajo el cuidado y el sostenimiento de un Padre espíritu que es omnisapiente, omniamante y omnipotente. Y es, de hecho, una paz que sobrepasa todo el entendimiento de la mente mortal, pero que un corazón humano que cree puede disfrutar en su máxima plenitud.
\usection{2. CONSEJOS PERSONALES DE DESPEDIDA}
\vs p181 2:1 Una vez que el Maestro, ante su partida, terminó de darles sus instrucciones y de impartirles unas últimas recomendaciones a los apóstoles como grupo, se dirigió a cada uno de ellos para decirles adiós individualmente, esta vez hablándoles palabras personales de consejo junto con su bendición de despedida. Los apóstoles estaban aún sentados alrededor de la mesa en la misma disposición en la que habían estado cuando compartieron la última cena y, a medida que el Maestro fue alrededor de la mesa hablando con ellos, estos, uno a uno, se iban poniendo de pie.
\vs p181 2:2 \pc A Juan, Jesús le dijo: “Tú, Juan, eres el más joven de mis hermanos. Has estado muy próximo a mí, y aunque os amo a todos con el mismo amor que un padre profesa hacia sus hijos, Andrés te encomendó que fueras uno de los tres apóstoles que permanecieran siempre junto a mí. Además de esto, has actuado por mí ---y debes seguir haciéndolo--- en los muchos asuntos relativos a mi familia terrenal. Y yo voy al Padre, Juan, confiando totalmente en que seguirás ocupándote de los míos en la carne. Mira que su confusión actual respecto a mi misión no evite en modo alguno que los comprendas, aconsejes y ayudes, como sabes que yo mismo haría si estuviera en la carne. Y cuando todos alcancen a ver la luz y entren plenamente en el reino, aunque todos vosotros los recibiréis con regocijo, dependo de ti, Juan, para que los acojas en mi nombre.
\vs p181 2:3 “Y ahora, conforme me adentro en las horas finales de mi andadura en la tierra, sigue a mi lado para que yo pueda dejarte palabras sobre mi familia. Respecto a la obra que mi Padre puso en mis manos, ya está acabada excepto por mi muerte en la carne, y estoy listo para beber esta última copa. Pero en cuanto a las responsabilidades que José, mi padre terrenal, me dejó, aunque he cumplido con ellos durante mi vida, debo ahora contar contigo para que obres en mi nombre en todos estos asuntos. Y te he escogido a ti para que hagas esto por mí, Juan, porque eres el más joven y es muy probable que sobrevivas a los demás apóstoles.
\vs p181 2:4 “Una vez os llamamos a ti y a tu hermano ‘hijos del trueno’. Empezaste con nosotros como alguien inflexible e intolerante, pero has cambiado mucho desde que querías que yo mandara descender fuego sobre las cabezas de aquellos no creyentes ignorantes e irreflexivos. Y aún debes cambiar más. Debes convertirte en el apóstol del nuevo mandamiento que yo os he dado esta noche. Dedica tu vida a enseñar a tus hermanos a amarse unos a otros, al igual que yo os he amado a vosotros”.
\vs p181 2:5 Estando Juan Zebedeo de pie allí, en el aposento alto, con las lágrimas rodándole por las mejillas, miró al rostro del Maestro y le dijo: “Y así lo haré, Maestro mío, pero, ¿cómo puedo aprender a amar más a mis hermanos?”. Entonces, Jesús le respondió: “Aprenderás a hacerlo cuando primero aprendas a amar más a su Padre de los cielos y cuando te intereses realmente más por el bien de tus hermanos en el tiempo y en la eternidad. Y este interés por los demás se fomenta mediante la compasión y la empatía, el servicio altruista y el perdón sin límites. Nadie debe menospreciar tu juventud, pero te recomiendo que prestes siempre atención al hecho de que, con frecuencia, la edad representa experiencia y que, en las cuestiones humanas, nada puede ocupar el lugar de la verdadera experiencia. Esfuérzate por vivir en paz con todos los hombres, en particular con tus amigos en la hermandad del reino celestial. Y, Juan, recuerda siempre, no pelees con las almas que quieras ganar para el reino”.
\vs p181 2:6 \pc Entonces el Maestro, rodeando su propio asiento, se detuvo un instante junto al sitio de Judas Iscariote. Los apóstoles estaban algo sorprendidos de que Judas no hubiera regresado aún, y sintieron una gran curiosidad por saber qué significaba el semblante triste de Jesús ante el asiento vacío del traidor. Pero ninguno de ellos, quizás excepto Andrés, llegó a tener el más mínimo atisbo de pensamiento de que su tesorero había salido para traicionar a su Maestro, como Jesús ya les había dado a entender temprano en la tarde y durante la cena. Pero estaban sucediendo tantas cosas que, en aquellos momentos, se les había olvidado enteramente el anuncio del Maestro de que uno de ellos lo traicionaría.
\vs p181 2:7 \pc Entonces, Jesús se acercó a Simón Zelotes, que se puso de pie y escuchó esta advertencia: “Tú eres un verdadero hijo de Abraham, pero qué difícil me ha resultado tratar de hacer de ti un hijo de este reino celestial. Te amo y también te aman todos tus hermanos. Sé que tú me amas, Simón, y también que amas el reino, pero aún estás decidido y determinado a hacer que este reino se adecue a tus deseos. Sé muy bien que terminarás por comprender la naturaleza espiritual y el significado de mi evangelio y que lo proclamarás con valentía, pero me inquieta lo que pueda sucederte cuando yo tenga que partir. Me regocijará saber que no vacilarás; me hará muy feliz saber que, después de que yo vaya al Padre, no dejarás de ser mi apóstol y que te comportarás como embajador del reino celestial, como corresponde”.
\vs p181 2:8 Jesús había apenas acabado de hablarle a Simón Zelotes, cuando el fervoroso patriota, secándose los ojos, respondió: “Maestro, no sientas temor por mi lealtad. Le di la espalda a todo para poder dedicar mi vida a instaurar tu reino en la tierra, y no flaquearé. Hasta ahora he sobrevivido a todas las contrariedades, y no te abandonaré”.
\vs p181 2:9 Y, entonces, poniendo su mano en el hombro de Simón, Jesús le dijo: “En verdad me reconforta oírte hablar así, particularmente en un momento como este, pero, mi buen amigo, aún no sabes que estás diciendo. Ni por un instante he dudado de tu lealtad, de tu devoción. Sé que no vacilarías en salir a la batalla y morir por mí, como también harían estos otros” (y todos ellos asintieron enérgicamente con la cabeza en señal de aprobación), “pero no se te exigirá que lo hagas. En repetidas ocasiones te he dicho que mi reino no es de este mundo, y que mis discípulos no pelearán para establecerlo. Te he dicho esto muchas veces, Simón, pero te niegas a enfrentarte a la verdad. No me preocupa tu lealtad hacia mí ni hacia el reino, pero ¿qué harás cuando me haya ido y te des finalmente cuenta de que no has sabido comprender el significado de mis enseñanzas, y que debes adaptar tus ideas equivocadas a la realidad de otro orden de asuntos del reino, el espiritual?”.
\vs p181 2:10 Simón quiso añadir algo, pero Jesús levantó la mano y, deteniéndolo, continuó diciendo: “Ninguno de mis apóstoles es más sincero y honesto de corazón que tú, pero ninguno se sentirá tan afectado ni descorazonado como tú tras mi partida. En todos tus desalientos, mi espíritu permanecerá contigo y estos, tus hermanos, no te abandonarán. No olvides lo que te he enseñado sobre la relación entre la ciudadanía en la tierra y la filiación en el reino espiritual del Padre. Reflexiona bien sobre todo lo que te he dicho de dar al césar lo que es del césar y a Dios lo que es de Dios. Dedica tu vida, Simón, a mostrar cómo el hombre mortal puede cumplir de manera satisfactoria mis instrucciones y reconocer al mismo tiempo entre el deber temporal a los poderes civiles y el servicio espiritual en la hermandad del reino. Si te dejas enseñar por el espíritu de la verdad, nunca habrá conflictos entre los requerimientos de la ciudadanía en la tierra y los de la filiación en los cielos, a no ser que los gobernantes temporales se atrevan a exigir de ti que le rindas el tributo y la adoración que solo pertenecen a Dios.
\vs p181 2:11 \pc “Y, ahora, Simón, cuando por fin veas todo esto, y una vez que te hayas librado de tu abatimiento y hayas salido a proclamar este evangelio con gran poder, jamás te olvides de que yo estuve contigo incluso en todos tus momentos de desaliento, y que lo estaré hasta el final. Siempre serás mi apóstol, y una vez que estés dispuesto a ver con los ojos del espíritu y a entregar en mayor medida tu voluntad a la voluntad del Padre en el cielo, entonces, volverás a trabajar conmigo como mi embajador, y nadie te quitará por causa de tu lentitud en comprender las verdades que te enseñé la autoridad que te he conferido. Así pues, Simón, de nuevo te prevengo que quien a espada pelea a espada perece, mientras que los que obran en el espíritu tendrán una vida sin fin en el reino venidero, y gozo y paz en el reino que ahora es. Cuando la labor que se encomendó a tus manos haya concluido en la tierra, tú, Simón, te sentarás conmigo en el reino de allí. Ciertamente verás el reino que has anhelado, pero no en esta vida. Continúa creyendo en mí y en lo que yo te he revelado, y recibirás el don de la vida eterna”.
\vs p181 2:12 \pc Cuando Jesús terminó de dirigirle estas palabras a Simón Zelotes, se adelantó hasta Mateo Leví y dijo: “Ya no recaerá por más tiempo sobre ti tener que proveer a las arcas del grupo apostólico. Pronto, muy pronto, estaréis todos dispersos; no se os permitirá tener el consuelo ni el sostén de uno solo de vuestros hermanos. A medida que sigáis con la predicación de este evangelio del reino, tendréis que encontrar por vuestra cuenta nuevos compañeros. Durante el tiempo de vuestra formación, os envié de dos en dos, pero ahora que os dejo, una vez que os hayáis recuperado de la conmoción, saldréis solos y hasta los confines de la tierra, anunciando esta buena nueva: que los mortales vivificados por la fe son los hijos de Dios”.
\vs p181 2:13 Entonces habló Mateo: “Pero, Maestro, ¿quién nos enviará, y cómo sabremos adónde ir? ¿Nos mostrará Andrés el camino?”. Jesús respondió: “No, Leví, Andrés ya no os dirigirá cuando proclaméis el evangelio. Él seguirá, de hecho, siendo vuestro amigo y consejero hasta el día en que llegue el nuevo maestro, y, entonces, el espíritu de la verdad os guiará a otras tierras a cada uno de vosotros en la labor de expandir el reino. Ha habido muchos cambios en ti desde aquel día en la casa de aduanas, cuando primeramente te dispusiste a seguirme; pero se producirán muchos más antes de que puedas tener la visión de una hermandad en la que los gentiles se sienten al lado de los judíos unidos fraternalmente. Pero continúa con tu ímpetu de ganar a tus hermanos judíos hasta que quedes totalmente satisfecho y, luego, ve con poder a los gentiles. De una cosa puedes estar seguro, Leví: te has ganado la confianza y el cariño de tus hermanos; todos ellos te aman”. (Y los diez afirmaron su conformidad con las palabras del Maestro).
\vs p181 2:14 “Leví, sé bien de tus preocupaciones, sacrificios y tareas para mantener las arcas repletas, algo que tus hermanos desconocen, y me regocija que, aunque el que llevaba la bolsa esté ausente, el embajador publicano está aquí, en mi reunión de despedida con los mensajeros del reino. Ruego para que puedas percibir el significado de mis enseñanzas con los ojos del espíritu. Y cuando el nuevo maestro llegue a tu corazón, ve adonde él te lleve y deja que tus hermanos vean ---al igual que el mundo entero--- qué es lo que puede hacer el Padre por un odiado cobrador de impuestos que se atrevió a seguir al Hijo del Hombre y a creer en el evangelio del reino. Leví, ya desde el principio, te amé, como amé a estos otros galileos. Sabiendo entonces tan bien que ni el Padre ni el Hijo hacen acepción de personas, cuídate de no hacer ninguna distinción entre los que se conviertan en creyentes en el evangelio mediante tu ministerio. Y, así pues, Mateo, dedica toda tu vida futura de servicio a mostrar a todos los hombres que Dios no hace acepción de personas; que, delante de Dios y en la fraternidad del reino, todos los hombres son iguales, todos los creyentes son hijos de Dios”.
\vs p181 2:15 \pc Jesús se acercó entonces a Santiago Zebedeo, que se quedó de pie, en silencio, mientras el Maestro le hablaba, diciéndole: “Santiago, cuando tú y tu hermano menor vinisteis a mí en una ocasión queriendo ocupar lugares de honor preferentes en el reino, os dije que era el Padre quien concedía tales honores, y os pregunté si podíais beber de mi copa, y ambos me respondisteis que sí. Incluso si no hubieras podido en aquel momento, como si no puedes ahora, pronto estarás preparado para realizar tal servicio por las experiencias que vas a atravesar. Debido a dicho comportamiento, tus hermanos se enojaron contigo en aquel momento. Si todavía no te han perdonado del todo, lo harán cuando te vean beber de mi copa. Sea tu ministerio largo o breve, que la paciencia se apodere de tu alma. Cuando venga el nuevo maestro, que él te enseñe la actitud de la compasión y esa comprensiva tolerancia que nace de la confianza sublime en mí y de la suprema sumisión a la voluntad del Padre. Dedica tu vida a demostrar esa unión de cariño humano y dignidad divina del discípulo que conoce a Dios y cree en el Hijo. Y todos los que viven así revelarán el evangelio incluso en su forma de morir. Tú y tu hermano Juan iréis por distintos caminos, y puede que uno de vosotros se siente conmigo en el reino eterno mucho antes que el otro. Te sería de gran ayuda aprender que la verdadera sabiduría entraña tacto a la vez que valor. Aprende a que tu combatividad vaya acompañada de buen juicio. Vendrán esos supremos momentos en los que mis discípulos no vacilarán en dar su vida por este evangelio, pero en circunstancias normales sería mucho mejor aplacar la ira de los no creyentes para poder vivir y seguir predicando la buena nueva. Hasta donde te sea posible, vive largamente en la tierra, para que tu vida de muchos años pueda ser fructífera, ganando almas para el reino celestial”.
\vs p181 2:16 \pc Cuando el Maestro acabó de hablarle a Santiago Zebedeo, rodeó la mesa hasta llegar a su extremo, en donde se sentaba Andrés, y, mirando a los ojos de su fiel ayudante, le dijo: “Andrés, tú me has representado con lealtad como jefe en funciones de los embajadores del reino celestial. Aunque algunas veces has dudado y otras has sido peligrosamente reticente, aun así, siempre has sido realmente justo y sumamente ecuánime en el trato con tus compañeros. Desde tu ordenación y la de tus hermanos como mensajeros del reino, habéis tenido autonomía en las cuestiones organizativas del grupo, por lo que te nombré jefe de estos elegidos. Yo no os he dirigido ni influido en tus decisiones en ningún otro asunto de orden temporal. Y, así lo hice, para que hubiera alguien que dirigiera todas vuestras futuras deliberaciones como grupo. En mi universo y en el universo de los universos de mi Padre, en todas sus relaciones espirituales, se trata a nuestros hermanos\hyp{}hijos, como a seres individuales, pero, en cuanto a las relaciones de grupo, es invariable que todos ellos dispongan de un claro liderazgo. Nuestro reino es un ámbito de orden, y cuando dos o más criaturas volitivas colaboran entre sí siempre se les facilita la autoridad de un jefe”.
\vs p181 2:17 “Y ahora, Andrés, puesto que tú eres el jefe de tus hermanos por la autoridad con la que te nombré, y puesto que has servido, pues, como mi representante personal, y estando como estoy a punto de dejaros para ir a mi Padre, te relevo de cualquier responsabilidad en lo que respecta a estos asuntos temporales y organizativos. A partir de ahora, no tendrás ninguna potestad sobre tus hermanos, excepto la que te hayas ganado en tu capacidad como líder espiritual, y que tus hermanos, por lo tanto, reconozcan con entera libertad. Desde este momento, no puedes ejercer ninguna autoridad sobre ellos a menos que estos restituyan en ti tal potestad, acordándolo claramente entre ellos una vez que yo me haya ido al Padre. Pero esta liberación de tus responsabilidades como jefe de este grupo, no atenúa de ninguna manera tu deber moral de hacer todo lo que esté en tu poder para mantener a tus hermanos unidos con mano firme y amorosa durante la difícil hora que se avecina, esos días que median entre mi partida en la carne y el envío del nuevo maestro, que vivirá en vuestros corazones y que terminará por llevaros a toda la verdad. Conforme me preparo para dejarte, quiero eximirte de cualquier responsabilidad material que tuvo sus inicios y autoridad con mi presencia entre vosotros como uno más. En lo sucesivo, solo ejerceré autoridad espiritual sobre ti y entre vosotros.
\vs p181 2:18 “Si tus hermanos desean retenerte como su consejero, te pido que, en todos los asuntos temporales y espirituales, hagas todo lo posible por promover la paz y la armonía entre los distintos grupos de creyentes sinceros del evangelio. Dedica el resto de tu vida a fomentar los aspectos prácticos del amor fraterno entre tus hermanos. Sé amable con mis hermanos en la carne cuando lleguen a creer completamente en este evangelio; manifiesta amor y ecuánime entrega a los griegos, en el oeste, y a Abner, en el este. Aunque estos, mis apóstoles, pronto se esparcirán por los cuatro ángulos de la tierra, para proclamar allí la buena nueva de la salvación por medio de la filiación con Dios, debes mantenerlos juntos durante los difíciles momentos que están al llegar, durante ese tiempo de severas pruebas en el que debéis aprender a creer en este evangelio sin mi presencia personal, mientras aguardáis pacientemente la llegada del nuevo maestro, el espíritu de la verdad. Y así, Andrés, aunque tal vez no recaiga en ti hacer grandes obras a la vista de los hombres, conténtate con ser el maestro y consejero de aquellos que sí las hacen. Sigue con tu labor en la tierra hasta el fin y, entonces, continuarás realizando este ministerio en el reino eterno, porque ¿es que no te he dicho muchas veces que tengo otras ovejas que no son de este redil?”.
\vs p181 2:19 \pc Jesús se dirigió luego hacia los gemelos Alfeo y, de pie entre los dos, dijo: “Hijitos míos, vosotros sois uno de los tres grupos de hermanos que escogieron seguirme. Los seis habéis hecho bien trabajando en paz con los de vuestra propia carne y sangre, pero ninguno de estos lo ha hecho mejor que vosotros. Se aproximan tiempos difíciles. Quizás no entendáis todo lo que os ocurrirá a vosotros y a vuestros hermanos, pero no dudéis jamás de que se os llamó hace tiempo para llevar a cabo la obra del reino. Por algún tiempo, no habrá muchedumbres que organizar, pero no os desaniméis; cuando vuestra tarea de vida concluya, yo os recibiré en lo alto, donde en gloria les hablaréis de vuestra salvación a las multitudes seráficas y al gran número de altos Hijos de Dios. Dedicad vuestra vida a realzar el trabajo ordinario. Mostrad a todos los hombres de la tierra y a los ángeles del cielo la alegría y el arrojo de quienes, tras haber sido llamados para trabajar durante un tiempo en el servicio especial de Dios, regresan a sus actividades de días anteriores. Si, por el momento, vuestra labor en los asuntos externos del reino está terminada, debéis volver a vuestras tareas cotidianas previas con el nuevo entendimiento que os da la experiencia de la filiación de Dios y con la excelsa conciencia de que, para quien conoce a Dios, no hay quehaceres ordinarios ni faenas seculares. Para vosotros, que habéis trabajado conmigo, todas las cosas se han vuelto sagradas, incluso cualquier labor terrenal se ha convertido en un servicio a Dios Padre. Y cuando oigáis las noticias de las obras de vuestros antiguos compañeros apostólicos, regocijaos con ellos y continuad vuestra actividad diaria como quienes aguardan a Dios y sirven mientras esperan. Habéis sido mis apóstoles, y siempre lo seréis, y me acordaré de vosotros en el reino venidero”.
\vs p181 2:20 \pc Y entonces Jesús se aproximó a Felipe, que, levantándose, oyó este mensaje de su Maestro: “Felipe, me has hecho muchas preguntas triviales, pero yo hice todo lo posible por dar respuesta a cada una de ellas, y ahora desearía responder a la última de dichas preguntas, que se plantean en tu mente, extremadamente honesta pero falta de espiritualidad. Todo el tiempo, mientras rodeaba la mesa para acercarme a ti, te has estado diciendo: ‘¿Qué haré yo si el Maestro se va y nos deja solos en el mundo?’ ¡Oh, tú, hombre de poca fe! Y, sin embargo, tienes casi tanta como muchos de tus hermanos. Felipe, tú has sido un buen encargado de abastecimientos. Tan solo nos fallaste en ciertas pocas ocasiones, y uno de esos fallos lo usamos para manifestar la gloria del Padre. Tu puesto de encargado está al terminar. Pronto deberás realizar con mayor plenitud la labor para la que se te llamó: predicar este evangelio del reino. Felipe, tú siempre querías demostraciones de todo y, muy pronto, verás grandes cosas. Habría sido mucho mejor que hubieras visto todo esto por la fe, pero, dado que fuiste honesto incluso en tu visión materialista de las cosas, vivirás para ver cómo se cumplen mis palabras. Y, entonces, cuando se te bendiga con la visión espiritual, sal para hacer tu labor, dedicando tu vida a la causa de llevar a la humanidad al encuentro de Dios y a la búsqueda de las realidades eternas con los ojos de la fe espiritual y no con los de la mente material. Recuerda, Felipe, tienes una gran misión en la tierra porque el mundo está lleno de seres que ven la vida tal como tú tiendes a verla. Tienes un gran trabajo por hacer, y cuando esté terminado en la fe, vendrás a mí en mi reino, y me será de gran gozo mostrarte lo que el ojo no ha visto, el oído no ha oído y la mente mortal no ha concebido. Entretanto, vuélvete como un niño pequeño en el reino del espíritu y permíteme, en el espíritu del nuevo maestro, guiarte hacia adelante en el reino espiritual. Y, de esta manera, podré hacer por ti mucho de lo que no pude lograr cuando vivía con vosotros como un mortal del mundo. Y siempre recuerda, Felipe, que quien me ha visto a mí, ha visto al Padre”.
\vs p181 2:21 \pc Entonces, el Maestro fue adonde estaba Natanael. Cuando este se puso de pie, Jesús le pidió que se sentara y, sentándose a su lado, le dijo: “Natanael, desde que te convertiste en mi apóstol, has aprendido a vivir por encima del prejuicio y ser cada vez más tolerante. Pero tienes aún muchas más cosas que aprender. Has sido una bendición para tus compañeros, porque tu sinceridad constante les ha servido de guía. Cuando yo me haya ido, es posible que tu franqueza sea un obstáculo para llevarte bien con tus hermanos, los antiguos y los nuevos. Debes aprender que incluso la expresión de un buen pensamiento debe modularse según sea el estatus intelectual y el desarrollo espiritual del oyente. La sinceridad es de suma utilidad en el trabajo del reino cuando está unida a la gentileza.
\vs p181 2:22 “Si quieres aprender a trabajar con tus hermanos, quizás puedas lograr cosas más permanentes, pero si te ves buscando a quienes piensan como tú, dedica tu vida, en tal eventualidad, a demostrar que el discípulo conocedor de Dios puede convertirse en constructor del reino incluso si está solo en el mundo y completamente aislado de sus compañeros creyentes. Yo sé que serás fiel hasta el fin, y algún día te recibiré en el servicio expandido de mi reino en lo alto”.
\vs p181 2:23 Entonces habló Natanael y le hizo a Jesús esta pregunta: “He escuchado tus enseñanzas desde que, por primera vez, me llamaste al servicio de este reino, pero, si te soy honesto, no puedo entender bien todo lo que nos estás diciendo. No sé qué debo esperar que ocurra, y creo que la mayoría de mis hermanos están igualmente desconcertados, pero dudan en confesar su confusión. ¿Puedes ayudarme?”. Jesús, colocando su mano sobre el hombro de Natanael, le dijo: “Amigo mío, no es extraño que te sientas perplejo cuando tratas de comprender el significado de mis enseñanzas espirituales, dado que te lo impiden en gran medida tus ideas preconcebidas de la tradición judía y estás muy confundido debido a tu insistente tendencia a interpretar mi evangelio según las enseñanzas de los escribas y de los fariseos.
\vs p181 2:24 “Mediante la palabra, te he enseñado muchas cosas, y he vivido mi vida entre vosotros. He hecho todo lo que podía hacerse para llevar luz a vuestras mentes y liberar vuestras almas, y lo que no hayas podido aprender de mis enseñanzas y de mi vida, debes ahora prepararte para aprenderlo de la mano de ese maestro de todos los maestros: la experiencia real. Y en todas estas nuevas experiencias que ya te aguardan, iré delante de vosotros y el espíritu de la verdad estará contigo. No temas; aquello que no consigas comprender en este momento, el nuevo maestro, cuando haya venido, te lo revelará durante lo que te resta por vivir en la tierra y, en adelante, por medio de tu formación en las eras eternas”.
\vs p181 2:25 Y, entonces, el Maestro, volviéndose hacia todos ellos, dijo: “No desmayéis si no llegáis a entender el significado del evangelio en su totalidad. Vosotros no sois sino hombres mortales y finitos, y lo que yo os he enseñado es infinito, divino y eterno. Sed pacientes y estad de buen ánimo, pues tenéis ante vosotros las eras eternas para continuar progresando hacia el logro de la perfección tal como vuestro Padre del Paraíso es perfecto”.
\vs p181 2:26 \pc Y, luego, Jesús se aproximó a Tomás, quien, levantándose, le escuchó decir: “Tomás, a menudo te ha faltado la fe; sin embargo, cuando tuviste tus momentos de duda, jamás careciste de coraje. Sé bien que no te dejas engañar por los falsos profetas ni por los maestros falaces. Una vez que me haya ido, tus hermanos valorarán más tu manera crítica de ver las nuevas enseñanzas. Y cuando todos vosotros estéis esparcidos por los confines de la tierra en los tiempos por venir, recuerda que sigues siendo mi embajador. Dedica tu vida a la gran labor de mostrar cómo la mente material crítica del hombre puede triunfar sobre la inercia de la duda intelectual cuando se encuentra ante la palpable manifestación de la verdad viva, tal como obra en la experiencia de los hombres y mujeres nacidos del espíritu que rinden los frutos del espíritu en su vida, y que se aman unos a otros, como yo os he amado. Tomás, me alegro de que te unieras a nosotros y sé que, tras un breve período de desconcierto, continuarás adelante en el servicio del reino. Tus dudas han creado confusión en tus hermanos, pero nunca tuve preocupación por ellos. Tengo confianza en ti, e iré delante de ti, incluso hasta los lugares más remotos de la tierra”.
\vs p181 2:27 \pc Después, el maestro fue hasta donde estaba Simón Pedro, que se puso de pie mientras Jesús le decía: “Pedro, yo sé que tú me amas, y que dedicarás tu vida a la proclamación pública de este evangelio del reino a los judíos y a los gentiles, pero me inquieta que tus años de tan estrecha relación conmigo no hayan podido hacer más para ayudarte a pensar antes de hablar. ¿Qué experiencias deberás pasar para que aprendas a poner un guardián en tus labios? ¡Cuántos problemas nos has ocasionado por tu forma irreflexiva de hablar, por tu pretenciosa confianza en ti mismo! Y te acarrearás muchos más problemas si no dominas esta flaqueza. Sabes que tus hermanos te aman a pesar de esta debilidad tuya, y debes también entender que tal inconveniente no afecta mi cariño por ti de ninguna de las maneras, pero sí disminuye la utilidad de tu labor y jamás dejará de crearte problemas. Pero, sin duda, la experiencia por la que pasarás esta misma noche te ayudará sobremanera. Y lo que ahora te digo, Simón Pedro, lo digo igualmente a todos tus hermanos aquí congregados: esta noche, todos vosotros estaréis en gran peligro de flaquear por lo que va a ocurrir. Sabéis que está escrito: ‘Herirán al pastor y las ovejas serán dispersadas’. Cuando esté ausente, existe el enorme riesgo de que algunos de vosotros sucumbáis a las dudas y vaciléis ante lo que me ocurra a mí. Pero ahora yo os prometo que volveré a vosotros durante un breve tiempo, y que iré entonces delante de vosotros a Galilea”.
\vs p181 2:28 Entonces Pedro, poniendo su mano sobre el hombro de Jesús, dijo: “Aunque todos mis hermanos puedan sucumbir a las duda por lo que te suceda, yo te prometo que no me faltará la fe por cualquier cosa que puedas hacer. Iré contigo y, si es necesario, moriré por ti”.
\vs p181 2:29 Estando Pedro allí, ante el Maestro, tembloroso por la intensa emoción y rebosante de auténtico amor por él, Jesús le miró a los ojos humedecidos y le dijo: “Pedro, de cierto, de cierto te digo que esta noche antes de que el gallo cante me habrás negado tres o cuatro veces. Y, de este modo, lo que no has logrado aprender de una apacible relación conmigo, lo aprenderás mediante graves dificultades y muchos pesares. Y tras haber verdaderamente aprendido esta necesaria lección, debes fortalecer a tus hermanos y continuar viviendo una vida dedicada a la predicación de este evangelio, aunque puede que te metan preso y que, quizás, sigas mis pasos, pagando el precio supremo por tu amoroso servicio a la edificación del reino del Padre.
\vs p181 2:30 “Pero recuerda mi promesa: cuando yo resucite, permaneceré con vosotros por un tiempo antes de ir al Padre. E incluso esta noche le suplicaré al Padre que os dé fuerzas a cada uno de vosotros para lo que tan pronto habéis de pasar. Os amo a todos con el mismo amor con el que el Padre me ama a mí y, por lo tanto, a partir de ahora, debéis amaros unos a otros como yo os he amado a vosotros”.
\vs p181 2:31 \pc Y, más tarde, tras haber cantado un himno, se marcharon al campamento del Monte de los Olivos.
