\upaper{34}{El espíritu materno del universo local}
\author{Mensajero poderoso}
\vs p034 0:1 Cuando el hijo creador se hace personal mediante la acción del Padre Universal y del Hijo Eterno, el Espíritu Infinito, entonces, individualiza una representación nueva y única de sí mismo para acompañar a este hijo creador a las regiones espaciales y ser allí su acompañante, primero, en la organización física y, luego, en la creación y el ministerio de las criaturas del recién proyectado universo.
\vs p034 0:2 El espíritu creativo responde tanto a las realidades físicas como a las espirituales, tal como lo hace el hijo creador, y son, por tanto, complementarios y colaboradores en la administración de uno de los universos locales del tiempo y el espacio.
\vs p034 0:3 Estas hijas espirituales son de la esencia del Espíritu Infinito, pero no pueden desempeñar su labor de forma simultánea en la creación física y en el ministerio espiritual. En la creación material, el hijo del universo proporciona el modelo mientras que el espíritu del universo inicia la materialización de las realidades físicas. El hijo obra en los diseños de la potencia, pero el espíritu transforma estas creaciones de energía en substancias físicas. Aunque resulta difícil describir en el universo esta presencia temprana del Espíritu Infinito como persona, para el hijo creador, el espíritu colaborador es personal y siempre ha obrado como un ser individual diferenciado.
\usection{1. LA MANIFESTACIÓN PERSONAL DEL ESPÍRITU CREATIVO}
\vs p034 1:1 Después de que se ha completado la organización física del conjunto planetario y estelar y se han establecido las vías circulatorias de la energía a través de los centros de potencia del suprauniverso, tras esta labor preliminar de creación de las instancias intermedias del Espíritu Infinito, que operan a través, y bajo la dirección, de su convergencia creativa en el universo local, el hijo miguel anuncia públicamente que acto seguido se establecerá la vida en el recién organizado universo. Cuando el Paraíso reconoce esta declaración de intención, se produce una respuesta aprobatoria en la Trinidad del Paraíso, seguida de la desaparición en el resplandor espiritual de estas Deidades del espíritu mayor en cuyo suprauniverso se está organizando esta nueva creación. Mientras tanto, los otros espíritus mayores se acercan a este alojamiento central de las Deidades del Paraíso y, posteriormente, cuando dicho espíritu mayor, acogido por la Deidad, emerge al reconocimiento de sus compañeros, tiene lugar, repentinamente, lo que se conoce como “estallido inaugural”. Es un fabuloso destello espiritual, un fenómeno claramente perceptible incluso desde la distante sede del suprauniverso en cuestión. Y en simultaneidad con esta manifestación trinitaria, poco entendida, se produce un cambio notable en la naturaleza de la presencia del espíritu creativo y en la potencia del Espíritu Infinito que reside en el universo local correspondiente. De forma inmediata, en respuesta a estos fenómenos con origen en el Paraíso, y ante la presencia misma del hijo creador, se hace personal una nueva representación individual del Espíritu Infinito. Se trata de la benefactora divina. El espíritu creativo, como ser individual, ayudante del hijo creador, se convierte en su estrecho colaborador creativo, en el espíritu materno del universo local.
\vs p034 1:2 A partir y a través de este nuevo ser que se ha segregado del Creador Conjunto, circulan las corrientes establecidas y las vías prescritas de la potencia espiritual y la influencia espiritual destinadas a infundir todos los mundos y seres de ese universo local. En realidad, esta nueva presencia no es más que la transformación de la colaboradora preexistente y menos personal del hijo creador, que lo había acompañado con anterioridad en la organización del universo físico.
\vs p034 1:3 \pc Este es el relato en pocas palabras de un formidable suceso que, no obstante, constituye casi todo lo que se puede decir con respecto a acontecimientos tan cruciales; son instantáneos, inescrutables e incomprensibles, y su secreto reside en el seno de la Trinidad del Paraíso. Tan solo estamos ciertos de algo: la presencia del espíritu creativo en el universo local, durante el momento de la creación u organización puramente física de este, no se diferencia totalmente del espíritu del Espíritu Infinito del Paraíso; en tanto que, después de que el espíritu mayor a cargo reaparece del acogimiento secreto de los Dioses y tras el destello de la energía espiritual, dicha manifestación del Espíritu Infinito en el universo local se transforma, de súbito y por completo, semejándose personalmente al espíritu mayor que mantuvo esa unión transformadora con el Espíritu Infinito. El espíritu materno del universo local adquiere así una naturaleza personal matizada por la del espíritu mayor del suprauniverso que ostenta esa jurisdicción astronómica.
\vs p034 1:4 A esta presencia personal del Espíritu Infinito, o espíritu materno creativo del universo local, se le conoce en Satania como la benefactora divina. En la práctica, para cualquier fin y propósito espiritual, esta manifestación de la Deidad es un ser individual divino, una persona espiritual. Y así lo reconoce y considera el hijo creador. Es a través de esta ubicación y expresión personal de la Tercera Fuente y Centro en nuestro universo local por la que el espíritu creativo pudo, con posterioridad, someterse con tanta plenitud al hijo creador y se pudo, en verdad, decir de él que “toda potestad en el cielo y en la tierra se le ha confiado”.
\usection{2. LA NATURALEZA DE LA BENEFACTORA DIVINA}
\vs p034 2:1 Habiendo experimentado una notable metamorfosis de su ser personal en el momento de la creación de la vida, la benefactora divina ejerce a partir de entonces su labor como persona y coopera, de una manera muy personal, con el hijo creador en la planificación y dirección de los muchos asuntos de su conjunta creación local. Para muchos tipos de seres del universo, incluso esta representación del Espíritu Infinito puede que no parezca completamente personal durante las eras que preceden al ministerio de gracia final de Miguel. No obstante, con posterioridad a la elevación del hijo creador a la autoridad soberana como hijo mayor, el espíritu creativo materno crece de tal manera en cualidades personales que es reconocido personalmente por todos aquellos con los que se relaciona.
\vs p034 2:2 Desde su más temprana vinculación con el hijo creador, el espíritu del universo posee todos los atributos del Espíritu Infinito en cuanto al control físico, incluyendo unos plenos atributos de antigravedad. Al lograr el estado personal, este espíritu ejerce en el universo local un dominio sobre la gravedad mental tan completo que se equipara al que ejercería el Espíritu Infinito si estuviese personalmente presente.
\vs p034 2:3 \pc En cada uno de los universos locales, la benefactora divina obra en conformidad con la naturaleza y las características consustanciales del Espíritu Infinito, tal como se reflejan en uno de los siete espíritus mayores del Paraíso. Aunque los espíritus del universo son básicamente uniformes en carácter, es igualmente cierto que ostentan funciones diversas, de acuerdo con el espíritu mayor a partir del que tuvieron su origen. Esta diferencia de origen en los diferentes suprauniversos explica los distintos modos de actuación de estos espíritus maternos de los universos locales. Si bien, en todos sus atributos espirituales esenciales, estos espíritus son idénticos, igualmente espirituales y totalmente divinos, sea cual fuere la diferencia existente entre los suprauniversos.
\vs p034 2:4 \pc El espíritu creativo es corresponsable con el hijo creador en la creación de las criaturas de los mundos y nunca le falla en la labor de sostener y conservar estas creaciones. La vida se provee y mantiene por mediación del espíritu creativo. “Envías tu Espíritu, son creados y renuevas la faz de la tierra”.
\vs p034 2:5 En la creación de un universo de criaturas inteligentes, el espíritu creativo materno obra, primero, en el ámbito de la perfección del universo, colaborando con el hijo en la creación de la brillante estrella de la mañana; después, sus vástagos se aproximan cada vez más al orden de seres creados de los planetas, del mismo modo que los hijos se escalonan en orden descendente desde los melquisedecs hasta los hijos materiales, que son los que realmente se ponen en contacto con los mortales de los mundos. En la etapa evolutiva posterior de las criaturas mortales, los hijos portadores de vida proporcionan el cuerpo físico, formado a partir de materia organizada existente en el mundo, mientras que el espíritu del universo aporta “el aliento de vida”.
\vs p034 2:6 \pc Aunque el séptimo segmento del gran universo pueda, en muchos aspectos, tener un desarrollo tardío, un meticuloso análisis de esta situación revela que, en épocas venideras, dicho segmento evolucionará en una creación excepcionalmente bien equilibrada. Predecimos este alto grado de simetría en Orvontón, porque el espíritu que preside este suprauniverso es el jefe de los espíritus mayores en las alturas; se trata de una inteligencia espiritual que engloba la unión armonizada y la perfecta coordinación de los rasgos y el carácter de las tres Deidades eternas. Avanzamos con lentitud y nos encontramos rezagados en comparación con otros sectores, pero sin duda nos aguarda algún día, en las eternas eras del futuro, un desarrollo extraordinario y un logro sin precedentes.
\usection{3. EL HIJO CREADOR Y EL ESPÍRITU CREATIVO EN EL TIEMPO Y EL ESPACIO}
\vs p034 3:1 Ni el Hijo Eterno ni el Espíritu Infinito están limitados ni condicionados por el tiempo o el espacio, pero la mayoría de sus vástagos sí.
\vs p034 3:2 El Espíritu Infinito se infunde sobre todo el espacio y habita en el círculo de la eternidad. No obstante, en su contacto personal con los hijos temporales, los seres personales del Espíritu Infinito deben a menudo tener en consideración el factor tiempo, no tanto el factor espacio. Muchos aspectos del ministerio de la mente no se ven afectados por el espacio, si bien, experimentan una dilación en el tiempo a la hora de coordinar los diversos niveles de la realidad del universo. Un mensajero solitario es prácticamente independiente del espacio aunque necesite realmente tiempo para trasladarse de un lugar a otro; en este sentido, hay otras entidades que desconocéis con características similares.
\vs p034 3:3 \pc En cuanto a sus prerrogativas personales, el espíritu creativo es completamente independiente del espacio, aunque no del tiempo. No hay presencia personal particular de este espíritu del universo ni en las sedes de las constelaciones ni en las de los sistemas. Se difunde por igual por todo su universo local y está, por tanto, genuinamente presente de forma personal tanto en un mundo como en otro.
\vs p034 3:4 El tiempo siempre limita al espíritu creativo en relación a su servicio en el universo. El hijo creador obra de forma instantánea en todo su universo, pero el espíritu creativo debe contar con el elemento tiempo en su ministerio de la mente universal, salvo cuando de manera consciente e intencional se vale de las prerrogativas personales del hijo del universo. En su labor relativa al espíritu puro, el espíritu creativo también actúa con independencia del tiempo al igual que lo hace en su colaboración con el misterioso funcionamiento de la reflectividad del universo.
\vs p034 3:5 \pc Aunque la vía circulatoria de la gravedad espiritual del Hijo Eterno obra independientemente del tiempo y del espacio, no toda la labor de los hijos creadores está exenta de las limitaciones del espacio. Si exceptuamos su actuación en los mundos evolutivos, estos hijos migueles parecen ser capaces de operar con relativa autonomía respecto al tiempo. Y, aunque este no representa un obstáculo para ellos, el espacio sí los condiciona. No pueden estar personalmente en dos lugares a la vez. Miguel de Nebadón actúa con independencia del tiempo dentro de su propio universo y, mediante la reflectividad, prácticamente de la misma manera en el suprauniverso. El tiempo no es barrera en su comunicación directa con el Hijo Eterno.
\vs p034 3:6 La benefactora divina ayuda y responde al hijo creador, facilitándole la superación y compensación de sus inherentes limitaciones con respecto al espacio. Así pues, cuando ambos desempeñan su actividad gestora unidos, se muestran, dentro de los confines de su propia creación local, virtualmente independientes del tiempo \bibemph{y} del espacio. En consecuencia, tal como se observa de manera práctica en todo el universo local, el hijo creador y el espíritu creativo actúan habitualmente con independencia del tiempo y del espacio, puesto que cada cual siempre dispone de la posibilidad de poder liberarse de estos factores por medio del otro.
\vs p034 3:7 \pc Solo los seres absolutos son independientes del tiempo y del espacio en un sentido absoluto. La mayoría de las personas de rango menor que el Hijo Eterno y el Espíritu Infinito están sujetas tanto al tiempo como al espacio.
\vs p034 3:8 Cuando el espíritu creativo se vuelve “consciente del espacio” es que se dispone a reconocer como suyo un “dominio espacial” delimitado, una zona en la que estará libre de espacio en contraposición con cualquier otra que lo llegaría a condicionar. Solamente dentro del entorno de la propia conciencia se es libre para elegir y actuar.
\usection{4. LAS VÍAS CIRCULATORIAS DEL UNIVERSO LOCAL}
\vs p034 4:1 En el universo local de Nebadón hay tres vías circulatorias espirituales diferentes:
\vs p034 4:2 \li{1.}El espíritu de gracia del hijo creador, el consolador, el espíritu de la verdad.
\vs p034 4:3 \li{2.}La vía circulatoria espiritual de la benefactora divina, el espíritu santo.
\vs p034 4:4 \li{3.}La vía del ministerio de la inteligencia, que incluye las acciones más o menos unificadas, aunque diversas, de los siete espíritus asistentes de la mente.
\vs p034 4:5 \pc Los hijos creadores están dotados de un espíritu con presencia en el universo, análogo, de muchas maneras, al de los siete espíritus mayores del Paraíso. Se trata del espíritu de la verdad, que el hijo de gracia hace derramar sobre el mundo tras recibir potestad espiritual sobre dicha esfera. El consolador, que otorga como don, es la fuerza espiritual que por siempre atrae a los buscadores de la verdad hacia Aquel que personifica la verdad en el universo local. Este espíritu, del que el hijo creador está dotado de forma consustancial, emerge de su naturaleza divina de la misma forma que las vías circulatorias mayores se derivan de las presencias de las personas de las Deidades del Paraíso.
\vs p034 4:6 El hijo creador va y viene; su presencia personal puede estar en el universo local o en cualquier otra parte. Sin embargo, el espíritu de la verdad obra sin perturbación en este sentido, porque dicha presencia divina, aunque procede del ser personal del hijo creador, está centrada operativamente en la persona de la benefactora divina.
\vs p034 4:7 El espíritu materno del universo, sin embargo, nunca se retira del mundo sede del universo local. El espíritu del hijo creador puede realizar su labor, y de hecho lo hace, independientemente de la presencia personal de este hijo creador, pero no sucede así con el espíritu perteneciente al espíritu materno. El espíritu santo de la benefactora divina se volvería inoperante si la presencia personal de esta se alejase de Lugar de Salvación. Dicha presencia espiritual parece situarse fija en la sede planetaria del universo, y es precisamente este hecho el que permite al espíritu del hijo creador actuar con independencia de la localización de dicho hijo creador. El espíritu materno del universo actúa como centro y punto de actividad del universo para el espíritu de la verdad y para el espíritu santo, su propia influencia personal.
\vs p034 4:8 \pc Tanto el Hijo\hyp{}Padre creador como el espíritu materno creativo contribuyen de distintas maneras a la dotación de mente de sus hijos del universo local. Si bien, el espíritu creativo no otorga la mente hasta que no se le confieren prerrogativas personales para ello.
\vs p034 4:9 Los órdenes personales supraevolutivos del universo local están dotados del tipo de mente correspondiente a ese universo local, según el modelo que rige en el suprauniverso al que pertenecen. Los órdenes humanos y subhumanos de vida evolutiva están, por su parte, dotados de los tipos de espíritus asistentes que sirven en la mente.
\vs p034 4:10 \pc Los siete espíritus asistentes de la mente son creación de la benefactora divina del universo local; son similares en carácter pero diferentes en cuanto a sus capacidades, y todos ellos comparten la naturaleza del espíritu del universo del mismo modo, aunque, al margen de su madre creadora, difícilmente se les puede considerar seres personales. A estos siete asistentes se les ha dado los siguientes nombres: espíritu de \bibemph{sabiduría,} espíritu de \bibemph{adoración,} espíritu de \bibemph{consejo,} espíritu de \bibemph{conocimiento,} espíritu de \bibemph{valentía,} espíritu de \bibemph{entendimiento} y espíritu de \bibemph{intuición} ---o de percepción diligente---.
\vs p034 4:11 \pc Estos son los “siete espíritus de Dios”, “como lámparas de fuego delante del trono”, que el profeta observó en los símbolos de su visión. Pero no vio los tronos de los veinticuatro centinelas alrededor de estos siete espíritus asistentes de la mente. Esto indica la confusión de ambos relatos; uno se refiere a la sede del universo y, el otro, a la capital del sistema. Los tronos de los veinticuatro ancianos están en Jerusem, la sede de vuestro sistema local de mundos habitados.
\vs p034 4:12 Pero fue de Lugar de Salvación que Juan escribió: “Y del trono salían relámpagos y truenos y voces” ---o transmisiones del universo a los sistemas locales---. También visualizó a las criaturas encargadas del control direccional del universo local, las brújulas vivas de su mundo sede. Estas cuatro criaturas de Lugar de Salvación, que operan sobre las corrientes del universo, mantienen dicho control direccional en Nebadón. Los asiste capazmente el primer espíritu\hyp{}mente en actuar, el asistente de intuición o el espíritu de “entendimiento diligente”. No obstante, la descripción de estas cuatro criaturas ---llamadas “bestias”--- ha sido lamentablemente deformada; son de una belleza incomparable y de una delicada apariencia.
\vs p034 4:13 Los cuatro puntos de la brújula son universales y consubstanciales a la vida de Nebadón. Todos los seres vivos poseen unidades corporales que son sensibles y responden a estas corrientes direccionales. Esta constitución de los seres se reproduce desde el universo hasta los distintos planetas y, en conjunción con las fuerzas magnéticas de los mundos, activan en el organismo animal a multitudes de cuerpos microscópicos de tal modo que estas células direccionales señalan siempre al norte y al sur. Por lo tanto, en los seres vivos del universo, está el sentido de orientación por siempre fijado. La humanidad no carece por completo de un conocimiento consciente de dicho sentido. Estos corpúsculos se observaron por primera vez en Urantia, aproximadamente en la época en la que se realizaba esta narrativa.
\usection{5. EL MINISTERIO DEL ESPÍRITU CREATIVO}
\vs p034 5:1 La benefactora divina coopera con el hijo creador en la formulación de la vida y en la creación de nuevos órdenes de seres hasta el momento del séptimo ministerio de gracia de este y, posteriormente, tras la elevación del hijo a la soberanía plena del universo, continúa colaborando con él y con el espíritu, que este hijo otorga, en la tarea añadida de servir al mundo y hacer progresar al planeta.
\vs p034 5:2 En los mundos habitados, el espíritu creativo inicia su tarea de alentar la evolución, comenzando con el material inanimado de estos mundos, otorgando primero la vida vegetal, luego los organismos animales, más tarde los primeros órdenes de existencia humana; y cada uno de estos dones sucesivos contribuye a un mayor desarrollo del potencial evolutivo de la vida planetaria desde las etapas iniciales y primitivas hasta la aparición de las criaturas de voluntad. Esta labor del espíritu se lleva a efecto, en gran parte, a través de los siete asistentes, los espíritus de la promesa, el espíritu\hyp{}mente unificador y coordinador de los planetas evolutivos, que guían a las razas de los hombres, por siempre y de manera unida, hacia ideas superiores y hacia los más altos ideales espirituales.
\vs p034 5:3 \pc El hombre mortal tiene una primera experiencia del ministerio del espíritu en concurrencia con la mente cuando la mente, puramente animal de las criaturas evolutivas, desarrolla cualidades que la hacen receptiva a los asistentes sexto y séptimo, los de adoración y de sabiduría. La acción de estos asistentes indica que la mente ha llegado a un desarrollo que la hace cruzar al umbral del ministerio espiritual. Y tales mentes, en su capacidad de adoración y sabiduría, se incorporan de inmediato a las vías espirituales de la benefactora divina.
\vs p034 5:4 Cuando la mente goza del ministerio del espíritu santo, posee la capacidad de elegir (consciente o inconscientemente) la presencia espiritual del Padre Universal ---el modelador del pensamiento---. No obstante, las mentes normales no están necesariamente preparadas para recibir a los modeladores del pensamiento hasta que el hijo de gracia no ha dado de sí el espíritu de la verdad para que lleve a cabo su ministerio planetario a todos los mortales. Este espíritu, en íntima conjunción con la presencia del espíritu de la benefactora divina, ronda los mundos, intentando impartir la verdad e iluminar espiritualmente la mente de los hombres, buscando inspirar las almas de las criaturas de las razas ascendentes y conducir a los seres que moran en los planetas evolutivos por siempre hacia su destino divino: llegar al Paraíso.
\vs p034 5:5 Aunque el espíritu de la verdad se derrame sobre toda carne, este espíritu del hijo creador está casi enteramente limitado en su acción y capacidad por la propia receptividad personal del hombre hacia lo que constituye la esencia de la misión del hijo de gracia. El espíritu santo es en parte independiente de la actitud humana y en parte está condicionado por las decisiones y la cooperación misma de la voluntad del hombre; no obstante, su ministerio se hace progresivamente más efectivo en la santificación y espiritualización de la vida interior de aquellos mortales que más plenamente \bibemph{se dejan regir} por las directrices divinas.
\vs p034 5:6 \pc Como seres individuales, no poseéis personalmente una parte o entidad separada del espíritu del hijo\hyp{}padre creador o del espíritu materno creativo. Estos espíritus servidores no se ponen en contacto con los centros pensantes de las mentes particulares ni moran en ellos como lo hacen los mentores misteriosos. Los modeladores del pensamiento constituyen individualizaciones concretas de la realidad prepersonal del Padre Universal, que realmente residen en la mente mortal como parte misma de ella y siempre obran en perfecta armonía con los espíritus, vinculados entre sí, del hijo creador y el espíritu creativo.
\vs p034 5:7 La presencia del espíritu santo de la hija del universo del Espíritu Infinito, del espíritu de la verdad del hijo del universo del Hijo Eterno y del espíritu modelador del Padre del Paraíso con un mortal evolutivo o en él indica que existe correspondencia de dones y ministerio espirituales, y lo capacita para comprender conscientemente el hecho de fe de su filiación con Dios.
\usection{6. EL ESPÍRITU EN EL HOMBRE}
\vs p034 6:1 Con el avance de la evolución de un planeta habitado y la consiguiente espiritualización de sus habitantes, esos seres personales, que han alcanzado un cierto grado de madurez, se hacen receptivos a otras influencias espirituales. A medida que los mortales progresan en dominio mental y en percepción espiritual, la acción de esta asistencia espiritual múltiple se armoniza cada vez más; aumenta de forma creciente en unión con el supraministerio de la Trinidad del Paraíso.
\vs p034 6:2 Aunque en su manifestación la Divinidad puede ser plural, en la experiencia humana la Deidad es singular, siempre \bibemph{una}. El ministerio espiritual tampoco es plural en la experiencia humana. Sea cual fuese la pluralidad de su origen, todas las influencias espirituales actúan como una sola. En realidad son una, pues el ministerio espiritual del Dios Séptuplo se da en y para las criaturas del gran universo; y a medida que las criaturas crecen en apreciación, y receptividad, de tal ministerio unificador, este, en la experiencia de dichas criaturas, se convierte en el ministerio del Dios Supremo.
\vs p034 6:3 \pc Mediante una larga serie de pasos, el espíritu divino desciende desde las alturas de la gloria eterna para encontrarse contigo tal cual eres y dondequiera que estés y entonces, en alianza de fe, abraza tiernamente tu alma de origen mortal y emprende el cierto y seguro camino de vuelta sobre esos pasos que descendieron, sin detenerse nunca, hasta que tu alma evolutiva sea firmemente exaltada hasta las alturas mismas de la dicha desde las que el espíritu partió originalmente en su misión de misericordia y ministerio.
\vs p034 6:4 Las fuerzas espirituales buscan y alcanzan infaliblemente sus propios niveles originales. Habiendo salido del Eterno, es seguro que regresarán allí, llevando consigo a todos aquellos hijos del tiempo y del espacio que han aceptado la guía y la enseñanza del modelador interior, a aquellos que verdaderamente han “nacido del Espíritu”, a los hijos de Dios por la fe.
\vs p034 6:5 \pc El espíritu divino es fuente de incesante ministerio y estímulo para los hijos de los hombres. Vuestra capacidad y vuestros logros son “conforme a su misericordia, por la renovación del Espíritu”. La vida espiritual, al igual que la energía física, se consume. El esfuerzo espiritual conduce a un cierto agotamiento espiritual. Toda la experiencia del camino ascendente es tan real como espiritual; por ello está en verdad escrito: “Es el Espíritu el que vivifica”. “El Espíritu da vida”.
\vs p034 6:6 Los supuestos teóricos de incluso las más elevadas doctrinas religiosas se muestran inertes; no son capaces de transformar el carácter humano ni de regir la conducta de los mortales. Lo que el mundo de hoy necesita es lo que vuestro antiguo maestro declaró: “No en palabras solamente sino también en poder y en el espíritu santo”. La semilla de la verdad teórica no tiene vida, los conceptos morales más elevados carecen de validez, a menos que, y hasta que, el espíritu divino no insufle su aliento sobre las formas de la verdad y vivifique los códigos de la rectitud.
\vs p034 6:7 Aquellos que han recibido y reconocido la presencia interior de Dios han nacido del espíritu. “Sois templos de Dios, y el espíritu de Dios está en vosotros”. No es suficiente con que este espíritu se haya derramado sobre vosotros; el espíritu divino debe regir todas las etapas de la experiencia humana.
\vs p034 6:8 La presencia del espíritu divino, el agua de la vida, previene de la sed abrasadora, que embarga al mortal insatisfecho, y del ansia indescriptible de la mente humana no espiritualizada. Aquellos a los que el espíritu impulsa “no tendrán sed jamás, porque el agua espiritual será en ellos una fuente que sacia y brota para vida eterna”. Estas almas así saciadas de modo divino por esta agua prácticamente no dependen del entorno material para hallar gozo en la vida o satisfacción en la existencia terrenal. Están iluminadas y renovadas espiritualmente, fortalecidas y dotadas moralmente.
\vs p034 6:9 \pc En todo mortal existe una doble naturaleza: la herencia de tendencias animales y el estímulo hacia lo superior del que está dotado espiritualmente. Durante vuestra breve vida en Urantia, estos dos impulsos disímiles y opuestos raramente podrán reconciliarse por completo; difícilmente se pueden armonizar ni unificar; si bien, a lo largo de vuestra vida, las influencias espirituales, obrando en combinación, jamás cesarán de daros asistencia para someter la carne cada vez más a la guía del espíritu. Y aunque debéis vivir vuestra vida material, aunque no podáis escapar al cuerpo y sus necesidades, no obstante, en propósito e ideales, cada vez os sentiréis más dotados de la capacidad de subordinar la naturaleza animal a la supremacía del Espíritu. En verdad existe en vosotros una afiliación de fuerzas espirituales, una coalición de poderes divinos, cuyo único propósito consiste en liberaros finalmente de la atadura a lo material y de los impedimentos finitos.
\vs p034 6:10 El propósito de todo este ministerio es “que podáis ser fortalecidos con poder mediante Su espíritu en el hombre interior”. Y todo esto no es más que los primeros pasos para poder lograr definitivamente la perfección de la fe y del servicio, ese estado en el que estaréis “llenos de la entera plenitud de Dios”, “porque todos los que son guiados por el espíritu de Dios son los hijos de Dios”.
\vs p034 6:11 \pc El espíritu nunca \bibemph{fuerza,} solo guía. Si aprendéis con voluntariedad, si queréis conseguir niveles espirituales y alcanzar las alturas divinas, si sinceramente deseáis lograr el objetivo eterno, entonces el espíritu divino os conducirá con delicadeza y cariño por el camino de la filiación y del progreso espiritual. Cada paso que deis debéis darlo de buena voluntad y cooperando con inteligencia y alegría. El predominio del espíritu nunca se distorsiona por la coacción ni se pone en peligro por la imposición.
\vs p034 6:12 Y cuando se acepta con libertad e inteligencia esa vida bajo la guía del espíritu, se desarrolla de forma gradual, en la mente humana, una inequívoca conciencia de contacto divino y una certitud de comunión espiritual; tarde o temprano “el Espíritu mismo se une a tu espíritu (el modelador) para dar testimonio de que eres hijo de Dios”. Ya tu propio modelador del pensamiento te ha hablado de tu relación de parentesco con Dios, por ello, las escrituras reflejan que el Espíritu da testimonio no \bibemph{a} tu espíritu sino en unión \bibemph{con} él.
\vs p034 6:13 La conciencia del predominio del espíritu en una vida humana viene acompañada de una creciente manifestación de características espirituales en la respuesta que, bajo su dirección, da el mortal ante la vida “porque los frutos del espíritu son amor, gozo, paz, mansedumbre, dulzura, bondad, fe, humildad, y templanza”. Estos mortales, divinamente iluminados, que siguen la guía del espíritu, incluso cuando recorren los humildes senderos del trabajo agotador y con lealtad humana cumplen con las obligaciones de sus deberes terrenales, han comenzado ya a percibir las luces de la vida eterna que brillan en las lejanas orillas de otro mundo; ya han comenzado a comprender la realidad de esa verdad inspiradora y reconfortante, “El reino de Dios no es comida ni bebida, sino justicia, paz, y gozo en el espíritu santo”. Y a través de cada prueba, frente a cada penuria, las almas nacidas del espíritu se sostienen en esa esperanza que trasciende todos los temores, porque el amor de Dios se esparce a todos los corazones mediante la presencia del espíritu divino.
\usection{7. EL ESPÍRITU Y LA CARNE}
\vs p034 7:1 La carne, intrínseca a las razas de origen animal, no da de manera natural los frutos del espíritu divino. Cuando se eleva la naturaleza de los mortales con la aportación de la naturaleza de los hijos materiales de Dios, como lo hicieron en cierta medida las razas de Urantia con la dádiva de Adán, entonces el camino se allana para que el espíritu de la verdad coopere con el modelador interior con el fin de producir en vuestro carácter la hermosa cosecha de los frutos del espíritu. Si no rechazáis a este espíritu, y aunque se necesite la eternidad para cumplir tal cometido, “él os guiará a toda la verdad”.
\vs p034 7:2 Los mortales evolutivos, habitantes de los mundos que progresan espiritualmente con normalidad, no experimentan los graves conflictos entre el espíritu y la carne que caracterizan a las razas actuales en Urantia. Pero incluso en los planetas más modélicos, el hombre preadánico debe realizar un decidido esfuerzo para ascender desde el plano existencial puramente animal hasta alcanzar niveles consecutivos de contenidos intelectuales crecientes y de valores espirituales superiores.
\vs p034 7:3 En un mundo normal, los mortales no experimentan esa guerra constante entre su naturaleza material y espiritual. Se enfrentan a la necesidad de elevarse desde los niveles de la existencia animal hasta los planos superiores de la vida espiritual, pero este movimiento ascendente parece más un recorrido educativo si se le compara con los intensos conflictos que experimentan los mortales de Urantia ante naturalezas tan divergentes como la material y la espiritual.
\vs p034 7:4 \pc Los pueblos de Urantia sufren el resultado de la falta de ayuda en nuestra labor de lograr un desarrollo espiritual planetario progresivo. Por un lado, la sublevación de Caligastia provocó una confusión a nivel mundial y despojó a todas las generaciones que siguieron de la asistencia moral que una sociedad bien organizada hubiese podido ofrecer; por otro, la transgresión de Adán, incluso de consecuencias más catastróficas, desposeyó a las razas de ese tipo superior de la naturaleza física que hubiese estado más en consonancia con las aspiraciones espirituales.
\vs p034 7:5 Los mortales de Urantia se ven obligados a experimentar esa manifiesta lucha entre el espíritu y la carne debido a que sus antepasados remotos no fueron completamente adanizados durante la misión de Adán. Según el plan divino, las razas mortales de Urantia deberían haber gozado de forma natural de una disposición física de mayor receptividad al espíritu.
\vs p034 7:6 \pc A pesar de este doble desastre para la naturaleza del hombre y para su entorno, los mortales de hoy en día experimentarían menos este patente enfrentamiento entre la carne y el espíritu si se dispusieran a entrar en el reino del espíritu, donde los hijos de Dios por la fe disfrutan de un relativo rescate de la servidumbre y las ataduras de la carne, dedicándose de todo corazón, en un servicio edificante y libertador, a cumplir la voluntad del Padre de los cielos. Jesús mostró a la humanidad el nuevo modo de vivir la vida mortal para que los seres humanos pudiesen, en gran manera, evitar las funestas consecuencias de la rebelión de Caligastia y compensar con mayor efectividad las privaciones resultantes de la transgresión de Adán. “El espíritu de vida de Cristo Jesús nos ha librado de la ley de la vida animal y de las tentaciones del mal y el pecado”. “Esta es la victoria que vence la carne, vuestra fe”.
\vs p034 7:7 Esos hombres y mujeres que conocen a Dios y que han nacido del espíritu no experimentan más conflicto con sus naturalezas mortales que los habitantes de los mundos, de planetas más normales que nunca se vieron contaminados por el pecado ni afectados por la rebelión. Los hijos de la fe actúan en niveles intelectuales y viven en planos espirituales que los hacen sobreponerse a los conflictos que pueden ocasionar los deseos físicos irrefrenables o innaturales de otras personas sin fe. Los deseos propios de los seres animales y los apetitos e impulsos ordinarios de la naturaleza física no entran en conflicto ni con los mayores logros espirituales, salvo en las mentes de personas ignorantes, mal instruidas o lamentablemente escrupulosas en extremo.
\vs p034 7:8 \pc Habiendo iniciado el camino de la vida eterna, habiendo aceptado la misión y recibido las instrucciones para avanzar, no temas los riesgos del olvido humano ni la veleidad del mortal, no te preocupes por el temor al fracaso ni te desconcierte la confusión, no vaciles ni cuestiones tu estatus ni tu posición, porque en esas horas oscuras, en cada encrucijada en la que te encuentres en tu lucha por seguir adelante, el espíritu de la verdad siempre hablará, diciendo: “Este es el camino”.
\vsetoff
\vs p034 7:9 [Exposición de un mensajero poderoso que realiza su servicio temporalmente en Urantia.]
