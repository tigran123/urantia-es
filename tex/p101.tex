\upaper{101}{Naturaleza real de la religión}
\author{Melquisedec}
\vs p101 0:1 La religión, como experiencia humana, comprende desde la esclavitud al temor primitivo que siente el hombre no civilizado, en su camino evolutivo, hasta la libertad de fe, sublime y magnífica, de esos mortales civilizados, espléndidamente conscientes de su filiación con el Dios eterno.
\vs p101 0:2 La religión antecede a la ética y a la moral superior de la progresiva evolución social. Pero la religión, como tal, no es meramente un avance moral, a pesar de que las manifestaciones exteriores y sociales de la religión estén poderosamente influidas por el empuje ético y moral de la sociedad humana. La religión es siempre el estímulo de la naturaleza evolutiva del hombre, pero no es el secreto de ese desarrollo.
\vs p101 0:3 La religión ---la convicción\hyp{}fe de la persona--- siempre puede triunfar sobre la lógica, superficialmente contradictoria, de la desesperación nacida en la mente material del no creyente. Hay realmente una voz interior, verdadera y genuina, esa “luz verdadera que alumbra a todo hombre que viene a este mundo”. Y esta guía espiritual se distingue del impulso ético de la conciencia humana. El sentimiento de esa certeza religiosa es más que una sensación de emoción. La convicción de la religión trasciende la razón de la mente, incluso la lógica de la filosofía. La religión \bibemph{es} fe, confianza y certeza.
\usection{1. LA VERDADERA RELIGIÓN}
\vs p101 1:1 La verdadera religión no es un sistema de creencias filosóficas, que se puedan razonar y corroborar mediante pruebas naturales, ni tampoco ninguna vivencia, formidable y mística, de inefables sentimientos de éxtasis de la que tan solo pueden gozar los devotos románticos del misticismo. La religión no es fruto de la razón, pero percibida desde dentro, es por completo razonable. La religión no proviene de la lógica de la filosofía humana, pero como experiencia humana es totalmente lógica. La religión es la experiencia de la divinidad en la conciencia del ser moral de origen evolutivo; representa la verdadera experiencia de las realidades eternas en el tiempo, la realización de las satisfacciones espirituales mientras se está en la carne.
\vs p101 1:2 \pc El modelador del pensamiento no posee mecanismos especiales por los que pueda conseguir expresarse a sí mismo; no existe ninguna capacidad religiosa mística para la recepción o manifestación de las emociones religiosas. Estas experiencias están disponibles mediante un mecanismo, naturalmente establecido, de la mente mortal. Y ahí radica la explicación de las dificultades del modelador para ponerse en comunicación directa con la mente material, en la que reside de forma permanente.
\vs p101 1:3 El espíritu divino se pone en contacto con el hombre mortal, no mediante sentimientos o emociones, sino en el ámbito del pensamiento más elevado y más espiritualizado. Son vuestros \bibemph{pensamientos,} no vuestros sentimientos, los que os llevan hacia Dios. La naturaleza divina se puede advertir solo con los ojos de la mente. Si bien, la mente que verdaderamente percibe a Dios, que escucha al modelador interior, es la mente pura. “Sin santidad nadie verá al Señor”. Cualquier comunión interna y espiritual de esa índole se denomina percepción espiritual. Estas experiencias religiosas resultan de la impresión producida en la mente del hombre por la acción conjunta del modelador y del espíritu de la verdad, a medida que estos obran en los ideales, percepción y afanes espirituales de los hijos evolutivos de Dios.
\vs p101 1:4 La religión vive y cobra fuerza, entonces, no por la vista ni por los sentimientos, sino más bien por la fe y la percepción. No consiste en el descubrimiento de nuevos hechos ni en el logro de vivencias singulares, sino más bien en el descubrimiento de nuevos \bibemph{contenidos} espirituales en hechos ya bien conocidos por la humanidad. La experiencia religiosa más elevada no depende de previas creencias, tradición ni autoridad; la religión tampoco es hija de sentimientos sublimes ni de emociones meramente místicas. Se trata, más bien, de una experiencia, sumamente profunda y real, de comunión espiritual con las influencias espirituales que residen en la mente humana y, por lo que respecta a su definición en términos de psicología, es sencillamente la vivencia de experimentar que la realidad de creer en Dios constituye la realidad de esa experiencia puramente personal.
\vs p101 1:5 \pc Aunque la religión no sea el resultado de los supuestos racionalistas de una cosmología material, consiste, no obstante, en la creación de una percepción interior, enteramente racional, que tiene su origen en la experiencia y en la mente del hombre. La religión no nace de la meditación mística ni de la contemplación aislada, aunque sea, por siempre, más o menos misteriosa y siempre indescriptible e inexplicable en términos de la pura razón intelectual y de la lógica filosófica. El germen de la verdadera religión nace en el ámbito de la conciencia moral del hombre, y se revela en el crecimiento de la percepción espiritual del hombre, en esa facultad de la persona humana conseguida como consecuencia de la presencia del modelador del pensamiento, que revela a Dios en la mente mortal sedienta de él.
\vs p101 1:6 La fe une la percepción moral a la percepción consciente de los valores, y el sentido de deber evolutivo, preexistente, completa la ascendencia de la verdadera religión. Con el tiempo, la experiencia de la religión da lugar a la conciencia cierta de Dios y a la incontestable seguridad de la supervivencia de la persona que cree.
\vs p101 1:7 Se observa, pues, que los anhelos religiosos y los impulsos espirituales no son de tal índole que sencillamente lleven a los hombres a \bibemph{querer} creer en Dios, sino más bien de una naturaleza y poder que infunden profundamente en ellos la convicción de que \bibemph{deberían} creer en Dios. El sentido del deber evolutivo y las obligaciones contraídas como consecuencia de la iluminación que brinda la revelación producen una impresión tan profunda en la naturaleza moral del hombre, que este acaba por alcanzar esa condición de mente y esa actitud del alma por las que llega a la conclusión de que \bibemph{es justo creer en Dios}. La sabiduría de mayor elevación y suprafilosófica de personas de tal iluminación y tan disciplinadas les indica, en última instancia, que dudar de Dios o desconfiar de su bondad equivaldría a cuestionar \bibemph{lo más real y profundo} que hay en la mente y en el alma: el modelador divino.
\usection{2. EL HECHO DE LA RELIGIÓN}
\vs p101 2:1 El hecho de la religión consiste enteramente en la experiencia religiosa de seres humanos corrientes, dotados de razón. Y este es el único sentido en el que la religión se puede considerar científica o incluso psicológica. La prueba de que la revelación es revelación radica en este mismo hecho de la experiencia humana: el hecho de que la revelación ciertamente sintetiza las ciencias aparentemente discrepantes de la naturaleza y de la teología de la religión en una filosofía del universo coherente y lógica, ofreciendo una explicación armonizada e ininterrumpida tanto de la ciencia como de la religión y creando consecuentemente una armonía de mente y una satisfacción de espíritu que responde, en la experiencia humana, a aquellas interrogantes de la mente mortal que anhela saber \bibemph{cómo} el Infinito hace su voluntad y lleva a cabo sus designios en la materia, en conjunción con las mentes y en cooperación con el espíritu.
\vs p101 2:2 La razón es el método de la ciencia; la fe es el método de la religión; la lógica es el sistema resolutivo de la filosofía. La revelación compensa la ausencia de perspectiva morontial al proporcionar un modo de alcanzar la unidad en la comprensión de la realidad y de las relaciones de la materia y el espíritu a través de la mediación de la mente. Y la verdadera revelación no convierte en innatural a la ciencia, en irrazonable a la religión ni en ilógica a la filosofía.
\vs p101 2:3 La razón, por medio del estudio de la ciencia, puede llevar de vuelta, a través de la naturaleza, a una Primera Causa, pero se precisa la fe religiosa para transformar la Primera Causa de la ciencia en un Dios de salvación; y se necesita además la revelación para validar tal fe, tal percepción espiritual.
\vs p101 2:4 Existen dos razones primordiales para creer en un Dios que propicia la supraexperiencia humana:
\vs p101 2:5 \li{1.}La experiencia humana, la garantía personal, la esperanza y la confianza puestas en marcha por el modelador del pensamiento interior, y de las que queda, de alguna manera, constancia.
\vs p101 2:6 \li{2.}La revelación de la verdad, ya sea mediante el ministerio personal directo del espíritu de la verdad, la dádiva para el mundo de los hijos divinos o las revelaciones de la palabra escrita.
\vs p101 2:7 \pc La ciencia finaliza su búsqueda razonada en la hipótesis de una Causa Primera. La religión no se detiene en su viaje de fe hasta tener la certeza de la existencia de un Dios de salvación. El estudio analítico de la ciencia propone por lógica la realidad y la existencia de un Absoluto. La religión cree incondicionalmente en la existencia y la realidad de un Dios que propicia la supervivencia del ser personal. Lo que la metafísica no logra hacer de ningún modo, e incluso la filosofía solo lo logra en parte, la revelación lo consigue; esto es, afirmar que esta Causa Primera de la ciencia y el Dios de salvación de la religión son \bibemph{una única y misma Deidad}.
\vs p101 2:8 \pc La razón es la prueba de la ciencia; la fe, de la religión; la lógica, de la filosofía, pero la revelación se valida solamente mediante la \bibemph{experiencia} humana. La ciencia genera conocimiento; la religión, felicidad; la filosofía, unidad; la revelación confirma la armonía vivencial de esta aproximación trina a la realidad universal.
\vs p101 2:9 La contemplación de la naturaleza puede tan solo revelar a un Dios de la naturaleza, a un Dios del movimiento. La naturaleza únicamente muestra la materia, el movimiento y la animación: la vida. La materia más la energía bajo ciertas condiciones se manifiesta como formas vivas, pero, aunque la vida natural sea pues relativamente continua como fenómeno, es completamente transitoria en cuanto a individualidades. La naturaleza no sirve de base para creer por lógica en la supervivencia de la persona humana. El hombre religioso que halla a Dios en la naturaleza ya ha encontrado primeramente, en su propia alma, a este mismo Dios personal.
\vs p101 2:10 \pc La fe revela a Dios en el alma. La revelación, sustituta de la percepción morontial en un mundo evolutivo, permite al hombre ver, en la naturaleza, al mismo Dios que la fe le muestra en el alma. De ese modo, la revelación consigue superar con éxito la brecha entre lo material y lo espiritual, incluso entre la criatura y el Creador, entre el hombre y Dios.
\vs p101 2:11 La contemplación de la naturaleza apunta lógicamente a la existencia de una guía inteligente, incluso de una supervisión activa, pero no revela adecuadamente a un Dios personal. Por otra parte, la naturaleza no desvela nada que impida que el universo se pueda considerar como la obra del Dios de la religión. No se puede encontrar a Dios a través de la naturaleza por sí sola, pero una vez que el hombre lo ha encontrado de alguna otra manera, su estudio se vuelve plenamente compatible con una interpretación más elevada y espiritual del universo.
\vs p101 2:12 \pc La revelación como fenómeno de una época es periódica; como experiencia personal humana, es continua. La divinidad obra en la persona humana por medio del modelador, o don del Padre; del espíritu de la verdad del hijo; y del espíritu santo del espíritu del universo; si bien, estas tres dotaciones supramortales se unifican en el desarrollo vivencial humano a través del ministerio del Supremo.
\vs p101 2:13 La verdadera religión es una visión de la realidad, la fe que nace de la conciencia moral, y no un mero asentimiento intelectual a cualquier conjunto de doctrinas dogmáticas. La verdadera religión consiste en la experiencia de “que El Espíritu mismo da testimonio a nuestro espíritu, de que somos los hijos de Dios”. La religión no radica en premisas teológicas, sino en la percepción espiritual y en la sublimidad de la confianza del alma.
\vs p101 2:14 Vuestra naturaleza más profunda ---el modelador divino--- crea dentro de vosotros hambre y sed de rectitud, un cierto anhelo de perfección divina. La religión es el acto de fe por el que se reconoce este impulso interior por lograr lo divino; y, así, se produce esa confianza y certeza del alma de la que os hacéis conscientes como vía de salvación: el modo de supervivencia del ser personal y de todos esos valores que habéis llegado a considerar como verdaderos y buenos.
\vs p101 2:15 \pc Llegar a comprender la religión no ha dependido nunca, ni dependerá, de un gran conocimiento ni de una lógica brillante. Es percepción espiritual, y esa es justo la razón por la que algunos de los más grandes maestros religiosos del mundo, incluso los profetas, tenían a menudo tan poca sabiduría del mundo. La fe religiosa está disponible tanto para la persona cultivada como la que no lo es.
\vs p101 2:16 Por siempre, la religión debe ser su propio crítico y juez; desde fuera no puede observase ni, mucho menos, comprenderse. La única certeza de un Dios personal consiste en la propia percepción de vuestras creencias y experiencias de las cosas espirituales. Todos esos semejantes vuestros que han tenido alguna experiencia similar no precisan de argumentaciones sobre la persona o la realidad de Dios; si bien, no hay argumento posible que pueda convencer verdaderamente a esos otros que carecen de tal certidumbre de Dios.
\vs p101 2:17 La psicología puede de hecho intentar estudiar el fenómeno de las respuestas religiosas al entorno social, pero jamás puede albergar la esperanza de adentrarse en los motivos y funcionamientos internos y auténticos de la religión. Tan solo la teología, el ámbito de la fe y el método de la revelación, puede proporcionar algún tipo de explicación inteligente acerca de la naturaleza y el contenido de la experiencia religiosa.
\usection{3. CARACTERÍSTICAS DE LA RELIGIÓN}
\vs p101 3:1 La religión es tan vital que subsiste en ausencia de la erudición. Vive a pesar de su contaminación con cosmologías y filosofías erróneas; sobrevive incluso a la confusión de la metafísica. A través de todas las vicisitudes históricas por las que la religión ha pasado, siempre perdura aquello que es indispensable para el progreso y la supervivencia del ser humano: la conciencia ética y la concienciación moral.
\vs p101 3:2 La fe\hyp{}percepción, o intuición espiritual, es el don de la mente cósmica en su vinculación con el modelador del pensamiento, que es el don del Padre al hombre. La razón espiritual, la inteligencia del alma, es el don del espíritu santo, que es el don del espíritu creativo al hombre. La filosofía espiritual, la sabiduría de las realidades espirituales, es el don del espíritu de la verdad, que es el don combinado de los hijos de gracia a los hijos de los hombres. Y la coordinación y correlación de estos dones espirituales hacen del hombre un ser personal espiritual con un destino potencial.
\vs p101 3:3 Este mismo ser personal espiritual, en su forma primitiva y embrionaria, que es posesión del modelador, sobrevive a la muerte natural de la carne. Mediante el camino vivo facilitado por los hijos divinos, esta entidad compuesta, de origen espiritual en conjunción con las experiencias humanas, está capacitada para sobrevivir (con la custodia del modelador) a la disolución del yo material de la mente y de la materia cuando tal unión transitoria de lo material y lo espiritual se disocia debido al cese de moción vital.
\vs p101 3:4 Mediante la fe religiosa, el alma del hombre se revela a sí misma, y demuestra la divinidad potencial de su nueva naturaleza por la forma característica en la que alienta al ser personal humano a reaccionar ante determinadas situaciones intelectuales y sociales difíciles y arduas. La genuina fe espiritual (la verdadera concienciación moral) se revela en que:
\vs p101 3:5 \li{1.}Causa el progreso de la ética y de la moral, pese a las tendencias animales innatas y desfavorables.
\vs p101 3:6 \li{2.}Produce una confianza sublime en la bondad de Dios, incluso frente a la amarga decepción y a la aplastante derrota.
\vs p101 3:7 \li{3.}Genera un profundo valor y confianza, a pesar de las adversidades naturales y de las calamidades físicas.
\vs p101 3:8 \li{4.}Concede una serenidad inexplicable y un sosiego continuo, pese a enfermedades desconcertantes e incluso padecimientos físicos agudos.
\vs p101 3:9 \li{5.}Mantiene un misterioso equilibrio y una compostura personal ante el maltrato y las injusticias más flagrantes.
\vs p101 3:10 \li{6.}Preserva la confianza divina en la victoria última, al margen de las crueldades de un azar supuestamente ciego y de la manifiesta indiferencia absoluta de las fuerzas naturales hacia el bien humano.
\vs p101 3:11 \li{7.}Persevera en la creencia en Dios de forma inquebrantable, pese a las demostraciones contrarias de la lógica, y se resiste con éxito a todos los otros sofismas intelectuales.
\vs p101 3:12 \li{8.}Continúa manifestando una fe impertérrita en la supervivencia del alma, pese a las enseñanzas engañosas de la falsa ciencia y a los convincentes delirios de una filosofía irracional.
\vs p101 3:13 \li{9.}Vive y triunfa con independencia de la sobrecarga aplastante por parte de las complicadas e incompletas civilizaciones de los tiempos modernos.
\vs p101 3:14 \li{10.}Contribuye a la supervivencia continuada del altruismo, a pesar del egoísmo humano, de los antagonismos sociales, de las codicias industriales y de los desajustes políticos.
\vs p101 3:15 \li{11.}Se adhiere firmemente a la sublime creencia en la unidad del universo y en la guía divina, sin que la presencia desconcertante del mal y del pecado importe.
\vs p101 3:16 \li{12.}Continúa sin descanso adorando a Dios pese a todo y a cualquier cosa. Se atreve a declarar: “Aunque él me mate, le serviré”.
\vs p101 3:17 \pc Sabemos, entonces, mediante tres fenómenos, que el hombre tiene un espíritu o unos espíritus divinos que habitan en él: primero, por la experiencia personal ---la fe religiosa---; segundo, por la revelación ---personal y planetaria---; y, tercero, por la asombrosa manifestación de reacciones tan extraordinarias e innaturales a su entorno material, como se demuestra por la precedente lista de doce modos de actuar de orden espiritual, en presencia de situaciones reales y difíciles de la auténtica existencia humana. Y todavía hay algunas más.
\vs p101 3:18 Y es justo este modo de actuar, vital y vigoroso, de la fe en el ámbito de la religión el que da derecho al hombre mortal a afirmar su posesión personal y la realidad espiritual de ese supremo don de la naturaleza humana: la experiencia religiosa.
\usection{4. LIMITACIONES DE LA REVELACIÓN}
\vs p101 4:1 Puesto que vuestro mundo ignora generalmente sus comienzos, incluso el de aquellos de origen físico, nos ha parecido en ocasiones sensato proporcionar alguna instrucción en cosmología. Y, siempre, esto ha ocasionado problemas para el futuro. En gran medida, las leyes de la revelación representan un obstáculo porque prohíben impartir conocimiento no ganado o prematuro. Cualquier cosmología que se presente como parte de la religión revelada está destinada a quedar superada en muy escaso tiempo. Por lo tanto, los futuros estudiosos de tal revelación, al descubrir supuestos errores en las cosmologías que en ella se adjuntan, se sienten tentados a descartar cualquier componente de auténtica verdad religiosa que esta pueda contener.
\vs p101 4:2 La humanidad debería comprender que nosotros, quienes participamos en la revelación de la verdad, estamos estrictamente restringidos por las instrucciones de nuestros superiores. No tenemos libertad para adelantar unos descubrimientos científicos que se producirán en mil años. Los reveladores deben actuar conforme a las directrices que forman parte del mandato de la revelación. No vemos manera alguna de superar esta dificultad, ni ahora ni en ningún momento futuro. Sabemos muy bien que, mientras que los hechos históricos y las verdades religiosas de esta serie de exposiciones reveladoras constarán en los registros de las eras venideras, en unos pocos años, muchas de nuestras afirmaciones sobre las ciencias físicas precisarán revisión como resultado de otros desarrollos científicos y de nuevos descubrimientos. Prevemos incluso ahora estos nuevos desarrollos, pero se nos prohíbe revelar e incluir, en estos informes, hechos que aún el hombre no ha descubierto. Que quede claro que las revelaciones no son necesariamente inspiradas. La cosmología de estas revelaciones \bibemph{no es inspirada}. Está limitada por nuestro permiso para coordinar y organizar el conocimiento actual. Aunque la percepción divina o espiritual sea un don, \bibemph{la sabiduría humana debe evolucionar}.
\vs p101 4:3 \pc La verdad es siempre una revelación: es autorrevelación cuando surge como resultado de la tarea del modelador interior; es revelación de una época cuando se presenta mediante la labor de algún otro actuante, grupo o ser personal celestial.
\vs p101 4:4 En última instancia, la religión debe juzgarse por sus frutos, de acuerdo con la manera y en la medida en la que muestra su propia excelencia, inherente y divina.
\vs p101 4:5 \pc La verdad no puede ser sino relativamente inspirada, aun cuando la revelación sea siempre un fenómeno espiritual. Mientras que las afirmaciones en referencia a la cosmología nunca son inspiradas, tales revelaciones son de inmenso valor porque, al menos de forma transitoria, clarifican el conocimiento mediante:
\vs p101 4:6 \li{1.}La reducción de la confusión al eliminar el error de forma fidedigna.
\vs p101 4:7 \li{2.}La coordinación de hechos y observaciones conocidos o a punto de conocerse.
\vs p101 4:8 \li{3.}La restauración de fragmentos importantes de conocimiento perdido referentes a acontecimientos decisivos del pasado remoto.
\vs p101 4:9 \li{4.}El suministro de información que llena espacios vacíos vitales en el conocimiento adquirido de otro modo.
\vs p101 4:10 \li{5.}La exposición de datos cósmicos de tal manera que ilumine las enseñanzas espirituales contenidas en la revelación a la que se adjuntan.
\usection{5. LA RELIGIÓN AMPLIADA POR LA REVELACIÓN}
\vs p101 5:1 La revelación es un modo de ahorrar extensos períodos en la necesaria labor de ordenar y separar los errores de la evolución de las verdades adquiridas de forma espiritual.
\vs p101 5:2 La ciencia se ocupa de los \bibemph{hechos;} la religión, solamente de los \bibemph{valores}. A través de una filosofía bien fundamentada, la mente procura unir los contenidos tanto de los hechos como de los valores, con lo que llega al concepto de la \bibemph{realidad} total. Recordad que la ciencia es el ámbito del conocimiento; la filosofía, el de la sabiduría; y, la religión, el de la experiencia de la fe. Si bien, en cuanto a su manifestación, la religión presenta dos etapas:
\vs p101 5:3 \li{1.}La religión evolutiva. La experiencia de la adoración primitiva, la religión derivada de la mente.
\vs p101 5:4 \li{2.}La religión revelada. La actitud universal que se deriva del espíritu; la seguridad y la creencia en la conservación de las realidades eternas, la supervivencia del ser personal y la consecución futura de la Deidad cósmica, cuyo propósito ha hecho todo esto posible. Es parte del plan del universo que, antes o después, la religión evolutiva esté llamada a recibir la ampliación espiritual de la revelación.
\vs p101 5:5 \pc Tanto la ciencia como la religión parten de la suposición de ciertos fundamentos, generalmente aceptados, para realizar deducciones lógicas. Asimismo, la filosofía debe comenzar su andadura basándose en la suposición de la realidad de tres cosas:
\vs p101 5:6 \li{1.}El cuerpo material.
\vs p101 5:7 \li{2.}La parte supramaterial del ser humano: el alma o incluso el espíritu interior.
\vs p101 5:8 \li{3.}La mente humana, el mecanismo para establecer la intercomunicación y la correlación entre el espíritu y la materia, entre lo material y lo espiritual.
\vs p101 5:9 \pc Los científicos reúnen hechos, los filósofos coordinan ideas, mientras que los profetas enaltecen ideales. El sentimiento y la emoción concurren invariablemente en la religión, pero no son la religión. La religión puede ser el sentimiento de una experiencia, pero difícilmente es la experiencia de un sentimiento. Ni la lógica (la racionalización) ni la emoción (el sentimiento) son parte esencial de la experiencia religiosa, aunque ambas puedan estar vinculadas indistintamente con el ejercicio de la fe al avanzar la percepción espiritual de la realidad, todo conforme al estatus y a la tendencia temperamental de la mente de la persona.
\vs p101 5:10 La religión evolutiva es consecuencia del don del asistente de la mente del universo local, encargado de crear e impulsar la característica propensión a la adoración del hombre evolutivo. Estas religiones primitivas se preocupan directamente por la ética y la moral: el sentido del \bibemph{deber} humano. Se basan en la certitud de la conciencia y dan lugar a la estabilización de civilizaciones relativamente éticas.
\vs p101 5:11 Las religiones reveladas personales están auspiciadas por los tres espíritus otorgados de gracia, que representan a las tres personas de la Trinidad del Paraíso y se ocupan en especial de la expansión de la \bibemph{verdad}. La religión evolutiva sensibiliza a la persona a la idea del deber personal; la religión revelada da creciente preponderancia al amor, la regla de oro.
\vs p101 5:12 La religión evolutiva se apoya por entero en la fe. La revelación conlleva además la garantía del relato ampliado de las verdades de la divinidad y de la realidad y del testimonio, incluso más valioso, de la experiencia real, acumulado como consecuencia de la actuación práctica y en unión de la fe evolutiva y de la verdad revelada. Tal actuación unida de la fe humana y de la verdad divina supone estar en posesión de un carácter bien encaminado hacia la auténtica adquisición de un ser personal morontial.
\vs p101 5:13 \pc La religión evolutiva proporciona solamente la certeza de la fe y la confirmación de la conciencia; la religión revelada aporta la certeza de la fe más la verdad de una experiencia viva en las realidades de la revelación. El tercer paso en el ámbito religioso, o la tercera etapa de la experiencia religiosa, tiene que ver con el estado morontial, con una comprensión más sólida de la mota. Al progresar morontialmente, las verdades de la religión revelada se expanden de forma creciente; cada vez más, conoceréis la verdad de los valores supremos, de las bondades divinas, de las relaciones universales, de las realidades eternas y de los destinos últimos.
\vs p101 5:14 Crecientemente, a lo largo del progreso morontial, la certeza de la verdad va reemplazando a la de la fe. Cuando finalmente os incorporéis al auténtico mundo espiritual, entonces, las certezas de la pura percepción espiritual obrarán en lugar de la fe y de la verdad o, más bien, en conjunción, y superposición, con estos antiguos modos personales de certidumbre.
\usection{6. EXPERIENCIA PROGRESIVA DE LA RELIGIÓN}
\vs p101 6:1 La etapa morontial de la religión revelada está relacionada con la \bibemph{experiencia de la supervivencia,} y su gran impulso consiste en el logro de la perfección espiritual. También hay un estímulo más elevado a la adoración unido a un imperioso llamamiento a un mayor servicio ético. La percepción morontial supone una creciente conciencia del Séptuplo, del Supremo e incluso el Último.
\vs p101 6:2 Durante cualquier experiencia religiosa, desde sus primeros comienzos en el nivel material hasta el momento de la consecución de la plena condición espiritual, el modelador es la clave de la comprensión personal de la realidad de la existencia del Supremo; y este mismo modelador posee también los secretos de vuestra fe en el trascendental alcance del Último. El ser personal experiencial del hombre en evolución, unido a la esencia del modelador del Dios existencial, constituye la culminación potencial de la existencia suprema y es inherentemente la base para el suprafinito devenir del ser personal trascendental.
\vs p101 6:3 \pc La voluntad moral incluye decisiones basadas en el conocimiento razonado, aumentadas por la sabiduría y aprobadas por la fe religiosa. Estas opciones son actos de naturaleza moral y demuestran la existencia del ser personal moral, el precursor del ser personal morontial y, finalmente, del verdadero estatus espiritual.
\vs p101 6:4 El conocimiento de tipo evolutivo no es sino la acumulación de información de la memoria protoplasmática; esta es la forma más primitiva de conciencia creatural. La sabiduría abarca las ideas formuladas a partir de la memoria protoplasmática en el proceso de asociación y recombinación, y estos fenómenos diferencian a la mente humana de la simple mente animal. Los animales tienen conocimiento, pero únicamente el hombre posee capacidad para la sabiduría. La verdad se hace accesible a la persona dotada de sabiduría mediante la dádiva a dicha mente de los espíritus del Padre y de los hijos: el modelador del pensamiento y el espíritu de la verdad.
\vs p101 6:5 \pc Cuando se dio de gracia en Urantia, Cristo Miguel vivió bajo el imperio de la religión evolutiva hasta el momento de su bautismo. Desde ese instante hasta el hecho de su crucifixión, incluida esta, llevó adelante su labor mediante la guía combinada de la religión evolutiva y la revelada. Desde la mañana de su resurrección hasta su ascensión, recorrió las múltiples etapas de la vida morontial propias de la transición humana desde el mundo de la materia hasta el del espíritu. Tras su ascensión, Miguel adquirió competencia total y completa para experimentar la Supremacía, tomar conciencia del Supremo; y, siendo la única persona de Nebadón que poseía capacidad ilimitada para experimentar la realidad del Supremo, logró de inmediato el estatus para lograr la soberanía de la supremacía en y para su universo local.
\vs p101 6:6 En el hombre, la fusión final con el modelador interior y la unicidad que resulta de esta ---el ser personal, síntesis del hombre y la esencia de Dios--- hacen de él, en potencia, una parte viva del Supremo, y aseguran a quien fue una vez mortal el derecho eterno por nacimiento a buscar, sin límites, la completud de servicio al universo para y con el Supremo.
\vs p101 6:7 \pc La revelación enseña al hombre mortal que, para emprender tan magnífica y fascinante aventura a través del espacio, por medio del acontecer del tiempo, debe comenzar por organizar su conocimiento en ideas\hyp{}decisiones; luego, encomendar a la sabiduría que labore incesantemente en su noble tarea de transformar las ideas que posee en ideales cada vez más prácticos, no obstante supremos, incluso aquellos conceptos que sean tan razonables como ideas y tan lógicos como ideales que el modelador procede a combinar y espiritualizar, a fin de hacerlos disponibles para su conjunción en la mente finita y constituirlos como dotación humana efectiva, ya listos para la acción del espíritu de la verdad de los hijos, las manifestaciones espacio\hyp{}temporales de la verdad del Paraíso ---la verdad universal---. La coordinación de ideas\hyp{}decisiones, ideales lógicos y verdad divina supone la posesión de un carácter recto, el prerrequisito para la admisión del mortal a las realidades, en constante expansión y crecientemente espirituales, de los mundos morontiales.
\vs p101 6:8 Las enseñanzas de Jesús establecieron la primera religión urantiana en integrar, con tanta plenitud, la armoniosa coordinación del conocimiento, la sabiduría, la fe, la verdad y el amor de forma tan completa y simultánea que proporcionó tranquilidad temporal, certeza intelectual, lucidez moral, estabilidad filosófica, sensibilidad ética, conciencia de Dios y la indudable garantía de la supraexperiencia personal. La fe de Jesús señaló el camino hacia la completud de la salvación humana, hacia la consecución última en el universo de los mortales, dado que ofrecía:
\vs p101 6:9 \li{1.}La salvación de las ataduras materiales en la realización personal de la filiación con Dios, que es espíritu.
\vs p101 6:10 \li{2.}La salvación de la esclavitud intelectual: el hombre conocerá la verdad y la verdad lo hará libre.
\vs p101 6:11 \li{3.}La salvación de la ceguera espiritual, la realización humana de la fraternidad de los seres mortales y la conciencia morontial de la hermandad de todas las criaturas del universo; el servicio\hyp{}descubrimiento de la realidad espiritual y el ministerio\hyp{}revelación de la bondad de los valores espirituales.
\vs p101 6:12 \li{4.}La salvación de la incompletitud del yo mediante el logro de los niveles espirituales del universo y a través de la futura realización de la armonía de Havona y de la perfección del Paraíso.
\vs p101 6:13 \li{5.}La salvación del yo, la liberación de las limitaciones de la propia conciencia mediante el alcance de los niveles cósmicos de la mente Suprema y a través de la armonización con los logros de todos los demás seres con consciencia de sí mismos.
\vs p101 6:14 \li{6.}La salvación del tiempo, el logro de una vida eterna de progreso sin fin en el reconocimiento y servicio de Dios.
\vs p101 6:15 \li{7.}La salvación de lo finito, la perfecta unicidad con la Deidad en y a través del Supremo mediante la cual la criatura finalizadora, emprende el descubrimiento trascendental del Último más allá de los niveles postreros de lo absonito.
\vs p101 6:16 \pc Esta salvación séptupla equivale a la realización, completa y perfecta, de la experiencia última del Padre Universal. Y todo esto está contenido, en potencia, en la realidad de fe de la experiencia humana de la religión. Y puede ser así porque la fe de Jesús se nutrió, y fue reveladora, de realidades existentes más allá de lo último; la fe de Jesús se aproximó al estatus de lo absoluto del universo en su manifestación posible en el cosmos espacio\hyp{}temporal en evolución.
\vs p101 6:17 Al tomar para sí la fe de Jesús, el hombre mortal puede anticipar en el tiempo las realidades de la eternidad. Jesús, como humano, descubrió vivencialmente al Padre Final, y sus hermanos en la carne, en su vida mortal, pueden seguirle en esta misma vivencia de descubrimiento del Padre. Incluso pueden experimentar, como son ahora, la misma satisfacción que experimentó Jesús, como él era entonces. Como consecuencia del último ministerio de gracia de Miguel, se actualizaron nuevos potenciales en el universo de Nebadón, y uno de ellos fue la iluminación, de forma nueva, de la senda eterna que conduce al Padre de todos y que los seres humanos materiales de carne y hueso, en su incipiente vida en los planetas del espacio, pueden asimismo recorrer. Jesús fue y es el camino vivo y nuevo, que él nos abrió para que el hombre pueda alcanzar la herencia divina que el Padre ha decretado y que será suya con tan solo pedirla. En Jesús, queda suficientemente demostrado a la vez el principio y el fin de la experiencia de la fe de la humanidad, incluso de la humanidad divina.
\usection{7. UNA FILOSOFÍA PERSONAL DE LA RELIGIÓN}
\vs p101 7:1 Una idea es solamente un plan de acción teórico, en tanto que una decisión positiva es un plan de acción validado. Un estereotipo es un plan de acción aceptado, sin validación. Los componentes que pueden conformar una filosofía personal de la religión se derivan tanto de la experiencia interior como de la experiencia de la persona con su entorno. El estatus social, las condiciones económicas, las oportunidades educativas, las tendencias morales, las influencias institucionales, los desarrollos políticos, las propensiones raciales y las enseñanzas religiosas del propio tiempo y lugar se convierten en elementos que intervienen en la definición de una filosofía personal de la religión. Incluso el temperamento innato y la inclinación intelectual determinan de manera significativa las características de la filosofía religiosa. La vocación, el matrimonio y los familiares también influyen en la evolución de las propias normas personales de vida.
\vs p101 7:2 Una filosofía de la religión evoluciona a partir del crecimiento de ciertas nociones básicas sumado a la experiencia de vivir, conforme ambas se ven modificadas por la tendencia a imitar a los que nos rodean. La validez de las conclusiones filosóficas depende de un pensamiento perspicaz, honesto y juicioso en unión a una sensibilidad hacia los contenidos y a una precisión de las valoraciones. Los cobardes morales jamás consiguen cotas elevadas de pensamiento filosófico; se requiere valentía para adentrarse en niveles nuevos de las experiencias e intentar explorar territorios desconocidos de la vida intelectual.
\vs p101 7:3 Pronto hacen su aparición nuevos sistemas de valores; se llega a nuevas formulaciones de los principios y las normas; se reconfiguran los hábitos y los ideales; se consigue cierta idea de un Dios personal, seguida de conceptos en cuanto a esta relación con un Dios personal.
\vs p101 7:4 \pc La gran diferencia entre una filosofía religiosa y una filosofía no religiosa de vida consiste en la naturaleza y en el nivel de los valores reconocidos y en el propósito de las lealtades. Se dan cuatro etapas en la evolución de la filosofía religiosa: el mortal puede convertirse en un conformista, resignarse a su sometimiento a la tradición y a la autoridad. O puede satisfacerse con pequeños logros, los suficientes como para dar estabilidad a su vida diaria y quedar, por lo tanto, detenido tempranamente en este nivel superficial. Estos mortales creen que hay que dejar las cosas tal como están. Un tercer grupo avanza hasta el plano de la intelectualidad lógica, pero allí se estanca como consecuencia de su esclavitud a la cultura. Es realmente penoso contemplar a grandes intelectos atrapados tan fuertemente en el cruel agarre de tal servidumbre cultural. Es igualmente lastimoso observar a aquellos que intercambian sus ataduras culturales por los grilletes materialistas de alguna ciencia, tan falazmente así llamada. En el cuarto nivel de la filosofía, se consigue la libertad de todos los obstáculos por las convenciones y las tradiciones, y se logra el arrojo para pensar, actuar y vivir con honestidad, lealtad, valentía y sinceridad.
\vs p101 7:5 La prueba de fuego para cualquier filosofía religiosa consiste en comprobar si distingue o no entre las realidades del mundo material y del mundo espiritual, mientras que reconoce, al mismo tiempo, su unificación por medio del esfuerzo intelectual y del servicio social. En una filosofía religiosa sólida, no se confunden las cosas de Dios con las cosas del césar. Tampoco se reconoce el culto estético a lo absolutamente maravilloso como sustituto de la religión.
\vs p101 7:6 La filosofía transforma esa religión primitiva, que era en gran medida un cuento de hadas de la conciencia, en una experiencia viva de los valores ascendentes de la realidad cósmica.
\usection{8. FE Y CREENCIAS}
\vs p101 8:1 Las creencias han alcanzado el nivel de la fe cuando motiva a la vida y moldea el modo de vivir. Aceptar una enseñanza como verdadera no es fe; es mera creencia; tampoco lo son ni la certidumbre ni la convicción. Un estado de mente llega a cotas de fe solo cuando estas creencias rigen realmente el modo de vida. La fe es un atributo vivo de la experiencia religiosa genuina y personal. Una persona cree en la verdad, admira la belleza y venera la bondad, pero no las adora; tal actitud de fe salvadora se centra únicamente en Dios, que personifica todas estas cosas, e infinitamente más.
\vs p101 8:2 Las creencias son siempre restrictivas e impuestas; la fe expande y libera. Las creencias fijan, la fe da libertad. Pero la fe religiosa viva es más que la conjunción de nobles creencias; es más que un excelso sistema filosófico; es una experiencia viva que entraña contenidos espirituales, ideales divinos y valores supremos; es conocedora de Dios y servidora del hombre. Las creencias pueden convertirse en posesión del grupo, pero la fe debe ser personal. Se pueden proponer creencias teológicas a un grupo, pero la fe solo puede crecer en el corazón del creyente individual.
\vs p101 8:3 La fe falta a su responsabilidad cuando pretende negar realidades e imparte a sus devotos un supuesto conocimiento. La fe es desleal cuando alienta la traición a la integridad intelectual y menosprecia la fidelidad a los valores supremos y a los ideales divinos. La fe nunca rehúye el deber de resolver los problemas de la existencia humana. La fe viva no fomenta el fanatismo, la persecución o la intolerancia.
\vs p101 8:4 La fe no constriñe la imaginación creativa como tampoco mantiene un prejuicio irracional hacia los descubrimientos de la investigación científica. La fe revitaliza la religión y compele al devoto religioso a vivir heroicamente de acuerdo con la regla de oro. El fervor de la fe está en consonancia con el conocimiento, y sus aspiraciones son el preludio de una paz sublime.
\usection{9. RELIGIÓN Y MORAL}
\vs p101 9:1 No se puede considerar auténtica ninguna pretendida revelación de la religión que afirma serlo, si no es capaz de identificar la responsabilidad de las obligaciones éticas que la anterior religión evolutiva ha creado y fomentado. La revelación indefectiblemente ensancha el horizonte ético de la religión evolutiva, expandiendo a la vez simultánea e invariablemente las obligaciones morales de todas las revelaciones previas.
\vs p101 9:2 Cuando os permitís emitir un juicio crítico de la religión primitiva del hombre (o de la religión del hombre primitivo), debéis recordar que hay que juzgar a estos seres no civilizados y valorar sus experiencias religiosas con arreglo a su lucidez y estatus de conciencia. No cometáis el error de pronunciaros sobre la religión de los demás con vuestros propios niveles de conocimiento y verdad.
\vs p101 9:3 La verdadera religión es esa convicción, sublime y firme, que viene de dentro del alma y advierte imperiosamente al hombre que sería erróneo no creer en esas realidades morontiales que conforman sus conceptos éticos y morales de mayor elevación, su más alta interpretación de los grandes valores de la vida y de las realidades más profundas del universo. Y dicha religión consiste, simplemente, en la experiencia de renunciar a la lealtad al intelecto y seguir los dictados más elevados de la conciencia espiritual.
\vs p101 9:4 La búsqueda de la belleza forma parte de la religión únicamente en la medida en la que es ética y hasta el punto en el que enriquece el concepto de la moral. El arte es religioso solo cuando queda imbuido de un propósito derivado de elevadas motivaciones espirituales.
\vs p101 9:5 La lúcida conciencia espiritual del hombre civilizado no se preocupa tanto por creencias intelectuales concretas o por determinada forma de vida como por el descubrimiento de la verdad de la vida, de la forma buena y correcta de responder a las situaciones, siempre recurrentes, de la existencia humana. La conciencia moral es solo un nombre aplicado al reconocimiento y la conciencia humanos de esos valores éticos y morontiales incipientes, de acuerdo a los cuales el hombre tiene el deber de regirse para gobernar y orientar su comportamiento día a día.
\vs p101 9:6 \pc A pesar de reconocer que la religión es imperfecta, hay al menos dos manifestaciones prácticas que describen su naturaleza y labor:
\vs p101 9:7 \li{1.}El impulso espiritual y la presión filosófica de la religión tienden a hacer que el hombre proyecte su consideración de los valores morales directamente hacia el exterior, hacia los intereses de sus semejantes: la reacción ética de la religión.
\vs p101 9:8 \li{2.}La religión crea para la mente humana una conciencia espiritualizada de la realidad divina, proveniente de la fe, que se basa en unos conceptos precedentes de los valores morales y se armoniza con unos conceptos sobrevenidos de los valores espirituales. De ahí que la religión se convierta en árbitro de las cuestiones humanas; es una forma glorificada de confianza moral y de seguridad en la realidad, en las realidades realzadas del tiempo y en las realidades más perdurables de la eternidad.
\vs p101 9:9 \pc La fe se convierte en el vínculo entre la conciencia moral y el concepto espiritual de la realidad perdurable. La religión llega a ser la vía de escape del hombre de las limitaciones materiales del mundo temporal y natural y se encamina hacia las excelsas realidades del mundo eterno y espiritual a través de la salvación, de la transformación morontial progresiva.
\usection{10. LA RELIGIÓN, LIBERADORA DEL HOMBRE}
\vs p101 10:1 El hombre inteligente sabe que es hijo de la naturaleza, que es parte del universo material; del mismo modo, no percibe que haya supervivencia del ser personal individual en los movimientos y tensiones del nivel matemático del universo energético. El hombre tampoco puede percibir la realidad espiritual mediante el análisis de las causas y los efectos físicos.
\vs p101 10:2 El ser humano es también consciente de que es parte de un universo conceptual, pero aunque este concepto puede perdurar más allá de la duración de una vida mortal, no hay nada intrínseco en él que sea indicativo de la supervivencia individual del ser personal que lo concibe. Tampoco el agotamiento de las posibilidades de la lógica y de la razón revelará jamás al lógico o al razonador la verdad eterna de la supervivencia de la persona.
\vs p101 10:3 El nivel material de la ley establece la continuidad de causalidad, la interminable respuesta del efecto a la acción antecedente; el nivel mental sugiere la perpetuación de la continuidad teórica, el flujo incesante de la potencialidad teórica a partir de nociones preexistentes. Pero ninguno de estos niveles del universo revela al inquisitivo mortal una vía de escape de la parcialidad de su estatus ni de la intolerable incertidumbre de ser una realidad transitoria en el universo, un ser personal temporal condenado a expirar cuando se agotan sus limitadas energías vitales.
\vs p101 10:4 En el universo, el hombre solo puede romper las cadenas intrínsecas a su estatus mortal mediante la vía morontial, que conduce a la percepción espiritual. La energía y la mente llevan de vuelta al Paraíso y a la Deidad, pero ni el don de la energía ni el de la mente del hombre proceden directamente de esta Deidad del Paraíso. El hombre es hijo de Dios únicamente en el sentido espiritual. Y esto es verdad porque solo en este sentido está, en este momento, dotado y habitado por el Padre del Paraíso. La humanidad nunca podrá descubrir la divinidad, salvo mediante la vía de la experiencia religiosa y por medio del ejercicio de la fe verdadera. La aceptación por la fe de la verdad de Dios permite al hombre escapar de los restringidos confines de las limitaciones materiales y le proporciona una esperanza racional de conseguir un salvoconducto para dirigirse desde el mundo material, donde está la muerte, hasta el mundo espiritual, donde la vida es eterna.
\vs p101 10:5 \pc El propósito de la religión no es satisfacer la curiosidad acerca de Dios, sino más bien ofrecer constancia intelectual y seguridad filosófica, estabilizar y enriquecer la vida humana combinando lo mortal con lo divino, lo parcial con lo perfecto, el hombre y Dios. Es a través de la experiencia religiosa por la que los conceptos del hombre sobre la idealidad se dotan de realidad.
\vs p101 10:6 \pc No podrán existir pruebas científicas o lógicas de la divinidad. La razón por sí sola no podrá jamás validar los valores ni las bondades de la experiencia religiosa. Pero seguirá siendo siempre verdad: todo aquel que quiera hacer la voluntad de Dios comprenderá la validez de los valores espirituales. En el nivel humano, esta es la mayor aproximación que se pueda hacer en cuanto a ofrecer pruebas acerca de la realidad de la experiencia religiosa. Esta fe otorga la única forma de escapar del arraigo mecánico del mundo material y de la distorsión equivocada de la incompletitud del mundo intelectual; es la única solución que se ha hallado al estancamiento del pensamiento mortal en relación a la supervivencia continuada del ser personal individual. Es el único pasaporte para finalizar la realidad y lograr la eternidad de la vida conforme a una creación universal de amor, de ley, de unidad y del logro progresivo de la Deidad.
\vs p101 10:7 La religión es una eficaz cura para el sentido de aislamiento imaginario o de soledad espiritual del hombre; otorga al creyente la condición de hijo de Dios, de ciudadano de un universo nuevo y significativo. La religión asegura al hombre que, al seguir el destello de la rectitud que percibe en su alma, llega a identificarse con el plan del Infinito y el propósito del Eterno. Un alma así liberada comienza, de inmediato, a sentirse como en casa en este nuevo universo, en su universo.
\vs p101 10:8 Cuando experimentéis tal transformación por la fe, ya no seréis una parte servil del cosmos matemático sino un hijo del Padre Universal, libre, con voluntad. Un hijo así liberado ya no luchará solo contra el inexorable destino del fin de la existencia temporal; ya no pugnará contra toda la naturaleza, con las probabilidades de ganar irremediablemente en su contra; ya no sentirá el temor paralizante de haber, tal vez, depositado su confianza en un fantasma sin esperanzas o haber puesto su fe en un descabellado error.
\vs p101 10:9 En cambio, ahora, los hijos de Dios se aúnan para luchar por una realidad que triunfe sobre las parciales sombras de la existencia. Por fin, todas las criaturas toman conciencia de que Dios, junto con las multitudes divinas del casi ilimitado universo, está a su lado en el sublime afán de conseguir la vida eterna y el estatus divino. Estos hijos que la fe ha hecho libres participan, de cierto, en las pugnas temporales del lado de las fuerzas supremas y de los seres personales divinos de la eternidad; incluso luchan en su curso las estrellas por ellos; por fin contemplan el universo desde dentro, desde la perspectiva de Dios y toda la incertidumbre de aislamiento material se transforma en la seguridad del eterno progreso. Incluso el tiempo mismo se convierte en una sombra de la eternidad proyectada por las realidades del Paraíso sobre la panoplia móvil del espacio.
\vsetoff
\vs p101 10:10 [Exposición de un melquisedec de Nebadón.]
