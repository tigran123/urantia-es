\upaper{185}{El juicio ante Pilato}
\author{Comisión de seres intermedios}
\vs p185 0:1 Poco después de las seis de la mañana de aquel viernes, 7 de abril del año 30 d. C., Jesús compareció ante Pilato, el procurador romano que gobernaba Judea, Samaria e Idumea bajo la supervisión directa del legado de Siria. Los guardias del templo llevaron al Maestro ante la presencia del gobernador romano, atado y acompañado por unos cincuenta de sus acusadores, incluido el tribunal sanedrita (saduceos principalmente), Judas Iscariote, Caifás el sumo sacerdote y el apóstol Juan. Anás no se presentó ante Pilato.
\vs p185 0:2 Temprano esa mañana, Pilato ya se había levantado y estaba listo para recibir a aquel grupo de visitantes. A última hora de la tarde anterior, aquellos mismos que le habían pedido su autorización para el empleo de soldados romanos en el arresto del Hijo del Hombre le habían informado de que llevarían a Jesús temprano ante él. Se había previsto que el juicio tuviera lugar delante del pretorio, en una ampliación de la fortaleza de Antonia, donde Pilato y su mujer residían cuando se quedaban en Jerusalén.
\vs p185 0:3 Aunque Pilato llevó a cabo gran parte del interrogatorio al que se sometió a Jesús en las salas del pretorio, el juicio público se celebró en el exterior, en las escalinatas que conducían a la entrada principal. Se les permitió aquel privilegio a los judíos, que se negaban a entrar en cualquier edificio gentil en el que se pudiese haber usado levadura ese día de la preparación de la Pascua. Aquello no solo los hubiera convertido, ceremonialmente, en impuros y les hubiera impedido, pues, compartir la fiesta de acción de gracias de la tarde, sino que habrían precisado de las ceremonias de purificación tras la puesta de sol, antes de ser aptos para comer la cena pascual.
\vs p185 0:4 Aunque estos judíos no sentían ningún remordimiento de conciencia por intrigar y llevar a efecto el asesinato legal de Jesús, eran, no obstante, muy escrupulosos en cuanto a estas cuestiones de limpieza ceremonial y de rutina tradicionalista. Y ellos no son los únicos en dejar de lado las obligaciones  elevadas y santas que dicta la naturaleza divina, y prestar, no obstante, una meticulosa atención a cosas insignificantes para el bien humano tanto en el tiempo como en la eternidad.
\usection{1. PONCIO PILATO}
\vs p185 1:1 Si Poncio Pilato no hubiera sido un gobernador razonablemente bueno de las provincias menores, Tiberio difícilmente hubiera permitido que permaneciera durante diez años como procurador de Judea. Pero, aunque se trataba de un regente relativamente aceptable, era un cobarde moral. No era un hombre lo suficientemente competente como para comprender la naturaleza de su tarea como gobernador de los judíos. No lograba entender el hecho de que estos hebreos tenían una religión \bibemph{real,} una fe por la que estaban dispuestos a morir, y que millones y millones de ellos, dispersos por todas partes del imperio, tenían a Jerusalén como el santuario de su fe y respetaban al sanedrín por considerarlo el más alto tribunal sobre la tierra.
\vs p185 1:2 Pilato sentía una gran desafección hacia los judíos, y este odio, profundamente arraigado, empezó a manifestarse con prontitud. De todas las provincias romanas, ninguna era tan complicada de gobernar como Judea. Pilato nunca entendió realmente las dificultades que entrañaba la gestión de la comunidad judía y, así pues, muy temprano en su cargo de gobernador, cometió una serie de equivocaciones, que resultaron fatales y prácticamente suicidas. Y fueron estos desaciertos los que dieron a los judíos tal poder sobre él. Cuando querían influir en sus decisiones, todo lo que tenían que hacer era amenazar con alguna revuelta, y Pilato capitulaba de inmediato. Y esta aparente vacilación, o falta de coraje moral del procurador, se debía esencialmente al recuerdo de un cierto número de polémicas, mantenidas con los judíos, en las que siempre había salido derrotado. Los judíos sabían que Pilato les tenía miedo, que temía por su puesto ante Tiberio, y, en múltiples ocasiones, emplearon esa información en profundo y grave detrimento del gobernador.
\vs p185 1:3 La aversión que los judíos sentían hacia Pilato surgió debido a una serie de desafortunados enfrentamientos con ellos. En primer lugar, Pilato no supo tomar en serio el prejuicio profundamente arraigado de los judíos contra todas las imágenes, que para ellos eran símbolos de idolatría. En consecuencia, permitió que sus soldados entraran en Jerusalén sin quitar las imágenes del César de sus estandartes militares, tal como había sido la práctica habitual de los soldados romanos bajo su predecesor. Una amplia delegación de judíos, durante cinco días, imploró a Pilato que retirara aquellas insignias. Él se negó rotundamente a concederles su petición y los amenazó con la muerte inmediata. Siendo él mismo un escéptico, Pilato no comprendía que hubiera hombres de fuertes sentimientos religiosos que no vacilarían en morir por sus convicciones religiosas; y, por lo tanto, quedó consternado cuando estos judíos se presentaron, desafiantes, ante su palacio e, inclinando el rostro a tierra, le mostraron que estaban dispuestos a morir. Pilato percibió entonces que había proferido una amenaza que no estaba dispuesto a cumplir. Se rindió y ordenó que se quitaran las imágenes de los estandartes de sus soldados en Jerusalén, y desde ese día se vio en gran medida sometido a los caprichos de los líderes judíos, que habían descubierto la debilidad de Pilato de no llevar a cabo sus amenazas.
\vs p185 1:4 Con posterioridad, Pilato, decidido a recuperar su prestigio perdido y, en este sentido, hizo colocar los escudos del emperador, utilizados generalmente para adorar al césar, en los muros del palacio de Herodes en Jerusalén. Cuando los judíos protestaron, él se mantuvo firme. Al negarse a escuchar sus protestas, estos apelaron enseguida a Roma y el emperador, con igual rapidez, mandó quitar los escudos ofensivos. Y, en lo sucesivo, Pilato contó incluso con menor estima que antes.
\vs p185 1:5 \pc Otra cuestión por la que cayó en gran desgracia ante los judíos fue cuando, a expensas de la tesorería del templo, Pilato construyó un nuevo acueducto para suministrar una mayor cantidad de agua a los millones de visitantes que acudían a Jerusalén en las grandes fiestas religiosas. Los judíos alegaban que solo el sanedrín podía hacer el desembolso de los fondos del templo, y no dejaron nunca de vilipendiar a Pilato por su prepotencia. A raíz de esta decisión se produjeron al menos una veintena de disturbios y un gran derramamiento de sangre. La última de estas graves revueltas se originó por la matanza de un amplio grupo de galileos que se encontraba en el altar del templo en su culto de adoración.
\vs p185 1:6 \pc Es significativo que, aunque este indeciso gobernante romano sacrificó a Jesús por temor a los judíos y para salvaguardar su posición personal, fue finalmente depuesto como consecuencia de una innecesaria masacre de samaritanos por las pretensiones de un falso Mesías que llevó tropas al monte Gerizim, donde afirmaba se encontraban enterrados los vasos del templo. Allí se desataron violentos disturbios cuando no logró revelar, como había prometido, el lugar en el que estas vasijas sagradas estaban escondidas. A causa de este incidente, el legado de Siria ordenó a Pilato que fuera a Roma. Tiberio murió mientras Pilato iba de camino a Roma, por lo que no se le nombró nuevamente procurador de Judea. No llegó a recuperarse nunca del todo de la insufrible culpabilidad que cargó sobre sí por haber consentido la crucifixión de Jesús. Al no hallar favor a los ojos del nuevo emperador, se retiró a la provincia de Lausana, donde posteriormente se suicidó.
\vs p185 1:7 \pc Claudia Prócula, la mujer de Pilato, había oído hablar mucho de Jesús por su doncella fenicia, creyente del evangelio del reino. Tras la muerte de Pilato, Claudia ocupó un lugar destacado en la difusión de la buena nueva.
\vs p185 1:8 \pc Y todo esto da una buena explicación de lo que sucedió aquella trágica mañana del viernes. Es fácil entender por qué los judíos se tomaban la libertad de dar instrucciones a Pilato ---hacer que se levantara a las seis de la mañana para juzgar a Jesús--- y también por qué no vacilaron en amenazarlo con acusarlo de traición ante el emperador, si se atrevía a negarse a sus exigencias de ordenar la muerte de Jesús.
\vs p185 1:9 Un digno gobernador romano, que no se hubiera visto en tal situación de desventaja con respecto a los dirigentes de los judíos, jamás habría permitido que estos religiosos, sanguinarios fanáticos, dieran muerte a un hombre a quien él mismo había declarado inocente de sus falsas acusaciones y exento de culpa. Roma cometió una gran equivocación, un error de gran repercusión en los asuntos terrenales, al mandar al mediocre de Pilato a gobernar Palestina. Tiberio debería haber enviado a los judíos al mejor regidor provincial de su imperio.
\usection{2. COMPARECENCIA DE JESÚS ANTE PILATO}
\vs p185 2:1 Cuando Jesús y sus acusadores estaban reunidos delante de la sala de juicios de Pilato, el gobernador romano salió y, hablándole al grupo allí congregado, preguntó: “¿Qué acusación traéis contra este hombre?”. Los saduceos y los miembros del consejo, encargados de deshacerse de Jesús, habían decidido ir ante Pilato con la intención de pedirle que confirmara la sentencia a muerte dictada contra él, sin presentar ninguna acusación concreta. Así pues, el portavoz del tribunal de los sanedritas respondió a Pilato: “Si este hombre no fuera malhechor, no te lo habríamos traído”.
\vs p185 2:2 Cuando Pilato observó que se resistían a exponer sus cargos contra Jesús, sabiendo que habían pasado toda la noche deliberando sobre su culpabilidad, les contestó: “Puesto que no os habéis puesto de acuerdo en presentar ninguna acusación firme, ¿por qué no tomáis vosotros a este hombre y lo juzgáis según vuestras leyes?”.
\vs p185 2:3 Entonces, el escribiente del tribunal del sanedrín le dijo a Pilato: “A nosotros no nos está permitido dar muerte a nadie, y este agitador de nuestra nación es digno de muerte por las cosas que ha dicho y hecho. Por ello hemos venido ante ti para que ratifiques este decreto”.
\vs p185 2:4 Ir ante el gobernador romano con este esquivo intento revela tanto la mala voluntad y la ira de los sanedritas hacia Jesús como su falta de respeto por la imparcialidad, el honor y la dignidad de Pilato. ¡Qué insolencia la de estos ciudadanos\hyp{}súbditos comparecer ante su gobernador provincial pidiendo un mandato de ejecución contra un hombre antes de dispensarle un juicio justo y sin ni siquiera presentar acusaciones incriminatorias concretas contra él!
\vs p185 2:5 Pilato conocía algo la labor de Jesús entre los judíos, y supuso que los cargos que se presentaran contra él tendrían que ver con el quebrantamiento de las leyes eclesiásticas judías; por ello, trató de devolver el caso a su propio tribunal de sanedritas. Nuevamente, Pilato se complacía en hacerles confesar públicamente su impotencia para pronunciar y ejecutar sentencias de muerte, incluso contra alguien de su propia raza, a quien habían llegado a despreciar con tal encarnizado rencor y envidia.
\vs p185 2:6 \pc Algunas horas antes, con anterioridad a la medianoche y, tras haber concedido el permiso para que los soldados procediesen al arresto clandestino de Jesús, Pilato había oído más cosas sobre Jesús y sus enseñanzas por parte de Claudia, su mujer, que era conversa parcial al judaísmo y que más tarde se convertiría en una verdadera creyente del evangelio de Jesús.
\vs p185 2:7 \pc Pilato hubiera deseado posponer aquella audiencia, pero vio que los líderes judíos estaban determinados a darle curso al caso. Sabía que aquella no solo era la mañana de la preparación para la Pascua, sino que ese día, siendo viernes, era también el día de la preparación para el \bibemph{sabbat} judío, de descanso y culto de adoración.
\vs p185 2:8 Pilato, siendo extremadamente sensible a la manera irrespetuosa en la que estos judíos lo abordaban, no estaba dispuesto a plegarse a sus exigencias de sentenciar a Jesús a muerte sin un juicio. Así pues, tras esperar unos momentos para que pudieran exponer sus cargos contra el preso, se volvió hacia ellos y dijo: “No condenaré a este hombre a muerte sin un juicio; tampoco accederé a interrogarlo hasta que no hayáis expuesto vuestras acusaciones contra él por escrito”.
\vs p185 2:9 Cuando el sumo sacerdote y los demás oyeron a Pilato decir aquello, le hicieron una señal al escribiente del tribunal, que entregó a Pilato los cargos escritos contra Jesús. Estos cargos eran:
\vs p185 2:10 \pc “Este tribunal de sanedritas ha hallado que este hombre es un malhechor y un agitador de nuestra nación, siendo culpable de:
\vs p185 2:11 \li{1.}Pervertir a nuestra nación e incitar a nuestro pueblo a la rebelión.
\vs p185 2:12 \li{2.}Prohibir que el pueblo pague tributos al césar.
\vs p185 2:13 \li{3.}Llamarse a sí mismo el rey de los judíos y enseñar la fundación de un nuevo reino”.
\vs p185 2:14 \pc A Jesús no se le había juzgado formalmente ni declarado legalmente culpable de ninguno de estos cargos. Ni siquiera los oyó cuando se formularon por primera vez, pero Pilato lo hizo traer del pretorio, donde estaba al cuidado de los guardias, e insistió en que dichos cargos se repitieran en la comparecencia de Jesús.
\vs p185 2:15 Cuando Jesús oyó estas acusaciones, supo muy bien que no se le había permitido hablar sobre estos asuntos ante el tribunal, al igual que lo sabían Juan Zebedeo y sus propios acusadores, pero no respondió a sus falsas acusaciones. Incluso cuando Pilato le ordenó que les respondiera, él no abrió la boca. Pilato quedó tan asombrado por la falta de equidad del procedimiento seguido y tan impresionado por la presencia silenciosa y señorial de Jesús que decidió llevar al preso al interior de la sala de juicios e interrogarlo privadamente.
\vs p185 2:16 Pilato se sentía mentalmente confuso, temeroso de los judíos en su corazón y fuertemente agitado en espíritu ante aquella imponente escena de Jesús allí de pie, en toda su majestuosidad, ante sus sanguinarios acusadores, mirándolos, no con un callado menosprecio, sino con una expresión de auténtica piedad y de doloroso afecto.
\usection{3. INTERROGATORIO PRIVADO DE PILATO}
\vs p185 3:1 Pilato llevó a Jesús y a Juan Zebedeo a una sala privada, dejando fuera a los guardias en la sala de juicios y le pidió al preso que se sentara. Él se sentó a su lado y le hizo algunas preguntas. Pilato comenzó su conversación con Jesús, asegurándole que no creía que fuera verdad el primer cargo contra él: que pervertía la nación e incitaba a la rebelión. Luego le preguntó: “¿Enseñaste alguna vez que debe negársele los tributos al césar?”. Jesús, señalando a Juan, dijo: “Pregúntale a él o a cualquier otro que haya oído mis enseñanzas”. Entonces Pilato interrogó a Juan sobre esta cuestión y Juan, dando su testimonio con respecto a las enseñanzas de su Maestro, explicó que Jesús y sus apóstoles pagaban impuestos tanto al césar como al templo. Tras preguntarle a Juan, Pilato dijo: “Cuida de no decirle a nadie que hablé contigo”. Y Juan nunca desveló este asunto.
\vs p185 3:2 Entonces Pilato se volvió para continuar interrogando a Jesús: “Y, ahora, en cuanto a la tercera acusación contra ti, ¿eres tú el rey de los judíos?”. Puesto que, al preguntar, había en la inflexión de la voz de Pilato cierta posible sinceridad, Jesús sonrió al procurador y dijo: “Pilato, ¿dices esto por ti mismo o te lo han dicho esos otros, mis acusadores?”. Ante lo cual, en un tono de parcial indignación, el gobernador respondió: “¿Soy yo acaso judío? Tu propio pueblo y los principales sacerdotes te han entregado a mí y me han pedido que te condene a muerte. Yo cuestiono la validez de sus cargos y solo estoy tratando de averiguar por mí mismo qué has hecho. Dime, ¿has dicho que eres el rey de los judíos, y has querido fundar un nuevo reino?”.
\vs p185 3:3 Entonces, le dijo Jesús a Pilato: “¿Es que no ves que mi reino no es de este mundo? Si mi reino fuera de este mundo, ciertamente mis discípulos pelearían para que no se me entregara a manos de los judíos. Mi presencia aquí ante ti, con estas ataduras, es suficiente para mostrar a todos los hombres que mi reino es de índole espiritual; es la hermandad de los hombres que mediante la fe y el amor se han convertido en los hijos de Dios. Y esta salvación es a la vez para los gentiles y los judíos”.
\vs p185 3:4 “Luego, ¿eres tú entonces rey?”, añadió Pilato. Y Jesús respondió: “Sí, soy ese rey, y mi reino es la familia de los hijos por la fe de mi Padre que está en el cielo. Yo para esto he nacido en este mundo: para mostrar a mi Padre a todos los hombres y dar testimonio de la verdad de Dios. Y aun ahora declaro ante ti que todo el que ama la verdad oye mi voz”.
\vs p185 3:5 Entonces, dijo Pilato, medio en mofa medio en serio: “La verdad, ¿qué es la verdad? ¿Alguien lo sabe?”.
\vs p185 3:6 Pilato no era capaz de comprender las palabras de Jesús, ni de entender tampoco la naturaleza de su reino espiritual, pero estaba ahora seguro de que el preso no había hecho nada por lo que fuera merecedor de muerte. Solo con mirar a Jesús, cara a cara, bastó para convencer, incluso a Pilato, de que este hombre afable y agotado, pero majestuoso e íntegro, no era un revolucionario fiero y peligroso que aspirara a establecerse en el trono temporal de Israel. Pilato creyó entender algo de lo que Jesús quería decir cuando se llamó a sí mismo rey, porque conocía las enseñanzas de los estoicos, que afirmaban que “el sabio es rey”. Pilato estaba totalmente convencido de que Jesús, en lugar de ser un peligroso instigador de sediciones, no era ni más ni menos que un visionario inofensivo, un inocente fanático.
\vs p185 3:7 Tras interrogar al Maestro, Pilato regresó al lugar donde estaban los principales sacerdotes y los acusadores de Jesús y dijo: “He interrogado a este hombre, y no hallo ningún delito en él. No creo que sea culpable de los cargos que habéis formulado contra él; pienso que debe ser puesto en libertad”. Y cuando los judíos oyeron aquello, se enfurecieron hasta tal punto que gritaron desenfrenados que Jesús debía morir; y uno de los sanedritas, se acercó con atrevimiento hasta situarse al lado de Pilato, y dijo: “Este hombre alborota al pueblo, desde Galilea hasta toda Judea. Es un agitador y un malhechor. Te lamentarás largamente si dejas ir libre a este perverso hombre”.
\vs p185 3:8 Pilato no sabía qué hacer con Jesús; así pues, cuando los oyó decir que había comenzado su labor en Galilea, pensó en eludir la responsabilidad de resolver el caso, al menos para ganar tiempo y poder pensar, enviando a Jesús a comparecer ante Herodes, que se encontraba entonces en la ciudad asistiendo a la Pascua. Pilato también pensó que aquel gesto contribuiría, quizás, como remedio para aliviar algunos de los agrios sentimientos que habían existido durante algún tiempo entre él y Herodes, debido a numerosos desencuentros sobre asuntos jurisdiccionales.
\vs p185 3:9 Pilato, llamando a los guardias, dijo: “Este hombre es galileo. Llevadlo de inmediato a Herodes, y cuando lo haya interrogado, comunicadme su dictamen”. Y remitió a Jesús a Herodes.
\usection{4. JESÚS ANTE HERODES}
\vs p185 4:1 Cuando Herodes Antipas estaba en Jerusalén, se quedaba en el antiguo palacio macabeo de Herodes el Grande, y fue a esta residencia del anterior rey donde los guardas del templo llevaron a Jesús, siendo seguidos por sus acusadores junto con una cada vez mayor multitud de personas. Durante mucho tiempo, Herodes había oído hablar de Jesús y sentía mucha curiosidad por él. Cuando el Hijo del Hombre estuvo ante él, aquel viernes por la mañana, el malvado idumeo no recordó ni por un solo momento al muchacho que años anteriores había aparecido ante él en Séforis, abogando por una decisión justa respecto al dinero que se le debía a su padre, fallecido accidentalmente mientras trabajaba en uno de los edificios públicos. A su entender, Herodes nunca había visto a Jesús, aunque le había preocupado bastante su labor en Galilea. Ahora, que Jesús estaba bajo la custodia de Pilato y de gente de Judea, Herodes estaba deseoso de verlo, teniendo la seguridad de que no tendría problemas con él en el futuro. Herodes había oído mucho acerca de los milagros que había obrado y esperaba que realizara algún prodigio.
\vs p185 4:2 Cuando llevaron a Jesús ante Herodes, el tetrarca quedó desconcertado ante su majestuoso aspecto y la serena calma de su semblante. Durante unos quince minutos, Herodes interrogó a Jesús, pero el Maestro no le contestó. Herodes lo ridiculizó y lo retó a que obrara algún milagro, pero Jesús no respondió a sus muchas preguntas ni a su sarcasmo.
\vs p185 4:3 Entonces Herodes se volvió a los principales sacerdotes y a los saduceos y, prestando atención a sus acusaciones, oyó todo lo que Pilato había oído, y algunas cosas más, sobre las supuestas fechorías del Hijo del Hombre. Finalmente, convenciéndose de que Jesús no hablaría ni realizaría ningún prodigio para él, Herodes, tras burlarse de él durante un tiempo, ataviándolo con un viejo manto real de color púrpura, lo mandó de vuelta a Pilato. Herodes sabía que no tenía jurisdicción sobre Jesús en Judea. Aunque le alegraba pensar que por fin se libraría de Jesús en Galilea, agradecía de que fuera Pilato quien tuviera la responsabilidad de darle muerte. Herodes jamás se había recuperado del temor que lo martirizaba a raíz de haber matado a Juan el Bautista. En ciertos momentos, Herodes tuvo miedo de que Jesús fuera Juan, resucitado de entre los muertos. Se sintió entonces aliviado de esos miedos al observar que Jesús era una clase de persona muy distinta del abiertamente crítico y fogoso profeta, que se había atrevido a desenmascarar y denunciar su vida privada.
\usection{5. JESÚS REGRESA A PILATO}
\vs p185 5:1 Cuando los guardias trajeron a Jesús de vuelta a Pilato, él salió a las escalinatas de fuera del pretorio, donde se había colocado su asiento de juicio, y, congregando a los principales sacerdotes y a los sanedritas, les dijo: “Me habéis presentado a este hombre, acusándolo de que pervierte al pueblo, prohíbe el pago de impuestos y afirma ser el rey de los judíos; pero, habiéndolo interrogado, no hallo en él delito alguno de aquello de lo que lo acusáis. Luego, lo remití a Herodes, y el tetrarca debe de haber llegado a la misma conclusión, dado que nos lo ha devuelto a nosotros. Ciertamente, nada digno de muerte ha hecho este hombre. Si aún pensáis que debe ser castigado, así lo haré antes de soltarlo”.
\vs p185 5:2 Precisamente en el momento en el que los judíos estaban a punto de protestar a gritos en contra de la puesta en libertad de Jesús, una gran muchedumbre venía marchando hacia el pretorio para pedirle a Pilato que soltara a un preso en honor de la fiesta de Pascua. Desde hacía algún tiempo, existía la costumbre en el día de Pascua que los gobernadores romanos permitieran al pueblo pedir el perdón de algún hombre, el que quisieran, que estuviera encarcelado o hubiera sido condenado a muerte. Y, entonces, aquella gente se presentaba ante él para pedirle que liberaran a un preso, y puesto que Jesús había gozado tan recientemente del gran favor de las multitudes, se le ocurrió a Pilato que quizás podría salir de su aprieto proponiéndoles que, al ser Jesús, en ese momento, ante su tribunal, un preso a la espera de sentencia, que les soltaría a este hombre de Galilea como señal de la buena voluntad de la Pascua.
\vs p185 5:3 Cuando la multitud subía precipitadamente las escalinatas del edificio, Pilato los oyó gritar el nombre de un cierto Barrabás. Barrabás era un destacado agitador político y un ladrón homicida, hijo de un sacerdote, que acababa de ser capturado en el acto mismo de robar y asesinar en la carretera de Jericó. Este hombre estaba sentenciado a muerte y sería ejecutado en cuanto finalizaran las fiestas de la Pascua.
\vs p185 5:4 Pilato se puso en pie y explicó a la multitud que los principales sacerdotes lo habían traído a Jesús ante él, queriendo condenarlo a muerte por determinados cargos, y que él no pensaba que el hombre mereciera la muerte. Dijo Pilato: “¿A quién queréis, pues, que os suelte: a este Barrabás, a un asesino, o a este Jesús de Galilea?”. Y cuando Pilato terminó de decir estas cosas, los sacerdotes y los miembros del consejo del sanedrín gritaron todos a viva voz: “¡A Barrabás, a Barrabás!”. Y, cuando la gente vio que los principales sacerdotes tenían la intención de dar muerte a Jesús, se unieron enseguida al clamor por su muerte, mientras daban fuertes gritos para que soltaran a Barrabás.
\vs p185 5:5 Pocos días antes, esta misma multitud había reverenciado a Jesús, pero aquella muchedumbre no respetaba a quien, habiendo afirmado que era el Hijo de Dios, se encontrara ahora a disposición de los principales sacerdotes y de los dirigentes judíos, siendo juzgado por su vida ante el tribunal de Pilato. Jesús podía ser un héroe a los ojos del pueblo cuando echaba del templo a los cambistas y a los mercaderes, aunque no estando preso, sin ofrecer resistencia, en manos de sus enemigos y enjuiciado a muerte.
\vs p185 5:6 Pilato se indignó ante la visión de los principales sacerdotes que vociferaban pidiendo el perdón de un notorio asesino, mientras gritaban exigiendo la sangre de Jesús. Vio su maldad y su odio y conocía sus prejuicios y envidia. Por lo tanto les dijo: “¿Cómo es que elegís la vida de un asesino antes que la de este hombre, cuyo peor crimen es llamarse a sí mismo en sentido figurado rey de los judíos?”. Si bien, aquellas no fueron unas palabras prudentes por parte de Pilato. Los judíos eran un pueblo orgulloso, sometido en aquel momento al yugo político de los romanos, aunque aguardando la llegada de un Mesías que los librara, con gran muestra de poder y gloria, de la esclavitud de los gentiles. Se sintieron contrariados, mucho más de lo que Pilato podía llegar a entrever, ante su insinuación de que este maestro de modales amables y extrañas doctrinas, ahora bajo arresto y acusado de delitos merecedores de muerte, pudiera referirse a sí mismo como “el rey de los judíos”. Percibían aquel comentario como un insulto a todo lo que ellos consideraban sagrado y honorable de su existencia como nación y, en consecuencia, dando rienda suelta a su indignación, gritaron con fuerzas que soltaran a Barrabás y dieran muerte a Jesús.
\vs p185 5:7 Pilato sabía que Jesús era inocente de los cargos presentados en su contra y, si hubiera sido un juez justo y valiente, lo habría absuelto y dejado libre. Pero tenía miedo de provocar a estos encolerizados judíos y, mientras vacilaba en cumplir con su deber, llegó un mensajero y le hizo entrega de un mensaje sellado de Claudia, su esposa.
\vs p185 5:8 Pilato indicó a los allí congregados ante él que deseaba leer aquella misiva que acababa de recibir antes de continuar adelante con aquel proceso. Cuando Pilato abrió la carta de su esposa, leyó: “Te ruego que no tengas nada que ver con ese hombre justo e inocente a quien llaman Jesús, porque esta noche he sufrido mucho en sueños por causa de él”. Esta nota de Claudia no solo afectó sobremanera a Pilato, haciendo que se demorara su dictamen sobre aquel asunto, sino que, lamentablemente, proporcionó bastante tiempo a los líderes judíos para poder circular libremente entre la multitud e instar al pueblo a que pidieran la libertad de Barrabás y clamaran por la crucifixión de Jesús.
\vs p185 5:9 Finalmente, Pilato se centró de nuevo en solucionar el problema al que se enfrentaba, preguntando a aquella concentración de personas, mezcla de dirigentes judíos y de gente en busca de un indulto: “¿Qué queréis, pues, que haga del que se llama rey de los judíos?”. Y todos ellos gritaron de común acuerdo: “¡Crucifícalo! ¡Crucifícalo!”. El consenso que halló en las exigencias de aquella variada muchedumbre sorprendió y alarmó a Pilato, a este juez injusto, dejado llevar por el miedo.
\vs p185 5:10 Entonces, una vez más, Pilato dijo: “¿Por qué queréis crucificar a este hombre? ¿Qué mal ha hecho? ¿Quién saldrá al frente para declarar contra él?”. Pero cuando oyeron a Pilato hablar en defensa de Jesús, se pusieron a gritar aun más: “¡Crucifícalo! ¡Crucifícalo!”.
\vs p185 5:11 Luego, Pilato se dirigió de nuevo a ellos sobre la puesta en libertad de un preso con motivo de la Pascua, diciéndoles: “Una vez más os pregunto: ¿A cuál de los dos queréis que os suelte en este tiempo de Pascua?”. Y, otra vez, la muchedumbre gritó: “¡Danos a Barrabás!”.
\vs p185 5:12 Entonces, Pilato añadió: “Si suelto al asesino, a Barrabás, ¿qué he de hacer con Jesús?”. Y, una vez más, la multitud gritó al unísono: “¡Sea crucifícado! ¡Sea crucificado!”.
\vs p185 5:13 Pilato estaba aterrorizado por el insistente clamor del gentío, que actuaba bajo la dirección personal de los principales sacerdotes y miembros del consejo del sanedrín; no obstante, decidió hacer al menos un intento más por calmar a la multitud y salvar a Jesús.
\usection{6. ÚLTIMA APELACIÓN DE PILATO}
\vs p185 6:1 Solo los enemigos de Jesús intervienen en todo lo que está sucediendo este viernes por la mañana temprano ante Pilato. Sus muchos amigos o bien no se han enterado aún de su arresto durante la noche ni de su juicio a aquellas horas del amanecer o bien están escondidos para no ser también arrestados y declarados dignos de muerte por creer en las enseñanzas de Jesús. En la multitud que clama por la muerte del Maestro, se hallan únicamente sus enemigos jurados y gente irreflexiva, fácilmente manipulable.
\vs p185 6:2 Pilato quiso hacer una última apelación a la piedad de la muchedumbre. Pero, por miedo a enfrentarse con el clamor de aquel mal inducido gentío, que gritaba pidiendo la sangre de Jesús, ordenó a los guardias judíos y a los soldados romanos que se llevaran a Jesús y lo azotaran, algo injusto e ilícito, puesto que la ley romana estipulaba que solamente se azotaran a los condenados a muerte por crucifixión. Los guardias trasladaron a Jesús al patio descubierto del pretorio para azotarle. Aunque sus enemigos no fueron testigos de aquello, Pilato sí lo fue, y antes de que acabaran con este perverso maltrato, Pilato mandó a los flageladores que parasen y les señaló que lo trajeran ante él. Antes de azotar a Jesús con sus látigos de nudos, estando atado al poste de la flagelación, lo vistieron de nuevo con el manto de púrpura y, entretejiendo una corona de espinas, la pusieron sobre su frente. Después de colocarle en la mano una caña a modo de cetro, doblando la rodilla delante de él, le hacían burla, diciendo: “¡Salve, rey de los judíos!”. Y le escupían y lo golpeaban en la cara con las manos. Y uno de ellos, antes de devolverlo a Pilato, le quitó la caña de la mano y lo golpeó con ella en la cabeza.
\vs p185 6:3 Entonces, Pilato llevó fuera a este preso sangrando y lacerado y, presentándoselo a aquella dispar multitud, dijo: “¡He aquí el hombre! Os digo otra vez no hallo ningún delito en él, y una vez que ha sido azotado, quiero soltarlo”.
\vs p185 6:4 Allí estaba pues Jesús el Nazareno, envuelto en un viejo manto real de púrpura con una corona de espinas que se hendían en su dulce frente. Su rostro estaba manchado de sangre y su cuerpo inclinado por el sufrimiento y la aflicción. Pero nada puede apelar a los insensibles corazones de quienes son víctimas de un odio intenso y visceral y esclavos de los prejuicios religiosos. Aquella visión produjo un fuerte estremecimiento por todos los dominios de un universo inmenso, pero no conmovió los corazones de aquellos que tenían en mente acabar con Jesús.
\vs p185 6:5 Cuando se habían recuperado de una primera impresión, tras ver la lastimosa condición en la que se encontraba el Maestro, gritaron aún más alto y durante más tiempo: “¡Crucifícalo! ¡Crucifícalo! ¡Crucifícalo!”.
\vs p185 6:6 Y entonces Pilato comprendió que resultaba en vano tratar de apelar a sus supuestos sentimientos de piedad. Dio unos pasos hacia adelante y dijo: “Veo que habéis tomado la decisión de que este hombre debe morir, ¿pero qué ha hecho para merecer la muerte? ¿Quién dirá cuál es su crimen?”.
\vs p185 6:7 Entonces el sumo sacerdote mismo se adelantó y, llegando hasta Pilato, afirmó furioso: “Nosotros tenemos una ley sagrada y, conforme a esa ley, él debe morir, porque ha afirmado que es el Hijo de Dios”. Cuando Pilato oyó aquello, sintió aún más temor, no solo de los judíos, sino que, recordando la misiva de su esposa y la mitología griega de los dioses que descendían sobre la tierra, tembló entonces ante el pensamiento de que Jesús fuera posiblemente un personaje divino. Hizo gestos a la muchedumbre para que se mantuviera en calma mientras tomaba a Jesús del brazo y de nuevo lo llevó al interior del edificio para poder interrogarlo una vez más. Se sentía confundido por el temor, a la vez que desconcertado por su propia superstición y hostigado por la actitud obstinada del gentío.
\usection{7. EL ÚLTIMO INTERROGATORIO DE PILATO}
\vs p185 7:1 Cuando Pilato, tembloroso por el miedo y la conmoción, se sentó al lado de Jesús, le preguntó: “¿De dónde eres tú? ¿Quién eres realmente? ¿Qué es eso que dicen que eres el Hijo de Dios?”.
\vs p185 7:2 Pero Jesús difícilmente podía contestar a las preguntas que le hacía un juez débil, dubitativo y amedrantado, que tan injustamente lo había hecho azotar incluso habiéndolo hallado inocente de cualquier delito, y antes de haber sido legalmente sentenciado a muerte. Jesús miró de frente a Pilato, pero no le respondió. Entonces le dijo Pilato: “¿A mí no me hablas? ¿No sabes que todavía tengo poder para soltarte o para crucificarte? Entonces respondió Jesús: “Ningún poder tendrías contra mí si no te fuera dado de arriba. No puedes ejercer ninguna autoridad sobre el Hijo del Hombre, a menos que el Padre de los cielos lo permita. Pero no eres tan culpable, porque ignoras el evangelio. El que me traicionó y quien me entregó a ti, esos tienen mayor pecado”.
\vs p185 7:3 Esta última conversación con Jesús aterró totalmente a Pilato. Aquel cobarde moral y pusilánime autoridad judicial actuaba ahora bajo el doble peso del temor supersticioso hacia Jesús y del terror mortal a los líderes judíos.
\vs p185 7:4 De nuevo, apareció Pilato ante la muchedumbre, diciendo: “Estoy seguro de que este hombre solo ha transgredido vuestra religión. Debéis llevároslo y juzgarlo conforme a vuestra ley. ¿Por qué queréis que dé mi consentimiento a su muerte por haber quebrantado vuestras tradiciones?”.
\vs p185 7:5 Pilato estaba prácticamente listo para liberar a Jesús cuando Caifás, el sumo sacerdote, se acercó al acobardado juez romano y, agitando vengativamente un dedo en la cara de Pilato, dijo, furioso, unas palabras que toda la multitud pudo oír: “Si sueltas a este hombre, no eres amigo del césar; y me cuidaré de que el emperador sepa todo esto”. Aquella amenaza, hecha públicamente, resultó ser demasiado para Pilato. El temor por su fortuna personal eclipsó en aquel momento cualquier otra consideración, y el medroso gobernador ordenó que trajesen a Jesús ante el asiento de juicio. Estando el Maestro allí, frente a ellos, Pilato señaló a él y dijo en son de burla: “Aquí tenéis a vuestro rey”. Y los judíos respondieron: “¡Que se lo lleven. Crucifícalo! ¡Crucifícalo!”. Y entonces Pilato dijo, con gran ironía y sarcasmo: “¿A vuestro rey he de crucificar?”. Y los judíos respondieron: “Sí, ¡crucifícalo! No tenemos más rey que el césar”. Y entonces Pilato comprendió que no existía ninguna esperanza de salvar a Jesús, no estando él dispuesto a enfrentarse a los judíos.
\usection{8. LA TRÁGICA CAPITULACIÓN DE PILATO}
\vs p185 8:1 Aquí estaba el Hijo de Dios encarnado como Hijo del Hombre. Había sido arrestado sin procesamiento; enjuiciado sin testigos; castigado sin veredicto; y, ahora, va a ser condenado pronto a muerte por un juez injusto, que había confesado que no hallaba delito en él. Si Pilato creyó que apelaría al sentimiento patriótico de la multitud al referirse a Jesús como el “rey de los judíos”, fracasó por completo. No estaba en las expectativas de los judíos tener un rey así. La declaración de los principales sacerdotes y de los saduceos de que “No tenemos más rey que el césar”, causó conmoción en el insensato gentío, pero era demasiado tarde ya para salvar a Jesús, incluso si la muchedumbre se hubiera atrevido a abrazar la causa del Maestro.
\vs p185 8:2 \pc Pilato tenía miedo de que produjera algún alboroto o alguna revuelta. No podía arriesgarse a que hubiera disturbios en Jerusalén durante la semana de Pascua. El césar lo había amonestado recientemente, y no quería arriesgarse a recibir otra amonestación. La gente vitoreó cuando Pilato ordenó que soltaran a Barrabás. Luego, mandó que le trajeran una vasija y agua y allí, ante la multitud, se lavó las manos, diciendo: “Inocente soy yo de la sangre de este hombre. Vosotros habéis tomado la decisión de que debe morir, pero yo no encontré delito en él. Allá vosotros. Los soldados se lo llevarán”. Y el gentío aclamó y contestó: “Que su sangre sea sobre nosotros y sobre nuestros hijos”.
