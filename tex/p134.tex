\upaper{134}{Los años de transición}
\author{Comisión de seres intermedios}
\vs p134 0:1 Durante su viaje por el Mediterráneo, Jesús había estudiado detenidamente a las personas que conoció y los países por donde pasaba y, sobre esa época, tomaría su decisión definitiva respecto a lo que le restaba por vivir en la tierra. Había considerado por completo, y ahora daba su aprobación final, al plan que establecía que nacería en Palestina, de padres judíos, y, por ello, regresó conscientemente a Galilea para aguardar allí el comienzo de su misión de vida como maestro público de la verdad; empezó a hacer proyectos para el inicio de esta andadura pública en la tierra del pueblo de su padre José, algo que hizo por voluntad propia.
\vs p134 0:2 Gracias a su experiencia personal y humana, Jesús se había percatado de que Palestina era el mejor lugar de todo el mundo romano para llevar a cabo los últimos episodios, y trazar las escenas finales, de su vida en la tierra. Por primera vez, se sintió totalmente satisfecho con la idea de manifestar abiertamente su verdadera naturaleza y revelar su identidad divina entre los judíos y los gentiles de su Palestina natal. Decidió expresamente terminar su vida en el planeta y completar su andadura como mortal en la misma tierra en la que había emprendido su experiencia humana como un niño indefenso. Su andadura en Urantia había comenzado en Palestina entre los judíos, y eligió terminar su vida allí mismo, y entre los judíos.
\usection{1. SU TRIGÉSIMO AÑO (AÑO 24 D. C.)}
\vs p134 1:1 Tras despedirse de Gonod y de Ganid en Charax (en diciembre del año 23 d. C.), Jesús regresó por el camino de Ur a Babilonia, donde se unió a una caravana del desierto que se encaminaba a Damasco. De Damasco fue a Nazaret, haciendo solo una parada de pocas horas en Cafarnaúm para visitar a la familia de Zebedeo. Se encontró allí con su hermano Santiago, que algún tiempo antes había acudido a trabajar, en su lugar, a la fábrica de barcos de Zebedeo. Jesús, después de charlar con Santiago y con Judá (que casualmente se encontraba también en Cafarnaúm), y una vez que hizo entrega a su hermano Santiago de la propiedad de la pequeña casa, que Juan Zebedeo había conseguido adquirir y poner a su nombre, continuó hacia Nazaret.
\vs p134 1:2 Al concluir su viaje por el Mediterráneo, Jesús había percibido suficiente dinero como para hacer frente a sus gastos diarios casi hasta el momento del comienzo de su ministerio público. Si bien, al margen de Zebedeo de Cafarnaúm y de las personas que conoció en este extraordinario viaje, el mundo nada supo de esta travesía. Su familia siempre creyó que había pasado este tiempo estudiando en Alejandría. Él, por su parte, nunca confirmó esta opinión, pero tampoco hizo nada por desmentir abiertamente tal malentendido.
\vs p134 1:3 Durante su estancia de algunas pocas semanas en Nazaret, Jesús conversó con su familia y amigos, pasó algún tiempo en el taller de reparaciones con su hermano José, aunque dedicó más su atención a María y a Ruth. Ruth rondaba los quince años, y aquella era la primera oportunidad que se le presentaba, desde que se había convertido en una joven mujer, de tener una larga charla con ella.
\vs p134 1:4 Desde hacía algún tiempo, tanto Simón como Judá deseaban contraer matrimonio, pero les habría disgustado hacerlo sin el consentimiento de Jesús; por ello, habían pospuesto estas celebraciones a la espera de la vuelta de su hermano mayor. Aunque todos consideraban a Santiago como el jefe de la familia en la mayoría de las cuestiones, cuando se trataba de casamientos, deseaban recibir la bendición de Jesús. Así pues, Simón y Judá se casaron en dobles nupcias a principios de marzo de este año, el 24 d. C. Ya todos los hijos mayores estaban casados; solamente Ruth, la más joven, permanecía en el hogar con María.
\vs p134 1:5 Jesús conversaba con bastante normalidad y naturalidad con los miembros de su familia de forma individual, pero, cuando estaban todos congregados, tenía tan poco que decir que esto era motivo de comentarios entre ellos mismos. A María, particularmente, le desconcertaba este comportamiento inusualmente peculiar de su hijo primogénito.
\vs p134 1:6 En la época en la que Jesús se preparaba para abandonar Nazaret, el guía de una gran caravana que pasaba por la ciudad cayó gravemente enfermo y Jesús, siendo experto en lenguas, se ofreció voluntario para ocupar su puesto. Debido a que este viaje requeriría que se ausentase durante un año y, dado que todos sus hermanos estaban casados y que su madre vivía en su casa con Ruth, Jesús convocó una reunión familiar en la que propuso que María y Ruth fueran a Cafarnaúm para residir en la casa que él tan recientemente le había cedido a Santiago. Por ello, pocos días después de que Jesús partiera con la caravana, María y Ruth se trasladaron a Cafarnaúm, donde vivieron, el resto de la vida de María, en la casa que él les había proporcionado. José se mudó con su familia a la antigua casa de Nazaret.
\vs p134 1:7 Este fue uno de los años más atípicos en la experiencia interior del Hijo del Hombre; logró un importante avance en llevar a efecto una armonía dinámica entre su mente humana y el modelador interior. El modelador había participado activamente en la reorganización de su pensamiento y en la formación de su mente para los magnos acontecimientos que sobrevendrían en un futuro, para entonces no distante. El ser personal de Jesús se estaba preparando para su gran cambio de actitud hacia el mundo. Este fue un intervalo de tiempo, una etapa de transición de este ser, que empezó su vida como Dios apareciendo como hombre, y que ahora se disponía a completar su andadura terrenal como hombre actuando como Dios.
\usection{2. VIAJE EN CARAVANA AL CASPIO}
\vs p134 2:1 Era el primero de abril del año 24 d. C. cuando Jesús salió de Nazaret en caravana para dirigirse a la región del mar Caspio. La caravana a la que se unió como guía iba desde Jerusalén, siguiendo el camino de Damasco y el Lago Urmia, a través de Asiria, Media y Partia, hasta la región sudoriental del mar Caspio. Trascurriría todo un año completo antes de que regresara de este viaje.
\vs p134 2:2 Esta caravana representó para Jesús otra aventura de exploración y de ministerio personal. El trato con la familia que para él constituía la caravana ---pasajeros, guardias y conductores de camellos--- significó para él una interesante experiencia. Un gran número de hombres, mujeres y niños, que residían a lo largo de la ruta seguida por la caravana, vivieron mejores vidas a raíz de su contacto con Jesús, para ellos, el extraordinario guía de una caravana común. No todos los que tuvieron la oportunidad de disfrutar de su ministerio personal se beneficiaron de este, pero una gran mayoría de quienes lo conocieron y hablaron con él se convirtieron en mejores personas para el resto de su vida mortal.
\vs p134 2:3 De todos sus viajes por el mundo, este del mar Caspio fue el que más aproximó a Jesús a Oriente, permitiéndole conseguir una mejor comprensión de los pueblos del Lejano Oriente. Tuvo un contacto estrecho y personal con cada una de las razas supervivientes de Urantia, exceptuando la roja. Asimismo, disfrutó de su ministerio personal con cada una de estas diferentes razas y pueblos mestizos, y todos ellos se mostraron sensibles a la verdad viva que él les traía. Los europeos del Lejano Occidente y los asiáticos del Lejano Oriente tomaron en consideración, de igual manera, sus palabras de esperanza y de vida eterna, y se dejaron influir igualmente por la vida de amoroso servicio y ministerio espiritual que él, con tanta generosidad, vivió entre ellos.
\vs p134 2:4 \pc En todos los sentidos, el viaje en caravana fue un éxito. Se trató del episodio más interesante de la vida humana de Jesús, porque, durante este año, ejerció labor de mando, al ser el responsable del material confiado a su cuidado y de la protección de los viajeros integrantes de la caravana. Y desempeñó sus múltiples deberes con gran lealtad, eficiencia y sabiduría.
\vs p134 2:5 Al volver de la región del Caspio, Jesús renunció a la dirección de la caravana en el lago Urmia, donde se detuvo por algo más de dos semanas. Regresó como pasajero en otra caravana que se dirigía a Damasco, donde los propietarios de los camellos le rogaron que se quedase a su servicio. Pero declinó esta oferta y prosiguió su viaje con la caravana hasta Cafarnaúm, lugar al que llegó el primero de abril del año 25 d. C. Había dejado de considerar Nazaret como su hogar, y Cafarnaúm se convirtió en el domicilio de Jesús, Santiago, María y Ruth. Pero Jesús jamás volvería a vivir con su familia; cuando se hallaba en Cafarnaúm, establecía su residencia con los Zebedeos.
\usection{3. LAS CONFERENCIAS DE URMIA}
\vs p134 3:1 Camino del mar Caspio, Jesús se había detenido varios días en la antigua ciudad persa de Urmia, emplazada en la orilla occidental del Lago Urmia, para descansar y recuperarse. En la isla más grande de un conjunto de islas situado a poca distancia de la costa cerca de Urmia, se erigía un gran edificio ---un anfiteatro para conferencias---, dedicado al “espíritu de la religión”. Esta edificación era realmente un centro destinado al estudio de la filosofía de las religiones.
\vs p134 3:2 Este centro se había construido por iniciativa de un rico comerciante, ciudadano de Urmia, y de sus tres hijos. Se llamaba Cimboitón y contaba con ancestros procedentes de una gran diversidad de pueblos.
\vs p134 3:3 En esta escuela de la religión, las conferencias y estudios comenzaban, todos los días de la semana, a las diez de la mañana. Las sesiones de la tarde se iniciaban a las tres y, los debates nocturnos, a las ocho. Cimboitón o uno de sus tres hijos presidían siempre estas reuniones de enseñanza, estudio y debate. El fundador de esta excepcional escuela vivió y murió sin desvelar jamás sus creencias religiosas personales.
\vs p134 3:4 Jesús participó en estos estudios en repetidas ocasiones, y antes de dejar Urmia, Cimboitón convino con Jesús que, en su viaje de vuelta, se alojara con ellos dos semanas, impartiera veinticuatro conferencias sobre “La hermandad de los hombres” y dirigiera doce sesiones nocturnas de preguntas, estudios y debates sobre sus conferencias en particular y sobre la hermandad de los hombres en general.
\vs p134 3:5 En virtud de este acuerdo, Jesús, en su viaje de regreso, se detuvo en Urmia y pronunció estas conferencias. De todas las enseñanzas impartidas por el Maestro en Urantia, estas fueron las más sistemáticas y formales. Nunca antes ni después diría tantas cosas sobre un mismo tema como en estas ponencias y estudios dedicados a la fraternidad de los hombres. En realidad, estas charlas trataban del “Reino de Dios” y de los “Reinos de los hombres”.
\vs p134 3:6 El personal docente de este templo de filosofía religiosa representaba a más de treinta religiones y sistemas de culto religioso. Los maestros se elegían, sustentaban y estaban completamente acreditados por sus respectivos grupos religiosos. En ese momento, había unos setenta y cinco de ellos. Vivían en simples casas, con capacidad cada cual para alojar a una docena de personas. Cada luna nueva, echándolo a suerte, se cambiaban los grupos. La intolerancia, algún ánimo contencioso o cualquier otro tipo de comportamiento que interfiriese con el buen funcionamiento de la comunidad eran óbice para la expulsión inmediata y sumaria del profesor infractor. Se le despedía sin ceremonias, y a su suplente en espera se le colocaba de inmediato en su lugar.
\vs p134 3:7 Estos maestros de las distintas religiones realizaban un denodado esfuerzo por mostrar lo similares que estas eran en cuanto a las cosas fundamentales de esta vida y de la venidera. Tenían que aceptar una sola doctrina para conseguir plaza en el cuerpo docente: cada maestro debía representar una religión que reconociera a Dios, esto es, a algún tipo de Deidad suprema. Entre los docentes había cinco maestros independientes, que no representaban a ninguna religión organizada y, como tal, se presentó Jesús ante ellos.
\vs p134 3:8 \pc [Cuando nosotros, los seres intermedios, iniciamos la preparación del resumen de las enseñanzas de Jesús en Urmia, surgieron discrepancias entre los serafines de las iglesias y los serafines del progreso sobre la conveniencia de incluir estas enseñanzas en la Revelación de Urantia. Las condiciones prevalentes en el siglo XX, tanto en religión como en los gobiernos humanos, son tan diferentes de las que imperaban en los tiempos de Jesús que resultaba, de hecho, difícil adaptar las enseñanzas del Maestro en Urmia a los problemas del reino de Dios y de los reinos de los hombres, a la manera en las que estos operan en el mundo en la actualidad. Fuimos incapaces de elaborar una exposición de las enseñanzas del Maestro que resultara aceptable para ambos grupos de serafines del gobierno planetario. Finalmente, el presidente melquisedec de la comisión reveladora nombró una comisión de tres de nosotros para que presentáramos nuestra opinión sobre las enseñanzas del Maestro en Urmia, adaptadas a las condiciones religiosas y políticas de Urantia en este siglo. En consecuencia, nosotros tres, seres intermedios secundarios, realizamos tal completo ajuste de las enseñanzas de Jesús, replanteamos sus afirmaciones como si las aplicáramos a las condiciones vigentes en el mundo hoy en día, y ahora exponemos estas aseveraciones en su versión actual, tras haberlas revisado el presidente melquisedec de la comisión reveladora.]
\usection{4. SOBERANÍA: DIVINA Y HUMANA}
\vs p134 4:1 La hermandad de los hombres se fundamenta en la paternidad de Dios. La familia de Dios dimana del amor de Dios ---Dios es amor. Dios Padre ama de forma divina a sus hijos, a todos ellos---.
\vs p134 4:2 El reino de los cielos, el gobierno divino, se fundamenta en el hecho de la soberanía divina ---Dios es espíritu---. Y dado que Dios es espíritu, este reino es espiritual. El reino de los cielos no es material ni únicamente intelectual; conlleva la relación espiritual entre Dios y el hombre.
\vs p134 4:3 Si las diferentes religiones reconocen la soberanía espiritual de Dios Padre, entonces todas estas religiones guardarán la paz. Solo cuando se dé por sentado que, de algún modo, alguna es superior a las demás y que posee autoridad exclusiva sobre las otras religiones, se atreverá la religión a ser intolerante con otras religiones y a perseguir a otros creyentes religiosos.
\vs p134 4:4 La paz religiosa ---la hermandad--- no puede existir a menos que todas las religiones estén dispuestas a desprenderse por entero de cualquier autoridad eclesiástica y a renunciar completamente a cualquier noción de soberanía espiritual. Solo Dios es el soberano espiritual.
\vs p134 4:5 No podéis disfrutar de igualdad religiosa (libertad religiosa) sin que haya guerras religiosas, a no ser que todas las religiones den su consentimiento a la transferencia de cualquier soberanía religiosa a un nivel sobrenatural, a Dios mismo.
\vs p134 4:6 El reino de los cielos presente en el corazón de los hombres creará la unidad religiosa (no necesariamente la uniformidad) porque todos y cada uno de los grupos religiosos, integrados por estos creyentes religiosos, estarán libres de cualquier noción de autoridad eclesiástica, esto es, de soberanía religiosa.
\vs p134 4:7 Dios es espíritu, y Dios otorga una fracción de su ser espiritual para que more en el corazón del hombre. Espiritualmente, todos los hombres son iguales. El reino de los cielos está libre de castas, clases, niveles sociales y grupos económicos. Vosotros sois todos hermanos.
\vs p134 4:8 Pero en el momento en el que perdéis de vista la soberanía espiritual de Dios Padre, alguna religión empezará a hacer valer su superioridad sobre las demás religiones; y, entonces, en lugar de paz en la tierra y buena voluntad entre los hombres, habrá disensiones, reproches e incluso guerras religiosas o, al menos, guerras entre fieles religiosos.
\vs p134 4:9 Los seres dotados de libertad de voluntad que se consideran a sí mismos como iguales, a no ser que se reconozcan mutuamente como súbditos de una suprasoberanía, de alguna autoridad que está más allá de ellos, tarde o temprano están tentados a probar su capacidad para adquirir poder y autoridad y estar por encima de otras personas y grupos. El concepto de igualdad jamás lleva a la paz, excepto ante el mutuo reconocimiento de la influencia rectora de dicha suprasoberanía.
\vs p134 4:10 Los devotos religiosos de Urmia vivían juntos en una paz y tranquilidad relativas porque habían abandonado enteramente cualquier noción de soberanía religiosa. Espiritualmente, todos ellos creían en un Dios soberano; socialmente, era en su presidente ---Cimboitón--- en quien recaía la autoridad, de forma plena e incuestionable. Todos sabían bien qué le ocurriría a cualquier maestro que pretendiese enseñorearse sobre sus compañeros. En Urantia, no puede haber ninguna paz perdurable hasta que todos los grupos religiosos no renuncien libremente a cualquier concepción de favor divino, de pueblo elegido y de soberanía religiosa. Solo cuando Dios Padre se convierta en supremo para los hombres, podrán estos convertirse en hermanos en el sentimiento religioso y vivir juntos sobre la tierra en paz religiosa.
\usection{5. SOBERANÍA POLÍTICA}
\vs p134 5:1 [Aunque la enseñanza del Maestro respecto a la soberanía de Dios es una verdad ---solo complicada por la posterior aparición, entre las religiones del mundo, de la religión sobre él---, sus charlas sobre la soberanía política se han visto sumamente dificultadas por la evolución política de la vida nacional durante los últimos, o más, mil novecientos años. En los tiempos de Jesús, había únicamente dos grandes potencias mundiales ---el Imperio romano en el oeste y el Imperio Han en el este---, y estos estaban extensamente separados por el reino de Partia y otras tierras intermedias de las regiones del Caspio y del Turquestán. Así pues, en la siguiente exposición, nos hemos apartado aún más del contenido de las enseñanzas del Maestro en Urmia en cuanto a la soberanía política, tratando, al mismo tiempo, de describir la importancia de dichas enseñanzas al aplicarlas a la etapa, peculiarmente crítica, de la evolución de la soberanía política en el siglo XX d. C.]
\vs p134 5:2 \pc Las guerras en Urantia no acabarán mientras las naciones se aferren a nociones ilusorias de una soberanía nacional ilimitada. Hay solamente dos niveles adecuados de soberanía en un mundo habitado: la libre voluntad espiritual del mortal individual y la soberanía colectiva de la humanidad en su totalidad. Entre el nivel del ser humano individual y el nivel de la totalidad de la humanidad, todas las agrupaciones y relaciones son relativas, transitorias y de valor solo si mejoran el bienestar, la protección y el progreso de cada criatura y de todas ellas en el conjunto planetario ---el hombre y la humanidad---.
\vs p134 5:3 Los maestros de religión deben recordar siempre que la soberanía espiritual de Dios prevalece sobre cualquier lealtad espiritual temporal que pueda interponerse. Algún día los gobernantes civiles aprenderán que los altísimos gobiernan en los reinos de los hombres.
\vs p134 5:4 El gobierno de los altísimos en los reinos de los hombres no tiene por objeto beneficiar en particular a un grupo especialmente favorecido de mortales. No existe tal cosa como un “pueblo elegido”. El gobierno de los altísimos, regidores del desarrollo político, está concebido para propiciar el mayor bien al mayor número de \bibemph{todos} los hombres y durante el mayor tiempo posible.
\vs p134 5:5 La soberanía es un poder que crece por medio de la organización. El desarrollo de la organización del poder político es válido y apropiado porque tiende a englobar segmentos cada vez más amplios del conjunto de la humanidad. Pero este mismo desarrollo de las organizaciones políticas conlleva un problema en cada etapa intermedia existente entre la organización inicial y natural del poder político ---la familia--- y la consumación final del desarrollo político ---el gobierno de toda la humanidad, por toda la humanidad y para toda la humanidad.
\vs p134 5:6 Empezando con el poder paternal en el grupo familiar, la soberanía política evoluciona a través de la organización conforme las familias se solapan en clanes consanguíneos que, por distintos motivos, se unen en unidades tribales ---agrupaciones políticas consanguíneas más grandes---. Y entonces, mediante el intercambio, el comercio y la conquista, las tribus se unifican como nación, mientras que las naciones mismas se unifican, en ocasiones, en un imperio.
\vs p134 5:7 Conforme la soberanía pasa de grupos más pequeños a los más grandes, las guerras disminuyen. Esto es, disminuyen las guerras a menor escala entre naciones más pequeñas, mientras que se incrementa el potencial de guerras a mayor escala a medida que las naciones que ostentan la soberanía se hacen más y más grandes. En poco tiempo, cuando se haya explorado y ocupado todo el mundo, cuando las naciones sean pocas, fuertes y poderosas, cuando estas naciones grandes y pretendidamente soberanas se toquen en sus fronteras, cuando solamente las separen los océanos, se habrá establecido entonces el escenario para guerras a gran escala, conflictos mundiales. Las denominadas naciones soberanas no pueden entrar en contacto sin generar conflictos y devenir en guerras.
\vs p134 5:8 La dificultad existente en la evolución de la soberanía política desde la familia al conjunto de la humanidad reside en la inercia\hyp{}resistencia que se presenta en todos los niveles intermedios. Las familias, en ocasiones, desafiaron a su clan, mientras que los clanes y las tribus subvirtieron con frecuencia la soberanía del estado territorial. Cada evolución nueva y progresiva de la soberanía política es desconcertante y se encuentra (y se ha encontrado siempre) obstaculizada por las “etapas de andamiaje” de desarrollos previos en la organización política. Y esto sucede porque las lealtades humanas, una vez que se movilizan, resultan difíciles de cambiar. La misma lealtad que posibilita la evolución de la tribu, dificulta la evolución de la supratribu ---el Estado territorial---. Y la misma lealtad (el patriotismo) que posibilita la evolución del Estado territorial complica sobremanera el desarrollo evolutivo del gobierno de toda la humanidad.
\vs p134 5:9 La soberanía política se crea a partir de la renuncia a la autodeterminación, primero, de la criatura a nivel individual en el seno de la familia y, luego, de las familias y del clan en relación a la tribu y a las agrupaciones mayores. Esta transferencia paulatina de la autodeterminación desde las organizaciones políticas más pequeñas a las progresivamente más grandes ha continuado generalmente sin interrupción en Oriente, desde el establecimiento de las dinastías Ming y Mogol. En Occidente, prevaleció esta trayectoria normal durante más de mil años hasta el fin de la Guerra Mundial, momento en el que un lamentable movimiento de retroceso la revocó temporalmente, al restablecer la soberanía política sumergida de numerosos grupos pequeños europeos.
\vs p134 5:10 Urantia no gozará de una paz duradera hasta que las llamadas naciones soberanas no depositen sus poderes soberanos, con inteligencia y enteramente, en las manos de la hermandad de los hombres ---el gobierno de la humanidad---. El internacionalismo ---las ligas de las naciones--- nunca puede ofrecer una paz permanente a la humanidad. Las confederaciones de las naciones a nivel mundial podrán evitar eficientemente las guerras a menor escala y controlar razonablemente a las naciones más pequeñas, pero no podrán evitar las guerras mundiales ni controlar a los tres, cuatro o cinco gobiernos más poderosos. Ante conflictos reales, una de estas potencias mundiales se retirará de la Liga y declarará la guerra. No se puede impedir que las naciones entren en guerra mientras permanezcan infectadas con el delirante virus de la soberanía nacional. El internacionalismo es un paso en la dirección correcta. Una policía internacional podrá evitar muchas guerras a menor escala, pero no será efectiva en la prevención de guerras a mayor escala, en conflictos entre los grandes gobiernos militares de la tierra.
\vs p134 5:11 Conforme decrece el número de naciones verdaderamente soberanas (grandes potencias), se incrementa a la vez la posibilidad y la necesidad de un gobierno de la humanidad. Cuando solamente haya unas pocas (grandes) potencias verdaderamente soberanas, estas deberán emprender una lucha a muerte por la supremacía nacional (imperial), o bien, mediante la renuncia voluntaria a determinadas prerrogativas de la soberanía, deberán crear el núcleo esencial de una potencia supranacional que servirá como comienzo de la soberanía real de toda la humanidad.
\vs p134 5:12 \pc La paz no vendrá a Urantia hasta que todas las naciones denominadas soberanas no depositen su poder de declarar la guerra en manos de un gobierno representativo de toda la humanidad. La soberanía política es algo innato en los pueblos del mundo. Cuando todos los pueblos de Urantia constituyan un gobierno mundial, tendrán el derecho y el poder de hacer SOBERANO a dicho gobierno; y cuando esa potencia mundial representativa o democrática controle las fuerzas terrestres, aéreas y navales del mundo, la paz en la tierra y la buena voluntad entre los hombres podrán prevalecer en la tierra ---pero no hasta entonces---.
\vs p134 5:13 Usando una importante ilustración de los siglos XIX y XX: los cuarenta y ocho estados de la Unión Federal Americana han gozado de paz por largo tiempo. No hay guerra entre ellos. Han cedido su soberanía a un gobierno federal y, mediante el arbitraje de la guerra, han abandonado cualquier reivindicación delirante de autodeterminación. Aunque cada estado regula sus asuntos internos, no se ocupa de las relaciones internacionales, aranceles, inmigración, cuestiones militares o del comercio interestatal. Los estados por separado tampoco se implican en temas de ciudadanía. Los cuarenta y ocho estados sufren los estragos de la guerra solamente cuando se pone de alguna manera en peligro la soberanía del gobierno federal.
\vs p134 5:14 \pc Estos cuarenta y ocho estados, al haber dejado atrás el doble sofisma de la soberanía y la autodeterminación, disfrutan de paz y tranquilidad interestatal. Las naciones de Urantia comenzarán de igual manera a disfrutar de la paz cuando depositen voluntariamente su soberanía en manos de un gobierno mundial ---la soberanía de la hermandad de los hombres---. En este Estado mundial, las naciones pequeñas serán tan poderosas como las más grandes, como en el caso del pequeño estado de Rhode Island que tiene dos senadores en el Congreso Americano, justo los mismos que el populoso estado de Nueva York o el extenso estado de Texas.
\vs p134 5:15 La soberanía (estatal) limitada de estos cuarenta y ocho estados se estableció por iniciativa de los hombres y para los hombres. La soberanía supraestatal (nacional) de la Unión Federal Americana se instituyó por los primeros trece estados para su propio beneficio y para el de los hombres. En algún momento y de manera similar, las naciones establecerán la soberanía supranacional del gobierno planetario de la humanidad para su propio beneficio y para el de todos los hombres.
\vs p134 5:16 Los ciudadanos no nacen para el beneficio de los gobiernos; los gobiernos son organizaciones creadas y diseñadas para el beneficio de los hombres. No puede haber otro fin para la evolución de la soberanía política que no sea la aparición del gobierno soberano de todos los hombres. Todas las demás soberanías son de valor relativo, transitorio en significado y subordinado en estatus.
\vs p134 5:17 Con el progreso científico, las guerras llegarán a ser cada vez más devastadoras hasta casi convertirse en racialmente suicidas. ¿Cuántas guerras mundiales deberán librarse y cuántas ligas de naciones deberán fracasar antes de que los hombres estén dispuestos a establecer el gobierno de la humanidad y comiencen a disfrutar de las bendiciones de la paz permanente y a progresar en la tranquilidad de la buena voluntad ---la buena voluntad mundial--- entre ellos?
\usection{6. LEY, LIBERTAD Y SOBERANÍA}
\vs p134 6:1 Si el hombre anhela ser libre ---la libertad---, debe recordar que \bibemph{todos} los demás hombres también lo desean. Los grupos de mortales que aman la libertad no pueden vivir juntos en paz a menos que se supediten a unas leyes, normas y regulaciones que concedan el mismo grado de libertad a cada uno de ellos, garantizando, al mismo tiempo, un grado igual de libertad para todos sus semejantes mortales. Si un hombre ha de ser absolutamente libre, habrá, entonces, quien se convierta en un absoluto esclavo. Y la naturaleza relativa de la libertad es verdadera social, económica y políticamente. La libertad es el don de la civilización posibilitado por el cumplimiento de la LEY.
\vs p134 6:2 La religión hace espiritualmente posible la realización de la hermandad de los hombres, pero hace falta un gobierno humano que regule los problemas sociales, económicos y políticos relacionados con ese objetivo de felicidad y eficiencia humanas.
\vs p134 6:3 Habrá guerras y rumores de guerras ---se levantarán nación contra nación--- mientras que la soberanía política del mundo esté repartida e injustamente en las manos de un grupo de Estados nacionales. Inglaterra, Escocia y Gales estuvieron constantemente peleándose entre ellos hasta que renunciaron a sus respectivas soberanías y las depositaron en el Reino Unido.
\vs p134 6:4 Una nueva guerra mundial enseñará a las llamadas naciones soberanas a formar algún tipo de federación, creando de este modo el mecanismo para impedir guerras a menor escala, esto es, guerras entre las naciones más pequeñas. Si bien, las guerras a nivel mundial continuarán hasta que no se establezca el gobierno de la humanidad. La soberanía global prevendrá guerras globales ---nada más puede hacerlo---.
\vs p134 6:5 Los cuarenta y ocho estados libres americanos coexisten en paz. Entre los ciudadanos de estos cuarenta y ocho estados se hallan todas las distintas nacionalidades y razas que viven en las naciones, siempre en guerra, de Europa. Estos norteamericanos representan a casi todas las religiones y sectas y sistemas de culto religioso de todo el mundo y, aún así, aquí en Norteamérica conviven en paz. Y todo esto es posible debido a que estos cuarenta y ocho estados han renunciado a su soberanía y abandonado cualquier noción de unos pretendidos derechos a la autodeterminación.
\vs p134 6:6 No se trata de una cuestión armamentística de desarme o no. Tampoco la cuestión del servicio militar obligatorio o voluntario forma parte de estas dificultades de mantener la paz mundial. Si le arrebatáis a las naciones poderosas cualquier forma de armas mecánicas modernas y cualquier tipo de explosivos, pelearán con los puños, piedras y palos siempre y cuando se aferren a sus delirios del derecho divino a la soberanía nacional.
\vs p134 6:7 La guerra no es la gran y terrible enfermedad del hombre; la guerra es un síntoma, una consecuencia. La verdadera enfermedad es el virus de la soberanía nacional.
\vs p134 6:8 Las naciones de Urantia nunca han poseído una verdadera soberanía; nunca han gozado de una soberanía que pudiese protegerlas de los estragos y de las devastaciones de las guerras mundiales. Al constituirse un gobierno global de la humanidad, las distintas naciones no renuncian a su soberanía, sino que establecen una soberanía mundial real, genuina y perdurable, que en adelante será capaz de protegerlas de todas las guerras. Los gobiernos locales gestionarán los asuntos locales; los gobiernos nacionales, los asuntos nacionales; el gobierno global administrará los asuntos internacionales.
\vs p134 6:9 La paz mundial no puede mantenerse mediante tratados, diplomacia, política exterior, alianzas, equilibrio de poder ni realizando cualquier otro tipo de juegos malabares improvisados con las soberanías del nacionalismo. Debe instituirse una ley a escala mundial y el gobierno global debe hacerla valer ---la soberanía de toda la humanidad---.
\vs p134 6:10 Bajo un gobierno mundial, las criaturas, a nivel individual, disfrutarán de mucha más libertad. Hoy día, los ciudadanos de las grandes potencias están sujetos a impuestos y se les regula y controla casi de forma opresiva. Y gran parte de esta actual injerencia en la libertad individual desaparecerá cuando los gobiernos nacionales estén dispuestos a confiar su soberanía, en cuanto a asuntos internacionales, en manos del gobierno global.
\vs p134 6:11 Bajo un gobierno mundial, se le brindará a las agrupaciones nacionales una verdadera oportunidad de realizar y gozar de las libertades personales de una genuina democracia. Se pondrá fin a la falacia de la autodeterminación. Con la regulación global de los asuntos monetarios y comerciales vendrá la nueva era de la paz mundial. Pronto se desarrollará una lengua global, y habrá al menos alguna esperanza de que haya, con el tiempo, una religión global ---o unas religiones con un punto de vista global---.
\vs p134 6:12 La seguridad colectiva jamás proporcionará la paz mientras que la colectividad no incluya a toda la humanidad.
\vs p134 6:13 La soberanía política de un gobierno representativo de la humanidad traerá a la tierra una paz perdurable, y la fraternidad espiritual de los hombres garantizará para siempre la buena voluntad entre todos los hombres. Y no hay ninguna otra forma de poder lograr la paz en la tierra y la buena voluntad entre los hombres.
\separatorline
\vs p134 6:14 Después de la muerte de Cimboitón, sus hijos tropezaron con grandes dificultades para mantener la paz en el profesorado. Las enseñanzas de Jesús habrían tenido una mayor repercusión si los maestros cristianos que se unirían más tarde al personal docente de Urmia hubieran mostrado más sabiduría y tolerancia.
\vs p134 6:15 El hijo mayor de Cimboitón solicitó la ayuda en Filadelfia de Abner, pero la elección de maestros de Abner fue muy desafortunada por el hecho de que estos resultaron ser inflexibles e intransigentes. Estos maestros trataron de hacer que su religión prevaleciera sobre todas las otras creencias. Nunca sospecharían que aquellas conferencias del guía de la caravana, a las que tan frecuentemente se aludía, las había impartido el mismo Jesús.
\vs p134 6:16 Al aumentar la confusión en el profesorado, los tres hermanos retiraron su apoyo económico y, cinco años después, la escuela se cerró. Se reabriría más tarde como templo mitraico y, con el tiempo, ardería por completo con motivo de una de sus celebraciones orgiásticas.
\usection{7. SU TRIGÉSIMO PRIMER AÑO (AÑO 25 D.C.)}
\vs p134 7:1 Cuando regresó del mar Caspio, Jesús sabía que sus viajes por el mundo estaban a punto de concluir. Solo hizo un viaje más fuera de Palestina, y fue para ir a Siria. Tras una breve estancia en Cafarnaúm, se dirigió a Nazaret, donde se quedó unos pocos días de visita. A mediados de abril, salió de Nazaret en dirección a Tiro. Desde allí viajó hacia el norte con destino a Antioquía, aunque se detuvo varios días en Sidón.
\vs p134 7:2 Este es el año de los recorridos en solitario de Jesús por Palestina y Siria. Durante este año de viajes, se le conoció por distintos nombres en diferentes partes del país: el carpintero de Nazaret, el fabricante de barcos de Cafarnaúm, el escriba de Damasco y el maestro de Alejandría.
\vs p134 7:3 En Antioquía, el Hijo del Hombre vivió más de dos meses, trabajando, observando, estudiando, visitando, asistiendo espiritualmente y aprendiendo a la vez cómo vive el hombre, cómo piensa, siente y responde a su entorno humano. En este período, trabajó como fabricante de tiendas durante tres semanas. Permaneció más tiempo en Antioquía que en ningún otro lugar de los que había visitado en este viaje. Diez años después, cuando Pablo predicaba en Antioquía y oyó a sus seguidores hablar de las doctrinas del \bibemph{escriba de Damasco,} poco imaginaba que sus discípulos habían oído la voz y escuchado las enseñanzas del Maestro mismo.
\vs p134 7:4 Desde Antioquía, Jesús viajó hacia el sur, a lo largo de la costa, hasta Cesarea, donde se quedó unas pocas semanas, continuando luego por la costa hasta Jope. Desde Jope se encaminó tierra adentro hasta Jamnia, Asdod y Gaza. Desde Gaza tomo un sendero interior hasta Beerseba, donde estuvo una semana.
\vs p134 7:5 Jesús comenzó entonces privadamente su recorrido final por el corazón de Palestina, dirigiéndose desde Beerseba en el sur hasta Dan en el norte. En este viaje en dirección norte, se detuvo en Hebrón, Belén (viendo la ciudad en la que nació), Jerusalén (no visitó Betania), Beerot, Lebona, Sicar, Siquem, Samaria, Geba, En\hyp{}Ganim, Endor, Madón; tras atravesar Magdala y Cafarnaúm, continuó hacia el norte; y, pasando por el este de las Aguas de Merom, se encaminó por Cárata hasta Dan o Cesarea de Filipo.
\vs p134 7:6 El modelador del pensamiento interior llevó ahora a Jesús a alejarse de los lugares donde moraban los hombres y lo hizo dirigirse hasta el monte Hermón para que pudiese acabar de tener dominio sobre su mente humana y llevar a cabo su consagración plena a lo que le restaba de su labor de vida en la tierra.
\vs p134 7:7 Se trató de una esas épocas singulares y extraordinarias de la vida terrenal del Maestro en Urantia. Otra vivencia de carácter muy similar fue la que experimentó al encontrarse a solas en las colinas cercanas a Pella, justo después de su bautismo. Este período de aislamiento en el monte Hermón señaló el final de su andadura puramente humana, es decir, la terminación formal de su ministerio de gracia como mortal, mientras que su posterior aislamiento señaló el comienzo de la faceta más divina de este ministerio. Y Jesús vivió en soledad con Dios durante seis semanas en las laderas del monte Hermón.
\usection{8. ESTANCIA EN EL MONTE HERMÓN}
\vs p134 8:1 Tras pasar algún tiempo en los alrededores de Cesarea de Filipo, Jesús se abasteció de provisiones y, una vez obtenidos un animal de carga y los servicios de un muchacho llamado Tiglat, continuó por el camino de Damasco hasta un poblado conocido en otro tiempo como Beit Jenn, a los pies del monte Hermón. Aquí, casi a mediados de agosto del año 25 d.C., estableció su campamento y, dejando sus provisiones en custodia de Tiglat, ascendió por las solitarias laderas de la montaña. Tiglat acompañó a Jesús este primer día de ascenso hasta un lugar determinado, a casi 2000 metros de altura sobre el nivel del mar. Allí construyeron un receptáculo de piedra, en el que el joven tendría que depositar alimentos dos veces a la semana.
\vs p134 8:2 El primer día, una vez que había dejado a Tiglat, Jesús no había ascendido sino un corto tramo de montaña cuando se detuvo para orar. Entre otras cosas rogó a su Padre que enviara de nuevo al serafín guardián para “estar con Tiglat”. Y le pidió que le permitiera afrontar, por sí solo, su última lucha con las realidades de la existencia mortal. Y se le concedió su petición. Emprendió la gran prueba solamente con la guía y el sostén de su modelador interior.
\vs p134 8:3 \pc Mientras permaneció en la montaña, Jesús comió con austeridad; se abstuvo de cualquier alimento solo un día o dos a la vez. Los seres sobrenaturales que se enfrentaron a él en la montaña, y con quienes contendió en espíritu y derrotó en poder, eran \bibemph{reales;} eran sus archienemigos del sistema de Satania; no eran fantasmas de la imaginación, resultado de los desvaríos intelectuales de un mortal debilitado y hambriento, que no pudiese distinguir la realidad de las visiones de una mente trastornada.
\vs p134 8:4 Jesús pasó las tres últimas semanas de agosto y las tres primeras semanas de septiembre en el monte Hermón. Durante estas semanas, finalizó la labor humana de lograr los círculos en cuanto a la comprensión de su mente y a la conquista de su ser personal. A lo largo de todo este período de comunión con su Padre celestial, el modelador interior también prestó el servicio al que estaba asignado. Se alcanzó allí la meta humana de esta criatura del mundo. Solamente faltaba por consumarse la fase final: la sintonización de la mente y el modelador.
\vs p134 8:5 Tras más de cinco semanas de comunión ininterrumpida con su Padre del Paraíso, Jesús llegó a tener la absoluta seguridad de su naturaleza y el convencimiento del triunfo de su persona sobre los niveles de expresión materiales y espacio temporales. Creyó firmemente en el predominio de su naturaleza divina sobre su naturaleza humana, y no vaciló en afirmarlo.
\vs p134 8:6 \pc Al acercarse el final de su estancia en la montaña, Jesús pidió a su Padre que le fuese permitido celebrar una conferencia con sus enemigos de Satania como Hijo del Hombre, como Josué ben José. Y se le concedió dicha petición. Durante su última semana en el monte Hermón, ocurrió la gran tentación, la prueba del universo. Satanás (representando a Lucifer) y Caligastia, el príncipe planetario rebelde, estuvieron presentes con Jesús y se hicieron perfectamente visibles. Y dicha “tentación”, tal prueba final de lealtad humana ante las falsedades de estos seres rebeldes, no tuvo nada que ver con la comida ni con los pináculos del templo ni con actos arrogantes. No guardó relación con los reinos de este mundo, sino con la soberanía sobre un poderoso y glorioso universo. El simbolismo de vuestros registros tenía como objeto las eras atrasadas en la que se desenvolvía el pueril pensamiento del mundo. Y las postreras generaciones deberían entender la gran contienda que libró el Hijo del Hombre en el monte Hermón aquel memorable día.
\vs p134 8:7 A las muchas propuestas y contrapropuestas de los emisarios de Lucifer, Jesús solo daba una respuesta: “Que impere la voluntad de mi Padre del Paraíso y que a ti, mi hijo rebelde, te juzguen los Ancianos de Días de manera divina. Soy tu padre\hyp{}creador; no podría juzgarte con justicia, y ya habéis despreciado mi misericordia. Te encomiendo al dictamen de los jueces de un universo más grande”.
\vs p134 8:8 Ante todas las maquinaciones y tácticas sugeridas por Lucifer, ante todas esas engañosas propuestas sobre el ministerio de gracia de su encarnación, Jesús solo tenía una respuesta: “Que se haga la voluntad de mi Padre del Paraíso”. Cuando la dura prueba terminó, el serafín guardián, que estaba apartado, retornó al lado de Jesús para asistirle.
\vs p134 8:9 \pc Una tarde de finales del verano, en medio de los árboles y en el silencio de la naturaleza, Miguel de Nebadón obtuvo la incuestionable soberanía sobre su universo. Ese día completó la tarea establecida para los hijos creadores: vivir plenamente la vida encarnada con la semejanza de un hombre mortal en los mundos evolutivos del tiempo y del espacio. No se anunció este trascendental logro al universo hasta el día de su bautismo, meses más tarde, aunque, en realidad, tuvo lugar ese día, en la montaña. Y, cuando Jesús descendió del monte Hermón, la rebelión de Lucifer en Satania y la secesión de Caligastia en Urantia quedaron prácticamente resueltas. Jesús había pagado el último precio exigido para la obtención de dicha soberanía, la cual, por sí misma, regula el estatus de todos los rebeldes y determina que tales levantamientos futuros (si ocurren alguna vez) se aborden de forma sumaria y efectiva. Por consiguiente, se puede observar que la llamada “gran tentación” de Jesús sucedió algún tiempo antes de su bautismo, y no justo después de este.
\vs p134 8:10 Al concluir dicha estancia en la montaña, conforme descendía, Jesús se encontró con Tiglat que ascendía al lugar fijado para dejar la comida. Al pedirle que se volviera, él solo dijo: “El período de descanso ha acabado; debo retornar a los asuntos de mi Padre”. Durante su viaje de regreso a Dan, iba silencioso y estaba muy cambiado. Allí se despidió del muchacho, regalándole el asno. Luego prosiguió hacia el sur, regresando por el mismo camino que había venido, en dirección a Cafarnaúm.
\usection{9. PERÍODO DE ESPERA}
\vs p134 9:1 El final del verano estaba ya cerca; era el momento del día de la expiación y de la fiesta de los Tabernáculos. Jesús se reunió durante el \bibemph{sabbat} con su familia en Cafarnaúm y, al día siguiente, salió para Jerusalén con Juan, el hijo de Zebedeo. Fueron hacia el este del lago y vía Gerasa, y bajaron por el valle del Jordán. Aunque durante el camino conversó ocasionalmente con su compañero de viaje, Juan notó un gran cambio en él.
\vs p134 9:2 Jesús y Juan pararon en Betania para pernoctar con Lázaro y sus hermanas, partiendo temprano a la mañana siguiente con dirección a Jerusalén. Pasaron casi tres semanas en la ciudad y sus alrededores, al menos Juan lo hizo así. Juan fue muchos días solo a Jerusalén, mientras Jesús paseaba por las colinas cercanas, dedicando muchos períodos a la comunión espiritual con su Padre celestial.
\vs p134 9:3 Ambos estuvieron presentes en los solemnes oficios del día de la expiación. A Juan le impresionaron mucho las ceremonias que tuvieron lugar durante este gran día del ritual de la religión judía, pero Jesús se mantuvo como un espectador, pensativo y silencioso. Al Hijo del Hombre estos ceremoniales le resultaban lamentables y patéticos. Lo percibía todo como una distorsión del carácter y de los atributos de su Padre celestial. Observaba las actividades de este día como si fuesen una parodia de los hechos de la justicia divina y de las verdades de la misericordia infinita. Le consumía por dentro no poder dar rienda suelta a su proclamación de la auténtica verdad sobre el carácter amoroso de su Padre y su misericordioso proceder en el universo, pero su leal mentor le advirtió que su hora aún no había llegado. No obstante, ese día por la noche, en Betania, Jesús sí hizo numerosos comentarios que perturbaron bastante a Juan, que jamás llegaría a entender del todo el auténtico significado de lo que Jesús dijo en aquella charla.
\vs p134 9:4 Jesús tenía previsto quedarse con Juan durante toda la semana de la fiesta de los Tabernáculos. Se trataba de una fiesta anual celebrada en toda Palestina; era la época de las vacaciones de los judíos. Aunque Jesús no participó del jolgorio de la ocasión, era evidente que le producía agrado y satisfacción contemplar cómo se abandonaban jóvenes y mayores a la despreocupación y al gozo.
\vs p134 9:5 En medio de la celebración y antes de que acabasen las festividades, Jesús se despidió de Juan, diciendo que deseaba retirarse a las colinas donde podría estar en mayor comunión con su Padre del Paraíso. Juan habría ido con él, pero Jesús le insistió que se quedara durante todos los festejos, diciendo: “No es necesario que soportes la carga del Hijo del Hombre; solo el centinela debe mantenerse vigilante mientras la ciudad duerme en paz”. Jesús no regresaría a Jerusalén. Tras una semana en soledad en las colinas próximas a Betania, partió para Cafarnaúm. De camino a casa, pasó un día y una noche solo en las laderas de Gilboa, cerca del lugar donde el rey Saúl se había quitado la vida; y, cuando llegó a Cafarnaúm, parecía más animado que cuando dejó a Juan en Jerusalén.
\vs p134 9:6 A la mañana siguiente, Jesús se encaminó al arca que contenía sus efectos personales, y que había permanecido en el taller de Zebedeo; se puso su delantal, y se presentó al trabajo, diciendo: “Me corresponde mantenerme ocupado mientras espero a que llegue mi hora”. Y trabajó algunos meses, hasta enero del año siguiente en la factoría de barcos al lado de su hermano Santiago. Tras trabajar junto a Jesús durante este período de tiempo y, al margen de las dudas que pudieran sobrevenirle y que enturbiaban su entendimiento sobre la labor de vida del Hijo del Hombre, Santiago nunca más volvería a cejar en su fe en la misión de Jesús, en la que creía verdadera y enteramente.
\vs p134 9:7 En el trascurso de este último periodo de Jesús como operario de la factoría, dedicó la mayor parte del tiempo al acabado interior de algunas de las embarcaciones más grandes. Se esmeró mucho en su labor manual y parecía sentir la satisfacción del logro humano cada vez que acababa un trabajo encomiable. Aunque no perdía el tiempo en nimiedades, era un artesano minucioso cuando se trataba de lo esencial de cualquier compromiso adquirido.
\vs p134 9:8 \pc Conforme pasaba el tiempo, llegaron a Cafarnaúm rumores de la aparición de cierto Juan que predicaba a la vez que bautizaba en el Jordán a los penitentes, y que predicaba: “Arrepentíos y sed bautizados, porque el reino de los cielos se ha acercado”. Jesús oía estas noticias mientras Juan se abría camino lentamente hasta el valle del Jordán desde el vado del río más cercano a Jerusalén. Pero Jesús continuó trabajando, fabricando embarcaciones, hasta que Juan se dirigió río arriba, a un lugar próximo a Pella, en el mes de enero del siguiente año, el 26 d. C. Entonces dejó sus herramientas, declarando, “Mi hora ha llegado”, y pronto se presentaría ante Juan para ser bautizado.
\vs p134 9:9 Si bien, se había producido un gran cambio en Jesús. Pocos de quienes habían disfrutado con sus charlas y ministerio en sus trayectos por el territorio llegarían alguna vez a reconocer más tarde en el maestro público, a la misma persona a la que habían conocido y amado privadamente en años anteriores. Y había una razón por la que aquellos que se beneficiaron primeramente de sus enseñanzas no lo reconocieran en su papel posterior de maestro público con autoridad. Durante muchos años, se había venido realizando esta transformación de mente y espíritu, y había concluido durante su memorable estancia en el monte Hermón.
