\upaper{194}{La dádiva del espíritu de la verdad}
\author{Comisión de seres intermedios}
\vs p194 0:1 En torno a la una, unos ciento cincuenta creyentes que se encontraban reunidos orando percibieron una extraña presencia en la sala. En ese mismo momento, todos estos discípulos tomaron conciencia de una sensación, nueva y profunda, de gozo, seguridad y confianza espiritual. Esta conciencia de fortaleza espiritual vino enseguida acompañada de un poderoso impulso a salir y a proclamar públicamente el evangelio del reino y la buena nueva de que Jesús había resucitado de entre los muertos.
\vs p194 0:2 Pedro se puso de pie y manifestó que debía tratarse de la venida del espíritu de la verdad que el Maestro les había prometido, y propuso que fueran al templo para comenzar la proclamación de la buena nueva encomendada a sus manos. E hicieron lo que Pedro les sugirió.
\vs p194 0:3 \pc Estos hombres se habían formado e instruido en la predicación del evangelio de la paternidad de Dios y de la filiación del hombre, pero en aquel preciso momento de euforia espiritual y de triunfo personal, la mejor noticia, la más grandiosa nueva, en la que estos hombres eran capaces de pensar era en \bibemph{el hecho} del Maestro resucitado. Y así, dotados de poder de lo alto, salieron a predicar a la gente las buenas nuevas ---e incluso la salvación por medio de Jesús---, pero indeliberadamente cayeron en el error de sustituir algunos de los hechos relacionados con el evangelio por el mensaje del evangelio mismo. Pedro, sin saberlo, inició este error, y otros que vinieron tras él lo siguieron hasta llegar a Pablo, que creó una nueva religión fundamentada en la reciente versión de la buena nueva.
\vs p194 0:4 El evangelio del reino consiste en el hecho de la paternidad de Dios, sumado a la consiguiente verdad de la filiación y de la hermandad de los hombres. El cristianismo, tal como se desarrolló desde ese día, consiste en el hecho de Dios como Padre del Señor Jesucristo, junto con la experiencia fraternal de los creyentes con el Cristo resucitado y glorificado.
\vs p194 0:5 No es extraño que estos hombres infundidos por el espíritu se hubieran servido de esta oportunidad para expresar sus sentimientos de triunfo sobre las fuerzas que habían intentado acabar con su Maestro, queriendo poner fin al influjo de sus enseñanzas. En un momento así, les resultó más fácil recordar su relación personal con Jesús y entusiasmarse con la certeza de que el Maestro seguía vivo, que su amistad no había terminado y que el espíritu, en efecto, había descendido sobre ellos tal como él había prometido.
\vs p194 0:6 Estos creyentes se sintieron de repente trasladados a otro mundo, a una nueva existencia de gozo, poder y gloria. El Maestro les había dicho que el reino vendría con poder y, algunos de ellos, empezaban a comprender lo que él había querido decir.
\vs p194 0:7 Y, cuando se tienen en cuenta todas estas circunstancias, no es difícil entender por qué estos hombres emprendieron la predicación de un \bibemph{nuevo evangelio sobre Jesús} en lugar de su anterior mensaje de la paternidad de Dios y de la fraternidad de los hombres.
\usection{1. EL SERMÓN DE PENTECOSTÉS}
\vs p194 1:1 Los apóstoles habían estado escondidos durante cuarenta días. Aquel día resultó ser la fiesta judía de Pentecostés, y miles de visitantes de todas las partes del mundo habían acudido a Jerusalén. Muchos habían venido para celebrar este día festivo, pero la mayoría permanecía en la ciudad desde la Pascua. En ese momento, estos amedrentados apóstoles resurgieron tras semanas de reclusión para presentarse valerosamente en el templo, donde comenzaron a predicar el nuevo mensaje de un Mesías resucitado. Todos los discípulos, asimismo, eran conscientes de haber sido dotados espiritualmente de un nuevo entendimiento y poder.
\vs p194 1:2 Eran sobre las dos cuando Pedro se puso de pie en el lugar mismo en el que su Maestro había enseñado por última vez en este templo e hizo aquel apasionado llamamiento a la gente, que tuvo como resultado que se ganaran más de dos mil almas para el reino. El Maestro ya no estaba, pero, de repente, descubrieron que el relato de su historia ejercía un poderoso efecto sobre las personas. No es de extrañar que se vieran movidos a continuar con la proclamación de aquello que justificaba su anterior devoción a Jesús y que, al mismo tiempo, impelía a los hombres a creer en él. Seis de los apóstoles participaron en esta ocasión: Pedro, Andrés, Santiago, Juan, Felipe y Mateo. Hablaron durante más de una hora y media y expusieron sus mensajes en griego, hebreo y arameo, en incluso dijeron algunas pocas palabras en otras lenguas de las que tenían un leve conocimiento.
\vs p194 1:3 Los líderes de los judíos se quedaron atónitos al ver la osadía de los apóstoles, pero temieron importunarlos por las grandes cantidades de personas que creían en su historia.
\vs p194 1:4 En torno a las cuatro y media, más de dos mil nuevos creyentes siguieron a los apóstoles hasta el estanque de Siloé, donde Pedro, Andrés, Santiago y Juan los bautizaron en nombre del Maestro. Ya había oscurecido cuando terminaron de bautizar a esta multitud.
\vs p194 1:5 Pentecostés era la gran celebración del bautismo, el momento en el que se aceptaban como miembros a los prosélitos de la puerta, o sea, gentiles que deseaban servir a Yahvé. Era, pues, más fácil convencerlos para que, en ese día, se sometieran un gran número de judíos y gentiles al bautismo. Al hacerlo, no se desvinculaban de ninguna manera de la fe judía. Incluso, durante algún tiempo después de esto, los apóstoles de Jesús fueron parte del judaísmo. Todos ellos, incluyendo a los apóstoles, continuaron siendo leales a los requerimientos fundamentales del sistema ceremonial judío.
\usection{2. EL SIGNIFICADO DE PENTECOSTÉS}
\vs p194 2:1 Jesús vivió en la tierra y enseñó un evangelio que rescataba al hombre de la superstición de que él era hijo del mal y lo elevaba a la dignidad de hijo de Dios por la fe. El mensaje de Jesús, tal como él lo predicó y vivió en su tiempo, disolvía perfectamente las dificultades espirituales del hombre de aquellos días. Y, ahora, que su persona ha dejado el mundo, envía en su lugar a su espíritu de la verdad, que está destinado a vivir en el hombre y a reformular, para cada nueva generación, el mensaje de Jesús, de manera que todos los nuevos grupos de mortales que aparezcan sobre la faz de la tierra tengan una versión nueva y actualizada del evangelio, proporcionándoles a la vez tal lucidez personal y tal guía a nivel de grupo como para convertirse en un eficiente disolutivo de las dificultades espirituales del hombre, siempre cambiantes y diversas.
\vs p194 2:2 \pc La primera misión de este espíritu es, manifiestamente, fomentar y hacer significativa la verdad a nivel personal, ya que la comprensión de la verdad representa la forma más elevada de libertad humana. Por otro lado, el propósito de este espíritu es acabar con la sensación de orfandad del creyente. Tras haber estado Jesús entre los hombres, todos los creyentes hubieran experimentado sensación de soledad de no haber venido el espíritu de la verdad para habitar en los corazones de los hombres.
\vs p194 2:3 Esta dádiva del espíritu del Hijo preparó satisfactoriamente la mente de todos los hombres normales para la posterior dádiva del espíritu del Padre (el modelador) a toda la humanidad. En cierto modo, este espíritu de la verdad es el espíritu a la vez del Padre Universal y del hijo creador.
\vs p194 2:4 No cometáis el error de esperar adquirir una fuerte conciencia intelectual del espíritu de la verdad que se ha derramado sobre vosotros. El espíritu nunca crea una conciencia de sí mismo, sino solo una conciencia de Miguel, del Hijo. Desde el comienzo, Jesús enseñó que el espíritu no hablaría de sí mismo. No es posible, por tanto, encontrar la prueba de vuestra fraternidad con el espíritu de la verdad en vuestra propia conciencia de él, sino más bien en el hecho del enaltecimiento de vuestra fraternidad con Miguel.
\vs p194 2:5 El espíritu también vino a ayudar a los hombres a recordar y a entender las palabras del Maestro, al igual que a iluminar espiritualmente y reinterpretar su vida en la tierra.
\vs p194 2:6 Por otro lado, el espíritu de la verdad llegó para ayudar al creyente a dar testimonio de las realidades de las enseñanzas de Jesús y de su vida tal como él la vivió en la carne, y tal como él ahora la vive nuevamente, y de una manera distinta, en cada uno de los creyentes de cualquier generación futura de hijos de Dios, llenos del espíritu.
\vs p194 2:7 Por consiguiente, parece que el espíritu de la verdad vino realmente para guiar a todos los creyentes a toda la verdad, al conocimiento que continúa expandiéndose conforme el creyente experimenta una conciencia espiritual, viva y creciente, de la realidad de su filiación eterna a medida que asciende hasta Dios.
\vs p194 2:8 \pc Jesús vivió una vida que representa la revelación del hombre cuando se somete a la voluntad del Padre; no quiso sentar un ejemplo que cualquier hombre pudiera intentar seguir literalmente. Su vida en la carne, junto con su muerte en la cruz y posterior resurrección, acabó por convertirse en un nuevo evangelio del rescate que se había pagado para liberar al hombre de las garras de la maldad ---de la condenación de un Dios ofendido---. No obstante, aunque el evangelio se distorsionó sobremanera, sigue siendo cierto que este nuevo mensaje sobre Jesús llevaba consigo muchas de las verdades y enseñanzas fundamentales de su temprano evangelio del reino. Y, más tarde o más temprano, estas verdades veladas de la paternidad de Dios y de la hermandad de los hombres resurgirán para transformar convenientemente las civilizaciones de toda la humanidad.
\vs p194 2:9 Pero estos errores del intelecto de ningún modo interfirieron con el gran avance del creyente en cuanto a su crecimiento espiritual. En menos de un mes, tras la dádiva del espíritu de la verdad, los apóstoles progresaron más espiritual y personalmente que durante los casi cuatro años de su afectiva relación personal con el Maestro. Tampoco, esta sustitución del \bibemph{hecho} de la resurrección de Jesús por la \bibemph{verdad} del evangelio salvífico de la filiación con Dios obstaculizó de manera alguna la rápida propagación de sus enseñanzas; al contrario, el hecho de que las nuevas enseñanzas sobre su persona y resurrección eclipsaran el mensaje de Jesús pareció facilitar considerablemente la predicación de esta buena nueva.
\vs p194 2:10 \pc El término “bautismo del espíritu”, cuyo uso llegó a generalizarse en esos días, significaba simplemente que se recibía conscientemente este don del espíritu de la verdad y se reconocía personalmente este nuevo poder espiritual, que incrementaba las influencias espirituales que las almas conocedoras de Dios percibían con anterioridad.
\vs p194 2:11 \pc Desde la dádiva del espíritu de la verdad, el hombre está bajo la influencia de la enseñanza y guía de una triple dotación espiritual: el espíritu del Padre, el modelador del pensamiento; el espíritu del hijo creador, el espíritu de la verdad; el espíritu del espíritu materno, el espíritu santo.
\vs p194 2:12 En cierto modo, la humanidad está supeditada a la doble influencia del séptuplo ministerio de las influencias espirituales del universo. Las primeras razas evolutivas de los mortales están sujetas al contacto consecutivo de los siete espíritus asistentes de la mente del espíritu materno del universo local. A medida que el hombre avanza de forma ascendente en la escala de la inteligencia y de la percepción espiritual, finalmente lo envuelven y habitan en él las siete influencias de los espíritus superiores. Y estos siete espíritus de los mundos en avance son:
\vs p194 2:13 \li{1.}El espíritu que el Padre universal confiere: los modeladores del pensamiento.
\vs p194 2:14 \li{2.}La presencia espiritual del Hijo Eterno: la gravedad espiritual del universo de los universos y el infalible canal de cualquier comunión espiritual.
\vs p194 2:15 \li{3.}La presencia espiritual del Espíritu Infinito: la mente\hyp{}espíritu universal de toda la creación, la fuente espiritual de la afinidad intelectual de todas las inteligencias en vías de desarrollo.
\vs p194 2:16 \li{4.}El espíritu del Padre Universal y del hijo creador: el espíritu de la verdad, normalmente considerado como el espíritu del Hijo del Universo.
\vs p194 2:17 \li{5.}El espíritu del Espíritu Infinito y del espíritu materno del universo: el espíritu santo, normalmente considerado como el espíritu del Espíritu del Universo.
\vs p194 2:18 \li{6.}El espíritu\hyp{}mente del espíritu materno del universo: los siete espíritus asistentes de la mente del universo local.
\vs p194 2:19 \li{7.}El espíritu del Padre, de los Hijos y de los Espíritus: el espíritu de nuevo nombre dado a los mortales ascendentes de los mundos, tras la fusión del alma mortal nacida del espíritu con el modelador del pensamiento del Paraíso y tras el posterior logro de la divinidad y la glorificación al convertirse en miembros del colectivo de finalizadores del Paraíso.
\vs p194 2:20 \pc Y, así, la dádiva del espíritu de la verdad brindó al mundo y a sus pueblos la última dotación espiritual concebida para asistir a los ascendentes mortales en su búsqueda de Dios.
\usection{3. LO OCURRIDO EN PENTECOSTÉS}
\vs p194 3:1 Se incorporaron muchas enseñanzas raras y extrañas a los primeros relatos del día de Pentecostés. En los tiempos que siguieron, los hechos acontecidos ese día de la llegada del espíritu de la verdad, del nuevo maestro, a inhabitar la humanidad, se han llegado a confundir con estallidos irreflexivos de desmesurado sentimentalismo. La misión principal de este espíritu, que el Padre y el hijo derramaron, es la de enseñar a los hombres sobre las verdades del amor del Padre y de la misericordia del Hijo. Se trata de verdades divinas que los hombres pueden comprender con más facilidad que todos los demás rasgos del carácter divino. El espíritu de la verdad se ocupa primordialmente de revelar la naturaleza espiritual del Padre y el carácter moral del Hijo. El hijo creador, en la carne, reveló Dios a los hombres; el espíritu de la verdad, en el corazón, revela el hijo creador a los hombres. Cuando el hombre rinde los “frutos del espíritu” en su vida, simplemente muestra los rasgos personales que el Maestro manifestó en su propia vida terrenal. Cuando Jesús estuvo en la tierra, vivió su vida como una persona única ---como Jesús de Nazaret---. Como espíritu morador del “nuevo maestro”, el Maestro, desde Pentecostés, ha podido vivir su vida de nuevo en la experiencia de todo creyente que haya sido instruido en la verdad.
\vs p194 3:2 Muchas de las cosas que acontecen en el transcurso de la vida son arduas de entender, son difícilmente reconciliables con la idea de que este es un universo en el que prevalece la verdad y triunfa la rectitud. Con mucha frecuencia, parece que lo que se impone es la injuria, la mentira, la deshonestidad y la falta de rectitud ---el pecado---. ¿Pero, después de todo, triunfa la fe sobre el mal, el pecado y la iniquidad? Sí, triunfa. La vida y la muerte de Jesús constituyen una prueba eterna de que la verdad de la bondad y la fe de la criatura guiada por el espíritu siempre se harán valer. Se burlaron de Jesús en la cruz diciendo: “Veamos si Dios viene y lo libera”. El día de la crucifixión estuvo en tinieblas, pero la mañana de la resurrección estuvo luminosa y gloriosa; el día de Pentecostés estuvo incluso más radiante y jubiloso. Las religiones del pesimismo y la desesperación buscan librarse de las cargas de la vida; ansían la extinción en un interminable sueño y descanso. Estas son las religiones del temor y del pavor primitivo. La religión de Jesús es un nuevo evangelio de fe que ha de proclamarse a la esforzada humanidad. Esta nueva religión se funda en la fe, la esperanza y el amor.
\vs p194 3:3 La vida mortal golpeó a Jesús con su mayor rigurosidad, crueldad y crudeza; pero como hombre se enfrentó a estas desesperantes vicisitudes con fe, arrojo y con la firme determinación de hacer la voluntad del Padre. Jesús se enfrentó a la vida en su más terrible faceta y la venció, incluso en la muerte. No usó la religión para liberarse de la vida. La religión de Jesús no busca escapar de esta vida para disfrutar de una esperada dicha en otra existencia. La religión de Jesús proporciona el gozo y la paz de otra existencia, de una nueva existencia espiritual para enaltecer y ennoblecer la vida que los hombres viven ahora en la carne.
\vs p194 3:4 Si la religión es el opio del pueblo, no lo es la religión de Jesús. En la cruz, él se negó a beber el brebaje anestesiante, y su espíritu, derramado sobre toda carne, imparte un magnífico ministerio a escala mundial que lleva al hombre hacia lo alto y lo insta a ir hacia adelante. El afán espiritual a continuar avanzando es la fuerza impulsora más extraordinaria que existe en este mundo; el creyente que aprende la verdad constituye la verdadera alma progresiva y enérgica de la tierra.
\vs p194 3:5 El día de Pentecostés, la religión de Jesús rompió todas las restricciones nacionales y las barreras raciales. Es para siempre verdad que “donde está el Espíritu del Señor, allí hay libertad”. En ese día, el espíritu de la verdad se convirtió en el don personal del Maestro para cada uno de los mortales. Se otorgó este espíritu con el propósito de capacitar a los creyentes para que predicaran el evangelio del reino con mayor efectividad, pero ellos confundieron el hecho de recibir el espíritu derramado sobre ellos con una parte del nuevo evangelio que, involuntariamente, habían empezado a elaborar.
\vs p194 3:6 \pc No obviéis el hecho de que el espíritu de la verdad se otorgó a todos los creyentes sinceros; este don del espíritu no vino únicamente a los apóstoles. En su totalidad, los ciento veinte hombres y mujeres congregados en el aposento alto recibieron al nuevo maestro, como también lo recibieron, en todo el mundo, todas las personas honestas de corazón. Este nuevo maestro se otorgó a la humanidad, y cada alma lo recibió conforme a su amor a la verdad y a su aptitud para comprender y aprehender las realidades espirituales. Por fin, la religión verdadera se liberaba de la custodia de los sacerdotes y de todas las castas sagradas, y hallaba su manifestación real e individual en las almas de los hombres.
\vs p194 3:7 \pc La religión de Jesús fomenta el desarrollo del más elevado orden de civilización humana, en cuanto que trae consigo el más elevado orden de persona espiritual y proclama la condición sagrada de tal persona.
\vs p194 3:8 La venida del espíritu de la verdad en Pentecostés hizo posible una religión que no era ni radical ni conservadora; ni antigua ni nueva; no dominada por mayores ni jóvenes. El hecho de la vida terrenal de Jesús facilita un punto estacionario para el anclaje del tiempo, mientras que la dádiva del espíritu de la verdad posibilita la perdurable expansión y el ilimitado desarrollo de la religión que él vivió y del evangelio que él proclamó. El espíritu guía a \bibemph{toda} la verdad; él es el maestro de una religión que se expande y crece continuamente hacia un progreso y un despliegue divino sin fin. Este nuevo maestro irá permanentemente desvelando al creyente, buscador de la verdad, lo que estaba divinamente contenido en la persona y en la naturaleza del Hijo del Hombre.
\vs p194 3:9 Las manifestaciones que acompañaron a la dádiva del “nuevo maestro”, y la buena acogida de la predicación de los apóstoles por parte de personas de distintas razas y naciones congregados en Jerusalén, manifiestan la universalidad de la religión de Jesús; el evangelio del reino no debía identificarse con ninguna raza, cultura o idioma en particular. Ese día de Pentecostés evidenció el gran esfuerzo del espíritu por liberar la religión de Jesús de las heredadas ataduras judías. Incluso después de esta demostración del derramamiento del espíritu sobre toda carne, los apóstoles, en un principio, trataron de imponer las exigencias del judaísmo a sus conversos. Hasta Pablo tuvo problemas con sus hermanos de Jerusalén por negarse a someter a los gentiles a estas prácticas judías. Ninguna religión revelada puede difundirse a todo el mundo si comete la grave equivocación de dejarse imbuir por alguna cultura nacional o de coligarse con prácticas raciales, sociales o económicas ya establecidas.
\vs p194 3:10 La dádiva del espíritu de la verdad se realizó al margen de formalismos, ceremonias, lugares sagrados y comportamiento especial de los que recibieron su manifestación en su plenitud. Cuando el espíritu vino sobre quienes se congregaban en el aposento alto, estaban simplemente sentados allí, justo tras haber estado silenciosamente recogidos en oración. El espíritu se otorgó en los campos al igual que en las ciudades. No fue necesario que los apóstoles se retiraran a un lugar aislado y que pasaran años meditando en soledad para recibir el espíritu. Para siempre, Pentecostés desvincula la idea de las vivencias espirituales de la noción de que estas se tienen que producir en un entorno especialmente favorable.
\vs p194 3:11 \pc Pentecostés, con su dotación espiritual, se concibió para que la religión del Maestro se desprendiera permanentemente de cualquier dependencia a la fuerza física; los maestros de esta nueva religión están provistos de armas espirituales. Deben salir a conquistar el mundo con inagotable clemencia, inigualable buena voluntad y amor abundante. Están equipados para vencer el mal con el bien, para derrotar el odio con el amor y para derribar el temor con la fe, valiente y viva, en la verdad. Jesús ya había enseñado a sus seguidores que su religión nunca era pasiva; sus apóstoles debían ser siempre activos y positivos en el ejercicio de su ministerio de la misericordia y en sus manifestaciones del amor. Ya no consideraban estos creyentes a Yahvé como “el Señor de los Ejércitos”, sino que ahora contemplaban a la Deidad eterna como “Dios y Padre del Señor Jesucristo”. Lograron, al menos, ese avance, incluso si, en cierta medida, no consiguieran llegar a comprender por completo la verdad de que Dios es asimismo el Padre espiritual de todos de manera individual
\vs p194 3:12 Pentecostés dotó al hombre mortal del poder de perdonar las injurias personales, de mantener su dulzura en medio de las más graves injusticias, de permanecer firme frente a los peligros más acechantes y de desafiar los males del odio y de la ira, actuando con valentía, amor y tolerancia. A través de su historia, Urantia ha sufrido los destructivos estragos de grandes guerras. Todos los que participaron en estas terribles luchas sufrieron la derrota. Solo hubo un vencedor; solo hubo alguien que consiguió salir de estas encarnizadas contiendas con mayor renombre, y se trata de Jesús de Nazaret y de su evangelio que vence el mal con el bien. El secreto de una mejor civilización está vinculado a las enseñanzas del Maestro sobre la hermandad del hombre, la buena voluntad del amor y la confianza mutua.
\vs p194 3:13 Hasta Pentecostés, la religión se había revelado tan solo al hombre que buscaba a Dios; desde Pentecostés, el hombre aún busca a Dios, pero brilla sobre el mundo el gran acontecimiento de que Dios igualmente busca al hombre y que envía a su espíritu para que viva en él cuando lo encuentra.
\vs p194 3:14 \pc Antes de las enseñanzas de Jesús, que culminarían en Pentecostés, las mujeres tenían poco o ningún estatus espiritual en los postulados de las religiones más antiguas. Después de Pentecostés, en la hermandad del reino, la mujer está ante Dios en igualdad con el hombre. Entre las ciento veinte personas que recibieron este don especial del espíritu había muchas de las mujeres discípulas, y ellas fueron copartícipes de estas bendiciones con los creyentes varones. Nunca más puede el hombre pretender atribuirse el ministerio del servicio religioso. El fariseo puede seguir agradeciendo a Dios el “no haber nacido mujer, leproso o gentil”, pero, entre los seguidores de Jesús, la mujer ha sido permanentemente emancipada de cualquier discriminación religiosa basada en el sexo. Pentecostés suprimió toda discriminación religiosa basada en distinciones raciales, diferencias culturales, castas sociales o prejuicios sexuales. No es de extrañar que estos creyentes de la nueva religión clamaran en voz alta: “Donde está el Espíritu del Señor, allí hay libertad”.
\vs p194 3:15 \pc La madre y un hermano de Jesús se encontraban también entre los ciento veinte creyentes y, como integrantes de este grupo ordinario de discípulos, participaron igualmente del derramamiento del espíritu. No recibieron de esta buena dádiva más que sus semejantes. A los miembros de la familia terrenal de Jesús no se les otorgó ningún don particular. Pentecostés señalaba el fin de sacerdocios especiales y de cualquier creencia en familias sagradas.
\vs p194 3:16 \pc Antes de Pentecostés los apóstoles habían hecho muchas renuncias por Jesús. Habían sacrificado sus hogares, familias, amigos, bienes terrenales y ocupaciones. En Pentecostés, se dieron a Dios, y el Padre y el Hijo respondieron dándose al hombre ---enviando sus espíritus para que habitaran en él---. Esta vivencia de perder el ego y de encontrar el espíritu no fue de índole emocional; fue un acto de entrega juiciosa de sí mismos y de consagración incondicional.
\vs p194 3:17 Pentecostés significó un llamamiento a la unidad espiritual de los creyentes del evangelio. Cuando el espíritu descendió sobre los discípulos en Jerusalén, también ocurrió en Filadelfia, en Alejandría y en todos los demás lugares donde vivían verdaderos creyentes. Era literalmente cierto que “la multitud de los creyentes era de un corazón y de un alma”. La religión de Jesús ejerce sobre los creyentes el mayor y extraordinario efecto unificador jamás conocido antes en el mundo.
\vs p194 3:18 \pc Con Pentecostés, se reducía la arrogancia de individuos, grupos, naciones y razas. Es esta actitud arrogante la que hace aumentar las tensiones hasta llegar a desatar periódicamente guerras destructivas. La humanidad solo podrá unificarse desde una posición espiritual, y el espíritu de la verdad ejerce esa influencia a nivel mundial, y asimismo universal.
\vs p194 3:19 La llegada del espíritu de la verdad purifica el corazón humano y lleva a sus receptores a plantearse vivir la vida con el único propósito de hacer la voluntad de Dios y buscar el bien de los hombres. La tendencia materialista hacia el egoísmo ha quedado absorbida por esta nueva dádiva espiritual de la solidaridad. Pentecostés, entonces y ahora, significa que el Jesús de la historia se ha convertido en el Hijo divino de la experiencia viva. El gozo de este espíritu derramado, cuando se vivencia conscientemente en la vida humana, tonifica la salud, estimula la mente y vigoriza constantemente el alma.
\vs p194 3:20 \pc No fue la oración la que trajo consigo al espíritu el día de Pentecostés, pero sí contribuyó considerablemente a determinar la capacidad de receptividad individual de cada creyente. La oración no mueve al corazón divino a incrementar la abundancia de sus bendiciones, pero, muy a menudo, horada canales más extensos y profundos por los que estas dádivas divinas pueden fluir a los corazones y a las almas de quienes recuerdan, pues, mantenerse en continua comunión con su Hacedor en ferviente oración y en auténtica adoración.
\usection{4. LOS COMIENZOS DE LA IGLESIA CRISTIANA}
\vs p194 4:1 Cuando, los enemigos de Jesús lo apresaron tan de repente y lo crucificaron tan rápidamente entre dos ladrones, sus apóstoles y discípulos cayeron en un gran desánimo. El pensamiento de su Maestro arrestado, atado, azotado y crucificado tuvo un efecto desolador en los apóstoles. Olvidaron sus enseñanzas y advertencias. Ciertamente, podía haber sido “un profeta poderoso en obra y en palabra delante de Dios y de todo el pueblo”, pero difícilmente podría ser el Mesías que ellos esperaban que hubiera restaurado el reino de Israel.
\vs p194 4:2 Seguidamente, viene la resurrección, que los libra de la desesperación y les devuelve su fe en la divinidad del Maestro. Una y otra vez lo ven y hablan con él, y él los lleva al Monte de los Olivos, donde se despide de ellos y les dice que vuelve al Padre. Les ha pedido que se queden en Jerusalén hasta que se les dote de poder ---hasta que el espíritu de la verdad venga---. Y, en el día de Pentecostés, llega este nuevo maestro, y ellos salen enseguida a predicar su evangelio, investidos con un nuevo poder. Son los intrépidos y valerosos seguidores de un Señor vivo, no de un líder muerto ni derrotado. El Maestro vive en los corazones de estos evangelistas; Dios no es una doctrina en sus mentes; se ha convertido en una presencia viva en sus almas.
\vs p194 4:3 “Día a día perseveraron unánime y firmemente juntos en el templo y partiendo el pan en casa. Comían con alegría y sencillez de corazón, alabando a Dios y teniendo favor con todo el pueblo. Todos estaban llenos del espíritu, y hablaban con arrojo la palabra de Dios. Y las multitudes de los que habían creído eran de un solo corazón y alma; y ninguno decía ser suyo propio nada de lo que poseía, sino que tenían todas las cosas en común”.
\vs p194 4:4 \pc ¿Qué les ha ocurrido a estos hombres a quienes Jesús había ordenado para que salieran a predicar el evangelio del reino ---la paternidad de Dios y la hermandad del hombre---? Tienen un nuevo evangelio; sus corazones arden tras esta nueva experiencia de recibir el espíritu; se llenan de una nueva energía espiritual. Su mensaje ha cambiado de repente y ahora proclaman al Cristo resucitado: “Jesús Nazareno, varón aprobado por Dios por sus obras poderosas y portentos; a él, entregado por el determinado consejo y anticipado conocimiento de Dios, en verdad lo crucificasteis y matasteis. Las cosas que Dios predijo por boca de todos los profetas, las ha cumplido. A este Jesús Dios lo levantó. Dios lo ha hecho Señor y Cristo a la vez. Siendo, por la diestra de Dios, exaltado y habiendo recibido del Padre la promesa del espíritu, ha derramado lo que veis y oís. Arrepentíos, para que sean borrados vuestros pecados; para que el Padre envíe al Cristo que os fue antes anunciado, a Jesús, a quien el cielo ha de recibir hasta los tiempos de la restauración de todas las cosas”.
\vs p194 4:5 De repente, el evangelio del reino, el mensaje de Cristo, se ha transformado en el evangelio del Señor Jesucristo. Ahora proclamaban los hechos de su vida, muerte y resurrección, y predicaban la esperanza de su pronto regreso a este mundo para culminar la labor comenzada. Así pues, el mensaje de los primeros creyentes abordaba la predicación de los hechos de su primera venida y enseñaba la esperanza de su segunda venida, un acontecimiento que creyeron inminente.
\vs p194 4:6 Cristo estaba próximo a convertirse en el credo de la Iglesia que se estaba formando con rapidez. Jesús vive; murió por los hombres; dio su espíritu; regresará de nuevo. Jesús llenaba todos sus pensamientos y definía la totalidad de su nuevo concepto de Dios y de todo lo demás. Estaban demasiado entusiasmados con la nueva doctrina de que “Dios es el Padre del Señor Jesús” como para interesarse por el antiguo mensaje de que “Dios es el Padre amoroso de todos los hombres” y de cada una de las personas. Es cierto que de estas tempranas comunidades de creyentes brotó una magnífica manifestación de amor fraternal y de inigualable buena voluntad. Pero se trataba de una comunidad de creyentes en Jesús, no de una comunidad de hermanos en la familia del reino del Padre de los cielos. Su buena voluntad surgía del amor nacido de su concepto del ministerio de gracia de Jesús y no del reconocimiento de la hermandad de los hombres mortales, Sin embargo, estaban plenos de gozo, y vivieron unas vidas tan nuevas y únicas que todos los hombres se sintieron atraídos hacia sus enseñanzas sobre Jesús. Cometieron la gran equivocación de emplear comentarios acerca del evangelio del reino, que ilustraban la vida de Jesús, en lugar del evangelio mismo, pero, incluso así, aquello representaba la más grandiosa religión que la humanidad hubiera jamás conocido.
\vs p194 4:7 Inequívocamente, en el mundo estaba erigiéndose una nueva comunidad. “Y la multitud de creyentes perseveraban en la doctrina de los apóstoles y en su comunión con ellos, en el partimiento del pan y en las oraciones”. Se llamaban unos a otros hermanos y hermanas; se saludaban entre ellos con un beso santo; llevaban ayuda a los pobres. Era una comunidad de vida al igual que de adoración. No participaban en la comunidad en virtud de ningún mandato, sino animados por el deseo de compartir sus bienes con otros creyentes. Esperaban confiados el regreso de Jesús para que completara la instauración del reino del Padre durante su generación. Esta repartición espontánea de bienes terrenales no derivaba directamente de las enseñanzas de Jesús; se produjo a raíz del ferviente convencimiento de estos hombres y mujeres de que él volvería en cualquier momento para poner fin a su labor y llevar a cabo la culminación del reino. Pero las consecuencias de este bien intencionado experimento de un impulsivo amor fraternal les trajo mucho pesar. Miles de creyentes sinceros vendieron sus propiedades y se deshicieron de todos sus bienes y haberes rentables. Con el paso del tiempo, los recursos, cada vez más escasos a causa del sistema cristiano de “compartir por igual”, legaron a su \bibemph{fin;} aunque no ocurrió así con el mundo. Muy pronto, los creyentes de Antioquía tuvieron que hacer una colecta para que sus compañeros de Jerusalén no murieran por inanición.
\vs p194 4:8 \pc En esos días, celebraban la Cena del Señor de la manera en la se había establecido; esto es, se congregaban entre ellos para tener una amistosa comida, en buena fraternidad, y participaban del sacramento al final de esta.
\vs p194 4:9 \pc En un principio, bautizaban en el nombre de Jesús; aquello fue casi veinte años antes de que empezaran a bautizar “en el nombre del Padre, del Hijo y del Espíritu Santo”. El bautismo era todo lo que se requería para ser admitido en la comunidad de creyentes. Hasta ese momento, carecían de organización; era, simplemente, la hermandad de Jesús.
\vs p194 4:10 \pc El grupo de los partidarios de Jesús creció rápidamente y, una vez más, los saduceos centraron en este su atención. A los fariseos les preocupaba poco aquella situación, ya que ninguna de las enseñanzas interfería en absoluto con el cumplimiento de las leyes judías. Pero los saduceos comenzaron a encarcelar a los líderes de tal grupo de adeptos hasta que aceptaron convencidos el consejo de Gamaliel, uno de los más prominentes rabinos, que les recomendó: “Apartaos de estos hombres y dejadlos, porque si este consejo o esta obra es de los hombres, se desvanecerá, pero si es de Dios, no los podréis destruir, no sea que seáis tal vez hallados luchando contra Dios”. Decidieron seguir el consejo de Gamaliel, y se produjo un tiempo de paz y tranquilidad en Jerusalén, durante en el que el nuevo evangelio sobre Jesús se extendió enseguida.
\vs p194 4:11 Y, así, todo marchó bien en Jerusalén hasta el momento de la venida de un gran número de griegos desde Alejandría. Dos de los pupilos de Rodán llegaron a Jerusalén e hicieron muchos conversos entre los helenistas. Entre sus primeros conversos, estaban Esteban y Bernabé. Estos capaces griegos no compartían el punto de vista judío, y no se conformaron tan bien al modo judío de adorar ni a las otras prácticas ceremoniales. Y fueron los actos de estos creyentes griegos los que pusieron fin a las relaciones pacíficas entre la hermandad de Jesús y los fariseos y saduceos. Esteban y su compañero griego comenzaron a predicar más de acuerdo a las enseñanzas de Jesús, y esto les ocasionó un inmediato conflicto con los dirigentes judíos. En uno de los sermones públicos de Esteban, cuando llegó a la parte censurable de su discurso, prescindieron de todos los formalismos judiciales y comenzaron a lapidarlo a muerte en aquel mismo lugar.
\vs p194 4:12 Esteban, el líder de la colonia griega de creyentes en Jesús en Jerusalén, se convirtió, pues, en el primer mártir de la nueva fe y constituyó la causa concreta por la que se organizó formalmente la primitiva Iglesia cristiana. Se afrontó esta nueva crisis reconociendo que los creyentes ya no podían seguir siendo una facción dentro de la fe judía. Todos acordaron que debían separarse de los no creyentes; y, en el transcurso de un mes desde la muerte de Esteban, la Iglesia de Jerusalén se organizó bajo el liderazgo de Pedro, nombrándose a Santiago, el hermano de Jesús, como su cabeza titular.
\vs p194 4:13 Y, entonces, estallaron las nuevas e implacables persecuciones de los judíos, por lo que los maestros en activo de la nueva religión sobre Jesús, que luego llegaría a llamarse cristianismo en Antioquía, salieron a los confines del Imperio para proclamar a Jesús. Al llevar este mensaje, antes de los tiempos de Pablo, el liderazgo recayó en manos griegas; y estos primeros misioneros, al igual que los que vendrían más tarde, siguieron la senda de los tempranos días de la marcha seguida por Alejandro, yendo por el camino de Gaza y Tiro a Antioquía y, luego, hacia Asia Menor y Macedonia para continuar hasta Roma y a las partes más remotas del imperio.
