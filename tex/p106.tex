\upaper{106}{Los niveles de la realidad del universo}
\author{Melquisedec}
\vs p106 0:1 No basta con que el mortal ascendente tenga algún entendimiento de las relaciones de la Deidad en cuanto a la génesis y a las manifestaciones de la realidad cósmica; debe también tenerlo de las relaciones existentes entre él mismo y los numerosos niveles de las realidades existenciales y experienciales, de las realidades potenciales y actuales. Una mayor comprensión de las realidades del universo y de sus modos de correlación, integración y unificación enriquecen la posición del hombre en la tierra, su percepción cósmica y su direccionalidad espiritual.
\vs p106 0:2 El presente gran universo y el emergente universo matriz están compuestos de muchas formas y facetas de la realidad que, a su vez, existen en diversos niveles de actuación y operatividad. Estas formas y facetas existentes y latentes ya se han propuesto en estos escritos con anterioridad, y ahora las agrupamos por conveniencia conceptual en las siguientes categorías:
\vs p106 0:3 \li{1.}\bibemph{Finitos incompletos.} Este es el estatus presente de las criaturas ascendentes del gran universo, el estatus actual de los mortales de Urantia. Este nivel abarca la existencia creatural desde los humanos planetarios hasta, aunque no inclusive, los que lograron su destino. Guarda relación con los universos desde sus primeros comienzos físicos hasta, aunque sin incluir, su asentamiento en luz y vida. Este nivel constituye la presente periferia de la actividad creativa en el tiempo y en el espacio. Parece estar moviéndose hacia afuera desde el Paraíso, porque el cierre de la actual era del universo, que presenciará la consecución del gran universo en luz y vida, de cierto asistirá igualmente a la aparición de algún nuevo orden de desarrollo evolutivo en el primer nivel del espacio exterior.
\vs p106 0:4 \li{2.}\bibemph{Finitos máximos.} Este es el estatus presente de todas las criaturas experienciales que han logrado alcanzar su destino ---en el sentido en el que este destino se revela dentro del ámbito de la presente era del universo---. Incluso los universos pueden llegar a un estatus máximo, tanto espiritual como físicamente. Pero el término “máximo” es en sí mismo un término relativo ---¿máximo en relación a qué?---. Y aquello que es máximo, supuestamente final, en la actual era del universo puede que no sea más que un verdadero comienzo respecto a eras venideras. Algunas facetas de Havona parecen encontrarse en este orden máximo.
\vs p106 0:5 \li{3.}\bibemph{Trascendentales.} Este nivel suprafinito (de forma antecedente) sigue al progreso de lo finito. Supone la génesis prefinita de los comienzos finitos y el significado posfinito de todas las finalizaciones o destinos aparentemente finitos. Gran parte del Paraíso\hyp{}Havona parece pertenecer a este orden trascendental.
\vs p106 0:6 \li{4.}\bibemph{Últimos.} Este nivel incluye aquello que es significativo para el universo matriz e incide en el nivel del destino del universo matriz en su completitud. El Paraíso\hyp{}Havona (en especial la vía circulatoria de los mundos del Padre) tiene, en muchos respectos, significación última.
\vs p106 0:7 \li{5.}\bibemph{Coabsolutos}. Este nivel supone la proyección de los experienciales sobre el ámbito de expresión creativa más allá del universo matriz.
\vs p106 0:8 \li{6.}\bibemph{Absolutos.} Este nivel connota la presencia, en la eternidad, de los siete Absolutos existenciales. También puede entrañar un cierto grado de consecución experiencial en vinculación; aunque, si es así, no entendemos cómo, quizás sea mediante el potencial de contacto del ser personal.
\vs p106 0:9 \li{7.}\bibemph{Infinitud.} Este nivel es preexistencial y posexperiencial. La unidad incondicionada de la infinitud es una realidad hipotética anterior a todos los comienzos y posterior a todos los destinos.
\vs p106 0:10 \pc Estos niveles de la realidad son símbolos convenientes, aunque imperfectos, sobre la presente era del universo y para la perspectiva humana. Existen otras maneras de observar la realidad desde un enfoque humano diferente y desde la óptica de otras eras del universo. Por lo tanto, cabe reconocer que los conceptos expuestos aquí son totalmente relativos, relativos en el sentido de estar condicionados y circunscritos por:
\vs p106 0:11 \li{1.}Las limitaciones del lenguaje humano.
\vs p106 0:12 \li{2.}Las limitaciones de la mente mortal.
\vs p106 0:13 \li{3.}El desarrollo limitado de los siete suprauniversos.
\vs p106 0:14 \li{4.}Vuestra ignorancia respecto a los seis fines primordiales del desarrollo del suprauniverso que no guardan relación con el ascenso de los mortales al Paraíso.
\vs p106 0:15 \li{5.}Vuestra incapacidad para lograr comprender, ni incluso parcialmente, la perspectiva eterna.
\vs p106 0:16 \li{6.}La imposibilidad de describir la evolución y el destino cósmico en relación a todas las eras del universo, no solo en cuanto a la era presente, en la que se produce el despliegue evolutivo de los siete suprauniversos.
\vs p106 0:17 \li{7.}La incapacidad de cualquier criatura de lograr comprender lo que realmente se entiende por preexistenciales o posexperienciales ---de aquello que se encuentra antes de los comienzos y después de los destinos---.
\vs p106 0:18 \pc El crecimiento de la realidad está condicionado por las circunstancias de las sucesivas eras del universo. El universo central no sufrió ningún cambio evolutivo en la era de Havona, pero, en las épocas actuales de la era del suprauniverso, está experimentando ciertos cambios graduales provocados por su coordinación con los suprauniversos evolutivos. Los siete suprauniversos, ahora en evolución, lograrán en algún momento asentarse en el estatus de luz y vida, conseguirán el límite de su crecimiento para la actual era del universo. Pero, sin lugar a dudas, la próxima era, la era del primer nivel del espacio exterior, liberará a los suprauniversos de las limitaciones respecto a su destino existente en la era actual. La repleción se superpone continuamente a la completitud.
\vs p106 0:19 \pc Estas son algunas de las limitaciones con las que nos encontramos al intentar exponer un concepto unificado del crecimiento cósmico de las cosas, los contenidos y los valores, y de su síntesis en niveles de la realidad en su constante ascenso.
\usection{1. VINCULACIÓN PRIMARIA DE LOS FUNCIONALES FINITOS}
\vs p106 1:1 Las facetas primarias o de origen espiritual de la realidad finita encuentran su expresión inmediata en los niveles creaturales, como el caso de los seres personales perfectos, y en los niveles del universo, como el caso de la creación perfecta de Havona. Incluso la Deidad experiencial se expresa, por consiguiente, en Havona, en la persona espiritual del Dios Supremo. Pero las facetas secundarias de lo finito, evolutivas, condicionadas por el tiempo y la materia, llegan a integrarse cósmicamente solo como resultado del crecimiento y del logro. En algún momento, todos los finitos secundarios o en perfeccionamiento han de alcanzar un nivel equivalente al de la perfección primaria, pero tal destino está supeditado a un tiempo de demora, un elemento restrictivo de los suprauniversos que no se encuentra intrínsecamente en la creación central. (Conocemos la existencia de finitos terciarios, pero su modo de integración aún no se ha revelado).
\vs p106 1:2 Este tiempo de demora del suprauniverso, este obstáculo para el alcance de la perfección, permite la participación de las criaturas en el crecimiento evolutivo. De este modo, es posible que la criatura se alíe con el Creador para la evolución de esa criatura misma. Y durante estos tiempos de desarrollo expansivo, lo incompleto se correlaciona con lo perfecto por medio del ministerio del Dios Séptuplo.
\vs p106 1:3 El Dios Séptuplo conlleva el reconocimiento por parte de la Deidad del Paraíso de las barreras del tiempo en los universos evolutivos del espacio. Por mucha distancia del Paraíso, por mucha profundidad en el espacio, en el que pueda tener su origen una persona material y superviviente, el Dios Séptuplo estará allí presente dedicado al ministerio amoroso y misericordioso de la verdad, la belleza y la bondad para esta criatura incompleta, afanada y evolutiva. El ministerio divino del Dios Séptuplo se extiende hacia dentro, por medio del Hijo Eterno, hasta el Padre del Paraíso y hacia fuera, por medio de los ancianos de días, a los padres de los universos ---a los hijos creadores---.
\vs p106 1:4 El hombre, siendo un ser personal, que asciende al avanzar espiritualmente, encuentra la divinidad personal y espiritual de la Deidad Séptupla; pero existen otras facetas de esta Deidad que no participan en el progreso del ser personal. Los aspectos divinos de esta agrupación de Deidades están, en la actualidad, integrados en el nexo entre los siete espíritus mayores y el Actor Conjunto, pero su destino es unificarse eternamente en el ser personal emergente del Ser Supremo. Las demás facetas de la Deidad Séptupla están indistintamente integradas en la presente era del universo, pero todas ellas están igualmente destinadas a unificarse en el Supremo. La Deidad Séptupla, en todas sus facetas, es la fuente de la unidad relativa de la realidad operativa del presente gran universo.
\usection{2. INTEGRACIÓN SECUNDARIA Y SUPREMA DE LO FINITO}
\vs p106 2:1 Al igual que el Dios Séptuplo coordina de forma operativa la evolución finita, el Ser Supremo sintetiza en última instancia la consecución del destino. El Ser Supremo es la culminación en cuanto deidad de la evolución del gran universo ---la evolución física en torno a un núcleo espiritual y el ulterior dominio del núcleo espiritual sobre los ámbitos circundantes y rotatorios de la evolución física---. Todo esto se realiza en conformidad con los imperativos del ser personal: el ser personal del Paraíso en el sentido más elevado, el ser personal del Creador en el sentido del universo, el ser personal del mortal en el sentido humano, el ser personal del Supremo en el sentido culminante o experiencialmente totalizador.
\vs p106 2:2 \pc La noción de Supremo debe aportar el reconocimiento de la diferencia entre la persona espíritu, la potencia evolutiva y la síntesis de la potencia\hyp{}ser personal ---la unificación de la potencia evolutiva con el ser personal espiritual junto a la posición dominante de este sobre aquella---.
\vs p106 2:3 El espíritu, en último término, procede del Paraíso por medio de Havona. La energía\hyp{}materia parece evolucionar en las profundidades del espacio, y los hijos del Espíritu Infinito, en conjunción con los hijos creadores de Dios, la organizan como potencia. Y todo esto es experiencial; es una interacción en el tiempo y el espacio en el que interviene una gran diversidad de seres vivos, que se extiende incluso hasta las divinidades creadoras y a las criaturas evolutivas. El dominio de la potencia en el gran universo, por parte de las divinidades creadoras, se amplía paulatinamente hasta englobar el establecimiento y la estabilización evolutivos de las creaciones espacio\hyp{}temporales, y este es el florecimiento de la potencia experiencial del Dios Séptuplo. Abarca toda la gama de logro divino en el tiempo y en el espacio desde la dádiva de los modeladores por parte del Padre Universal hasta la vida de gracia de los hijos del Paraíso. Esta es la potencia ganada, la potencia demostrada, la potencia experiencial; contrasta con la potencia eterna, la potencia inconmensurable, la potencia existencial de las Deidades del Paraíso.
\vs p106 2:4 Esta potencia experiencial, que surge de los logros divinos del mismo Dios Séptuplo, manifiesta las características cohesivas de la divinidad, sintetizándose ---totalizándose--- como la potencia todopoderosa del dominio experiencial conseguido sobre las creaciones evolutivas. Y esta todopoderosa potencia encuentra a su vez cohesión espíritu\hyp{}ser personal en la esfera piloto del cinturón externo de los mundos de Havona en unión con el ser personal espiritual de la presencia en Havona del Dios Supremo. De esta manera, la Deidad experiencial culmina su larga lucha evolutiva al otorgar al resultado de la potencia del tiempo y del espacio la presencia espiritual y el ser personal divino que mora en la creación central.
\vs p106 2:5 Por consiguiente, el Ser Supremo, en última instancia, consigue englobar todo lo que evoluciona en el tiempo y en el espacio, dotando así, a estas cualidades, de ser personal espiritual. Puesto que las criaturas, incluso los mortales, son seres personales que participan en esta majestuosa interacción, ciertamente logran la capacidad de conocer al Supremo y de percibirlo como verdaderos hijos de tal Deidad evolutiva.
\vs p106 2:6 \pc Miguel de Nebadón es como el Padre del Paraíso porque comparte con él su perfección paradisíaca; por ello, en algún momento, los mortales evolutivos lograrán relacionarse con el Supremo experiencial, porque realmente compartirán su perfección evolutiva.
\vs p106 2:7 \pc El Dios Supremo es experiencial; en consecuencia, es completamente experimentable. Las realidades existenciales de los siete Absolutos no son perceptibles de modo experiencial; el ser personal de la criatura finita solo puede captar las \bibemph{realidades personales} del Padre, del Hijo y del Espíritu a través de una actitud de oración y adoración.
\vs p106 2:8 Dentro de la síntesis completada de la potencia\hyp{}ser personal del Ser Supremo, se agrupará toda la absolutidad de las distintas triodidades que así puedan relacionarse, y todos los seres personales finitos alcanzarán de forma experiencial, y comprenderán, a este majestuoso ser personal. Cuando los ascendentes alcancen la presumible séptima etapa de su existencia espiritual, experimentarán en ella la realización de un nuevo valor\hyp{}significado de la absolutidad y de la infinitud de las triodidades, tal como se revela en los niveles subabsolutos del Ser Supremo, que es experimentable. Pero la consecución de estas etapas de desarrollo máximo aguardará probablemente el asentamiento coordinado de todo el gran universo en luz y vida.
\usection{3. VINCULACIÓN TERCIARIA Y TRASCENDENTAL DE LA REALIDAD}
\vs p106 3:1 Los arquitectos absonitos devienen el plan; los creadores supremos lo hacen realidad; el Ser Supremo se consumará en plenitud tal como lo crearon los creadores supremos y lo idearon en el espacio los arquitectos mayores.
\vs p106 3:2 Durante la presente era del universo, la labor de los arquitectos mayores es la coordinación administrativa del universo matriz. Si bien, la aparición del Todopoderoso Supremo al final de la actual era del universo significará que lo finito evolutivo ha alcanzado la primera etapa de su destino experiencial. Este acontecimiento llevará ciertamente a la actuación plena de la primera Trinidad experiencial: la unión de los creadores supremos, el Ser Supremo y los arquitectos del universo matriz. Esta Trinidad está destinada a efectuar la postrera integración evolutiva de la creación matriz.
\vs p106 3:3 La Trinidad del Paraíso es verdaderamente la Trinidad de la infinitud, y ninguna Trinidad puede ser infinita si no incluye a esta Trinidad primigenia. Pero la Trinidad primigenia es el devenir de la agrupación exclusiva de las Deidades absolutas; los seres subabsolutos nada tuvieron que ver con tal relación primordial. Las trinidades que aparecen más tarde y que son experienciales engloban las contribuciones de incluso seres personales creaturales. Sin duda, esto es verdad en el caso de la Trinidad Última, en la que la presencia misma de los hijos creadores mayores entre los miembros de los creadores supremos denota la consiguiente presencia de la experiencia real y genuina de las criaturas \bibemph{dentro} de esta vinculación Trinitaria.
\vs p106 3:4 La primera Trinidad experiencial establece el logro grupal de los devenires últimos. Las vinculaciones grupales pueden anticipar, incluso trascender, las capacidades individuales; esto es cierto incluso más allá del nivel finito. En las eras venideras, una vez que los siete suprauniversos se hayan asentado en luz y vida, el colectivo de finalizadores promulgará sin duda los propósitos de las Deidades del Paraíso tal como la Trinidad Última los dicte, y tal como se unifican como potencia\hyp{}ser personal en el Ser Supremo.
\vs p106 3:5 \pc A lo largo de todos los grandiosos acontecimientos del universo de la eternidad pasada y futura, apreciamos la expansión de los elementos comprensibles del Padre Universal. Como el YO SOY, postulamos filosóficamente su permeación de la infinitud total, pero ninguna criatura es capaz de entender experiencialmente tal postulado. A medida que los universos se expanden y que la gravedad y el amor se extienden al espacio en su proceso de organización en el tiempo, podemos comprender más y más la Primera Fuente y Centro. Observamos la acción de la gravedad penetrando en la presencia espacial del Absoluto Indeterminado, y descubrimos criaturas espirituales que evolucionan y se extienden en los ámbitos de la presencia divina del Absoluto de la Deidad, mientras que la evolución cósmica y espiritual se unifica por medio de la mente y la experiencia en niveles finitos en cuanto deidad como el Ser Supremo y se coordinan en niveles trascendentales como la Trinidad Última.
\usection{4. INTEGRACIÓN CUATERNARIA ÚLTIMA}
\vs p106 4:1 La Trinidad del Paraíso ciertamente coordina en el sentido último, aunque obra en este respecto como un absoluto condicionado en sí mismo; la Trinidad Última experiencial coordina lo trascendental como un trascendental. En el futuro eterno, esta Trinidad experiencial, mediante el aumento de la unidad, activará postreramente el devenir de la presencia de la Deidad Última.
\vs p106 4:2 Aunque la Trinidad Última está llamada a coordinar la creación matriz, el Dios Último es la potencia\hyp{}manifestación personal trascendental del curso direccional de todo el universo matriz. El completo devenir del Último supone la completitud de la creación matriz y connota la completa aparición de esta Deidad trascendental.
\vs p106 4:3 No conocemos los cambios que se iniciarán con la plena aparición del Último. Pero, al igual que el Supremo está ahora espiritual y personalmente presente en Havona, del mismo modo el Último está allí presente aunque en un sentido absonito y suprapersonal. Y se os ha informado de la existencia de los vicerregentes condicionados del Último, aunque no se os ha hecho referencia a su localización o cometido actual.
\vs p106 4:4 Pero al margen de las repercusiones administrativas conexas a la gradual aparición de la Deidad Última, los valores personales de su divinidad trascendental serán experimentables por todos los seres personales que hayan participado en la actualización de este nivel de la Deidad. La trascendencia de lo finito puede solo llevar a la consecución última. El Dios Último existe en la trascendencia del tiempo y el espacio, si bien, es subabsoluto a pesar de su intrínseca capacidad de conjunción de carácter operativo con los absolutos.
\usection{5. VINCULACIÓN DE LA QUINTA FASE O COABSOLUTA}
\vs p106 5:1 El Último es el ápice de la realidad trascendental, al igual que el Supremo es la cúspide de la realidad evolutiva\hyp{}experiencial. Y la aparición real de estas dos Deidades experienciales sienta las bases para la segunda Trinidad experiencial. Se trata de la Trinidad Absoluta, la unión del Dios Supremo, del Dios Último y del Consumador del Destino del Universo, no revelado. Y esta Trinidad tiene la capacidad teórica de activar los Absolutos de la potencialidad ---el de la Deidad, el Universal y el Indeterminado---. Pero la formación en su completitud de esta Trinidad Absoluta no puede tener lugar hasta que no se complete la evolución de todo el universo matriz, desde Havona hasta el cuarto nivel ulterior del espacio.
\vs p106 5:2 Debe quedar claro que las Trinidades experienciales no solo correlacionan las cualidades del ser personal de la Divinidad experiencial, sino también todas las cualidades distintas a las personales que caracterizan su lograda unidad como Deidad. Aunque esta exposición trata fundamentalmente de las facetas personales de la unificación del cosmos, es, no obstante, verdad que los aspectos impersonales del universo de los universos están asimismo destinados a ser objeto de la unificación tal como lo ilustra la síntesis de la potencia\hyp{}ser personal que en este momento se está efectuando con respecto a la evolución del Ser Supremo. Las cualidades espirituales\hyp{}personales del Supremo son inseparables de las prerrogativas sobre la potencia del Todopoderoso, y ambas se complementan gracias al potencial desconocido de la mente Suprema. Tampoco puede el Dios Último como persona considerarse al margen de los aspectos distintos de lo personal de la Deidad Última. Y en el nivel absoluto de los Absolutos de la Deidad e Indeterminado son inseparables e indistinguibles en presencia del Absoluto Universal.
\vs p106 5:3 Las Trinidades son, en sí mismas y por sí mismas, no personales, pero tampoco contravienen el ser personal. Más bien, lo engloban y correlacionan, en sentido colectivo, con las acciones de lo impersonal. Las Trinidades son, pues, siempre realidad en cuanto \bibemph{deidad} pero nunca realidad \bibemph{personal.} Los aspectos personales de una trinidad son intrínsecos a sus miembros individuales, y como personas individuales \bibemph{no} constituyen esa trinidad. Solamente son trinidad de forma colectiva; esto es trinidad. Pero en la trinidad siempre se integra toda la deidad incluyente; la trinidad es la unidad en cuanto deidad.
\vs p106 5:4 Los tres Absolutos ---de la Deidad, Universal e Indeterminado--- no son trinidad, porque no todos son deidad. Únicamente lo deificado puede llegar a ser trinidad; cualquier otra agrupación es triunidad o triodidad.
\usection{6. INTEGRACIÓN DE LA SEXTA FASE O ABSOLUTA}
\vs p106 6:1 El presente potencial del universo matriz no puede considerarse absoluto, aunque bien pudiera ser casi último, y estimamos que es imposible comprender por completo los contenidos\hyp{}valores absolutos dentro del ámbito del cosmos subabsoluto. Nos resulta, por lo tanto, muy difícil intentar concebir la expresión total de las posibilidades ilimitadas de los tres Absolutos e incluso tratar de visualizar la manifestación personal y experiencial del Dios Absoluto en el nivel impersonal actual del Absoluto de la Deidad.
\vs p106 6:2 El espacio\hyp{}escenario del universo matriz parece ser el adecuado para la actualización del Ser Supremo, para la formación y la plena acción de la Trinidad Última, para el devenir del Dios Último e incluso para el inicio de la Trinidad Absoluta. Pero, respecto a la plena actuación de esta segunda Trinidad experiencial, nuestras nociones parecen llevarnos a algo que está más allá del inmenso universo matriz.
\vs p106 6:3 Si suponemos un cosmos\hyp{}infinito ---un cosmos ilimitable más allá del universo matriz--- y si concebimos que el desarrollo final de la Trinidad Absoluta se llevará a cabo allí en tal fase de acción supraúltima, entonces resulta posible conjeturar que la actuación completa de la Trinidad Absoluta alcanzará su expresión final en las creaciones de la infinitud y consumará la actualización absoluta de todos los potenciales. La integración y agrupación de segmentos cada vez mayores de la realidad se irá aproximando a un estatus absoluto en proporción a la inclusión de toda la realidad dentro de los segmentos así agrupados.
\vs p106 6:4 Dicho de otra manera: la Trinidad Absoluta, como su nombre indica, es realmente absoluta en cuanto a su acción total. No sabemos cómo una acción absoluta puede lograr una expresión total sobre una base condicionada, limitada o restringida de alguna otra manera. Así pues, debemos suponer que cualquier totalidad de actuación será no condicionada (potencialmente). Y parecería también que lo no condicionado también sería ilimitado, al menos desde un punto de vista cualitativo, aunque no estamos tan seguros con respecto a las relaciones cuantitativas.
\vs p106 6:5 Sí, no obstante, estamos ciertos de esto: aunque la Trinidad existencial del Paraíso es infinita, y aunque la Trinidad Última experiencial es subinfinita, la Trinidad Absoluta no es tan fácil de clasificar. Aunque experiencial en su génesis y constitución, claramente incide en los Absolutos existenciales de la potencialidad.
\vs p106 6:6 A pesar de que resulta difícilmente provechoso para la mente humana tratar de comprender estos conceptos suprahumanos tan remotos, nos gustaría sugerir que la acción eterna de la Trinidad Absoluta puede considerarse como la culminación de algún tipo de experiencialización de los Absolutos de la potencialidad. Parecería ser una conclusión razonable con respecto al Absoluto Universal, si no al Absoluto Indeterminado; al menos sabemos que el Absoluto Universal no es tan solo estático y potencial sino también vinculado en el sentido de la Deidad total al que aluden esas palabras. Pero en lo que se refiere a los valores concebibles de la divinidad y el ser personal, estos acontecimientos que se han conjeturado entrañan la manifestación personal del Absoluto de la Deidad y la aparición de esos valores suprapersonales y de esos contenidos ultrapersonales inherentes a la compleción del ser personal del Dios Absoluto ---la tercera y última de las Deidades experienciales---.
\usection{7. LA COMPLETUD DE DESTINO}
\vs p106 7:1 Algunas de las dificultades que surgen al elaborar conceptos sobre la integración de la realidad infinita son consustanciales al hecho de que tales nociones abarcan algo de la completud del desarrollo universal, algún tipo de realización experiencial de todo lo que podría ser alguna vez. Y es inconcebible que la infinitud cuantitativa pueda en algún momento llegar a efectuarse enteramente en completud. Siempre deben quedar posibilidades no exploradas en los tres Absolutos potenciales que ninguna cantidad de desarrollo experiencial podría jamás agotar. La eternidad misma, aunque absoluta, no es más que absoluta.
\vs p106 7:2 Incluso el concepto preliminar de la integración final es inseparable de la consecución de la eternidad incondicionada y es, por lo tanto, prácticamente irrealizable en cualquier futuro imaginable.
\vs p106 7:3 \pc El destino se establece gracias al acto volitivo de las Deidades que constituyen la Trinidad del Paraíso; el destino se instaura en la inmensidad de los tres grandes potenciales cuya absolutidad abarca las posibilidades de todo el desarrollo futuro; el destino probablemente se consuma mediante el acto del Consumador del Destino Universal, y es probable que, en este acto, participen el Supremo y el Último en la Trinidad Absoluta. Todo destino experiencial puede ser al menos parcialmente comprendido por parte de las criaturas experienciales; pero un destino que incide en los existenciales infinitos es difícilmente comprensible. La completud de destino es un logro existencial\hyp{}experiencial que parece involucrar al Absoluto de la Deidad. Pero el Absoluto de la Deidad está en relación eterna con el Absoluto Indeterminado por virtud del Absoluto Universal. Y estos tres Absolutos, potencialmente experienciales, son realmente existenciales, y más, al estar desprovistos de limitaciones, de tiempo, de espacio, de límites, de medidas ---verdaderamente infinitos---.
\vs p106 7:4 El improbable logro de la meta, no impide, sin embargo, que se teorice filosóficamente sobre tales hipotéticos destinos. La actualización del Absoluto de la Deidad como un Dios absoluto factible de alcanzarse puede ser prácticamente imposible de realizar; sin embargo, tal consecución de la completud sigue siendo una posibilidad teórica. La participación del Absoluto Indeterminado en algún cosmos infinito inconcebible puede ser inconmensurablemente remota en el futuro devenir de la ilimitada eternidad, pero dicha hipótesis es, no obstante, válida. Los mortales, los morontiales, los espíritus, los finalizadores, los trascendentales y otros, junto con los universos mismos y todas las demás facetas de la realidad, ciertamente tienen un destino \bibemph{potencialmente final que es absoluto en valor;} pero dudamos de que algún ser o universo pueda jamás llegar a alcanzar por completo todos los aspectos de dicho destino.
\vs p106 7:5 \pc Por más que crezcáis en la comprensión del Padre, vuestra mente siempre se va a quedar estupefacta ante la infinitud no revelada del Padre\hyp{}YO SOY, la inmensidad inexplorada que siempre permanecerá insondable e incomprensible a lo largo de todos los ciclos de la eternidad. Aunque lleguéis a alcanzar bastante de Dios, siempre habrá más de él, una existencia de la que jamás ni siquiera sospecharéis. Y creemos que esto es igualmente verdad en los niveles trascendentales como lo es en los ámbitos de la existencia finita. ¡La búsqueda de Dios no tiene fin!
\vs p106 7:6 Tal incapacidad para llegar a Dios en un sentido final no debería de modo alguno disuadir a las criaturas del universo; en verdad, podéis alcanzar y de hecho alcanzáis niveles de la Deidad Séptupla, del Supremo y del Último, lo cual significa para vosotros lo que la consecución infinita del Dios Padre significa para el Hijo Eterno y el Actor Conjunto en su estatus absoluto de la existencia eterna. Lejos de intimidar a las criaturas, la infinitud de Dios debería convertirse en la suprema certidumbre de que, a lo largo de todo su interminable devenir futuro, la persona ascendente tendrá ante sí posibilidades de desarrollar su ser personal y de vincularse con la Deidad que ni siquiera la eternidad podrá agotar ni dar fin.
\vs p106 7:7 \pc Para las criaturas finitas del gran universo, el concepto del universo matriz parece ser prácticamente infinito, pero, sin duda alguna, sus arquitectos absonitos perciben su relación con los desarrollos futuros e insospechados dentro del ámbito del interminable YO SOY. Incluso el mismo espacio no es sino una condición última, una condición delimitada \bibemph{dentro} de la absolutidad relativa de las zonas quietas del espacio intermedio.
\vs p106 7:8 En un momento de la eternidad futura, inconcebiblemente distante, en el que se produzca la compleción final de todo el universo matriz, no cabe duda de que volveremos nuestras miradas atrás, hacia la totalidad de su historia, solo como el principio, sencillamente como la creación de ciertos pilares finitos y trascendentales para el logro de metamorfosis incluso mayores y más fascinantes en la inexplorada infinitud. En tal momento de la eternidad futura, el universo matriz parecerá todavía joven; realmente, será siempre joven ante las ilimitadas posibilidades de la eternidad sin fin.
\vs p106 7:9 \pc La improbabilidad de la consecución del destino infinito no impide en lo más mínimo que se alberguen ideas sobre este, y no dudamos en afirmar que si los tres potenciales absolutos pudiesen alguna vez actualizarse en su totalidad, sería posible concebir la integración final de la realidad total. Este logro evolutivo se basa en la completa actualización de los Absolutos Indeterminado, Universal y de la Deidad, las tres potencialidades cuya unión constituye el carácter latente del YO SOY, las realidades en suspensión de la eternidad, las posibilidades temporalmente inactivas de todo el devenir futuro, y mucho más.
\vs p106 7:10 Tales contingencias son bastante remotas, por decir algo; no obstante, en las reacciones de orden material, en los seres personales y en las agrupaciones de las tres Trinidades creemos descubrir la posibilidad teórica de reunir las siete facetas absolutas del Padre\hyp{}YO SOY. Y esto nos coloca frente al concepto de la Trinidad triple que incluye a la Trinidad del Paraíso, cuyo estatus es existencial, y a las dos Trinidades, que aparecen con posterioridad, cuya naturaleza y origen es experiencial.
\usection{8. LA TRINIDAD DE TRINIDADES}
\vs p106 8:1 Resulta difícil describir la naturaleza de la Trinidad de Trinidades para la mente humana; es la suma real de la totalidad de la infinitud experiencial, tal como esta se manifiesta en la realización en la eternidad de la infinitud teórica. En la Trinidad de Trinidades, el infinito experiencial consigue la identidad con el infinito existencial, y ambos son como uno en el YO SOY preexperiencial, preexistencial. La Trinidad de Trinidades es la expresión final de todo lo que está implícito en las quince triunidades y en las correspondientes triodidades. Las completitudes, ya sean existenciales o experienciales, son de difícil comprensión para los seres relativos; por consiguiente, deben exponerse siempre como relatividades.
\vs p106 8:2 La Trinidad de Trinidades tiene su existencia en diversas facetas. Contiene posibilidades, probabilidades e inevitabilidades que retan la imaginación de seres muy por encima del nivel humano. Posee significaciones insospechadas por los filósofos celestiales, porque estas están en las triunidades, y las triunidades son, en última instancia, insondables.
\vs p106 8:3 Hay distintas formas de describir la Trinidad de Trinidades. Hemos optado por exponer el concepto en tres niveles, que son los siguientes:
\vs p106 8:4 \li{1.}El nivel de las tres Trinidades.
\vs p106 8:5 \li{2.}El nivel de la Deidad experiencial.
\vs p106 8:6 \li{3.}El nivel del YO SOY.
\vs p106 8:7 \pc Se trata de niveles que están en unificación creciente. En realidad, la Trinidad de Trinidades es el primer nivel, mientras que el segundo nivel y el tercero son unificaciones\hyp{}derivaciones del primero.
\vs p106 8:8 \pc EL PRIMER NIVEL: En este nivel inicial de vinculación, se cree que las tres Trinidades obran como agrupaciones, perfectamente sincronizadas aunque diferenciadas, de los seres personales de la Deidad.
\vs p106 8:9 \li{1.}\bibemph{La Trinidad del Paraíso,} la vinculación de las tres Deidades del Paraíso ---Padre, Hijo y Espíritu---. Conviene recordar que la Trinidad del Paraíso conlleva una triple acción: absoluta, trascendental (la Trinidad de Ultimidad) y finita (la Trinidad de la Supremacía). La Trinidad del Paraíso es cualquiera y todas ellas en cualquier momento y en todo momento.
\vs p106 8:10 \li{2.}\bibemph{La Trinidad Última}. Se trata de la conjunción en cuanto deidad de los creadores supremos, el Dios Supremo y los arquitectos del universo matriz. Aunque la presente sea una exposición adecuada de los aspectos divinos de esta Trinidad, se debe dejar constancia de que existen otras facetas de esta Trinidad, que, sin embargo, parecen estar en perfecta coordinación con los aspectos divinos.
\vs p106 8:11 \li{3.}\bibemph{La Trinidad Absoluta}. Se trata de la agrupación del Dios Supremo, el Dios Último y el Consumador del Destino del Universo en lo que respecta a todos los valores divinos. Determinadas otras facetas de esta agrupación trina tienen que ver con valores distintos de los divinos en el cosmos en expansión. Pero estos se unifican con las facetas divinas, al igual que los aspectos de la potencia y el ser personal de las Deidades experienciales están ahora en proceso de síntesis experiencial.
\vs p106 8:12 \pc La vinculación de estas tres Trinidades en la Trinidad de Trinidades proporciona una posible integración ilimitada de la realidad. Esta agrupación contiene causas, intermediaciones y finalizaciones; iniciadores, realizadores y consumadores; comienzos, existencias y destinos. La cooperación Padre\hyp{}Hijo se ha convertido en el Hijo\hyp{}Espíritu y, luego, en el Espíritu\hyp{}Supremo y, de ahí, en el Supremo\hyp{}Último y el Último\hyp{}Absoluto, incluso en el Absoluto y el Padre\hyp{}Infinito ---la compleción del ciclo de la realidad---. Asimismo, en otras facetas no tan inmediatamente relacionadas con la divinidad y el ser personal, la Primera Gran Fuente y Centro realiza en sí misma el estado ilimitado de la realidad alrededor del círculo de la eternidad, desde la absolutidad de la autoexistencia, a través de la no terminación de la autorrevelación, hasta la completud de la autorrealización ---desde el absoluto de los existenciales hasta la completud de los experienciales---.
\vs p106 8:13 \pc EL SEGUNDO NIVEL: La coordinación de las tres Trinidades entraña inevitablemente la unión y vinculación de las Deidades experienciales, que están intrínsecamente relacionadas con estas Trinidades. La naturaleza de este segundo nivel se ha expuesto algunas veces tal como sigue:
\vs p106 8:14 \li{1.}\bibemph{El Supremo}. Se trata de la consecuencia en cuanto deidad de la unidad de la Trinidad del Paraíso en su enlace experiencial con los hijos creadores e hijas creativas de las Deidades del Paraíso. El Supremo es la personificación como deidad de la completitud de la primera etapa de la evolución finita.
\vs p106 8:15 \li{2.}\bibemph{El Último}. Se trata de la consecuencia en cuanto deidad de la unidad devenida de la segunda Trinidad, la personificación trascendental y absonita de la divinidad. El Último consiste en una unidad, considerada de modo variable, de muchas cualidades, y su concepción humana haría bien en incluir al menos esas facetas de la ultimidad que ejercen un control directivo, que son experimentables personalmente y unificadoras tensionales, pero existen muchos otros aspectos no revelados de la Deidad devenida. Aunque el Último y el Supremo son equiparables, no son idénticos, tampoco es el Último simplemente una ampliación del Supremo.
\vs p106 8:16 \li{3.}\bibemph{El Absoluto}. Se han propuesto muchas teorías respecto al carácter del tercer miembro del segundo nivel de la Trinidad de Trinidades. El Dios Absoluto está sin duda implicado en esta agrupación como consecuencia personal de la acción final de la Trinidad Absoluta; no obstante, el Absoluto de la Deidad es una realidad existencial de estatus eterno.
\vs p106 8:17 La dificultad nocional que se plantea en cuanto al tercer miembro es inherente al hecho de que la premisa de su membrecía supone realmente un solo Absoluto. Teóricamente, si tal acontecimiento tuviera lugar, deberíamos presenciar la \bibemph{unificación experiencial} de los tres Absolutos en uno. Y se nos enseña que, en la infinitud y \bibemph{de forma existencial,} hay un solo Absoluto. Aunque está menos claro quién puede ser este tercer miembro, a menudo se propugna que este puede consistir en el Absoluto de la Deidad, el Absoluto Universal y el Absoluto Indeterminado en algún tipo de enlace inimaginable y de manifestación cósmica. Ciertamente, la Trinidad de Trinidades difícilmente podría lograr su actuación plena en ausencia de la total unificación de los tres Absolutos, y los tres Absolutos difícilmente podrían unificarse en ausencia de la completa realización de todas las potencialidades infinitas.
\vs p106 8:18 Probablemente, sería tan solo una alteración mínima de la verdad si se asumiera que el Absoluto Universal fuese el tercer miembro de la Trinidad de Trinidades, siempre y cuando esta concepción contemplara al Universal, no solo como estático y potencial, sino también como vinculado. Pero no percibimos aún su relación con los aspectos creativos y evolutivos de la acción de la Deidad total.
\vs p106 8:19 Aunque resulta difícil elaborar un concepto completo de la Trinidad de Trinidades, no lo es si lo llevamos a cabo de forma parcial. Si el segundo nivel de la Trinidad de Trinidades se concibe como esencialmente personal, se vuelve totalmente posible plantear la unión del Dios Supremo, el Dios Último y el Dios Absoluto como la repercusión personal de la unión de las Trinidades personales, las cuales son ancestrales a estas Deidades experienciales. Nos aventuramos a opinar que estas tres Deidades experienciales ciertamente se unificarán en el segundo nivel como consecuencia directa de la creciente unidad de sus Trinidades ancestrales y causales, que constituyen el primer nivel.
\vs p106 8:20 El primer nivel consta de tres Trinidades; el segundo nivel existe en la vinculación personal consistente en los seres personales experienciales\hyp{}evolucionados, experienciales\hyp{}devenidos y experienciales\hyp{}existenciales de la Deidad. Y, con independencia de cualquier dificultad conceptual en la comprensión de la completitud de la Trinidad de Trinidades, la agrupación personal de estas tres Deidades en el segundo nivel se ha manifestado en nuestra propia era del universo en el fenómeno de la deidización de Majestón, que se actualizó en este segundo nivel mediante el Absoluto de la Deidad, actuando por medio del Último y en respuesta al preliminar mandato creativo del Ser Supremo.
\vs p106 8:21 EL TERCER NIVEL: En una hipótesis incondicionada del segundo nivel de la Trinidad de Trinidades, se engloba la correlación de cualquier faceta de cualquier clase de realidad que es, fue o puede ser en la totalidad de la infinitud. El Ser Supremo no es solamente espíritu sino también mente y potencia y experiencia. El Último es todo esto y mucho más, mientras que, en el concepto conjunto de la unicidad de los Absolutos de la Deidad, Universal e Indeterminado se incluye la completud absoluta de toda la realización de la realidad.
\vs p106 8:22 En la unión del Supremo, del Último y del Absoluto completo, podía ocurrir el reensamblaje con carácter operativo de aquellos aspectos de la infinitud que fueron primigeniamente segmentados por el YO SOY, y que dieron lugar a la aparición de los Siete Absolutos de la Infinitud. Aunque los filósofos del universo consideran este hecho como una probabilidad muy remota, hacemos, no obstante, con frecuencia esta pregunta: si el segundo nivel de la Trinidad de Trinidades pudiera alguna vez lograr la unidad trinitaria, ¿qué consecuencias se derivarían entonces de tal unidad en cuanto deidad? No lo sabemos, pero confiamos en que llevaría directamente a la realización del YO SOY como experiencial alcanzable. Desde la perspectiva de los seres personales podría significar que el incognoscible YO SOY se convirtiera en experimentable como Padre\hyp{}Infinito. Lo que estos destinos absolutos podrían significar desde la perspectiva no personal es otra cuestión que tan solo la eternidad podría posiblemente dilucidar. Pero, tal como nosotros vemos estas remotas contingencias como criaturas personales, deducimos que el destino final de todos los seres personales es el conocimiento final del Padre Universal de estos mismos seres personales.
\vs p106 8:23 Tal como lo concebimos filosóficamente en la eternidad pasada, el YO SOY está solo, no hay nadie fuera de él. Dirigiendo la mirada hacia adelante, hacia la eternidad futura, no contemplamos la posibilidad de que el YO SOY, como existencial, pueda cambiar, pero tendemos a prever una inmensa diferencia experiencial. Este concepto del YO SOY comporta una plena autorrealización ---abarca esa ilimitada constelación de seres personales que se han convertido en participantes volitivos en la autorrevelación del YO SOY, y que permanecerán eternamente como componentes absolutos con voluntad de la totalidad de la infinitud: los hijos finales del Padre absoluto---.
\usection{9. UNIFICACIÓN EXISTENCIAL INFINITA}
\vs p106 9:1 En el concepto de la Trinidad de Trinidades postulamos la posible unificación experiencial de la realidad ilimitada y, a veces, teorizamos que todo esto puede acontecer en la absoluta lejanía de la más distante eternidad. Pero, pese a todo, hay una unificación actual y presente de la infinitud en esta misma era al igual que en todas las eras pasadas y futuras del universo; tal unificación es existencial en la Trinidad del Paraíso. La unificación de la infinitud como realidad experiencial es inconcebiblemente remota, pero una unidad incondicionada de la infinitud ahora domina el momento actual de la existencia del universo y une las divergencias de toda la realidad con una majestuosidad existencial que es \bibemph{absoluta}.
\vs p106 9:2 Cuando las criaturas finitas tratan de concebir la unificación infinita de los niveles de la completud de la eternidad consumada, se enfrentan con las limitaciones intelectuales intrínsecas a sus existencias finitas. El tiempo, el espacio y la experiencia dificultan tal conceptualización de las criaturas; y aun así, sin tiempo, al margen del espacio y dejando aparte la experiencia, ninguna criatura podría lograr ni siquiera una comprensión limitada de la realidad del universo. Sin la sensibilidad del tiempo, ninguna criatura evolutiva podría posiblemente percibir las relaciones secuenciales. Sin la percepción del espacio, ninguna criatura podría imaginar las relaciones de simultaneidad. Sin la experiencia, ninguna criatura evolutiva podría ni siquiera existir; solo los Siete Absolutos de la Infinitud trascienden realmente la experiencia, e incluso estos pueden ser, en determinadas facetas, experienciales.
\vs p106 9:3 El tiempo, el espacio y la experiencia son las mayores ayudas con las que cuenta el hombre para la percepción relativa de la realidad y, sin embargo, son sus más ingentes obstáculos para su percepción completa de la realidad. Los mortales y muchas otras criaturas del universo consideran oportuno pensar en los potenciales como si se actualizaran en el espacio y evolucionaran hasta su consecución en el tiempo, pero todo este proceso es un fenómeno espacio\hyp{}temporal que no ocurre realmente en el Paraíso ni en la eternidad. En el nivel absoluto no hay tiempo ni espacio; todos los potenciales pueden percibirse allí como actuales.
\vs p106 9:4 El concepto de la unificación de toda la realidad, sea en esta o en cualquier otra era del universo, es esencialmente doble: existencial y experiencial. Tal unidad está en vías de realización experiencial en la Trinidad de Trinidades, pero el grado de la actualización perceptible de esta Trinidad triple es directamente proporcional a la desaparición de las delimitaciones e imperfecciones de la realidad cósmica. Si bien, la total integración de la realidad está presente, incondicional y existencialmente, en la Trinidad del Paraíso, dentro de la que, en este mismo momento del universo, la realidad infinita está absolutamente unificada.
\vs p106 9:5 \pc La paradoja creada por las perspectivas experiencial y existencial es inevitable y, en parte, se basa en el hecho de que la Trinidad del Paraíso y la Trinidad de Trinidades constituyen cada cual una relación en la eternidad que los mortales solo pueden percibir como relatividad espacio\hyp{}temporal. El concepto humano de la gradual actualización experiencial de la Trinidad de Trinidades ---la perspectiva temporal--- debe complementarse con el postulado adicional de que \bibemph{es} ya una realidad fehaciente ---la perspectiva de la eternidad---. Pero ¿cómo se pueden reconciliar estas dos perspectivas? Sugerimos a los mortales finitos que acepten la verdad de que la Trinidad del Paraíso es la unificación existencial de la infinitud, y que su incapacidad de descubrir la presencia actual y la manifestación completa de la Trinidad de Trinidades experienciales se debe en parte a una mutua distorsión debido a:
\vs p106 9:6 \li{1.}La limitada perspectiva humana, la incapacidad de lograr comprender el concepto de la eternidad incondicionada.
\vs p106 9:7 \li{2.}El estatus imperfecto del ser humano, su lejanía del nivel absoluto de los experienciales.
\vs p106 9:8 \li{3.}El propósito de la existencia humana, el hecho de que la humanidad está concebida para evolucionar por medio de la experiencia y, como consecuencia, debe ser intrínseca y esencialmente dependiente de ella. Tan solo un Absoluto puede ser a la vez existencial y experiencial.
\vs p106 9:9 \pc En la Trinidad del Paraíso, el Padre Universal es el YO SOY de la Trinidad de Trinidades, y la imposibilidad de vivenciar al Padre como infinito se debe a las limitaciones finitas. El concepto del YO SOY \bibemph{existencial,} solitario, previo a la Trinidad e inalcanzable y el postulado del YO SOY \bibemph{experiencial} post\hyp{}Trinidad de Trinidades y alcanzable constituyen una única y misma hipótesis; en el Infinito no ha ocurrido ningún cambio real; todo desarrollo aparente se debe a las crecientes capacidades de la recepción de la realidad y de la apreciación cósmica.
\vs p106 9:10 El YO SOY, en última instancia, debe existir \bibemph{antes} que todos los existenciales y \bibemph{después} que todos los experienciales. Aunque estas ideas puedan no aclarar para la mente humana las paradojas de la eternidad y la infinitud, deberían al menos estimular a tales intelectos finitos a abordar de nuevo estas inacabables problemáticas, problemáticas que continuarán intrigándoos en Lugar de Salvación y, más tarde, como finalizadores y, en adelante, durante todo el interminable futuro de vuestras andaduras eternas en los universos, en su amplia expansión.
\vs p106 9:11 \pc Tarde o temprano todos los seres personales del universo comienzan a percatarse de que la búsqueda final de la eternidad es la inacabable exploración de la infinitud, el interminable viaje de descubrimiento hacia la absolutidad de la Primera Fuente y Centro. Tarde o temprano todos nos haremos conscientes de que cualquier crecimiento creatural es proporcional a su identificación con el Padre. Llegaremos al entendimiento de que vivir la voluntad de Dios constituye el pasaporte eterno hacia la posibilidad sin límites de la infinitud misma. Los mortales comprenderán alguna vez que el éxito en la búsqueda del Infinito es directamente proporcional al logro de su semejanza con el Padre, y que en esta era del universo se revelarán las realidades del Padre conforme a las cualidades de la divinidad. Y las criaturas del universo pueden tomar posesión personal de estas cualidades de la divinidad mediante la experiencia de vivir de forma divina, y vivir de forma divina significa realmente vivir la voluntad de Dios.
\vs p106 9:12 Para las criaturas materiales, evolutivas, finitas, una vida basada en vivir la voluntad del Padre conduce directamente al logro de la supremacía del espíritu en el ámbito del ser personal y lleva a dichas criaturas a dar un paso más hacia la comprensión del Padre\hyp{}Infinito. Esta vida en el Padre es una vida basada en la verdad, sensible a la belleza y dominada por la bondad. Tal persona conocedora de Dios está iluminada interiormente por la adoración y exteriormente dedicada al servicio incondicional de la hermandad universal de todas las personas, un ministerio asistencial pleno en misericordia y motivado por el amor, mientras que todas estas cualidades de vida se unifican en el ser personal evolutivo, en niveles siempre ascendentes de sabiduría cósmica, autorrealización, encuentro de Dios y adoración del Padre.
\vsetoff
\vs p106 9:13 [Exposición de un melquisedec de Nebadón.]
