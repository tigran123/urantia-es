\upaper{168}{La resurrección de Lázaro}
\author{Comisión de seres intermedios}
\vs p168 0:1 Poco después del mediodía, Marta salió a encontrarse con Jesús cuando él llegó a lo alto de la colina cercana a Betania. Su hermano Lázaro hacía cuatro días, desde la media tarde del domingo, que yacía en un sepulcro privado situado en un extremo del jardín. La mañana de ese día, jueves, la piedra de la entrada a la tumba se había hecho rodar y se había colocado en su sitio.
\vs p168 0:2 Cuando Marta y María enviaron noticia a Jesús de la enfermedad de Lázaro, estaban convencidas de que el Maestro haría algo al respecto. Sabían que su hermano se encontraba gravemente enfermo y, aunque no esperaban que Jesús dejara sus enseñanzas y predicaciones para venir en su ayuda, confiaban tanto en su poder para sanar las enfermedades, que pensaron que bastaría con que pronunciara las palabras de curación y Lázaro, de inmediato, recuperaría la salud. Y cuando Lázaro murió, pocas horas después de que el mensajero saliera de Betania hacia Filadelfia, llegaron a la conclusión de que el Maestro no había sabido de la enfermedad de su hermano hasta que era demasiado tarde y llevaba ya muerto varias horas.
\vs p168 0:3 Pero ellas, junto con todos sus amigos creyentes, se quedaron muy desconcertadas por el mensaje que el corredor trajo al llegar a Betania la mañana del martes. El mensajero insistió en que había oído decir a Jesús: “\ldots esta enfermedad no es en verdad para muerte”. Tampoco podían comprender por qué no había mandado a decirles algo ni les había brindado cualquier otra forma de ayuda.
\vs p168 0:4 Muchos amigos de poblados cercanos y de otros lugares de Jerusalén se acercaron para consolar a las afligidas hermanas. Lázaro y sus hermanas eran hijos de un judío acomodado y venerable, que había sido el más destacado residente de aquella pequeña aldea de Betania. Y pese a que los tres eran desde hacía mucho tiempo seguidores fervientes de Jesús, todos los que los conocían sentían un gran respeto hacia ellos. Habían heredado extensos viñedos y olivares en la zona, y su fortuna se ponía además de manifiesto en el hecho de que podían permitirse una sepultura privada en su propiedad. Sus padres también estaban sepultados allí.
\vs p168 0:5 María había renunciado a la idea de que Jesús llegara y se había dejado llevar por la pena, pero Marta acariciaba la esperanza de que vendría, incluso hasta el mismo momento, esa misma mañana, en el que rodaron la piedra hasta el sepulcro y sellaron la entrada. E incluso entonces, ella le dio instrucciones a un joven vecino para que vigilara la carretera de Jericó desde lo alto de la colina al este de Betania; fue él quien trajo a Marta la noticia de que Jesús y sus amigos se acercaban.
\vs p168 0:6 Cuando Marta se encontró con Jesús, se postró a sus pies, exclamando: ¡Maestro, si hubieras estado aquí, mi hermano no habría muerto!”. Muchos miedos cruzaron por la mente de Marta, pero no manifestó ninguna duda, ni se atrevió a criticar o a cuestionar la conducta del Maestro en relación a la muerte de Lázaro. Cuando Marta terminó de hablar, Jesús se inclinó y, levantándola sobre sus pies, le dijo: “Solo ten fe, Marta, y tu hermano resucitará”. Entonces, Marta respondió: “Yo sé que resucitará en la resurrección, en el día final; pero también sé ahora que todo lo que pidas a Dios, nuestro Padre te lo dará”.
\vs p168 0:7 Entonces Jesús, mirándola fijamente a los ojos, dijo: “Yo soy la resurrección y la vida; el que cree en mí, aunque esté muerto, vivirá. En verdad, todo aquel que vive y cree en mí, nunca morirá realmente. Marta, ¿crees esto?”. Y ella respondió al Maestro: “Sí, durante mucho tiempo he creído que tú eres el Libertador, el Hijo de Dios vivo, incluso aquel que habría de venir a este mundo”.
\vs p168 0:8 Como Jesús preguntó por María, Marta entró enseguida en la casa y, susurrándole al oído, le dijo a su hermana: “El Maestro está aquí y ha preguntado por ti”. Y cuando María oyó esto, se levantó rápidamente y se apresuró a reunirse con Jesús, que seguía en el mismo sitio en el que Marta se había encontrado antes con él, a cierta distancia de la casa. Los amigos que estaban con María, tratando de consolarla, cuando vieron que María se había levantado de prisa y que había salido, la siguieron, creyendo que iba al sepulcro a llorar.
\vs p168 0:9 Muchos de los presentes eran enemigos acérrimos de Jesús. Aquel era el motivo por el que Marta había ido sola a encontrarse con Jesús y por el que había vuelto a la casa para informar a María en secreto que Jesús preguntaba por ella. Aunque Marta ansiaba ver a Jesús, quería evitar en lo posible cualquier inconveniente que la repentina llegada de Jesús, en medio de un gran grupo de sus enemigos de Jerusalén, pudiera ocasionar. La intención de Marta había sido quedarse en la casa con sus amigos mientras María iba a recibir a Jesús, pero su intento se vio frustrado, porque todos ellos siguieron a María y se encontraron, pues, inesperadamente, ante la presencia del Maestro.
\vs p168 0:10 Marta llevó a María hasta Jesús y, cuando ella lo vio, se postró a sus pies, exclamando: “¡Si hubieras estado aquí, mi hermano no habría muerto!”. Y, cuando Jesús vio que todos lloraban tan afligidos, su alma se estremeció de compasión.
\vs p168 0:11 Cuando los sufrientes vieron que María había ido a recibir a Jesús, se apartaron a escasa distancia, mientras Marta y María hablaban con el Maestro y recibían sus palabras de consuelo, y ellas alentaba a mantenerse fuertes en su fe al Padre y resignarse por entero a la voluntad divina.
\vs p168 0:12 La mente humana de Jesús se sintió muy perturbada por el conflicto entre su amor por Lázaro y sus desconsoladas hermanas y su desdén y desprecio por las superficiales muestras de afecto de algunos de estos judíos descreídos y de intenciones criminales. Jesús detestaba y se indignaba ante una demostración, forzada, de duelo aparente por la muerte de Lázaro por parte de algunos de estos supuestos amigos, al mezclarse en sus corazones este falso sentimiento de pesar con la enconada enemistad que tenían contra él. No obstante, algunos de estos judíos, verdaderos amigos de la familia, eran sinceros en su duelo.
\usection{1. EN EL SEPULCRO DE LÁZARO}
\vs p168 1:1 Tras consolar unos momentos a Marta y a María, apartados de los dolientes, les preguntó: “¿Dónde lo pusisteis?”. Entonces Marta dijo: “Ven y ve”. Y cuando el Maestro caminaba en silencio con las dos afligidas hermanas, lloró. Al ver sus lágrimas los judíos amigos que iban tras ellos, uno de ellos dijo: ¡Mirad cuánto lo amaba! ¿No podía este, que abrió los ojos al ciego, haber hecho también que este hombre no muriera? En ese momento ya se hallaban ante el sepulcro familiar, que era una pequeña cueva natural, o pronunciada pendiente, en la cornisa de una roca que se alzaba a más de nueve metros de altura, al otro lado de la parcela del jardín.
\vs p168 1:2 \pc Resulta difícil explicar a las mentes humanas exactamente por qué lloró Jesús. Aunque tenemos acceso al historial de sus emociones humanas junto con su pensamiento divino, tal como constan en la mente del modelador personificado, no estamos del todo seguros de la verdadera causa de estas manifestaciones de emoción. Nos inclinamos a creer que Jesús lloró debido a una serie de pensamientos y sentimientos, como los que siguen, que cruzaban por su mente en aquel momento:
\vs p168 1:3 \li{1.}Se compadecía auténticamente del dolor de Marta y María; sentía un afecto humano, real y profundo por estas hermanas que habían perdido a su hermano.
\vs p168 1:4 \li{2.}Su mente se agitó por la presencia de la multitud de dolientes, algunos sinceros y otros simples farsantes. Le disgustaban siempre estas demostraciones externas de duelo. Sabía que las hermanas amaban a su hermano y que tenían fe en la supervivencia de los creyentes. Es posible que estas emociones contradictorias expliquen por qué gimió al aproximarse a la tumba.
\vs p168 1:5 \li{3.}En verdad, Jesús dudó si devolver a Lázaro a la vida mortal. Sus hermanas realmente lo necesitaban, pero Jesús lamentaba tener que hacer que su amigo volviera y que pasara por la amarga persecución, que sabía que Lázaro tendría que padecer, por haber sido el destinatario de la más grande de todas las demostraciones de poder divino del Hijo del Hombre.
\vs p168 1:6 \pc Y ahora podemos relatar un hecho interesante y aleccionador: aunque esta narrativa se desenvuelva en apariencias como un hecho natural y normal en los asuntos humanos, hay otros aspectos secundarios fascinantes. Cuando el mensajero acudió a Jesús el domingo para informarle de la enfermedad de Lázaro y, aunque Jesús envió palabras diciendo que “no era para muerte”, al mismo tiempo, él, en persona, fue a Betania e incluso le preguntó a las hermanas: “¿Dónde lo pusisteis?”. A pesar de que todo esto parece indicar que el proceder del Maestro era el ordinario en estos hechos de la vida y según el conocimiento limitado de la mente humana, no obstante, los archivos del universo revelan que, tras la muerte de Lázaro, el modelador personificado de Jesús dictó órdenes para que su modelador del pensamiento se quedara de forma indefinida en el planeta, y este mandato quedó registrado solo quince minutos antes de que Lázaro exhalara su último suspiro.
\vs p168 1:7 ¿Sabía la mente divina de Jesús, incluso antes de que Lázaro falleciera, que lo resucitaría de entre los muertos? No tenemos conocimiento de ello. Solo sabemos lo que hacemos constar aquí.
\vs p168 1:8 \pc Muchos de los enemigos de Jesús se burlaron de sus expresiones de afecto, y decían entre ellos: “Si en tan alto aprecio tenía a este hombre, ¿por qué tardó tanto tiempo en venir a Betania? Si es lo que se afirma de él, ¿por qué no salvó a su querido amigo? ¿Qué bueno tiene sanar desconocidos en Galilea si no puede salvar a quienes amas?”. Y también se mofaron de Jesús y menospreciaron sus enseñanzas y su labor.
\vs p168 1:9 Y, así, ese jueves por la tarde, sobre las dos y media, se dispuso todo para que en este pequeño poblado de Betania se realizara la mayor de las obras del ministerio en la tierra de Miguel de Nebadón, la más grande manifestación de poder divino de su encarnación, ya que su propia resurrección ocurrió tras haberse liberado de los lazos de su vida mortal.
\vs p168 1:10 El pequeño grupo que se había reunido ante el sepulcro de Lázaro poco podría llegar a imaginarse que, cerca de ellos, había una gran concurrencia de todos los órdenes de seres celestiales, congregados bajo el liderazgo de Gabriel y, en ese momento a la espera de las directrices del modelador personificado de Jesús, vibrando expectantes y listos para dar cumplimiento al mandato de su amado Soberano.
\vs p168 1:11 Cuando Jesús pronunció esas palabras, ordenando: “Quitad la piedra”, las multitudes celestiales allí congregadas se prepararon para llevar a cabo el acto de resucitar a Lázaro a su cuerpo mortal. Tal forma de resurrección conlleva, en su realización, dificultades que trascienden por mucho el procedimiento ordinario de la resurrección de las criaturas mortales a una forma morontial, y precisa de muchos más seres celestiales y una organización mucho mayor de los recursos del universo.
\vs p168 1:12 Cuando Marta y María oyeron el mandato de Jesús de hacer rodar a un lado la piedra de la entrada del sepulcro, experimentaron emociones encontradas. María tenía la esperanza de que Lázaro resucitara de entre los muertos. Pero a Marta, aunque compartía en cierta medida la fe de su hermana, la movía más el temor de que Lázaro no estuviera presentable, en su apariencia, ante Jesús, los apóstoles y sus amigos. Dijo Marta: “¿Debemos apartar la piedra? Mi hermano lleva ya cuatro días muerto, y su cuerpo habrá empezado a descomponerse”. Marta dijo también aquello porque no estaba segura de por qué el Maestro había pedido que se quitara la piedra; quizás pensó que Jesús únicamente quería ver a Lázaro por última vez. Marta no era ni estable ni firme en su disposición de ánimo. Al haber vacilaciones en cuanto a hacerlo o no, Jesús dijo: “¿Es que no te dije en un principio que esta enfermedad no era para muerte? ¿Es que no he venido para cumplir mi promesa? Y después de venir a vosotras, ¿es que no te dije que si creías, verías la gloria de Dios? ¿Por qué dudas entonces? ¿Cuánto tiempo ha de pasar para que creas y obedezcas?”.
\vs p168 1:13 Cuando Jesús acabó de hablar, sus apóstoles, con la ayuda de vecinos voluntariosos, agarraron la piedra y la apartaron de la entrada al sepulcro rodándola.
\vs p168 1:14 \pc Por lo común, los judíos pensaban que la gota de hiel sobre la punta de la espada del ángel de la muerte empezaba a actuar al fin del tercer día, por lo que surtía totalmente efecto al cuarto día. Admitían que el alma del hombre podía quedarse alrededor del sepulcro hasta el final del tercer día, intentando reanimar el cuerpo muerto; pero creían firmemente que esa alma se iba a la morada de los espíritus de los fallecidos antes de que el cuarto día hubiera amanecido.
\vs p168 1:15 Estas creencias y opiniones sobre los muertos y la partida de los espíritus de los fallecidos contribuyeron a dar seguridad en la mente de todos los que estaban entonces allí presentes en el sepulcro de Lázaro y después a todos los que oyeran lo que estaba a punto de ocurrir, que aquel era, real y verdaderamente, un caso de resurrección de entre los muertos mediante un acto personal de quien afirmó que era “la resurrección y la vida”.
\usection{2. LA RESURRECCIÓN DE LÁZARO}
\vs p168 2:1 Estando este grupo de unos cuarenta y cinco mortales ante el sepulcro, pudieron entrever la silueta de Lázaro, envuelto en vendas de lino, depositado en el nicho inferior derecho de la cueva funeraria. Mientras estas criaturas terrenales se hallaban allí en un atónito silencio, una inmensa multitud de seres celestiales se habían colocado en sus lugares, preparados para pasar a la acción cuando Gabriel, su comandante, así lo indicara.
\vs p168 2:2 Jesús alzó los ojos y dijo: “Padre, gracias te doy porque me has oído y has accedido a mi petición. Sé que siempre me oyes, te hablo de este modo por los que están ahora aquí junto a mí, para que puedan creer que tú me has enviado al mundo, para que sepan que obras conmigo en lo que estamos a punto de hacer”. Y cuando acabó de orar, clamó a gran voz: “¡Lázaro, ven fuera!”.
\vs p168 2:3 Aunque los expectantes humanos permanecieron inmóviles, la inmensa multitud celestial entró en acción conjuntamente, en obediencia a la voz del creador. En solo doce segundos de tiempo de la tierra, el cuerpo, hasta aquel entonces inerte de Lázaro, comenzó a moverse y se sentó pronto en el estante de piedra sobre el que había yacido. Su cuerpo estaba atado con vendas mortuorias y, su rostro, envuelto en un sudario. Y, al ponerse de pie ante ellos ---vivo---, Jesús dijo: “Desatadlo y dejadlo ir”.
\vs p168 2:4 Todos, excepto los apóstoles y Marta y María, huyeron a la casa. Estaban pálidos de miedo y sobrecogidos por el estupor. Aunque algunos se quedaron allí, muchos otros se fueron apresuradamente a sus hogares.
\vs p168 2:5 Lázaro saludó a Jesús y a los apóstoles y preguntó qué significaban las vendas mortuorias y por qué se había despertado en el jardín. Jesús y los apóstoles se apartaron, mientras Marta le hablaba a Lázaro de su muerte, entierro y resurrección. Tuvo que explicarle que había muerto el domingo y que se le había traído de vuelta a la vida el jueves, ya que no había tenido conciencia del tiempo desde que se había quedado dormido en el sueño de la muerte.
\vs p168 2:6 \pc Mientras Lázaro salía del sepulcro, el modelador personificado de Jesús, ahora era jefe de los suyos en este universo local, dio la orden al antiguo modelador de Lázaro, que se hallaba en espera en aquel momento, para que retomara su morada en la mente y alma del hombre resucitado.
\vs p168 2:7 \pc Luego, Lázaro se dirigió hacia donde estaba Jesús y, con sus hermanas, se arrodilló a los pies del Maestro para darle gracias y rendir alabanzas a Dios. Jesús, tomándolo de la mano, lo alzó, diciéndole: “Hijo mío, lo que te ha sucedido a ti lo experimentarán también todos aquellos que creen en este evangelio, salvo que resucitarán con una forma más gloriosa. Tú serás un testigo vivo de la verdad que he hablado: yo soy la resurrección y la vida. Pero entremos todos ahora a la casa y alimentemos nuestros cuerpos físicos”.
\vs p168 2:8 \pc Conforme caminaban hasta la casa, Gabriel despidió a los grupos auxiliares de la multitud celestial allí congregada, mientras dejaba constancia del primer caso en Urantia, y el último, en el que una criatura mortal había resucitado de la muerte con el mismo cuerpo físico que tenía al morir.
\vs p168 2:9 \pc Lázaro apenas podía comprender lo que había ocurrido. Sabía que había estado muy enfermo, pero solo recordaba haberse quedado dormido y haber sido despertado. Nunca le fue posible contar nada de lo ocurrido sobre esos cuatro días en el sepulcro, porque había estado completamente inconsciente. El tiempo es inexistente para quienes duermen el sueño de la muerte.
\vs p168 2:10 Aunque a raíz de este asombroso acto, muchos creyeron en Jesús, otros solo endurecieron aún más sus corazones para rechazarlo. Hacia el mediodía del día siguiente, este hecho se había difundido por todo Jerusalén. Un buen número de hombres y mujeres acudieron a Betania para ver a Lázaro y hablar con él, y los fariseos, alarmados y desconcertados, convocaron a toda prisa al sanedrín a una reunión para decidir qué se debía hacer ante estos últimos acontecimientos.
\usection{3. REUNIÓN DEL SANEDRÍN}
\vs p168 3:1 Pese a que el testimonio de este hombre, resucitado de entre los muertos, contribuyó en mucho a consolidar la fe de una gran cantidad de creyentes en el evangelio del reino, ejerció poca o ninguna influencia sobre la actitud de los líderes y dirigentes religiosos de Jerusalén salvo para acelerar su decisión de matar a Jesús y poner fin a su labor.
\vs p168 3:2 \pc A la una del día siguiente, viernes, el sanedrín se reunió para deliberar más a fondo sobre la cuestión, “¿qué vamos a hacer con Jesús de Nazaret?”. Tras más de dos horas de análisis y de agrio debate, cierto fariseo presentó una resolución en la que se pedía la muerte inmediata de Jesús, afirmando que era una amenaza para todo Israel y comprometiendo oficialmente al sanedrín a votar la resolución que instaba a matar a Jesús, sin juicio, algo contrario a todo precedente.
\vs p168 3:3 Repetidas veces este respetable órgano de líderes judíos había decretado que se prendiera a Jesús y se sometiera a juicio bajo cargos de blasfemia y de numerosas otras acusaciones de menosprecio de la ley sagrada judía. Anteriormente, ya habían llegado incluso a declarar que debía morir, pero aquella era la primera vez que el sanedrín dejaba constancia de su deseo de decretar su muerte sin juicio previo. Si bien, no se votó esta resolución porque catorce miembros del sanedrín renunciaron a la vez cuando se propuso aquella insólita actuación. Aunque estas dimisiones tardaban en aprobarse oficialmente casi dos semanas, este grupo de catorce abandonó el sanedrín aquel mismo día para no volver jamás a sentarse en el consejo. Cuando posteriormente se admitieron dichas dimisiones, se expulsaron a otros cinco miembros porque sus compañeros creían que albergaban sentimientos de amistad hacia Jesús. Con la destitución de estos diecinueve hombres, el sanedrín estaba en condiciones de juzgar y condenar a Jesús con un acuerdo que rozaba la unanimidad.
\vs p168 3:4 La semana siguiente, Lázaro y sus hermanas fueron citados a comparecer ante el sanedrín. Tras oírse su testimonio, no quedaba ninguna duda de que Lázaro había resucitado de entre los muertos. Aunque en sus diligencias, el sanedrín prácticamente admitió la resurrección de Lázaro, al hacerlas constar en acta se hizo anotar una resolución en la que atribuía este y otros prodigios obrados por Jesús al poder del príncipe de los diablos, aliado de Jesús según se declaró.
\vs p168 3:5 Al margen de cuál fuese la fuente de ese poder obrador de prodigios, estos líderes judíos estaban convencidos de que, si no lo paraban de inmediato, en muy poco tiempo toda la gente ordinaria creería en él; y se producirían, además, graves conflictos con las autoridades romanas, ya que muchos de sus creyentes lo consideraban el Mesías, el libertador de Israel.
\vs p168 3:6 Fue en esa misma reunión del sanedrín en la que Caifás, el sumo sacerdote, expresó por primera vez ese viejo adagio judío, que repetiría después tantas veces: “Conviene que un solo hombre muera a que lo haga el pueblo”.
\vs p168 3:7 Aunque a Jesús se le había advertido del proceder del sanedrín en aquel tenebroso viernes por la tarde, no se sintió perturbado en lo más mínimo y, durante el \bibemph{sabbat,} continuó descansando con algunos amigos en Betfagé, un poblado cercano a Betania. El domingo por la mañana temprano, Jesús y los apóstoles se congregaron, como habían acordado previamente, en la casa de Lázaro, y, tras despedirse de la familia de Betania, iniciaron su camino de regreso al campamento de Pella.
\usection{4. RESPUESTA A LA ORACIÓN}
\vs p168 4:1 En el camino de Betania a Pella, los apóstoles hicieron muchas preguntas a Jesús y excepto a las relacionadas con los detalles de la resurrección de los muertos, el Maestro respondió extensamente a las demás. Esas cuestiones sobrepasaban la capacidad de comprensión de sus apóstoles, por lo que declinó comentarlas con ellos. Al haber partido de Betania en secreto, iban solos. Jesús, por tanto, aprovechó la ocasión para explicarles a los diez muchas cosas con la idea de prepararlos para los días difíciles que tenían por delante.
\vs p168 4:2 Los apóstoles estaban mentalmente muy agitados y dedicaron un tiempo considerable a comentar sus últimas experiencias relacionadas con la oración y la respuesta a esta. Todos recordaban la afirmación que le hizo Jesús en Filadelfia al mensajero de Betania, cuando dijo simplemente: “Esta enfermedad no es en verdad para muerte”. Y, sin embargo, a pesar de esta promesa, Lázaro de hecho había muerto. Durante todo el día, una y otra vez, volvieron a conversar sobre este tema de la respuesta a la oración.
\vs p168 4:3 Las respuestas que les dio Jesús a sus numerosas preguntas pueden resumirse de la siguiente manera:
\vs p168 4:4 \li{1.}La mente finita se sirve de la oración para tratar de acercarse al Infinito. Consiguientemente, la acción de orar debe estar limitada por el conocimiento, la sabiduría y las cualidades de lo finito; del igual modo, su respuesta debe estar condicionada por la perspectiva, los objetivos, los ideales y las prerrogativas del Infinito. No puede observarse que haya una interrupción de la continuidad de los fenómenos materiales entre la acción de orar y el hecho de recibir una plena respuesta espiritual.
\vs p168 4:5 \li{2.}Cuando una oración queda aparentemente sin respuesta, este retraso, con frecuencia, anuncia una respuesta mejor, aunque, por alguna buena razón, esta llegue a sufrir alguna demora considerable. Cuando Jesús dijo que la enfermedad de Lázaro no era en verdad para la muerte, Lázaro ya llevaba muerto once horas. No se niega la respuesta a ninguna oración sincera, salvo cuando la perspectiva superior del mundo espiritual ha concebido una respuesta más favorable, una respuesta que satisfaga la petición del espíritu del hombre a diferencia de hacerlo a la que parte simplemente de la mente humana.
\vs p168 4:6 \li{3.}Las oraciones que se hacen en el tiempo, cuando las guía el espíritu y se expresan en la fe, son normalmente tan extensas y globales que solo pueden responderse en la eternidad; la petición finita está a veces tan llena de deseo de alcanzar el Infinito, que su respuesta se ha de posponer largamente en espera de la creación en el orante de una adecuada capacidad de receptividad; la oración de fe puede llegar a ser tan totalizadora que solo en el Paraíso podrá recibir respuesta.
\vs p168 4:7 \li{4.}Las respuestas a la oración surgida de la mente mortal son, habitualmente, de tal naturaleza que únicamente se pueden recibir y reconocer después de que esa misma mente orante haya alcanzado el estado inmortal. La oración de un ser material muchas veces puede recibir respuesta exclusivamente cuando este haya progresado hasta llegar al nivel del espíritu.
\vs p168 4:8 \li{5.}La oración de una persona que conoce a Dios puede desvirtuarse de tal manera por la ignorancia y la superstición, que la respuesta a esta sería sumamente inaceptable. En ese caso, los seres espirituales mediadores traducen de tal forma esa oración que, cuando la respuesta llega, el suplicante no puede en absoluto reconocerla como tal.
\vs p168 4:9 \li{6.}Todas las oraciones que son verdaderas van dirigidas a seres espirituales, y todas estas peticiones deben responderse en términos espirituales, y todas estas respuestas deben consistir en realidades espirituales. Los seres espirituales no pueden responder materialmente a las peticiones espirituales, ni siquiera las que les hacen los seres materiales. La oración de los seres materiales es eficaz solo cuando “oran en espíritu”.
\vs p168 4:10 \li{7.}No cabe esperar respuesta a ninguna oración a menos que nazca del espíritu y se nutra de la fe. La oración sincera entraña que habéis concedido, prácticamente y por adelantado, a quienes la oyen, pleno derecho para responder a vuestras peticiones según esa sabiduría suprema y ese amor divino que, de acuerdo con vuestra fe, los llevan a responder siempre a vuestras oraciones.
\vs p168 4:11 \li{8.}El niño está siempre en su derecho cuando decide pedirle algo al padre; y el padre tiene siempre sus obligaciones paternas hacia el niño inmaduro, cuando su mayor sabiduría dicta que la respuesta a la petición del niño se demore, modifique, aparte, transcienda o posponga hasta otro estadio de su ascenso espiritual.
\vs p168 4:12 \li{9.}No vaciléis en formular oraciones que expresen vuestros anhelos espirituales; no dudéis de que recibiréis respuesta a vuestras peticiones. Estas respuestas estarán atesoradas, aguardando a que consigáis alcanzar esos niveles espirituales futuros que hayáis realmente conseguido en el cosmos, en este mundo o en los demás, en los que os será posible reconocer y tomar posesión de esas respuestas, por tanto tiempo esperadas, a vuestras tempranas pero prematuras peticiones.
\vs p168 4:13 \li{10.}Todas las peticiones nacidas genuinamente del espíritu recibirán cumplida respuesta. Pedid, y se os dará. Pero debéis recordar que sois criaturas que avanzáis en el tiempo y el espacio; debéis, por tanto, tener siempre presente el factor espacio\hyp{}temporal cuando esperéis recibir personalmente respuestas completas a vuestras múltiples oraciones y peticiones.
\usection{5. QUÉ FUE DE LÁZARO}
\vs p168 5:1 Lázaro permaneció en su casa de Betania, convirtiéndose en el principal centro de interés de muchos creyentes sinceros y de numerosos curiosos, hasta los días de la crucifixión de Jesús, cuando recibió el aviso de que el sanedrín había decretado su muerte. Los dirigentes de los judíos estaban determinados a evitar que las enseñanzas de Jesús siguieran propagándose, y consideraron oportuno que sería inútil matar a Jesús si permitían que Lázaro, que representaba su momento álgido como obrador de prodigios, viviera y diera testimonio del hecho de que Jesús lo había resucitado de entre los muertos. Lázaro ya había padecido una amarga persecución por parte de ellos.
\vs p168 5:2 Por tanto, Lázaro se despidió precipitadamente de sus hermanas en Betania, huyendo a través de Jericó y cruzando el Jordán, no permitiéndose descansar hasta llegar a Filadelfia. Lázaro conocía bien a Abner, y allí se sentía a salvo de las maquinaciones homicidas del malvado sanedrín.
\vs p168 5:3 Poco después de aquello, Marta y María se deshicieron de sus tierras de Betania y se unieron a su hermano en Perea. Entretanto, Lázaro se había convertido en el tesorero de la Iglesia de Filadelfia. Fue un firme defensor de Abner en su controversia con Pablo y con la Iglesia de Jerusalén, muriendo, por último, a los 67 años de edad, de la misma enfermedad que se lo llevó siendo más joven en Betania.
