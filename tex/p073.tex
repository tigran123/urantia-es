\upaper{73}{El jardín de Edén}
\author{Solonia}
\vs p073 0:1 La decadencia cultural y la pobreza espiritual ocasionadas por la caída de Caligastia y la consiguiente confusión social tuvieron poco efecto sobre la condición física o biológica de los pueblos de Urantia. La evolución orgánica prosiguió a buen ritmo, con bastante independencia del revés moral y cultural que con tanta celeridad siguieron a la deslealtad de Caligastia y Daligastia. Y llegó un momento en la historia planetaria, hace casi cuarenta mil años, cuando los portadores de vida en servicio comprobaron que, desde un punto de vista puramente biológico, el progreso evolutivo de las razas de Urantia estaba llegando a su culmen. Siendo de esta misma opinión, los síndicos melquisedecs acordaron de inmediato unirse a los portadores de vida para presentar una petición a los altísimos de Edentia, solicitando que Urantia se sometiera a inspección con el fin de que se autorizara el envío de mejoradores biológicos, esto es, de un hijo y una hija material.
\vs p073 0:2 La petición iba dirigida a los altísimos de Edentia, porque ellos habían ejercido una jurisdicción directa sobre muchos de los asuntos de Urantia desde la caída de Caligastia y el vacío temporal de autoridad que se había producido en Jerusem.
\vs p073 0:3 Tabamantia, supervisor soberano de la serie de mundos decimales o experimentales, vino a inspeccionar el planeta y, tras su examen del progreso racial, propuso debidamente que se le concediese a Urantia unos hijos materiales. En algo menos de cien años después de esta inspección, Adán y Eva, un hijo y una hija material del sistema local, llegaron y comenzaron la difícil labor de intentar desenredar los confusos asuntos de un planeta atrasado por la rebelión y bajo el obligado estado de aislamiento espiritual.
\usection{1. NODITAS Y AMADONITAS}
\vs p073 1:1 En un planeta normal, la llegada del hijo material habitualmente anunciaría la proximidad de una gran era de invenciones, progreso material y lucidez intelectual. La era posadánica es el gran período de la ciencia en la mayoría de los planetas, pero no era así en Urantia. Aunque el planeta estaba poblado por razas aptas físicamente, las tribus languidecían en las profundidades del salvajismo y del estancamiento moral.
\vs p073 1:2 Diez mil años después de la rebelión, se había borrado prácticamente todo lo ganado durante el gobierno del príncipe; las razas del mundo habían avanzado tan poco que parecía que este errado hijo nunca había venido a Urantia. Solo entre los noditas y los amadonitas perduraban las tradiciones de Dalamatia y la cultura del príncipe planetario.
\vs p073 1:3 Los \bibemph{noditas} eran los descendientes de los miembros rebeldes de la comitiva del príncipe; su nombre se deriva de su primer líder, Nod, antiguo presidente de la comisión de manufactura y comercio de Dalamatia. Los \bibemph{amadonitas} eran los descendientes de aquellos andonitas que eligieron permanecer leales a Van y Amadón. “Amadonita” es más una denominación cultural y religiosa que un término racial; desde una perspectiva racial, los amadonitas eran esencialmente \bibemph{andonitas}. “Nodita” es un término cultural al igual que racial, ya que los mismos noditas constituyeron la octava raza de Urantia.
\vs p073 1:4 Entre los noditas y los amadonitas existía una tradicional enemistad. Esta hostilidad heredada afloraba constantemente a la superficie siempre que los vástagos de ambos grupos intentaban emprender alguna iniciativa en común. Incluso más tarde, en relación a Edén, les resultó muy difícil trabajar juntos sosegadamente.
\vs p073 1:5 Poco tiempo después de la destrucción de Dalamatia, los seguidores de Nod se dividieron en tres grupos principales. El grupo central permaneció en las inmediaciones de su lugar de origen, cerca de la cabecera del Golfo Pérsico. El grupo oriental emigró hacia las regiones de las altiplanicies de Elam, justo al este del valle del Éufrates. El grupo occidental se situó en las costas sirias del nordeste del Mediterráneo y en el territorio aledaño.
\vs p073 1:6 Estos noditas habían procreado sin restricción con las razas sangiks y habían dejado atrás una capaz progenie. Y algunos de los descendientes de los dalamatianos rebeldes se unieron después a Van y a sus leales seguidores en las tierras del norte de Mesopotamia. Aquí, en las proximidades del lago de Van y en la región sur del Mar Caspio, los noditas socializaron y se cruzaron con los amadonitas, y se contaban entre los “poderosos hombres de la antigüedad”.
\vs p073 1:7 Con anterioridad a la llegada de Adán y Eva, estos grupos ---los noditas y los amadonitas--- eran las razas más avanzadas y cultas de la Tierra.
\usection{2. PLANIFICACIÓN DEL JARDÍN}
\vs p073 2:1 Por casi los cien años que precedieron a la inspección de Tabamantia, Van y sus colaboradores desde su sede en las altiplanicies de la ética y la cultura del mundo, habían estado predicando la venida de un hijo de Dios prometido, un mejorador de las razas, un maestro de la verdad y el digno sucesor del traidor Caligastia. Aunque la mayoría de los habitantes del mundo de aquellos días mostraba poco o ningún interés por tales predicciones, aquellos cercanos a Van y Amadón sí tomaron en serio estas enseñanzas y comenzaron a hacer planes para dar verdaderamente acogida al hijo prometido.
\vs p073 2:2 Van contó a sus colaboradores más próximos la historia de los hijos materiales de Jerusem; lo que había conocido de ellos antes de venir a Urantia. Sabía bien que estos hijos adánicos siempre residían en viviendas ajardinadas sencillas pero placenteras y propuso, ochenta y tres años antes de la llegada de Adán y Eva, que se dedicaran a proclamar su venida y a preparar un hogar ajardinado para su recibimiento.
\vs p073 2:3 Desde esta sede en las altiplanicies y desde sesenta y un asentamientos bastante dispersos, Van y Amadón reclutaron a un colectivo de más de tres mil trabajadores dispuestos y entusiastas, los cuales, en una solemne asamblea, se consagraron a la misión de preparar la venida del hijo prometido ---o, por lo menos, esperado---.
\vs p073 2:4 Van dividió a sus voluntarios en cien compañías con un capitán a cargo de cada una de ellas y un colaborador de su equipo personal que servía de oficial de enlace, manteniendo a Amadón como su colaborador personal. Todas estas comisiones comenzaron activamente a realizar su trabajo preliminar, y la comisión para la ubicación del Jardín salió a la búsqueda del lugar idóneo.
\vs p073 2:5 \pc Aunque a Caligastia y Daligastia se les había despojado de buena parte de su capacidad para obrar el mal, hicieron todo lo posible para frustrar y dificultar la tarea de preparación del Jardín. Si bien, sus malvadas maquinaciones fueron en gran medida neutralizadas por la acción incondicional de las casi diez mil criaturas intermedias leales que trabajaron incansablemente para hacer avanzar aquella iniciativa.
\usection{3. EMPLAZAMIENTO DEL JARDÍN}
\vs p073 3:1 La comisión para la ubicación del Jardín, tras haberse ausentado durante casi tres años, presentó un informe favorable de tres posibles emplazamientos: el primero era una isla en el Golfo Pérsico; el segundo, un emplazamiento fluvial, que sería utilizado más tarde como segundo jardín; y, el tercero, una península larga y estrecha ---casi una isla--- que sobresalía en dirección oeste desde las orillas orientales del Mar Mediterráneo.
\vs p073 3:2 La comisión prefirió casi unánimemente la tercera elección. Se optó por este lugar, teniéndose que emplear dos años en el traslado de la sede mundial de la cultura, incluido el árbol de la vida, a esta península mediterránea. Menos un solo grupo, todos los habitantes de la península la desalojaron pacíficamente a la llegada de Van y sus acompañantes.
\vs p073 3:3 \pc La península mediterránea tenía un clima saludable y una temperatura constante; este estable tiempo atmosférico se debía a las montañas que la rodeaban y al hecho de que dicha región era prácticamente una isla en un mar interior. Aunque llovía abundantemente en las altiplanicies circundantes, raras veces lo hacía en Edén mismo. Pero cada noche, desde la extensa red artificial de canales de irrigación, “subía un vapor” que revitalizaba la vegetación del Jardín.
\vs p073 3:4 El litoral costero de esta masa terrestre era bastante elevado, y el istmo que la enlazaba con el continente tenía solo algo más de cuarenta y tres kilómetros de ancho en su punto más estrecho. El gran río que regaba el Jardín descendía de las tierras más elevadas de la península, fluía en dirección este por esta franja peninsular hacia el continente y, de ahí, a través de las tierras bajas de Mesopotamia, hasta el mar distante. Se nutría por cuatro afluentes que tenían su origen en las lomas costeras de la península edénica, y tales afluentes son los “cuatro brazos” del río que “salía de Edén” y que después se confundirían con los ramales de los ríos que rodeaban al segundo jardín.
\vs p073 3:5 En las montañas que circundaban al Jardín había abundantes piedras y metales preciosos, aunque se les prestó muy poca atención. La idea preponderante había de ser la glorificación de la horticultura y la exaltación de la agricultura.
\vs p073 3:6 El sitio elegido para el Jardín era probablemente el más bello paraje del mundo en su género; el clima era entonces excelente. En ninguna otra parte había un lugar que se pudiera haber prestado con tanta perfección a convertirse en tal paraíso de manifestación botánica. Aquí se daba cita lo más selecto de la civilización de Urantia. Fuera de allí, el mundo yacía en tinieblas, ignorancia y salvajismo. Edén era el único sitio luciente de Urantia; era por naturaleza un ensueño de hermosura, y pronto su paisaje glorioso se hizo poesía espléndida, perfecta.
\usection{4. FUNDACIÓN DEL JARDÍN}
\vs p073 4:1 Cuando los hijos materiales, los mejoradores biológicos, comienzan su estancia en un mundo evolutivo, a su lugar de residencia se le suele llamar el Jardín de Edén, porque se caracteriza por la belleza floral y la grandeza botánica de Edentia, la capital de la constelación. Van conocía bien estas costumbres y, en consecuencia, dispuso que toda la península se dedicase al Jardín. Se hicieron planes en cuanto el pastoreo y la ganadería para el territorio continental contiguo. Respecto a la vida animal, en el parque solo habría aves y las distintas especies domesticadas. Las indicaciones de Van eran que Edén debía ser un jardín, y solamente un jardín. Nunca se sacrificaron animales dentro de sus recintos. A lo largo de todos los años de su construcción, toda la carne que los trabajadores del Jardín consumían se traía de los rebaños que se cuidaban en el continente.
\vs p073 4:2 La primera tarea que se realizó fue la construcción de una muralla de ladrillo cruzando el istmo de la península. Una vez finalizada, se pudo proceder sin obstáculos a la verdadera labor de embellecimiento paisajístico y a la edificación de las viviendas.
\vs p073 4:3 Se creó un jardín zoológico construyendo una muralla más pequeña justo fuera de la muralla principal; el espacio intermedio, que estaba ocupado por todo tipo de animales salvajes, servía de defensa adicional contra ataques hostiles. Esta colección de animales se dividió en doce grandes grupos, con caminos amurallados entre ellos que conducían a las doce puertas del Jardín; el río y sus pastizales limítrofes ocupaban el área central.
\vs p073 4:4 Para la preparación del Jardín solo se emplearon trabajadores voluntarios; jamás se usaron asalariados. Cultivaban el Jardín y apacentaban sus rebaños para sustentarse; también recibieron aportaciones de alimentos de creyentes de las proximidades. Y esta gran iniciativa se llevó a cabo hasta su conclusión, a pesar de las correspondientes dificultades por el estado confuso del mundo durante estos tiempos turbulentos.
\vs p073 4:5 Pero se suscitó una gran decepción cuando Van, al no saber cuánto tiempo tardarían los esperados hijo e hija en llegar, propuso que, en caso de que su venida se retrasara, también se formara a la generación más joven en la labor de continuar con la tarea emprendida. Aquello pareció un reconocimiento de falta de fe por parte de Van y provocó muchos problemas, causando numerosas deserciones; pero Van siguió adelante con los preparativos planificados, cubriendo, entretanto, los puestos de los desertores con voluntarios más jóvenes.
\usection{5. UN HOGAR AJARDINADO}
\vs p073 5:1 En el centro de la península edénica se hallaba el excelente templo de piedra del Padre Universal, el sagrado santuario del Jardín. Al norte se estableció la sede administrativa; al sur se construyeron las viviendas para los trabajadores y sus familias; al oeste se facilitó una parcela de suelo para las proyectadas escuelas del sistema educativo del hijo esperado, mientras que al “oriente de Edén” se construyeron las residencias destinadas al hijo prometido y a sus vástagos inmediatos. En los planos arquitectónicos de Edén, se asignaban hogares y suelo abundante para un millón de seres humanos.
\vs p073 5:2 En el momento de la llegada de Adán, aunque solo se había completado una cuarta parte del Jardín, este ya disponía de miles de kilómetros de acequias de riego y de más de diecinueve mil kilómetros de caminos y carreteras pavimentados. Había algo más de cinco mil edificios de ladrillo en los distintos sectores, y los árboles y las plantas eran casi innumerables. Cualquier agrupación habitacional en el parque no podía constar de más de siete casas. Y, aunque sencillas, las estructuras del Jardín eran, al mismo tiempo, muy artísticas. Las carreteras y los caminos estaban bien construidos, y la arquitectura paisajista era exquisita.
\vs p073 5:3 Las disposiciones sanitarias del Jardín eran mucho más avanzadas que cualquiera de las que se habían probado hasta entonces en Urantia. El agua para beber de Edén se mantenía salubre mediante el estricto cumplimiento de las regulaciones sanitarias diseñadas para conservar su pureza. Durante estos tiempos primitivos, se produjeron muchos problemas porque se descuidaban estas normas, pero Van, paulatinamente, inculcó en sus colaboradores la importancia de no permitir que cayese nada en el suministro de agua del Jardín.
\vs p073 5:4 Antes de que se llegara a instalar un sistema de alcantarillado, los edenitas solían enterrar rigurosamente todos los desechos o materia en descomposición. Los inspectores de Amadón hacían sus rondas diariamente en busca de posibles causas de enfermedad. Los urantianos no volverían a tomar conciencia de la importancia de la prevención de las enfermedades humanas hasta los últimos tiempos de los siglos XIX y XX. Antes de la disrupción del régimen de Adán, se había construido un sistema de alcantarillado de ladrillo que se extendía por debajo de los muros y desembocaba en el río de Edén a más de un kilómetro y medio más allá del muro exterior o menor del Jardín.
\vs p073 5:5 En el momento de la llegada de Adán, la mayoría de las plantas de aquella zona del mundo crecía en Edén. Ya se habían mejorado de forma notable muchos frutos, cereales y frutos de cáscara dura. Aquí se cultivaron por primera vez muchos vegetales y cereales modernos; pero decenas de variedades de plantas alimenticias se perderían posteriormente para el mundo.
\vs p073 5:6 Aproximadamente, el cinco por ciento del Jardín era objeto de un cultivo artificial intensivo, el quince por ciento estaba parcialmente cultivado, el resto se dejó más o menos en su estado natural pendiente de la llegada de Adán, pues se pensaba que era mejor terminar el parque atendiendo a sus ideas.
\vs p073 5:7 Y de este modo quedó el Jardín de Edén listo para recibir al Adán prometido y a su consorte. Y este Jardín hubiese hecho honor a un mundo bajo una óptima administración y una normal gobernación. Adán y Eva se sintieron bastante complacidos con el plan general de Edén; no obstante, hicieron muchos cambios en el mobiliario de su vivienda personal.
\vs p073 5:8 Aunque en el momento de la llegada de Adán, la tarea de embellecimiento aún no estaba acabada del todo, el lugar era ya una joya de belleza botánica; y, durante los primeros días de su estancia en Edén, todo el Jardín adquirió una nueva forma y adoptó nuevas proporciones de belleza y magnificencia. Nunca antes ni después de este momento, se dio cobijo en Urantia a una manifestación tan hermosa y completa de horticultura y agricultura.
\usection{6. EL ÁRBOL DE LA VIDA}
\vs p073 6:1 En el centro del templo del Jardín, Van plantó el árbol de la vida, durante mucho tiempo custodiado, cuyas hojas eran para la “sanidad de las naciones”, y cuyo fruto le había servido de sustento a él por largo tiempo durante su estancia en la tierra. Van sabía bien que Adán y Eva, una vez que aparecieran en Urantia con una forma material, dependerían igualmente de este don de Edentia para su mantenimiento vital.
\vs p073 6:2 En las capitales de los sistemas, los hijos materiales no precisan del árbol de la vida para su sostenimiento. Solo en la reconstitución planetaria de sus personas dependen de este suplemento para la inmortalidad física.
\vs p073 6:3 \pc El “árbol del conocimiento del bien y del mal” puede ser una figura retórica, una designación simbólica que abarque un gran número de experiencias humanas, pero el “árbol de la vida” no fue un mito; era real y estuvo presente en Urantia durante mucho tiempo. Cuando los altísimos de Edentia dieron su aprobación a la designación de Caligastia como príncipe planetario de Urantia y a la de los cien ciudadanos de Jerusem como su personal administrativo, enviaron al planeta, con los melquisedecs, un arbusto de Edentia, y esta planta creció en Urantia hasta convertirse en el árbol de la vida. Esta forma de vida no inteligente es originaria de las esferas sedes de las constelaciones, encontrándose igualmente en los mundos sedes de los universos locales y de los suprauniversos así como en las esferas de Havona, pero no en las capitales de los sistemas.
\vs p073 6:4 Esta planta extraordinaria acumulaba ciertas energías del espacio que constituían un antídoto contra los elementos causantes de la vejez en la existencia animal. El fruto del árbol de la vida era como una batería de almacenaje extra\hyp{}químico que, cuando se consumía, liberaba misteriosamente una fuerza del universo que prolongaba la vida. Este modo de sustentarse era totalmente ineficaz para el común de los seres evolutivos de Urantia; pero sí le fue expresamente de utilidad a los cien miembros materializados de la comitiva de Caligastia y a los cien andonitas modificados, que habían contribuido con su plasma vital al personal del príncipe y a quienes, en compensación, se les hizo poseedores de ese complemento vital, que les posibilitó el uso del fruto del árbol de la vida para prolongar indefinidamente una existencia que, de otra manera, hubiera sido mortal.
\vs p073 6:5 \pc Durante los días del gobierno del príncipe, el árbol crecía en tierra, en el patio central y circular del templo del Padre. Al estallar la rebelión, Van y sus colaboradores lo rebrotaron a partir de su núcleo central en su campamento provisional. Luego se llevaron a este arbusto de Edentia a su refugio en las altiplanicies, donde sirvió de utilidad a Van y a Amadón durante más de ciento cincuenta mil años.
\vs p073 6:6 Cuando Van y sus colaboradores preparaban el Jardín para Adán y Eva, trasplantaron el árbol de Edentia al Jardín de Edén, donde creció, una vez más, en un patio central y circular de otro templo del Padre. Periódicamente, Adán y Eva consumían su fruto para mantener su forma doble de vida física.
\vs p073 6:7 \pc Al tomar los planes del hijo material un rumbo equivocado, a Adán y a su familia no se les permitió llevarse el núcleo del árbol del Jardín. Cuando los noditas invadieron Edén, se les dijo que serían como “dioses si comían del fruto del árbol”. Se sorprendieron mucho al encontrarlo sin protección. Durante años, comieron este fruto sin restricción, pero no tuvo efecto en ellos; todos eran seres humanos materiales del mundo y carecían de la dotación que actuaba como complemento de dicho fruto. Se enfurecieron ante su impotencia por no poder sacar provecho del árbol de la vida y, con motivo de sus guerras internas, un incendio destruyó tanto el árbol como el templo; solo la muralla de piedra quedó en pie, hasta que posteriormente se sumergió el Jardín. Este fue el segundo templo del Padre que sufrió la destrucción.
\vs p073 6:8 Y ahora, todo ser vivo de Urantia debe seguir el curso natural de la vida y la muerte. Todos, Adán, Eva, sus hijos y los hijos de sus hijos, junto con sus colaboradores, perecieron con el transcurso del tiempo, quedando así sujetos al plan de ascensión del universo local en el que la resurrección en los mundos de las moradas sigue a la muerte física.
\usection{7. El DESTINO DE EDÉN}
\vs p073 7:1 Una vez que Adán desalojó el primer jardín, este fue ocupado distintamente por noditas, cutitas y suntitas. Más tarde se convirtió en la morada de los noditas del norte que se opusieron a cooperar con los adanitas. La península había estado dominada por estos noditas peor dotados durante casi cuatro mil años tras la salida de Adán del Jardín, cuando, en conexión con la actividad violenta de los volcanes circundantes y el sumergimiento del puente terrestre entre Sicilia y África, el fondo oriental del Mar Mediterráneo se hundió, arrastrando bajo sus aguas a toda la península edénica. En conjunción con este enorme sumergimiento, el litoral del Mediterráneo oriental se elevó considerablemente. Y este fue el final de la más bella creación natural que Urantia haya albergado jamás. El hundimiento no fue repentino, sino que se necesitaron varios cientos de años para que toda la península quedara completamente sumergida.
\vs p073 7:2 No podemos considerar de ningún modo que tal desaparición del Jardín se debiese al malogro de los planes divinos o como resultado de los errores de Adán y Eva. Creemos que el sumergimiento de Edén no fue sino un suceso natural, pero parece evidente que el hundimiento del Jardín se programó para que ocurriera casi en el momento en el que se habían acumulado las suficientes reservas de la raza violeta para emprender la labor de rehabilitar a los pueblos del mundo.
\vs p073 7:3 \pc Los melquisedecs aconsejaron a Adán que no iniciara el programa de mejora y mezcla de las razas hasta que su propia familia no alcanzase el medio millón. Nunca se pretendió que el Jardín fuese el hogar permanente de los adanitas. Tenían que convertirse en emisarios de una nueva vida para el mundo entero; tenían que movilizarse para darse desinteresadamente a las razas necesitadas de la tierra.
\vs p073 7:4 Las instrucciones impartidas a Adán por parte de los melquisedecs daban a entender que se tenían que establecer sedes raciales, continentales y locales que estarían a cargo de sus hijos e hijas inmediatos, mientras que él y Eva tendrían que dividir su tiempo entre estas distintas capitales mundiales, en calidad de asesores y coordinadores del ministerio mundial de mejora biológica, de avance intelectual y de rehabilitación moral.
\vsetoff
\vs p073 7:5 [Exposición de Solonia, la “voz” seráfica “del Jardín”.]
