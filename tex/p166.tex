\upaper{166}{Última visita al norte de Perea}
\author{Comisión de seres intermedios}
\vs p166 0:1 Desde el 11 hasta el 20 de febrero, Jesús y los doce viajaron por todas las ciudades y aldeas del norte de Perea donde trabajaban los compañeros de Abner y los miembros del colectivo de mujeres. Se encontraron con que estos mensajeros del evangelio estaban teniendo éxito en su misión, y Jesús dirigió reiteradamente la atención de sus apóstoles a diseminar el evangelio del reino sin recurrir a milagros y prodigios.
\vs p166 0:2 Esta misión en Perea, de tres meses de duración, se llevó por completo a cabo con éxito y con poca ayuda de los doce apóstoles, y el evangelio a partir de ese momento reflejó, no tanto la persona de Jesús, como sus \bibemph{enseñanzas}. Pero sus seguidores no siguieron sus instrucciones por mucho tiempo, ya que, poco después de la muerte y resurrección de Jesús, se apartaron de sus enseñanzas y comenzaron a construir la Iglesia primitiva, fundamentándola en nociones milagrosas y en los recuerdos glorificados de su persona divina\hyp{}humana.
\usection{1. LOS FARISEOS EN RAGABA}
\vs p166 1:1 El sábado, 18 de febrero, Jesús estaba en Ragaba. Allí vivía un fariseo acomodado llamado Natanael; y puesto que un buen número de sus compañeros fariseos seguía a Jesús y a los doce por todo el país, ese \bibemph{sabbat} por la mañana Natanael preparó un desayuno para todos ellos, unos veinte, y convidó a Jesús como invitado de honor.
\vs p166 1:2 Cuando Jesús llegó a este desayuno, la mayoría de los fariseos, con dos o tres intérpretes de la ley, estaban ya allí sentados a la mesa. De inmediato, el Maestro tomó asiento a la izquierda de Natanael sin dirigirse a la vasija de agua a lavarse las manos. Muchos de los fariseos, en especial aquellos que estaban a favor de las enseñanzas de Jesús, sabían que él solo se lavaba las manos por motivos de limpieza, que detestaba esos actos puramente ceremoniales; por ello, no se sorprendieron de que fuera directamente a la mesa sin lavarse dos veces las manos. Pero a Natanael le impactó que el Maestro incumpliera los estrictos requerimientos de las prácticas fariseas. Jesús tampoco se lavó las manos, como hacían los fariseos, después de cada plato ni al terminar de comer.
\vs p166 1:3 Tras bastantes murmuraciones entre Natanael y un fariseo poco amistoso situado a su derecha y tras mucho alzamiento de cejas y de muecas de desdén en los labios de los que estaban sentados frente al Maestro, Jesús finalmente dijo: “Había pensado que me habíais invitado a esta casa para partir el pan con vosotros y quizás para preguntarme sobre la proclamación del nuevo evangelio del reino de Dios; pero observo que me habéis traído aquí para presenciar una demostración de culto ceremonial a la manera de vuestra propia santurronería. Esa es la atención que habéis tenido ahora hacia mí; ¿con qué otra cosa me honraréis en esta ocasión como vuestro invitado?”.
\vs p166 1:4 Cuando el Maestro terminó de decir estas cosas, todos agacharon sus miradas y guardaron silencio. Y como nadie hablaba, Jesús continuó: “Muchos de vosotros, fariseos, estáis aquí conmigo como mis amigos, algunos incluso sois mis discípulos, pero la mayoría de los fariseos se obstinan en su negativa de ver la luz y reconocer la verdad, aun cuando la labor del evangelio se les manifieste con gran poder. ¡Limpiáis cuidadosamente lo de fuera del vaso y del plato, pero vuestras vasijas de comida espiritual están sucias y corroídas! Procuráis mostrar ante el pueblo una apariencia piadosa y santa, pero dentro de vosotros vuestras almas están llenas de hipocresía, codicia, extorsión y de todo tipo de maldad espiritual. Vuestros líderes se atreven incluso a conspirar y a planear el asesinato del Hijo del Hombre. ¿Es que no entendéis, necios, que el Dios del cielo ve tanto los motivos internos de vuestras almas como vuestras pretensiones externas y vuestras pías ocupaciones? No creáis que dar limosnas y pagar diezmos os limpia de vuestra falta de rectitud y os hace aparecer limpios en la presencia del Juez de todos los hombres. ¡Ay de vosotros, fariseos, que insistís en rechazar la luz de la vida! Sois escrupulosos en pagar el diezmo y ostentosos en dar limosnas, pero deliberadamente desdeñáis la visita de Dios y rechazáis la revelación de su amor. Os es necesario hacer estos deberes menores, sin dejar de cumplir aquello de mayor peso. ¡Ay de los que pasáis por alto la justicia, desecháis la misericordia y repudiáis la verdad! ¡Ay de todos los que desprecian la revelación del Padre mientras buscan las primeras sillas en la sinagoga y las salutaciones en las plazas!”.
\vs p166 1:5 \pc Cuando Jesús se levantó para marcharse, uno de los intérpretes de la Ley que estaba a la mesa, dirigiéndose a él, dijo: “Pero, Maestro, con algunas de tus afirmaciones también nos recriminas a nosotros. ¿Es que no hay nada bueno entre los escribas, los fariseos o los intérpretes de la Ley?”. Y Jesús, de pie, le contestó: “Vosotros, al igual que los fariseos, amáis los primeros asientos en las cenas y andar con largas ropas mientras cargáis a los hombres con pesos gravosos de llevar. Y cuando las almas de los hombres se tambalean bajo esas pesadas cargas, vosotros no levantáis ni uno solo de vuestros dedos. ¡Ay de vosotros que os regocijáis grandemente en edificar los sepulcros de los profetas a quienes mataron vuestros padres! Y de que sois consentidores de los hechos de vuestros padres se pone de manifiesto cuando planeáis ahora matar a quienes vienen en este día para hacer lo que hicieron los profetas en su día: proclamar la rectitud de Dios y revelar la misericordia del Padre celestial. Pero de todas las generaciones pasadas, a esta generación perversa e hipócrita se le demandará la sangre de los profetas y de los apóstoles. ¡Ay de vosotros, intérpretes de la Ley!, porque habéis quitado la llave de la ciencia a la gente común! Vosotros mismos os negáis a entrar en el camino de la verdad y, al mismo tiempo, queréis impedírselo a los que quieren entrar. Pero de ninguna manera podéis cerrar las puertas del reino de los cielos; pues estas están abiertas a todos los que tienen la fe para entrar; y estos portales de misericordia no se cerrarán por el prejuicio y la arrogancia de engañosos maestros y de falsos pastores, porque sois semejantes a sepulcros blanqueados, que por fuera se muestran hermosos, pero por dentro están llenos de huesos de muertos y de toda inmundicia espiritual”.
\vs p166 1:6 Y cuando Jesús acabó de hablar en la mesa de Natanael, salió de la casa sin haber comido. Y de los fariseos que oyeron estas palabras, algunos se convirtieron en creyentes de sus enseñanzas y entraron al reino, pero la gran mayoría prosiguió el camino de la oscuridad, cada vez más determinados a acecharlo y a procurar cazar algunas de sus palabras para someterlo a juicio ante el sanedrín de Jerusalén.
\vs p166 1:7 \pc Había solo tres cosas a las que los fariseos prestaban una especial atención:
\vs p166 1:8 \li{1.}A la práctica estricta del diezmo.
\vs p166 1:9 \li{2.}A la escrupulosa observancia de las leyes de purificación.
\vs p166 1:10 \li{3.}Al rechazo a relacionarse con todo los que no fueran fariseos.
\vs p166 1:11 \pc En ese momento, Jesús quería denunciar la esterilidad espiritual de las primeras dos prácticas, mientras que reservaba sus comentarios de reproche a los fariseos, por su rechazo de todo tipo de interacción social con los no fariseos, para otra siguiente ocasión en la que estuviera de nuevo comiendo con muchos de estos mismos hombres.
\usection{2. LOS DIEZ LEPROSOS}
\vs p166 2:1 Al día siguiente, Jesús fue con los doce a Amatus, cerca de la frontera de Samaria y, al aproximarse a la ciudad, se encontraron con un grupo de diez leprosos que vivía en las inmediaciones de aquel lugar. Nueve de ellos eran judíos y, uno, samaritano. Por lo general, estos judíos se habrían guardado de cualquier relación o contacto con este samaritano, pero la afección que tenían en común era más que suficiente para vencer cualquier prejuicio religioso. Habían oído hablar mucho de Jesús y de sus primeros milagros de curación y, puesto que los setenta tenían como norma anunciar el momento de la llegada prevista de Jesús cuando salía con los doce en estos viajes, los diez leprosos se habían enterado de que sobre aquella hora se esperaba que él apareciera por aquellos alrededores; y estaban, pues, aguardándolo en las afueras de la ciudad con la idea de atraer su atención y pedirle que los sanara. Cuando los leprosos vieron que Jesús se acercaba a ellos, no atreviéndose a aproximarse a él, se pararon de lejos y alzaron la voz, diciendo: “Maestro, ten misericordia de nosotros; límpianos de nuestra aflicción. Sánanos tal como has sanado a otros”.
\vs p166 2:2 Jesús acababa de explicar a los doce por qué los gentiles de Perea, junto con los judíos menos ortodoxos, se mostraban más dispuestos a creer en el evangelio predicado por los setenta que los judíos de Judea, más ortodoxos y más apegados a la tradición. Les había hecho notar el hecho de que los galileos, e incluso los samaritanos, habían asimismo recibido su mensaje con mayor prontitud, pero los doce apóstoles aún no estaban del todo dispuestos a abrigar buenos sentimientos hacia los samaritanos, durante tanto tiempo despreciados.
\vs p166 2:3 En este sentido, cuando Simón Zelotes vio al samaritano entre los leprosos trató de convencer al Maestro para que pasara de largo hacia la ciudad sin ni siquiera preocuparse por intercambiar saludos con ellos. Jesús dijo a Simón: “Pero, ¿qué pasa si el samaritano ama a Dios tanto como los judíos? ¿Es que debemos impartir juicio a nuestros semejantes? ¿Quién puede decirlo? Si sanamos a estos diez hombres, quizás el samaritano resulte ser más agradecido que los mismos judíos. ¿Das tus opiniones como ciertas, Simón?”. Y Simón contestó enseguida: “Si los limpias, pronto lo averiguarás”. Y Jesús respondió: “Así será pues, Simón, y pronto sabrás la verdad sobre la gratitud de los hombres y la amorosa misericordia de Dios”.
\vs p166 2:4 Jesús, acercándose a los leprosos, dijo: “Si queréis ser sanados, id de inmediato y mostraos a los sacerdotes como exige la ley de Moisés”. Y mientras iban, sanaron. Pero, cuando el samaritano vio que se había curado, se volvió y, yendo en busca de Jesús, comenzó a glorificar a Dios a gran voz. Y cuando halló al Maestro, se postró en tierra de rodillas a sus pies y dio gracias por haber quedado limpio. Los otros nueve, los judíos, supieron también de su curación y, aunque estaban agradecidos por haber quedado limpios, continuaron su camino para mostrarse a los sacerdotes.
\vs p166 2:5 Mientras el samaritano permanecía arrodillado a los pies de Jesús, el Maestro, mirando a los doce y particularmente a Simón Zelotes, dijo: “¿No son diez los que han quedado limpios? ¿Dónde están entonces los otros nueve, los judíos? No hubo quien volviera y diera gloria a Dios sino este extranjero”. Luego dijo al samaritano: “Levántate, vete; tu fe te ha sanado”.
\vs p166 2:6 Jesús miró de nuevo a sus apóstoles al alejarse el extranjero. Todos los apóstoles le devolvieron la mirada, salvo Simón Zelotes, cuyos ojos estaban cabizbajos. Los doce no dijeron palabra; tampoco Jesús. No era necesario que lo hiciera.
\vs p166 2:7 \pc Aunque estos diez hombres realmente creían que eran leprosos, solo cuatro estaban afectados por esta enfermedad. Los otros seis se curaron de una afección cutánea que se confundía con la lepra. Pero el samaritano sí la padecía en verdad.
\vs p166 2:8 \pc Jesús mandó a los doce que no dijeran nada sobre la limpieza de los leprosos, y cuando se adentraban en Amatus, comentó: “Veis como los hijos de la casa, incluso cuando se insubordinan contra la voluntad de su Padre, dan por hecho las bendiciones que les vengan. Piensan que no tiene la menor importancia dejar de agradecer al Padre por haberles concedido la curación, pero los extranjeros, cuando reciben dádivas del señor de la casa, se maravillan y se sienten movidos a dar las gracias en reconocimiento de las buenas cosas que se les ha concedido”. Y los apóstoles aún no dijeron nada en respuesta a las palabras del Maestro.
\usection{3. EL SERMÓN EN GERASA}
\vs p166 3:1 Cuando Jesús y los doce se encontraron con los mensajeros del reino en Gerasa, un fariseo de los que creían en él hizo esta pregunta: “Señor, ¿serán pocos o muchos los que se salvarán?”. Y Jesús, respondiéndole, dijo:
\vs p166 3:2 \pc “Se os ha enseñado que solo los hijos de Abraham serán salvos; que solo los gentiles de adopción pueden esperar la salvación. Algunos de vosotros habéis llegado a la conclusión de que, puesto que consta en las Escrituras que solo Caleb y Josué, entre todas las multitudes que salieron de Egipto, vivieron para entrar a la tierra prometida, relativamente pocos de los que buscan el reino de los cielos encontrarán entrada en él.
\vs p166 3:3 “Entre vosotros hay también otro dicho, y que dice mucha verdad: que el camino que lleva a la vida eterna es recto y angosto, que la puerta que conduce hasta allí es igualmente angosta, de modo que, de todos los que buscan la salvación, son pocos los que la hallan. También tenéis una enseñanza que dice que el camino que lleva a la perdición es espacioso, que la entrada a la misma es ancha, y que son muchos los que deciden entrar por ella. Y este proverbio no está falto de significado. Pero yo declaro que la salvación es primeramente una cuestión de opción personal. Aunque la puerta que lleva a la vida sea angosta, es lo suficientemente ancha como para admitir a todos los que sinceramente quieran entrar, porque yo soy esa puerta. Y el Hijo jamás le negará la entrada a ningún hijo del universo que, mediante la fe, desee encontrar al Padre a través del Hijo.
\vs p166 3:4 “Pero el peligro para todos los que quieren demorar su entrada al reino mientras persiguen los placeres de la inmadurez y encuentran gratificación en el egoísmo es este: habiéndose negado a vivenciar espiritualmente su entrada en el reino, querrán después entrar en él, en el momento en el que se revele la gloria de un camino mejor en la era por venir. Y, por ello, si quienes despreciaron el reino cuando yo vine como hombre, intentan hallar la entrada cuando esta se revele en su naturaleza divina, yo les diré a todos esos egoístas: no sé de dónde sois. Tuvisteis la oportunidad de prepararos para esta ciudadanía celestial, pero no quisisteis aceptar los ofrecimientos de misericordia que se os hizo; desdeñasteis las invitaciones a entrar mientras la puerta estaba abierta. Ahora, para vosotros que habéis rechazado la salvación, la puerta está cerrada. No está abierta para aquellos que quieren entrar en el reino y glorificarse a sí mismos egoístamente. La salvación no es para los que se niegan a pagar el precio, que es su dedicación incondicional a hacer la voluntad de mi Padre. Cuando en espíritu y alma le dais la espalda al reino del Padre, es inútil estar en mente y cuerpo ante esta puerta y llamar, diciendo: ‘Señor, ábrenos’; nosotros también queremos ser grandes en el reino’. Entonces yo os diré que no sois de mi redil. No os acogeré para que estéis entre quienes pelearon la buena batalla de la fe y ganaron la recompensa por su abnegado servicio en el reino sobre la tierra. Y cuando comencéis a decir: ‘¿Es que no comimos y bebimos contigo, y enseñaste en nuestras calles?’; y entonces yo diré de nuevo que espiritualmente sois unos desconocidos; que no éramos consiervos en el ministerio de misericordia del Padre en la tierra; que no os conozco; y entonces el Juez de toda la tierra os dirá: ‘apartaos de nosotros, todos vosotros que os habéis complacido obrando iniquidad’.
\vs p166 3:5 “Pero, no temáis, todo aquel que desee sinceramente encontrar la vida eterna entrando en el reino de Dios de cierto hallará esa perdurable salvación. Pero vosotros, que la rechazáis, veréis algún día a los profetas de la simiente de Abraham sentarse con los creyentes de las naciones gentiles en este reino glorificado para compartir el pan de la vida y reconfortarse con el agua de la vida. Y vendrán del norte y del sur y del oriente y del occidente quienes tomarán el reino en poder espiritual y con el asalto persistente de la fe viva. Y, he aquí, que muchos de los primeros serán los últimos, y los últimos muchas veces serán los primeros”.
\vs p166 3:6 Se trataba de hecho de una versión nueva y desconocida del antiguo y habitual proverbio sobre el camino recto y el camino angosto.
\vs p166 3:7 Lentamente los apóstoles y muchos de los discípulos fueron aprendiendo lo que Jesús quería decir cuando había reiteradamente proclamado: “A no ser que nazcáis de nuevo, que lo hagáis del espíritu, no podréis entrar en el reino de Dios”. No obstante, para todos aquellos honestos de corazón y sinceros de fe, es eternamente verdad que: “He aquí que yo estoy a la puerta del corazón de los hombres y llamo, y si alguno me abre, yo entraré y cenaré con él y lo alimentaré con el pan de la vida; seremos uno solo, en espíritu y propósito, y seremos, pues, para siempre, hermanos en el servicio, largo y fructífero, de buscar al Padre del Paraíso”. Y por tanto, los muchos o pocos que se salven depende por completo de los muchos o pocos que oigan mi invitación: “Yo soy la puerta, yo soy el camino nuevo y vivo, y aquel que quiera podrá entrar para emprender la interminable búsqueda de la verdad que lleva a la vida eterna”.
\vs p166 3:8 Los mismos apóstoles se mostraban incapaces de comprender del todo sus enseñanzas sobre la necesidad de hacer uso de la fuerza espiritual para abrirse camino ante cualquier oposición material y poder superar cualquier obstáculo terrenal, que se interpusiera en su entendimiento de los tan importantes valores espirituales de la nueva vida en el espíritu como hijos liberados de Dios.
\usection{4. ENSEÑANZA SOBRE LOS ACCIDENTES}
\vs p166 4:1 Aunque la mayor parte de los palestinos solo hacían dos comidas al día, Jesús y los apóstoles, cuando viajaban, tenían la costumbre de hacer una pausa al mediodía para descansar y tomar algo. Y, así, en una de estas paradas, yendo de camino a Filadelfia, Tomás le preguntó a Jesús: “Maestro, tras oír tus comentarios cuando viajábamos esta mañana, me gustaría preguntarte si los seres espirituales son responsables de los sucesos extraños y extraordinarios que ocurren en el mundo material y, además, si los ángeles y otros seres espirituales pueden prevenir los accidentes”.
\vs p166 4:2 \pc Respondiendo a la pregunta de Tomás, Jesús dijo: “¿Seguís haciéndome estas preguntas tras haber estado tanto tiempo con vosotros? ¿Es que no habéis observado que el Hijo del Hombre vive como uno de vosotros y se niega permanentemente a emplear las fuerzas del cielo para su sustento personal? ¿Es que no nos ganamos la vida tal como lo hacen todos los hombres de este mundo? ¿Es que veis acaso que el poder del mundo espiritual se manifieste en la vida material de este mundo a no ser en la revelación del Padre y en alguna curación de sus sufridos hijos?
\vs p166 4:3 “Durante demasiado tiempo, vuestros padres han creído que la prosperidad era un signo de aprobación divina; que la adversidad era la prueba del descontento de Dios. Yo os digo que tales creencias son supersticiones. ¿Es que no observáis que hay un número mucho mayor de pobres que reciben con alegría el evangelio y que entran de inmediato en el reino? Si las riquezas evidencian el favor divino, ¿por qué se niegan los ricos tantas veces a creer en esta buena nueva del cielo?
\vs p166 4:4 “El Padre hace caer su lluvia sobre justos e injustos; el sol, de la misma manera, brilla sobre los rectos y los impíos. Sabéis de aquellos galileos cuya sangre Pilato mezcló con los sacrificios de ellos, pero yo os digo que esos galileos no eran de ningún modo más pecadores que sus semejantes simplemente porque padecieron tales cosas. También habéis oído hablar de aquellos dieciocho sobre los cuales cayó la torre en Siloé, matándolos a todos. No penséis que estos hombres que murieron eran más culpables que todos sus hermanos de Jerusalén. Estos fueron sencillamente víctimas inocentes de uno de los accidentes del tiempo.
\vs p166 4:5 “Existen tres grupos de sucesos que pueden ocurrir en vuestras vidas:
\vs p166 4:6 “1. Podéis ser partícipes de esos acontecimientos normales que son parte de vuestra vida y de la de semejantes vuestros que viven en la tierra.
\vs p166 4:7 “2. Puede suceder que seáis víctimas de alguno de los accidentes de la naturaleza, de algunos de los infortunios que acaecen a los hombres, aunque conocéis perfectamente bien que estos hechos no son, de modo alguno, preestablecidos ni las ocasionan las fuerzas espirituales del mundo.
\vs p166 4:8 “3. Podéis recoger la cosecha de vuestros propios afanes por cumplir las leyes naturales que gobiernan el mundo.
\vs p166 4:9 \pc “Había un hombre que plantó una higuera en su propiedad, y después de ir muchas veces a buscar fruto en ella sin hallarlo, llamó a los viñadores a su presencia y les dijo: ‘Ya hace tres temporadas que vengo por fruto en esta higuera y no lo hallo. ¡Cortad este árbol estéril; ¿para qué inutilizar la tierra?’ Pero el jardinero principal respondió a su señor ‘Déjala todavía un año más hasta que yo cave alrededor de ella y la abone, y si el año que viene no da fruto, la cortaremos entonces’. Y cuando siguieron las leyes naturales de la fertilidad, puesto que el árbol estaba vivo y en buenas condiciones, se vieron recompensados con una abundante cosecha.
\vs p166 4:10 “En los asuntos de la enfermedad y de la salud, debéis saber que estos dos estados fisiológicos son frutos de causas materiales; la salud no es la sonrisa del cielo ni es la aflicción la desaprobación de Dios.
\vs p166 4:11 “Todos los hijos humanos del Padre tienen las mismas posibilidades de recibir las bendiciones materiales; por lo tanto, él otorga las cosas físicas a los hijos de los hombres sin hacer discriminación. Por lo que respecta a los dones espirituales, el Padre está limitado por la capacidad del hombre para recibir estas dádivas espirituales. Aunque el Padre no hace acepción de personas, al conceder dichos dones espirituales se ve condicionado por la fe del hombre y por su disposición a seguir siempre la voluntad del Padre”.
\vs p166 4:12 \pc Conforme viajaban a Filadelfia, Jesús continuó enseñándoles y respondiendo a sus preguntas relacionadas con los accidentes, la enfermedad y los milagros, pero no fueron capaces de comprender del todo sus palabras. Una hora de enseñanza no puede cambiar por completo las creencias aprendidas durante toda una vida, por lo que Jesús consideró necesario repetir su mensaje, decirles una y otra vez lo que él deseaba que ellos entendieran; e, incluso así, no lograron captar el significado de su misión en la tierra hasta después de su muerte y resurrección.
\usection{5. LA CONGREGACIÓN DE FILADELFIA}
\vs p166 5:1 Jesús y los doce fueron de camino a visitar a Abner y a sus compañeros, que predicaban y enseñaban en Filadelfia. Entre todas las ciudades de Perea, Filadelfia era la que albergaba al grupo más numeroso de judíos y gentiles, ricos y pobres, eruditos e iletrados acogidos a las enseñanzas de los setenta, por lo que habían entrado en el reino de los cielos. La sinagoga de Filadelfia nunca se había sometido a la supervisión del sanedrín de Jerusalén y, por lo tanto, jamás había estado cerrada a las enseñanzas de Jesús y sus acompañantes. En aquel mismo momento, Abner impartía sus enseñanzas tres veces al día en la sinagoga de Filadelfia.
\vs p166 5:2 Esta misma sinagoga se convirtió más tarde en una Iglesia cristiana y fue la sede misionera desde la que se promulgaría el evangelio a las regiones del este. Fue por mucho tiempo el bastión de las enseñanzas del Maestro y, durante siglos, se mantuvo en esta región como el más prominente centro de conocimiento cristiano.
\vs p166 5:3 Los judíos de Jerusalén habían tenido continuos problemas con los judíos de Filadelfia. Y, tras la muerte y resurrección de Jesús, la Iglesia de Jerusalén, de la que estaba al frente Santiago, el hermano del Señor, comenzó a tener serias dificultades con la congregación de creyentes de Filadelfia. Abner se convirtió en cabeza de la Iglesia de Filadelfia, continuando como tal hasta su muerte. Y este distanciamiento de Jerusalén explica por qué no se menciona nada de Abner y de su labor en los relatos evangélicos del Nuevo Testamento. Esta enemistad entre Jerusalén y Filadelfia se prolongó a lo largo de toda las vidas de Santiago y de Abner y prosiguió durante cierto tiempo después de la destrucción de Jerusalén. Filadelfia fue realmente la sede de la Iglesia primitiva en el sur y en el este, al igual que Antioquía lo fue en el norte y en el oeste.
\vs p166 5:4 \pc Parece que es de lamentar que Abner disintiera con todos los líderes de la Iglesia cristiana primitiva. Perdió la confianza de Pedro y de Santiago (el hermano de Jesús) debido a cuestiones administrativas y a la jurisdicción de la Iglesia de Jerusalén; terminó su relación con Pablo por diferencias filosóficas y teológicas. En su filosofía, Abner era más babilónico que helénico, y se resistió obstinadamente a todos los intentos de Pablo por rehacer las enseñanzas de Jesús para presentarla lo menos ofensiva posible, primero a los judíos y, luego, a los greco\hyp{}romanos creyentes de los misterios.
\vs p166 5:5 Por ello, Abner se vio obligado a llevar una vida de aislamiento. Era cabeza de una Iglesia desacreditada en Jerusalén. Se había atrevido a desafiar a Santiago, el hermano del Señor, que posteriormente obtuvo el respaldo de Pedro. Esta conducta lo separó de hecho de todos sus anteriores compañeros. Además, osó enfrentarse a Pablo. Aunque se mostró totalmente favorable a la misión de Pablo entre los gentiles y, aunque lo apoyó en sus disputas con la Iglesia de Jerusalén, se opuso drásticamente a la versión de las enseñanzas de Jesús que Pablo eligió predicar. En sus últimos años, Abner denunció a Pablo como el “taimado corruptor de las enseñanzas de la vida de Jesús de Nazaret, el Hijo del Dios vivo”.
\vs p166 5:6 Durante los últimos años de Abner y algún tiempo después, los creyentes de Filadelfia se aferraron más estrictamente a la religión de Jesús, tal como él la había vivido y enseñado, que cualquier otro grupo sobre la tierra.
\vs p166 5:7 Abner vivió hasta los 89 años de edad; murió en Filadelfia el 21 de noviembre del año 74 d. C. Y, hasta el final, fue un creyente fiel y un maestro del evangelio del reino celestial.
