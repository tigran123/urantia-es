\upaper{35}{Los Hijos de Dios de los universos locales}
\author{Jefe de los arcángeles}
\vs p035 0:1 Los Hijos de Dios anteriormente mencionados tuvieron su origen en el Paraíso. Son los vástagos de los Gobernantes Divinos de los ámbitos universales. Los hijos creadores pertenecen al primer orden de filiación del Paraíso; en Nebadón solamente hay uno de ellos: Miguel, el padre y soberano del universo. Los avonales o hijos magistrados pertenecen al segundo orden. Nebadón tiene su contingente íntegro: un total de 1062. Y estos “cristos menores” son tan eficientes y todopoderosos en sus respectivos ministerios de gracia planetarios como lo fue el hijo creador y mayor en Urantia. Por ser de origen trinitario, no hay registro en el universo local del tercer orden de Hijos de Dios, pero calculo que hay en Nebadón entre quince y veinte mil hijos preceptores de la Trinidad, excluyendo los 9642 asistentes trinitizados por criaturas que sí constan en registro. Estos dainales del Paraíso no son ni magistrados ni administradores; son maestros consumados.
\vs p035 0:2 Los Hijos de Dios que se van a examinar tienen su origen en el universo local; descienden de un hijo creador del Paraíso en distintos modos de conjunción con el espíritu materno del universo, su complementario. En estas narraciones se mencionan los siguientes órdenes de filiación:
\vs p035 0:3 \li{1.}Los hijos melquisedecs.
\vs p035 0:4 \li{2.}Los hijos vorondadecs.
\vs p035 0:5 \li{3.}Los hijos lanonandecs.
\vs p035 0:6 \li{4.}Los hijos portadores de vida.
\vs p035 0:7 \pc La Deidad trina del Paraíso crea tres órdenes de filiación: los migueles, los avonales y los dainales. En el universo local, la Deidad doble, el Hijo y el Espíritu, coopera también en la creación de tres elevados órdenes de Hijos de Dios: los melquisedecs, los vorondadecs y los lanonandecs; y, habiendo logrado esta triple expresión, colaboran con el siguiente nivel de Dios Séptuplo en la creación de la versátil orden de los portadores de vida. Estos seres están clasificados con los Hijos de Dios que descienden, pero en el universo constituyen una forma de vida única y primigenia. Todo el siguiente escrito está dedicado a su estudio.
\usection{1. EL PADRE MELQUISEDEC}
\vs p035 1:1 Una vez que han tenido origen los ayudantes personales, tales como la brillante estrella de la mañana y otros seres personales con cometidos gobernativos, según el propósito divino y los planes creativos de un universo dado tiene lugar una nueva forma de unión creativa entre el hijo creador y el espíritu creativo del universo local, o hija del Espíritu Infinito. De esta alianza nace un ser personal, el melquisedec primigenio o padre Melquisedec, ese ser único que con posterioridad colaborará con el hijo creador y el espíritu creativo en dar a su vez existencia al grupo completo que lleva ese nombre.
\vs p035 1:2 En el universo de Nebadón, el padre Melquisedec actúa como el primer mandatario adjunto de la brillante estrella de la mañana. Gabriel se ocupa más de las políticas del universo y Melquisedec lo hace de los procedimientos prácticos. Gabriel preside los tribunales y consejos que se constituyen regularmente en Nebadón, mientras que Melquisedec preside las comisiones especiales, extraordinarias y de emergencia al igual que los órganos consultivos. Gabriel y el padre Melquisedec nunca se alejan al mismo tiempo de Lugar de Salvación; en ausencia de Gabriel, el padre Melquisedec ejerce la función de mandatario en jefe de Nebadón.
\vs p035 1:3 Todos los melquisedecs de nuestro universo se crearon durante el transcurso de un milenio de tiempo regular por el hijo creador y el espíritu creativo en conjunción con el padre Melquisedec. Siendo un orden de filiación en el que uno de sus propios integrantes obró como cocreador, los melquisedecs, en su constitución, se originaron en parte a sí mismos y, por tanto, pueden aspirar a tener una excelsa forma de autogobierno. Eligen periódicamente a su propio jefe de gobernación por un término de siete años de este tiempo estándar y actúan por lo demás como un orden de seres que se regula a sí mismo, aunque el melquisedec primigenio ejerce ciertas prerrogativas consustanciales a su calidad de coprogenitor. Ocasionalmente, este padre Melquisedec designa a ciertos miembros de su orden para que desempeñen la labor de portadores de vida especiales en los mundos midsonitas, un tipo de planeta habitado no revelado hasta ahora en Urantia.
\vs p035 1:4 Los melquisedecs no realizan una gran actividad fuera del universo local, excepto cuando se los convoca como testigos en asuntos pendientes ante los tribunales del suprauniverso y cuando se les designa como embajadores especiales, como a veces sucede, para representar a un universo ante otro en el mismo suprauniverso. El melquisedec primigenio o primogénito de cada uno de los universos puede siempre viajar libremente a los universos cercanos o al Paraíso en misiones que guardan relación con los intereses y obligaciones de su orden.
\usection{2. LOS HIJOS MELQUISEDECS}
\vs p035 2:1 Los melquisedecs constituyen el primer orden de hijos divinos que se acercan lo suficientemente a la vida de las criaturas de menor rango como para poder realizar, de forma directa, su ministerio de elevar a los mortales, de servir a las razas evolutivas sin necesidad de encarnarse. Estos hijos se hallan por naturaleza en el punto medio de la gloriosa escala descendente de seres personales, encontrándose por su origen aproximadamente a medio camino entre la Divinidad más elevada y las criaturas de voluntad más modestas. Son, por tanto, intermediarios naturales entre los niveles más elevados y divinos de la existencia viva y de las formas de vida de los mundos evolutivos, de orden inferior e incluso materiales. A los órdenes seráficos, o ángeles, les deleita trabajar con los melquisedecs; de hecho, todas las formas de vida inteligente hallan en ellos amigos comprensivos, maestros receptivos y consejeros sensatos.
\vs p035 2:2 Los melquisedecs se rigen de manera autónoma. En este singular grupo, encontramos la primera iniciativa de autodeterminación por parte de seres del universo local y contemplamos la forma más notable de un verdadero autogobierno. Estos hijos establecen sus propios mecanismos para la gobernación de su grupo y de su planeta de residencia, así como para las seis esferas vinculadas y sus mundos integrantes. Y debe quedar constancia de que estos hijos melquisedecs jamás han abusado de sus prerrogativas; ni una sola vez han traicionado en todo el suprauniverso de Orvontón la confianza depositada en ellos. Representan la esperanza de cualquier grupo del universo que aspire al autogobierno; son el modelo al igual que los maestros del autogobierno para todas las esferas de Nebadón. Todos los órdenes de seres inteligentes, los superiores de arriba y los de menor rango de abajo, sinceramente elogian el gobierno de los melquisedecs.
\vs p035 2:3 \pc El orden de filiación de los melquisedecs ocupa la posición, y asume la responsabilidad, del hijo mayor en una familia numerosa. Realizan un trabajo mayormente sin cambio y algo rutinario, si bien, buena parte del mismo es voluntario y enteramente autoimpuesto. La mayoría de las asambleas especiales que se reúnen ocasionalmente en Lugar de Salvación se convocan a petición de los melquisedecs. Por iniciativa propia, estos hijos patrullan su universo nativo; mantienen una estructura autónoma dedicada a la recogida de información relativa al universo y presentan informes periódicos al hijo creador, con independencia de cualquier otra información que llegue a la sede del universo, a través de las instancias intermedias regulares encargadas de la gestión rutinaria de este. Por naturaleza, son observadores imparciales; cuentan con la total confianza de todas las clases de seres inteligentes.
\vs p035 2:4 Los melquisedecs actúan como tribunales de apelación itinerantes y consultivos de los mundos; estos hijos del universo acuden en pequeños grupos a estos mundos con el fin de prestar servicio como comisiones asesoras para tomar declaración, recibir sugerencias y ejercer de letrados, ayudando así a solventar las dificultades importantes y a resolver las graves diferencias que surgen ocasionalmente en los asuntos de los dominios evolutivos.
\vs p035 2:5 Estos hijos mayores del universo son los principales ayudantes de la brillante estrella de la mañana en llevar a cabo los mandatos del hijo creador. Cuando un melquisedec va a un mundo remoto en nombre de Gabriel, se le puede designar, a efecto de esa misión particular, como sustituto de aquel que lo envía; en cuyo caso, aparecerá en dicho planeta con toda la autoridad de la brillante estrella de la mañana. Esto es especialmente cierto en aquellas esferas donde un hijo de mayor rango aún no ha hecho su aparición semejando a las criaturas que las habitan.
\vs p035 2:6 Cuando un hijo creador emprende su andadura de gracia en un mundo evolutivo, va solo; pero cuando uno de sus hermanos del Paraíso, un hijo avonal, inicia su propio ministerio de gracia, lo hace acompañado y apoyado por melquisedecs, doce en número, que tan eficazmente contribuyen al éxito de tal misión. Los melquisedecs apoyan igualmente a los avonales del Paraíso en sus misiones como magistrados a los mundos habitados y, durante este cometido, son visibles a los ojos de los mortales si el hijo avonal se manifiesta también de esta forma.
\vs p035 2:7 No hay faceta alguna de las necesidades espirituales planetarias a la que no dejen de atender. Son los maestros que tan frecuentemente ganan mundos completos de avanzada vida para el reconocimiento último y total del hijo creador y de su Padre del Paraíso.
\vs p035 2:8 \pc Los melquisedecs son casi perfectos en sabiduría, pero no son infalibles en su juicio. Cuando se han encontrado aislados y solos en misiones planetarias, a veces han incurrido en errores sobre cuestiones menores; esto es, han optado por hacer ciertas cosas que no contaban con la aprobación posterior de sus supervisores. Este error de apreciación incapacita temporalmente a un melquisedec hasta que acude a Lugar de Salvación y, en audiencia con el hijo creador, recibe esa instrucción que lo exonera de la disonancia que provocó la desavenencia con sus compañeros; y, entonces, tras un cese de carácter correctivo, se reincorporará a su servicio al tercer día. En Nebadón, raras veces se han producido estas pequeñas disfunciones en la actividad de los melquisedecs.
\vs p035 2:9 Estos Hijos de Dios no son un orden de seres en aumento; su número, aunque varía en función del universo local, es fijo. El número de melquisedecs registrados en su planeta sede en Nebadón supera los diez millones.
\usection{3. LOS MUNDOS MELQUISEDECS}
\vs p035 3:1 Los melquisedecs tienen su propio mundo cerca de Lugar de Salvación, la sede del universo local. Esta esfera, llamada Melquisedec, es el mundo piloto de la vía circulatoria de Lugar de Salvación, que está compuesta de setenta esferas primarias, cada una de las cuales está circundada a su vez por seis esferas dependientes dedicadas a alguna actividad especial. A estas maravillosas esferas ---70 primarias y 420 dependientes--- se las denomina con frecuencia Universidad Melquisedec. Los mortales ascendentes de todas las constelaciones de Nebadón pasan por un proceso de formación en todos estos 490 mundos con el fin de adquirir la condición de residentes en Lugar de Salvación. Sin embargo, tal instrucción no es más que un aspecto de la actividad plural que tiene lugar en el conjunto de esferas arquitectónicas de esta sede del universo local.
\vs p035 3:2 Las 490 esferas que circulan alrededor de Lugar de Salvación se dividen en diez grupos, cada uno de los cuales contiene siete esferas primarias y cuarenta y dos dependientes. Cada uno de estos grupos está bajo la supervisión general de uno de los órdenes principales de vida existente en el universo. El primer grupo, que comprende el mundo piloto y las seis esferas primarias siguientes en la circundante procesión planetaria, se encuentra bajo la supervisión de los melquisedecs. Estos mundos melquisedecs son:
\vs p035 3:3 \li{1.}El mundo piloto: el mundo de residencia de los hijos melquisedecs.
\vs p035 3:4 \li{2.}El mundo de las escuelas de la vida física y los laboratorios de energías vivas.
\vs p035 3:5 \li{3.}El mundo de la vida morontial.
\vs p035 3:6 \li{4.}La esfera de la vida espiritual inicial.
\vs p035 3:7 \li{5.}El mundo de la vida espiritual intermedia.
\vs p035 3:8 \li{6.}La esfera de la vida espiritual en avance.
\vs p035 3:9 \li{7.}El ámbito de la autorrealización correlacionada y suprema.
\vs p035 3:10 \pc Los seis mundos dependientes de cada una de estas esferas de los melquisedecs están dedicados a actividades relacionadas con el trabajo de la esfera primaria a la que están vinculados.
\vs p035 3:11 \pc El mundo piloto, la esfera \bibemph{Melquisedec,} es el espacio común de encuentro para todos los seres que participan en la instrucción y espiritualización de los ascendentes mortales del tiempo y del espacio. Para dichos ascendentes este mundo es probablemente el lugar más interesante de todo Nebadón. Todos los mortales que han completado su formación en las constelaciones están destinados a arribar en el planeta Melquisedec, donde se les inicia en el régimen de disciplinas y de perfeccionamiento espiritual del sistema educativo de Lugar de Salvación. No olvidaréis jamás vuestras reacciones aquel primer día de vida ante este singular mundo, ni siquiera cuando hayáis alcanzado vuestro destino en el Paraíso.
\vs p035 3:12 Los mortales ascendentes residen en el mundo Melquisedec mientras realizan su formación en los seis planetas de educación especializada que circundan dicho mundo. Esta es la pauta que se sigue durante toda su estancia en los setenta mundos culturales, las esferas primarias de la vía circulatoria de Lugar de Salvación.
\vs p035 3:13 \pc La actividad que ocupa el tiempo de los numerosos seres que moran en los seis mundos dependientes de la esfera Melquisedec es plural y variada; si bien, en cuanto a los mortales ascendentes, estos satélites están destinados a las siguientes facetas de estudio:
\vs p035 3:14 \li{1.}La esfera número uno se dedica a la reflexión sobre la vida planetaria inicial de tales ascendentes, y se lleva a cabo en clases integradas por aquellos que proceden de un determinado mundo de origen mortal. Los que provienen de Urantia realizan juntos este examen de sus experiencias.
\vs p035 3:15 \li{2.}La tarea especial de la esfera número dos consiste igualmente en una reflexión, aunque, en este caso, de las experiencias vividas al paso por los mundos de las moradas que rodean el satélite primario de la sede del sistema local.
\vs p035 3:16 \li{3.}La reflexión que se realiza en la esfera tercera se centra en la estancia de los ascendentes en la capital del sistema local y abarca la actividad realizada en los restantes mundos arquitectónicos del conjunto que forma la sede del sistema.
\vs p035 3:17 \li{4.}La cuarta esfera se dedica a la reflexión de las experiencias en los setenta mundos dependientes de la constelación y de las esferas vinculadas a estos mundos.
\vs p035 3:18 \li{5.}En la quinta esfera, se lleva a cabo la reflexión sobre la estancia del ascendente en el mundo sede de la constelación.
\vs p035 3:19 \li{6.}En la esfera número seis, el tiempo se dedica a intentar correlacionar estas cinco épocas y lograr así una coordinación de las experiencias con el fin de prepararse para el ingreso en las escuelas primarias Melquisedec, que imparten formación en los asuntos del universo.
\vs p035 3:20 \pc Las escuelas de administración del universo y de sabiduría espiritual están situadas en el mundo de residencia de los melquisedecs, donde también se encuentran las escuelas dedicadas a una sola línea de investigación, tales como energía, materia, organización, comunicación, archivo, ética y existencia comparada de las criaturas.
\vs p035 3:21 En la Facultad de Dotación Espiritual de los melquisedecs, todos los órdenes de Hijos de Dios ---incluso aquellos órdenes del Paraíso--- cooperan con los melquisedecs y los maestros seráficos en la formación de las multitudes que acuden como evangelistas de destino a los mundos remotos del universo para proclamar la libertad espiritual y la filiación divina. Esta división especial de la Universidad Melquisedec es una institución exclusiva del universo; no se admiten visitantes estudiantiles de otras regiones espaciales.
\vs p035 3:22 Los melquisedecs imparten, en su mundo de residencia, el curso superior de formación en administración del universo. El primigenio padre Melquisedec preside esta Facultad de Alta Ética. A estos centros de formación se envían a los estudiantes de intercambio desde los distintos universos. Aunque el joven universo de Nebadón se halla en un bajo nivel en la escala de los universos en cuanto a logros espirituales y a un elevado desarrollo ético, nuestros problemas de gobernación han sido de tal grado que el universo completo se ha convertido en un enorme campo de prácticas para otras creaciones cercanas y las facultades de los melquisedecs están atestadas de visitantes estudiantiles y observadores de otros mundos. Además del inmenso grupo de estudiantes locales matriculados, siempre hay más de cien mil estudiantes extranjeros que asisten a estas clases, porque el orden de los melquisedecs de Nebadón goza de renombre en todo Esplandón.
\usection{4. EL TRABAJO ESPECIAL DE LOS MELQUISEDECS}
\vs p035 4:1 Dentro de la actividad de los melquisedecs, hay un área sumamente especializada que guarda relación con la supervisión de la trayectoria morontial progresiva de los mortales ascendentes. Una gran parte de esta formación está a cargo de los sabios y pacientes servidores seráficos, asistidos por mortales que han alcanzado un nivel de logro relativamente superior en la escala del universo; si bien, toda esta labor educativa está bajo la supervisión general de los melquisedecs en colaboración con los hijos preceptores de la Trinidad.
\vs p035 4:2 \pc Aunque los órdenes de los melquisedecs se dedican mayormente al inmenso sistema educativo y al régimen de formación experiencial del universo local, también prestan su servicio en misiones excepcionales y en circunstancias extraordinarias. En un universo en evolución, que acabará por sumar aproximadamente diez millones de mundos habitados, muchas cosas fuera de lo común están llamadas a suceder, y es en esas urgencias cuando actúan los melquisedecs. En Edentia, la sede de vuestra constelación, se les conoce como hijos de emergencia. Están siempre preparados para servir en cualquier situación de dificultad ---física, intelectual o espiritual---, ya sea en un planeta, en un sistema, en una constelación o en el universo. Cuando y dondequiera que se precise ayuda de carácter especial, allí encontraréis a uno o más hijos melquisedecs.
\vs p035 4:3 Cuando algún aspecto del plan del hijo creador corre peligro de malograrse, un melquisedec acudirá de inmediato a prestar ayuda. Pero con frecuencia no se requieren sus servicios como se requirieron en el caso de la impía rebelión ocurrida en Satania.
\vs p035 4:4 En cualquier mundo habitado por criaturas de voluntad, cuando surge una situación de emergencia, sea cual fuese su naturaleza, los melquisedecs son los primeros en operar. A veces lo hacen como custodios temporales de planetas insubordinados, sirviendo como síndicos de gobiernos rebeldes. En una crisis planetaria, estos hijos melquisedecs prestan sus servicios realizando muchos cometidos singulares. A este tipo de hijos les resulta fácil hacerse visible a los seres mortales y, a veces, alguno de ellos ha llegado incluso a encarnarse con la semejanza de un hombre mortal. En Nebadón, un melquisedec ha servido siete veces en un mundo evolutivo manifestándose de esta manera y, en numerosas ocasiones, estos hijos han aparecido semejando otros órdenes de criaturas del universo. Son, en efecto, los versátiles y voluntarios servidores en casos de urgencia para todos los órdenes de inteligencia del universo y para todos los mundos y sistemas de mundos.
\vs p035 4:5 \pc Al melquisedec que vivió en Urantia en los tiempos de Abraham se le conoció en la región como el Príncipe de Salem porque presidía una pequeña comunidad de buscadores de la verdad que residía en un lugar llamado Salem. Se ofreció como voluntario para encarnarse como hombre mortal, y lo hizo con la aprobación de los síndicos melquisedecs del planeta, que temían que se extinguiese la luz de la vida durante ese período de creciente oscuridad espiritual. Y este melquisedec realmente impulsó la verdad de sus días, transmitiéndola con éxito a Abraham y a sus colaboradores.
\usection{5. LOS HIJOS VORONDADECS}
\vs p035 5:1 Tras la creación de los auxiliares personales y del primer grupo de los versátiles melquisedecs, el hijo creador y el espíritu creativo del universo local planificaron y dieron existencia al segundo gran orden de filiación del universo: los vorondadecs. A este variado orden de hijos se les conoce de manera más general como los padres de las constelaciones, porque en todos los universos locales hay regularmente uno de ellos a la cabeza del gobierno de cada una de las constelaciones.
\vs p035 5:2 \pc El número de vorondadecs varía en cada universo local. Los que están registrados en Nebadón se elevan a un millón y, al igual que sus coiguales, los melquisedecs, no poseen el poder de reproducirse. No existe método alguno conocido por el que puedan aumentar su número.
\vs p035 5:3 \pc En muchos aspectos, estos hijos constituyen un órgano autónomo; de forma individual y grupal, e incluso como totalidad, estos hijos gozan, en buena parte, de autodeterminación, de la misma manera que los melquisedecs; si bien, los vorondadecs no actúan en un campo de actividad tan amplio. No igualan a sus hermanos melquisedecs respecto a su brillante versatilidad, pero son incluso más dignos de confianza y eficientes como gobernantes y administradores previsores. Tampoco son exactamente iguales a sus subordinados, los lanonandecs, soberanos de los sistemas, en cuanto a sus destrezas como administradores, pero superan a cualquier otro orden de filiación del universo respecto a la estabilidad de sus propósitos y a la divinidad de sus juicios.
\vs p035 5:4 Aunque las decisiones y reglamentaciones de este orden de hijos están siempre de acuerdo con el espíritu de la filiación divina y en armonía con las políticas del hijo creador, se les ha llegado a citar a causa de sus errores ante el hijo creador y, en relación a detalles técnicos, sus decisiones se han revocado a veces por apelación ante los tribunales superiores del universo. No obstante, los vorondadecs rara vez caen en el error y jamás se han rebelado; nunca en toda la historia de Nebadón se ha hallado a uno de ellos en desacato del gobierno del universo.
\vs p035 5:5 Los vorondadecs realizan un amplio y variado servicio en los universos locales. Sirven como embajadores ante otros universos al igual que como cónsules representando a las constelaciones dentro de su universo nativo. De todos los órdenes de filiación del universo local, es a ellos a quienes con más frecuencia se les confían y delegan plenos poderes soberanos para que se ejerzan en las situaciones críticas del universo.
\vs p035 5:6 En esos mundos segregados en oscuridad espiritual, en esas esferas que, por rebelión y transgresión, han sufrido aislamiento planetario, un observador vorondadec está generalmente presente hasta el restablecimiento de su condición normal. En ciertos casos de emergencias, este observador altísimo podría ejercer autoridad absoluta y discrecional sobre todos los seres celestiales destinados en ese planeta. En Lugar de Salvación, hay constancia de que los vorondadecs han hecho a veces uso de dicha autoridad en tales planetas como regentes altísimos. Esto también ha ocurrido incluso en mundos habitados no tocados por rebeliones.
\vs p035 5:7 Con frecuencia, un grupo de doce o más hijos vorondadecs se constituye en tribunal superior de revisión judicial y apelación con respecto a casos especiales que afecten el estado de un planeta o de un sistema. No obstante, su trabajo está normalmente relacionado con las funciones legislativas autóctonas de los gobiernos de las constelaciones. Como resultado de todos estos actos de servicio, los vorondadecs se han convertido en los historiadores de los universos locales; están familiarizados personalmente con todas las disensiones políticas y agitaciones sociales de los mundos habitados.
\usection{6. LOS PADRES DE LAS CONSTELACIONES}
\vs p035 6:1 Al menos tres vorondadecs están asignados al gobierno de cada una de las cien constelaciones de un universo local. El hijo creador los elige y Gabriel los nombra como los \bibemph{altísimos} de las constelaciones para servir durante un decamilenio ---10\,000 años de tiempo estándar, unos 50\,000 años de tiempo de Urantia---. El altísimo reinante, el padre de la constelación, tiene dos colaboradores, uno de mayor rango y otro de menor rango. En cada cambio de gobierno, el de mayor rango pasa a ser el jefe de gobierno y el de menor rango asume las funciones del de mayor rango; al mismo tiempo, los vorondadecs sin destino, con residencia en los mundos de Lugar de Salvación, proponen a uno de sus propios miembros como candidato para asumir las responsabilidades del colaborador de menor rango. Así pues, según la política actual, cada uno de los gobernantes altísimos sirve en la sede de una constelación durante un periodo de tres decamilenios, unos 150\,000 años de Urantia.
\vs p035 6:2 Los cien padres de las constelaciones, los que en verdad presiden los gobiernos de las constelaciones, componen el supremo órgano de consulta del hijo creador. Este consejo celebra sesiones frecuentes en la sede del universo y no tiene límites en cuanto al ámbito y alcance de sus deliberaciones; no obstante, se ocupa principalmente del bienestar de las constelaciones y de la unificación de la administración de todo el universo local.
\vs p035 6:3 Cuando un padre de la constelación está atendiendo sus obligaciones en la sede del universo, como hace con frecuencia, el colaborador de mayor rango pasa a ser el director en funciones de los asuntos de la constelación. La actividad normal de este colaborador es la supervisión de los asuntos espirituales, mientras que el colaborador de menor rango se ocupa personalmente del bienestar físico de la constelación. Sin embargo, no se lleva a cabo ninguna política de importancia en la constelación a menos que los tres altísimos estén de acuerdo en todos los detalles de su ejecución.
\vs p035 6:4 Todo el mecanismo de la información espiritual y de los canales de comunicación están a disposición de los altísimos de la constelación. Están en perfecto contacto con sus superiores de Lugar de Salvación y con sus subordinados directos, los soberanos de los sistemas locales. Con frecuencia, se reúnen en consejo con estos soberanos para deliberar sobre el estado de la constelación.
\vs p035 6:5 Los altísimos se rodean de un equipo de asesores, que regularmente varía en cantidad y en personal, con arreglo a la presencia de los distintos grupos en las sedes de las constelaciones e, igualmente, a medida que las condiciones locales cambien. En momentos de tensión, se pueden solicitar más hijos del orden de los vorondadecs, que se recibirán con celeridad, para ayudar en la labor de gobernación. Norlatiadec, vuestra propia constelación, está actualmente regida por doce hijos vorondadecs.
\usection{7. LOS MUNDOS VORONDADECS}
\vs p035 7:1 Los planetas de los vorondadecs constituyen el segundo grupo de siete mundos de la vía circulatoria de las setenta esferas primarias que rodea a Lugar de Salvación. Cada una de estas esferas, con sus seis satélites circundantes, está dedicada a una faceta especial de la actividad de los vorondadecs. En estos cuarenta y nueve ámbitos espaciales, los mortales ascendentes llegan a la cima de su educación con respecto a la legislación del universo.
\vs p035 7:2 En los mundos sedes de las constelaciones, los mortales ascendentes han observado el funcionamiento de las asambleas legislativas, pero aquí, en estos mundos de los vorondadecs, participan en la promulgación de la legislación general real del universo local bajo la tutela de los vorondadecs de mayor rango. Tales promulgaciones están encaminadas a coordinar los variados pronunciamientos de las asambleas legislativas autónomas de las cien constelaciones. La instrucción que se recibe en las escuelas vorondadecs no tiene parangón ni incluso en Uversa. Esta formación es progresiva, extendiéndose desde la primera esfera, con trabajo suplementario en sus seis satélites, hasta las seis esferas primarias restantes y a sus grupos de satélites vinculados.
\vs p035 7:3 En estos mundos de estudio y de trabajo práctico, los peregrinos ascendentes se iniciarán en numerosas actividades nuevas a medida que prosiguen su instrucción en estos cuarenta y nueve mundos. No se nos prohíbe dar a conocer estas inconcebibles y excelsas ocupaciones, pero no tenemos esperanza de ser capaces de describirlas para la mente material de los seres mortales. No encontramos palabras para transmitir su significado y no existe labor humana análoga que se pueda utilizar para ilustrarlas. En los mundos vorondadecs de la vía circulatoria de Lugar de Salvación, también se efectúan otras muchas actividades que no forman parte del régimen dispuesto para los mortales ascendentes.
\usection{8. LOS HIJOS LANONANDECS}
\vs p035 8:1 Después de la creación de los vorondadecs, el hijo creador y el espíritu materno del universo se unen con el propósito de dar existencia al tercer orden de filiación del universo: los lanonandecs. Y aunque estos seres se encargan de diversas tareas relacionadas con la administración de los sistemas, se les conoce mejor como los soberanos de los sistemas, o gobernantes de los sistemas locales, y como príncipes planetarios, o jefes del gobierno de los mundos habitados.
\vs p035 8:2 Al ser seres creados de un orden de filiación último y de inferior rango ---en cuanto a niveles de divinidad---, necesitaron superar ciertos cursos de formación en los mundos melquisedecs a fin de prepararse para el servicio que tenían que prestar. Fueron los primeros alumnos de la Universidad Melquisedec y sus maestros y examinadores melquisedecs los clasificaron y certificaron según su aptitud, ser personal y logros.
\vs p035 8:3 El universo de Nebadón comenzó su existencia exactamente con doce millones de lanonandecs y, tras pasar por la esfera Melquisedec, en las pruebas finales, se les dividió en tres clases:
\vs p035 8:4 \li{1.}\bibemph{Lanonandecs primarios}. Había 709\,841 de los de más alto rango. A estos hijos se les nombra soberanos del sistema y asistentes de los consejos supremos de las constelaciones al igual que asesores de la administración superior del universo.
\vs p035 8:5 \li{2.}\bibemph{Lanonandecs secundarios}. De este orden emergente de la esfera Melquisedec, había 10\,234\,601. Se les nombra príncipes planetarios y se les destina a las reservas de dicho orden.
\vs p035 8:6 \li{3.}\bibemph{Lanonandecs terciarios}. Este grupo estaba formado por 1\,055\,558 hijos de este orden. Actúan de asistentes de rango menor, mensajeros, custodios, comisionados, observadores, y llevan a cabo diversos cometidos relacionados con el sistema y los mundos que lo componen.
\vs p035 8:7 \pc Tal como ocurre con los seres evolutivos, a estos hijos no les es posible avanzar de un grupo a otro. Una vez expuestos a la formación auspiciada por los melquisedecs, una vez que se les ha probado y clasificado, sirven de forma continuada en la categoría asignada. Tampoco pueden reproducirse. Su número en el universo es fijo.
\vs p035 8:8 Redondeando su número, en Lugar de Salvación, el orden de hijos lanonandecs se divide de la siguiente manera:
\vs p035 8:9 \pc Coordinadores del universo y asesores de las constelaciones\bibdf100\,000
\vs p035 8:10 Soberanos de los sistemas y asistentes\bibdf600\,000
\vs p035 8:11 Príncipes planetarios y reservas\bibdf10\,000\,000
\vs p035 8:12 Colectivo de mensajeros\bibdf400\,000
\vs p035 8:13 Custodios y archivistas\bibdf100\,000
\vs p035 8:14 Colectivo de reserva\bibdf800\,000
\vs p035 8:15 \pc Al ser un orden de filiación de rango algo menor que los melquisedecs y los vorondadecs, los lanonandecs pueden realizar un servicio incluso mayor en las unidades menores del universo, ya que son capaces de acercarse más a las modestas criaturas de las razas inteligentes. No obstante, también se hallan en un mayor riesgo de descarriarse, de apartarse de un modo de proceder aceptable del gobierno del universo. Si bien, estos lanonandecs, especialmente los de orden primario, son los más aptos y versátiles de todos los administradores de los universos locales. En capacidad ejecutiva solamente los superan Gabriel y sus colaboradores no revelados.
\usection{9. LOS GOBERNANTES LANONANDECS}
\vs p035 9:1 Los lanonandecs gobiernan los planetas de forma continuada y los sistemas en turno rotatorio. Uno de ellos ostenta en este momento el gobierno de Jerusem, la sede de vuestro sistema local de mundos habitados.
\vs p035 9:2 Los soberanos del sistema ejercen su labor de gobierno desde las sedes de cada uno de los sistemas de mundos habitados con un régimen de comisiones de dos o tres miembros. Cada decamilenio, el padre de la constelación nombra jefe a uno de estos lanonandecs. No obstante, al ser algo totalmente facultativo para los gobernantes de la constelación, a veces no se producen cambios en la jefatura del trío. Los gobiernos del sistema no cambian repentinamente su personal a no ser que ocurra algún tipo de tragedia.
\vs p035 9:3 Cuando los soberanos del sistema o sus asistentes están llamados a retirarse, el consejo supremo, situado en la sede de la constelación, elige a aquellos que ocuparán sus puestos a partir de las reservas existentes de ese orden, un grupo más numeroso en Edentia que el promedio indicado anteriormente.
\vs p035 9:4 Los consejos supremos de los lanonandecs están emplazados en las distintas sedes de las constelaciones. El colaborador altísimo de mayor rango del Padre de la Constelación preside este grupo mientras que el de menor rango supervisa las reservas de los lanonandecs del orden secundario.
\vs p035 9:5 \pc Los soberanos del sistema hacen honor a sus propios nombres; es prácticamente la actividad que realizan en relación a los asuntos locales de los mundos habitados. Ejercen la dirección de los príncipes planetarios, de los hijos materiales y de los espíritus servidores de una manera que se aproxima a lo paternal. Su dominio personal es prácticamente completo. No están supervisados por los observadores trinitarios del universo central. Forman la división ejecutiva del universo local y, como custodios del cumplimiento de los mandatos legislativos y encargados de la aplicación de veredictos judiciales, se encuentran en una posición dentro de toda la administración del universo en la que la deslealtad personal a la voluntad del hijo miguel podría fácil y rápidamente afianzarse e intentar imponerse.
\vs p035 9:6 Es de lamentar que en nuestro universo local más de setecientos hijos lanonandecs se rebelaran contra el gobierno del universo, provocando así la confusión en algunos sistemas y en numerosos planetas. De todos estos casos de malogro, únicamente tres de ellos eran soberanos del sistema; prácticamente todos pertenecían a los órdenes segundo y tercero, esto es, a príncipes planetarios y a lanonandecs terciarios.
\vs p035 9:7 El gran número de estos hijos que ha faltado a su integridad no es atribuible a fallo alguno en su creación. Se podrían haber creado divinamente perfectos, pero se hicieron así para que pudiesen entender mejor, y acercarse más, a las criaturas evolutivas que habitan en los mundos del tiempo y el espacio.
\vs p035 9:8 De todos los universos locales de Orvontón, nuestro suprauniverso, exceptuando Henselón, es el que ha perdido a un mayor número de miembros de este orden de hijos. Existe consenso en Uversa de que la causa de muchos de los problemas de gobernación que se han dado en Nebadón se ha debido a que nuestros hijos lanonandecs se crearon con un amplio grado de libertad personal para tomar decisiones. No realizo esta observación en un sentido crítico. El creador de nuestro universo tiene pleno poder y autoridad para hacer esto. Nuestros altos gobernantes sostienen que, aunque estos hijos en su libertad de elección ocasionan un desmesurado problema en las etapas tempranas del universo, cuando las circunstancias se han sometido a un análisis detallado y finalmente se han aclarado, la mayor lealtad y el más completo servicio por voluntad propia de parte de estos hijos totalmente probados traerán beneficios que resarcirán con creces la confusión y las tribulaciones de tiempos anteriores.
\vs p035 9:9 \pc En caso de rebelión en la sede de un sistema, por lo general, en un plazo relativamente corto, un nuevo soberano toma posesión, pero esto no sucede así con los distintos planetas. Estos constituyen las unidades básicas de la creación material, y la libertad de elección de las criaturas es un factor a tener en cuenta en la resolución final de tales problemas. Se nombran príncipes planetarios sucesores para los mundos aislados, para esos planetas cuyos príncipes gobernantes se hayan descarriado, aunque no asumen el gobierno efectivo de tales mundos hasta que las consecuencias de la insurrección no se hayan superado y eliminado parcialmente gracias a las medidas correctoras adoptadas por los melquisedecs y otros seres personales servidores. La rebelión de un príncipe planetario conlleva el aislamiento instantáneo de su planeta; las vías espirituales locales se cortan de inmediato. Solamente un hijo de gracia puede restablecer las líneas interplanetarias de comunicación en un mundo espiritualmente aislado.
\vs p035 9:10 Existe un plan para salvar a estos hijos rebeldes e imprudentes, y muchos han hecho uso de esta medida misericordiosa; no obstante, nunca más podrán actuar en aquellos puestos en los que fallaron. Tras su rehabilitación, se les asigna a tareas de custodia y a departamentos de administración física.
\usection{10. LOS MUNDOS LANONANDECS}
\vs p035 10:1 El conjunto de las esferas de administración de los lanonandecs conforman el tercer grupo de siete mundos de la vía circulatoria de Lugar de Salvación compuesta de setenta planetas, con sus cuarenta y dos satélites respectivos. En estos ámbitos los lanonandecs experimentados, pertenecientes al colectivo de los ex soberanos de los sistemas ofician como docentes, impartiendo administración a los peregrinos ascendentes y a las multitudes seráficas. Los mortales evolutivos observan el trabajo de los administradores de los sistemas en sus propias capitales, aunque aquí participan igualmente en la coordinación real de los pronunciamientos administrativos de los diez mil sistemas locales.
\vs p035 10:2 Estas escuelas de administración del universo local están bajo la supervisión de un colectivo de hijos lanonandecs con una larga experiencia como soberanos de los sistemas y como consejeros de las constelaciones. Solo las escuelas de mandatarios de Ensa superan en excelencia a estas facultades para líderes.
\vs p035 10:3 Aunque sirven como esferas de instrucción para los mortales ascendentes, los mundos lanonandecs son igualmente centros de gran actividad en relación a las gestiones normales y rutinarias de la administración del universo. En todo el camino hacia el Paraíso, los peregrinos ascendentes continúan sus estudios en las escuelas de prácticas y conocimiento aplicado, en las que verdaderamente se les entrena para hacer realmente lo que se les imparte. El sistema educativo del universo auspiciado por los melquisedecs es práctico, progresivo, significativo y experiencial. Abarca la formación en las cosas materiales, intelectuales, morontiales y espirituales.
\vs p035 10:4 \pc Con respecto a estas esferas de administración de los lanonandecs, la mayoría de los hijos rescatados de ese orden sirven como custodios y directores de asuntos planetarios. Estos príncipes planetarios transgresores y sus colaboradores en rebelión, que optan por aceptar la rehabilitación que se les ofrece, continuarán sirviendo en estas tareas rutinarias, al menos hasta que el universo de Nebadón se asiente en luz y vida.
\vs p035 10:5 \pc Sin embargo, muchos de los hijos lanonandecs de los sistemas de mayor antigüedad han dejado constancia de espléndidos historiales de servicio, administración y logros espirituales. Constituyen un grupo noble, fiel y leal, a pesar de su tendencia a caer en el error por las falacias de la libertad personal y las ilusiones de autodeterminación.
\vsetoff
\vs p035 10:6 [Auspiciado por el jefe de los arcángeles que actúa con el beneplácito de Gabriel de Lugar de Salvación.]
