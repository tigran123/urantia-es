\upaper{57}{El origen de Urantia}
\author{Portador de vida}
\vs p057 0:1 Al presentar estos extractos de los archivos existentes en Jerusem respecto a los antecedentes e historia temprana de Urantia para los registros de este planeta, se nos dan instrucciones para que calculemos el tiempo según su uso ordinario ---el actual calendario bisiesto producto de los 365¼ días que tiene el año---. Por lo general, intentaremos no dar años exactos, aunque haya constancia de ellos. Redondearemos los números hasta uno entero más próximo, porque consideramos que es el mejor método de exponer estos hechos históricos.
\vs p057 0:2 Cuando hagamos referencia a algún acontecimiento que haya tenido lugar hace uno o dos millones de años, es nuestra intención determinar la fecha de tal hecho en número de años, trazándola en relación a las primeras décadas del siglo XX de la era cristiana. Describiremos, por tanto, estos sucesos distantes en el tiempo como si hubieran ocurrido en periodos uniformes de miles, millones y miles de millones de años.
\usection{1. LA NEBULOSA ANDRÓNOVER}
\vs p057 1:1 Urantia tiene su origen en vuestro Sol, y vuestro Sol es uno de los múltiples descendientes de la nebulosa Andrónover, la cual, en otro tiempo, estaba organizada como parte componente de la potencia física y de la sustancia material del universo local de Nebadón. Y esta gran nebulosa se originó a su vez en la carga de la fuerza universal del espacio en el suprauniverso de Orvontón hace muchísimo tiempo.
\vs p057 1:2 En el momento de comenzar este relato, los organizadores mayores y primarios de la fuerza del Paraíso, desde hacía mucho tiempo, habían tenido el control completo de las energías espaciales, que después se organizarían bajo la forma de la nebulosa Andrónover.
\vs p057 1:3 \pc Hace \bibemph{987\,000\,000\,000} de años, el organizador de la fuerza adjunto, en aquel momento el inspector en funciones número 811\,307 del orden de Orvontón, que viajaba desde Uversa, informó a los ancianos de días de que las condiciones espaciales eran favorables para iniciar los fenómenos de la materialización en cierto sector del entonces segmento oriental de Orvontón.
\vs p057 1:4 \pc Hace \bibemph{900\,000\,000\,000} de años, tal como consta en los archivos de Uversa, se registró un permiso expedido por el Consejo de Equilibrio de Uversa, dirigido al gobierno del suprauniverso, autorizando el envío de un organizador de la fuerza y sus asistentes a la región previamente designada por el inspector número 811\,307. Las autoridades de Orvontón encomendaron a quien originariamente había descubierto este universo potencial que cumpliera el mandato de los ancianos de días en el que se pedía la organización de una nueva creación material.
\vs p057 1:5 El registro de este permiso indica que el organizador de la fuerza y su comitiva ya habían partido de Uversa en el largo viaje a ese sector oriental del espacio, donde, posteriormente, emprenderían un extenso período de actividad que daría como resultado la aparición gradual de una nueva creación física en Orvontón.
\vs p057 1:6 \pc Hace \bibemph{875\,000\,000\,000} de años, se dio inicio, como estaba previsto, a la enorme nebulosa de Andrónover número 876\,926. Solo se precisaba la presencia del organizador de la fuerza y sus asistentes de enlace para dar comienzo al remolino de energía que acabó por convertirse en este inmenso ciclón del espacio. Una vez iniciadas dichas rotaciones nebulares, los organizadores vivos de la fuerza se retiran simplemente en ángulos rectos respecto al plano del disco de rotación y, desde ese momento en adelante, son las cualidades intrínsecas de la energía las que garantizan la evolución progresiva y ordenada de este nuevo sistema físico.
\vs p057 1:7 Aproximadamente en este momento, la narración da un cambio y se ocupa de la actuación de los seres personales del suprauniverso. En realidad, la historia propiamente dicha comienza en este punto ---en ese momento aproximado en el que los organizadores de la fuerza del Paraíso se disponen a retirarse, habiendo dejado las condiciones energéticas y espaciales listas para la acción de los directores de la potencia y de los controladores físicos del suprauniverso de Orvontón---.
\usection{2. LA ETAPA NEBULAR PRIMARIA}
\vs p057 2:1 Todas las creaciones materiales evolutivas nacen de nebulosas circulares y gaseosas, y todas estas nebulosas primarias son circulares durante la primera parte de su existencia gaseosa. A medida que envejecen, se vuelven por lo general espirales, y cuando su función de formar soles llega a su fin, terminan a menudo siendo cúmulos de estrellas o enormes soles rodeados por un número variable de planetas, satélites y grupos más pequeños de materia, en muchos sentidos parecidos a vuestro diminuto sistema solar.
\vs p057 2:2 \pc Hace \bibemph{800\,000\,000\,000} de años, Andrónover se consolidó claramente como una de las magníficas nebulosas primarias de Orvontón. Cuando los astrónomos de universos cercanos observaban este fenómeno espacial, veían muy poco que les llamara la atención. Los cálculos aproximados de la gravedad realizados en otras creaciones adyacentes mostraban que en las regiones de Andrónover estaban teniendo lugar materializaciones del espacio, pero eso era todo.
\vs p057 2:3 \pc Hace \bibemph{700\,000\,000\,000} de años, el sistema de Andrónover estaba alcanzando proporciones gigantescas, y se enviaron más controladores físicos a nueve creaciones materiales circundantes con el fin de dar apoyo y cooperar con los centros de la potencia de este nuevo sistema material que evolucionaba con tanta rapidez. En esta remota época, toda la materia que se legaba para las creaciones posteriores se contenía dentro de los confines de esta gigantesca rueda espacial, que continuaba rotando y que, tras alcanzar su diámetro máximo, rotaba cada vez con mayor rapidez a medida que continuaba condensándose y contrayéndose.
\vs p057 2:4 \pc Hace \bibemph{600\,000\,000\,000} de años se alcanzó el punto álgido del período de activaciones energéticas del sistema de Andrónover; la nebulosa había adquirido su masa máxima. En ese momento era una gigantesca nube circular de gas con una forma algo parecida a un esferoide aplanado. Este fue el período inicial de la formación diferenciada de la masa y de la velocidad de rotación variable. La gravedad y otras influencias estaban a punto de comenzar la tarea de convertir los gases del espacio en materia organizada.
\usection{3. LA ETAPA NEBULAR SECUNDARIA}
\vs p057 3:1 La enorme nebulosa comenzó entonces, paulatinamente, a adoptar forma espiral y a hacerse claramente visible incluso para los astrónomos de universos distantes. Así es la historia natural de la mayoría de las nebulosas; antes de empezar a arrojar soles e iniciar la labor de dar origen a un universo, estas nebulosas secundarias del espacio se suelen ver como \bibemph{fenómenos helicoidales}.
\vs p057 3:2 Los estudiosos de las estrellas de aquella lejana época que estaban en las proximidades, al observar esta metamorfosis de la nebulosa Andrónover, veían exactamente lo que los astrónomos del siglo XX ven cuando dirigen su telescopio hacia el espacio y visualizan, en la presente era, las nebulosas helicoidales del espacio exterior cercano.
\vs p057 3:3 Aproximadamente en el momento en el que se alcanzó la masa máxima, el control gravitatorio del contenido gaseoso comenzó a hacerse más débil, y a ello siguió la etapa de fuga de gas. El gas salía a raudales en dos brazos gigantescos y distintos, que se habían originado en los lados opuestos de la masa matriz. Las rápidas rotaciones de este enorme núcleo central pronto dieron un aspecto helicoidal a estos dos chorros de gas que se proyectaban hacia el espacio. El enfriamiento y posterior condensación de algunas partes de estos protuberantes brazos acabaron por producirle su apariencia nudosa. Estas partes de mayor densidad eran inmensos sistemas y subsistemas de materia física que giraban por el espacio en medio de la nube gaseosa de la nebulosa, mientras que se mantenían firmemente sujetos a la gravedad de la rueda matriz.
\vs p057 3:4 Pero la nebulosa había empezado a contraerse, y el aumento del índice de rotación redujo todavía más el control de la gravedad; y, al poco tiempo, las regiones gaseosas exteriores comenzaron de hecho a escaparse de la sujeción directa del núcleo nebular, saliendo al espacio por vías de circulación de trazado irregular, regresando a las regiones nucleares para completar su movimiento circulatorio, y así sucesivamente. Pero esto no fue sino una etapa temporal dentro del avance progresivo de la nebulosa. El índice de rotación, siempre creciente, arrojó enseguida al espacio, en vías circulatorias independientes, enormes soles.
\vs p057 3:5 Y esto fue lo que le sucedió a Andrónover hace muchísimas eras. La rueda de energía se fue haciendo cada vez mayor hasta que alcanzó su máxima expansión y, entonces, cuando la contracción se inició, esta giró cada vez más rápidamente hasta que terminó por alcanzar la etapa centrífuga crítica y comenzó la gran desintegración.
\vs p057 3:6 \pc Hace \bibemph{500\,000\,000\,000} de años nació el primer Sol de Andrónover. Este rayo abrasador se desprendió de la sujeción gravitatoria matriz y salió disparado a gran velocidad al espacio tomando un rumbo independiente por el cosmos creado. Su ruta de escape determinó su propia órbita. Estos soles tan jóvenes se hacen esféricos muy pronto e inician sus largas e intensas trayectorias como estrellas del espacio. Con la excepción de los núcleos nebulares terminales, la gran mayoría de los soles de Orvontón nacieron de forma análoga. Estos soles, al desprenderse, pasan por distintos períodos de evolución y posterior servicio al universo.
\vs p057 3:7 \pc Hace \bibemph{400\,000\,000\,000} de años comenzó el período de recaptación de la nebulosa de Andrónover. Muchos de los soles más cercanos y pequeños fueron recapturados de nuevo como resultado del agrandamiento gradual y de la posterior condensación de su núcleo materno. Muy pronto se dio principio a la fase terminal de la condensación nebular, el período que siempre precede a la separación final de estos inmensos cúmulos espaciales de energía y materia.
\vs p057 3:8 Apenas un millón de años después de esta época, Miguel de Nebadón, un hijo creador del Paraíso, eligió esta nebulosa en desintegración como sitio para su aventura de formar un universo. Casi de inmediato, tuvo lugar la creación de los mundos arquitectónicos de Lugar de Salvación y los cien grupos de planetas que forman las sedes de las constelaciones. Se precisó casi un millón de años para terminar estos grupos de mundos, especialmente creados. Los planetas sede de los sistemas locales se construyeron durante el periodo de tiempo que se extendía desde aquel momento hasta hace alrededor de cinco mil millones de años.
\vs p057 3:9 \pc Hace \bibemph{300\,000\,000\,000} de años, las vías circulatorias solares de Andrónover ya estaban bien establecidas, y el sistema nebular pasaba por un período transitorio de relativa estabilidad física. Sobre esta fecha, los asistentes de Miguel llegaron a Lugar de Salvación y, en Orvontón, el gobierno de Uversa reconoció la existencia física del universo local de Nebadón.
\vs p057 3:10 \pc Hace \bibemph{200\,000\,000\,000} de años se evidenció un avance progresivo del proceso de contracción y condensación con una enorme generación de calor en el cúmulo central o masa nuclear de Andrónover. Incluso en las regiones próximas a la rueda central y matriz\hyp{}solar apareció un espacio relativo. Las regiones exteriores se volvieron más estables y mejor organizadas; algunos planetas que rotaban en torno a los soles recién nacidos se habían enfriado lo suficiente como para ser aptos para la implantación de la vida. Los planetas habitados más antiguos de Nebadón datan de estos tiempos.
\vs p057 3:11 Se produce ahora, por primera vez, el funcionamiento del mecanismo universal de Nebadón ya completado, y la creación de Miguel queda registrada en Uversa como un universo propio para la habitación y la ascensión progresiva de los mortales.
\vs p057 3:12 \pc Hace \bibemph{100\,000\,000\,000} de años la tensión de la condensación nebular llegó a su momento álgido; se alcanzó el punto máximo de tensión calórica. Esta etapa crucial de pugna entre la gravedad y el calor dura a veces muchas eras, pero, tarde o temprano, el calor gana la contienda a la gravedad, y comienza el período espectacular de la dispersión de soles. Y esto supone el fin de la trayectoria secundaria de una nebulosa espacial.
\usection{4. LAS ETAPAS TERCIARIA Y CUATERNARIA}
\vs p057 4:1 En su etapa primaria, la nebulosa es circular; en la secundaria, espiral; en la terciaria se produce la primera dispersión de soles, mientras que, en la cuaternaria, se engloba el segundo y último ciclo de dispersión de soles, en el que el núcleo matriz acaba siendo un cúmulo globular o un sol solitario que funciona como centro de un sistema solar delimitado.
\vs p057 4:2 \pc Hace \bibemph{75\,000\,000\,000} de años, esta nebulosa había alcanzado el punto álgido de su etapa de formación de soles. Este fue el culmen del primer período de pérdidas de soles. Desde entonces, la mayoría de estos soles se han apoderado de extensos sistemas planetarios, satélites, islas oscuras, cometas, meteoros y nubes de polvo cósmico.
\vs p057 4:3 \pc Hace \bibemph{50\,000\,000\,000} de años concluyó este primer período de dispersión de soles; la nebulosa finalizaba rápidamente su ciclo terciario, durante el que había dado origen a 876\,926 sistemas solares.
\vs p057 4:4 \pc Hace \bibemph{25\,000\,000\,000} de años se evidenció el final del ciclo terciario de vida nebular, que trajo consigo la organización y la relativa estabilización de los lejanos sistemas estrellados derivados de esta nebulosa matriz. Pero el proceso de contracción física y de creciente producción de calor continuó en la masa central del remanente nebular.
\vs p057 4:5 \pc Hace \bibemph{10\,000\,000\,000} de años comenzó el ciclo cuaternario de Andrónover. La masa nuclear había alcanzado su temperatura máxima; se había llegado al punto crítico de condensación. El núcleo materno original convulsionaba bajo la presión combinada de la tensión por condensación de su propio calor interno y del flujo creciente de la atracción gravitatoria del enjambre de sistemas solares liberados que lo rodeaban. Las erupciones nucleares que iban a dar inicio al segundo ciclo solar nebular eran inminentes. El ciclo cuaternario de la existencia nebular estaba a punto de empezar.
\vs p057 4:6 \pc Hace \bibemph{8\,000\,000\,000} de años empezó la descomunal erupción final. Solamente los sistemas exteriores están a salvo en momentos de tal agitación cósmica. Y este fue el principio del final de la nebulosa. Esta descarga última de soles se extendió por un período de casi dos mil millones de años.
\vs p057 4:7 \pc Hace \bibemph{7\,000\,000\,000}de años se evidenció el punto álgido de la desintegración final de Andrónover. Este fue el período en el que terminaron por nacer los soles de mayor tamaño y se produjo el culmen de las perturbaciones físicas locales.
\vs p057 4:8 \pc Hace \bibemph{6\,000\,000\,000} de años se señala el término de la desintegración final y el nacimiento de vuestro Sol, el quincuagésimo sexto sol antes del último sol de la familia solar de Andrónover. Esta erupción final del núcleo nebular dio nacimiento a 136\,702 soles, esferas solitarias en su mayoría. El número total de soles y sistemas solares que tuvieron su origen en la nebulosa Andrónover es de 1\,013\,628. El Sol del sistema solar es el número 1\,013\,572.
\vs p057 4:9 Y la gran nebulosa Andrónover ha dejado de existir, pero sigue viviendo en los muchos soles y sus familias planetarias que se originaron en esta nube matriz del espacio. El último resto nuclear de esta magnífica nebulosa todavía arde con un brillo rojizo y continúa emitiendo luz y calor de intensidad moderada a su familia planetaria residual de ciento sesenta y cinco mundos, que en este momento giran en torno a esta venerable madre de dos poderosas generaciones de monarcas de luz.
\usection{5. EL ORIGEN DE MONMATIA: EL SISTEMA SOLAR DE URANTIA}
\vs p057 5:1 Hace \bibemph{5\,000\,000\,000} de años, vuestro Sol era una llameante esfera prácticamente aislada, que había atraído hacia sí la mayor parte de la materia cercana que circulaba por el espacio, residuos de la reciente convulsión vinculada a su propio nacimiento.
\vs p057 5:2 Hoy en día, vuestro Sol tiene una relativa estabilidad; sin embargo, sus ciclos de manchas solares cada once años y medio ponen de manifiesto que en su juventud era una estrella variable. Durante tempranos tiempos, la contracción continua y el consiguiente aumento gradual de la temperatura produjeron formidables convulsiones en su superficie. Estas bruscas sacudidas de proporciones titánicas necesitaban tres días y medio para completar un ciclo de luminosidad variable. Tal estado de variabilidad, tal pulsación periódica, hizo a vuestro Sol altamente reactivo a ciertas influencias externas con las que enseguida se encontraría.
\vs p057 5:3 Así se disponía el escenario del espacio local para el excepcional origen de \bibemph{Monmatia,} nombre dado a la familia planetaria de vuestro Sol, el sistema solar al cual pertenece vuestro mundo. Menos del uno por ciento de los sistemas planetarios de Orvontón se han originado de forma similar.
\vs p057 5:4 \pc Hace \bibemph{4500\,000\,000} de años, el enorme sistema Angona comenzó a aproximarse a este sol solitario. El centro de este gran sistema era un gigante oscuro del espacio, sólido, altamente cargado, que ejercía una extraordinaria atracción gravitatoria.
\vs p057 5:5 A medida que Angona se aproximaba más al Sol, en los momentos de máxima expansión durante las pulsaciones solares, se proyectaban al espacio, en forma de gigantescas lenguas solares, chorros de materia gaseosa. Al principio, estas lenguas llameantes de gas invariablemente volvían a caer en el Sol; pero a medida que Angona se iba acercando cada vez más, la atracción de la gravedad del gigantesco visitante se hizo tan poderosa que estas lenguas de gas se desprendían en algunos puntos, y sus raíces caían de nuevo en el Sol mientras que las secciones exteriores se separaban para formar cuerpos independientes de materia, o meteoritos solares, los cuales de inmediato comenzaban a girar alrededor del Sol en sus propias órbitas elípticas.
\vs p057 5:6 A medida que el sistema de Angona se aproximaba, las eyecciones solares se hacían cada vez más grandes; más y más materia se desprendía del Sol convirtiéndose en cuerpos independientes que circulaban por el espacio circundante. Esta situación se desarrolló durante quinientos mil años hasta que Angona alcanzó su punto más cercano al Sol; con lo que, en conjunción con una de sus convulsiones periódicas internas, el Sol sufrió un fraccionamiento parcial; enormes volúmenes de materia se vertieron simultáneamente desde lados opuestos. Por el lado de Angona, se separó una inmensa columna de gases solares, más bien puntiaguda en los dos extremos y notablemente protuberante en el centro, y se desprendió definitivamente del control gravitatorio inmediato del Sol.
\vs p057 5:7 Esta gran columna de gases solares que se separó así del Sol luego evolucionó para convertirse en los doce planetas del sistema solar. Los gases expulsados como efecto recíproco desde el lado opuesto del Sol, en un flujo afín a la extrusión de este gigantesco ancestro del sistema solar, se han condensado desde entonces para formar los meteoros y el polvo espacial del sistema solar; si bien, buena parte de esta materia se recobró luego por la gravedad solar a medida que el sistema Angona retrocedía al espacio remoto.
\vs p057 5:8 Aunque Angona logró alejar la materia ancestral de los planetas del sistema solar y el enorme volumen de materia que ahora circula alrededor del Sol como asteroides y meteoros, no obtuvo para sí nada de esta materia solar. El sistema visitante no se aproximó lo suficiente como para sustraer, de hecho, sustancia alguna al Sol, aunque sí pasó lo bastante cerca como para trasvasar hacia el espacio intermedio todo la materia de la que hoy en día se compone el sistema solar.
\vs p057 5:9 Los cinco planetas interiores y los cinco exteriores pronto se formaron, a escala reducida, a partir de los núcleos que se iban enfriando y condesando en los extremos estrechos y menos masivos de la gigantesca protuberancia gravitatoria que Angona había logrado separar del Sol; mientras que Saturno y Júpiter se formaron a partir de las partes centrales más masivas y protuberantes. La poderosa atracción gravitatoria de Júpiter y Saturno capturó enseguida la mayor parte de la materia que se le había extraído a Angona, tal como atestigua el movimiento retrógrado de algunos de sus satélites.
\vs p057 5:10 Júpiter y Saturno, al derivarse del centro mismo de la enorme columna de gases solares sobrecalentados, contenían tanta materia solar y de tan alta temperatura que brillaban con una luz resplandeciente y emitían enormes cantidades de calor; durante un corto periodo de tiempo, tras su formación como cuerpos espaciales separados, fueron en realidad soles secundarios. Estos dos planetas, los más grandes del sistema solar, siguen siendo, hasta hoy día, gaseosos en gran medida; no han llegado todavía a enfriarse hasta el punto de condensarse o solidificarse por completo.
\vs p057 5:11 Los núcleos gaseosos, en proceso de contracción, de los otros diez planetas alcanzaron pronto la fase de solidificación y comenzaron pues a atraer hacia sí crecientes cantidades de materia meteórica que circulaba por el espacio cercano. Los mundos del sistema solar tuvieron, por tanto, un doble origen: eran núcleos de condensación gaseosa que se vieron incrementados más tarde por la captura de enormes cantidades de meteoros. De hecho, todavía continúan capturando meteoros, aunque en un número mucho menor.
\vs p057 5:12 Los planetas no giran alrededor del Sol en el plano ecuatorial de su matriz solar, algo que harían si hubiesen sido lanzados por la rotación solar. En vez de ello, se desplazan en el plano de la extrusión solar de Angona, que formaba un amplio ángulo con respecto al plano del ecuador solar.
\vs p057 5:13 \pc Aunque Angona no pudo capturar nada de la masa solar, vuestro Sol sí añadió a su familia planetaria en metamorfosis alguna materia que circulaba por el espacio del sistema visitante. Debido al intenso campo de gravedad de Angona, su familia planetaria accesoria seguía órbitas a considerable distancia del gigante oscuro; y, poco después de la extrusión de la masa ancestral del sistema solar, mientras que Angona todavía se encontraba en las proximidades del Sol, tres de los planetas mayores del sistema Angona se acercaron tanto al masivo antepasado del sistema solar que su atracción gravitatoria, aumentada por la del Sol, resultó suficiente para desequilibrar el dominio gravitatorio de Angona y separar, de forma permanente, a estos tres elementos dependientes del errante celestial.
\vs p057 5:14 Toda la materia del sistema solar derivada del Sol estaba originariamente dotada de un movimiento orbital con una dirección homogénea, y de no haber sido por la interposición de estos tres cuerpos espaciales foráneos, toda ella seguiría manteniendo esta misma dirección orbital. Sin embargo, el impacto de tales elementos dependientes de Angona introdujo fuerzas direccionales, nuevas y ajenas, en el sistema solar que emergía, resultando en la aparición del movimiento retrógrado. El movimiento \bibemph{retrógrado} de cualquier sistema astronómico es siempre accidental y siempre aparece como consecuencia del impacto de una colisión de cuerpos extraños del espacio. Estas colisiones no siempre producen un movimiento retrógrado, pero este siempre aparece en sistemas que contienen masas de orígenes diversos.
\usection{6. LA ETAPA DEL SISTEMA SOLAR: LA ERA DE LA FORMACIÓN DE LOS PLANETAS}
\vs p057 6:1 Después del nacimiento del sistema solar sobrevino un período de reducción de descargas solares. De forma decreciente, durante otros quinientos mil años, el Sol continuó vertiendo cantidades cada vez menores de materia al espacio adyacente. No obstante, durante estos tiempos primitivos de órbitas erráticas, cuando los cuerpos circundantes realizaban su mayor acercamiento al Sol, la matriz solar conseguía recuperar una gran parte de este material meteórico.
\vs p057 6:2 \pc Los planetas más próximos al Sol fueron los primeros en aminorar su rotación debido a la fricción de marea. Tales influencias gravitatorias contribuyen, además, a la estabilización de las órbitas planetarias, mientras que actúan de freno al grado de rotación axial del planeta. Esto hace que un planeta gire cada vez más lentamente hasta cesar su rotación axial, quedando un hemisferio del planeta siempre virado hacia el Sol o hacia algún cuerpo mayor, tal como queda ilustrado por el planeta Mercurio y la Luna, que siempre presenta la misma cara a Urantia.
\vs p057 6:3 Cuando las fricciones de marea de la Luna y de la Tierra se igualen, la Tierra siempre presentará el mismo hemisferio hacia la Luna, y el día y el mes serán análogos, ---durarán unos cuarenta y siete días---. Cuando se llegue a tal estabilidad de órbitas, las fricciones de marea provocarán la acción contraria; no impulsarán más a la Luna para que se aleje de la Tierra, sino que, paulatinamente, acercarán este satélite al planeta. Y, entonces, en ese remoto futuro, cuando la Luna se acerque a unos diecisiete mil quinientos kilómetros de la Tierra, la acción de la gravedad de esta última hará que la Luna convulsione, y esta explosión, ocasionada por la gravedad de marea, la destruirá convirtiéndola en pequeñas partículas, que pueden acumularse en torno al mundo como anillos de materia parecidos a los de Saturno o ser atraídas de forma gradual a la Tierra como meteoros.
\vs p057 6:4 Si los cuerpos espaciales son similares en tamaño y densidad, pueden colisionar. Pero cuando dos cuerpos espaciales de densidad similar son relativamente desiguales en tamaño, entonces, si el más pequeño se aproxima paulatinamente al más grande, se producirá la desintegración del cuerpo más pequeño cuando el radio de su órbita se vuelva inferior a dos veces y media el radio del cuerpo más grande. Las colisiones entre los gigantes del espacio son de hecho raras, pero las explosiones de los cuerpos menores por causa de la marea gravitatoria son bastante comunes.
\vs p057 6:5 Las estrellas fugaces ocurren en enjambres porque son los fragmentos de cuerpos mayores de materia que han convulsionado por la gravedad de marea ejercida por cuerpos espaciales cercanos y aún mayores. Los anillos de Saturno son fragmentos de un satélite desintegrado. Una de las lunas de Júpiter se está acercando ahora peligrosamente a la zona crítica de la disrupción de marea y, dentro de algunos millones de años, o será atrapada por el planeta o convulsionará por la gravedad de marea. El quinto planeta del sistema solar de antaño atravesó, hace mucho tiempo, una órbita irregular, acercándose, periódicamente, cada vez más a Júpiter, hasta que entró en la zona crítica de convulsión por esta gravedad de marea, y se fragmentó rápidamente, convirtiéndose en el cúmulo de asteroides de la época presente.
\vs p057 6:6 \pc Hace \bibemph{4\,000\,000\,000} de años se evidenció la organización de los sistemas de Júpiter y Saturno prácticamente como se observa en la actualidad salvo por sus lunas, las cuales continuaron aumentando de tamaño durante varios miles de millones de años. De hecho, todos los planetas y satélites del sistema solar siguen todavía creciendo como resultado de continuas capturas de meteoros.
\vs p057 6:7 \pc Hace \bibemph{3\,500\,000\,000} de años los núcleos de condensación de los otros diez planetas estaban bien formados y los de la mayoría de las lunas estaban intactos, aunque algunos de los satélites más pequeños se unieron después para formar las lunas más grandes de hoy en día. Este periodo se puede considerar como la era de formación planetaria por acumulación.
\vs p057 6:8 \pc Hace \bibemph{3\,000\,000\,000} de años el sistema solar funcionaba prácticamente como hoy. Sus componentes seguían creciendo en tamaño a medida que los meteoros del espacio continuaban cayendo sobre los planetas y sus satélites a un ritmo prodigioso.
\vs p057 6:9 Sobre esta época, vuestro sistema solar quedó inscrito en el registro físico de Nebadón y se le dio el nombre de Monmatia.
\vs p057 6:10 \pc Hace \bibemph{2\,500\,000\,000} de años, los planetas habían aumentado inmensamente. Urantia era una esfera bien desarrollada, de aproximadamente una décima parte de su masa actual y aún continuaba creciendo de forma rápida por acumulación de meteoros.
\vs p057 6:11 Toda esta tremenda actividad es parte normal de la creación de un mundo evolutivo del tipo de Urantia y constituye los preliminares astronómicos que establecen el escenario para el comienzo de la evolución física de estos mundos del espacio, en preparación para las aventuras de la vida en el tiempo.
\usection{7. LA ERA METEÓRICA: LA ERA VOLCÁNICA. LA ATMÓSFERA PLANETARIA PRIMITIVA}
\vs p057 7:1 Durante estos tiempos primitivos, las regiones espaciales del sistema solar se saturaron de pequeños cuerpos fragmentados y condensados, y, en ausencia del efecto protector por combustión de una atmósfera protectora, dichos cuerpos espaciales se estrellaban directamente sobre la superficie de Urantia. Estos constantes impactos mantenían la superficie del planeta más o menos recalentada, y ello, junto con la acción de la gravedad en aumento al hacerse más grande la esfera, empezó a poner en marcha aquellas influencias que gradualmente provocaron que los elementos más pesados, como el hierro, se asentaran cada vez más en el centro del planeta.
\vs p057 7:2 \pc Hace \bibemph{2\,000\,000\,000} de años la Tierra comenzó a superar con claridad a la Luna en tamaño. El planeta siempre había sido más grande que su satélite, pero no existía tanta diferencia entre ellos hasta ese momento durante el que la Tierra capturó enormes cuerpos espaciales. Urantia tenía entonces una quinta parte de su tamaño actual, y se había hecho lo bastante grande como para mantener la atmósfera primitiva que había comenzado a aparecer por efecto de la tensión interna y elemental entre su interior caliente y su corteza en proceso de enfriamiento.
\vs p057 7:3 La acción volcánica aparece claramente en estos tiempos. El calor interno de la Tierra continuaba aumentando a causa del soterramiento cada vez a mayor profundidad de elementos radiactivos o más pesados traídos del espacio por los meteoros. El estudio de estos elementos radiactivos revelará que Urantia en su superficie tiene más de mil millones de años. La medición con radio es vuestro cronómetro más fiable para hacer estimaciones científicas sobre la antigüedad del planeta; no obstante, cualquier cálculo de este orden queda demasiado corto porque los materiales radiactivos disponibles a vuestro análisis se obtienen de la superficie terrestre y, por consiguiente, su presencia en Urantia es relativamente reciente.
\vs p057 7:4 \pc Hace \bibemph{1\,500\,000\,000} de años, el tamaño de la Tierra era dos tercios del actual, en tanto que la Luna se aproximaba a la masa que tiene hoy en día. El haber superado rápidamente la Tierra a la Luna en tamaño permitió a esta ir substrayéndole lentamente la poca atmósfera que el satélite tenía originariamente.
\vs p057 7:5 La acción volcánica está en este momento en su punto álgido. Toda la Tierra viene a ser un auténtico infierno candente; su superficie, derretida, tiene parecido con su estado temprano antes de que los metales más pesados gravitaran hacia el centro. \bibemph{Se trata de la era volcánica}. No obstante, gradualmente, se estaba formando una corteza compuesta fundamentalmente de granito, relativamente más ligero. Se está estableciendo el escenario para que el planeta pueda algún día albergar la vida.
\vs p057 7:6 \pc La atmósfera primitiva del planeta está lentamente evolucionando; contiene ahora algún vapor de agua, monóxido de carbono, dióxido de carbono y cloruro de hidrógeno, pero hay poco o ningún nitrógeno libre u oxígeno libre. La atmósfera de un mundo en la era volcánica ofrece un raro espectáculo. Además de los gases mencionados, está sobrecargada de numerosos gases volcánicos y, a medida que el cinturón de aire se desarrolla, del producto de la combustión de las intensas lluvias meteóricas que se precipitan de forma constante sobre la superficie del planeta. Tal combustión meteórica mantiene el oxígeno atmosférico prácticamente agotado, y el ritmo del bombardeo meteórico sigue aún siendo enorme.
\vs p057 7:7 \pc En poco tiempo, la atmósfera se hizo más estable y se enfrió lo suficiente como para dar comienzo a la precipitación de lluvias sobre la candente superficie rocosa del planeta. Por miles de años, Urantia estuvo envuelta en un inmenso y continuo manto de vapor. Y, durante estas eras, nunca brilló el Sol sobre la superficie de la Tierra.
\vs p057 7:8 Gran parte del carbono de la atmósfera se separó para formar los carbonatos de los distintos metales que abundaban en los estratos superficiales del planeta. Más tarde, la prolífica vida vegetal primitiva consumió cantidades mucho mayores de estos gases carbónicos.
\vs p057 7:9 Incluso en los períodos posteriores, el continuo flujo de lava y los meteoros que caían mermaron casi por completo el oxígeno del aire. Incluso los primeros depósitos del océano primitivo que pronto surgiría no contenían rocas de colores ni esquistos. Y, durante mucho tiempo, tras la aparición de este océano, no hubo prácticamente oxígeno libre en la atmósfera; y no se llegó a producir en cantidades significativas hasta que las algas marinas y otras formas de vida vegetal lo generaran más tarde.
\vs p057 7:10 La atmósfera primitiva del planeta durante la era volcánica provee poca protección contra los impactos por colisión de los enjambres meteóricos. Millones y millones de ellos pueden penetrar este cinturón de aire y estrellarse contra la corteza planetaria en forma de cuerpos sólidos. Si bien, al pasar el tiempo, cada vez menos de ellos resultan lo bastante grandes como para poder atravesar el escudo de fricción, progresivamente más resistente, de la atmósfera rica en oxígeno de las eras posteriores.
\usection{8. LA ESTABILIZACIÓN DE LA CORTEZA TERRESTRE. LA ERA DE LOS TERREMOTOS, EL OCÉANO MUNDIAL Y EL PRIMER CONTINENTE.}
\vs p057 8:1 Hace \bibemph{1\,000\,000\,000} de años se marca la fecha en la que realmente da comienzo la historia de Urantia. El planeta había alcanzado casi su tamaño actual. Y por este tiempo se inscribió en los registros físicos de Nebadón y se le dio el nombre de \bibemph{Urantia}.
\vs p057 8:2 La atmósfera, junto a la incesante precipitación de humedad, facilitó el enfriamiento de la corteza terrestre. La acción volcánica pronto igualó a la presión del calor interno y la contracción de la corteza y, a medida que los volcanes se reducían rápidamente, y avanzaba esta época de enfriamiento y ajuste de la corteza, lo terremotos hicieron su aparición.
\vs p057 8:3 La verdadera historia geológica de Urantia comienza cuando la corteza terrestre se enfría lo suficiente como para dar lugar a la formación del primer océano. La condensación del vapor de agua existente sobre la superficie de la Tierra, en proceso de enfriamiento, una vez iniciada, continuó hasta que quedó virtualmente completa. Hacia el final de este período, el océano abarcaba todo el mundo, cubriendo por completo el planeta con una profundidad media de más de kilómetro y medio. Las mareas operaban de manera muy parecida a las de hoy en día, pero este océano primitivo no era salado, sino que prácticamente cubría el mundo de agua dulce. En esos días, la mayor parte del cloro se combinaba con diversos metales, pero era suficiente, en unión con el hidrógeno, para contribuir a que esta agua fuese ligeramente ácida.
\vs p057 8:4 Al inicio de esta lejana era, Urantia debe considerarse como un planeta dominado por el agua. Luego, flujos más profundos de lava y, por consiguiente más densos, surgieron del fondo del Océano Pacífico, y esta parte de la superficie cubierta de agua se hundió de manera considerable. La primera masa de tierra continental emergió del océano del mundo como ajuste y compensación del equilibrio de la corteza de la Tierra, que sufría un proceso paulatino de engrosamiento.
\vs p057 8:5 \pc Hace \bibemph{950\,000\,000} de años Urantia presenta la imagen de un único gran continente de tierra y una única gran extensión de agua: el Océano Pacífico. Los volcanes están aún bastante extendidos y los terremotos son frecuentes e intensos a la vez. Los meteoros continúan bombardeando la Tierra, aunque están disminuyendo tanto en frecuencia como en tamaño. La atmósfera se va despejando, pero la cantidad de dióxido de carbono continúa elevada. La corteza de la tierra está estabilizándose de manera gradual.
\vs p057 8:6 Fue aproximadamente en este momento en el que se asignó Urantia al sistema de Satania para su administración planetaria y se le inscribió en el registro de vida de Norlatiadek. Comenzó entonces el reconocimiento a nivel administrativo de la pequeña e insignificante esfera que estaba destinada a ser el planeta en el que Miguel acometería más tarde su formidable tarea de darse de gracia en la forma de una criatura mortal y de ser partícipe de esas experiencias que, desde entonces, han hecho que Urantia llegara a conocerse a nivel local como “el mundo de la cruz”.
\vs p057 8:7 \pc Hace \bibemph{900\,000\,000} de años se testimonió la llegada a Urantia del primer grupo de exploración enviado de Jerusem para examinar el planeta y realizar un informe respecto a su adaptación como sede de vida experimental. Esta comisión constaba de veinticuatro miembros en la que se incluían portadores de vida, hijos lanonandecs, melquisedecs, serafines y otros órdenes de vida celestial relacionados con la organización y la administración planetarias de los primeros días.
\vs p057 8:8 Tras haber realizado un concienzudo estudio del planeta, esta comisión regresó a Jerusem e informó favorablemente al soberano del sistema, recomendando que Urantia constara en el registro de vida experimental. Vuestro mundo, en consecuencia, quedó inscrito como planeta decimal, y se notificó a los portadores de vida que se les otorgaría permiso para establecer nuevos modelos de activación de orden mecánico, químico y eléctrico en el momento de su subsiguiente llegada con mandatos para trasplantar e implantar la vida.
\vs p057 8:9 En su momento, los preparativos para la ocupación del planeta se completaron por la comisión mixta de los doce de Jerusem y se aprobaron por la comisión planetaria de los setenta de Edentia. Estos planes, propuestos por los consejeros asesores de los portadores de vida, se aceptaron finalmente en Lugar de Salvación. Transcurrido poco tiempo, los servicios de transmisión de Nebadón difundían el anuncio de que Urantia sería el escenario en el que los portadores de vida llevarían a cabo, en Satania, su sexagésimo experimento, diseñado para ampliar y mejorar los modelos de vida de Nebadón relativos al tipo de Satania.
\vs p057 8:10 Poco después de haberse reconocido por primera vez a Urantia en las transmisiones del universo difundidas a todo Nebadón, se le confirió pleno estatus en el universo. Tras esto, quedó inscrita en los registros de los planetas sede de los sectores mayor y menor del suprauniverso; y, antes de finalizar esta era, Urantia constaba en el registro de vida planetaria de Uversa.
\vs p057 8:11 \pc En su totalidad, esta era se caracterizó por tormentas frecuentes y violentas. La corteza primitiva de la Tierra estaba en un estado de cambios continuos. El enfriamiento de su superficie alternaba con inmensos flujos de lava. En ninguna parte de esta puede hallarse nada relacionado con la corteza planetaria originaria. Todo se ha mezclado demasiadas veces con las lavas expulsadas desde sus profundos orígenes y se ha amalgamado con los depósitos posteriores procedentes del primitivo océano mundial.
\vs p057 8:12 En ninguna parte de la superficie del mundo se encontrarán más restos mutados de estas antiguas rocas pre\hyp{}oceánicas que en el nordeste de Canadá, alrededor de la Bahía de Hudson. Esta gran elevación de granito está compuesta de roca perteneciente a las eras pre\hyp{}oceánicas. Estas capas rocosas se han recalentado, doblado, torcido, aplastado, y han pasado muchas veces por estos estados de distorsiones metamórficas.
\vs p057 8:13 A lo largo de las eras oceánicas, enormes capas de roca estratificada sin fósiles se depositaron en este ancestral fondo oceánico. (La piedra caliza puede formarse como resultado de la precipitación química; no toda la piedra caliza de más antigüedad se formó a partir de los depósitos de vida marina.) En ninguna de estas antiguas formaciones rocosas se hallarán indicios de vida; no contienen fósiles salvo que, de manera fortuita, los depósitos posteriores de las eras acuáticas llegasen a mezclarse con estas capas más antiguas previas a la vida.
\vs p057 8:14 La corteza terrestre primitiva era altamente inestable, pero las montañas no estaban en proceso de formación. El planeta se contraía bajo la presión de la gravedad a medida que se iba formando. Las montañas no se originan como resultado del derrumbe de la corteza, en curso de enfriamiento, de una esfera en contracción, sino que aparecen más tarde por efecto de la acción de la lluvia, la gravedad y la erosión.
\vs p057 8:15 La masa de tierra continental de esta era aumentó hasta cubrir casi el diez por ciento de la superficie de la Tierra. Los intensos terremotos no comenzaron hasta que esta masa de tierra emergió bien por encima del nivel del agua. Una vez que comenzaron, se incrementaron en frecuencia e intensidad durante eras. Por millones y millones de años, los terremotos han ido disminuyendo, pero Urantia aún tiene una media de quince al día.
\vs p057 8:16 \pc Hace \bibemph{850\,000\,000} de años comenzó la primera época efectiva de estabilización de la corteza de la Tierra. La mayoría de los metales más pesados se habían asentado en el centro del globo; la corteza, en proceso de enfriamiento, había cesado de derrumbarse en tan grandes proporciones como en épocas anteriores. Se estableció un mejor equilibrio entre el afloramiento de la tierra y el lecho oceánico más pesado. El flujo de lava bajo la corteza se extendió prácticamente por todo el mundo, lo que compensó y estabilizó las fluctuaciones debidas al enfriamiento, a la contracción y a los deslizamientos de la superficie.
\vs p057 8:17 Las erupciones volcánicas y los terremotos continuaron disminuyendo en frecuencia e intensidad. La atmósfera se limpiaba de gases volcánicos y de vapor de agua, pero el porcentaje de dióxido de carbono era todavía alto.
\vs p057 8:18 Las perturbaciones eléctricas iban también disminuyendo en el aire y en la tierra. Los flujos de lava habían hecho aflorar a la superficie una mezcla de elementos que diversificó la corteza y aisló mejor al planeta de algunas energías espaciales. Y todo esto contribuyó bastante a facilitar el control de la energía terrestre y a regular su flujo, tal como se revela observando el funcionamiento de los polos magnéticos.
\vs p057 8:19 \pc Hace \bibemph{800\,000\,000} de años se evidenció el inicio de la primera gran época terrestre: la era de una mayor emersión continental.
\vs p057 8:20 En este punto, la hidrosfera de la Tierra se condensó primero en el océano mundial y después en el Océano Pacífico, esta última masa de agua cubría entonces las nueve décimas partes de la superficie de la tierra. Los meteoros que caían en el mar se acumulaban en el fondo oceánico ---los meteoros están compuestos, por lo general, de materiales pesados--- y los que caían en la tierra en gran medida se oxidaban, luego se desgastaban por la erosión y eran arrastrados a las cuencas oceánicas. Por lo tanto, el fondo del océano se hizo cada vez más pesado, a lo que había que añadir el peso de una masa de agua que en algunas partes tenía una profundidad de más de dieciséis kilómetros.
\vs p057 8:21 El creciente empuje hacia abajo del Océano Pacífico operó para impulsar más hacia arriba a la masa de tierra continental. Europa y África comenzaron a elevarse de las profundidades del Pacífico junto con las masas de tierra denominadas en la actualidad Australia, América del Norte y del Sur y el continente de la Antártida, mientras que en el lecho del Océano Pacífico se añadieron ajustes mediante hundimiento compensatorio. Hacia el final de este período casi un tercio de la superficie del planeta consistía en tierra, toda en un solo cuerpo continental.
\vs p057 8:22 Con este aumento de la elevación del suelo terrestre, aparecieron las primeras diferencias climáticas del planeta. Tal elevación, las nubes cósmicas y las influencias oceánicas constituyen los factores principales de la fluctuación del clima. El núcleo central de la masa de tierra asiática alcanzó una altura de más de catorce kilómetros en el momento de la máxima emersión del suelo. Si hubiese habido más humedad en el aire flotando sobre estas regiones tan elevadas, se habrían formado enormes mantos de hielo; la era glacial hubiese llegado mucho antes de lo que lo hizo. Transcurrieron algunos cientos de millones de años antes de que apareciera de nuevo tanta tierra por encima del agua.
\vs p057 8:23 \pc Hace \bibemph{750\,000\,000} de años comenzaron las primeras fracturas en la masa de tierra continental tal como el gran agrietamiento de norte\hyp{}sur, el cual luego dejó entrar las aguas del océano y preparó el camino para la deriva hacia el oeste de los continentes de América del Norte y del Sur, incluyendo Groenlandia. La extensa fisura este\hyp{}oeste separó a África de Europa y apartó del continente asiático a las masas de tierra de Australia, las Islas del Pacífico y la Antártida.
\vs p057 8:24 \pc Hace \bibemph{700\,000\,000} de años, Urantia se aproximaba a un grado de evolución que le hacía tener las condiciones adecuadas para albergar la vida. La deriva continental del suelo proseguía; cada vez más, los océanos penetraban en la tierra en forma de mares de largos brazos, proporcionando esas aguas poco profundas y bahías protegidas que tan aptas son para el hábitat de la vida marina.
\vs p057 8:25 \pc Hace \bibemph{650\,000\,000} de años se presenció otra separación de las masas de tierra y, como consecuencia, se produjo una nueva extensión de los mares continentales. Y estas aguas comenzaron a alcanzar rápidamente ese grado de salinidad tan indispensable para la vida en Urantia.
\vs p057 8:26 Fueron estos mares y sus sucesores los que determinaron los registros de vida de Urantia, tal como después se descubrió en páginas de piedra bien conservadas, volumen tras volumen, a medida que una era sucedía a la otra, y nacía una edad de la otra. Estos mares internos de tiempos antiguos fueron en verdad la cuna de la evolución.
\vsetoff
\vs p057 8:27 [Exposición de un portador de vida, miembro del colectivo originario enviado a Urantia y actualmente observador residente.]
