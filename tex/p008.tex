\upaper{8}{El Espíritu Infinito}
\author{Consejero divino}
\vs p008 0:1 Allá por la eternidad, cuando el “primer” pensamiento infinito y absoluto del Padre Universal halla en el Hijo Eterno un verbo tan perfecto y adecuado para su expresión divina, se sucede seguidamente el deseo supremo tanto del Dios pensamiento como del Dios verbo de tener un ser universal e infinito que manifieste su expresión mutua y su acción combinada.
\vs p008 0:2 En los albores de la eternidad, tanto el Padre como el Hijo se hacen infinitamente conocedores de su mutua interdependencia, de su eterna y absoluta unicidad, y, por consiguiente, celebran un pacto infinito y sempiterno de su vinculación divina. Este pacto sin fin se celebra para la realización de sus conceptos unidos a través de todo el círculo de la eternidad; y, a partir de este acontecimiento eterno, el Padre y el Hijo permanecen en esta unión divina.
\vs p008 0:3 Nos encontramos ahora frente a frente con el origen en la eternidad del Espíritu Infinito, de la Tercera Persona de la Deidad. En el instante mismo en el que el Dios Padre y el Dios Hijo conciben de forma conjunta una acción idéntica e infinita ---sustentar un plan de pensamiento absoluto---, en ese mismo momento, el Espíritu Infinito comienza a existir, ya en su completa totalidad.
\vs p008 0:4 \pc Expreso así el orden del origen de las Deidades únicamente para permitiros pensar en su relación. En la realidad, las tres existen desde la eternidad; son existenciales. No tienen ni principio ni fin en el tiempo; son iguales en rango, supremas, últimas, absolutas e infinitas. Son, siempre han sido y siempre serán tres personas claramente individualizadas pero eternamente vinculadas: el Dios Padre, el Dios Hijo y el Dios Espíritu.
\usection{1. EL DIOS DE ACCIÓN}
\vs p008 1:1 En la eternidad del pasado, al hacerse personal el Espíritu Infinito, el ciclo del ser personal divino se torna perfecto y completo. El Dios de Acción existe, y el inmenso escenario del espacio está preparado para la acción formidable de la creación ---la aventura universal---, el panorama divino de las eras eternas.
\vs p008 1:2 El primer acto del Espíritu Infinito es la observación y el reconocimiento de sus padres divinos, el Padre\hyp{}Padre y la Madre\hyp{}Hijo. Él, el Espíritu, se identifica de forma incondicional con ambos. Es plenamente conocedor por separado de sus personas y de sus atributos infinitos así como de su naturaleza combinada y de su labor conjunta. Luego, de forma voluntaria, con una disposición suprema y una alentadora espontaneidad, la tercera persona de la Deidad, a pesar de su igualdad con la primera y la segunda personas, promete eterna lealtad al Dios Padre y reconoce su perpetua dependencia del Dios Hijo.
\vs p008 1:3 Inherente a la naturaleza de este proceso y en reconocimiento mutuo de la independencia personal de cada uno y de la unión rectora de los tres, se establece el ciclo de la eternidad. La Trinidad del Paraíso existe. El escenario del espacio universal está listo para el panorama múltiple e ilimitado del despliegue creativo del propósito del Padre Universal, que se realza por medio de la persona del Hijo Eterno y mediante la intervención del Dios de Acción, a través de quien se lleva a cabo la actuación sobre la realidad de la colaboración creadora del Padre\hyp{}Hijo.
\vs p008 1:4 \pc El Dios de Acción obra y las inertes bóvedas del espacio se ponen en movimiento. Mil millones de esferas perfectas vienen a existir en un instante. Antes de este momento hipotético en la eternidad, las energías del espacio intrínsecas al Paraíso existen y son potencialmente operativas, pero no están actualizadas; ni tampoco puede medirse la gravedad física excepto por la reacción de las realidades materiales a su incesante atracción. No hay ningún universo material en este (supuesto) momento eternamente distante, pero en el mismo instante en que se materializan mil millones de mundos, se manifiesta la gravedad suficiente y adecuada para mantenerlos dentro de la perpetua atracción del Paraíso.
\vs p008 1:5 Centellea ahora por la creación de los Dioses la segunda forma de la energía, y este espíritu que efluye es de manera instantánea aprehendido por la gravedad espiritual del Hijo Eterno. Así pues, el universo, abarcado de forma doble por la gravedad, es rozado por la energía del infinito y se sumerge en el espíritu de la divinidad. De este modo se prepara el terreno de la vida para la conciencia de la mente, que se manifiesta en las vías circulatorias de la inteligencia vinculadas al Espíritu Infinito.
\vs p008 1:6 Sobre estas simientes de existencia potencial, difundidas a lo largo y ancho de la creación central de los Dioses, el Padre actúa y aparece el ser personal creatural. Luego, la presencia de las Deidades del Paraíso llena todo el espacio organizado y comienza de forma efectiva a atraer todas las cosas y seres hacia el Paraíso.
\vs p008 1:7 \pc El Espíritu Infinito se eterniza de forma simultánea al nacimiento de los mundos de Havona, este universo central que se crea por él y con él y en él en obediencia a los conceptos combinados y a las voluntades unidas del Padre y el Hijo. La tercera persona se hace Deidad por este mismo acto de creación conjunta, convirtiéndose así por siempre en el Creador Conjunto.
\vs p008 1:8 \pc Estos son los tiempos grandiosos y asombrosos de la expansión creadora del Padre y del Hijo por medio de la acción y en la acción de su colaborador conjunto y mandatario exclusivo, la Tercera Fuente y Centro. No hay ninguna constancia de estos agitados tiempos. Contamos tan solo con las escasas revelaciones del Espíritu Infinito para sustanciar estos portentosos procesos, y él simplemente verifica el hecho de que el universo central y todo lo que pertenece a este se eternizaron simultáneamente al lograr él su ser personal y su existencia consciente.
\vs p008 1:9 En resumen, el Espíritu Infinito demuestra que, puesto que él es eterno, así también el universo central es eterno. Y este es el punto de partida tradicional de la historia del universo de los universos. No se sabe nada en absoluto, y no existe documentación respecto a acontecimientos o procesos anteriores a esta formidable erupción de energía creativa y de sabiduría rectora que cristalizó el inmenso universo que existe, y que con tanta excelencia obra en el centro de todas las cosas. Más allá de este acontecimiento, se encuentran los inescrutables procesos de la eternidad y las profundidades del infinito, un absoluto misterio.
\vs p008 1:10 \pc Así pues, describimos el origen secuencial de la Tercera Fuente y Centro intentando ser condescendientes con las mentes de las criaturas mortales, sujetas al tiempo y condicionadas al espacio. La mente del hombre necesita tener un punto de partida para la visualización de la historia del universo, y se me ha instruido que proponga este método histórico de acceso al concepto de la eternidad. En la mente material, la lógica exige una Primera Causa; por tanto, consideramos que el Padre Universal es la Primera Fuente y Centro absoluta de toda la creación y, al mismo tiempo, instruimos a todas las mentes creaturales en el hecho de que el Hijo y el Espíritu son coeternos con el Padre en todas las fases de la historia universal y en todos los ámbitos de la actividad creativa. Lo hacemos sin dejar, en modo alguno, de respetar la realidad y la eternidad de la Isla del Paraíso y de los Absolutos Indeterminado, Universal y de la Deidad.
\vs p008 1:11 Es bastante con que la mente material de los hijos del tiempo alcance a concebir al Padre en la eternidad. Sabemos que un niño se relaciona mejor con la realidad si comprende primero las relaciones del núcleo padre\hyp{}hijo y luego ensancha este concepto hasta abarcar la familia como un todo. Posteriormente, la mente en formación del niño podrá asimilar el concepto de las relaciones familiares, de las relaciones de la comunidad, de la raza y del mundo y, luego, las del universo, del suprauniverso e incluso del universo de los universos.
\usection{2. LA NATURALEZA DEL ESPÍRITU INFINITO}
\vs p008 2:1 El Creador Conjunto pertenece a la eternidad y es total e incondicionalmente uno con el Padre Universal y con el Hijo Eterno. El Espíritu Infinito refleja en perfección no solo la naturaleza del Padre del Paraíso, sino también la del Hijo Primigenio.
\vs p008 2:2 \pc A la Tercera Fuente y Centro se le conoce por numerosos títulos: Espíritu Universal, Guía Suprema, Creador Conjunto, Mandatario Divino, Mente Infinita, Espíritu de los Espíritus, Espíritu Materno del Paraíso, Actor Conjunto, Coordinador Final, Espíritu Onmipresente, Inteligencia Absoluta, Acción Divina; y, en Urantia, algunas veces se le confunde con la mente cósmica.
\vs p008 2:3 Es del todo apropiado denominar Espíritu Infinito a la Tercera Persona de la Deidad, porque Dios es espíritu. Pero las criaturas materiales que cometen el error de considerar la materia como realidad básica y la mente, junto al espíritu, como postulados enraizados en la materia, comprenderían mejor la Tercera Fuente y Centro si se le llamara Realidad Infinita, Organizador Universal o Coordinador del Ser Personal.
\vs p008 2:4 \pc El Espíritu Infinito, como revelación universal de la divinidad, es inescrutable y está totalmente fuera del alcance de la comprensión humana. Para percibir la absolutidad del Espíritu solo necesitáis contemplar la infinitud del Padre Universal y maravillaros de la eternidad del Hijo Primigenio.
\vs p008 2:5 \pc Hay ciertamente un misterio en la persona del Espíritu Infinito, pero no tanto como en las personas del Padre y el Hijo. De todos los aspectos de la naturaleza del Padre, el Creador Conjunto es quien revela su infinitud de la manera más espectacular. Incluso si el universo matriz finalmente se expandiera a la infinitud, la presencia espiritual, la potestad sobre la energía y el potencial intelectual del Actor Conjunto serían adecuados para hacer frente a las demandas de semejante creación ilimitada.
\vs p008 2:6 Aunque comparte en todo sentido la perfección, la rectitud y el amor del Padre Universal, el Espíritu Infinito manifiesta una inclinación hacia los atributos de la misericordia del Hijo Eterno, convirtiéndose así en el benefactor de la misericordia de las Deidades del Paraíso para el gran universo. Desde siempre y por siempre ---de forma universal y eterna--- el Espíritu es este benefactor porque, así como los hijos divinos revelan el amor de Dios, el Espíritu Divino representa la misericordia de Dios.
\vs p008 2:7 No es posible que el Espíritu pudiera tener más bondad que el Padre, puesto que toda bondad se origina en el Padre, pero en las acciones del Espíritu podemos comprender mejor dicha bondad. La fidelidad del Padre y la constancia del Hijo se hacen muy reales para los seres espirituales y las criaturas materiales de las esferas, mediante el ministerio amoroso y el servicio incesante de los seres personales del Espíritu Infinito.
\vs p008 2:8 El Creador Conjunto hereda toda la belleza de pensamiento y el carácter genuino del Padre. Mas estos sublimes rasgos de la divinidad se coordinan en los niveles casi supremos de la mente cósmica, los cuales se subordinan a la sabiduría eterna e infinita de la mente incondicionada e ilimitada de la Tercera Fuente y Centro.
\usection{3. LA RELACIÓN DEL ESPÍRITU CON EL PADRE Y EL HIJO}
\vs p008 3:1 Así como el Hijo Eterno es la expresión verbal del “primer” pensamiento absoluto e infinito del Padre Universal, el Actor Conjunto es el cumplimiento perfecto del “primer” concepto o plan creativo completo para la acción combinada de la colaboración de las personas del Padre\hyp{}Hijo, con unión absoluta de pensamiento y verbo. La Tercera Fuente y Centro se eterniza de forma simultánea al decreto de la creación central, y, entre los universos, solo esta creación central tiene existencia eterna.
\vs p008 3:2 Desde que la Tercera Fuente se hizo personal, la Primera Fuente ya no participa personalmente en la creación del universo. El Padre Universal delega todo lo posible en su Hijo Eterno; asimismo, el Hijo Eterno otorga toda la autoridad y poder posibles al Creador Conjunto.
\vs p008 3:3 El Hijo Eterno y el Creador Conjunto, como compañeros, han planeado e ideado a través de sus seres personales homólogos, todos los universos que han surgido con posterioridad a Havona. En todas las creaciones posteriores, el Espíritu mantiene con el Hijo la misma relación personal que el Hijo mantiene con el Padre en la primigenia creación central.
\vs p008 3:4 Un hijo creador del Hijo Eterno y un espíritu creativo del Espíritu Infinito os crearon a vosotros y crearon vuestro universo y, mientras que el Padre sostiene fielmente lo que ha organizado, incumbe a este hijo del universo y a este espíritu del universo impulsar y sustentar su obra así como servir a sus criaturas.
\vs p008 3:5 \pc El Espíritu Infinito representa efectivamente al Padre omniamante y al Hijo omnimisericordioso en el propósito de realizar su proyecto conjunto de atraer hacia ellos a todas las almas amantes de la verdad de todos los mundos del tiempo y del espacio. En el mismo instante en el que el Hijo Eterno aceptó el plan de su Padre para que las criaturas de los universos alcanzaran la perfección, en el momento en que el proyecto de ascensión se convirtió en un plan del Padre\hyp{}Hijo, en ese instante el Espíritu Infinito se erigió como administrador conjunto al Padre y al Hijo para el cumplimiento de su propósito unido y eterno. Y, al hacerlo, el Espíritu Infinito, de este modo, ofrece al Padre y al Hijo todos sus recursos en cuanto a presencia divina y seres personales espirituales; él lo ha dedicado \bibemph{todo} al formidable plan de elevar a las alturas divinas de la perfección del Paraíso a las criaturas de voluntad supervivientes.
\vs p008 3:6 El Espíritu Infinito es una revelación completa, exclusiva y universal del Padre Universal y de su Hijo Eterno. Todo conocimiento de la colaboración Padre\hyp{}Hijo debe obtenerse a través del Espíritu Infinito, el representante conjunto de la divina unión del verbo\hyp{}pensamiento.
\vs p008 3:7 El Hijo Eterno es el único camino de acceso al Padre Universal, y el Espíritu Infinito es el único medio de alcanzar al Hijo Eterno. Los seres ascendentes del tiempo solo pueden descubrir al Hijo mediante el paciente ministerio del Espíritu.
\vs p008 3:8 En el centro de todas las cosas, el Espíritu Infinito es la primera de las Deidades del Paraíso que alcanzan los peregrinos ascendentes. La Tercera Persona envuelve a la Segunda y a la Primera Persona y, por tanto, siempre debe ser reconocida primero por todos los que eligen presentarse ante el Hijo y su Padre.
\vs p008 3:9 Y de muchas otras maneras el Espíritu igualmente representa y sirve al Padre y al Hijo.
\usection{4. EL ESPÍRITU DEL MINISTERIO DIVINO}
\vs p008 4:1 Paralelamente al universo físico donde la gravedad del Paraíso mantiene todas las cosas juntas, existe el universo espiritual en el que la palabra del Hijo interpreta el pensamiento de Dios y, cuando “se hace carne”, demuestra la amorosa misericordia de la naturaleza combinada de los creadores conjuntos. Pero en toda esta creación material y espiritual, y a través de ella, hay un inmenso entorno en el que el Espíritu Infinito y su progenie espiritual manifiestan la misericordia, la paciencia y el afecto perpetuo combinados de los padres divinos hacia los hijos de inteligencia que ellos han concebido y creado colaborando entre sí. El ministerio perpetuo de la mente es la esencia del carácter divino del Espíritu. Toda la descendencia espiritual del Creador Conjunto participa de este deseo de ofrecer su ministerio, de este impulso divino a servir.
\vs p008 4:2 Dios es amor, el Hijo es misericordia, el Espíritu es ministerio: ministerio de amor divino y de misericordia sin fin para toda la creación inteligente. El Espíritu es la personificación del amor del Padre y de la misericordia del Hijo; en él ellos están eternamente unidos para el servicio universal. El Espíritu es \bibemph{amor aplicado} a la creación creatural, el amor combinado del Padre y el Hijo.
\vs p008 4:3 En Urantia el Espíritu Infinito se conoce como influencia omnipresente, como presencia universal, pero en Havona vosotros lo conoceréis como una presencia personal que presta un servicio real. Aquí el ministerio del Espíritu del Paraíso es el modelo ejemplar e inspirador de cada uno de sus espíritus homólogos y de seres personales de menor rango que sirven a los seres creados en los mundos del tiempo y del espacio. En este universo divino, el Espíritu Infinito participó plenamente en las siete apariciones trascendentales del Hijo Eterno; asimismo participó con el hijo miguel primigenio en sus siete ministerios de gracia en las vías circulatorias de Havona, erigiéndose así como el benefactor espiritual, comprensivo y compasivo, de todos los peregrinos del tiempo que cruzan estos círculos perfectos de las alturas.
\vs p008 4:4 \pc Cuando un hijo creador de Dios acepta hacerse cargo de la responsabilidad de un universo local en proyecto, los seres personales del Espíritu Infinito se comprometen a servir incansablemente a este hijo miguel en la aventura creativa de su misión. Especialmente en las personas de las hijas creativas, los espíritus maternos del universo local, encontramos al Espíritu Infinito dedicado a la labor de impulsar la ascensión de las criaturas materiales hacia niveles de realización espiritual cada vez más elevados. Todo este ministerio de servicio a las criaturas se lleva a cabo en perfecta armonía con los propósitos de los hijos creadores de estos universos locales y en íntima colaboración con sus seres personales.
\vs p008 4:5 Así como los Hijos de Dios emprenden la enorme tarea de revelar al universo el amoroso ser personal del Padre, el Espíritu Infinito se dedica a la interminable tarea de revelar el amor combinado del Padre y del Hijo a las mentes individuales de los hijos de cada uno de los universos. En estas creaciones locales, el Espíritu no desciende a las razas materiales con la semejanza de un hombre mortal como lo hacen algunos de los Hijos de Dios, sino que el Espíritu Infinito y sus espíritus de igual rango descienden, sometiéndose con júbilo a una serie sorprendente de atenuaciones de su divinidad, hasta aparecer como ángeles para estar a vuestro lado y guiaros por las sendas más humildes de la existencia terrenal.
\vs p008 4:6 Por esta misma secuencia decreciente, el Espíritu Infinito realmente, y como persona, se acerca bastante a todos los seres de las esferas de origen animal. Y el Espíritu hace todo esto sin invalidar en lo más mínimo su existencia como Tercera Persona de la Deidad en el centro de todas las cosas.
\vs p008 4:7 \pc El Creador Conjunto es verdaderamente, y para siempre, el gran espíritu servidor de las criaturas, el benefactor de la misericordia universal. Para comprender el ministerio del Espíritu, ponderad la verdad de que él es la imagen combinada del amor infinito del Padre y de la eterna misericordia del Hijo. Sin embargo, este ministerio no se limita a representar solamente al Hijo Eterno y al Padre Universal. El Espíritu Infinito posee también la facultad de impartir su ministerio a las criaturas del mundo a título personal; la Tercera Persona tiene dignidad divina y otorga también, en su propio nombre, el ministerio universal de la misericordia.
\vs p008 4:8 A medida que el hombre aprende más sobre el ministerio amoroso e incansable de los órdenes menores de seres descendientes de este Espíritu Infinito, tanto más admirará y adorará la naturaleza suprema y el carácter sin igual de esta Acción combinada del Padre Universal y del Hijo Eterno. En verdad, este Espíritu es “los ojos del Señor que siempre están sobre los justos” y “los divinos oídos que siempre están abiertos a sus oraciones”.
\usection{5. LA PRESENCIA DE DIOS}
\vs p008 5:1 El atributo más notable del Espíritu Infinito es la omnipresencia. Por todo el universo de los universos está siempre presente este espíritu que todo lo infunde y que es tan similar a la presencia de una mente universal y divina. Tanto la Segunda Persona como la Tercera Persona de la Deidad están representadas en todos los mundos por sus espíritus omnipresentes.
\vs p008 5:2 El Padre es \bibemph{infinito} y, por tanto, solo la volición lo limita. Al otorgar los modeladores y encauzar el ser personal, el Padre actúa solo, pero en el contacto de las fuerzas espirituales con los seres inteligentes, hace uso de los espíritus y de los seres personales del Hijo Eterno y del Espíritu Infinito. Él, de su propia voluntad, está espiritualmente presente de igual forma con el Hijo o con el Actor Conjunto; está presente \bibemph{con} el Hijo y \bibemph{en} el Espíritu. El Padre, con toda seguridad, está presente en todas partes, y discernimos su presencia por y a través de todas estas diversas, pero vinculadas, fuerzas, influencias y presencias.
\vs p008 5:3 \pc En vuestras escrituras sagradas el término \bibemph{Espíritu de Dios} parece usarse de forma indistinta para designar tanto al Espíritu Infinito del Paraíso como al espíritu creativo de vuestro universo local. El espíritu santo es la vía circulatoria espiritual de esta hija creativa del Espíritu Infinito del Paraíso. El espíritu santo constituye una vía que es propia a cada universo local y tiene sus confines en el ámbito espiritual de esa creación; pero el Espíritu Infinito es omnipresente.
\vs p008 5:4 \pc Existen muchas influencias espirituales, y todas son como \bibemph{una}. Incluso la labor de los modeladores del pensamiento, aunque independiente de todas las otras influencias, coincide invariablemente con el ministerio espiritual de las influencias combinadas del Espíritu Infinito y del espíritu materno del universo local. Estas presencias espirituales no son perceptibles en la vida de los habitantes de Urantia. En vuestras mentes y sobre vuestras almas obran como un espíritu, a pesar de la diversidad de sus orígenes. Y conforme se vivencia este ministerio espiritual unido, se convierte para vosotros en la influencia del Supremo, “quien siempre puede libraros de flaquezas y presentaros intachables ante vuestro Padre en las alturas”.
\vs p008 5:5 Recordad siempre que el Espíritu Infinito es el Actor \bibemph{Conjunto;} tanto el Padre como el Hijo obran en él y a través de él; él está presente no solo como él mismo sino también como Padre y como Hijo y como Padre\hyp{}Hijo. En reconocimiento de esto y por muchas otras razones adicionales, a la presencia del Espíritu Infinito se la llama a menudo “el espíritu de Dios”.
\vs p008 5:6 Sería lógico también aludir a este espíritu de Dios como el nexo de unión de todo ministerio espiritual, porque tal nexo es verdaderamente la unión de los espíritus del Dios Padre, del Dios Hijo, del Dios Espíritu y del Dios Séptuplo ---e incluso del espíritu del Dios Supremo---.
\usection{6. EL SER PERSONAL DEL ESPÍRITU INFINITO}
\vs p008 6:1 No dejéis que el gran don y la inmensa distribución de la Tercera Fuente y Centro nublen u os distraigan del hecho de su ser personal. El Espíritu Infinito es una presencia universal, una acción eterna, un poder cósmico, una influencia santa y una mente universal; es todas estas cosas e infinitamente más, pero es también un ser personal verdadero y divino.
\vs p008 6:2 El Espíritu Infinito es un ser personal completo y perfecto, el divino equivalente e igual en rango del Padre Universal y del Hijo Eterno. El Creador Conjunto es tan real y visible para las inteligencias más elevadas de los universos como lo son el Padre y el Hijo; e incluso más, porque es el Espíritu a quien todos los ascendentes deben alcanzar antes de aproximarse al Padre, a través del Hijo.
\vs p008 6:3 El Espíritu Infinito, la Tercera Persona de la Deidad, es poseedor de todos los atributos que vosotros asociáis con el ser personal. El Espíritu está dotado de mente absoluta: “El Espíritu descubre todas las cosas, incluso las cosas profundas de Dios”. El Espíritu no está solamente dotado de mente, sino también de voluntad. De la dádiva de sus dones está dicho: “Pero todas estas cosas las hace uno y el mismo Espíritu, repartiendo a cada uno en particular como él quiere”.
\vs p008 6:4 “El amor del Espíritu” es real, como también lo son sus pesares; por tanto “no contristéis al Espíritu de Dios”. Ya sea que contemplemos al Espíritu Infinito como Deidad del Paraíso o como espíritu creativo del universo local, hallamos que el Creador Conjunto no es solamente la Tercera Fuente y Centro sino también una persona divina. Este ser personal divino también reacciona como persona ante el universo. El Espíritu os habla: “El que tenga oídos, que escuche lo que dice el Espíritu”. “El Espíritu mismo intercede por vosotros”. El Espíritu ejerce una influencia directa y personal sobre los seres creados, “porque todos los que son guiados por el Espíritu de Dios, estos son los hijos de Dios”.
\vs p008 6:5 Aunque contemplemos la acción del ministerio del Espíritu Infinito en los mundos remotos del universo de los universos, aunque concibamos a esta misma Deidad coordinadora actuando en y mediante indecibles multitudes de seres que tienen su origen en la Tercera Fuente y Centro, aunque reconozcamos la omnipresencia del Espíritu, seguimos, no obstante, afirmando que esta misma Tercera Fuente y Centro es una persona, el Creador Conjunto de todas las cosas y de todos los seres y de todos los universos.
\vs p008 6:6 \pc En la administración de los universos, el Padre, el Hijo y el Espíritu están perfecta y eternamente correlacionados. Aunque cada uno está consagrado al ministerio personal para toda la creación, los tres se entrelazan divina y absolutamente al realizar un servicio de creación y de dirección que por siempre los hace uno.
\vs p008 6:7 En la persona del Espíritu Infinito, el Padre y el Hijo están mutuamente presentes, siempre y en perfección incondicionada, porque el Espíritu es semejante al Padre y semejante al Hijo, y semejante también al Padre y al Hijo ya que ellos dos son eternamente uno.
\vsetoff
\vs p008 6:8 [Exposición en Urantia de un consejero divino de Uversa encargado por los ancianos de días de describir la naturaleza y la labor del Espíritu Infinito.]
