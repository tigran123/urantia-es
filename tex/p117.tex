\upaper{117}{El Dios Supremo}
\author{Mensajero poderoso}
\vs p117 0:1 A medida que hacemos la voluntad de Dios en cualquiera de los lugares del universo en los que tengamos nuestra existencia, en esa misma medida, el potencial todopoderoso del Supremo se actualiza un paso más. La voluntad de Dios es el propósito de la Primera Fuente y Centro tal como se potencializa en los tres Absolutos, se manifiesta personalmente en el Hijo Eterno, se conjunta con el Espíritu Infinito para obrar en el universo y se eterniza en los sempiternos modelos del Paraíso. Y el Dios Supremo está llegando a ser la más elevada manifestación finita de la voluntad total de Dios.
\vs p117 0:2 Si todos los moradores del gran universo lograran, relativamente, vivir con plenitud alguna vez la voluntad de Dios, entonces, las creaciones espacio\hyp{}temporales se asentarían en luz y vida, y el Todopoderoso, el potencial en cuanto deidad de la Supremacía se hará efectivo con la gradual aparición del ser personal divino del Dios Supremo.
\vs p117 0:3 Cuando una mente evolutiva llega a estar en sintonía con las vías circulatorias de la mente cósmica, cuando un universo evolutivo se estabiliza conforme al modelo del universo central, cuando, en su avance, un espíritu entra en contacto con el ministerio unido de los espíritus mayores, cuando un ser personal ascendente finalmente se sintoniza con la guía divina del modelador interior, entonces la actualidad del Supremo se hace un grado más real en los universos; entonces la divinidad de la Supremacía avanza un paso más hacia su realización cósmica.
\vs p117 0:4 Las partes y las individualidades del gran universo evolucionan como reflejo de la evolución total del Supremo, mientras que, a su vez, el Supremo sintetiza el total acumulado de toda la evolución del gran universo. Desde la perspectiva humana, ambos son reciprocidades evolutivas y experienciales.
\usection{1. NATURALEZA DEL SER SUPREMO}
\vs p117 1:1 El Supremo es la belleza de la armonía física, la verdad del significado intelectual y la bondad del valor espiritual. Él es la dulzura del verdadero éxito y el gozo del logro perdurable. Él es la sobrealma del gran universo, la conciencia del cosmos finito, lo compleción de la realidad finita y la manifestación personal de la experiencia del Creador\hyp{}criatura. Por toda la eternidad futura, el Dios Supremo proclamará la realidad de la experiencia volitiva en las relaciones trinitarias de la Deidad.
\vs p117 1:2 \pc En las personas de los creadores supremos, los Dioses han descendido del Paraíso a los ámbitos del tiempo y del espacio, para crear y hacer evolucionar allí a criaturas con capacidad para alcanzar el Paraíso y que puedan ascender hasta allí en busca del Padre. Esta procesión en el universo de creadores que descienden para revelar a Dios y de criaturas que ascienden para buscar a Dios desvela la evolución de la Deidad del Supremo, en la que tanto los que descienden como los que ascienden logran un mutuo entendimiento, el descubrimiento de la hermandad eterna y universal. El Ser Supremo de este modo se vuelve la síntesis finita de la experiencia de la causa iniciada por el Creador perfecto y de la respuesta de la criatura en vías de perfección.
\vs p117 1:3 El gran universo alberga la posibilidad, y siempre está en búsqueda, de una completa unificación, y esto se origina del hecho de que la existencia cósmica es consecuencia de los actos creativos y de los mandatos en cuanto a la potencia de la Trinidad del Paraíso, que es unidad incondicionada. Esta misma unidad trinitaria se expresa, en el cosmos finito, en el Supremo, cuya realidad se hace crecientemente patente a medida que los universos alcanzan su máximo nivel de identificación con la Trinidad.
\vs p117 1:4 \pc La voluntad del Creador y la voluntad de la criatura son cualitativamente diferentes, aunque también experiencialmente semejantes, ya que la criatura y el Creador pueden colaborar en el logro de la perfección del universo. El hombre puede obrar en cooperación con Dios y, en consecuencia, cocrear un finalizador eterno. Dios puede obrar incluso como humanidad en las encarnaciones de sus hijos del Paraíso, que, de ese modo, alcanzan la supremacía de la experiencia creatural.
\vs p117 1:5 En el Ser Supremo, el Creador y la criatura se unen en una Deidad cuya voluntad es expresión de un ser personal divino. Y esta voluntad del Supremo es algo más que la voluntad de la criatura o del Creador, así como la voluntad soberana del hijo mayor de Nebadón es algo más que la combinación de las voluntades divina y humana. La unión de la perfección del Paraíso y de la experiencia espacio\hyp{}temporal genera un nuevo valor significativo en los niveles de la realidad en cuanto deidad.
\vs p117 1:6 La naturaleza divina evolutiva del Supremo se está convirtiendo en una representación fiel de la inigualable experiencia de todas las criaturas y de todos los creadores del gran universo. En el Supremo, condición de creador y estado creatural son una misma cosa; están por siempre unidos por esa experiencia que nace de las vicisitudes conexas a la solución de los múltiples problemas que acucian a toda creación finita, mientras prosigue por el eterno sendero en búsqueda de la perfección y de la liberación de las ataduras de la imperfección.
\vs p117 1:7 \pc La verdad, la belleza y la bondad están correlacionadas en el ministerio del Espíritu, en la grandeza del Paraíso, en la misericordia del Hijo y en la experiencia del Supremo. El Dios Supremo \bibemph{es} verdad, belleza y bondad, porque estas ideas de la divinidad representan máximos finitos de la experiencia conceptual. Las fuentes eternas de estas cualidades trinas de la divinidad están en los niveles suprafinitos, pero una criatura solamente puede concebir tales fuentes como supraverdad, suprabelleza y suprabondad.
\vs p117 1:8 Miguel, creador, reveló el amor divino del Padre Creador por sus hijos terrenales. Y habiendo descubierto y recibido este cariño divino, el hombre puede aspirar a revelar dicho amor a sus hermanos en la carne. Este afecto que siente la criatura es un verdadero reflejo del amor del Supremo.
\vs p117 1:9 El Supremo muestra una naturaleza bien proporcionada. La Primera Fuente y Centro es potencial en los tres grandes Absolutos, es actual en el Paraíso, en el Hijo y en el Espíritu; pero el Supremo es actual y potencial a la vez, es un ser de supremacía personal y de potencia todopoderosa, receptivo por igual tanto al esfuerzo de la criatura como al propósito del Creador; actúa en sí mismo sobre el universo y reacciona a la suma total del universo; y es, al mismo tiempo, creador supremo y criatura suprema. La Deidad de la Supremacía expresa, pues, la suma total de la totalidad de lo finito.
\usection{2. LA FUENTE DEL CRECIMIENTO EVOLUTIVO}
\vs p117 2:1 El Supremo es Dios en el tiempo; suyo es el secreto del crecimiento de la criatura en el tiempo; suya es también la conquista del presente incompleto y la consumación del futuro en vías de perfección. Y los frutos finales de todo crecimiento finito son: la potencia regida por el espíritu mediante la mente en virtud de la presencia unificadora y creativa del ser personal. La consecuencia culminante de todo este crecimiento es el Ser Supremo.
\vs p117 2:2 Para el hombre mortal, existencia equivale a crecimiento. Y, de hecho, parecería ser así, incluso en el más amplio sentido del universo, porque una existencia guiada por el espíritu parece ciertamente dar lugar a un crecimiento experiencial ---un aumento de estatus---. Durante mucho tiempo, sin embargo, hemos sostenido que el crecimiento actual que caracteriza la existencia de la criatura en la presente era del universo es labor del Supremo. Igualmente, sostenemos que este tipo de crecimiento es propio de la era del crecimiento del Supremo, y que terminará al completarse tal crecimiento,
\vs p117 2:3 \pc Considerad el estatus de los hijos trinitizados por criaturas: nacen y viven en la presente era del universo; tienen ser personal, además de estar dotados de mente y de espíritu. Poseen experiencias y memoria de las mismas, pero no \bibemph{crecen} como los ascendentes. Es nuestro entender que estos hijos trinitizados por criaturas, aunque se encuentran \bibemph{en} esta era del universo, pertenecen en realidad \bibemph{a} la próxima ---a la era que seguirá al finalizar el crecimiento del Supremo---. Así pues, no están \bibemph{en} el Supremo en su actual estatus de incompletitud y consiguiente crecimiento. No participan, pues, del crecimiento experiencial de la actual era, sino que se mantienen en reserva para la próxima.
\vs p117 2:4 Mi propio orden, los mensajeros poderosos, al ser acogido por la Trinidad, no participa en el crecimiento de la presente era del universo. En cierto sentido, nuestro estatus es el perteneciente a la anterior era, como lo son de hecho los hijos estacionarios de la Trinidad. Hay una cosa cierta: nuestro estatus es fijo al estar acogidos por la Trinidad, y la experiencia ya no deviene en crecimiento.
\vs p117 2:5 Esto no ocurre en el caso de los finalizadores ni con ningún otro orden evolutivo y experiencial que sea partícipe del proceso de crecimiento del Ser Supremo. Vosotros los mortales que vivís ahora en Urantia y que podéis aspirar a conseguir el Paraíso y el estatus de finalizadores deberíais entender que dicho destino solo es viable porque estáis en el Supremo y sois de él, por ello participáis en el ciclo del crecimiento del Supremo.
\vs p117 2:6 \pc En algún momento, llegará a su fin el crecimiento del Supremo; su estatus logrará completarse (en el sentido de la energía\hyp{}espíritu). Esta conclusión evolutiva del Supremo también será testigo del final de la evolución de la criatura como parte de la Supremacía. No sabemos qué tipo de crecimiento podrá caracterizar a los universos del espacio exterior. Pero estamos muy seguros de que será algo muy diferente a todo lo que se ha visto en la era presente de la evolución de los siete suprauniversos. Sin duda, será tarea de los ciudadanos evolutivos del gran universo compensar a los moradores del espacio exterior por esta privación del crecimiento de la Supremacía.
\vs p117 2:7 En cuanto a su existencia en el momento de la consumación de la presente era del universo, el Ser Supremo obrará como soberano experiencial en el gran universo. Los moradores del espacio exterior ---ciudadanos de la próxima era del universo--- tendrán una potencialidad de crecimiento postrero al de los suprauniversos, una capacidad de alcance evolutivo que presupondrá la soberanía del Todopoderoso Supremo, quedando por ello excluida la participación de las criaturas en la síntesis de la potencia\hyp{}personalidad de la presente era del universo.
\vs p117 2:8 Por lo tanto, el estatus incompleto del Supremo se puede considerar una virtud, puesto que posibilita el crecimiento evolutivo de la criatura\hyp{}creación de los actuales universos. El vacío tiene ciertamente sus bondades, porque se puede colmar de forma experiencial.
\vs p117 2:9 \pc Una de las preguntas más fascinantes de la filosofía finita es la siguiente: ¿Se actualiza el Ser Supremo como respuesta a la evolución del gran universo, o evoluciona este cosmos finito progresivamente en respuesta a la actualización gradual del Supremo? O, ¿es posible que sean mutuamente interdependientes en su desarrollo?, ¿que sean reciprocidades evolutivas, cada cual iniciando el crecimiento del otro? Estamos convencidos de algo: las criaturas y los universos, de mayor o menor rango, están evolucionando en el seno del Supremo, y, a medida que evolucionan, va haciendo su aparición la suma unificada de toda la actividad finita de esta era del universo. Y esta es la aparición del Ser Supremo, que para todos los seres personales, se trata de la evolución de la potencia todopoderosa del Dios Supremo.
\usection{3. EL SIGNIFICADO DEL SUPREMO PARA LAS CRIATURAS DEL UNIVERSO}
\vs p117 3:1 Esa realidad cósmica que, indistintamente se designa como Ser Supremo, Dios Supremo y Todopoderoso Supremo, es la síntesis compleja y universal de las facetas emergentes de todas las realidades finitas. La extensa diversificación de la energía eterna, del espíritu divino y de la mente universal alcanza su punto culminante finito en la evolución del Supremo, que es la suma total de todo el crecimiento finito, realizado en sí mismo en los niveles en cuanto deidad de la máxima compleción de lo finito.
\vs p117 3:2 El Supremo es el canal divino por el que fluye la infinitud creativa de las triodidades que se cristaliza en el panorama galáctico del espacio, en el que tiene lugar la magnífica actuación del ser personal en el tiempo: la conquista espiritual de la energía\hyp{}materia por mediación de la mente.
\vs p117 3:3 \pc Dijo Jesús: “Yo soy el camino vivo”, y de cierto él es el camino vivo desde el nivel material de la autoconciencia hasta el nivel espiritual de la conciencia de Dios. Y así como él es el camino vivo de ascensión desde el yo hasta Dios, de igual modo el Supremo es el camino vivo desde la conciencia finita hasta la trascendencia de la conciencia, incluso hasta la percepción de lo absonito.
\vs p117 3:4 Vuestro hijo creador puede ser realmente tal cauce vivo desde la humanidad hasta la divinidad, dado que ha experimentado personalmente y en su plenitud la travesía por este sendero universal de progreso, desde la verdadera humanidad de Josué ben José, el Hijo del Hombre, hasta la divinidad del Paraíso de Miguel de Nebadón, el hijo del Dios infinito. Asimismo, el Ser Supremo puede obrar en el universo como vía de acceso para trascender las limitaciones finitas, porque él es la personificación genuina y el epítome personal de toda evolución, avance y espiritualización creatural. En el gran universo, incluso las experiencias de los seres personales descendentes del Paraíso constituyen esa parte de su experiencia que complementa la suma de las experiencias de ascensión de los peregrinos del tiempo.
\vs p117 3:5 \pc El hombre mortal está más que figurativamente hecho a imagen de Dios. Desde una perspectiva física, esta afirmación es difícilmente verdad, pero en relación a ciertas potencialidades dentro del ámbito del universo es un hecho real. En la raza humana, se está desplegando algo del mismo acontecimiento de logro evolutivo que tiene lugar, a una escala inmensamente más grande, en el universo de los universos. El hombre, un ser personal volitivo, se vuelve creativo en colaboración con un modelador, una entidad impersonal, en la presencia de las potencialidades finitas del Supremo, y el resultado es el florecimiento de un alma inmortal. En los universos, las personas creadoras del tiempo y del espacio obran en colaboración con el espíritu impersonal de la Trinidad del Paraíso y se vuelven por tanto capaces de crear un nuevo potencial de la potencia de la realidad de Deidad.
\vs p117 3:6 El hombre mortal, siendo una criatura, no es exactamente como el Ser Supremo, que es deidad, pero la evolución del hombre se asemeja al crecimiento del Supremo de alguna manera. El hombre crece conscientemente desde lo material hacia lo espiritual mediante la fuerza, el poder y la determinación de sus propias decisiones; también crece a medida que su modelador del pensamiento desarrolla nuevas formas de descender desde el nivel espiritual hasta el morontial del alma; y una vez que el alma llega a existir, comienza a crecer en sí misma y por sí misma.
\vs p117 3:7 Esto de algún modo es la manera en la que el Ser Supremo se expande. Su soberanía crece en y a partir de los hechos y los logros de los seres personales creadores supremos; tal es la evolución de la majestad de su poder como gobernante del gran universo. Su naturaleza en cuanto deidad es asimismo dependiente de la unidad preexistente de la Trinidad del Paraíso. Pero hay todavía otro aspecto en la evolución del Dios Supremo: no solo evoluciona a partir de los creadores y se deriva de la Trinidad, sino que también evoluciona por sí mismo y se deriva de sí mismo. El mismo Dios Supremo es partícipe volitivo y creativo de su propia actualización en cuanto deidad. El alma humana morontial es igualmente acompañante volitivo y cocreativo de su propia inmortalización.
\vs p117 3:8 \pc El Padre colabora con el Actor Conjunto en su actuación sobre las energías del Paraíso para hacerlas receptivas al Supremo. El Padre colabora con el Hijo Eterno para dar nacimiento a seres personales creadores cuyos actos culminarán en algún momento en la soberanía del Supremo. El Padre colabora tanto con el Hijo como con el Espíritu en la creación de las personas de la Trinidad para que obren como gobernantes del gran universo hasta ese momento en que se complete la evolución del Supremo y se califique para asumir esa soberanía. El Padre coopera con sus correlatos, Deidades y no Deidades, de estas y de otras muchas formas, en el progreso evolutivo de la Supremacía, pero también obra por sí mismo en estas cuestiones. Y, probablemente, la mejor manera en la que se revela esta labor solitaria es en el ministerio de los modeladores del pensamiento y de sus entidades afines.
\vs p117 3:9 La Deidad es unidad, existencial en la Trinidad, experiencial en el Supremo y, en los mortales, se realiza en la criatura al fusionarse con el modelador. La presencia de los modeladores del pensamiento en el hombre mortal revela la unidad esencial del universo, puesto que el hombre, el orden más modesto de ser personal del universo, alberga dentro de sí una fracción real de la realidad más elevada y eterna, el Padre primigenio de todos los seres personales.
\vs p117 3:10 El Ser Supremo evoluciona en razón de su coordinación con la Trinidad del Paraíso y como consecuencia de los éxitos divinos de los hijos de esa Trinidad: los creadores y los administradores del gran universo. El alma inmortal del hombre desarrolla su propio destino eterno en conjunción con la presencia divina del Padre del Paraíso y conforme a las decisiones de la mente humana en el ámbito de su ser personal. Lo que la Trinidad es para el Dios Supremo, el modelador es para el hombre evolutivo.
\vs p117 3:11 \pc Durante la presente era del universo, el Ser Supremo no parece estar capacitado para obrar directamente como creador salvo en esos casos en los que las instancias intermedias creativas del tiempo y del espacio hayan agotado sus posibilidades de acción a nivel finito. Hasta este momento, esto ha devenido una sola vez en la historia del universo; cuando se habían agotado las posibilidades de una acción finita en cuanto a la reflectividad del universo, entonces el Supremo actuó llevando creativamente a su culminación todos los actos creadores previos. Y creemos que actuará de nuevo en ese mismo cometido en eras futuras siempre que la anterior facultad creadora haya completado su ciclo correspondiente de actividad creativa.
\vs p117 3:12 El Ser Supremo no creó al hombre, pero, literalmente, el hombre se creó, y su propia vida se derivó, a partir de la potencialidad del Supremo. Tampoco hace evolucionar al hombre; no obstante, el propio Supremo es la esencia misma de la evolución. Desde la perspectiva finita, realmente vivimos, nos movemos y tenemos nuestro ser dentro de la inmanencia del Supremo.
\vs p117 3:13 Al parecer, el Supremo no puede iniciar una causa primigenia, pero parece ser el catalizador de todo el crecimiento del universo y está aparentemente destinado a llevar a su total culminación el destino de todos los seres experienciales\hyp{}evolutivos. El Padre da origen al concepto del cosmos finito; los hijos creadores llevan a efecto esta idea en el tiempo y en el espacio con la aprobación y cooperación de los espíritus creativos; el Supremo da culminación al finito total y establece la relación de este con el destino de lo absonito.
\usection{4. EL DIOS FINITO}
\vs p117 4:1 Al ver la incesante pugna de la creación creatural por lograr la perfección de su estatus y la divinidad de su ser, no podemos sino creer que este inacabable esfuerzo atestigua a su vez la incesante pugna del Supremo por su propia realización divina. El Dios Supremo es la Deidad finita, y debe hacer frente a los problemas que conlleva lo finito en el sentido total de este término. Nuestras luchas con las vicisitudes del tiempo en las evoluciones del espacio son reflejos de sus esfuerzos para conseguir la realidad de su yo y completar su soberanía dentro de la esfera de acción en la que, hasta los confines más extremos de sus posibilidades, se expande su naturaleza evolutiva.
\vs p117 4:2 El Supremo se esfuerza por lograr su expresión a lo largo de todo el gran universo. Su evolución divina se basa, en alguna medida, en la sabiduría\hyp{}acción de todo ser personal existente. Cuando un ser humano opta por la supervivencia eterna, está cocreando su destino; y, en la vida de este mortal ascendente, el Dios finito encuentra un mayor grado de crecimiento en su propia realización personal y un engrandecimiento de su soberanía experiencial. Pero si una criatura rechaza la andadura eterna, esa parte del Supremo que depende de la elección de dicha criatura experimenta una inevitable demora, una privación que debe compensarse por alguna experiencia substitutiva o adicional; por lo que respecta al ser personal del no superviviente, este se absorbe en la sobrealma de la creación, convirtiéndose en parte de la Deidad del Supremo.
\vs p117 4:3 Dios es tan confiado, tan amoroso, que entrega una parte de su naturaleza incluso en manos de los seres humanos para su custodia y autorrealización. La naturaleza del Padre, la presencia del modelador, es indestructible al margen de la decisión del mortal. El hijo del Supremo, el yo evolutivo, puede llegar a su término, a pesar de que el ser personal, potencialmente unificador de tal errado yo, persista como un componente de la Deidad de la Supremacía.
\vs p117 4:4 La persona humana puede verdaderamente invalidar la individualidad de su estado creatural y, aunque todo lo que era valioso en la vida de tal suicida cósmico subsistirá, \bibemph{estas cualidades no lo harán como criatura individual}. De nuevo, el Supremo encontrará su expresión en las criaturas de los universos, pero nunca más en esa persona particular; el ser personal único de un no ascendente vuelve al Supremo como una gota de agua vuelve al mar.
\vs p117 4:5 . Cualquier acción aislada de las partes personales de lo finito es relativamente irrelevante para la eventual aparición del Todo Supremo, pero, no obstante, el todo depende de los actos totales de sus múltiples partes. El ser personal del mortal individual es insignificante ante la totalidad de la Supremacía, pero el ser personal de cada ser humano representa un valor\hyp{}significado irremplazable en lo finito; el ser personal, una vez expresado, nunca más encuentra una expresión idéntica, salvo en la persistencia de ese ser personal vivo.
\vs p117 4:6 Y, de esta manera, mientras pugnamos por nuestra propia expresión, el Supremo lo hace en nosotros, y con nosotros, por su expresión en cuanto deidad. Cuando encontramos al Padre, el Supremo encuentra de nuevo al Creador del Paraíso de todas las cosas. A medida que superamos las dificultades de nuestra propia realización, de igual manera el Dios de la experiencia va adquiriendo la todopoderosa supremacía en los universos del tiempo y del espacio.
\vs p117 4:7 \pc La humanidad no asciende en el universo sin esforzarse, ni el Supremo evoluciona sin una acción decidida e inteligente. Las criaturas no logran la perfección de forma meramente pasiva, como tampoco el espíritu de la Supremacía puede llevar a efecto la potencia del Todopoderoso sin su incesante ministerio asistencial a la creación finita.
\vs p117 4:8 La relación temporal del hombre con el Supremo constituye la base de la moral cósmica, la sensibilidad universal y la aceptación, con respecto al \bibemph{deber}. Se trata de una moral que transciende el sentido temporal del bien y del mal relativos; es una moral que se fundamenta directamente en la apreciación consciente por parte de la criatura de su obligación experiencial a la Deidad experiencial. El hombre mortal y todas las otras criaturas finitas se crean a partir del potencial vivo de energía, mente y espíritu existentes en el Supremo. El modelador\hyp{}ascendente recurre al Supremo para crear el carácter inmortal y divino de un finalizador. Es a partir de la propia realidad del Supremo que el modelador, con la aprobación de la voluntad humana, teje los patrones de la naturaleza eterna de un hijo ascendente de Dios.
\vs p117 4:9 La evolución del progreso del modelador, respecto a la espiritualización y eternización de un ser personal humano, produce directamente un engrandecimiento de la soberanía del Supremo. Tales éxitos en la evolución humana son, al mismo tiempo, éxitos en la actualización evolutiva del Supremo. Aunque es cierto que las criaturas no podrían evolucionar sin el Supremo, es también probablemente cierto que la evolución del Supremo nunca podrá realizarse plenamente al margen de la evolución completa de todas las criaturas. Aquí reside la gran responsabilidad cósmica de los seres personales conscientes de sí mismos: que la Deidad Suprema depende en cierto sentido de la elección de la voluntad de los mortales. Y, mediante los mecanismos inescrutables de la reflectividad del universo, los ancianos de días tienen constancia, fiel y completamente, del mutuo avance evolutivo de las criaturas y del Supremo.
\vs p117 4:10 El gran reto que se ha planteado al hombre mortal es el siguiente: ¿Optarás por hacer personal los contenidos y valores experimentables del cosmos en tu yo evolutivo? O, al rechazar la supervivencia, ¿permitirás que estos secretos de la Supremacía permanezcan latentes, aguardando la acción de otra criatura de algún otro tiempo que, a \bibemph{su} manera, trate de hacer su contribución a la evolución del Dios finito? Pero esa sería su aportación al Supremo; no la tuya.
\vs p117 4:11 \pc En esta era, la gran pugna del universo se libra entre lo potencial y lo actual ---la búsqueda de la actualización por parte de todo aquello que todavía no está expresado---. Si el hombre mortal avanza en su aventura al Paraíso, está siguiendo los movimientos del tiempo, que fluyen como corrientes en el cauce de la eternidad; si el hombre mortal rechaza la andadura eterna, se mueve contrario a la corriente de la serie de acontecimientos que tienen lugar en los universos finitos. La creación mecánica continúa, de forma inexorable, según el propósito en despliegue del Padre del Paraíso, pero la creación volitiva tiene la opción de aceptar o rechazar el papel de la participación del ser personal en la aventura de la eternidad. El hombre mortal no puede destruir los valores supremos de la existencia humana, pero puede, con claridad, impedir la evolución de estos valores en su propia experiencia personal. En la medida en la que el yo humano se niega de esta manera a tomar parte de la ascensión al Paraíso, precisamente en esa medida el Supremo se demora en lograr su expresión divina en el gran universo.
\vs p117 4:12 Se le ha concedido al hombre mortal no solo la custodia de la presencia del modelador del Padre del Paraíso, sino también el dominio sobre el destino de una fracción infinitesimal del futuro del Supremo. Porque así como el hombre alcanza su destino humano, de igual manera el Supremo logra su destino en los niveles en cuanto deidad.
\vs p117 4:13 Así pues, al igual que una vez sucedió con nosotros, os aguarda una toma de decisión: ¿Fallaréis al Dios del tiempo, que tan dependiente es de las decisiones de la mente finita?, ¿fallaréis a la persona suprema de los universos por la indolencia de vuestra vuelta al nivel animal?, ¿fallaréis al gran hermano de todas las criaturas, que tan dependiente es de cada una de ellas?, ¿podéis permitiros pasar al ámbito de lo no realizado cuando tenéis ante vosotros la fascinante visión de la andadura en el universo ---el descubrimiento divino del Padre del Paraíso y la participación divina en la búsqueda, y en la evolución, del Dios de la Supremacía?---.
\vs p117 4:14 \pc Los dones de Dios ---su dádiva de la realidad--- no son particiones de sí mismo; él no aleja la creación de sí mismo, pero ha establecido tensiones en las creaciones que circundan al Paraíso. Dios primero ama al hombre y le confiere el potencial de la inmortalidad ---la realidad eterna---. Y conforme ama a Dios, el hombre se hace eterno en actualidad. Y he aquí el misterio: cuanto más estrechamente se aproxime el hombre a Dios mediante el amor, más grande es la realidad ---la actualidad--- de ese hombre. Cuanto más se retire el hombre de Dios, más cercano está a la no realidad ---al cese de su existencia---. Cuando el hombre consagra su voluntad a hacer la voluntad del Padre, cuando el hombre da a Dios todo lo que \bibemph{tiene,} entonces Dios hace de ese hombre más de lo que es.
\usection{5. LA SOBREALMA DE LA CREACIÓN}
\vs p117 5:1 El gran Supremo es la sobrealma cósmica del gran universo. En él, las cualidades y cantidades del cosmos encuentran ciertamente su reflejo propio de la deidad; su naturaleza en cuanto deidad es un mosaico compuesto por la inmensa totalidad de toda la naturaleza de las criaturas y los creadores de todos los universos evolutivos. El Supremo es también una Deidad en actualización que personifica una voluntad creativa, que abarca a su vez el propósito evolutivo del universo.
\vs p117 5:2 Los yos intelectuales y potencialmente personales de lo finito emergen de la Tercera Fuente y Centro y consiguen su síntesis, finita y espacio\hyp{}temporal como Deidad en el Supremo. Cuando la criatura se somete a la voluntad del Creador no soterra ni renuncia a su ser personal; los seres personales individuales que participan en la actualización del Dios finito no pierden su yo volitivo al obrar de esta manera. Más bien, estos seres personales se engrandecen progresivamente por su participación en esta gran aventura de la Deidad; mediante tal unión con la divinidad, el hombre enaltece, enriquece, espiritualiza y unifica su yo evolutivo hasta el mismo umbral de la Supremacía.
\vs p117 5:3 \pc El alma inmortal evolutiva del hombre, creación conjunta de la mente material y del modelador, asciende como tal hasta el Paraíso y posteriormente, cuando se incorpora en el colectivo final, se coliga, de alguna manera nueva, con la vía circulatoria de la gravedad espiritual del Hijo Eterno, mediante un método experiencial conocido como \bibemph{trascendencia del finalizador}. Estos finalizadores se convierten, de este modo, en aspirantes que se aceptan para su reconocimiento experiencial como seres personales del Dios Supremo. Y cuando estos intelectos mortales, en misiones futuras no reveladas del colectivo final, logran la séptima etapa de su existencia espiritual, tales mentes dobles se harán trinas. Estas dos mentes, la humana y la divina, armonizadas quedarán glorificadas en unión con la mente experiencial del Ser Supremo, que en ese momento ya estará actualizado.
\vs p117 5:4 En el futuro eterno, el Dios Supremo estará actualizado ---expresado creativamente y representado espiritualmente--- en la mente espiritualizada, en el alma inmortal, del hombre ascendente, así incluso como el Padre Universal se reveló en la vida terrena de Jesús.
\vs p117 5:5 \pc El hombre no se une con el Supremo y soterra su identidad personal, sino que las repercusiones en el universo de la experiencia de todos los hombres forman, pues, parte de la experiencia divina del Supremo. “El acto es nuestro, las consecuencias de Dios”.
\vs p117 5:6 En su progreso, el ser personal deja una estela de realidad actualizada a su paso por los niveles ascendentes de los universos. Sean mentales, espirituales o energéticas las creaciones en crecimiento del tiempo y el espacio, estas se modifican al avanzar el ser personal por sus ámbitos. Cuando el hombre actúa, el Supremo reacciona, y esta interactuación constituye el hecho del progreso.
\vs p117 5:7 Las grandes vías circulatorias de la energía, la mente y el espíritu nunca son posesión permanente del ser personal ascendente; estos ministerios se quedan para siempre como parte de la Supremacía. En la experiencia del mortal, el intelecto humano reside en las pulsaciones rítmicas de los espíritus asistentes de la mente y efectúa sus decisiones en el escenario que se despliega al encauzarse la mente en este ministerio. En el momento de la muerte del mortal, el yo humano se separa perpetuamente de la vía circulatoria de los asistentes. Aunque estos asistentes no parecen trasmitir experiencia de un ser personal a otro, pueden, y ciertamente transmiten, las repercusiones impersonales de la decisión\hyp{}acción al Dios Supremo a través del Dios Séptuplo. (Por lo menos esto es verdad en cuanto a los asistentes de la adoración y de la sabiduría).
\vs p117 5:8 Y esto es también así en cuanto a las vías circulatorias del espíritu: el hombre las emplea en su ascenso a través de los universos pero nunca las posee como parte de su ser personal eterno. Si bien, estas vías circulatorias del ministerio espiritual, ya sean el espíritu de la verdad, el espíritu santo u otras presencias espirituales del suprauniverso, son receptivas y reactivas a los valores emergentes en el ser personal ascendente, y estos valores se transmiten fielmente al Supremo por medio del Séptuplo.
\vs p117 5:9 \pc Aunque tales influencias espirituales como el espíritu santo y el espíritu de la verdad son ministerios del universo local, sus directrices no están por completo confinadas a los límites geográficos de una creación local dada. Conforme el mortal ascendente va más allá de las fronteras de su universo local originario, no se ve enteramente privado del ministerio del espíritu de la verdad, que tan constantemente le ha enseñado y guiado a través de los laberintos filosóficos de los mundos materiales y morontiales, y ha dirigido infaliblemente al peregrino al Paraíso en cualquier crisis surgida al ascender, siempre diciéndole: “Este es el camino”. Cuando os alejéis de los ámbitos del universo local, mediante el ministerio del espíritu del Ser Supremo emergente y los servicios de la reflectividad del suprauniverso, estaréis todavía guiados en vuestro ascenso al Paraíso por el espíritu directivo y consolador de los Hijos de Dios de gracia del Paraíso.
\vs p117 5:10 ¿De qué modo se registran en el Supremo estas múltiples vías circulatorias, al ejercer su ministerio cósmico, los contenidos, los valores y los hechos de la experiencia evolutiva? No estamos muy seguros, pero creemos que dicho registro ocurre por mediación de las personas de los creadores supremos de origen en el Paraíso, que son los otorgadores inmediatos de dichas vías circulatorias del tiempo y del espacio. El cúmulo de experiencia mental que poseen los siete espíritus asistentes de la mente, en su ministerio al nivel físico del intelecto, es una parte de la experiencia del universo local de la benefactora divina y, a través de este espíritu creativo, probablemente quede registrada en la mente de la Supremacía. Asimismo, las experiencias del mortal con el espíritu de la verdad y con el espíritu santo tal vez se registren en la persona de la Supremacía por métodos similares.
\vs p117 5:11 Incluso la experiencia del hombre y del modelador debe encontrar un eco en la divinidad del Dios Supremo, porque, a medida que los modeladores adquieren experiencia, son como el Supremo, y el alma evolutiva del hombre mortal se crea a partir de la preexistente posibilidad de dicha experiencia en el Supremo.
\vs p117 5:12 De este modo, las experiencias múltiples de toda la creación se convierten en parte de la evolución de la Supremacía. Las criaturas meramente hacen uso de las cualidades y cantidades de lo finito a medida que ascienden al Padre; las consecuencias impersonales de dicho uso permanecen para siempre como parte del cosmos vivo, de la persona Suprema.
\vs p117 5:13 Lo que el hombre lleva consigo como posesión personal son las consecuencias que tienen sobre el carácter la experiencia de haber empleado, en su ascenso al Paraíso, las vías circulatorias de la mente y del espíritu del gran universo. Cuando decide y su decisión culmina en acción, el hombre vive la experiencia, y los contenidos y los valores de dicha experiencia son, por siempre, parte de su carácter eterno en todos los niveles, desde el finito hasta el final. El carácter cósmicamente moral y divinamente espiritual representa el cúmulo, el patrimonio de la criatura, de decisiones personales que han sido iluminadas por la adoración sincera, glorificadas por el amor inteligente y culminadas en el servicio fraternal.
\vs p117 5:14 En su evolución, el Supremo llegará a compensar a las criaturas finitas por su inhabilidad de no lograr más que un limitado contacto experiencial con el universo de los universos. Las criaturas pueden llegar al Padre del Paraíso, pero sus mentes evolutivas, siendo finitas, son incapaces de comprender realmente al Padre infinito y absoluto. Pero, puesto que cualquier experiencia de la criatura se registra en el Supremo y es parte de él, cuando todas las criaturas consigan alcanzar el nivel final de su existencia finita, y después de que el desarrollo total del universo posibilite el logro del Dios Supremo como presencia real y divina, entonces, inherente al hecho de tal contacto, estará el contacto con la experiencia total. La finitud del tiempo contiene en sí misma el germen de la eternidad; y, se nos enseña que, cuando la plenitud de la evolución atestigua el agotamiento de su capacidad para el crecimiento cósmico, la finitud total emprenderá las facetas absonitas de la andadura eterna en la búsqueda del Padre como Último.
\usection{6. LA BÚSQUEDA DEL SUPREMO}
\vs p117 6:1 Buscamos al Supremo en los universos, pero no lo encontramos. “Él es está fuera y dentro de todas las cosas y seres, en movimiento y en reposo. Irreconocible en su misterio, aunque distante, él está también cerca”. El Todopoderoso Supremo es “la forma de lo que aún no está formado, el modelo de lo que aún no está creado”. El Supremo es vuestro hogar en el universo y, cuando lo halléis, será como volver a casa. Él es vuestro padre experiencial, y así como en la experiencia de los seres humanos, así ha crecido él en la experiencia de la paternidad divina. Os conoce porque es semejante a una criatura al igual que a un creador.
\vs p117 6:2 Si en verdad deseáis encontrar a Dios, no podéis evitar el nacimiento en vuestras mentes de la conciencia del Supremo. Así como Dios es vuestro Padre divino, del mismo modo el Supremo es vuestra Madre divina, de quien os nutrís a lo largo de vuestra vida como criaturas del universo. “¡Cuán universal es el Supremo ---él está en todas partes---! Las cosas sin límites de la creación dependen de su presencia para la vida, y no rechaza a ninguna”.
\vs p117 6:3 Lo que Miguel es para Nebadón, el Supremo es para el cosmos finito; su Deidad es la gran vía por la que el amor del Padre fluye hacia fuera, a toda la creación, y él es la gran vía por la que las criaturas finitas se encaminan hacia dentro en su búsqueda del Padre, que es amor. Incluso los modeladores del pensamiento guardan relación con él; en naturaleza y divinidad primigenias, son como el Padre, pero cuando acumulan las experiencias de sus interacciones en el tiempo, en los universos del espacio, llegan a semejarse al Supremo.
\vs p117 6:4 \pc El gesto de la criatura de hacer la voluntad del Creador es un valor cósmico y posee, en el universo, un significado al que responde de inmediato una fuerza ubicua coordinadora, no revelada, que indica probablemente la acción, siempre creciente, del Ser Supremo.
\vs p117 6:5 El alma morontial de un mortal evolutivo es realmente progenie de la acción del Padre Universal, mediante el modelador, y de la respuesta cósmica del Ser Supremo, de la Madre Universal. La influencia materna domina el ser personal humano durante la infancia del alma en crecimiento en el universo local. La influencia, como Deidad, de ambos padres llega a equipararse tras la fusión con el modelador y durante la andadura en el suprauniverso, pero cuando las criaturas del tiempo comienzan su travesía por el universo central eterno, la naturaleza paterna se hace cada vez más manifiesta, alcanzando el máximo de su manifestación finita con su reconocimiento del Padre Universal y la admisión en el colectivo final.
\vs p117 6:6 En y a través de la labor de consecución del estatus de finalizador, las cualidades maternas y experienciales del yo ascendente se ven influenciadas considerablemente por el contacto y la infusión de la presencia espiritual del Hijo Eterno y la presencia mental del Espíritu Infinito. Luego, en todos los ámbitos de actuación de los finalizadores en el gran universo, aparece un nuevo despertar del potencial materno latente del Supremo, una nueva realización de los contenidos experienciales, y una nueva síntesis de valores experienciales de la completa andadura de ascensión. Parece que esta realización del yo proseguirá en las andaduras en el universo de los finalizadores de la sexta etapa, hasta que la herencia materna del Supremo logre su sintonía finita con la herencia del Padre, a través del modelador. Este fascinante período de actividad en el ámbito del gran universo representa una continuidad de la andadura adulta del mortal ascendente perfeccionado.
\vs p117 6:7 Al completarse la sexta etapa de la existencia y entrar en la etapa séptima y final del estatus espiritual, tendrán lugar, probablemente, las eras avanzadas de la experiencia enriquecedora, de la sabiduría madura y de la realización divina. En la naturaleza del finalizador, esto será probablemente equivalente a la total consecución resultante de la pugna de la mente por su autorrealización espiritual, la completa coordinación de la naturaleza ascendente del hombre con la naturaleza del modelador divino, dentro de los límites de las posibilidades finitas. En el universo, un yo de tanta magnificencia se convierte, pues, en el hijo finalizador eterno del Padre del Paraíso al igual que, en el universo, en el hijo eterno de la Madre Suprema, un yo, del universo, cualificado para representar por igual al Padre y a la Madre de los universos y de los seres personales, en cualquier actividad o cometido, pertenecientes a la administración finita de las cosas y seres creados, creadores o en evolución.
\vs p117 6:8 Todos los humanos cuyas almas están en evolución son, literalmente, los hijos evolutivos del Dios Padre y del Dios Madre, o Ser Supremo. Pero hasta ese momento en el que el hombre mortal no se haga consciente de su alma, de su heredad divina, debe adquirir por la fe esta certeza de su parentesco con la Deidad. La experiencia de la vida humana es la crisálida cósmica en la que los dones en el ámbito del universo del Ser Supremo y la presencia del Padre Universal (ninguno de los cuales tienen estado personal) están haciendo evolucionar el alma morontial del tiempo y el carácter del finalizador humano\hyp{}divino, poseedores de destino en el universo y servidores en la eternidad.
\vs p117 6:9 \pc Con excesiva frecuencia, los hombres se olvidan de que Dios constituye la más grande experiencia de la existencia humana. Otras experiencias están limitadas en cuanto a su naturaleza y contenido, pero la experiencia de Dios no tiene límites salvo aquellos impuestos por la capacidad de comprensión de las criaturas, y esta misma experiencia aumenta por sí misma dicha capacidad. Cuando los hombres buscan a Dios, lo buscan todo. Cuando encuentran a Dios, lo han encontrado todo. La búsqueda de Dios constituye un irrestricta dádiva de amor, acompañada por el sorprendente descubrimiento de un amor nuevo y más grande que está por otorgarse.
\vs p117 6:10 Todo el amor verdadero proviene de Dios, y el hombre recibe el cariño divino a medida que él mismo da este amor a sus semejantes. El amor es dinámico. Nunca puede apresarse; está vivo, es libre, estremecedor y siempre está en movimiento. El hombre no puede tomar el amor del Padre y recluirlo en su corazón. El amor del Padre se convierte en algo real en el hombre mortal solo cuando pasa a través del ser personal del hombre y él, a su vez, lo entrega a sus semejantes. La gran vía circulatoria del amor procede del Padre, de hijos a hermanos y, de ahí, al Supremo. El amor del Padre se manifiesta en el ser personal humano gracias al ministerio del modelador interior. Un hijo que conoce a Dios revela este amor a sus hermanos del universo, y este cariño fraternal es la esencia del amor del Supremo.
\vs p117 6:11 \pc La única aproximación posible al Supremo es mediante la experiencia y, en las épocas presentes de la creación, solo existen tres vías por las que la criatura se puede acercar a la Supremacía:
\vs p117 6:12 \li{1.}Los ciudadanos del Paraíso descienden de la Isla Eterna a través de Havona, donde adquieren la capacidad para comprender la Supremacía mediante la observación del diferencial de la realidad Paraíso\hyp{}Havona y el descubrimiento exploratorio de la múltiple actividad de los seres personales creadores supremos, que abarcan desde los espíritus mayores hasta los hijos creadores.
\vs p117 6:13 \li{2.}Los ascendentes del espacio\hyp{}tiempo que suben desde los universos evolutivos de los creadores supremos se aproximan bastante al Supremo al atravesar Havona, como paso preliminar para aumentar su apreciación de la unidad de la Trinidad del Paraíso.
\vs p117 6:14 \li{3.}Los nativos de Havona adquieren una comprensión del Supremo a través de sus contactos con los peregrinos descendentes del Paraíso y con los peregrinos ascendentes de los siete suprauniversos. Los nativos de Havona están, inherentemente, en posición de armonizar las perspectivas, esencialmente distintas, de los ciudadanos de la Isla eterna y de los ciudadanos de los universos evolutivos.
\vs p117 6:15 \pc Para las criaturas evolutivas hay siete formas de aproximarse al Padre Universal, y cada uno de estos actos de ascensión al Paraíso pasa por la divinidad de uno de los siete espíritus mayores; y cada aproximación se hace posible gracias al aumento de su receptividad a la experiencia como resultado del hecho de que la criatura ha servido en el suprauniverso, reflejo de la naturaleza de ese espíritu mayor. La suma total de estas siete experiencias constituye los límites presentes conocidos que delimitan la conciencia de la criatura sobre la realidad y a la actualidad del Dios Supremo.
\vs p117 6:16 No son solamente sus propias limitaciones, las que impiden al hombre hallar al Dios finito; es también la incompletitud del universo, incluso la incompletitud de todas las criaturas ---pasadas, presentes y futuras--- la que hace al Supremo inaccesible. Individualmente, cualquier persona que haya alcanzado el nivel divino de similitud con Dios puede encontrar al Dios Padre, pero ninguna criatura \bibemph{de forma individual} descubrirá jamás personalmente al Supremo, hasta ese momento bastante distante en el tiempo en el que, mediante el logro universal de la perfección, \bibemph{todas} las criaturas lo encuentren de forma simultánea.
\vs p117 6:17 A pesar del hecho de que no podéis, en esta era del universo, encontrarlo personalmente como podéis encontrar y encontraréis al Padre, al Hijo y al Espíritu, no obstante, vuestro ascenso al Paraíso y posterior andadura en el universo crearán, paulatinamente, en vuestra conciencia, el reconocimiento de la presencia en el universo y de la acción cósmica del Dios de toda experiencia. Los frutos del espíritu son la sustancia del Supremo tal como él se realiza en la experiencia humana.
\vs p117 6:18 El logro, en algún momento, del Supremo por parte del hombre se produce como consecuencia de su fusión con el espíritu de la Deidad del Paraíso. En el caso de los urantianos, este espíritu es la presencia del modelador del Padre Universal; y aunque el mentor misterioso procede del Padre y es como el Padre, dudamos de que incluso tal don divino pueda llevar a cabo la imposible tarea de revelar la naturaleza del Dios infinito a una criatura finita. Sospechamos que lo que los modeladores revelarán a los futuros finalizadores de la séptima etapa será la divinidad y la naturaleza del Dios Supremo. Y esta revelación será, para una criatura finita, lo que la revelación del infinito sería para un ser absoluto.
\vs p117 6:19 El Supremo no es infinito, pero probablemente abarque toda la infinitud que una criatura finita sea capaz realmente de entender alguna vez. ¡Comprender más de lo que representa el Supremo sería ser más que finito!
\vs p117 6:20 Todas las creaciones experienciales son interdependientes en cuanto a la realización de su destino. Solo la realidad existencial se contiene a sí misma y existe por sí misma. Havona y los siete suprauniversos se precisan mutuamente para alcanzar el máximo de su realización finita; asimismo, dependerán en algún momento de los universos futuros del espacio exterior para trascender la finitud.
\vs p117 6:21 Un ascendente humano puede hallar al Padre; Dios es existencial y, por consiguiente, real, con independencia del estatus experiencial reinante en la totalidad del universo. Pero ningún ascendente encontrará jamás al Supremo de forma individual hasta que todos los ascendentes no hayan alcanzado en el universo esa máxima madurez que los faculte, simultáneamente, para participar en dicho descubrimiento.
\vs p117 6:22 El Padre no hace acepción de personas; trata a cada uno de sus hijos ascendentes como personas cósmicas. Igualmente, el Supremo no hace distinción de personas; trata a sus hijos experienciales como un solo todo cósmico.
\vs p117 6:23 El hombre puede descubrir al Padre en su corazón, pero tendrá que buscar al Supremo en el corazón de todos los demás hombres; y cuando todas las criaturas revelen perfectamente el amor del Supremo, entonces, él se convertirá, en el universo, en actualidad para todas las criaturas. Lo que constituye otra manera de decir que los universos se asentarán en luz y vida.
\vs p117 6:24 El logro de la propia realización en perfección de todos los seres personales sumado a la conquista del perfecto equilibrio en todos los universos equivale a la realización del Supremo y testimonia la liberación de toda realidad finita de las limitaciones de la existencia incompleta. Tal agotamiento de todos los potenciales finitos produce esta realización, en su completitud, del Supremo y, definiéndolo de otra manera, la completa actualización evolutiva del mismo Ser Supremo.
\vs p117 6:25 \pc Los hombres no encuentra al Supremo repentinamente y de forma espectacular como un terremoto que abriese grandes grietas en las rocas, sino que lo encuentran con lentitud y paciencia como un río que erosionase silenciosamente su lecho.
\vs p117 6:26 Cuando encontráis al Padre, encontraréis la magna causa de vuestra ascensión espiritual en los universos; cuando encontráis al Supremo descubriréis el magno efecto de vuestra andadura de progreso al Paraíso.
\vs p117 6:27 Pero ningún mortal conocedor de Dios estará jamás solo en su viaje a través del cosmos, porque sabe que el Padre camina a su lado a cada paso del camino, mientras que el camino mismo que atraviesa es la presencia del Supremo.
\usection{7. EL FUTURO DEL SUPREMO}
\vs p117 7:1 La realización completa de todos los potenciales finitos equivale a la realización en su completitud de toda la experiencia evolutiva. Esto supone la aparición final del Supremo como presencia todopoderosa en cuanto Deidad en los universos. Creemos que el Supremo, en esta etapa de su desarrollo, poseerá un estado personal tan diferenciado como el del Hijo Eterno, estará tan específicamente empoderado como la Isla del Paraíso y tan totalmente unificado como el Actor Conjunto, y todo esto dentro de los límites de las posibilidades finitas de la Supremacía en la culminación de la presente era del universo.
\vs p117 7:2 Aunque se trata de un concepto enteramente adecuado acerca del futuro del Supremo, queremos llamar la atención respecto a determinados problemas consustanciales a este concepto:
\vs p117 7:3 \li{1.}Los supervisores incondicionados del Supremo difícilmente pueden deidificarse en una etapa previa a la completa evolución de este y, sin embargo, estos mismos supervisores, incluso en este momento, ejercen, de forma limitada, la soberanía de la supremacía con respecto a los universos asentados en luz y vida.
\vs p117 7:4 \li{2.}Es difícil que el Supremo pueda desempeñar su actividad en la Trinidad Última hasta no haber alcanzado la completa actualización de su estatus en el universo, y la Trinidad Última es, todavía ahora, una realidad delimitada, y se os ha informado de la existencia de los vicerregentes condicionados del Último.
\vs p117 7:5 \li{3.}El Supremo no es completamente real para las criaturas del universo, pero hay muchas razones que indican que es muy real para la Deidad Séptupla, que se extiende desde el Padre Universal del Paraíso hasta los hijos creadores y los espíritus creativos de los universos locales.
\vs p117 7:6 \pc Puede resultar que en los límites superiores de lo finito, donde el tiempo se une con el tiempo transcendido, exista algún tipo de desdibujamiento y de mezcla de secuencia. Puede resultar que el Supremo sea capaz de proyectar, en el universo, su presencia sobre estos niveles supratemporales y, entonces, hasta cierto punto, anticipar su evolución venidera reflejando dicha proyección futura de vuelta a los niveles creados como la Inmanencia del Incompleto Proyectado. Se pueden observar tales fenómenos dondequiera que lo finito haga contacto con lo suprafinito, como en las experiencias de los seres humanos en los que moran los modeladores del pensamiento, que son auténticas predicciones de los éxitos futuros del hombre en el universo, por toda la eternidad.
\vs p117 7:7 \pc Al ser admitidos en el colectivo de los finalizadores del Paraíso, los mortales ascendentes prestan juramento a la Trinidad del Paraíso, y al hacer este juramento de lealtad, están al mismo tiempo prometiendo su eterna fidelidad al Dios Supremo, que \bibemph{es} la Trinidad tal como los seres personales creaturales finitos la entienden. Posteriormente, cuando desempeñan su actividad por todos los universos evolutivos, las compañías de finalizadores son únicamente receptivas a los mandatos venidos del Paraíso, hasta esos memorables tiempos en los que los universos locales se asienten en luz y vida. A medida que las nuevas organizaciones gubernamentales de estas creaciones perfeccionadas empiezan a ser un reflejo de la soberanía emergente del Supremo, observamos que estas distantes compañías reconocen la autoridad jurisdiccional de tales nuevos gobiernos. Parece que el Dios Supremo a medida que evoluciona unifica los colectivos finales evolutivos, pero es muy probable que el Supremo, cuando se convierta en miembro de la Trinidad Última, dirija estos siete colectivos.
\vs p117 7:8 \pc El Ser Supremo dispone de tres posibilidades suprafinitas para manifestarse en el universo:
\vs p117 7:9 \li{1.}Colaboración Absonita en la primera Trinidad experiencial.
\vs p117 7:10 \li{2.}Relación coabsoluta en la segunda Trinidad experiencial.
\vs p117 7:11 \li{3.}Participación coinfinita en la Trinidad de Trinidades, pero no tenemos una idea satisfactoria de lo que esto realmente significa.
\vs p117 7:12 \pc Se trata de una de las hipótesis generalmente aceptadas respecto al futuro del Supremo, pero también existe mucha suposición en cuanto a sus relaciones con el presente gran universo, con posterioridad a su logro del estatus de luz y vida.
\vs p117 7:13 El objetivo actual de los suprauniversos es convertirse, tal como son y dentro de sus potenciales, en perfectos, incluso como lo es Havona. Esta perfección guarda relación con sus logros físicos y espirituales, inclusive con su desarrollo administrativo, gubernamental y fraternal. Se cree que, en las eras venideras, se disiparán de los suprauniversos las posibilidades de desarmonía, desajuste e inadaptación. Las vías circulatorias de la energía encontrarán el equilibrio perfecto y se someterán por completo a la mente, mientras que el espíritu, en la presencia del ser personal, habrá conseguido el dominio sobre la mente.
\vs p117 7:14 Es de suponer que, en ese momento remoto en el tiempo, el espíritu\hyp{}persona del Supremo y la potencia adquirida por el Todopoderoso habrán conseguido un desarrollo equiparado, y que ambos, unificados en y por la Mente Suprema, se harán efectivos como Ser Supremo, serán actualidad completa en los universos ---una actualidad que será observable por todas las inteligencias creaturales, a la que reaccionarán todas las energías creadas, que se coordinará en el ámbito de todas las entidades espirituales y que experimentarán todos los seres personales del universo---.
\vs p117 7:15 Este planteamiento supone la soberanía real del Supremo en el gran universo. Es enteramente posible que los presentes administradores de la Trinidad continúen siendo sus vicerregentes, pero creemos que las demarcaciones imperantes entre los siete universos desaparecerán paulatinamente, y que el gran universo en su totalidad obrará como un todo perfeccionado.
\vs p117 7:16 Es posible que, en ese momento, el Supremo resida personalmente en Uversa, la sede central de Orvontón, desde donde dirigirá la administración de las creaciones del tiempo, pero esto es real y únicamente una suposición. Ciertamente, no obstante, la persona del Supremo será claramente contactable en algún lugar concreto, aunque la ubicuidad de su presencia como Deidad continuará probablemente permeando el universo de los universos. No sabemos cuál será la relación de los ciudadanos de los suprauniversos de esa era con el Supremo, pero quizás sea algo parecido a la que existe en la actualidad entre los nativos de Havona y la Trinidad del Paraíso.
\vs p117 7:17 \pc El gran universo ya perfeccionado de esos futuros días será inmensamente diferente de como es ahora. Habrán desaparecido la apasionante aventura que entraña la organización de las galaxias del espacio, la implantación de la vida en los inestables mundos del tiempo, y el desarrollo de la armonía a partir del caos, de la belleza a partir de los potenciales, de la verdad a partir de los contenidos y de la bondad a partir de los valores. ¡Estos universos del tiempo habrán logrado cumplir su destino finito! Y quizás durante un intervalo de tiempo, habrá reposo, descanso de la lucha milenaria por buscar la perfección evolutiva. ¡Pero no durante mucho tiempo! Ciertamente, sin lugar a dudas, el inexorablemente enigma de la Deidad emergente del Dios Último planteará un reto a estos ciudadanos perfeccionados de los universos establecidos, de la misma manera que la búsqueda del Dios Supremo significó un reto para sus tenaces antepasados evolutivos. El telón del destino cósmico se abrirá para revelar la trascendental grandeza de la fascinante búsqueda absonita por lograr al Padre Universal en aquellos niveles revelados, nuevos y más elevados, de la ultimidad de la experiencia creatural.
\vsetoff
\vs p117 7:18 [Auspiciado por un mensajero poderoso con residencia temporal en Urantia.]
