\upaper{77}{Las criaturas intermedias}
\author{Arcángel}
\vs p077 0:1 La mayoría de los mundos habitados albergan uno o más grupos de seres singulares, que operan en un nivel de vida aproximadamente a medio camino entre el de los mortales de los mundos y el de los órdenes angélicos; de ahí que se llamen criaturas \bibemph{intermedias}. Parecen ser un accidente del tiempo, pero están tan extendidos y son tan valiosos como ayudantes que, desde hace mucho tiempo, todos los hemos aceptado como uno de los órdenes de seres fundamentales para nuestro ministerio planetario combinado.
\vs p077 0:2 En Urantia desempeñan su actividad dos órdenes distintos de seres intermedios: el colectivo primario o más antiguo, que surgió en los tiempos de Dalamatia, y el grupo secundario o más joven, cuyo origen se remonta a los tiempos de Adán.
\usection{1. LOS SERES INTERMEDIOS PRIMARIOS}
\vs p077 1:1 Los seres intermedios primarios de Urantia tienen su origen en una excepcional interrelación de lo material y lo espiritual. Conocemos de la existencia de criaturas similares en otros mundos y en otros sistemas, pero se originaron de modo distinto.
\vs p077 1:2 Es conveniente tener siempre en cuenta que los sucesivos ministerios de gracia de los Hijos de Dios en un planeta evolutivo producen cambios notables en la organización espiritual del mundo y, a veces, modifican tanto el funcionamiento de la interrelación de los agentes espirituales y materiales del planeta que se crean situaciones verdaderamente difíciles de entender. El estatus de los cien miembros corpóreos de la comitiva del príncipe Caligastia ilustra precisamente esta excepcional interrelación: como ascendentes morontiales y ciudadanos de Jerusem eran criaturas supramateriales sin capacidad de reproducción; como servidores planetarios descendentes en Urantia, eran criaturas sexuales materiales capaces de procrear progenie humana (tal como algunos de ellos harían después). Si bien, no nos es posible explicar, de forma satisfactoria, cómo estos cien miembros pudieron ser progenitores en un nivel supramaterial, pero esto es precisamente lo que ocurrió. La vinculación supramaterial (no sexual) de un hombre y una mujer, miembros de la comitiva corpórea, dio como resultado la aparición del primogénito de los seres intermedios primarios.
\vs p077 1:3 Se descubrió inmediatamente que una criatura de este orden, a medio camino entre el nivel mortal y el angélico, sería de gran ayuda para llevar a cabo los asuntos de la sede central del príncipe y, en consecuencia, se concedió permiso a las parejas de la comitiva corpórea para procrear seres similares. Esta iniciativa dio como resultado el nacimiento del primer grupo de cincuenta criaturas intermedias.
\vs p077 1:4 Tras observar durante un año la labor de este excepcional grupo de seres, el príncipe planetario autorizó la generación sin restricción de criaturas intermedias. Este plan se llevó a efecto mientras persistió la capacidad de reproducción, y así fue como se dio nacimiento al colectivo original de 50\,000 seres intermedios.
\vs p077 1:5 Entre la procreación de cada una de las criaturas intermedias mediaba un período de medio año y, cuando cada pareja daba nacimiento a mil de ellas, ya no nacían más. Y no hay ninguna explicación viable que aclare por qué, cuando cada pareja engendraba al que hacía el número mil, su capacidad de reproducción se agotaba. Por más que se intentó, el resultado siempre era el fracaso.
\vs p077 1:6 \pc Estas criaturas constituyeron el cuerpo de inteligencia de la administración del príncipe. Recorrían todos los lugares, estudiando y observando las razas del mundo y prestando otros inestimables servicios al príncipe y a su comitiva en la labor de influir sobre la sociedad humana que se hallaba alejada de la sede planetaria.
\vs p077 1:7 Este régimen continuó hasta los trágicos días de la rebelión planetaria, que atrapó a algo más de las cuatro quintas partes de los seres intermedios primarios. Las fuerzas leales entraron al servicio de los síndicos melquisedecs, actuando bajo el liderazgo nominal de Van hasta los días de Adán.
\usection{2. LA RAZA NODITA}
\vs p077 2:1 Aunque este es el relato del origen, naturaleza y labor de las criaturas intermedias de Urantia, la afinidad entre ambos órdenes ---el primario y el secundario--- obliga a interrumpir la historia de los seres intermedios primarios en este punto, a fin de seguir la línea de descendencia de los miembros rebeldes de la comitiva corpórea del príncipe Caligastia, desde los días de la rebelión planetaria hasta los tiempos de Adán. Fue esta línea hereditaria la que, en los primeros días del segundo jardín, aportó la mitad de los ancestros del orden secundario de seres criaturas intermedias.
\vs p077 2:2 \pc Los miembros físicos de la comitiva del príncipe habían tomado forma como criaturas sexuadas a fin de participar en el plan de procrear una progenie que incorporara las cualidades combinadas de su orden especial unidas a las de los linajes seleccionados de las tribus andonitas, algo que se haría con anticipación a la posterior aparición de Adán. Los portadores de vida habían proyectado un nuevo tipo de mortal que englobase la unión de la progenie conjunta de la comitiva del príncipe y la de la primera generación de Adán y Eva. Así pues, habían diseñado un plan previendo un nuevo orden de criaturas planetarias que esperaban se convirtiesen en los gobernantes\hyp{}maestros de la sociedad humana. Estos seres estaban concebidos para asumir la soberanía social, no la soberanía civil. Pero, como este proyecto se malogró casi por completo, nunca sabremos de qué clase de aristocracia de benévolos líderes e inigualable cultura quedó Urantia privada. Porque cuando la comitiva corpórea se reprodujo más tarde, lo hizo con posterioridad a la rebelión y después de ser desposeídos de su vínculo con las corrientes vitales del sistema.
\vs p077 2:3 En la era posterior a la rebelión en Urantia se presenciaron muchos sucesos insólitos. Una gran civilización ---la cultura de Dalamatia--- se estaba desmoronando. “Había nefilim (noditas) en la tierra en aquellos días, y cuando estos hijos de los dioses se llegaron a las hijas de los hombres y les engendraron hijos, estos fueron ‘los valientes de antaño', ‘los varones de renombre'”. Aunque eran difícilmente “hijos de los dioses”, los mortales evolutivos de aquellos remotos días consideraban así a la comitiva y a sus primeros descendientes; en la tradición se llegó incluso a magnificar su estatura. Este es, pues, el origen del relato popular, prácticamente universal, de dioses que descendieron a la tierra y procrearon allí con las hijas de los hombres una antigua raza de héroes. Y esta leyenda se volvió aún más confusa con las mezclas raciales de los adanitas, que posteriormente harían su aparición en el segundo jardín.
\vs p077 2:4 Puesto que los cien miembros corpóreos de la comitiva del príncipe portaban plasma germinal de las estirpes humanas andónicas, era natural esperar que, si se reproducían sexualmente, su progenie se parecería por completo a los vástagos de otros padres andonitas. Pero cuando los sesenta rebeldes de la comitiva, los seguidores de Nod, acometieron de hecho la reproducción sexual, sus hijos resultaron ser muy superiores en casi todos los sentidos tanto a los pueblos andonitas como a los sangiks. Esta inesperada excelencia no solo caracterizaba sus cualidades físicas e intelectuales sino también sus aptitudes espirituales.
\vs p077 2:5 Dichos rasgos mutacionales que aparecieron en la primera generación de noditas se debían a ciertos cambios producidos en la configuración y en los componentes químicos de los factores hereditarios del plasma germinal andónico. Estos cambios fueron causados por la presencia, en los cuerpos de los miembros de la comitiva, de las poderosas vías de sustento vital del sistema de Satania. Tales vías vitales provocaron que los cromosomas del modelo específico de Urantia se reorganizaran siguiendo más los modelos de la variación normalizada de Satania, respecto a la manifestación de vida establecida para Nebadón, que el modelo del plasma vital andónico. El método de esta metamorfosis del plasma germinal, mediante la acción de las corrientes vitales del sistema, no es diferente de esos procedimientos mediante los cuales los científicos de Urantia modifican el plasma germinal de las plantas y animales mediante los rayos X.
\vs p077 2:6 Por consiguiente, los pueblos noditas surgieron de ciertas modificaciones particulares e inesperadas ocurridas en el plasma vital, que los cirujanos de Avalón habían transferido desde los cuerpos de los donantes andonitas hasta los de los miembros de la comitiva corpórea.
\vs p077 2:7 \pc Cabe recordar que los cien andonitas donantes de plasma germinal se convirtieron a su vez en receptores del complemento orgánico del árbol de la vida, de modo que las corrientes vitales de Satania se infundieron igualmente en sus cuerpos. Los cuarenta y cuatro andonitas modificados que siguieron a la comitiva en su rebelión también procrearon entre sí e hicieron una gran contribución a la mejora de las estirpes del pueblo nodita.
\vs p077 2:8 Estos dos grupos, que sumaban un total de 104 personas portadoras del plasma germinal andonita modificado, constituirían los ancestros de los noditas, la octava raza que apareció en Urantia. Y estos nuevos rasgos distintivos de la vida humana en Urantia representan otra faceta de la ejecución del plan original destinado a utilizar a este planeta como mundo de modificación de la vida, salvo que esta fue una de las circunstancias no previstas.
\vs p077 2:9 \pc Los noditas de linaje puro fueron una raza magnífica, pero paulatinamente se mezclaron con los pueblos evolutivos de Urantia y, en poco tiempo, se produjo en ellos un gran deterioro. Diez mil años después de la rebelión, los noditas habían sufrido reveses hasta el extremo de que la duración media de su vida solo era escasamente superior a la de las razas evolutivas.
\vs p077 2:10 Cuando los arqueólogos desentierran las tablillas de arcilla con las crónicas de los últimos descendientes sumerios, descubren listas de reyes sumerios que se extienden varios miles de años atrás; y, a medida que estas crónicas retroceden más y más en el tiempo, los reinados de cada uno de estos reyes se prolonga desde alrededor de veinticinco o treinta años hasta ciento cincuenta o más. Esta extensión de los reinados de los reyes más antiguos significa que algunos de los primeros gobernantes noditas (los descendientes inmediatos de la comitiva del príncipe) vivieron realmente más tiempo que sus sucesores más recientes; también indica un esfuerzo por remontar sus dinastías hasta la época de Dalamatia.
\vs p077 2:11 La documentación de personas tan longevas se debe igualmente a la confusión entre los meses y los años como períodos de tiempo. Esto también se puede observar en la genealogía bíblica de Abraham y en los primeros registros de los chinos. La confusión del mes de veintiocho días, o estación, con el año de más de trescientos cincuenta días, introducido después, explica la alusión tradicional a existencias humanas tan longevas. Hay datos de un hombre que vivió más de novecientos “años”. Este período no supone ni setenta años y, durante mucho tiempo, estas vidas se consideraron muy largas; posteriormente se les designó “sesenta años más diez”.
\vs p077 2:12 El cálculo del tiempo por meses de veintiocho días perduró por mucho tiempo tras los días de Adán. Si bien, cuando los egipcios emprendieron la reforma del calendario, hace unos siete mil años, la llevaron a cabo con gran exactitud, introduciendo el año de 365 días.
\usection{3. LA TORRE DE BABEL}
\vs p077 3:1 Tras el sumergimiento de Dalamatia, los noditas se desplazaron hacia el norte y el este y fundaron en poco tiempo la nueva ciudad de Dilmún como su sede racial y cultural. Y, cerca de cincuenta mil años después de la muerte de Nod, cuando los descendientes de la comitiva del príncipe habían llegado a ser demasiado numerosos como para subsistir en las tierras de las inmediaciones de la nueva ciudad y, una vez que habían contactado con las tribus andonitas y sangiks de zonas limítrofes a sus fronteras para contraer matrimonio, se les ocurrió a sus líderes que se debería hacer algo para preservar su unidad racial. En consecuencia, se convocó un consejo de tribus y, tras muchas deliberaciones, se dio aprobación al plan de Bablot, un descendiente de Nod.
\vs p077 3:2 Bablot proponía erigir un pretencioso templo de glorificación racial en el centro del territorio que entonces ocupaban. Este templo tendría una torre sin igual en el mundo. Iba a ser un colosal monumento conmemorativo de su pasado grandioso. Había muchos que querían que este monumento se edificara en Dilmún, pero otros, recordando los relatos sobre el hundimiento de Dalamatia, su primera capital, alegaban que una construcción tan enorme se debería situar a una distancia prudencial del mar y sus peligros.
\vs p077 3:3 Bablot tenía previsto que los nuevos edificios se convirtieran en el núcleo del futuro centro de cultura y civilización noditas. Sus recomendaciones acabaron por prevalecer y se dio comienzo a las obras con arreglo a sus planes. La nueva ciudad se llamaría \bibemph{Bablot,} en honor al arquitecto y constructor de la torre. Más tarde, este lugar llegaría a conocerse como Bablod y, finalmente, como Babel.
\vs p077 3:4 Pero entre los noditas existían ciertas discrepancias de opinión en cuanto a los planes y propósitos de esta iniciativa. Sus líderes tampoco estaban totalmente de acuerdo respecto al plan de obras o al uso de los edificios una vez construidos. Tras cuatro años y medio de trabajo, se produjo una gran controversia sobre el objetivo y el motivo por los que se erigiría la torre, y las polémicas se enconaron tanto que se detuvo todo el trabajo. Los porteadores de alimentos difundieron la noticia de la disensión, y un gran número de tribus se fue congregando en el recinto de la obra. Se plantearon tres posturas diferentes sobre el propósito por el que se construía la torre:
\vs p077 3:5 \li{1.}El grupo más grande, casi la mitad, deseaba que la torre se construyera como un monumento conmemorativo a la historia y a la superioridad racial noditas. Pensaban que debería ser una edificación grande e imponente que se ganara la admiración de todas las generaciones futuras.
\vs p077 3:6 \li{2.}La siguiente facción en importancia pretendía que la torre se destinara a conmemorar la cultura de Dilmún. Preveían que Bablot se convertiría en un gran centro de comercio, arte y manufactura.
\vs p077 3:7 \li{3.}El contingente más pequeño y minoritario consideraba que la construcción de la torre brindaba la oportunidad de expiar el desatino de sus progenitores por participar en la rebelión de Caligastia. Sostenían que la torre debía dedicarse al culto del Padre de todos, que el único propósito de la nueva ciudad debía ser ocupar el lugar de Dalamatia: ejercer como centro cultural y religioso para los primitivos habitantes de los alrededores.
\vs p077 3:8 \pc Rápidamente, se votó en contra del grupo con motivaciones religiosas. La mayoría rechazaba la doctrina de que sus ancestros hubiesen sido culpables de rebelión; les enojaba tal estigma racial. Tras haberse librado de una de las tres perspectivas en litigio y no logrando decidirse por ninguna de las otras dos mediante un debate, recurrieron a la guerra. El grupo religioso, los no combatientes, huyeron a sus lugares de origen en el sur, mientras que sus compañeros lucharon hasta casi aniquilarse mutuamente.
\vs p077 3:9 \pc Hace unos doce mil años, se intentó por segunda vez erigir la torre de Babel. Las razas mezcladas de los anditas (noditas con adanitas) se comprometieron a levantar un nuevo templo sobre las ruinas de la primera construcción, pero la iniciativa no tuvo el respaldo suficiente; cayó por su propio ostentoso peso. Durante mucho tiempo, se conoció esta región con el nombre de la tierra de Babel.
\usection{4. LOS CENTROS DE CIVILIZACIÓN NODITAS}
\vs p077 4:1 La dispersión de los noditas fue el resultado inmediato del conflicto, mutuamente pernicioso, surgido en relación a la torre de Babel. Esta guerra interna disminuyó considerablemente el número de los noditas más puros y fue, en muchos sentidos, la causante de su incapacidad para establecer una gran civilización preadánica. A partir de ese momento, la cultura nodita decayó durante más de ciento veinte mil años hasta que fue elevada por la infusión del linaje adánico. Pero incluso en los tiempos de Adán, los noditas eran todavía un pueblo capaz. Muchos de sus descendientes mestizos se contaban entre los constructores del Jardín, y varios capitanes de los grupos de Van eran noditas. Algunas de las mentes más hábiles que prestaban servicio en el equipo de Adán pertenecían a esta raza.
\vs p077 4:2 Inmediatamente tras el conflicto de Bablot, se establecieron tres de los cuatro grandes centros noditas:
\vs p077 4:3 \li{1.}\bibemph{Los noditas occidentales o sirios}. Los restantes miembros de los nacionalistas o memorialistas raciales viajaron hacia el norte donde se unieron con los andonitas, fundando los más recientes centros noditas al noroeste de Mesopotamia. Este era el grupo más numeroso de los noditas dispersos, y contribuyeron mucho a la estirpe asiria que apareció con posterioridad.
\vs p077 4:4 \li{2.}\bibemph{Los noditas orientales o elamitas}. Los defensores de la cultura y del comercio emigraron en gran número hacia el Oriente adentrándose en Elam y allí se unieron con las tribus mestizas sangiks. Los elamitas de hace treinta o cuarenta mil años se habían convertido en cuanto a su naturaleza en sangik, aunque continuaban manteniendo una civilización superior a la de las primitivas tribus de los alrededores.
\vs p077 4:5 Tras establecerse en el segundo jardín, se habituaba a hacer alusión a este asentamiento nodita cercano como “la tierra de Nod”; y, durante el largo período de paz relativa entre este grupo nodita y los adanitas, las dos razas se mezclaron considerablemente, porque se convirtió cada vez más en costumbre que los Hijos de Dios (los adanitas) se casaran con las hijas de los hombres (los noditas).
\vs p077 4:6 \li{3.}\bibemph{Los noditas centrales o presumerios}. En la desembocadura de los ríos Tigris y Éufrates había un reducido grupo que conservaba mejor su integridad racial. Subsistieron durante miles de años y, con el tiempo, aportaron el ancestro nodita que se mezcló con los adanitas para fundar los pueblos sumerios de los tiempos históricos.
\vs p077 4:7 Y todo esto explica cómo aparecieron de forma tan repentina y misteriosa los sumerios en la escena mesopotámica. Los investigadores jamás podrán seguir el rastro de estas tribus hasta los inicios de los sumerios, que surgieron hace doscientos mil años, tras el sumergimiento de Dalamatia. Sin vestigios de su origen en otros lugares del mundo, estas antiguas tribus se vislumbran súbitamente en el horizonte de la civilización con una cultura completamente desarrollada y superior, que incluía templos, metalurgia, agricultura, ganadería, alfarería, tejeduría, derecho mercantil, códigos civiles, ceremonial religioso y un antiguo sistema de escritura. Al principio de la era histórica, hacía bastante tiempo que se había perdido el alfabeto de Dalamatia al haberse adoptado el peculiar sistema de escritura originado en Dilmún. La lengua sumeria, aunque prácticamente perdida para el mundo, no era semítica; tenía mucho en común con las llamadas lenguas arias.
\vs p077 4:8 Los detallados registros que dejaron los sumerios se hacen eco del lugar de un extraordinario asentamiento situado en el Golfo Pérsico, cerca de la temprana ciudad de Dilmún. Los egipcios llamaron Dilmat a esta ciudad de ancestral gloria, mientras que los posteriores sumerios adanizados confundieron la primera y la segunda ciudad nodita con Dalamatia y llamaron Dilmún a las tres. Y los arqueólogos ya han encontrado estas antiguas tablillas sumerias de arcilla que hablan de este paraíso terrenal “donde los dioses bendijeron primeramente a la humanidad con el ejemplo de vida civilizada y culta”. Y estas tablillas, referidas a Dilmún, el paraíso de los hombres y Dios, reposan ahora silenciosamente sobre los polvorientos estantes de muchos museos.
\vs p077 4:9 Los sumerios conocían bien de la existencia del primer Jardín y del segundo, pero, a pesar de la gran cantidad de matrimonios mixtos con los adanitas, continuaron considerando a los moradores del jardín del norte como una raza extraña. El orgullo sumerio por la ancestral cultura nodita les indujo a ignorar estos vislumbres tardíos de gloria en beneficio de la grandeza y las tradiciones paradisíacas de la ciudad de Dilmún.
\vs p077 4:10 \li{4.}\bibemph{Los noditas del norte y los amadonitas:} los vanitas. Este grupo surgió con anterioridad al conflicto de Bablot. Estos noditas más septentrionales eran descendientes de los que habían dado la espalda al liderazgo de Nod y de sus sucesores en favor del de Van y Amadón.
\vs p077 4:11 \pc Algunos de los primeros acompañantes de Van se establecieron después cerca de las orillas del lago, que todavía sigue llevando su nombre, y en torno a esta localidad nacieron sus leyendas. Ararat se convirtió en su montaña sagrada, que para los vanitas de los últimos tiempos adquirió un significado muy similar al que tiene el monte Sinaí para los hebreos. Hace diez mil años, los ancestros vanitas de los asirios enseñaban que los dioses habían dado a Van su ley moral de siete mandamientos en el monte Ararat. Tenían la firme convicción de que Van y su compañero Amadón fueron llevados vivos del planeta mientras estaban en la montaña absortos en adoración.
\vs p077 4:12 El monte Ararat era la montaña sagrada del norte de Mesopotamia y, dado que gran parte de vuestras leyendas de estos tiempos ancestrales nacieron en el marco de la historia babilónica del diluvio, no es extraño que el monte Ararat y su región se incorporaran en la historia judía de Noé y el diluvio universal surgida con posterioridad.
\vs p077 4:13 Hacia el año 35\,000 a. C., Adánez visitó uno de los antiguos asentamientos vanitas más orientales para fundar allí su centro de civilización.
\usection{5. ADÁNEZ Y RATTA}
\vs p077 5:1 Habiendo esbozado los antecedentes noditas de los ancestros de los seres intermedios secundarios, en esta narrativa se debe considerar ahora la mitad adánica de tales ancestros, pues los seres intermedios también son los nietos de Adánez, el primogénito de la raza violeta de Urantia.
\vs p077 5:2 \pc Adánez formaba parte de aquel grupo de hijos de Adán y de Eva que eligieron permanecer en la tierra con su padre y su madre. Pues bien, este hijo, el mayor de Adán, había oído contar frecuentemente a Van y Amadón la historia de su hogar en las altiplanicies del norte y, algún tiempo después del establecimiento del segundo jardín, decidió ir en busca de estas tierras de sus sueños juveniles.
\vs p077 5:3 En aquel momento, Adánez tenía ciento veinte años de edad y había sido padre de treinta y dos hijos del linaje puro del primer jardín. Quería quedarse con sus padres y ayudarles a la construcción del segundo jardín, pero estaba muy abatido por la pérdida de su compañera y de sus hijos. Todos habían elegido ir a Edentia juntos con los demás hijos adánicos que escogieron convertirse en pupilos de los altísimos.
\vs p077 5:4 Adánez no quería abandonar a sus padres en Urantia; era reacio a huir de las dificultades y de los peligros, pero se percató de que las relaciones en el segundo jardín distaban de ser satisfactorias. Hizo mucho por hacer avanzar las primeras iniciativas de defensa y construcción, aunque decidió ir al norte lo antes posible. Y a pesar de que su despedida fue muy grata, Adán y Eva quedaron bastante afligidos por la pérdida de su hijo mayor; al verle partir a un mundo extraño y hostil, temieron que no volviera jamás.
\vs p077 5:5 Un grupo de veintisiete personas acompañó a Adánez al norte en la búsqueda del poblado de sus fantasías infantiles. En poco más de tres años, el grupo de Adánez encontró en efecto el objeto de su aventura y, entre la gente de este poblado, él halló a una mujer maravillosa y bella, de veinte años de edad, que afirmaba ser la última descendiente de puro linaje de la comitiva del príncipe. Esta mujer, Ratta, decía que sus ancestros eran todos descendientes de dos de los miembros caídos de la comitiva del príncipe. Ella era la última de su raza; no tenía hermanos ni hermanas vivos. Ratta estaba a punto de tomar la decisión de no emparejarse con nadie, de morir sin descendencia, pero le dio su corazón al majestuoso Adánez. Y cuando oyó la historia de Edén y cómo las predicciones de Van y Amadón se habían de hecho cumplido, al escuchar el relato de la transgresión cometida en el Jardín, le embargó un único pensamiento: casarse con este hijo y heredero de Adán. Rápidamente, aquella idea maduró en Adánez y, en algo más de tres meses, se casaron.
\vs p077 5:6 \pc Adánez y Ratta tuvieron sesenta y siete hijos. Dieron origen a un gran linaje de líderes mundiales, pero hicieron algo más. Se debe recordar que estos dos seres eran en realidad suprahumanos. Cada cuarto hijo que les nacía era de un orden único, a menudo invisible. Jamás en la historia del mundo había ocurrido una cosa semejante. Ratta estaba muy consternada ---hasta cayó en la superstición---, pero Adánez sabía bien de la existencia de los seres intermedios primarios y llegó a la conclusión de que, ante sus ojos, estaba sucediendo algo similar. Cuando nació el segundo hijo de tan extraño comportamiento, decidió que procrearan, puesto que uno era masculino y el otro femenino, y este es el origen del orden secundario de seres intermedios. Al cabo de cien años, antes de que este fenómeno cesara, habían nacido casi dos mil de ellos.
\vs p077 5:7 \pc Adánez vivió 396 años. Volvió muchas veces a visitar a su padre y a su madre. Cada siete años él y Ratta viajaban al sur, hacia el segundo jardín y, mientras tanto, los seres intermedios los mantenían informados sobre el bienestar de su pueblo. Durante la vida de Adánez, estos seres prestaron un gran servicio en la construcción de un centro mundial nuevo e independiente para promover la verdad y la rectitud.
\vs p077 5:8 De este modo, Adánez y Ratta tuvieron a su disposición a este colectivo de magníficos ayudantes, que trabajó con ellos durante toda sus largas vidas asistiéndoles en la propagación de la verdad avanzada y en la difusión de niveles superiores de vida intelectual, espiritual y física. Y los resultados de este esfuerzo por mejorar el mundo jamás llegaron a ser eclipsados del todo por los retrocesos que seguirían.
\vs p077 5:9 \pc Desde los tiempos de Adánez y Ratta y durante casi siete mil años, los adanecitas mantuvieron una elevada cultura. Más tarde, se mezclaron con los noditas y andonitas vecinos y se incluyeron también entre los “hombres valientes de la antigüedad”. Y algunos de los avances de aquella era persistieron llegando a ser una parte latente del potencial cultural que floreció después hasta convertirse en la civilización europea.
\vs p077 5:10 Este centro de civilización estaba situado en la región este del extremo sur del Mar Caspio, cerca del Kopet Dagh. A poca altura, en las estribaciones de Turquestán, están los vestigios de lo que una vez fue la sede adanecita de la raza violeta. En estas altiplanicies, localizadas en una antigua franja fértil de las estribaciones más bajas de la cordillera Kopet, surgieron sucesivamente, en diversos períodos, cuatro culturas distintas, que fueron respectivamente promovidas por cuatro grupos diferentes de descendientes de Adánez. El segundo de estos grupos fue el que emigró hacia el oeste, a Grecia y a las islas del Mediterráneo. El resto de los descendientes de Adánez lo hizo hacia el norte y el oeste, adentrándose en Europa con la estirpe mixta de la última ola andita venida de Mesopotamia, y que también se contaban entre los invasores andita\hyp{}arios de la India.
\usection{6. LOS SERES INTERMEDIOS SECUNDARIOS}
\vs p077 6:1 Aunque los seres intermedios primarios tenían un origen prácticamente suprahumano, el orden secundario es vástago del linaje adánico puro unido a una descendiente humanizada cuyos progenitores eran ancestros directos de la antigua comitiva del príncipe.
\vs p077 6:2 Entre los hijos de Adánez, había exactamente dieciséis de los peculiares progenitores de los seres intermedios secundarios. Estos singulares hijos estaban igualmente divididos en cuanto a sexo, y cada pareja era capaz de engendrar un ser intermedio secundario cada setenta días mediante un procedimiento consistente en una combinación de unión sexual y no sexual. Y nunca, antes de ese momento, había sido factible que se produjese en la tierra tal fenómeno, como tampoco se ha dado desde entonces.
\vs p077 6:3 Estos dieciséis hijos vivieron y murieron (salvo por sus peculiaridades) como mortales del mundo, pero sus vástagos, eléctricamente energizados, viven y continúan viviendo, sin estar sujetos a las limitaciones de la carne mortal.
\vs p077 6:4 Cada una de las ocho parejas dieron lugar finalmente a 248 seres intermedios secundarios, y de este modo se dio existencia el colectivo original de estas criaturas: una cantidad de 1984. Hay ocho subgrupos de estos seres. Se les denomina A\hyp{}B\hyp{}C el primero, el segundo, el tercero y así sucesivamente. Y luego hay D\hyp{}E\hyp{}F primero, segundo y así sucesivamente.
\vs p077 6:5 \pc Tras la transgresión de Adán, los seres intermedios primarios se reincorporaron al servicio de los síndicos melquisedecs, en tanto que el grupo secundario se adscribió al centro de Adánez hasta que este falleció. Treinta y tres de estos seres secundarios intermedios, los jefes de su organización al morir Adánez, intentaron alentar a todos los de su orden a que entrasen al servicio de los melquisedecs y unirse así con el colectivo primario. Pero, al no lograrlo, abandonaron a sus compañeros y pasaron como grupo al servicio de los síndicos planetarios.
\vs p077 6:6 Después de la muerte de Adánez, el resto de los seres intermedios ejercieron sobre Urantia una influencia extraña, desorganizada e inconexa. Desde aquel momento hasta los días de Maquiventa Melquisedec, llevaron una existencia errática y desordenada. Este melquisedec los tuvo parcialmente bajo su control, pero continuaron haciendo mucho daño hasta los días de Cristo Miguel. Y durante la estancia de Miguel en la tierra, todos se decidieron definitivamente en cuanto a su futuro destino; la mayoría leal se alistó bajo el liderazgo de los seres intermedios primarios.
\usection{7. LOS SERES INTERMEDIOS REBELDES}
\vs p077 7:1 En los tiempos de la rebelión de Lucifer, la mayoría de los seres intermedios cayeron en pecado. Al hacer un recuento de la magnitud de la devastación ocurrida en la rebelión planetaria, se halló, entre otras pérdidas, que de los 50\,000 seres intermedios originarios, 40\,119 se habían unido a la secesión de Caligastia.
\vs p077 7:2 El número inicial de seres intermedios secundarios era de 1984; de estos, 873 no se adhirieron al gobierno de Miguel y fueron convenientemente confinados el día de Pentecostés con motivo del juicio planetario de Urantia. Nadie puede predecir el futuro de estas criaturas caídas.
\vs p077 7:3 Ambos grupos de seres intermedios rebeldes están ahora detenidos a la espera del dictamen final de los asuntos de la rebelión del sistema. No obstante, antes de inaugurarse la actual dispensación planetaria, hicieron en la tierra muchas cosas extrañas.
\vs p077 7:4 Estos seres intermedios desleales eran capaces de dejarse ver ante los ojos de los mortales bajo ciertas circunstancias; esto era particularmente cierto en el caso de los compañeros de Beelzebú, el líder de los seres intermedios secundario apóstatas. Pero no se debe confundir a estas singulares criaturas con determinados querubines y serafines rebeldes que también estuvieron en la tierra hasta el momento de la muerte y resurrección de Cristo. Algunos de los escritores más antiguos denominaban espíritus malignos y demonios a estas criaturas intermedias rebeldes y, ángeles malvados, a los serafines apóstatas.
\vs p077 7:5 En ningún mundo pueden los espíritus malignos tomar posesión de la mente mortal una vez que un hijo de gracia del Paraíso ha vivido allí su vida. Pero antes de los días de Cristo Miguel en Urantia ---antes de la llegada generalizada de los modeladores del pensamiento y de que se derramase el espíritu del Maestro sobre toda carne--- estos seres intermedios rebeldes eran realmente capaces de influir en las mentes de ciertos mortales peor dotados y controlar de alguna manera sus acciones. Esto se conseguía de forma muy similar a como ejercen su labor las criaturas intermedias leales, cuando actúan como eficaces guardianes de contacto de las mentes humanas del colectivo de reserva de destino de Urantia, en esos momentos en los que el modelador, en efecto, se separa del ser personal durante un tiempo en el que contacta con inteligencias sobrehumanas.
\vs p077 7:6 Lo que los escritos afirman no es una mera figura retórica: “Y le trajeron a personas con todo tipo de dolencias, los endemoniados y los lunáticos”. Jesús sabía y reconocía la diferencia entre demencia y posesión demoníaca, aunque en las mentes de quienes vivieron en su época y generación había una gran confusión respecto a estos estados de conciencia.
\vs p077 7:7 Incluso antes de Pentecostés, ningún espíritu rebelde podía dominar una mente humana normal y, desde aquel día, hasta las débiles mentes de los mortales peor dotados están libres de esta posibilidad. Desde la llegada del espíritu de la verdad, las supuestas expulsiones de demonios han sido producto de la confusión de la creencia en la posesión demoniaca con la histeria, la locura y la deficiencia mental. Pero por el hecho de que el ministerio de gracia de Miguel haya liberado para siempre a todas las mentes humanas de Urantia de la posibilidad de ser poseídas por el demonio, no creáis que no fue una realidad en tiempos pasados.
\vs p077 7:8 En estos momentos, todo el grupo de seres intermedios rebeldes está preso por orden de los altísimos de Edentia. Ya no vagan más por este mundo con su proclividad a hacer el mal. Con independencia de los modeladores del pensamiento, el derramamiento del espíritu de la verdad sobre toda carne imposibilitó para siempre que los espíritus desleales, sean del tipo que sean, invadiesen de nuevo incluso a la más deficiente de las mentes humanas. Desde el día de Pentecostés, jamás podrá darse de nuevo una posesión demoniaca.
\usection{8. LOS SERES INTERMEDIOS UNIDOS}
\vs p077 8:1 Durante el último juicio de este mundo, cuando Miguel trasladó a los supervivientes dormidos del planeta, las criaturas intermedias se quedaron atrás para ayudar en la labor espiritual y semiespiritual del planeta. En este momento, obran como un único colectivo, que incluye a los dos órdenes y asciende a un total de 10\,992 miembros. Actualmente, los miembros más antiguos de cada orden se alternan en el gobierno de \bibemph{Los Seres Intermedios Unidos de Urantia}. Este sistema de gobierno continúa desde que se unificaron en un solo grupo poco después de Pentecostés.
\vs p077 8:2 Los miembros del orden más antiguo o primario se conocen generalmente por números; a menudo reciben nombres como 1\hyp{}2\hyp{}3 el primero, 4\hyp{}5\hyp{}6 el primero y así sucesivamente. En Urantia, se designan por orden alfabético a los seres intermedios adánicos para distinguirlos de la denominación numérica de los seres intermedios primarios.
\vs p077 8:3 Los componentes de ambos órdenes son seres no materiales en cuanto a la nutrición y absorción de la energía, pero comparten muchos rasgos humanos y son capaces de apreciar y entender vuestro humor al igual que vuestra adoración. Cuando se asignan a mortales, participan del trabajo, descanso y entretenimientos humanos. Pero los seres intermedios no duermen, ni poseen la capacidad de procrear. En cierto modo, los miembros del grupo secundario se diferencian en términos de masculinidad y feminidad, y con frecuencia se alude a ellos como “él” o “ella”. A menudo trabajan juntos en parejas mixtas.
\vs p077 8:4 Los seres intermedios no son hombres, ni tampoco ángeles, pero, por su naturaleza, los del orden secundario están más cerca de los hombres que de los ángeles; son, de alguna manera, de vuestras razas y, por lo tanto, muy comprensivos y compasivos en su relación con los seres humanos; son de un inestimable valor para los serafines en su tarea para y con las distintas razas de la humanidad, y ambos órdenes son imprescindibles para los serafines que sirven como guardianes personales de los mortales.
\vs p077 8:5 \pc Los Seres Intermedios Unidos de Urantia se organizan para prestar servicios con los serafines planetarios en virtud de sus dones innatos y destrezas adquiridas en los siguientes grupos:
\vs p077 8:6 \li{1.}\bibemph{Seres intermedios mensajeros.} Los componentes de este grupo tienen nombres; forman un colectivo reducido y son de gran ayuda en un mundo evolutivo al estar al servicio de comunicaciones personales rápidas y fidedignas.
\vs p077 8:7 \li{2.}\bibemph{Centinelas planetarios}. Los seres intermedios son los guardianes, los centinelas, de los mundos del espacio. Desempeñan la importante función de observadores de los todos numerosos fenómenos y tipos de comunicación de relevancia para los seres sobrenaturales del mundo. Patrullan el ámbito espiritual invisible del planeta.
\vs p077 8:8 \li{3.}\bibemph{Seres personales de contacto.} En los contactos que se realizan con los seres mortales de los mundos materiales siempre se emplean las criaturas intermedias, como en el caso de la persona mediante la que se transmitieron estas comunicaciones. Son un elemento esencial en estos enlaces de los niveles espiritual y material.
\vs p077 8:9 \li{4.}\bibemph{Ayudantes del progreso}. Son las criaturas intermedias más espirituales, y se asignan como asistentes de los diversos órdenes de serafines que actúan en el planeta en grupos especiales.
\vs p077 8:10 \pc Los seres intermedios difieren considerablemente en sus habilidades para ponerse en contacto con los serafines por encima de ellos y con sus primos humanos por debajo. Por ejemplo, es extremadamente difícil para los seres intermedios primarios establecer contacto directo con agentes materiales. Se hallan considerablemente más cercanos a los seres de tipo angélico y, por ello, se les suele asignar a trabajar con las fuerzas espirituales residentes en el planeta y a prestarles sus servicios. Actúan como acompañantes y guías de visitantes celestiales y de estudiantes con residencia transitoria, en tanto que las criaturas secundarias casi exclusivamente se adscriben al servicio de los seres materiales del planeta.
\vs p077 8:11 Los 1111 seres intermedios secundarios leales llevan a cabo importantes misiones en la tierra. En comparación con sus compañeros primarios, son indudablemente materiales. Existen justo fuera del rango de la visión humana y poseen el suficiente margen de adaptación para ponerse en contacto físico a voluntad con lo que los humanos llaman “cosas materiales”. Estas singulares criaturas tienen algunos determinados poderes sobre las cosas del tiempo y del espacio, sin excluir a los animales del planeta.
\vs p077 8:12 Una gran parte de los fenómenos más tangibles atribuidos a los ángeles se han realizado por los seres intermedios secundarios. Cuando los ignorantes líderes religiosos de ese tiempo encarcelaron a los primeros maestros del evangelio de Jesús, un verdadero “ángel del Señor” “abrió por la noche las puertas de la cárcel y los sacó”. Pero en el caso de la liberación de Pedro, tras la muerte de Santiago por mandato de Herodes, fue un ser intermedio secundario quien llevó a cabo la labor atribuida a un ángel.
\vs p077 8:13 Hoy en día, su tarea principal consiste en ser, inadvertidamente, los compañeros de enlace personal con aquellos hombres y mujeres que componen el colectivo planetario de reserva de destino. Fue la labor de este grupo secundario, hábilmente secundada por determinados miembros del colectivo primario, la que propició la conjunción de seres personales y de circunstancias en Urantia, que finalmente llevó a los supervisores celestiales del planeta a iniciar aquellas peticiones conducentes a la concesión de los mandatos que posibilitaron la serie de revelaciones de las que forma parte la exposición de este tema. Pero conviene dejar claro que los seres intermedios no intervienen en las sórdidas prácticas que se llevan a cabo bajo la denominación general de “espiritismo”. Los seres intermedios residentes actualmente en Urantia, todos de una honorable reputación, no están relacionados con los fenómenos de la llamada “mediumnidad”; y no suelen permitir, habitualmente, que los humanos sean testigos de su actividad física, a veces necesaria, ni de otros contactos con el mundo material, tal como se perciben por los sentidos humanos.
\usection{9. LOS CIUDADANOS PERMANENTES DE URANTIA}
\vs p077 9:1 Se puede considerar a los seres intermedios como el primer grupo de habitantes permanentes de los distintos tipos de mundos de todo el universo, en contraste con los ascendentes evolutivos como son las criaturas mortales y las multitudes angélicas. Estos ciudadanos permanentes se hallan en diferentes puntos del ascenso al Paraíso.
\vs p077 9:2 A diferencia de los diversos órdenes celestiales nombrados para \bibemph{servir} en los planetas, los seres intermedios \bibemph{viven} en los mundos habitados. Los serafines van y vienen, pero las criaturas intermedias permanecen y permanecerán, aunque sean, no obstante, servidores de los seres nativos del planeta y provean el único régimen de carácter permanente, que armoniza y relaciona los gobiernos cambiantes de las multitudes seráficas.
\vs p077 9:3 Como verdaderos ciudadanos de Urantia, los seres intermedios tienen un interés familiar en el destino de esta esfera. Conforman una decidida asociación que constantemente trabaja para el progreso de su planeta natal. El lema de su orden muestra su propia determinación: “Lo que los Seres Intermedios Unidos emprenden, los Seres Intermedios Unidos lo llevan a cabo”.
\vs p077 9:4 Aunque su habilidad para viajar por las vías circulatorias de la energía posibilita su partida del planeta, se han comprometido de forma individual a no abandonarlo antes de que las autoridades del universo los eximan de sus obligaciones. Los seres intermedios están arraigados en el planeta hasta las eras de luz y vida. Exceptuando a 1\hyp{}2\hyp{}3 el primero, jamás ninguna de las criaturas intermedias leales ha partido de Urantia.
\vs p077 9:5 A 1\hyp{}2\hyp{}3 el primero, el de más antigüedad del orden primario, se le relevó de sus deberes planetarios inmediatos poco después de Pentecostés. Este noble ser intermedio se mantuvo firme con Van y Amadón durante los trágicos días de la rebelión planetaria, y su intrépido liderazgo contribuyó decididamente a reducir las bajas de su orden. En el momento actual, sirve en Jerusem en calidad de miembro de los veinticuatro consejeros, habiendo ya desempeñado una vez, desde Pentecostés, la función de gobernador general de Urantia.
\vs p077 9:6 \pc Los seres intermedios están vinculados al planeta, pero, de la misma manera que los mortales hablan con los viajeros venidos de lejos para saber de los lugares distantes del planeta, los seres intermedios conversan con los viajeros celestiales para saber de los lugares distantes del universos. De este modo, se familiarizan con este sistema y este universo, incluso con Orvontón y sus creaciones hermanas y se preparan, de este modo, para su ciudadanía en los niveles superiores existenciales de las criaturas.
\vs p077 9:7 Aunque nacieron plenamente desarrollados ---sin experimentar período alguno de crecimiento ni de desarrollo desde la inmadurez---, los seres intermedios nunca cesan de crecer en sabiduría y experiencia. Como los mortales, son criaturas en desarrollo y tienen una cultura que es un auténtico logro evolutivo. Hay muchas grandes mentes y espíritus poderosos entre los miembros del colectivo de seres intermedios de Urantia.
\vs p077 9:8 Desde una perspectiva más amplia, la civilización de Urantia es resultado de la conjunción de los mortales urantianos y de los seres intermedios urantianos, y esto es cierto a pesar de la actual diferencia entre los dos niveles de cultura, una diferencia que no se compensará con anterioridad a las eras de luz y vida.
\vs p077 9:9 La cultura de los seres intermedios, siendo producto de una ciudadanía planetaria inmortal, es relativamente inmune a las vicisitudes temporales que acosan a la civilización humana. Las generaciones de hombres olvidan; el colectivo de seres intermedios recuerda, y esa memoria es el tesoro de las tradiciones de vuestro mundo habitado. De esta forma, la cultura de un planeta está siempre presente en ese planeta y, en las circunstancias debidas, estas preciadas memorias de los hechos pasados se vuelven disponibles. Así fue como los seres intermedios de Urantia dieron a sus primos en la carne la historia de la vida y enseñanzas de Jesús.
\vs p077 9:10 Los seres intermedios son los expertos servidores que compensan esa brecha existente entre los asuntos materiales y los espirituales de Urantia, surgida al morir Adán y Eva. Asimismo, son vuestros hermanos mayores, compañeros en la larga lucha por lograr en Urantia un estado de luz y vida afianzado. Los Seres Intermedios Unidos son un colectivo a prueba de rebelión y cumplirán fielmente su labor en la evolución planetaria, hasta que este mundo alcance la meta de los tiempos, hasta ese día distante en el que reine realmente la paz en la tierra y haya en verdad buena voluntad en el corazón de los hombres.
\vs p077 9:11 Debido a la valiosa labor realizada por estos seres intermedios, hemos llegado a la conclusión de que son en verdad una parte fundamental de la organización espiritual de los mundos. Y allí donde la rebelión no ha perturbado los asuntos de algún planeta, resultan incluso de mayor ayuda a los serafines.
\vs p077 9:12 \pc En su totalidad, la organización de espíritus elevados, multitudes angélicas y seres intermedios está entusiásticamente consagrada a fomentar el plan del Paraíso para el ascenso progresivo y el perfeccionamiento de los mortales evolutivos, una de las supremas ocupaciones del universo ---el magnífico plan de supervivencia diseñado para que Dios descienda hasta los hombres y, luego, mediante cierta forma de colaboración sublime, ascender a los hombres hasta Dios y hacia la eternidad de servicio y al logro de la divinidad--- tanto para los mortales como para los seres intermedios ---.
\vsetoff
\vs p077 9:13 [Exposición de un arcángel de Nebadón.]
