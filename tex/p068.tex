\upaper{68}{Los albores de la civilización}
\author{Melquisedec}
\vs p068 0:1 Comienza aquí la narración de la larguísima y continuada lucha de la especie humana partiendo de una existencia algo superior a la animal, a través de las eras intermedias, y hasta épocas posteriores, cuando una civilización real, aunque imperfecta, evolucionó de las razas mejor dotadas de la humanidad.
\vs p068 0:2 La civilización es una adquisición de la raza humana; no es biológicamente consustancial a ella; de ahí que todos los niños tengan que crecer en un entorno de cultura, en tanto que cada generación venidera de jóvenes ha de obtener nuevamente su educación. Las cualidades superiores de la civilización ---científicas, filosóficas y religiosas--- no se transmiten de una generación a otra por herencia directa. Tales logros culturales se preservan únicamente mediante la protección inteligente de la herencia social.
\vs p068 0:3 Los maestros de Dalamatia introdujeron el desarrollo social de tipo cooperativo y, durante trescientos mil años, se formó a la humanidad en la idea de actividades grupales. El hombre azul, sobre todo, se benefició de estas primeras enseñanzas sociales, en cierta medida lo hizo el hombre rojo y, menos que los demás, el hombre negro. En los últimos tiempos, son la raza amarilla y la blanca las que han demostrado tener el desarrollo social más avanzado de Urantia.
\usection{1. SOCIALIZACIÓN PROTECTORA}
\vs p068 1:1 Cuando han de vivir en estrecho contacto, los seres humanos a menudo aprenden a gustarse unos a otros, pero el hombre primitivo no rebosaba por naturaleza del espíritu de los sentimientos fraternales ni del deseo de tener un contacto social con sus semejantes. Más bien, las razas primitivas aprendieron a través de experiencias dolorosas que “la unión hace la fuerza”; y es esta falta de atracción fraternal natural la que, en la actualidad, obstaculiza la realización inmediata de la hermandad del hombre en Urantia.
\vs p068 1:2 La asociación se convirtió pronto en el precio a pagar por la supervivencia. El hombre solitario estaba indefenso, a no ser que llevara la marca tribal que evidenciara que pertenecía a un grupo, que ciertamente se vengaría de cualquier agresión que sufriese. Incluso en la época de Caín, resultaba letal salir de su territorio solo sin llevar alguna señal de pertenencia a un grupo. La civilización se ha convertido en el seguro del hombre contra la muerte violenta, mientras se paga su coste sometiéndose a las numerosas exigencias de las leyes de la sociedad.
\vs p068 1:3 La sociedad primitiva se fundó, por tanto, sobre una reciprocidad de necesidades y sobre el reforzamiento de la seguridad que proporcionaba la asociación. Y la sociedad humana ha evolucionado en inacabables ciclos como resultado de este temor al aislamiento y la aversión a la cooperación.
\vs p068 1:4 \pc Los seres humanos primitivos aprendieron tempranamente que los grupos son inmensamente superiores y más fuertes que la simple suma de cada uno de los individuos que los componen. Cien hombres unidos y trabajando al unísono pueden mover una gran piedra; una veintena de guardianes de la paz bien entrenados pueden contener una multitud enfurecida. Y así nació la sociedad, no de la mera asociación numérica, sino más bien \bibemph{organizándose} y cooperando de forma inteligente. Si bien, la cooperación no es un rasgo natural del hombre; en un principio, el hombre aprende a cooperar gracias al miedo y, luego, porque descubre entonces que le resulta muy beneficioso para enfrentarse a las dificultades del tiempo y protegerse de los supuestos peligros de la eternidad.
\vs p068 1:5 Así pues, los pueblos que se organizaron pronto en sociedades primitivas consiguieron tener más éxito en su lucha contra la naturaleza al igual que en su defensa frente a sus semejantes; tenían mayores posibilidades de sobrevivir; de ahí que la civilización en Urantia haya avanzado de forma ininterrumpida, a pesar de sus numerosos reveses. Y es el fortalecimiento del valor de la supervivencia, por medio de la asociación, la causa de que muchos errores del hombre no hayan conseguido frenar ni destruir, hasta el momento, la civilización humana.
\vs p068 1:6 \pc La sociedad cultural contemporánea es más bien un fenómeno reciente, y este hecho está bien demostrado por la supervivencia actual de unas condiciones sociales tan primitivas como las que caracterizan a los aborígenes australianos y a los bosquimanos y los pigmeos de África. Entre estos pueblos atrasados se puede observar algo de la antigua hostilidad tribal, la desconfianza personal y otros rasgos extremadamente antisociales, tan característicos de todas las razas primitivas. Estos restos deplorables de los pueblos antisociales de los tiempos antiguos atestiguan, elocuentemente, el hecho de que la tendencia individualista natural del hombre no puede competir con éxito con las organizaciones y asociaciones más potentes y poderosas que promueven el progreso social. Estas razas antisociales atrasadas y desconfiadas, que hablan un dialecto diferente cada sesenta u ochenta kilómetros, demuestran en qué tipo de mundo estaríais viviendo ahora si no hubiera sido por las enseñanzas combinadas de la comitiva corpórea del príncipe planetario y la labor posterior del grupo adánico de mejoradores raciales.
\vs p068 1:7 La expresión moderna, “volver a la naturaleza”, es un engaño nacido de la ignorancia, una creencia en la realidad de una antigua e imaginaria “edad de oro”. La única base sobre la que se forjó la leyenda de la edad de oro es el hecho histórico de la existencia de Dalamatia y del Edén. Pero estas desarrolladas sociedades estaban lejos de ser unos sueños utópicos hechos realidad.
\usection{2. FACTORES DEL PROGRESO SOCIAL}
\vs p068 2:1 La sociedad civilizada es el resultado de los primeros intentos del hombre por sobreponerse a su aversión al \bibemph{aislamiento}. Pero esto no significa necesariamente que existiese afecto mutuo, y el actual estado tumultuoso de ciertos grupos subdesarrollados son una buena muestra de las circunstancias por las que atravesaron las primeras tribus; y aunque los integrantes de una civilización puedan entrar en conflicto y luchar unos contra otros, y aunque la civilización en sí pueda parecer una muchedumbre incompatible, que se esfuerza y lucha, se evidencia en ella un esfuerzo entusiasta; no la letal monotonía del estancamiento.
\vs p068 2:2 Aunque el nivel de inteligencia ha contribuido de forma notable al ritmo del progreso cultural, la sociedad está fundamentalmente diseñada para reducir el factor de riesgo en el modo de vivir de las personas, y ha avanzado tan rápidamente como ha conseguido aminorar el dolor y aumentar el factor de placer en la vida. Por ello, todo el órgano social sigue adelante de forma lenta hacia su meta y destino ---supervivencia o extinción---, dependiendo de si la meta es autoconservación o autogratificación. La autoconservación da origen a la sociedad, mientras que la excesiva autogratificación la destruye.
\vs p068 2:3 La sociedad se preocupa de la autoperpetuación, de la autoconservación y de la autogratificación, pero la autorrealización humana merece convertirse en el objetivo inmediato de muchos grupos culturales.
\vs p068 2:4 El instinto de la manada del hombre natural es apenas suficiente para justificar el desarrollo de una organización social tal como la que existe ahora en Urantia. Aunque esta innata tendencia gregaria subyace en la sociedad humana, una gran parte de la sociabilidad del hombre es adquirida. Hubo dos grandes factores que contribuyeron a la temprana asociación de los seres humanos: el hambre de alimentos y el deseo sexual; el hombre comparte estos impulsos instintivos con el mundo animal. Las otras dos emociones que llevaron a los seres humanos a unirse y a \bibemph{mantenerse} juntos fueron la vanidad y el miedo, muy especialmente el miedo a los espíritus.
\vs p068 2:5 \pc La historia no es sino la documentación de la permanente lucha del hombre por el alimento. \bibemph{El hombre primitivo solo pensaba cuando tenía hambre;} guardar alimentos fue su primer acto de abnegación y de autodisciplina. Con el desarrollo de la sociedad, el hambre de comida cesó de ser el único aliciente para la asociación mutua. Numerosos otros tipos de hambre y la satisfacción de diversas necesidades llevaron a la asociación más estrecha de la humanidad. Pero, hoy día, la sociedad es inestable a causa de la proliferación desorbitada de supuestas necesidades humanas. La civilización occidental del siglo XX gime cansadamente bajo la presión de la ingente sobrecarga del lujo y de la multiplicación desmedida de deseos y anhelos humanos. La sociedad moderna se encuentra bajo la presión de uno de sus más peligrosos periodos debido a su extensa interasociación y a su sumamente compleja interdependencia.
\vs p068 2:6 El hambre, la vanidad y el miedo a los espíritus efectuaron una continua presión social, pero la gratificación sexual fue transitoria y errática. El impulso sexual por sí solo no indujo a los hombres y mujeres primitivos a asumir las pesadas cargas del mantenimiento del hogar. El hogar primitivo se fundó sobre la agitación sexual del varón, cuando se le privaba de una asidua gratificación, y sobre el dedicado amor maternal de la hembra humana, el cual, hasta cierto grado, comparte con las hembras de todos los animales superiores. La presencia de un niño indefenso determinó la primera diferenciación de las actividades del hombre y de la mujer; la mujer tenía que mantener una residencia fija donde pudiera cultivar la tierra. Y, desde los tiempos más primitivos, siempre se consideraba hogar allí donde se encontrara la mujer.
\vs p068 2:7 Así pues, la mujer se convirtió pronto en indispensable para la estructura social que se iba desarrollando, no tanto por la efímera pasión sexual sino como consecuencia de las \bibemph{necesidades alimenticias;} ella era una parte fundamental en la manutención. Fue proveedora de alimentos, bestia de carga y compañera capaz de soportar graves abusos sin resentimientos violentos y, además de todos estos rasgos deseables, era un permanente medio de gratificación sexual.
\vs p068 2:8 Casi todo lo que hay de valor perdurable en la civilización tiene sus raíces en la familia. La familia fue el primer exitoso grupo pacifista, pues, en ella, el hombre y la mujer aprendieron a reconciliar sus antagonismos y, al mismo tiempo, a enseñar a sus hijos a buscar la paz.
\vs p068 2:9 La función del matrimonio en la evolución es asegurar la supervivencia de la raza, no simplemente el logro de la felicidad personal; la autoconservación y la autoperpetuación son los verdaderos objetivos del hogar. La autocomplacencia es algo secundario y no esencial salvo como incentivo para garantizar la unión entre los sexos. La naturaleza exige la supervivencia, pero las artes de la civilización continúan incrementando los placeres del matrimonio y las satisfacciones de la vida familiar.
\vs p068 2:10 \pc Si se amplía el concepto de vanidad hasta abarcar el orgullo, la ambición y el honor, entonces se puede apreciar no solo cómo estas inclinaciones contribuyen a la formación de las asociaciones humanas, sino también cómo mantienen a los hombres unidos, ya que dichas emociones son fútiles sin un público ante el que exhibirse. Pronto la vanidad se relacionó con otras emociones e impulsos que implicaban un ámbito social en donde manifestarse y sentirse gratificados. Este grupo de emociones dio origen a los tempranos comienzos de todas las artes, de los ceremoniales y de todas las formas de juegos deportivos y competiciones.
\vs p068 2:11 La vanidad contribuyó poderosamente al nacimiento de la sociedad; pero, en el momento de estas revelaciones, los tortuosos afanes de una generación envanecida amenazan con empantanar toda la compleja estructura de una civilización altamente especializada. Hace mucho tiempo que la querencia del placer sustituyó a la de los alimentos; los fines sociales legítimos de la autoconservación se están transformando rápidamente en formas abyectas y amenazantes de autogratificación. La autoconservación forja sociedades; la autogratificación desmedida destruye indefectiblemente la civilización.
\usection{3. INFLUENCIA SOCIALIZADORA DEL MIEDO A LOS ESPÍRITUS}
\vs p068 3:1 Los deseos primitivos dieron lugar a la sociedad primigenia, pero el miedo a los espíritus de los muertos la mantuvo unida y trasmitió a su existencia un aspecto extrahumano. El miedo común tenía un origen fisiológico: miedo al dolor físico, al hambre insatisfecha o a alguna calamidad terrenal; pero el miedo a los espíritus era un tipo de terror nuevo y reverencial.
\vs p068 3:2 Probablemente, el soñar con los espíritus fue por sí solo el factor principal que facilitó el desarrollo de la sociedad humana. Aunque la mayoría de los sueños perturbaba enormemente a la mente primitiva, el soñar con espíritus realmente aterrorizaba a los primeros hombres, llevando a estos soñadores supersticiosos a extender sus brazos a los demás en su disposición seria a asociarse para protegerse mutuamente contra peligros que se mostraban vagos, invisibles e imaginarios del mundo de los espíritus. El sueño fantasmal constituyó una de las más tempranas diferencias entre la mente de orden animal y la de orden humano. Los animales no visibilizan la supervivencia después de la muerte.
\vs p068 3:3 Salvo por este elemento fantasmal, toda la sociedad se fundó sobre las necesidades fundamentales y los impulsos biológicos básicos. Si bien, el miedo a los espíritus introdujo un nuevo componente en la civilización, un miedo que sobrepasaba las necesidades elementales de las personas y que superaba las luchas por preservar el grupo. El temor a los espíritus de los difuntos puso de manifiesto una nueva y sorprendente forma de miedo, un terror atroz y poderoso que provocó que los dispersos órdenes sociales de las eras primitivas se hiciesen más sólidamente disciplinados y mejor dirigidos que los de los tiempos antiguos. Esta superstición sin sentido, algo que aún persiste, preparó las mentes de los hombres, mediante el miedo supersticioso a lo irreal y a lo sobrenatural, para el futuro reconocimiento de que “el principio de la sabiduría es el temor del Señor”. Los miedos infundados de la evolución están destinados a ser reemplazados por el sobrecogimiento reverente hacia la Deidad, que la Revelación inspira. El temprano culto del miedo a los espíritus se convirtió en un fuerte vínculo social y, desde aquel día remoto, la humanidad continúa esforzándose en mayor o menor medida para lograr la espiritualidad.
\vs p068 3:4 \pc El hambre y el amor impulsaron a los hombres a unirse; la vanidad y el miedo a los espíritus los mantuvieron unidos. Si bien, estas emociones por sí solas, sin la influencia de revelaciones promotoras de la paz, son incapaces de soportar la tensión de las desconfianzas y sentimientos de ira que se originan en las interrelaciones humanas. Sin la ayuda de causas sobrehumanas, las tensiones en la sociedad hacen que esta se quiebre, y estos mismos factores que causan la movilización social ---el hambre, el amor, la vanidad y el miedo--- se confabulan para sumir a la humanidad en la guerra y en el derramamiento de sangre.
\vs p068 3:5 La predisposición a la paz de la raza humana no es un bien natural; es producto de las enseñanzas de la religión revelada, de la experiencia acumulada de las razas avanzadas, pero sobre todo de las enseñanzas de Jesús, el Príncipe de la Paz.
\usection{4. EVOLUCIÓN DE LAS COSTUMBRES}
\vs p068 4:1 Todas las instituciones sociales modernas resultan de la evolución de las costumbres primitivas de vuestros ancestros salvajes; las convenciones de hoy son las costumbres, modificadas y ampliadas, del ayer. El hábito es para la persona lo que la costumbre es para el grupo; y las costumbres de los grupos se convierten con el tiempo en tradiciones populares o tribales ---en convenciones colectivas---. A partir de estos tempranos y humildes inicios se han originado todas las instituciones actuales de la sociedad humana.
\vs p068 4:2 Hay que tener en cuenta que las costumbres provienen del afán por adaptar la vida del grupo a las condiciones de la existencia colectiva; las costumbres fueron la primera institución social del hombre. Y todas estas reacciones tribales surgieron de iniciativas tomadas para evitar el dolor y la humillación, procurando, al mismo tiempo, conjugar placer y poder. El origen de las tradiciones populares, tal como el de las lenguas, es siempre inconsciente e involuntario y, por consiguiente, siempre envuelto en un halo de misterio.
\vs p068 4:3 \pc El miedo a los espíritus llevó al hombre primitivo a visibilizar lo sobrenatural, estableciendo así unas bases firmes para esas poderosas influencias sociales de la ética y la religión que, a su vez, conservaron inalteradas las costumbres y las tradiciones de la sociedad de generación en generación. Lo único que tempranamente estableció y cristalizó las costumbres fue la creencia de que los difuntos eran celosos guardianes de los hábitos en los que habían vivido y muerto; por tal motivo, infligirían un terrible castigo a los mortales vivos que osaran tratar con despreocupación y desdén las normas de vida que ellos habían respetado cuando estaban en la carne. Todo esto queda perfectamente ilustrado por la actual veneración que siente la raza amarilla por sus ancestros. La religión primitiva que se desarrolló posteriormente reforzó sensiblemente el miedo a los espíritus al preservar las costumbres; sin embargo, la civilización en su avance ha venido liberando cada vez más a la humanidad de las ataduras al miedo y de la esclavitud de la superstición.
\vs p068 4:4 Antes de las enseñanzas de los maestros de Dalamatia que lo liberaban y hacían más receptivos a nuevas ideas, el hombre ancestral era una víctima indefensa de los usos ritualistas; el salvaje primitivo estaba cercado por un sinfín de ceremoniales. Todo lo que hacía, desde que se despertaba por la mañana hasta el momento de dormir por la noche en su cueva, tenía que hacerlo de una determinada manera ---según las tradiciones de la tribu---. Era esclavo de la tiranía de los usos establecidos; en su vida no había nada libre, espontáneo ni original. No existía ningún progreso natural hacia una existencia mental, moral o social de orden superior.
\vs p068 4:5 El hombre primitivo estaba poderosamente atenazado por las costumbres; el salvaje era auténtico esclavo de los usos establecidos; pero en ocasiones surgían quienes se atrevían a variar tales usos e introducían nuevas formas de pensar y mejores sistemas de vida. No obstante, la inercia del hombre primitivo representa un seguro freno de tipo biológico, que evita una precipitación demasiado repentina en los desastrosos desajustes de una civilización que avance con extremada rapidez.
\vs p068 4:6 Sin embargo, estas costumbres no son un mal absoluto; su avance debe continuar. Emprender la transformación masiva de la civilización mediante una revolución radical podría resultar prácticamente letal. La costumbre ha sido el hilo conductor que ha mantenido a la civilización unida. El curso de la historia humana está lleno de los vestigios de costumbres descartadas y de prácticas sociales obsoletas; pero no ha perdurado ninguna civilización que haya abandonado sus costumbres, excepto para adoptar otras mejores y más aptas.
\vs p068 4:7 La supervivencia de una sociedad depende mayormente de la evolución gradual de sus costumbres, que surge del deseo de experimentación. Se plantean nuevas ideas ---y se entabla una oposición entre ellas---. Una civilización que progresa adopta la idea avanzada y perdura; el tiempo y las circunstancias acaban por seleccionar a la sociedad más apta para sobrevivir. Pero esto no significa que cada uno de los cambios individuales y aislados que se realizan en la composición de la sociedad humana haya sido para su mejora. ¡No! ¡Por supuesto que no! En Urantia, en la larga lucha por avanzar de su civilización, ha habido muchísimos retrocesos.
\usection{5. TÉCNICAS DEL MANEJO DEL SUELO: LAS ARTES DEL SUSTENTO}
\vs p068 5:1 El suelo es el escenario de la sociedad; los hombres, sus actores. Y el hombre debe adaptar constantemente sus actuaciones para amoldarse a las condiciones del suelo. La evolución de las costumbres siempre depende de la relación suelo\hyp{}hombre. Esto es cierto a pesar de que sea difícil de comprender. La técnica del manejo del suelo de parte del hombre, o las artes del sustento, más sus condiciones de vida, es igual a la suma total de las tradiciones populares, de las costumbres. Y la suma total de la adaptación del hombre a las exigencias de la vida es igual a su civilización cultural.
\vs p068 5:2 Las primeras culturas humanas surgieron a lo largo de los ríos del hemisferio oriental, dando lugar a las cuatro grandes etapas en el progreso de la civilización:
\vs p068 5:3 \li{1.}\bibemph{Etapa de recolección}. La compulsión por la comida, el hambre, llevó a la primera forma de trabajo organizado, a hileras de seres recopilando alimentos. A veces, en su marcha contra el hambre, al pasar por las tierras en donde rebuscaban la comida, se formaban filas de quince kilómetros de largo. Esta era constituye la etapa de la cultura nómada primitiva, y es el modo de vida que en la actualidad siguen los bosquimanos de África.
\vs p068 5:4 \li{2.}\bibemph{Etapa de caza.} La invención de armas rudimentarias permitió al hombre convertirse en cazador, consiguiendo pues liberarse de forma notable de su esclavitud a la comida. Un reflexivo andonita, que se había contusionado gravemente el puño en un duro combate, redescubrió la idea de usar un palo largo como su propio brazo y un trozo de pedernal duro, atado en el extremo con tendones, como su puño. Por su cuenta, muchas tribus hicieron descubrimientos de este tipo, y estas diferentes clases de mazos llegaron a representar uno de los más grandes avances de la civilización humana. Hoy en día algunos aborígenes australianos no han progresado mucho más allá de esta etapa.
\vs p068 5:5 Los hombres azules se convirtieron en cazadores y tramperos expertos; cercando los ríos, capturaban una gran cantidad de peces, desecando el excedente para consumirlo en el invierno. Para atrapar las presas, se usaban muchas formas ingeniosas de lazos y trampas, pero las razas más primitivas no cazaban animales de gran tamaño.
\vs p068 5:6 \li{3.}\bibemph{Etapa de pastoreo}. La domesticación de los animales posibilitó este estadio de la civilización. Los árabes y los originarios de África figuran entre los pueblos pastores más recientes.
\vs p068 5:7 La vida pastoril propició un alivió añadido ante la esclavitud de la comida; el hombre aprendió a vivir de los rendimientos de sus recursos, del incremento de sus rebaños; y esto le proporcionó más tiempo libre para la cultura y el progreso.
\vs p068 5:8 La sociedad prepastoril se caracterizó por la cooperación entre los sexos, pero la expansión de la cría de animales rebajó a la mujer al abismo de la esclavitud social. En tiempos pasados, era obligación del hombre conseguir alimentos de origen animal; la mujer se ocupaba de proporcionar productos vegetales comestibles. Por consiguiente, al incorporarse el hombre a la era pastoril, la dignidad de la mujer decayó enormemente. La mujer aún debía trabajar duro a fin de proveer los vegetales necesarios para la vida; mientras que el hombre solo tenía que acudir a sus rebaños para conseguir abundante alimentos de origen animal. El hombre se volvió, de este modo, relativamente independiente de la mujer; durante toda la época pastoril, el estatus de la mujer declinó constantemente. Al cierre de esta era, las mujeres se habían convertido en apenas algo más que en animales humanos, relegadas a trabajar y a dar a luz a la prole humana, de forma muy parecida a los animales del rebaño, de los que se esperaba que trabajaran y parieran las crías. Los hombres de las épocas pastoriles sentían un gran cariño por su ganado; resulta por ello tan lamentable que no hubiesen desarrollado un afecto más profundo hacia sus esposas.
\vs p068 5:9 \pc \bibemph{4. Etapa agrícola}. Esta era se produjo gracias a la domesticación de las plantas, y es representativa del más alto nivel de civilización de orden material. Tanto Caligastia como Adán procuraron enseñar horticultura y agricultura. Adán y Eva fueron horticultores, y no pastores, y la horticultura constituía en esos días una forma de cultura avanzada. El cultivo de plantas tiene un efecto ennoblecedor en todas las razas del género humano.
\vs p068 5:10 La agricultura cuadriplicó con creces la proporción suelo\hyp{}hombre del mundo. Se puede combinar con las actividades pastoriles de la anterior etapa cultural. Cuando se solapan las tres etapas, los hombres cazan y las mujeres cultivan la tierra.
\vs p068 5:11 Siempre han existido roces entre los pastores y los cultivadores del suelo. El cazador y el pastor eran combativos, belicosos; el agricultor es más amante de la paz. La asociación con los animales apunta a lucha y a fuerza; la asociación con las plantas infunde paciencia, paz y tranquilidad. La agricultura y el industrialismo son las actividades de la paz. Pero el punto débil de ambos, como actuaciones sociales en el mundo, es que carecen de emoción y aventura.
\vs p068 5:12 \pc La sociedad humana ha evolucionado desde la etapa de la caza hasta la etapa territorial de la agricultura, pasando por la pastoril. Y al progresar la civilización, cada una de estas etapas iba acompañada de la paulatina disminución del nomadismo; el hombre empezó a vivir en su hogar cada vez más.
\vs p068 5:13 Y, en la actualidad, es la industria la que está complementando a la agricultura, con el consiguiente aumento de la urbanización y de la multiplicación de grupos de ciudadanos no agrícolas. Pero una era industrial no tiene esperanzas de sobrevivir, si sus líderes no consiguen reconocer que incluso las más altas cotas de desarrollo social han de apoyarse siempre sobre una sólida base agrícola.
\usection{6. EVOLUCIÓN DE LA CULTURA}
\vs p068 6:1 El hombre es una criatura de la tierra, un hijo de la naturaleza; por muy encarecidamente que intente escapar del suelo, está, en definitiva, destinado al fracaso. La expresión “polvo eres y al polvo volverás” es verdaderamente aplicable, de forma literal, a toda la humanidad. La lucha fundamental del hombre fue, sigue siendo y siempre será una pugna por el suelo. Con el propósito de ganar este orden de lucha, se crearon las primeras sociedades de seres humanos primitivos. La proporción suelo\hyp{}hombre subyace en toda civilización social.
\vs p068 6:2 La inteligencia del hombre, sirviéndose de las artes y de las ciencias, incrementó el rendimiento del suelo; al mismo tiempo, el aumento natural de vástagos se pudo regular en cierta medida y, por consiguiente, se facilitó el sustento y el esparcimiento para construir una civilización cultural.
\vs p068 6:3 \pc La sociedad humana se rige por una ley que estipula que la población debe variar en proporción directa a las artes del manejo del suelo y en proporción inversa a un nivel de vida dado. A lo largo de todas estas tempranas eras, incluso más que en la actualidad, la ley de la oferta y la demanda en lo que se refiere a hombres y a suelo determinaba el valor previsto de ambos. Durante los tiempos en los que había cuantioso suelo ---territorios desocupados---, había una gran necesidad de hombres y, como consecuencia, el valor de la vida humana mejoró bastante en apreciación; de ahí que la pérdida de vidas fuese más terrible. Durante los períodos de escasez de suelo, a lo que se unía la sobrepoblación, la vida humana se desvalorizó correspondientemente, con lo que la guerra, el hambre y la peste se consideraban menos preocupantes.
\vs p068 6:4 Cuando decrece el rendimiento del suelo o aumenta la población, se reanuda la inevitable lucha; aflora la peor faceta de la naturaleza humana. La mejora en el rendimiento del suelo, la ampliación de las artes mecánicas y la reducción de la población tienden, en su totalidad, a promover el desarrollo del mejor lado de la naturaleza humana.
\vs p068 6:5 \pc Una sociedad fronteriza desarrolla el lado no especializado de la humanidad; las bellas artes y el verdadero progreso científico, junto con la cultura espiritual, han llegado a florecer más óptimamente en los mayores núcleos habitados, siempre que estuviesen apoyados por una población agrícola e industrial algo por debajo de la proporción suelo\hyp{}hombre. Las ciudades siempre multiplican la capacidad de sus habitantes para el bien o para el mal.
\vs p068 6:6 Los niveles de vida han tenido siempre efecto sobre el tamaño de la familia. Cuanto más elevado es el nivel de vida, más pequeña es la familia, hasta el punto de quedar fijada o extinguirse gradualmente.
\vs p068 6:7 A través de los tiempos, los niveles de vida han determinado la condición de la población superviviente en contraste con la mera cantidad. Los niveles de vida de las clases locales dan origen a nuevas castas sociales, a nuevas costumbres. Cuando dichos niveles llegan a ser demasiado complejos o excesivamente lujosos se convierten, rápidamente, en autodestructivos. Las castas son una consecuencia directa de la gran tensión social formada a raíz de la fuerte competencia que produce la densidad de la población.
\vs p068 6:8 A menudo, las primeras razas recurrían a prácticas encaminadas a restringir la población; todas las tribus primitivas mataban a los niños deformes o enfermizos. Con frecuencia, se mataba a las niñas pequeñas antes de la práctica de la venta de esposas. Algunas veces se estrangulaba a los hijos al nacer, pero el método preferido era exponerlos a las inclemencias del tiempo u a otros peligros. El padre de gemelos insistía normalmente en que se matara a uno de los dos, porque se pensaba que los nacimientos múltiples se debían a la magia o a la infidelidad. Por lo general, sin embargo, a los gemelos del mismo sexo se les perdonaba. Aunque, alguna vez, estos tabúes sobre los gemelos estaban prácticamente generalizados, nunca formaron parte de las costumbres de los andonitas; estos pueblos siempre consideraron a los gemelos como una señal de buena suerte.
\vs p068 6:9 Muchas razas aprendieron la técnica del aborto, y esta práctica se convirtió en algo bastante común después de que se implantara el tabú sobre el parto entre las no casadas. Durante mucho tiempo, las jóvenes solteras tuvieron la costumbre de matar a sus vástagos, pero, entre los grupos más civilizados, estos hijos ilegítimos quedaban bajo la custodia de la madre de la joven. La práctica del aborto y del infanticidio casi llevó a muchos clanes primitivos al exterminio. Pero, independientemente de los dictados de las costumbres, a muy pocos niños se les quitaba la vida tras haber sido alguna vez amamantados ---el cariño maternal es muy fuerte---.
\vs p068 6:10 Todavía en el siglo XX persisten vestigios de este control primitivo sobre la población. En Australia existe una tribu cuyas madres se niegan a criar a más de dos o tres hijos. No hace mucho que una tribu caníbal se comía a cada quinto hijo que naciera. En Madagascar, algunas tribus siguen quitándole la vida a todos los niños nacidos en ciertos días de mala suerte, resultando en la muerte de aproximadamente el veinticinco por ciento de todos los recién nacidos.
\vs p068 6:11 \pc Desde una perspectiva mundial, la sobrepoblación nunca fue un problema serio en el pasado, pero, si las guerras disminuyen y la ciencia adquiere un creciente control sobre las enfermedades humanas, esta puede convertirse, en un futuro cercano, en un grave dilema. En tal momento, la sabiduría de los líderes del mundo se verá sometida a una gran prueba. ¿Tendrán los dirigentes de Urantia la percepción y la valentía para fomentar la multiplicación del ser humano promedio o estable, en lugar de facilitársela a grupos extremos que o bien sobrepasan la media de normalidad o que, protagonistas de un enorme crecimiento, están por debajo de ella? Se debe fomentar al hombre normal; él es la columna vertebral de la civilización y el origen de los genios mutantes de la raza. Se ha de mantener al hombre por debajo de la normalidad bajo el control de la sociedad; no debe haber más de los que se necesiten para gestionar los niveles inferiores de la industria, esto es, esas tareas que precisan una inteligencia por encima del nivel animal, pero cuyo bajo grado de exigencia resultan una verdadera esclavitud y una servidumbre para los individuos mejores dotados de la humanidad.
\vsetoff
\vs p068 6:12 [Exposición de un melquisedec emplazado en otro tiempo a Urantia.]
