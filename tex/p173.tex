\upaper{173}{Lunes en Jerusalén}
\author{Comisión de seres intermedios}
\vs p173 0:1 Ese lunes por la mañana temprano, como se había acordado previamente, Jesús y los apóstoles se reunieron en Betania, en la casa de Simón y, tras una breve conversación, partieron para Jerusalén. De camino al templo, los doce estaban extrañamente silenciosos; no se habían recuperado de lo vivido el día anterior. Iban expectantes, temerosos y profundamente afectados por un cierto sentimiento de desapego de la realidad debido al repentino cambio de táctica del Maestro, junto a sus instrucciones de que no impartieran ninguna enseñanza pública durante la semana de Pascua.
\vs p173 0:2 Cuando el grupo bajaba del Monte de los Olivos, Jesús marchaba delante y los apóstoles lo seguían de cerca en un reflexivo silencio. Exceptuando a Judas Iscariote, solo un pensamiento predominaba en sus mentes, y este era: ¿Qué hará hoy el Maestro? En el caso de Judas, el pensamiento que lo absorbía era: ¿Qué haré? ¿Continúo con Jesús y mis compañeros o debo alejarme? Y si los dejo, ¿cómo romperé con ellos?
\vs p173 0:3 Estos hombres llegaron al templo sobre las nueve de aquella hermosa mañana. Enseguida se encaminaron al gran patio en el que Jesús tantas veces había enseñado y, después de saludar a los creyentes que lo estaban esperando, subió a una de las tarimas de los enseñantes y empezó a dirigirse a la multitud que se aglomeraba allí. Los apóstoles se apartaron a corta distancia y aguardaron acontecimientos.
\usection{1. PURIFICACIÓN DEL TEMPLO}
\vs p173 1:1 En relación con los servicios y las ceremonias del sistema de culto del templo se había creado una enorme actividad comercial, un negocio en el que se proporcionaban animales propicios para los distintos sacrificios. Aunque estaba permitido que los devotos aportaran su propio animal para el sacrificio, lo cierto era que este animal debía estar libre de todo “defecto” en el sentido de la ley levítica y tal como la interpretaban los inspectores oficiales del templo. Muchos de los fieles se habían visto humillados al rechazárseles un animal, supuestamente perfecto, por estos evaluadores del templo. Por lo tanto, se generalizó la práctica de adquirir a los animales sacrificiales en el templo mismo y, aunque había varios lugares de venta en las cercanías del Monte de los Olivos, estaba en boga obtenerlos directamente de los corrales del templo. De forma paulatina, se había desarrollado esta costumbre de vender toda clase de animales para los sacrificios en los patios del templo. Se había generado un gran negocio del que se obtenían importantes lucros. Una parte de estas ganancias se reservaban para la tesorería del templo, pero la mayor parte de esta iba indirectamente a las manos de las familias gobernantes de sumos sacerdotes.
\vs p173 1:2 Esta venta de animales del templo se impuso porque, cuando el devoto compraba algún animal, aunque el precio podría ser algo alto, no tenía que pagar ninguna otra tarifa, y podía estar seguro de que no se lo rechazarían por razones de defectos reales o legalistas. En distintos momentos del pasado, se hizo habitual la práctica de hacer pagar un recargo exorbitante a la gente común, en especial durante las grandes fiestas nacionales. Hubo ocasiones en las que los avariciosos sacerdotes llegaron hasta el extremo de exigir el equivalente de una semana de trabajo por un par de palomas, cuando deberían haberse vendido a los pobres por unos pocos cuartos. Los “hijos de Anás” habían ya empezado a instalar sus bazares en los recintos del templo; se trataban de verdaderos puestos de mercaderías, que subsistirían hasta que una turba los derribó tres años antes de la destrucción del mismo templo.
\vs p173 1:3 \pc Pero el comercio de animales para los sacrificios y el de los distintos tipos de mercaderías no era la única manera en la que se profanaban los patios del templo. En esos días, se propició una extensa red de intercambio bancario y comercial que realizaba su actividad directamente en los recintos del templo. Y todo esto ocurrió de la siguiente manera: durante la dinastía de los Asmoneos, los judíos acuñaban su propia moneda de plata, y se había regularizado la práctica de exigir que las cuotas del templo de medio siclo y todas las otras tarifas del templo se pagaran en esta moneda judía. Esta norma exigía que se autorizara a los cambistas a canjear los numerosos tipos de moneda que circulaban por toda Palestina y las demás provincias del Imperio romano por este tradicional siclo de acuñación judía. El impuesto personal del templo, a pagar por todos excepto por las mujeres, los esclavos y los menores, era de medio siclo, una moneda del tamaño aproximado a la de diez centavos de dólar, pero del doble de grosor. En la época de Jesús, los sacerdotes también estaban eximidos del pago de las cuotas del templo. Así pues, entre los días 15 y 25 del mes previo a la Pascua, los cambistas acreditados montaban sus puestos en las principales ciudades de Palestina con el fin de proporcionar a los judíos el dinero conveniente para abonar las cuotas del templo tras llegar a Jerusalén. Después de este período de diez días, estos cambistas se trasladaban a Jerusalén y colocaban sus mesas de canje en los patios del templo. Se les permitía cobrar una comisión equivalente a tres o cuatro centavos cuando cambiaban una moneda valorada en diez, y en el caso del cambio de una moneda de mayor valor, se les autorizaba a cobrar el doble. De igual manera, estos banqueros del templo sacaban provecho del cambio de cualquier dinero destinado a la compra de animales para el sacrificio y el pago de sus solemnes promesas y ofrendas.
\vs p173 1:4 Estos cambistas del templo no solo desarrollaban regularmente una lucrativa actividad bancaria en el intercambio de más de veinte clases de monedas que los peregrinos visitantes traían periódicamente a Jerusalén, sino que también gestionaban todos los demás tipos de operaciones bancarias. Tanto la tesorería como los dirigentes del templo se beneficiaban enormemente de estos intercambios comerciales. No era infrecuente que la tesorería del templo dispusiera de más de diez millones de dólares, mientras que la gente común se sumía en la pobreza y pagaba estas injustas tasas.
\vs p173 1:5 \pc Jesús, ese lunes por la mañana, en medio de esta aglomeración ruidosa de cambistas, comerciantes y vendedores de ganado, trató de enseñar el evangelio del reino celestial. No era el único en sentirse ofendido por aquella profanación del templo; a la gente común, en particular a los visitantes judíos de las provincias extranjeras, también les molestaba enormemente la desacralización especulativa de su casa de culto nacional. En esos días, el mismo sanedrín celebraba sus reuniones periódicas en una sala rodeada de toda esta algarabía y confusión del comercio y del trueque.
\vs p173 1:6 Cuando Jesús se disponía a dar su charla, hubo dos incidentes que captaron su atención. En la mesa de uno de los cambistas allí cercanos, se había originado una violenta y acalorada riña por el supuesto recargo cobrado a un judío de Alejandría, mientras que, en el mismo momento, el aire se cargó con los mugidos de una manada de unos cien bueyes que se llevaban desde una de las zonas de los corrales de animales a otra. Al detenerse Jesús y contemplar silenciosa pero pensativamente aquella escena de comercio y confusión, vio, en la proximidad, a un sencillo galileo, un hombre con el que había hablado cierta vez en Irón, a quien ridiculizaban y empujaban unos arrogantes ciudadanos de Judea con aires de superioridad; y todo este cúmulo de cosas produjo en el alma de Jesús uno de esos extraños y ocasionales brotes de indignación.
\vs p173 1:7 Ante el estupor de sus apóstoles, que se hallaban cerca, y que se abstuvieron de participar en lo que tan repentinamente ocurriría, Jesús bajó de la tarima desde la que enseñaba y, yendo hasta el joven que llevaba el ganado por el patio, le quitó el látigo de cuerdas y llevó él mismo rápidamente a los animales fuera del templo. Pero aquello no fue todo; ante la mirada de asombro de los miles de personas aglomeradas en el patio del templo, se encaminó majestuosamente y con determinación hasta el corral de ganado más alejado y se puso a abrir las puertas de cada uno de los establos, sacando a los animales de su encierro. En ese momento, los peregrinos, allí reunidos, conmocionados y gritando estrepitosamente, se abalanzaron a los bazares y empezaron a volcar las mesas de los cambistas. En menos de cinco minutos, se barrió todo comercio del templo. Cuando aparecieron en escena los guardias romanos que estaban en las proximidades, todo estaba tranquilo y las multitudes habían vuelto a la calma; Jesús, regresando al estrado de los oradores, habló a la multitud con estas palabras: “Hoy sois testigos de que está escrito en las Escrituras: ‘Mi casa será llamada casa de oración para todas las naciones, pero vosotros la habéis hecho cueva de ladrones’”.
\vs p173 1:8 Pero antes de que pudiera expresar otras palabras, aquella gran congregación de personas prorrumpió en hosannas de alabanza y, al momento, un grupo de jóvenes se separó de la multitud cantando himnos en agradecimiento de que aquellos comerciantes profanos y especuladores hubieran sido expulsados del templo sagrado. Para entonces, algunos sacerdotes habían llegado a la escena, y uno de ellos dijo a Jesús: “¿Es que no oyes lo que dicen los hijos de los levitas?”. Y el Maestro contestó: “¿Nunca leísteis ‘de la boca de los niños y de los que aún maman se ha perfeccionado la alabanza?’”. Y todo el resto del día, mientras Jesús enseñaba, unos guardianes, puestos por la gente, estuvieron vigilando cada entrada y no permitieron que nadie portara ni siquiera una vasija vacía por los patios del templo.
\vs p173 1:9 \pc Cuando los sumos sacerdotes y los escribas conocieron estos hechos, se quedaron mudos de estupor. Todavía temían más al Maestro que antes, y aún estaban más empeñados en matarlo. Pero estaban desconcertados. No sabían cómo hacerlo por este gran temor a las multitudes que, de forma tan impulsiva, habían dado entonces su aprobación a la expulsión por parte de Jesús de los irreverentes especuladores. Y durante todo ese día, un día de tranquilidad y paz en los patios del templo, la gente oyó las enseñanzas de Jesús y estuvo muy pendiente de sus palabras.
\vs p173 1:10 Este sorprendente acto de Jesús sobrepasó la comprensión de sus apóstoles. Estaban tan estupefactos por aquel repentino e inesperado comportamiento de su Maestro que permanecieron apiñados cerca de la tarima de los oradores durante todo el incidente; no levantaron un dedo por dar su apoyo a aquella purificación del templo. Si este espectacular suceso hubiera ocurrido un día antes, en el momento de la llegada triunfal de Jesús al templo tras el paso de la tumultuosa comitiva por las puertas de la ciudad, y mientras la multitud lo aclamaba a voces, habrían estado listos para aquello, pero tal como se presentaron los acontecimientos, no estaban en absoluto preparados para involucrarse.
\vs p173 1:11 Esta purificación del templo desvela la actitud del Maestro con respecto a la comercialización de la religión al igual que su aborrecimiento de cualquier forma de injusticia y especulación a expensas de los pobres y de los iletrados. Este incidente demuestra asimismo que Jesús daba su aprobación al empleo de la fuerza, cuando era necesario defender a la mayoría de cualquier grupo humano dado contra las prácticas abusivas y esclavizantes de minorías injustas, que pudieran escudarse tras el poder político, económico o eclesiástico. No se debe permitir que haya gente astuta, perversa y maquinadora que se organice para la explotación y opresión de quienes, debido a su idealismo, no están dispuestos a recurrir a la fuerza para protegerse a sí mismos ni para promover sus encomiables proyectos de vida.
\usection{2. RETO A LA AUTORIDAD DEL MAESTRO}
\vs p173 2:1 El domingo, la entrada triunfal del Maestro en Jerusalén alarmó tanto a los líderes judíos que se retuvieron de arrestar a Jesús. Ese día, aquella espectacular purificación del templo logró posponer igualmente su detención. Día tras día, los dirigentes judíos estaban cada vez más convencidos de la necesidad de matarlo, pero les angustiaban dos temores, que contribuían a demorar la hora de llevar esto a cabo. Los sumos sacerdotes y los escribas eran reticentes a arrestar a Jesús en público, porque temían que la multitud se volviera contra ellos enfurecida de indignación; también les aterraba la posibilidad de que se llamara a los guardias romanos para sofocar un levantamiento popular.
\vs p173 2:2 En su sesión del mediodía, el sanedrín, puesto que ningún amigo del Maestro asistía a dicha reunión, acordó por unanimidad que Jesús debía morir rápidamente. Si bien, no pudieron llegar a ningún acuerdo respecto a cuándo y cómo debía ser detenido. Por último, convinieron en nombrar, entre ellos, a cinco grupos que fueran en medio de la gente e intentaran enredarlo en sus enseñanzas o desacreditarlo de cualquier otra manera en presencia de quienes lo escuchaban. Por consiguiente, a eso de las dos de la tarde, cuando Jesús acababa de comenzar su charla sobre “La libertad de la filiación”, un grupo de estos ancianos de Israel se abrió paso hasta acercarse a Jesús e, interrumpiéndolo como era habitual en ellos, le hicieron esta pregunta: “¿Con qué autoridad haces estas cosas? ¿Quién te dio esta autoridad?”.
\vs p173 2:3 Era por completo pertinente que los dirigentes del templo y los oficiales del sanedrín judío formularan esta pregunta a quien pretendiera enseñar y actuar de la forma excepcional que había caracterizado a Jesús, en especial en cuanto a su reacción de limpiar el templo de mercaderías. Los comerciantes y cambistas operaban con un permiso directamente concedido por los más altos dirigentes del templo, y se suponía que un porcentaje de sus ganancias iba de inmediato a la tesorería. Que no se os olvide que la consigna del judaísmo era la \bibemph{autoridad}. Los profetas siempre suscitaban problemas porque tenían el atrevimiento de enseñar sin autoridad, sin haberse formado debidamente en las academias rabínicas ni haber sido ordenados legítimamente por el sanedrín. La falta de esta autoridad en esta pretendida enseñanza pública se veía como un indicativo de ignorante arrogancia o de manifiesta rebeldía. En esos días, solo el sanedrín podía ordenar a un líder religioso o a un maestro, y tal ceremonia debía celebrarse en presencia de un mínimo de tres personas que hubieran sido previamente ordenadas del mismo modo. Esta ordenación confería el título de “rabino” al maestro y también lo cualificaba para actuar como juez, “atando y desatando aquellos asuntos que pudieran llevarle ante él para su resolución”.
\vs p173 2:4 En esa hora de la tarde, los dirigentes del templo se presentaron ante Jesús desafiando no solo sus enseñanzas sino sus actos. Jesús era bien consciente de que aquellos mismos hombres habían enseñado públicamente durante mucho tiempo que la autoridad con la que él enseñaba era satánica, y que todas sus portentosas obras habían sido hechas por el poder del príncipe de los diablos. Por ello, el Maestro comenzó su respuesta, haciéndoles otra pregunta. Dijo Jesús: “Yo también os haré una pregunta, y si me la contestáis, también os diré con qué autoridad hago estas cosas. El bautismo de Juan, ¿de dónde era? ¿Venía su autoridad del cielo o de los hombres?”.
\vs p173 2:5 Y cuando los que lo interrogaban oyeron esto, se apartaron a un lado para consultarse entre ellos sobre la respuesta a dar. Habían planeado avergonzar a Jesús en presencia de la multitud, pero ahora eran ellos mismos quienes se encontraban muy confundidos ante todos lo que se encontraban en ese momento congregados en el patio del templo. Y su desconcierto fue incluso más evidente cuando regresaron hasta Jesús, diciendo: “En cuanto al bautismo de Juan, no podemos responder; no lo sabemos”. Respondieron así al Maestro porque habían discutido entre sí, diciendo: si decimos ‘del cielo’, nos dirá entonces: ‘Por qué no lo creísteis’, y quizás añada que él recibió su autoridad de Juan; y si decimos que de los hombres, seguidamente la multitud puede volverse contra nosotros, porque la mayoría tiene a Juan por profeta; y, así pues, se vieron forzados a venir ante Jesús y la multitud, y confesar que ellos, los maestros y líderes religiosos de Israel, no podían (o no querían) dar su opinión respecto a la misión de Juan. Y cuando acabaron de hablar, Jesús, bajando su mirada hacia ellos desde su posición en alto, les dijo: “Tampoco yo os diré con qué autoridad hago estas cosas”.
\vs p173 2:6 \pc Jesús nunca pretendió apelar a Juan para reivindicar su propia autoridad; el sanedrín nunca había ordenado a Juan. La autoridad de Jesús estaba en él mismo y en la supremacía eterna de su Padre.
\vs p173 2:7 Al hacer frente a sus adversarios de aquel modo, Jesús no tuvo la intención de esquivar la pregunta. En un principio, parecería que se le podría culpar de evadirla habilidosamente, pero no fue así. Jesús nunca quería aprovecharse injustamente ni incluso de sus enemigos. Con esta aparente evasiva, lo que en realidad hizo fue dar a los asistentes la respuesta a la pregunta de los fariseos respecto a la autoridad que sustentaba su misión. Estos habían declarado que él actuaba por autoridad del príncipe de los diablos. Y, repetidas veces, Jesús había afirmado que todas sus enseñanzas y obras se realizaban por poder y autoridad de su Padre de los cielos. Los líderes judíos se negaban a aceptar aquello y estaban tratando de acorralarlo para que admitiera que era un maestro no autorizado por el sanedrín. Al responderles como lo hizo, y aunque no reclamó que su autoridad viniera de Juan, sí satisfizo a la gente al sacar la conclusión de que el esfuerzo de sus enemigos por enredarlo se había vuelto de hecho en contra de ellos mismos, y que los había desacreditado considerablemente a los ojos de todos los presentes.
\vs p173 2:8 Y este excepcional talento del Maestro en el trato con sus enemigos lo hacía muy temido. No intentaron hacer más preguntas aquel día; se retiraron para seguir consultándose entre ellos. Pero la multitud no tardó en percibir la falta de honestidad e insinceridad de estas preguntas de los dirigentes judíos. Incluso la gente común logró distinguir entre la majestuosidad moral del Maestro y la taimada hipocresía de sus enemigos. Pero la purificación del templo hizo que los saduceos se unieran a los fariseos para planificar de la mejor forma la muerte de Jesús. Y los saduceos representaban entonces una mayoría en el sanedrín.
\usection{3. LA PARÁBOLA DE LOS DOS HIJOS}
\vs p173 3:1 Estando los capciosos fariseos allí, en silencio, ante Jesús, él miró hacia abajo, desde el estrado, y les dijo: “Puesto que ponéis en duda la misión de Juan y sentís animadversión por las enseñanzas y las obras del Hijo del Hombre, prestad oído mientras os digo una parábola: Un gran y respetado propietario tenía dos hijos, y buscando su ayuda para administrar sus grandes heredades, se acercó a uno de ellos y le dijo: ‘Hijo, vete hoy a trabajar en mi viña’. Y este irreflexivo hijo respondió al padre, diciendo: ‘No iré’; pero después, arrepentido, fue. Cuando encontró a su hijo mayor, le dijo lo mismo: ‘Hijo, vete hoy a trabajar en mi viña’. Y este hijo hipócrita y desleal contestó: ‘Sí, padre mío, iré’. Pero cuando su padre se marchó, no fue. Os pregunto ahora, ¿cuál de los dos hijos hizo realmente la voluntad de su padre?”.
\vs p173 3:2 Y, de común acuerdo, la gente habló, diciendo: “El primero”. Y entonces dijo Jesús: “Así es; y ahora de cierto os digo que los publicanos y las rameras, aunque parecen rechazar la llamada al arrepentimiento, verán el error en su camino y van delante de vosotros al reino de Dios, que con vanagloria pretendéis servir al Padre de los cielos, mientras os negáis a hacer sus obras. No fuisteis vosotros, fariseos y escribas, los que creísteis en Juan, sino los publicanos y los pecadores; tampoco creéis en mis enseñanzas, pero la gente común oye con alegría mis palabras”.
\vs p173 3:3 Jesús no despreciaba personalmente ni a fariseos ni a saduceos. Era sus sistemas de enseñanzas y prácticas los que quería desacreditar. No era hostil a ningún hombre, pero se estaba produciendo aquí la inevitable confrontación entre una religión del espíritu, nueva y viva, y la más antigua religión de las ceremonias, la tradición y la autoridad.
\vs p173 3:4 Todo este tiempo, los doce apóstoles estuvieron cerca del Maestro, pero no intervinieron en estos hechos. Cada uno de los doce fue reaccionando según su propia y peculiar manera a los acontecimientos de aquellos últimos días del ministerio de Jesús en la carne, y cada cual permaneció obediente al mandato del Maestro de abstenerse de enseñar y predicar públicamente durante la semana de Pascua.
\usection{4. LA PARÁBOLA DEL ARRENDADOR AUSENTE}
\vs p173 4:1 Cuando los principales fariseos y escribas que habían tratado de enredar a Jesús con sus preguntas acabaron de escuchar la historia de los dos hijos, se apartaron para deliberar nuevamente entre ellos, y el Maestro, dirigiendo su atención a la multitud que le escuchaba, dijo otra parábola:
\vs p173 4:2 \pc “Un buen hombre, propietario de unas tierras, plantó una viña. La rodeó con una cerca de setos, cavó un lagar y edificó una torre para los guardias. Luego la alquiló a unos arrendatarios y se fue en un largo viaje a otro país. Cuando se acercaba la temporada de los frutos, envió a sus siervos a los arrendatarios para recibir su renta. Pero estos, consultándose entre ellos, se negaron a dar a los siervos los frutos que le debían al amo; al contrario, cayendo sobre ello, golpearon a uno, apedrearon a otro y enviaron a los demás con las manos vacías. Y cuando el propietario oyó todo esto, envió a otros siervos suyos, de mayor confianza, para que trataran con estos malvados arrendatarios, y a estos los hirieron y también los insultaron. Y entonces el amo envió a su siervo favorito, a su mayordomo, y a este lo mataron. Y, aun así, con paciencia e indulgencia, volvió a mandar a otros muchos siervos, pero no recibieron a ninguno. A unos los golpearon y a otros los mataron, y, ante este comportamiento, el propietario este decidió enviar a su hijo a tratar con estos ingratos arrendatarios, diciéndose: ‘Puede que traten mal a mis siervos, pero seguro que tendrán respeto a mi hijo amado’. Pero cuando los arrendatarios impenitentes y perversos vieron al hijo, dijeron entre sí: ‘Este es el heredero; venid, matémoslo y la heredad será nuestra’. Entonces lo tomaron y, después de echarlo de la viña, lo mataron. Cuando el señor de la viña oiga cómo rechazaron y mataron a su hijo, ¿qué hará con estos arrendatarios ingratos y malvados?’”.
\vs p173 4:3 \pc Y cuando la gente oyó esta parábola y la pregunta que Jesús les hizo, contestaron: “Destruirá a esos miserables y arrendará su viña a otros labradores honrados que le paguen el fruto a su tiempo”. Y cuando algunos de los asistentes percibieron que esta parábola se refería a la nación judía y al trato que le había dado a los profetas y al inminente rechazo de Jesús y del evangelio del reino, dijeron afligidos: “No permita Dios que sigamos haciendo estas cosas”.
\vs p173 4:4 Jesús vio a un grupo de saduceos y fariseos abriéndose camino entre la multitud, e hizo una pausa por un momento hasta que llegaron junto a él, entonces dijo: “Sabéis de qué manera vuestros padres rechazaron a los profetas, y sabéis bien que habéis determinado en vuestros corazones a rechazar al Hijo del Hombre”. Y, luego, mirando inquisitivamente a los sacerdotes y ancianos que estaban próximos a él, Jesús dijo: “¿Ni aun habéis leído en las Escrituras sobre la piedra que desecharon los edificadores, y que, cuando la gente la descubrió, ha venido a ser cabeza del ángulo? Y así pues, os advierto una vez más que, si continuáis rechazando este evangelio, el reino de Dios pronto será quitado de vosotros y será dado a un pueblo que esté dispuesto a recibir la buena nueva y a producir los frutos del espíritu. Y hay un misterio en esta piedra, ya que el que caiga sobre ella, aunque por ello será quebrantado, se salvará; pero sobre quien ella caiga será desmenuzado y sus cenizas esparcidas a los cuatro vientos.
\vs p173 4:5 Cuando los fariseos oyeron estas palabras, entendieron que Jesús se refería a ellos mismos y a los demás líderes judíos. Desearon vivamente prenderlo allí y en aquel momento, pero tenían miedo de la multitud. No obstante, las palabras del Maestro les habían enfurecido tanto que se apartaron y otra vez deliberaron entre ellos sobre cómo acabar con su vida. Esa noche, tanto los saduceos como los fariseos aunaron sus esfuerzos para planear cómo tenderle una trampa al día siguiente.
\usection{5. LA PARÁBOLA DE LA FIESTA DE BODAS}
\vs p173 5:1 Tras retirarse los escribas y los dirigentes de los judíos, Jesús se dirigió de nuevo a la multitud allí congregada y le contó la parábola de la fiesta de bodas, diciendo:
\vs p173 5:2 \pc “El reino de los cielos es semejante a un rey que hizo una fiesta de boda a su hijo y envió mensajeros a llamar a los que ya estaban invitados a la fiesta, diciendo: ‘Todo está preparado para la cena de boda en el palacio del rey’. Si bien, muchos de los que ya habían prometido asistir, se negaron a ir. Cuando el rey oyó que rechazaban su invitación, envió a otros siervos y mensajeros, diciendo: ‘Decid a todos los invitados que vengan, porque he aquí que he preparado mi comida. He hecho matar a mis bueyes y a mis animales engordados, y todo está dispuesto para la celebración de la próxima boda de mi hijo’. Pero, de nuevo, estos desconsiderados, sin hacer caso de la llamada de su rey, se fueron cada uno por su camino, uno a su labranza, otro a su cerámica y otros a sus negocios. Otros además no se contentaron con menospreciar esta llamada del rey, sino que, rebelándose abiertamente, tomaron a los mensajeros y los escarnecieron, matando además a algunos de ellos. Y cuando el rey se enteró de que los convidados que él había elegido, incluso los que ya habían aceptado su invitación y habían prometido asistir a la fiesta de bodas, habían acabado por rechazar su llamada y en rebeldía habían asaltado y matado a mensajeros suyos escogidos, se enojó enormemente. Y, entonces, este rey, sintiéndose insultado, envió a sus ejércitos y a los ejércitos de sus aliados y los mandó que mataran a aquellos homicidas rebeldes y que quemaran su ciudad.
\vs p173 5:3 “Y después de haber castigado a los que habían desdeñado su invitación, fijó otro día para la fiesta de bodas y dijo a sus mensajeros: ‘Los que fueron invitados a la boda no eran dignos; id, pues, a las salidas de los caminos y a las carreteras e incluso más allá de los confines de la ciudad, y a cuantos halléis, aunque esos sean desconocidos, invitadlos a que vengan y asistan a esta fiesta de bodas’. Y entonces salieron los siervos por las carreteras y por lugares apartados, y reunieron a todos los que encontraron, tantos malos como buenos, ricos y pobres, hasta que finalmente la sala nupcial se llenó de invitados. Cuando todo estaba preparado, el rey entró para ver a sus invitados y, para su gran sorpresa, vio allí a un hombre que no estaba vestido apropiadamente. El rey, puesto que había proporcionado gratuitamente trajes de boda a todos sus invitados, le dijo: ‘Amigo, ¿cómo entraste aquí a mi sala de invitados sin estar vestido para esta ocasión’? Y este hombre negligente guardó silencio. Entonces dijo el rey a los que servían: ‘Echad a este irrespetuoso invitado de mi casa que corra la misma suerte de todos los demás que despreciaron mi hospitalidad y rechazaron mi llamada. No tendré aquí a nadie sino a los que se complacen en aceptar mi invitación y me hacen el honor de llevar esos trajes de boda que tan generosamente les han facilitado a todos’”.
\vs p173 5:4 \pc Tras contarles esta parábola, Jesús estaba ya a punto de despedir a la multitud cuando un creyente, que lo respaldaba, abriéndose paso entre ellos, y llegándose a él, preguntó: “Pero, Maestro, ¿cómo sabremos de estas cosas? ¿Cómo podremos estar preparados para la invitación del rey? ¿Qué señal nos darás para que sepamos que eres el Hijo de Dios?”. Y cuando el Maestro oyó esto, dijo: “Solo os será dada una señal”. Y, entonces, señalando hacia su propio cuerpo, continuó: “Destruid este templo, y en tres días lo levantaré”. Pero no lo entendieron y, cuando se dispersaban, hablaban entre ellos diciendo: “En casi cincuenta años fue edificado este templo, y sin embargo él dice que lo destruirá y levantará en tres días”. Ni incluso sus propios apóstoles comprendieron el significado de sus palabras, pero, luego, después de su resurrección, recordaron lo que había dicho.
\vs p173 5:5 Sobre las cuatro de esa tarde, Jesús hizo señas a sus apóstoles y les indicó que deseaba salir del templo e ir a Betania para cenar y descansar por la noche. Mientras subían el Monte de los Olivos, Jesús dio instrucciones a Andrés, Felipe y Tomás para que, al día siguiente, instalaran un campamento cerca de la ciudad, que ocuparían durante el resto de la semana de Pascua. Cumpliendo estas instrucciones, la mañana siguiente plantaron sus tiendas en la quebrada de una colina, desde la que se veía el parque del campamento público de Getsemaní, en una parcela de terreno perteneciente a Simón de Betania.
\vs p173 5:6 Aquel lunes por la noche, una vez más, un callado grupo de judíos se dirigía camino arriba por la ladera occidental del Monte de los Olivos. Como nunca antes, estos doce hombres comenzaron a sentir que algo trágico les sobrevendría. Mientras que, en aquella mañana temprano, la asombrosa purificación del templo había hecho que se incrementaran sus esperanzas de ver al Maestro hacerse valer y manifestar sus portentosos poderes, los sucesos de toda la tarde solo les causaba decepción ante el innegable rechazo de las enseñanzas de Jesús que todos habían observado por parte de las autoridades judías. Se sintieron atenazados por la aprehensión y presos de una temible incertidumbre. Se daban cuenta de que solo mediarían unos pocos y cortos días entre los sucesos del día que acababa de terminar y el colapso de una inminente fatalidad. Todos percibían que algo terrible estaba a punto de suceder, pero no sabían qué esperar. Fueron a sus respectivos lugares para descansar, pero durmieron muy poco. Incluso los gemelos Alfeo llegaron por fin a tomar conciencia de que los acontecimientos de la vida del Maestro avanzaban con rapidez hacia su desenlace final.
