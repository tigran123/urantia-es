\upaper{143}{A través de Samaria}
\author{Comisión de seres intermedios}
\vs p143 0:1 A final de junio del año 27 d. C., debido a la creciente hostilidad de los dignatarios religiosos judíos, Jesús y los doce abandonaron Jerusalén tras enviar sus tiendas y exiguos efectos personales a la casa de Lázaro, en Betania, para dejarlos allí almacenados. De camino al norte, hacia Samaria, permanecieron durante el \bibemph{sabbat} en Betel, predicando allí durante varios días a la gente que venía de Gofna y Efraín. Un grupo de ciudadanos de Arimatea y Tamna llegó para invitar a Jesús a que visitara sus aldeas. El Maestro y sus apóstoles pasaron más de dos semanas enseñando a los judíos y a los samaritanos de esta región, muchos de los cuales acudían hasta de lugares tan lejanos como Antipatris para oír la buena nueva del reino.
\vs p143 0:2 Los habitantes de Samaria del sur oyeron gratamente a Jesús, y los apóstoles, salvo Judas Iscariote, lograron superar gran parte de sus prejuicios contra los samaritanos. A Judas le resultaba muy difícil sentir afecto por estos samaritanos. La última semana de julio, Jesús y sus acompañantes se dispusieron a partir hacia las nuevas ciudades griegas de Fasaelis y Arquelais, cercanas al Jordán.
\usection{1. PREDICACIÓN EN ARQUELAIS}
\vs p143 1:1 Durante la primera mitad del mes de agosto, el grupo apostólico estableció su sede en las ciudades griegas de Arquelais y Fasaelis, donde tuvieron su primera experiencia enseñando casi exclusivamente a concentraciones de gentiles ---griegos, romanos y sirios---; eran pocos los judíos que vivían en estas dos ciudades griegas. Al entrar en contacto con estos ciudadanos romanos, los apóstoles encontraron dificultades a la hora de proclamar el mensaje del reino venidero e inconvenientes hacia las enseñanzas de Jesús. En una de las muchas charlas nocturnas con sus apóstoles, Jesús escuchó con atención las objeciones surgidas contra el evangelio del reino, a medida que los doce comentaban las experiencias tenidas en su labor personal con la gente.
\vs p143 1:2 Felipe formuló una pregunta que evidenciaba las dificultades por las que estaban atravesando. Felipe dijo: “Maestro, estos griegos y romanos restan importancia a nuestro mensaje, aduciendo que estas enseñanzas son válidas solo para los débiles y los esclavos. Afirman que la religión de los paganos es superior a nuestras enseñanzas, porque estimulan la formación de un carácter fuerte, sólido y vigoroso. Piensan que queremos convertir a los hombres en especímenes endebles de sumisos pusilánimes, que muy pronto desaparecerían de la faz de la tierra. Tú les agradas, Maestro, y reconocen abiertamente que tus enseñanzas son celestiales e ideales, pero no nos toman en serio a nosotros. Creen que tu religión no es para este mundo; que los hombres no pueden vivir de acuerdo a lo que enseñas. Y, entonces, Maestro, ¿qué le podemos decir estos gentiles?”.
\vs p143 1:3 Después de oír otras objeciones similares contra el evangelio del reino por parte de Tomás, Natanael, Simón el Zelote y Mateo, Jesús dijo a los doce:
\vs p143 1:4 “He venido a este mundo para hacer la voluntad de mi Padre y para revelar su carácter amoroso a toda la humanidad. Esa es, hermanos míos, mi misión. Y es lo único que haré sin importar que los judíos o los gentiles de este día o de otras generaciones malinterpreten mis enseñanzas. Pero no debéis ignorar que incluso el amor divino conlleva una severa disciplina. El amor de un padre por su hijo obliga muchas veces al padre a reprimir los actos imprudentes de su irreflexiva progenie. El hijo no siempre comprende los motivos sensatos y amorosos que mueven a un padre a imponer tal disciplina restrictiva. Pero yo os declaro que mi Padre del Paraíso gobierna de cierto un universo de universos gracias al poder irresistible de su amor. El amor es la mayor de todas las realidades espirituales. La verdad es una revelación que libera, pero el amor significa relación suprema. Y, al margen de los desatinos de vuestros semejantes en la dirección del mundo de hoy, en una era venidera, el evangelio que os doy a conocer gobernará este mismo mundo. El objetivo último del progreso humano es el reconocimiento reverencial de la paternidad de Dios y la manifestación amorosa de la hermandad del hombre.
\vs p143 1:5 “Pero, ¿quién os dijo que mi evangelio está exclusivamente destinado a los esclavos y los débiles? ¿Es que vosotros, mis apóstoles elegidos, parecéis débiles? ¿Aparentaba Juan ser débil? ¿Veis que a mí el temor me atenace? Cierto, predicamos el evangelio a los pobres y a los oprimidos de esta generación. Las religiones de este mundo han desatendido al pobre, pero mi Padre no hace acepción de personas. Además, los pobres de hoy son los primeros en escuchar la llamada al arrepentimiento y a la aceptación de la filiación. El evangelio del reino debe predicarse a todos los hombres ---judíos y gentiles, griegos y romanos, ricos y pobres, libres y esclavos--- y asimismo a los jóvenes y a los mayores, hombres y mujeres.
\vs p143 1:6 “Porque mi Padre sea un Dios de amor y se deleite en ministrar misericordia, no alberguéis la idea de que servir al reino será algo plácidamente monótono. La ascensión al Paraíso representa la suprema aventura de todos los tiempos, el accidentado logro de la eternidad. En la tierra, servir al reino exigirá de vosotros todo el arrojo del que, como hombres, podáis hacer acopio tanto vosotros como vuestros colaboradores. Muchos seréis llevados a la muerte por vuestra lealtad al evangelio de este reino. Resulta fácil morir en la línea de la batalla física cuando vuestro valor se ve fortalecido por la presencia de vuestros camaradas de combate, pero se necesita una forma de valor y de devoción humanas más elevada y profunda para dar tu vida con calma y a solas por el amor de una verdad que se atesora en vuestros corazones mortales.
\vs p143 1:7 “Hoy, los descreídos se burlan de vosotros porque predicáis un evangelio de no confrontación y vivís una vida de no violencia, pero vosotros sois los primeros voluntarios de una larga lista de sinceros creyentes en este evangelio del reino, que asombrará a toda la humanidad por su devoción heroica hacia estas enseñanzas. Ningún ejército del mundo ha manifestado jamás mayor valentía y bravura que los que vosotros y vuestros leales sucesores exhibiréis cuando salgáis al mundo a proclamar la buena nueva ---la paternidad de Dios y la hermandad de los hombres---. La valentía de la carne es la bravura en su menor grado. La bravura de la mente es la valentía humana en su mayor grado, pero la de mayor elevación y supremacía es la inquebrantable lealtad a las elevadas convicciones de las realidades profundas del espíritu. Y tal valor constituye el heroísmo del hombre que conoce a Dios. Y todos vosotros sois hombres conocedores de Dios; sois en verdad los compañeros personales del Hijo del Hombre”.
\vs p143 1:8 \pc Esto no fue todo lo que Jesús dijo en aquella ocasión, sino una introducción a su discurso; continuó después, detenidamente, ampliando e ilustrando sus aseveraciones. Esta fue una de las charlas más apasionadas jamás pronunciada por Jesús a los doce. Resultaba raro que Jesús hablara a sus discípulos expresando tan fuertes sentimientos; se trataba de una de esas escasas ocasiones en las que se expresó con manifiesto fervor, junto a tan acusada emoción.
\vs p143 1:9 \pc Estas palabras tuvieron un resultado inmediato sobre la predicación pública y el ministerio personal de los apóstoles; desde ese mismo día, su mensaje tomó un nuevo tono de valerosa fortaleza. Los doce siguieron adquiriendo en su enseñanza del evangelio del reino un espíritu de positivismo y reafirmación. Desde ese día en adelante, ya no se dedicaron tanto a predicar las virtudes negativas y los mandatos pasivos de la multifacética enseñanza de su Maestro.
\usection{2. LECCIÓN SOBRE EL DOMINIO DE SÍ MISMO}
\vs p143 2:1 El Maestro era un ejemplo de ser humano que había llegado a tener un perfecto dominio de sí mismo. Cuando lo insultaban, no devolvía el insulto; cuando padecía, no amenazaba a sus torturadores; cuando sus enemigos lo acusaron, él simplemente encomendó su causa al juicio justo del Padre de los cielos.
\vs p143 2:2 \pc En una de esas charlas nocturnas, Andrés le preguntó a Jesús: “Maestro, ¿hemos de practicar la abnegación como nos enseñó Juan, o hemos de esforzarnos por conseguir el control de nosotros mismos que tú nos impartes? ¿En qué se diferencia lo que tú enseñas de lo que enseñaba Juan?”. Jesús respondió: “Juan, en efecto, os instruyó en el camino de la justicia conforme a la luz y a las leyes de sus padres, esto es, la religión de la introspección y de la abnegación. Pero yo vengo con un mensaje nuevo que entraña el olvido y el control de uno mismo. Os muestro el camino de la vida tal como me lo reveló mi Padre de los cielos.
\vs p143 2:3 “De cierto, de cierto os digo, que quien se gobierne a sí mismo es más grande que quien captura una ciudad. El dominio de sí es la medida de la naturaleza moral del hombre y el indicador de su desarrollo espiritual. En el viejo orden, vosotros ayunabais y orabais; como criaturas nuevas que renacéis del espíritu, se os enseña a creer y a regocijaros. En el reino del Padre, os convertiréis en criaturas nuevas; las cosas viejas pasarán. ¡Mirad! Os muestro cómo todas las cosas son hechas nuevas. Y por el amor que os tenéis los unos por los otros, convenceréis al mundo de que habéis pasado de la esclavitud a la libertad, de la muerte a la vida eterna.
\vs p143 2:4 “En la antigua usanza, os esforzáis por suprimir, obedecer y conformaros a las reglas de la vida; en la nueva, sois primero \bibemph{transformados} por el espíritu de la verdad y, por consiguiente, vuestra alma se verá fortalecida por la constante renovación espiritual de vuestro entendimiento, y así se os dota de poder para realizar, con certeza y gozo, la voluntad benevolente, loable y perfecta de Dios. No lo olvidéis: es vuestra fe personal en las excepcionalmente grandes y preciadas promesas de Dios la que os garantiza que llegaréis a ser partícipes de la naturaleza divina. Por consiguiente, por medio de vuestra fe y vuestra transformación por el espíritu, os convertís realmente en templos de Dios y su espíritu verdaderamente habita en vosotros. Si el espíritu mora, entonces, en vosotros, ya nunca más seréis esclavos serviles de la carne, sino hijos liberados del espíritu. La nueva ley del espíritu os confiere la libertad de ser dueños de vosotros mismos, en sustitución de la antigua ley del temor, arraigada en un cautiverio autoimpuesto y en la esclavitud de la abnegación.
\vs p143 2:5 “Muchas veces, cuando habéis hecho el mal, habéis pensado en inculpar al maligno por vuestros actos, cuando la verdad es que han sido vuestras propias tendencias naturales las que os han descarriado. ¿No os dijo ya hace algún tiempo el profeta Jeremías que el corazón humano es más engañoso que todas las cosas y que, a veces, incluso extremadamente perverso? ¡Qué fácil os resulta engañaros a vosotros mismos y, veros, por tanto, arrastrados por insensatos temores, diversas pasiones, placeres esclavizantes, malicia, envidia e incluso por un odio vengativo!
\vs p143 2:6 Se logra la salvación por medio de la regeneración del espíritu y no por las pretendidas buenas obras de los seres humanos. Estáis justificados por la fe y hermanados por la gracia, no por el temor y la negación de las necesidades y deseos humanos, aunque los hijos del Padre que han nacido del espíritu son siempre y por siempre \bibemph{dueños} de su yo y de todo lo que atañe a los deseos de la carne. Cuando conocéis que la fe os salva, tenéis paz verdadera con Dios. Y todos los que siguen en el camino de esta paz celestial están destinados a ser bendecidos con el servicio eterno de los hijos, siempre en avance, del Dios eterno. Para vosotros, en lo sucesivo, no constituirá un deber sino más bien un sublime privilegio estar limpios de todos los males de la mente y del cuerpo, en tanto que buscáis la perfección en el amor de Dios.
\vs p143 2:7 “Vuestra filiación se asienta en la fe, y debéis manteneros inconmovibles ante el temor. Vuestro gozo nace de la confianza en la palabra divina y, por ello, no os debéis dejaros llevar por la duda de la realidad del amor y de la misericordia del Padre. Es la bondad misma de Dios la que guía al hombre a un arrepentimiento verdadero y genuino. El secreto del dominio de vuestro yo está vinculado a vuestra fe en el espíritu interior, que siempre obra movido por el amor. Ni siquiera esta fe salvadora es vuestra; es también el don de Dios. Y si sois los hijos de esta fe viva, no sois por más tiempo esclavos serviles del yo, sino los dueños victoriosos de vosotros mismos, los hijos liberados de Dios.
\vs p143 2:8 “Hijos míos, si nacéis, pues, del espíritu, seréis por siempre libres de la consciente servidumbre a una vida de abnegación y vigilancia de los deseos del cuerpo; y se os trasladará al reino gozoso del espíritu, desde donde manifestaréis de forma espontánea sus frutos en vuestra vida diaria; y los frutos del espíritu constituyen la esencia del orden más elevado de un autocontrol placentero y ennoblecedor, hasta llegar incluso a la cima del logro humano en la tierra ---el verdadero dominio de uno mismo---”.
\usection{3. ENTRETENIMIENTO Y ESPARCIMIENTO}
\vs p143 3:1 Sobre ese tiempo, entre los apóstoles y sus más cercanos colaboradores, se produjo una situación de gran nerviosismo y tensión emocional. Aún no se habían acostumbrado del todo a vivir y a trabajar juntos. Cada vez les resultaba más difícil estar en armonía con los discípulos de Juan. El contacto con los gentiles y samaritanos constituía una dura prueba para estos judíos. Y, además de esto, las afirmaciones recientes de Jesús habían aumentado sus estados de perturbación mental. Andrés estaba prácticamente fuera de sí; ya no sabía qué más hacer, así que fue al Maestro para consultarle sus problemas y perplejidad. Cuando Jesús terminó de oír a su jefe apostólico contarle sus preocupaciones, dijo: “Andrés, no puedes convencer a nadie de que olvide su desasosiego cuando ha llegado a tal grado de implicación y cuando hay tantas personas con sentimientos intensos involucradas. No puedo hacer lo que me pides. No participaré en estas dificultades personales de índole social, pero me uniré a vosotros para disfrutar de tres días de reposo y esparcimiento. Ve a tus hermanos y anúnciales que todos vosotros subiréis conmigo al monte Sartaba, donde deseo descansar durante uno o dos días.
\vs p143 3:2 “Ahora, debes ir a cada uno de tus once hermanos y hablar con ellos en privado, diciéndoles: ‘El Maestro quiere que nos vayamos con él aparte durante un tiempo para reposar y relajarnos. Ya que últimamente hemos pasado todos por un período de aflicción espiritual y tensión mental, te sugiero que no menciones nuestras pruebas y preocupaciones mientras estemos en vacaciones. ¿Puedo confiar en ti para que cooperes conmigo en esto?’ Comunícate con cada uno de tus hermanos de esta manera particular y personal”. Y Andrés siguió las instrucciones de su Maestro.
\vs p143 3:3 \pc Para ellos esto fue una maravillosa experiencia; jamás olvidarían el día que subieron a la montaña. A lo largo de todo el viaje, apenas si se hizo algún comentario de sus dificultades. Al llegar a la cima de la montaña, Jesús los sentó a su alrededor mientras les decía: “Hermanos míos, todos vosotros debéis aprender el valor del descanso y la eficacia del esparcimiento. Tenéis que daros cuenta de que el mejor modo de resolver problemas tan enmarañados es apartarse de ellos por un tiempo. Así, cuando se vuelve renovado de estos momentos de descanso y adoración, estos se pueden abordar con mayor claridad de mente y mano más firme, por no mencionar con un corazón más dispuesto. Se descubre, entonces, que, cuando se deja descansar la mente y el cuerpo, muchas veces estos inconvenientes se han reducido en tamaño y proporción”.
\vs p143 3:4 Al día siguiente, Jesús asignó a cada uno de los doce, un tema para comentar. Todo el día se dedicó al recuerdo de anécdotas y a hablar de asuntos no relacionados con su labor religiosa. Se quedaron por un momento sorprendidos cuando Jesús incluso no dio las gracias ---oralmente--- al partir el pan para el almuerzo del mediodía. Era la primera vez que lo habían visto desatender tal costumbre.
\vs p143 3:5 Cuando subieron a la montaña, los problemas asaltaban la cabeza de Andrés. La perplejidad hacía que el corazón de Juan latiese desmesuradamente. Santiago sentía una dolorosa perturbación en el alma. Mateo se veía apremiado por la falta de fondos a raíz de su estancia entre los gentiles. Pedro estaba sobreexcitado y había estado últimamente más enfadado de lo habitual. Judas padecía uno de sus ataques periódicos de susceptibilidad y egoísmo. Simón estaba inusualmente alterado debido a su preocupación por conciliar su patriotismo con el amor de la hermandad de los hombres. Felipe estaba cada vez más confundido por el derrotero que habían tomado las cosas. El buen humor de Natanael había decrecido desde que habían entrado en contacto con las poblaciones gentiles, y Tomás estaba atravesando un serio episodio de depresión. Solo los gemelos se comportaban con normalidad y de forma imperturbable. Todos estaban enormemente desconcertados al desconocer cómo convivir pacíficamente con los discípulos de Juan.
\vs p143 3:6 Al tercer día, cuando iniciaron el descenso de la montaña para volver al campamento, se había producido un gran cambio en ellos. Habían descubierto el hecho importante de que muchas perplejidades humanas son en realidad inexistentes, de que muchos apremiantes problemas no son sino fruto de un exagerado temor y de una excesiva aprensión. Habían aprendido que la mejor manera de hacer frente al desconcierto es apartándose de él; al alejarse, habían dado pie a que tales problemas se resolviesen por sí mismos.
\vs p143 3:7 Su regreso de estos días de ocio señaló el comienzo de un período de gran mejoría en sus relaciones con los seguidores de Juan. Muchos de los doce se dejaron llevar por el júbilo en cuanto observaron el cambio de ánimo que se había producido en las mentes de todos y se sintieron liberados de su irritabilidad nerviosa, algo que había sucedido gracias a aquellos tres días de distracción de las responsabilidades rutinarias de la vida. Existe siempre el peligro de que la monotonía de las relaciones humanas multiplique notablemente la perplejidad y exacerbe las dificultades.
\vs p143 3:8 \pc No fueron muchos los gentiles de las dos ciudades griegas de Arquelais y Fasaelis que creyeron en el evangelio, pero los doce apóstoles adquirieron una valiosa experiencia atribuible a su amplia labor en las poblaciones exclusivamente gentiles. Un lunes por la mañana, a mediados de mes, Jesús le dijo a Andrés: “Adentrémonos en Samaria”. Y partieron de inmediato a la ciudad de Sicar, cerca del pozo de Jacob.
\usection{4. JUDÍOS Y SAMARITANOS}
\vs p143 4:1 Durante más de seiscientos años, los judíos de Judea y, más tarde los de Galilea también, habían estado enemistados con los samaritanos. Esta sensación de malestar entre los judíos y los samaritanos se ocasionó de la forma siguiente: unos setecientos años antes de Cristo, Sargón, rey de Asiria, al frenar una revuelta en la Palestina central, se llevó en cautividad a más de veinticinco mil judíos del reino septentrional de Israel, dando asentamiento en su lugar a un número prácticamente igual de descendientes de los cutitas, sefarvitas y hamateos. Posteriormente, Asurbanipal envió además a otros grupos de colonos para que habitaran Samaria.
\vs p143 4:2 La enemistad religiosa entre los judíos y los samaritanos se remontaba al regreso de los primeros de su cautiverio en Babilonia, cuando los samaritanos trataron de impedir la reconstrucción de Jerusalén. Más tarde, perpetraron una ofensa contra los judíos al ofrecer su ayuda a los ejércitos de Alejandro. A cambio de su amistad, Alejandro dio permiso a los samaritanos para que construyeran un templo sobre el monte Gerizim, en el que adoraban a Yahvé y a sus dioses tribales, y ofrecían sacrificios siguiendo el tipo de servicio que se realizaba en el templo de Jerusalén. Continuaron practicando este culto de adoración al menos hasta el tiempo de los macabeos, momento en el que Juan Hircano destruyó este templo. El apóstol Felipe, en su labor por los samaritanos tras la muerte de Jesús, celebró numerosas reuniones en el emplazamiento de este viejo templo samaritano.
\vs p143 4:3 El antagonismo entre los judíos y los samaritanos era ancestral y estaba basado en sucesos históricos; desde los tiempos de Alejandro, las relaciones entre ellos se habían ido deteriorando continuamente. Los doce apóstoles no eran reacios a predicar en las ciudades griegas ni en otras ciudades gentiles de la Decápolis y Siria, pero para ellos conllevó una difícil prueba de su lealtad al Maestro cuando dijo, “Nos vamos a Samaria”. Si bien, durante más de un año que llevaban con Jesús, habían desarrollado una forma de lealtad personal que trascendía incluso su fe en las enseñanzas que él impartía y sus prejuicios contra los samaritanos.
\usection{5. LA MUJER DE SICAR}
\vs p143 5:1 Cuando el Maestro y los doce llegaron al pozo de Jacob, Jesús, cansado por el viaje, se paró junto a este pozo, mientras Felipe se llevaba a los apóstoles para que le ayudasen a traer la comida y las tiendas desde Sicar. Estaban dispuestos a permanecer en la zona durante algún tiempo. Pedro y los hijos de Zebedeo se hubiesen quedado con Jesús, pero él les pidió que se fueran con sus hermanos diciendo: “No temáis por mí; estos samaritanos son amigables; solo nuestros hermanos, los judíos, quieren hacernos daño”. Y eran casi las seis de aquella tarde de verano cuando Jesús se sentó junto al pozo para esperar el regreso de los apóstoles.
\vs p143 5:2 El agua del pozo de Jacob contenía menos minerales que el agua de los pozos de Sicar y era, por lo tanto, bastante apreciada por su potabilidad. Jesús tenía sed, pero no tenía forma alguna de extraer agua. Por consiguiente, cuando apareció una mujer de Sicar con su cántaro y se dispuso a sacar agua del pozo, Jesús le dijo: “Dame de beber”. Esta mujer de Samaria sabía que Jesús era judío por su aspecto y vestimenta, y supuso que era de Galilea por su acento. Su nombre era Nalda, y era una atractiva criatura. Ella se quedó muy sorprendida al ver que un hombre judío le hablaba de esta manera junto al pozo y le pidiera de beber; no se consideraba adecuado en aquellos días que un hombre respetable hablara en público con una mujer, ni mucho menos que un judío conversara con una samaritana. Por ello, Nalda preguntó a Jesús: ¿Cómo tú, siendo judío, me pides a mí de beber, que soy mujer samaritana?”. Jesús le respondió: “Ciertamente te he pedido de beber, pero si tan solo pudieras comprender, tú me pedirías a mí que te diera de beber el agua viva”. Entonces dijo Nalda: “Pero, Señor, no tienes con qué sacarla y el pozo es hondo; ¿de dónde, pues, tienes esa agua viva? ¿Acaso eres tú mayor que nuestro padre Jacob, que nos dio este pozo, del cual bebieron él, sus hijos y sus ganados?”.
\vs p143 5:3 Jesús respondió: “Cualquiera que beba de esta agua volverá a tener sed, pero el que beba del agua del espíritu vivo no tendrá sed jamás. Y esta agua viva se convertirá en él en fuente de agua fresca que brota para vida eterna”. Nalda dijo entonces: “Dame de esa agua para que no tenga yo sed, ni venga aquí a sacarla. Además, cualquier cosa que una mujer samaritana pueda recibir de un judío tan digno de encomio será un honor”.
\vs p143 5:4 Nalda no sabía cómo interpretar la actitud abierta de Jesús para dirigirse a ella. En el rostro del Maestro percibió el semblante de un hombre recto y santo, pero malinterpretó su lenguaje figurado como si la tratase con familiaridad y como forma de insinuarse a ella. Y, al ser una mujer de moral relajada, se dispuso a coquetear con él, pero Jesús, mirándola fijamente a los ojos, le dijo con voz imperativa: “Mujer, ve, llama a tu marido, y ven acá”. Este mandato hizo a Nalda entrar en razón. Se dio cuenta de que había juzgado equivocadamente la amabilidad del Maestro; notó que había tergiversado su manera de hablar. Se amedrantó. Se dio cuenta de que se encontraba en la presencia de una persona fuera de lo común, y tanteando su mente para encontrar una respuesta adecuada, dijo muy confundida: “Pero Señor, no puedo llamar a mi marido, porque no tengo marido”. Entonces Jesús le dijo: “Has hablado con verdad porque, aunque una vez tuviste marido, con el que ahora vives no lo es. Sería mejor que dejaras de jugar con mis palabras y buscaras el agua viva que este día te he ofrecido”.
\vs p143 5:5 En ese momento, Nalda se puso seria, y su mejor yo se despertó. No era una mujer inmoral solo por propia elección. Había sido repudiada de forma despiadada e injusta por su marido y, al encontrarse en graves aprietos, había accedido a vivir con un cierto griego como su esposa, pero sin contraer matrimonio. Entonces Nalda, bastante avergonzada de haber hablado a Jesús de forma tan desconsiderada, se dirigió al Maestro con gran contrición, diciéndole: “Señor mío, me arrepiento de la forma en que te he hablado, porque percibo que eres un hombre santo o quizás un profeta”. Y estaba a punto de pedirle directamente al Maestro su ayuda personal, cuando hizo lo que tantos han hecho antes y desde entonces: eludir el tema de la salvación personal para hacer alusión a una cuestión teológica y filosófica. Desvió rápidamente la conversación sobre sus propias necesidades espirituales para referirse a un asunto polémico. Señalando al monte Gerizim, continuó: “Nuestros padres adoraban en este monte y, sin embargo, \bibemph{vosotros} decís que en Jerusalén es el lugar donde se debe adorar; ¿cuál es pues el sitio más conveniente para adorar a Dios?”.
\vs p143 5:6 Jesús percibió el intento del alma de esta mujer por evitar un contacto directo e inquisitivo con su Hacedor, pero también vio en su alma el deseo de conocer la mejor forma de vivir. Al fin y al cabo, en el corazón de Nalda había una verdadera sed de agua viva; por ello, la trató con paciencia, diciéndole: “Mujer, créeme que la hora viene cuando no adoraréis al Padre ni en este monte ni en Jerusalén. Vosotros adoráis lo que no sabéis, una mezcla de religiones de muchos dioses paganos y de filosofías gentiles. Al menos los judíos saben a quién adoran. Han borrado su confusión al centrar su adoración en un solo Dios, Yahvé. Pero créeme cuando te digo que la hora viene ---y ahora es--- cuando los verdaderos adoradores adorarán al Padre en espíritu y en verdad, porque así quiere el Padre que sean los que le adoren. Dios es espíritu, y los que lo adoran deben adorarlo en espíritu y en verdad. Tu salvación no viene de conocer cómo deben adorar otros o dónde, sino de recibir en tu propio corazón esa agua viva que te ofrezco ya ahora”.
\vs p143 5:7 Pero Nalda trataría nuevamente de evitar referirse a la embarazosa cuestión de su vida personal en la tierra y del estatus de su alma ante Dios. Una vez más recurrió a preguntas de carácter general sobre la religión, diciendo: “Sí, sé, Señor, que Juan ha predicado acerca de la llegada del Propiciador, de aquel que se le llamará el Libertador, y que, cuando venga, nos lo desvelará todo\ldots ”. y Jesús, interrumpiendo a Nalda, le dijo con aplastante seguridad: “Yo soy, el que está hablando contigo”.
\vs p143 5:8 Era el primer pronunciamiento, directo, concluyente y manifiesto, de su naturaleza y filiación divinas que Jesús había realizado en la tierra; y lo había hecho a una mujer, a una mujer samaritana, a una mujer de cuestionable reputación a los ojos de los hombres hasta este momento, pero una mujer en quien los ojos divinos vieron a alguien contra la que se había pecado más que alguien que había pecado por propia voluntad, un alma humana que \bibemph{ahora} deseaba la salvación, la deseaba sincera e incondicionalmente, y aquello era suficiente.
\vs p143 5:9 Cuando Nalda estaba a punto de manifestar su anhelo real y personal de cosas mejores y de una forma más noble de vivir, justo cuando estaba preparaba para revelar el verdadero deseo de su corazón, volvieron los doce apóstoles de Sicar y, al ver aquella escena, se asombraron sobremanera de que Jesús estuviese hablando con tal cercanía y a solas con esta mujer samaritana. Rápidamente, emplazaron sus provisiones y se apartaron a un lado, sin atreverse a reprobarle nada, mientras Jesús le decía a Nalda: “Mujer, sigue tu camino; Dios te ha perdonado. En adelante vivirás una nueva vida. Has recibido el agua viva, y un nuevo gozo brotará en tu alma y te convertirás en hija del Altísimo”. Y la mujer, notando la actitud de desaprobación de los apóstoles, dejó su cántaro y corrió a la ciudad.
\vs p143 5:10 Al entrar en la ciudad, fue proclamando a quien se encontraba: “Id al pozo de Jacob e id rápidamente. Venid a ver a un hombre que me ha dicho todo lo que he hecho. ¿Será él el Propiciador?”. Y, antes de que se pusiera el sol, una gran multitud se había congregado junto al pozo de Jacob para oír a Jesús, y el Maestro les habló sobre el agua viva, sobre el don del espíritu interior.
\vs p143 5:11 Los apóstoles no dejaban de sorprenderse por la disposición de Jesús para hablar con mujeres, mujeres de dudosa reputación e incluso de conducta inmoral. A Jesús le resultaba muy difícil enseñar a sus apóstoles que las mujeres, incluso las denominadas como inmorales, tienen un alma capaz de elegir a Dios como a su Padre, llegando a ser por ello hijas de Dios y aspirantes a la vida eterna. Incluso diecinueve siglos más tarde, hay muchos que muestran la misma incomprensión de las enseñanzas del Maestro. Hasta la religión cristiana se ha edificado obstinadamente sobre el hecho de la muerte de Cristo, en lugar de sobre la verdad de su vida. El mundo debería preocuparse más de su vida, gozosa y reveladora de Dios, que de su muerte, trágica y lamentable.
\vs p143 5:12 Nalda relató toda la historia al apóstol Juan al día siguiente, pero él nunca llegaría a desvelarla por completo a los demás apóstoles, y Jesús no habló de este hecho a los doce en detalle.
\vs p143 5:13 Nalda le contó a Juan que Jesús le había dicho “todo lo que he hecho”. Muchas veces Juan quiso preguntarle a Jesús acerca de su conversación con Nalda, pero nunca lo hizo. Jesús le dijo a Nalda solo una cosa acerca de ella misma, pero el hecho de mirarla a los ojos y el modo en el que se dirigió a ella le hizo rememorar, de forma instantánea, tal extenso pasaje de toda su accidentada vida. Relacionó el reconocimiento de su propia vida anterior con la mirada y las palabras del Maestro. Jesús nunca le comentó que había tenido cinco maridos. Desde que su marido la repudió, ella había vivido con cuatro hombres diferentes y esto, junto a todo su pasado, se le vino tan vívidamente a la mente cuando tomó conciencia de que Jesús era un hombre de Dios, que le haría decir posteriormente a Juan que, efectivamente, Jesús le había dicho todo acerca de ella.
\usection{6. EL RESURGIMIENTO SAMARITANO}
\vs p143 6:1 La noche en la que Nalda incitó a las multitudes de Sicar a ir a ver a Jesús, los doce acababan de regresar con alimentos y rogaron a Jesús que comiera con ellos en lugar de hablarle a la gente; llevaban todo el día sin comer y tenían hambre. Pero Jesús sabía que llegaría pronto la noche y persistió en su actitud de dirigirles unas palabras antes de despedirlos. Cuando Andrés insistió en que comiera algo antes de hablarles, Jesús le dijo: “Tengo para comer un alimento que vosotros no sabéis”. Al oír esto los apóstoles, se dijeron entre sí: “¿Le habrá traído alguien de comer? ¿Podría ser que la mujer le diese comida además de bebida?”. Cuando Jesús les oyó conversar entre ellos, antes de hablarle a la gente, se apartó y dijo a los doce: “Mi comida es que se haga la voluntad del que me envió y que acabe su obra. No debéis decir nunca más que falta más o menos tanto tiempo para que llegue la siega. Alzad vuestros ojos y mirad a esta gente que viene de una ciudad samaritana para oírnos; yo os digo que los campos ya están blancos para la siega. El que siega recibe salario y recoge este fruto para la vida eterna, para que el que siembra se goce juntamente con el que siega, así pues, los sembradores y los segadores se alegran juntos. En esto es verdadero el dicho: ‘Uno es el que siembra y el otro es el que siega’. Y os envío a segar lo que vosotros no labrasteis; otros labraron y vosotros estáis a punto de entrar en sus labores”. Jesús dijo esto en referencia a la predicación de Juan el Bautista.
\vs p143 6:2 Jesús y los apóstoles fueron a Sicar y predicaron dos días antes de emplazar su campamento en el monte Gerizim. Y muchos de los habitantes de Sicar creyeron en el evangelio y pidieron que se les bautizara, pero los apóstoles de Jesús aún no lo hacían.
\vs p143 6:3 \pc En su primera noche en el campamento del monte Gerizim, los apóstoles esperaban que Jesús les reprendiera por su actitud hacia la mujer del pozo de Jacob, pero él no hizo alusión a este tema. En cambio, les dio una memorable charla sobre “las realidades fundamentales del reino de Dios”. En cualquier religión, resulta fácil permitir que los valores se vuelvan desproporcionados y que los hechos ocupen el lugar de la verdad en la teología personal. El hecho de la cruz se convirtió en el centro mismo del posterior cristianismo; si bien, esto no constituye la verdad más importante de la religión que pueda inferirse de la vida y enseñanzas de Jesús de Nazaret.
\vs p143 6:4 La enseñanza de Jesús en el monte Gerizim consistió en: su deseo de que todos los hombres vean a Dios como a un Padre\hyp{}amigo así como él (Jesús) es un hermano\hyp{}amigo. Y, repetidas veces, les recalcó que una relación basada en el amor es lo más grandioso del mundo ---del universo--- así como la verdad es la más extraordinaria proclamación del cumplimiento de estas relaciones divinas.
\vs p143 6:5 Jesús se dio a conocer a los samaritanos con tanta claridad porque podía hacerlo sin percances, y porque sabía que no volvería a visitar la zona central de Samaria para predicar el evangelio del reino.
\vs p143 6:6 Jesús y los doce estuvieron acampados en el monte Gerizim hasta finales de agosto. Predicaban la buena nueva del reino ---la paternidad de Dios--- a los samaritanos de las ciudades durante el día y pasaban la noche en el campamento. La labor que Jesús y los doce realizaron en estas ciudades samaritanas dio su fruto en la cantidad de almas que se ganaron para el reino y contribuyó notablemente a preparar la magnífica labor de Felipe en estas regiones tras la muerte y resurrección de Jesús, con posterioridad a la dispersión de los apóstoles hasta los últimos confines de la tierra por la severa persecución de los creyentes en Jerusalén.
\usection{7. ENSEÑANZA SOBRE LA ORACIÓN Y LA ADORACIÓN}
\vs p143 7:1 En las charlas nocturnas que impartió en el monte Gerizim, Jesús enseñó muchas grandes verdades e hizo hincapié, particularmente, en las siguientes:
\vs p143 7:2 \pc La verdadera religión consiste en el proceder del alma en sus relaciones personales con el Creador; la religión organizada es el intento del hombre por \bibemph{socializar} la adoración de cada creyente.
\vs p143 7:3 \pc La adoración ---la contemplación de lo espiritual--- debe alternarse con el servicio, el contacto con la realidad material. El trabajo debe alternarse con el entretenimiento; la religión debe equilibrarse con el buen humor. La filosofía profunda debe atenuarse con el ritmo de la poesía. Las presiones de la vida ---la tensión del tiempo en el ser personal--- debe sosegarse con la tranquilidad de la adoración. Las sensaciones de inseguridad por el temor al aislamiento del ser personal en el universo deben remediarse con la meditación de fe en el Padre y con el intento de lograr el Supremo.
\vs p143 7:4 \pc La oración tiene por objeto hacer que el hombre piense menos pero que sea más \bibemph{perceptivo;} no está hecha para aumentar el conocimiento, sino más bien para expandir la percepción.
\vs p143 7:5 \pc La adoración tiene el objetivo de anticipar una futura vida mejor y hacer, luego, que estos nuevos significados espirituales tengan su reflejo en la vida presente. La oración es sustento espiritual, pero la adoración es creatividad divina.
\vs p143 7:6 \pc La adoración es el modo de buscar en el \bibemph{Uno} la inspiración para servir a \bibemph{otros muchos}. La adoración es el rasero con el que se mide el grado del desprendimiento del alma del universo material y su adhesión, simultánea y segura, a las realidades espirituales de toda la creación.
\vs p143 7:7 \pc Orar es recordarse a sí mismo ---pensamiento sublime---; adorar es olvidarse de sí mismo ---suprapensamiento---. La adoración es atención sin esfuerzo, reposo verdadero e ideal del alma, una forma de apacible ejercicio espiritual.
\vs p143 7:8 \pc La adoración es el acto en el que una parte se identifica con el Todo; lo finito con lo Infinito; el hijo con el Padre; el tiempo es el acto de conformar el paso a la eternidad. La adoración es el acto de comunión personal del hijo con el Padre divino, es la adopción de actitudes reconfortantes, creativas, fraternales y románticas por parte del alma\hyp{}espíritu del ser humano.
\vs p143 7:9 \pc Aunque los apóstoles solo llegarían a entender algunas de las enseñanzas que Jesús impartió en el campamento, en otros mundos si lo hicieron, y lo harán otras generaciones en la tierra.
