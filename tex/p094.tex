\upaper{94}{Las enseñanzas de Melquisedec en oriente}
\author{Melquisedec}
\vs p094 0:1 Los primeros maestros de la religión de Salem se adentraron en las tribus más remotas de África y Eurasia, predicando constantemente el evangelio de Maquiventa de la fe y la confianza del hombre en un Dios universal como único precio para obtener el favor divino. El pacto de Melquisedec con Abraham fue la pauta seguida en la temprana difusión de las doctrinas que partieron de Salem y de otros centros de irradiación de estas. Urantia nunca ha tenido, en cualquier otra religión, misioneros más entusiastas y enérgicos que estos nobles hombres y mujeres que llevaron las enseñanzas de Melquisedec por todo el hemisferio oriental. Estos misioneros se reclutaron de numerosos pueblos y razas, y difundieron sus enseñanzas en gran medida por medio de nativos convertidos. Establecieron centros de formación en diferentes partes del mundo en las que enseñaron la religión de Salem a estos pupilos nativos y, luego, les encargaron que desempeñaran la labor de maestros entre su propia gente.
\usection{1. LAS ENSEÑANZAS DE SALEM EN LA INDIA VÉDICA}
\vs p094 1:1 En los días de Melquisedec, la India era un país cosmopolita que había caído recientemente bajo el dominio político y religioso de los invasores ario\hyp{}anditas procedentes del norte y del oeste. En este momento, solo las regiones septentrionales y occidentales de la península estaban en buena parte invadidas por los arios. Estos recién llegados védicos habían llevado con ellos sus muchas deidades tribales. Sus formas religiosas de adoración seguían estrechamente las prácticas ceremoniales de sus más tempranos ancestros anditas, por cuanto que el padre seguía actuando de sacerdote y la madre de sacerdotisa, y el fogón familiar se seguía aún usando como altar.
\vs p094 1:2 El sistema de culto védico estaba entonces en proceso de crecimiento y transformación bajo la dirección de la casta brahmánica de maestros\hyp{}sacerdotes, que paulatinamente estaban asumiendo el control sobre este ritual de adoración en expansión. La fusión de las antiguas treinta y tres deidades arias estaba bien avanzada cuando los misioneros de Salem penetraron en el norte de la India.
\vs p094 1:3 El politeísmo de estos arios representaba una degradación de su previo monoteísmo, a causa de su división en unidades tribales: cada tribu veneraba a su propio dios. Tal degradación del monoteísmo originario y del trinitarismo de la Mesopotamia andita estaba en proceso de una nueva síntesis en los primeros siglos del segundo milenio a. C. Los numerosos dioses estaban organizados en un panteón bajo la dirección trina de Dyaus pitar, el señor de los cielos; Indra, el tempestuoso señor de la atmósfera; y Agni, el dios tricéfalo del fuego, señor de la tierra y símbolo vestigial de un concepto anterior de la Trinidad.
\vs p094 1:4 Se estaban dando determinados desarrollos henoteístas que allanaban el camino para un monoteísmo evolucionado. Agni, la deidad más antigua, a menudo se exaltaba como padre\hyp{}cabeza de todo el panteón. El postulado sobre la deidad\hyp{}padre, a veces llamada Prayápati, a veces, Brahma, quedó oscurecido en la batalla teológica que los sacerdotes brahmánicos librarían después con los maestros de Salem. El Brahmán se concebía como el principio de energía\hyp{}divinidad que activaba la totalidad del panteón védico.
\vs p094 1:5 \pc Los misioneros de Salem predicaban el Dios único de Melquisedec, el Altísimo de los cielos. Esta imagen de Dios no era del todo discordante con el incipiente concepto del Padre Brahma como fuente de todos los dioses, pero la doctrina de Salem no era ritualista y, de ahí, que entrara en flagrante contradicción con los dogmas, tradiciones y enseñanzas de los sacerdotes brahmánicos. Estos se negaban a aceptar las enseñanzas de Salem en cuanto la salvación por medio de la fe, esto es, la obtención del favor de Dios sin prácticas ritualistas ni ceremonias sacrificiales.
\vs p094 1:6 \pc El rechazo del evangelio de Melquisedec de la confianza en Dios y en la salvación mediante la fe constituyó un punto de inflexión fundamental para la India. Los misioneros de Salem habían contribuido en gran medida a la pérdida de la fe en todos los ancestrales dioses védicos, pero los líderes, los sacerdotes del vedismo, se negaron a aceptar las enseñanzas de Melquisedec de un solo Dios y de sencillez de fe.
\vs p094 1:7 Los brahmanes hicieron una selección de los escritos sagrados de su época en un intento de combatir a los maestros de Salem, y esta recopilación, tal como después se revisó, ha llegado hasta los tiempos modernos como Rig Veda, uno de los libros sagrados de mayor antigüedad. El segundo, tercero y cuarto Veda siguieron en el empeño de los brahmanes por cristalizar, formalizar y fijar sus ritos de adoración y sacrificios para los pueblos de aquellos días. En el mejor de los casos, estos escritos están a la altura de cualquier otra colección de carácter similar respecto a la belleza de sus conceptos y a la verdad de sus apreciaciones. Pero, a medida que esta religión superior quedó contaminada con miles y miles de supersticiones, sistemas de culto y ritos del sur de India, se fue transformando paulatinamente en el sistema teológico más heterogéneo jamás desarrollado por el hombre mortal. Un examen de los Vedas revelará tanto algunos de los conceptos más elevados como los más adulterados sobre la Deidad jamás concebidos.
\usection{2. EL BRAHMANISMO}
\vs p094 2:1 A medida que los misioneros de Salem se adentraban hacia el sur, en el Decán dravidiano, se fueron encontrando con un creciente sistema de castas; se trataba del plan de los arios para prevenir la pérdida de identidad racial frente a la oleada en aumento de pueblos sangiks secundarios. Puesto que la casta de los sacerdotes brahmánicos era la esencia misma de este sistema, este orden social retrasó considerablemente el avance de los maestros de Salem. Tal sistema de castas no logró salvar a la raza aria, pero sí tuvo éxito en la perpetuación de los brahmanes, los cuales, a su vez, han mantenido su hegemonía religiosa en la India hasta los tiempos presentes.
\vs p094 2:2 Y, así pues, con el debilitamiento del vedismo por el rechazo de una verdad más elevada, el sistema de culto de los arios se vio sometido a crecientes incursiones procedentes del Decán. En un desesperado esfuerzo por frenar la ola de extinción racial y aniquilación religiosa, la casta brahmánica trató de ensalzarse a sí misma por encima de todo lo demás. Enseñaron que el sacrificio a la deidad era en sí mismo eficaz por completo, que tenía un poderoso efecto. Proclamaron que, de los dos principios divinos esenciales del universo, uno era la deidad Brahmán y, el otro, el sacerdocio brahmánico. En ningún otro pueblo de Urantia había habido sacerdotes que se atreviesen a exaltarse a sí mismos incluso por encima de sus dioses, a arrogarse para sí mismos los honores debidos a estos dioses. Pero fueron tan absurdamente lejos con sus presuntuosas declaraciones que todo su precario sistema se derrumbó ante las prácticas religiosas degradantes que llegaban desde las civilizaciones circundantes, menos avanzadas. El mismo gran sacerdocio védico flaqueó y se hundió en la marea negra de la inercia y el pesimismo, que su propia presunción egoísta y poco juiciosa había hecho caer sobre toda la India.
\vs p094 2:3 La excesiva concentración en sí mismo llevó, sin duda, a temer la perpetuación no evolutiva del yo en un círculo interminable de encarnaciones sucesivas de hombres, animales o plantas. Y de todas las creencias contaminantes que podían haber incidido en lo que pudo haber sido un incipiente monoteísmo, ninguna fue tan invalidante como la creencia en la transmigración ---la doctrina de la reencarnación de las almas--- que provenía del Decán dravidiano. Esta creencia en un extenuante y monótono ciclo repetitivo de transmigraciones despojó a los afanados mortales de su largamente anhelada esperanza de encontrar esa liberación y ese avance espiritual en la muerte que habían formado parte de la temprana fe védica.
\vs p094 2:4 A esta enseñanza filosóficamente incapacitante le siguió pronto la creación de la doctrina del escape eterno del yo, sumergiéndose en el descanso y la paz universal de la unión absoluta con Brahmán, la sobrealma de toda la creación. El deseo y las aspiraciones humanas quedaron casi arrasados, prácticamente eliminados. Durante más de dos mil años, las mejores mentes de la India han procurado huir de todo deseo, abriendo, de esta manera, las puertas de par en par a la entrada de esos sistemas de culto y enseñanzas posteriores que, esencialmente, han maniatado el alma de muchos pueblos hindúes en las cadenas de la falta de esperanza espiritual. De todas las civilizaciones, la védica\hyp{}aria fue la que pagó el más terrible precio por su rechazo del evangelio de Salem.
\vs p094 2:5 \pc Las castas por sí mismas no podían perpetuar el sistema religioso\hyp{}cultural ario y, a medida que las religiones inferiores del Decán penetraban en el norte, se dio una era de desesperación y desesperanza. Fue durante estos días aciagos cuando surgió la tradición de no quitar la vida, la cual ha persistido desde entonces. Muchos de los nuevos sistemas de culto fueron abiertamente ateístas; se alegaba que alcanzar la salvación solo podía provenir del propio esfuerzo del hombre sin ayuda de nadie. Si bien, por toda una gran parte de esta lamentable filosofía, se pueden rastrear algunos vestigios distorsionados de las enseñanzas de Melquisedec e incluso de las de Adán.
\vs p094 2:6 \pc Esos fueron los tiempos de la recopilación de las escrituras posteriores de la fe hindú, los Bráhmanas y los Upanishads. Habiendo rechazado las enseñanzas de la religión personal mediante la vivencia de la fe personal con el Dios único y resultando contaminados por las oleadas de sistemas de cultos y credos degradantes e incapacitante del Decán, con sus antropomorfismos y reencarnaciones, el sacerdocio brahmánico reaccionó de forma violenta contra estas perjudiciales creencias; se produjo un claro esfuerzo por buscar y encontrar la \bibemph{verdadera realidad}. Los brahmanes trataron de suprimir el antropomorfismo de la conceptualización india de la deidad, pero, al hacerlo, cometieron el grave error de despersonalizar el concepto de Dios y surgieron, no con un ideal elevado y espiritual del Padre del Paraíso, sino con una idea distante y metafísica de un Absoluto que lo abarcaba todo.
\vs p094 2:7 En su afán por protegerse a sí mismos, los brahmanes habían rechazado al Dios único de Melquisedec, y se encontraban ahora con la hipótesis de Brahmán, ese ser filosófico indefinido e ilusorio, ese \bibemph{ente} impersonal e inefectivo que, desde ese día desafortunado hasta el siglo XX, ha dejado la vida espiritual de la India desamparada y abatida.
\vs p094 2:8 \pc Fue en los tiempos de la redacción de los Upanishads cuando surgió el budismo en la India. Pero a pesar de sus logros durante mil años, no pudo competir con el hinduismo que aparecería más tarde; aunque su moral era más elevada, su temprana descripción de Dios era incluso menos bien definida que la del hinduismo, que preveía deidades menores y personales. En el norte de la India, el budismo cedió finalmente ante la embestida de un islam militante con un concepto claro de Alá como Dios supremo del universo.
\usection{3. LA FILOSOFÍA BRAHMÁNICA}
\vs p094 3:1 Aunque el brahmanismo, en su etapa más preeminente, apenas era una religión, sí fue verdaderamente uno de los más nobles logros de la mente mortal en los ámbitos de la filosofía y de la metafísica. Habiendo comenzado con la idea de descubrir la realidad última, la mente india no se detuvo hasta no haber reflexionado sobre casi todas las facetas de la teología salvo el concepto doble esencial de la religión: la existencia del Padre Universal de todas las criaturas del universo y el hecho de la experiencia ascendente en el universo de estas mismas criaturas, a medida que tratan de alcanzar al Padre eterno, que les ha mandado que sean perfectas, así como él es perfecto.
\vs p094 3:2 Por el concepto de Brahmán, las mentes de aquellos días ciertamente captaban la idea de un omnipresente Absoluto, porque este postulado se identificaba, al mismo tiempo, como energía creadora y reacción cósmica. Se creía que el Brahmán estaba más allá de cualquier definición, susceptible de comprenderse solo mediante la negación sucesiva de todas sus cualidades finitas. Era decididamente una creencia en un ser absoluto, incluso infinito, pero se trataba de un concepto desprovisto de atributos personales y, por consiguiente, no era experimentable de manera individual por el devoto religioso.
\vs p094 3:3 El Brahmán\hyp{}Narayana se concebía como el Absoluto, el infinito ELLO ES, la fuerza creativa primordial del cosmos potencial, el Yo Universal que existe de modo estático y potencial a lo largo de toda la eternidad. Si los filósofos de aquellos días hubiesen sido capaces de dar el siguiente paso adelante en la conceptualización de la deidad, si hubiesen sido capaces de concebir el Brahmán como relacional y creativo, como un ser personal accesible por los seres creados y en evolución, entonces tal enseñanza quizás podría haberse convertido en la descripción más avanzada de la Deidad realizada en Urantia, puesto que habría abarcado los primeros cinco niveles en los que obra la deidad total y, tal vez, podría haber previsto los dos restantes.
\vs p094 3:4 En algunas de sus facetas, el concepto de la Sobrealma Universal Única como totalidad de la suma de toda la existencia creatural llevó a los filósofos indios muy cerca de la verdad del Ser Supremo, pero esta verdad no les sirvió de nada, porque fracasaron en elaborar un método de acercamiento personal razonable a su objetivo teórico monoteísta del Brahmán\hyp{}Narayana.
\vs p094 3:5 El principio kármico de la continuidad causal está, de nuevo, muy próximo a la verdad de la síntesis de las repercusiones de todas las acciones espacio\hyp{}temporales en la presencia conforme Deidad del Supremo; pero este postulado nunca contempló la respectiva consecución personal de la Deidad por parte del devoto religioso a nivel individual, sino únicamente el sumergimiento último de todos los seres personales en la Sobrealma Universal.
\vs p094 3:6 La filosofía del brahmanismo estuvo también muy cerca de la concreción de la inhabitación de los modeladores del pensamiento, aunque solo para verse distorsionada por un concepto erróneo de la verdad. La enseñanza de que el alma es la inhabitación del Brahmán habría allanado el camino para la creación de una religión avanzada, si este concepto no hubiese sido totalmente deformado por la creencia de que no existe individualidad humana aparte de esta inhabitación del Uno Universal.
\vs p094 3:7 En la doctrina de la fusión del alma individual con la Sobrealma, los teólogos de la India no contemplaron la supervivencia de algo humano, algo nuevo y único, algo nacido de la unión de la voluntad del hombre y de la voluntad de Dios. La enseñanza del regreso del alma al Brahmán tiene bastante analogía con la realidad del regreso del modelador al seno del Padre Universal, pero existe algo distinto del modelador que también sobrevive, el homólogo morontial del ser personal del mortal. Y este vital concepto estaba fatídicamente ausente de la filosofía brahmánica.
\vs p094 3:8 La filosofía brahmánica se ha aproximado a muchos de los hechos del universo y ha abordado numerosas verdades cósmicas, pero, con demasiada frecuencia, ha sido víctima del error de no diferenciar entre los diversos niveles de realidad, como el absoluto, el trascendental y el finito. No tomó en consideración que lo que puede ser finito\hyp{}ilusorio en el nivel absoluto puede ser absolutamente real en el nivel finito. Y tampoco ha sabido reconocer la necesaria persona del Padre Universal, con quien otras personas de todos los niveles pueden comunicarse personalmente, partiendo de la limitada experiencia de la criatura evolutiva con Dios hasta la experiencia ilimitada del Hijo Eterno con el Padre del Paraíso.
\usection{4. LA RELIGIÓN HINDÚ}
\vs p094 4:1 Con el paso de los siglos, la población de la India volvió en cierta medida a los rituales antiguos de los Vedas, tal como se habían modificado por los misioneros de Melquisedec y cristalizado por el posterior sacerdocio brahmánico. Esta religión, la más antigua y cosmopolita de las religiones del mundo, experimentó nuevos cambios en respuesta al budismo y al jainismo y a las influencias del mahometismo y el cristianismo, aparecidos más tarde. Pero a la llegada de las enseñanzas de Jesús, estas se habían vuelto tan occidentalizadas como para ser una “religión del hombre blanco”, y, por ello, extrañas y desconocidas para la mente hindú.
\vs p094 4:2 \pc En el momento actual, la teología hindú describe cuatro niveles descendentes de la deidad y la divinidad:
\vs p094 4:3 \li{1.}\bibemph{El Brahmán,} el Absoluto, el Uno Infinito, el ELLO ES.
\vs p094 4:4 \li{2.}\bibemph{El Trimurti,} la Trinidad suprema del hinduismo. En esta agrupación, \bibemph{Brahma,} el primero de sus miembros, se concibe como creado a sí mismo a partir del Brahmán ---la infinitud---. Si no fuese por su identificación estrecha con el Uno Infinito panteísta, Brahma podría constituir la base de una conceptualización del Padre Universal. Brahma también se identifica con el hado.
\vs p094 4:5 La adoración del segundo y tercer miembro, Shiva y Vishnu, apareció en el primer milenio después de Cristo. \bibemph{Shiva} es el señor de la vida y de la muerte, el dios de la fertilidad y el maestro de la destrucción. \bibemph{Vishnu} es extremadamente popular debido a la creencia de que se encarna periódicamente en forma humana. De esta manera, Vishnu se vuelve real y vivo en la imaginación de los indios. Se considera a Shiva y a Vishnu supremos sobre todas las cosas.
\vs p094 4:6 \li{3.}\bibemph{Las deidades védicas y posvédicas}. Muchos de los ancestrales dioses de los arios, tales como Agni, Indra y Soma, han perdurado como dioses secundarios frente a los tres miembros del Trimurti. Desde los días primitivos de la India védica, han surgido numerosos otros dioses, y estos también se han incorporado al panteón hindú.
\vs p094 4:7 \li{4.}\bibemph{Los semidioses:} superhombres, semidioses, héroes, demonios, espectros, espíritus malignos, diablillos, monstruos, duendes y santos de los posteriores sistemas de culto.
\vs p094 4:8 \pc Aunque el hinduismo no ha logrado revitalizar al pueblo indio, ha sido, al mismo tiempo, una religión generalmente tolerante. Su gran fuerza yace en el hecho de que ha demostrado ser la religión más moldeable y amorfa de las aparecidas en Urantia. Es capaz de cambios casi ilimitados y posee una inusual capacidad de flexibilidad y adaptación, desde las elevadas consideraciones semimonoteístas de los intelectuales brahmanes hasta el notorio fetichismo y las prácticas cultuales primitivas de las clases degradadas y desfavorecidas de creyentes ignorantes.
\vs p094 4:9 El hinduismo ha sobrevivido porque es esencialmente una parte integral del entramado social básico de la India. Carece de una gran estructura jerárquica que se pueda desorganizar o destruir; está entrelazada en el patrón de vida del pueblo. Tiene una adaptabilidad a las condiciones cambiantes que destaca de todos los demás sistemas de culto, y demuestra una actitud tolerante de adopción hacia otras muchas religiones, afirmando que Gautama Buda e incluso el mismo Cristo fueron encarnaciones de Vishnu.
\vs p094 4:10 Hoy, en la India, hay una gran necesidad del evangelio de Jesús: la paternidad de Dios y la filiación y consiguiente fraternidad de todos los hombres, que se realiza de manera personal en el ministerio amante y en el servicio social. En la India, existe el marco filosófico, hay presencia de una estructura cultual; todo lo que se necesita es la chispa revitalizadora del amor dinámico ejemplificado en el evangelio primigenio del Hijo del Hombre, despojado de dogmas y doctrinas occidentales, que han tendido a hacer del ministerio de vida de Miguel una religión del hombre blanco.
\usection{5. PUGNA POR LA VERDAD EN CHINA}
\vs p094 5:1 A medida que los misioneros de Salem atravesaron Asia, difundiendo las doctrinas del Dios Altísimo y la salvación por medio de la fe, absorbieron una buena parte de la filosofía y del pensamiento religioso de los distintos países por los que cruzaron. Pero los maestros enviados por Melquisedec y sus sucesores no incumplieron la responsabilidad encomendada; se adentraron en todos los pueblos del continente eurasiático, y no llegaron a China hasta mediados del segundo milenio antes de Jesucristo. Durante más de cien años, los salemitas mantuvieron su sede en See Fuch, formando allí a los maestros chinos, que impartieron sus enseñanzas en todos los territorios habitados por la raza amarilla.
\vs p094 5:2 Como consecuencia directa de esta enseñanza, surgieron en China las formas más tempranas del taoísmo, una religión inmensamente diferente a la que hoy en día lleva ese nombre. El taoísmo primitivo o prototaoísmo estaba compuesto de los siguientes elementos:
\vs p094 5:3 \li{1.}Las persistentes enseñanzas de Singlangtón, que perduraban en el concepto de Shang\hyp{}ti, el Dios del Cielo. En los tiempos de Singlangtón, el pueblo chino se volvió prácticamente monoteísta; concentraron su adoración en la Verdad Única, conocida más tarde como Espíritu del Cielo, el gobernante del universo. Y la raza amarilla nunca perdió por completo este concepto primitivo de la Deidad, aunque, en siglos posteriores, muchos dioses y espíritus menores se infiltraron insidiosamente en su religión.
\vs p094 5:4 \li{2.}La religión de Salem de una Altísima Deidad Creadora que otorgaría su favor a la humanidad en respuesta a la fe del hombre. Pero es muy cierto que, en el momento en el que los misioneros de Melquisedec habían penetrado en los territorios de la raza amarilla, su mensaje original había cambiado considerablemente con respecto a las sencillas doctrinas de Salem de los días de Maquiventa.
\vs p094 5:5 \li{3.}El concepto del Brahmán\hyp{}Absoluto de los filósofos indios, unido al deseo de escapar de todo mal. Quizás la mayor influencia externa en la propagación hacia el este de la religión de Salem la ejercieron los maestros indios de la fe védica, que inyectaron su concepto del Brahmán ---el Absoluto--- en el pensamiento salvacionista de los salemitas.
\vs p094 5:6 \pc Esta creencia mixta se difundió por los territorios de la raza amarilla y cobriza de forma subyacente en el pensamiento religioso\hyp{}filosófico. En Japón, este prototaoísmo se conoció con el nombre de sintoísmo y, en este país, bastante alejado de Salem, en Palestina, los pueblos supieron de la encarnación de Maquiventa Melquisedec, que habitó en la tierra para que la humanidad no se olvidara del nombre de Dios.
\vs p094 5:7 En China, todas estas creencias más tarde se confundieron y se mezclaron con el creciente sistema de culto de adoración de los ancestros. Pero los chinos, desde los tiempos de Singlangtón, jamás cayeron en la desesperante esclavitud de la superchería sacerdotal. La raza amarilla fue la primera en salir de esta brutal servidumbre y entrar en una civilización ordenada; fue la primera en conseguir un cierto grado de libertad del terror abyecto a los dioses, y ni siquiera llegó a temer a los espectros de los muertos, tal como lo hacían otras razas. China halló su derrota porque no logró progresar más allá de su temprana emancipación de los sacerdotes; no obstante, se vio envuelta en un error casi igualmente trágico: la adoración de los ancestros.
\vs p094 5:8 \pc Pero los salemitas no trabajaron en vano; sobre los fundamentos de su evangelio conformarían sus enseñanzas los grandes filósofos del siglo VI de China. La atmósfera moral y los sentimientos espirituales de los tiempos de Lao\hyp{}Tse y Confucio surgieron de las enseñanzas de los misioneros salemitas de una era más temprana.
\usection{6. LAO\hyp{}TSE Y CONFUCIO}
\vs p094 6:1 Alrededor de seiscientos años antes de la llegada de Miguel, le pareció a Melquisedec, que había abandonado hacía mucho su forma carnal, que la pureza de sus enseñanzas en la tierra se estaban viendo excesivamente comprometidas por su generalizada integración en las creencias más antiguas de Urantia. Durante cierto tiempo, tuvo la impresión de que su misión como precursor de Miguel estaba al borde del fracaso. Y en el siglo VI a. C., mediante una coordinación inusitada de instancias espirituales, no del todo comprendida ni siquiera por los supervisores planetarios, Urantia fue testigo de una manifestación, extraordinariamente insólita, de una múltiple verdad religiosa. Mediante la intermediación de distintos maestros humanos, el evangelio de Salem se reafirmó y revitalizó, y gran parte de lo que entonces se predicó ha persistido hasta los tiempos de la redacción de este texto escrito.
\vs p094 6:2 Este excepcional siglo de progreso espiritual se caracterizó por la presencia, en todo el mundo civilizado, de grandes maestros religiosos, morales y filosóficos. En China, hubo dos destacados maestros:  y Confucio.
\vs p094 6:3 \pc \bibemph{Lao\hyp{}Tse} se basó directamente en los conceptos de la tradición de Salem cuando afirmó que el Tao era la Única Primera Causa de toda la creación. Lao fue un hombre de gran visión espiritual. Enseñó que el destino eterno del hombre era “su unión perpetua con el Tao, Dios Supremo y Rey Universal”. Tenía una profunda comprensión de la causalidad última, pues escribió: “La unidad surge del Tao Absoluto, y de la Unidad aparece la Dualidad cósmica, y de esta Dualidad, emerge la Trinidad, y la Trinidad es la fuente primigenia de toda realidad”. “Toda realidad está siempre en equilibrio entre los potenciales y los actuales del cosmos, y estos están eternamente armonizados por el espíritu de la divinidad”.
\vs p094 6:4 Lao\hyp{}Tse realizó también una de las más tempranas exposiciones de la doctrina de devolver bien por mal: “La bondad engendra bondad, pero para quien es verdaderamente bueno, el mal también engendra bondad”.
\vs p094 6:5 Enseñó el regreso de la criatura al Creador y describió la vida como la gradual aparición de un ser personal de los potenciales cósmicos, mientras que la muerte era como la vuelta al hogar de este ser personal creatural. Su concepto de la fe genuina era inusual, y lo comparó además a la “actitud de un niño pequeño”.
\vs p094 6:6 Tenía un claro entendimiento del propósito eterno de Dios, pues dijo: “La Deidad Absoluta no pugna pero es siempre victoriosa; no coacciona a la humanidad pero está siempre dispuesta a responder a sus auténticos deseos; la voluntad de Dios es eterna en paciencia y eterna en la inevitabilidad de su expresión”. Y del verdadero devoto religioso dijo, al expresar la verdad de que es más bienaventurado dar que recibir: “El hombre bueno no intenta retener la verdad sino que más bien desea entregar estas riquezas a sus semejantes, porque esa es la realización de la verdad. La voluntad del Dios Absoluto siempre beneficia, nunca destruye; el objetivo del verdadero creyente es actuar siempre, pero jamás coaccionar”.
\vs p094 6:7 Las enseñanzas de Lao respecto a no resistencia y a la distinción que hizo entre \bibemph{acción} y \bibemph{coacción} se distorsionaron más tarde al convertirse en la creencia de “no ver, no hacer y no pensar nada”. Si bien, Lao jamás impartió ese error, aunque su argumentación de la no resistencia haya influenciado el desarrollo ulterior de la propensión pacífica del pueblo chino.
\vs p094 6:8 Pero el taoísmo popular en el siglo XX de Urantia tiene muy poco en común con los elevados sentimientos y los conceptos cósmicos del viejo filósofo que enseñó la verdad como la percibía, la cual era: esa fe en el Dios Absoluto es la fuente de la energía divina la cual rehará el mundo, y mediante la que el hombre asciende a la unión espiritual con el Tao, la Deidad Eterna y el Creador Absoluto de los universos.
\vs p094 6:9 \pc \bibemph{Confucio} (Kung Fu\hyp{}Tze), más joven, fue contemporáneo de Lao en China del siglo VI. Confucio basó sus doctrinas en las mejores tradiciones morales de la larga historia de la raza amarilla, y también se vio de alguna manera influenciado por las persistentes tradiciones de los misioneros de Salem. Su tarea principal fue la recopilación de los sabios proverbios de los filósofos antiguos. Durante toda su vida, se le rechazó como maestro, pero sus escritos y enseñanzas han ejercido desde entonces una gran influencia en China y Japón. Confucio marcó una nueva dirección para los chamanes puesto que colocó la moral en el lugar de la magia. Pero lo hizo sumamente bien. Creó un nuevo fetiche del \bibemph{orden} y estableció el respeto por la conducta de los ancestros, que los chinos aún veneran en el momento de este texto escrito.
\vs p094 6:10 La predicación de Confucio sobre la moral se fundamentaba en la teoría de que el camino terrenal es la sombra deformada del camino celestial; que el verdadero modelo de la civilización temporal es el reflejo del eterno orden del cielo. El concepto potencial de Dios en el confucionismo estaba casi completamente subordinado al énfasis puesto en el Camino del Cielo, el modelo del cosmos.
\vs p094 6:11 En Oriente, las enseñanzas de Lao se han perdido para todos, excepto para algunos pocos, pero los escritos de Confucio han constituido desde entonces la base del entramado moral de la cultura para casi un tercio de los urantianos. Estos preceptos de Confucio, aunque perpetúan lo mejor del pasado, eran más bien contrarios al mismo espíritu chino de indagación que había producido esos logros tan venerados. La influencia de estas doctrinas se combatieron sin éxito tanto por iniciativas imperiales de Ch'in Shih Huang Ti como por las enseñanzas de Mo Ti, que proclamó una fraternidad fundada no en el deber ético sino en el amor a Dios. Trató de reactivar la ancestral búsqueda de la nueva verdad, pero sus enseñanzas fracasaron ante la vigorosa oposición de los discípulos de Confucio.
\vs p094 6:12 Con el tiempo, como a muchos otros maestros espirituales y morales, tanto a Confucio como a Lao\hyp{}Tse se les deificó por parte de sus seguidores en esas eras espiritualmente oscuras de China, que mediaron entre el declive y la degradación de la fe taoísta y la llegada de los misioneros budistas desde la India. Durante estos siglos de decadencia espiritual, la religión de la raza amarilla degeneró en una penosa teología poblada de diablos, dragones y espíritus malignos, que revelaban el retorno a sus miedos de la mente mortal poco formada. Y China, que una vez estuvo en el primer plano de la sociedad humana debido a su religión avanzada, quedó atrás a causa de su fracaso temporal para progresar en el auténtico sendero del desarrollo de esa conciencia de Dios indispensable para el verdadero progreso, no solamente del mortal de forma individual, sino también de las civilizaciones intrincadas y complejas que caracterizan el avance de la cultura y de la sociedad en un planeta evolutivo del tiempo y el espacio.
\usection{7. GAUTAMA SIDDHARTA}
\vs p094 7:1 Contemporáneo de Lao\hyp{}Tse y Confucio en China, surgió en la India otro gran maestro de la verdad: Gautama Siddharta. Nació en el siglo VI a. C., en la provincia del Nepal, al norte de la India. Más tarde, sus seguidores hicieron creer que era hijo de un gobernante extraordinariamente rico, cuando, en realidad, era el presunto heredero al trono de un jefezuelo que gobernaba, con la aprobación de la gente, un pequeño valle montañoso aislado del sur del Himalaya.
\vs p094 7:2 Tras practicar vanamente el yoga durante seis años, Gautama enunció las teorías que se convirtieron en la filosofía del budismo. Siddharta libró una lucha decidida pero inútil contra el creciente sistema de castas. Este joven príncipe profeta irradiaba una noble honestidad y un singular altruismo que atraían grandemente a los hombres de aquellos días. Gautama restó valor a la práctica de buscar salvación individual a través del sufrimiento físico y del dolor personal. Y exhortó a sus seguidores a que llevaran su evangelio por todo el mundo.
\vs p094 7:3 En medio de la confusión y de las radicales prácticas de los sistemas de culto de la India, las enseñanzas más sensatas y moderadas de Gautama fueron un reconfortante remedio. Censuró el uso de los dioses y a los sacerdotes y sus sacrificios, pero tampoco supo percibir \bibemph{el ser personal} del Uno Universal. Al no creer en la existencia individual del alma humana, Gautama luchó, sin duda, valientemente contra la tradicional creencia de la transmigración de las almas. Hizo un noble esfuerzo por liberar a los hombres del miedo y hacer que se sintieran tranquilos y como en casa en el gran universo, pero no logró mostrarles la senda que los llevaba hacia el auténtico y sublime hogar de los mortales ascendentes ---el Paraíso--- ni el camino de creciente servicio de la existencia eterna.
\vs p094 7:4 Gautama fue un verdadero profeta, y si hubiese atendido las instrucciones del ermitaño Godad, quizás habría suscitado en toda la India el renacimiento del evangelio de Salem de la salvación por la fe. Godad descendía de una familia que nunca había abandonado las tradiciones de los misioneros de Melquisedec.
\vs p094 7:5 Gautama fundó su escuela en Benarés, y fue durante su segundo año cuando Baután, un pupilo suyo, impartió a su maestro las tradiciones de los misioneros de Salem sobre el pacto de Melquisedec con Abraham; y, aunque Siddharta no tenía una noción muy clara del Padre Universal, adoptó una posición avanzada respecto a la salvación por la fe ---la sencilla creencia---. Así lo declaró ante sus seguidores y comenzó a enviar a sus discípulos en grupos de sesenta para proclamar al pueblo de la India “la buena nueva de la salvación gratuita; de que todos los hombres, de clase alta o humilde, pueden alcanzar la felicidad por medio de la fe en la rectitud y en la justicia”.
\vs p094 7:6 La esposa de Gautama creía en el evangelio de su marido y fue la fundadora de una orden de monjas. Su hijo se convirtió en sucesor de su padre y expandió considerablemente su sistema de culto; captó la nueva idea de la salvación por la fe, pero en sus últimos años se desvió respecto al evangelio de Salem del favor divino solo mediante la fe y, en su vejez, sus últimas palabras fueron: “Hallad vuestra propia salvación”.
\vs p094 7:7 \pc En su mejor momento, el evangelio que predicó Gautama de la salvación universal, libre de sacrificios, torturas, rituales y sacerdotes, fue una doctrina revolucionaria y asombrosa para su tiempo. Y estuvo sorprendentemente cerca de ser el resurgimiento del evangelio de Salem. Socorrió a millones de almas desesperadas y, a pesar de su grotesca degradación en los siglos que siguieron, aún continúa siendo la esperanza de millones de seres humanos.
\vs p094 7:8 Siddharta enseñó muchas más verdades de las que han sobrevivido en los sistemas de culto modernos que llevan su nombre. El budismo moderno no se corresponde más a las enseñanzas de Gautama Siddharta que el cristianismo a las de Jesús de Nazaret.
\usection{8. LA FE BUDISTA}
\vs p094 8:1 Para convertirse en budista, bastaba con hacer profesión pública de fe recitando el Refugio: “Me refugio en el Buda; me refugio en la Doctrina; me refugio en la Fraternidad”.
\vs p094 8:2 El budismo tuvo su origen en una persona histórica, no en un mito. Los seguidores de Gautama lo llamaban Sasta, que significaba amo o maestro. Aunque no tenía pretensiones sobrehumanas ni para él mismo ni para sus enseñanzas, sus discípulos comenzaron pronto a llamarle el \bibemph{iluminado,} el Buda; y, más tarde, Sakyamuni Buda.
\vs p094 8:3 \pc El evangelio primigenio de Gautama se basaba en cuatro nobles verdades:
\vs p094 8:4 \li{1.}Las nobles verdades del sufrimiento.
\vs p094 8:5 \li{2.}Los orígenes del sufrimiento.
\vs p094 8:6 \li{3.}La eliminación del sufrimiento.
\vs p094 8:7 \li{4.}El camino para eliminar el sufrimiento.
\vs p094 8:8 \pc Estrechamente vinculada a la doctrina del sufrimiento y al modo de eludirlo, estaba la filosofía del Óctuple Sendero: recta visión, recto anhelo, recto lenguaje, recta acción, recto modo de vida, recto esfuerzo, recta conciencia y recta concentración. Al escapar del sufrimiento, Gautama no intentaba eliminar cualquier esfuerzo, deseo y afecto. Más bien, su enseñanza estaba concebida para que el hombre mortal percibiera la inutilidad de poner por completo sus esperanzas y aspiraciones en metas temporales y en objetivos materiales. No se trataba tanto de evitar amar a sus semejantes, sino de que el verdadero creyente debía también mirar más allá de las relaciones de este mundo material y centrarse en las realidades del futuro eterno.
\vs p094 8:9 \pc Gautama predicaba cinco mandamientos morales:
\vs p094 8:10 \li{1.}No matarás.
\vs p094 8:11 \li{2.}No robarás.
\vs p094 8:12 \li{3.}No serás impúdico.
\vs p094 8:13 \li{4.}No mentirás.
\vs p094 8:14 \li{5.}No tomarás bebidas embriagantes.
\vs p094 8:15 \pc Había otros mandamientos secundarios, cuyo debido cumplimiento era facultativo para los creyentes.
\vs p094 8:16 \pc Siddharta apenas creía en la inmortalidad de la persona humana; su filosofía solo preveía alguna especie de continuidad operativa. Nunca definió con claridad lo que pretendía incluir en su doctrina del Nirvana. El hecho de que podría experimentarse teóricamente durante la existencia mortal indicaría que no se percibía como un estado de aniquilación total. Suponía una condición de iluminación suprema y de felicidad inefable en el que todas las amarras que ataban al mundo material se habían roto; se estaba libre de los deseos de la vida mortal y liberado del peligro de experimentar de nuevo la encarnación.
\vs p094 8:17 Conforme a las enseñanzas primigenias de Gautama, la salvación se consigue mediante el esfuerzo humano, aparte de la ayuda divina; no hay cabida para la fe salvadora ni para las oraciones a poderes sobrehumanos. Gautama, en su intento de reducir al mínimo las supersticiones de la India, procuró alejar a los hombres de cualquier alusión explícita a una salvación mágica. Y, al hacerlo así, dejó la puerta abierta para que sus sucesores malinterpretaran sus enseñanzas y predicaran que cualquier afán humano por conseguir algo es indeseable y penoso. Sus seguidores pasaron por alto el hecho de que la mayor felicidad va unida a la consecución inteligente y entusiasta de nobles objetivos, y que esos logros constituyen el verdadero progreso en la autorrealización cósmica.
\vs p094 8:18 La gran verdad de las enseñanzas de Siddharta fue su proclamación de un universo de absoluta justicia. Impartió la mejor filosofía sin Dios jamás inventada por el hombre mortal; era un humanismo ideal y, de la manera más efectiva, erradicó cualquier motivo que impulsara a las supersticiones, a los rituales mágicos y al temor a los espectros o demonios.
\vs p094 8:19 La gran debilidad del evangelio primigenio del budismo fue que no produjo una religión que incitara al servicio social altruista. La hermandad budista no fue, durante mucho tiempo, una fraternidad de creyentes, sino más bien una comunidad de maestros en formación. Gautama les prohibió recibir dinero, intentando, de esta manera, prevenir el desarrollo de propensiones jerárquicas. Gautama mismo era eminentemente social; en efecto, su vida fue mucho más grandiosa que su propia predicación.
\usection{9. PROPAGACIÓN DEL BUDISMO}
\vs p094 9:1 El budismo prosperó porque ofrecía salvación a través de la creencia en el Buda, en el iluminado. Representó mejor las verdades de Melquisedec que cualquier otro sistema religioso de Asia oriental. Pero el budismo no se generalizó como religión hasta que Asoka no lo propugnó como autodefensa. Asoka, un monarca de casta inferior, fue, después de Akenatón en Egipto, uno de los gobernantes civiles más notables habidos entre Melquisedec y Miguel. Construyó un gran imperio indio gracias a la propaganda de sus misioneros budistas. Durante un período de veinticinco años, formó y envió a más de diecisiete mil misioneros a las fronteras más alejadas de todo el mundo conocido. En una generación, hizo del budismo la religión dominante de la mitad del mundo. Se asentó pronto en el Tíbet, Cachemira, Ceilán, Birmania, Java, Siam, Corea, China y Japón. Y, en términos generales, fue una religión inmensamente superior a las que reemplazó o reforzó.
\vs p094 9:2 La difusión del budismo desde la India, su país de origen, a toda Asia es una de las historias más apasionantes de la expansión misionera debido a la dedicación espiritual y la perseverancia de sinceros devotos religiosos que emprendieron tal tarea. Los maestros del evangelio de Gautama no solo desafiaron por tierra los peligros de las rutas de las caravanas sino que también se enfrentaron a los peligros de los mares de China, a medida que llevaban a cabo su misión en el continente asiático, llevando a todos los pueblos el mensaje de su fe. Pero este budismo ya no era la sencilla doctrina de Gautama; era el evangelio milagrero que lo había convertido en un dios. Y cuanto más se alejaba el budismo de su origen en las altiplanicies de la India, más diferente se volvía de las auténticas enseñanzas de Gautama, y más parecido tenía a las religiones que reemplazaba.
\vs p094 9:3 Más tarde, el budismo se vio muy influenciado por el taoísmo de China, el sintoísmo de Japón y el cristianismo del Tíbet. Tras mil años, en la India el budismo simplemente se debilitó y desapareció. Se brahmanizó y más tarde se rindió deplorablemente ante el islam, mientras que, en gran parte de Oriente, degeneró en un ritual que Gautama Siddharta jamás hubiese reconocido.
\vs p094 9:4 En el sur, el estereotipo fundamentalista de las enseñanzas de Siddharta subsistió en Ceilán, Birmania y en la península de Indochina. Se trata de la rama hinayana del budismo que se aferra a la doctrina primitiva, de carácter antisocial.
\vs p094 9:5 Pero incluso antes de su desplome en la India, los grupos de seguidores chinos y del norte de la India de Gautama habían comenzado el desarrollo de las enseñanzas mahayanas del “camino mayor” a la salvación, a diferencia de los puristas del sur que se atuvieron a la hinayana o “camino menor”. Y estos mahayanistas se liberaron de las limitaciones sociales inherentes a la doctrina budista y, desde entonces, esta rama septentrional del budismo ha seguido evolucionando en China y Japón.
\vs p094 9:6 Hoy en día, el budismo es una religión viva y en crecimiento, porque tiene éxito en la conservación de muchos de los más altos valores morales de sus partidarios. Fomenta la calma y el autocontrol, aumenta la serenidad y la felicidad y hace mucho por evitar la pena y el llanto. Aquellos que creen en esta filosofía viven vidas mejores que muchos que no creen en ella.
\usection{10. LA RELIGIÓN EN EL TÍBET}
\vs p094 10:1 En el Tíbet se puede encontrar la combinación más insólita de las enseñanzas de Melquisedec junto con las del budismo, hinduismo, taoísmo y cristianismo. Cuando los misioneros budistas se adentraron en el Tíbet, hallaron un estado de barbarie primitiva muy similar al que descubrieron los primeros misioneros cristianos en las tribus nórdicas de Europa.
\vs p094 10:2 Estos tibetanos, de mentes sencillas, no quisieron renunciar por completo a su magia y a sus amuletos ancestrales. El examen de los ceremoniales religiosos de los ritos actuales tibetanos revela una comunidad excesivamente grande de sacerdotes con la cabeza rapada que practican un complejo ritual de campanas, cánticos, incienso, procesiones, rosarios, imágenes, amuletos, pinturas, agua bendita, magnificas vestiduras y refinados coros. Poseen estrictos dogmas y credos cristalizados, ritos místicos y peculiares ayunos. Su jerarquía incluye monjes, monjas, abades y el Gran Lama. Oran a los ángeles, a los santos, a la Madre Sagrada y a los dioses. Practican confesiones y creen en el purgatorio. Sus monasterios son amplios y sus catedrales grandiosas. Perseveran en la interminable repetición de sus ritos sagrados y creen que tales ceremoniales otorgan la salvación. Las plegarias se vinculan a una rueda cuyo giro creen que las convierten en más eficaces. En estos tiempos modernos, no es posible encontrar ningún otro pueblo cuyas celebraciones abarquen tantas religiones; y es inevitable que tal acumulación litúrgica se vuelva desmesuradamente engorrosa e intolerablemente onerosa.
\vs p094 10:3 Los tibetanos tienen algo de todas las religiones principales del mundo excepto las simples enseñanzas del evangelio de Jesús: la filiación con Dios, la fraternidad de los hombres y la ciudadanía por siempre ascendente en el universo eterno.
\usection{11. LA FILOSOFÍA BUDISTA}
\vs p094 11:1 El budismo entró en China en el primer milenio después de Cristo, y se adaptó bien a las costumbres religiosas de la raza amarilla. Durante mucho tiempo, en su adoración a los ancestros, habían orado a los muertos; ahora también podían orar por ellos. El budismo pronto se fusionó con los vestigios de las prácticas rituales de un taoísmo en vías de desaparición. La religión nueva que se formó, síntesis de ambas tradiciones, con sus templos de adoración y un determinado ceremonial religioso pronto se convirtió en el sistema de culto generalmente aceptado por los pueblos de China, Corea y Japón.
\vs p094 11:2 \pc Aunque, en algunos aspectos, es lamentable que el budismo no se difundiese en el mundo hasta después de que los seguidores de Gautama hubiesen distorsionado en tan gran medida las tradiciones y enseñanzas de su sistema de culto como para convertir a Gautama en un ser divino, este mito de su vida humana, adornado, como lo fue, con una gran cantidad de milagros, resultó muy atrayente para quienes oían el evangelio del budismo norteño o mahayanaista.
\vs p094 11:3 Más adelante, algunos de sus seguidores enseñaron que el espíritu de Sakyamuni Buda regresaba periódicamente a la tierra como un Buda vivo, abriendo así el camino para la perpetuación indefinida de imágenes, templos, rituales a Buda y de falsos “Budas vivos”. De esta manera, la religión del gran discrepante indio se encontró finalmente constreñida por aquellas mismas prácticas ceremoniales e invocaciones ritualistas contra las que él había luchado tan decididamente, y que tan valientemente había denunciado.
\vs p094 11:4 \pc El gran logro de la filosofía budista consistió en su comprensión de que toda verdad es relativa. Mediante esta hipótesis, los budistas han sido capaces de compatibilizar y correlacionar las disparidades existentes dentro de sus propias escrituras religiosas al igual que las diferencias entre las suyas y las de muchos otros. Se enseñaba que la pequeña verdad era para las mentes pequeñas y la gran verdad para las grandes mentes.
\vs p094 11:5 Esta filosofía sostenía también que la naturaleza (divina) del Buda residía en todos los hombres; que el hombre, a través de su propio esfuerzo, podía alcanzar la realización de esta divinidad interior. Y esta enseñanza es una de las expresiones más claras de la verdad sobre los modeladores interiores que una religión haya hecho jamás en Urantia.
\vs p094 11:6 Pero un gran inconveniente del evangelio primigenio de Siddharta, tal como sus seguidores lo interpretaban, era que pretendía liberar por completo al yo humano de todas las limitaciones de la naturaleza mortal aislando al yo de la realidad objetiva. La auténtica autorrealización cósmica es consecuencia de la identificación con la realidad cósmica y con el cosmos finito de la energía, la mente y el espíritu, limitado por el espacio y condicionado por el tiempo.
\vs p094 11:7 Aunque las ceremonias y las celebraciones externas del budismo quedaron gravemente contaminadas con las de los territorios a los que se desplazaban, esta degradación no sucedió del todo en el caso de la vida filosófica de los grandes pensadores que, ocasionalmente, abrazaron este sistema de pensamiento y creencias. Durante más de dos mil años, muchas de las mejores mentes de Asia se han centrado en la problemática de determinar la verdad absoluta y la verdad del Absoluto.
\vs p094 11:8 La evolución de un elaborado concepto del Absoluto vino a través de múltiples cauces de pensamiento y por enrevesados caminos de razonamiento. El creciente ascenso de esta doctrina de la infinitud no se definió con tanta claridad como lo fue la evolución del concepto de Dios en la teología hebrea. Sin embargo, las mentes de los budistas alcanzaron, sopesaron y pasaron por ciertos niveles de profundidad en su camino a la conceptualización de la Fuente Primigenia de los universos.
\vs p094 11:9 \li{1.}\bibemph{La leyenda de Gautama}. En la base de esta percepción, estaba el hecho histórico de la vida y enseñanzas de Siddharta, el príncipe profeta de la India. Esta leyenda se convirtió en mito conforme discurrió a través de los siglos y de los extensos territorios de Asia hasta llegar a trascender la idea del estatus de Gautama como ser iluminado y le empezaron a añadir otros atributos.
\vs p094 11:10 \li{2.}\bibemph{Los muchos Budas}. Se argumentaba que, si Gautama había venido a los pueblos de la India, entonces, en el distante pasado y en el futuro remoto las razas de la humanidad debían haber sido e indudablemente serían, bendecidas con otros maestros de la verdad. Esto dio origen a la enseñanza de que había muchos Budas, un número ilimitado e infinito de ellos, e incluso de que cualquiera podía aspirar a convertirse en uno de ellos ---alcanzar la divinidad de un Buda---.
\vs p094 11:11 \li{3.}\bibemph{El Buda absoluto}. En el momento en que el número de Budas se aproximaba a la infinitud, se hizo necesario que las mentes de aquellos días reunificaran este concepto tan difícil de manejar. Por consiguiente, comenzó a impartirse la enseñanza de que todos los Budas no eran sino manifestaciones de una esencia superior, de un Uno Eterno de existencia infinita e incondicionada, de una Fuente Absoluta de toda la realidad. A partir de ahí, el concepto budista de la Deidad, en su acepción más elevada, se separa de la persona humana de Gautama Siddharta y se libera de las limitaciones antropomórficas a las que ha estado sujeto. Esta conceptualización última del Buda Eterno se puede muy bien identificar con el Absoluto, e incluso a veces con el infinito YO SOY.
\vs p094 11:12 \pc Aunque esta idea de la Deidad Absoluta no encontró nunca gran apoyo popular en los pueblos de Asia, sí permitió que los intelectuales de esas tierras unificaran su filosofía y armonizaran su cosmología. El concepto del Buda Absoluto es, en algunas ocasiones, casi personal y, en otras, completamente impersonal ---e incluso una infinita fuerza creativa---. Tales conceptos, aunque útiles para la filosofía, no son vitales para el desarrollo religioso. Incluso un Yahvé antropomórfico es de mayor valor religioso que el Absoluto infinitamente remoto del budismo o del brahmanismo.
\vs p094 11:13 Hubo momentos en los que incluso se pensó que el Absoluto estaba contenido dentro del infinito YO SOY. Pero estas consideraciones resultaban un frío consuelo para las multitudes hambrientas que anhelaban oír palabras de promesa, escuchar el sencillo evangelio de Salem de que la fe en Dios garantizaba el favor divino y la supervivencia eterna.
\usection{12. CONCEPTO DE DIOS DEL BUDISMO}
\vs p094 12:1 En la cosmología del budismo había dos grandes puntos débiles: su contaminación de las numerosas supersticiones de la India y China y su sublimación de Gautama, primero, como el iluminado y, después, como el Buda Eterno. Al igual que el cristianismo se ha visto afectado por la absorción de mucha filosofía humana equivocada, de la misma manera, el budismo porta también su marca de nacimiento de índole humana. Pero las enseñanzas de Gautama han seguido evolucionando en los últimos dos mil quinientos años. Para un budista bien informado, el concepto de Buda ya no es la persona humana de Gautama como, para un cristiano bien informado, el concepto de Jehová no es el espíritu demoníaco del Horeb. La pobreza terminológica, junto con la conservación sentimental de la nomenclatura antigua, provoca a menudo la incapacidad de comprender el verdadero significado de la evolución de los conceptos religiosos.
\vs p094 12:2 \pc Gradualmente, el concepto de Dios, en su contraste con el Absoluto, comenzó a aparecer en el budismo. Sus fuentes se remontan a los tempranos días en los que se dio la diferenciación entre los seguidores del Camino Menor y los del Camino Mayor. Fue en esta última rama del budismo donde la doble conceptualización de Dios y el Absoluto alcanzó finalmente su madurez. Paso a paso, siglo tras siglo, el concepto de Dios ha evolucionado hasta que, con las enseñanzas de Ryonin, Honen Shonin y Shinran en Japón, este concepto llegaría por último a fructificar en la creencia en Amida Buda.
\vs p094 12:3 Entre estos creyentes, se enseña que el alma, al experimentar la muerte, puede optar por gozar de una estancia en el Paraíso antes de entrar en el Nirvana, la existencia última. Se anuncia que esta nueva salvación se alcanza por la fe en la misericordia divina y por el amoroso cuidado de Amida, Dios del Paraíso en el oeste. En su filosofía, los amidistas creen en una Realidad Infinita que está más allá de toda comprensión mortal finita; en su religión, se aferran a la fe en un Amida todo misericordioso, que tanto ama al mundo que no consentirá que mortal alguno que invoque su nombre con verdadera fe y un corazón puro deje de conseguir la inefable felicidad del Paraíso.
\vs p094 12:4 La gran fuerza del budismo reside en el hecho de que sus partidarios son libres de elegir la verdad proveniente de todas las religiones; tal libertad de elección rara veces ha caracterizado la fe urantiana. En este aspecto, la denominación Shin de Japón se ha convertido en uno de los grupos religiosos más avanzados del mundo; ha reavivado el ancestral espíritu misionero de los seguidores de Gautama y ha comenzado a enviar maestros a otros pueblos. Esta disposición para beneficiarse de la verdad de cualquier otra fuente es ciertamente una tendencia encomiable, que aparece entre los creyentes religiosos durante la primera mitad del siglo XX d. C.
\vs p094 12:5 En el siglo XX, el mismo budismo está experimentando un renacimiento. A través de su contacto con el cristianismo, los aspectos sociales del budismo han mejorado significativamente. El deseo de aprender ha renacido en los corazones de los monjes sacerdotes de la comunidad, y la difusión de la educación en esta fe religiosa provocará con toda certeza nuevos logros en su desarrollo religioso.
\vs p094 12:6 En el momento de este texto escrito, una gran parte de Asia tiene su esperanza puesta en el budismo. ¿Podrá esta noble comunidad de fe, que con tanta valentía resistió las oscuras eras del pasado, ser receptiva una vez más a la verdad de las realidades cósmicas expandidas, tal como los discípulos del gran maestro de la India lo fueron en su momento a la proclamación de la nueva verdad? ¿Responderá de nuevo esta ancestral comunidad de fe al revitalizante estímulo de la exposición de los nuevos conceptos de Dios y del Absoluto que ha buscado durante tanto tiempo?
\vs p094 12:7 \pc En toda Urantia se está a la espera de la proclamación del mensaje ennoblecedor de Miguel, sin la carga de las doctrinas y dogmas acumulados en diecinueve siglos de contacto con las religiones de origen evolutivo. Está llegando la hora de llevar al budismo, al cristianismo, al hinduismo e incluso a los pueblos de cualquier fe religiosa, no el evangelio sobre Jesús, sino la realidad viva, espiritual, del evangelio de Jesús.
\vsetoff
\vs p094 12:8 [Exposición de un melquisedec de Nebadón.]
