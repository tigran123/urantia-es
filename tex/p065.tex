\upaper{65}{La acción directiva sobre la evolución}
\author{Portador de vida}
\vs p065 0:1 La vida material evolutiva de orden básico ---la vida previa a la mente--- resulta de la formulación de los controladores físicos mayores y del ministerio de impartición de la vida de los siete espíritus mayores, en conjunción con la cooperación activa de los portadores de vida asignados a tal fin. Como resultado de la acción correlacionada de este triple proceso creativo, se desarrolla la capacidad física y la mente del organismo ---mecanismos materiales para la reacción inteligente a los estímulos ambientales externos y, más adelante, a los estímulos internos o influencias que se originan en la mente misma del organismo---.
\vs p065 0:2 \pc Existen, pues, tres niveles diferenciados en la creación y evolución de la vida:
\vs p065 0:3 \li{1.}El entorno físico\hyp{}energético, que conforma la capacidad mental.
\vs p065 0:4 \li{2.}El ministerio ejercido en la mente por los espíritus asistentes, que incide en la capacidad espiritual.
\vs p065 0:5 \li{3.}La dote espiritual de la mente mortal, que culmina en la dádiva del modelador del pensamiento.
\vs p065 0:6 \pc Los niveles no educables y mecánicos de respuesta del organismo al medio ambiente son los ámbitos de los que se ocupan los controladores físicos. Los espíritus asistentes de la mente activan y regulan los tipos de mente que son adaptables o no mecánicos, esto es, educables ---esos mecanismos de respuesta de los organismos capaces de aprender de la experiencia---. Y, del mismo modo en el que los espíritus asistentes actúan sobre los potenciales de la mente, los portadores de vida ejercen una considerable acción directiva, aunque discrecional, sobre los aspectos ambientales de los procesos evolutivos, hasta el mismo momento en el que aparece la voluntad humana ---la capacidad de conocer a Dios y la facultad de elegir adorarle---.
\vs p065 0:7 Es la labor integrada de los portadores de vida, de los controladores físicos y de los espíritus asistentes la que condiciona el curso de la evolución orgánica en los mundos habitados. Y, por este motivo, la evolución ---en Urantia y en otros lugares--- siempre es intencionada y nunca accidental.
\usection{1. LAS FUNCIONES DE LOS PORTADORES DE VIDA}
\vs p065 1:1 Los portadores de vida están dotados de unos potenciales de metamorfosis de su ser personal que muy pocos órdenes de criaturas poseen. Estos hijos del universo local son capaces de actuar en tres fases distintas del ser. Habitualmente, desempeñan sus deberes en calidad de hijos de la fase intermedia, al ser esta su condición de origen. Si bien, no es posible que un portador de vida en tal grado de existencia obre en los ámbitos electroquímicos, combinando las energías físicas y las partículas materiales para crear entidades vivas.
\vs p065 1:2 Los portadores de vida pueden actuar y en efecto lo hacen en los tres niveles siguientes:
\vs p065 1:3 \li{1.}El nivel físico de la electroquímica.
\vs p065 1:4 \li{2.}La fase intermedia usual de la existencia cuasi\hyp{}morontial.
\vs p065 1:5 \li{3.}El nivel semiespiritual avanzado.
\vs p065 1:6 \pc Cuando los portadores de vida se preparan para implantar la vida, y tras haber elegido los emplazamientos para tal tarea, convocan a la comisión de los arcángeles encargada de la transmutación de los portadores. Este grupo consta de diez órdenes de distintos seres personales que incluyen a los controladores físicos y a sus compañeros, y está presidida por el jefe de los arcángeles, que actúa en esta función por mandato de Gabriel y con el permiso de los ancianos de días. Cuando estos seres están debidamente interconectados, pueden efectuar unas modificaciones en los portadores de vida que les permitirán actuar de inmediato en los niveles físicos de la electroquímica.
\vs p065 1:7 Una vez que se han formulado los modelos de vida y se ha completado debidamente la organización material, las fuerzas supramateriales implicadas en la propagación de la vida se activan al instante y la vida comienza a existir. A raíz de lo cual, a los portadores de vida se les restituye de inmediato a su habitual fase intermedia de su existencia como seres personales, en cuyo estado pueden actuar sobre las entidades vivas y guiar a los organismos en evolución, aun cuando se les despoja de toda capacidad de organizar ---de crear--- nuevos modelos de materia viva.
\vs p065 1:8 Después de que la evolución orgánica ha completado su recorrido y ha aparecido la libre voluntad de orden humano en los organismos evolutivos superiores, los portadores de vida deben o bien abandonar el planeta o hacer votos de renunciación; es decir, han de comprometerse a abstenerse de cualquier nuevo intento por influir en el curso de la evolución orgánica. Y, cuando estos votos se hacen de forma voluntaria, por parte de aquellos portadores de vida que eligen permanecer en el planeta como asesores futuros de aquellos a los que se les encomendará el fomento de las criaturas volitivas recién evolucionadas, se convoca una comisión de doce miembros, presidida por el jefe de las estrellas vespertinas, en representación del soberano del sistema, actuando mediante la autoridad de Gabriel; e, inmediatamente, se transmutan estos portadores de vida a la tercera fase de su existencia como seres personales ---el nivel semiespiritual del ser---. Vengo ejerciendo mis funciones en Urantia en esta tercera fase desde los tiempos de Andón y Fonta.
\vs p065 1:9 Anhelamos ese momento en el que el universo se asiente en luz y vida y alcancemos una posible cuarta etapa de nuestra existencia en la que seremos completamente espirituales; si bien, nunca se nos ha revelado la forma en la podamos lograr ese estatus deseable y avanzado.
\usection{2. VISIÓN DE CONJUNTO DE LA EVOLUCIÓN}
\vs p065 2:1 El relato del ascenso del hombre desde las algas marinas hasta el dominio de la creación terrestre es, de hecho, una aventura de lucha biológica y de supervivencia de la mente. Los ancestros primigenios del hombre fueron, literalmente, el limo y el fango del lecho oceánico contenidos en las bahías y lagunas de aguas cálidas y tranquilas de los inmensos litorales que recorrían los ancestrales mares interiores; en aquellas mismas aguas, los portadores de vida establecieron las tres independientes implantaciones de vida ocurridas en Urantia.
\vs p065 2:2 Existen en la actualidad muy pocas de las tempranas especies de vegetación marina que tomaron parte en aquellos decisivos cambios, y que resultaron en los organismos situados en la zona fronteriza de lo animal. Las esponjas son los supervivientes de uno de estos tempranos tipos intermedios, de esos organismos gracias a los que se produjo la transición \bibemph{gradual} desde lo vegetal hasta lo animal. Estas primeras formas transitorias, aunque muy parecidas, no eran idénticas a las esponjas modernas; eran verdaderos organismos limítrofes ---ni vegetal ni animal---, pero acabaron por abrir camino al desarrollo de las verdaderas formas de vida animal.
\vs p065 2:3 Las bacterias, unos organismos vegetales simples de naturaleza muy primitiva, han cambiado muy poco desde los tempranos albores de la vida; incluso muestran algún grado de retrogresión en su comportamiento parasitario. Muchos hongos presentan también un movimiento retrógrado en su evolución, al ser plantas que han perdido su capacidad de producir clorofila y se han vuelto más o menos parasitarias. La mayoría de las bacterias productoras de enfermedades y sus auxiliares, los organismos víricos, pertenecen a este grupo de hongos parasitarios disidentes. Durante las eras intermedias, todo el inmenso reino de la vida vegetal evolucionó a partir de unos ancestros de los que también descienden las bacterias.
\vs p065 2:4 Pronto apareció, y lo hizo \bibemph{de repente,} el tipo de vida animal protozoario superior. Y, de aquellos lejanos tiempos, proviene la ameba, el típico organismo animal unicelular, que ha llegado a nuestros días con pocas modificaciones. Hoy pasa su tiempo de forma muy parecida a como lo hacía cuando era el último y más importante logro de la evolución de la vida. Esta criatura diminuta y sus primos protozoarios son para la creación animal lo que las bacterias para el reino vegetal; constituyen la supervivencia de las tempranas etapas evolutivas en la diferenciación de la vida junto con el \bibemph{fracaso de su desarrollo posterior}.
\vs p065 2:5 En poco tiempo, los tempranos tipos de animales unicelulares se unieron en colonias, al principio siguiendo el modo del volvox y, en breve, el de la hidra y la medusa. Más tarde, evolucionaron la estrella de mar, los lirios de piedra, los erizos de mar, los pepinos de mar, los ciempiés, los insectos, las arañas, los crustáceos y los grupos muy afines de los gusanos y las sanguijuelas, a los que pronto siguieron los moluscos ---la ostra, el pulpo y el caracol---. Cientos y cientos de especies surgieron y perecieron; solo se mencionan a esas especies que sobrevivieron a aquella prolongada lucha. Dichos especímenes sin evolucionar, junto a la familia de los peces que apareció después, constituyen hoy en día los tipos estacionarios de animales primitivos e inferiores, esto es, las ramas del árbol de la vida que no consiguió avanzar.
\vs p065 2:6 \pc Se disponía así el escenario para la aparición de los primeros animales vertebrados: los peces. A partir de la familia de los peces se originaron dos excepcionales modificaciones: la rana y la salamandra. Y, a partir de la rana, comenzaría aquella serie de diferenciaciones progresivas que culminó por fin en el hombre mismo.
\vs p065 2:7 La rana es uno de los ancestros supervivientes más tempranos de la raza humana, pero tampoco logró avanzar; subsiste en el presente de forma muy similar a la que tenía en esos tiempos remotos. La rana es la única especie antecesora de las primeras razas precursoras que vive en este momento sobre la faz de la tierra. La raza humana no tiene ningún antepasado vivo entre la rana y el esquimal.
\vs p065 2:8 \pc Las ranas dieron lugar a los reptiles, una gran familia animal prácticamente extinta, pero que, antes de desaparecer, dio origen a toda la familia de las aves y a numerosos órdenes de mamíferos.
\vs p065 2:9 Probablemente, el salto único más grande de toda la evolución prehumana se haya realizado cuando el reptil se volvió ave. Los tipos de aves de hoy ---águilas, patos, palomas, y avestruces--- descendieron todos de los enormes reptiles de hace muchísimo tiempo.
\vs p065 2:10 El reino de los reptiles, descendiente de la familia de la rana, se representa hoy mediante cuatro ramificaciones de supervivientes: dos de ellas, no evolucionadas, las serpientes y los lagartos, junto con sus primos, los cocodrilos y las tortugas; una de ellas, parcialmente evolucionada, la familia de las aves; y la cuarta, los ancestros de los mamíferos y la línea directa de la que desciende la especie humana. Pero, la imponente cantidad de reptiles, aunque extintos desde hace mucho tiempo, tuvo resonancia en el elefante y el mastodonte, mientras que sus particulares formas se perpetuaron en los canguros saltadores.
\vs p065 2:11 \pc En Urantia solamente han aparecido catorce filos; el de los peces es el último, y no se han desarrollado nuevas clases después de la de las aves y la de los mamíferos.
\vs p065 2:12 \pc Los mamíferos placentarios surgieron \bibemph{de repente} de un dinosaurio reptil, ágil y pequeño, de hábitos carnívoros, pero que poseía un cerebro relativamente grande. Estos mamíferos se desarrollaron rápidamente y de muchas formas diferentes, dando origen no solo a las variedades comunes modernas, sino que también evolucionaron en especies marinas, como la ballena y las focas, y en los navegantes aéreos como la familia de los murciélagos.
\vs p065 2:13 Así pues, el hombre evolucionó a partir de los mamíferos superiores derivados, principalmente, de la \bibemph{implantación occidental} de vida en los ancestrales mares resguardados del este hasta el oeste. Los grupos \bibemph{oriental} y \bibemph{central} de organismos vivos pronto avanzaron de forma favorable hacia el logro de los niveles prehumanos de la existencia animal. Pero, al paso de las eras, el foco oriental del emplazamiento de la vida no llegó a alcanzar un nivel satisfactorio de inteligencia prehumana; sufrió, repetidas veces y de modo irrecuperable, tales pérdidas de sus tipos superiores de plasma germinal, que quedó desprovisto para siempre de su capacidad de rehabilitar las potencialidades humanas.
\vs p065 2:14 Puesto que la cualidad de la capacidad mental de este grupo oriental para un postrero desarrollo era tan categóricamente inferior a la de los otros dos grupos, los portadores de vida, con el consentimiento de sus superiores, actuaron sobre el medio ambiente a fin de delimitar más estos linajes prehumanos inferiores de vida evolutiva. Visto exteriormente, la exclusión de estos grupos de criaturas de índole inferior parece accidental; pero, en realidad, fue totalmente deliberada.
\vs p065 2:15 En el despliegue evolutivo de la inteligencia que siguió, los lémures, los ancestros de las especies humanas, estaban mucho más avanzados en América del Norte que en otras regiones; y, por ello, se les impulsó a emigrar desde el entorno de la \bibemph{implantación occidental} de la vida, cruzando el puente terrestre de Bering y continuando por la costa, hasta el sudoeste de Asia, donde prosiguieron su evolución y se beneficiaron de la incorporación de algunas estirpes del grupo central de la vida. Así pues, el hombre evolucionó a partir de determinadas estirpes occidentales y centrales de la vida, pero desde las regiones centrales hasta el Próximo Oriente.
\vs p065 2:16 De esta forma, la vida implantada en Urantia evolucionó hasta la edad del hielo, momento en el que el hombre mismo apareció por primera vez y comenzó su azarosa andadura planetaria. Y esta aparición del hombre primitivo en la tierra durante el período glacial no fue por puro azar; fue deliberada. Los rigores y la dureza climática de la era glacial se adecuaron, en todos los sentidos, al propósito de fomentar la creación de un tipo resistente de ser humano en posesión de una enorme capacidad para sobrevivir.
\usection{3. FOMENTO DE LA EVOLUCIÓN}
\vs p065 3:1 Resulta prácticamente imposible explicarle a la mente humana de hoy día muchos de los hechos extraños y, en apariencia grotescos, del  primitivo progreso evolutivo. Un plan calculado operaba bajo esta aparentemente extraña evolución de los seres vivos; pero no se nos permite interferir arbitrariamente en el desarrollo de los modelos de vida una vez que se han puesto en marcha.
\vs p065 3:2 \pc Los portadores de vida pueden emplear cualquier recurso natural posible y utilizar todas y cada una de las circunstancias fortuitas que ayuden al progreso evolutivo de las formas de vida experimentales; pero no estamos autorizados a intervenir de forma mecánica ni actuar de forma arbitraria sobre la dirección ni el curso de la evolución, ya sea vegetal o animal.
\vs p065 3:3 Se os ha informado de que los mortales de Urantia evolucionaron mediante el desarrollo de una rana primitiva y de que, a esta estirpe ascendente, contenida potencialmente en una sola rana, le faltó poco en cierta ocasión para extinguirse. Pero no se debe deducir de ahí que se hubiese puesto fin en tal punto crítico a la evolución de la humanidad por un mero azar. En ese mismo momento, estaban a examen y fomentábamos no menos de mil variedades de linajes diferentes de vida remotamente situadas y en estado de mutación, que podrían haberse destinado para formar distintos modelos de desarrollo prehumano. Esta particular rana antecesora constituía nuestra tercera elección; las dos estirpes de vida anteriores habían perecido, a pesar de nuestro empeño por conservarlas.
\vs p065 3:4 \pc Incluso la pérdida de Andón y Fonta antes de que tuvieran vástagos, aunque habría retrasado la evolución humana, no la hubiese evitado. Tras la aparición de Andón y Fonta y antes de que se agotaran los potenciales humanos de la vida animal, en proceso de mutación, evolucionaron no menos de siete mil linajes favorables que podrían haber alcanzado alguna clase de desarrollo de tipo humano. Y muchas de estas excelentes estirpes se asimilaron posteriormente por las diversas ramas de las especies humanas en expansión.
\vs p065 3:5 Mucho antes de que el hijo y la hija material ---los mejoradores biológicos--- lleguen a un planeta, los potenciales humanos de las especies animales evolutivas ya se han agotado. Esta condición biológica de la vida animal se le desvela a los portadores de vida mediante el fenómeno de la tercera fase de activación de los espíritus asistentes, que ocurre de forma automática y coincidente con tal agotamiento de la capacidad de toda vida animal para dar origen a los potenciales de mutación de individuos prehumanos.
\vs p065 3:6 \pc La humanidad en Urantia ha de resolver sus problemas de desarrollo humano con los linajes que posee ---en todo el tiempo futuro por llegar no volverán a evolucionar más razas a partir de fuentes prehumanas---. Pero este hecho no excluye la posibilidad de alcanzar niveles de desarrollo humano considerablemente superiores por medio del fomento inteligente de los potenciales evolutivos que aún residen en las razas mortales. Aquello que nosotros, los portadores de vida, hacemos en favor de la promoción y la conservación de las estirpes de vida antes de la aparición de la voluntad humana, lo ha de hacer el hombre por sí mismo, tras tal suceso, y con posterioridad a nuestro retiro de la participación activa en la evolución. De manera general, el destino evolutivo del hombre está en sus propias manos y la inteligencia científica, antes o después, debe reemplazar el funcionamiento aleatorio de la selección natural incontrolada y de la supervivencia fortuita.
\vs p065 3:7 Y al tratar del fomento de la evolución, no sería superfluo indicar que si, en el largo futuro que tenéis por delante, alguna vez os unís a un colectivo de portadores de vida, dispondréis de numerosas oportunidades para ofrecer vuestras propuestas y mejorar en lo posible los planes y métodos de la dirección y el trasplante de la vida. ¡Sed pacientes! Si tenéis buenas ideas, si vuestras mentes son fértiles en mejores procedimientos administrativos para cualquier parte de los ámbitos universales, de cierto tendréis la oportunidad de presentarlos a vuestros compañeros gestores de las eras venideras.
\usection{4. LA AVENTURA URANTIANA}
\vs p065 4:1 No paséis por alto que se nos asignó Urantia como mundo de vida experimental. En este planeta realizamos nuestro sexagésimo intento de modificar y, en la medida de lo posible, mejorar la adaptación al modo de Satania de los diseños de vida de Nebadón, y queda constancia de que hemos conseguido numerosas modificaciones beneficiosas de los modelos de vida regulares. Concretamente, en Urantia hemos elaborado y presentado, de manera satisfactoria, no menos de veintiocho características de modificación de la vida, que serán de utilidad para la totalidad de Nebadón a lo largo de todo el tiempo futuro.
\vs p065 4:2 Si bien, el establecimiento de la vida en un planeta nunca es experimental en el aspecto de que se intente llevar a cabo algo no probado y desconocido. El método usado para evolucionar la vida es siempre progresivo, diferenciado y variable, pero nunca aleatorio, incontrolado ni completamente experimental, en el sentido de imprevisión.
\vs p065 4:3 \pc Existen muchos rasgos de la vida humana que aportan un gran número de pruebas de que el fenómeno de la existencia mortal está planeado de forma inteligente, y de que la evolución orgánica no es un mero accidente cósmico. Cuando una célula viva resulta dañada, esta dispone de la capacidad de elaborar ciertas sustancias químicas facultadas para estimular y activar las células vecinas normales, de tal modo que estas comienzan de inmediato a secretar ciertas sustancias facilitadoras de los procesos curativos de la herida; y, al mismo tiempo, dichas células normales no dañadas empiezan a proliferar ---en realidad, se ponen a trabajar para crear nuevas células y reemplazar a aquellas otras que pudiesen haber quedado destruidas en tal circunstancia---.
\vs p065 4:4 Esta acción y reacción química implicada en la sanación de la herida y en la reproducción de las células constituye la elección de los portadores de vida de un procedimiento que abarca más de cien mil fases y rasgos característicos de reacciones químicas y repercusiones biológicas posibles. Los portadores de vida realizaron en sus laboratorios más de medio millón de experimentos concretos antes de decidirse finalmente por esta fórmula para su experimento de vida en Urantia.
\vs p065 4:5 Cuando los científicos de Urantia conozcan mejor estas sustancias químicas curativas, se volverán más eficaces en el tratamiento de las lesiones e, indirectamente, sabrán mejor cómo controlar determinadas enfermedades graves.
\vs p065 4:6 Desde que se estableció la vida en Urantia, los portadores de vida han mejorado este método de sanación hasta el punto de que se ha implantado en otro mundo de Satania, en donde proporciona allí un mejor alivio del dolor y ejerce un mayor control sobre la capacidad de proliferación de las células aledañas normales.
\vs p065 4:7 \pc En el experimento con la vida realizado en Urantia, hubo elementos singulares, pero los dos sucesos más sobresalientes fueron la aparición de la raza andónica con anterioridad a la evolución de los seis pueblos de color y la aparición posterior y simultánea de los mutantes sangik en una única familia. Urantia es el primer mundo de Satania en el que las seis razas de color nacieron de la misma familia humana. Normalmente surgen estirpes diversificadas a partir de mutaciones independientes dentro del linaje animal prehumano, y suelen aparecer en la tierra de una en una y de modo sucesivo a lo largo de periodos prolongados de tiempo, comenzando con el hombre rojo y pasando por todos los colores hasta el índigo.
\vs p065 4:8 Otra destacable variación del procedimiento fue la llegada tardía del príncipe planetario. Por lo general, el príncipe aparece en un planeta en torno al momento del desarrollo de la voluntad; y si se hubiese seguido tal plan, Caligastia podría haber venido a Urantia durante las vidas de Andón y Fonta en lugar de casi quinientos mil años después, en simultaneidad con la aparición de las seis razas sangik.
\vs p065 4:9 A petición de los portadores de vida, en el caso de un mundo habitado ordinario se le hubiese concedido un príncipe planetario en el momento de la aparición de Andón y Fonta, o algún tiempo después. Pero, puesto que se había designado a Urantia como planeta de modificación de la vida, hubo un preacuerdo para enviar a los observadores melquisedecs, doce en total, en calidad de asesores de los portadores de vida y de supervisores del planeta hasta la posterior llegada del príncipe planetario. Estos melquisedecs vinieron en el momento en el que Andón y Fonta tomaron las decisiones que permitirían a los modeladores del pensamiento habitar en sus mentes mortales.
\vs p065 4:10 \pc En Urantia, las iniciativas de los portadores de vida para mejorar los modelos de vida de Satania resultaron necesariamente en la formación de muchas formas transitorias de vida supuestamente inutilizables. No obstante, los logros ya generados bastan para justificar las modificaciones realizadas en Urantia de los diseños regulares de la vida.
\vs p065 4:11 Nuestra intención fue la de crear una temprana manifestación de la voluntad en la vida evolutiva de Urantia, y tuvimos éxito. Habitualmente, la voluntad no emerge hasta mucho tiempo después de originarse las razas de color y suele surgir por primera vez entre los grupos más dotados del hombre rojo. Vuestro mundo es el único planeta de Satania donde el orden humano de voluntad apareció en una raza previa a las de color.
\vs p065 4:12 Si bien, en nuestro empeño por facilitar esa combinación y correlación de factores hereditarios que acabarían por dar nacimiento a los ancestros mamíferos de la raza humana, nos vimos en la necesidad de permitir que se produjeran cientos de miles de otras combinaciones y correlaciones de factores hereditarios relativamente superfluos. Al ahondar en el pasado del planeta, es seguro que vuestra mirada se encontrará con muchos de estos resultados colaterales, aparentemente extraños, de nuestras tentativas, y comprendo bien lo desconcertante que algunas de estas cosas pueden parecer para el limitado punto de vista humano.
\usection{5. VICISITUDES DE LA EVOLUCIÓN DE LA VIDA}
\vs p065 5:1 Los portadores de vida lamentamos que nuestro particular empeño por modificar la vida inteligente de Urantia se hubiese visto tan impedido por trágicas aberraciones, fuera de nuestro control, como fueron la traición de Caligastia y la transgresión de Adán.
\vs p065 5:2 Si bien, durante toda esta aventura biológica, nuestra mayor decepción fue la reversión de alguna vida primitiva vegetal hasta los niveles preclorofílicos de la bacteria parasitaria en tan grande e imprevista magnitud. Esta eventualidad en la evolución de la vida vegetal ocasionó muchas inquietantes enfermedades en los mamíferos superiores, en particular en las especies humanas más vulnerables. Cuando hicimos frente a esta desconcertante situación, desechamos en cierto modo los inconvenientes que suponían, porque sabíamos que la posterior dotación del plasma vital adánico reforzaría la capacidad de resistencia de la raza mestiza que se derivaría, inmunizándola prácticamente contra todas las enfermedades producidas por este tipo de organismo vegetal. Pero nuestras expectativas estaban condenadas al fracaso debido a la fatalidad de tal transgresión adánica.
\vs p065 5:3 El universo de los universos, en el que se incluye este pequeño mundo llamado Urantia, no se gobierna simplemente para merecer nuestra aprobación ni para acomodarse a nuestra conveniencia ni mucho menos para complacer nuestros caprichos y satisfacer nuestra curiosidad. Los seres de sabiduría y omnipotencia que tienen la responsabilidad de dirigir el universo, sin duda, saben muy bien lo que tienen que hacer; y así le sucede a los portadores de vida y le corresponde a la mente mortal procurar esperar con paciencia y cooperar fervientemente con el régimen de la sabiduría, el reino del poder y la marcha del progreso.
\vs p065 5:4 Existen, naturalmente, ciertas compensaciones por las adversidades padecidas, tal como el ministerio de gracia de Miguel en Urantia. Pero, con independencia de tales consideraciones, los más recientes supervisores celestiales de este planeta manifiestan su entera confianza en el postrero triunfo evolutivo de la raza humana y en la futura vindicación de nuestros planes y modelos de vida primigenios.
\usection{6. MÉTODOS DE EVOLUCIÓN DE LA VIDA}
\vs p065 6:1 Es imposible determinar con precisión, de forma simultánea, la ubicación exacta y la velocidad de un objeto en movimiento; cualquier intento de medir uno de estos dos factores supondría inevitablemente la modificación del otro. El hombre mortal se enfrenta con el mismo tipo de paradoja al emprender el análisis químico del protoplasma. El químico puede dilucidar la constitución del protoplasma \bibemph{muerto,} pero no puede percibir la organización física ni el comportamiento dinámico del protoplasma vivo. El científico se irá acercando cada vez más a los secretos de la vida, pero nunca los descubrirá por la sencilla razón de que debe matar al protoplasma para analizarlo. El protoplasma muerto pesa lo mismo que el protoplasma vivo, pero no es lo mismo.
\vs p065 6:2 \pc Las cosas y los seres vivos están dotados de una capacidad propia para la adaptación. En cada célula \bibemph{viva,} vegetal o animal, en cada organismo \bibemph{vivo,} material o espiritual, existe un ansia insaciable por alcanzar una perfección creciente de su ajuste al medio ambiente, de su adaptación orgánica y de hallar una mejor realización en la vida. Estos interminables esfuerzos de todos los seres vivos demuestran la existencia en ellos de un afán innato de perfección.
\vs p065 6:3 El paso más importante de la evolución de las plantas fue el desarrollo de la capacidad de producir clorofila, y el segundo gran avance fue la evolución de la espora en una semilla compleja. La espora es sumamente eficiente como agente reproductor, pero carece de los potenciales de variedad y versatilidad inherentes a la semilla.
\vs p065 6:4 Uno de los episodios más útiles y complejos de la evolución de los tipos superiores de animales consistió en el desarrollo de la capacidad del hierro contenido en las células sanguíneas circulantes para realizar la doble labor de transportar el oxígeno y de eliminar el dióxido de carbono. Y esta actuación de los glóbulos rojos ilustra cómo los organismos en evolución pueden adaptar sus funciones a un medio ambiente variable o cambiante. Los animales superiores, incluido el hombre, oxigenan sus tejidos mediante la acción del hierro contenido en los glóbulos rojos de la sangre, que transporta el oxígeno a las células vivas y, con la misma eficiencia, elimina el dióxido de carbono. Pero hay otros metales que pueden servir para el mismo propósito. La sepia emplea el cobre para esta función, y la ascidia utiliza el vanadio.
\vs p065 6:5 La continuación de dichas adaptaciones biológicas se muestra en la evolución de los dientes de los mamíferos superiores de Urantia, que llegaron hasta treinta y seis en los ancestros remotos del hombre, para comenzar después un reajuste adaptativo hacia los treinta y dos en el hombre primitivo y sus parientes cercanos. En este momento, la especie humana se inclina lentamente hacia los veintiocho dientes. En este planeta, el proceso evolutivo sigue avanzando de forma activa en su capacidad adaptativa.
\vs p065 6:6 Pero muchas de las adaptaciones de los organismos vivos, aparentemente misteriosas, son de carácter puramente químico, totalmente físico. En cualquier momento, existe, en la corriente sanguínea de cualquier ser humano, la posibilidad de más 15\,000\,000 de reacciones químicas entre las hormonas segregadas por una docena de glándulas endocrinas.
\vs p065 6:7 \pc Las formas inferiores de la vida vegetal son enteramente reactivas al entorno físico, químico y eléctrico. Pero, a medida que se asciende en la escala de la vida, los servicios que los siete espíritus asistentes ofrecen a la mente inician su actividad, uno tras otro, y la mente se hace cada vez más adaptativa, creativa, coordinativa y dominante. La capacidad de los animales para adaptarse al aire, al agua y a la tierra no es un don sobrenatural, sino un ajuste suprafísico.
\vs p065 6:8 Por sí solas, la física y la química no pueden explicar cómo el ser humano evolucionó a partir del protoplasma primigenio de los primeros mares. La capacidad de \bibemph{aprender} ---la memoria y la respuesta diferenciada al medio ambiente--- es facultad de la mente. Las leyes de la física no son sensibles al aprendizaje; son inmutables e invariables. Las reacciones de la química no se modifican por la educación; son uniformes y fiables. Aparte de la presencia del Absoluto Indeterminado, las reacciones eléctricas y químicas son previsibles. Pero la mente puede beneficiarse de la experiencia, puede aprender de los hábitos reactivos del comportamiento en respuesta a la repetición de los estímulos.
\vs p065 6:9 Los organismos preinteligentes reaccionan a los estímulos del medio ambiente, pero aquellos organismos que son reactivos al ministerio de la mente pueden adaptar y actuar sobre el entorno mismo.
\vs p065 6:10 El cerebro físico, con el sistema nervioso al que está vinculado, posee una capacidad innata para responder al ministerio de la mente, al igual que la mente en desarrollo de un ser personal dispone de cierta capacidad innata de receptividad espiritual y contiene, por tanto, los potenciales de progreso y logro espirituales. La evolución intelectual, social, moral y espiritual depende del ministerio mental de los siete espíritus asistentes y de sus colaboradores suprafísicos.
\usection{7. NIVELES EVOLUTIVOS DE LA MENTE}
\vs p065 7:1 Estos siete espíritus asistentes son los versátiles servidores de la mente que ofrecen su ayuda a las existencias inteligentes de inferior orden del universo local. Se asiste a este orden de mente desde la sede del universo local o desde algún mundo relacionado con dicha sede, pero, desde las capitales de los sistemas, se ejerce una influencia directiva sobre este tipo de mente de menor rango.
\vs p065 7:2 En un mundo evolutivo, hay muchísimas cosas que dependen de la labor de estos siete asistentes. Pero son servidores de la mente; no se ocupan de la evolución física, que es el ámbito de los portadores de vida. No obstante, la perfecta integración de estas dotes espirituales con el procedimiento natural y ordenado del régimen innato y en despliegue de los portadores de vida es responsable por la incapacidad del mortal para percibir, en el fenómeno de la mente, nada que no sea la mano de la naturaleza y el desarrollo de los procesos naturales; aunque, ocasionalmente, os sintáis algo desconcertados cuando intentáis explicar todo lo relacionado con las reacciones naturales de la mente en su vinculación con la materia. Y si Urantia operara más en conformidad con los planes primigenios, observaríais incluso menos cosas que captarían vuestra atención en cuanto al fenómeno de la mente.
\vs p065 7:3 Los siete espíritus asistentes son más parecidos a interconexiones que a entidades y, en los mundos ordinarios, están interconectados con la actividad de otros asistentes a través de todo el universo local. En los planetas de experimentación de la vida, sin embargo, están relativamente aislados. Y en Urantia, debido a la naturaleza excepcional de los modelos de vida, los asistentes menores experimentaron mucha más dificultad en contactar con los organismos evolutivos de la que hubiesen tenido en el caso de un tipo de dotación de vida más normalizado.
\vs p065 7:4 Además, en un mundo evolutivo promedio, los siete espíritus asistentes están mucho mejor sincronizados con las etapas progresivas del desarrollo animal de lo que estuvieron en Urantia. Con una sola excepción, en toda su actividad por todo el universo de Nebadón, los asistentes jamás habían experimentado unos inconvenientes tan grandes en contactar con las mentes en evolución de los organismos de Urantia. En este mundo, se desarrollaron muchas formas de fenómenos marginales ---de combinaciones confusas de los tipos de respuesta del organismo no educables y mecánicos y de los no mecánicos y educables---.
\vs p065 7:5 Los siete espíritus asistentes no entran en contacto con esos órdenes de organismos que responden al medio ambiente de manera puramente mecánica. Tal respuesta de los organismos vivos preinteligentes pertenece totalmente a los ámbitos energéticos de los centros de la potencia, de los controladores físicos y de sus colaboradores.
\vs p065 7:6 La adquisición del potencial de la capacidad de aprender de la experiencia supone el inicio de la actuación de los espíritus asistentes, y estos ejercen su labor desde las mentes más humildes de las existencias primitivas e invisibles hasta los tipos superiores en la escala evolutiva de los seres humanos. Los asistentes son la fuente y el modelo del, por lo demás, más o menos misterioso comportamiento y de las insuficientemente comprendidas rápidas reacciones de la mente al entorno material. Durante mucho tiempo, estas influencias fieles y siempre cumplidoras han de llevar adelante la etapa preliminar de su ministerio antes de que la mente animal alcance los niveles humanos de receptividad espiritual.
\vs p065 7:7 Los asistentes obran exclusivamente en la evolución de la mente experiencial hasta el nivel de la sexta fase, el espíritu de adoración. En este nivel ocurre ese solapamiento inevitable de los ministerios ---o fenómeno en el que lo superior desciende para alcanzar y coordinarse con lo más humilde ante la expectativa del logro de niveles avanzados de desarrollo---. Y hay otro ministerio espiritual más que acompaña la acción del séptimo y último asistente: el espíritu de la sabiduría. Durante todo el ministerio del mundo espiritual, las personas nunca experimentan transiciones bruscas en relación a la cooperación espiritual; estos cambios siempre son graduales y recíprocos.
\vs p065 7:8 Siempre han de diferenciarse los ámbitos de las respuestas físicas (electroquímicas) a los estímulos medioambientales de las respuestas mentales, y, a su vez, todas ellas deben reconocerse como fenómenos separados de la actividad espiritual. Los ámbitos de la gravedad física, mental y espiritual son reinos diferenciados de la realidad cósmica, a pesar de su estrecha correlación.
\usection{8. LA EVOLUCIÓN EN EL TIEMPO Y EN EL ESPACIO}
\vs p065 8:1 El tiempo y el espacio están indisociablemente unidos; existe una intrínseca vinculación entre ellos. Las dilaciones del tiempo son inevitables en presencia de ciertas condiciones del espacio.
\vs p065 8:2 Si tardar tanto tiempo en efectuar los cambios evolutivos del desarrollo de la vida os causa perplejidad, os diría que en el terreno temporal no podemos hacer que los procesos de la vida se desplieguen más rápido de lo que permiten las metamorfosis físicas del planeta. Hemos de esperar el desarrollo físico, natural de un planeta; no tenemos ningún tipo de control sobre la evolución geológica. Si las condiciones físicas lo permitiesen, podríamos hacer planes para que la evolución completa de la vida durara mucho menos de un millón de años. Pero todos estamos bajo la jurisdicción de los gobernantes supremos del Paraíso, y el tiempo no existe en el Paraíso.
\vs p065 8:3 El patrón que las personas usan para medir el tiempo es la duración de sus vidas. Por ello todas las criaturas están condicionadas por el tiempo y, consiguientemente, consideran que la evolución es un proceso interminable. Para aquellos como nosotros cuya esperanza de vida no está limitada por la existencia temporal, la evolución no parece ser una tarea tan prolongada. En el Paraíso, donde el tiempo no existe, todas estas cosas están \bibemph{presentes} en la mente de la Infinitud y en los actos de la Eternidad.
\vs p065 8:4 Del mismo modo que la evolución de la mente depende del lento desarrollo de las condiciones físicas y se dilata en el tiempo por estas mismas condiciones, así el progreso espiritual depende de la expansión mental y es indefectiblemente postergado por la retardación intelectual. Pero esto no significa que la evolución espiritual dependa de la educación, de la cultura o de la sabiduría. El alma puede evolucionar a pesar del cultivo de la mente, pero no en la ausencia de la capacidad mental y el deseo ---la elección de la supervivencia y la decisión de lograr crecientemente una mayor perfección--- de hacer la voluntad del Padre celestial. Aunque la supervivencia puede no ser dependiente de la posesión del conocimiento y de la sabiduría, el progreso ciertamente lo es.
\vs p065 8:5 \pc En los laboratorios evolutivos cósmicos, la mente siempre domina sobre la materia, y el espíritu siempre se correlaciona con la mente. Si estas distintas dotes no consiguen sincronizarse y coordinarse, se pueden ocasionar retrasos; pero, si la persona conoce verdaderamente a Dios y desea encontrarle y llegar a ser como él, la supervivencia está asegurada con independencia de las trabas del tiempo. El estado físico puede obstaculizar la mente, y la perversión mental puede demorar el logro espiritual; sin embargo, ninguno de estos impedimentos puede vencer la elección de la voluntad cuando surge enteramente del alma.
\vs p065 8:6 Cuando las condiciones físicas son favorables, pueden producirse desarrollos mentales \bibemph{repentinos,} cuando el estado de la mente es propicio, pueden ocurrir transformaciones espirituales \bibemph{repentinas;} cuando los valores espirituales reciben un apropiado reconocimiento, entonces los contenidos cósmicos se vuelven perceptibles y, cada vez más, el ser personal se libera de las trabas del tiempo y de las limitaciones del espacio.
\vsetoff
\vs p065 8:7 [Auspiciado por un portador de vida de Nebadón residente en Urantia.]
