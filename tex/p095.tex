\upaper{95}{Las enseñanzas de melquisedec en el Levante}
\author{Melquisedec}
\vs p095 0:1 Así como en la India se originaron muchas de las religiones y filosofías de Asia oriental, también el Levante fue la patria de las fes religiosas del mundo occidental. Los misioneros de Salem se repartieron por todo el suroeste de Asia, a través de Palestina, Mesopotamia, Egipto, Irán y Arabia, proclamando la buena nueva del evangelio de Maquiventa Melquisedec en todos los lugares. En algunas de estas tierras, sus enseñanzas dieron sus frutos; en otras, obtuvieron resultados desiguales. A veces, su fracaso se debió a la falta de acierto, otras, a circunstancias más allá de su control.
\usection{1. LA RELIGIÓN DE SALEM EN MESOPOTAMIA}
\vs p095 1:1 Hacia el año 2000 a. C., las religiones de Mesopotamia habían perdido prácticamente las enseñanzas de los setitas y estaban, en gran medida, bajo la influencia de las creencias primitivas de dos grupos de invasores: los beduinos semitas que se habían infiltrado desde el desierto occidental y los jinetes bárbaros que habían venido desde el norte.
\vs p095 1:2 Pero la costumbre de los primeros pueblos adanitas de honrar el séptimo día de la semana nunca desapareció del todo en Mesopotamia. Si bien, durante la era de Melquisedec, el séptimo día se consideraba como el peor día en cuanto a la mala suerte. Estaba plagado de tabúes; era ilícito ir de viaje, cocinar o encender un fuego en tal maligno séptimo día. Los judíos llevaron de vuelta a Palestina muchos de los tabúes mesopotámicos que encontraron respecto a la observancia babilónica del séptimo día, o sábado.
\vs p095 1:3 Aunque los maestros de Salem hicieron mucho por perfeccionar y enaltecer las religiones de Mesopotamia, no lograron que los distintos pueblos reconocieran de forma permanente a un solo Dios. Estas enseñanzas prevalecieron durante más de ciento cincuenta años y luego, paulatinamente, cedieron el paso a las creencias más antiguas en multiplicidad de deidades.
\vs p095 1:4 Los maestros de Salem redujeron considerablemente el número de dioses de Mesopotamia, disminuyendo en cierto momento las deidades principales a siete: Bel, Shamash, Nabu, Anu, Ea, Marduc y Sin. En el apogeo de la nueva enseñanza, exaltaron la supremacía de tres de estos dioses sobre todos los demás; se trataba de la tríada de Babilonia: Bel, Ea y Anu, los dioses de la tierra, el mar y el cielo. En diferentes localidades, aparecieron otras tríadas, todas reminiscencias de las enseñanzas de la Trinidad de los anditas y los sumerios, basadas en la creencia de los salemitas respecto en la insignia de los tres círculos de Melquisedec.
\vs p095 1:5 Los maestros salemitas nunca lograron vencer completamente la popularidad de Ishtar, madre de dioses y el espíritu de la fertilidad sexual. Contribuyeron bastante a depurar la adoración prestada a esta diosa, pero los babilónicos y sus vecinos nunca habían superado por completo estas formas encubiertas de los sistemas de culto sexuales. Se había convertido en una práctica generalizada en toda Mesopotamia que las mujeres se sometieran, al menos una vez en su juventud, a los abrazos de extraños; se pensaba que era un acto de entrega exigido por Ishtar, y se creía que la fertilidad dependía en gran parte de este sacrificio sexual.
\vs p095 1:6 \pc En sus inicios, las enseñanzas de Melquisedec eran sumamente alentadoras hasta que Nabodad, el líder de la escuela de Kish, decidió coordinar un ataque contra las frecuentes prácticas de prostitución en el templo. Pero los misioneros de Salem fracasaron en su intento de llevar a cabo esta reforma social, y sus enseñanzas espirituales y filosóficas más importantes se hundieron en este naufragio.
\vs p095 1:7 A este colapso del evangelio de Salem le siguió de inmediato un gran incremento del culto a Ishtar, ritual que ya había penetrado en Palestina como Astoret, en Egipto como Isis, en Grecia como Afrodita y, en las tribus norteñas, como Astarté. Y fue en relación con este resurgimiento de la adoración a Ishtar cuando los sacerdotes babilónicos volvieron de nuevo a la observación de las estrellas; la astrología experimentó su último gran renacimiento en Mesopotamia, la adivinación se puso de moda y, durante siglos, el sacerdocio se fue progresivamente deteriorando.
\vs p095 1:8 Melquisedec había aleccionado a sus seguidores a que impartieran la enseñanza del Dios único, el Padre y Hacedor de todo, y que predicaran únicamente el evangelio del favor divino solo mediante la fe. Pero, con frecuencia, los maestros de una nueva verdad han cometido el error de intentar abarcar excesivamente, tratando de reemplazar la evolución lenta por la revolución repentina. Los misioneros de Melquisedec en Mesopotamia dictaron al pueblo unos patrones morales demasiado elevados; pretendieron demasiado, y su noble causa cayó derrotada. Se les había encomendado que predicaran un determinado evangelio, que proclamaran la verdad de la realidad del Padre Universal, pero quedaron atrapados en la causa, aparentemente noble, de la reforma de las costumbres y, por ello, su gran misión se desvió de su objetivo y sucumbió ante la frustración y el olvido.
\vs p095 1:9 En una generación, la sede de Salem en Kish llegó a su fin, y el anuncio de la creencia en un solo Dios casi cesó por toda Mesopotamia. Pero quedaron algunos vestigios de las escuelas de Salem. Hubo reducidos grupos de personas dispersos por distintos lugares que continuaron con su creencia en un solo Creador y siguieron luchando contra la idolatría e inmoralidad de los sacerdotes mesopotámicos.
\vs p095 1:10 \pc Fueron los misioneros de Salem del período que siguió al rechazo de sus enseñanzas quienes escribieron muchos de los salmos del Antiguo Testamento, grabándolos en piedra. Los sacerdotes hebreos de tiempos posteriores los encontraron durante su cautiverio, incorporándolos más tarde en su colección de himnos atribuidos a autores judíos. Estos hermosos salmos de Babilonia no se redactaron en los templos de Bel\hyp{}Marduc; fueron obra de los descendientes de los primeros misioneros salemitas, y contrastan notablemente con las colecciones mágicas de los sacerdotes babilónicos. El Libro de Job resume bastante bien las enseñanzas de la escuela de Salem en Kish y en toda Mesopotamia.
\vs p095 1:11 Gran parte de la cultura religiosa de Mesopotamia se abrió camino en la literatura hebrea y en su liturgia por medio de Egipto, a través de los trabajos de Amenemope y Akenatón. Los egipcios preservaron excepcionalmente bien las enseñanzas sobre la obligación social derivada de los primeros mesopotámicos anditas, mayormente perdidas por los babilónicos que ocuparían más tarde el valle del Éufrates.
\usection{2. LA RELIGIÓN EGIPCIA PRIMITIVA}
\vs p095 2:1 En realidad, las enseñanzas primigenias de Melquisedec echaron sus raíces más profundas en Egipto y, desde allí, se expandieron más tarde a Europa. La religión evolutiva del valle del Nilo se vio periódicamente reforzada por la llegada de estirpes mejor dotadas de noditas, adanitas y de pueblos anditas más tardíos procedentes del valle del Éufrates. Ocasionalmente, muchos de los administradores civiles egipcios eran sumerios. Al igual que la India, que en esos días albergaba la mayor mezcla de razas del mundo, Egipto, de forma similar, propició el orden de filosofía religiosa más exhaustivamente mezclado jamás hallado en Urantia y, desde el valle del Nilo, este se difundió a muchos lugares del mundo. Los judíos recibieron gran parte de su idea de la creación del mundo de los babilónicos, si bien, recibieron su concepto de la divina Providencia de los egipcios.
\vs p095 2:2 Las tendencias políticas y morales, en lugar de las filosóficas o religiosas, convirtieron a Egipto en un territorio más favorable que Mesopotamia para las enseñanzas de Salem. Cada líder tribal en Egipto, tras luchar por conseguir el trono, trataba de perpetuar su dinastía proclamando a su dios tribal como la deidad primigenia y creadora de todos los otros dioses. De esa forma, los egipcios se fueron acostumbrando progresivamente a la noción de un supradiós, lo que fue un trampolín para la posterior doctrina de una Deidad universal y creadora. La idea del monoteísmo tuvo en Egipto sus momentos álgidos y bajos durante muchos siglos; la creencia en un solo Dios siempre ganaba terreno, aunque nunca llegaría realmente a predominar sobre los conceptos evolutivos del politeísmo.
\vs p095 2:3 Por mucho tiempo, los pueblos egipcios se habían entregado a la adoración de los dioses de la naturaleza; más concretamente, cada una de las cuarenta tribus por separado tenía su particular dios grupal: una adoraba al toro, otra al león, una tercera al carnero y así sucesivamente. Incluso antes, habían sido tribus totémicas muy parecidas a las tribus amerindias.
\vs p095 2:4 \pc Con el tiempo, los egipcios observaron que los cadáveres colocados en los sepulcros sin ladrillos se conservaban ---quedaban embalsamados--- por la acción de la arena impregnada de sosa, mientras que los enterrados en bóvedas de ladrillos se descomponían. Estas observaciones condujeron a ciertos experimentos que dieron lugar a la práctica posterior de embalsamar a los muertos. Los egipcios creían que la conservación del cuerpo facilitaba su pasaje por la vida futura. Con el fin de que la persona pudiera ser identificada convenientemente en el futuro distante tras la descomposición del cuerpo, colocaban una estatua fúnebre en la tumba junto con el cuerpo y esculpían un retrato sobre el ataúd. La elaboración de estas estatuas fúnebres trajo una gran mejora del arte egipcio.
\vs p095 2:5 Durante siglos, los egipcios ponían su fe en las tumbas como salvaguardia del cuerpo y su consiguiente grata supervivencia tras la muerte. La evolución posterior de las prácticas mágicas, aunque significaban una onerosa carga de vida desde la cuna hasta la tumba los libró muy eficazmente de la religión de las tumbas. Los sacerdotes inscribían sobre el ataúd textos mágicos que pensaban constituían una protección contra el peligro de que “le arrebataran el corazón al hombre en el inframundo”. Pronto, se hizo una variada selección de estos textos y se coleccionaron y conservaron como el Libro de los Muertos. Pero, en el valle del Nilo, el ritual mágico se incorporó pronto a los ámbitos de la conciencia y del carácter hasta un punto pocas veces alcanzado por los rituales de esos días. Y, después, se dependió más de estos ideales éticos y morales para la salvación que de las sofisticadas tumbas.
\vs p095 2:6 \pc Las supersticiones de estos tiempos tienen un buen ejemplo en la creencia generalizada acerca de la eficacia del esputo como agente curativo, una idea que tuvo su origen en Egipto y se difundió desde allí a Arabia y Mesopotamia. En la legendaria batalla entre Horus y Set, el joven dios perdió el ojo, pero una vez que Set fue derrotado, el ojo de Horus quedó restituido por el sabio dios Thot, que escupió sobre la herida y la sanó.
\vs p095 2:7 \pc Los egipcios creyeron por mucho tiempo que las estrellas que parpadeaban en el cielo nocturno representaban la supervivencia de las almas de los muertos honorables; otros supervivientes, según pensaban, eran absorbidos por el sol. Durante cierto período, la veneración solar se convirtió en una especie de adoración de los antepasados. El pasadizo de entrada inclinado de la gran pirámide apuntaba directamente a la estrella polar para que el alma del rey, cuando saliese de la tumba, pudiese ir directo a las constelaciones, estacionarias y establecidas, de las estrellas fijas, la supuesta morada de los reyes.
\vs p095 2:8 Cuando se observaba que los rayos oblicuos del sol penetraban en la tierra a través de un claro entre las nubes, se creía que representaban el descenso de una escalinata celestial por las que el rey y otras almas justas podían ascender. “El rey Pepi ha derramado su fulgor como una escalinata bajo sus pies para ascender hasta su madre”.
\vs p095 2:9 Cuando Melquisedec apareció en la carne, los egipcios tenían una religión muy por encima de la de los pueblos circundantes. Consideraban que un alma desencarnada, si estaba debidamente armada con fórmulas mágicas, podía eludir a los espíritus malignos intermedios y abrirse paso hasta la sala de juicios de Osiris, en la que, si era inocente de “asesinato, robo, falsedad, adulterio, hurto y egoísmo”, sería admitida en los reinos de la dicha. Si se pesaba esta alma en las balanzas y se la hallaba deficiente, sería enviada al infierno, a la Devoradora. Se trataba, relativamente, de una noción avanzada en cuanto a la vida futura en comparación con las creencias de muchos pueblos de los alrededores.
\vs p095 2:10 El concepto de un juicio en el más allá por los pecados cometidos en la tierra durante la vida en la carne se transmitió a la teología hebrea a través de Egipto. La palabra juicio aparece solo una vez en todo el Libro de los Salmos hebreos, y ese salmo concreto lo escribió un egipcio.
\usection{3. EVOLUCIÓN DE LOS CONCEPTOS MORALES}
\vs p095 3:1 Aunque la cultura y la religión de Egipto procedían fundamentalmente de la Mesopotamia andita y fueron, en su mayoría, trasladadas a las civilizaciones venideras a través de los hebreos y los griegos, en muy considerable medida, el idealismo social y ético de los egipcios surgió y se desarrolló en el valle del Nilo de forma puramente evolutiva. A pesar de la importación de tanta verdad y cultura por parte de los anditas, en Egipto evolucionó, de manera estrictamente humana, más cultura moral de la que apareció de modo similarmente natural en ninguna otra zona concreta antes de la encarnación de Miguel.
\vs p095 3:2 La evolución moral no depende de la revelación en su totalidad. Se pueden extraer elevados conceptos morales de la propia experiencia del hombre. El hombre es incluso capaz de desarrollar valores espirituales y adquirir percepción cósmica a partir de sus experiencias personales, porque en él habita un espíritu divino. Estos desarrollos naturales de la conciencia y del carácter también se vieron incrementados por la llegada periódica de maestros de la verdad procedentes, en tiempos ancestrales, del segundo Edén y, con posterioridad, de la sede central de Melquisedec en Salem.
\vs p095 3:3 Miles de años antes de que el evangelio de Salem penetrara en Egipto, sus líderes morales enseñaban la justicia, la equidad y la necesidad de evitar la avaricia. Tres mil años antes de que se escribieran las escrituras hebreas, el lema de los egipcios era: “Firme es el hombre que tiene la rectitud como norma; que camina siguiendo su senda”. Enseñaban dulzura, moderación y discreción. El mensaje de uno de los grandes maestros de esa época fue: “Haz lo correcto y trata a todos con justicia”. La tríada egipcia de este tiempo era Verdad\hyp{}Justicia\hyp{}Rectitud. De todas las religiones puramente humanas de Urantia, ninguna llegó a superar jamás los ideales sociales y la grandeza moral de este antiguo humanismo del valle del Nilo.
\vs p095 3:4 Las doctrinas que sobrevivieron de la religión de Salem florecieron en el fértil terreno de estas ideas éticas e ideales morales evolutivos. Los conceptos del bien y el mal hallaron pronta respuesta en el corazón de un pueblo que creía que “se da la vida a los pacíficos y la muerte a los culpables”. “El pacífico es aquel que hace lo deseado; el culpable es aquel que hace lo detestable”. A lo largo de los siglos, los habitantes del valle del Nilo habían vivido según estas nuevas normas éticas y sociales antes de que llegaran a considerar los conceptos más tardíos de lo correcto y lo equivocado ---de lo bueno y de lo malo---.
\vs p095 3:5 \pc Egipto era un país de índole intelectual y moral, pero no excesivamente espiritual. En seis mil años, solo surgieron cuatro grandes profetas entre los egipcios: Amenemope, a quien siguieron durante algún tiempo; Okhbán, que fue asesinado; Akenatón, a quien aceptaron sin entusiasmo y solamente por una corta generación; y Moisés, a quien se le rechazó. Por otra parte, fueron las circunstancias políticas en lugar de las religiosas las que facilitaron que Abraham y, más tarde, José ejercieran una gran influencia en todo Egipto a favor de las enseñanzas de Salem de un solo Dios. Pero cuando los misioneros de Salem entraron por primera vez en Egipto, encontraron esta cultura evolutiva sumamente ética mezclada con las normas morales modificadas de los inmigrantes mesopotámicos. Estos primeros maestros del valle del Nilo fueron los primeros en proclamar la conciencia como el mandato de Dios, como la voz de la Deidad.
\usection{4. LAS ENSEÑANZAS DE AMENEMOPE}
\vs p095 4:1 En su momento, creció en Egipto un maestro llamado por muchos el “hijo del hombre” y por otros Amenemope. Este vidente ensalzó la conciencia hasta su cima más alta en el arbitraje entre el bien y el mal, enseñó el castigo por el pecado y proclamó la salvación mediante la súplica a la deidad solar.
\vs p095 4:2 Amenemope enseñó que las riquezas y la fortuna eran dones de Dios, y este concepto influyó profundamente en la filosofía hebrea surgida más tarde. Este noble maestro creía que la conciencia de Dios era el factor determinante de cualquier conducta; que cada momento debía vivirse en el reconocimiento de la presencia de Dios y de la responsabilidad hacia él. Las enseñanzas de este sabio se traducirían posteriormente al hebreo, convirtiéndose en el libro sagrado de ese pueblo mucho antes de que el Antiguo Testamento se consignara por escrito. La lección principal de este hombre bueno estaba destinada a su hijo al que instruyó en la rectitud y honestidad de los puestos de confianza gubernamentales, y estos nobles sentimientos de antaño honrarían a cualquier estadista moderno.
\vs p095 4:3 Este sabio del Nilo impartió que “las riquezas hacen alas y vuelan” ---que todas las cosas terrenales son evanescentes---. Su gran oración fue “líbrame del temor”. Instó a todos a que se apartaran de las “palabras de los hombres” y se acercaran a “los actos de Dios”. En esencia, enseñó: el hombre propone pero Dios dispone. Sus enseñanzas, traducidas al hebreo, determinaron la filosofía del Libro de los Proverbios del Antiguo Testamento. Traducidos al griego, estos influyeron en toda la filosofía religiosa helénica posterior. El filósofo alejandrino más tardío, Filón, poseía un ejemplar del Libro de la Sabiduría.
\vs p095 4:4 Amenemope supo conservar la ética evolutiva y la moral revelada y, en sus escritos, las transmitió a los hebreos y a los griegos. No fue el más grande de los maestros religiosos de esta época, pero fue el más influyente, ya que iluminaría el pensamiento futuro de dos pueblos fundamentales en el crecimiento de la civilización occidental: los hebreos, entre los que la fe religiosa occidental llegó a su apogeo y, los griegos, que desarrollaron el pensamiento filosófico puro hasta sus más altos niveles europeos.
\vs p095 4:5 \pc En el Libro de los Proverbios hebreos, los capítulos quince, diecisiete, veinte y desde el capítulo veintidós, verso diecisiete, hasta el capítulo veinticuatro, verso veintidós, se extrajeron casi literalmente del Libro de la Sabiduría de Amenemope. Este maestro escribió el primer salmo del Libro hebreo de los salmos y es la esencia de las enseñanzas de Akenatón.
\usection{5. EL EXCEPCIONAL AKENATÓN}
\vs p095 5:1 Las enseñanzas de Amenemope iban lentamente perdiendo su autoridad sobre la mente egipcia cuando, mediante la influencia de un médico salemita egipcio, una mujer de la familia real abrazó las enseñanzas de Melquisedec. Esta mujer convenció a su hijo Akenatón, faraón de Egipto, para que aceptara la doctrina de Un Solo Dios.
\vs p095 5:2 Desde la desaparición carnal de Melquisedec, ningún ser humano hasta ese momento había poseído un concepto tan increíblemente claro de la religión revelada de Salem como Akenatón. En determinados aspectos, este joven rey egipcio es una de las personas más excepcionales de la historia humana. Durante este tiempo, de creciente depresión espiritual en Mesopotamia, él supo conservar viva la doctrina de El Elyón, de Un Solo Dios, en Egipto, manteniendo así la senda monoteísta filosófica, que resultaría más adelante fundamental para el contexto religioso del entonces futuro ministerio de gracia de Miguel. Y en reconocimiento de este logro, entre otras razones, el niño Jesús fue llevado a Egipto, donde algunos de los sucesores espirituales de Akenatón lo vieron y, en cierto modo, comprendieron algunos aspectos de su misión divina en Urantia.
\vs p095 5:3 Moisés, el personaje más importante entre Melquisedec y Jesús, fue un común obsequio al mundo de la raza hebrea y de la familia real egipcia; y si Akenatón hubiese poseído la versatilidad y la capacidad de Moisés, se habría manifestado como un genio político equiparable a su sorprendente liderazgo religioso, Egipto se habría convertido, entonces, en la gran nación monoteísta de esa era; y si esto hubiese sucedido, habría habido alguna posibilidad de que Jesús hubiese vivido la mayor parte de su vida mortal en Egipto.
\vs p095 5:4 Jamás en la historia llevó un rey a cabo una transformación tan sistemática de toda una nación del politeísmo al monoteísmo como el extraordinario Akenatón. Con la más asombrosa determinación, este joven gobernante rompió con el pasado, cambió su nombre, abandonó su capital, construyó una ciudad enteramente distinta y creó un arte y una literatura nuevos para un pueblo completo. Pero fue demasiado rápido; quiso hacer demasiadas cosas, muchas más de las que podrían permanecer una vez que él ya no estuviese. Por otra parte, no logró proporcionar a su pueblo estabilidad y prosperidad material, todo lo cual provocó una reacción desfavorable contra sus enseñanzas religiosas cuando, más adelante, las mareas de la adversidad y de la opresión azotaron a los egipcios.
\vs p095 5:5 Si este hombre, que poseía una sorprendente clara visión y una extraordinaria determinación, hubiese tenido la sagacidad política de Moisés, habría cambiado por completo la historia de la evolución de la religión y de la revelación de la verdad en el mundo occidental. Durante su vida, fue capaz de frenar la actividad de los sacerdotes, a los que generalmente desprestigió, pero mantuvieron sus sistemas de culto en secreto y entraron en acción en cuanto el joven rey desapareció del poder; y no fueron lentos en relacionar todos los problemas que sobrevinieron a Egipto con la institución del monoteísmo durante su reinado.
\vs p095 5:6 Muy acertadamente, Akenatón trató de establecer el monoteísmo bajo la apariencia del dios solar. Esta decisión de abordar la adoración del Padre Universal absorbiendo a todos los dioses en la adoración del sol se debió al consejo de un médico salemita. Akenatón tomó las doctrinas generalizadas de la entonces existente fe hacia Atón sobre la paternidad y maternidad de la Deidad y creó una religión que reconocía una íntima relación de adoración entre el hombre y Dios.
\vs p095 5:7 Akenatón fue lo suficientemente prudente como para mantener la adoración externa a Atón, el dios\hyp{}sol, mientras que condujo a sus compatriotas a la adoración de un solo Dios, creador de Atón y supremo Padre de todos. Este joven rey\hyp{}maestro fue un escritor prolífico, siendo el autor del tratado que lleva por título “Un solo Dios”, un libro de treinta y un capítulos, que los sacerdotes, cuando regresaron al poder, destruyeron totalmente. Akenatón también escribió ciento treinta y siete himnos, doce de los cuales están preservados actualmente en el Libro de los Salmos del Antiguo Testamento, atribuidos a un autor hebreo.
\vs p095 5:8 \pc La palabra suprema de la religión de Akenatón en la vida diaria era “rectitud” y, rápidamente, difundió la idea de obrar bien hasta albergar la ética internacional y la nacional. Era una generación de sorprendente piedad personal, que se caracterizó por una genuina aspiración entre los hombres y las mujeres más inteligentes por encontrar a Dios y conocerlo. En esos días, ante los ojos de la ley, la posición social o la riqueza no otorgaba privilegios añadidos a los egipcios. La vida familiar en Egipto contribuyó mucho a conservar y a aumentar la cultura moral y sirvió más tarde de inspiración para la magnífica vida familiar de los judíos en Palestina.
\vs p095 5:9 La fatídica debilidad del evangelio de Akenatón consistió en su verdad más grande: la enseñanza de que Atón no era solamente el creador de Egipto sino también de “todo el mundo, hombre y animales, y de todas las tierras extranjeras, incluso de Siria y Kush, además de esta tierra de Egipto. Coloca a todos en su lugar y les proporciona lo que precisan”. Estos conceptos de la Deidad eran elevados y excelsos, pero no eran nacionalistas. Tales sentimientos de internacionalidad en la religión no lograron incrementar el estado anímico del ejército egipcio en el campo de batalla, aunque sí facilitaban armas eficaces a los sacerdotes que podrían emplear contra el joven rey y su nueva religión. Tenía un concepto de la Deidad muy por encima del que tendrían posteriormente los hebreos, pero era demasiado avanzado para servir los propósitos del constructor de una nación.
\vs p095 5:10 \pc Aunque el ideal monoteísta languideció con la muerte de Akenatón, la idea de un solo Dios perduró en la mente de muchos grupos de personas. El yerno de Akenatón decidió secundar a los sacerdotes, volviendo a la adoración de los viejos dioses y cambiando su nombre a Tutankamón. La capital regresó a Tebas y los sacerdotes se lucraron con la tierra, obteniendo con el tiempo la tenencia de un séptimo de todo Egipto; y, enseguida, uno de la misma orden de sacerdotes se atrevió a tomar el trono.
\vs p095 5:11 Pero los sacerdotes no pudieron vencer del todo la ola monoteísta. De forma creciente, se vieron obligados a combinar y a unir sus dioses; la familia de dioses se redujo cada vez más. Akenatón había relacionado el disco candente de los cielos con el Dios creador, y esta idea continuó llameando en el corazón de los hombres, incluso en el de los sacerdotes, mucho después de que el joven reformador hubiese fallecido. El concepto del monoteísmo nunca se extinguió del corazón de los hombres de Egipto, ni del mundo. Perduró incluso hasta la llegada del hijo creador de ese mismo Padre divino, el único Dios, a quien Akenatón había proclamado con tanto fervor para que todo Egipto lo adorara.
\vs p095 5:12 La debilidad de la doctrina de Akenatón radica en el hecho de que planteaba una religión tan avanzada que tan solo los egipcios educados podían entender sus enseñanzas por completo. Las bases de los trabajadores agrícolas nunca comprendieron realmente su evangelio y, por lo tanto, estaban dispuestos a volver con los sacerdotes a la antigua adoración de Isis y de su consorte Osiris, que se suponía había resucitado de forma milagrosa a la muerte cruel de manos de Set, el dios de las tinieblas y del mal.
\vs p095 5:13 La enseñanza de una inmortalidad al alcance de todos los hombres era demasiado adelantada para los egipcios. Solo a los reyes y a los ricos se les prometía la resurrección; de ahí que embalsamaran y preservaran tan cuidadosamente sus cuerpos en las tumbas para el día del juicio. Pero el igualitarismo en la salvación y en la resurrección, tal como lo enseñó Akenatón, prevalecieron con el tiempo, incluso hasta tal punto de que los egipcios llegaran a creer después en la supervivencia de los animales.
\vs p095 5:14 \pc Aunque el esfuerzo de este gobernante egipcio por imponer a su pueblo la adoración de un solo Dios aparentemente fracasó, es necesario hacer constar que las repercusiones de su labor perduraron durante siglos tanto en Palestina como en Grecia, y que Egipto, de este modo, se convirtió en el agente transmisor de la cultura evolutiva combinada del Nilo y de la religión revelada del Éufrates a todos los posteriores pueblos de Occidente.
\vs p095 5:15 La gloria de esta gran era de desarrollo moral y de crecimiento espiritual en el valle del Nilo se desvanecía con rapidez hacia la época en la que comenzaba la vida nacional de los hebreos, y, como consecuencia de su estancia en Egipto, estos beduinos se llevaron consigo un buen número de estas enseñanzas y perpetuaron muchas de las doctrinas de Akenatón en la religión de su raza.
\usection{6. LAS DOCTRINAS DE SALEM EN IRÁN}
\vs p095 6:1 Desde Palestina, algunos de los misioneros de Melquisedec pasaron a través de Mesopotamia a la gran meseta iraní. Durante más de quinientos años, los maestros de Salem hicieron progresos en Irán, y la totalidad de la nación se estaba desplazando hacia la religión de Melquisedec, cuando un cambio de gobernantes precipitó una amarga persecución, que prácticamente puso fin a las enseñanzas monoteístas del sistema de culto de Salem. La doctrina del pacto abrahámico estaba casi extinta en Persia cuando, en aquel gran siglo del renacimiento moral, el siglo VI a. C., apareció Zoroastro, reactivando los rescoldos humeantes del evangelio de Salem.
\vs p095 6:2 Este fundador de una nueva religión era un joven varonil e intrépido, que, en su primera peregrinación a Ur, en Mesopotamia, había conocido las tradiciones de Caligastia y la rebelión de Lucifer ---junto con muchas otras tradiciones--- todo lo cual apeló con fuerzas a su naturaleza religiosa. En consecuencia, a raíz de un sueño que tuvo mientras estaba en Ur, concibió el plan de volver a su hogar en el norte para emprender la remodelación de la religión de su pueblo. Había asimilado la idea hebraica de un Dios de la justicia, el concepto mosaico de la divinidad. Tenía clara en su mente la idea de un Dios supremo, y definió a todos los demás dioses como diablos, relegándolos a engrosar las filas de los demonios de los que había oído en Mesopotamia. Había aprendido de la historia de los siete espíritus mayores, puesto que esta tradición aún perduraba en Ur y, por ello, creó un firmamento de siete dioses supremos con Ahura\hyp{}Mazda a la cabeza. Relacionó a estos dioses de menor rango con la idealización de la Ley Justa, el Buen Pensamiento, el Gobierno Noble, el Carácter Santo, la Salud y la Inmortalidad.
\vs p095 6:3 Y esta nueva religión era una religión de acción ---de trabajo--- no de oraciones ni de rituales. Su Dios era un ser de suprema sabiduría y el benefactor de la civilización; era una filosofía religiosa militante que se atrevía a batallar con el mal, la inacción y el retraso.
\vs p095 6:4 Zoroastro no impartió la enseñanza de adorar al fuego, sino que procuró usar la llama como símbolo del Espíritu puro y sabio y de su predominio universal y supremo. (Es muy cierto que sus seguidores posteriores reverenciaron y adoraron este fuego simbólico). Finalmente, al convertirse un príncipe iraní, esta nueva religión se difundió con la fuerza de la espada. Y Zoroastro murió heroicamente en una batalla por lo que creía que era la “verdad del Señor de la luz”.
\vs p095 6:5 \pc El zoroastrismo es el único credo urantiano que perpetúa las enseñanzas edénicas y dalamatianas sobre los Siete Espíritus Mayores. Aunque no consiguió desarrollar el concepto de la Trinidad, se acercó, de alguna manera, al concepto del Dios Séptuplo. El zoroastrismo primigenio no era puramente dualista; aunque las primeras enseñanzas sí definían el mal como un equivalente temporal de la bondad, este estaba de cierto sumergido eternamente en la realidad última de la bondad. Solo en tiempos más tardíos se abrió camino la creencia de que el bien y el mal competían en condiciones de igualdad.
\vs p095 6:6 Las tradiciones judías acerca del cielo y del infierno y la doctrina de los demonios, tal como constan en las escrituras hebreas, aunque fundamentadas en tradiciones que aún persistían sobre Lucifer y Caligastia, procedían principalmente del zoroastrismo de los tiempos en el que los judíos estaban bajo el dominio político y cultural de los persas. Zoroastro, tal como los egipcios, instruyó sobre el “día del juicio”, pero relacionó este hecho con el fin del mundo.
\vs p095 6:7 Incluso la religión que sucedió al zoroastrismo en Persia estuvo influenciada de manera muy significativa por esta religión. Cuando los sacerdotes iraníes trataron de acabar con las enseñanzas de Zoroastro, retomaron la antigua adoración de Mitra. Y el mitraísmo se difundió por todo el Levante y las regiones mediterráneas; llegando a ser, durante cierto tiempo, contemporáneo del judaísmo y del cristianismo. Por lo tanto, las enseñanzas de Zoroastro llegaron a ejercer su influencia sobre tres grandes religiones: el judaísmo y el cristianismo y, a través de estos, sobre el mahometismo.
\vs p095 6:8 \pc Pero distan mucho las excelsas enseñanzas y los nobles salmos de Zoroastro de la tergiversación moderna de su evangelio por parte de los parsis, con su gran temor a los muertos, sumado a la aceptación de creencias en unos sofismas que Zoroastro nunca accedió a permitir.
\vs p095 6:9 Este gran hombre formó parte de ese singular grupo surgido en el siglo VI a. C. para que la luz de Salem no se extinguiera total y definitivamente, dado que ardía tan tenuemente como para mostrar al hombre en su oscuro mundo la senda de luz que lleva a la vida eterna.
\usection{7. LAS ENSEÑANZAS DE SALEM EN ARABIA}
\vs p095 7:1 Las enseñanzas de Melquisedec en relación a un solo Dios se impartieron en el desierto de Arabia en una fecha relativamente reciente. Y, tal como sucedió en Grecia, los misioneros de Salem fracasaron igualmente en Arabia debido a su deficiente entendimiento de las instrucciones de Maquiventa respecto a la sobrecarga organizativa. Pero no se vieron así obstaculizados por su interpretación de sus advertencias en contra de cualquier intento de difundir el evangelio mediante la fuerza militar o la coacción civil.
\vs p095 7:2 Ni siquiera en China o Roma fracasaron tan rotundamente las enseñanzas de Melquisedec como en esta región desértica, tan próxima al mismo Salem. Mucho después de que la mayoría de los pueblos de Oriente y Occidente se habían convertido en budistas y en cristianos respectivamente, el desierto de Arabia seguía como lo había hecho durante miles de años. Cada tribu adoraba a sus antiguos fetiches y muchas familias de forma individual tenían sus propios dioses domésticos. Durante mucho tiempo, continuó la lucha entre la Ishtar babilónica, el Yahvé hebreo, el Ahura iraní y el Padre cristiano del Señor Jesucristo. Nunca pudo ninguno de ellos reemplazar del todo a los demás.
\vs p095 7:3 Aquí y allá, a lo largo de toda Arabia, había familias y clanes que se aferraron a la difusa idea de un solo Dios. Estos grupos atesoraban las tradiciones de Melquisedec, Abraham, Moisés y Zoroastro. Había numerosos núcleos poblacionales que podían haber respondido al evangelio de Jesús, pero los misioneros cristianos de las tierras desérticas formaban un colectivo austero e inflexible en contraste con los misioneros, más transigentes e innovadores, que actuaban en los países mediterráneos. Si los seguidores de Jesús se hubiesen tomado más en serio su mandato de “ir a todo el mundo y predicar el evangelio”, y hubiesen sido más compasivos en esa predicación, menos estrictos en las exigencias sociales indirectas, ideadas por ellos mismos, se habría recibido con alegría el sencillo evangelio del hijo del carpintero en muchos territorios, Arabia entre ellos.
\vs p095 7:4 A pesar del hecho de que los grandes monoteísmos levantinos no lograron enraizar en Arabia, esta tierra desértica fue capaz de generar una fe que, aunque menos exigente en sus aspectos sociales, era, no obstante, monoteísta.
\vs p095 7:5 Había un único componente compartido de índole tribal, racial o nacional en las creencias primitivas y desorganizadas del desierto, y se trataba del respeto, peculiar y generalizado, que casi todas las tribus árabes estaban dispuestas a rendir, a una piedra fetiche de color negro de cierto templo de la Meca. Este elemento de veneración y de contacto común llevaría más tarde a la creación de la religión islámica. Lo que Yahvé, el espíritu del volcán, fue para los semitas judíos, la piedra de la Kaaba lo sería para sus primos árabes.
\vs p095 7:6 La fuerza del Islam ha consistido en su relato claro y bien definido de Alá como la única y exclusiva Deidad; su debilidad ha sido el uso de la fuerza militar para su propagación, junto con la degradación de la mujer. Pero esta religión se ha mantenido fiel a la idea de la Única Deidad Universal de todos, “que conoce lo invisible y lo visible. Él es el misericordioso y compasivo”. “En verdad Dios colma de bondad a todos los hombres”. “Y cuando estoy enfermo, él es quien me sana”. “Porque cada vez que tres hablen juntos, Dios es el cuarto que está presente”, porque ¿acaso no es él “el primero y el último, y también quien es invisible y está oculto”?
\vsetoff
\vs p095 7:7 [Exposición de un melquisedec de Nebadón.]
