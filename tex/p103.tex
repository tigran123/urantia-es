\upaper{103}{La realidad de la experiencia religiosa}
\author{Melquisedec}
\vs p103 0:1 Cualquier reacción verdaderamente religiosa del hombre se realiza bajo el auspicio del temprano ministerio del asistente de la adoración y con la valoración del asistente de la sabiduría. El hombre se ve dotado primeramente de una supramente a raíz del encauzamiento de su ser personal en la vía circulatoria del espíritu santo, proveniente del espíritu creativo del universo; y, mucho antes de los ministerios de gracia de los hijos divinos o de la dádiva universal de los modeladores, esta influencia espiritual actúa ampliando la visión del hombre respecto a la ética, la religión y la espiritualidad. Tras estos ministerios de los hijos del Paraíso, el espíritu de la verdad, liberado, contribuye poderosamente al engrandecimiento de la capacidad humana para percibir las verdades religiosas. A medida que la evolución avanza en un mundo habitado, se incrementa la participación de los modeladores del pensamiento en el desarrollo del orden más elevado de percepción religiosa humana. El modelador es la ventana cósmica a través de la que la criatura finita puede vislumbrar, mediante la fe, las certezas y divinidades de la Deidad ilimitada, del Padre Universal.
\vs p103 0:2 Las inclinaciones religiosas de las razas humanas son innatas; se manifiestan de forma generalizada y tienen un origen aparentemente natural; las religiones primitivas son siempre evolutivas en sus comienzos. Conforme la experiencia religiosa natural va desarrollándose, las revelaciones periódicas de la verdad intervienen en el trascurso, por otra parte lento, de la evolución planetaria.
\vs p103 0:3 \pc En Urantia existen hoy en día cuatro tipos de religión:
\vs p103 0:4 \li{1.}La religión natural o evolutiva.
\vs p103 0:5 \li{2.}La religión sobrenatural o revelada.
\vs p103 0:6 \li{3.}La religión práctica u ordinaria, una mezcla de religión natural y sobrenatural en distintos grados.
\vs p103 0:7 \li{4.}Las religiones filosóficas, las doctrinas teológicas formuladas por el hombre o concebidas filosóficamente y las religiones creadas por la razón.
\usection{1. LA FILOSOFÍA DE LA RELIGIÓN}
\vs p103 1:1 La unidad de la experiencia religiosa de un grupo social o racial proviene de la naturaleza idéntica de la fracción de Dios que mora en la persona. Es esta parte divina presente en el hombre la que da origen a su interés altruista por el bien de los demás. Pero puesto que la persona es única ---no hay dos mortales idénticos--- se deduce, inevitablemente, que no hay dos seres humanos que puedan interpretar del mismo modo las directrices y los impulsos del espíritu de la divinidad que vive en su mente. Cualquier grupo de mortales puede experimentar la unidad espiritual, pero jamás podrá llegar a alcanzar la uniformidad filosófica. Y esta diversidad interpretativa del pensamiento religioso y de la experiencia se demuestra por el hecho de que los teólogos y los filósofos del siglo XX han formulado más de quinientas definiciones distintas de la religión. En realidad, cada ser humano define la religión en términos de su propia interpretación vivencial de los impulsos divinos que emanan del espíritu de Dios que mora en él y, por ello, debe ser única y completamente diferente de la filosofía religiosa de todos los demás seres humanos.
\vs p103 1:2 Cuando un mortal está plenamente de acuerdo con la filosofía religiosa de algún semejante mortal, ese fenómeno indica que estos dos seres han tenido una \bibemph{experiencia religiosa} similar en lo tocante al contenido de la interpretación filosófica religiosa que comparten.
\vs p103 1:3 Aunque vuestra religión sea una cuestión de experiencia personal, es muy importante que os expongáis al conocimiento del inmenso número de otras experiencias religiosas (las distintas interpretaciones de mortales diferentes) para evitar que vuestra vida religiosa se vuelva egocéntrica ---circunscrita, egoísta e insociable---.
\vs p103 1:4 El racionalismo está equivocado cuando supone que la religión es primeramente una creencia primitiva en algo, a la que sigue luego la búsqueda de los valores. La religión es fundamentalmente una búsqueda de valores y, luego, la elaboración de un sistema de creencias interpretativas. Es mucho más fácil para los hombres ponerse de acuerdo respecto a los valores religiosos ---objetivos--- que respecto a las creencias ---interpretaciones---. Y esto explica por qué las religiones pueden estar de acuerdo en valores y objetivos, mientras manifiestan el confuso fenómeno de mantener una creencia ante cientos de creencias contradictorias ---los credos---. Esto también explica por qué una determinada persona puede mantener su experiencia religiosa frente al abandono o cambio de muchas de sus creencias religiosas. La religión subsiste a pesar de cambios radicales en las creencias religiosas. La teología no genera la religión; es la religión la que genera la filosofía teológica.
\vs p103 1:5 El hecho de que los devotos religiosos hayan creído en tantas cosas falsas no invalida la religión, porque esta se funda en el reconocimiento de los valores y se valida por la fe de la experiencia religiosa personal. La religión, pues, se fundamenta en la experiencia y en el pensamiento religioso. La teología, la filosofía de la religión, es un intento sincero de interpretar esa experiencia. Estas creencias interpretativas pueden ser correctas o erróneas, o una mezcla de verdad y error.
\vs p103 1:6 Llegar a reconocer los valores espirituales es una experiencia que va más allá de lo conceptual. No existe ninguna palabra en idioma humano alguno que pueda emplearse para designar esta “sensación”, “sentimiento”, “intuición” o “vivencia” que hemos decidido llamar conciencia de Dios. El espíritu de Dios que mora en el hombre no es personal ---el modelador es prepersonal--- pero este mentor manifiesta un valor, emana un sabor de divinidad que es personal en el sentido más elevado e infinito. Si Dios no fuese por lo menos personal, no podría ser consciente y, si no fuese consciente, sería entonces infrahumano.
\usection{2. LA RELIGIÓN Y EL SER HUMANO}
\vs p103 2:1 La religión actúa en la mente humana y se ha realizado en la experiencia con anterioridad a su aparición en la conciencia humana. Un niño ha existido unos nueve meses antes de experimentar su \bibemph{nacimiento}. Pero el “nacimiento” de la religión no es repentino; aparece más bien de forma gradual. No obstante, tarde o temprano, hay un “día de nacimiento”. No entraréis en el reino de los cielos a menos que hayáis “nacido de nuevo” ---hayáis nacido del espíritu---. Gran parte de los nacimientos espirituales van acompañados de mucha angustia espiritual y de considerables perturbaciones psicológicas, al igual que muchos nacimientos físicos se caracterizan por tener un “parto difícil” y por otras anomalías del “alumbramiento”. Otros nacimientos espirituales conllevan un crecimiento natural y normal en cuanto al reconocimiento de los valores humanos junto a un fortalecimiento de la experiencia espiritual, aunque no ocurre ningún desarrollo religioso sin un esfuerzo consciente y determinaciones positivas e individuales. La religión jamás es una experiencia pasiva, una actitud negativa. Lo que se denomina “nacimiento de la religión” no está relacionado directamente con las llamadas experiencias de conversión, propias de episodios religiosos, que ocurren más adelante en la vida como resultado de conflictos mentales, represiones emocionales y trastornos temperamentales.
\vs p103 2:2 Pero aquellas personas criadas por sus padres de tal forma que crecieron en la conciencia de ser hijos de un amoroso Padre celestial no deben mirar con recelo a sus semejantes mortales que solo consiguen tal conciencia de la fraternidad con Dios a través de alguna crisis psicológica o de trastornos emocionales.
\vs p103 2:3 El terreno evolutivo de la mente del hombre, en el que germina la semilla de la religión revelada, constituye la naturaleza moral que tan prontamente da origen a la conciencia social. Los primeros impulsos de la naturaleza moral de un niño no tienen nada que ver con el sexo, la culpa o el orgullo personal, sino más bien con el incentivo a la justicia, a la equidad y con el estímulo hacia la bondad ---el ministerio de servir a sus semejantes---. Cuando se cultiva tempranamente este despertar moral, se produce un desarrollo gradual de la vida religiosa, que está relativamente libre de conflictos, trastornos y crisis.
\vs p103 2:4 Todo ser humano padece muy temprano algún tipo de conflicto entre el impulso hacia sus propios intereses y el impulso al altruismo, y muchas veces se logra una primera experiencia de la conciencia de Dios como resultado de la búsqueda de una ayuda sobrehumana para resolver tales conflictos morales.
\vs p103 2:5 La psicología de un niño es positiva por naturaleza, no negativa. Hay tantos mortales negativos porque así se les formó. Cuando se dice que el niño es positivo se hace alusión a sus impulsos morales, a esos poderes de la mente cuya aparición señala la llegada del modelador del pensamiento.
\vs p103 2:6 En ausencia de enseñanzas erróneas, al aparecer gradualmente la conciencia religiosa, la mente del niño normal avanza de forma positiva hacia la rectitud moral y el ministerio social, en lugar de hacerlo de forma negativa, alejándose del pecado y de la culpa. Puede haber o no haber conflicto en el desarrollo de la experiencia religiosa, pero siempre están presentes las inevitables decisiones, el esfuerzo y el ejercicio de la voluntad humana.
\vs p103 2:7 La elección moral suele ir acompañada, en mayor o menor grado, de algún conflicto moral. Y este primer conflicto se desencadena en la mente del niño entre el acicate al egoísmo y los impulsos altruistas. El modelador del pensamiento no deja de lado los valores personales de la motivación interesada, pero obra de hecho mostrando una ligera preferencia por el impulso altruista conducente a la meta de la felicidad humana y a los gozos del reino de los cielos.
\vs p103 2:8 Cuando un ser moral elige ser altruista y enfrentarse al impulso de no serlo, tal hecho constituye una primera experiencia religiosa. Ningún animal puede realizar dicha elección; esta decisión es humana y religiosa. Engloba el hecho de la conciencia de Dios y muestra el incentivo al servicio social, la base de la hermandad de los hombres. Cuando la mente elige, por medio de un acto de la libre voluntad, discernir lo moralmente correcto, tal decisión es una experiencia religiosa.
\vs p103 2:9 Pero antes de que el niño haya evolucionado lo suficiente como para adquirir una aptitud moral y ser, por consiguiente, capaz de elegir el servicio altruista, ya ha desarrollado una naturaleza egocéntrica, fuerte y bien unificada. Y es esta concreta situación la que da origen a la teoría de la pugna entre las naturalezas “superior” e “inferior”, entre el “viejo hombre del pecado” y la “nueva naturaleza” de la gracia. Muy pronto en su vida, el niño normal comienza a aprender que “es más bienaventurado dar que recibir”.
\vs p103 2:10 El hombre tiene tendencia a identificar su afán por atender sus propios intereses con su ego ---con su yo---. Como contraste, se inclina a identificar su voluntad altruista con algún tipo de influencia externa a sí mismo ---Dios---.Y, de hecho, esta consideración es correcta, puesto que todos esos deseos que no les son propios se originan, realmente, en la guía del modelador del pensamiento interior, que es una fracción de Dios. El impulso del mentor espiritual se efectúa en la conciencia humana en forma de incentivo al altruismo, al cuidado de unas criaturas semejantes a él. Por lo menos, esta es la experiencia temprana y fundamental de la mente del niño. Si al crecer el niño no logra unificar su ser personal, el estímulo hacia el altruismo puede desarrollarse tanto como para perjudicar seriamente su propio bien. Una conciencia errada puede llegar a ser la responsable de muchos conflictos, preocupaciones y pesares, y de una interminable infelicidad humana.
\usection{3. LA RELIGIÓN Y LA RAZA HUMANA}
\vs p103 3:1 Aunque la creencia en los espíritus, los sueños y distintas otras supersticiones tuvieron todas un papel importante en el origen evolutivo de las religiones primitivas, no debéis pasar por alto la influencia del espíritu de solidaridad del clan o de la tribu. En las relaciones de grupo se daba la situación social exacta que hacía que se planteara el conflicto egocentrismo\hyp{}altruismo en la naturaleza moral de la mente humana primitiva. A pesar de su creencia en los espíritus, la atención religiosa de los primitivos australianos está aún en el clan. Con el tiempo, estos conceptos religiosos tienden a tomar forma personal, primero, como animales y, luego como seres sobrehumanos o Dios. Incluso razas peor dotadas como los bosquimanos africanos, que ni siquiera llegan a ser totémicos en sus creencias, reconocen la diferencia entre el propio interés y el interés del grupo, una distinción primera entre los valores de lo secular y de lo sagrado. Pero el grupo social no es el motivo de la experiencia religiosa. Al margen del influjo de todas estas tempranas contribuciones a la religión primitiva del hombre, lo cierto es que el verdadero impulso religioso tiene su origen en presencias espirituales genuinas que activan la voluntad hacia la generosidad.
\vs p103 3:2 \pc La religión posterior se prefigura sobre la creencia primitiva en las maravillas y en los misterios naturales, en el impersonal ‘maná’. Pero tarde o temprano, la religión evolutiva exige que la persona realice algún sacrificio personal para el bien de su grupo social, algo que haga a las personas más felices y mejores. En última instancia, la religión está destinada a estar al servicio de Dios y del hombre.
\vs p103 3:3 La religión tiene como objeto cambiar el entorno del hombre; si bien, gran parte de la religión que se practica entre los mortales de hoy en día se ha vuelto incapaz de llevar esto a cabo. Con excesiva frecuencia, el entorno ha prevalecido sobre la religión.
\vs p103 3:4 \pc Recordad que en la religión de todos los tiempos, la experiencia fundamental es el sentimiento de los valores morales y de los contenidos sociales, no el pensamiento sobre los dogmas teológicos o las teorías filosóficas. La religión se desarrolla favorablemente conforme se sustituye el elemento mágico por el concepto de los principios morales.
\vs p103 3:5 El hombre evolucionó desde las supersticiones del maná, la magia, la adoración de la naturaleza, el temor de los espíritus y la adoración de los animales hasta los diversos ceremoniales, mediante los cuales la actitud religiosa de la persona se transformó en las reacciones grupales del clan. Más tarde, estas ceremonias se concentraron y cristalizaron en las creencias tribales y, con el tiempo, estos temores y credos tomaron forma personal como dioses. Pero, en todo este desarrollo religioso, la dimensión moral nunca fue totalmente inexistente. El impulso del Dios interior del hombre fue siempre intenso. Y estos poderosos influjos ---uno humano y el otro divino--- aseguraron la supervivencia de la religión a través de las vicisitudes del tiempo, pese a estar, tan a menudo, en peligro de extinción a causa de miles de tendencias destructivas y antagonismos hostiles.
\usection{4. COMUNIÓN ESPIRITUAL}
\vs p103 4:1 El rasgo distintivo entre una reunión social y otra religiosa es que, en contraste con la secular, la religiosa está impregnada de una atmósfera de \bibemph{comunión}. De esta manera, la asociación humana genera un sentimiento de fraternidad con lo divino, y este es el comienzo de la adoración colectiva. Compartir una comida comunitaria significó el tipo más temprano de comunión social, así que, en las primeras religiones, quedó establecido que los creyentes consumiesen una parte del sacrificio ceremonial. Incluso en el cristianismo, la Santa Cena conserva este modo de comunión. La atmósfera que se crea en la comunión proporciona una estimulante y reconfortante tregua en el conflicto entre el ego que busca su propio interés y el impulso altruista del mentor espiritual interior. Y este es el preámbulo de la verdadera adoración ---la práctica de la presencia de Dios deviene en la gradual aparición de la hermandad de los hombres---.
\vs p103 4:2 Cuando el hombre primitivo sentía que su comunión con Dios se veía interrumpida, recurría a algún tipo de sacrificio en un intento por buscar la expiación, por restablecer la relación de amistad. El hambre y la sed de rectitud llevan al descubrimiento de la verdad, y la verdad engrandece los ideales, y esto crea nuevos problemas para el creyente individual, dado que nuestros ideales tienden a crecer en progresión geométrica, mientras que nuestra capacidad para estar a la altura de estos solo aumenta en progresión aritmética.
\vs p103 4:3 El sentimiento de culpa (no la conciencia del pecado) se manifiesta bien cuando se interrumpe la comunión espiritual o bien cuando disminuyen los ideales morales. Liberarse de tan difícil situación solo es posible cuando se reconoce que los más elevados ideales de la persona no son necesariamente sinónimos de la voluntad de Dios. El hombre no puede pretender estar a la altura de sus más altos ideales, aunque puede ser fiel a su propósito de encontrar a Dios y de llegar a ser como él, cada vez más.
\vs p103 4:4 Jesús eliminó todos los ceremoniales sacrificiales y expiatorios. Abatió el fundamento de toda esta culpa ficticia y de cualquier sentido de aislamiento en el universo cuando declaró que el hombre era hijo de Dios; la relación criatura\hyp{}Creador se asentó sobre la base de una relación hijo\hyp{}padre. Dios se convierte en un Padre amoroso para sus hijos e hijas mortales. Todos los ceremoniales que no sean parte legítima de tal estrecha relación de familia quedan abrogados para siempre.
\vs p103 4:5 Dios Padre trata con el hombre, su hijo, de conformidad, no con su virtud o valía concretas, sino con el reconocimiento de la motivación del hijo ---el propósito y la intención de la criatura---. Esta relación es paterno filial y está motivada por el amor divino.
\usection{5. EL ORIGEN DE LOS IDEALES}
\vs p103 5:1 La mente evolutiva primitiva da origen a un sentimiento de deber social y de obligación moral que se deriva principalmente del temor emocional. El incentivo más positivo hacia el servicio social y el idealismo altruista se deriva del impulso directo del espíritu divino que mora en la mente humana.
\vs p103 5:2 Esta idea\hyp{}ideal de hacer el bien a los demás ---el impulso de negarle algo al ego en beneficio del prójimo--- está en un principio muy circunscrita. El hombre primitivo considera como prójimo solo a aquellos muy cercanos a él, a aquellos que lo tratan a él como a su prójimo; a medida que avanza la civilización religiosa, el prójimo se expande conceptualmente hasta incluir el clan, la tribu, la nación. Luego Jesús amplió el ámbito del prójimo hasta abarcar la humanidad en su totalidad, e incluso deberíamos amar a nuestros enemigos. Y hay algo en el interior de cada ser humano normal que le dice que esta enseñanza es moral ---justa---. Incluso aquellos que practican menos este ideal admiten que es teóricamente correcto.
\vs p103 5:3 Todos los hombres reconocen la moral de este impulso humano universal a ser desinteresados y altruistas. El humanista lo atribuye a la acción natural de la mente material; el devoto religioso reconoce, con más acierto, que el verdadero estímulo hacia la generosidad de la mente mortal es en respuesta a la guía espiritual interior del modelador del pensamiento.
\vs p103 5:4 Pero la interpretación que hace el hombre de este temprano conflicto entre la voluntad del ego y la voluntad del bien del otro no es siempre digna de confianza. Solamente un ser personal bien unificado puede dirimir las múltiples formas de disputa entre las apetencias del ego y la incipiente conciencia social. El yo tiene derechos al igual que el prójimo. Ninguno de los dos tiene la exclusiva sobre la atención y el servicio de la persona. La falta de solución de este problema da origen al tipo más temprano del sentimiento de culpa del hombre.
\vs p103 5:5 La felicidad humana se alcanza únicamente cuando el deseo egocéntrico del yo y el impulso altruista del yo superior (espíritu divino) se coordinan y reconcilian mediante la voluntad unificada del ser personal, integradora y rectora. La mente del hombre evolutivo se enfrenta siempre al complejo problema de mediar en el conflicto entre la expansión natural de los impulsos emocionales y el crecimiento moral de los impulsos altruistas basados en la percepción espiritual ---la genuina reflexión religiosa---.
\vs p103 5:6 El intento de asegurar un bien igualmente al yo y al superior número de otros yos plantea un problema, que no siempre se puede resolver de forma satisfactoria dentro de un contexto espacio\hyp{}temporal. Durante la vida eterna, estos antagonismos pueden encontrar solución, pero en una corta vida humana resulta imposible hacerlo. Jesús aludió a dicha paradoja cuando dijo: “El que salve su vida la perderá, pero el que pierda su vida en nombre del reino, la hallará”.
\vs p103 5:7 \pc La búsqueda del ideal ---el empeño por semejarse a Dios--- es un esfuerzo continuo antes y después de la muerte. En lo esencial, la vida tras la muerte no difiere de la existencia mortal. Todo lo bueno que hagamos en esta vida contribuye de manera directa al perfeccionamiento de la vida futura. La religión real no promueve la indolencia moral ni la pereza espiritual, alentando la vana esperanza de adquirir todas las virtudes de un carácter noble por el hecho de cruzar las puertas de la muerte natural. La verdadera religión no subestima el esfuerzo del hombre por progresar mientras está en posesión de la vida mortal. Cualquier logro del ser humano contribuye directamente al enriquecimiento de las primeras etapas de su disfrute de la supervivencia inmortal.
\vs p103 5:8 \pc Resulta fatal para el idealismo del hombre cuando se le enseña que todos los impulsos altruistas son el mero desarrollo de sus instintos gregarios naturales. Sin embargo, se ennoblece y revitaliza poderosamente cuando aprende que estos impulsos superiores de su alma emanan de fuerzas espirituales que moran en su mente mortal.
\vs p103 5:9 Cuando el hombre toma plena conciencia de que dentro de él vive y obra vigorosamente algo eterno y divino, se eleva por encima y más allá de sí mismo. Y es así como, en el origen sobrehumano de nuestros ideales, una fe viva valida nuestra confianza de que somos los hijos de Dios y hace verdaderas nuestras convicciones altruistas ---los sentimientos de la hermandad del hombre---.
\vs p103 5:10 En su ámbito espiritual, el hombre ciertamente goza de libre voluntad. El hombre mortal no es un esclavo indefenso de la soberanía inflexible de un Dios todopoderoso ni la víctima de la desesperanzadora fatalidad de un determinismo cósmico mecanicista. El hombre es realmente el arquitecto de su propio destino eterno.
\vs p103 5:11 \pc Pero el hombre no se salva ni ennoblece sintiéndose presionado. El crecimiento espiritual brota desde dentro del alma evolutiva. La presión puede deformar la persona, pero nunca estimular el crecimiento. Incluso la presión educativa es útil, aunque solo de forma negativa, por cuanto que ayuda a prevenir experiencias desastrosas. El crecimiento espiritual es mucho mayor cuando todas las presiones externas son mínimas. “Donde está el Espíritu del Señor, allí hay libertad”. El hombre se desarrolla mejor cuando las presiones del hogar, la comunidad, la Iglesia y el Estado son menores. Si bien, esto no debe interpretarse en el sentido de que, en una sociedad en crecimiento, no haya lugar para el hogar, las instituciones sociales, la Iglesia y el Estado.
\vs p103 5:12 Cuando algún miembro de un grupo social religioso haya cumplido con los requisitos que este le impone, se le debería animar a gozar de libertad religiosa para manifestar plenamente su propia interpretación personal de las verdades de las creencias religiosas y de los hechos de la experiencia religiosa. La continuidad de un grupo religioso depende de su unidad espiritual, no de su uniformidad teológica. En un grupo religioso se debería poder gozar de la libertad de pensar libremente, sin tener que convertirse en “librepensadores”. Hay grandes esperanzas para cualquier Iglesia que adore al Dios vivo, valide la hermandad de los hombres y se atreva a aliviar a sus miembros de la presión que el credo ejerce sobre ellos.
\usection{6. ARMONIZACIÓN FILOSÓFICA}
\vs p103 6:1 La teología es el estudio de los actos y reacciones del espíritu humano; nunca puede convertirse en ciencia dado que siempre ha de combinarse, en mayor o menor grado, con la psicología para su expresión personal y, con la filosofía, para su representación sistemática. La teología es siempre el estudio de \bibemph{vuestra} religión; la psicología, el de la religión del otro.
\vs p103 6:2 \pc Cuando el hombre aborda el estudio y el examen de su universo desde \bibemph{fuera,} crea las distintas ciencias físicas; cuando aborda la investigación de sí mismo y del universo desde \bibemph{dentro,} da origen a la teología y a la metafísica. El posterior arte de la filosofía se desarrolla cuando se intentan armonizar las múltiples discrepancias que están llamadas a aparecer en un principio entre los hallazgos y las enseñanzas de estas dos vías diametralmente opuestas de acercarse al universo de las cosas y de los seres.
\vs p103 6:3 La religión está relacionada con la perspectiva espiritual, con la conciencia de la \bibemph{interioridad} de la experiencia humana. La naturaleza espiritual del hombre le brinda la oportunidad de darle la vuelta al universo desde fuera hacia dentro. Por consiguiente, es cierto que, vista exclusivamente desde esta interioridad de la experiencia personal, toda creación parece ser espiritual por su propia naturaleza.
\vs p103 6:4 Cuando el hombre examina analíticamente el universo a través de su dote material de sus sentidos físicos y de la percepción mental correspondiente, el cosmos parece ser mecánico y compuesto de materia\hyp{}energía. Esta forma de estudiar la realidad consiste en darle la vuelta al universo desde dentro hacia fuera.
\vs p103 6:5 \pc No se puede elaborar un concepto filosófico lógico y coherente del universo ni sobre los postulados del materialismo ni sobre los del espiritualismo, puesto que ambos sistemas de pensamiento, cuando se aplican de forma generalizada, se ven compelidos a ver el cosmos de forma distorsionada: el primero, entrando en contacto con un universo vuelto desde dentro hacia fuera; el segundo, percibiendo la naturaleza del universo desde fuera hacia dentro. En consecuencia, ni la ciencia ni la religión podrán jamás en sí mismas, por sí solas, adquirir una adecuada comprensión de las verdades universales y de su relación sin la guía de la filosofía humana y la dilucidación de la revelación divina.
\vs p103 6:6 Para su expresión y autorrealización, el espíritu interior del hombre debe depender siempre del mecanismo y de los procedimientos de la mente. De igual manera, la experiencia externa del hombre respecto a la realidad material debe basarse en la conciencia mental de la persona que la experimenta. Por lo tanto, las experiencias humanas espirituales y materiales ---internas y externas--- están invariablemente correlacionadas con la acción de la mente y condicionadas, en cuanto a su reconocimiento consciente, por la actividad de la mente. El hombre experimenta la materia en su mente; experimenta la realidad espiritual en el alma, pero se hace consciente de esta experiencia en la mente. El intelecto armoniza y, constantemente, condiciona y determina la totalidad de la experiencia humana. Tanto las cosas\hyp{}energía como los valores espirituales están influenciados por la propia interpretación que se hace de ellos a través de los medios mentales de los que hace uso la conciencia.
\vs p103 6:7 Vuestra dificultad para alcanzar una coordinación más armoniosa entre la ciencia y la religión se debe a vuestro total desconocimiento del ámbito intermedio del mundo morontial de cosas y seres. El universo local consta de tres grados, o niveles, de manifestación de la realidad: la materia, la morontia y el espíritu. El ángulo de aproximación morontial borra cualquier divergencia existente entre los hallazgos de las ciencias físicas y la acción del espíritu de la religión. La razón es el método de comprensión de las ciencias; la fe es el método de percepción de la religión; la mota es el método característico del nivel morontial. La mota es una sensibilidad a la realidad supramaterial que empieza a compensar el crecimiento incompleto, teniendo como su sustancia el conocimiento\hyp{}razón y, como su esencia, la fe\hyp{}percepción. La mota es una conciliación suprafilosófica de la apreciación de la realidad divergente, inalcanzable por los seres personales materiales; está basada, en parte, en la experiencia de haber sobrevivido a la vida material en la carne. Pero hay muchos mortales que reconocen la conveniencia de tener algún método que reconcilie los ámbitos científicos y religiosos, inmensamente separados; y la metafísica es el resultado del infructuoso intento del hombre por salvar esta brecha, ampliamente reconocida. Pero la metafísica humana ha demostrado ser más confusa que lúcida. La metafísica representa el esfuerzo del hombre, bien intencionado aunque vano por suplir la ausencia de la mota morontial.
\vs p103 6:8 \pc La metafísica ha demostrado ser un fracaso; la mota no puede percibirse por el hombre. La revelación es el único modo de compensar la ausencia en el mundo material de la sensibilidad hacia la verdad de la mota. La revelación clarifica de forma fehaciente el enmarañamiento de la metafísica, que es creación de la razón en una esfera evolutiva.
\vs p103 6:9 La ciencia es el intento del hombre por estudiar su entorno físico, el mundo de la energía\hyp{}materia; la religión constituye la experiencia del hombre con el cosmos de los valores espirituales; la filosofía se ha desarrollado gracias al empeño de la mente del hombre por organizar y correlacionar los hallazgos de estos conceptos, tan distantes entre ellos, en algo parecido a una actitud razonable y unificada hacia el cosmos. La filosofía, clarificada por la revelación, obra aceptablemente en ausencia de la mota y en presencia del colapso y fracaso de la razón del hombre por encontrar un sustituto de la mota: la metafísica
\vs p103 6:10 \pc El hombre primitivo no diferenciaba entre el nivel de la energía y el del espíritu. Fue la raza violeta y sus sucesores anditas los que, en primer lugar, trataron de separar lo matemático de lo volitivo. El hombre civilizado ha seguido crecientemente los pasos de los primeros griegos y sumerios, que distinguían entre lo inanimado y lo animado. Y, a medida que la civilización avanza, la filosofía tendrá que tender puentes sobre la brecha cada vez más grande que separa el concepto del espíritu del concepto de la energía. Pero, en el tiempo del espacio, estas divergencias se hacen una en el Supremo.
\vs p103 6:11 \pc La ciencia ha de basarse siempre en la razón, aunque la imaginación y las suposiciones resulten de utilidad en la extensión de sus límites. La religión depende por siempre de la fe, pese a que la razón sea una influencia estabilizadora y una conveniente servidora. Y siempre ha habido, y por siempre habrá, interpretaciones erróneas de los fenómenos tanto del mundo natural como del mundo espiritual, de las llamadas equivocadamente ciencias y religiones.
\vs p103 6:12 Es debido a su incompleta comprensión de la ciencia, a su débil dominio de la religión y a sus intentos fallidos en la metafísica, por lo que el hombre ha tratado de formular sus enunciados filosóficos. Y el hombre moderno podría elaborar, de hecho, una filosofía digna y atractiva de sí mismo y de su universo si no fuera por el colapso de su nexo metafísico, importantísimo e indispensable, entre los mundos de la materia y del espíritu: el fracaso de la metafísica para salvar la brecha morontial entre lo físico y lo espiritual. El hombre mortal carece de conceptos mentales y materiales de rango morontial, y la \bibemph{revelación} representa el único modo de suplir esta insuficiencia de datos conceptuales, que tan imperiosamente necesita a fin de definir una filosofía lógica del universo y llegar a un entendimiento satisfactorio de su lugar, seguro y estable, en ese universo.
\vs p103 6:13 La revelación es la única esperanza que tiene el hombre evolutivo de salvar tal brecha morontial. La fe y la razón, sin ayuda de la mota, no pueden concebir ni elaborar un universo lógico. Sin la percepción de la mota, el hombre mortal no puede discernir la bondad, el amor y la verdad en los fenómenos del mundo material.
\vs p103 6:14 Cuando la filosofía del hombre se inclina fuertemente hacia el mundo de la materia, se vuelve racionalista o \bibemph{naturalista;} cuando lo hace especialmente hacia el nivel espiritual, se convierte en \bibemph{idealista} o incluso mística; cuando es tan desafortunada como para apoyarse en la metafísica, indefectiblemente se vuelve \bibemph{escéptica,} confusa. En eras pasadas, la mayoría del conocimiento del hombre y de sus valoraciones intelectuales se incluía dentro de una de estas tres distorsiones perceptivas. La filosofía no debería proyectar sus interpretaciones de la realidad siguiendo la manera lineal de la lógica; nunca debe dejar de considerar la simetría elíptica de la realidad y la curvatura esencial de todos los conceptos relacionales.
\vs p103 6:15 El grado máximo de filosofía asequible para el hombre mortal debe basarse lógicamente en la razón de la ciencia, en la fe de la religión y en la percepción de la verdad facilitada por la revelación. Por medio de esta unión, el hombre puede compensar, de algún modo, su falta de éxito al tratar de desarrollar una metafísica adecuada y su incapacidad de comprender la mota de la morontia.
\usection{7. CIENCIA Y RELIGIÓN}
\vs p103 7:1 La ciencia se sustenta en la razón, la religión en la fe. La fe, aunque no se base en la razón, es razonable; aunque independiente de la lógica, está no obstante alentada por una lógica plausible. La fe no puede nutrirse ni siquiera de una filosofía ideal; de hecho, es, con la ciencia, la fuente misma de tal filosofía. La fe, la percepción religiosa humana, solamente puede verdaderamente instruirse mediante la revelación, tan solo puede indudablemente elevarse por medio de la experiencia humana personal junto con la presencia espiritual del modelador de Dios, que es espíritu.
\vs p103 7:2 \pc La verdadera salvación consiste en la evolución divina de la mente mortal desde su identificación con la materia, pasando por la mediación de los ámbitos morontiales, hasta alcanzar un estatus elevado de identificación espiritual en el universo. Y, como el instinto intuitivo material precede a la aparición del conocimiento razonado en la evolución terrestre, igualmente la manifestación de la percepción intuitiva espiritual augura la consiguiente aparición de la morontia y de la razón y de la experiencia espirituales en el supremo plan de la evolución celestial, cuyo objeto consiste en transmutar los potenciales del hombre temporal en la realidad y divinidad del hombre eterno, de un finalizador del Paraíso.
\vs p103 7:3 Pero a medida que el hombre, en su ascenso, se extiende hacia dentro y en dirección al Paraíso buscando la experiencia de Dios, asimismo se extiende hacia fuera y en dirección al espacio buscando entender el cosmos material a niveles energéticos. El entendimiento del hombre del progreso de la ciencia no se limita a su vida terrena. Durante su ascenso en el universo y en el suprauniverso, su formación consistirá, en gran proporción, en el estudio de la transmutación de la energía y de la metamorfosis de la materia. Dios es espíritu, pero la Deidad es unidad, y la unidad de la Deidad no solo abarca los valores espirituales del Padre Universal y del Hijo Eterno, sino que también es consciente de los hechos energéticos del Rector Universal y de la Isla del Paraíso, mientras que estas dos facetas de la realidad universal están perfectamente correlacionadas en las relaciones mentales del Actor Conjunto y unificadas, en el nivel finito, en la Deidad emergente del Ser Supremo.
\vs p103 7:4 \pc La unión de la actitud científica y de la percepción religiosa por medio de la filosofía experiencial forma parte de la experiencia del hombre en su largo ascenso al Paraíso. Las estimaciones de las matemáticas y las certezas de la percepción siempre precisarán de la labor armonizadora de la lógica mental en todos los niveles experienciales inferiores a la consecución máxima del Supremo.
\vs p103 7:5 Pero la lógica nunca tendrá éxito en armonizar los hallazgos de la ciencia y las percepciones de la religión a menos que tanto los aspectos científicos como los religiosos de la persona se rijan por la verdad, estén sinceramente incentivados a seguirla adondequiera que esta lleve con independencia de las conclusiones a las que pueda llegar.
\vs p103 7:6 La lógica es el modo de proceder de la filosofía, su método de expresión. Dentro del ámbito de la verdadera ciencia, la razón es siempre susceptible a una lógica auténtica; dentro del ámbito de la verdadera religión, la fe es siempre lógica partiendo de la base de una perspectiva interiorista, aunque dicha fe pueda parecer bastante infundada partiendo del examen interno del enfoque científico. Desde fuera, mirando hacia dentro, el universo quizás parezca material; desde dentro, mirando hacia fuera, el mismo universo parece ser completamente espiritual. La razón surge de la conciencia material; la fe lo hace de la conciencia espiritual. Si bien, por mediación de una filosofía reforzada por la revelación, la lógica puede confirmar tanto la visión interior como la exterior, con lo que lleva a efecto la compaginación de la ciencia y de la religión. Por consiguiente, mediante su común contacto con la lógica de la filosofía, la ciencia y la religión se harán crecientemente más tolerantes la una con la otra, y cada vez menos escépticas.
\vs p103 7:7 Lo que necesitan, en su desarrollo, tanto la ciencia como la religión es una autocrítica más profunda y valiente, una mayor conciencia de la incompletitud de su estado evolutivo. Los maestros de la ciencia y de la religión por igual están a menudo demasiado seguros de sí mismos y son dogmáticos. La ciencia y la religión únicamente pueden hacer autocrítica de sus \bibemph{hechos}. A partir del momento en el que se alejan del estadio de los hechos, la razón abdica o bien degenera rápidamente asociándose a una falsa lógica.
\vs p103 7:8 \pc La verdad ---la comprensión de las relaciones cósmicas, de los hechos del universo y de los valores espirituales--- se puede alcanzar idóneamente por medio del ministerio del espíritu de la verdad y juzgarse de forma más conveniente mediante la \bibemph{revelación}. Pero la revelación no da origen ni a la ciencia ni a la religión; su tarea es coordinar a las dos con la verdad de la realidad. Siempre, a falta de la revelación o ante la incapacidad de aceptarla o comprenderla, el hombre mortal ha recurrido al gesto inútil de la metafísica, el único sucedáneo humano de la revelación de la verdad o de la mota del ser personal morontial.
\vs p103 7:9 La ciencia del mundo material permite al hombre ejercer su influencia y, hasta cierto punto, su dominio sobre el entorno físico. La religión de la experiencia espiritual es el origen del impulso a la fraternidad, que posibilita a los hombres convivir en las complejidades de la civilización de una era científica. La metafísica, pero más ciertamente la revelación, favorece un espacio de encuentro común para los descubrimientos de la ciencia y de la religión, y hace posible el intento humano de correlacionar lógicamente estos ámbitos separados del pensamiento, aunque interdependientes, en una filosofía bien equilibrada que contenga consistencia científica y certeza religiosa.
\vs p103 7:10 \pc En el estado mortal, nada se puede probar de forma absoluta; tanto la ciencia como la religión se fundamentan en premisas. En el nivel morontial, los postulados de ambas son susceptibles de parcial comprobación por la lógica de la mota. En el estatus máximo del nivel espiritual, la necesidad de alguna prueba de carácter finito se desvanece paulatinamente ante la verdadera experiencia de la realidad; pero, incluso entonces, hay muchas cuestiones, más allá de la finitud, que continúan sin probarse.
\vs p103 7:11 Todas las categorías del pensamiento humano se basan en determinadas premisas que se aceptan, sin comprobarse, mediante una sensibilidad constitutiva de la realidad de la dote mental del hombre. La ciencia emprende su alardeada andadura en el razonamiento \bibemph{suponiendo} la realidad de tres cosas: la materia, el movimiento y la vida. La religión emprende su trayectoria con la premisa de la validez de tres cosas: la mente, el espíritu y el universo ---el Ser Supremo---.
\vs p103 7:12 La ciencia se convierte en el ámbito intelectual de las matemáticas, de la energía y de la materia existentes en el tiempo y el espacio. La religión no solo asume la responsabilidad de ocuparse del espíritu finito y temporal, sino también del espíritu de la eternidad y de la supremacía. Solo por medio de una amplia experiencia en la mota, pueden estas dos perspectivas extremas del universo ofrecer interpretaciones análogas sobre sus orígenes, funciones, relaciones, realidades y destinos. La máxima armonización de la divergencia energía\hyp{}espíritu está en la vía circulatoria de los siete espíritus mayores; su primera unificación, en la Deidad del Supremo; la completud de su unidad, en la infinitud de la Primera Fuente y Centro, en el YO SOY.
\vs p103 7:13 \pc La \bibemph{razón} es el acto de reconocer las conclusiones de la conciencia respecto a la experiencia en y con el mundo físico de la energía y de la materia. La \bibemph{fe} es el acto de reconocer la validez de la conciencia espiritual ---algo que no es susceptible de otra demostración humana---. La \bibemph{lógica} es el avance y síntesis de la búsqueda de la verdad de la unidad de la fe y de la razón, y se fundamenta en la dote característica de la mente de los seres mortales: el reconocimiento innato de las cosas, de los contenidos y de los valores.
\vs p103 7:14 \pc En la presencia del modelador del pensamiento, se halla la verdadera evidencia de la realidad espiritual, pero la validez de esta presencia no es constatable para el mundo exterior, sino solo para aquel que experimenta tal inhabitación de Dios. La conciencia del modelador se basa en la recepción intelectual de la verdad, en la percepción supramental de la bondad y en la motivación de la persona al amor.
\vs p103 7:15 La ciencia descubre el mundo material, la religión lo evalúa y la filosofía trata de interpretar sus contenidos, mientras armoniza la perspectiva material y la científica con el concepto religioso espiritual. Pero la historia es un ámbito en el que la ciencia y la religión tal vez nunca lleguen a estar totalmente de acuerdo.
\usection{8. FILOSOFÍA Y RELIGIÓN}
\vs p103 8:1 Aunque, mediante la razón y la lógica, la ciencia y la filosofía puedan determinar la probabilidad de la existencia de Dios, solo la experiencia religiosa personal del hombre guiado por el espíritu puede afirmar la certeza de tal Deidad suprema y personal. Por medio de esta encarnación de la verdad viva, la hipótesis filosófica de la probabilidad de que Dios exista se vuelve una realidad religiosa.
\vs p103 8:2 La confusión acerca de la experiencia de la certeza de Dios surge a raíz de interpretaciones disimilares y relacionales de tal experiencia por parte de personas distintas y de las diferentes razas humanas. La experiencia de Dios puede ser enteramente válida, pero el enunciado \bibemph{sobre} Dios, siendo intelectual y filosófico, es dispar y, con frecuencia, desconcertantemente engañoso.
\vs p103 8:3 Un hombre bueno y noble puede estar completamente enamorado de su esposa pero ser totalmente incapaz de superar satisfactoriamente un examen escrito sobre la psicología del amor marital. Otro hombre, sintiendo poco o ningún amor hacia su esposa, puede superar ese mismo examen más aceptablemente. La imperfección de la idea del amante respecto a la verdadera naturaleza del ser querido no invalida en lo más mínimo ni la realidad ni la sinceridad de su amor.
\vs p103 8:4 \pc Si verdaderamente creéis en Dios ---por la fe lo conocéis y lo amáis--- no permitáis de ningún modo que la realidad de tal experiencia se vea reducida o se resienta por las dudosas implicaciones de la ciencia, por las objeciones de la lógica, por los postulados de la filosofía o por las hábiles recomendaciones de almas bien intencionadas que quieren crear una religión sin Dios.
\vs p103 8:5 La certidumbre de la persona religiosa que conoce a Dios no debe verse perturbada por la incertidumbre del materialista incrédulo; más bien, es la incertidumbre del no creyente la que debe verse rigurosamente cuestionada por la profunda fe y la inquebrantable certeza del creyente que ha tenido esa experiencia.
\vs p103 8:6 \pc Para aportar el mayor de los servicios a la ciencia y la religión, la filosofía debe evitar los dos extremos del materialismo y del panteísmo. Únicamente una filosofía que reconozca la realidad del ser personal ---permanencia en presencia de cambio--- puede ser de valor moral para el hombre, puede hacer de enlace entre las teorías de la ciencia material y de la religión espiritual. La revelación compensa las debilidades de la filosofía evolutiva.
\usection{9. LA ESENCIA DE LA RELIGIÓN}
\vs p103 9:1 La teología aborda el contenido intelectual de la religión; la metafísica (la revelación), los aspectos filosóficos. La experiencia religiosa \bibemph{es} el contenido espiritual de la religión. A pesar de las veleidades mitológicas y de las quimeras psicológicas del contenido intelectual de la religión, de las erróneas premisas de la metafísica y de las formas de autoengaño, de las distorsiones políticas y las perversiones socioeconómicas del contenido filosófico de la religión, la experiencia espiritual de la religión personal sigue siendo genuina y válida.
\vs p103 9:2 La religión tiene que ver con el sentimiento, la acción y la vida, y no meramente con el pensamiento. El pensamiento guarda una estrecha relación con la vida material y debería regirse principalmente, aunque no en su totalidad, por la razón y los hechos de la ciencia y, en su expansión no material hacia los reinos del espíritu, por la verdad. Sin perjuicio de lo ilusoria y errónea que sea la teología de cada cual, la propia religión puede ser totalmente auténtica y perpetuamente verdadera.
\vs p103 9:3 En su forma original, el budismo es una de las más excelentes religiones no teístas que haya surgido a lo largo de la historia evolutiva de Urantia, aunque, conforme esta creencia se desarrolló, no permaneció sin Dios. La religión sin fe es una contradicción; sin Dios es una inconsistencia filosófica y un despropósito intelectual.
\vs p103 9:4 La filiación mágica y mitológica de la religión natural no invalida la realidad y la verdad de las posteriores religiones reveladas ni del supremo evangelio salvífico de la religión de Jesús. La vida de Jesús y sus enseñanzas liberaron finalmente a la religión de las supersticiones de la magia, de las quimeras de la mitología y de la servidumbre al dogmatismo tradicional. Si bien, esta magia y mitología primitivas prepararon con gran eficacia el camino para una religión posterior de mayor excelencia, al asumir la existencia y la realidad de valores y seres supramateriales.
\vs p103 9:5 Aunque sea un fenómeno subjetivo puramente espiritual, la experiencia religiosa entraña una actitud positiva y de fe viva hacia los reinos más elevados de la realidad objetiva del universo. El ideal de la filosofía religiosa consiste en ese grado de fe\hyp{}confianza que puede llevar al hombre a depender incondicionalmente del amor absoluto del Padre Infinito del universo de los universos. Esta genuina experiencia religiosa trasciende con creces la objetivación filosófica del deseo idealista; de hecho, da por sentado la salvación y solo se ocupa de aprender y hacer la voluntad del Padre del Paraíso. Las características distintivas de semejante religión son: la fe en una Deidad suprema, la esperanza de la supervivencia eterna y el amor, en particular al prójimo.
\vs p103 9:6 \pc Cuando la teología rige la religión, la religión muere; se hace doctrina en lugar de convertirse en vida. La misión de la teología consiste sencillamente en facilitar la autoconciencia de la experiencia personal espiritual. La teología supone un intento de índole religiosa por definir, clarificar, exponer y justificar las afirmaciones de la religión en cuanto a su carácter vivencial, las cuales, en último término, solo pueden ser validadas por la fe viva. En la filosofía superior del universo, la sabiduría, al igual que la razón, se coaligan con la fe. La razón, la sabiduría y la fe son los más altos logros del hombre. La razón incorpora al hombre al mundo de los hechos, de las cosas; la sabiduría, al mundo de la verdad, de la relación; la fe lo inicia en el mundo de la divinidad, de la experiencia espiritual.
\vs p103 9:7 La fe con gran voluntariedad porta consigo a la razón hasta donde esta puede llegar y, luego, continúa con la sabiduría, guiándola hasta el máximo de su límite filosófico; entonces, se atreve a emprender un viaje por un universo sin límite ni final con la única compañía de la VERDAD.
\vs p103 9:8 \pc La ciencia (el conocimiento) se fundamenta en la premisa intrínseca (el espíritu asistente) de que la razón es válida, de que el universo puede comprenderse. La filosofía (la comprensión correlacionada de ambos) se fundamenta en la premisa intrínseca (el espíritu de la sabiduría) de que la sabiduría es válida, de que el universo material puede coordinarse con el espiritual. La religión (la verdad de la experiencia espiritual personal) se fundamenta en la premisa intrínseca (el modelador del pensamiento) de que la fe es válida, de que se puede conocer y llegar a Dios.
\vs p103 9:9 La toma plena de conciencia de la realidad de la vida mortal consiste en la gradual disposición a creer en estas premisas de la razón, la sabiduría y la fe. Una vida así está motivada por la verdad y regida por el amor; y estos son los ideales de la realidad cósmica objetiva, cuya existencia no puede demostrarse de forma material.
\vs p103 9:10 Cuando la razón alguna vez distingue entre el bien y el mal, manifiesta sabiduría; cuando la sabiduría elige entre el bien y el mal, entre la verdad y el error, demuestra que está guiada por el espíritu. Y, por consiguiente, la mente, el alma y el espíritu obran por siempre en estrecha unión y de forma correlacionada. La razón trata del conocimiento fáctico; la sabiduría, de la filosofía y de la revelación; la fe, de la experiencia espiritual viva. A través de la verdad, el hombre alcanza la belleza y, por medio del amor espiritual, asciende a la bondad.
\vs p103 9:11 La fe lleva a conocer a Dios, y no a un mero sentimiento místico de la presencia divina. La fe no debe verse excesivamente condicionada por sus efectos emocionales. La verdadera religión consiste en el hecho de creer y en conocer al igual que en la satisfacción de los sentimientos.
\vs p103 9:12 \pc Existe una realidad en la experiencia religiosa que es proporcional a su contenido espiritual, y esta realidad trasciende la razón, la ciencia, la filosofía, la sabiduría y todos los otros logros humanos. La convicción de tal experiencia es irrefutable; la lógica del vivir religioso es irrebatible; la certidumbre de tal conocimiento es sobrehumana; la satisfacción es magníficamente divina; la valentía, irreductible; la profunda dedicación, incuestionable; la lealtad, suprema; y los destinos, finales ---eternos, últimos y universales---.
\vsetoff
\vs p103 9:13 [Exposición de un melquisedec de Nebadón.]
