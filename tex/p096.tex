\upaper{96}{Yahvé, el Dios de los hebreos}
\author{Melquisedec}
\vs p096 0:1 Al conceptualizar la Deidad, el hombre incluye primero a todos los dioses, después subordina los dioses ajenos a su deidad tribal y, finalmente, excluye a todos salvo al Dios único de valor último y supremo. Los judíos integraron a todos sus dioses en su más sublime concepto del Señor Dios de Israel. Igualmente, los hindúes unieron a sus muy variadas deidades en “una espiritualidad de dioses” tal como se describe en el Rig Veda, mientras que los mesopotámicos redujeron el número de sus dioses englobándolos en el concepto de Bel Marduc. Estas ideas monoteístas se desarrollaron en todo el mundo no mucho después de la aparición en Salem, Palestina, de Maquiventa Melquisedec. Pero el concepto que tenía Melquisedec de la Deidad era distinto al de la filosofía evolutiva de inclusión, subordinación y exclusión; se basaba exclusivamente en un \bibemph{poder creativo} y muy pronto ejerció su influencia en las nociones más elevadas de la deidad existentes en Mesopotamia, la India y Egipto.
\vs p096 0:2 \pc Por tradición, la religión de Salem se veneraba por los ceneos y por algunas otras tribus cananeas. Y este era uno de los propósitos de la encarnación de Melquisedec: que una religión de un solo Dios se fomentase y allanase el camino para el ministerio de gracia en la tierra de un hijo de ese Dios único. Miguel no podía venir a Urantia hasta que no existiese una población creyente en el Padre Universal entre los que pudiese hacer su aparición.
\vs p096 0:3 La religión salemita perduró en Palestina entre los ceneos, que la tenían como su credo, y esta religión, al ser adoptada más tarde por los hebreos, se vio influenciada, primeramente, por las enseñanzas morales de los egipcios; después, por el pensamiento teológico babilónico; y, finalmente, por los conceptos iraníes del bien y del mal. Objetivamente, la religión hebrea se fundamenta en el pacto hecho entre Abraham y Maquiventa Melquisedec; evolutivamente, es producto de circunstancias situacionales únicas, pero culturalmente ha tomado prestado libremente de la religión, de la moralidad y de la filosofía de todo el Levante. A través de la religión hebrea, una gran parte de la moral y del pensamiento religioso de Egipto, Mesopotamia e Irán se transmitiría a los pueblos occidentales.
\usection{1. CONCEPTOS DE LA DEIDAD ENTRE LOS SEMITAS}
\vs p096 1:1 Los primeros semitas creían que todas las cosas estaban habitadas por espíritus. Había espíritus del mundo animal y del mundo vegetal; espíritus del año, o señor de la progenie; espíritus del fuego, del agua y del aire; un verdadero panteón de espíritus que debían temerse o adorarse. Y las enseñanzas de Melquisedec sobre un Creador Universal nunca disiparon por completo la creencia en estos espíritus subordinados o dioses de la naturaleza.
\vs p096 1:2 El paso de los hebreos desde el politeísmo, a través del henoteísmo, al monoteísmo no tuvo un desarrollo conceptual ininterrumpido y continuo. Experimentaron muchos retrocesos en la evolución de sus conceptos sobre la Deidad, a la vez que, en cualquier época dada, existían ideas diversas acerca de Dios entre los diferentes grupos de creyentes semitas. En ocasiones, asignaban numerosos términos a sus conceptos de Dios y, a fin de evitar confusiones, se definirán estos distintos apelativos de la Deidad, por su relación con la evolución de la teología judía:
\vs p096 1:3 \li{1.}\bibemph{Yahvé} era el dios de las tribus del sur de Palestina, que relacionaron este concepto de la deidad con el Monte Horeb, el volcán del Sinaí. Yahvé era meramente uno de los cientos de miles de dioses de la naturaleza que mantenían la atención y demandaban la adoración de las tribus y los pueblos semitas.
\vs p096 1:4 \li{2.}\bibemph{El Elyón}. Durante siglos, tras la estancia de Melquisedec en Salem, su doctrina de la Deidad subsistió en diferentes versiones pero, por lo general, se designaba con el término de El Elyón, el Dios Altísimo del cielo. Muchos semitas, incluyendo los descendientes inmediatos de Abraham, adoraron en distintos momentos tanto a Yahvé como a El Elyón.
\vs p096 1:5 \li{3.}\bibemph{El Shaddai}. Es difícil explicar lo que representaba El Shaddai. Esta idea de Dios era un término compuesto derivado de las enseñanzas del Libro de la Sabiduría de Amenemope, modificadas por la doctrina de Akenatón sobre Atón e influenciadas más tarde por las enseñanzas de Melquisedec, tal como se plasmaban en la noción de El Elyón. Pero a medida que el concepto de El Shaddai impregnaba la mente hebrea, este se vio influido con las creencias existentes en el desierto sobre Yahvé.
\vs p096 1:6 En esta era, una de las ideas predominantes de la religión fue la idea egipcia sobre la divina Providencia: la creencia de que la prosperidad material era una recompensa por servir a El Shaddai.
\vs p096 1:7 \li{4.}\bibemph{El}. En medio de esta confusión terminológica y de la imprecisión conceptual, muchos devotos creyentes se esforzaron honestamente por rendir culto a todas estas ideas evolutivas de la divinidad, y se adquirió la costumbre de referirse a esta Deidad compuesta como El. Este término incluía además a otros dioses de la naturaleza de los beduinos.
\vs p096 1:8 \li{5.}\bibemph{Elohim}. Durante mucho tiempo, en Kish y en Ur continuaron existiendo grupos sumerio\hyp{}caldeos que impartieron el concepto trino de Dios, basado en las tradiciones de los días de Adán y Melquisedec. Dicha doctrina se llevó a Egipto, donde esta Trinidad se adoraba con el nombre de Elohim, o en singular, como Eloah. En el ámbito filosófico de Egipto y, más tarde, a través de los maestros alejandrinos de bagaje hebraico, se enseñó esta unidad de dioses plurales, y muchos de los asesores de Moisés, en el momento del éxodo, creían en esta Trinidad. Pero el concepto del Elohim trinitario nunca formó verdaderamente parte de la teología hebrea, hasta después de que los hebreos cayeran bajo la influencia política de los babilonios.
\vs p096 1:9 \li{6.}\bibemph{Diversas denominaciones}. A los semitas no les era grato pronunciar el nombre de su Deidad, por lo que, en ocasiones, recurrieron a numerosos apelativos, tales como: el Espíritu de Dios, el Señor, el Ángel del Señor, el Todopoderoso, el Santo, el Altísimo, Adonai, el Anciano de Días, el Señor Dios de Israel, el Creador del Cielo y de la Tierra, Kyrios, Yah, el Señor de las Huestes y el Padre de los Cielos.
\vs p096 1:10 \pc \bibemph{Jehová} es un término que se ha empleado en tiempos recientes para designar el concepto final de Yahvé, desarrollado definitivamente a partir de la larga experiencia de los hebreos. Pero el nombre de Jehová no comenzó a emplearse hasta mil quinientos años después de los tiempos de Jesús.
\vs p096 1:11 \pc Hasta hacia el año 2000 a. C., el Monte Sinaí estuvo activo de forma intermitente como volcán, con erupciones ocasionales ocurridas hasta fechas tan tardías como la llegada de los israelitas a esta región. El fuego y el humo, junto con las atronadoras explosiones propias de las erupciones de esta montaña volcánica, impresionaban y sobrecogían a los beduinos de las regiones circundantes y les provocaba un gran temor hacia Yahvé. Más tarde, este espíritu del Monte Horeb se convirtió en el Dios de los semitas hebreos, que acabarían por creer en su supremacía sobre todos los demás dioses.
\vs p096 1:12 Durante mucho tiempo, los cananeos habían venerado a Yahvé y, aunque muchos de los ceneos creían en mayor o menor grado en El Elyón, el supradiós de la religión de Salem, una mayoría de cananeos se aferraba en parte a las viejas deidades tribales. Estaban poco dispuestos a abandonar a sus deidades nacionales en favor de un Dios de ámbito internacional, por no decir interplanetario. No eran propensos a aceptar una deidad universal y, consecuentemente, estas tribus siguieron adorando a sus deidades tribales, incluyendo a Yahvé y a los becerros de plata y oro que simbolizaban la idea que los pastores beduinos tenían del espíritu del volcán del Sinaí.
\vs p096 1:13 Mientras adoraban a sus dioses, los sirios también creían en el Yahvé de los hebreos, porque sus profetas dijeron al rey sirio: “Sus dioses son dioses de los montes, por eso nos han vencido, pero si peleamos con ellos en la llanura, de seguro los venceremos”.
\vs p096 1:14 A medida que el hombre avanza culturalmente, los dioses menores se subordinan a una deidad suprema; el gran Júpiter persiste solo como una exclamación. Los monoteístas conservan sus dioses de menor rango tales como los espíritus, demonios, parcas, nereidas, hadas, duendecillos, enanitos, almas en pena y mal de ojo. Los hebreos pasaron por el henoteísmo y, durante mucho tiempo, creyeron en la existencia de otros dioses además de Yahvé, pero, cada vez más, fueron percibiendo a estas deidades ajenas como subordinadas a Yahvé. Admitían la realidad de Quemós, dios de los amoritas, pero sostenían que estaba subordinado a Yahvé.
\vs p096 1:15 La idea de Yahvé ha experimentado el desarrollo conceptual más amplio jamás habido en las teorías humanas sobre Dios. Su paulatina evolución solo se puede comparar con la metamorfosis del concepto de Buda en Asia, que al final condujo a la conceptualización del Absoluto Universal, al igual que el concepto de Yahvé llevaría finalmente a la idea del Padre Universal. Pero en lo tocante al hecho histórico, es necesario comprender que, aunque los judíos cambiaron, por tanto, sus perspectivas sobre la Deidad desde el dios tribal del Monte Horeb hasta el amoroso y misericordioso Padre Creador de épocas posteriores, no modificaron su nombre; continuaron en todo momento llamando Yahvé a este concepto evolutivo de la Deidad.
\usection{2. LOS PUEBLOS SEMITAS}
\vs p096 2:1 Los semitas del este eran jinetes bien organizados y bien liderados que invadieron las regiones orientales del creciente fértil y allí se unieron con los babilonios. Los caldeos próximos a Ur estaban entre los semitas orientales más avanzados. Los fenicios formaban un grupo, superior y bien organizado, de semitas mixtos que ocupaban la zona occidental de Palestina, a lo largo de la costa mediterránea. Racialmente, los semitas figuraban entre los pueblos de Urantia más mezclados: contenían factores hereditarios de casi todas las nueve razas del mundo.
\vs p096 2:2 Repetidamente, los semitas árabes, lucharon para llegar hasta la tierra prometida del norte, la tierra que “fluía leche y miel”, pero con igual frecuencia se les rechazó por los semitas e hititas del norte, mejor organizados y mucho más civilizados. Más tarde, durante una hambruna inusitadamente grave, estos beduinos itinerantes entraron en gran número en Egipto como obreros contratados para las obras públicas egipcias, solo para pasar por la amarga experiencia de la esclavitud, debido al duro trabajo diario de explotación que soportaban los obreros habituales del valle del Nilo.
\vs p096 2:3 Fue solamente tras los días de Maquiventa Melquisedec y Abraham cuando a determinadas tribus de los semitas, a causa de sus particulares creencias religiosas, se les llamó hijos de Israel y, más adelante, hebreos, judíos y “pueblo elegido”. Abraham no fue el padre racial de todos los hebreos; ni siquiera fue el progenitor de todos los semitas beduinos que estuvieron cautivos en Egipto. Es cierto que sus descendientes, al salir de Egipto, formaron el núcleo del posterior pueblo judío, pero la inmensa mayoría de los hombres y las mujeres que llegarían a integrarse en los clanes de Israel nunca habían vivido en Egipto. Eran simplemente nómadas como ellos mismos que optaron por seguir el liderazgo de Moisés cuando los hijos de Abraham, acompañados de los semitas de Egipto, viajaban a través del norte de Arabia.
\vs p096 2:4 \pc Las enseñanzas de Melquisedec sobre El Elyón, el Altísimo, y el pacto del favor divino mediante la fe, se habían olvidado, en buena parte, en el momento de la esclavización por parte de Egipto de los pueblos semitas, que pronto formarían la nación hebrea. Si bien, durante todo este período de cautiverio, persistía entre estos nómadas de Arabia una creencia tradicional en Yahvé como su deidad racial.
\vs p096 2:5 Más de cien tribus arábigas distintas adoraban a Yahvé y, excepto por los restos del concepto de El Elyón de Melquisedec, que perduraba entre las clases más cultas de Egipto, incluyendo a linajes mixtos de hebreos y egipcios, la religión del colectivo general de esclavos hebreos cautivos era una versión modificada del antiguo ritual yahvista de magia y sacrificio.
\usection{3. EL INIGUALABLE MOISÉS}
\vs p096 3:1 El comienzo de la evolución de los conceptos e ideales hebreos de un Creador Supremo se remonta a la salida de los semitas de Egipto bajo la dirección de un formidable líder, maestro y persona de gran aptitud organizativa: Moisés. Su madre pertenecía a la familia real egipcia; su padre era un oficial semita de enlace entre el gobierno y los beduinos cautivos. Por este motivo, Moisés poseía cualidades derivadas de fuentes raciales superiores; su ascendencia estaba tan sumamente mezclada que es imposible adscribirlo a un solo grupo racial. Si su linaje no hubiese tenido tal mezcla, jamás habría exhibido esa inusual versatilidad y adaptabilidad, que le posibilitó dirigir a la multitud tan plural que acabó por sumarse a los semitas beduinos y que, bajo su liderazgo, huyeron de Egipto con dirección al desierto arábigo.
\vs p096 3:2 A pesar de los alicientes de la cultura del reino del Nilo, Moisés decidió unirse al pueblo de su padre. En el momento en el que, con su gran capacidad organizativa, Moisés elaboraba sus planes para una futura liberación de estos beduinos cautivos, ellos apenas disponían de religión digna de tal nombre; no poseían una verdadera idea de Dios ni tenían esperanzas en el mundo.
\vs p096 3:3 \pc Nunca líder alguno se había comprometido antes a reformar y hacer progresar a un colectivo de seres humanos más desamparados, abatidos, desmoralizados e ignorantes. Pero, en cuanto a su desarrollo, estos esclavos poseían posibilidades latentes en sus estirpes hereditarias, y había un número suficiente de líderes formados, instruidos por Moisés en preparación al día de la revuelta y del logro de la libertad, que coordinarían eficientemente tal acción. Se había empleado a estos hombres, superiormente dotados, como capataces de su propio pueblo nativo; habían recibido alguna educación gracias a la influencia de Moisés con los gobernantes egipcios.
\vs p096 3:4 Moisés intentó negociar por medios diplomáticos la liberación de sus semejantes semitas. Él y su hermano subscribieron un pacto con el rey de Egipto, mediante el que se les concedió permiso para salir de forma pacífica del valle del Nilo en dirección al desierto arábigo. Recibirían una cantidad modesta de dinero y mercancías como muestra de sus prolongados servicios en Egipto. Los hebreos, por su parte, concertaron el acuerdo de mantener relaciones amistosas con los faraones y de no sumarse a ninguna alianza contra Egipto. Pero el rey, más tarde, creyó conveniente rechazar dicho tratado, dando como razón la excusa de que sus espías habían descubierto que los esclavos beduinos eran desleales. Afirmaba que buscaban la libertad con el fin de ir al desierto y organizar a los nómadas contra Egipto.
\vs p096 3:5 Pero Moisés no se desanimó; esperó el momento propicio y, en menos de un año, cuando las fuerzas militares egipcias estaban muy atareadas resistiendo de forma simultánea un fuerte ataque libio desde el sur y una invasión naval griega desde el norte, y, con su audacia y destreza organizativa, sacó a sus compatriotas fuera de Egipto en una espectacular fuga nocturna. Esta rápida huida hacia la libertad se planeó de forma minuciosa y se llevó a cabo hábilmente. Y tuvieron éxito, a pesar de la tenaz persecución del faraón y de un pequeño grupo de egipcios, que en su totalidad cayeron ante las defensas de los fugitivos, dejando un buen botín. Este botín se incrementaría por el saqueo de la multitud de esclavos huidos en su marcha hacia su ancestral hogar en el desierto.
\usection{4. PROCLAMACIÓN DE YAHVÉ}
\vs p096 4:1 La evolución y el avance de las enseñanzas mosaicas han ejercido su influencia sobre casi la mitad del mundo, y todavía la ejercen incluso en el siglo XX. Aunque en Moisés se sustanciaba la más adelantada filosofía religiosa egipcia, los esclavos beduinos conocían poco estas enseñanzas; si bien, nunca habían llegado a olvidar del todo al dios del Monte Horeb, a quien sus ancestros habían llamado Yahvé.
\vs p096 4:2 Moisés había oído hablar de las enseñanzas de Maquiventa Melquisedec tanto por parte de su padre como de su madre; de hecho, esta creencia religiosa común de ambos explica la insólita unión entre una mujer de sangre real y un hombre de una raza cautiva. El suegro de Moisés adoraba a El Elyón, pero los padres del libertador creían en El Shaddai y Moisés se educó, pues, en esta enseñanza. Si bien, por la influencia de su suegro, un ceneo, se convirtió a El Elyón; y, para el momento en el que los hebreos acamparon alrededor del Monte Sinaí, tras la huida de Egipto, Moisés había definido un concepto, nuevo y ampliado, de la Deidad (derivado de todas sus creencias anteriores), que acertadamente decidió proclamar a su pueblo a modo de una noción más elaborada de su antiguo dios tribal, Yahvé.
\vs p096 4:3 Moisés había intentado enseñar a estos beduinos la idea de El Elyón, aunque antes de abandonar Egipto, se había convencido de que nunca comprenderían esta doctrina plenamente. Así pues, decidió deliberadamente comprometerse a adoptar a su dios tribal del desierto como el solo y único dios de sus seguidores. Moisés no impartió concretamente que otros pueblos y naciones no pudiesen tener otros dioses, pero sí sostenía con firmeza que, en particular para los hebreos, Yahvé estaba por encima de todos ellos. No obstante, constantemente se sintió contrariado por la difícil situación de intentar explicar su nueva y más elevada idea de la Deidad a estos esclavos faltos de conocimiento bajo la forma del término antiguo de Yahvé, que siempre había tenido como símbolo el becerro de oro de las tribus beduinas.
\vs p096 4:4 \pc El hecho de que Yahvé fuese el Dios de los hebreos huidos explica por qué permanecieron tanto tiempo a los pies de la montaña sagrada del Sinaí, y por qué recibieron allí los diez mandamientos que Moisés dio a conocer en nombre de Yahvé, el dios del Horeb. Durante esta larga estancia ante el Sinaí, se perfeccionaron aún más los ceremoniales religiosos del culto de adoración hebreo recientemente surgido.
\vs p096 4:5 \pc No parece que Moisés hubiese logrado jamás instaurar un ceremonial de adoración relativamente avanzado, y mantener íntegros a sus seguidores durante un cuarto de siglo, si no hubiese sido por la violenta erupción del Horeb durante la tercera semana de su ferviente estancia en su base. “El monte de Yahvé se consumía en el fuego, y el humo subía como el humo de un horno, y todo el monte se estremecía violentamente”. En vista de este cataclismo, no sorprende que Moisés pudiera inculcar en sus hermanos la enseñanza de que su Dios era “poderoso, terrible, un fuego devorador, temible y todopoderoso”.
\vs p096 4:6 Moisés proclamó que Yahvé era el Señor Dios de Israel, que había elegido a los hebreos como su pueblo elegido; estaba construyendo una nueva nación, y con sabiduría nacionalizó sus enseñanzas religiosas, diciendo a sus seguidores que Yahvé era muy estricto, un “Dios celoso”. Pero no obstante intentó engrandecer su concepto de divinidad cuando les enseñó que Yahvé era el “Dios de los espíritus de toda carne” y cuando dijo: “El eterno Dios es tu refugio, y acá abajo los brazos eternos”. Moisés enseñó que Yahvé era un Dios que cumplía sus pactos; que él “no te dejará, ni te destruirá, ni se olvidará del pacto que juró a tus padres, porque el Señor os ama y no olvidará el juramento que hizo a vuestros padres”.
\vs p096 4:7 Moisés hizo un esfuerzo heroico por elevar a Yahvé a la dignidad de una Deidad suprema cuando lo definió diciendo: “Es un Dios de verdad y no hay iniquidad en él; es justo y recto en todas sus maneras”. Y, sin embargo, a pesar de esta excelsa enseñanza, el limitado entendimiento de sus seguidores hizo necesario hablar de Dios a la imagen del hombre, como sometido a ataques de ira, cólera y severidad, e incluso que era vengativo y fácilmente influenciable por el comportamiento del hombre.
\vs p096 4:8 Bajo las enseñanzas de Moisés, este dios tribal de la naturaleza, Yahvé, se convirtió en el Señor Dios de Israel, que los siguió en el desierto e incluso en el exilio, donde se le concibió enseguida como el Dios de todos los pueblos. El cautiverio que sufrirían más tarde y que esclavizó a los judíos en Babilonia hizo que la noción evolutiva de Yahvé se liberara y adoptara la función monoteísta de Dios de todas las naciones.
\vs p096 4:9 La característica más única y asombrosa de la historia religiosa de los hebreos consiste en este continuo desarrollo de la conceptualización de la Deidad desde el dios primitivo del Monte Horeb, pasando por las enseñanzas de sus sucesivos líderes espirituales, hasta el alto nivel de evolución alcanzado, ilustrado en la doctrina sobre la Deidad de los Isaías, que proclamaron esa magnífica idea de un Padre Creador de amor y misericordia.
\usection{5. LAS ENSEÑANZAS DE MOISÉS}
\vs p096 5:1 Moisés combinaba, de forma extraordinaria en su persona, liderazgo militar, capacidad de organización y magisterio religioso. Fue el maestro y líder más importante del mundo durante el tiempo transcurrido entre Maquiventa y Jesús. Moisés trató de introducir muchas reformas en Israel, de las que no quedan constancia. En el espacio de vida de un solo hombre, él sacó a la heterogénea multitud de los llamados hebreos de la esclavitud e incivilizada itinerancia, mientras sentaba las bases para el nacimiento venidero de una nación y la perpetuación de una raza.
\vs p096 5:2 Hay tan poca documentación sobre la gran obra de Moisés porque, en el momento del éxodo, los hebreos no poseían lengua escrita. El relato de los tiempos y de la labor de Moisés partió de tradiciones existentes más de mil años después de la muerte del gran líder.
\vs p096 5:3 Muchos de los avances conseguidos por Moisés se debieron, al margen de la religión de los egipcios y de las tribus levantinas circundantes, a las tradiciones ceneas de la época de Melquisedec. Sin las enseñanzas que Maquiventa impartió a Abraham y a sus contemporáneos, los hebreos hubiesen salido de Egipto en una desesperanzadora oscuridad. Moisés y Jetro, su suegro, recogieron los restos que quedaban de las tradiciones de los días de Melquisedec, y estas enseñanzas, unidas al conocimiento de los egipcios, llevaron a Moisés a perfeccionar la religión y del ritual de los israelitas. Moisés tenía una especial aptitud para organizar; seleccionó lo mejor de la religión y de las costumbres de Egipto y Palestina y, sumando estas prácticas a las tradiciones de las enseñanzas de Melquisedec, organizó el sistema ceremonial hebreo de adoración.
\vs p096 5:4 \pc Moisés creía en la Providencia; estaba enteramente contagiado por las doctrinas de Egipto respecto al control sobrenatural del Nilo y de los demás elementos de la naturaleza. Tenía una gran visión de Dios, pero fue totalmente sincero al enseñar a los hebreos que, si obedecían a Dios, “él te amará, te bendecirá y te multiplicará; multiplicará el fruto de tu vientre y el fruto de tu tierra: tu grano, tu vino, tu aceite y tus rebaños. Bendito serás más que todos los pueblos, y el Señor, tu Dios, apartará de ti toda enfermedad, y no hará recaer sobre ti ninguna de las malas plagas de Egipto”. Incluso dijo: “Acuérdate del Señor, tu Dios, porque él es quien te da el poder para adquirir las riquezas”. “Prestarás a muchas naciones, pero tú no tomarás prestado; tendrás dominio sobre muchas naciones, pero sobre ti no tendrán dominio”.
\vs p096 5:5 \pc Pero resultaba verdaderamente lastimoso ver a Moisés, un hombre de gran mente, tratando de adaptar su concepto sublime de El Elyón, el Altísimo, a la comprensión de ignorantes e iliteratos hebreos. A sus líderes congregados les clamó: “El Señor vuestro Dios es Dios; no hay otro fuera de él”; mientras que ante la dispar multitud declaró: “¿Quién como vuestro Dios entre los dioses?”. Moisés tomó una postura valiente y parcialmente efectiva contra los fetiches y la idolatría cuando manifestó: “Ninguna figura visteis en el día en que vuestro Dios os habló en Horeb de en medio del fuego”. También prohibió que se hicieran imágenes de cualquier tipo.
\vs p096 5:6 Moisés temía proclamar la misericordia de Yahvé; prefería causar espanto en el pueblo con el temor a la justicia de Dios, diciendo: “Vuestro Dios es Dios de Dioses y Señor de Señores, Dios grande, poderoso y temible, que no hace acepción de personas”. De nuevo intentó controlar a los convulsos clanes cuando declaró que “vuestro Dios hace morir cuando lo desobedecéis; sana y hace vivir cuando lo obedecéis”. Pero Moisés enseñó a estas tribus que se convertirían en el pueblo elegido de Dios solo con la condición de que “guardasen sus mandamientos y obedecieran sus estatutos”.
\vs p096 5:7 Durante los primeros tiempos, se enseñó a los hebreos poco acerca de la misericordia de Dios. Aprendieron que Dios era “el Todopoderoso; el Señor guerrero, el Dios de las batallas, glorioso en el poder, que aplasta a sus enemigos”. “El Señor vuestro Dios anda en medio de tu campamento para librarte”. Los israelitas pensaban de su Dios como de alguien que los amaba, pero que también “endureció el corazón del faraón” y “maldijo a sus enemigos”.
\vs p096 5:8 Aunque Moisés ofreció a los hijos de Israel breves atisbos de una Deidad universal y benefactora, por lo general, la noción que estos tenían de Yahvé en su vida cotidiana era la de un Dios, aunque algo mejor que los dioses tribales de los pueblos de los alrededores. Su concepto de Dios era primitivo, tosco y antropomórfico; cuando Moisés murió, estas tribus beduinas volvieron rápidamente a las ideas semibárbaras de sus dioses antiguos del Horeb y del desierto. La visión ampliada y más sublime de Dios que Moisés exponía en ocasiones a sus líderes se olvidó pronto, mientras que la mayoría de la gente volvía a la adoración fetichista de sus becerros de oro, el símbolo de Yahvé para el pastor palestino.
\vs p096 5:9 \pc Cuando Moisés entregó el mando de los hebreos a Josué, ya había reunido a miles de descendientes indirectos de Abraham, Nacor, Lot y de otras de las tribus afines y, con firmeza, los había organizado como nación autosuficiente y parcialmente autorregulada de guerreros pastores.
\usection{6. CONCEPTO DE DIOS TRAS LA MUERTE DE MOISÉS}
\vs p096 6:1 Cuando Moisés falleció, el alto concepto de Yahvé que había impartido se deterioró rápidamente. Josué y los líderes de Israel siguieron conservando las tradiciones mosaicas del Dios todo sapiente, benefactor y todopoderoso, pero la gente común volvió con prontitud a las viejas ideas del desierto sobre Yahvé. Y esta deriva hacia atrás de la noción de la Deidad continuó cada vez con los sucesivos gobiernos de los distintos jeques tribales, los llamados jueces.
\vs p096 6:2 La fascinación que la extraordinaria persona de Moisés ejercía había mantenido vivo en el corazón de sus seguidores el estímulo de un progresivo engrandecimiento de la idea de Dios; pero, cuando llegaron a las tierras fértiles de Palestina, evolucionaron con celeridad desde pastores nómadas hasta agricultores asentados y algo tranquilos. Y esta evolución de su práctica de vida y el cambio de perspectiva religiosa requerían una modificación más o menos completa del carácter de su conceptualización de la naturaleza de su Dios, de Yahvé. Durante los tiempos del comienzo de la transformación del austero, tosco, estricto y atronador Dios del desierto del Sinaí en el concepto más tardío de un Dios de amor, justicia y misericordia, los hebreos casi se olvidaron de las excelsas enseñanzas de Moisés. Estuvieron a punto de perder toda noción de monoteísmo; prácticamente desaprovecharon su oportunidad de convertirse en el pueblo que serviría como nexo fundamental en la evolución espiritual de Urantia, en el colectivo que conservaría las enseñanzas de Melquisedec de un solo Dios hasta los tiempos de la encarnación de un hijo de gracia de ese Padre de todos.
\vs p096 6:3 Josué trató imperiosamente de mantener la idea de un Yahvé supremo en la mente de los miembros de las tribus, haciendo que se proclamara: “como estuve con Moisés, estaré contigo; no te dejaré ni te desampararé”. Josué estimó necesario predicar un evangelio severo a su incrédulo pueblo, un pueblo demasiado dispuesto a creer en su religión antigua y nativa, aunque reticente a avanzar en la religión de la fe y de la rectitud. El peso de las enseñanzas de Josué recaía en esta afirmación: “Yahvé es un Dios santo y celoso que no sufrirá vuestras rebeliones y vuestros pecados”. El concepto más elevado de esta era describía a Yahvé como un “Dios de poder, juicio y justicia”.
\vs p096 6:4 Pero incluso en esta época de oscuridad, surgía, ocasionalmente, algún maestro solitario proclamando la idea mosaica de la divinidad: “Vosotros hijos del mal no podéis servir al Señor, porque él es un Dios santo”. ¿Será el mortal más justo que Dios? ¿Será el hombre más puro que el que lo hizo?”. “¿Descubrirás tú los secretos de Dios? ¿Hallarás la perfección del Todopoderoso? Mirad, Dios es grande y nosotros no lo conocemos. Él es el Todopoderoso, al cual no alcanzamos”.
\usection{7. LOS SALMOS Y EL LIBRO DE JOB}
\vs p096 7:1 Bajo el liderazgo de sus jeques y sacerdotes, los hebreos comenzaron a extenderse ampliamente por Palestina. Pero pronto retornaron a las primitivas creencias del desierto y se contaminaron con las prácticas religiosas menos avanzadas de los cananeos. Se convirtieron en idólatras y libertinos, y su idea de la Deidad cayó muy por debajo de los conceptos egipcios y mesopotámicos sobre Dios que algunos grupos salemitas supervivientes preservaban, y que constan en algunos de los salmos y en el llamado Libro de Job.
\vs p096 7:2 \pc Los salmos son obra de una veintena o más de autores; muchos de ellos los escribieron maestros egipcios y mesopotámicos. Durante estos tiempos, cuando en el Levante se adoraba a los dioses de la naturaleza, existía aún un buen número de personas que creía en la supremacía de El Elyón, el Altísimo.
\vs p096 7:3 En ninguna colección de escritos religiosos se expresa tal riqueza de devoción y de ideas inspiradoras como en el Libro de los Salmos. Y sería muy conveniente que, en la atenta lectura de esta maravillosa recopilación de textos literarios piadosos, se preste atención, por separado, a la fuente y a la cronología de cada uno de los himnos de alabanza y adoración, teniendo en cuenta que no existe ninguna otra colección individual que abarque un intervalo tan extenso de tiempo. Este Libro de los Salmos es el relato de los distintos conceptos desarrollados sobre Dios por los creyentes en la religión de Salem de todo el Levante e incluye la totalidad del periodo comprendido entre Amenemope e Isaías. En los salmos se describe la conceptualización de Dios en todas sus facetas, desde la tosca idea de una deidad tribal hasta el ideal, considerablemente ampliado, de los hebreos más tardíos, en donde Yahvé aparece como un soberano amoroso y un Padre misericordioso.
\vs p096 7:4 Y así considerados, este conjunto de salmos constituye la selección más valiosa y útil de sentimientos devocionales jamás reunida por el hombre hasta los días del siglo XX. El espíritu piadoso de esta colección de himnos trasciende al de cualquiera de los demás libros sagrados del mundo.
\vs p096 7:5 \pc La heterogénea imagen de la Deidad que aparece en el Libro de Job fue fruto de la labor de más de una veintena de maestros religiosos mesopotámicos y se prolongó por un período de casi trescientos años. Y cuando leáis los excelsos conceptos de la divinidad representados en esta recopilación de las creencias mesopotámicas, reconoceréis que fue en los alrededores de Ur de Caldea donde la idea de un Dios real se conservó con más excelencia durante los días aciagos de Palestina.
\vs p096 7:6 En Palestina, la sabiduría y la omnipresencia de Dios se entendían a menudo, pero raras veces su amor y su misericordia. El Yahvé de aquellos tiempos “manda espíritus malos para dominar las almas de sus enemigos”; hace prosperar a sus propios y obedientes hijos, mientras que maldice e inflige terribles castigos a todos los demás. “Frustra los pensamientos de los astutos y atrapa a los sabios en su propia astucia”.
\vs p096 7:7 Solamente en Ur surgió una voz para declarar la misericordia de Dios diciendo: “orará a Dios y obtendrá su favor. Verá su faz con júbilo, y él restaurará al hombre su justicia divina”. De esta manera, se predicaba desde Ur la salvación, el favor divino, por la fe: “Es clemente para el que se arrepiente y dice: ‘que lo libra de descender al sepulcro, que hay redención en él'. Si alguien dice, ‘He pecado y he pervertido lo recto, pero de nada me ha aprovechado’, Dios redimirá su alma para que no pase al sepulcro, y verá la luz”. En el mundo levantino, no se habían oído desde los tiempos de Melquisedec un mensaje tan clamoroso y alentador para la salvación humana como esta extraordinaria enseñanza de Eliú, profeta de Ur y sacerdote de los creyentes salemitas, esto es, la que quedaba de la antigua colonia mesopotámica de Melquisedec.
\vs p096 7:8 Y, así pues, grupos remanentes de los misioneros de Salem de Mesopotamia mantuvieron la luz de la verdad durante el período de la desvertebración de los pueblos hebreos y hasta la aparición del primero de una larga lista de maestros de Israel, que nunca cesarían de desarrollar, noción a noción, el concepto de Yahvé, hasta haber alcanzado el reconocimiento del ideal del Padre Universal y Creador de todos, el súmmun de su evolución conceptual.
\vsetoff
\vs p096 7:9 [Exposición de un melquisedec de Nebadón.]
