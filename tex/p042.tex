\upaper{42}{Energía, mente y materia}
\author{Mensajero poderoso}
\vs p042 0:1 Los pilares del universo son materiales en el sentido de que la energía es la base de toda existencia, y el Padre Universal rige la energía pura. La fuerza, la energía, es la única cosa que constituye un monumento imperecedero en demostración y prueba de la existencia y la presencia del Absoluto Universal. Esta inmensa corriente de energía que emana de las Presencias del Paraíso nunca ha cesado, jamás ha fallado; nunca ha interrumpido su sostenimiento infinito.
\vs p042 0:2 La actuación sobre la energía del universo está siempre en conformidad con la voluntad personal y los mandatos omnisapientes del Padre Universal. Este dominio personal de la potencia manifestada y de la energía circulatoria se modifica por los actos y las decisiones correlacionados del Hijo Eterno, así como también por los propósitos unidos del Hijo y del Padre, que el Actor conjunto lleva a efecto. Estos seres divinos actúan personalmente y de forma individual; también lo hacen en las personas y potestades de un número casi ilimitado de seres de rango inferior, cada uno de los cuales expresa de forma diversa, por todo el universo de los universos, este propósito eterno y divino. Pero tales modificaciones o transmutaciones funcionales y provisionales de la potencia divina no disminuyen de modo alguno la verdad de la declaración de que toda la fuerza\hyp{}energía está bajo la total dirección de un Dios personal que reside en el centro de todas las cosas.
\usection{1. LAS FUERZAS Y LAS ENERGÍAS DEL PARAÍSO}
\vs p042 1:1 El fundamento del universo es la materia, pero la esencia de la vida es el espíritu. El Padre de los espíritus es también el predecesor de los universos; el Padre eterno del Hijo Primigenio es igualmente la fuente eterna del modelo primigenio, la Isla del Paraíso.
\vs p042 1:2 La materia ---la energía--- al no ser sino diferentes manifestaciones de la misma realidad cósmica, como fenómeno del universo es algo intrínseco al Padre Universal. “Todas las cosas en él subsisten”. La materia quizás parezca poner de manifiesto una energía inherente y exhibir una potencia autocontenida, pero las líneas de la gravedad que participan de las energías e intervienen en todos estos fenómenos físicos proceden, y dependen, del Paraíso. El ultimatón, la primera forma mensurable de energía, tiene el Paraíso como su núcleo.
\vs p042 1:3 \pc Existe una forma de energía desconocida en Urantia que es innata a la materia y que está presente en el espacio universal. Cuando por fin se realice este descubrimiento, los físicos creerán que han resuelto el misterio de la materia, al menos de manera parcial. Y así habrán dado un paso más en el acercamiento al Creador; así habrán llegado a comprender una faceta más de la manera divina; pero, de ningún modo, habrán encontrado a Dios ni tampoco habrán establecido la existencia de la materia o el funcionamiento de las leyes naturales, al margen del procedimiento cósmico que parte del Paraíso y del propósito incentivador del Padre Universal.
\vs p042 1:4 Tras haber alcanzado un progreso incluso mayor y haber realizado nuevos descubrimientos, una vez que Urantia haya avanzado de forma considerable en comparación con el conocimiento actual, aunque consigáis el dominio de las rotaciones energéticas de las unidades eléctricas de la materia hasta el punto de modificar sus manifestaciones físicas ---incluso después de todos estos logros, los científicos serán siempre incapaces de crear un solo átomo de materia o de originar un destello de energía o de añadir a la materia aquello que llamamos vida---.
\vs p042 1:5 \pc La creación de la energía y el don de la vida son prerrogativas del Padre Universal y de los seres personales creadores vinculados a él. El río de la energía y de la vida es un flujo continuo que proviene de las Deidades, es una corriente universal conjunta de la fuerza del Paraíso que fluye hacia todo el espacio. Esta energía divina se difunde por toda la creación. Los organizadores de la fuerza inician esos cambios e instituyen esas modificaciones de la fuerza espacial que resultan en la energía; los directores de la potencia transmutan la energía en materia; así nacen los mundos materiales. Los portadores de vida inician ciertos procesos en la materia inerte que dan lugar a lo que llamamos vida, vida material. Los supervisores de la potencia morontial desempeñan igualmente su actividad en todos los ámbitos de transición entre los mundos materiales y los espirituales. Los creadores espirituales de más elevado orden inauguran procesos similares en las formas divinas de la energía, y de aquí surgen las formas espirituales superiores de vida inteligente.
\vs p042 1:6 \pc La energía procede del Paraíso; se modela según directiva divina. La energía ---la energía pura--- participa de la naturaleza de lo divino; se diseña a semejanza de los tres Dioses que forman uno solo, al igual que obran en la sede del universo de los universos. Y toda la fuerza se encauza en el Paraíso, viene de las Presencias del Paraíso y regresa a ellas, y es en esencia una manifestación de la Causa Incausada ---el Padre Universal---; y sin el Padre nada de lo que existe existiría.
\vs p042 1:7 La fuerza que procede de la Deidad autoexistente es en sí misma por siempre existente. La fuerza\hyp{}energía es imperecedera, indestructible; estas manifestaciones del Infinito pueden estar sujetas a transmutaciones ilimitadas, a transformaciones sin fin, y a eternas metamorfosis, pero en ningún sentido o grado, ni siquiera en la más mínima magnitud imaginable, pueden o podrían experimentar jamás la extinción. Pero la energía, aunque surge del infinito, no se manifiesta de modo infinito; tal como en la actualidad se concibe, existen límites exteriores en el universo matriz.
\vs p042 1:8 La energía es eterna pero no infinita; responde siempre a la atracción globalizada de la Infinitud. La fuerza y la energía siempre tienen una continuidad; habiendo salido del Paraíso, deben regresar allí, aunque se necesiten era tras era para que se complete la vía circulatoria decretada. Aquello que se origina en la Deidad del Paraíso solo puede tener el Paraíso o la Deidad como destino.
\vs p042 1:9 \pc Y todo esto confirma nuestra convicción de un universo de universos circular, limitado de alguna manera, pero inmenso y ordenado. Si esto no fuese así, entonces, en algún momento, más tarde o más temprano, aparecería algún indicio del agotamiento de la energía. Todas las leyes, las configuraciones, la administración y el testimonio de los exploradores del universo apuntan a la existencia de un Dios infinito y, sin embargo, por ahora, indican, igualmente, la existencia de un universo finito, de una existencia circular sin fin, casi ilimitada, pero, no obstante, finita en contraste con la infinitud.
\usection{2. LOS SISTEMAS DE ENERGÍA UNIVERSALES NO ESPIRITUALES (LAS ENERGÍAS FÍSICAS)}
\vs p042 2:1 Es de hecho difícil encontrar en lengua inglesa palabras que puedan designar y describir con propiedad los distintos niveles de la fuerza y de la energía ---física, mental o espiritual---. En estas narraciones no podemos adoptar vuestras definiciones, generalmente aceptadas, de fuerza, energía y potencia. Existe tal insuficiencia lingüística que nos vemos obligados a asignar múltiples significados a estos términos. En este escrito, por ejemplo, la palabra \bibemph{energía} se usa para significar todas las fases y formas de movimiento, acción y potencial fenomenales, mientras que \bibemph{fuerza} se aplica a los estadios pregravitatorios de la energía y \bibemph{potencia} a los estadios posgravitatorios de la energía.
\vs p042 2:2 Trataré, sin embargo, de reducir la confusión conceptual proponiendo la conveniencia de adoptar la siguiente categorización para la fuerza cósmica, la energía emergente y la potencia del universo ---o energía física---:
\vs p042 2:3 \li{1.}\bibemph{La potencia del espacio}. Alude a la incontestable y libre presencia espacial del Absoluto Indeterminado. En un sentido extrínseco, este concepto se aplica al potencial de la fuerza\hyp{}espacio del universo que es inherente a la totalidad operativa del Absoluto Indeterminado, mientras que, en un sentido intrínseco, implica la totalidad de la realidad cósmica ---universos--- que emanaron en la eternidad de la Isla del Paraíso, isla que no tiene ni principio, ni fin, ni movimiento, ni cambio.
\vs p042 2:4 En el lado inferior del Paraíso se dan con probabilidad tres fenómenos característicos que se reparten en tres zonas donde la fuerza absoluta está presente y opera: la zona axial del Absoluto Indeterminado, la zona de la Isla del Paraíso misma y la zona intermedia, compuesta por ciertas instancias intermedias o funciones ecualizadoras y compensadoras no identificadas. Estas tres zonas concéntricas constituyen el centro del ciclo paradisíaco de la realidad cósmica.
\vs p042 2:5 La potencia espacial es una prerrealidad; es el ámbito del Absoluto Indeterminado y es sensible solamente a la atracción personal del Padre Universal, a pesar de que sea aparentemente modificable por la presencia de los organizadores mayores primarios de la fuerza.
\vs p042 2:6 En Uversa, se hace referencia a la potencia espacial como ABSOLUTA.
\vs p042 2:7 \li{2.}\bibemph{La fuerza primordial}. Representa el primer cambio fundamental en la potencia espacial y puede ser una de las funciones del Paraíso Inferior en su vinculación con el Absoluto Indeterminado. Sabemos que la presencia espacial que sale del Paraíso inferior está de algún modo modificada respecto a la entrante. Mas a pesar de cualquiera de estas posibles correlaciones, la transmutación, claramente identificada, de la potencia espacial en fuerza primordial es la función diferenciadora primaria debida a la tensión\hyp{}presencia de los organizadores vivos de la fuerza del Paraíso.
\vs p042 2:8 La fuerza pasiva y potencial se convierte en activa y primordial en respuesta a la resistencia ofrecida por la presencia espacial de los organizadores mayores de la fuerza y devenidos primarios. La fuerza emerge entonces del dominio exclusivo del Absoluto Indeterminado a los ámbitos de respuesta múltiple ---respuesta a ciertos movimientos primigenios iniciados por el Dios de Acción y acto seguido a ciertos movimientos compensatorios que emanan del Absoluto Universal---. La fuerza primordial es, según parece, reactiva a la causalidad trascendental en proporción a la absolutidad.
\vs p042 2:9 A la fuerza primordial se la denomina a veces \bibemph{energía pura;} en Uversa nos referimos a ella como SEGREGATA.
\vs p042 2:10 \li{3.}\bibemph{Las energías emergentes}. La presencia pasiva de los organizadores primarios de la fuerza es suficiente para transformar la potencia espacial en fuerza primordial y, sobre este campo espacial activado, estos mismos organizadores de la fuerza dan comienzo a su labor de activación. En los ámbitos de la manifestación de la energía, la fuerza primordial, antes de aparecer como potencia del universo, está llamada a pasar por dos fases distintas de transmutación. Estos dos niveles de energía emergente son:
\vs p042 2:11 \pc a. \bibemph{La energía poderosa}. Es una energía efectivamente orientable, movible de forma masiva, enormemente tensionada y de fuerte reacción ---son sistemas gigantescos de energía puestos en movimiento por la acción de los organizadores primarios de la fuerza---. Al principio, esta energía primaria o poderosa no es claramente susceptible a la atracción de la gravedad del Paraíso, aunque es probable que se haga susceptible como agregado de masa o en función de su orientación espacial al conjunto de influencias absolutas que operan desde la zona del Paraíso inferior. Cuando la energía emerge hasta un nivel en el que comienza a responder a la atracción circular y absoluta de la gravedad del Paraíso, los organizadores primarios de la fuerza ceden paso a la actividad de sus colaboradores secundarios.
\vs p042 2:12 \pc b. \bibemph{La energía gravitatoria}. La energía que hace entonces su aparición responde a la atracción de la gravedad y lleva consigo el potencial de la potencia del universo; se convierte en la predecesora activa de toda la materia del universo. Esta energía secundaria, o energía gravitatoria, es el efecto de la elaboración de la energía que resulta de la presión de su presencia y de las tendencias en tensión establecidas por los organizadores mayores de la fuerza y trascendentales adjuntos. En respuesta a la labor de estos operadores de la fuerza, la energía del espacio pasa rápidamente de su estadio de energía poderosa al de gravitatorio, volviéndose así directamente susceptible a la atracción circular de la gravedad (absoluta) del Paraíso, revelando, al mismo tiempo, cierto potencial de sensibilidad a la atracción de la gravedad lineal, intrínseca a la masa material, que hace pronto su aparición como resultado de los estadios electrónico y poselectrónico de la energía y de la materia. Cuando surge dicha respuesta a la gravedad, estos organizadores mayores de la fuerza adjuntos pueden retirarse de los ciclones energéticos del espacio siempre y cuando los directores de la potencia del universo sean factibles de ser asignados a ese campo de actuación.
\vs p042 2:13 \pc Estamos bastante inseguros respecto a las causas exactas de los primeros estadios de la evolución de la fuerza, pero reconocemos la acción inteligente del Último en los dos niveles en los que se manifiesta la energía emergente. En Uversa, cuando se considera a las energías poderosa y gravitatoria de forma conjunta, se las denomina ULTIMATA.
\vs p042 2:14 \li{4.}\bibemph{La potencia del universo}. La fuerza del espacio se transforma primero en energía del espacio y, de ahí, en energía regida por la gravedad. De este modo, la energía física ha alcanzado un desarrollo tal que se puede conducir hacia los canales de la potencia y ponerse al servicio de los múltiples propósitos de los creadores del universo. Los versátiles directores, centros y controladores de la energía física llevan a cabo esta tarea en el gran universo ---las creaciones organizadas y habitadas---. Estos directores de la potencia del universo asumen el control, más o menos completo, de veintiuna de las treinta fases de la energía, que constituyen el sistema energético actual de los siete suprauniversos. Este ámbito de la potencia\hyp{}energía\hyp{}materia es el campo de la acción inteligente del Séptuplo, que actúa bajo la acción directiva espacio temporal del Supremo.
\vs p042 2:15 En Uversa nos referimos a los reinos de la potencia del universo como GRAVITA.
\vs p042 2:16 \li{5.}\bibemph{La energía de Havona}. Conceptualmente hablando, este escrito se ha acercado al Paraíso a medida que hemos narrado la transmutación de la fuerza del espacio, nivel tras nivel, hasta el nivel de carácter operativo de la energía\hyp{}potencia de los universos del tiempo y del espacio. Continuando hacia el Paraíso, nos encontramos, entonces, con una fase preexistente de la energía característica del universo central. Aquí el ciclo evolutivo parece revertirse; la energía\hyp{}potencia parece ahora comenzar a replegarse hacia atrás, hacia la fuerza, pero una fuerza de naturaleza muy distinta a la potencia del espacio y a la fuerza primordial. Los sistemas de energía de Havona no son dobles; son trinos. Este es el ámbito de la energía existencial del Actor Conjunto, que obra en nombre de la Trinidad del Paraíso.
\vs p042 2:17 En Uversa estas energías de Havona se conocen como TRIATA.
\vs p042 2:18 \li{6.}\bibemph{La energía trascendental}. Este sistema de energía opera en el nivel superior del Paraíso y desde este, y solo con respecto a los seres absonitos. En Uversa se la denomina TRANOSTA.
\vs p042 2:19 \li{7.}\bibemph{La monota}. La energía tiene una estrecha afinidad con la divinidad cuando se trata de energía del Paraíso. Nos inclinamos a creer que la monota es la energía viva y no espiritual del Paraíso ---un equivalente en la eternidad de la energía viva y espiritual del Hijo Primigenio---, de ahí, el sistema de energía no espiritual del Padre Universal.
\vs p042 2:20 No nos es posible diferenciar entre la \bibemph{naturaleza} del espíritu del Paraíso y la \bibemph{monota} del Paraíso. Aparentemente son iguales. Poseen nombres distintos, pero resulta difícil comentar mucho sobre una realidad cuyas manifestaciones espirituales y no espirituales solo se pueden distinguir por el \bibemph{nombre}.
\vs p042 2:21 \pc Sabemos que las criaturas finitas pueden tener la experiencia de adorar al Padre Universal mediante el ministerio del Dios Séptuplo y de los modeladores del pensamiento, pero albergamos la duda de que un ser personal subabsoluto, ni incluso los directores de la potencia, pueda comprender la infinitud de la energía de la Primera Gran Fuente y Centro. Hay algo de lo que podemos tener certeza: si los directores de la potencia son conocedores del método por el que la fuerza del espacio experimenta esa metamorfosis, no nos revelan su secreto al resto de nosotros. Soy de la opinión de que no entienden del todo la labor de los organizadores de la fuerza.
\vs p042 2:22 Estos mismos directores de la potencia son catalizadores de la energía; esto es, mediante su presencia hacen que la energía se segmente, se organice o se acumule formando unidades. Todo esto supone que debe haber algo inherente a la energía que provoque este modo de actuación en presencia de estas criaturas de la potencia. Hace mucho tiempo que los melquisedecs de Nebadón denominaron al fenómeno de la transmutación de la fuerza cósmica en potencia del universo como una de las siete “infinitudes de la divinidad”. Y esto es todo lo que podréis conocer al respecto durante vuestro ascenso en el universo local.
\vs p042 2:23 \pc A pesar de nuestra incapacidad para comprender enteramente el origen, la naturaleza y las transmutaciones de la fuerza cósmica, estamos totalmente familiarizados con todas las fases del comportamiento de la energía emergente desde los tiempos de su respuesta directa e inequívoca a la acción de la gravedad del Paraíso ---aproximadamente desde el momento en que los directores de la potencia de los suprauniversos comienzan a desempeñar su actividad---.
\usection{3. CLASIFICACIÓN DE LA MATERIA}
\vs p042 3:1 Salvo en el universo central, la materia es idéntica en todos los universos. En cuanto a sus propiedades físicas, la materia depende del índice de rotación de sus componentes, del número y tamaño de los elementos rotatorios, de su distancia al cuerpo nuclear o del contenido espacial de la materia, al igual que de la presencia de ciertas fuerzas todavía por descubrir en Urantia.
\vs p042 3:2 En los diversos soles, planetas y cuerpos del espacio existen diez grandes divisiones de la materia:
\vs p042 3:3 \li{1.}La materia ultimatónica: los componentes físicos primigenios de la existencia material, las partículas de energía de la que se forman los electrones.
\vs p042 3:4 \li{2.}La materia subelectrónica: el estadio de explosión y repulsión de los extraordinarios gases solares.
\vs p042 3:5 \li{3.}La materia electrónica (o estadio eléctrico de diferenciación de la materia): electrones, protones y distintos otros elementos que entran a formar parte de la diversa configuración de las agrupaciones de electrones.
\vs p042 3:6 \li{4.}La materia subatómica: materia que existe abundantemente en el interior de los soles calientes.
\vs p042 3:7 \li{5.}Átomos fragmentados: que se encuentran en los soles en enfriamiento y por todo el espacio.
\vs p042 3:8 \li{6.}La materia ionizada: átomos individuales que han perdido sus electrones exteriores (activos químicamente) a causa de la actividad eléctrica, térmica, de los rayos X o por los solventes.
\vs p042 3:9 \li{7.}La materia atómica: el estadio químico de la organización de los elementos, las unidades constituyentes de la materia molecular o visible.
\vs p042 3:10 \li{8.}El estadio molecular de la materia: materia en un estado de materialización relativamente estable tal como existe en Urantia en condiciones ordinarias.
\vs p042 3:11 \li{9.}La materia radioactiva: la tendencia y acción disruptiva de los elementos más pesados en condiciones de calor moderado y de presión gravitatoria atenuada.
\vs p042 3:12 \li{10.}La materia colapsada: la materia relativamente estacionaria que se halla en el interior de los soles fríos o muertos. Esta forma de materia no es realmente estacionaria; todavía tiene cierta actividad ultimatónica e incluso electrónica, pero dichos elementos están muy próximos el uno del otro, y sus índices de rotación están enormemente disminuidos.
\vs p042 3:13 \pc La clasificación de la materia anteriormente expuesta se ha realizado en base a la configuración de esta y no en base a la forma en la que aparece ante los seres creados. Tampoco tiene en cuenta los estadios preemergentes de la energía ni sus materializaciones eternas en el Paraíso y en el universo central.
\usection{4. TRANSMUTACIONES DE LA ENERGÍA Y DE LA MATERIA}
\vs p042 4:1 La luz, el calor, la electricidad, el magnetismo, la acción química, la energía y la materia son ---en su origen, naturaleza y destino--- una misma cosa, junto con otras realidades materiales aún por conocer en Urantia.
\vs p042 4:2 No comprendemos del todo los casi interminables cambios a los que la energía física se ve sometida. En un universo, aparece como luz; en otro, como luz más calor; en otro, toma formas desconocidas en Urantia; en incontables millones de años puede reaparecer bajo la forma de energía eléctrica activa, emergente y expansiva o de potencia magnética; e incluso después puede aparecer de nuevo en otro universo bajo alguna forma de materia variable, atravesando una serie de metamorfosis, con su consiguiente desaparición física externa en algún gran cataclismo cósmico. Y luego, tras indecibles eras y tras deambular de modo interminable por innumerables universos, de nuevo esta misma energía puede resurgir y modificar muchas veces su forma y potencial; de este modo, continúan estas transformaciones era tras era y por incontables regiones espaciales. Así pues, la materia prosigue su recorrido, sufriendo las transmutaciones del tiempo pero girando siempre fiel alrededor del círculo de la eternidad; y aunque por mucho tiempo se le impida retornar a su fuente, siempre es susceptible a esta, y siempre avanza por la ruta ordenada por la Persona Infinita que la envió.
\vs p042 4:3 Los centros de la potencia y sus colaboradores se implican bastante en la labor de transmutar el ultimatón en las vías circulatorias y rotaciones del electrón. Estos singulares seres rigen y componen la potencia al actuar capazmente sobre las unidades básicas de la energía materializada, los ultimatones. Son los regidores de la energía que circula en su estado primitivo. En conjunción con los controladores físicos, son capaces de controlar y dirigir eficazmente la energía incluso después de que se haya transmutado al nivel eléctrico, el llamado estadio electrónico. Pero su radio de acción se restringe sobremanera cuando la energía organizada electrónicamente vira en las circunvoluciones de los sistemas atómicos. En el momento de tal materialización, estas energías se sitúan al completo alcance del poder de atracción de la gravedad lineal.
\vs p042 4:4 La gravedad actúa positivamente en las líneas de fuerza y en los canales de energía de los centros de la potencia y de los controladores físicos, pero estos seres, en el ejercicio de sus atributos de antigravedad, solo se relacionan negativamente con ella.
\vs p042 4:5 Por todo el espacio, el frío y otras influencias entran en acción para organizar de forma creativa a los ultimatones en electrones. El calor es proporcional a la actividad electrónica, mientras que el frío significa simplemente ausencia de calor ---reposo relativo de la energía---, el estatus de la carga de fuerza universal del espacio a condición de que ni la energía emergente ni la materia organizada estén presentes y respondiendo a la gravedad.
\vs p042 4:6 La presencia y acción de la gravedad impiden que aparezca el cero absoluto teórico; en el espacio interestelar no existe la temperatura de cero absoluto. En todo el espacio organizado, hay corrientes de energía, vías circulatorias de la potencia y actuaciones de los ultimatones, así como también energías electrónicas organizadoras, que responden a la gravedad. A efectos prácticos, el espacio no está vacío. Incluso la atmósfera de Urantia se hace cada vez menos densa hasta que a unos cinco mil kilómetros comienza a desvanecerse en la materia espacial media de esta sección del universo. El espacio más vacío que se conoce en Nebadón contiene unos cien ultimatones ---el equivalente de un electrón--- por cada 16,38 centímetro cúbico. En la práctica, esta escasez de materia se considera como espacio vacío.
\vs p042 4:7 En los ámbitos de la evolución de la energía y de la materia, la temperatura ---el calor y el frío--- constituye solo un elemento secundario respecto de la gravedad. Los ultimatones obedecen sin resistencia a las temperaturas extremas. Las bajas temperaturas favorecen ciertas formas de construcción electrónica y ensamblaje atómico, mientras que las altas temperaturas facilitan toda clase de fragmentación atómica y desintegración de la materia.
\vs p042 4:8 Cuando se someten al calor y a la presión de ciertos estados solares internos, todas las combinaciones de la materia salvo las más básicas pueden llegar a fragmentarse. El calor puede, por consiguiente, vencer, en gran manera, a la estabilidad gravitatoria. Si bien, no existe calor ni presión solar conocidos que puedan convertir de nuevo a los ultimatones en energía poderosa.
\vs p042 4:9 Los soles abrasadores pueden transformar la materia en diversas formas de energía, pero los mundos oscuros y el conjunto del espacio exterior pueden retrasar la acción de los electrones y de los ultimatones hasta el punto de convertir a estas energías en la materia de los universos. Ciertas combinaciones de electrones de naturaleza similar, al igual que muchas de las combinaciones elementales de la materia nuclear, se forman en las temperaturas extremadamente bajas del espacio abierto, aumentando más tarde al agregarse a concentraciones mayores de energía en proceso de materialización.
\vs p042 4:10 Durante toda esta interminable metamorfosis de la energía y de la materia debemos tener en cuenta el efecto de la presión gravitatoria y el comportamiento antigravitatorio de las energías ultimatónicas bajo ciertas condiciones de temperatura, velocidad y rotación. La temperatura, las corrientes de energía, la distancia y la presencia de los organizadores vivos de la fuerza y de los directores de la potencia repercuten igualmente sobre todos los fenómenos de transmutación de la energía y de la materia.
\vs p042 4:11 El aumento de la masa en la materia es equivalente al aumento de la energía dividido por el cuadrado de la velocidad de la luz. En un sentido dinámico, el trabajo que la materia en reposo puede llevar a cabo es equivalente a la energía consumida en reunir sus partes desde el Paraíso, menos la resistencia de las fuerzas vencidas durante el tránsito y la mutua atracción ejercida por las partes de la materia.
\vs p042 4:12 \pc La existencia de formas preelectrónicas de la materia se indica por los dos pesos atómicos del plomo. En su formación inicial, el plomo pesa levemente más que el que se origina por la desintegración del uranio mediante las emanaciones de radio; y esta diferencia de peso atómico representa la pérdida real de energía al producirse la fragmentación atómica.
\vs p042 4:13 \pc La integridad relativa de la materia está asegurada por el hecho de que puede absorber o liberar la energía solamente en esas cantidades exactas que los científicos de Urantia han denominado “\bibemph{quanta}”. En los mundos materiales, esta inteligente disposición sirve para mantener a los universos en funcionamiento.
\vs p042 4:14 La magnitud de energía que se absorbe o emite cuando varían las posiciones de los electrones o de otros elementos es siempre un “\bibemph{quantum}” o algún múltiplo de este, pero el comportamiento de las vibraciones u ondulaciones de dichas unidades de energía se determina enteramente por las dimensiones de las estructuras materiales implicadas. Las ondulaciones de esta energía tienen un diámetro 860 veces mayor que el diámetro de los ultimatones, electrones, átomos u otras unidades que actúan de este modo. La interminable confusión que acompaña a la observación del comportamiento de la mecánica cuántica ondulatoria se debe a la sobreimposición de las ondas de energía: dos crestas pueden combinarse para formar una cresta de doble altura, mientras que al encontrarse una cresta y una depresión se anulan mutuamente.
\usection{5. LAS MANIFESTACIONES DE LA ENERGÍA ONDULATORIA}
\vs p042 5:1 En el suprauniverso de Orvontón hay cien octavas de energía ondulatoria. De estos cien grupos de manifestaciones energéticas, en Urantia se conocen, en su totalidad o en parte, sesenta y cuatro. En la escala del suprauniverso, los rayos del Sol constituyen cuatro octavas: los rayos visibles abarcan una sola octava, la número cuarenta y seis de esta serie; el próximo es el grupo de rayos ultravioletas, mientras que diez octavas más arriba están los rayos X, seguidos por los rayos gamma del radio. Treinta y dos octavas por encima de la luz visible del Sol están los rayos de energía del espacio exterior, que con tanta frecuencia se mezclan con partículas diminutas de materia altamente energizadas, a ellos vinculadas. Justo por debajo de la luz solar visible, se hallan los rayos infrarrojos, y treinta octavas más abajo está el grupo de radiotransmisión.
\vs p042 5:2 \pc Desde el punto de vista del conocimiento científico de la Urantia del siglo XX, estas manifestaciones de la energía ondulatoria se pueden clasificar en los diez grupos siguientes:
\vs p042 5:3 \li{1.}\bibemph{Los rayos infraultimatónicos} ---o rotaciones limítrofes de ultimatones al empezar a adoptar una forma definitiva---. Se trata del primer estadio de la energía emergente en el que los fenómenos ondulatorios se pueden detectar y medir.
\vs p042 5:4 \li{2.}\bibemph{Los rayos ultimatónicos}. La acumulación de la energía en las diminutas esferas de los ultimatones ocasiona vibraciones en el contenido del espacio que son perceptibles y mensurables. Y mucho antes de que los físicos descubran el ultimatón, harán sin duda observaciones de estos rayos que se precipitan sobre Urantia. Estos rayos, cortos y potentes, son el resultado de la actividad inicial de los ultimatones cuando su velocidad disminuye hasta ese punto en el que hacen un viraje hacia la organización electrónica de la materia. A medida que los ultimatones se combinan para formar electrones, se produce una condensación y un consiguiente almacenamiento de energía.
\vs p042 5:5 \li{3.}\bibemph{Los rayos espaciales cortos}. Tienen las longitudes de onda más cortas de todas aquellas específicamente electrónicas y conforman la etapa preatómica de esta forma de materia. Para que se produzcan estos rayos espaciales se precisan temperaturas extraordinariamente altas o bajas. Son de dos clases: una que acompaña el nacimiento de los átomos y la otra que es indicativo de la desintegración atómica. Emanan en mayores cantidades desde el plano más denso del suprauniverso, la Vía Láctea, que es también el plano más denso de los universos exteriores.
\vs p042 5:6 \li{4.}\bibemph{El estadio electrónico}. Este estadio de la energía es la base de toda materialización en los siete suprauniversos. Cuando los electrones pasan desde los niveles de energía superiores a los inferiores en su traslación orbital, siempre se emite \bibemph{quanta}. El cambio orbital de los electrones resulta en la eyección o absorción de partículas mensurables, muy precisas y uniformes, de luz\hyp{}energía, mientras que cualquier electrón cede una partícula de luz\hyp{}energía cuando colisiona. Las manifestaciones de la energía ondulatoria contribuyen también al comportamiento de los cuerpos con carga positiva y de otros elementos de la etapa electrónica.
\vs p042 5:7 \li{5.}\bibemph{Los rayos gamma} ---o emanaciones que son características de la disociación espontánea de la materia atómica---. El mejor ejemplo de esta forma de actividad electrónica se encuentra en los fenómenos asociados con la desintegración del radio.
\vs p042 5:8 \li{6.}\bibemph{El grupo de rayos X}. El siguiente paso en la desaceleración del electrón produce las distintas formas de rayos X solares junto con aquellos rayos X que se generan de forma artificial. La carga electrónica crea un campo eléctrico; el movimiento da lugar a una corriente eléctrica; la corriente produce un campo magnético. Cuando un electrón se detiene de repente, la perturbación electromagnética resultante produce el rayo X; el rayo X es \bibemph{esa} perturbación. Los rayos X solares son idénticos a los que se generan de forma mecánica para explorar el interior del cuerpo humano, salvo que son nimiamente más largos.
\vs p042 5:9 \li{7.}\bibemph{Los rayos ultravioletas} o rayos químicos de la luz solar y sus diversas producciones por medios mecánicos.
\vs p042 5:10 \li{8.}\bibemph{La luz blanca:} la totalidad de la luz visible de los soles.
\vs p042 5:11 \li{9.}\bibemph{Los rayos infrarrojos:} la desaceleración de la actividad electrónica hasta acercarse al estadio de calor apreciable.
\vs p042 5:12 \li{10.}\bibemph{Las ondas hertzianas:} energías utilizadas en Urantia para la radiotransmisión.
\vs p042 5:13 \pc De todas estas diez fases del movimiento ondulatorio de la energía, el ojo humano es sensible solamente a una octava, o totalidad de la luz solar ordinaria.
\vs p042 5:14 \pc El llamado éter es simplemente un nombre colectivo utilizado para designar al conjunto de actividades de la fuerza y de la energía que se producen en el espacio. Los ultimatones, los electrones y otras acumulaciones masivas de energía constituyen partículas uniformes de materia y, en su recorrido por el espacio, avanzan en realidad en línea recta. La luz y otras formas de manifestaciones reconocibles de la energía consisten en una sucesión de determinadas partículas de energía, que se mueven en línea recta salvo cuando la gravedad o la intervención de otras fuerzas las modifican. El hecho de que estas secuencias de partículas de energía aparezcan como fenómenos ondulatorios, cuando son objeto de ciertas observaciones se debe a la resistencia de la capa no diferenciada de energía que impregna todo el espacio, al hipotético éter, y a la atracción intergravitatoria entre las distintas partículas de materia que las acompañan. El espaciamiento de los intervalos entre las partículas de materia, junto con la velocidad inicial de los haces de energía, determina el aspecto ondulado de muchas formas de energía\hyp{}materia.
\vs p042 5:15 La excitación del contenido del espacio produce una reacción de tipo ondulatorio al paso de partículas de materia que se mueven con rapidez, tal como el paso de un barco por el agua trae consigo olas de amplitud e intervalo variables.
\vs p042 5:16 El comportamiento de la fuerza primordial da realmente lugar a fenómenos que, en muchos sentidos, son análogos a vuestro supuesto éter. El espacio no está vacío; por todo el espacio, las esferas giran y se sumergen en un inmenso océano desplegado de fuerza\hyp{}energía; el contenido espacial de un átomo tampoco está vacío. No obstante, el éter no existe; y la ausencia misma de este hipotético elemento posibilita que los planetas habitados se puedan librar de caer en el Sol y de que los envolventes electrones logren resistirse a precipitarse en el núcleo atómico.
\usection{6. ULTIMATONES, ELECTRONES Y ÁTOMOS}
\vs p042 6:1 Aunque la carga espacial de la fuerza universal es homogénea e indiferenciada, la organización de la energía evolucionada en materia implica la concentración de la energía en masas específicas con dimensiones determinadas y peso establecido ---la concretización de la respuesta a la gravedad---.
\vs p042 6:2 La gravedad local o lineal se vuelve totalmente operativa al aparecer la organización atómica de la materia. La materia preatómica desarrolla una leve respuesta a la gravedad cuando se la activa mediante rayos X y otras energías similares; si bien, la gravedad lineal no ejerce atracción alguna apreciable sobre las partículas de energía electrónica libres, sin ligación, no cargadas, ni sobre los ultimatones sueltos.
\vs p042 6:3 \pc Los ultimatones operan por atracción mutua, respondiendo solamente a la gravedad circular del Paraíso. Al no responder a la gravedad lineal, se mantienen, pues, en la deriva universal del espacio. Los ultimatones son capaces de acelerar su velocidad de rotación hasta el punto de poseer un efecto parcialmente antigravitatorio, pero no pueden, con independencia de los organizadores de la fuerza o de los directores de la potencia, alcanzar la velocidad crítica y escapar de su pérdida de individualidad, o sea, volver al estadio de energía poderosa. Por naturaleza, los ultimatones cesan su estado físico solo cuando forman parte de la desintegración final de un Sol enfriado y moribundo.
\vs p042 6:4 \pc Los ultimatones, desconocidos en Urantia, antes de cumplir los condicionamientos previos de rotación\hyp{}energía para su organización electrónica, pasan por un periodo de desaceleración que entraña muchos estadios de actividad de orden físico. Realizan tres tipos diferentes de acciones: resistencia mutua a la fuerza cósmica, rotaciones individuales con potencial antigravitatorio y posicionamiento intraelectrónico de los cien ultimatones mutuamente correlacionados.
\vs p042 6:5 La atracción que se ejercen mutuamente mantiene a cien ultimatones unidos en la formación de un electrón; y nunca hay ni más ni menos que cien ultimatones en un electrón típico. La pérdida de uno o más ultimatones merma la configuración característica del electrón, lo que da origen a una de las diez formas modificadas del electrón.
\vs p042 6:6 Los ultimatones no describen órbitas ni giran siguiendo vías circulatorias en el interior de los electrones, pero se despliegan o agrupan conforme a su velocidad de rotación axial, estableciendo de este modo diferentes dimensiones electrónicas. Esta misma velocidad de rotación axial del ultimatón determina igualmente la reacción negativa o positiva de los distintos tipos de unidades electrónicas. La totalidad de la separación y agrupación de la materia electrónica, junto con la diferenciación eléctrica de los cuerpos con carga negativa y positiva de la energía\hyp{}materia, resultan de esta diversidad operativa de la interacción entre los ultimatones que la componen.
\vs p042 6:7 \pc El átomo posee un diámetro algo por encima de 1/39\,370\,000 de centímetro, mientras que un electrón pesa un poco más de 1/2000 del átomo más pequeño, el hidrógeno. El protón, de carga positiva, característico del núcleo atómico, aunque puede que no sea más grande que un electrón, de carga negativa, pesa casi dos mil veces más.
\vs p042 6:8 \pc Si se aumentase la masa de la materia hasta que la masa de un electrón fuese equivalente a la décima parte de una onza, y entonces se ampliase la escala proporcionalmente, el volumen de dicho electrón llegaría a ser tan grande como el de la tierra. Si aumentáramos el volumen de un protón ---mil ochocientas veces más pesado que un electrón--- hasta el tamaño de la cabeza de un alfiler, entonces, comparativamente, esta alcanzaría un diámetro equivalente al de la órbita de la Tierra alrededor del Sol.
\usection{7. LA MATERIA ATÓMICA}
\vs p042 7:1 La configuración de la materia se asemeja a la del sistema solar. En el centro de cualquier diminuto universo de energía existe una porción nuclear de orden material relativamente estable, más o menos inmóvil. Esta unidad central posee la posibilidad de manifestarse de manera triple. Rodeando este centro de energía, en una profusión ilimitada aunque en orbitas fluctuantes, dan vueltas las unidades de energía, comparables de algún modo a los planetas que circundan el Sol en algún conjunto estrellado como vuestro propio sistema solar.
\vs p042 7:2 \pc Dentro del átomo, los electrones giran alrededor del protón central en un espacio que es en comparación casi igual al de los planetas que giran alrededor del Sol en el espacio del sistema solar. Existe la misma distancia relativa, en comparación con el tamaño real, entre el núcleo atómico y la órbita electrónica interna que entre el planeta interno, Mercurio, y vuestro Sol.
\vs p042 7:3 Las rotaciones axiales de los electrones y sus velocidades orbitales alrededor del núcleo atómico sobrepasan la imaginación humana, por no mencionar la velocidad de los ultimatones que los componen. Las partículas positivas de radio se alejan volando hacia el espacio a una velocidad de dieciséis mil kilómetros por segundo, mientras que las partículas negativas alcanzan una velocidad que se aproxima a la de la luz.
\vs p042 7:4 \pc Los universos locales se crean siguiendo el sistema decimal. En un universo doble, hay exactamente cien materializaciones atómicas diferenciables de energía espacial, la máxima configuración posible de la materia en Nebadón. Dichas cien formas de materia constan, en sucesión regular, de entre uno a cien electrones que giran alrededor de un núcleo central relativamente compacto. Es esta acumulación ordenada y constante de distintas energías la que constituye la materia.
\vs p042 7:5 Estos cien elementos no se muestran de modo reconocible en la superficie de un planeta, pero están presentes en algún lugar, lo han estado o están en proceso de evolución. Las condiciones que rodean el origen y la consiguiente evolución de un planeta determinan cuántas de estas cien clases de átomos serán observables. Los átomos más pesados no se hallan en la superficie de muchos mundos. También en Urantia, los elementos conocidos más pesados manifiestan una tendencia a desintegrarse, tal como se ilustra en el comportamiento del radio.
\vs p042 7:6 La estabilidad del átomo depende del número de neutrones eléctricamente inactivos situados en el cuerpo central. El comportamiento químico se deriva enteramente de la actividad de los electrones en su libre rotación.
\vs p042 7:7 \pc Nunca ha sido posible en Orvontón acumular, de forma natural, más de cien electrones orbitales en un sistema atómico. Cuando, de modo artificial, se han añadido ciento un electrones en un campo orbital, el resultado ha sido siempre la desintegración instantánea del protón central junto con la violenta dispersión de los electrones y de las otras energías liberadas.
\vs p042 7:8 \pc Aunque los átomos puedan contener de entre uno a cien electrones orbitales, únicamente los diez electrones externos de los átomos de mayor tamaño giran alrededor del núcleo central de forma diferenciada y discreta, dando vueltas de manera intacta y compacta alrededor de órbitas exactas y definidas. Los treinta electrones más cercanos al centro son difíciles de observar o detectar como corpúsculos separados y organizados. Este mismo índice relativo del comportamiento electrónico con relación a su proximidad al núcleo ocurre en todos los átomos, independientemente del número de electrones que posean. A mayor cercanía al núcleo, menor individualidad de orden electrónico existe. La extensión ondulatoria de la energía de un electrón puede expandirse tanto que puede llegar a ocupar la totalidad de las órbitas atómicas menores; esto es especialmente cierto de los electrones más cercanos al núcleo atómico.
\vs p042 7:9 Los treinta electrones orbitales más internos poseen individualidad, pero sus sistemas energéticos tienden a entremezclarse, extendiéndose de un electrón al otro y casi de órbita en órbita. Los treinta electrones siguientes constituyen la segunda familia, o zona energética, y poseen una mayor individualidad al ser corpúsculos de materia que ejercen un control más completo sobre estos sistemas conjuntos de energía. Los otros treinta que le siguen, correspondientes a la tercera zona energética, están más individualizados y circulan en órbitas más exactas y definidas. Los últimos diez electrones, presentes solamente en los diez elementos más pesados, poseen el distintivo de la independencia y son, por lo tanto, capaces de escapar con mayor o menor libertad del control del núcleo matriz. Con un mínimo de variación de temperatura y presión, los componentes de este cuarto grupo más externo de electrones escapan de la atracción del núcleo central, como queda ilustrado en la desintegración espontánea del uranio y de otros elementos semejantes.
\vs p042 7:10 Los primeros veintisiete átomos, aquellos que contienen de uno a veintisiete electrones orbitales, resultan más fáciles de apreciar que los demás. Del veintiocho en adelante nos encontramos, cada vez más, con la imprevisibilidad de la supuesta presencia del Absoluto Indeterminado. No obstante, parte de esta imprevisibilidad electrónica se debe al diferencial de velocidad de la rotación axial de los ultimatones y a su inexplicable propensión a “amontonarse”. Hay otras fuerzas ---de orden físico, eléctrico, magnético y gravitatorio--- que operan igualmente y traen consigo un comportamiento electrónico variable. Por lo tanto, los átomos se asemejan a las personas en cuanto a su previsibilidad. Los estadísticos pueden formular leyes aplicables a un gran número de átomos o de personas, pero no a un solo átomo o a una sola persona.
\usection{8. LA COHESIÓN DEL ÁTOMO}
\vs p042 8:1 Aunque la gravedad es uno de los diversos factores que mantienen unidos a los minúsculos sistemas de energía atómica dentro de estas unidades físicas elementales y entre ellas mismas, hay también presente una energía fuerte y desconocida, la clave de su constitución básica y de su comportamiento fundamental, una fuerza, todavía por descubrir en Urantia. Se trata de una influencia de rango universal que impregna todo el espacio existente dentro de esta diminuta estructura energética.
\vs p042 8:2 El espacio interelectrónico de un átomo no está vacío. Por todo el átomo se producen manifestaciones ondulatorias, perfectamente sincronizadas con la velocidad de los electrones y las rotaciones de los ultimatones, que activan dicho espacio interelectrónico. Las leyes que conocéis sobre la atracción positiva y negativa no resultan del todo válidas para esta fuerza; su comportamiento, pues, es a veces imprevisible. Esta influencia no identificada parece ser la respuesta del Absoluto Indeterminado en relación a la fuerza espacial.
\vs p042 8:3 \pc Los protones cargados y los neutrones no cargados del núcleo del átomo se mantienen unidos debido a la función alternante del mesotrón, una partícula de materia ciento ochenta veces más pesada que el electrón. Sin este elemento, la carga eléctrica transportada por los protones tendría un efecto desestabilizador en el núcleo atómico.
\vs p042 8:4 Por la constitución de los átomos, ni las fuerzas eléctricas ni las gravitatorias podrían mantener al núcleo unido. La integridad del núcleo se preserva gracias a la acción de reciprocidad y cohesión del mesotrón, que es capaz de integrar partículas cargadas y no cargadas debido a su potencia de fuerza\hyp{}masa de orden superior y por la actividad añadida que hace que los protones y los neutrones cambien constantemente de lugar. El mesotrón consigue que la carga eléctrica de las partículas nucleares se proyecte incesantemente de un lado a otro entre los protones y los neutrones. En una infinitésima parte de un segundo, una partícula nuclear es un protón cargado y, en la siguiente, un neutrón no cargado. Y estas alternancias de la condición energética son tan increíblemente rápidas que se despoja a la carga eléctrica de cualquier posibilidad de actuar como fuerza desestabilizadora. De este modo, el mesotrón obra como un “portador de energía” que contribuye sobremanera a la estabilidad nuclear del átomo.
\vs p042 8:5 La presencia y la acción del mesotrón dan también explicación a otro misterio atómico. Cuando los átomos tienen un comportamiento radioactivo, emiten mucha más energía de la que cabría esperar. Este exceso de radiación se deriva de la desintegración del mesotrón, o “portador de energía”, que, de ese modo, se convierte en un simple electrón. En su desintegración, el mesotrón emite también ciertas pequeñas partículas no cargadas.
\vs p042 8:6 La presencia del mesotrón explica ciertas propiedades cohesivas del núcleo atómico, pero no justifica la cohesión entre protones ni la adhesión entre neutrones. La paradójica fuerza fuerte que da integridad y cohesión al átomo es una forma de energía todavía no descubierta en Urantia.
\vs p042 8:7 Estos mesotrones se encuentran con abundancia en los rayos espaciales que, de forma tan incesante, inciden sobre vuestro planeta.
\usection{9. LA FILOSOFÍA NATURAL}
\vs p042 9:1 No solo la religión es dogmática, también la filosofía natural tiende al dogmatismo. Cuando un célebre maestro religioso llegó a la conclusión de que el número siete era esencial en la naturaleza porque hay siete orificios en la cabeza humana, si hubiese conocido más química, podría haber argumentado dicha opinión basándose en un fenómeno real del mundo físico. En todos los universos físicos del tiempo y del espacio, a pesar de que la energía se constituya de forma universal siguiendo el sistema decimal, existe un recordatorio siempre presente y real de que la configuración electrónica de la premateria tiene un orden séptuplo.
\vs p042 9:2 El número siete es clave en el universo central y en el sistema espiritual de transmisión de los rasgos esenciales del carácter, pero el número diez, el sistema decimal, es intrínseco a la energía, la materia y a la creación material. No obstante, el mundo del átomo sí despliega ciertos caracteres que recurren periódicamente en grupos de siete ---un distintivo de nacimiento que porta este mundo material y que indica su remoto origen espiritual---.
\vs p042 9:3 Esta persistente constitución creativa de carácter séptuplo se presenta en los reinos de la química en forma de propiedades físicas y químicas semejantes, que recurren en periodos discriminados de siete cuando los elementos básicos se disponen por orden de sus pesos atómicos. Cuando los elementos químicos de Urantia se ordenan así en hileras siguiendo esta disposición, cualquier cualidad o propiedad dada tiende a recurrir de siete en siete. Este cambio periódico en orden de siete se repite de forma decreciente y con variaciones en toda la tabla química, siendo observable de forma más acusada en las agrupaciones atómicas primeras o más ligeras. Empezando con uno cualquiera de los elementos, una vez anotada alguna de sus propiedades, dicha cualidad cambiará en los seis elementos consecutivos, pero al llegar al octavo suele reaparecer, esto es, el octavo elemento químicamente activo se parece al primero, el noveno al segundo y así sucesivamente. Este hecho del mundo físico muestra, de modo inconfundible, la constitución séptupla de la energía ancestral y es indicativo de la realidad fundamental que subyace a la diversidad séptupla de las creaciones del tiempo y el espacio. El hombre debería también tomar nota de que existen siete colores en el espectro natural.
\vs p042 9:4 Pero no todas las hipótesis de la filosofía natural tienen validez; el éter hipotético, por ejemplo, no representa sino un intento ingenioso del hombre por vertebrar su desconocimiento de los fenómenos espaciales. La filosofía del universo no se puede basar en las observaciones de la llamada ciencia. Si no presenciase la metamorfosis, el científico tendería a negar la posibilidad de que una oruga se transforme en mariposa.
\vs p042 9:5 La estabilidad física en conjunción con la plasticidad biológica está presente en la naturaleza solamente gracias a la casi infinita sabiduría de los arquitectos mayores de la creación. Nada inferior a esta sabiduría trascendental podría jamás diseñar unidades de materia que son al mismo tiempo tan estables y tan eficientemente adaptables.
\usection{10. LOS SISTEMAS DE ENERGÍA UNIVERSALES NO ESPIRITUALES (LOS SISTEMAS DE LA MENTE MATERIAL)}
\vs p042 10:1 La extensión sin fin de la realidad cósmica relativa, desde la absolutidad de la monota del Paraíso hasta la absolutidad de la potencia del espacio, parece indicar que existe cierta evolución en la relación entre las realidades no espirituales de la Primera Fuente y Centro ---aquellas realidades que están ocultas en la potencia del espacio, que se revelan en la monota y se manifiestan de forma provisional en los niveles cósmicos intermedios---. Este ciclo eterno de energía, al encauzarse en el Padre de los universos, es absoluto y, siendo absoluto, no es susceptible de expansión ni como hecho ni como valor; sin embargo, el Padre Primigenio incluso en este momento ---como siempre--- se realiza a sí mismo en un escenario de significados espacio temporales en continuo despliegue, y de contenidos espacio temporales transcendidos, en un escenario de relaciones cambiantes en donde la energía\hyp{}materia se somete a la gradual acción directiva del espíritu vivo y divino a través del afán experiencial de la mente personal viva.
\vs p042 10:2 Las energías universales no espirituales se vinculan de nuevo a los sistemas vivos de las mentes no creadoras en distintos niveles, algunos de los cuales se pueden describir de la siguiente manera:
\vs p042 10:3 \li{1.}\bibemph{La mente con anterioridad a los espíritus asistentes}. Este nivel de la mente es no experiencial y en los mundos habitados están bajo el ministerio de los controladores físicos mayores. Se trata de la mente mecánica, el intelecto no educable de las formas más primitivas de la vida material, pero esta mente no educable opera en muchos niveles además del de la vida planetaria primitiva.
\vs p042 10:4 \li{2.}\bibemph{La mente atendida por los espíritus asistentes}. Se corresponde al ministerio del espíritu materno del universo local que obra a través de sus siete espíritus asistentes en un nivel de la mente material, susceptible de ser educada (nivel no mecánico). En este nivel la mente material adquiere experiencias como intelecto subhumano (animal) a través de los cinco primero asistentes de la mente; como intelecto humano (moral), en los siete asistentes; como intelecto sobrehumano (seres intermedios) en los últimos dos ayudantes.
\vs p042 10:5 \li{3.}\bibemph{Las mentes morontiales en evolución} ---o conciencia en expansión de los seres personales evolutivos en su andadura ascendente en el universo local---. Es el don del espíritu materno del universo local en conjunción con el hijo creador. Este nivel de la mente conlleva la configuración de un vehículo vital de tipo morontial, una síntesis de lo material y de lo espiritual, que se lleva a efecto por los supervisores de la potencia morontial del universo local. La mente morontial opera de forma diferenciada en respuesta a los 570 niveles de vida morontial, revelando una creciente capacidad de vinculación con la mente cósmica en los niveles de realización de carácter superior. Este es el curso evolutivo de las criaturas mortales, pero el hijo del universo y el espíritu del universo también otorgan la mente de orden no morontial a los hijos no morontiales de las creaciones locales.
\vs p042 10:6 \pc \bibemph{La mente cósmica}. Se trata de la mente séptupla diversificada del tiempo y del espacio, cada faceta de la cual cuenta con el ministerio de uno de los siete espíritus mayores en cada uno de los siete suprauniversos. La mente cósmica abarca todos los niveles de la mente finita y se coordina de forma experiencial con los niveles evolutivos en cuanto deidad de la Mente Suprema y de forma trascendental con los niveles existenciales de la mente absoluta ---las vías circulatorias directas del Actor Conjunto---.
\vs p042 10:7 En el Paraíso, la mente es absoluta; en Havona, es absonita; en Orvontón, es finita. La mente siempre implica la presencia\hyp{}actividad de un ministerio vivo además de diversos sistemas de energía, y esto es cierto de todos los niveles y de todos los tipos de mente. Pero, más allá de la mente cósmica resulta, cada vez más difícil representar las relaciones de la mente con la energía no espiritual. La mente de Havona es subabsoluta pero supraevolutiva; al ser existencial\hyp{}experiencial, está más cerca de lo absonito que cualquier otro concepto que se os haya revelado. La mente del Paraíso sobrepasa la comprensión humana; es existencial, no espacial y no temporal. Sin embargo, todos estos niveles mentales están eclipsados por la presencia universal del Actor Conjunto ---por la atracción de la gravedad mental del Dios de la mente del Paraíso---.
\usection{11. LOS MECANISMOS DEL UNIVERSO}
\vs p042 11:1 En la valoración y el reconocimiento de la mente se debe tener presente que el universo no es ni mecánico ni mágico; es una creación de la mente y un mecanismo que se rige mediante leyes. Sin bien, aunque en su aplicación práctica, las leyes de la naturaleza operan en lo que parece ser el doble ámbito de lo físico y lo espiritual, en la realidad ambos niveles son uno solo. La Primera Fuente y Centro es la causa primordial de toda materialización y, al mismo tiempo, el Padre primero y final de todos los espíritus. El Padre del Paraíso se manifiesta de forma personal en los universos externos a Havona solo como energía pura y espíritu puro ---como los modeladores del pensamiento y otras fracciones similares de la Deidad---.
\vs p042 11:2 \pc Los mecanismos no gobiernan en absoluto toda la creación; el universo de los universos \bibemph{en su totalidad} está planificado por una mente, creado por una mente y regido por una mente. Pero el componente divino que dirige el universo de los universos es demasiado perfecto para que, mediante los métodos científicos de la mente finita del hombre, se pueda apreciar rastro alguno de la preeminencia de la mente infinita. Esta mente creadora, rectora y sostenedora no es ni mente material ni mente creatural; es mente espiritual que obra en y desde los niveles creadores de la realidad divina.
\vs p042 11:3 La facultad de distinguir y descubrir una mente en los mecanismos del universo depende por completo de las dotes, el alcance y la capacidad de la mente investigadora que se implique en esta tarea de observación. La mente del espacio\hyp{}tiempo, configurada a partir de las energías del tiempo y del espacio, está sujeta al mecanismo de acción de estos dos factores.
\vs p042 11:4 \pc En el espacio\hyp{}tiempo, el movimiento y la gravitación del universo constituyen facetas gemelas del mecanismo impersonal del universo de los universos. El grado de respuesta a la gravedad del espíritu, de la mente y de la materia es mayormente independiente del tiempo, pero solo los niveles espirituales verdaderos de la realidad lo son también del espacio (son no espaciales). Los niveles mentales superiores del universo ---los niveles mentales espirituales--- pueden ser igualmente no espaciales, pero los de la mente material, como la mente humana, son susceptibles a las interacciones de la gravitación del universo, y pierden esta respuesta solamente en proporción a su grado de identificación con el espíritu. Los niveles de la realidad espiritual se determinan por su contenido espiritual, y la espiritualidad en el tiempo y el espacio es inversamente proporcional a su respuesta a la gravedad lineal.
\vs p042 11:5 La respuesta a la gravedad lineal mide de modo cuantitativo la energía no espiritual. Toda masa ---o energía organizada--- está sujeta a esta atracción de la gravedad excepto cuando el movimiento y la acción de la mente actúan sobre ella. La gravedad lineal es la fuerza cohesiva de corto alcance del macrocosmos, algo así como las fuerzas de cohesión intraatómica constituyen las fuerzas de corto alcance del microcosmos. La energía física materializada, organizada como la conocida materia, no puede atravesar el espacio sin tener un efecto de respuesta en la gravedad. Aunque dicha respuesta es directamente proporcional a la masa, esta sufre tal modificación por el espacio que media que el resultado final no es más que una somera aproximación cuando se expresa como inversamente proporcional al cuadrado de la distancia. El espacio acaba por imperar sobre la gravitación lineal gracias a la presencia en él de las influencias antigravitatorias de numerosas fuerzas supramateriales que operan para neutralizar la acción de la gravedad y toda reacción a ella.
\vs p042 11:6 \pc Los mecanismos cósmicos, extremadamente complejos y sumamente automatizados, siempre tienen tendencia a ocultar la presencia de una mente latente, inventiva o creativa, a todas y cada una de las inteligencias que están muy por debajo de los niveles del universo en los que naturalmente se desenvuelven estos mismos mecanismos. Por ello, resulta inevitable que los mecanismos superiores del universo hayan de parecer como desprovistos de mente para los órdenes inferiores de criaturas. La única excepción posible a esta afirmación sería suponer la existencia de algún tipo de mente subyacente al asombroso fenómeno de un universo que \bibemph{aparentemente se mantiene por sí mismo} ---pero esto es una cuestión más bien filosófica que de la experiencia real---.
\vs p042 11:7 No existen mecanismos inmutables cuando es una mente la que coordina el universo. Hay un fenómeno universal que hace que evolución progresiva y automantenimiento cósmico estén enlazados. La capacidad evolutiva del universo es inagotable en la infinitud de la espontaneidad. El avance hacia una unidad armoniosa, hacia una creciente síntesis experiencial que se superpone a una cada vez mayor complejidad de relaciones, solo se puede llevar a efecto por una mente resoluta y directiva.
\vs p042 11:8 Cuanto más elevada sea la mente del universo en su vinculación con cualquier fenómeno de este, más difícil les resultará descubrirla a mentes de orden inferior. Y puesto que la mente que gobierna el mecanismo del universo es espíritu\hyp{}mente creativo (la mente misma del Infinito), a las mentes de menor rango no les resulta posible ni descubrirla ni percibirla, mucho menos le será factible hacerlo a la mente \bibemph{más humilde} de todas, a la mente humana. La mente animal evolutiva, aunque busca por naturaleza a Dios, no es por sí misma ni en sí misma intrínsecamente conocedora de Dios.
\usection{12. MODELO Y FORMA: EL PREDOMINIO DE LA MENTE}
\vs p042 12:1 La evolución de los mecanismos del universo supone e indica la presencia y el predominio no visible de la mente creativa. La facultad que posee el intelecto mortal para concebir, diseñar y crear mecanismos automáticos demuestra que las cualidades superiores, creativas y resolutivas de la mente del hombre tienen una predominante influencia en el planeta. La mente siempre tiene tendencia a:
\vs p042 12:2 \li{1.}Crear mecanismos materiales.
\vs p042 12:3 \li{2.}Descubrir misterios ocultos.
\vs p042 12:4 \li{3.}Explorar circunstancias remotas.
\vs p042 12:5 \li{4.}Construir sistemas mentales.
\vs p042 12:6 \li{5.}Alcanzar objetivos de sabiduría.
\vs p042 12:7 \li{6.}Conseguir niveles espirituales.
\vs p042 12:8 \li{7.}Lograr los destinos divinos: supremo, último y absoluto.
\vs p042 12:9 \pc La mente es siempre creativa. La mente, de la que están dotados los seres, ya sea animal, mortal, morontial, ascendente espiritual o de un finalizador, es siempre apta para dar origen a un cuerpo adecuado para servir los propósitos de la identidad de la criatura viva. Pero el fenómeno de la presencia del ser personal o del modelo de una identidad, como tal, no es una manifestación de la energía, ni física ni mental ni espiritual. La forma personal es el elemento \bibemph{configurativo} del ser vivo; supone la \bibemph{organización} de las energías, y esto, más la vida y el movimiento, es el \bibemph{mecanismo} de la existencia creatural.
\vs p042 12:10 Incluso los seres espirituales tienen formas, y estas formas espirituales (modelos) son reales. Incluso los órdenes más elevados de seres personales espirituales tienen formas ---presencias personales en todos los sentidos análogas a los cuerpos mortales de Urantia---. Casi todos los seres que concurren en los siete suprauniversos poseen formas. Pero existen algunas excepciones a esta regla general: los modeladores del pensamiento parecen no disponer de una forma hasta después de fusionarse con las almas supervivientes de sus compañeros mortales. Los mensajeros solitarios, los espíritus inspirados de la Trinidad, los auxiliares personales del Espíritu Infinito, los mensajeros de la gravedad, los archivistas trascendentales y algunos otros más tampoco tienen una forma perceptible. Pero estos son algunos pocos casos de carácter excepcional; la inmensa mayoría posee genuinas formas personales, formas que caracterizan individualmente a los seres, y que son reconocibles y personalmente diferenciables.
\vs p042 12:11 La conjunción de la mente cósmica y del ministerio de los espíritus asistentes de la mente trae consigo el tabernáculo físico adecuado para el ser humano en evolución. Asimismo, la mente morontial particulariza la forma morontial para todos los supervivientes mortales. Al igual que el cuerpo mortal es personal y característico de cada ser humano, la forma morontial será, del mismo modo, sumamente particularizada y adecuadamente característica de la mente creativa que lo rige. Dos formas morontiales no tienen más parecido entre ellas que dos cuerpos, cualesquiera que sean. Los supervisores de la potencia morontial auspician, y los serafines acompañantes proporcionan, el material morontial indiferenciado a partir del cual puede la vida morontial comenzar su actividad. Y tras la vida morontial, se descubrirá que las formas espirituales son igualmente diferentes, personales y características de sus respectivos moradores de espíritu\hyp{}mente.
\vs p042 12:12 \pc En el mundo material vosotros creéis que el cuerpo tiene un espíritu, pero nosotros consideramos que es el espíritu el que tiene un cuerpo. Los ojos materiales son en verdad las ventanas del alma nacida del espíritu. El espíritu es el arquitecto; la mente, el constructor; el cuerpo, el edificio material.
\vs p042 12:13 \pc En los universos fenoménicos, las energías físicas, espirituales y mentales, como tales y en sus estados puros, no interactúan por completo como realidades. En el Paraíso, las tres energías son coiguales, en Havona se coordinan, mientras que en los niveles existentes en el universo, en los que se desarrolla una actividad de orden finito, se han de hallar todos los rangos en cuanto a la preeminencia de lo material, lo mental y lo espiritual. En el espacio y el tiempo, en situaciones no personales, la energía física parece predominar, pero también parece que cuanto más se aproxima la actividad del espíritu\hyp{}mente a la divinidad de propósito y a la supremacía de acción, más preeminente se vuelve la faceta espiritual de la energía y, en su nivel último, el espíritu\hyp{}mente puede adquirir casi una total preponderancia. En el nivel absoluto, el espíritu es sin duda soberano. Y a partir de ahí hacia el exterior, a través de los reinos del tiempo y del espacio, siempre y cuando haya presente una realidad espiritual divina, dondequiera que obre un espíritu\hyp{}mente genuino, existirá siempre una tendencia a crearse un equivalente material o físico de esa realidad espiritual.
\vs p042 12:14 El espíritu es la realidad creativa; su equivalente físico es el reflejo en el espacio\hyp{}tiempo de la realidad espiritual, la consecuencia física de la acción creativa del espíritu\hyp{}mente.
\vs p042 12:15 La mente predomina con carácter universal sobre la materia, al igual que es, a su vez, sensible a la acción directiva última del espíritu. Y en el hombre mortal, solo esa mente que libremente se rinde a la dirección del espíritu puede tener la esperanza de sobrevivir a la existencia material del espacio\hyp{}tiempo como hijo inmortal del mundo espiritual eterno del Supremo, del Último y del Absoluto: del Infinito.
\vsetoff
\vs p042 12:16 [Exposición, a instancias de Gabriel, de un mensajero poderoso de servicio en Nebadón.]
