\upaper{177}{Miércoles, día de descanso}
\author{Comisión de seres intermedios}
\vs p177 0:1 Cuando no se sentían con el apremio de enseñar a la gente, Jesús y sus apóstoles tenían la costumbre de descansar los miércoles de sus tareas. Este particular miércoles desayunaron algo más tarde de lo habitual; en el campamento había un silencio que resultaba premonitorio y no se habló mucho durante la primera mitad de esta temprana comida. Finalmente, Jesús dijo: “Deseo que hoy reposéis. Tomaos vuestro tiempo en pensar en todo lo que ha sucedido desde que llegamos a Jerusalén y en meditar sobre lo que está por llegar, de lo que ya os hablé claramente. Aseguraos de que la verdad more en vuestras vidas y de crecer diariamente en la gracia”.
\vs p177 0:2 Después del desayuno, el Maestro informó a Andrés que tenía la intención de ausentarse durante el día y le indicó que se les permitiera a los apóstoles pasar el tiempo como gustasen, exceptuando que, bajo ninguna circunstancia, cruzaran las puertas de Jerusalén.
\vs p177 0:3 Cuando Jesús estaba listo para ir a las colinas, él solo, David Zebedeo lo abordó, diciendo: “Bien sabes, Maestro, que los fariseos y dirigentes judíos desean acabar contigo, y aún así te dispones a ir a las colinas sin que nadie te acompañe. Sería una imprudencia hacerlo; enviaré, pues, a tres hombres contigo bien preparados para cerciorarnos de que no te ocurra nada malo”. Jesús miró a los tres galileos, fornidos y bien armados, y dijo a David: “Tus intenciones son buenas, pero te equivocas; no entiendes que el Hijo del Hombre no necesita a nadie que lo defienda. Ningún hombre pondrá sus manos sobre mí hasta la hora en la que esté listo para dar mi vida según la voluntad de mi Padre. Estos hombres no me acompañarán. Deseo ir solo, para poder estar en comunión con el Padre”.
\vs p177 0:4 Al oír estas palabras, David y sus guardias armados se retiraron; pero cuando Jesús partía en solitario, Juan Marcos se presentó con una pequeña cesta conteniendo alimentos y agua y sugirió que, si el Maestro tenía la intención de estar todo el día fuera, puede que sintiera hambre. El Maestro sonrió a Juan y extendió su brazo para agarrar la cesta.
\usection{1. UN DÍA CON DIOS, A SOLAS}
\vs p177 1:1 Cuando Jesús estaba a punto de tomar la cesta del almuerzo de las manos de Juan, el joven se aventuró a decir: “Pero, Maestro, puede que dejes la cesta en el suelo porque quieras apartarte para orar y te vayas sin ella. Además, si voy contigo para llevarte el almuerzo, podrás orar con más libertad y, ten por cierto que guardaré silencio. No haré ninguna pregunta y me quedaré con la cesta mientras te alejas”.
\vs p177 1:2 Mientras decía aquellas palabras, cuya temeridad dejó atónitos a quienes estaban cerca y las oyeron, Juan continuaba atrevidamente sosteniendo la cesta. Y allí estaban, pues, Juan y Jesús sujetando la cesta. Al momento, el Maestro la soltó y mirando abajo, hacia el muchacho, dijo: “Puesto que deseas venir conmigo con todo tu corazón, no se te negará. Nos marcharemos los dos solos y tendremos una buena charla. Podrás hacerme cualquier pregunta que te brote del corazón, y nos confortaremos y consolaremos el uno al otro. Empieza tú llevando el almuerzo y, cuando te canses, yo te ayudaré. Sígueme pues”.
\vs p177 1:3 Jesús no regresó al campamento hasta después de la puesta del sol. El Maestro pasó este último día de calma en la tierra con este joven, hambriento de la verdad, en conversación con su Padre del Paraíso. Este acontecimiento se conoció en las alturas como “el día en el que un joven estuvo con Dios en las colinas”. Y tal hecho ilustra, para siempre, el deseo del creador de confraternizar con sus criaturas. Hasta un muchacho, si el deseo de su corazón es realmente supremo, puede recibir la atención y disfrutar de la amorosa compañía del Dios de un universo, tener realmente la inolvidable dicha de estar a solas con Dios en las colinas y durante todo un día entero. Aquella fue la excepcional experiencia vivida por Juan Marcos ese miércoles en las colinas de Judea.
\vs p177 1:4 Jesús conversó extensamente con Juan Marcos, comentando abiertamente los asuntos de este mundo y del próximo. Juan le contó a Jesús lo mucho que lamentaba no haber tenido la edad suficiente para ser uno de los apóstoles, y le manifestó su gran agradecimiento porque se le había permitido seguirlos desde que dieron su primera predicación junto al vado del Jordán cerca de Jericó, salvo en su viaje a Fenicia. Jesús previno al joven de que no se descorazonara por los acontecimientos que estaban al llegar y le aseguró que viviría para convertirse en un poderoso mensajero del reino.
\vs p177 1:5 A Juan Marcos lo emocionaba el recuerdo de aquel día en el que estuvo con Jesús en las colinas, pero nunca olvidó la advertencia final que le hizo el Maestro, ya a punto de regresar al campamento de Getsemaní, cuando le dijo: “Bien, Juan, hemos tenido una buena charla, un verdadero día de descanso, pero mira que no le cuentes a nadie las cosas que te he dicho”. Y Juan Marcos nunca reveló nada de lo que había acontecido ese día que pasó con Jesús en las colinas.
\vs p177 1:6 Durante las pocas horas que le restaban de vida a Jesús en la tierra, Juan Marcos nunca perdería por mucho tiempo de vista al Maestro. El muchacho estaba a poca distancia de él; solo dormía cuando Jesús dormía.
\usection{2. TEMPRANA VIDA EN FAMILIA}
\vs p177 2:1 Aquel día, en el transcurso de su charla con Juan Marcos, Jesús pasó un tiempo considerable comparando las experiencias vividas por ambos en su primera infancia y en los últimos años de la niñez. Aunque los padres de Juan poseían más posesiones materiales de las que habían tenido los padres de Jesús, había experiencias juveniles muy similares entre ambos. Jesús le comentó a Juan muchas cosas que le ayudaron a comprender mejor a sus padres y a otros miembros de su familia. Cuando el muchacho le preguntó al Maestro cómo podía saber que él se convertiría en un “poderoso mensajero del reino”, Jesús le dijo:
\vs p177 2:2 “Sé que demostrarás tu lealtad al evangelio del reino, porque actualmente puedo contar con tu fe y tu amor, una cualidades basadas en la formación que tan tempranamente has recibido en tu propia casa. Eres fruto de un hogar en el que los padres se profesan un afecto sincero, por lo que no te han dado un cariño tan excesivo como para tener un concepto exageradamente alto de ti mismo. Ni tu persona ha sufrido distorsiones como consecuencia de las maniobras, faltas de amor, de unos padres enfrentados, cada cual queriendo ganarse tu confianza y lealtad. Has gozado de ese amor paternal que garantiza el hecho de tener una encomiable confianza en uno mismo y que favorece unos sentimientos normales de seguridad. Pero has sido también afortunado porque tus padres eran sensatos a la vez que cariñosos; fue su sabiduría la que los llevó a negarte la mayoría de los caprichos y los muchos lujos que el dinero puede comprar; te enviaron a la escuela de la sinagoga con los compañeros de juegos de tu vecindario y, asimismo, te animaron a aprender a vivir en este mundo de modo que tuvieras tus propias experiencias. Con tu joven amigo Amós, llegaste al Jordán, donde nosotros predicábamos y los discípulos de Juan bautizaban. Los dos os queríais venir con nosotros. Cuando regresasteis a Jerusalén, tus padres dieron su consentimiento, pero no así los de Amós; amaban tanto a su hijo que le impidieron tener la dicha que has experimentado tú y que incluso hoy mismo estás viviendo. Si se hubiese fugado de casa para unirse a nosotros, podría haber herido el amor y sacrificado la lealtad debida a sus padres. Incluso si tal forma de proceder hubiera sido acertada, habría sido un terrible precio a pagar a cambio de adquirir experiencia, independencia y libertad. Unos padres juiciosos, como los tuyos, procuran que sus hijos no se vean obligados a dañar el amor ni a reprimir su lealtad para desarrollar su independencia y disfrutar de una estimulante libertad al alcanzar tu edad.
\vs p177 2:3 “El amor, Juan, es la realidad suprema del universo cuando son seres omnisapientes quienes lo profesan, pero es un atributo peligroso y a menudo raya en el egoísmo si se manifiesta en la experiencia de los padres mortales. Cuando contraigas matrimonio y tengas que criar a tus propios hijos, asegúrate de que tu amor se rija por la sabiduría y se guíe por la inteligencia.
\vs p177 2:4 “Tu joven amigo Amós cree en este evangelio del reino tanto como tú, pero no puedo enteramente confiar en él; no estoy seguro de lo que hará en los años venideros. Su vida temprana en el hogar no ha sido del tipo que diera de sí una persona totalmente digna de confianza. Amós se parece demasiado a uno de los apóstoles, que no pudo tener en su hogar una formación ordinaria, con amor y sensatez. Serás feliz y leal en toda tu vida adulta, porque pasaste tus primeros ocho años en un hogar normal y bien estructurado. Posees un carácter fuerte y bien equilibrado porque te educaste en un entorno en el que prevalecía el cariño y reinaba la sabiduría. Este tipo de formación durante la niñez genera un tipo de lealtad que me asegura que seguirás con la manera de proceder que comenzaste”.
\vs p177 2:5 Durante más de una hora, Jesús y Juan prosiguieron esta conversación sobre la vida en el hogar. El Maestro continuó comentando a Juan la dependencia total del niño a sus padres, en correlación con la vida en el hogar, para adquirir sus tempranos conceptos sobre las cosas intelectuales, sociales, morales e incluso espirituales; para el niño pequeño, la familia representa todo lo que puede en principio conocer respecto a las relaciones humanas o divinas. El niño debe derivar sus primeras nociones del universo de los cuidados de su madre; y es completamente dependiente de su padre terrenal para albergar sus primeras ideas sobre el Padre celestial. Estas relaciones sociales y espirituales del hogar condicionan la vida mental y emocional temprana del niño, haciendo por ello que su vida posterior sea feliz o infeliz, fácil o difícil. Toda la vida del ser humano se ve enormemente influenciada por lo que le acontece durante los pocos primeros años de su existencia.
\vs p177 2:6 \pc Sinceramente creemos que el evangelio enunciado en las enseñanzas de Jesús, fundado tal como lo está en la relación padre\hyp{}hijo, difícilmente tendrá aceptación a niveles mundiales hasta que la vida en familia de los pueblos modernos civilizados no conlleve más amor y sabiduría. A pesar del gran conocimiento y mayor verdad de que disponen los padres del siglo XX para mejorar el hogar y ennoblecer la vida familiar, sigue siendo cierto que hay muy pocos hogares modernos que constituyan lugares óptimos para criar a niños y niñas como lo fue el hogar de Jesús en Galilea y el de Juan Marcos en Judea, aunque el reconocimiento del evangelio de Jesús llevará de inmediato a su enaltecimiento. La vida que se vive con amor en un hogar donde reine la sabiduría y la fiel devoción a la verdadera religión ejercen entre sí una profunda influencia mutua. Esta forma de vida familiar eleva la religión, y la genuina religión glorifica siempre el hogar.
\vs p177 2:7 Es verdad que los hogares modernos, mejor atendidos, han eliminado prácticamente muchos de los factores deplorablemente anquilosantes y restrictivos de estos antiguos hogares judíos. Existe, de hecho, mayor permisividad y mucha más libertad personal, pero esta libertad no está limitada por el amor, motivada por la lealtad ni dirigida por la sesuda disciplina de la sabiduría. Siempre y cuando enseñemos al niño a orar, “Padre nuestro que estás en los cielos”, pesa sobre todos los padres terrenales la enorme responsabilidad de vivir y ordenar sus hogares de tal manera que la palabra \bibemph{padre} se consagre dignamente en las mentes y en los corazones de todos los niños al ir creciendo.
\usection{3. EL DÍA EN EL CAMPAMENTO}
\vs p177 3:1 Los apóstoles dedicaron la mayor parte de ese día a caminar por el Monte de los Olivos y a charlar con los discípulos acampados allí con ellos, pero, a primera hora de la tarde, ardían en deseos de ver regresar a Jesús. Conforme pasaba el día, más ansiosos estaban por su seguridad; se sentían indeciblemente solos sin él. Durante todo el día, se debatió intensamente si se debería haber permitido al Maestro ir solo a las colinas, acompañado solamente del muchacho de los recados. Aunque nadie expresó abiertamente sus pensamientos, no hubo nadie, excepto Judas Iscariote, que no anhelara estar en el lugar de Juan Marcos.
\vs p177 3:2 \pc Fue a media tarde cuando Natanael dio una charla sobre “El deseo supremo” a una media docena de apóstoles y a un número similar de discípulos. La charla acabó con estas palabras: “El problema de la mayoría de nosotros es nuestra falta de entusiasmo. No amamos al Maestro como él nos ama a nosotros. Si todos nosotros hubiéramos deseado verdaderamente ir con él tanto como Juan Marcos, Jesús sin duda nos habría llevado a todos. Nos quedamos simplemente mirando mientras el muchacho se acercaba al Maestro y le daba la cesta, pero cuando el Maestro la tomó, el muchacho no la soltó. Así pues, el Maestro nos dejó aquí mientras se iba a las colinas con cesta, muchacho y demás”.
\vs p177 3:3 \pc Sobre las cuatro de la tarde, unos corredores le llevaron a David Zebedeo noticia de su madre en Betsaida y de la madre de Jesús. Algunos días antes, David había tomado conciencia de que los sumos sacerdotes y los dirigentes judíos matarían a Jesús. Sabía que estaban determinados a acabar con el Maestro, y estaba casi convencido de que Jesús no haría uso de su poder divino para salvarse ni permitiría a sus seguidores que emplearan la fuerza en su defensa. Habiendo llegado a estas conclusiones, no tardó en enviar un mensajero a su madre, urgiéndola a que viniera enseguida a Jerusalén y que trajera con ella a María, la madre de Jesús, y a todos los miembros de su familia.
\vs p177 3:4 La madre de David hizo lo que le pidió su hijo, y ahora los corredores habían venido de vuelta a David para informarle de que su madre y toda la familia de Jesús se dirigían hacia Jerusalén y deberían llegar en algún momento, ya fuera tarde el día siguiente o muy temprano al otro día por la mañana. Puesto que David había actuado por iniciativa propia, creyó prudente mantener el asunto en privado. No le dijo, pues, a nadie que la familia de Jesús venía de camino a Jerusalén.
\vs p177 3:5 \pc Poco después del mediodía, llegaron al campamento más de veinte de los griegos que se habían reunido con Jesús y los doce en la casa de José de Arimatea, y Pedro y Juan estuvieron varias horas conversando con ellos. Al haber sido instruidos por Rodán de Alejandría, estos griegos, al menos algunos de ellos, tenían un buen nivel de conocimiento del reino.
\vs p177 3:6 Al final de la tarde, tras volver al campamento, Jesús tuvo una charla con los griegos y, de no haber sido porque aquello hubiera disgustado bastante a sus apóstoles y a muchos de sus más destacados discípulos, él habría ordenado a estos veinte griegos, al igual que había hecho con los setenta.
\vs p177 3:7 \pc Mientras que en el campamento sucedía todo aquello, en Jerusalén los sumos sacerdotes y los ancianos estaban asombrados de que Jesús no hubiera vuelto para hablar a las multitudes. Era verdad que el día anterior, al salir del templo, había dicho, “os dejo vuestra casa desierta”, pero ellos no alcanzaban a entender que estuviera dispuesto a renunciar a la gran ventaja que le suponía contar con la actitud favorable de las multitudes. Y aunque temían que se pudiera organizar un gran alboroto en la gente, las últimas palabras del Maestro a la multitud habían sido, no obstante, para instarles a actuar de modo razonable conforme a la autoridad de los “que se sientan en la cátedra de Moisés”. Pero, para ellos, aquel día era de mucho ajetreo, ya que se preparaban simultáneamente tanto para celebrar la Pascua como para terminar de elaborar sus planes de acabar con Jesús.
\vs p177 3:8 \pc No llegó mucha gente al campamento, porque su situación había sido un secreto bien guardado por todos aquellos que sabían que Jesús pensaba quedarse allí por la noche, en lugar de ir a Betania.
\usection{4. JUDAS Y LOS SUMOS SACERDOTES}
\vs p177 4:1 Poco después de que Jesús y Juan Marcos dejaran el campamento, Judas Iscariote desapareció de la presencia de sus hermanos y no regresó hasta última hora de la tarde. Este apóstol, en su confusión y descontento, y pese a la exigencia expresa de su Maestro de que se abstuvieran de entrar a Jerusalén, acudió a toda prisa a su cita con los enemigos de Jesús en la casa del sumo sacerdote Caifás. Se trataba de una reunión informal del sanedrín, programada para algo más tarde de las diez de esa mañana. Esta reunión se convocó para examinar el tipo de cargos que debían imputarse a Jesús y decidir el procedimiento a seguir para llevarlo ante las autoridades romanas, y asegurarse la necesaria ratificación civil para la sentencia a muerte que ya habían dictado.
\vs p177 4:2 El día anterior, Judas había desvelado a algunos de sus parientes y a ciertos amigos saduceos de la familia de su padre que había llegado a la conclusión de que, aunque Jesús era un soñador y un idealista con buenos propósitos, no era el libertador esperado de Israel. Señaló que le gustaría muchísimo encontrar alguna forma de retirarse dignamente de todo el movimiento. Sus amigos, adulándolo, le garantizaron que su retirada sería aclamada por los líderes judíos como un gran acontecimiento, y que conseguiría el máximo de ellos. Le hicieron creer que recibiría de inmediato altos honores de parte del sanedrín, y que acabaría por estar en una posición que le permitiría borrar el estigma de su bienintencionada, pero “lamentable relación con ignorantes galileos”.
\vs p177 4:3 Judas no podía creer por completo que las portentosas obras del Maestro se habían realizado mediante el poder del príncipe de los demonios, pero en aquel momento tenía la plena convicción de que Jesús no ejercería su poder para su propio engrandecimiento; al final, se había convencido de que Jesús permitiría su muerte a mano de los líderes judíos, y él no podía soportar el humillante pensamiento de que se le identificara con un movimiento fracasado. Se negaba a considerar la idea de una aparente derrota. Entendía muy bien el carácter firme de su Maestro y la agudeza de su mente, majestuosa y misericordiosa; sin embargo, le complacía reconocer incluso la mínima sugerencia de uno de sus familiares de que Jesús, siendo un fanático con buenos propósitos, quizás no estuviese realmente en su sano juicio; que siempre había parecido ser una persona extraña e incomprendida.
\vs p177 4:4 Y, entonces, como nunca antes, Judas se encontró a sí mismo convertido en alguien extrañamente rencoroso; le afectaba que Jesús nunca le hubiera asignado un puesto más honroso. Desde un principio, Judas había agradecido el honor de ser el tesorero apostólico, pero ahora comenzaba a sentir que no se le valoraba lo suficiente; que no se reconocía su talento. Y de repente, se vio abrumado por la indignación de que a Pedro, Santiago y Juan se les hubiese encomendando el privilegio de estar en relación estrecha con Jesús, por lo que, yendo en aquel momento de camino a la casa del sumo sacerdote, su pensamiento estaba más centrado en vengarse de Pedro, Santiago y Juan que en el pensamiento mismo de traicionar a Jesús. Pero, por encima de todas estas cosas, justo entonces, un nueva idea empezó a rondarle por la cabeza hasta llegar a ocupar su conciencia de forma preeminente: se había propuesto buscar honores para sí mismo y, si podía lograrlo desquitándose al mismo tiempo de quienes habían contribuido a la mayor decepción de su vida, mucho mejor. Se vio asaltado por un cúmulo terrible de confusión, soberbia, desesperación y determinación. Y, así pues, debe quedar claro que no fue por dinero por lo que Judas se dirigía en esa hora a la casa de Caifás para planificar la traición de Jesús.
\vs p177 4:5 Conforme Judas se aproximaba a la casa de Caifás, acabó por decidirse definitivamente a abandonar a Jesús y a sus compañeros apóstoles; y habiendo tomado, pues, la determinación de desertar de la causa del reino de los cielos, estaba resuelto a obtener para sí mismo la mayor parte posible de ese honor y gloria que había pensado que serían suyos algún día cuando se identificó primeramente con Jesús y con el nuevo evangelio del reino. Todos los apóstoles habían compartido alguna vez esta aspiración de Judas, pero, a medida que trascurría el tiempo, aprendieron a admirar la verdad y a amar a Jesús, al menos más que lo que había hecho Judas.
\vs p177 4:6 Un primo del traidor le presentó a Caifás y a los líderes judíos, aduciendo que Judas, habiéndose percatado del error cometido a dejarse engañar por las sutiles enseñanzas de Jesús, había llegado a un punto en el que deseaba renunciar pública y formalmente a su vinculación con el galileo, pidiendo asimismo que se le readmitiera en la confianza y fraternidad de sus hermanos de Judea. Este portavoz de Judas continuó explicando que su primo reconocía que sería mejor, para la paz de Israel, poner a Jesús bajo custodia y que, como prueba de su pesar por haber participado en aquel errado movimiento y de su sinceridad en su retorno ahora a las enseñanzas de Moisés, había venido para ofrecerse al sanedrín y organizar, junto con el capitán que tenía órdenes de arrestar a Jesús, la forma de prenderlo discretamente; con ello, evitaba, pues, el peligro de exaltar a las multitudes o la necesidad de posponer su detención hasta después de la Pascua.
\vs p177 4:7 Cuando el primo de Judas terminó de hablar, lo presentó, y este, dando un paso adelante para acercarse al sumo sacerdote dijo: “Haré todo lo que mi primo ha prometido, pero ¿qué me queréis dar por este servicio?”. Judas pareció no percibir la expresión de desdén e incluso de repugnancia que se reflejó en el rostro del despiadado y jactancioso Caifás; su corazón estaba demasiado entregado a su propia gloria personal y al afán de satisfacer su propio ensalzamiento.
\vs p177 4:8 Y entonces Caifás bajó la mirada hacia el traidor, diciendo: “Judas, ve al capitán de la guardia y dispón con ese oficial lo necesario para traernos a tu Maestro esta noche o mañana por la noche y, cuando nos lo hayas entregado, recibirás tu recompensa por tal servicio”. Cuando Judas oyó aquello, se retiró de la presencia de los sumos sacerdotes y de los dirigentes judíos y fue a consultar con el capitán de los guardas del templo sobre la manera en la que Jesús debía ser prendido. Judas sabía que Jesús estaba en ese momento fuera del campamento y no tenía idea de cuándo regresaría aquella noche, por lo que acordaron entre ellos arrestar a Jesús la noche siguiente (jueves), después de que tanto la gente de Jerusalén como todos los peregrinos visitantes se hubieran retirado a descansar.
\vs p177 4:9 Cuando Judas regresó al campamento con sus compañeros, estaba intoxicado con ideas de grandeza y de gloria no experimentadas desde hacía mucho tiempo. Se había unido a Jesús esperando convertirse algún día en alguien importante en el nuevo reino. Al final, se había dado cuenta de que no habría ese nuevo reino tal como él había imaginado. Pero se alegraba de ser tan astuto como para poder compensar su decepción por la gloria que había anticipado sin conseguirla con el inmediato honor y la recompensa que lograría en el viejo orden, que ahora pensaba que sobreviviría y que acabaría de cierto con Jesús y con todo lo que él representaba. En última instancia, el incentivo de Judas, en su plan consciente de traicionar a Jesús, fue solo el pensamiento de su propia seguridad y glorificación. Fue el acto cobarde de un desertor egoísta, que no le importó las posibles consecuencias de su conducta sobre su Maestro y sus antiguos compañeros.
\vs p177 4:10 Pero siempre había sido precisamente así. Judas llevaba mucho tiempo inmerso en esta forma de conciencia deliberada, insistente y revanchista, que lo llevaba más y más a acumular en su mente, y a alojar en su corazón, estos deseos aborrecibles y malévolos de venganza y deslealtad. Jesús amaba a Judas y confiaba en él tal como amaba y confiaba en los demás apóstoles, pero Judas jamás llegó a corresponderle, confiando lealmente en él ni amándolo sin reservas. Y ¡qué peligrosa puede llegar a ser la ambición cuando se enraíza enteramente en el interés propio y se motiva en exceso por una hosca venganza, tan largamente reprimida! ¡Qué demoledora resulta la decepción en la vida de esas personas insensatas que, al fijar sus miradas en los atractivos, sombríos y evanescentes, del tiempo, quedan ciegos para lograr las metas más elevadas, reales y perdurables de los mundos eternos, de valores divinos y realidades verdaderamente espirituales. En su mente, Judas ansiaba los honores del mundo y aprendió a amarlos con todo su corazón; los demás apóstoles, en sus mentes, también los ansiaban, pero con sus corazones amaban a Jesús y hacían lo que estaba en sus manos para aprender a amar las verdades que él les enseñaba.
\vs p177 4:11 Judas no se daba cuenta en aquel momento, pero había sido inconscientemente crítico con Jesús desde el instante en el que Herodes decapitó a Juan el Bautista. En el fondo de su corazón, siempre le causó indignación el hecho de que Jesús no hubiera salvado a Juan. No debéis olvidar que Judas había sido discípulo de Juan mucho antes convertirse en seguidor de Jesús. Este cúmulo de resentimiento humano y de amarga decepción, que Judas había acaparado en su alma, revestido de odio, tomó entonces forma en su mente subconsciente y se preparó para aparecer y asolarlo una vez que se atrevió a separarse del respaldo de sus hermanos a la vez que se exponía a las maliciosas insinuaciones y a la sutil mofa de los enemigos de Jesús. Cada vez que Judas permitía que sus esperanzas remontaran el vuelo y Jesús decía o hacía algo que las destrozaba, siempre quedaba en el corazón de Judas la cicatriz de un amargo rencor; y, al multiplicarse estas cicatrices, su corazón, tantas veces herido, perdió en poco tiempo cualquier afecto verdadero hacia quien le infligía estas desagradables experiencias a su persona, bien intencionada, pero cobarde y egocéntrica. Judas no se percataba de ello, pero era un cobarde. Por lo cual, era muy propenso a acusar a Jesús de cobardía por negarse a tomar el poder y la gloria, cuando parecían estar fácilmente a su alcance. Y cualquier hombre mortal sabe muy bien que el amor, incluso si es genuino en un principio, puede acabar por convertirse, por culpa de las decepciones, los celos y el resentimiento, albergado por largo tiempo, en auténtico odio.
\vs p177 4:12 Finalmente, los sumos sacerdotes y los ancianos pudieron respirar tranquilos durante algunas pocas horas. No tendrían que arrestar a Jesús en público y, el haberse granjeado la alianza del traidor Judas les garantizaba que Jesús no escaparía de su jurisdicción, como había hecho tantas veces en el pasado.
\usection{5. ÚLTIMOS MOMENTOS DE SOCIALIZACIÓN}
\vs p177 5:1 Como era miércoles, a última hora de la tarde hubo entre ellos, en el campamento, un tiempo de socialización. El Maestro trató de animar a sus abatidos apóstoles, pero fue prácticamente imposible. Todos comenzaban a darse cuenta de que se avecinaban momentos de desconcierto y consternación. No podían sentirse animados ni incluso cuando el Maestro recordó sus atareados años de relación y cariño. Jesús se interesó con delicadeza por la familia de todos los apóstoles y, fijándose en David Zebedeo, preguntó si alguien había oído recientemente de su madre, hermana pequeña o de otros miembros de su familia. David bajó la mirada, temiendo responder.
\vs p177 5:2 Fue en aquella ocasión cuando Jesús advirtió a sus seguidores respecto al apoyo de la gente. Hizo memoria de aquellos momentos en Galilea cuando los seguían a todos lados, continuamente y con entusiasmo, enormes muchedumbres de personas, y cómo, con el mismo ardor, se volvieron en su contra y retomaron sus antiguas formas de creer y vivir. Y entonces dijo: “Y, por tanto, no os dejéis engañar por las grandes multitudes que nos oyeron en el templo, y que parecían creer nuestras enseñanzas. Escuchan la verdad y la creen superficialmente con sus mentes, pero son pocos los que dejan que la palabra de la verdad llegue a lo más profundo de sus corazones y eche raíces vivas. Cuando surjan los verdaderos problemas, no se puede depender del apoyo de quienes conocen el evangelio solamente con la mente y no lo han vivido con el corazón. Cuando los dirigentes de los judíos alcancen un acuerdo para acabar con el Hijo del Hombre, y cuando ataquen de común acuerdo, veréis cómo la multitud huirá consternada o guardará silencio asombrada mientras que ellos, enfurecidos y ciegos, llevan a los maestros de la verdad evangélica a la muerte. Y, entonces, cuando las adversidades y las persecuciones os sobrevengan, habrá otros, que creéis que aman la verdad, que se dispersarán, y quienes renunciarán al evangelio y os desertarán. Algunos de los que estuvieron muy cerca de nosotros ya han tomado la decisión de desertar. Habéis tenido hoy un día de descanso para prepararos para los tiempos que ya están ahora sobre nosotros. Vigilad, pues, y orad para que mañana os sintáis fortalecidos ante los días que tenemos por delante”.
\vs p177 5:3 Una inexplicable tensión embargaba el ambiente del campamento. Silenciosos mensajeros iban y venían, comunicándose solamente con David Zebedeo. Antes de que se acabara la tarde, hubo quienes conocieron la noticia de que Lázaro había huido a toda prisa de Betania. Juan Marcos estaba preocupadamente callado después de volver al campamento, pese a haber pasado todo el día en compañía del Maestro. Cualquier intento que hacían los apóstoles para persuadirlo a hablar solo les indicaba que Jesús le había pedido que no dijera nada.
\vs p177 5:4 Incluso el buen humor y la inusual sociabilidad del Maestro los atemorizaba. Todos sentían la cierta proximidad de una atroz soledad que entendían estaba a punto de descender sobre ellos de manera repentina y aplastante, y que los aterrorizaba sin poder evitarlo. Vagamente percibían lo que se les venía encima, y ninguno se veía preparado para afrontar aquella prueba. El Maestro había estado fuera todo el día; lo habían echado tremendamente de menos
\vs p177 5:5 Aquel miércoles, al final ya de la tarde, significó para ellos el momento más bajo en su estatus espiritual hasta la misma hora de la muerte del Maestro. El siguiente día, un día aún más cerca del trágico viernes, él estaba, aún, con ellos, y soportaron aquellas horas de ansiedad con mayor serenidad.
\vs p177 5:6 Justo antes de la medianoche, Jesús, sabiendo que aquella sería la última noche que pasaría con la familia que él había escogido en la tierra, dijo, al despedirlos para irse a descansar: “Id a dormir hermanos míos, y que la paz esté con vosotros hasta que os levantéis mañana, un día más para hacer la voluntad del Padre y vivir el gozo de saber que somos sus hijos”.
