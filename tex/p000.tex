\upaper{0}{Prólogo}
\author{Consejero divino}
\vs p000 0:1 En la mente de los mortales de Urantia ---así se llama vuestro mundo--- existe una gran confusión respecto al significado de términos como Dios, divinidad y deidad. Los seres humanos se sienten aún más desconcertados e inciertos en cuanto a las relaciones entre los seres personales divinos designados con estas numerosas denominaciones. Es debido a tal pobreza conceptual y a tan gran confusión de ideas por lo que se me ha solicitado que desarrolle esta introducción y explique los significados que han de atribuirse a ciertos signos verbales para su uso, en lo sucesivo, en los escritos que los miembros del colectivo de reveladores de la verdad de Orvontón han traducido, mediante autorización, a la lengua inglesa de Urantia.
\vs p000 0:2 A pesar de nuestro empeño por ampliar la conciencia cósmica y aumentar la percepción espiritual, nos resulta sumamente difícil exponer conceptos profundos y verdades avanzadas cuando hemos de ceñirnos al uso de una lengua específica de vuestro mundo. Pero en nuestro mandato se nos aconseja que nos esforcemos todo lo posible por transmitir nuestros significados usando los signos verbales del idioma inglés. Se nos han dado instrucciones para acuñar términos nuevos solamente cuando el nuevo concepto a describir carezca de una terminología en este idioma que se pueda emplear para expresarlo, ya sea de manera parcial o incluso con mayor o menor distorsión de su significado.
\vs p000 0:3 Con la esperanza de facilitar la comprensión y de prevenir la confusión por parte de cualquier mortal que lea con detenimiento estos escritos, hemos considerado conveniente exponer, en este enunciado inicial, un bosquejo de las acepciones que se van a atribuir a las numerosas palabras inglesas que se emplearán para designar a la Deidad y a ciertos conceptos afines con respecto a las cosas, los contenidos y los valores de la realidad universal.
\vs p000 0:4 No obstante, las definiciones y acotaciones terminológicas contenidas en este Prólogo necesitan contar con el desarrollo que se hace de dichos términos en exposiciones posteriores. El presente Prólogo, por tanto, no constituye un enunciado completo en sí mismo, sino que representa solo una guía rigurosa, diseñada para asistir a quienes lean los escritos adjuntos que tratan de la Deidad y del universo de los universos, y que han sido desarrollados por una comisión de Orvontón enviada a Urantia con dicho propósito.
\vs p000 0:5 \pc Vuestro mundo, Urantia, es uno de los muchos planetas habitados similares, que componen el universo local de \bibemph{Nebadón}. Este universo, junto a creaciones similares, forma el suprauniverso de \bibemph{Orvontón,} de cuya capital, Uversa, procede nuestra comisión. Orvontón es uno de los siete suprauniversos evolutivos del tiempo y del espacio que circundan la creación sin principio ni fin de la perfección divina: el universo central de \bibemph{Havona}. En el corazón de este universo central y eterno, se encuentra la Isla estacionaria del Paraíso, centro geográfico de la infinitud y morada del Dios Eterno.
\vs p000 0:6 Al conjunto de los siete suprauniversos en evolución más el universo central y divino lo denominamos comúnmente \bibemph{gran universo;} estos constituyen en este momento las creaciones organizadas y habitadas. Todas ellas forman parte del \bibemph{universo matriz,} que a su vez engloba a los universos deshabitados, aunque en activación, del espacio exterior.
\usection{I. DEIDAD Y DIVINIDAD}
\vs p000 1:1 En el universo de los universos se manifiestan fenómenos relativos a la actividad en cuanto deidad en niveles diversos de la realidad cósmica, de los contenidos mentales y de los valores espirituales, pero, en su totalidad, este ministerio ---personal o de otra índole--- está coordinado de forma divina.
\vs p000 1:2 \pc LA DEIDAD es susceptible de ser personal como Dios; es prepersonal y suprapersonal de maneras no del todo comprensibles para el hombre. La Deidad se caracteriza por la cualidad de la unidad ---actual o potencial--- en todos los niveles supramateriales de la realidad. Las criaturas comprenden mejor esta cualidad unificadora como divinidad.
\vs p000 1:3 \pc La Deidad obra en el nivel personal, prepersonal y suprapersonal. La Deidad Total tiene la capacidad de obrar en los siete niveles siguientes:
\vs p000 1:4 \li{1.}\bibemph{Estático:} Deidad que se contiene a sí misma y existe por sí misma.
\vs p000 1:5 \li{2.}\bibemph{Potencial:} Deidad con volición y determinación propias.
\vs p000 1:6 \li{3.}\bibemph{Relacional:} Deidad divinamente fraternal que se hace personal por sí misma.
\vs p000 1:7 \li{4.}\bibemph{Creativo:} Deidad que se distribuye a sí misma y se revela por medios divinos.
\vs p000 1:8 \li{5.}\bibemph{Evolutivo:} Deidad que se expande a sí misma y se identifica con las criaturas.
\vs p000 1:9 \li{6.}\bibemph{Supremo:} Deidad con experiencias propias que unifica a criaturas y Creador. Deidad que obra en el primer nivel de identificación creatural, como regidoras en el espacio\hyp{}tiempo del gran universo; a veces se le designa Supremacía de la Deidad.
\vs p000 1:10 \li{7.}\bibemph{Último:} Deidad que se proyecta a sí misma y que trasciende el espacio\hyp{}tiempo. Deidad omnipotente, omnisciente y omnipresente. Deidad que obra en el segundo nivel de expresión de la divinidad unificadora como regidoras efectivas y sostenedoras absonitas del universo matriz. Comparada con el ministerio de las Deidades en el gran universo, tal acción absonita en el universo matriz es equiparable a la acción directiva y al suprasustento universales; a veces se le llama Ultimidad de la Deidad.
\vs p000 1:11 \pc \bibemph{El nivel finito} de la realidad se caracteriza por la vida de las criaturas y por las limitaciones del espacio y del tiempo. Las realidades finitas quizás no tengan fin, pero siempre tienen principio: son creadas. El nivel de Supremacía de la Deidad se puede concebir en términos de su acción en relación con las existencias finitas.
\vs p000 1:12 \pc \bibemph{El nivel absonito} de la realidad se caracteriza por los seres y cosas sin principio ni fin y por la trascendencia del tiempo y del espacio. Los absonitos no son creados; devienen: sencillamente son. El nivel de Ultimidad de la Deidad se infiere en términos de su acción en relación a las realidades absonitas. En cualquier parte del universo matriz, siempre que se trascienda el tiempo y el espacio, este fenómeno absonito es un acto de la Ultimidad de la Deidad.
\vs p000 1:13 \pc \bibemph{El nivel absoluto} está desprovisto de principio, de fin, de tiempo y de espacio. En el Paraíso, por ejemplo, el tiempo y el espacio no existen; la condición espacio\hyp{}temporal del Paraíso es absoluta. Dicho nivel lo alcanzan de forma trinitaria, existencialmente, las Deidades del Paraíso; no obstante, este tercer nivel de expresión de la Deidad unificadora no está plenamente unificado de forma experiencial. Los valores y contenidos absolutos del Paraíso se manifiestan cuando, donde y como quiera que obre el nivel absoluto de la Deidad.
\vs p000 1:14 \pc La Deidad puede ser existencial, como en el caso del Hijo Eterno; experiencial, como en el caso del Ser Supremo; relacional, como en el caso del Dios Séptuplo; indivisa, como en el caso de la Trinidad del Paraíso.
\vs p000 1:15 La Deidad es la fuente de todo lo que es divino. La Deidad es divina de forma característica e invariable, pero no todo lo divino es Deidad necesariamente, aunque se armonice con la Deidad y se incline hacia alguna faceta de unidad con ella, ya sea espiritual, mental o personal.
\vs p000 1:16 \pc La DIVINIDAD es la cualidad característica, unificadora y coordinadora de la Deidad.
\vs p000 1:17 La divinidad es comprensible para las criaturas como verdad, belleza y bondad; se correlaciona en el ser personal como amor, misericordia y ministerio; y se desvela en los niveles impersonales como justicia, potencia y soberanía.
\vs p000 1:18 La divinidad puede ser perfecta ---completa--- como en los niveles perfectos existenciales y creadores del Paraíso; imperfecta, como en los niveles experienciales y creaturales en evolución espacio\hyp{}temporal; o puede ser relativa, ni perfecta ni imperfecta, como en ciertos niveles de relaciones existenciales\hyp{}experienciales de Havona.
\vs p000 1:19 \pc Cuando intentamos concebir la perfección en la totalidad de sus facetas y formas relativas, nos encontramos con siete tipos posibles:
\vs p000 1:20 \li{1.}Perfección absoluta en todos los aspectos.
\vs p000 1:21 \li{2.}Perfección absoluta en alguna faceta y perfección relativa en todos los demás aspectos.
\vs p000 1:22 \li{3.}Aspectos absolutos, relativos e imperfectos en diversas relaciones.
\vs p000 1:23 \li{4.}Perfección absoluta en algún sentido, imperfección en los otros.
\vs p000 1:24 \li{5.}Perfección absoluta en ninguna dirección, perfección relativa en todas las manifestaciones.
\vs p000 1:25 \li{6.}Perfección absoluta en ninguna faceta, relativa en algunas, imperfecta en otras.
\vs p000 1:26 \li{7.}Perfección absoluta sin atributo alguno, imperfección en todo.
\usection{II. DIOS}
\vs p000 2:1 Las criaturas mortales en evolución sienten un impulso irresistible por representar con signos sus conceptos finitos de Dios. Sin embargo, la conciencia que tiene el hombre de sus deberes morales y de su idealismo espiritual representa un nivel de valores ---una realidad experiencial--- que resulta difícil de representar con signos.
\vs p000 2:2 La conciencia cósmica conlleva el reconocimiento de una Causa Primera, la sola y única realidad incausada. Dios, el Padre Universal, obra en tres niveles de la Deidad\hyp{}ser personal de valor subinfinito y de relativa expresión divina:
\vs p000 2:3 \li{1.}\bibemph{Prepersonal:} como en el ministerio de las fracciones del Padre, los modeladores del pensamiento.
\vs p000 2:4 \li{2.}\bibemph{Personal:} como en la experiencia evolutiva de los seres creados y procreados.
\vs p000 2:5 \li{3.}\bibemph{Suprapersonal:} como en el devenir de las existencias de ciertos seres absonitos y afines.
\vs p000 2:6 DIOS es un símbolo verbal con el que se designa a todas las manifestaciones personales de la Deidad. Este término necesita una definición diferente en cada nivel personal en el que obre la Deidad e incluso debe redefinirse de nuevo en cada uno de estos niveles, ya que este término puede designar a las distintas manifestaciones personales de igual y de menor rango de la Deidad, como, por ejemplo, los hijos creadores del Paraíso ---los padres de los universos locales---.
\vs p000 2:7 \pc El término Dios, tal como lo usamos, puede entenderse:
\vs p000 2:8 \bibemph{Por designación:} como Dios Padre.
\vs p000 2:9 \bibemph{Por contexto:} como cuando se usa al hablar de algún nivel o relación en cuanto deidad. Cuando existan dudas sobre la interpretación exacta de la palabra Dios, sería aconsejable relacionarla con la persona del Padre Universal.
\vs p000 2:10 \pc El término Dios denota siempre \bibemph{ser personal}. Deidad puede referirse o no a seres personales divinos.
\vs p000 2:11 \pc La palabra DIOS se usa en estos escritos con los siguientes significados:
\vs p000 2:12 \li{1.}\bibemph{Dios Padre:} Creador, Rector y Sostenedor; el Padre Universal, la Primera Persona de la Deidad.
\vs p000 2:13 \li{2.}\bibemph{Dios Hijo:} Creador Homólogo, Rector\hyp{}Espíritu y Administrador Espiritual; el Hijo Eterno, la Segunda Persona de la Deidad.
\vs p000 2:14 \li{3.}\bibemph{Dios Espíritu:} Actor Conjunto, Integrador Universal y Otorgador de la Mente; el Espíritu Infinito, la Tercera Persona de la Deidad.
\vs p000 2:15 \li{4.}\bibemph{Dios Supremo:} Dios del tiempo y del espacio en actualización o evolución; Deidad personal que está realizando de forma relacional la consecución experiencial y espacio\hyp{}temporal de la identidad criatura\hyp{}Creador. El Ser Supremo experimenta personalmente el logro de la unidad de la Deidad como Dios evolutivo y experiencial de las criaturas evolutivas del tiempo y del espacio.
\vs p000 2:16 \li{5.}\bibemph{Dios Séptuplo:} Ser personal de la Deidad que obra de hecho por dondequiera del tiempo y el espacio; Deidades personales del Paraíso y sus colaboradores creativos que obran dentro y fuera de las fronteras del universo central y se potencian y hacen personales como Ser Supremo, en el primer nivel creatural de revelación de la Deidad unificadora en el tiempo y el espacio. Este nivel, el gran universo, constituye el entorno donde los seres personales del Paraíso descienden al espacio\hyp{}tiempo y se vinculan con las criaturas evolutivas que ascienden del espacio\hyp{}tiempo.
\vs p000 2:17 \li{6.}\bibemph{Dios Último:} El Dios que deviene del supratiempo y del espacio trascendido; el segundo nivel experiencial en el que se manifiesta la Deidad unificadora. El Dios Último conlleva alcanzar la realización de los valores suprapersonales\hyp{}absonitos sintetizados, los valores del espacio\hyp{}tiempo\hyp{}trascendido y los valores devenidos\hyp{}experienciales, armonizados en los niveles creativos finales de la realidad de la Deidad.
\vs p000 2:18 \li{7.}\bibemph{Dios Absoluto:} El Dios que experimenta los valores suprapersonales trascendidos y los contenidos divinos, ahora existencial como \bibemph{Absoluto de la Deidad.} Constituye el tercer nivel de expresión y expansión de la Deidad. En este nivel supracreativo, la Deidad culmina su potencial para hacerse personal, encuentra compleción de divinidad y disminuye su capacidad para revelarse a sí misma en niveles sucesivos y progresivos respecto a otra manifestación personal. La Deidad ahora encuentra al \bibemph{Absoluto Indeterminado,} incide en él y experimenta identidad con él.
\usection{III. LA PRIMERA FUENTE Y CENTRO}
\vs p000 3:1 La realidad infinita y total es existencial en siete facetas y en la acción de siete absolutos correlacionados:
\vs p000 3:2 \li{1.}La Primera Fuente y Centro.
\vs p000 3:3 \li{2.}La Segunda Fuente y Centro.
\vs p000 3:4 \li{3.}La Tercera Fuente y Centro.
\vs p000 3:5 \li{4.}La Isla del Paraíso.
\vs p000 3:6 \li{5.}El Absoluto de la Deidad.
\vs p000 3:7 \li{6.}El Absoluto Universal.
\vs p000 3:8 \li{7.}El Absoluto Indeterminado.
\vs p000 3:9 \pc Dios, como Primera Fuente y Centro, es primordial con relación a la realidad total ---de forma incondicional---. La Primera Fuente y Centro es infinita al igual que eterna y, por tanto, solo la volición la limita o condiciona.
\vs p000 3:10 Dios ---el Padre Universal--- es el ser personal de la Primera Fuente y Centro y, como tal, mantiene relaciones personales de potestad infinita sobre todas las fuentes y centros de igual y de menor rango. Dicha potestad es personal e infinita en \bibemph{potencia,} si bien nunca realmente obra debido a la perfecta acción de tales fuentes y centros y de los seres personales de igual y menor rango.
\vs p000 3:11 La Primera Fuente y Centro es, por consiguiente, primordial en todos los ámbitos: deificado o no deificado, personal o impersonal, actual o potencial, finito o infinito. No existe ningún ser o cosa, ni relatividad ni completividad, excepto en su relación directa o indirecta, y de dependencia, con la primacía de la Primera Fuente y Centro.
\vs p000 3:12 \pc \bibemph{La Primera Fuente y Centro} se relaciona con el universo de la siguiente manera:
\vs p000 3:13 \li{1.}Las fuerzas de gravedad de los universos materiales convergen en el centro de gravedad del Paraíso inferior. Precisamente, es por esto por lo que la ubicación geográfica de su persona está eternamente fija con relación absoluta al centro de fuerza\hyp{}energía del plano inferior o material del Paraíso. Pero el ser personal absoluto de la Deidad existe en el plano superior o espiritual del Paraíso.
\vs p000 3:14 \li{2.}Las fuerzas de la mente convergen en el Espíritu Infinito; la mente cósmica diferenciada y divergente, en los siete espíritus mayores; la mente del Supremo, que se hace efectiva como experiencia espacio\hyp{}temporal, en Majestón.
\vs p000 3:15 \li{3.}Las fuerzas espirituales del universo convergen en el Hijo Eterno.
\vs p000 3:16 \li{4.}La capacidad ilimitada de acción en cuanto deidad reside en el Absoluto de la Deidad.
\vs p000 3:17 \li{5.}La capacidad ilimitada de respuesta del infinito existe en el Absoluto Indeterminado.
\vs p000 3:18 \li{6.}Los dos Absolutos, el Determinado y el Indeterminado, se armonizan y unifican en y mediante el Absoluto Universal.
\vs p000 3:19 \li{7.}El ser personal potencial de un ser moral evolutivo o de cualquier otro ser moral converge en el ser personal del Padre Universal.
\vs p000 3:20 \pc La REALIDAD, tal como la comprenden los seres finitos, es parcial, relativa y vaga. En el Ser Supremo se incluye la máxima conceptualización posible que las criaturas finitas y evolutivas pueden tener sobre la realidad de la Deidad. Sin embargo, existen realidades anteriores y eternas, realidades suprafinitas, que son ancestrales a esta Deidad suprema de las criaturas evolutivas del espacio\hyp{}tiempo. Al intentar describir el origen y la naturaleza de la realidad universal, nos vemos forzados a emplear un modo de razonamiento espacio\hyp{}temporal para que sea asequible a la mente finita. En consecuencia, muchos acontecimientos simultáneos de la eternidad se deben exponer en forma de interacciones secuenciales.
\vs p000 3:21 Tal como la criatura del espacio\hyp{}tiempo percibiría el origen y la diferenciación de la Realidad, el eterno e infinito YO SOY conseguía desligarse, como Deidad, de las ataduras de la infinitud incondicionada mediante el ejercicio de su libre voluntad, consustancial y eterna, y esta separación de la infinitud incondicionada produjo la primera \bibemph{tensión absoluta de la divinidad}. Dicha tensión diferenciada de la infinitud se resuelve mediante el Absoluto Universal, que tiene la capacidad de unificar y armonizar la infinitud dinámica de la Deidad Total y la infinitud estática del Absoluto Indeterminado.
\vs p000 3:22 Con este acto primigenio, el YO SOY teórico conseguía la realización de su ser personal al erigirse, de manera simultánea, en Padre Eterno del Hijo Primigenio y Fuente Eterna de la Isla del Paraíso. Coexistentes con la diferenciación del Hijo en relación con el Padre, y en presencia del Paraíso, aparecieron la persona del Espíritu Infinito y el universo central de Havona. Con la aparición en coexistencia de la Deidad personal, el Hijo Eterno y el Espíritu Infinito, el Padre se liberó, como ser personal, de la difusión, de otro modo inevitable, en el potencial de la Deidad Total. Desde entonces, es solamente en relación trinitaria con sus dos iguales en la Deidad que el Padre colma todo su potencial como Deidad, en tanto que la Deidad experiencial se actualiza progresivamente en los niveles divinos de la Supremacía, la Ultimidad y la Absolutidad.
\vs p000 3:23 \pc Hemos concedido \bibemph{el concepto filosófico del YO SOY} a la mente finita del hombre, sujeta al tiempo y ligada al espacio, ante la imposibilidad de que las criaturas comprendan las existencias eternas: las realidades y relaciones sin principio ni fin. Para la criatura espacio\hyp{}temporal, todas las cosas han de tener un principio, salvo únicamente la INCAUSADA: la causa primigenia de las causas. Por ello, conceptualizamos este valor\hyp{}nivel filosófico como el YO SOY, al mismo tiempo que impartimos a todas las criaturas la enseñanza de que el Hijo Eterno y el Espíritu Infinito son coeternos con el YO SOY; dicho de otra manera, que nunca hubo un momento en que el YO SOY no fuera el \bibemph{Padre} del Hijo y, con él, del Espíritu.
\vs p000 3:24 \pc \bibemph{El Infinito} se usa para indicar la plenitud ---la completud--- implícita en la primacía de la Primera Fuente y Centro. El YO SOY \bibemph{teórico} es una ampliación filosófica para las criaturas de “la infinitud de la voluntad”, pero el Infinito es un valor\hyp{}nivel \bibemph{real} que representa la eternidad\hyp{}intensión de la verdadera infinitud de la absoluta e incoercible libre voluntad del Padre Universal. A este concepto a veces se le designa el Padre\hyp{}Infinito.
\vs p000 3:25 Mucha de la confusión existente en todos los órdenes de seres, de cualquier rango, cuando tratan de concebir al Padre\hyp{}Infinito, es connatural a su limitada comprensión. La primacía absoluta del Padre Universal no es apreciable a niveles subinfinitos; por ello, es probable que solamente el Hijo Eterno y el Espíritu Infinito conozcan verdaderamente la infinitud del Padre; para todos los demás seres personales, dicho concepto supone un acto de fe.
\usection{IV. LA REALIDAD DEL UNIVERSO}
\vs p000 4:1 La realidad se actualiza, de manera diferenciada, en diversos niveles del universo; la realidad se origina en y mediante la volición infinita del Padre Universal, y se realiza en tres facetas primordiales, en los numerosos niveles diferentes de actualización en el universo:
\vs p000 4:2 \li{1.}\bibemph{La realidad no deificada} se extiende desde los ámbitos energéticos de lo no personal hasta los reinos de la realidad de los valores de la existencia universal no susceptibles de hacerse personales, e incluso hasta la presencia del Absoluto Indeterminado.
\vs p000 4:3 \li{2.}\bibemph{La realidad deificada} comprende todos los potenciales infinitos de la Deidad extendiéndose, en sentido ascendente, a todos los reinos de lo personal desde lo finito más modesto hasta lo infinito más elevado, abarcando el ámbito de todo lo que es susceptible de ser personal e incluso más: hasta la presencia del Absoluto de la Deidad.
\vs p000 4:4 \li{3.}\bibemph{La realidad correlacionada}. La realidad del universo es, según cabe suponer, deificada o no deificada, pero para los seres subdeificados existe un inmenso ámbito de realidad correlacionada, potencial y en actualización, que resulta difícil de reconocer. Gran parte de esta realidad interrelacionada se incluye en los reinos del Absoluto Universal.
\vs p000 4:5 Este es el concepto primordial de la realidad primigenia: El Padre inicia y mantiene la Realidad. Los \bibemph{diferenciales} primordiales de la realidad son lo deificado y lo no deificado ---el Absoluto de la Deidad y el Absoluto Indeterminado---. La \bibemph{relación} primordial es la tensión entre los dos. Dicha tensión divina iniciada por el Padre la resuelve en perfección el Absoluto Universal que se eterniza como tal.
\vs p000 4:6 \pc Desde el punto de vista del tiempo y del espacio, la realidad es además divisible en:
\vs p000 4:7 \li{1.}\bibemph{Actual y potencial}. Realidades que existen en su plenitud de expresión en contraste con las que conllevan una capacidad de desarrollo sin desvelar. El Hijo Eterno es realidad espiritual absoluta en actualidad; el hombre mortal es, en una gran parte, potencialidad espiritual no realizada.
\vs p000 4:8 \li{2.}\bibemph{Absoluta y subabsoluta}. Las realidades absolutas existen en la eternidad. Las realidades subabsolutas se proyectan en dos niveles: las absonitas, o realidades que son relativas con respecto al tiempo y la eternidad, y las finitas, o realidades que se proyectan en el espacio y se actualizan en el tiempo.
\vs p000 4:9 \li{3.}\bibemph{Existencial y experiencial}. La Deidad del Paraíso es existencial, pero los emergentes Supremo y Último son experienciales.
\vs p000 4:10 \li{4.}\bibemph{Personal e impersonal}. La expansión de la Deidad, la expresión de lo personal y la evolución del universo se condicionan, por siempre, al acto de la libre voluntad del Padre que ha separado, para siempre, los contenidos mentales, espirituales y personales y los valores de actualidad y potencialidad, centrados en el Hijo Eterno, de aquellas cosas que se centran y consisten en la Isla eterna del Paraíso.
\vs p000 4:11 \pc El PARAÍSO es un término que incluye a los Absolutos personales y no personales, centros de atracción de todas las facetas de la realidad del universo. El Paraíso, determinado de forma apropiada, alude a todas y cada una de las formas de la realidad, de la Deidad, de la divinidad, de lo personal, de la energía ---espiritual, mental o material---. Todas comparten el Paraíso como lugar de origen, acción y destino, en lo tocante a valores, contenidos y existencia efectiva.
\vs p000 4:12 \pc \bibemph{La Isla del Paraíso} ---el Paraíso determinado de otro modo--- constituye el absoluto de la potestad sobre la materia\hyp{}gravedad de la Primera Fuente y Centro. El Paraíso está inmóvil; es lo único estacionario en el universo de los universos. La Isla del Paraíso tiene ubicación en el universo, pero ninguna posición en el espacio. Esta isla eterna es la fuente real de los universos físicos: los pasados, los presentes y los futuros. La Isla Nuclear de Luz se deriva de la Deidad, pero de ninguna manera es Deidad; tampoco las creaciones materiales son parte de la Deidad: son una consecuencia.
\vs p000 4:13 El Paraíso no crea, sino que rige de manera singular mucha de la actividad del universo; es mucho más rector que reactor. El Paraíso regula, en todos los universos materiales, las reacciones y la conducta de todos los seres que tienen que ver con la fuerza, la energía y la potencia, pero el Paraíso es, en sí mismo único, exclusivo y una excepción en los universos. El Paraíso no es equivalente a nada ni nada es equivalente al Paraíso. No es ni una fuerza ni una presencia; sino simplemente \bibemph{Paraíso}.
\usection{V. REALIDADES PERSONALES}
\vs p000 5:1 El ser personal es un nivel de la realidad deificada y se extiende, en sentido ascendente, desde la mente superior del nivel humano y del ser intermedio, motivada a la adoración y a la sabiduría, a través del nivel morontial y del espiritual, hasta la completud del estatus del ser personal. Este es el modo en que el ser personal creatural humano, y similares, ascienden y evolucionan, pero hay muchos otros órdenes de seres personales en el universo.
\vs p000 5:2 La realidad está sometida a una expansión universal, el ser personal a una diversificación infinita, y ambos son susceptibles de una casi ilimitada coordinación y estabilización eterna por parte de la Deidad. Mientras que el alcance metamórfico de la realidad no personal está definitivamente limitado, no conocemos límite alguno para la evolución progresiva de las realidades personales.
\vs p000 5:3 Al alcanzar niveles experienciales, todos los órdenes o valores personales son relacionables e incluso cocreativos. Hasta Dios y el hombre pueden coexistir en un ser personal unificado, tal como lo demuestra, con excelencia, la condición presente de Cristo Miguel: Hijo del Hombre e Hijo de Dios.
\vs p000 5:4 Todos los órdenes y facetas subinfinitas del ser personal se alcanzan de forma relacional; incluso son potencialmente cocreativos. Lo prepersonal, lo personal y lo suprapersonal están por completo enlazados por un potencial mutuo de alcance armonizado, consecución progresiva y capacidad cocreativa. Pero nunca lo impersonal se transmuta directamente en personal. El ser personal nunca es espontáneo; es un don del Padre del Paraíso. El ser personal se superpone a la energía y solo se vincula a los sistemas de energía viva; la identidad puede vincularse a los modelos de energía no viva.
\vs p000 5:5 \pc El Padre Universal es el secreto de la realidad personal, de la dádiva del ser personal y del destino del ser personal. El Hijo Eterno es el ser personal absoluto, el secreto de la energía espiritual, de los espíritus morontiales y de los espíritus perfeccionados. El Actor Conjunto es el ser personal del espíritu\hyp{}mente, la fuente de la inteligencia, de la razón y de la mente universal. Pero la Isla del Paraíso es no personal y extraespiritual; es la esencia del cuerpo universal, la fuente y centro de la materia física y el modelo matriz absoluto de la realidad material universal.
\vs p000 5:6 \pc Estas cualidades de la realidad universal se ponen de manifiesto en la experiencia humana de los urantianos en los niveles siguientes:
\vs p000 5:7 \li{1.}\bibemph{El cuerpo:} Es el organismo físico o material del hombre; es el mecanismo electroquímico vivo, de naturaleza y origen animal.
\vs p000 5:8 \li{2.}\bibemph{La mente:} Es el mecanismo del organismo humano que piensa, percibe y siente; es toda experiencia consciente e inconsciente; es la inteligencia vinculada a la vida emocional que se eleva mediante la adoración y la sabiduría hasta el nivel del espíritu.
\vs p000 5:9 \li{3.}\bibemph{El espíritu:} Es el ser divino que mora en la mente del hombre, el modelador del pensamiento. Este espíritu inmortal es prepersonal ---no es un ser personal--- aunque esté llamado a ser parte del ser personal de la criatura mortal que sobrevive.
\vs p000 5:10 \li{4.}\bibemph{El alma:} Es el alma del hombre que se adquiere de forma experiencial. A medida que la criatura mortal elige “hacer la voluntad del Padre que está en los cielos”, el espíritu morador se erige, al mismo tiempo, como padre de una \bibemph{nueva realidad} en la experiencia humana. La mente humana y material es la madre de esta misma realidad emergente. La sustancia de esta nueva realidad no es ni material ni espiritual: es \bibemph{morontial}. Así es el alma inmortal emergente, que está destinada a sobrevivir a la muerte física y comenzar su ascensión al Paraíso.
\vs p000 5:11 \pc \bibemph{El ser personal:} El ser personal del hombre mortal no es ni su cuerpo ni su mente ni su espíritu; ni tampoco su alma. El ser personal es la única realidad inmutable a las experiencias, por otro lado siempre cambiantes, de las criaturas y unifica todos los demás componentes de que consta la individualidad. El ser personal constituye la dádiva inigualable del Padre Universal a las energías vivas vinculadas a la materia, a la mente y al espíritu, y que sobrevive al sobrevivir el alma morontial.
\vs p000 5:12 \pc \bibemph{Morontia} es un término que designa al inmenso nivel que media entre lo material y lo espiritual. Puede designar realidades personales o impersonales, energías vivas o no vivas. La urdimbre de la morontia es espiritual; su trama es física.
\usection{VI. ENERGÍA Y MODELO}
\vs p000 6:1 Llamamos personal a todo lo que responde a la vía del Padre por donde circula el ser personal. Llamamos espiritual a todo lo que responde a la vía del Hijo por donde circula el espíritu. Llamamos mente a todo lo que responde a la vía del Actor Conjunto por donde circula la mente, mente como atributo del Espíritu Infinito ---mente en todas sus facetas---. Llamamos materia a todo lo que responde a la vía por donde circula la gravedad material con centro en el Paraíso inferior ---energía\hyp{}materia en todos sus estados metamórficos---.
\vs p000 6:2 \pc Usamos el término ENERGÍA en su más amplio sentido aplicado al ámbito espiritual, mental y material. \bibemph{Fuerza} también se usa ampliamente. \bibemph{Potencia} se limita regularmente a designar el nivel electrónico de la materia o la materia sensible a la gravedad lineal del gran universo. Potencia también se emplea para designar soberanía. No podemos adoptar vuestras definiciones de fuerza, energía y potencia, generalmente aceptadas. Hay tal insuficiencia de términos lingüísticos que nos vemos obligados a asignar significados múltiples a estos términos.
\vs p000 6:3 \pc \bibemph{Energía física:} término que indica todas las facetas y formas del movimiento, de la acción y del potencial fenoménicos.
\vs p000 6:4 Al hablar de las manifestaciones de la energía física, usamos, por lo general, los términos fuerza cósmica, energía emergente y potencia del universo. Estos se emplean a menudo de la manera siguiente:
\vs p000 6:5 \li{1.}\bibemph{Fuerza cósmica:} abarca todas las energías derivadas del Absoluto Indeterminado, pero que aún no son receptivas a la gravedad del Paraíso.
\vs p000 6:6 \li{2.}\bibemph{Energía emergente:} abarca aquellas energías que responden a la gravedad del Paraíso, pero que aún no son receptivas a la gravedad local o lineal. Este es el nivel preelectrónico de la energía\hyp{}materia.
\vs p000 6:7 \li{3.}\bibemph{Potencia del universo:} incluye todas las formas de energía que, a pesar de responder a la gravedad del Paraíso, son de modo directo receptivas a la gravedad lineal. Este es el nivel electrónico de la energía\hyp{}materia y de todas sus evoluciones posteriores.
\vs p000 6:8 \pc \bibemph{Mente:} fenómeno que supone la presencia y actividad del \bibemph{ministerio vivo,} además de sistemas variados de energía; y esto es cierto en todos los grados de inteligencia. En los seres personales, la mente siempre media entre el espíritu y la materia. Así pues, el universo está iluminado por tres clases de luces: la luz material, la percepción intelectual y la luminosidad del espíritu.
\vs p000 6:9 \pc \bibemph{Luz} ---la luminosidad del espíritu--- es un símbolo verbal, una imagen, que indica la manifestación característica del ser personal de los seres espirituales de las distintas órdenes. Esta emanación luminosa no está relacionada en ningún aspecto ni con la percepción intelectual ni con las manifestaciones de la luz física.
\vs p000 6:10 \pc El MODELO puede proyectarse como material, espiritual, mental o cualquier combinación de estas energías. Puede infundirse sobre seres personales, identidades, entidades o materia no viva. Pero el modelo es el modelo y permanece siendo el modelo; solo las \bibemph{copias} se multiplican.
\vs p000 6:11 El modelo configura la energía, pero no la dirige. Únicamente la gravedad dirige la energía\hyp{}materia. Ni el espacio ni el modelo son receptivos a la gravedad, pero no existe relación entre el espacio y el modelo; el espacio no es ni un modelo ni un modelo potencial. El modelo es una configuración de la realidad que ya ha satisfecho su débito con la gravedad; la \bibemph{realidad} de cualquier modelo radica en sus energías, en sus componentes mentales, espirituales o materiales.
\vs p000 6:12 En contraposición al aspecto de lo \bibemph{total,} el modelo desvela el aspecto \bibemph{individual} de la energía y del ser personal. Las formas del ser personal o de la identidad son modelos resultantes de la energía (física, espiritual o mental) pero no son inherentes a ella. Esa cualidad de la energía o del ser personal, por virtud de la que se origina la aparición del modelo, se atribuye a Dios ---a la Deidad--- a la dotación de fuerza del Paraíso, a la coexistencia del ser personal y de la potencia.
\vs p000 6:13 El modelo es un diseño original del que se hacen las copias. El Paraíso Eterno es el absoluto de los modelos; el Hijo Eterno es el modelo del ser personal; el Padre Universal es la fuente ancestral directa de ambos. Pero ni el Paraíso otorga el modelo, ni el Hijo otorga el ser personal.
\usection{VII. EL SER SUPREMO}
\vs p000 7:1 El mecanismo de la Deidad del universo matriz es doble en lo que respecta a las relaciones en la eternidad. El Dios Padre, el Dios Hijo y el Dios Espíritu son eternos ---son seres existenciales--- mientras que el Dios Supremo, el Dios Último y el Dios Absoluto son Deidades personales de las épocas posteriores a Havona que \bibemph{se actualizan} en las esferas del espacio\hyp{}tiempo y del espacio\hyp{}tiempo trascendido consecuentes a la expansión evolutiva del universo matriz. Dichas Deidades personales en actualización son eternas en el futuro desde el momento, y a medida, que se potencian y se hacen personales en los universos en desarrollo por medio de la actualización experiencial de los potenciales creativos y relacionales de las Deidades eternas del Paraíso.
\vs p000 7:2 La Deidad tiene, por tanto, una doble presencia:
\vs p000 7:3 \li{1.}\bibemph{Existencial:} seres con existencia eterna, pasada, presente y futura.
\vs p000 7:4 \li{2.}\bibemph{Experiencial:} seres que se actualizan en el presente posterior a Havona, pero cuya existencia no tendrá fin en toda la eternidad futura.
\vs p000 7:5 \pc El Padre, el Hijo y el Espíritu son existenciales: existenciales en actualidad (aunque todos los potenciales son presumiblemente experienciales). El Supremo y el Último son totalmente experienciales. El Absoluto de la Deidad es experiencial en actualización, pero existencial en potencia. La esencia de la Deidad es eterna, pero únicamente las tres personas primigenias de la Deidad son incondicionalmente eternas. Los demás seres personales de la Deidad han tenido origen, pero son eternos en cuanto a destino.
\vs p000 7:6 Al conseguir expresarse como Deidad existencial en el Hijo y el Espíritu, el Padre está ahora consiguiendo expresarse de manera experiencial en niveles en cuanto deidad, hasta aquí impersonales y no revelados, como Dios Supremo, Dios Último y Dios Absoluto; pero dichas Deidades experienciales no existen plenamente en este momento, sino que están en proceso de actualización.
\vs p000 7:7 \pc El \bibemph{Dios Supremo} en Havona constituye el reflejo espiritual y personal de la Deidad trina del Paraíso. Esta Deidad vinculada y relacional se expande ahora creativamente hacia fuera en el Dios Séptuplo y se sintetiza en la potencia experiencial del Todopoderoso Supremo en el gran universo. La Deidad del Paraíso, existencial en tres personas, evoluciona así de forma experiencial en dos facetas de la Supremacía, mientras que estas facetas dobles unifican la potencia\hyp{}ser personal en un único Señor: el Ser Supremo.
\vs p000 7:8 El Padre Universal consigue por medio de su libre voluntad liberarse del yugo de la infinitud y de las ataduras de la eternidad mediante la trinitización, el triple estado personal de la Deidad. El Ser Supremo, incluso hoy en día, está evolucionando como unificación personal subeterna de la séptupla manifestación de la Deidad en los segmentos espacio\hyp{}temporales del gran universo.
\vs p000 7:9 \pc \bibemph{El Ser Supremo} no es un creador directo, exceptuando que es el padre de Majestón; no obstante, coordina y sintetiza toda la actividad del universo entre criatura y Creador. El Ser Supremo, que ahora se actualiza en los universos evolutivos, es la Deidad que correlaciona y sintetiza la divinidad espacio\hyp{}temporal y, asimismo, correlaciona y sintetiza la Deidad trina del Paraíso que se relaciona experiencialmente con los creadores supremos del tiempo y del espacio. Cuando termine de actualizarse, esta Deidad evolutiva constituirá la fusión eterna de lo finito y de lo infinito: la unión perpetua e indisoluble de la potencia adquirida de forma experiencial con el ser personal\hyp{}espíritu.
\vs p000 7:10 Toda la realidad finita del espacio\hyp{}tiempo, bajo el impulso directivo del Ser Supremo en evolución, está dedicada a la movilización en constante ascenso y a la unificación en perfección (la síntesis de la potencia y el ser personal) de todas las facetas y valores de la realidad finita, en combinación con facetas variadas de la realidad del Paraíso, con la finalidad y el propósito de emprender con posterioridad la conquista de los niveles absonitos que alcanzan las supracriaturas.
\usection{VIII. EL DIOS SÉPTUPLO}
\vs p000 8:1 Para remediar la condición finita y compensar las limitaciones conceptuales de las criaturas evolutivas, el Padre Universal ha establecido para ellas una aproximación séptupla a la Deidad:
\vs p000 8:2 \li{1.}Los hijos creadores del Paraíso.
\vs p000 8:3 \li{2.}Los ancianos de días.
\vs p000 8:4 \li{3.}Los siete espíritus mayores.
\vs p000 8:5 \li{4.}El Ser Supremo.
\vs p000 8:6 \li{5.}El Dios Espíritu.
\vs p000 8:7 \li{6.}El Dios Hijo.
\vs p000 8:8 \li{7.}El Dios Padre.
\vs p000 8:9 \pc Esta séptupla manifestación personal de la Deidad en el tiempo y en el espacio y para los siete suprauniversos hace posible que el hombre mortal alcance la presencia de Dios, que es espíritu. Esta Deidad séptupla, que para las criaturas finitas del espacio\hyp{}tiempo en algún momento se hará potencia\hyp{}ser personal en el Ser Supremo, es la Deidad efectiva de las criaturas evolutivas mortales en su andadura de ascensión al Paraíso. Tal andadura de descubrimiento experiencial, por la que se llega a la cognición de Dios, comienza con el reconocimiento de la divinidad del hijo creador del universo local y conduce, en sentido ascendente, a través de los ancianos de días del suprauniverso, y mediante la persona de uno de los siete espíritus mayores, hasta el descubrimiento y reconocimiento en el Paraíso del ser personal divino del Padre Universal.
\vs p000 8:10 \pc El gran universo constituye el triple ámbito, relativo a la Deidad, de la Trinidad de la Supremacía, del Dios Séptuplo y del Ser Supremo. El Dios Supremo existe en potencia en la Trinidad del Paraíso, de la que deriva su ser personal y sus atributos como espíritu; pero en este momento está actualizándose en los hijos creadores, en los ancianos de días y en los espíritus mayores, de los que deriva su potencia como el Todopoderoso de los suprauniversos del tiempo y del espacio. Esta poderosa manifestación del Dios inmediato de las criaturas evolutivas existentes en el tiempo y el espacio evoluciona, de hecho, conjuntamente con ellas. El Todopoderoso Supremo, que evoluciona en el valor\hyp{}nivel de la actividad no personal, y la persona espiritual del Dios Supremo son \bibemph{una realidad:} el Ser Supremo.
\vs p000 8:11 Los hijos creadores vinculados como Deidad en el Dios Séptuplo proporcionan el mecanismo mediante el cual lo mortal se hace inmortal y lo finito logra el acogimiento de lo infinito. El Ser Supremo proporciona el modo de activación de la potencia y del ser personal, la síntesis divina, de \bibemph{todas} estas interacciones múltiples, permitiendo así que lo finito alcance lo absonito y, a través de otras posibles actualizaciones futuras, intentar alcanzar el Último. Los hijos creadores acompañados de sus benefactoras divinas son partícipes de esta suprema activación, pero es probable que los ancianos de días y los siete espíritus mayores estén asignados eternamente como administradores permanentes del gran universo.
\vs p000 8:12 La acción del Dios Séptuplo data de la organización de los siete suprauniversos, y se expandirá con probabilidad en conexión con la evolución futura de las creaciones del espacio exterior. Con la organización de estos futuros universos que evolucionan de forma progresiva en los niveles espaciales primario, secundario, terciario y cuaternario, se declarará inaugurado, sin lugar a dudas, el acercamiento supremo y absonito a la Deidad.
\usection{IX. EL DIOS ÚLTIMO}
\vs p000 9:1 Al igual que el Ser Supremo evoluciona progresivamente a partir de una previa dotación divina del potencial de la energía y del ser personal contenidos en el gran universo, el Dios Último deviene a partir de los potenciales divinos que residen en los ámbitos del espacio\hyp{}tiempo transcendido del universo matriz. La actualización de la Deidad Última señala la unificación absonita de la primera Trinidad experiencial e indica la expansión de la Deidad unificadora en el segundo nivel de la realización creativa de sí misma. Esto constituye el equivalente del ser personal\hyp{}potencia de la actualización de la Deidad experiencial del universo de las realidades absonitas del Paraíso en los niveles que devienen de los valores del espacio\hyp{}tiempo transcendido. La compleción de tal despliegue experiencial tiene como objeto facilitar el servicio\hyp{}destino último a todas las criaturas del espacio\hyp{}tiempo que hayan alcanzado los niveles absonitos por medio de la consecución completa del Ser Supremo y mediante el ministerio del Dios Séptuplo.
\vs p000 9:2 \pc \bibemph{El Dios Último} designa la Deidad personal que obra en los niveles divinos de lo absonito y en las esferas del universo del supratiempo y del espacio trascendido. El Último constituye el suprasupremo devenir de la Deidad. El Supremo es la unificación de la Trinidad, tal como la comprenden los seres finitos; el Último es la unificación de la Trinidad del Paraíso, tal como la comprenden los seres absonitos.
\vs p000 9:3 El Padre Universal, mediante el mecanismo de la Deidad evolutiva, se ocupa de hecho del formidable e increíble \bibemph{acto} de converger en el ser personal y de activar la potencia, en sus respectivos contenidos\hyp{}niveles del universo, de los valores de la realidad divina de lo finito, de lo absonito e incluso de lo absoluto.
\vs p000 9:4 Las primeras tres Deidades del Paraíso eternas en el pasado ---el Padre Universal, el Hijo Eterno y el Espíritu Infinito--- complementarán su ser personal en el eterno futuro, con la actualización experiencial de las Deidades evolutivas compañeras: el Dios Supremo, el Dios Último y, posiblemente, el Dios Absoluto.
\vs p000 9:5 \pc El Dios Supremo y el Dios Último, que evolucionan en este momento en los universos experienciales, no son existenciales: no son eternos en el pasado, solamente eternos en el futuro, eternos condicionados por el espacio\hyp{}tiempo y por la trascendencia. Son Deidades dotadas de supremacía, ultimidad y, posiblemente, de supremacía\hyp{}ultimidad, pero han experimentado orígenes históricos en el universo. Nunca tendrán fin, pero su ser personal ha tenido efectivamente un principio. Son, de hecho, actualizaciones de los potenciales eternos e infinitos de la Deidad, pero, por sí mismos, no son ni incondicionalmente eternos ni infinitos.
\usection{X. EL DIOS ABSOLUTO}
\vs p000 10:1 Hay muchos rasgos de la realidad eterna de la \bibemph{Deidad Absoluta} para los que la mente finita del espacio\hyp{}tiempo no puede encontrar una explicación satisfactoria, pero la actualización del \bibemph{Dios Absoluto} se produciría como consecuencia de la unificación de la segunda Trinidad experiencial, de la Trinidad Absoluta. Esto constituiría la realización experiencial de la divinidad absoluta, la unificación de los contenidos absolutos en niveles absolutos; pero no estamos seguros de que se engloben todos los valores absolutos, puesto que no se nos ha informado en ningún momento de que el Absoluto Determinado sea el equivalente del Infinito. Los destinos supraúltimos incluyen contenidos absolutos y espiritualidad infinita, y no podemos establecer valores absolutos sin estas dos realidades inconclusas.
\vs p000 10:2 Para todos los seres supraabsonitos, el Dios Absoluto representa la realización\hyp{}consecución de la última meta, pero la potencia y el ser personal potencial de la Deidad Absoluta superan nuestras nociones, y no nos decidimos a comentar esas realidades tan alejadas de la actualización experiencial.
\usection{XI. LOS TRES ABSOLUTOS}
\vs p000 11:1 Cuando el pensamiento combinado del Padre Universal y del Hijo Eterno, obrando en el Dios de Acción, constituyó la creación del universo central y divino, el Padre dirigió la expresión de su pensamiento hacia la palabra de su Hijo y al acto de su Mandatario Conjunto al diferenciar su presencia en Havona de los potenciales de la infinitud. Y esos potenciales sin desvelar de la infinitud permanecen ocultos en el espacio del Absoluto Indeterminado y bajo cobijo divino en el Absoluto de la Deidad, mientras que estos dos se erigen como uno en la acción del Absoluto Universal, la infinitud\hyp{}unidad no revelada del Padre del Paraíso.
\vs p000 11:2 Tanto la potencia de la fuerza cósmica como la potencia de la fuerza espiritual están en proceso progresivo de revelación\hyp{}realización a medida que se efectúa el engrandecimiento de toda la realidad con el desarrollo experiencial y mediante la correlación de lo experiencial con lo existencial a través del Absoluto Universal. En virtud de la presencia equilibradora del Absoluto Universal, la Primera Fuente y Centro realiza la extensión de la potencia experiencial, se complace en la identificación con sus criaturas evolutivas y consigue la expansión de la Deidad experiencial en los niveles de la Supremacía, la Ultimidad y la Absolutidad.
\vs p000 11:3 \pc Cuando no es posible distinguir por completo entre el Absoluto de la Deidad y el Absoluto Indeterminado, a su presumible labor conjunta o a su presencia entrelazada se la denomina la acción del Absoluto Universal.
\vs p000 11:4 \li{1.}\bibemph{El Absoluto de la Deidad} parece ser un todopoderoso activador, en tanto que el Absoluto Indeterminado parece ser un mecanizador de total eficacia del universo de los universos, unificado en supremacía y armonizado en ultimidad, e incluso de otros muchos universos, ya creados, en proceso de creación y aún por crearse.
\vs p000 11:5 El Absoluto de la Deidad no puede dar respuesta de una manera subabsoluta, o simplemente no lo hace, a ninguna circunstancia del universo. Cada una de las respuestas de este Absoluto parece hacerse en función del bien común del conjunto de seres y cosas creadas, no solo en su presente estado existencial, sino también considerando las infinitas posibilidades de toda la eternidad futura.
\vs p000 11:6 El Absoluto de la Deidad es ese potencial que fue disgregado de la realidad total e infinita, mediante la libertad de elección del Padre Universal, y en el que tiene lugar toda la actividad ---existencial y experiencial--- de la divinidad. Este es el \bibemph{Absoluto Determinado} en contraposición con el \bibemph{Absoluto Indeterminado;} pero el Absoluto Universal se sobreañade a los dos, englobando todo el potencial absoluto.
\vs p000 11:7 \li{2.}\bibemph{El Absoluto Indeterminado} es no personal, extradivino y no deificado. Este Absoluto está desprovisto, por tanto, de ser personal, de divinidad y de todas las prerrogativas creadoras. Ningún hecho o verdad, ninguna experiencia o revelación, ninguna filosofía o absonitidad es capaz de desentrañar la naturaleza y el carácter de este Absoluto sin condicionamiento en el universo.
\vs p000 11:8 Es preciso aclarar que el Absoluto Indeterminado es una \bibemph{realidad positiva} que se infunde sobre el gran universo y que, al parecer, se difunde, con presencia espacial equivalente, por la actividad de la fuerza y por las evoluciones premateriales de las impresionantes expansiones de las regiones espaciales, más allá de los siete suprauniversos. El Absoluto Indeterminado no es el mero negativismo de un concepto filosófico predicado sobre supuestos sofismas metafísicos concernientes a la universalidad, dominación y primacía de lo incondicionado e indeterminado. El Absoluto Indeterminado ejerce una acción directiva sobre el universo en la infinitud; dicha acción directiva es ilimitada en cuanto a la fuerza\hyp{}espacio, pero está claramente condicionada por la presencia de la vida, la mente, el espíritu y el ser personal, y está condicionado, además, por las resoluciones y los determinados mandatos de la Trinidad del Paraíso.
\vs p000 11:9 Estamos convencidos de que el Absoluto Indeterminado no ejerce una influencia indiscriminada y ubicua comparable a los conceptos panteístas de la metafísica o la antigua hipótesis científica del éter. El Absoluto Indeterminado es ilimitado en cuanto a fuerza, y está condicionado por la Deidad, pero no percibimos del todo su relación con las realidades espirituales del universo.
\vs p000 11:10 \li{3.}El \bibemph{Absoluto Universal,} por lógica, resultaba inevitable al acto de la absoluta libre voluntad del Padre Universal de establecer, en la realidad del universo, la diferencia entre los valores deificados y no deificados (susceptibles de ser personales y no personales). El Absoluto Universal constituye ese fenómeno de la Deidad que representa la resolución de la tensión creada por dicho acto de la libertad de la voluntad al diferenciar así la realidad del universo, y obra coordinando y relacionando las sumas totales de estas potencialidades existenciales.
\vs p000 11:11 \pc La tensión\hyp{}presencia del Absoluto Universal señala el ajuste de la diferencia entre la realidad en cuanto deidad y la realidad no deificada, inherentes a la separación entre el dinamismo de la libertad de la voluntad divina y el estatismo de la infinitud incondicionada.
\vs p000 11:12 \pc Recordad siempre: la infinitud potencial es absoluta e inseparable de la eternidad. La infinitud actual en el tiempo no puede ser nunca más que parcial y debe ser, por consiguiente, no absoluta; tampoco puede ser absoluta la infinitud del ser personal en actualidad excepto en la Deidad incondicionada. Y este diferencial de potencial infinito entre el Absoluto Indeterminado y el Absoluto de la Deidad es el que eterniza al Absoluto Universal, posibilitándolo así, cósmicamente, para contar con universos materiales en el espacio y, espiritualmente, para albergar en el tiempo a seres personales finitos.
\vs p000 11:13 Lo finito puede coexistir en el cosmos junto con lo Infinito solamente porque la presencia relacional del Absoluto Universal iguala con gran perfección las tensiones entre el tiempo y la eternidad, entre la finitud y la infinitud, entre el potencial de la realidad y la manifestación de la realidad, entre el Paraíso y el espacio, entre el hombre y Dios. Relacionalmente, el Absoluto Universal constituye la identificación de la zona de la realidad en evolución progresiva existente en los universos del espacio\hyp{}tiempo y en los del espacio\hyp{}tiempo trascendido, en los que se manifiesta la Deidad subinfinita.
\vs p000 11:14 El Absoluto Universal es el potencial de la Deidad estática\hyp{}dinámica con capacidad de realizarse en los niveles del tiempo\hyp{}eternidad como valores finitos\hyp{}absolutos y de ser susceptible a un acercamiento experiencial\hyp{}existencial. Este aspecto incomprensible de la Deidad puede ser estático, potencial y vinculado, pero no es, experiencialmente, ni creativo ni evolutivo en lo que concierne a los seres personales de inteligencia que actúan ahora en el universo matriz.
\vs p000 11:15 \pc \bibemph{El Absoluto}. Los dos Absolutos ---Determinado e Indeterminado---, aunque desde la perspectiva de las criaturas con mente parezcan obrar de forma divergente, se unifican de manera perfecta y divina en y a través del Absoluto Universal. En última instancia y comprensión final, los tres forman un solo Absoluto. En los niveles subinfinitos, se les diferencia en cuanto a su acción, pero en la infinitud constituyen UNO SOLO.
\vs p000 11:16 \pc Nunca usamos el término Absoluto para negar nada o para negarlo todo. Tampoco consideramos que el Absoluto Universal se determine a sí mismo, como una especie de Deidad panteísta e impersonal. El Absoluto, en todo lo que se refiere a un universo personal, está estrictamente limitado por la Trinidad y sujeto a la Deidad.
\usection{XII. LAS TRINIDADES}
\vs p000 12:1 La Trinidad del Paraíso primigenia y eterna es existencial y resultaba inevitable. Esta Trinidad sin principio era consustancial al hecho mismo de la diferenciación de lo personal y de lo no personal realizada por la incoercible voluntad del Padre Universal, y se hizo efectiva cuando su voluntad personal enlazó estas realidades dobles mediante la mente. Las Trinidades posteriores a Havona son experienciales: son inherentes a la creación de los dos niveles, el subabsoluto y el evolutivo, de la manifestación de la potencia\hyp{}ser personal en el universo matriz.
\vs p000 12:2 \pc \bibemph{La Trinidad del Paraíso} ---la unión como Deidad del Padre Universal, del Hijo Eterno y del Espíritu Infinito--- es existencial en actualidad, pero todos sus potenciales son experienciales. Por ello, esta Trinidad constituye la única realidad de la Deidad que engloba la infinitud, y por ello suceden en el universo los fenómenos de actualización del Dios Supremo, del Dios Último y del Dios Absoluto.
\vs p000 12:3 \pc La primera y la segunda Trinidades experienciales, las Trinidades posteriores a Havona, no pueden ser infinitas porque incluyen a las \bibemph{Deidades derivadas,} Deidades evolucionadas por medio de la actualización experiencial de las realidades creadas o devenidas mediante la Trinidad existencial del Paraíso. La infinitud de la divinidad se agranda constantemente, si no se engrandece, mediante la experiencia finita y absonita de la criatura y el Creador.
\vs p000 12:4 Las Trinidades constituyen verdades relacionales y hechos pertinentes a la manifestación de la Deidad de igual rango. Dentro de la labor de la Trinidad se engloban las realidades de la Deidad, y las realidades de la Deidad siempre procuran realizarse y manifestarse en el estado personal. El Dios Supremo, el Dios Último e incluso el Dios Absoluto son, por tanto, inevitabilidades divinas. Estas tres Deidades experienciales estaban potencialmente en la Trinidad existencial, en la Trinidad del Paraíso, pero su gradual aparición en el universo como seres personales de potencia depende, en parte, de su propia acción experiencial en los universos de la potencia y del ser personal y, en parte, de la consecución experiencial de los creadores y las Trinidades posteriores a Havona.
\vs p000 12:5 \pc Las dos Trinidades experienciales posteriores a Havona, la Trinidad Última y la Trinidad Absoluta, no están en este momento plenamente manifestadas; están en proceso de realización en el universo. Estas conjunciones de la Deidad se pueden describir como sigue:
\vs p000 12:6 \li{1.}\bibemph{La Trinidad Última,} ahora en evolución, constará finalmente del Ser Supremo, de las personas creadoras supremas y de los arquitectos absonitos del universo matriz, esos singulares planificadores del universo, que no son ni creadores ni criaturas. El Dios Último terminará inevitablemente por potenciarse y hacerse personal como consecuencia divina de la unificación de esta Trinidad Última experiencial, en el escenario creciente del casi ilimitado universo matriz.
\vs p000 12:7 \li{2.}\bibemph{La Trinidad Absoluta} ---la segunda Trinidad experiencial---, ahora en proceso de actualización, constará del Dios Supremo, del Dios Último y del Consumador del Destino del Universo, no revelado. Dicha Trinidad obra en los dos niveles, personal y suprapersonal, llegando incluso hasta los límites de lo no personal, y su unificación en universalidad haría experiencial a la Deidad Absoluta.
\vs p000 12:8 \pc La Trinidad Última se unifica de manera experiencial hasta su compleción, pero en verdad dudamos de que una unificación tan plena sea posible en la Trinidad Absoluta. Nuestro concepto, sin embargo, de la Trinidad eterna del Paraíso constituye un recordatorio, siempre presente, de que la trinitización de la Deidad quizás logre lo que de otra manera resulta inalcanzable; por eso presuponemos la aparición futura del \bibemph{Supremo\hyp{}Último,} al igual que la posible trinitización\hyp{}efectuación del Dios Absoluto.
\vs p000 12:9 \pc Los filósofos de los universos presuponen la existencia de una \bibemph{Trinidad de Trinidades,} una Trinidad infinita existencial\hyp{}experiencial, pero no son capaces de imaginar su condición personal; tal vez equivaliera a la persona del Padre Universal en el nivel conceptual del YO SOY. Pero, aparte de todo esto, la Trinidad del Paraíso primigenia es infinita en potencialidad puesto que el Padre Universal es infinito en actualidad.
\usection{\bibemph{RECONOCIMIENTO}}
\vs p000 12:11 Al elaborar los escritos siguientes que describen el carácter del Padre Universal y la naturaleza de sus colaboradores del Paraíso, y que tratan, además, de describir el perfecto universo central y los siete suprauniversos que lo circundan, tenemos que guiarnos por los mandatos de los gobernantes de los suprauniversos, que han dispuesto que, en cualquier intento por revelar la verdad y organizar el conocimiento fundamental, demos preferencia a los conceptos humanos más elevados que existen referentes a los temas que se van a tratar. Solo podemos recurrir a la pura revelación cuando el concepto a exponer no haya sido previamente expresado adecuadamente por la mente humana.
\vs p000 12:12 Las continuadas revelaciones planetarias de la verdad divina contienen, invariablemente, los conceptos más elevados que existen de los valores espirituales, como parte integrante de una nueva y mejor coordinación del conocimiento planetario. Como consecuencia, para exponer estos escritos sobre Dios y sus colaboradores del universo, nos hemos basado en una selección de más de mil conceptos humanos que representan el conocimiento planetario más elevado y avanzado existente de los valores espirituales y de los contenidos del universo. Cuando estos conceptos humanos, recogidos de mortales del pasado y del presente que conocieron a Dios, se muestren inadecuados para describir la verdad tal como se nos ha solicitado que la revelemos, no dudaremos en complementarlos, recurriendo para tal fin a nuestro propio conocimiento superior de la realidad y de la divinidad de las Deidades del Paraíso y del universo trascendente, lugar donde residen.
\vs p000 12:13 Somos plenamente conscientes de las dificultades de nuestro cometido; reconocemos la imposibilidad de traducir del todo el lenguaje de los conceptos de la divinidad y de la eternidad a los signos lingüísticos de las nociones finitas de la mente de los mortales. Pero sabemos que en la mente humana mora una fracción de Dios, y que con el alma humana reside el espíritu de la verdad; también sabemos que tales fuerzas espirituales se aúnan para hacer posible que el hombre material alcance a comprender la realidad de los valores espirituales y la filosofía de los contenidos del universo. Además, sabemos, incluso con mayor certeza, que estos espíritus de la Presencia Divina son capaces de asistir al hombre a aprehender espiritualmente cualquier verdad que contribuya a ampliar la realidad, siempre en progreso, de la experiencia religiosa personal ---de la conciencia de Dios---.
\vsetoff
\vs p000 12:14 [Redactado por un consejero divino de Orvontón, jefe del colectivo de seres personales del suprauniverso designado para describir en Urantia la verdad acerca de las Deidades del Paraíso y el universo de los universos.]
