\upaper{21}{Los hijos creadores del Paraíso}
\author{Perfeccionador de la sabiduría}
\vs p021 0:1 Los hijos creadores son los hacedores y gobernantes de los universos locales del tiempo y del espacio. Estos creadores y soberanos de los universos tienen doble origen, incorporando las características de Dios Padre y de Dios Hijo. Si bien, cada hijo creador es diferente de los demás; cada cual tiene una naturaleza y un ser personal únicos; cada cual es el “Hijo unigénito” del perfecto ideal, en cuanto deidad, de su origen.
\vs p021 0:2 En su inmensa labor de organizar, hacer evolucionar y perfeccionar un universo local, estos elevados hijos siempre disfrutan de la continuada aprobación del Padre Universal. La relación de los hijos creadores con su Padre del Paraíso es enternecedora e inigualable. Sin duda, el profundo cariño que profesa la Deidad, como progenitora, por su progenie divina constituye la fuente de ese amor hermoso y casi divino que incluso los padres mortales albergan por sus hijos.
\vs p021 0:3 Estos hijos primarios del Paraíso adquieren su estado personal como migueles. Cuando parten del Paraíso para fundar sus universos, se les conoce como migueles creadores; cuando se establecen en autoridad suprema, se les llama migueles mayores. A veces aludimos al soberano de vuestro universo de Nebadón como Cristo Miguel. Para siempre y por siempre reinan a la manera del orden de Miguel. Este es el nombre que se le da al primer hijo nacido de este orden y con esta naturaleza.
\vs p021 0:4 \pc El miguel primigenio o primogénito nunca ha vivenciado la encarnación como ser material, pero ha experimentado siete veces el camino de ascenso que siguen las criaturas espirituales por las siete vías planetarias de Havona, avanzando desde las esferas exteriores hasta la vía más interior de la creación central. El orden de Miguel conoce el gran universo de un extremo a otro; no hay experiencia esencial alguna sentida por los hijos del tiempo y del espacio en la que los migueles no hayan participado personalmente. De hecho, son partícipes no solo de la naturaleza divina sino también de la vuestra, es decir, de todas las naturalezas, desde la más elevada hasta la más modesta.
\vs p021 0:5 El miguel primigenio es quien preside sobre los hijos del Paraíso primarios cuando estos se reúnen para conferenciar en el centro de todas las cosas. No hace mucho, en Uversa, se registró la emisión al universo de un cónclave extraordinario de ciento cincuenta mil hijos creadores reunidos en la Isla eterna, en presencia de sus progenitores, que deliberaban sobre el progreso de la unificación y estabilización del universo de los universos. Se trataba de un grupo de migueles mayores seleccionados, de hijos creadores que habían llevado a cabo sus siete ministerios de gracia.
\usection{1. ORIGEN Y NATURALEZA DE LOS HIJOS CREADORES}
\vs p021 1:1 Cuando la plenitud de la ideación espiritual absoluta del Hijo Eterno se encuentra con la plenitud del concepto absoluto del ser personal del Padre Universal, cuando dicha unión creativa se logra de forma final y plena, cuando suceden tan absoluta identidad de espíritu y tan unicidad infinita de concepto personal, entonces, en ese mismo momento, sin que se pierda nada del ser personal ni de la prerrogativa de ninguna de las dos Deidades infinitas, aparece destellante, en todo su ser, un nuevo y primigenio hijo creador ---el hijo unigénito---, del ideal perfecto y de la poderosa idea, cuya unión da origen a este nuevo ser personal creador de poder y perfección.
\vs p021 1:2 Cada hijo creador es el unigénito y el único vástago posible de la unión perfecta de los conceptos primigenios de las dos mentes infinitas y eternas y perfectas, de los creadores sempiternos del universo de los universos. Jamás puede haber otro hijo así, porque cada uno de ellos es la expresión y la manifestación incondicionada, completa y última de todas y cada faceta de cada rasgo de cada posibilidad de cada realidad divina que pudiera hallarse, por toda la eternidad, expresarse o desarrollarse, a partir de esos potenciales creativos divinos que se unieron para dar nacimiento a este hijo miguel. Cada hijo creador es el absoluto de los conceptos unidos en cuanto deidad, constitutivos de su origen divino.
\vs p021 1:3 La naturaleza divina de estos hijos creadores tiene su origen, en principio, por igual, en los atributos de sus dos padres del Paraíso. Todos participan de la plenitud de la naturaleza divina del Padre Universal y de las prerrogativas creativas del Hijo Eterno, pero a medida que observamos la forma práctica de proceder de los migueles en los universos, percibimos aparentes diferencias. Algunos hijos creadores parecen ser más como Dios Padre; otros, más como Dios Hijo. Por ejemplo, la administración del universo de Nebadón muestra que la naturaleza y carácter de su creador e hijo gobernante lo asemejan más a la Madre Hijo Eterna. Además, tenemos que señalar que los migueles que presiden algunos universos parecen semejarse por igual a Dios Padre y Dios Hijo. Y estas observaciones no implican, en sentido alguno, ningún juicio de valor sino una mera exposición de los hechos.
\vs p021 1:4 No conozco el número exacto de hijos creadores, pero tengo buenas razones para creer que hay más de setecientos mil. Si bien, sabemos que existen exactamente setecientos mil uniones de días y que no se están creando más. También hemos observado que los planes diseñados para la actual era del universo parecen indicar que un unión de días tendrá que emplazarse en cada universo local como embajador consejero de la Trinidad. Hemos notado además que el número en constante aumento de hijos creadores excede ya el inalterable número de los uniones de días, pero nunca se nos ha informado del destino de esos migueles que exceden los setecientos mil.
\usection{2. LOS CREADORES DE LOS UNIVERSOS LOCALES}
\vs p021 2:1 Los hijos del Paraíso del orden primario son los diseñadores, creadores, constructores y administradores de sus respectivos dominios, de los universos locales del tiempo y del espacio ---las unidades creativas básicas de los siete suprauniversos evolutivos---. A un hijo creador se le permite elegir el emplazamiento espacial en el que tendrá lugar su futura actividad cósmica, pero antes de que pueda incluso comenzar a organizar físicamente su universo, debe dedicar un largo período a la observación y estudio de la labor de sus hermanos mayores en las distintas creaciones del suprauniverso donde proyecta actuar. Y con anterioridad a todo esto, el hijo miguel habrá completado su prolongado y excepcional período de observación en el Paraíso y de formación en Havona.
\vs p021 2:2 \pc Cuando un hijo creador parte del Paraíso para embarcarse en la aventura de la creación de un universo, para erigirse en rector ---prácticamente el Dios--- del universo local que él mismo ha organizado, entonces, por primera vez, se encuentra que está en estrecho contacto con la Tercera Fuente y Centro y, en muchos aspectos, dependiente de ella. El Espíritu Infinito, aunque habita con el Padre y el Hijo en el centro de todas las cosas, está destinado a proceder como auténtico y eficaz ayudante de todo hijo creador. Por tanto, a cada uno de los hijos creadores lo acompaña una hija creativa del Espíritu Infinito, ese ser que está llamado a convertirse en la benefactora divina, en el espíritu materno del nuevo universo local.
\vs p021 2:3 La partida de un hijo miguel en tal ocasión libera para siempre sus prerrogativas como creador procedentes de las fuentes y centros del Paraíso, sujetas solamente a ciertas limitaciones consustanciales a la preexistencia de estas mismas fuentes y centros y a ciertas otras potencias y presencias anteriores. Entre estas limitaciones a dichas prerrogativas creadoras, por lo demás todopoderosas, del Padre de un universo local están las siguientes:
\vs p021 2:4 \li{1.}El Espíritu Infinito domina \bibemph{la energía\hyp{}materia}. Antes de poder crear nuevas formas de las cosas, grandes o pequeñas, antes de poder intentar cualquier transformación nueva de la energía\hyp{}materia, el hijo creador debe conseguir el consentimiento y la cooperación activa del Espíritu Infinito.
\vs p021 2:5 \li{2.}El Hijo Eterno rige \bibemph{los diseños y los tipos de criaturas}. Antes de que un hijo creador pueda comenzar a crear un nuevo tipo de ser, un nuevo diseño creatural, debe conseguir el consentimiento de la Madre Hijo Eterna y Primigenia.
\vs p021 2:6 \li{3.}El Padre Universal diseña y otorga \bibemph{el ser personal}.
\vs p021 2:7 \pc Los factores pre\hyp{}creaturales del ser determinan los tipos y los modelos de la \bibemph{mente}. Una vez que estos se han vinculado para formar una criatura (personal o no), se añade la mente que constituye la dote de la Tercera Fuente y Centro, la fuente universal del ministerio de la mente para todos los seres por debajo del nivel de los creadores del Paraíso.
\vs p021 2:8 \pc La potestad sobre los diseños y tipos \bibemph{espirituales} depende de su nivel de manifestación. En última instancia, el diseño espiritual se dirige por la Trinidad o mediante las dotes espirituales pre\hyp{}trinitarias de las personas de la Trinidad ---Padre, Hijo y Espíritu---.
\vs p021 2:9 \pc Cuando tal hijo perfecto y divino ha tomado posesión del lugar espacial para el universo que ha elegido; cuando se han resuelto los problemas iniciales en relación a la materialización del universo y al equilibrio bruto; cuando ha formado con la hija complementaria del Espíritu Infinito un vínculo eficaz de trabajo, entonces, este hijo del universo y su espíritu del universo se coligan para dar origen a las innumerables multitudes de sus hijos del universo local. Con relación a este suceso, la convergencia del espíritu creativo en el Espíritu Infinito del Paraíso cambia de naturaleza, adquiriendo las cualidades personales del espíritu materno de un universo local.
\vs p021 2:10 Pese a que todos los hijos creadores son, desde una perspectiva divina, como sus padres del Paraíso, ninguno se asemeja exactamente a los demás, cada cual es excepcional, distinto, único y primigenio en su \bibemph{naturaleza} así como también en su persona. Y puesto que son los arquitectos y hacedores de los planes de vida de sus respectivos entornos, esta misma diversidad asegura que sus dominios sean también distintos en cuanto a cualquier forma y etapa de vida, derivadas de cada miguel, que puedan crearse o evolucionar con posterioridad. Por lo tanto, los órdenes de criaturas nativas de los universos locales varían bastante. No hay dos universos locales que estén regidos o habitados por seres nativos de origen doble que sean idénticos en todos los sentidos. Dentro de cada suprauniverso, se asemejan bastante en la mitad de los atributos que les son consustanciales, porque tienen su origen en los uniformes espíritus creativos; la otra mitad varía porque tienen su origen en los diversificados hijos creadores. Si bien, tal diversidad no caracteriza ni a aquellas criaturas que tienen un único origen en el espíritu creativo ni a aquellos seres traídos de fuera y que son originarios del universo central o de los suprauniversos.
\vs p021 2:11 \pc Cuando un hijo miguel se ausenta de su universo, su gobierno lo dirige el primer ser allí nacido, la brillante estrella de la mañana, jefe del poder ejecutivo del universo local. En esos momentos, el consejo y el asesoramiento del unión de días son inestimables. Durante estas ausencias, el hijo creador puede conferir al espíritu materno, vinculado a él, el pleno poder de su presencia espiritual en los mundos habitados y en los corazones de sus hijos mortales. El espíritu materno del universo local permanece siempre en la sede central, haciendo extensibles sus cuidados protectores y su ministerio espiritual hasta las zonas más lejanas de tal entorno evolutivo.
\vs p021 2:12 No es necesaria la presencia personal de un hijo creador en su universo local para que una creación material establecida opere sin complicaciones. Estos hijos pueden viajar al Paraíso y sus universos seguirán todavía girando por el espacio. Pueden dejar sus ámbitos de poder para encarnarse como hijos del tiempo, pero, incluso así, sus mundos darán vueltas sobre sus respectivos centros. Ninguna organización material es independiente de la atracción de la gravedad absoluta del Paraíso ni del pleno poder cósmico inherente en la presencia espacial del Absoluto Indeterminado.
\usection{3. LA SOBERANÍA DE UN UNIVERSO LOCAL}
\vs p021 3:1 Un hijo creador recibe los dominios de un universo con el consentimiento de la Trinidad del Paraíso y la confirmación del espíritu mayor que supervisa el correspondiente suprauniverso. Con ello, obtiene el derecho a la posesión física, a la tenencia cósmica de tal universo. Pero la elevación de un hijo miguel, desde esta etapa inicial y autolimitada de gobierno a la supremacía experiencial de una merecida soberanía, es producto de sus propias experiencias personales en la labor de crear un universo y de su encarnación. Si bien, hasta que no consigue lograr su soberanía por medio de sus ministerios de gracia, gobierna como vicerregente del Padre Universal.
\vs p021 3:2 \pc Un hijo creador podría, en cualquier momento, hacer valer su plena soberanía sobre su creación personal, pero prudentemente decide no hacerlo así. Si asumiera una soberanía suprema de forma inmerecida, antes de pasar por los ministerios de gracia en semejanza de las criaturas, los seres personales del Paraíso que residen en su universo local se retirarían. Sin embargo, esto no ha ocurrido nunca en ninguna de las creaciones del tiempo y del espacio.
\vs p021 3:3 El hecho de ser creador supone plenitud de soberanía, pero los migueles eligen \bibemph{merecerla} de forma experiencial, conservando así la cooperación plena de todos los seres personales del Paraíso asignados a la administración del universo local. No tenemos noticia de que ningún miguel haya actuado de otra manera; si bien, todos podrían al ser verdaderamente hijos con libre voluntad.
\vs p021 3:4 \pc La soberanía de un hijo creador en un universo local pasa por seis etapas, o quizás siete, que se manifiestan de forma experiencial. Estas aparecen en el orden siguiente:
\vs p021 3:5 \li{1.}Soberanía inicial como vicerregente: la autoridad solitaria provisional que un hijo creador ejerce antes de que el espíritu creativo vinculado a él adquiera cualidades personales.
\vs p021 3:6 \li{2.}Soberanía conjunta como vicerregente: el gobierno conjunto de las dos personas del Paraíso con posterioridad a que el espíritu materno del universo adquiera su ser personal.
\vs p021 3:7 \li{3.}Soberanía en incremento como vicerregente: el aumento de la autoridad de un hijo creador durante el período de sus siete ministerios de gracia en la forma de las criaturas.
\vs p021 3:8 \li{4.}Soberanía suprema: la autoridad establecida tras haber concluido su séptimo ministerio de gracia. En Nebadón, la soberanía suprema data desde el momento en el que Miguel completó esta misión en Urantia. Lleva algo más de mil novecientos años de vuestro tiempo planetario.
\vs p021 3:9 \li{5.}Soberanía suprema en incremento: la relación en aumento que surge al asentarse en luz y vida la mayoría de los entornos creaturales. Esta etapa pertenece a un futuro al que aún no se ha llegado en vuestro universo local.
\vs p021 3:10 \li{6.}Soberanía trinitaria: ejercida con posterioridad al asentamiento de todo el universo local en luz y vida.
\vs p021 3:11 \li{7.}Soberanía no revelada: las relaciones desconocidas de una futura era del universo.
\vs p021 3:12 \pc Al aceptar la soberanía inicial como vicerregente de un universo local que se ha proyectado, un miguel creador jura ante la Trinidad no asumir la soberanía suprema hasta haber completado sus siete ministerios de gracia y estos hayan sido certificados por los gobernantes del suprauniverso. Pero si un hijo miguel no pudiera, a voluntad, hacer valer una soberanía no merecida, no tendría sentido jurar que no lo va a hacer.
\vs p021 3:13 Incluso en las eras anteriores a estos ministerios, el hijo creador gobierna en su dominio de forma casi suprema cuando no hay disensiones en ninguna de las partes de este. Difícilmente se manifestarían los límites de un gobierno si no se desafiara la soberanía. La soberanía que ejerce un hijo creador antes de darse a un universo sin rebeliones no es mayor que en un universo con rebeliones; en el primer caso, los límites a la soberanía no son manifiestos; en el segundo, sí los son.
\vs p021 3:14 Si se desafía, ataca o se hace peligrar la autoridad o la administración del hijo creador, él ha prometido eternamente sostener, proteger, defender y, si es necesario, rescatar su creación personal. Tan solo las criaturas que ellos mismos han creado o seres más elevados elegidos por ellos pueden preocupar u hostigar a estos hijos. Se podría deducir que con toda probabilidad, “los seres más elevados”, aquellos que se han originado en niveles situados por encima del universo local, no causarían problemas al hijo creador, y esto es así. Sin embargo, podrían hacerlo si así lo eligieran. La virtud es volitiva en el ser personal; la rectitud no es espontánea en las criaturas de libre voluntad.
\vs p021 3:15 Antes de acabar su andadura de gracia, un hijo creador gobierna con unos límites de soberanía autoimpuestos, pero tras haber realizado dicho servicio, gobierna en virtud de la experiencia real adquirida en la forma y semejanza de sus múltiples criaturas. Cuando un creador ha morado siete veces entre sus criaturas, cuando termina tal andadura de gracia, entonces se erige, de forma suprema, como autoridad del universo; se convierte en un hijo mayor, en un soberano y en un gobernante supremo.
\vs p021 3:16 \pc Para conseguir la soberanía suprema de un universo local hay que seguir siete pasos que tienen un componente experiencial:
\vs p021 3:17 \li{1.}Adentrarse de forma experiencial en los siete niveles creaturales mediante tales ministerios de encarnación con la semejanza de las criaturas de cada uno de estos niveles.
\vs p021 3:18 \li{2.}Consagrarse experiencialmente a cada una de las facetas de la voluntad séptupla de la Deidad del Paraíso, tal como se personifica en los siete espíritus mayores.
\vs p021 3:19 \li{3.}Pasar por cada una de estas siete experiencias en los niveles creaturales, en simultaneidad con el desempeño de una de las siete consagraciones a la voluntad de la Deidad del Paraíso.
\vs p021 3:20 \li{4.}En cada nivel creatural, poner de manifiesto, de forma experiencial, la cúspide de la vida creatural para la Deidad del Paraíso y para todas las inteligencias del universo.
\vs p021 3:21 \li{5.}En cada nivel creatural, revelar experiencialmente una de las facetas de la voluntad séptupla de la Deidad en el nivel en el que se otorga y en todo el universo.
\vs p021 3:22 \li{6.}De modo experiencial, unificar su experiencia creatural séptupla con su séptupla experiencia de consagración a la revelación de la naturaleza y la voluntad de la Deidad.
\vs p021 3:23 \li{7.}Lograr una nueva y más elevada relación con el Ser Supremo. La repercusión de la totalidad de esta experiencia como creador\hyp{}criatura incrementa la realidad en el entorno del suprauniverso del Dios Supremo y la soberanía espacio\hyp{}temporal del Supremo Todopoderoso, y lleva a efecto la suprema soberanía en el universo local de un miguel del Paraíso.
\vs p021 3:24 \pc Al solucionarse la cuestión de su soberanía en un universo local, el hijo creador no solo demuestra su propia capacidad para gobernar sino que también revela la naturaleza y representa la actitud séptupla de la Deidad del Paraíso. La comprensión finita y el reconocimiento, por parte de las criaturas, de que la primacía del Padre tiene importancia en la aventura del hijo creador, cuando condesciende a hacer suya la forma y las experiencias de sus criaturas. Estos hijos primarios del Paraíso verdaderamente revelan la naturaleza amorosa y la autoridad benefactora del Padre, del mismo Padre quien, en colaboración con el Hijo y el Espíritu, tiene a su cargo toda potencia, ser personal y gobierno a lo largo y ancho de todos los ámbitos universales.
\usection{4. LOS MINISTERIOS DE GRACIA DE LOS MIGUELES}
\vs p021 4:1 Existen siete grupos de hijos creadores de gracia y se clasifican así de acuerdo con el número de veces que se han dado a sí mismos a las criaturas de sus dominios. Varían desde los que tienen su primera experiencia de gracia, pasando por otras cinco esferas en las que se dan de forma progresiva, hasta que llevan a efecto el séptimo y último episodio en su experiencia como criatura\hyp{}creador.
\vs p021 4:2 Los ministerios de gracia que realizan los avonales son siempre semejando un hombre mortal, pero los que realiza un hijo creador implican su aparición en siete niveles creaturales y conllevan la revelación de las siete expresiones primarias de la voluntad y naturaleza de la Deidad. Sin excepción, todos los hijos creadores pasan por estas siete etapas, dándose a las criaturas que ellos mismos han creado antes de asumir la establecida y suprema jurisdicción sobre los universos, igualmente creados por ellos.
\vs p021 4:3 A pesar de que estos siete ministerios varían según los diferentes sectores y universos, siempre conllevan la aventura de tener que darse en forma mortal. En el último de estos servicios, el hijo creador aparece en algún planeta habitado como miembro de una de las razas mortales mejor dotadas, normalmente del grupo racial que contenga un mayor legado hereditario del linaje adánico, importado previamente para elevar la condición física de los pueblos de origen animal. Solo una vez en el curso de estas siete misiones, nace un miguel del Paraíso de mujer como en el caso del niño de Belén, del que tenéis constancia. Solo una vez vive y muere como miembro del orden más modesto de criaturas evolutivas de voluntad.
\vs p021 4:4 Después de cada uno de sus ministerios de gracia, el hijo creador continúa hasta “la derecha del Padre” para lograr allí su aceptación respecto a ellos y recibir instrucciones preliminares en relación a su siguiente servicio en el universo. Tras su séptimo y último ministerio, el hijo creador recibe, de parte del Padre Universal, autoridad y jurisdicción suprema sobre todo su universo.
\vs p021 4:5 \pc Hay constancia de que el último hijo divino que apareció en vuestro planeta era un hijo creador del Paraíso que había acabado las seis etapas de su andadura de gracia; por consiguiente, cuando dejó atrás la conciencia adquirida de su vida encarnada en Urantia, él pudo decir, y verdaderamente dijo: “consumado es”; y literalmente estaba consumado. Con su muerte en Urantia, completó su andadura de gracia; aquel fue su último paso para cumplir el juramento sagrado de un hijo creador del Paraíso. Y cuando adquieren dicha experiencia, estos hijos son soberanos supremos del universo; ya no gobiernan como vicerregentes del Padre, sino en su propio derecho y nombre como “Rey de Reyes y Señor de Señores”. Con algunas excepciones mencionadas, estos hijos creadores, que se dan de gracia siete veces, son incondicionalmente soberanos en los universos en los que habitan. En cuanto a su universo local, “todo el poder en el cielo y en la tierra” se delegó a este hijo mayor triunfante y coronado.
\vs p021 4:6 \pc Tras completar estas andaduras, se considera a los hijos creadores como un orden aparte, como hijos mayores séptuplos. En cuanto a su persona, son idénticos a los hijos creadores, pero al haber pasado por tal experiencia única, comúnmente se les considera de un orden distinto. Cuando un creador condesciende a darse de gracia, está destinado a experimentar un cambio verdadero y permanente. Es cierto que el hijo de gracia continúa siendo, no obstante, un creador, pero ha añadido a su naturaleza una experiencia de tipo creatural, lo que por siempre lo eleva desde el nivel divino del hijo creador hasta el plano experiencial del hijo mayor, a alguien que ha conseguido pleno derecho a gobernar un universo y a regir sus mundos. Estos seres incorporan en sí todo lo que la paternidad divina conlleva y engloban todo lo que se puede derivar de su experiencia como criatura perfeccionada. ¡Cómo es posible que el hombre se lamente de su origen humilde y de la andadura evolutiva que ha de cumplir cuando los mismos Dioses tienen que pasar por una vivencia equivalente, antes de que se les pueda considerar experiencialmente merecedores y capaces de gobernar su universo de forma plena y final!
\usection{5. RELACIÓN DE LOS HIJOS MAYORES CON EL UNIVERSO}
\vs p021 5:1 El poder de un miguel mayor es ilimitado porque procede de su experimentada relación con la Trinidad del Paraíso, es incuestionable porque procede de su experiencia real de estar sujeto, al igual que sus criaturas mismas, a tal autoridad. La naturaleza de la soberanía de un hijo creador séptuplo es suprema porque:
\vs p021 5:2 \li{1.}Abarca el punto de vista séptuplo de la Deidad del Paraíso.
\vs p021 5:3 \li{2.}Incorpora la actitud séptupla de las criaturas del tiempo y del espacio.
\vs p021 5:4 \li{3.}Armoniza de manera perfecta la actitud del Paraíso y el punto de vista creatural.
\vs p021 5:5 \pc Esta soberanía experiencial incluye, pues, toda la divinidad del Dios Séptuplo que culmina en el Ser Supremo. Y la soberanía personal de un hijo séptuplo es como la soberanía futura del Ser Supremo que en algún momento se completará, al abarcar, como lo hace, el contenido más pleno posible de la potencia y la autoridad de la Trinidad del Paraíso capaz de manifestarse dentro de los correspondientes límites espacio\hyp{}temporales.
\vs p021 5:6 \pc Con el logro de la soberanía suprema en el universo local, desaparece del hijo miguel el poder y la ocasión de crear tipos de criaturas completamente nuevos durante la presente era del universo. Pero esta pérdida de poder del hijo mayor para originar órdenes de seres completamente nuevos no interfiere de ninguna manera con su labor de elaborar la vida ya establecida y en desarrollo; este inmenso programa de evolución del universo continúa sin interrupción ni restricción. La adquisición de la soberanía suprema de un hijo mayor conlleva la responsabilidad de su dedicación personal al impulso y a la administración de lo que ya se ha diseñado y creado, y de lo que posteriormente se creará por aquellos que han sido, de este modo, diseñados y creados. Es posible que con el tiempo se dé una evolución casi infinita de seres distintos, pero ningún modelo primigenio ni ningún tipo completamente nuevo de criatura inteligente se originará de forma directa por un hijo mayor, a partir de ese momento. Ese es el primer paso, el principio, de una administración de carácter estable en cualquier universo local.
\vs p021 5:7 La elevación de un hijo que se ha dado de gracia siete veces para alcanzar la soberanía incuestionable de su universo marca el principio del fin de la incertidumbre y de la relativa confusión de una era. Tras este hecho, lo que no pueda alguna vez espiritualizarse acabará por desorganizarse; aquello que no pueda coordinarse con la realidad cósmica acabará por deshacerse. Cuando se han agotado las disposiciones de una misericordia sin fin y la paciencia indecible en el afán por ganar la lealtad y la devoción de todas las criaturas de voluntad de los mundos, prevalecerán la justicia y la rectitud. Aquello que la misericordia no puede rehabilitar, la justicia, al final, lo reducirá a la nada.
\vs p021 5:8 \pc Los migueles mayores son supremos en sus propios universos locales una vez que se les ha nombrado gobernantes soberanos. Las pocas limitaciones a su gobierno resultan de aquellas inherentes a la preexistencia cósmica de ciertas fuerzas y seres personales. Aparte de esto, estos hijos soberanos son supremos en autoridad, responsabilidad y poder administrativo en sus respectivos universos; son como Creadores y Dioses, prácticamente supremos en todas las cosas. No existe mayor sabiduría en cuanto al funcionamiento de un universo que la que ellos poseen.
\vs p021 5:9 Cuando tras su elevación se establece la soberanía de un miguel del Paraíso en un universo local, este asume potestad plena sobre todos los demás Hijos de Dios que operan en sus dominios, y puede gobernar libremente de acuerdo con su idea de las necesidades de tales dominios. Un hijo mayor puede a voluntad cambiar el tipo de juicio espiritual y de ajuste evolutivo de los planetas habitados. Y estos hijos elaboran y llevan a cabo los planes que ellos mismos eligen en todos los asuntos referidos a las necesidades planetarias especiales, en particular en lo que respecta a los mundos en donde residen sus criaturas e incluso más en lo que respecta al lugar de su último ministerio de gracia, al planeta de su encarnación semejando un hombre mortal.
\vs p021 5:10 Los hijos soberanos parecen estar en comunicación perfecta con los mundos de gracia, pero no solo con los mundos donde residen personalmente sino con todos los mundos en los que un hijo magistrado se haya dado de gracia. Este contacto se mantiene debido a su propia presencia espiritual, mediante el espíritu de la verdad, que ellos pueden “derramar sobre toda carne”. Estos hijos mayores también mantienen un contacto ininterrumpido con la Madre Hijo Eterna, en el centro de todas las cosas. Poseen un rango de compasión que se extiende desde el Padre Universal en las alturas hasta las modestas razas de vida planetaria en los mundos del tiempo.
\usection{6. EL DESTINO DE LOS MIGUELES MAYORES}
\vs p021 6:1 Nadie se atrevería a hablar con completa autoridad ni de la naturaleza ni del destino de los soberanos mayores séptuplos de los universos locales; no obstante, todos especulamos mucho sobre estos temas. Se nos enseña, y nosotros así lo creemos, que el miguel del Paraíso es el \bibemph{absoluto} de los conceptos dobles en cuanto deidad, que lo ha originado; por tanto, incorpora en sí facetas reales de la infinitud del Padre Universal y del Hijo Eterno. Los migueles deben ser una parte de la infinitud total, pero probablemente son absolutos en relación a esa parte de la infinitud que concierne a su origen. Si bien, al observar su labor en la presente era del universo, no notamos actuación alguna que sea más que finita; cualquier supuesta capacidad suprafinita debe de ser autocontenida y hasta el momento no revelada.
\vs p021 6:2 El cumplimiento de los ministerios de gracia en forma creatural y su elevación a la soberanía suprema de un universo han de significar la total liberación de la capacidad de acción finita de un miguel y la aparición de su capacidad para un servicio más que finito. Con respecto a esto, observamos que estos hijos mayores tienen entonces restricciones en cuanto a la creación de nuevos órdenes de seres, una restricción, sin duda, que se hace necesaria para liberar sus potenciales suprafinitos.
\vs p021 6:3 Es muy probable que estos poderes creadores no revelados permanezcan autocontenidos a lo largo de la presente era del universo. Si bien, en algún momento del futuro remoto, en los universos del espacio exterior que se están movilizando en este momento, creemos que el contacto entre un hijo mayor séptuplo y un espíritu creativo de la séptima etapa puede alcanzar niveles absonitos de servicio, en concurrencia con la aparición de nuevas cosas, contenidos y valores en niveles trascendentales que entrañan significados últimos para el universo.
\vs p021 6:4 Al igual que la Deidad del Supremo se está actualizando por virtud de su servicio experiencial, del mismo modo los hijos creadores están logrando la realización personal de los potenciales divinos paradisíacos ligados a sus insondables naturalezas. Cuando Cristo Miguel estaba en Urantia, dijo cierta vez: “Yo soy el camino, la verdad y la vida”. Y creemos que en la eternidad los migueles están prácticamente destinados a ser “el camino, la verdad y la vida”, e iluminar por siempre la senda para todos los seres personales del universo, una senda que lleva desde la divinidad suprema hasta la eterna completud en cuanto deidad, pasando por la absonitidad última.
\vsetoff
\vs p021 6:5 [Exposición de un perfeccionador de la sabiduría.]
