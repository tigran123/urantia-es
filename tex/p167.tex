\upaper{167}{Visita a Filadelfia}
\author{Comisión de seres intermedios}
\vs p167 0:1 A lo largo de todo este periodo de servicio ministerial en Perea, cuando se comenta que Jesús y los apóstoles visitaban las distintas localidades en las que trabajaban los setenta, conviene recordar que, por lo general, solo diez apóstoles iban con él; era habitual dejar al menos a dos de ellos en Pella para instruir a la multitud. Al prepararse Jesús para continuar su viaje a Filadelfia, Simón Pedro y su hermano Andrés regresaron al campamento de Pella para enseñar a las personas allí congregadas. Cuando el Maestro dejaba el campamento de Pella para hacer sus visitas en Perea, no era raro que lo siguieran entre trescientos y quinientos de los acampados. Cuando llegó a Filadelfia, lo acompañaban más de seiscientos seguidores.
\vs p167 0:2 Durante el reciente viaje de predicación por la Decápolis, no se había producido ningún milagro y, exceptuando la curación de los diez leprosos, hasta entonces, no había ocurrido tampoco ninguno en esta misión de Perea. Fue un espacio de tiempo en el que el evangelio se proclamaba con poder, sin milagros, y, la mayor parte del tiempo, sin la presencia personal de Jesús ni incluso la de sus apóstoles.
\vs p167 0:3 \pc Jesús y los diez apóstoles llegaron a Filadelfia el miércoles, 22 de febrero, y pasaron el jueves y el viernes descansando de sus últimos viajes y actividades. Aquel viernes por la noche, Santiago habló en la sinagoga, y se convocó un consejo general para las últimas horas de la tarde del siguiente día. Había un gran alborozo por el progreso del evangelio en Filadelfia y en las aldeas cercanas. Los mensajeros de David trajeron noticia de los nuevos avances del reino por toda Palestina, al igual que buenas nuevas de Alejandría y Damasco.
\usection{1. DESAYUNO CON LOS FARISEOS}
\vs p167 1:1 En Filadelfia, vivía un fariseo muy rico e influyente que había aceptado las enseñanzas de Abner, y que había invitado a Jesús a su casa para desayunar el \bibemph{sabbat} por la mañana. Se sabía que Jesús llegaría a Filadelfia en aquel momento, por lo que habían acudido de Jerusalén y de otros lugares un gran número de visitantes, entre ellos muchos fariseos. Por lo tanto, se había invitado a este desayuno, organizado en honor al Maestro, a unos cuarenta de estos hombres prominentes y a algunos intérpretes de la Ley.
\vs p167 1:2 Mientras Jesús se quedó en la puerta hablando con Abner, y una vez que el anfitrión hubiese tomado asiento, entró en la sala uno de los fariseos principales de Jerusalén, miembro del sanedrín, y como tenía por costumbre, se fue directamente hacia el asiento de honor, a la izquierda del anfitrión. Pero, puesto que este lugar había sido reservado para el Maestro y el de la derecha para Abner, el anfitrión hizo un gesto al fariseo de Jerusalén para que se sentara cuatro asientos a la izquierda. Este dignatario se sintió muy agraviado al no recibir su puesto de honor.
\vs p167 1:3 Pronto estuvieron todos sentados y disfrutando de la conversación, puesto que la mayoría de los presentes eran discípulos de Jesús o, al menos, favorables al evangelio. Solo sus enemigos se fijaron en el hecho de que él no cumplía con la ceremonia del lavado de manos antes de sentarse a la mesa para comer. Abner se lavó las manos al comienzo de la comida, aunque no mientras la servían.
\vs p167 1:4 Casi al final de la comida, llegó de la calle un hombre que durante mucho tiempo había padecido una enfermedad crónica que se había exacerbado ahora y le provocaba un estado hidrópico. Este hombre era un creyente a quien los compañeros de Abner habían bautizado recientemente. No pidió a Jesús que lo sanara, pero el Maestro sabía bien que aquel hombre enfermo había venido a este desayuno con la esperanza de eludir las multitudes que se agolpaban a su alrededor y tener así más posibilidades de llamar la atención del Maestro. Aquel hombre estaba informado de que entonces se obraban pocos milagros; sin embargo, había meditado en su corazón que tal vez su penosa situación pudiera apelar a la compasión del Maestro. Y no estaba equivocado porque, cuando entró en la sala, tanto Jesús como el arrogante fariseo de Jerusalén se percataron de él. El fariseo no tardó en expresar su indignación de que se le fuese permitida la entrada allí a alguien así. Pero Jesús miró al enfermo y le sonrió con tanta benevolencia que él se le acercó y se sentó en el suelo. Como el desayuno estaba acabando, el Maestro contempló a los otros invitados y luego, después de fijar su mirada significativamente en el hombre con hidropesía, dijo: “Amigos míos, maestros de Israel y eruditos intérpretes de la Ley, me gustaría haceros una pregunta: ¿Es lícito sanar a los enfermos y afligidos el día del \bibemph{sabbat?}”. Pero los allí presentes conocían muy bien a Jesús y callaron; no respondieron a su pregunta.
\vs p167 1:5 \pc Entonces, Jesús fue adonde estaba sentado el enfermo y, tomándolo de la mano, le dijo: “Levántate y vete. No has pedido que te sane, pero conozco el deseo de tu corazón y la fe de tu alma”. Antes de que el hombre abandonara la sala, Jesús volvió a su asiento y, dirigiéndose a los que estaban a la mesa, dijo: “Mi Padre hace tales obras no para induciros a que entréis al reino, sino para revelarse a los que ya están en el reino. Podéis entender que es propio del Padre hacer precisamente estas cosas, porque ¿quién de vosotros, si su animal favorito cae en algún pozo, no va y lo saca inmediatamente, aunque sea \bibemph{sabbat?}”. Y como nadie le replicó y, puesto que su anfitrión aprobaba evidentemente lo que estaba sucediendo, Jesús se puso en pie y habló a todos los presentes, diciéndoles: “Hermanos míos, cuando se os convide a unas bodas, no os sentéis en el primer lugar, no sea que otro más distinguido que vosotros esté también convidado y, viniendo el anfitrión, os pida que dejéis vuestro lugar a este otro convidado honorable. Y, en este caso, se os haga ocupar avergonzados el último lugar en la mesa. Cuando seáis convidados a alguna celebración, sería sensato que, al llegar a la mesa, vayáis y os sentéis en el último lugar, para que, cuando el anfitrión vea a sus convidados, él os diga: ‘Amigo mío, ¿por qué te sientas en el último asiento? Sube más arriba’; así tendréis el reconocimiento de los otros convidados. No os olvidéis: cualquiera que se enaltece será humillado, y el que de verdad se humilla será enaltecido. Por lo tanto, cuando hagáis comida o cena, no llaméis siempre a vuestros amigos ni a vuestros hermanos ni a vuestros parientes ni a vuestros vecinos ricos para que ellos os vuelvan a convidar a vosotros a sus celebraciones, y seáis pues recompensados. Cuando hagáis banquete, llamad algunas veces a los pobres, a los lisiados y a los ciegos. Y por estas cosas seréis bienaventurados en vuestro corazón, porque sabéis bien que los cojos y los ciegos no os pueden recompensar por vuestro amoroso servicio”.
\usection{2. PARÁBOLA DE LA GRAN CENA}
\vs p167 2:1 Cuando Jesús acabó de hablar en la mesa del desayuno del fariseo, uno de los intérpretes de la Ley allí presente, deseando romper el silencio, dijo de forma irreflexiva: “Bienaventurado el que coma pan en el reino de Dios” ---un dicho común en aquellos días---. Y entonces Jesús contó una parábola, que incluso su amigable anfitrión se sintió obligado a tomar en serio. Dijo:
\vs p167 2:2 “Cierto gobernante dio una gran cena, y convidó a muchos. A la hora de la cena envió a sus siervos a decir a los convidados: ‘Venid, que ya está todo listo’. Y todos a una comenzaron a excusarse. El primero dijo: ‘Acabo de comprar una hacienda, y necesito inspeccionarla; te ruego que me excuses’. Otro dijo: ‘he comprado cinco yuntas de bueyes, y debo ir por ellas; te ruego que me excuses’. Y otro dijo: ‘Acabo de casarme y por tanto no puedo ir’. Los siervos regresaron e hicieron saber estas cosas a su amo. Cuando el dueño de la casa oyó esto, se enojó bastante y, dirigiéndose a sus siervos, dijo: ‘He preparado esta fiesta de boda: se ha sacrificado a los animales cebados, y todo está dispuesto para mis convidados, pero han despreciado mi invitación; fueron cada cual a su labranza y a sus negocios, y otros le faltaron el respeto a mis siervos que los invitaron a venir a mi fiesta. Id pronto, pues, por calles y callejuelas de la ciudad, por grandes carreteras y vallados y traed de allí a los pobres y a los marginados, a los ciegos y a los cojos, que haya muchos convidados en la fiesta de boda’. Y los siervos hicieron lo que su señor les mandó, e incluso hubo lugar para más convidados. Entonces dijo el señor a sus siervos: ‘Id ahora por calzadas y campos y obligad a los que estén allí a que vengan para que mi casa esté llena. Os digo que ninguno de los que fueron convidados primero probará de mi cena’. Y los siervos hicieron lo que su amo les había mandado, y la casa se llenó”.
\vs p167 2:3 \pc Y cuando oyeron estas palabras, se marcharon; cada uno se fue a su casa. Al menos uno de los despectivos fariseos presentes esa mañana, comprendió el significado de esta parábola, porque se bautizó aquel día e hizo confesión pública de su fe en el evangelio del reino. Esa noche, Abner predicó sobre esta parábola en el consejo general de los creyentes.
\vs p167 2:4 Al día siguiente todos los apóstoles participaron en el ejercicio filosófico de tratar de interpretar el significado de esta parábola de la gran cena. Aunque Jesús escuchó con interés las distintas interpretaciones, se negó rotundamente a ofrecerles cualquier otra ayuda para que entendieran la parábola. Tan solo dijo: “Que cada cual encuentre el significado por sí mismo y en su propia alma”.
\usection{3. LA MUJER CON ESPÍRITU DE ENFERMEDAD}
\vs p167 3:1 Abner había hecho arreglos para que el Maestro enseñara en la sinagoga ese día de \bibemph{sabbat;} se trataba de la primera vez que Jesús hacía su aparición en una sinagoga desde que, por orden del sanedrín, todas se habían cerrado a sus enseñanzas. Al concluir el servicio, Jesús miró hacia abajo y, ante él, vio a una mujer de edad avanzada, muy encorvada, que denotaba abatimiento. Hacía mucho tiempo que se sentía atenazada por el miedo y que la alegría se había desvanecido de su vida. Al bajar del púlpito, Jesús se dirigió a ella y, poniendo la mano en el hombro de su cuerpo encorvado, dijo: “Mujer, con tan solo creer, estarías por completo libre de tu espíritu de enfermedad”. Y esta mujer, que llevaba dieciocho años encorvada y deprimida por miedos que la atenazaban, creyó las palabras del Maestro y gracias a su fe se enderezó al momento. Cuando esta mujer vio que se había puesto erguida, alzó la voz glorificando a Dios.
\vs p167 3:2 A pesar de que la aflicción de esta mujer era enteramente mental y estaba encorvada como resultado de su depresión mental, la gente pensó que Jesús la había curado de un desorden físico real. Aunque la congregación de la sinagoga de Filadelfia era favorable a las enseñanzas de Jesús, el jefe principal de la sinagoga era un fariseo poco amistoso. Y al compartir la opinión de la congregación de que Jesús acababa de curar un trastorno físico, e indignarse por el atrevimiento de Jesús de hacer tal cosa el día del \bibemph{sabbat,} se puso de pie ante la congregación y dijo: “¿No hay seis días en que se debe trabajar? En estos, pues, venid y sed sanados, y no en el día del \bibemph{sabbat}”.
\vs p167 3:3 Cuando aquel hostil dignatario habló así, Jesús volvió a la tarima de los oradores y dijo: “¿Por qué actuáis con hipocresía? ¿No desatáis cada uno de vosotros vuestro buey del pesebre y lo lleváis a beber en el \bibemph{sabbat?} Si tal servicio es admisible el día del \bibemph{sabbat,} ¿por qué no puede esta mujer, hija de Abraham, sujeta a su mal durante estos dieciocho años, desatarse de esta ligadura y ser llevada a compartir del agua de la libertad y de la vida, aunque sea en este día del \bibemph{sabbat?}”. Y mientras la mujer continuaba glorificando a Dios, su detractor se avergonzó, y la congregación se regocijó con ella por su curación.
\vs p167 3:4 A raíz de haber criticado a Jesús en público en aquel \bibemph{sabbat,} el jefe principal de la sinagoga fue destituido, y se puso en su lugar a un seguidor de Jesús.
\vs p167 3:5 \pc Con frecuencia, Jesús liberaba a estas víctimas del temor de su espíritu de enfermedad, de su depresión mental y de sus ataduras al miedo. Pero la gente creía que todas estas aflicciones o eran enfermedades físicas o posesión de espíritus malignos.
\vs p167 3:6 \pc Jesús enseñó de nuevo en la sinagoga el domingo y Abner, al mediodía de aquella jornada, bautizó a muchos en el río que fluía por el sur de la ciudad. Al día siguiente, Jesús y los diez apóstoles habrían emprendido la vuelta al campamento de Pella, a no ser por la llegada de uno de los mensajeros de David, que trajo a Jesús un mensaje urgente de sus amigos de Betania, aldea cercana a Jerusalén.
\usection{4. NOTICIAS DE BETANIA}
\vs p167 4:1 El domingo26 de febrero, tarde ya en la noche, llegó a Filadelfia, desde Betania, un corredor con un mensaje de Marta y María, que decía: “Señor, el que amas está muy enfermo”. El mensaje le llegó a Jesús al término de su charla vespertina y justo en el momento en el que se despedía de sus apóstoles para ir a dormir. Al principio, Jesús no respondió. Sucedió entonces una de esas extrañas pausas, un lapso de tiempo en el que parecía estar en comunicación con algo que estaba fuera, y más allá, de él mismo. Y, luego, levantando la mirada, se dirigió al mensajero a oídos de los apóstoles, diciendo: “Esta enfermedad no es en verdad para muerte. No dudéis de que es para la gloria de Dios y para que el Hijo sea enaltecido por ella”.
\vs p167 4:2 \pc Jesús sentía un gran cariño por Marta, María y su hermano Lázaro; los amaba con fervoroso afecto. Su primer pensamiento humano fue ir enseguida en su ayuda, pero otra idea acudió a su mente humana\hyp{}divina. Casi había renunciado a la esperanza de que los líderes judíos de Jerusalén aceptaran alguna vez el reino, pero aún amaba a su pueblo, y se le ocurrió entonces un plan por el que los escribas y los fariseos de Jerusalén pudieran tener otra oportunidad de aceptar sus enseñanzas; por lo que, conforme a la voluntad del Padre, decidió darle visibilidad, en este último llamamiento a Jerusalén, a la obra más formidable y de mayor envergadura de toda su andadura en la tierra. Los judíos se aferraban a la idea de un libertador, obrador de prodigios. Y aunque él se negaba a realizar portentos materiales o cualquier expresión temporal de poder político, ciertamente pidió en aquel momento la aprobación del Padre para manifestar su poder, hasta entonces no mostrado, sobre la vida y la muerte.
\vs p167 4:3 \pc Los judíos tenían por costumbre dar sepultura a sus muertos el día de su deceso; era necesario proceder así en un clima tan cálido. Sucedía a menudo que se sepultaba a alguien que simplemente estaba en estado comatoso, de modo que al segundo o incluso al tercer día, dicha persona salía de la tumba. Pero los judíos creían que, aunque el espíritu o el alma pudieran permanecer cerca del cuerpo durante dos o tres días, nunca lo hacían tras el tercer día; que, para el cuarto, el proceso de descomposición del cuerpo ya estaba bastante avanzado y que nadie regresaba jamás de la tumba, transcurrido ese período de tiempo. Y fue por estas razones por las que Jesús se quedó dos días más completos en Filadelfia antes de disponerse a partir para Betania.
\vs p167 4:4 \pc Así pues, temprano en la mañana del miércoles, Jesús dijo a sus apóstoles: “Dispongámonos enseguida para ir de nuevo a Judea”. Y cuando los apóstoles oyeron decir aquello a su Maestro, se apartaron un momento para hablar entre ellos. Santiago dirigió la conversación, y todos acordaron que era una gran insensatez permitirle que regresara a Judea, y volvieron todos a una para comunicárselo a Jesús. Santiago dijo: “Maestro, estuviste en Jerusalén hace unas pocas semanas, y los líderes judíos querían matarte y la gente intentó apedrearte. En ese momento les diste a esos hombres su oportunidad de recibir la verdad, y no permitiremos que vuelvas a Judea”.
\vs p167 4:5 Entonces Jesús les dijo: “Pero, ¿no entendéis que el día tiene doce horas en las que se puede hacer el trabajo sin peligro? El que anda de día no tropieza, porque tiene luz, pero el que anda de noche, puede que tropiece porque no hay luz en él. Siempre que dure mi día, no temo entrar a Judea. Quiero obrar otra imponente labor para esos judíos; quiero darles una nueva oportunidad de que crean, incluso en sus propios términos ---aceptando sus condiciones de querer ver la manifestación visible de la gloria y el poder del Padre y del amor del Hijo---. Además, ¡no os dais cuenta de que nuestro amigo Lázaro duerme, y quiero ir a despertarlo!”.
\vs p167 4:6 Dijo entonces, uno de los apóstoles: “Maestro, si Lázaro duerme, es más seguro que sane”. En aquel tiempo, los judíos tenían la costumbre de hablar de la muerte como del reposar del sueño, pero como los apóstoles no entendían que Jesús quería decir que Lázaro había partido de este mundo, entonces les dijo claramente: “Lázaro ha muerto. Y me alegro por vosotros de no haber estado allí, incluso si los demás no sean salvos por ello, para que así podáis tener un nuevo motivo de creer en mí; y lo que presenciaréis os fortalezca y os sirva de preparación para ese día en el que me despediré de vosotros para ir al Padre”.
\vs p167 4:7 Al no poder convencerlo para que no fuera a Judea y, como algunos de los apóstoles se mostraban reacios incluso a acompañarlo, Tomás se dirigió a sus compañeros, diciendo: “Ya le hemos comunicado al Maestro nuestros temores, pero él está dispuesto a ir a Betania. Estoy seguro de que esto significa el fin; sin duda lo matarán, pero si esa es la elección del Maestro, actuemos como hombres valientes; vayamos también nosotros, para que muramos con él”. Y siempre fue así; en los asuntos que requerían de decidido y permanente arrojo, Tomás siempre fue el apoyo principal de los doce apóstoles.
\usection{5. DE CAMINO A BETANIA}
\vs p167 5:1 De camino a Judea, a Jesús lo seguía un grupo de casi cincuenta personas entre amigos y enemigos. Al mediodía del miércoles, a la hora del almuerzo, habló a sus apóstoles y a este grupo de seguidores sobre “Las condiciones para ser salvos”; y al terminar esta lección contó la parábola del fariseo y del publicano (un recaudador de impuestos). Jesús dijo: “Veis, pues, que el Padre da la salvación a los hijos de los hombres, y esta salvación es un don gratuito para todos los que poseen fe para recibir la filiación en la familia divina. No hay nada que el hombre pueda hacer para ganar esta salvación. Los actos de santurronería no compran el favor de Dios, las oraciones en público no compensan la falta de fe viva en el corazón. Podréis engañar con vuestro servicio ante los demás, pero Dios ve el interior de vuestras almas. Lo que yo os digo está bien ejemplificado por dos hombres que fueron al templo a orar, uno era fariseo y, el otro, publicano. El fariseo, puesto en pie, oraba consigo mismo de esta manera: ‘Dios, te doy gracias porque no soy como los otros hombres, que son estafadores, ignorantes, injustos, adúlteros ni aun como este publicano; ayuno dos veces a la semana; diezmo todo lo que gano’. Pero el publicano, estando lejos, no quería ni aun alzar los ojos al cielo, sino que se golpeaba el pecho, diciendo, ‘Dios, sé propicio a mí, pecador’. Os digo que el publicano se fue a casa con la aprobación de Dios antes que el fariseo, porque cualquiera que se enaltece será humillado y el que se humilla será enaltecido”.
\vs p167 5:2 \pc Esa noche en Jericó, los fariseos hostiles quisieron tender al Maestro una trampa, haciendo que hablase del matrimonio y del divorcio, como hicieron otros compañeros suyos cierta ocasión en Galilea, pero Jesús habilidosamente esquivó entrar en conflicto con sus leyes sobre el divorcio. Al igual que el publicano y el fariseo ejemplificaban la buena y la mala religión, sus métodos de divorcio indicaban la diferencia entre las mejores leyes matrimoniales del código judío, comparada con la lamentable laxitud de la interpretación farisea de estos estatutos mosaicos sobre el divorcio. El fariseo se juzgaba a sí mismo por los peores criterios; el publicano se evaluaba a sí mismo según los más altos ideales. Para el fariseo, la devoción era un medio de incitarle a una inactividad santurrona y a la fiabilidad de una falsa seguridad espiritual; para el publicano, la devoción era un medio de estimular el alma a la concienciación de la necesidad de arrepentirse, confesar y aceptar, mediante la fe, el perdón misericordioso. El fariseo buscaba justicia; el publicano, misericordia. La ley del universo es: Pedid, y se os dará; buscad, y hallaréis.
\vs p167 5:3 Aunque Jesús rehusó involucrarse en controversias con los fariseos sobre el divorcio, sí enseñó de manera constructiva los ideales más altos respecto al matrimonio. Realzó el matrimonio como la relación humana más elevada e ideal de todas. De igual manera, dio a entender su enérgica reprobación de las costumbres laxas e injustas sobre el divorcio de los judíos de Jerusalén, que permitían, en aquella época, que un hombre divorciara a su esposa por las razones más nimias, como ser una mala cocinera o una deficiente ama de casa o, por la no mejor razón, de haberse enamorado de una mujer más bella.
\vs p167 5:4 Los fariseos habían incluso llegado a enseñar que tal forma de divorcio permisivo era una exención especial concedida al pueblo judío, particularmente a los fariseos. Y, por ello, aunque Jesús evitó pronunciarse sobre el matrimonio y el divorcio, denunció con gran contundencia estas vergonzantes muestras de desprecio hacia las relaciones matrimoniales y puso de relieve su injusticia hacia las mujeres y los niños. Nunca dio su aprobación a práctica alguna de divorcio que diera al hombre cualquier tipo de ventaja sobre la mujer; el Maestro tan solo aprobaba aquellas enseñanzas que permitían a las mujeres estar en igualdad con los hombres.
\vs p167 5:5 Aunque Jesús no impartió nuevos mandatos que rigiesen el matrimonio y el divorcio, sí instó a que vivieran a la altura de sus propias leyes y enseñanzas superiores. Con frecuencia, hizo referencias a las Escrituras en su afán por mejorar sus costumbres en cuestiones sociales. Aunque defendió pues un concepto elevado e ideal del matrimonio, eludió hábilmente el enfrentamiento con sus interrogadores sobre las prácticas sociales que figuraban tanto en sus leyes escritas como en sus tan preciados privilegios sobre el divorcio.
\vs p167 5:6 A los apóstoles les resultaba muy difícil comprender la reticencia del Maestro a pronunciarse claramente sobre cuestiones científicas, sociales, económicas y políticas. No comprendían bien que su misión en la tierra suponía dedicarse exclusivamente a la revelación de las verdades espirituales y religiosas.
\vs p167 5:7 Después de que Jesús hablara sobre el matrimonio y el divorcio, ya más avanzada la noche, sus apóstoles le hicieron privadamente otras muchas preguntas, y las respuestas que él les dio sirvieron para liberar sus mentes de muchas ideas erróneas. Al terminar esta charla, Jesús dijo: “El matrimonio es una digna institución y ha de ser deseado por todos los hombres. El hecho de que el Hijo del Hombre desarrolle por sí solo su misión en la tierra no significa que crea que el matrimonio sea desaconsejable. Que yo deba ejercer así mi labor es voluntad del Padre, pero el mismo Padre ha ordenado la creación del hombre y de la mujer, y es voluntad divina que los hombres y las mujeres hallen su servicio más elevado y su consiguiente gozo en la instauración de hogares para recibir y formar a los hijos, en cuya creación estos padres se convierten en copartícipes con los Hacedores del cielo y de la tierra. “Por esto el hombre dejará padre y madre, y se unirá a su mujer, y los dos serán uno solo”.
\vs p167 5:8 Y de este modo, Jesús alivió la mente de los apóstoles de muchas de sus preocupaciones sobre el matrimonio y aclaró sus numerosas nociones equivocadas sobre el divorcio; al mismo tiempo, hizo bastante por enaltecer sus ideales de los nexos sociales y ampliar su respeto hacia las mujeres, los niños y el hogar.
\usection{6. BENDICIÓN DE LOS NIÑOS}
\vs p167 6:1 Esa noche, el mensaje de Jesús sobre el matrimonio y la bendición que significan los niños se difundió por todo Jericó, de manera que, a la mañana siguiente, mucho antes de que Jesús y los apóstoles se dispusieran a partir, incluso antes de la hora del desayuno, un gran número de madres acudieron adonde Jesús se alojaba; traían a sus hijos en brazos y agarrados de sus manos con el deseo de que él los bendijera. Cuando los apóstoles salieron y vieron a aquella aglomeración de mujeres con sus niños, trataron de alejarlas de allí, pero las mujeres se negaron a irse hasta que el Maestro no impusiera sus manos sobre los pequeños y los bendijera. Y cuando los apóstoles las reprendieron en voz alta, Jesús, oyendo el tumulto, salió, y con indignación les reprochó su actitud, diciendo: Dejad que los niños vengan a mí y no se lo impidáis, porque de ellos es el reino de Dios. De cierto, de cierto, os digo que el que no recibe el reino de Dios como un niño difícilmente podrá entrar en él y crecer hasta la plena estatura de la madurez espiritual”.
\vs p167 6:2 Y cuando el Maestro habló con sus apóstoles, recibió a todos los niños y les impuso las manos, mientras les daba a las madres valor y esperanza con sus palabras.
\vs p167 6:3 \pc A menudo, Jesús hablaba a sus apóstoles de las estancias celestiales y les enseñó que los hijos de Dios, al avanzar, deben crecer espiritualmente en ellas al igual que los niños crecen físicamente en este mundo. Y, así pues, a menudo lo sagrado tiene apariencia de común como sucedió aquel día, cuando estos hijos y sus madres no se dieron cuenta de que las inteligencias expectantes de Nebadón contemplaban a los niños de Jericó jugar con el creador de un universo.
\vs p167 6:4 \pc En Palestina, el estatus de la mujer mejoró bastante gracias a las enseñanzas de Jesús; algo que habría sucedido en todo el mundo si sus seguidores no se hubieran apartado tanto de lo que él, con tanta laboriosidad, les había enseñado.
\vs p167 6:5 \pc Fue también en Jericó, con motivo de una conversación sobre la formación religiosa inicial de los niños en los hábitos de la adoración divina, cuando Jesús recalcó a sus apóstoles el gran valor de la belleza cuya apreciación lleva a la adoración, especialmente en el caso de los niños. El Maestro mediante preceptos y ejemplos enseñó el valor de adorar al Creador en los entornos naturales de la creación. Prefería estar en comunión con el Padre celestial en medio de los árboles y entre las criaturas humildes del mundo natural. Gozaba y le inspiraba contemplar al Padre a través del espectacular fenómeno de las regiones estelares de los hijos creadores.
\vs p167 6:6 Cuando no sea posible adorar a Dios en los tabernáculos de la naturaleza, los hombres deberían hacer lo posible por disponer de bellas casas de culto, santuarios de atrayente sencillez y embellecimiento artístico, capaces de suscitar las emociones humanas de mayor elevación junto con la senda intelectual que lleva a la comunión espiritual con Dios. La verdad, la belleza y la veneración ayudan poderosa y eficazmente a lograr una verdadera adoración. Pero la comunión espiritual no se fomenta con la mera ornamentación excesiva ni con el embellecimiento sobrecargado del intrincado y ostentoso arte del hombre. La belleza es más religiosa cuando más sencilla sea y más se asemeje a la naturaleza. ¡Qué lástima iniciar a los niños pequeños a la idea de la adoración pública en espacios fríos y baldíos, tan desprovistos del atractivo de la belleza y exentos de cualquier apelación a los sentimientos de alegría y a una sacralidad que los inspire! El niño debería iniciarse a la adoración al aire libre, en la naturaleza y, más tarde, acompañado por sus padres, asistir a lugares asamblearios religiosos que sean al menos tan atractivos en el sentido material y tan artísticamente hermosos como el hogar en el que residen diariamente.
\usection{7. CHARLA SOBRE LOS ÁNGELES}
\vs p167 7:1 Mientras viajaban colina arriba desde Jericó hasta Betania, Natanael caminó la mayoría del tiempo al lado de Jesús, y su conversación sobre los niños en relación al reino de los cielos llevó indirectamente al examen del ministerio de los ángeles. Natanael acabó por hacerle al Maestro la siguiente pregunta: “Dado que el sumo sacerdote es un saduceo y, puesto que los saduceos no creen en los ángeles, ¿qué le enseñaremos a la gente respecto a los servidores celestiales?”. Entonces, entre otras cosas, Jesús dijo:
\vs p167 7:2 \pc “Las multitudes angélicas son un orden separado de seres creados; son completamente distintos del orden material de las criaturas mortales, y actúan como un grupo diferenciado de inteligencias del universo. Los ángeles no forman parte de ese colectivo de criaturas, que las Escrituras califican como “los Hijos de Dios”; como tampoco son espíritus glorificados de los hombres mortales que han avanzado a través de las estancias de lo alto. Los ángeles se crean de forma directa y no se reproducen. Las multitudes angélicas y la raza humana solo son afines espiritualmente. A medida que el hombre progresa en su viaje hacia el Padre del Paraíso, durante algún tiempo, pasa de hecho por un estado de existencia análogo al de los ángeles, pero el hombre mortal jamás se convierte en ángel.
\vs p167 7:3 “Los ángeles no mueren nunca, como sí le pasa al hombre. Los ángeles son inmortales, salvo que se involucren en el pecado, tal como sucedió con algunos de los que creyeron en las mentiras de Lucifer. Los ángeles son los sirvientes espirituales del cielo, y no son omnisapientes ni todopoderosos. Pero todos los ángeles leales son verdaderamente puros y santos.
\vs p167 7:4 “¿Y es que no recuerdas lo que ya os dije alguna vez que si vuestros ojos espirituales estuvieran ungidos, veríais entonces los cielos abiertos y a los ángeles de Dios subiendo y bajando? Gracias al ministerio de los ángeles, los mundos se mantienen en contacto unos con otros, porque ¿no os he dicho repetidas veces que tengo otras ovejas que no son de este redil? Y estos ángeles no son los espías del mundo espiritual que os vigilan y luego van al Padre y le cuentan los pensamientos de vuestro corazón y le informan de los actos de la carne. El Padre no precisa de tal servicio porque su propio espíritu vive en vosotros. Pero estos espíritus angélicos sí realizan la tarea de mantener informada a una parte de la creación celestial de lo que sucede en otras partes distantes del universo. Y a muchos de los ángeles, mientras ejercen su labor en el gobierno del Padre y en los universos de los hijos, se les asigna al servicio de las razas humanas. Cuando os enseñé que muchos de estos serafines son espíritus servidores, no os hablaba en lenguaje figurativo ni en un sentido poético. Y todo esto es verdad, al margen de vuestra dificultad para comprender estas cuestiones.
\vs p167 7:5 “Muchos de estos ángeles se dedican a la labor de salvar a los hombres, porque, ¿es que no os he hablado del gozo de los serafines cuando un alma elige renunciar al pecado y comenzar la búsqueda de Dios? Ciertamente, os he hablado incluso del gozo en la \bibemph{presencia de los ángeles} del cielo cuando un pecador se arrepiente, indicando con ello la existencia de otros órdenes superiores de seres celestiales, que asimismo se ocupan del bienestar espiritual y del progreso divino del hombre mortal.
\vs p167 7:6 “Estos ángeles están de igual manera sumamente implicados respecto a los medios por los que el espíritu del hombre se ve liberado de los tabernáculos de la carne y su alma escoltada a las estancias del cielo. Los ángeles son los guías inequívocos y celestiales del alma del hombre durante ese período de tiempo ignoto e indefinido que media entre la muerte de la carne y la vida nueva en las moradas espirituales”.
\vs p167 7:7 \pc Y habría continuado hablando con Natanael sobre el ministerio de los ángeles pero se vio interrumpido al acercarse Marta; unos amigos que habían visto al Maestro ascender las colinas por el este la habían informado de que el Maestro se estaba aproximando a Betania. Y ahora ella se apresuraba a saludarlo.
