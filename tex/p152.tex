\upaper{152}{Los sucesos que desembocaron en la crisis de Cafarnaúm}
\author{Comisión de seres intermedios}
\vs p152 0:1 La noticia de la curación de Amós, el lunático de Queresa, ya había llegado hasta Betsaida y Cafarnaúm, por lo que una gran multitud esperaba a Jesús cuando la barca tocó tierra aquel martes por la mañana. Entre esta gran afluencia de personas se encontraban los nuevos enviados del sanedrín de Jerusalén, llegados a Cafarnaúm para vigilar al Maestro y buscar pruebas para su captura y condena. Mientras Jesús hablaba con los que se habían congregado allí para darle la bienvenida, Jairo, uno de los jefes de la sinagoga, se abrió paso entre la multitud y, arrodillándose a sus pies, lo tomó de la mano y le suplicó que se diera prisa en acompañarlo, diciendo: “Maestro, mi joven y única hija está agonizando. Te ruego que vengas y la sanes”. Cuando Jesús oyó lo que este padre le pedía, dijo: “Iré contigo”.
\vs p152 0:2 Jesús fue pues con él, seguido de una gran muchedumbre que había oído el ruego de aquel padre y quería saber qué pasaría. Poco antes de llegar a la casa del dignatario, mientras caminaban de prisa por una calle estrecha, rodeado del gentío que se agolpaba en torno a él, Jesús se detuvo de repente, y exclamó: “Alguien me ha tocado”. Y cuando todos los que estaban a su lado lo negaron, Pedro habló: “Ves que la multitud te aprieta, amenazando con aplastarnos, y dices “alguien me ha tocado”. ¿Qué quieres decir?”. Entonces Jesús contestó: “Pregunté quién me tocó, porque percibí que de mí había salido energía de vida”. Al mirar Jesús a su alrededor, sus ojos se posaron sobre una mujer próxima a él, la cual, adelantándose, se postró a sus pies y dijo: “Hace años que padezco de flujo de sangre. He sufrido bastante a manos de muchos médicos; he gastado toda mi fortuna, pero ninguno ha podido curarme. Entonces supe de ti y me dije que si tan solo pudiera tocar la orilla de tu manto, sanaría sin duda. Y me metí entre la gente que te sigue hasta llegar hasta ti, Maestro, y toqué el borde de tu manto, y se me restauró la salud; sé que he sido curada de mi aflicción”.
\vs p152 0:3 Cuando Jesús oyó estas palabras, tomó a la mujer de la mano y, levantándola, le dijo: “Hija, tu fe te ha salvado. Vete en paz”. Era su \bibemph{fe} y no su \bibemph{contacto} lo que le había hecho recuperar la salud. Y este caso es un buen ejemplo de muchas curas aparentemente milagrosas que se produjeron durante la andadura terrenal de Jesús, pero que, de ningún modo, él deseaba conscientemente que ocurrieran. Con el trascurso del tiempo, se demostró que aquella mujer había quedado realmente curada de su mal. La suya era un tipo de fe que se hacía directamente del poder creativo presente en la persona del Maestro. Con la fe que sentía solo le era suficiente con acercarse a la persona del Maestro. No necesitaba en absoluto tocar su manto; sino que aquello era la parte supersticiosa de su creencia. Jesús llamó a Verónica, esta mujer de Cesarea de Filipo, a su presencia para corregir dos errores que podría albergar en su mente o que persistiesen en las de aquellos que habían sido testigos de tal curación. No quería que Verónica se marchase pensando que su temido intento de robarle la cura había sido premiado, o que su actitud, supersticiosa, de relacionar el hecho de tocar su manto con su sanación había surtido efecto. Quería que todos supieran que era su \bibemph{fe,} pura y viva, la que había obrado la cura.
\usection{1. EN CASA DE JAIRO}
\vs p152 1:1 Como era natural, Jairo estaba muy impaciente por el retraso surgido en llegar a su casa; así que aceleraron bastante el paso. Pero ya antes de entrar en el patio del dignatario, uno de sus siervos salió diciendo: “No molestes más al Maestro; tu hija ha muerto”. Pero Jesús pareció no oír las palabras del siervo, porque, llevando consigo a Pedro, Santiago y Juan, se volvió al desconsolado padre diciéndole: “No temas; cree solamente”. Cuando se adentró en la casa, encontró ya allí a los flautistas con las plañideras, formando un alboroto impropio; también estaban los parientes llorando y lamentándose. Y cuando había echado a las plañideras fuera de la habitación, Jesús entró con el padre, la madre y sus tres apóstoles. Les había dicho a las plañideras que la joven no estaba muerta, pero se burlaron de él y lo despreciaron. Entonces, Jesús se volvió a la madre, diciéndole: “Tu hija no está muerta, sino que duerme”. Y cuando la casa estaba en calma, yendo adonde la niña estaba tendida la tomo de la mano y clamó: “Hija, a ti te digo, ¡despierta y levántate!”. Y cuando la muchacha oyó estas palabras, inmediatamente se levantó y caminó por la habitación. Y, seguidamente, una vez recuperada de su aturdimiento, Jesús mandó que le dieran algo de comer, porque llevaba mucho tiempo sin haber tomado alimentos.
\vs p152 1:2 Al haberse levantado un gran revuelo en Cafarnaún en contra de Jesús, él llamó a toda la familia y les explicó que la joven había estado en coma tras una larga fiebre y que no la había resucitado de entre los muertos, sino simplemente despertado. De igual manera, les comunicó todo esto a sus apóstoles, pero resultó inútil; todos ellos creyeron que la había devuelto a la vida. Cualquier cosa que dijera para aclarar muchos de estos aparentes milagros tenía poco efecto sobre sus seguidores. Estaban obstinados en los milagros, y no dejaban pasar la oportunidad de atribuirle a Jesús un prodigio más. Jesús y los apóstoles regresaron a Betsaida una vez que él les encargara expresamente a todos ellos que no dijeran nada a nadie sobre aquello.
\vs p152 1:3 \pc Cuando Jesús salió de la casa de Jairo, dos ciegos, a quienes un muchacho mudo les servía de guía, lo siguieron y le decían a gritos que los curara. En aquel momento, la fama de Jesús como sanador estaba en su punto más álgido. Dondequiera que iba, allí estaban los enfermos y los afligidos esperándolo. El Maestro se veía ahora agotado, y esto empezaba a preocupar a todos sus amigos temiendo que su continua labor de enseñanza y curación lo llevara hasta el colapso físico.
\vs p152 1:4 \pc Los apóstoles de Jesús no podían entender la naturaleza y atributos de este Dios\hyp{}hombre, y mucho menos la gente común. Como tampoco ninguna de las generaciones venideras han sido capaces de determinar lo que sucedió mientras Jesús de Nazaret estaba en persona en la tierra. Y ni la ciencia ni la religión tendrán posibilidad alguna de examinar estos hechos excepcionales por la sencilla razón de que no volverán a reunirse tales extraordinarias circunstancias, ni en este mundo ni en ningún otro mundo de Nebadón. Jamás aparecerá de nuevo, en ningún mundo del universo, un ser con la semejanza de un hombre mortal que personifique, al mismo tiempo, todos los atributos de la energía creativa sumados a unos dones espirituales que trascienden el tiempo y la mayoría de las otras limitaciones materiales.
\vs p152 1:5 Nunca antes de que Jesús estuviera en la tierra, ni desde entonces, han podido los hombres y las mujeres mortales obtener, por su fe, fuerte y viva, resultados de forma tan directa y vívida. Para que estos fenómenos se repitieran, tendríamos que llegar hasta la presencia inmediata de Miguel, el Creador, y encontrarlo tal como fue en aquellos días ---el Hijo del Hombre---. Asimismo, hoy, aunque su ausencia impide tales manifestaciones materiales, deberíais absteneros de imponer cualquier índole de limitación sobre la posible expresión de su \bibemph{poder espiritual}. Aunque el Maestro esté físicamente ausente, está espiritualmente presente y actúa en los corazones de los hombres. Al irse de este mundo, Jesús facilitó que su espíritu viviera al lado del de su Padre, que habita en las mentes de todos los seres humanos.
\usection{2. ALIMENTACIÓN DE LOS CINCO MIL}
\vs p152 2:1 Jesús continuó enseñando a la gente durante el día e instruyendo a los apóstoles y a los evangelistas por la noche. El viernes les anunció un permiso de una semana para que en esos días todos sus seguidores pudieran ir a casa o ir a ver a sus amigos, antes de prepararse y dirigirse a Jerusalén para la Pascua. Pero más de la mitad de sus discípulos se negaron a abandonarlo, y la multitud iba incrementando día a día, tanto que David Zebedeo quiso establecer un nuevo campamento, pero Jesús no dio su consentimiento. El Maestro había descansado tan poco el \bibemph{sabbat} que la mañana del domingo, 27 de marzo, procuró alejarse de la gente. Algunos de los evangelistas se quedaron allí para hablar a la multitud, mientras que Jesús y los doce planeaban marcharse, inadvertidamente, hacia la otra orilla del lago. Allí podrían disfrutar de un necesario descanso en un hermoso parque al sur de Betsaida\hyp{}Julias. Esta región era el lugar de vacaciones favorito de los habitantes de Cafarnaúm; todos estaban familiarizados con estos parques de la costa oriental.
\vs p152 2:2 Pero, la gente no estaba dispuesta a permitírselo. Vieron la dirección que la barca había tomado, y alquilando cualquier embarcación posible, comenzaron a perseguirlos. Los que no lograron conseguir barcas, partieron a pie, rodeando el tramo superior del lago.
\vs p152 2:3 A última hora de la tarde, más de mil personas habían localizado ya al Maestro en uno de los parques, y él les dirigió unas breves palabras; Pedro les habló a continuación. Muchos de ellos habían traído su propia comida y, después de cenar, se reunieron en pequeños grupos mientras los apóstoles y discípulos de Jesús les impartían enseñanzas.
\vs p152 2:4 El lunes por la tarde, la multitud había aumentado hasta más de tres mil personas. Y todavía ---hasta bien entrada la noche--- la gente, trayendo a todo tipo de enfermos con ellos, continuaba congregándose. Cientos de personas, de camino a la Pascua, habían decidido hacer una parada en Cafarnaúm interesados en ver y oír a Jesús, y no querían de ninguna manera sentirse defraudados. Hacia el mediodía del miércoles, en este parque del sur de Betsaida\hyp{}Julias, se habían reunido unos cinco mil hombres, mujeres y niños. El tiempo era placentero; se acercaba el fin de la estación lluviosa de esta zona.
\vs p152 2:5 \pc Felipe había traído víveres, que estaban al cuidado de Marcos, el joven recadero, con la intención de alimentar a Jesús y los doce durante tres días. Hacia la tarde de este día, el tercero para casi la mitad de los que se habían congregado allí, la multitud ya había consumido casi toda la comida que había traído. David Zebedeo no disponía aquí de una ciudad hecha de tiendas para poder alimentar y albergar a tal número de personas ni Felipe se había aprovisionado de tanto alimento. No obstante, la gente, a pesar de estar hambrienta, no quería irse. Se murmuraba que Jesús, queriendo evitar problemas tanto con Herodes como con los líderes de Jerusalén, había elegido este sitio tranquilo, fuera de la jurisdicción de todos sus enemigos, por ser el lugar más conveniente para ser coronado rey. De hora en hora el entusiasmo de la gente iba en aumento. A Jesús no se le había dicho ni una sola palabra de esto, aunque él sabía claramente todo lo que estaba sucediendo. Incluso los doce apóstoles se habían contagiado de estas ideas y los evangelistas más jóvenes, aún más. Los apóstoles que estaban a favor de proclamar rey a Jesús eran Pedro, Juan, Simón Zelotes y Judas Iscariote. Los que se oponían a este plan eran Andrés, Santiago, Natanael y Tomás. Mateo, Felipe y los gemelos Alfeo estaban indecisos. Joab, uno de los jóvenes evangelistas, era el cabecilla de esta trama para convertirlo en rey.
\vs p152 2:6 \pc Esta era la situación hacia las cinco de la tarde del miércoles, cuando Jesús le pidió a Santiago Alfeo que llamara a Andrés y Felipe. Jesús dijo: “¿Qué vamos a hacer con la multitud? Ya llevan tres días con nosotros, y hay muchos que tienen hambre. No tienen comida”. Felipe y Andrés se miraron entre sí, y entonces Felipe respondió: “Maestro, mándalos a que vayan a las aldeas de alrededor y compren alimentos”. Y Andrés, temeroso de que se llevara a cabo la trama urdida para coronar rey a Jesús, se sumó rápidamente a Felipe, diciendo: “Sí, Maestro, pienso que sería mejor despedirlos para que sigan su camino y consigan comida, mientras descansas un rato”. En aquel momento, otros apóstoles se habían sumado a la conversación. Entonces, Jesús dijo: “Pero no deseo dejarles marchar hambrientos; ¿es que no podéis darles de comer?”. Aquello enervó a Felipe, que contestó inmediatamente: “Maestro, ¿dónde podemos ir a comprar pan para la multitud aquí en medio del campo? Ni doscientos denarios serían suficientes para el almuerzo”.
\vs p152 2:7 Antes de que los apóstoles pudiesen decir nada, Jesús se volvió a Andrés y Felipe, y les dijo: “No quiero despedir a esta gente. Están aquí como ovejas sin pastor. Me gustaría darles de comer. ¿Qué comida traemos?”. Mientras Felipe conversaba con Mateo y Judas, Andrés fue a buscar al joven Marcos para comprobar con qué provisiones contaban aún. Cuando regresó a Jesús, dijo: “Al muchacho solo le quedan cinco panes de cebada y dos pescados secos”; y Pedro añadió rápidamente: “Aún tenemos que cenar”.
\vs p152 2:8 Jesús permaneció en silencio durante un instante. Su mirada estaba distante. Los apóstoles no dijeron nada. Jesús se volvió de repente hacia Andrés y dijo: “Tráeme los panes y los peces”. Y cuando Andrés le llevó a Jesús la cesta, el Maestro dijo: “Haced que la gente se siente en la hierba por grupo, de ciento en ciento, y nombrad a un líder en cada uno de los grupos, mientras vosotros traéis hasta aquí a todos los evangelistas”.
\vs p152 2:9 Entonces tomó el pan en sus manos y, después de haber dado las gracias, lo partió y se lo dio a los apóstoles, que lo pasaron a sus compañeros; estos a su vez se lo llevaron a la multitud. De la misma manera, Jesús partió y repartió los peces. Y comieron todos y se saciaron. Y cuando acabaron de comer, Jesús dijo a los discípulos: “Recoged los pedazos que sobran para que nada se pierda”. Cuando terminaron de recogerlos, tenían doce cestas llenas. Fueron unos cinco mil hombres, mujeres y niños los que comieron de aquella extraordinaria y abundante comida.
\vs p152 2:10 \pc Y este fue el primero y único milagro de la naturaleza que obró Jesús habiéndolo planeado previamente y de forma consciente. Es verdad que sus discípulos estaban predispuestos a considerar como milagros muchas cosas que no lo eran, pero aquí se realizó un ministerio auténticamente sobrenatural. En este caso, se nos explicó que Miguel multiplicó los componentes de la comida, tal como siempre lo hace, salvo que eliminó el factor tiempo y el visible cauce vital.
\usection{3. EL INCIDENTE DE LA CORONACIÓN}
\vs p152 3:1 La alimentación de los cinco mil por medio de una energía de carácter sobrenatural fue otro de esos casos en los que la piedad humana se sumó al poder creativo que tuvo su equivalencia en lo ocurrido. Una vez que se había dado de comer a la multitud hasta saciarse y, dado que la fama de Jesús aumentó allí y entonces por este formidable prodigio, el plan de agarrar al Maestro y proclamarlo rey ya no precisaba de más instrucciones de índole personal. La idea pareció contagiar a toda la multitud. Al estar satisfechas sus necesidades físicas tan súbitamente y, al mismo tiempo, de forma tan espectacular, la muchedumbre, abrumada, reaccionó emocionalmente. Durante mucho tiempo, se había enseñado a los judíos que con la llegada del Mesías, el hijo de David, nuevamente fluiría la tierra con leche y miel, y que el pan de vida se derramaría sobre ellos como se suponía había caído el maná del cielo sobre sus ancestros en el desierto. ¿Y es que no se habían cumplido ya, por completo y ante ellos, todas sus expectativas? Cuando esta multitud hambrienta y malnutrida se sintió saciada con el prodigioso alimento, su reacción fue unánime: “He aquí a nuestro rey”. El libertador de Israel, hacedor de portentos había venido. En los ojos de esta gente sencilla, el poder de alimentar conllevaba el derecho a gobernar. No era, pues, de extrañar que la multitud, cuando acabó su copiosa comida, se puso de pie al unísono gritando: “¡Hacedlo rey!”.
\vs p152 3:2 Este portentoso grito entusiasmó a Pedro y a aquellos apóstoles que seguían esperanzados en que Jesús reivindicara alguna vez su derecho a gobernar. Pero, estas vanas esperanzas no iban a persistir mucho tiempo. Cuando el fuerte grito de la multitud aún reverberaba por las rocas cercanas, Jesús se subió sobre una gran piedra y, levantando la mano derecha para que le prestaran atención, dijo: “Hijos míos, vuestras intenciones son buenas, pero sois estrechos de miras y materialistas de mente”. Se produjo una breve pausa; allí estaba este fornido galileo de porte majestuoso ante el hechizante resplandor del crepúsculo de aquella costa oriental. Todo él parecía un rey mientras continuó hablándole a la atónita multitud: “Deseáis hacerme rey no porque vuestras almas se hayan iluminado por alguna gran verdad, sino porque vuestros estómagos están llenos de pan. ¿Cuántas veces os he dicho que mi reino no es de este mundo? Este reino de los cielos que nosotros proclamamos es una hermandad espiritual, y ningún hombre la gobierna sentado en trono material. Mi Padre del cielo es el omnisapiente y todopoderoso Soberano de esta hermandad espiritual de los hijos de Dios en la tierra. ¿Tanto he errado al revelaros al Padre de los espíritus que queréis hacer rey a su Hijo en la carne? Ahora iros todos a vuestras casas. Si habéis de tener un rey, que el Padre de las luces se entrone en los corazones de cada uno de vosotros como el Soberano espiritual de todas las cosas”.
\vs p152 3:3 \pc Estas palabras de Jesús hicieron que la multitud se fuese de allí ofuscada y descorazonada. Muchos de los que habían creído en él lo dejaron y no volvieron desde aquel día a seguirlo nunca más. Los apóstoles estaban consternados; permanecieron en silencio, de pie, alrededor de las doce canastas con la comida sobrante; solo el recadero, el joven Marcos, habló diciendo: “Y se negó a ser nuestro rey”. Jesús, antes de marcharse a las colinas para estar solo, se volvió hacia Andrés y le dijo: “Lleva a tus hermanos de regreso a la casa de Zebedeo y ora con ellos; pide en particular por Simón Pedro, tu hermano”.
\usection{4. LA VISIÓN EN LA NOCHE DE SIMÓN PEDRO}
\vs p152 4:1 Los apóstoles, sin su Maestro ---que había hecho que se fueran por sí mismos--- entraron en la barca y, en silencio, empezaron a remar hacia Betsaida, en la orilla occidental del lago. De los doce, ninguno se sentía tan atribulado ni abatido como Simón Pedro. Apenas si dijeron palabra; todos pensaban en el Maestro, solitario en las colinas. ¿Es que los había abandonado? Nunca antes les había dicho que se fueran, negándose a ir con ellos. ¿Qué significado tenía todo aquello?
\vs p152 4:2 La oscuridad descendió sobre ellos, porque soplaba un viento contrario que les impedía avanzar. Conforme transcurrían las horas de oscuridad, y remaban duramente, Pedro se sintió muy cansado y cayó, exhausto, en un profundo sueño. Andrés y Santiago lo pusieron en el asiento acolchado de la popa de la barca para que descansara. Mientras los demás apóstoles bregaban contra el viento y las olas, Pedro tuvo un sueño en el que vio a Jesús venir hacia ellos caminando sobre el mar. Cuando pareció estar a la altura de la barca, Pedro gritó: “Sálvanos, Maestro, sálvanos”. Y los que se encontraban cercanos a la popa le oyeron decir algunas de estas palabras. Al continuar esta aparición nocturna en la mente de Pedro, soñó que oyó decir a Jesús: “¡Tened ánimo! Soy yo, no temáis”. Esto fue como el bálsamo de Gilead para el alma atormentada de Pedro; calmó su espíritu atribulado, así que (en su sueño) gritó al Maestro: “Señor, si eres tú, manda que yo vaya a ti sobre las aguas”. Y cuando Pedro se puso a andar sobre las aguas, tuvo miedo de las alborotadas olas, y cuando estaba a punto de hundirse, gritó: “Señor, ¡sálvame!”. Y la mayoría de los doce le oyeron proferir este grito. Entonces Pedro soñó que Jesús vino a su rescate y, extendiéndole la mano, lo sostuvo y lo levantó, diciéndole: “¡Hombre de poca fe! ¿Por qué dudaste?”.
\vs p152 4:3 En cuanto a la última parte de su sueño, Pedro se levantó del asiento sobre el que dormía y realmente pasó por encima de la barca hasta caer al agua. Y despertó de su sueño cuando Andrés, Santiago y Juan se inclinaron hacia él para sacarlo del mar.
\vs p152 4:4 Pedro siempre consideró que lo ocurrido fue real. Creía sinceramente que Jesús había venido a ellos aquella noche. Solo llegó a convencer, y de forma parcial, a Juan Marcos, lo que explica por qué Marcos omitió de su narrativa un fragmento de esta historia. Lucas, el médico, que investigó esta cuestión detenidamente, llegó a la conclusión que aquel suceso no fue sino una visión de Pedro y, por lo tanto, no quiso incluirlo en su propia narrativa.
\usection{5. VUELTA A BETSAIDA}
\vs p152 5:1 El jueves por la mañana, antes del amanecer, anclaron su barca a corta distancia de la costa, cerca de la casa de Zebedeo y trataron de dormir hasta casi el mediodía. Andrés fue el primero en levantarse y, al ir a caminar junto al mar, encontró a Jesús en compañía del muchacho de los recados; estaba sentado sobre una piedra a orillas del agua. Pese a que muchos de la multitud y los jóvenes evangelistas buscaron a Jesús toda esa noche y gran parte del siguiente día por las colinas orientales, poco después de la medianoche, él y el joven Marcos habían comenzado a caminar alrededor del lago y cruzaron el río de vuelta a Betsaida.
\vs p152 5:2 \pc De los cinco mil que se habían alimentado milagrosamente y que, con el estómago repleto y el corazón vacío, habían querido proclamarlo rey, solo unos quinientos persistieron en ir tras él. Pero antes de que estos recibieran palabra de que había regresado a Betsaida, Jesús pidió a Andrés que reuniera a los doce apóstoles y a sus acompañantes, las mujeres incluidas, diciéndoles que “deseo hablar con ellos”. Y cuando estaban todos listos ante él, Jesús dijo:
\vs p152 5:3 \pc “¿Hasta cuándo os he de soportar? ¿Tan tardos sois de entendimiento espiritual y tan faltos de fe viva? Todos estos meses os he enseñado las verdades del reino y, sin embargo, os dejáis llevar por motivos materiales dando de lado los espirituales. ¿Es que ni siquiera habéis leído en las Escrituras cómo Moisés exhorta a los hijos descreídos de Israel, diciendo: ‘No temáis; estad firmes y ved la salvación que el Señor os dará’? El cantor dijo: ‘Pon tu confianza en el Señor’. ‘Sé paciente, espera en el Señor y ten fortaleza. Él alentará tu corazón’. ‘Echa sobre el Señor tu carga, y él te sostendrá. Esperad en él en todo tiempo y derramad delante de él vuestro corazón, porque Dios es vuestro refugio’. ‘El que habita al abrigo del Altísimo morará bajo la sombra del Omnipotente’. ‘Mejor es confiar en el Señor que en príncipes humanos’.
\vs p152 5:4 ¿Y es que no veis todavía que cuando se obran milagros y prodigios de orden material no se ganan almas para el reino espiritual? Alimentamos a la multitud pero esto no les llevó a tener hambre del pan de vida ni sed de las aguas de la rectitud espiritual. Cuando se sació su hambre, no buscaron entrar en el reino de los cielos, sino que quisieron proclamar rey al Hijo del Hombre según el modo de los reyes de este mundo, solo para poder seguir comiendo pan sin tener que afanarse por ganarlo. Y todo esto, en lo que muchos de vosotros participasteis en mayor o menor manera, no hace nada por revelar al Padre celestial ni por hacer avanzar su reino en la tierra. ¿Es que no tenemos suficientes enemigos entre los líderes religiosos del país y sin necesidad de hacer algo que probablemente pueda también enfurecer a los gobernantes civiles? Oro para que mi Padre unja vuestros ojos y veáis y abráis vuestros oídos para poder oír a fin de que pongáis toda vuestra fe en el evangelio que os he enseñado”.
\vs p152 5:5 \pc Entonces Jesús anunció que era su deseo retirarse unos días a descansar con sus apóstoles, antes de estar listos para dirigirse a Jerusalén y celebrar la Pascua, y prohibió a los discípulos y a la multitud que lo siguieran. Por consiguiente, navegaron hasta la región de Genesaret para reposar y dormir durante dos o tres días. Jesús se estaba preparando para una gran crisis de su vida en la tierra y, por lo tanto, pasó mucho tiempo en comunión con su Padre de los cielos.
\vs p152 5:6 La noticia de la alimentación de los cinco mil y del intento de hacer rey a Jesús despertó una amplia curiosidad e hizo cundir el miedo tanto entre los líderes religiosos como entre los gobernantes de toda Galilea y Judea. Aunque este gran milagro no aportó nada al fomento del evangelio del reino en las almas de los creyentes poco convencidos y de mentalidad materialista, sí sirvió para llevar a un punto crítico la propensión de la familia de apóstoles inmediata de Jesús y discípulos cercanos a la búsqueda de milagros y a su afán por tener un rey. Este espectacular suceso puso fin a la temprana etapa de enseñanza, entrenamiento y curación, preparando pues el camino para la inauguración de este último año de proclamación de ideas más elevadas y espirituales del nuevo evangelio del reino: filiación divina, libertad espiritual y salvación eterna.
\usection{6. EN GENESARET}
\vs p152 6:1 Estando de descanso en la casa de un rico creyente de la región de Genesaret, Jesús tuvo charlas informales con los doce cada tarde. Los embajadores del reino no eran sino un grupo de hombres, que además de sentirse decepcionados, estaban serios, apesadumbrados y habían sido reprendidos. Pero incluso tras lo ocurrido, y como los siguientes acontecimientos desvelaron, estos doce hombres todavía no habían conseguido librarse por completo de sus ideas, por mucho tiempo anheladas y de las que estaban imbuidos, sobre la venida del Mesías judío. Los hechos de las últimas pocas semanas habían ocurrido con demasiada celeridad para que estos atónitos pescadores pudieran llegar a comprender todo su significado. Se precisa tiempo para que puedan producirse en hombres y mujeres cambios radicales y de envergadura en sus conceptos básicos y fundamentales sobre conducta social, actitudes filosóficas y convicciones religiosas.
\vs p152 6:2 Mientras Jesús y los doce se tomaban ese descanso en Genesaret, las multitudes se dispersaron; algunas personas volvieron a sus casas y otras se dirigieron a Jerusalén para la Pascua. En menos de un mes, el número de aquellos entusiastas y declarados seguidores de Jesús, que sobrepasaba los cincuenta mil solamente en Galilea, se redujo a menos de quinientos. Jesús deseaba que sus apóstoles experimentaran la veleidad de sentirse aclamados por la gente para que no se vieran en la tentación de confiar en tales expresiones temporales de histeria religiosa, cuando los dejara y tuvieran ellos por si mismos que continuar con la labor del reino; pero su intención tuvo un efecto limitado sobre ellos.
\vs p152 6:3 \pc Durante su segunda noche en Genesaret, el Maestro contó de nuevo a los apóstoles la parábola del sembrador, añadiendo estas palabras: “Como podéis ver, hijos míos, cuando se apela a los sentimientos humanos, el resultado es algo transitorio y absolutamente decepcionante, e igualmente vacío y yermo cuando apelamos exclusivamente al intelecto humano; solo si apeláis al espíritu que vive en la mente humana podréis tener la esperanza de conseguir un éxito perdurable y lograr esas portentosas transformaciones del carácter humano que pronto rinden, copiosamente, los auténticos frutos del espíritu en la vida diaria de quienes quedan así liberados de la oscuridad de la duda y entran en la luz de la fe ---el reino de los cielos--- por el nacimiento del espíritu”.
\vs p152 6:4 \pc Jesús enseñó que cuando se recurre a las emociones se consigue atraer y concentrar la atención intelectual. Denominó a esa mente, así despierta y avivada, como la puerta de entrada al alma. Es en ella donde reside esa naturaleza espiritual del hombre que debe reconocer la verdad y responder al llamamiento espiritual del evangelio, lo que resultará en verdaderas y permanentes transformaciones del carácter.
\vs p152 6:5 De este modo, Jesús trataba de preparar a los apóstoles para la inminente conmoción que les aguardaba ---la crisis en la actitud pública hacia él, que se produciría en solo unos pocos días---. Explicó a los doce que los dignatarios religiosos de Jerusalén conspirarían con Herodes Antipas para acabar con ellos. Los doce empezaron a ser más conscientes (aunque no de modo definitivo) de que Jesús no iba a sentarse en el trono de David. Vieron con mayor claridad que la verdad espiritual no progresaba mediante portentos materiales. Empezaron a percatarse de que lo sucedido con los cinco mil y el intento de la gente por hacer rey a Jesús fueron el punto álgido en las expectativas de un pueblo anhelante de milagros y prodigios y la culminación de la aceptación de Jesús por las multitudes. Confusamente percibían y apenas anticipaban los tiempos de criba espiritual y de cruel adversidad que se aproximaban. Estos doce hombres iban lentamente tomando conciencia de la verdadera naturaleza de su misión como embajadores del reino, y se aprestaron para las difíciles y duras pruebas con las que se enfrentarían el último año del ministerio del Maestro sobre la tierra.
\vs p152 6:6 \pc Antes de salir de Genesaret, Jesús los instruyó respecto a la milagrosa alimentación de los cinco mil, diciéndoles concretamente por qué había realizado tal extraordinaria manifestación de poder creativo, y asegurándoles, al mismo tiempo, que no se había rendido, pues, ante la compasión que sentía por la multitud hambrienta, hasta que no había determinado si aquello era “en conformidad con la voluntad del Padre”.
\usection{7. EN JERUSALÉN}
\vs p152 7:1 El domingo 3 de abril, Jesús, acompañado tan solo por los doce apóstoles, se dirigió a Jerusalén desde Betsaida. Para evitar las multitudes y atraer la menor atención posible, viajaron vía Gérasa y Filadelfia. En este viaje, les prohibió que realizaran cualquier enseñanza pública, como tampoco les permitió enseñar ni predicar mientras estaban en Jerusalén. Llegaron a Betania, una aldea cercana a Jerusalén, el miércoles 6 de abril, a última hora de la tarde. Estuvieron solo esa noche en casa de Lázaro, Marta y María, pero al día siguiente se separaron. Jesús, junto con Juan, se quedó en la casa de un creyente llamado Simón, también en Betania, cerca de la casa de Lázaro. Judas Iscariote y Simón Zelotes pararon en Jerusalén, con unos amigos, mientras que el resto de los apóstoles se alojó, de dos en dos, en diferentes casas.
\vs p152 7:2 Jesús solo una vez entró en Jerusalén durante esta Pascua, y lo hizo en el día grande de la fiesta. Abner llevó a Betania a muchos creyentes de Jerusalén para encontrarse con Jesús. Durante esta estancia en Jerusalén, los doce se percataron de la creciente hostilidad reinante contra su Maestro. Todos partieron de Jerusalén con la idea de que se avecinaba una crisis.
\vs p152 7:3 El domingo, 24 de abril, Jesús y los apóstoles dejaron Jerusalén para dirigirse a Betsaida, yendo por tierra a las ciudades costeras de Jope, Cesarea y Tolemaida. Desde allí, se encaminaron vía Ramá y Corazín hacia Betsaida, llegando el viernes, 29 de abril. En cuanto estuvieron allí, Jesús envió inmediatamente a Andrés a solicitar permiso al jefe de la sinagoga para hablar al día siguiente, siendo \bibemph{sabbat,} en el servicio de la tarde. Jesús sabía bien que aquella sería la última vez que se le permitiría hablar en la sinagoga de Cafarnaúm.
