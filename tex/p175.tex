\upaper{175}{El último discurso en el templo}
\author{Comisión de seres intermedios}
\vs p175 0:1 Poco después de las dos de la tarde de ese martes, Jesús, acompañado por once apóstoles, José de Arimatea, los treinta griegos y algunos otros discípulos, llegó al templo y comenzó a impartir su última charla en los patios del edificio sagrado. Este discurso tenía por objeto ser su último llamamiento al pueblo judío y su acusación definitiva contra sus vehementes enemigos y potenciales asesinos: los escribas, los fariseos, los saduceos y los dirigentes principales de Israel. Durante toda la mañana, los distintos grupos habían tenido la oportunidad de hacerle preguntas a Jesús, pero por la tarde nadie preguntó nada.
\vs p175 0:2 Cuando el Maestro comenzó a hablar, el patio del templo estaba tranquilo y en orden. Los cambistas y los comerciantes no se habían atrevido a entrar nuevamente en el templo después de que Jesús y la exaltada multitud los habían echado de allí el día anterior. Antes de iniciar el discurso, Jesús miró con ternura a las personas allí presentes, las cuales oirían pronto sus palabras públicas de despedida, sus palabras de misericordia hacia la humanidad, junto a su última denuncia de los falsos maestros y de los intolerantes dirigentes de los judíos.
\usection{1. EL DISCURSO}
\vs p175 1:1 “He estado con vosotros durante todo este largo tiempo, recorriendo estas tierras de lado a lado, proclamando el amor del Padre por los hijos de los hombres, y muchos han visto la luz y han entrado por la fe en el reino de los cielos. Con respecto a estas enseñanzas y predicaciones, el Padre ha obrado maravillas, hasta la resurrección de la muerte. Son muchos los enfermos y afligidos que se han curado porque han creído; si bien, ni la proclamación de la verdad ni las curaciones han abierto los ojos de quienes se niegan a ver la luz, de quienes están decididos a rechazar este evangelio del reino.
\vs p175 1:2 “Acorde siempre con la voluntad de mi Padre, mis apóstoles y yo hemos hecho todo lo posible para vivir en paz con nuestros hermanos, por cumplir razonablemente con las exigencias de las leyes de Moisés y las tradiciones de Israel. Hemos buscado insistentemente la paz, pero los líderes de Israel no han querido aceptarla. Al rechazar la verdad de Dios y la luz del cielo, están tomando partido por el error y la oscuridad. Y no puede haber paz entre la luz y la oscuridad, entre la vida y la muerte, entre la verdad y el error.
\vs p175 1:3 “Muchos de vosotros habéis tenido el coraje de creer en mis enseñanzas y ya habéis entrado en el gozo y en la libertad de la conciencia de estar en filiación con Dios. Y daréis testimonio de que he ofrecido esa misma filiación de Dios a toda la nación judía, incluso a esos mismos que ahora desean mi muerte. Pero aun así, mi Padre recibirá a esos maestros enceguecidos y líderes hipócritas con tan solo volverse hacia él y aceptar su misericordia. A pesar de todo, no es demasiado tarde para que este pueblo reciba la palabra del cielo y acoja al Hijo del Hombre.
\vs p175 1:4 “Durante mucho tiempo, mi Padre ha tratado a las gentes de este pueblo con misericordia. Generación tras generación, hemos enviado a nuestros profetas para enseñarles y prevenirles y, generación tras generación, ellos mataron a estos maestros enviados del cielo. Y ahora, vuestros pertinaces sumos sacerdotes y gobernantes siguen haciendo lo mismo. Tal como Herodes propició la muerte de Juan, también vosotros ahora estáis dispuestos a acabar con el Hijo del Hombre.
\vs p175 1:5 “Mientras haya alguna posibilidad de que los judíos se vuelvan a mi Padre y busquen la salvación, el Dios de Abraham, Isaac y Jacob mantendrá sus manos de misericordia tendidas hacia vosotros; pero cuando hayáis llenado vuestra copa de impenitencia y hayáis rechazado definitivamente la misericordia de mi Padre, esta nación quedará abandonada a su propia suerte, y se precipitará rápidamente hacia su ignominioso fin. Este pueblo estaba llamado a convertirse en la luz del mundo y a manifestar la gloria espiritual de ser una raza conocedora de Dios, pero os habéis separado tanto de la observancia de vuestros privilegios divinos que vuestros líderes están a punto de cometer la mayor insensatez de todos los tiempos; están al borde de rechazar totalmente el don de Dios otorgado a todos los hombres y para todas las eras ---la revelación del amor del Padre de los cielos por todas sus criaturas de la tierra---.
\vs p175 1:6 “Y una vez que rechacéis esta revelación de Dios al hombre, el reino de los cielos se concederá a otros pueblos, a los que lo reciban con regocijo y alegría. En nombre del Padre que me envió, os prevengo solemnemente que estáis a punto de perder vuestra posición en el mundo como abanderados de la verdad eterna y custodios de la ley divina. En este momento, yo os ofrezco una última oportunidad de que os presentéis ante mí y os arrepintáis, y manifestéis vuestra intención de buscar a Dios con todo vuestro corazón y de entrar, como niños pequeños y mediante una fe sincera, en la seguridad y la salvación del reino de los cielos.
\vs p175 1:7 “Mi Padre se ocupa de vuestra salvación desde hace mucho tiempo, y yo vine aquí para vivir entre vosotros y mostraros personalmente el camino. Muchos, tanto judíos como samaritanos, e incluso gentiles, han creído en el evangelio del reino, pero quienes deberían ser los primeros en dar un paso adelante y aceptar la luz del cielo se han negado rotundamente a creer en la revelación de la verdad de Dios ---Dios revelado en el hombre y el hombre elevado hacia Dios---.
\vs p175 1:8 “Esta tarde mis apóstoles están aquí, ante vosotros, en silencio, pero pronto oiréis sus voces resonando, llamándoos a la salvación e instándoos a uniros al reino celestial como hijos del Dios vivo. Y, ahora, pongo por testigos a mis discípulos y creyentes en el evangelio del reino, al igual que a los mensajeros invisibles que están a su lado, de que, una vez más, he ofrecido a Israel y a sus dirigentes, liberación y salvación. Pero todos veis cómo se menosprecia la misericordia del Padre y se rechaza a los mensajeros de la verdad. No obstante, os advierto que estos escribas y fariseos se sientan todavía en la cátedra de Moisés y, por lo tanto, hasta que los Altísimos que gobiernan en los reinos de los hombres no acaben por derrocar a esta nación y destruyan el sitio de estos dirigentes, yo os mando que cooperéis con estos ancianos de Israel. No se os pide que os unáis a ellos en sus planes para matar al Hijo del Hombre, pero en todo lo relacionado con la paz de Israel, os someteréis a ellos. En todos estos asuntos, haced todo los que os digan que hagáis y observad las leyes importantes, pero no imitéis sus perversas obras. Recordad que este es su pecado: dicen lo que es bueno, pero no lo hacen. Sabéis bien cómo estos líderes atan cargas pesadas, difíciles de llevar, y las ponen sobre vuestros hombros, pero ellos ni con un dedo quieren moverlas para ayudaros a llevarlas. Os han oprimido con ceremonias y esclavizado con tradiciones.
\vs p175 1:9 “Asimismo, estos egocéntricos dirigentes se complacen en hacer sus buenas obras para ser vistos por los hombres. Ensanchan sus filacterías y extienden los flecos de sus mantos oficiales. Les gusta ocupar los primeros lugares en las cenas y piden las primeras sillas en las sinagogas. Ambicionan las salutaciones y alabanzas en las plazas del mercado y desean que los hombres los llamen Rabí. E incluso, buscando este honor de la gente, se apropian en secreto de las casas de las viudas y sacan provecho de los servicios del templo sagrado. Fingidamente, estos hipócritas oran largo rato en público y dan limosnas para atraer la atención de sus semejantes.
\vs p175 1:10 “Aunque debéis honrar a vuestros dirigentes y reverenciar a vuestros maestros, no debéis llamar Padre en el sentido espiritual a ningún hombre, porque uno es vuestro Padre, que es Dios. No tratéis tampoco de enseñorearos sobre vuestros hermanos en el reino. Recordad que os he enseñado que el que quiera ser el mayor de vosotros sea vuestro servidor. Los que pretendan enaltecerse ante Dios serán humillados; pero quien verdaderamente se humille será enaltecido. En vuestras vidas diarias, no busquéis vuestra propia glorificación, sino la gloria de Dios. Con inteligencia, subordinad vuestra propia voluntad a la voluntad del Padre de los cielos.
\vs p175 1:11 “No confundáis mis palabras. No albergo animosidad contra los sumos sacerdotes ni contra los dirigentes judíos que en este momento quieren incluso acabar conmigo; no tengo mala voluntad hacia estos escribas y fariseos que rechazan mis enseñanzas. Sé que muchos de vosotros creéis en secreto, y sé que profesaréis públicamente vuestra fidelidad al reino cuando llegue mi hora. Pero, ¿cómo podrán justificar sus actos vuestros rabinos, que claman hablar con Dios y seguidamente se atreven a rechazar y matar a Aquel que viene para revelar el Padre a los mundos?
\vs p175 1:12 “¡Ay de vosotros, escribas y fariseos, hipócritas!, porque cerráis las puertas del reino de los cielos a los hombres sinceros, indoctos en vuestras doctrinas. Os negáis a entrar en el reino y, al mismo tiempo, hacéis todo lo posible para evitar que los demás lo hagan. Estáis de espaldas a las puertas de la salvación y pleiteáis con quienes quieren entrar.
\vs p175 1:13 “¡Ay de vosotros, escribas y fariseos, que tan hipócritas sois!, porque recorréis mar y tierra para hacer un prosélito, y cuando lo habéis conseguido, no os contentáis hasta hacerlos dos veces peor de lo que era como hijo de los paganos.
\vs p175 1:14 “¡Ay de vosotros, sumos sacerdotes y dirigentes que os apoderáis de los bienes de los pobres y exigís gravosos tributos a los que quieren servir a Dios como creen que mandó Moisés! Vosotros que os negáis a mostrar misericordia, ¿podéis esperar recibirla en los mundos venideros?
\vs p175 1:15 “¡Ay de vosotros, falsos maestros, guías ciegos! ¿Qué se puede esperar de una nación cuando el ciego guía al ciego? Los dos tropezarán y caerán en el foso de la destrucción.
\vs p175 1:16 “¡Ay de vosotros que disimuláis cuando hacéis juramento! Sois embaucadores porque decís que alguien puede jurar por el templo y romper su juramento; pero si alguien jura por el oro del templo, queda obligado. ¡Insensatos y ciegos! Ni siquiera en vuestra deshonestidad sois consecuentes, porque, ¿cuál es mayor, el oro o el templo que supuestamente santifica al oro? También decís que si alguien jura por el altar, no es nada; pero que si jura por la ofrenda que está sobre él, es deudor. De nuevo estáis ciegos a la verdad, porque ¿cuál es mayor, la ofrenda o el altar que santifica a la ofrenda? ¿Cómo podéis justificar esta hipocresía y falta de honradez delante del Dios de los cielos?
\vs p175 1:17 “¡Ay de vosotros, escribas y fariseos y todos los demás hipócritas que os aseguráis de diezmar la menta, el anís y el comino, y descuidáis lo más importante de la ley: la fe, la misericordia y la justicia! Razonablemente, esto era necesario hacerlo sin haber descuidado lo otro. Sois en verdad guías ciegos y maestros necios, que coláis el mosquito y tragáis el camello.
\vs p175 1:18 “¡Ay de vosotros, escribas, fariseos e hipócritas!, porque sois escrupulosos para limpiar lo de fuera del vaso y del plato, pero dentro queda la suciedad de la extorsión, de los excesos y del engaño. Espiritualmente estáis ciegos. ¿Es que no reconocéis que es mucho mejor limpiar primero lo de dentro del vaso, y entonces dejar que lo que se derrame limpie por sí mismo lo de fuera? ¡Sois perversos y depravados!, porque hacéis que vuestra religión se ajuste externamente a la letra de vuestra interpretación de la ley de Moisés, mientras que vuestras almas están sumidas en la iniquidad y llenas de asesinatos.
\vs p175 1:19 “¡Ay de todos vosotros que rechazáis la verdad y menospreciáis la misericordia! Muchos sois semejantes a sepulcros blanqueados, que por fuera se muestran hermosos, pero por dentro están llenos de huesos de muertos y de toda inmundicia. Y, así igual, vosotros que rechazáis deliberadamente el plan de Dios, os mostráis por fuera santos y justos a los hombres, pero por dentro vuestros corazones están llenos de hipocresía e iniquidad.
\vs p175 1:20 “¡Ay de vosotros, falsos guías de una nación!, porque habéis edificado cerca un monumento a los profetas mártires de la antigüedad, mientras conspiráis para acabar con Aquel de quien ellos hablaban. Adornáis las tumbas de los justos y hacéis alarde de que si hubierais vivido en los días de vuestros padres, no habríais matado a los profetas; y, entonces, frente a tal pensamiento hipócrita, os disponéis a aniquilar a Aquel de quien hablaban los profetas, al Hijo del Hombre. Con esto dais testimonio contra vosotros mismos de que sois los malvados hijos de aquellos que mataron a los profetas. ¡Continuad, pues, y colmad la copa de vuestra condenación!
\vs p175 1:21 “¡Ay de vosotros, hijos del mal! Juan de cierto os llamó generación de víboras, y yo os pregunto, ¿cómo escaparéis del juicio que él pronunció sobre vosotros?
\vs p175 1:22 “Pero aun así, os ofrezco ahora misericordia y perdón en nombre de mi Padre; a pesar de todo os extiendo la mano amorosa de la fraternidad eterna. Mi Padre os ha enviado a los sabios y a los profetas; a unos perseguisteis y a otros matasteis. Luego llegó Juan y proclamó la venida del Hijo del Hombre, y acabasteis con su vida después de que muchos hubieran creído en sus enseñanzas. Y ahora os disponéis a derramar más sangre inocente. ¿Es que no comprendéis que llegará el terrible día del juicio final, cuando el Juez de toda la tierra pedirá a este pueblo explicación de la forma en la que han rechazado, perseguido y exterminado a estos mensajeros del cielo? ¿Es que no entendéis que debéis rendir cuentas de toda esta sangre justa, desde el primer profeta al que distéis muerte hasta los tiempos de Zacarías, a quien matasteis entre el templo y el altar? Y si continuáis por esos malvados caminos, quizás se os exija que respondáis por ello en esta misma generación.
\vs p175 1:23 “¡Oh Jerusalén e hijos de Abraham, que habéis apedreado a los profetas y asesinado a los maestros que os enviaron, pero aún así yo quisiera juntar ahora a vuestros hijos como la gallina junta a sus polluelos bajo las alas, pero no queréis.
\vs p175 1:24 “Y ahora me despido de vosotros. Habéis oído mi mensaje y habéis tomado vuestra decisión. Los que creyeron mi evangelio están incluso ahora a salvo en el reino de Dios. A vosotros, que habéis optado por rechazar la ofrenda de Dios, os digo que ya no me volveréis a ver más enseñando en el templo. Mi obra para vosotros ha acabado. ¡Mirad que ahora salgo con mis hijos, y vuestra casa se os deja desierta!”.
\vs p175 1:25 Y, seguidamente, el Maestro hizo señas a sus seguidores para abandonar el templo.
\usection{2. ESTATUS DE LOS JUDÍOS A NIVEL PERSONAL}
\vs p175 2:1 El hecho de que los líderes espirituales y los maestros religiosos de la nación judía rechazaran en algún momento las enseñanzas de Jesús y conspiraran para acarrearle una cruel muerte, no afecta de ninguna manera el estatus individual del judío en su posición ante Dios. Y esto no debe ser causa para que aquellos que profesan ser seguidores del Cristo tengan prejuicios contra sus semejantes judíos. Los judíos, como nación, como grupo sociopolítico, pagaron en su totalidad el terrible precio de rechazar al Príncipe de la Paz. Hace mucho tiempo que dejaron de ser los adalides espirituales de la verdad divina para las razas de la humanidad, pero esto no justifica que los descendientes de estos antiguos judíos deban padecer las persecuciones a las que se han visto expuestos por parte de profesos seguidores de Jesús de Nazaret, de seres intolerantes, indignos y fanáticos. Jesús mismo fue judío por nacimiento.
\vs p175 2:2 Muchas veces esta persecución y este odio irracional, tan poco cristiano, contra los judíos modernos acabaron en el sufrimiento y la muerte de judíos inocentes e inofensivos, cuyos mismos ancestros, en los tiempos de Jesús, aceptaron fervorosamente su evangelio y al final murieron sin vacilar por esa verdad en la que creían tan incondicionalmente. ¡Cómo se estremecen de horror los seres celestiales al ver cómo los profesos seguidores de Jesús se complacen en perseguir, hostigar e incluso asesinar a los posteriores descendientes de Pedro, Felipe y Mateo y de otros judíos palestinos, de esos primeros mártires que tan gloriosamente dieron sus vidas por el evangelio del reino celestial!
\vs p175 2:3 ¡Qué cruel e irrazonable es hacer que personas inocentes sufran por los pecados de sus progenitores por actos errados que ignoran por completo y sobre los que no tienen ninguna responsabilidad! ¡Y perpetrar estos perversos actos en nombre de quien enseñó a sus discípulos a que amaran incluso a sus enemigos! Al redactar este relato de la vida de Jesús, fue necesario describir cómo algunos de sus compatriotas judíos lo rechazaron y se confabularon contra él para propiciar su ignominiosa muerte; pero queremos avisar a los lectores de esta narrativa que su argumentación histórica no justifica en absoluto el odio injusto ni legitima la desleal actitud, que tantos cristianos declarados han mantenido contra los judíos durante tantos siglos. Los creyentes del reino, aquellos que siguen las enseñanzas de Jesús, no deben seguir denostando a ningún judío como si este fuera culpable del rechazo y la crucifixión de Jesús. El Padre y su Hijo Creador nunca han dejado de amar a los judíos. Dios no hace acepción de personas, y la salvación es tanto para los judíos como para los gentiles.
\usection{3. LA FATÍDICA REUNIÓN DEL SANEDRÍN}
\vs p175 3:1 La fatídica reunión del sanedrín tuvo lugar oficialmente a las ocho de la noche de aquel martes. Anteriormente, en numerosas ocasiones, este supremo tribunal de la nación judía había decretado oficiosamente la muerte de Jesús. Fueron muchas las veces en las que dicho augusto órgano gobernante se había propuesto poner fin a su labor, pero, nunca antes, se había decidido a arrestar y darle muerte a cualquier precio. Fue justo antes de la medianoche de ese mismo martes, 4 de abril del año 30 d. C., cuando el sanedrín, tal como estaba entonces constituido, votó, de manera oficial y \bibemph{por unanimidad,} condenar a muerte a Jesús y a Lázaro. Aquella era su forma de responder al llamamiento hecho por Jesús en el templo solo unas horas antes a los dirigentes de los judíos, y demostraba su gran resentimiento hacia Jesús por aquella su última y enérgica acusación contra estos mismos despiadados sumos sacerdotes, saduceos y fariseos. La imposición de la pena de muerte sobre el Hijo de Dios (incluso antes de su juicio) fue la reacción del sanedrín a su último ofrecimiento de misericordia celestial jamás antes extendido a la nación judía, como tal.
\vs p175 3:2 Desde aquel momento, quedó en manos de los judíos poner término al breve y corto período de vida nacional que les restaba, en total conformidad con su estatus meramente humano entre las naciones de Urantia. Israel había repudiado al Hijo de aquel Dios, que había hecho un pacto con Abraham; y el plan destinado a que los hijos de Abraham fueran los portadores de la luz de la verdad en el mundo había fracasado por completo. El pacto divino estaba derogado, y se acercaba a grandes pasos el fin de la nación hebrea.
\vs p175 3:3 Temprano, la mañana siguiente, los oficiales del sanedrín recibieron órdenes de arrestar a Jesús, pero se les notificó que no lo prendieran en público. Se les dijo que planearan hacerlo en secreto, a ser posible de repente y por la noche. Suponiendo que Jesús no volvería aquel día (miércoles) al templo para enseñar, mandaron a estos oficiales del sanedrín que “lo llevaran ante el alto tribunal judío en algún momento antes de la medianoche del jueves”.
\usection{4. LA SITUACIÓN EN JERUSALÉN}
\vs p175 4:1 Con este último discurso de Jesús en el templo, los apóstoles de nuevo se sumieron en la confusión y en la consternación. Antes de que el Maestro comenzara su terrible denuncia de los dirigentes judíos, Judas había regresado al templo; todos los doce oyeron, pues, la última mitad del discurso. Es de lamentar que Judas Iscariote no hubiera podido oír la primera parte de este discurso de despedida ni el definitivo ofrecimiento de misericordia a estos dirigentes por parte de Jesús. Judas no pudo acudir antes porque se encontraba aún en conversaciones con cierto grupo de parientes y amigos saduceos con los que había almorzado, y con quienes había estado consultando sobre la mejor manera de desvincularse de Jesús y de sus compañeros apóstoles. Fue al escuchar las palabras acusatorias finales del Maestro contra los líderes y dirigentes judíos, cuando Judas acabó de tomar la decisión de abandonar el movimiento evangélico y desentenderse de todo. No obstante, dejó el templo junto con los doce, fue con ellos al Monte de los Olivos, donde escuchó, con sus compañeros el profético discurso sobre la destrucción de Jerusalén y el fin de la nación judía, y permaneció con ellos ese martes por la noche en el nuevo campamento, cercano a Getsemaní.
\vs p175 4:2 \pc La multitud, que oyó a Jesús discurrir desde su misericordioso llamamiento a los líderes judíos hasta su repentino y áspero reproche, llegando a rozar la implacable denuncia, quedó atónita y conmocionada. Esa noche, mientras que el sanedrín sentenciaba a muerte a Jesús, y el Maestro se sentaba con sus apóstoles y algunos de sus discípulos en el Monte de los Olivos, anunciando el fin de la nación judía, toda Jerusalén estaba absorta comentando con seriedad, pero discreción, una única pregunta: “¿Qué harán con Jesús?”.
\vs p175 4:3 \pc Más de treinta judíos prominentes, creyentes secretos del reino, se reunieron en la casa de Nicodemo, y debatieron sobre qué derrotero seguirían en caso de romper abiertamente con el sanedrín. Todos los presentes acordaron que reconocerían públicamente su lealtad hacia el Maestro en el mismo momento en el que supieran de su arresto. Y aquello fue exactamente lo que hicieron.
\vs p175 4:4 Los saduceos, que entonces ejercían una gran influencia y dominio sobre el sanedrín estaban deseosos de deshacerse de Jesús por las siguientes razones:
\vs p175 4:5 \li{1.}Temían que el creciente favor popular que la multitud le dispensaba pusiera en peligro la existencia de la nación judía por posibles dificultades con las autoridades romanas.
\vs p175 4:6 \li{2.}Su empeño en reformar el templo repercutía directamente sobre sus ingresos; la purificación del templo afectaba su economía.
\vs p175 4:7 \li{3.}Se sentían responsables del mantenimiento del orden social, y temían las consecuencias de una mayor propagación de la extraña y nueva doctrina de Jesús sobre la hermandad de los hombres.
\vs p175 4:8 \pc Los fariseos tenían otros motivos para querer ver a Jesús sentenciado a muerte. Le temían porque:
\vs p175 4:9 \li{1.}Se oponía claramente al tradicional dominio que ellos ejercían sobre la gente. Los fariseos eran ultraconservadores, y se sentían tremendamente indignados por los ataques, supuestamente radicales de Jesús, al prestigio del que estaban investidos como maestros religiosos.
\vs p175 4:10 \li{2.}Mantenían que Jesús había infringido la ley; que había mostrado un absoluto desprecio por el \bibemph{sabbat} y por otros numerosos requerimientos legales y ceremoniales.
\vs p175 4:11 \li{3.}Lo acusaban de blasfemia porque aludía a Dios como su Padre.
\vs p175 4:12 \li{4.}Y, ahora, estaban enormemente furiosos contra él a causa del discurso de Jesús de fuerte condena hacia ellos, pronunciado aquel mismo día en el templo, en la última parte de sus palabras de despedida.
\vs p175 4:13 \pc El sanedrín, después de haber dictado oficialmente la condena a muerte de Jesús y haber emitido órdenes para su arresto, hizo un receso aquel martes cerca de la medianoche, tras fijar una reunión para las diez de la mañana del día siguiente en la casa del sumo sacerdote Caifás; se proponían formular allí los cargos contra Jesús para llevarlo a juicio.
\vs p175 4:14 Un pequeño grupo de saduceos había propuesto de hecho el asesinato como medio para deshacerse de Jesús, pero los fariseos se negaron tajantemente a aprobar tal procedimiento.
\vs p175 4:15 \pc Y esa era la situación en Jerusalén y entre los hombres aquel azaroso día, mientras una gran confluencia de seres celestiales contemplaba aquella aciaga escena en la tierra. Estaban ansiosos por hacer algo para auxiliar a su amado Soberano, pero se encontraban impotentes para actuar por las firmes restricciones impuestas por sus superiores.
