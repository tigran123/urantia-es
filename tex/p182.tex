\upaper{182}{En Getsemaní}
\author{Comisión de seres intermedios}
\vs p182 0:1 Sobre las diez de la noche de ese jueves, Jesús llevó a los once apóstoles de vuelta desde la casa de Elías y María Marcos al campamento de Getsemaní. Desde aquel día en el que estuvo con él en las colinas, Juan Marcos se había ocupado de velar por Jesús. Juan, teniendo necesidad de dormir, estuvo durmiendo durante varias horas, mientras que el Maestro había estado con sus apóstoles en el aposento alto, pero al oírles bajar las escaleras, se levantó y, echándose enseguida encima un manto de lino, los siguió a través de la ciudad, cruzando el torrente Cedrón, hasta llegar a su campamento privado, colindante con el parque de Getsemaní. Juan Marcos permaneció tan cerca del Maestro a lo largo de esa noche y del día siguiente que fue testigo de todo y pudo casualmente oír mucho de lo que el Maestro dijo desde aquel momento hasta la hora de su crucifixión.
\vs p182 0:2 A medida que Jesús y los once regresaban al campamento, los apóstoles empezaron a preguntarse por las razones de la prolongada ausencia de Judas, y hablaron entre sí sobre la precognición del Maestro de que uno de ellos lo traicionaría; por primera vez, sospecharon que no todo estaba bien con Judas Iscariote. Pero no hicieron ningún comentario explícito sobre él hasta llegar al campamento y observar que no estaba allí para recibirlos. Cuando todos ellos asediaron a Andrés a preguntas para saber qué le había pasado a Judas, su jefe tan solo indicó: “No sé dónde está Judas, pero me temo que nos ha desertado”.
\usection{1. ÚLTIMA ORACIÓN EN GRUPO}
\vs p182 1:1 Unos momentos después de llegar al campamento, Jesús les dijo: “Amigos y hermanos míos, mi tiempo con vosotros se acaba, y deseo que vayamos a un lugar apartado para orar a nuestro Padre de los cielos que nos dé fortaleza en la que apoyarnos en esta hora y, en adelante, en toda la labor que nos queda por hacer en su nombre”.
\vs p182 1:2 Cuando Jesús había dicho estas cosas, los condujo durante un corto trayecto, ascendiendo el Monte de los Olivos, a un lugar desde el que se tenía una amplia visión de Jerusalén. Allí les pidió que se arrodillaran sobre una gran roca plana, formando un círculo a su alrededor tal como habían hecho el día de su ordenación; y, entonces, de pie, en medio de ellos, resplandeciente por la luz de la luna, levantó los ojos al cielo y oró:
\vs p182 1:3 “Padre, mi hora ha llegado; glorifica ahora a tu Hijo para que también tu hijo te glorifique a ti. Sé que me has dado plena potestad sobre todas las criaturas vivas de mis dominios, y daré la vida eterna a todos los que se conviertan en los hijos de Dios por la fe. Y esta es la vida eterna: que mis criaturas te conozcan a ti, el único Dios verdadero y Padre de todos, y crean en Aquel a quien tú enviaste a este mundo. Padre, te he enaltecido en la tierra y he acabado la obra que me diste para hacer. Casi he terminado de darme de gracia a los hijos de nuestra propia creación; solo me resta dar mi vida en la carne. Y, ahora, ¡oh! Padre mío, glorifícame con la gloria que tenía contigo antes de que este mundo existiera y recíbeme una vez más a tu diestra.
\vs p182 1:4 “Te he manifestado a los hombres que elegiste del mundo y me diste. Ellos son tuyos ---como toda vida está en tus manos--- y tú me los diste a mí, y yo he vivido entre ellos, enseñándoles el camino de la vida, y ellos han creído. Ahora han conocido que todas las cosas que me has dado proceden de ti, y que la vida que vivo en la carne es para dar a conocer a mi Padre a los mundos. La verdad que tú me has dado se la he revelado a ellos. Estos, mis amigos y embajadores, han deseado sinceramente recibir tu palabra. Les he dicho que salí de ti, que tú me enviaste a este mundo, y que estoy a punto de regresar a ti. Padre, te ruego por estos hombres escogidos. Y lo hago no como lo haría por el mundo, sino como por aquellos a quienes he elegido del mundo para representarme ante este tras haber vuelto a tu labor, tal como te he representado en este mundo durante mi estancia en la carne. Estos hombres son míos; tú me los diste; pero todas las cosas que son mías son tuyas por siempre y, todo lo que era tuyo, tú ahora has hecho que sea mío. Tú has sido enaltecido en mí, y ahora oro para poder ser yo honrado en estos hombres. Ya no puedo estar en este mundo; estoy a punto de volver a la obra que me has encargado que haga. Debo dejar a estos hombres atrás para que nos representen a nosotros y a nuestro reino entre los hombres. Padre, mantén a estos hombres fieles mientras me preparo para dejar mi vida en la carne. Ayuda a estos, mis amigos, para que sean uno en espíritu, tal como nosotros somos uno. Mientras pude estar con ellos, pude velar por ellos y guiarlos, pero ahora estoy a punto de irme. Estate tú cerca de ellos, Padre, hasta que podamos enviar al nuevo maestro para que los consuele y fortalezca.
\vs p182 1:5 “Tú me diste doce hombres, y a todos he conservado menos a uno, el hijo de la venganza, que se niega a continuar en la fraternidad con nosotros. Estos hombres son débiles y frágiles, pero sé que podemos confiar en ellos; los he puesto a prueba; me aman, al igual que te veneran a ti. Aunque deberán sufrir mucho por mi causa, deseo también que puedan rebosar de gozo por la certeza de su filiación en el reino celestial. Les he dado tu palabra y les he enseñado la verdad. El mundo puede odiarlos, tal como me ha odiado a mí, pero no te pido que los saques del mundo, sino solamente que los protejas del mal que existe en el mundo. Santifícalos en la verdad; tu palabra es la verdad. Y tal como tú me enviaste a este mundo, así estoy yo a punto de enviar a estos hombres al mundo. Por su bien, he vivido entre hombres y he consagrado mi vida a tu servicio para poder inspirarlos a que se purifiquen por la verdad que les he enseñado y el amor que les he revelado. Bien sé, Padre mío, que no hay necesidad de que te pida que veles por estos hermanos después de mi partida; sé que tú los amas al igual que yo, pero hago esto para que comprendan mejor que el Padre ama a los hombres mortales tal como los ama el Hijo.
\vs p182 1:6 “Y, ahora, Padre mío, quiero pedir no solo por estos once hombres sino también por todos aquellos que ahora creen o que en adelante puedan creer en el evangelio del reino mediante la palabra de su ministerio futuro. Quiero que todos ellos sean uno, al igual que tú y yo somos uno. Tú estás en mí, y yo en ti, y deseo que estos creyentes estén igualmente en nosotros; que nuestros dos espíritus habiten en ellos. Si mis hijos son uno como nosotros somos uno, y si se aman unos a otros como yo los he amado, todos los hombres creerán pues que he venido de ti y estarán dispuestos a recibir la revelación que he realizado de la verdad y de la gloria. He revelado a estos creyentes la gloria que me diste. Tal como tú has vivido conmigo en espíritu, así he vivido yo con ellos en la carne. Tal como tú has sido uno conmigo, he sido yo uno con ellos, y así el nuevo maestro será siempre uno con ellos y en ellos. Y he hecho todo esto para que mis hermanos en la carne sepan que el Padre los ama tal como el Hijo los ama, y que tú los amas a ellos al igual que me amas a mí. Padre, obra conmigo para salvar a estos creyentes y que alguna vez vengan a estar a mi lado en la gloria y luego continúen para unirse a ti en el acogimiento del Paraíso. A quienes sirven conmigo como hombres humildes, los quiero tener conmigo en altos mundos de la gloria para que puedan ver todo lo que tú has entregado a mis manos como cosecha eterna de la siembra del tiempo en semejanza de la carne mortal. Anhelo mostrar a mis hermanos terrenales la gloria que tenía contigo antes de la fundación de este mundo. Padre justo, este mundo sabe muy poco de ti, pero yo te conozco, y he hecho que estos creyentes te conozcan, y ellos darán a conocer tu nombre a otras generaciones. Y, ahora, les prometo que tú estarás con ellos en el mundo tal como has estado conmigo ---que así sea---”.
\vs p182 1:7 Los once permanecieron arrodillados en aquel circulo formado alrededor de Jesús durante varios minutos antes de levantarse y encaminarse silenciosos de vuelta al campamento cercano.
\vs p182 1:8 \pc Jesús oró por la \bibemph{unidad} entre sus seguidores, pero no deseaba uniformidad. El pecado causa una inercia, que es nociva, pero la rectitud fomenta el espíritu creativo mediante el que una persona, de manera individual, puede experimentar las realidades vivas de la verdad eterna y la comunión progresiva con los espíritus divinos del Padre y del Hijo. En la fraternidad espiritual del hijo\hyp{}creyente con el Padre divino nunca puede haber doctrinas absolutas ni ningún grupo que imponga la superioridad sectaria de sus creencias.
\vs p182 1:9 El Maestro, durante el transcurso de esta su última oración con sus apóstoles, se refirió al hecho de que había manifestado el \bibemph{nombre} del Padre al mundo. Y eso es verdaderamente lo que hizo, al revelar a Dios mediante su vida de perfección en la carne. El Padre de los cielos había querido revelarse a Moisés, pero no pudo desvelar más sobre él que la afirmación: “YO SOY”. Y cuando se le suscitó a hacer una mayor revelación de sí mismo, solo añadió: “YO SOY el que SOY”. Pero, una vez consumada la vida terrenal de Jesús, este nombre del Padre se había revelado de tal manera que el Maestro, que era el Padre encarnado, podía verdaderamente decir:
\vs p182 1:10 Yo soy el pan de la vida.
\vs p182 1:11 Yo soy el agua viva.
\vs p182 1:12 Yo soy la luz del mundo.
\vs p182 1:13 Yo soy el deseado de todos los tiempos.
\vs p182 1:14 Yo soy la puerta abierta a la salvación eterna.
\vs p182 1:15 Yo soy la realidad de la vida sin fin.
\vs p182 1:16 Yo soy el buen pastor.
\vs p182 1:17 Yo soy la senda a la perfección infinita.
\vs p182 1:18 Yo soy la resurrección y la vida.
\vs p182 1:19 Yo soy el secreto de la supervivencia eterna.
\vs p182 1:20 Yo soy el camino, la verdad y la vida.
\vs p182 1:21 Yo soy el Padre infinito de mis hijos finitos.
\vs p182 1:22 Yo soy la verdadera vid, vosotros sois los pámpanos.
\vs p182 1:23 Yo soy la esperanza de todos los que conocen la verdad viva.
\vs p182 1:24 Yo soy el puente vivo entre un mundo y otro.
\vs p182 1:25 Yo soy el vínculo vivo entre el tiempo y la eternidad.
\vs p182 1:26 \pc 182:1.26 (1965.20) De esa manera, Jesús engrandeció la revelación viva del nombre de Dios para todas las generaciones. Tal como el amor divino revela la naturaleza de Dios, la verdad eterna desvela su nombre en dimensiones siempre en aumento.
\usection{2. ÚLTIMA HORA ANTES DE LA TRAICIÓN}
\vs p182 2:1 Cuando regresaron al campamento, los apóstoles se sorprendieron enormemente al darse cuenta de que Judas no estaba allí. Mientras los once se ensalzaron en una acalorada discusión sobre su compañero apóstol traidor, David Zebedeo y Juan Marcos llevaron a Jesús a un lado y le comunicaron que habían estado observando a Judas durante varios días, y que sabían que tenía la intención de traicionarlo, entregándolo en manos de sus enemigos. Jesús los escuchó pero tan solo dijo: “Amigos míos, nada puede suceder al Hijo del Hombre a no ser que el Padre de los cielos así lo quiera. Que no se atribule vuestro corazón; todas las cosas ayudarán a la gloria de Dios y a la salvación de los hombres”.
\vs p182 2:2 La actitud animosa de Jesús había empezado a decaer. A medida que pasaba la hora estaba cada vez más serio, e incluso triste. Los apóstoles, en su gran agitación, eran reticentes a volver a sus tiendas, incluso cuando el Maestro mismo les pedía que lo hicieran. Al volver de su charla con David y Juan, Jesús dirigió sus últimas palabras a los once, diciéndoles: “Amigos míos, id a descansar. Preparaos para la tarea de mañana. Recordad: todos debemos someternos a la voluntad del Padre de los cielos. Mi paz os dejo con vosotros”. Y habiendo dicho esto, les indicó que se fueran a sus tiendas, si bien, conforme se iban, Jesús llamó a Pedro, a Santiago y a Juan, y les dijo: “Deseo que os quedéis conmigo un rato”.
\vs p182 2:3 Los apóstoles se quedaron dormidos solamente porque estaban literalmente exhaustos. No habían dormido mucho desde su llegada a Jerusalén. Antes de irse a sus alojamientos cada uno por separado, Simón Zelotes se los llevó a su propia tienda, donde guardaba espadas y otras armas, y proporcionó dicho armamento a cada uno de ellos. Excepto Natanael, todos ellos recogieron estas armas y se las ciñeron a la cintura. Natanael, al negarse a recogerlas, dijo: “Hermanos míos, el Maestro nos ha dicho muchas veces que su reino no es de este mundo, y que sus discípulos no deben pelear con la espada para instaurarlo. Así lo pienso; y no creo que el Maestro necesite que usemos la espada para defenderlo. Todos hemos visto su gran poder y sabemos que podría defenderse de sus enemigos si así lo deseara. Si no quiere confrontar a sus enemigos, debe ser porque tal proceder es su forma de cumplir la voluntad de su Padre. Yo oraré, pero no empuñaré la espada”. Cuando Andrés oyó a Natanael, devolvió su espada a Simón Zelotes. Y, sí pues, cuando se separaron para descansar por la noche, eran nueve los que estaban armados.
\vs p182 2:4 Por el momento, su indignación ante la traición de Judas eclipsó cualquier otra cosa de las mentes de los apóstoles. El comentario del Maestro referente a Judas, realizado en el transcurso de su última oración, les había abierto los ojos al hecho de que los había abandonado.
\vs p182 2:5 \pc Una vez que los ocho apóstoles volvieron finalmente a sus tiendas, y mientras que Pedro, Santiago y Juan esperaban recibir órdenes del Maestro, Jesús llamó a David Zebedeo y le dijo: “Envíame a tu mensajero más rápido y digno de confianza”. Cuando David llevó hasta el Maestro a un tal Santiago, anteriormente corredor de mensajería nocturna que prestaba servicio entre Jerusalén y Betsaida, Jesús, dirigiéndose a él, le dijo: “Ve a toda prisa a Filadelfia y dile a Abner lo siguiente: ‘El Maestro te envía saludos de paz y dice que ha llegado la hora en que será entregado en manos de sus enemigos, que será llevado a la muerte, pero que resucitará de entre los muertos y pronto aparecerá ante ti, antes de ir al Padre, y que entonces te asesorará hasta el momento en que el nuevo maestro venga a vivir en tu corazón’”. Y cuando Santiago había repetido su mensaje hasta que el Maestro quedó satisfecho, Jesús lo envió por su camino, diciendo: “No temas lo que nadie pueda hacerte, Santiago, porque esta noche un mensajero invisible correrá a tu lado”.
\vs p182 2:6 Después Jesús se volvió al jefe del grupo de los griegos visitantes, que estaban acampados con ellos y le dijo: “Hermano mío, que no te turbe lo que está a punto de tener lugar, puesto que ya te he prevenido. El Hijo del Hombre será condenado a muerte por instigación de sus enemigos, los sumos sacerdotes y los dirigentes de los judíos, pero resucitaré para estar con vosotros durante un breve tiempo antes de ir al Padre. Y cuando hayas visto que esto acontece, glorifica a Dios y fortalece a tus hermanos”.
\vs p182 2:7 \pc En circunstancias corrientes, los apóstoles le hubieran dado personalmente las buenas noches al Maestro, pero aquella noche estaban tan preocupados por el repentino descubrimiento de la deserción de Judas y tan abrumados por el carácter especial de la oración de despedida del Maestro, que escucharon su saludo de adiós y se marcharon en silencio.
\vs p182 2:8 Jesús sí le dijo a Andrés al alejarse aquella noche de su lado: “Andrés, haz lo que puedas para mantener a tus hermanos unidos hasta que yo vuelva a vosotros después de haber bebido esta copa. Fortalece a tus hermanos, ya que te lo he dicho todo. La paz esté contigo”.
\vs p182 2:9 Ninguno de los apóstoles esperaba que ocurriera nada fuera de lo ordinario esa noche, pues era ya muy tarde. Intentaron dormir para poder levantarse temprano por la mañana y estar preparados para lo peor. Pensaban que los sumos sacerdotes tratarían de prender a su Maestro por la mañana temprano, porque no se realizaba ninguna tarea secular después del mediodía del día de la preparación de la Pascua. Solo David Zebedeo y Juan Marcos se percataron de que los enemigos de Jesús vendrían con Judas aquella misma noche.
\vs p182 2:10 \pc David había planeado montar guardia esa noche en el sendero alto que llevaba a la carretera de Betania a Jerusalén, mientras que Juan Marcos lo haría en la carretera que subía del Cedrón a Getsemaní. Antes de ir David a su puesto de vigilancia, tal como él mismo se había impuesto, se despidió de Jesús diciendo: “Maestro, servirte me ha sido de gran gozo. Mis hermanos son tus apóstoles, pero yo me he regocijado haciendo las cosas menores que debían hacerse, y, cuando te hayas ido, te echaré de menos con todo mi corazón”. Y, entonces, Jesús dijo a David: “David, hijo mío, otros han hecho lo que se les pidió que hicieran, pero tú has hecho este servicio siguiendo tu propio corazón, y no he ignorado tu devoción. Algún día, tú también servirás conmigo en el reino eterno”.
\vs p182 2:11 Y, después, al prepararse para ir hacer guardia en el sendero alto, David le dijo a Jesús: “Sabes, Maestro que envié por tu familia, y tengo noticias por un mensajero de que esta noche están en Jericó. Mañana temprano, antes del mediodía, estarán aquí, puesto que sería peligroso que recorrieran por la noche el camino sangriento”. Y Jesús, bajando la mirada hacia David, solo dijo: “Que así sea, David”.
\vs p182 2:12 \pc Cuando David subió el Monte de los Olivos, Juan Marcos se dirigió hasta su punto de vigilancia, cerca de la carretera que descendía a lo largo del arroyo hasta Jerusalén. Y Juan habría permanecido en este puesto de no haber sido por su gran deseo de estar cerca de Jesús y saber lo que estaba sucediendo. Poco después de que David se alejara de él y, al observar que Jesús se retiraba con Pedro, Santiago y Juan a una quebrada cercana, Juan Marcos se vio tan desbordado por una mezcla de devoción y curiosidad que abandonó su puesto de centinela y los siguió, ocultándose entre los arbustos, desde donde vio y pudo oír todo lo que aconteció durante estos últimos momentos en el jardín y justo antes de que Judas y los guardias armados aparecieran para arrestar a Jesús.
\vs p182 2:13 \pc Mientras todo esto ocurría en el campamento del Maestro, Judas Iscariote se encontraba con el capitán de los guardianes del templo, que había reunido a sus hombres y se preparaban para salir a arrestar a Jesús, siguiendo la guía del traidor.
\usection{3. A SOLAS EN GETSEMANÍ}
\vs p182 3:1 Una vez que reinó la quietud y el silencio en el campamento, Jesús, acompañado de Pedro, Santiago y Juan, ascendió un corto trayecto hacia una quebrada cercana donde solía ir en ocasiones anteriores para orar y estar en comunión. Los tres apóstoles no pudieron evitar percibir que el Maestro sentía una gran pesadumbre. Jamás antes lo habían visto tan abrumado y angustiado. Cuando llegaron a su lugar de oración, los pidió que se sentaran y velaran con él mientras él se apartó de ellos a distancia como de un tiro de piedra para orar. Y cuando se postró sobre su rostro, oró: “Padre mío, he venido a este mundo para hacer tu voluntad, y así lo he hecho. Sé que ha llegado la hora de entregar esta vida en la carne, y no trato de evitarlo, pero quiero saber que tu voluntad es que yo beba esta copa. Asegúrame de que te contentaré en mi muerte al igual que lo hice en mi vida”.
\vs p182 3:2 El Maestro permaneció en actitud orante durante unos momentos, y volvió luego a los tres apóstoles; los halló profundamente dormidos porque los ojos de ellos estaban cargados de sueño y no lograban mantenerse despiertos. Al despertarlos Jesús, les dijo: “¡Qué! ¿Es que no habéis podido velar conmigo ni al menos una hora? ¿Es que no veis que mi alma está sumamente afligida hasta el límite de sus fuerzas, y que deseo intensamente vuestra compañía?”. Una vez que los tres se despertaron de su adormecimiento, el Maestro se fue nuevamente aparte, por sí solo, y, cayendo en tierra, oró de nuevo: “Padre, yo sé que es posible eludir esta copa ---todas las cosas son posibles para ti---, pero he venido para hacer tu voluntad, y aunque esta sea una copa amarga, la beberé si es tu voluntad”. Y cuando acabó de orar, un ángel poderoso bajó hasta su lado y, hablándole, lo tocó y le dio fuerzas.
\vs p182 3:3 Cuando Jesús regresó para hablar con ellos, los encontró otra vez profundamente dormidos. Los despertó diciendo: “En esta hora necesito que vosotros veléis y oréis conmigo ---pero incluso más que eso, necesitáis orar para que no entréis en la tentación---; ¿por qué os quedáis dormidos cuando os dejo?”.
\vs p182 3:4 Y, entonces, dejándolos, el Maestro oró por tercera vez: “Padre, tú ves a mis apóstoles dormidos, ten misericordia de ellos. A la verdad, el espíritu está dispuesto, pero la carne es débil. Y, ahora, ¡Oh Padre si no puede pasar de mi esta copa, la beberé. Que se haga tu voluntad y no la mía”. Y cuando terminó de orar, quedó postrado por un momento en el suelo. Cuando se levantó y volvió a donde estaban sus apóstoles, los encontró dormidos una vez más. Los contempló y, con un gesto compasivo, dijo tiernamente: “Dormid ya y descansad; el momento de la decisión ha pasado. Ha llegado la hora en la que el Hijo del Hombre será traicionado y entregado en manos de sus enemigos”. Al agacharse para sacudirlos con la intención de despertarlos, les dijo: “Levantaos, volvamos al campamento, porque se acerca el que me traiciona, y la hora ha llegado cuando mi redil será dispersado. Pero yo os he hablado ya de estas cosas”.
\vs p182 3:5 \pc Durante los años que Jesús vivió entre sus seguidores, estos, de hecho, tuvieron muchas pruebas de su naturaleza divina, pero en este momento están a punto de ser testigos de nuevos signos de su humanidad. Justo antes de la más grande de todas las revelaciones de su divinidad, su resurrección, debían ahora venir las más grandes pruebas de su naturaleza mortal: su humillación y crucifixión.
\vs p182 3:6 Cada vez que Jesús oraba en el jardín, con mayor firmeza su humanidad se aferraba a su divinidad por la fe; su voluntad humana se hizo más completamente una con la voluntad divina de su Padre. Entre las palabras del ángel poderoso estaba el mensaje de que el Padre deseaba que su Hijo terminara su ministerio de gracia en la tierra y pasara por la experiencia de la muerte como una criatura más, al igual que todas las criaturas mortales deben experimentar la disolución material cuando pasan desde la existencia del tiempo al camino de progreso en la eternidad.
\vs p182 3:7 Más temprano, a última hora de la tarde, no parecía haber sido tan difícil beber la copa, pero cuando el Jesús humano se despidió de sus apóstoles y los mandó a descansar, la prueba se volvió más atroz. Jesús pasó por esos altibajos de sentimientos, habituales en cualquier experiencia humana, y justo en aquel momento estaba agotado por el trabajo, exhausto por las largas horas de ardua labor y de dolorosa ansiedad por la seguridad de sus apóstoles. Aunque ningún mortal puede pretender comprender los pensamientos y sentimientos del Hijo de Dios encarnado, en un momento como este, sabemos que padeció una gran angustia y un sufrimiento indecible, porque grandes gotas de sudor rodaban por su rostro. Finalmente, se convenció de que era la intención del Padre permitir que los acontecimientos naturales siguieran su curso y se determinó totalmente a no emplear ninguno de sus poderes soberanos como jefe supremo de su universo para salvarse a sí mismo.
\vs p182 3:8 Las multitudes celestiales de toda una inmensa creación están ahora congregadas, sobrevolando aquella escena bajo el mando temporal conjunto de Gabriel y del modelador personificado de Jesús. A los comandantes de división de estos ejércitos del cielo se les ha advertido repetidas veces que no interfieran en estos acontecimientos terrenales, a no ser que Jesús mismo les ordene que intervengan.
\vs p182 3:9 \pc A Jesús, tener que separarse de sus apóstoles le producía una gran pena en su corazón humano; y este dolor por el amor que sentía hacia ellos se hizo pesaroso sobre él y le hacía más difícil enfrentarse a una muerte como la que él sabía bien que le aguardaba. Se daba cuenta de cuán débiles y faltos de conocimientos eran sus apóstoles, y temió dejarlos. Sabía bien que había llegado la hora de su partida, pero su corazón humano buscaba una forma posible, legítima, de poder eludir aquel terrible sufrimiento y angustia. Y cuando su corazón trató pues de hallar un escape, y no lograrlo, estuvo dispuesto a beber la copa. La mente divina de Miguel era consciente de que había hecho todo lo posible por los doce apóstoles; pero el corazón humano de Jesús deseaba poder haber hecho más por ellos antes de dejarlos solos en el mundo. El corazón de Jesús estaba totalmente abatido; realmente amaba a sus hermanos. Estaba aislado de su familia en la carne; uno de sus elegidos lo estaba traicionando. El pueblo de José, su padre, lo había rechazado, acabando con su destino como pueblo con una misión especial en la tierra. Su alma se sentía atormentada por su amor defraudado y por el rechazo de su misericordia. Fue uno de esos horrendos momentos como ser humano en lo que todo parecía golpearlo con una devastadora crueldad y una terrible agonía.
\vs p182 3:10 Lo humano en Jesús no era insensible a esta situación de íntima soledad, de escarnio público y de aparente fracaso de su causa. Todos estos sentimientos lo presionaban de manera indecible. En esta gran aflicción, su mente volvió a los días de su niñez en Nazaret y a su primera labor en Galilea. En el momento de aquella gran prueba vinieron a su mente placenteras escenas de su ministerio en la tierra. Y estos antiguos recuerdos de Nazaret, Cafarnaúm, el monte Hermón, y de los atardeceres y amaneceres del reluciente mar de Galilea lo sosegaron mientras su corazón humano cobraba fuerzas y se preparaba para recibir al traidor que tan pronto lo entregaría.
\vs p182 3:11 Antes de que Judas y los soldados llegaran, el Maestro ya había recobrado por completo su habitual compostura: el espíritu había triunfado sobre la carne; la fe se había impuesto sobre cualquier tendencia humana al albergar el temor y la duda. Había superado y pasado satisfactoriamente el test supremo, logrando ser completamente consciente de su naturaleza humana. Una vez más, el Hijo del Hombre estaba listo para enfrentarse a sus enemigos con calma y con la plena certeza de su invencibilidad como hombre mortal, incondicionalmente dedicado a hacer la voluntad de su Padre.
