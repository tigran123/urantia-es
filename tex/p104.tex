\upaper{104}{Expansión del concepto de la Trinidad}
\author{Melquisedec}
\vs p104 0:1 El concepto de la Trinidad de la religión revelada no debe confundirse con la creencia acerca de las tríadas de las religiones evolutivas. Las ideas respecto a las tríadas surgieron de múltiples y sugerentes analogías, pero, sobre todo, debido a las tres articulaciones de los dedos, a que se precisaba un mínimo de tres patas para dar estabilidad a un taburete y a que eran tres los puntos de apoyo necesarios para sostener una tienda; además, el hombre primitivo, durante mucho tiempo, no sabía contar más allá del número tres.
\vs p104 0:2 Aparte de ciertos dobletes naturales, tales como el pasado y el presente, el día y la noche, el calor y el frío, lo masculino y lo femenino, el hombre tiende, por lo general, a pensar en tríadas: ayer, hoy y mañana; amanecer, mediodía y atardecer; padre, madre e hijo. Se vitorea tres veces al ganador. Los muertos se entierran al tercer día, y se apacigua a los espectros por medio de tres abluciones de agua.
\vs p104 0:3 A raíz de estas correlaciones naturales de la experiencia humana, la tríada apareció en la religión mucho antes de que se revelara a la humanidad la Trinidad de Deidades del Paraíso, o incluso algunos de sus representantes. Más adelante, persas, hindúes, griegos, egipcios, babilonios, romanos y escandinavos tuvieron tríadas de dioses, pero estas seguían sin ser verdaderas trinidades. Todas las tríadas de deidades tuvieron un origen natural y han aparecido, en un momento u otro, en la mayoría de los pueblos inteligentes de Urantia. En ocasiones, el concepto de tríada evolutiva se ha llegado a mezclar con el de la Trinidad revelada; en estos casos, resulta a menudo imposible distinguir la una de la otra.
\usection{1. CONCEPTOS URANTIANOS SOBRE LA TRINIDAD}
\vs p104 1:1 La primera revelación en Urantia conducente a la comprensión de la Trinidad del Paraíso la realizó la comitiva del príncipe Caligastia hace aproximadamente medio millón de años. Este primer concepto de la Trinidad se perdió para el mundo durante los agitados tiempos que siguieron a la rebelión planetaria.
\vs p104 1:2 La segunda exposición de la Trinidad la efectuaron Adán y Eva en el primer y segundo jardín. Estas enseñanzas no llegaron a borrarse del todo ni siquiera en la época de Maquiventa Melquisedec, unos treinta y cinco mil años más tarde. El concepto de la Trinidad de los setitas persistió en Mesopotamia y en Egipto aunque, más particularmente, en la India, donde se perpetuó por mucho tiempo en Agni, el dios védico trifásico del fuego.
\vs p104 1:3 La tercera exposición de la Trinidad la realizó Maquiventa Melquisedec, y esta doctrina se simbolizó con los tres círculos concéntricos que el sabio de Salem portaba en el pecho. Pero Maquiventa tuvo muchas dificultades para impartir a los beduinos palestinos enseñanzas sobre el Padre Universal, el Hijo Eterno y el Espíritu Infinito. La mayoría de sus discípulos pensaban que la Trinidad constaba de los tres altísimos de Norlatiadek; algunos pocos concebían la Trinidad como si estuviese constituida por el Soberano del Sistema, el Padre de la Constelación y la Deidad Creadora del universo local; aún eran menos los que lejanamente captaron la idea de la vinculación en el Paraíso del Padre, el Hijo y el Espíritu.
\vs p104 1:4 A través de la actividad de los misioneros de Salem, las enseñanzas de Melquisedec sobre la Trinidad se difundieron, poco a poco, por gran parte de Eurasia y el norte de África. Suele ser difícil distinguir entre las tríadas y las trinidades en las eras más tardías de los anditas y la posterior a Melquisedec, cuando ambos conceptos, en cierto modo, se entremezclaron y aglutinaron.
\vs p104 1:5 \pc Entre los hindúes, el concepto trinitario arraigó como Ser, Inteligencia y Felicidad. (Un concepto indio más tardío fue el de Brahma, Shiva y Vishnu.) Aunque los sacerdotes setitas llevaron a la India unas primeras descripciones de la Trinidad, las ideas posteriores acerca de esta se importaron por los misioneros de Salem y se desarrollaron por personas nativas de la India de gran capacidad intelectual, que combinaron estas doctrinas con los conceptos evolutivos de las tríadas.
\vs p104 1:6 La fe budista desarrolló dos doctrinas de naturaleza trinitaria: la primera fue Maestro, Ley y Hermandad, definida por Gautama Siddharta. La idea más tardía, que se desarrolló en la rama norteña de los seguidores de Buda, incluía el Señor Supremo, el Espíritu Santo y el Salvador Encarnado.
\vs p104 1:7 Y estas ideas de los hindúes y budistas eran auténticos postulados trinitarios, esto es, se enunciaba la manifestación triple de un Dios monoteísta. La concepción de la Trinidad no radica en una mera agrupación de tres dioses separados.
\vs p104 1:8 \pc Los hebreos sabían de la Trinidad por las tradiciones ceneas de los días de Melquisedec, pero su fervor monoteísta por el Dios único, Yahvé, eclipsó de tal manera todas estas enseñanzas que, en el momento de la aparición de Jesús, la doctrina de Elohim se había erradicado prácticamente de la teología judía. La mente hebrea no podía conciliar el concepto trinitario con la creencia monoteísta en el Señor Único, el Dios de Israel.
\vs p104 1:9 Los partidarios de la fe islámica tampoco lograron comprender la idea de la Trinidad. Le resulta siempre difícil al monoteísmo incipiente admitir el trinitarismo cuando se ve confrontado por el politeísmo. La idea de la trinidad se afianza más en aquellas religiones que poseen una firme tradición monoteísta sumada a una flexibilidad doctrinal. Los grandes monoteístas, los hebreos y los mahometanos, tuvieron dificultades para distinguir entre la adoración a tres dioses, el politeísmo, y el trinitarismo, la adoración a una Deidad que existe como manifestación trina de la divinidad y del ser personal.
\vs p104 1:10 \pc Jesús enseñó a sus apóstoles la verdad en relación a las personas de la Trinidad del Paraíso, pero pensaron que les hablaba de manera figurada y simbólica. Habiéndose educado en el monoteísmo hebreo, les era difícil albergar cualquier creencia que pudiera estar en conflicto con su predominante concepto de Yahvé. Y los cristianos primitivos heredaron el prejuicio hebreo contra la idea de la Trinidad.
\vs p104 1:11 En Antioquía se dio a conocer la primera Trinidad del cristianismo y consistía en Dios, su Verbo y su Sabiduría. Pablo tenía conocimiento de la Trinidad del Paraíso del Padre, el Hijo y el Espíritu, pero raras veces la predicó y solo la mencionó en algunas de sus epístolas a las iglesias que se estaban formando. Aún así, tal como les ocurrió a sus compañeros apóstoles, Pablo confundió a Jesús, el hijo creador del universo local, con la Segunda Persona de la Deidad, el Hijo Eterno del Paraíso.
\vs p104 1:12 El concepto cristiano de la Trinidad, que empezó a obtener reconocimiento casi al final del siglo I d. C., estaba integrado por el Padre Universal, el hijo creador de Nebadón y la benefactora divina de Lugar de Salvación ---el espíritu materno del universo local y consorte creativa del hijo creador---.
\vs p104 1:13 Desde los tiempos de Jesús no se ha conocido en Urantia la identidad fehaciente de la Trinidad del Paraíso (salvo por algunas pocas personas a quienes les fue expresamente revelada) hasta que su naturaleza se divulgó en esta revelación. Si bien, aunque el concepto cristiano de la Trinidad incurrió en un manifiesto error, estaba prácticamente en lo cierto en término de sus relaciones espirituales. Este concepto resulta problemático únicamente en sus implicaciones filosóficas y en sus consecuencias cosmológicas: a muchas personas conscientes del cosmos les resulta difícil creer que la Segunda Persona de la Deidad, el segundo miembro de la Trinidad Infinita, habitara alguna vez en Urantia; y, aunque en espíritu, esto sea verdad, no es un hecho en realidad. Los creadores del orden de los migueles personifican en plenitud la divinidad del Hijo Eterno, pero no son el ser personal absoluto.
\usection{2. LA UNIDAD DE LA TRINIDAD Y LA PLURALIDAD DE LA DEIDAD}
\vs p104 2:1 El monoteísmo surgió como disconformidad filosófica contra la incongruencia del politeísmo. Se organizó primero como panteón, con una compartimentación de la actividad sobrenatural; luego, se desarrolló, a través de la exaltación henoteísta de un dios superior a los demás dioses, hasta llegar por fin a excluirlos a todos salvo al Dios Único de valor final.
\vs p104 2:2 El trinitarismo surge ante la objeción, a niveles experienciales, de la imposibilidad de experimentar la unicidad de una Deidad desantropomorfizada y solitaria, cuya significación está desvinculada del universo. Con el suficiente tiempo, la filosofía tiende a abstraer las cualidades personales del concepto de la Deidad del monoteísmo puro, reduciendo así esta idea de un Dios inconexo a la condición de un Absoluto panteísta. Siempre resultó difícil comprender la naturaleza personal de un Dios que no tiene relaciones personales de igualdad con otros seres personales del mismo rango. En la Deidad, el ser personal exige que tal Deidad exista en relación de igualdad con otra Deidad personal.
\vs p104 2:3 A través del reconocimiento del concepto de la Trinidad, la mente del hombre puede abrigar la esperanza de comprender algo acerca de la relación entre el amor y la ley existente en las creaciones espacio\hyp{}temporales. Mediante la fe espiritual, el hombre alcanza a percibir el amor de Dios, pero pronto descubre que dicha fe no tiene influencia sobre las leyes establecidas en el universo material. Con independencia de la solidez de la creencia del hombre en Dios como Padre del Paraíso, sus horizontes cósmicos en expansión demandan que también reconozca la realidad de la Deidad del Paraíso como ley universal, que se reconozca la soberanía de la Trinidad que se extiende hacia fuera desde el Paraíso e incluso prepondera en los universos locales evolutivos de los hijos creadores y de las hijas creativas de las tres personas eternas, cuya unión como Deidad \bibemph{es} el hecho y la realidad de la indivisibilidad eterna de la Trinidad del Paraíso.
\vs p104 2:4 Y esta misma Trinidad del Paraíso tiene entidad real: no es un ser personal, pero es, no obstante, una realidad verdadera y absoluta; no es un ser personal, pero está, no obstante, en armonía con seres personales coexistentes ---las personas del Padre, del Hijo y del Espíritu---. La Trinidad es una realidad extraaditiva de la Deidad que deviene de la conjunción de las tres Deidades del Paraíso. Las cualidades, características y labor de la Trinidad no son la simple suma de los atributos de las tres Deidades del Paraíso; la labor de la Trinidad es única, original y totalmente imprevisible a partir del análisis de los atributos del Padre, del Hijo y del Espíritu.
\vs p104 2:5 Por ejemplo, el Maestro, cuando estaba en la tierra, advirtió a sus seguidores que la justicia no es nunca un acto \bibemph{persona}l; está siempre en función del \bibemph{grupo}. Tampoco los Dioses, como personas imparten la justicia. Pero sí ejercen esta labor como un todo colectivo, como la Trinidad del Paraíso.
\vs p104 2:6 La comprensión conceptual de la conjunción trinitaria del Padre, del Hijo y del Espíritu prepara a la mente humana para el posterior relato de determinadas otras relaciones triples. La razón teológica puede quedar plenamente satisfecha por el concepto de la Trinidad del Paraíso, pero la razón filosófica y la cosmológica reclaman el reconocimiento de otras agrupaciones trinas de la Primera Fuente y Centro, unas triunidades en las que el Infinito actúa en diversas funciones no paternales que se manifiestan universalmente: las relaciones del Dios de la fuerza, la energía, la potencia, la causalidad, la reacción, la potencialidad, la actualidad, la gravedad, la tensión, el modelo, el principio y la unidad.
\usection{3. TRINIDADES Y TRIUNIDADES}
\vs p104 3:1 Aunque en ocasiones la humanidad ha intentado entender la Trinidad de las tres personas de la Deidad, la congruencia impone que el intelecto humano perciba que existen determinadas relaciones entre los siete Absolutos. Pero todo lo que es verdad de la Trinidad del Paraíso no lo es necesariamente de una \bibemph{triunidad,} porque una triunidad es algo distinto a una trinidad. En algunos aspectos de carácter operativo, una triunidad puede ser análoga a una trinidad, pero nunca es homóloga en cuanto a su naturaleza.
\vs p104 3:2 En Urantia, el hombre mortal está atravesando una gran era en la ampliación de sus horizontes y conceptos, y su filosofía cósmica debe acelerar su desarrollo para poder seguir el ritmo expansivo del entorno intelectual del pensamiento humano. Conforme se expande su conciencia cósmica, el hombre mortal percibe la interrelación de todo lo que halla en su ciencia material, filosofía intelectual y percepción espiritual. No obstante, debido a esta creencia en la unidad del cosmos, el hombre aprecia la diversidad de todo lo que existe. A pesar de todos los conceptos sobre la inmutabilidad de la Deidad, el hombre considera que vive en un universo de cambio constante y de crecimiento experiencial. Al margen de la cognición de la supervivencia de los valores espirituales, el hombre debe tener siempre en cuenta las matemáticas y las prematemáticas de la fuerza, la energía y la potencia.
\vs p104 3:3 De alguna manera, la plenitud eterna de la infinitud debe conciliarse con el crecimiento temporal de los universos evolutivos y con la incompletitud de sus habitantes experienciales. En cierta forma, el concepto de infinidad total debe segmentarse y delimitarse de tal modo que el intelecto mortal y el alma morontial puedan comprender tal concepto de valor final y de significación espiritual.
\vs p104 3:4 Aunque la razón impone una unidad monoteísta de la realidad cósmica, la experiencia a niveles finitos requiere el postulado de Absolutos plurales y su coordinada actuación en las relaciones cósmicas. Sin la correlación de sus existencias no hay posibilidad de que se diversifiquen las relaciones absolutas, de que operen los diferenciales, las variables, los modificadores, los atenuadores, los delimitadores y los reductores.
\vs p104 3:5 \pc En estos escritos, la realidad total (la infinitud) se ha presentado tal y como existe en los siete Absolutos:
\vs p104 3:6 \li{1.}El Padre Universal.
\vs p104 3:7 \li{2.}El Hijo Eterno.
\vs p104 3:8 \li{3.}El Espíritu Infinito.
\vs p104 3:9 \li{4.}La Isla del Paraíso.
\vs p104 3:10 \li{5.}El Absoluto de la Deidad.
\vs p104 3:11 \li{6.}El Absoluto Universal.
\vs p104 3:12 \li{7.}El Absoluto Indeterminado.
\vs p104 3:13 \pc La Primera Fuente y Centro, que es Padre del Hijo Eterno, es también el Modelo de la Isla del Paraíso. Su ser personal es incondicionado en el Hijo pero potencial en el Absoluto de la Deidad. El Padre es energía revelada en el Paraíso\hyp{}Havona y, al mismo tiempo, energía oculta en el Absoluto Indeterminado. El Infinito se desvela por siempre en la actuación constante del Actor Conjunto mientras que está eternamente obrando en la actividad compensatoria pero velada del Absoluto Universal. De este modo, el Padre está vinculado a los seis Absolutos, a su vez correlacionados entre sí, y, por ello, los siete, en su conjunto, abarcan el círculo de la infinitud a lo largo de los ciclos interminables de la eternidad.
\vs p104 3:14 \pc Podría parecer que la triunidad de las relaciones absolutas fuera inevitable. El ser personal busca vincularse con otros seres personales en un nivel absoluto al igual que en todos los otros niveles. Y la agrupación de los tres seres personales del Paraíso eterniza la primera triunidad, la unión de las personas del Padre, del Hijo y del Espíritu. Porque cuando estas tres personas, \bibemph{como personas,} se reúnen para actuar unificadamente constituyen, por consiguiente, una triunidad de acción unitaria, no una trinidad ---que es una entidad orgánica---, pero, aún así, una triunidad, esto es, una unanimidad triple de acción conjunta.
\vs p104 3:15 La Trinidad del Paraíso no es una triunidad; no conlleva unanimidad de acción; es más bien una Deidad indivisa e indivisible. El Padre, el Hijo y el Espíritu (como personas) pueden relacionarse con la Trinidad del Paraíso, porque la Trinidad \bibemph{es} su Deidad indivisa. El Padre, el Hijo y el Espíritu no mantienen tales relaciones personales con la primera triunidad, porque esta \bibemph{constituye} la unidad de acción de sus tres personas. Solamente como Trinidad ---como Deidad indivisa--- mantienen de forma colectiva una relación externa con la triunidad conformada por la suma de sus personas.
\vs p104 3:16 De este modo, la Trinidad del Paraíso es única entre las relaciones absolutas; hay varias triunidades existenciales, pero solo una Trinidad existencial. Una triunidad \bibemph{no} es una entidad. Tiene carácter operativo en lugar de orgánico. Sus miembros son colaboradores en lugar de ser mutuamente interdependientes. Los componentes de las triunidades pueden ser entidades, pero la triunidad es en sí misma una agrupación trina.
\vs p104 3:17 Existe, sin embargo, un aspecto en el que trinidad y triunidad son equiparables: ambas devienen como resultado de actuaciones que son algo más que la suma perceptible de los atributos de los miembros que las componen. Si bien, aunque, desde una perspectiva de carácter operativo sean, en ese sentido, análogas, por lo demás no dan muestra manifiesta de conexión. Aproximadamente, se corresponden entre sí como la relación existente entre función y estructura. Si bien, la función de una agrupación triunitaria no es la función de la estructura o entidad trinitaria.
\vs p104 3:18 Las triunidades son, no obstante, reales; muy reales. Con ellas, se activa la realidad total, y, a través de ellas, el Padre Universal ejerce un dominio inmediato y personal sobre las funciones prevalentes de la infinitud.
\usection{4. LAS SIETE TRIUNIDADES}
\vs p104 4:1 Al tratar de describir las siete triunidades, se señala la atención al hecho de que el Padre Universal constituye el miembro primordial de cada una de ellas. Él es, fue y siempre será: el Primer Padre\hyp{}Fuente Universal, el Centro Absoluto, la Causa Primordial, el Rector Universal, el Energizador Sin Límites, la Unidad Primigenia, el Sostenedor Incondicionado, la Primera Persona de la Deidad, el Modelo Cósmico Primordial y la Esencia de la Infinitud. El Padre Universal es la causa personal de los Absolutos; él es el absoluto de los Absolutos.
\vs p104 4:2 \pc Se puede plantear la naturaleza y el significado de las siete triunidades de la manera siguiente:
\vs p104 4:3 \pc \bibemph{La primera triunidad ---la triunidad personal\hyp{}intencional---.} Se trata de la agrupación de los tres seres personales de la Deidad:
\vs p104 4:4 \li{1.}El Padre Universal.
\vs p104 4:5 \li{2.}El Hijo Eterno.
\vs p104 4:6 \li{3.}El Espíritu Infinito.
\vs p104 4:7 \pc Es la unión triple del amor, la misericordia y el ministerio ---la conjunción personal e intencional de los tres seres personales eternos del Paraíso---. Constituyen una confluencia divinamente fraternal que ama a las criaturas, que actúa paternalmente y favorece la ascensión. Los seres personales divinos de esta primera triunidad son Dioses que imparten el ser personal, conceden el espíritu y otorgan la mente.
\vs p104 4:8 Es la triunidad de la volición infinita; actúa a lo largo del presente eterno y durante todo el curso del tiempo, pasado presente y futuro. La agrupación trina de estos seres personales aporta la infinitud volitiva y facilita los mecanismos por los que la Deidad personal se revela a sí misma a las criaturas del cosmos evolutivo.
\vs p104 4:9 \pc \bibemph{La segunda triunidad ---la triunidad potencia\hyp{}modelo}---. Ya sea un diminuto ultimatón, una estrella llameante o nebulosa rotante, incluso el universo central o los suprauniversos, desde la más pequeña hasta la más grande de las organizaciones materiales, siempre se deriva su modelo físico ---su configuración cósmica--- de la acción de esta triunidad, que consta de:
\vs p104 4:10 \li{1.}El Padre\hyp{}Hijo.
\vs p104 4:11 \li{2.}La Isla del Paraíso.
\vs p104 4:12 \li{3.}El Actor Conjunto.
\vs p104 4:13 \pc Las instancias intermedias cósmicas de la Tercera Fuente y Centro organizan la energía, que está diseñada de acuerdo con el modelo del Paraíso, la materialización absoluta; pero detrás de toda esta incesante actuación sobre la energía está la presencia del Padre\hyp{}Hijo, cuya unión activó por primera vez el modelo primigenio del Paraíso, dando lugar a la aparición de Havona en conjunción con el nacimiento del Espíritu Infinito: el Actor Conjunto.
\vs p104 4:14 En la experiencia religiosa, las criaturas se ponen en contacto con el Dios que es amor, pero tal percepción espiritual nunca debe eclipsar el reconocimiento inteligente del hecho universal del modelo que representa el Paraíso. Los seres personales del Paraíso suscitan la adoración voluntaria de todas las criaturas gracias al imperioso poder del amor divino y llevan a todos estos seres personales, nacidos del espíritu, al gozo inefable del servicio sin fin de los hijos finalizadores de Dios. La segunda triunidad es el arquitecto del escenario espacial en el que se desarrollan estas interacciones y determina los modelos de la configuración cósmica.
\vs p104 4:15 El amor puede caracterizar a la divinidad de la primera triunidad, pero el modelo es la expresión galáctica de la segunda triunidad. Lo que la primera triunidad es para los seres personales evolutivos, la segunda triunidad lo es para los universos evolutivos. El modelo y el ser personal son dos de las grandes manifestaciones de las acciones de la Primera Fuente y Centro; y, por difícil que resulte de comprender, es, sin embargo, cierto que la potencia\hyp{}modelo y la persona amorosa son una única y misma realidad universal; la Isla del Paraíso y el Hijo Eterno constituyen revelaciones coiguales, aunque antitéticas, de la inescrutable naturaleza del Padre\hyp{}Fuerza Universal.
\vs p104 4:16 \pc \bibemph{La tercera triunidad ---la triunidad espíritu\hyp{}evolutiva}---. La totalidad de la manifestación espiritual tiene su comienzo y su fin en esta triunidad, que consta de:
\vs p104 4:17 \li{1.}El Padre Universal.
\vs p104 4:18 \li{2.}El Hijo\hyp{}Espíritu.
\vs p104 4:19 \li{3.}El Absoluto de la Deidad.
\vs p104 4:20 \pc Desde el potencial espiritual, hasta el espíritu del Paraíso, todo espíritu encuentra la expresión de su realidad en esta relación trina de la esencia del espíritu puro del Padre, de los valores espirituales activos del Hijo\hyp{}Espíritu y de los potenciales espirituales ilimitados del Absoluto de la Deidad. Los valores existenciales del espíritu tienen su génesis primordial, su manifestación completa y su destino final en esta triunidad.
\vs p104 4:21 El Padre existe con anterioridad al espíritu; el Hijo\hyp{}Espíritu obra como espíritu activo creativo; el Absoluto de la Deidad existe como espíritu que todo lo abarca, incluso más allá del espíritu.
\vs p104 4:22 \pc \bibemph{La cuarta triunidad ---la triunidad de la infinitud de la energía---.} En esta triunidad se eternizan los inicios y los finales de toda la realidad de la energía, desde la potencia del espacio hasta la monota. Esta agrupación trina incluye los siguientes miembros:
\vs p104 4:23 \li{1.}El Padre\hyp{}Espíritu.
\vs p104 4:24 \li{2.}La Isla del Paraíso.
\vs p104 4:25 \li{3.}El Absoluto Indeterminado.
\vs p104 4:26 \pc El Paraíso es el centro de la activación de la fuerza\hyp{}energía del cosmos ---la ubicación en el universo de la Primera Fuente y Centro, el punto de la actividad cósmica del Absoluto Indeterminado y la fuente de toda la energía---. Existencialmente presente en esta triunidad está el potencial energético del cosmos infinito, del que el gran universo y el universo matriz solo son manifestaciones parciales.
\vs p104 4:27 La cuarta triunidad rige de forma absoluta las unidades fundamentales de la energía cósmica y las libera del control del Absoluto Indeterminado en proporción directa a la aparición en las Deidades experienciales de capacidad subabsoluta para regir y estabilizar el cosmos en su metamorfosis.
\vs p104 4:28 Esta triunidad \bibemph{es} fuerza y energía. Las ilimitadas posibilidades del Absoluto Indeterminado se centran en torno al absolutum de la Isla del Paraíso, de donde emanan unas inconcebibles perturbaciones del Indeterminado, por lo demás, en quiescencia estática. Y la palpitación sinfín del corazón material paradisíaco del cosmos infinito late en armonía con el inconmensurable modelo y el impenetrable plan del Energizador Infinito, de la Primera Fuente y Centro.
\vs p104 4:29 \pc \bibemph{La quinta triunidad ---la triunidad de la infinitud reactiva}---, que consta de:
\vs p104 4:30 \li{1.}El Padre Universal.
\vs p104 4:31 \li{2.}El Absoluto Universal.
\vs p104 4:32 \li{3.}El Absoluto Indeterminado.
\vs p104 4:33 \pc Este grupo consigue eternizar y llevar a efecto en la infinitud la realización de todo lo que es actualizable del ámbito de la realidad no relativa a la Deidad. Esta triunidad manifiesta una capacidad para reaccionar a la acción y presencia de otras triunidades, en cuanto a la volición, la causación, la tensión y el modelo.
\vs p104 4:34 \pc \bibemph{La sexta triunidad ---la triunidad de la Deidad cósmico\hyp{}vinculada---.} Este grupo consiste de:
\vs p104 4:35 \li{1.}El Padre Universal.
\vs p104 4:36 \li{2.}El Absoluto de la Deidad.
\vs p104 4:37 \li{3.}El Absoluto Universal.
\vs p104 4:38 Se trata de la relación trina de la Deidad en el cosmos, la inmanencia de la Deidad en conjunción con la trascendencia de la Deidad. Esta es la última prolongación de la divinidad, en los niveles de la infinitud, hacia aquellas realidades que están fuera del ámbito de la realidad en cuanto deidad.
\vs p104 4:39 \pc \bibemph{La séptima triunidad ---la triunidad de la unidad infinita---.} Conlleva la unidad de la infinitud operativamente manifiesta en el tiempo y en la eternidad, la unificación equilibrada de los actuales y de los potenciales. Este grupo consiste en:
\vs p104 4:40 \li{1.}El Padre Universal.
\vs p104 4:41 \li{2.}El Actor Conjunto.
\vs p104 4:42 \li{3.}El Absoluto Universal.
\vs p104 4:43 \pc El Actor Conjunto integra universalmente los diversos aspectos de carácter operativo de toda la realidad actualizada en todos sus niveles de manifestación, desde los finitos, a través de los trascendentales, hasta los absolutos. El Absoluto Universal compensa perfectamente los diferenciales inherentes en los diversos aspectos de cualquier realidad incompleta, desde las ilimitadas potencialidades de la realidad activo\hyp{}volitiva y causativa de la Deidad hasta las inagotables posibilidades de la realidad en cuanto no deidad, estática y reactiva, en las incomprensibles esferas de acción del Absoluto Indeterminado.
\vs p104 4:44 Tal como actúan en esta triunidad, el Actor Conjunto y el Absoluto Universal son igualmente receptivos a la Deidad y a las presencias en cuanto no deidad, así como también lo es la Primera Fuente y Centro, que en esta relación es, para todos los efectos, conceptualmente indistinguible del YO SOY.
\vs p104 4:45 \pc Estas consideraciones son suficientes para dilucidar el concepto de las triunidades. No conociendo el nivel último de las triunidades, no es posible que comprendáis por completo las primeras siete. Aunque no estimamos que sea sensato tratar de realizar una elaboración más detallada de ellas, podemos indicar que existen quince relaciones trinas de la Primera Fuente y Centro, de las que ocho no se revelan en estos documentos. Las que están sin revelar aluden a realidades, manifestaciones y potencialidades, que están más allá del nivel experiencial de la supremacía.
\vs p104 4:46 Las triunidades son, operativamente, la rueda de balance de la infinitud, la unificación de la especificidad de los Siete Absolutos de la Infinitud. Es la presencia existencial de las triunidades la que posibilita que el Padre\hyp{}YO SOY experimente la unidad de la infinitud operativa a pesar de la diversificación de esta infinitud en siete Absolutos. La Primera Fuente y Centro es el miembro que unifica todas las triunidades; en él todas las cosas tienen sus orígenes absolutos, sus existencias eternas, sus destinos infinitos ---“todas las cosas en él subsisten”---.
\vs p104 4:47 Aunque estas relaciones trinas no puedan acrecentar la infinitud del Padre\hyp{}YO SOY, parecen posibilitar las manifestaciones subinfinitas y subabsolutas de su realidad. Las siete triunidades multiplican la capacidad adaptativa del infinito, eternizan sus nuevas profundidades, deidifican sus nuevos valores, desvelan sus nuevas potencialidades, revelan sus nuevos contenidos; y todas estas diversas manifestaciones que ocurren en el tiempo y el espacio y en el cosmos eterno tienen su existencia en la inactividad hipotética de la infinitud primigenia del YO SOY.
\usection{5. LAS TRIODIDADES}
\vs p104 5:1 Existen determinadas otras relaciones trinas, no formadas por el Padre, que no son verdaderas triunidades, y se distinguen siempre de las triunidades que sí incluyen al Padre. Se les han llamado, indistintamente, triunidades de compañeros, triunidades correlacionadas y \bibemph{triodidades}. Resultan como consecuencia de la existencia de las triunidades. Dos de estos grupos se constituyen tal como sigue:
\vs p104 5:2 \bibemph{La triodidad de actualidad}. Esta triodidad consiste en la interrelación de tres absolutos actuales:
\vs p104 5:3 \li{1.}El Hijo Eterno.
\vs p104 5:4 \li{2.}La Isla del Paraíso.
\vs p104 5:5 \li{3.}El Actor Conjunto.
\vs p104 5:6 \pc El Hijo Eterno es el absoluto de la realidad espiritual, el ser personal absoluto. La Isla del Paraíso es el absoluto de la realidad cósmica, el modelo absoluto. El Actor Conjunto es el absoluto de la realidad de la mente, el correlato de la realidad espiritual absoluta y la síntesis del ser personal y de la potencia de la Deidad existencial. Esta relación trina deviene en la coordinación de la suma total de la realidad actualizada ---espiritual, cósmica o mental---. Es incondicionada en actualidad.
\vs p104 5:7 \pc \bibemph{La triodidad de potencialidad}. Esta triodidad consiste en la relación de los tres Absolutos de la potencialidad:
\vs p104 5:8 \li{1.}El Absoluto de la Deidad.
\vs p104 5:9 \li{2.}El Absoluto Universal.
\vs p104 5:10 \li{3.}El Absoluto Indeterminado.
\vs p104 5:11 \pc De esta manera se correlacionan los depósitos infinitos de toda la realidad energética latente ---espiritual, mental o cósmica---. Esta conjunción trina aporta la integración de toda esta realidad energética inactiva. Es potencialmente infinita.
\vs p104 5:12 \pc Las triunidades se ocupan fundamentalmente de la unificación de carácter operativo de la infinitud, así pues, las triodidades intervienen en la aparición cósmica de las Deidades experienciales. Las triunidades se ocupan indirectamente, pero las triodidades intervienen de forma directa en las Deidades experienciales: Suprema, Última y Absoluta. Aparecen en la síntesis emergente de la potencia\hyp{}ser personal del Ser Supremo. Y, para las criaturas del tiempo y del espacio, el Ser Supremo es una revelación de la unidad del YO SOY.
\vsetoff
\vs p104 5:13 [Exposición de un melquisedec de Nebadón.]
