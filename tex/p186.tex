\upaper{186}{Justo antes de la crucifixión}
\author{Comisión de seres intermedios}
\vs p186 0:1 En el momento en el que Jesús y sus acusadores se pusieron en marcha para ir a ver a Herodes, el Maestro se volvió al apóstol Juan y le dijo: “Juan, no puedes hacer ya nada más por mí. Ve a buscar a mi madre y tráela para que me vea antes de morir”. Cuando Juan oyó la petición del Maestro, aunque se resistía a dejarlo solo entre sus enemigos, se apresuró hacia Betania, donde toda la familia de Jesús estaba reunida, a la espera de noticias, en la casa de Marta y María, las hermanas de Lázaro, a quien Jesús había resucitado de entre los muertos.
\vs p186 0:2 Varias veces, durante la mañana, los mensajeros habían informado a Marta y a María sobre el desarrollo del juicio contra Jesús. Pero su familia no llegó a Betania hasta algunos minutos antes de que lo hiciera Juan llevando el deseo de Jesús de ver a su madre antes de ser ejecutado. Una vez que Juan Zebedeo les contó todo lo ocurrido desde el arresto de Jesús a media noche, María, su madre, fue enseguida, junto con Juan, a ver a su hijo mayor. Cuando María y Juan accedieron a la ciudad, Jesús, acompañado de los soldados romanos que iban a crucificarlo, ya había llegado al Gólgota.
\vs p186 0:3 Cuando María, la madre de Jesús, partió con Juan para ir a ver a su hijo, su hermana Ruth no quiso quedarse atrás con el resto de la familia; y, dado que tenía decidido acompañar a su madre, su hermano Judá fue con ella. Los demás miembros de la familia del Maestro permanecieron en Betania bajo la guía de Santiago y, casi cada hora, los mensajeros de David Zebedeo les informaban paso a paso sobre el hecho terrible de la ejecución de Jesús de Nazaret, su hermano mayor.
\usection{1. EL FINAL DE JUDAS ISCARIOTE}
\vs p186 1:1 Sobre las ocho y media de ese viernes por la mañana, al acabar la comparecencia de Jesús ante Pilato, se colocó al Maestro bajo la custodia de los soldados romanos que iban a crucificarlo. En cuanto los romanos tuvieron en sus manos a Jesús, el capitán de los guardias judíos partió de vuelta con sus hombres a su cuartel general del templo. El sacerdote principal y sus acompañantes sanedritas siguieron de cerca a los guardias, y se dirigieron directamente a su lugar acostumbrado de reunión en la cámara de piedras labradas del templo. Allí encontraron a muchos otros miembros del sanedrín que estaban a la espera de saber qué se había hecho con Jesús. Mientras Caifás presentaba su informe al sanedrín sobre el juicio y la condena de Jesús, Judas apareció ante ellos para reclamar su recompensa por haber participado en el arresto y la sentencia a muerte de su Maestro.
\vs p186 1:2 Todos estos judíos aborrecían a Judas; veían en el traidor a alguien totalmente despreciable. Mientras duró el juicio de Jesús ante Caifás y durante su aparición ante Pilato, a Judas le remordía la conciencia por su alevoso comportamiento hacia él. Y empezaba a sentirse algo decepcionado respecto a la recompensa que había de recibir como pago por traicionar a Jesús. No le gustaba la frialdad y el desinterés de las autoridades judías; no obstante, esperaba que se le retribuyera generosamente por su cobarde conducta. Tenía la expectativa de que lo llamaran ante la asamblea plenaria del sanedrín y de que oiría sus alabanzas a la vez que le otorgaban los merecidos honores como muestra por el gran servicio que, jactanciosamente, creía haber rendido a su nación. Imaginad, pues, la gran sorpresa de este egocéntrico traidor cuando un sirviente del sumo sacerdote, dándole una palmada en el hombro, lo hizo salir de la sala y le dijo: “Judas, se me ha encargado que te pague por la traición de Jesús. Aquí está tu recompensa”. Y diciéndole esto, este criado de Caifás le entregó a Judas una bolsa con treinta piezas de plata ---el precio ordinario de un buen esclavo que gozara de salud---.
\vs p186 1:3 Judas se quedó estupefacto, atónito. Se dio la vuelta deprisa para entrar en la sala, pero el portero no se lo permitió. Quería presentar una apelación al sanedrín, pero no le dieron su consentimiento. Judas no podía concebir cómo estos líderes de los judíos le habían permitido traicionar a sus amigos y a su Maestro para luego ofrecerle treinta piezas de plata como recompensa. Se sentía humillado, desilusionado y totalmente deshecho. Se alejó del templo como en estado de trance. De modo instintivo, metió la bolsa del dinero en su gran bolsillo, el mismo en el que, durante tanto tiempo, había llevado la bolsa con los fondos apostólicos. Y vagó por las calles de la ciudad tras las multitudes que iban a presenciar las crucifixiones.
\vs p186 1:4 Desde la distancia, Judas vio cómo izaban el travesaño de la cruz con Jesús clavado en ella; ante esta visión, se dio prisa para volver al templo y, abriéndose paso a la fuerza frente al portero, se encontró a sí mismo en presencia del sanedrín, que aún seguía reunido. El traidor tenía la respiración entrecortada y estaba muy consternado, pero logró balbucear estas palabras: “He pecado entregando sangre inocente. Vosotros me habéis insultado. Me habéis ofrecido dinero como retribución por mi servicio ---lo que cuesta un esclavo---. Me arrepiento de lo que he hecho; aquí tenéis vuestro dinero. Quiero quitarme de encima la culpa de mi acto”.
\vs p186 1:5 Cuando los dirigentes de los judíos oyeron a Judas, se burlaron de él. Uno de ellos, que estaba sentado cerca de donde se hallaba Judas, le señaló con un gesto que saliera de la sala, diciéndole: “Tu Maestro ya ha sido castigado a muerte por los romanos y, en cuanto a tu culpa, ¿qué nos importa a nosotros? ¡Allá tú, y fuera de aquí!”.
\vs p186 1:6 Cuando abandonaba la cámara del sanedrín, Judas sacó las treinta piezas de plata de la bolsa y las arrojó sobre el suelo del templo, por donde se esparcieron. Cuando el traidor salió del templo, estaba casi fuera de sí. Judas estaba tomando conciencia y padeciendo los efectos de la verdadera naturaleza del pecado. El atractivo, la fascinación y la embriaguez de la mala acción habían desaparecido. Ahora, el malhechor estaba a solas, y cara a cara, ante el dictamen acusatorio de su alma desencantada y decepcionada. Al cometerse, el pecado resulta cautivador y trepidante, pero, luego, se debe hacer frente a la cosecha de los hechos, desnudos y nada románticos.
\vs p186 1:7 Este antiguo embajador del reino de los cielos en la tierra recorría ahora las calles de Jerusalén, repudiado y solo. Su desesperación era abrumadora y casi absoluta. Así estuvo deambulando por la ciudad y por fuera de sus muros hasta sentir la terrible soledad del valle de Hinom. Allí subió por las rocas escarpadas y, quitándose el cinto de su manto, ató un extremo a un pequeño árbol, se anudó el otro alrededor del cuello y se arrojó al precipicio. Antes de morir, el nudo, que había atado nerviosamente con sus manos, se soltó, y el cuerpo del traidor se destrozó al caer sobre los afilados riscos.
\usection{2. LA ACTITUD DEL MAESTRO}
\vs p186 2:1 Cuando fue arrestado, Jesús sabía que su labor en la tierra como hombre mortal estaba acabada. Era plenamente consciente del tipo de muerte que le esperaba, y no le preocupaban mucho los pormenores de aquellos pretendidos juicios.
\vs p186 2:2 Ante el tribunal de los sanedritas, Jesús declinó dar respuesta al testimonio de los testigos perjuros. Había solo una pregunta que siempre obtenía respuesta, ya fuera amigo o enemigo quien la formulara, y era la referida a la naturaleza y a la divinidad de su misión en la tierra. Cuando se le preguntaba si él era el Hijo de Dios, contestaba indefectiblemente. Con rotundidad, se negó hablar en presencia del curioso y perverso Herodes. Ante Pilato, habló únicamente cuando creyó que podría ayudarlo a él, o a alguna otra persona honesta, a tener un mejor conocimiento de la verdad a través de sus palabras. Jesús había enseñado a sus apóstoles la inutilidad de echar perlas delante de los cerdos, y ahora practicaba lo que había enseñado. Su forma de proceder en ese momento daba ejemplo del paciente autocontrol de la naturaleza humana junto al silencio majestuoso y a la solemne dignidad de la naturaleza divina. Estaba totalmente dispuesto a tratar con Pilato sobre cualquier cuestión relacionada con los cargos que se le imputaban, sobre cualquier pregunta que él reconociera como pertinente a la jurisdicción del gobernador.
\vs p186 2:3 Jesús estaba convencido de que cumplía la voluntad del Padre al someterse al curso natural y ordinario de los acontecimientos humanos, tal como debía hacerlo cualquier otra criatura mortal, y, por consiguiente, se negó a hacer uso incluso del poder, puramente humano, de su persuasiva elocuencia, para influenciar el resultado de las maquinaciones de sus semejantes mortales, socialmente cortos de mira y espiritualmente ciegos. Aunque Jesús vivió y murió en Urantia, toda su andadura humana, desde el principio hasta el fin, consistió en un plan formidable que tenía el objeto de instruir a todo el universo que él había creado e, incesantemente, sustentaba.
\vs p186 2:4 \pc Estos judíos estrechos de mente clamaron indecorosamente por la muerte del Maestro, mientras él estaba allí, de pie, en un silencio reverencial, contemplando aquella escena que significaba la muerte de una nación ---el propio pueblo de su padre terrenal---.
\vs p186 2:5 \pc Jesús había adquirido esa clase de carácter humano que le permitía conservar su serenidad y afirmar su dignidad incluso ante insultos continuados y arbitrarios. No se le podía intimidar. Cuando el sirviente de Anás primeramente lo golpeó, solo había sugerido la pertinencia de llamar a testigos que testificaran contra él sin falsedades.
\vs p186 2:6 Desde el principio hasta el fin de aquel supuesto juicio ante Pilato, las expectantes multitudes celestiales no pudieron evitar transmitir al universo la descripción de aquella escena como “Pilato siendo juzgado ante Jesús”.
\vs p186 2:7 Cuando se encontró frente a Caifás, y cuando todos los falsos testimonios se habían venido abajo, Jesús no vaciló en responder a la pregunta del sacerdote principal, aportando así, en su propio testimonio, la justificación que buscaban para condenarlo por blasfemia.
\vs p186 2:8 El Maestro nunca mostró el menor interés por los esfuerzos que hizo Pilato por liberarlo, bien intencionados pero a medias. En realidad, se apiadó de él y ciertamente trató de dar luz a su obnubilada mente. Se mantuvo totalmente impasible ante las apelaciones que el gobernador romano les hizo a los judíos para que retiraran sus cargos penales contra él. Durante toda esta penosa experiencia, se comportó de una manera dignamente sencilla y con una modesta majestuosidad. Ni siquiera quiso insinuar la falta de sinceridad de los que serían sus asesinos cuando le preguntaron si él era “el rey de los judíos”. Dando el mínimo de explicaciones, aceptó aquel calificativo sabiendo que, habiendo ellos ya decidido rechazarlo, él sería el último en aportarles un verdadero liderazgo nacional, ni incluso en un sentido espiritual.
\vs p186 2:9 Durante estos juicios, Jesús habló poco, aunque sí lo suficiente como para demostrar a todos los mortales de qué manera puede enaltecerse el carácter humano del hombre cuando está en alianza con Dios y para revelar a todo el universo la manera en la que Dios puede manifestarse en la vida de la criatura, cuando esta criatura verdaderamente opta por hacer la voluntad del Padre, convirtiéndose, pues, en un diligente hijo del Dios vivo.
\vs p186 2:10 Su amor por los mortales faltos de conocimiento se desvela perfectamente en su paciencia y en su gran serenidad frente a las burlas, los golpes y los continuos castigos de los zafios soldados y de los irreflexivos sirvientes. Ni siquiera se indignó cuando le vendaron los ojos y, ridiculizándolo, le pegaban en la cara, gritándole: “Profetízanos quién fue el que te golpeó”.
\vs p186 2:11 Pilato hablo con más verdad de la que sabía cuando, después de haber sido Jesús azotado, lo presentó ante la multitud, exclamando: “¡He aquí el hombre!”. En efecto, el amedrentado gobernador romano poco podía imaginarse que en aquel preciso instante todo el universo estaba atento, mirando fijamente la inaudita escena de su amado Soberano sometido de aquella manera a las humillantes burlas y golpes de sus propios súbditos mortales, en tinieblas y degradados. Y al hablar Pilato, resonó por todo Nebadón: “¡He aquí a Dios y al hombre!”. Por todos los lugares de un universo, indecibles millones de seres, desde ese día, han continuado contemplando a aquel hombre, mientras que el Dios de Havona, gobernante supremo del universo de los universos, acepta al hombre de Nazaret como quien ha colmado el ideal de las criaturas mortales de este universo local del tiempo y del espacio. En su inigualable vida, nunca dejó de revelar Dios al hombre. Ahora, en estos episodios finales de su andadura mortal y su consiguiente muerte, realizaba una nueva y conmovedora revelación del hombre a Dios.
\usection{3. DAVID ZEBEDEO, UN HOMBRE DIGNO DE CONFIANZA}
\vs p186 3:1 Poco después de que Jesús fuese entregado a los soldados romanos, tras concluir la audiencia ante Pilato, un destacamento de guardias del templo se encaminó a toda prisa a Getsemaní para dispersar o arrestar a los seguidores del Maestro. Pero mucho antes de que llegaran, estos seguidores ya se habían diseminado. Los apóstoles se habían retirado a lugares ya previstos para ocultarse; los griegos se habían separado y habían partido hacia distintas casas de Jerusalén; los demás discípulos habían asimismo desaparecido. Pensando que los enemigos de Jesús regresarían, David Zebedeo quitó enseguida unas cinco o seis tiendas y las llevó a la quebrada de la colina, donde el Maestro se retiraba a menudo para orar y realizar su culto de adoración al Padre. Pensaba ocultarse allí y, al mismo tiempo, mantener un centro, o estación de coordinación, para su servicio de mensajería. Apenas se había marchado David del campamento, llegaron los guardias del templo. Como no encontraron a nadie allí, se contentaron con incendiar el campamento, para luego regresar rápidamente al templo. Cuando oyó el informe de los soldados, el sanedrín quedó satisfecho al enterarse de que los seguidores de Jesús estaban tan completamente asustados y subyugados que ya no habría peligro de revueltas ni se produciría ningún intento por rescatar a Jesús de las manos de sus ejecutores. Pudieron finalmente respirar tranquilos y, así, dieron por terminada la reunión, marchándose cada cual por su lado para prepararse para la Pascua.
\vs p186 3:2 Tan pronto como Pilato entregó a Jesús a los soldados romanos para que fuera crucificado, un mensajero se apresuró a Getsemaní para informar a David y, en cinco minutos, ya estaban los corredores de camino a Betsaida, Pella, Filadelfia, Sidón, Siquem, Hebrón, Damasco y Alejandría. Y estos mensajeros portaban la noticia de que Jesús estaba a punto de ser crucificado por los romanos, debido a las insistentes exigencias de los dirigentes de los judíos.
\vs p186 3:3 Durante todo aquel trágico día, y hasta que se remitió finalmente el mensaje de que el cuerpo del Maestro estaba depositado en la tumba, David enviaba a sus mensajeros prácticamente cada media hora para mantener informados a los apóstoles, a los griegos y a la familia terrenal de Jesús, congregada en Betania, en la casa de Lázaro. Cuando los mensajeros partieron con la noticia de que Jesús había sido sepultado, David despidió a su grupo de corredores locales con motivo de la celebración de la Pascua y del \bibemph{sabbat,} el día de reposo, dándole instrucciones de que se encontraran con él de manera discreta el domingo por la mañana, en la casa de Nicodemo, donde planeaba esconderse con Andrés y Simón Pedro por unos días.
\vs p186 3:4 David Zebedeo, de una atípica mentalidad, fue el único de los más destacados discípulos de Jesús que interpretaba de forma literal el hecho manifiesto, afirmado por el Maestro, de que moriría y “resucitaría al tercer día”. Cierta vez, David le había oído hacer esta predicción y, siendo tan literal en su forma de pensar, se propuso entonces reunir a sus mensajeros el domingo por la mañana a horas tempranas, en la casa de Nicodemo, para que estuvieran disponibles y divulgar la noticia, en caso de que Jesús resucitara de entre los muertos. David descubrió tempranamente que ninguno de los seguidores de Jesús esperaba que Jesús volviera tan pronto del sepulcro; por ello, no comentó mucho sobre su idea al respecto y nada sobre la movilización de todos los efectivos de sus mensajeros para aquel domingo por la mañana temprano, salvo a los corredores enviados a ciudades distantes y a núcleos de creyentes durante la mañana del viernes.
\vs p186 3:5 Y, así pues, estos seguidores de Jesús, esparcidos por todo Jerusalén y su entorno, compartieron la Pascua esa noche y, al día siguiente permanecieron recluidos.
\usection{4. EN PREPARACIÓN A LA CRUCIFIXIÓN}
\vs p186 4:1 Una vez que Pilato se lavó las manos ante la multitud, tratando así de exculparse por haber sentenciado a un hombre inocente a morir en la cruz por la simple razón de temer contrariar el clamor de los dirigentes de los judíos, ordenó que entregaran al Maestro a los soldados romanos y dio instrucciones al capitán para que lo crucificaran de inmediato. Cuando los soldados se hicieron cargo de Jesús, lo llevaron de vuelta al patio del pretorio y, tras despojarlo del manto que le había puesto Herodes, lo vistieron con sus propias vestimentas. Estos soldados se burlaron de él y lo insultaron, pero no le infligieron ningún otro castigo físico. Jesús estaba ahora a solas con estos soldados romanos. Sus amigos estaban escondidos; sus enemigos se habían marchado por su camino; ni incluso Juan Zebedeo estaba ya a su lado.
\vs p186 4:2 Eran poco después de las ocho cuando Pilato entregó a Jesús a los soldados y, algo antes de las nueve, se dirigieron con él al sitio de la crucifixión. Durante este espacio de tiempo de más de media hora, Jesús no pronunció una sola palabra. La actividad gobernativa de todo un gran universo estaba prácticamente paralizada. Gabriel y los principales gobernantes de Nebadón se hallaban congregados, o bien aquí en Urantia, o bien prestaban suma atención a los informes espaciales de los arcángeles, con el fin de mantenerse al corriente de lo que le ocurría al Hijo del Hombre en Urantia.
\vs p186 4:3 Para el momento en el que se disponían a partir con Jesús al Gólgota, los soldados ya habían empezado a sentirse impresionados por su insólita calma y extraordinaria dignidad, por su resignado silencio.
\vs p186 4:4 Gran parte del retraso respecto a la salida con Jesús al lugar de la crucifixión se debió a que el capitán había decidido, en el último minuto, llevarse con él a dos ladrones, que estaban condenados a muerte; dado que Jesús iba a ser crucificado aquella mañana, el capitán romano pensó que sería mejor que ambos murieran con él en lugar de esperar hasta el fin de las celebraciones de la Pascua.
\vs p186 4:5 En el momento en el que los ladrones pudieron estar listos, se los condujo al patio. Allí contemplaron a Jesús, uno de ellos lo hacía por primera vez, pero el otro lo había oído hablar con frecuencia tanto en el templo como, muchos meses antes, en el campamento de Pella.
\usection{5. LA MUERTE DE JESÚS EN RELACIÓN CON LA PASCUA}
\vs p186 5:1 No hay ninguna relación directa entre la muerte de Jesús y la Pascua judía. Ciertamente, el Maestro abandonó su vida en la carne ese día, el día de la preparación para la Pascua judía y en torno a la hora en la que se sacrificaban en el templo los corderos pascuales. Pero esta coincidencia de hechos no indica de manera alguna que la muerte del Hijo del Hombre en la tierra tenga vinculación alguna con el sistema sacrificial judío. Jesús era judío, pero como Hijo del Hombre era un mortal más del mundo. Los acontecimientos ya narrados y que condujeron a esta hora de la inmediata crucifixión del Maestro indican suficientemente que su muerte en este momento fue sencillamente algo natural y organizado por los hombres.
\vs p186 5:2 Fue el hombre y no Dios quien planeó y acometió la muerte de Jesús en la cruz. Es verdad que el Padre no quiso interferir en el desarrollo de los acontecimientos humanos en Urantia, pero el Padre del Paraíso no decretó, impuso ni exigió la muerte de su Hijo tal como se llevó a cabo en la tierra. Es un hecho que Jesús, de alguna manera, ya fuese más tarde o más temprano, habría tenido que despojarse de su cuerpo mortal, poner fin a su encarnación, pero podría haberlo hecho de innumerables maneras, sin morir en una cruz entre dos ladrones. Todo esto fue obra del hombre, no de Dios.
\vs p186 5:3 Ya en el instante de su bautismo, el Maestro había llevado a término todo el proceso requerido en cuanto a su experiencia en la tierra y en la carne para culminar su séptimo y último ministerio de gracia en el universo. En este mismo momento, Jesús dio cumplimiento a su deber en la tierra. Toda la vida que vivió después, e incluso la forma en la que murió, constituyó su ministerio, puramente personal, para el bien y la elevación de las criaturas mortales de este mundo y de otros mundos.
\vs p186 5:4 El evangelio de la buena nueva en la que el hombre mortal puede, mediante la fe, llegar a ser consciente espiritualmente de que él es un hijo de Dios, no depende de la muerte de Jesús. Es un hecho, en verdad, que todo este evangelio del reino se ha visto sumamente iluminado por la muerte del Maestro, pero, mucho más, por su propia vida.
\vs p186 5:5 Todo lo que el Hijo del Hombre dijo o hizo en la tierra embelleció notablemente las doctrinas de la filiación con Dios y de la hermandad de los hombres, pero esta relación fundamental entre Dios y los hombres es intrínseca al hecho universal del amor de Dios por sus criaturas y a la misericordia innata de sus hijos divinos. Esta relación, enternecedora y divinamente hermosa, entre el hombre y su Hacedor, en este mundo y en todos los demás a lo largo de todo el universo de los universos, ha existido desde la eternidad; y no está condicionada, en ningún sentido, por estos periódicos ministerios de gracia de los hijos creadores de Dios, que adoptan, de esta manera, la naturaleza y semejanza de las inteligencias por ellos creadas, y que son parte de las condiciones que han de satisfacer para lograr finalmente la soberanía ilimitada sobre sus universos locales respectivos.
\vs p186 5:6 El Padre de los cielos amaba al hombre mortal de la tierra de igual manera antes de la vida y la muerte de Jesús en Urantia e igualmente después de esta transcendente demostración de la coligación del hombre con Dios. Esta poderosa misión de la encarnación del Dios de Nebadón como hombre en Urantia no podía acrecentar los atributos del Padre eterno, infinito y universal, pero sí enriqueció e iluminó a todos los otros regidores y criaturas del universo de Nebadón. Aunque el Padre de los cielos no nos ama más en virtud de este ministerio de gracia de Miguel, todas las demás inteligencias celestiales sí lo hacen. Y esto ocurre porque Jesús no solo efectuó una revelación de Dios al hombre, sino que, de la misma manera, efectuó una nueva revelación del hombre a los Dioses y a las inteligencias celestiales del universo de los universos.
\vs p186 5:7 Jesús no va a morir como un acto de sacrificio por el pecado ni va a redimir ninguna culpa moral innata a la raza humana. La humanidad no tiene esta ancestral culpa ante Dios. La culpa es, sencillamente, cuestión del pecado personal y de la rebeldía, deliberada y con conocimiento de causa, contra la voluntad del Padre y el gobierno de sus hijos creadores.
\vs p186 5:8 El pecado y la rebelión no guardan relación con el plan ordinario de ministerio de gracia de los Hijos de Dios del Paraíso, aunque nos parezca que el plan de salvación sea un rasgo más del plan de gracia.
\vs p186 5:9 La salvación que Dios ofrece a los mortales de Urantia habría sido igual de efectiva e inequívocamente cierta incluso si las crueles manos de mortales ignorantes no hubieran dado muerte a Jesús. Si los mortales de la tierra hubieran acogido favorablemente al Maestro y él hubiera partido de Urantia mediante la renuncia voluntaria de su vida en la carne, el hecho del amor de Dios y de la misericordia del Hijo ---el hecho de la filiación con Dios--- no se habría visto condicionado de ninguna manera. Vosotros los mortales sois hijos de Dios, y solo se precisa una cosa para que esta verdad se convierta en un hecho en vuestra experiencia personal, y se trata de vuestra fe nacida del espíritu.
