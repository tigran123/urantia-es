\upaper{18}{Los seres personales supremos de la Trinidad}
\author{Consejero divino}
\vs p018 0:1 Todos los seres personales supremos de la Trinidad se crean con un propósito específico de servicio. La Trinidad Divina los concibe para que cumplan ciertos deberes específicos. Están capacitados para servir con devoción completa y siguiendo un procedimiento perfecto. Existen siete órdenes de seres personales supremos de la Trinidad:
\vs p018 0:2 \li{1.}Los seres secretos trinitizados de supremacía.
\vs p018 0:3 \li{2.}Los eternos de días.
\vs p018 0:4 \li{3.}Los ancianos de días.
\vs p018 0:5 \li{4.}Los perfectos de días.
\vs p018 0:6 \li{5.}Los recientes de días.
\vs p018 0:7 \li{6.}Los uniones de días.
\vs p018 0:8 \li{7.}Los fieles de días.
\vs p018 0:9 \pc Estos seres, cuyo número está cerrado de forma definitiva, son administradores perfectos. Su creación es un acontecimiento pasado; no se hacen personales a ninguno más de ellos.
\vs p018 0:10 En todo el gran universo, estos seres personales supremos de la Trinidad representan los procedimientos administrativos de la Trinidad del Paraíso; representan la justicia y \bibemph{son} el juicio ejecutivo de la Trinidad del Paraíso. Forman una perfecta línea de administración interrelacionada, que se extiende desde las esferas del Padre en el Paraíso hasta los mundos donde se localizan las sedes centrales de los universos locales, incluyendo las capitales de las constelaciones que los componen.
\vs p018 0:11 Todos los seres de origen en la Trinidad se crean en todos sus atributos divinos con la perfección del Paraíso. Su preparación para el servicio cósmico solamente se ha visto aumentada con el paso del tiempo en el terreno de la experiencia. No existe peligro alguno de incumplimiento ni riesgo de rebelión con los seres de origen en la Trinidad. Son de esencia divina, y nunca se ha sabido que se hayan apartado de la senda divina y perfecta de la conducta que les es personalmente propia.
\usection{1. LOS SERES SECRETOS TRINITIZADOS DE SUPREMACÍA}
\vs p018 1:1 Hay siete mundos en la vía circulatoria más interior de los satélites del Paraíso, y cada uno de estos mundos excelsos está presidido por un colectivo de diez seres secretos trinitizados de supremacía. No son creadores sino administradores inigualables y supremos. La gestión de los asuntos relativos a estas siete esferas fraternales está totalmente encomendada a este colectivo de setenta directores supremos. Aunque estos vástagos de la Trinidad supervisan estas siete esferas sagradas tan próximas al Paraíso, a este grupo de mundos se le conoce de forma universal como la vía personal del Padre Universal.
\vs p018 1:2 Los seres secretos trinitizados de supremacía, en grupos de diez, obran como directores conjuntos, en igualdad de rango, de sus respectivas esferas, pero también tienen, individualmente, bajo su responsabilidad áreas específicas. La tarea a realizar en cada uno de estos singulares mundos está dividida en siete secciones principales. Cada uno de estos gobernantes coiguales preside cada uno de estos sectores de actividad especial. Los tres restantes sirven de representantes personales de la Deidad trina en relación con los otros siete, uno representando al Padre, otro al Hijo y, el último, al Espíritu.
\vs p018 1:3 Aunque haya un manifiesto parecido entre los seres secretos trinitizados de supremacía propio de la clase a la que pertenecen, estos seres desvelan a su vez siete características diferentes propias del grupo al que pertenecen. Los diez directores supremos encargados de los asuntos relativos a Lugar de la Divinidad reflejan el carácter y la naturaleza personales del Padre Universal; y así ocurre con cada una de estas siete esferas: cada grupo de diez se asemeja a esa Deidad o conjunción de la Deidad, característica de su entorno. Los diez directores que gobiernan Lugar de la Ascensión reflejan la naturaleza combinada del Padre, del Hijo y del Espíritu.
\vs p018 1:4 \pc Poco puedo revelar acerca de la labor de estos elevados seres personales de los siete mundos sagrados del Padre, porque en verdad son \bibemph{seres secretos} de supremacía. No existen secretos que estén de manera arbitraria relacionados con el acercamiento al Padre Universal, al Hijo Eterno o al Espíritu Infinito. Las Deidades son como un libro abierto para todos los que alcanzan la perfección divina, pero jamás se puede acceder por completo a todos los seres secretos de supremacía. Siempre seremos incapaces de penetrar totalmente en los dominios que contienen los secretos personales que tienen lugar entre la conjunción de las personas de la Deidad con los grupos séptuplos de seres creados.
\vs p018 1:5 Puesto que la labor que realizan estos directores supremos guarda relación con el contacto estrecho y personal de las Deidades con estas siete fundamentales agrupaciones de seres del universo, cuando residen en estos siete mundos especiales o mientras actúan en todas partes del gran universo, es propio que estas relaciones tan personales y estos extraordinarios contactos se mantengan en un secreto sagrado. Los creadores del Paraíso respetan la privacidad y la santidad del ser personal incluso en sus criaturas más modestas. Esto es verdad tanto de manera individual como en cuanto a los diversos y diferenciados órdenes de seres personales.
\vs p018 1:6 Incluso para los seres que han logrado altos niveles de realización en el universo, estos mundos secretos representarán para siempre una prueba de lealtad. Se nos permite conocer plena y personalmente a los Dioses eternos, conocer libremente su carácter divino y perfecto, pero no se nos ha dado poder comprender por completo todas las relaciones personales de los soberanos del Paraíso con sus criaturas.
\usection{2. LOS ETERNOS DE DÍAS}
\vs p018 2:1 Un ser personal supremo de la Trinidad dirige cada uno de los mil millones de mundos de Havona. Estos gobernantes se conocen como los eternos de días, y su número es exactamente mil millones, uno por cada una de las esferas de Havona. Son vástagos de la Trinidad del Paraíso, pero, al igual que los seres secretos de supremacía, no existen datos sobre su origen. Estos dos grupos de padres omnisapientes han gobernado por siempre los excelentes mundos del sistema Paraíso\hyp{}Havona, y obran sin rotación ni reasignación.
\vs p018 2:2 Los eternos de días son visibles para todas las criaturas de voluntad que habitan en sus dominios. Presiden los cónclaves planetarios que se convocan con regularidad. De forma periódica, y rotativa, visitan las esferas donde se localizan las sedes centrales de los siete suprauniversos. Son parientes cercanos, e iguales divinos, de los ancianos de días, que presiden los destinos de los siete gobiernos de los suprauniversos. Cuando un eterno de días está ausente de su esfera, un hijo magistrado de la Trinidad dirige su mundo.
\vs p018 2:3 Exceptuando los órdenes de vida establecidos tales como los nativos de Havona y otras criaturas vivas del universo central, los eternos de días que allí residen han desarrollado sus respectivas esferas enteramente de acuerdo con sus propias ideas e ideales personales. Hacen visitas a los planetas de los demás, pero no se copian ni imitan; siempre son completamente originales.
\vs p018 2:4 La arquitectura, el embellecimiento natural, las construcciones morontiales y las creaciones espirituales son en cada esfera exclusivas y singulares. Cada mundo es un lugar de belleza perpetua y completamente diferente de cualquier otro mundo en el universo central. Cada uno de vosotros pasará un tiempo más largo o más corto en cada una de estas inigualables y emocionantes esferas, a medida que os adentréis en el Paraíso a través de Havona. Es natural en vuestro mundo hablar del Paraíso como de lo que está \bibemph{arriba,} pero sería más correcto referirse a la meta divina de ascensión como \bibemph{hacia dentro}.
\usection{3. LOS ANCIANOS DE DÍAS}
\vs p018 3:1 Cuando los mortales del tiempo finalizan su preparación en los mundos de formación que rodean la sede central de su universo local y se les promueve a las esferas de instrucción de su suprauniverso, han progresado en su desarrollo espiritual hasta el punto que pueden reconocer y comunicarse con los elevados gobernantes y directores espirituales de estas adelantadas regiones espaciales, incluyendo a los ancianos de días.
\vs p018 3:2 Todos los ancianos de días son esencialmente idénticos; revelan el carácter combinado y la naturaleza unificada de la Trinidad. Poseen individualidad y son distintos en cuanto a su ser personal, pero no difieren uno del otro como los siete espíritus mayores. Proporcionan una dirección uniforme a los siete suprauniversos que de alguna manera son diferentes; cada cual es una creación distinta de la otra, demarcada y única. Los siete espíritus mayores son distintos en su naturaleza y atributos, pero los ancianos de días, los gobernantes personales de los suprauniversos, son todos vástagos homogéneos y supraperfectos de la Trinidad del Paraíso.
\vs p018 3:3 Los elevados siete espíritus mayores determinan la \bibemph{naturaleza} de sus respectivos suprauniversos, pero los ancianos de días dan instrucciones sobre la \bibemph{administración} de estos mismos suprauniversos. Superponen la uniformidad administrativa a la diversidad creativa y aseguran la armonía del todo frente a las subyacentes diferencias de creación de los siete agrupamientos segmentados del gran universo.
\vs p018 3:4 \pc Todos los ancianos de días fueron trinitizados al mismo tiempo. Constituyen el inicio de los datos existentes sobre el ser personal del universo de los universos, de aquí su nombre: \bibemph{ancianos} de días. Cuando lleguéis al Paraíso y busquéis registros escritos sobre el comienzo de las cosas, descubriréis que la primera anotación que aparece en la sección destinada al ser personal es el relato de la trinitización de estos veintiún ancianos de días.
\vs p018 3:5 \pc Estos elevados seres gobiernan siempre en grupos de tres. Existen áreas de actividad en la que obran de forma individual; hay otras en las que dos cualesquiera pueden obrar; pero en los ámbitos superiores de su actividad gobernativa deben actuar de forma conjunta. Nunca abandonan personalmente sus mundos de residencia, aunque no necesitan hacerlo, porque estos mundos son los puntos de convergencia en el suprauniverso del extenso sistema de la reflectividad.
\vs p018 3:6 Las moradas personales de cada trío de ancianos de días están situadas en el punto de polaridad espiritual de la esfera donde se localiza su sede central. Esta esfera se divide en setenta sectores administrativos, cada cual con una capital en la que los ancianos de días residen en ocasiones.
\vs p018 3:7 En poder, magnitud de autoridad y grado de jurisdicción, los ancianos de días son los más poderosos y magníficos de entre los gobernantes directos de las creaciones espacio\hyp{}temporales. En todo el inmenso universo de los universos solamente ellos están investidos de los altos poderes de juicio potestativo final respecto a la extinción eterna de las criaturas de voluntad. Y los tres ancianos de días deben participar en los decretos finales del tribunal supremo de un suprauniverso.
\vs p018 3:8 \pc Aparte de las Deidades y sus colaboradores del Paraíso, los ancianos de días son los gobernantes más perfectos, versátiles y divinamente dotados que existen en el espacio\hyp{}tiempo. Aparentemente, son los gobernantes supremos de los suprauniversos; pero este derecho a gobernar no lo han ganado experiencialmente, así pues, están destinados en un futuro a ser reemplazados por el Ser Supremo, un soberano de índole experiencial, de quien sin duda serán sus vicerregentes.
\vs p018 3:9 El Ser Supremo está en vías de alcanzar la soberanía de los siete suprauniversos mediante su servicio de orden experiencial, del mismo modo que un hijo creador obtiene experiencialmente la soberanía de su universo local. Pero durante la presente era en la que la evolución del Supremo está inconclusa, los ancianos de días, en coordinación y de manera perfecta, rigen plenamente los universos en evolución del tiempo y del espacio. En todos sus decretos y dictámenes, los ancianos de días se caracterizan por su primordial sabiduría y sus iniciativas individuales.
\usection{4. LOS PERFECTOS DE DÍAS}
\vs p018 4:1 Hay exactamente doscientos diez perfectos de días; estos presiden los gobiernos de los diez sectores mayores de cada suprauniverso. Fueron trinitizados para la tarea especial de asistir a los directores del suprauniverso, y gobiernan como vicerregentes directos y personales de los ancianos de días.
\vs p018 4:2 Tres perfectos de días se asignan a cada capital de sector mayor, pero, a diferencia de los ancianos de días, no es necesario que los tres estén en todo momento presentes. Periódicamente, uno de los tres puede ausentarse para conferenciar en persona con los ancianos de días sobre el bienestar de las regiones espaciales a su cargo.
\vs p018 4:3 \pc Estos gobernantes trinos de los sectores mayores son peculiarmente perfectos en su maestría de los detalles administrativos, de ahí su nombre: \bibemph{perfectos} de días. Al citar los nombres de estos seres del mundo espiritual, nos enfrentamos con el problema de traducirlos a vuestro idioma, y muy a menudo es una ardua tarea realizar una traducción suficientemente satisfactoria. Nos disgusta usar denominaciones arbitrarias que tendrían para vosotros poco significado; y, por lo tanto, nos resulta con frecuencia difícil elegir el nombre apropiado, un nombre que os resulte claro y que, al mismo tiempo, de algún modo, designe al original.
\vs p018 4:4 \pc Los perfectos de días tienen adscritos a su gobierno a un colectivo con un moderado número de consejeros divinos, perfeccionadores de la sabiduría y censores universales. Tienen incluso un número mayor de mensajeros poderosos, de aquellos elevados en autoridad y de los sin nombre ni número. Aunque gran parte del trabajo rutinario de los asuntos relativos al sector mayor lo llevan a cabo los guardianes celestiales y los asistentes de los hijos elevados. Estos dos grupos se eligen entre los vástagos trinitizados por seres personales del Paraíso\hyp{}Havona o por finalizadores mortales glorificados. Las Deidades del Paraíso vuelven a trinitizar a alguno de estos dos órdenes de seres trinitizados por criaturas para posteriormente enviarlos a prestar su ayuda en la administración del suprauniverso.
\vs p018 4:5 La mayoría de los guardianes celestiales y de los asistentes de los hijos elevados se asigna al servicio de los sectores mayores y menores, pero los custodios trinitizados (serafines y seres intermedios acogidos por la Trinidad) son los funcionarios del poder judicial de los tres sectores, que ejercen su labor en los tribunales de los ancianos de días, de los perfectos de días y de los recientes de días. Los embajadores trinitizados (mortales ascendentes acogidos por la Trinidad del tipo que se fusiona con el Hijo o con el Espíritu) pueden encontrarse en cualquier parte del suprauniverso, aunque la mayoría presta sus servicios en los sectores menores.
\vs p018 4:6 Antes de los tiempos del pleno desarrollo del plan de administración de los siete suprauniversos, casi todos los administradores de los distintos sectores de estos gobiernos, exceptuando a los ancianos de días, pasaron períodos de formación de distinta duración, bajo la dirección de los eternos de días, en los distintos mundos del perfecto universo de Havona. Los seres que luego fueron trinitizados pasaron, asimismo, por una temporada de formación bajo los eternos de días antes de que se les adscribiera al servicio de los ancianos de días, de los perfectos de días y de los recientes de días. Todos ellos son administradores avezados, probados y experimentados.
\vs p018 4:7 \pc Veréis con prontitud a los perfectos de días cuando avancéis hasta la sede central de gobierno de Esplandón después de vuestra permanencia en los mundos de vuestro sector menor, ya que estos gobernantes excelsos están estrechamente vinculados a los setenta mundos de los sectores mayores para la formación superior de las criaturas que ascienden del tiempo y del espacio. Los perfectos de días, en persona, toman el juramento al grupo de ascendentes que se ha graduado en las facultades de los sectores mayores.
\vs p018 4:8 La labor de los peregrinos del tiempo en los mundos que rodean la sede central de un sector mayor es principalmente de naturaleza intelectual, a diferencia del carácter más físico y material de la formación en las siete esferas educativas de un sector menor y de la actividad espiritual de los cuatrocientos noventa mundos universitarios de la sede central del suprauniverso.
\vs p018 4:9 Aunque únicamente quedáis registrados en Esplandón, vuestro sector mayor, que abarca el universo local en donde tenéis vuestro origen, tendréis que pasar por cada uno de los sectores mayores que rodean nuestro suprauniverso. Veréis a cada uno de los treinta perfectos de días de Orvontón antes de que lleguéis a Uversa.
\usection{5. LOS RECIENTES DE DÍAS}
\vs p018 5:1 Los recientes de días son los más jóvenes de los directores supremos de los suprauniversos; presiden en grupos de tres los asuntos relativos a los sectores menores. En naturaleza, son coiguales con los perfectos de días, pero de menor rango en cuanto a su autoridad administrativa. Hay exactamente veintiún mil de estos seres personales de la Trinidad; son seres personalmente gloriosos y eficaces de manera divina. Se crearon simultáneamente y se formaron juntos en Havona bajo los eternos de días.
\vs p018 5:2 Los recientes de días tienen un colectivo de colaboradores y asistentes semejante al de los perfectos de días. Además tienen asignado a ellos un número enorme de distintos órdenes de menor rango de seres celestiales. En la administración de los sectores menores incorporan a un gran número de mortales ascendentes que residen allí, el personal de las distintas colonias de cortesía y los distintos grupos con origen en el Espíritu Infinito.
\vs p018 5:3 Los gobiernos de los sectores menores se ocupan en gran parte, aunque no exclusivamente, de los grandes problemas físicos de los suprauniversos. Las esferas del sector menor son las sedes centrales de los controladores físicos mayores. En estos mundos, los mortales ascendentes prosiguen sus estudios y experimentos que guardan relación con la observación de la actividad del tercer orden de los centros supremos de la potencia y de los siete órdenes de los controladores físicos mayores.
\vs p018 5:4 Puesto que el régimen de un sector menor está dedicado mayoritariamente a los problemas físicos, sus tres recientes de días rara vez están juntos en la esfera sede de la capital. La mayor parte del tiempo, alguno de ellos se encuentra de viaje, ya sea en reunión con los perfectos de días que supervisan el sector mayor, o, ausente, representando a los ancianos de días en los cónclaves de los seres elevados de origen en la Trinidad que tienen lugar en el Paraíso. Estos se turnan con los perfectos de días en la representación de los ancianos de días ante los consejos supremos del Paraíso. Mientras tanto, otro de estos recientes de días puede estar en viaje de inspección de los mundos donde tienen sus sedes centrales los universos locales que pertenece a su jurisdicción. Pero al menos uno de estos gobernantes permanece siempre de servicio en la sede central del sector menor.
\vs p018 5:5 Todos conoceréis alguna vez a los tres recientes de días que están a cargo de Ensa, vuestro sector menor, puesto que debéis pasar por ellos cuando os adentréis en los mundos de formación de los sectores mayores. Al ascender a Uversa, solamente atravesaréis un grupo de esferas de formación del sector menor.
\usection{6. LOS UNIONES DE DÍAS}
\vs p018 6:1 Los seres personales de la Trinidad del orden de “días” no tienen funciones administrativas por debajo del nivel de los gobiernos del suprauniverso. En los universos locales en evolución actúan solo de consejeros y asesores. Los uniones de días son un grupo de seres personales de enlace acreditados por la Trinidad del Paraíso ante los dos gobernantes de los universos locales. Cada universo local organizado y habitado tiene asignado uno de estos asesores del Paraíso, el cual sirve de representante de la Trinidad y, en algunos respectos, del Padre Universal ante la creación local.
\vs p018 6:2 Existen setecientos mil de estos seres, aunque a todos no se les ha nombrado de manera oficial. El colectivo de reserva de los uniones de días actúa en el Paraíso como Consejo Supremo Regulador del Universo.
\vs p018 6:3 De una manera especial, estos observadores de la Trinidad coordinan la actividad administrativa de todos los poderes del gobierno universal, desde los de los universos locales hasta los de los suprauniversos, pasando por los gobiernos del sector, de aquí su nombre: \bibemph{uniones} de días. Presentan un informe por triplicado a sus superiores: informan a los recientes de días de datos relevantes de naturaleza física y semiintelectual de su sector menor, a los perfectos de días de los acontecimientos intelectuales y cuasi espirituales de su sector mayor y a los ancianos de días, emplazados en la capital de su suprauniverso, de los asuntos espirituales y semiparadisíacos de dicho suprauniverso.
\vs p018 6:4 Puesto que son seres de origen en la Trinidad, todas las vías circulatorias del Paraíso están disponibles para su intercomunicación y, por tanto, están siempre en contacto entre ellos mismos y con todos los demás seres personales que necesiten hasta llegar a los consejos supremos del Paraíso.
\vs p018 6:5 \pc Los uniones de días no tienen una relación orgánica con el gobierno del universo local al que están asignados. Aparte de sus deberes como observadores, actúan solo por solicitud de las autoridades locales. Son miembros oficiales de todos los consejos principales y de todos los cónclaves importantes de la creación local, pero no participan en las consideraciones técnicas de los problemas de tipo administrativo.
\vs p018 6:6 Cuando un universo local se asienta en luz y vida, sus seres glorificados se vinculan libremente al unión de días, el cual adquiere entonces una mayor competencia en esa área de perfección evolutiva. Pero sigue siendo esencialmente un embajador de la Trinidad y un consejero del Paraíso.
\vs p018 6:7 Un hijo divino de doble origen en la Deidad gobierna de forma directa el universo local, pero tiene constantemente a su lado a un hermano del Paraíso, a un ser personal de origen en la Trinidad. En caso de que un hijo creador se ausente de forma temporal de la sede central de su universo local, los gobernantes suplentes, a la hora de tomar sus decisiones más importantes, siguen, en buena parte, las recomendaciones de su unión de días.
\usection{7. LOS FIELES DE DÍAS}
\vs p018 7:1 Estos elevados seres personales de origen en la Trinidad son los asesores del Paraíso de los gobernantes de las cien constelaciones de cada universo local. Hay setenta millones de fieles de días y, al igual que los uniones de días, no todos están de servicio. Su colectivo de reserva en el Paraíso constituye la comisión asesora de ética interuniversal y autogobierno. Los fieles de días tienen un servicio de rotación de acuerdo con las decisiones del consejo supremo de su colectivo de reserva.
\vs p018 7:2 Todo lo que un unión de días es para un hijo creador de un universo local, lo son los fieles de días para los hijos vorondadecs que gobiernan las constelaciones de esa creación local. Son de una suprema dedicación, además de ser divinamente fieles al bienestar de las constelaciones que se les ha asignado, de aquí su nombre: \bibemph{fieles} de días. Actúan tan solo de asesores; no participan nunca en actividades de tipo administrativo excepto por invitación de las autoridades de la constelación. Tampoco se ocupan de forma directa de la instrucción de los peregrinos en ascenso de las esferas arquitectónicas de formación que rodean la sede central de una constelación. Todas esas tareas están bajo la supervisión de los hijos vorondadecs.
\vs p018 7:3 Todos los fieles de días que actúan en las constelaciones de un universo local están bajo la jurisdicción del unión de días ante quien responden directamente. No cuentan con un extenso sistema de comunicación ya que de ordinario se limitan a interrelacionarse entre ellos mismos dentro de los límites del universo local. Cualquier fiel de días que se encuentre de servicio en Nebadón puede comunicarse con todos los demás de su orden que se hallen de servicio en este universo local, como de hecho hace.
\vs p018 7:4 Al igual que el unión de días en una sede central del universo, los fieles de días mantienen sus residencias personales en las capitales de la constelación, separadas de las de los directores administrativos de dichas zonas. Sus moradas son, en efecto, modestas si se las compara con las de los gobernantes vorondadecs de las constelaciones.
\vs p018 7:5 Los fieles de días constituyen el último eslabón en la larga cadena de administración y asesoramiento que se extiende desde las esferas sagradas del Padre Universal, cerca del centro de todas las cosas, hasta las divisiones principales de los universos locales. El régimen de origen en la Trinidad termina con las constelaciones; no hay ningún consejero del Paraíso permanentemente emplazado en los sistemas que las componen ni en los mundos habitados. Estas últimas unidades administrativas están por completo bajo la jurisdicción de los seres nativos de los universos locales.
\vsetoff
\vs p018 7:6 [Exposición de un consejero divino de Uversa.]
