\upaper{15}{Los siete suprauniversos}
\author{Censor universal}
\vs p015 0:1 En lo que concierne al Padre Universal ---como Padre--- los universos son prácticamente inexistentes; él trata con seres personales; él es el Padre de los seres personales. En lo que concierne al Hijo Eterno y al Espíritu Infinito ---como compañeros creadores--- los universos tienen ubicación y difieren, y están bajo el gobierno conjunto de los hijos creadores y de los espíritus creativos. En lo que concierne a la Trinidad del Paraíso, fuera de Havona solo existen siete universos habitados, los siete suprauniversos que mantienen jurisdicción sobre el círculo del primer nivel espacial posterior a Havona. Los siete espíritus mayores irradian su influencia desde la Isla central hacia fuera, formando así pues esta inmensa creación, una rueda gigantesca con su núcleo central en la Isla eterna del Paraíso; sus siete rayos son las irradiaciones de los siete espíritus mayores; y su perímetro, las regiones exteriores del gran universo.
\vs p015 0:2 Al iniciarse la materialización de la creación universal se designó el esquema séptuplo en la organización y el gobierno del suprauniverso. La primera creación posterior a Havona se dividió en siete formidables segmentos, y se diseñaron y construyeron los mundos que se convertirían en las sedes centrales de los gobiernos de los suprauniversos. El actual esquema de administración ha existido desde casi la eternidad, y los gobernantes de estos siete suprauniversos se llaman, con toda razón, ancianos de días.
\vs p015 0:3 De la inmensa cantidad de conocimiento existente en relación a los suprauniversos, poco puedo deciros, pero hay, por todas estas regiones, un modo de proceder que indica la existencia de una dirección inteligente tanto de las fuerzas físicas como de las espirituales, y la presencia de la gravedad universal obra ahí con poder majestuoso y armonía perfecta. Es importante que os forméis primero una idea adecuada de la constitución física y de la organización material del suprauniverso, porque así estaréis mejor preparados para captar el significado de la maravillosa organización dispuesta para su gobierno espiritual y como lugar del avance intelectual de las criaturas de voluntad, que moran en las miríadas de planetas habitados esparcidos por todos los lugares de los siete suprauniversos.
\usection{1. EL NIVEL ESPACIAL DEL SUPRAUNIVERSO}
\vs p015 1:1 Dentro del limitado espectro de registros, observaciones y recuerdos de las generaciones de un millón o de mil millones de vuestros cortos años, a efectos prácticos, Urantia y el universo del que forma parte están sumidos en una inmersión prolongada y desconocida de un espacio nuevo; pero, de acuerdo con los archivos de Uversa, según observaciones más antiguas, en armonía con la experiencia y los cálculos más amplios de seres del orden al que pertenezco, y como resultado de las conclusiones basadas en estos y otros hallazgos, sabemos que los universos están involucrados en una procesión ordenada, bien entendida y perfectamente controlada, que gira con majestuosidad y grandeza alrededor de la Primera Gran Fuente y Centro y del universo donde reside.
\vs p015 1:2 Hace mucho tiempo que descubrimos que los siete suprauniversos se mueven a lo largo de una gran elipse, de un gigantesco círculo alargado. Vuestro sistema solar y otros mundos del tiempo no se están precipitando sin mapas ni brújulas en un espacio desconocido. El universo local al que pertenece vuestro sistema sigue un curso definido y bien conocido, en sentido contrario a las manecillas del reloj, alrededor del inmenso círculo que rodea al universo central. Esta ruta cósmica está bien trazada, y los observadores de las estrellas del suprauniverso la conocen en detalle, tal como los astrónomos de Urantia pueden conocer las órbitas de los planetas que constituyen vuestro sistema solar.
\vs p015 1:3 Urantia está situada en un universo local y en un suprauniverso no organizado por completo. Vuestro universo local está en inmediata proximidad a numerosas creaciones físicas que no están totalmente completas. Pertenecéis a uno de los universos relativamente recientes. Pero, en este momento, no os precipitáis sin control en el espacio inexplorado ni rotáis ciegamente en regiones desconocidas. Os movéis siguiendo la ruta ordenada y predeterminada del nivel espacial del suprauniverso. Estáis ahora pasando a través del mismo espacio que vuestro sistema planetario, o sus predecesores, recorrieron tiempos atrás; y algún día, en el futuro remoto, vuestro sistema o sus sucesores nuevamente recorrerán el mismo espacio en el que con tanta celeridad os precipitáis.
\vs p015 1:4 \pc En esta era y según se percibe en Urantia la dirección, el suprauniverso número uno gira casi en dirección norte, aproximadamente en posición opuesta, hacia el este, a la residencia en el Paraíso de las Grandes Fuentes y Centros y del universo central de Havona. Esta posición, con la correspondiente del oeste, representa el mayor acercamiento físico de las esferas del tiempo a la Isla eterna. El suprauniverso número dos está en el norte, preparándose para girar hacia el oeste, mientras que el número tres en estos momentos ocupa el segmento más septentrional de la gran ruta espacial, habiendo ya virado la curva que lo conduce en su desplazamiento hacia el sur. El número cuatro sigue una trayectoria relativamente recta hacia el sur, con las regiones adelantadas situándose ya en oposición a los Grandes Centros. El número cinco casi ha dejado su posición frente al Centro de los Centros, siguiendo un curso directo hacia el sur justo antes de girar hacia el este. El número seis ocupa la mayor parte de la curva meridional, el segmento que vuestro suprauniverso casi ha franqueado.
\vs p015 1:5 Nebadón, vuestro universo local, pertenece a Orvontón, el séptimo suprauniverso, que se mueve entre los suprauniversos uno y seis, habiendo virado desde no hace mucho (según calculamos nosotros el tiempo) la curva meridional del nivel espacial del suprauniverso. Hoy en día, el sistema solar al que pertenece Urantia dejó atrás, hace unos cuantos miles de millones de años, su recorrido alrededor de la curva meridional, de manera que, en este momento, vosotros estáis avanzando más allá de la curva meridional y os desplazáis rápidamente a través de la larga y relativamente recta ruta septentrional. Durante incalculables eras, Orvontón seguirá este curso septentrional casi directo.
\vs p015 1:6 Urantia pertenece a un sistema que está bien al exterior hacia los límites de vuestro universo local; y vuestro universo local está en este momento atravesando la periferia de Orvontón. Más allá de vosotros aún hay otros, pero vosotros estáis muy alejados en el espacio de aquellos sistemas físicos que giran alrededor del gran círculo en relativa proximidad a la Gran Fuente y Centro.
\usection{2. ORGANIZACIÓN DE LOS SUPRAUNIVERSOS}
\vs p015 2:1 Solo el Padre Universal conoce la ubicación y el número real de los mundos habitados del espacio; a todos llama por su nombre y número. Yo tan solo puedo daros el número aproximado de planetas habitados o habitables, porque algunos universos locales tienen más mundos adecuados para la vida inteligente que otros. Tampoco están organizados todos los universos en proyecto. Por tanto, los cálculos que os ofrezco son únicamente con el propósito de ofreceros una idea de la inmensidad de la creación material.
\vs p015 2:2 \pc Hay siete suprauniversos en el gran universo, y están constituidos aproximadamente de la siguiente manera:
\vs p015 2:3 \li{1.}\bibemph{Sistemas}. Son las unidades básicas del gobierno de los suprauniversos y constan de unos mil mundos habitados o habitables. No se han incluido en este grupo ni soles resplandecientes ni mundos helados ni planetas demasiado cercanos a los ardientes soles ni otras esferas no adecuadas para ser habitadas por criaturas. Se llama sistema a estos mil mundos adaptados para mantener la vida, pero en los sistemas más jóvenes tan solo un número relativamente pequeño de estos mundos puede estar habitado. Cada planeta habitado está presidido por un príncipe planetario, y cada sistema local tiene una esfera arquitectónica como su sede central, que está gobernada por un soberano del sistema.
\vs p015 2:4 \li{2.}\bibemph{Constelaciones}. Cien sistemas (unos 100\,000 planetas habitables) forman una constelación. Cada constelación tiene como sede central arquitectónica una esfera y residen en ella tres hijos vorondadecs, o altísimos. Cada constelación también tiene un fiel de días como observador, esto es, un embajador de la Trinidad del Paraíso.
\vs p015 2:5 \li{3.}\bibemph{Universos locales}. Cien constelaciones (unos 10\,000\,000 de planetas habitables) constituyen un universo local. Cada universo local tiene un magnífico mundo como sede central arquitectónica, que está gobernada por uno de los hijos creadores de Dios, de igual rango, del orden de Miguel. Cada universo está bendecido por la presencia de un unión de días, esto es, un representante de la Trinidad del Paraíso.
\vs p015 2:6 \li{4.}\bibemph{Sectores menores}. Cien universos locales (unos 1\,000\,000\,000 de planetas habitables) constituyen un sector menor del gobierno del suprauniverso. Poseen un maravilloso mundo como sede desde el que sus gobernantes, los recientes de días, rigen los asuntos del sector menor. En cada sede central de un sector menor hay tres recientes de días, que son seres personales supremos de la Trinidad.
\vs p015 2:7 \li{5.}\bibemph{Sectores mayores}. Cien sectores menores (unos 100\,000\,000\,000 de mundos habitables) constituyen un sector mayor. Cada sector mayor posee una espléndida sede central y está presidido por tres perfectos de días, que son seres personales supremos de la Trinidad.
\vs p015 2:8 \li{6.}\bibemph{Suprauniversos}. Diez sectores mayores (sobre 1\,000\,000\,000\,000 de planetas habitables) constituyen un suprauniverso. Cada suprauniverso posee un mundo enorme y glorioso como sede central y está gobernado por tres ancianos de días.
\vs p015 2:9 \li{7.}\bibemph{El gran universo}. Siete suprauniversos constituyen el actual gran universo organizado, que consiste en aproximadamente siete billones de mundos habitables además de las esferas arquitectónicas y de los mil millones de esferas habitadas de Havona. Los suprauniversos están gobernados y regidos de forma indirecta y mediante la reflectividad desde el Paraíso por los siete espíritus mayores. Los eternos de días rigen directamente cada uno de los mil millones de mundos de Havona. Cada uno de estos seres personales supremos de la Trinidad preside una de estas esferas perfectas.
\vs p015 2:10 \pc Excluyendo las esferas del Paraíso\hyp{}Havona, en el plan de organización del universo se establecen las siguientes unidades:
\vs p015 2:11 Suprauniversos:\bibdf7
\vs p015 2:12 Sectores mayores:\bibdf70
\vs p015 2:13 Sectores menores:\bibdf7000
\vs p015 2:14 Universos locales:\bibdf700\,000
\vs p015 2:15 Constelaciones:\bibdf70\,000\,000
\vs p015 2:16 Sistemas locales:\bibdf7\,000\,000\,000
\vs p015 2:17 \pc Planetas habitables:\bibdf7\,000\,000\,000\,000
\vs p015 2:18 Cada uno de los siete suprauniversos está constituido, aproximadamente, de la siguiente manera:
\vs p015 2:19 Un sistema:\bibdf1000 mundos.
\vs p015 2:20 Una constelación (100 sistemas):\bibdf100\,000 mundos.
\vs p015 2:21 Un universo (100 constelaciones):\bibdf10\,000\,000 de mundos.
\vs p015 2:22 Un sector menor (100 universos):\bibdf1\,000\,000\,000 de mundos.
\vs p015 2:23 Un sector mayor (100 sectores menores):\bibdf100\,000\,000\,000 de mundos.
\vs p015 2:24 Un suprauniverso (10 sectores mayores):\bibdf1\,000\,000\,000\,000 de mundos.
\vs p015 2:25 \pc Todos estos cálculos son, como mucho, aproximaciones, porque hay nuevos sistemas que evolucionan constantemente, a la vez que hay otras formaciones que temporalmente dejan de tener existencia material.
\usection{3. EL SUPRAUNIVERSO DE ORVONTÓN}
\vs p015 3:1 Prácticamente, todas las regiones estelares visibles a simple vista desde Urantia pertenecen a la séptima sección del gran universo, el suprauniverso de Orvontón. El inmenso sistema estrellado de la Vía Láctea representa el núcleo central de Orvontón, que está en gran parte más allá de los límites de vuestro universo local. Este gran conjunto de soles, de islas oscuras del espacio, de estrellas dobles, de acumulaciones globulares, de nubes de estrellas, de nebulosas espirales y de otras formas de nebulosas, junto con miríadas de planetas, está agrupado a la manera de un reloj circular alargado de aproximadamente un séptimo de los universos habitados evolutivos.
\vs p015 3:2 Desde la posición astronómica de Urantia, si examináis una sección transversal de los sistemas cercanos a la gran Vía Láctea, observaréis que las esferas de Orvontón se mueven en una inmensa superficie plana alargada, siendo mucho más grande de ancho que de grueso y más largo que ancho.
\vs p015 3:3 La observación de la llamada Vía Láctea desvela el aumento relativo de la densidad estelar en Orvontón cuando los cielos se ven en una dirección, mientras que en ambos lados la densidad disminuye; el número de estrellas y de otras esferas disminuye al alejarse del plano principal de nuestro suprauniverso material. Cuando el ángulo de observación es favorable, al mirar a través del cuerpo principal de esta zona de máxima densidad, dirigís vuestras miradas al universo residencial y centro de todas las cosas.
\vs p015 3:4 \pc Los astrónomos de Urantia han identificado aproximadamente ocho de las diez divisiones mayores de Orvontón. Es difícil reconocer por separado las otras dos porque estáis forzados a visualizar estos fenómenos desde el interior. Si pudierais observar el suprauniverso de Orvontón desde una remota posición en el espacio, inmediatamente reconoceríais los diez sectores principales de la séptima galaxia.
\vs p015 3:5 El centro de rotación de vuestro sector menor está situado lejos en la enorme y densa nube de estrellas de Sagitario, alrededor de la cual giran vuestro universo local y las creaciones que lo acompañan, y desde los lados opuestos del inmenso sistema subgaláctico de Sagitario podéis observar dos grandes flujos de nubes de estrellas que surgen en formidables espirales estelares.
\vs p015 3:6 El núcleo del sistema físico al que pertenecen vuestro sol y los planetas que lo acompañan constituye el centro de la primitiva nebulosa Andrónover. Esta antigua nebulosa espiral se distorsionó ligeramente debido a los trastornos de la gravedad junto con los acontecimientos que acompañaron al nacimiento de vuestro sistema solar, y que se ocasionaron ante la inminente aproximación de una gran nebulosa cercana. Esta colisión directa transformó de alguna manera a Andrónover en un conjunto globular, pero no destruyó totalmente la doble procesión de soles ni las aglomeraciones físicas en torno a estos. Vuestro sistema solar ocupa en la actualidad una posición claramente central en uno de los brazos de esta distorsionada espiral y está situado alrededor del punto intermedio desde el centro hacia fuera, en la dirección del borde del flujo de estrellas.
\vs p015 3:7 \pc El sector de Sagitario y todos los demás sectores y divisiones de Orvontón están en rotación alrededor de Uversa, y parte de la confusión de los astrónomos de Urantia surge de las impresiones falsas y de las distorsiones relativas producidas por los múltiples movimientos rotatorios siguientes:
\vs p015 3:8 \li{1.}La rotación de Urantia alrededor de su Sol.
\vs p015 3:9 \li{2.}La circulación de vuestro sistema solar alrededor del núcleo de la antigua nebulosa Andrónover.
\vs p015 3:10 \li{3.}La rotación del conjunto estrellado de Andrónover y de las acumulaciones que lo acompañan alrededor del centro compuesto de rotación y gravedad de la nube de estrellas de Nebadón.
\vs p015 3:11 \li{4.}El viraje de la nube de estrellas locales de Nebadón y de las creaciones que lo acompañan alrededor del centro en Sagitario de su sector menor.
\vs p015 3:12 \li{5.}El giro alrededor de su sector mayor de los cien sectores menores, incluyendo Sagitario.
\vs p015 3:13 \li{6.}El movimiento giratorio de los diez sectores mayores, el llamado desplazamiento de estrellas, alrededor de la sede central de Uversa en Orvontón.
\vs p015 3:14 \li{7.}El movimiento de Orvontón y de los seis suprauniversos que lo acompañan alrededor del Paraíso y de Havona; es una procesión, en sentido contrario a las manecillas del reloj, del nivel espacial del suprauniverso.
\vs p015 3:15 \pc Estos movimientos múltiples son de varios órdenes: las rutas espaciales de vuestro planeta y de vuestro sistema solar son innatos, connaturales a su origen. El movimiento absoluto en sentido contrario a las manecillas del reloj de Orvontón también es innato, connatural al diseño arquitectónico del universo matriz. Pero los movimientos intermedios son de origen diverso, derivándose en parte de la segmentación característica de la materia y de la energía en los suprauniversos y en parte producidos por la acción inteligente e intencionada de los organizadores de la fuerza del Paraíso.
\vs p015 3:16 \pc Los universos locales están mucho más próximos a medida que se acercan a Havona; las vías circulatorias son más grandes en número, y hay una mayor superposición, nivel sobre nivel. Pero cuando nos alejamos del centro eterno hay cada vez menos sistemas, niveles, vías circulatorias y universos.
\usection{4. LAS NEBULOSAS: LOS ANCESTROS DE LOS UNIVERSOS}
\vs p015 4:1 Aunque la creación y la organización de los universos permanecen por siempre bajo la dirección de los Creadores infinitos y de sus colaboradores, todo el fenómeno opera de acuerdo con un procedimiento estipulado y en conformidad con las leyes gravitatorias de la fuerza, de la energía y de la materia. No obstante, hay algo misterioso en relación con la carga de fuerza universal del espacio. Nosotros comprendemos mucho de la organización de las creaciones materiales desde la etapa ultimatónica en adelante, pero no comprendemos del todo el antecedente cósmico de los ultimatones. Estamos convencidos de que estas fuerzas ancestrales tienen su origen en el Paraíso porque siempre se mueven a través del espacio infundido en los contornos exactos y gigantescos del Paraíso. Aunque no responde a la gravedad del Paraíso, esta carga de fuerza del espacio, el antecedente de toda materialización, siempre responde a la presencia del Paraíso inferior, teniendo al parecer su vía de entrada y salida en el centro de dicho Paraíso inferior.
\vs p015 4:2 Los organizadores de la fuerza del Paraíso transmutan la potencia del espacio en fuerza primordial y hacen evolucionar este potencial prematerial en manifestaciones energéticas primarias y secundarias de la realidad física. Cuando esta energía alcanza niveles de respuesta a la gravedad, los directores de la potencia y sus colaboradores en la gestión del suprauniverso aparecen en escena y comienzan sus interminables operaciones, diseñadas para establecer las múltiples vías por donde circula la potencia y los canales de energía de los universos del tiempo y del espacio. Así aparece la materia física en el espacio, y así se establece el escenario para el inicio de la organización del universo.
\vs p015 4:3 Esta segmentación de la energía es un fenómeno al que jamás se ha dado una explicación por parte de los físicos de Nebadón. Su dificultad principal radica en la relativa inaccesibilidad de los organizadores de la fuerza del Paraíso, ya que los directores vivos de la potencia, aunque son competentes para ocuparse de la energía del espacio, no tienen noción alguna del origen de las energías que de forma tan inteligente y hábil utilizan.
\vs p015 4:4 \pc Estos organizadores de la fuerza del Paraíso son los que dan origen a las nebulosas; son capaces de iniciar alrededor de su presencia espacial los tremendos ciclones de fuerza que, una vez que comienzan, no se pueden detener ni limitar jamás hasta que estas fuerzas que se difunden por todos lados se activan para la aparición final de las unidades ultimatónicas de la materia universal. Así se crean nebulosas espirales y nebulosas con otras formas: las ruedas matrices de los soles que se originan directamente y de sus distintos sistemas. En el espacio exterior se pueden ver diez formas diferentes de nebulosas, de fases de la evolución universal primaria, y estas inmensas ruedas de energía tienen el mismo origen que tuvieron las de los siete suprauniversos.
\vs p015 4:5 \pc Las nebulosas varían mucho en tamaño y en el consiguiente número y masa total de sus descendientes estelares y planetarios. Existe una nebulosa productora de soles situada justo al norte de las fronteras de Orvontón, aunque todavía dentro del nivel espacial de este suprauniverso, que ya ha dado origen a unos cuarenta mil soles, y esta rueda matriz sigue arrojando soles, la mayoría muchas veces más grandes que el vuestro. Algunas de las nebulosas más grandes del espacio exterior están produciendo hasta cien millones de soles.
\vs p015 4:6 Las nebulosas no están directamente relacionadas con ninguna de las unidades administrativas como los sectores menores o los universos locales, aunque algunos universos locales se han organizado a partir de una sola nebulosa. Cada universo local contiene exactamente una cien milésima parte de la carga total de energía de un suprauniverso sea cual fuere su relación con las nebulosas, porque las nebulosas no organizan la energía; esta se distribuye de forma universal.
\vs p015 4:7 No todas las nebulosas espirales producen soles. Algunas han mantenido su control sobre muchos de los descendientes estelares que segregaron, y su apariencia espiral resulta por el hecho de que sus soles salen en estrecha formación del brazo nebular pero retornan por diversas rutas, facilitando así la observación en un determinado punto pero dificultando su visualización cuando están sumamente dispersos en sus diferentes rutas de retorno más alejadas del brazo de la nebulosa. No existen en este momento muchas nebulosas productoras de soles activas en Orvontón, aunque Andrómeda, que está fuera del suprauniverso habitado, es muy activa. Esta remota nebulosa es visible a simple vista, y cuando la visualicéis, considerad que la luz que contempláis de ella salió de esos distantes soles hace casi un millón de años.
\vs p015 4:8 La galaxia denominada Vía Láctea está compuesta por un inmenso número de antiguas nebulosas tanto espirales como de otras formas; muchas aún retienen su estructura original. Si bien, como resultado de cataclismos interiores y de la atracción exterior, otras muchas se han deformado y han cambiado su estructura hasta el punto de que estos enormes conjuntos tienen la apariencia de gigantescas masas luminosas de resplandecientes soles, como la Nube de Magallanes. Cerca de los límites exteriores de Orvontón, predomina una acumulación de estrellas del tipo globular.
\vs p015 4:9 Las inmensas nubes de estrellas de Orvontón deben considerarse como aglomeraciones de materia comparables a las nebulosas que, por separado, se pueden observar en las regiones espaciales exteriores a la Vía Láctea. Muchas de estas denominadas nubes de estrellas del espacio, sin embargo, únicamente contienen material gaseoso. El potencial de energía de estas nubes de estrellas gaseosas es increíblemente enorme, y parte de este lo toman los soles cercanos para enviarlos de nuevo al espacio en forma de emanaciones solares.
\usection{5. EL ORIGEN DE LOS CUERPOS ESPACIALES}
\vs p015 5:1 La mayor parte de la masa contenida en los soles y planetas de un suprauniverso se origina en las ruedas nebulares. Muy poco de la masa del suprauniverso se organiza mediante la acción directa de los directores de la potencia (tal como se hace en la construcción de las esferas arquitectónicas), aunque en el espacio abierto se origina una cantidad constantemente variable de materia.
\vs p015 5:2 En cuanto a su origen, la mayoría de los soles, planetas y otras esferas se pueden clasificar dentro de los siguientes diez grupos:
\vs p015 5:3 \li{1.}\bibemph{Anillos concéntricos por contracción}. No todas las nebulosas son espirales. Muchas inmensas nebulosas en lugar de partirse en un sistema estelar doble o evolucionar como espiral se condensan mediante la formación de anillos múltiples. Durante largos períodos de tiempo, este tipo de nebulosa aparece como un enorme sol central que estuviera circundado y envuelto por numerosas nubes gigantescas de aglomeraciones de materia.
\vs p015 5:4 \li{2.}\bibemph{Las estrellas de remolino} son aquellos soles que salen despedidos de las grandes ruedas matrices de gases sumamente recalentados. No salen como anillos sino que lo hacen sucesivamente hacia la derecha y hacia la izquierda. Las estrellas de remolino también se originan en nebulosas que no son espirales.
\vs p015 5:5 \li{3.}\bibemph{Planetas por explosión de la gravedad}. Cuando nace un sol de una nebulosa en espiral o de una nebulosa en forma de cilindro, no es infrecuente que salga despedido a una distancia considerable. Dicho sol es sumamente gaseoso y, posteriormente, después de haberse enfriado un poco y condensado, puede suceder que se encuentre girando cerca de alguna masa enorme de energía, ya sea un sol gigantesco o una isla oscura del espacio. Aunque no se acerque lo suficiente como para provocar una colisión, sí puede situarse lo suficientemente cerca como para que el efecto de la atracción de la gravedad del cuerpo más grande dé origen a oleadas de convulsiones en el más pequeño, iniciando así una serie de oleadas de disrupciones que ocurren de forma simultánea en los lados opuestos del convulsionado sol. En su punto álgido, estas fulminantes erupciones producen una serie de aglomeraciones de materia de diverso tamaño que pueden proyectarse más allá de la zona de atracción gravitatoria del sol en erupción, estabilizándose así en sus propias órbitas alrededor de uno de los dos cuerpos implicados. Más tarde las acumulaciones más grandes de materia se unen y gradualmente atraen hacia sí a los cuerpos más pequeños. Así es como comienza la existencia de muchos de los planetas sólidos de los sistemas menores. Vuestro propio sistema solar se originó precisamente así.
\vs p015 5:6 \li{4.}\bibemph{Planetas de origen centrífugo}. En ciertas etapas de su desarrollo, hay enormes soles cuya fuerza de rotación se acelera en grado sumo y comienzan a arrojar grandes cantidades de materia que con posterioridad se acumulan para formar pequeños mundos que continúan circundando al sol que les dio origen.
\vs p015 5:7 \li{5.}\bibemph{Esferas por falta de gravedad}. Existe un límite definido para el tamaño de las estrellas solas. Cuando un sol alcanza ese límite, a menos que desacelere su velocidad de rotación, está destinado a fragmentarse. Se produce una fisura en el sol y, de esta división, se origina una nueva estrella doble. Se pueden formar con posterioridad numerosos planetas pequeños como consecuencia de esta gigantesca escisión.
\vs p015 5:8 \li{6.}\bibemph{Estrellas por contracción}. En los sistemas más pequeños, el planeta exterior más grande a veces atrae a sí mismo a los mundos cercanos, mientras que aquellos planetas próximos al sol comienzan su desplazamiento final. En vuestro sistema solar, tal fin significaría que los cuatro planetas más interiores serían atraídos por el sol, mientras que el planeta más grande, Júpiter, se agrandaría de forma considerable porque atraparía a los mundos restantes. Tal fin de un sistema solar daría como resultado la creación de dos soles adyacentes pero desiguales: la formación de una variedad de estrella doble. Este tipo de catástrofes se dan con poca frecuencia excepto en los bordes de los conjuntos estrellados del suprauniverso.
\vs p015 5:9 \li{7.}\bibemph{Esferas por acumulación}. De la inmensa cantidad de materia que circula en el espacio, se pueden acumular lentamente pequeños planetas. Estos crecen por aditamento meteórico y por colisiones menores. En algunos sectores del espacio, las condiciones favorecen que nazcan planetas de este modo. Muchos de los mundos habitados han tenido este origen.
\vs p015 5:10 Algunas islas oscuras densas son el resultado directo del aditamento de la transmutación de energía en el espacio. Otro grupo de estas islas oscuras se ha originado por la acumulación de enormes cantidades de materia fría, de simples fragmentos y meteoros que circulan por el espacio. Estas aglomeraciones de materia nunca han poseído calor y, excepto por su densidad, su composición es muy similar a la de Urantia.
\vs p015 5:11 \li{8.}\bibemph{Soles apagados}. Algunas de las islas oscuras del espacio son soles aislados que se han apagado por haber emitido toda la energía espacial de que disponían. Los elementos que componen la materia se aproximan a la plena condensación, a una consolidación prácticamente completa; y se requieren eras tras eras para que estas enormes masas de materia de tan alta condensación vuelvan a cargarse en las vías que circulan en el espacio y prepararse así para nuevos ciclos de actividad en el universo, tras una colisión o algún suceso cósmico que de igual manera las active.
\vs p015 5:12 \li{9.}\bibemph{Esferas por colisión}. No son raras las colisiones en aquellas regiones con acumulaciones más densas. A este ajuste astronómico lo acompañan tremendos cambios de energía y de transmutaciones de materia. Las colisiones en las que están implicados soles muertos son claramente las causantes de la creación de fluctuaciones generalizadas de energía. Los restos de las colisiones constituyen, con frecuencia, los núcleos materiales que posteriormente forman cuerpos planetarios aptos para ser habitados por los mortales.
\vs p015 5:13 \li{10.}\bibemph{Mundos arquitectónicos}. Estos mundos se construyen de acuerdo con un plan específico y un propósito preciso, tal como Lugar de Salvación, sede central de vuestro universo local, y Uversa, donde reside el gobierno de nuestro suprauniverso.
\vs p015 5:14 \pc Existen otros numerosos modos de desarrollar soles y de separar planetas, pero los que se han mencionado reflejan la forma de originarse de la inmensa mayoría de los sistemas estelares y de los conjuntos planetarios. Describir todos los modos diferentes de metamorfosis de las estrellas y de evolución de los planetas requeriría relatar casi cien modos distintos en que se forman los soles y se originan los planetas. A medida que vuestros estudiosos de las estrellas escudriñen los cielos, observarán fenómenos que les indicarán todos estos modos de evolución estelar, pero pocas veces detectarán la formación de esas pequeñas agrupaciones no luminosas de materia que sirven de planetas habitados, los más importantes de las inmensas creaciones materiales.
\usection{6. LAS ESFERAS DEL ESPACIO}
\vs p015 6:1 Sea cual fuere su origen, las distintas esferas del espacio se pueden clasificar en las siguientes divisiones principales:
\vs p015 6:2 \li{1.}Soles: las estrellas del espacio.
\vs p015 6:3 \li{2.}Islas oscuras del espacio.
\vs p015 6:4 \li{3.}Cuerpos espaciales menores: cometas, meteoros y planetesimales.
\vs p015 6:5 \li{4.}Planetas, incluyendo los mundos habitados.
\vs p015 6:6 \li{5.}Esferas arquitectónicas: mundos hechos a medida.
\vs p015 6:7 \pc Con la excepción de las esferas arquitectónicas, todos los cuerpos espaciales han tenido un origen evolutivo, evolutivo en el sentido de que no han llegado a existir por decreto de la Deidad, evolutivo en el sentido de que los actos creativos de Dios se han desarrollado de forma espacio\hyp{}temporal a través de la actuación de muchas de las inteligencias creadas y acontecidas por la Deidad.
\vs p015 6:8 \pc \bibemph{ Los soles}. Son las estrellas del espacio en todas las distintas etapas de su existencia. Algunos son sistemas espaciales solitarios en evolución; otros son estrellas dobles, sistemas planetarios en contracción o en desaparición. Las estrellas del espacio existen en no menos de mil distintos estados y etapas. Vosotros estáis familiarizados con los soles que emiten luz acompañada de calor; pero también hay soles que brillan sin calor.
\vs p015 6:9 Los billones y billones de años que un sol ordinario continuará emitiendo calor y luz son un signo del inmenso almacenamiento de energía que contiene cada unidad de materia. La energía real almacenada en estas partículas invisibles de materia física es casi inimaginable. Y esta energía se vuelve casi totalmente disponible en forma de luz cuando se la somete a la tremenda presión del calor y a la actividad energética conexa que prevalecen en el interior de los resplandecientes soles. Hay todavía otras condiciones que permiten a estos soles transformar y enviar mucha de la energía espacial que les llega por las vías establecidas que circulan en el espacio. La energía física en muchas de sus fases y todas las formas de la materia son atraídas hacia las dínamos solares y posteriormente se distribuyen por ellas. De esta manera los soles sirven como aceleradores locales de la circulación de la energía, actuando como habituales estaciones de control de la potencia.
\vs p015 6:10 Más de diez billones de resplandecientes soles calientan e iluminan el suprauniverso de Orvontón. Estos soles son las estrellas que podéis observar en vuestro sistema astronómico. Más de dos billones están demasiado distantes y son demasiado pequeños como para poder verse desde Urantia. Pero en el universo matriz existen tantos soles como gotas de agua hay en los océanos de vuestro mundo.
\vs p015 6:11 \pc \bibemph{Las islas oscuras del espacio}. Son soles muertos y otras aglomeraciones grandes de materia carentes de luz y calor. Las islas oscuras a veces son enormes en cuanto a su masa y ejercen una influencia poderosa en el equilibrio y en la utilización de la energía en el universo. La densidad de algunas de estas grandes masas es casi increíble. Y esta gran concentración de masa permite que estas islas oscuras actúen como poderosas ruedas equilibrantes, manteniendo sistemas adyacentes suficientemente unidos. Sostienen el equilibrio gravitatorio de la potencia en muchas constelaciones; muchos sistemas físicos que, de otro modo, se destruirían rápidamente al ser arrastrados hacia soles cercanos, se mantienen firmemente dentro de la atracción de la gravedad de estas protectoras islas oscuras. Es debido a esta acción por lo que podemos ser precisos en su localización. Hemos medido la atracción de la gravedad de los cuerpos luminosos y, por tanto, podemos calcular el tamaño y ubicación exacta de las islas oscuras del espacio que con tanta eficacia mantienen a cualquier sistema fijo en su curso.
\vs p015 6:12 \pc \bibemph{Los cuerpos espaciales menores}. Los meteoros y otras pequeñas partículas de materia que circulan y evolucionan en el espacio constituyen un enorme agregado de energía y de sustancia material.
\vs p015 6:13 Muchos cometas surgen sin control alguno como resultado de las ruedas solares matrices, que paulatinamente se someten al control del sol central dominante. Los cometas tienen también numerosos otros orígenes. La cola de un cometa señala la dirección contraria al cuerpo o sol que lo atrae debido a la reacción eléctrica de sus gases sumamente expandidos y a la presión real de la luz y otras energías que emanan del sol. Este fenómeno constituye una de las pruebas inequívocas de la existencia física de la luz y de las energías a ella vinculadas; demuestra que la luz tiene peso. La luz es una sustancia real; no son simplemente olas de un hipotético éter.
\vs p015 6:14 \pc \bibemph{Los planetas}. Son las aglomeraciones más grandes de materia que siguen una órbita alrededor de un sol o de algún otro cuerpo espacial; oscilan en tamaño desde lo planetesimal hasta enormes esferas gaseosas, líquidas o sólidas. Los mundos fríos que se han acumulado mediante el ensamblaje de material flotante del espacio, cuando están en relación adecuada con un sol cercano, son los mejores planetas para albergar vida inteligente. Los soles muertos no reúnen, en general, condiciones para la vida; están normalmente demasiado lejos de un sol activo y resplandeciente y, además, son en su totalidad demasiado grandes; la gravedad en su superficie es tremenda.
\vs p015 6:15 En vuestro suprauniverso, de cuarenta planetas fríos, solamente uno reúne las condiciones para que lo habite vuestro orden de seres. Y, por supuesto, los soles supercalentados y los helados mundos exteriores no son adecuados para albergar una vida de orden superior. En vuestro sistema solar, en el presente, existen solo tres planetas que pueden cobijar la vida. Urantia, por su tamaño, densidad y ubicación, es, en muchos aspectos, ideal para la vida humana.
\vs p015 6:16 Las leyes del comportamiento de la energía física son intrínsecamente universales, pero su efecto a nivel local tiene mucho que ver con las condiciones físicas predominantes en cada uno de los planetas y sistemas locales. Los incontables mundos del espacio se caracterizan por una casi ilimitada variedad de criaturas y de otras manifestaciones de la vida. Hay, sin embargo, ciertos rasgos similares en los grupos de mundos vinculados entre sí a un sistema específico, al igual que también existe un modelo universal de vida inteligente. Existe una afinidad física entre esos sistemas planetarios que pertenecen a la misma vía de circulación física, y que se siguen uno a otro en su casi ilimitado recorrido alrededor del círculo de los universos.
\usection{7. LAS ESFERAS ARQUITECTÓNICAS}
\vs p015 7:1 Aunque cada gobierno del suprauniverso opera cerca del centro de los universos evolutivos de su segmento espacial, este ocupa un mundo hecho a medida y poblado por seres personales autorizados. Estos mundos, donde se ubican sus sedes centrales, son esferas arquitectónicas, cuerpos espaciales expresamente construidos con un propósito específico. Aunque comparten la luz de soles cercanos, estas esferas se iluminan y calientan de forma autónoma. Cada una tiene un sol que emite luz sin calor, como los satélites del Paraíso, y se calientan mediante la circulación de ciertas corrientes de energía cerca de la superficie de la esfera. Estos mundos sedes pertenecen a uno de los sistemas más grandes situados cerca del centro astronómico de sus suprauniversos respectivos.
\vs p015 7:2 \pc El tiempo está reglado en las sedes centrales de los suprauniversos. Un día ordinario en el suprauniverso de Orvontón equivale a casi treinta días del tiempo de Urantia, y un año de Orvontón equivale a cien días ordinarios. El año de Uversa es el ordinario en el séptimo suprauniverso, y corresponde a tres mil días menos veintidós minutos del tiempo de Urantia, alrededor de ocho de vuestros años más un quinto de año.
\vs p015 7:3 \pc Los mundos donde se localizan las sedes de los siete suprauniversos participan de la naturaleza y grandeza del Paraíso, su modelo central de perfección. En realidad, todos los mundos sedes son paradisíacos. Son, en efecto, moradas celestiales, y aumentan en tamaño material, belleza morontial y gloria espiritual desde Jerusem hasta la Isla central. Y todos los satélites de estos mundos sedes son también esferas arquitectónicas.
\vs p015 7:4 Los distintos mundos sedes están provistos de todas las formas de creación material y espiritual. Todas las clases de seres materiales, morontiales y espirituales se sienten en casa en estos mundos de encuentro de los universos. A medida que las criaturas mortales ascienden en el universo, pasando desde los mundos materiales a los mundos espirituales, jamás dejan de reconocer el valor ni el disfrute de esos niveles existenciales que les precedieron.
\vs p015 7:5 \pc \bibemph{Jerusem,} sede central de vuestro sistema local de Satania, tiene sus siete mundos de cultura y transición, cada uno circundado por siete satélites, entre los que están los siete mundos de las moradas, parada obligada morontial, y la primera residencia posmortal del hombre. La palabra cielo tal como a veces se la ha utilizado en Urantia, en ocasiones alude a estos siete mundos de las moradas, denominándose el primer mundo de morada el primer cielo, y así sucesivamente hasta el séptimo.
\vs p015 7:6 \pc \bibemph{Edentia,} sede central de vuestra constelación de Norlatiadec, tiene sus setenta satélites de cultura social y formación; en estos residen los seres ascendentes tras completar el régimen de Jerusem en cuanto a activación, unificación y realización del ser personal.
\vs p015 7:7 \pc \bibemph{Lugar de Salvación,} la capital de Nebadón, vuestro universo local, está rodeada de diez conjuntos universitarios de cuarenta y nueve esferas cada uno. Aquí el hombre se espiritualiza tras su etapa de confraternización en la constelación.
\vs p015 7:8 \pc \bibemph{Umenor tercero,} sede central de Ensa, vuestro sector menor, está rodeado de siete esferas para que la vida que asciende se prepare en estudios físicos superiores.
\vs p015 7:9 \pc \bibemph{Umayor quinto,} la sede central de Esplandón, vuestro sector mayor, está rodeado de setenta esferas dedicadas al perfeccionamiento intelectual del suprauniverso.
\vs p015 7:10 \pc \bibemph{Uversa,} sede central de Orvontón, vuestro suprauniverso, está muy de cerca rodeada de siete centros universitarios superiores dedicados al perfeccionamiento espiritual de las criaturas de voluntad en su camino ascendente. Cada uno de estos siete conjuntos de esferas extraordinarias consta de setenta mundos especializados que contienen miles y miles de instituciones y organismos bien provistos, dedicados a la formación y al cultivo espiritual del universo, en los que los peregrinos del tiempo de nuevo se instruyen y examinan como preparación para su largo viaje a Havona. A los peregrinos del tiempo que llegan se les recibe siempre en estos mundos afines, pero los que parten, al haberse graduado, siempre salen para Havona directamente desde las orillas de Uversa.
\vs p015 7:11 Uversa es la sede central espiritual y administrativa de aproximadamente un billón de mundos habitados o habitables. La gloria, grandeza y perfección de la capital de Orvontón sobrepasa todas las maravillas de las creaciones del tiempo y del espacio.
\vs p015 7:12 \pc Si se instaurasen todos los universos locales que están en proyecto, junto con los elementos que los componen, habría en los siete suprauniversos algo menos de quinientos mil millones de mundos arquitectónicos.
\usection{8. CONTROL Y REGULACIÓN DE LA ENERGÍA}
\vs p015 8:1 Las esferas donde se localizan las sedes de los suprauniversos están construidas de forma que pueden actuar como eficaces reguladores de la potencia y de la energía para sus distintos sectores, sirviendo de puntos de convergencia desde donde se dirige la energía a los universos locales que los componen. Ejercen una poderosa influencia sobre el equilibrio y el control de la energía física que circula a través del espacio organizado.
\vs p015 8:2 Los centros de la potencia del suprauniverso y los controladores físicos, entidades inteligentes vivas y semivivas, originadas con este preciso propósito, realizan otras funciones de regulación. Resultan difíciles de entender. Estos órdenes menores de seres no son volitivos, no poseen voluntad, no tienen decisión, actúan de forma muy inteligente, aunque al parecer esto es automático y consustancial a la elevada especialización de su grupo. Los centros de la potencia y los controladores físicos de los suprauniversos tienen a su cargo la dirección y el control parcial de los treinta sistemas de energía que comprenden el área de la gravita. Las vías de circulación de la energía física que rigen los centros de la potencia de Uversa necesitan algo más de 968 millones de años para completar su circunvalación del suprauniverso.
\vs p015 8:3 \pc La energía en evolución es real; tiene peso, aunque el peso es siempre relativo, dependiendo de la velocidad de rotación, la masa y la antigravedad. La masa de la materia tiende a retardar la velocidad en la energía; y la velocidad de la energía, que está presente por todos lados, representa su dotación inicial de velocidad menos la retardación por la masa que se encuentra en tránsito más la función reguladora de los controladores de la energía viva del suprauniverso y la influencia física de los cuerpos cercanos sumamente recalentados o sumamente cargados.
\vs p015 8:4 El plan universal para el mantenimiento del equilibrio entre la materia y la energía hace necesario que se hagan y deshagan constantemente unidades materiales menores. Los directores de la potencia del universo tienen la habilidad de condensar y detener, o expandir y liberar, cantidades variables de energía.
\vs p015 8:5 Si el efecto retardador durara lo suficiente, la gravedad acabaría por convertir toda la energía en materia si no fuese por dos factores: primero, por el efecto antigravitatorio de los controladores de la energía y, segundo, porque la materia organizada tiende a desintegrarse bajo ciertas condiciones existentes en estrellas de temperaturas muy elevadas y bajo ciertas condiciones peculiares en el espacio, cerca de cuerpos fríos sumamente energizados de materia condensada.
\vs p015 8:6 Cuando la masa se sobrecarga y amenaza con desequilibrar la energía, para reabastecer las vías de circulación de la potencia física, intervienen los controladores físicos, a menos que la misma tendencia adicional de la gravedad a materializar por completo la energía sea vencida en la eventualidad de una colisión entre los gigantes extintos del espacio, disipando así completamente, en un instante, las acumulaciones de gravedad. En estas circunstancias en las que sobreviene una colisión, enormes masas de materia se convierten de repente en la más rara forma de energía, y comienza de nuevo la pugna por el equilibrio universal. Finalmente, los sistemas físicos más grandes se estabilizan, se fijan físicamente y giran en las estables y equilibradas vías que circulan en los suprauniversos. Con posterioridad a este suceso, ya no ocurrirán más colisiones ni otras catástrofes devastadoras en estos sistemas estables.
\vs p015 8:7 Durante los períodos de exceso de energía se dan turbulencias energéticas y fluctuaciones caloríficas acompañadas de manifestaciones eléctricas. Durante los períodos de menos energía aumenta la tendencia de la materia a agruparse, condensarse y descontrolarse en las vías circulatorias de más frágil equilibrio, resultando en ajustes mediante ondas o colisión que rápidamente restablecen el equilibrio entre la energía circulante y la materia más realmente estabilizada. Prever y comprender, además, la posible conducta de los resplandecientes soles y de las islas oscuras del espacio es una de las tareas de los observadores celestiales de las estrellas.
\vs p015 8:8 Somos capaces de identificar la mayoría de las leyes que gobiernan el equilibrio universal y de predecir bastante lo relacionado con la estabilidad del universo. Prácticamente, nuestros pronósticos son fidedignos, pero siempre nos enfrentamos con ciertas fuerzas que no responden del todo a las leyes del control de la energía ni a la conducta de la materia que nos son conocidas. La previsibilidad de todos los fenómenos físicos se vuelve cada vez más difícil a medida que procedemos desde el Paraíso hacia fuera, hacia los universos. Al cruzar los límites de la administración personal de los gobernantes del Paraíso, nos encontramos cada vez más incapaces de hacer cálculos en base a las reglas establecidas y a la experiencia adquirida en relación a las observaciones que guardan relación exclusivamente con los fenómenos físicos de los sistemas astronómicos cercanos. Incluso en el ámbito de los siete suprauniversos, vivimos en medio de acciones de fuerza y reacciones de energía que impregnan todos nuestros dominios y se extienden de forma unida y equilibrada a través de todas las regiones del espacio exterior.
\vs p015 8:9 Cuanto más nos alejamos, con más claridad encontramos esos fenómenos variables e imprevisibles que son tan infaliblemente característicos de la insondable presencia\hyp{}actuación de los Absolutos y de las Deidades experienciales. Y estos fenómenos han de indicar alguna acción directiva sobre todas las cosas.
\vs p015 8:10 Aparentemente, el suprauniverso de Orvontón está ahora descargándose; los universos exteriores parecen estar cargándose para actividades futuras sin precedentes; el universo central de Havona está eternamente estabilizado. La gravedad y la ausencia de calor (frío) organizan y mantienen la materia; el calor y la antigravedad alteran la materia y disipan la energía. Los directores de la potencia y los organizadores de la fuerza vivos constituyen la clave del control especial y de la dirección inteligente de la interminable metamorfosis de la construcción, desintegración y reconstrucción del universo. Las nebulosas pueden dispersarse, los soles extinguirse, los sistemas desaparecer y los planetas perecer, pero los universos no se descargan.
\usection{9. LAS VÍAS CIRCULATORIAS DE LOS SUPRAUNIVERSOS}
\vs p015 9:1 Las vías circulatorias universales del Paraíso literalmente se difunden por los siete suprauniversos. Estas vías presenciales son: la gravedad del ser personal del Padre Universal, la gravedad espiritual del Hijo Eterno, la gravedad mental del Actor Conjunto y la gravedad material de la Isla eterna.
\vs p015 9:2 Además de las vías circulatorias universales del Paraíso y de la presencia\hyp{}actuación de los Absolutos y de las Deidades experienciales, dentro del nivel espacial del suprauniverso operan únicamente dos secciones de la vía de circulación de la energía o segregaciones de la potencia: las vías circulatorias del suprauniverso y de los universos locales.
\vs p015 9:3 \pc \bibemph{Las vías circulatorias del suprauniverso:}
\vs p015 9:4 \li{1.}La vía circulatoria de uno de los siete espíritus mayores del Paraíso que unifica la inteligencia. Esta vía por donde circula la mente cósmica está limitada a un solo suprauniverso.
\vs p015 9:5 \li{2.}La vía de los siete espíritus reflectores por donde circula el sistema de reflectancia de cada suprauniverso.
\vs p015 9:6 \li{3.}Las vías circulatorias secretas de los mentores misteriosos, que de alguna manera están correlacionadas y se dirigen hacia el Padre en el Paraíso a partir de Lugar de la Divinidad.
\vs p015 9:7 \li{4.}La vía de comunicación entre el Hijo Eterno y sus hijos del Paraíso.
\vs p015 9:8 \li{5.}La presencia instantánea del Espíritu Infinito.
\vs p015 9:9 \li{6.}Las transmisiones del Paraíso, los informes espaciales de Havona.
\vs p015 9:10 \li{7.}Las vías de los centros de la potencia y de los controladores físicos por donde circula la energía.
\vs p015 9:11 \pc \bibemph{Las vías circulatorias del universo local:}
\vs p015 9:12 \li{1.}El espíritu de gracia de los hijos del Paraíso, el consolador de los mundos de gracia. El espíritu de la verdad, el espíritu de Miguel en Urantia.
\vs p015 9:13 \li{2.}La vía circulatoria de las benefactoras divinas, los espíritus maternos del universo local, el espíritu santo de vuestro mundo.
\vs p015 9:14 \li{3.}La vía circulatoria del ministerio de la inteligencia de un universo local, que incluye la presencia de los espíritus asistentes de la mente en su variada capacidad de actuación.
\vs p015 9:15 \pc Cuando se desarrolla en un universo local una cierta armonía espiritual en la que sus vías circulatorias individuales y combinadas se vuelven indistintas de las del suprauniverso, cuando en verdad prevalece tal similitud de actuación y unicidad de ministerio, entonces el universo local de inmediato se mueve circulando por las vías establecidas de luz y vida, se hace apto para su admisión en la alianza espiritual formada por la unión perfeccionada de dicha supracreación. Los requisitos para la admisión en los consejos de los ancianos de días, o sea, para ser miembro de la alianza del suprauniverso, son:
\vs p015 9:16 \li{1.}\bibemph{Estabilidad física}. Las estrellas y los planetas de un universo local deben estar en equilibrio; los períodos de metamorfosis estelar inmediata deben haber terminado. El universo debe seguir una ruta definitiva; su órbita debe ser invulnerable e irrevocablemente estable.
\vs p015 9:17 \li{2.}\bibemph{Lealtad espiritual}. Debe existir un estado de reconocimiento universal, y de lealtad, al hijo soberano de Dios que dirige los asuntos de dicho universo local. Debe haber entrado en un estado de cooperación en buenos términos entre los planetas, sistemas y constelaciones de todo el universo local.
\vs p015 9:18 \pc Ni siquiera se considera a vuestro universo local como perteneciente al orden físico establecido del suprauniverso ni tampoco capaz de ser miembro de la familia espiritual reconocida en el gobierno del suprauniverso. Aunque Nebadón todavía no tiene representación en Uversa, a nosotros, pertenecientes al gobierno del suprauniverso, se nos envía periódicamente a sus mundos en misiones especiales, tal como yo he hecho al venir a Urantia, directamente desde Uversa. Prestamos toda la ayuda posible a vuestros directores y gobernantes para que puedan solucionar sus difíciles problemas; estamos deseosos de ver que vuestro universo cumpla los requisitos para ser admitido con pleno derecho en las creaciones vinculadas entre sí de la familia del suprauniverso.
\usection{10. LOS GOBERNANTES DE LOS SUPRAUNIVERSOS}
\vs p015 10:1 En las sedes centrales de los suprauniversos residen los miembros del elevado gobierno espiritual de las regiones del tiempo y del espacio. El poder ejecutivo del gobierno del suprauniverso, que tiene su origen en los consejos de la Trinidad, está inmediatamente dirigido, y supremamente supervisado, por uno de los siete espíritus mayores. Estos seres representan la autoridad del Paraíso y dirigen los suprauniversos a través de los siete mandatarios supremos emplazados en los siete mundos especiales del Espíritu Infinito, los satélites más exteriores del Paraíso.
\vs p015 10:2 La sede central del suprauniverso es la morada de los espíritus reflectores y de los auxiliares reflectores de imagen. Desde esta posición intermedia, estos seres maravillosos llevan a cabo su extraordinaria actividad en el campo de la reflectividad, sirviendo así, por encima y por debajo, al universo central y a los universos locales respectivamente.
\vs p015 10:3 \pc Cada suprauniverso está presidido por tres ancianos de días, mandatarios conjuntos que ostentan la jefatura del gobierno del suprauniverso. En su poder ejecutivo, el personal a cargo del gobierno del suprauniverso se divide en siete grupos diferentes:
\vs p015 10:4 \li{1.}Ancianos de días.
\vs p015 10:5 \li{2.}Perfeccionadores de la sabiduría.
\vs p015 10:6 \li{3.}Consejeros divinos.
\vs p015 10:7 \li{4.}Censores universales.
\vs p015 10:8 \li{5.}Mensajeros poderosos.
\vs p015 10:9 \li{6.}Aquellos elevados en autoridad.
\vs p015 10:10 \li{7.}Aquellos sin nombre ni número.
\vs p015 10:11 \pc A los tres ancianos de días les asiste directamente un colectivo de mil millones de perfeccionadores de la sabiduría, con quienes están vinculados tres mil millones de consejeros divinos. Mil millones de censores universales se asignan a la administración de cada uno de los suprauniversos. Estos tres grupos son Seres Personales Iguales en Rango de la Trinidad, y se originan de forma divina y directa en la Trinidad del Paraíso.
\vs p015 10:12 Los tres órdenes restantes, los mensajeros poderosos, aquellos elevados en autoridad y aquellos sin nombre ni número son mortales ascendentes glorificados. El primero de estos órdenes se elevó siguiendo el régimen de ascensión y pasó a través de Havona en los días de Granfanda. Cuando llegaron al Paraíso se incorporaron al colectivo final, la Trinidad del Paraíso los acogió y, posteriormente, se les asignó al servicio excelso de los ancianos de días. Como clase, estos tres órdenes se conocen como los hijos finalizadores trinitizados; son de origen doble pero ahora están al servicio de la Trinidad. De esta manera, el poder ejecutivo del gobierno del suprauniverso se amplió para incluir a los hijos glorificados y perfeccionados de los mundos evolutivos.
\vs p015 10:13 El consejo de acción conjunta, del suprauniverso, está compuesto por los siete grupos de mandatarios arriba nombrados y por los siguientes gobernantes de los sectores y de otros supervisores regionales:
\vs p015 10:14 \li{1.}Los perfectos de días: los gobernantes de los sectores mayores de un suprauniverso.
\vs p015 10:15 \li{2.}Los recientes de días: los directores de los sectores menores de un suprauniverso.
\vs p015 10:16 \li{3.}Los uniones de días: los asesores del Paraíso para los gobernantes de los universos locales.
\vs p015 10:17 \li{4.}Los fieles de días: los consejeros del Paraíso para los gobernantes altísimos de los gobiernos de las constelaciones.
\vs p015 10:18 \li{5.}Los hijos preceptores de la Trinidad que casualmente se encuentren de servicio en la sede central del suprauniverso.
\vs p015 10:19 \li{6.}Los eternos de días que puedan encontrarse presentes en la sede central del suprauniverso.
\vs p015 10:20 \li{7.}Los siete auxiliares reflectores de imagen: portavoces de los siete espíritus reflectores y, a través de ellos, los representantes de los siete espíritus mayores del Paraíso.
\vs p015 10:21 \pc Los auxiliares reflectores de imagen también actúan como representantes de numerosos grupos de seres influyentes en los gobiernos del suprauniverso, pero por distintas razones no se encuentran, por el momento, plenamente activos en su cargo. Dentro de este grupo se encuentran la manifestación en el suprauniverso del ser personal del Ser Supremo en evolución, los supervisores incondicionados del Supremo, los vicerregentes condicionados del Último, los reflectores de enlace innominados de Majestón y los representantes espirituales suprapersonales del Hijo Eterno.
\vs p015 10:22 \pc En casi cualquier momento, es posible encontrar representantes de todos los grupos de seres creados en los mundos donde se localizan las sedes centrales de los suprauniversos. La tarea rutinaria de ministerio de los suprauniversos la realizan los poderosos seconafines y otros miembros de la inmensa familia del Espíritu Infinito. En la tarea de estos maravillosos centros de administración, control, ministerio y juicio ejecutivo, las inteligencias de cada esfera de la vida universal se combinan en un servicio eficiente, en una administración inteligente, en un ministerio amoroso y en un juicio justo.
\vs p015 10:23 Los suprauniversos no mantienen ningún tipo de legación entre sí; están completamente aislados unos de los otros. En ellos se tiene conocimiento de los asuntos mutuos solo a través de un centro de intercambio de información en el Paraíso mantenido por los siete espíritus mayores. Estos gobernantes trabajan en los consejos de la sabiduría divina para el bien de sus propios suprauniversos, a pesar de lo que pueda estar ocurriendo en otros segmentos de la creación universal. Este aislamiento de los suprauniversos persistirá hasta el momento en que se logre su coordinación mediante una efectuación más completa del ser personal\hyp{}soberanía del Ser Supremo experiencial, ahora en evolución.
\usection{11. LA ASAMBLEA DELIBERANTE}
\vs p015 11:1 Es en mundos como Uversa donde se encuentran cara a cara los seres representantes de la autocracia de la perfección y los de la democracia de la evolución. El poder ejecutivo del gobierno del suprauniverso se origina en los ámbitos de la perfección; el poder legislativo surge del florecimiento de los universos evolutivos.
\vs p015 11:2 La asamblea deliberante del suprauniverso está circunscrita al mundo donde está la sede central. Este consejo legislativo o consultivo consta de siete cámaras, y cada universo local que se admite a los consejos del suprauniverso elige a su representante local para cada una de ellas. Los consejos superiores de dichos universos locales eligen a estos representantes de entre los peregrinos ascendentes que se han graduado en Orvontón y que permanecen en Uversa, con autorización para transportarse a Havona. El término medio de servicio es alrededor de cien años de tiempo regular del suprauniverso.
\vs p015 11:3 No he conocido nunca desavenencia alguna entre los mandatarios de Orvontón y la asamblea de Uversa. Jamás hasta ahora, en la historia de nuestro suprauniverso, el cuerpo deliberante ha aprobado una recomendación que la división ejecutiva del gobierno del suprauniverso haya siquiera titubeado en ejecutar. Ha prevalecido siempre la más perfecta armonía y un acuerdo de cooperación, todo lo cual da testimonio del hecho de que los seres evolutivos pueden en verdad lograr las alturas de la sabiduría perfeccionada que les faculta para concordar con seres personales de origen perfecto y de naturaleza divina. La presencia de las asambleas deliberantes en la sede central de los suprauniversos revela la sabiduría, y preconiza el triunfo último de todo el inmenso concepto evolutivo del Padre Universal y de su Hijo Eterno.
\usection{12. LOS TRIBUNALES SUPREMOS}
\vs p015 12:1 Cuando hablamos de los poderes ejecutivo y deliberante del gobierno de Uversa, podríais, por analogía con ciertas formas de gobierno civil de Urantia, razonar que hemos de tener un tercer poder o poder judicial, y así es; pero no forman un colectivo separado. Nuestros tribunales están constituidos de la siguiente manera: los preside, según la naturaleza y gravedad del caso, un anciano de días, un perfeccionador de la sabiduría o un consejero divino. Las pruebas a favor o en contra de un ser, planeta, sistema, constelación o universo se exponen a los censores, que son quienes las interpretan. La defensa de los hijos del tiempo y de los planetas evolutivos está a cargo de los mensajeros poderosos, u observadores oficiales del gobierno del suprauniverso ante los universos y sistemas locales. La actitud del gobierno superior la representan aquellos elevados en autoridad. Y pronuncia generalmente el veredicto una comisión, cuyo número varía, que consiste por igual en aquellos sin nombre ni número y en un grupo de seres personales comprensivos elegidos desde la asamblea deliberante.
\vs p015 12:2 Los tribunales de los ancianos de días son los órganos jurisdiccionales superiores de apelación ante el dictamen espiritual de todos los universos locales que componen los suprauniversos. Los hijos soberanos de los universos locales son supremos en sus propios dominios, están sujetos al gobierno del suprauniverso solamente en su voluntariedad de presentar asuntos para el asesoramiento o pronunciamiento de los ancianos de días, excepto en los asuntos relativos a la extinción de las criaturas de voluntad. Los mandatos judiciales se originan en los universos locales, pero las sentencias relativas a la extinción de una criatura de voluntad siempre se formulan en la sede central del suprauniverso y allí se ejecutan. Los hijos de los universos locales pueden decretar la supervivencia de un hombre mortal, pero solo los ancianos de días pueden constituir un tribunal que realice un juicio ejecutivo en los asuntos de la vida y la muerte eternas.
\vs p015 12:3 En todos los asuntos que no requieren un juicio, o presentación de pruebas, los ancianos de días o sus colaboradores toman las decisiones, y estas decisiones son siempre unánimes. Aquí nos encontramos ante consejos perfectos. No hay desacuerdos ni opiniones minoritarias en los decretos de estos tribunales supremos y excepcionales.
\vs p015 12:4 Con pocas excepciones, los gobiernos de los suprauniversos ejercen jurisdicción sobre todas las cosas y todos los seres de su entorno respectivo. No se recurren las decisiones y decretos de las autoridades del suprauniverso, puesto que estas representan las opiniones acordadas por los ancianos de días y el espíritu mayor que, desde el Paraíso, preside los destinos del suprauniverso correspondiente.
\usection{13. LOS GOBIERNOS DE LOS SECTORES}
\vs p015 13:1 Un \bibemph{sector mayor} abarca aproximadamente un décimo de un suprauniverso y consiste en cien sectores menores, diez mil universos locales, unos cien mil millones de mundos habitables. Tres perfectos de días ---seres personales supremos de la Trinidad--- rigen estos sectores mayores.
\vs p015 13:2 Los tribunales de los perfectos de días están constituidos de forma muy semejante a las de los ancianos de días, excepto que no se constituyen en órgano jurisdiccional para juzgar espiritualmente a los mundos. La labor de estos gobiernos del sector mayor guarda relación fundamentalmente con la condición intelectual de esta extensa creación. En los sectores mayores se retienen, arbitran, ofrecen y organizan, con el objeto de informar a los tribunales de los ancianos de días, todos los asuntos de importancia del suprauniverso de naturaleza rutinaria y administrativa no relacionados directamente con la administración espiritual de los mundos o con la realización de los planes de ascensión de los mortales diseñados por los gobernantes del Paraíso. El personal a cargo del gobierno de un sector mayor no es diferente al de un suprauniverso.
\vs p015 13:3 Así como los magníficos satélites de Uversa se destinan a la etapa final de vuestra preparación espiritual para llegar a Havona, los setenta satélites de Umayor Quinto se dedican a la formación y al desarrollo intelectual que se imparten en vuestro suprauniverso. Desde todo Orvontón, se congregan aquí seres de sabiduría que incansablemente trabajan para preparar a los mortales del tiempo para su posterior camino de avance en la eternidad. La mayor parte de esta instrucción dirigida a los mortales ascendentes se realiza en los setenta mundos de estudio.
\vs p015 13:4 \pc Tres recientes de días presiden los gobiernos del \bibemph{sector menor}. Se encargan mayormente del control físico, unificación, estabilización y coordinación rutinaria de la administración de los universos locales que lo componen. Cada sector menor abarca hasta cien universos locales, diez mil constelaciones, un millón de sistemas o unos mil millones de mundos habitables.
\vs p015 13:5 Los mundos donde se ubica la sede central del sector menor son el gran punto de encuentro de los controladores físicos mayores. Estos mundos sedes están rodeados de siete esferas de formación que constituyen las escuelas de ingreso en el suprauniverso y son los centros de enseñanza que imparten el conocimiento físico y administrativo del universo de los universos.
\vs p015 13:6 Los administradores del sector menor están bajo la jurisdicción inmediata de los gobernantes del sector mayor. Los recientes de días reciben todos los informes de las observaciones y coordinan todas las recomendaciones, que se elevan al suprauniverso desde los uniones de días, emplazados como observadores y asesores de la Trinidad en las esferas sedes de los universos locales, y de los fieles de días, que de igual manera están asignados a los consejos de los altísimos localizados en las sedes centrales de las constelaciones. Todos estos informes se transmiten a los perfectos de días de los sectores mayores para que sean posteriormente remitidos a los tribunales de los ancianos de días. Así el régimen de la Trinidad se extiende desde las constelaciones de los universos locales hacia arriba hasta la sede central del suprauniverso. Las sedes centrales del sistema local no tienen representantes de la Trinidad.
\usection{14. PROPÓSITOS DE LOS SIETE SUPRAUNIVERSOS}
\vs p015 14:1 En la evolución de los siete suprauniversos se desvelan siete propósitos principales. Cada uno de ellos, trazado para la evolución del suprauniverso, hallará su más plena expresión únicamente en uno de los siete suprauniversos y, por tanto, cada suprauniverso tiene una función específica y una naturaleza singular.
\vs p015 14:2 Orvontón, el séptimo suprauniverso, al cual pertenece vuestro universo local, se conoce principalmente por su extraordinaria y generosa aportación de ministerio piadoso a los mortales de los mundos. Se le conoce por la forma en la que prevalece la justicia atemperada por la misericordia y por el modo en el que el gobierno rige condicionado por la paciencia, mientras que con libertad se realizan los sacrificios que el tiempo impone con el fin de asegurar la continuidad en la eternidad. Orvontón muestra al universo amor y misericordia.
\vs p015 14:3 Es muy difícil describir nuestra idea de la verdadera naturaleza del fin evolutivo que se desvela en Orvontón, pero se podría mostrar diciendo que en esta supracreación creemos que los seis propósitos singulares de la evolución cósmica, tal como se manifiestan en las seis supracreaciones que lo acompañan, se correlacionan aquí en un todo significativo; y es por esta razón por la que a veces creemos que el Dios Supremo, al desarrollar y completar su estado personal, gobernará, en el futuro remoto, desde Uversa, los siete suprauniversos perfeccionados en toda la majestad experiencial de su para entonces logrado omnipotente poder soberano.
\vs p015 14:4 Así como Orvontón es singular en naturaleza y único en cuanto a su destino, así también lo es cada uno de los seis suprauniversos que lo acompañan. Gran parte de lo que sucede en Orvontón, sin embargo, no se os revela, y de estos rasgos no revelados de la vida de Orvontón, muchos encontrarán una más completa expresión en algún otro suprauniverso. Los siete propósitos de la evolución del suprauniverso operan en todos los siete suprauniversos, pero cada supracreación ofrecerá la más plena expresión de tan solo uno de ellos. Para entender más sobre dichos propósitos, mucho de lo que vosotros no entendéis tendría que revelarse, pero incluso así poco comprenderíais. Toda esta narrativa no presenta sino un atisbo de la inmensa creación de la que vuestro mundo y sistema local forman parte.
\vs p015 14:5 \pc Vuestro mundo se llama Urantia, y es el número 606 en el conjunto planetario, o sistema, de Satania. Este sistema tiene actualmente 619 mundos habitados, y más de doscientos otros planetas evolucionan de forma propicia para en el futuro transformarse en mundos habitados.
\vs p015 14:6 Satania tiene como sede central un mundo llamado Jerusem, y es el sistema número veinticuatro de la constelación de Norlatiadec. Vuestra constelación, Norlatiadec, consta de cien sistemas locales y tiene como sede central un mundo llamado Edentia. Norlatiadec es la constelación número setenta del universo de Nebadón. El universo local de Nebadón consta de cien constelaciones y tiene una capital conocida como Lugar de Salvación. El universo de Nebadón es el número ochenta y cuatro del sector menor de Ensa.
\vs p015 14:7 El sector menor de Ensa consta de cien universos locales y tiene una capital llamada Umenor Tercero. Este sector menor es el número tres del sector mayor de Esplandón. Esplandón consta de cien sectores menores y tiene como sede central un mundo llamado Umayor Quinto. Es el quinto sector mayor del suprauniverso de Orvontón, el séptimo segmento del gran universo. Así pues, podéis ubicar vuestro planeta en el esquema de la organización y administración del universo de los universos.
\vs p015 14:8 En el gran universo, el número de vuestro mundo es 5\,342\,482\,337\,666. Ese es el número registrado en Uversa y en el Paraíso, vuestro número en el censo de mundos habitados. Conozco el número de registro de las esferas físicas, pero tiene unas dimensiones tan extraordinarias que sería de poca utilidad para la mente mortal.
\vs p015 14:9 \pc Vuestro planeta forma parte de un cosmos enorme; vosotros pertenecéis a un conjunto casi infinito de mundos, pero vuestra esfera se rige con igual precisión y se apoya con tanto amor como si fuese el único mundo habitado que existiera.
\vsetoff
\vs p015 14:10 [Exposición de un censor universal procedente de Uversa.]
