\upaper{83}{La institución del matrimonio}
\author{Jefe de los serafines}
\vs p083 0:1 Este es el relato de los primeros comienzos de la institución del matrimonio. Ha avanzado continuamente desde los emparejamientos desatados y promiscuos dentro de la horda, pasando por numerosas variaciones y adaptaciones, hasta la aparición de esas normas matrimoniales, que culminarían en la consecución de las uniones en pareja, el enlace de un hombre con una mujer para establecer un hogar del más elevado orden social.
\vs p083 0:2 El matrimonio ha estado en peligro, en múltiples ocasiones, y las costumbres matrimoniales se han basado, en gran parte, en la propiedad privada a la vez que en la religión; pero la verdadera influencia, que por siempre salvaguarda al matrimonio y su consiguiente familia, es el sencillo e innato hecho biológico de que los hombres y las mujeres realmente no pueden vivir los unos sin los otros, ya se trate de los salvajes más primitivos o de los mortales más cultivados.
\vs p083 0:3 Debido al impulso sexual, el hombre egoísta se ve llevado a superar el nivel animal. Las relaciones sexuales, egocéntricas y autogratificantes, traen como consecuencia la abnegación y garantizan que se asuman obligaciones de carácter altruista al igual que numerosas responsabilidades familiares, beneficiosas a su vez para la raza humana. De aquí que el sexo haya sido un civilizador desapercibido e insospechado del salvaje; porque este mismo impulso sexual de forma automática e inequívocamente \bibemph{obliga al hombre a pensar} y, en último término, \bibemph{lo conduce a amar}.
\usection{1. EL MATRIMONIO COMO INSTITUCIÓN DE LA SOCIEDAD}
\vs p083 1:1 El matrimonio es instrumento de la sociedad diseñado para regular y controlar esas múltiples relaciones humanas que nacen del hecho físico de la existencia de ambos sexos. Como tal institución, el matrimonio opera en dos direcciones:
\vs p083 1:2 \li{1.}En la regulación de las relaciones sexuales personales.
\vs p083 1:3 \li{2.}En la regulación de la descendencia, la herencia, la sucesión y el orden social, siendo esta su función inicial más antigua.
\vs p083 1:4 \pc La familia humana, que surge de la institución del matrimonio, es, en sí misma, un estabilizador de tal institución junto con las costumbres sobre la propiedad. Otros poderosos factores que influyen en la estabilidad matrimonial son el orgullo, la vanidad, la galantería, el deber y las convicciones religiosas. Pero, aunque los matrimonios puedan aprobarse o desaprobarse desde las alturas, no se disponen en el cielo. La familia es una institución claramente humana, un desarrollo evolutivo. El matrimonio es una institución de la sociedad, no un ministerio de la iglesia. Es verdad que la religión debe tener una poderosa influencia sobre esta institución, pero no debe atribuirse exclusivamente su control y regulación.
\vs p083 1:5 El matrimonio primitivo era fundamentalmente una actividad comercial; y hasta en los tiempos modernos es a menudo un asunto social o comercial. Por medio de la influencia de la mezcla del linaje andita y, como resultado de las costumbres de la civilización en avance, el matrimonio se está volviendo lentamente recíproco, romántico, paternal, poético, afectuoso, ético e incluso idealista. Sin embargo, la elección y el llamado amor romántico estaban a nivel mínimo en el emparejamiento primitivo. Durante los primeros tiempos, el marido y la mujer no pasaban mucho tiempo juntos; ni siquiera comían juntos con demasiada frecuencia. Si bien, entre los antiguos, el afecto personal no estaba ligado estrechamente a la atracción sexual; se encariñaban unos de otros principalmente por el hecho de vivir y trabajar juntos.
\usection{2. CORTEJO Y ESPONSALES}
\vs p083 2:1 Los matrimonios primitivos estaban siempre planeados por los padres del joven y la joven. La etapa de transición entre esta costumbre y la de la libre elección era el ámbito de los agentes matrimoniales o casamenteros profesionales. Al principio, estos casamenteros eran los barberos; más adelante, los sacerdotes. Originariamente, el matrimonio era asunto del grupo; después, un tema familiar; solo recientemente se ha convertido en una aventura personal.
\vs p083 2:2 La coacción, no la atracción, era el planteamiento del matrimonio primitivo. En los primeros tiempos, la mujer no tenía ninguna actitud sexual distante, sino solamente la inferioridad sexual que le inculcaban las costumbres. Así como el saqueo precedió al comercio, el matrimonio por captura precedió al matrimonio por contrato. Algunas mujeres contribuían a esta para escapar a la dominación de los hombres más viejos de su tribu; preferían caer en las manos de hombres de su propia edad aunque fuese de otra tribu. Estas fugas simuladas fue una etapa de transición entre la captura por la fuerza y el posterior cortejo por seducción.
\vs p083 2:3 Existía un tipo primitivo de ceremonia nupcial consistente en la imitación de una huida, una especie de simulacro de la fuga, que había sido antes práctica habitual. Más adelante, la captura fingida se convirtió en parte de la ceremonia ordinaria de la boda. La pretendida actitud de una joven moderna por resistirse a su “captura”, su reticencia hacia el matrimonio, son restos de estas viejas costumbres. La tradición de cruzar el umbral de la puerta con la novia en brazos es una reminiscencia de un buen número de tradiciones ancestrales, entre otras, la de esos días en los que se robaba a las esposas.
\vs p083 2:4 Durante mucho tiempo, se le negó a la mujer su plena libertad y disposición personal respecto al matrimonio, pero las mujeres más inteligentes han sabido siempre eludir esta restricción mediante el ejercicio de su astucia e ingenio. El hombre, por lo general, ha tomado la iniciativa en el cortejo, aunque no siempre. La mujer a veces formalmente, al igual que de forma encubierta, da inicio al matrimonio. Y, a medida que la civilización ha avanzado, las mujeres han ido asumiendo una parte cada vez más activa en todas las facetas del cortejo y del matrimonio.
\vs p083 2:5 El amor, el romance y la opción personal crecientes del cortejo prematrimonial constituyen una aportación de los anditas a las razas del mundo. Las relaciones entre los sexos van evolucionando de forma favorable; en su avance, numerosos pueblos están sustituyendo paulatinamente las motivaciones más antiguas de utilidad y propiedad por concepciones hasta cierto punto idealizadas de la atracción sexual. El impulso sexual y la afectividad están comenzando a desplazar a la fría premeditación en la elección de compañeros de vida.
\vs p083 2:6 Inicialmente, los esponsales equivalían al matrimonio; y entre los pueblos primitivos, las relaciones sexuales eran algo habitual durante el período de compromiso. En tiempos recientes, la religión ha establecido un tabú sexual sobre ese período comprendido entre los esponsales y el casamiento.
\usection{3. COMPRA Y DOTE}
\vs p083 3:1 Los antiguos desconfiaban del amor y de las promesas; creían que las uniones duraderas debían estar garantizadas por algo tangible: por la propiedad privada. Por esta misma razón, el precio de compra de una esposa se consideraba como un bien o un depósito que el marido perdería en caso de divorcio o abandono. Una vez que se había realizado la compra de una novia, muchas tribus permitían que la marca del marido se grabase a fuego en ella. Los africanos aún compran a sus esposas. Comparan a una esposa por amor, o la esposa del hombre blanco, con un gato porque no cuesta nada.
\vs p083 3:2 Las presentaciones de las novias constituían la ocasión de vestir elegantemente y ornamentar a las hijas para su exhibición pública, con la idea de conseguir precios más altos por ellas como esposas. Pero no se vendían como animales ---entre las tribus posteriores, las esposas no eran transferibles---. Tampoco su compra era siempre una fría transacción monetaria; en la adquisición de una esposa, el servicio era equivalente a dinero en efectivo. Si un candidato, por otra parte conveniente, no podía pagar por su esposa, el padre de la muchacha podía adoptarlo como hijo y luego contraer matrimonio. Y si un hombre pobre buscaba esposa y no podía afrontar el precio exigido por el codicioso padre, con frecuencia los ancianos lo presionaban para que enmendara sus exigencias, o de lo contrario la pareja podría fugarse.
\vs p083 3:3 Conforme la civilización avanzaba, los padres ya no deseaban parecer que vendían a sus hijas, así pues, aunque seguían aceptando el precio por la compra de la novia, introdujeron la costumbre de dar a la pareja unos obsequios valiosos que aproximadamente equivalían al coste de la adquisición. Más adelante, cuando desapareció la costumbre de pagar por la novia, estos obsequios se convirtieron en su dote.
\vs p083 3:4 La idea de la dote pretendía dar la impresión de que la novia era independiente, dando a suponer que se estaba bastante lejos de los tiempos en los que las esposas eran esclavas o una propiedad. El hombre no podía divorciar a su esposa con dote sin restituírsela por completo. Entre algunas tribus, los padres de la novia y el novio hacían respectivamente un depósito, que se perdía en caso de que alguno de ellos abandonara al otro; era en realidad una fianza matrimonial. Durante el período de transición de la compra a la dote, si se compraba a la esposa, los hijos pertenecían al padre; si no, pertenecían a la familia de la madre.
\usection{4. LA CEREMONIA DE LA BODA}
\vs p083 4:1 La ceremonia nupcial surgió a partir del hecho de que el matrimonio era originariamente un asunto de la comunidad, no solo el punto culminante en la decisión de dos personas. El emparejamiento era interés del grupo, al mismo tiempo que una cuestión personal.
\vs p083 4:2 \pc Toda la vida de los antiguos estaba rodeada de magia, rituales y ceremonias, y el matrimonio no era una salvedad. A medida que avanzó la civilización, a medida que el matrimonio llegó a considerarse con mayor seriedad, la ceremonia de la boda se volvió cada vez más pretenciosa. El matrimonio primitivo era un bien con intereses patrimoniales, tal como incluso lo es hoy, por lo que requería una ceremonia legal, mientras que la posición social de los hijos futuros exigía la más amplia difusión. El hombre primitivo no dejaba constancia de la ceremonia nupcial, por lo que debía presenciarse por un gran número de personas.
\vs p083 4:3 Al principio, la ceremonia de la boda era más bien del orden de unos esponsales y consistía en la notificación pública de la intención de convivir; después, consistió en una comida formal que hacían juntos. Entre algunas tribus, los padres simplemente llevaban a su hija al marido; en otros casos, la única ceremonia era el intercambio oficial de obsequios, tras lo cual, el padre de la novia la entregaba al novio. Entre muchos pueblos levantinos, era costumbre prescindir de todas las formalidades; el matrimonio se consumaba mediante las relaciones sexuales. El hombre rojo fue el primero en desarrollar la celebración más elaborada de las bodas.
\vs p083 4:4 \pc La infecundidad era muy temida y, puesto que se atribuía la esterilidad a las maquinaciones de los espíritus, las iniciativas que se tomaban para asegurar la fecundidad también llevaron a vincular la boda con ciertos ceremoniales mágicos o religiosos. Y, en este intento por garantizar un matrimonio feliz y fértil, se empleaban muchos encantos; incluso se consultaba con los astrólogos para determinar las estrellas de nacimiento de las partes contrayentes. En cierta época, en todas las bodas de gente acomodada era habitual el sacrificio humano.
\vs p083 4:5 Se buscaban los días de suerte; el jueves se consideraba el más favorable, y se pensaba que las bodas celebradas en luna llena eran extraordinariamente afortunadas. Era costumbre en muchos pueblos de Oriente Próximo arrojar grano sobre los recién casados; se trataba de un rito mágico que se suponía aseguraba la fecundidad. Algunos pueblos orientales usaban arroz a tal efecto.
\vs p083 4:6 Se creyó siempre que el fuego y el agua eran los mejores medios para resistirse a los espectros y a los espíritus malignos; de ahí que los fuegos del altar y las velas encendidas, al igual que la aspersión bautismal de agua bendita, estaban normalmente presentes en las bodas. Durante mucho tiempo, se acostumbró a fijar un día de boda falso y luego posponer el acontecimiento de forma repentina para desorientar a los espectros y espíritus.
\vs p083 4:7 Las burlas que se hacen a los recién casados y las bromas que se les gastan en su luna de miel son reliquias de esos remotos días en los que se creía que era mejor parecer desdichados e incómodos a la vista de los ojos de los espíritus para evitar así despertar su envidia. El uso del velo nupcial es un vestigio de los tiempos en los que se consideraba necesario disfrazar a la novia para que los espectros no la reconociesen y ocultar al mismo tiempo su belleza de la mirada, por otra parte celosa y envidiosa de los espíritus. Los pies de la novia no deben tocar el suelo justo antes de la ceremonia. Incluso en el siglo XX sigue siendo habitual, siguiendo las costumbres cristianas, de extender alfombras desde el vehículo en el que se llegaba hasta el altar de la iglesia.
\vs p083 4:8 Una de las formas más antiguas de la ceremonia nupcial era contar con un sacerdote que bendijera el lecho nupcial para garantizar la fertilidad de la unión; esto se hacía mucho antes de que se estableciese un rito nupcial formal. Durante este período de la evolución de las costumbres matrimoniales, se esperaba que los invitados a la boda desfilaran de noche por la cámara nupcial para testimoniar legalmente la consumación del matrimonio.
\vs p083 4:9 El factor suerte que hacía que, a pesar de todas las pruebas prenupciales, algunos matrimonios saliesen mal, llevó al hombre primitivo a buscar un seguro de protección contra este fracaso, recurriendo a sacerdotes y a la magia. Y esta tendencia culminó directamente en las modernas bodas eclesiales. Pero, durante mucho tiempo, se dio por sentado que el matrimonio consistía en el acuerdo de los padres contratantes ---después de la pareja---, mientras que, durante los últimos quinientos años, la Iglesia y el Estado han asumido dicha competencia y se toman la libertad de emitir pronunciamientos respecto al matrimonio.
\usection{5. MATRIMONIOS MÚLTIPLES}
\vs p083 5:1 En la historia primitiva del matrimonio, las mujeres solteras pertenecían a los hombres de la tribu. Más tarde, las mujeres solo tendrían un marido a la vez. Este hábito de \bibemph{un solo hombre a la vez} fue el primer paso que se dio para dejar atrás la promiscuidad de la horda. Aunque a la mujer tan solo se le permitía un hombre, su marido podía romper a voluntad estas relaciones temporales. Pero estas uniones tan escasamente reguladas significaron el primer paso hacia una vida en pareja en lugar de una vida en la horda. Durante esta etapa del desarrollo del matrimonio, los hijos pertenecían por lo general a la madre.
\vs p083 5:2 El paso siguiente en la evolución del emparejamiento fue el \bibemph{matrimonio colectivo}. Esta faceta comunal del matrimonio tuvo que ocurrir durante el desarrollo de la vida familiar, porque las costumbres matrimoniales no eran lo suficientemente sólidas como para hacer que las uniones de pareja fuesen permanentes. Los matrimonios entre hermanos pertenecían a este grupo; cinco hermanos de una familia se casaban con cinco hermanas de otra. En todo el mundo, las formas más laxas de las bodas comunales evolucionaron paulatinamente hasta convertirse en distintos tipos de matrimonios colectivos. Y estas uniones grupales se regulaban, en gran parte, por las costumbres en torno al tótem. La vida familiar se desarrolló de forma lenta y segura porque la regulación del matrimonio y del sexo propició la supervivencia de la correspondiente tribu, al asegurar la supervivencia de un gran número de hijos.
\vs p083 5:3 Poco a poco, los matrimonios colectivos dieron paso a los incipientes hábitos de la poligamia ---poliginia y poliandria--- entre las tribus más avanzadas. Pero la poliandria nunca se generalizó; solía circunscribirse a las reinas y a las mujeres ricas. Además, era normalmente un asunto de familia: una esposa para varios hermanos. Las restricciones económicas y de casta hicieron a veces necesario que diversos hombres se conformaran con una sola esposa. Incluso entonces, la mujer solo se casaba con uno solo; los demás se aceptaban vagamente como “tíos” de la progenie conjunta.
\vs p083 5:4 La costumbre judía de exigir que el hombre se juntase con la viuda de su hermano muerto con el fin de que “levantara descendencia a su hermano” fue también habitual en más de la mitad del mundo antiguo. Se trataba de un vestigio del tiempo en el que el matrimonio era un asunto familiar más que un acuerdo de orden individual.
\vs p083 5:5 La institución de la poliginia reconocía, en distintas épocas, cuatro clases de esposas:
\vs p083 5:6 \li{1.}Las esposas ceremoniales o legales.
\vs p083 5:7 \li{2.}Las esposas de afecto y consentimiento.
\vs p083 5:8 \li{3.}Las concubinas, o esposas contractuales.
\vs p083 5:9 \li{4.}Las esposas esclavas.
\vs p083 5:10 \pc La verdadera poliginia, en la que todas las esposas eran del mismo rango y todos los hijos iguales, ha sido escasamente común. Por lo general, incluso en el caso de los matrimonios múltiples, la esposa predominante, o compañera formal, era preponderante en el hogar. Solo ella merecía una ceremonia ritual de boda y solo los hijos de esta esposa, comprada o con dote, podían heredar, salvo que existiese algún acuerdo especial con ella.
\vs p083 5:11 La esposa formal no era necesariamente la esposa amada; en los tiempos primitivos no era lo habitual. La esposa por amor, o enamorada, no surgió hasta que las razas no avanzaron de forma considerable, más concretamente después de la unión de las tribus evolutivas con los noditas y adanitas.
\vs p083 5:12 La esposa tabú ---la única esposa con estatus legal--- dio origen a las costumbres de poseer concubinas. Bajo dichas costumbres, un hombre podía tener solamente una esposa, si bien, podía mantener relaciones sexuales con un gran número de concubinas. El concubinato fue el peldaño que llevó hasta la monogamia, el primer paso para alejarse de la poliginia real. Las concubinas de los judíos, los romanos y los chinos eran, con bastante frecuencia, las sirvientas de la esposa. Más adelante, como en el caso de los judíos, se consideraba a la esposa legal la madre de todos los hijos del marido.
\vs p083 5:13 Los antiguos tabúes sobre las relaciones sexuales con una esposa embarazada o amamantando tendieron en gran medida a fomentar la poliginia. Las mujeres primitivas envejecían muy pronto debido a su frecuente maternidad unida al duro trabajo. (Estas esposas sobrecargadas solo se las arreglaban para existir por el hecho de que se las aislaba una semana al mes cuando no estaban encinta.) Estas mujeres a menudo se cansaban de tener hijos y pedían a sus maridos que tomaran una segunda esposa, más joven, que fuese capaz de ayudar tanto a la maternidad como en el trabajo doméstico. Las nuevas esposas, por lo tanto, eran normalmente acogidas con entusiasmo por las esposas mayores; no existía nada parecido a los celos sexuales.
\vs p083 5:14 El número de esposas estaba únicamente limitado por las posibilidades del hombre para mantenerlas. Los hombres ricos y capacitados querían un gran número de hijos, y, puesto que la mortalidad infantil era muy elevada, se precisaba un grupo de esposas para conseguir una familia grande. Muchas de estas esposas múltiples eran simples trabajadoras, esposas esclavas.
\vs p083 5:15 Las costumbres humanas evolucionan, aunque lo hacen con mucha lentitud. La finalidad del harén era crear un conjunto, numeroso y fuerte, de parientes de sangre que apoyaran el trono. Cierta vez, se persuadió a un determinado jefe para que no tuviera un harén, que debería contentarse con una sola esposa; así pues, rápidamente se deshizo de su harén. Las esposas, descontentas, volvieron a sus casas, y sus parientes, ofendidos, corrieron hacia el jefe enfurecidos y acabaron con él de inmediato.
\usection{6. LA VERDADERA MONOGAMIA: EL MATRIMONIO EN PAREJA}
\vs p083 6:1 La monogamia es un privilegio exclusivista; es buena para aquellos que logran tan deseable estado, pero tiende a actuar como impedimento biológico para los que no son tan afortunados. Si bien, con total independencia del efecto sobre el individuo, la monogamia es sin duda lo mejor para los hijos.
\vs p083 6:2 La monogamia más primitiva se debió a la fuerza de las circunstancias: a la pobreza. La monogamia es cultural y social, artificial e innatural, esto es, innatural para el hombre evolutivo. Era totalmente natural para los noditas y adanitas más puros, y ha sido de un gran valor cultural para todas las razas avanzadas.
\vs p083 6:3 Las tribus caldeas reconocían el derecho de la esposa a imponer al marido una promesa prenupcial de que no tomaría una segunda esposa ni una concubina; tanto los griegos como los romanos propiciaron el matrimonio monógamo. El culto a los ancestros siempre ha fomentado la monogamia, al igual que el cristianismo al considerar, equivocadamente, el matrimonio como un sacramento. Incluso la elevación del nivel de vida ha desaconsejado invariablemente la multiplicidad de esposas. Hacia el tiempo de la llegada de Miguel a Urantia, prácticamente la totalidad del mundo civilizado había llegado a una monogamia teórica. Pero esta monogamia pasiva no significaba que la humanidad se hubiese habituado a la práctica del verdadero matrimonio en pareja.
\vs p083 6:4 \pc Al tiempo que se trata de lograr la monogamia como el ideal del matrimonio en pareja, que es, en definitiva, una especie de relación sexual monopolizadora, la sociedad no debe pasar por alto la nada envidiable condición de esos desdichados hombres y mujeres que no consiguen encontrar su lugar en este orden social nuevo y mejorado, incluso cuando han hecho todo lo posible por asumir y cooperar con sus exigencias. El hecho de no encontrar una pareja en el competitivo entorno social puede deberse a dificultades insuperables o a las innumerables restricciones que las costumbres vigentes imponen. Realmente, la monogamia es ideal para aquellos que están insertos en ella, pero inevitablemente debe provocar graves dificultades a aquellos que quedan abandonados al frío de una solitaria existencia.
\vs p083 6:5 Siempre ha habido un pequeño número de personas desafortunadas que han tenido que sufrir el avance de la mayoría bajo las costumbres en desarrollo de la civilización evolutiva; pero la mayoría favorecida debería continuamente mirar con amabilidad y consideración a aquellos semejantes menos afortunados que han de pagar el precio por no poder unirse a las filas de esas compañías sexuales ideales, que permiten la satisfacción de todos los impulsos biológicos bajo la aprobación de las costumbres más elevadas de una evolución social en avance.
\vs p083 6:6 \pc La monogamia siempre ha sido, es y será por siempre la meta ideal de la evolución sexual humana. Este ideal del verdadero matrimonio en pareja significa renunciamiento y, por consiguiente, fracasa, con tanta frecuencia, precisamente porque una o ambas de las partes contrayentes son deficientes en el súmmum de las virtudes humanas: el fuerte dominio de sí mismo.
\vs p083 6:7 La monogamia es el rasero por el que se mide el avance de la civilización social, que contrasta con la evolución puramente biológica. La monogamia no es necesariamente biológica o natural, pero es indispensable para el mantenimiento inmediato y el desarrollo ulterior de la civilización social. Contribuye a la delicadeza de los sentimientos, al refinamiento del carácter moral y a un crecimiento espiritual que son absolutamente imposibles en la poligamia. La mujer nunca podrá convertirse en una madre ideal si se ve todo el tiempo obligada a rivalizar por el afecto de su marido.
\vs p083 6:8 El matrimonio en pareja favorece y fomenta ese íntimo entendimiento y esa cooperación verdadera que es más conveniente para la felicidad de los padres, el bienestar de los hijos y la eficiencia social. El matrimonio, que comenzó como ruda imposición, va evolucionando gradualmente en una magnífica institución cultural de autocontrol, de expresión y perpetuación de uno mismo.
\usection{7. DISOLUCIÓN DEL MATRIMONIO}
\vs p083 7:1 En la temprana evolución de las costumbres maritales, el matrimonio era una unión indefinida que se podía terminar a voluntad, y los hijos siempre seguían a la madre; el vínculo madre\hyp{}hijo es instintivo y ha obrado sin relación alguna con el estadio del desarrollo de las costumbres.
\vs p083 7:2 Entre los pueblos primitivos, solo alrededor de la mitad de los matrimonios resultaban ser satisfactorios. La causa más frecuente de la separación era la esterilidad, de la que se culpaba siempre a la esposa; y se creía que las esposas sin hijos se convertían en serpientes en el mundo de los espíritus. Bajo las costumbres más primitivas, el divorcio era únicamente opción del hombre, y estas normas han persistido hasta el siglo XX entre algunos pueblos.
\vs p083 7:3 Conforme las costumbres evolucionaron, determinadas tribus desarrollaron dos formas de matrimonio: el ordinario, que permitía el divorcio, y el sacerdotal, que no lo permitía. El comienzo de la compra de las esposas y de sus dotes, al introducir una penalización de los bienes por fracaso del matrimonio, contribuyó bastante a reducir las separaciones. Y de hecho, muchas uniones modernas se estabilizan por medio de este ancestral elemento patrimonial.
\vs p083 7:4 La presión social de la situación dentro de la comunidad y de los privilegios patrimoniales siempre ha sido una fuerza poderosa en el mantenimiento de los tabúes y de las costumbres matrimoniales. A lo largo de las eras, el matrimonio ha estado progresando de forma continua y se encuentra en una posición avanzada en el mundo moderno, pese a estar amenazadoramente asediado por la generalizada insatisfacción de esos pueblos en los que la opción individual ---una nueva libertad--- resulta de suma importancia. Aunque estos trastornos de adaptación aparecen entre las razas más progresivas como consecuencia de una evolución social acelerada repentinamente, entre los pueblos menos avanzados, el matrimonio continúa prosperando y perfeccionándose lentamente bajo las directrices de las viejas costumbres.
\vs p083 7:5 La sustitución, nueva y repentina, del más idealista, pero extremadamente individualista, motivo del amor en el matrimonio, en lugar del motivo patrimonial, más antiguo y largamente establecido, ha provocado de forma inevitable que la institución del matrimonio se vuelva temporalmente inestable. Las razones del hombre para casarse siempre han ido mucho más allá de los verdaderos principios morales del matrimonio; en los siglos XIX y XX, el ideal occidental del matrimonio ha dejado repentinamente bastante atrás a los impulsos sexuales egocéntricos de las razas, aunque se hayan controlado de forma parcial. La presencia en cualquier sociedad de un gran número de solteros indica o bien el colapso transitorio de las costumbres o bien una etapa de transición en ellas.
\vs p083 7:6 A lo largo de las eras, la verdadera prueba del matrimonio ha sido esa intimidad continuada ineludible en toda vida familiar. Dos jóvenes, mimados y malcriados, educados para esperar todo tipo de complacencia y gratificación plena de su vanidad y ego, difícilmente pueden abrigar esperanzas de tener buenos resultados en su matrimonio y en la construcción de un hogar ---una alianza permanente que implica modestia, compromiso, devoción y dedicación desinteresada a la formación de los hijos---.
\vs p083 7:7 El alto grado de imaginación y de formidable romance que forman parte del cortejo es, en gran medida, responsable por el aumento de las tendencias hacia el divorcio de los pueblos occidentales modernos, todo lo cual se complica incluso más por la mayor libertad personal de la mujer y el aumento de su libertad económica. El divorcio fácil, cuando resulta de la falta de autocontrol o de un fallo de adaptación normal de la persona, solo lleva directamente de vuelta a esas rudas etapas de la sociedad de las que el hombre ha surgido tan recientemente, como resultado de tanta angustia personal y sufrimiento racial.
\vs p083 7:8 No obstante, mientras que la sociedad no consiga educar debidamente a sus hijos y a su juventud, mientras que el orden social no logre proporcionar una adecuada formación prematrimonial y mientras que el idealismo juvenil, insensato e inmaduro, sea el árbitro para acceder al matrimonio, el divorcio seguirá siendo frecuente durante mucho tiempo. Y, tanto en cuanto el grupo social no alcance a ofrecer una buena preparación matrimonial para sus jóvenes, el divorcio debe actuar como una válvula de seguridad social hasta el punto de prevenir situaciones, aun peores, durante las eras del rápido crecimiento de las costumbres en evolución.
\vs p083 7:9 \pc Los antiguos parecen haber considerado el matrimonio con tanta seriedad como algunos de los pueblos de hoy en día. Y muchos de los matrimonios apresurados y fallidos de los tiempos modernos no parecen representar mejora alguna respecto a las antiguas prácticas de formar a los jóvenes y las jóvenes para el emparejamiento. La gran incongruencia de la sociedad moderna consiste en enaltecer el amor e idealizar el matrimonio, mientras que desaprueba llevar a cabo un examen a fondo de ambos.
\usection{8. IDEALIZACIÓN DEL MATRIMONIO}
\vs p083 8:1 El matrimonio, que tiene su punto culminante en el hogar, es, en efecto, la institución más excelsa del hombre, pero es esencialmente humana; nunca se le debería haber tildado de sacramento. Los sacerdotes setitas hicieron del matrimonio un ritual religioso; pero, durante miles de años después de Edén, el emparejamiento continuó siendo una institución puramente social y civil.
\vs p083 8:2 Es lamentable comparar los vínculos humanos con los divinos. La unión de marido y mujer en la relación matrimonio\hyp{}hogar es un acto material de los mortales de los mundos evolutivos. Es cierto, de hecho, que se puede derivar un gran progreso espiritual como consecuencia del esfuerzo humano sincero del marido y la esposa por avanzar, pero esto no significa que el matrimonio sea necesariamente sagrado. El desarrollo espiritual comporta una dedicación auténtica hacia otras parcelas de la actividad humana.
\vs p083 8:3 En verdad, tampoco se puede comparar el matrimonio con la relación del modelador con el hombre ni con la fraternidad de Cristo Miguel hacia sus hermanos humanos. Estas relaciones son difícilmente comparables, en modo alguno, con la vinculación del marido con la esposa. Y es muy desafortunado que la equivocada noción humana de esta relación haya producido tanta confusión en cuanto al estatus del matrimonio.
\vs p083 8:4 También es de lamentar que determinados grupos de mortales hayan pensado que el matrimonio se lleva a efecto por mediación divina. Estas creencias llevan directamente al concepto de la indisolubilidad del estado marital, pese a las circunstancias o a los deseos de las partes contrayentes. Pero el mero hecho de que el matrimonio se pueda disolver indica que la Deidad no participa de estas uniones. Una vez que Dios hubiese unido dos cosas o dos personas, estas permanecerían pues unidas hasta el momento en que la voluntad divina decretara su separación. Pero, con respecto al matrimonio, como institución humana que es, ¿quién se puede atrever a emitir un juicio y decir cuál de ellos son uniones que los supervisores del universo aprobarían a diferencia de aquellas otras que son puramente humanas en naturaleza y origen?
\vs p083 8:5 Sin embargo, existe un ideal del matrimonio en las esferas de lo alto. En la capital de cada sistema local, los hijos y las hijas materiales de Dios son un ejemplo del apogeo de los ideales de la unión entre hombre y mujer en los vínculos matrimoniales y con el fin de procrear y criar una progenie. En definitiva, el matrimonio ideal de los mortales es humanamente sagrado.
\vs p083 8:6 \pc El matrimonio siempre ha sido, y aún es, el sueño supremo de la idealidad temporal del hombre. Aunque este hermoso sueño raras veces llega a cumplirse en su totalidad, perdura como un ideal glorioso, siempre atrayendo a la humanidad en su avance hacia la realización de mayores esfuerzos por lograr la felicidad humana. Pero se ha de enseñar a los jóvenes, hombres y mujeres, algo sobre las realidades del matrimonio, antes de que se vean inmersos en las rigurosas exigencias de las interrelaciones de la vida familiar; la idealización juvenil debe atenuarse con un cierto grado de desencanto prenupcial.
\vs p083 8:7 Sin embargo, no se debe desalentar la idealización juvenil del matrimonio; tales sueños son la plasmación de la meta futura de una vida familiar. Esta actitud es estimulante y útil, siempre que no produzca falta de sensibilidad frente a la consecución de los requisitos prácticos y habituales del matrimonio y de la posterior vida familiar.
\vs p083 8:8 Los ideales del matrimonio han progresado mucho en los últimos tiempos; en algunos pueblos, la mujer disfruta prácticamente de los mismos derechos que su consorte. En teoría, al menos, la familia se está convirtiendo en una fiel alianza para criar a los hijos, junto a la lealtad sexual. Pero incluso esta forma más reciente del matrimonio no debe presuponer una fluctuación extrema hasta concederse un control exclusivo mutuo de toda la otra persona y de su individualidad. El matrimonio no es solo un ideal individualista; es la unión social evolutiva de un hombre y una mujer, que existe y actúa bajo las costumbres vigentes, que está limitada por los tabúes y que opera bajo las leyes y las regulaciones de la sociedad.
\vs p083 8:9 Los matrimonios del siglo XX tienen una posición elevada en comparación con los de esas épocas anteriores, pese a que la institución del hogar esté experimentando en estos momentos una dura prueba, debido a los problemas a los que se ha visto repentinamente sometida la organización social por el rápido aumento de las libertades de la mujer, unos derechos que durante tanto tiempo les fueron negados en la tardía evolución de las costumbres de las generaciones pasadas.
\vsetoff
\vs p083 8:10 [Exposición del jefe de los serafines emplazado en Urantia.]
