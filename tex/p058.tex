\upaper{58}{Establecimiento de la vida en Urantia}
\author{Portador de vida}
\vs p058 0:1 En todo Satania solo hay sesenta y un mundos similares a Urantia, esto es, planetas donde se han realizado modificación de la vida. La mayoría de los mundos habitados se pueblan según métodos establecidos; en estas esferas, a los portadores de vida se les da poco margen de acción en relación a sus planes para la implantación de la vida. Si bien, aproximadamente a un mundo de cada diez se le designa como \bibemph{planeta decimal} y se le inscribe en el registro especial de los portadores de vida; y, en tales planetas, se nos permite llevar a cabo ciertos experimentos a fin de modificar o posiblemente mejorar los órdenes habituales de seres vivos del universo.
\usection{1. CONDICIONES PREVIAS PARA LA VIDA FÍSICA}
\vs p058 1:1 Hace \bibemph{600\,000\,000} de años, la comisión de portadores de vida enviada desde Jerusem llegó a Urantia y empezó el estudio de las condiciones físicas previas a la puesta en marcha de la vida en el mundo número 606 del sistema de Satania. Esta iba a ser nuestra actuación seiscientos seis en cuanto a dar inicio a los modelos de vida de Nebadón en Satania y nuestra sexagésima oportunidad de realizar cambios y establecer modificaciones en los diseños de vida básicos y ordinarios del universo local.
\vs p058 1:2 \pc Debe quedar claro que los portadores de vida no pueden iniciar la vida hasta que la esfera no esté lista para comenzar su ciclo evolutivo. Tampoco podemos proporcionar un desarrollo vital más rápido que aquel que el progreso físico del planeta puede sostener y albergar.
\vs p058 1:3 Los portadores de vida de Satania habían diseñado un modelo de vida compuesto de cloruro de sodio; por lo tanto, no se podía tomar ninguna medida respecto a su implantación hasta que las aguas del océano no se volvieran suficientemente salobres. El tipo urantiano de protoplasma puede solamente actuar en una adecuada solución salina. Toda vida ancestral ---vegetal y animal--- evolucionó en un hábitat de este tipo de solución. E incluso los animales terrestres más altamente organizados no podrían continuar viviendo de no circular esta misma solución salina esencial por todo su cuerpo en el torrente circulatorio, que generosamente baña cada minúscula célula viva, sumergiéndola en estas “profundidades salobres”.
\vs p058 1:4 Vuestros antepasados primitivos circulaban con libertad por el océano salado; hoy en día, esta misma solución salada circula a modo de océano, sin restricción, por vuestro cuerpo, bañando cada una de las células con un líquido químico comparable, fundamentalmente, al agua salada que suscitó en el planeta las primeras reacciones protoplásmicas de las primeras células vivas.
\vs p058 1:5 Si bien, al inaugurarse esta era, Urantia está, en todos los sentidos, evolucionando hacia condiciones favorables para el mantenimiento de las formas iniciales de vida marina. Lenta, pero claramente, los avances físicos en la Tierra y en las regiones adyacentes del espacio van preparando el escenario para intentos adicionales destinados a implantar esas formas de vida, que, como habíamos determinado, se adaptarían mejor al medio ambiente físico que se desplegaba tanto en el ámbito terrestre como en el espacial.
\vs p058 1:6 Tras ello, la comisión de Satania de portadores de vida regresó a Jerusem, prefiriendo aguardar a que la masa de tierra continental se disgregara nuevamente, lo cual proporcionaría todavía más mares interiores y bahías resguardadas, antes de empezar realmente a implantar la vida.
\vs p058 1:7 \pc En un planeta donde la vida tiene un origen marino, las condiciones ideales para la implantación de la vida vienen de la mano de un gran número de mares interiores, de extensas líneas costeras de aguas poco profundas y de bahías resguardadas; y dicha distribución de las aguas de la Tierra se desarrollaba rápidamente. Estos antiguos mares interiores raramente tenían una profundidad de más de ciento cincuenta o ciento ochenta metros, y la luz del Sol puede penetrar en el agua oceánica hasta algo más de ciento ochenta metros de profundidad.
\vs p058 1:8 Y fue en esas costas de clima moderado y estable, de una era posterior, donde la vida vegetal primitiva se adentró en la tierra. Allí, el alto grado de carbono de la atmósfera proporcionó, a las nuevas variedades de vida terrestre, la oportunidad de crecer de forma rápida y exuberante. Aunque esta atmósfera era entonces ideal para el crecimiento de las plantas, contenía tan alto grado de dióxido de carbono que ningún animal, y mucho menos el hombre, podría haber vivido sobre la faz del mundo.
\usection{2. LA ATMÓSFERA DE URANTIA}
\vs p058 2:1 La atmósfera planetaria filtra a la Tierra alrededor de dos mil millonésimas partes de la emanación total de la luz del Sol. Si la luz que cae sobre América del Norte se pagara a razón de dos céntimos por kilovatio hora, la factura anual de la luz ascendería a 800 mil billones de dólares. La factura por la luz solar de Chicago tendría un importe bastante superior a los 100 millones de dólares diarios. Y convendría recordar que recibís del Sol otras formas de energía ---la luz no es la única aportación del Sol que llega a vuestra atmósfera---. Se vierte sobre Urantia una inmensa cantidad de energía solar, abarcando longitudes de onda que oscilan entre las que están por encima del alcance de la visión humana hasta las que están por debajo.
\vs p058 2:2 \pc La atmósfera de la Tierra es casi completamente opaca para una gran parte de la radiación del extremo ultravioleta del espectro. La mayoría de estas longitudes de onda corta se absorben por la capa de ozono situada a unos dieciséis kilómetros de altitud por encima de la superficie, y que se extiende otros dieciséis kilómetros hacia el espacio. En las condiciones que prevalecen en la superficie de la Tierra, el ozono que impregna esta región formaría una capa de solo dos milímetros y medio de grosor; no obstante, esta cantidad de ozono relativamente pequeña y, al parecer, insignificante, protege a los habitantes de Urantia del exceso de estas peligrosas y destructivas radiaciones ultravioletas presentes en la luz del Sol. Pero, si esta capa de ozono fuese solo algo más densa, os veríais privados de esos rayos violetas, tan importantes y saludables, que actualmente alcanzan la superficie de la Tierra, y que son la fuente ancestral de una de vuestras vitaminas más esenciales.
\vs p058 2:3 No obstante, algunos de vuestros mecanicistas mortales menos imaginativos insisten en percibir la creación material y la evolución humana como producto del azar. Los seres intermedios de Urantia han reunido más de cincuenta mil hechos físicos y químicos que estiman son incompatibles con las leyes de lo accidentalmente fortuito; hechos que, según sostienen, demuestran de manera inequívoca la presencia de un propósito inteligente en la creación material. Todo lo cual no tiene en cuenta su catálogo de más de cien mil hallazgos ajenos al ámbito de la física y de la química que, según afirman, prueban la presencia de una mente en la planificación, creación y mantenimiento del cosmos material.
\vs p058 2:4 Vuestro Sol derrama verdaderos aluviones de rayos mortíferos, y vuestra placentera vida en Urantia se debe al efecto “fortuito” de más de dos veintenas de actuaciones protectoras, aparentemente accidentales, similares a la acción de esta singular capa de ozono.
\vs p058 2:5 A no ser por el efecto “manta” de la atmósfera durante la noche, el calor se perdería por la radiación con tanta rapidez que la vida sería imposible de mantener excepto acudiendo a medidas de índole artificial.
\vs p058 2:6 \pc Los ocho o diez kilómetros de la parte inferior de la atmósfera terrestre conforman la troposfera o región de los vientos y de las corrientes de aire que producen los fenómenos meteorológicos. Por encima de esta región, está la ionosfera interior y, a continuación, por encima, está la estratosfera. Ascendiendo desde la superficie de la Tierra, la temperatura desciende de forma continuada durante diez o trece kilómetros, a cuya altura se registran alrededor de 57 grados (C) bajo cero. Esta franja de temperatura que se extiende desde unos 54 a 57 grados (C) bajo cero permanece invariable al ascender otros sesenta y cuatro kilómetros; este sector de temperatura constante es la estratosfera. A una altura de setenta y dos o de ochenta kilómetros, la temperatura empieza a elevarse, y este incremento continúa hasta que, al nivel de los despliegues aurorales, se alcanza una temperatura de 650 grados (C), y es este calor intenso el que ioniza el oxígeno. Pero la temperatura en una atmósfera tan enrarecida es difícilmente comparable con la medición del calor en la superficie de la Tierra. Tened en cuenta que la mitad de toda vuestra atmósfera se halla en los primeros cinco kilómetros. Las cintas de luz más altas de la aurora boreal ---unos seiscientos cuarenta kilómetros--- dan muestra de la altura de la atmósfera de la Tierra.
\vs p058 2:7 Los fenómenos de las auroras están directamente relacionados con las manchas solares, con esos ciclones solares que giran en sentidos opuestos por encima y por debajo del ecuador del Sol, al igual que los huracanes tropicales terrestres. Tales perturbaciones atmosféricas giran en sentidos opuestos según se produzcan por encima o por debajo del ecuador.
\vs p058 2:8 La capacidad de las manchas solares para alterar las frecuencias de la luz demuestra que estos centros de tormentas solares operan como enormes imanes. Tales campos magnéticos son capaces de lanzar partículas cargadas desde los cráteres de las manchas solares, a través del espacio, hasta la atmósfera exterior de la Tierra, en donde su influencia ionizante produce los espectaculares despliegues aurorales. Así pues, tenéis los mayores fenómenos de las auroras cuando las manchas solares están en su apogeo ---o poco después---, momento en el que estas se sitúan generalmente cercanas al ecuador.
\vs p058 2:9 Hasta la aguja de la brújula es sensible a esta influencia solar puesto que rota ligeramente hacia el este cuando sale el sol y hacia el oeste cuando el sol está al ponerse. Esto sucede todos los días, pero durante el apogeo de los ciclos de manchas solares, esta variación de la brújula es doblemente mayor. Estas desviaciones diurnas de la brújula responden al aumento de la ionización de la atmósfera superior, producido por la luz solar.
\vs p058 2:10 Es la presencia de dos diferentes niveles de zonas de conducción electrizadas en la superestratosfera lo que explica la transmisión a larga distancia de vuestras emisiones de radio en onda larga y corta. Vuestras transmisiones se ven en ocasiones interrumpidas debido a las descomunales tormentas que periódicamente azotan los ámbitos de estas ionosferas exteriores.
\usection{3. EL ENTORNO ESPACIAL}
\vs p058 3:1 Durante los tiempos más primitivos de la materialización de un universo, las regiones espaciales están salpicadas de inmensas nubes de hidrógeno, similares a las acumulaciones de polvo astronómico que caracterizan hoy día muchas regiones de todo el espacio remoto. Una gran parte de la materia organizada que los soles llameantes fragmentan y dispersan en forma de energía radiante originariamente se acumulaba en estas tempranas nubes de hidrógeno. Bajo ciertas condiciones excepcionales, las alteraciones atómicas también se producen en el núcleo de las masas de hidrógeno de mayor tamaño. Y todos estos fenómenos de formación y de disolución de átomos, como ocurre en las nebulosas altamente recalentadas, vienen seguidos de la gradual aparición de una gran cantidad de rayos espaciales cortos de energía radiante. Acompañando a estas distintas radiaciones, se halla una forma de energía espacial desconocida en Urantia.
\vs p058 3:2 Esta carga energética de rayo corto del espacio del universo es cuatrocientas veces mayor que todas las otras formas de energía radiante que existen en los ámbitos organizados del espacio. La producción de rayos espaciales cortos, ya procedan de nebulosas llameantes, de potentes campos eléctricos, del espacio exterior o de inmensas nubes de polvo e hidrógeno se modifican de forma cualitativa y cuantitativa por las fluctuaciones, y por los cambios repentinos de la tensión, de la temperatura, de la gravedad y de presiones electrónicas.
\vs p058 3:3 Estas contingencias en el origen de los rayos espaciales están determinadas por muchos sucesos cósmicos al igual que por las órbitas de la materia circulante, que varían desde círculos modificados hasta elipses extremas. Las condiciones físicas pueden alterarse también, de manera considerable, porque a veces los electrones giran en sentido contrario al del comportamiento de la materia de mayor tamaño, incluso en la misma zona física.
\vs p058 3:4 Las inmensas nubes de hidrógeno son verdaderos laboratorios químicos del cosmos; albergan todas las fases de la energía en evolución y de la materia en metamorfosis. También se dan grandes actividades energéticas en los gases marginales de las grandes estrellas binarias que tan a menudo se solapan y, por consiguiente, se entremezclan en gran parte. Pero ninguna de estas enormes y extensas actuaciones de la energía del espacio ejerce el menor efecto sobre los fenómenos de la vida organizada ---el plasma germinal de los seres vivos---. Estas condiciones de la energía del espacio guardan relación con el entorno esencial para la implantación de la vida, pero no son efectivas en las modificaciones posteriores de los factores hereditarios del plasma germinal como lo son algunos de los rayos más largos de la energía radiante. La vida implantada por los portadores de vida es completamente resistente a toda esta sorprendente avalancha de rayos cortos espaciales de la energía del universo.
\vs p058 3:5 \pc Todas estas condiciones cósmicas esenciales tenían que desarrollarse hasta su nivel propicio, antes de que los portadores de vida pudieran empezar realmente a implantar la vida en Urantia.
\usection{4. LA ÉPOCA DE LOS ALBORES DE LA VIDA}
\vs p058 4:1 El hecho de que se nos llame “portadores de vida” no debe confundiros. Podemos portar y portamos vida a los planetas, pero no trajimos a Urantia vida alguna. La vida en Urantia es única, original del planeta. Esta esfera es un mundo de modificación de la vida; toda vida que ha aparecido aquí en el planeta la formulamos nosotros; y no hay otro mundo en toda Satania, ni siquiera en todo Nebadón, en el que la vida existente allí sea exactamente igual a la de Urantia.
\vs p058 4:2 \pc Hace \bibemph{550\,000\,000} de años, el colectivo de portadores de vida regresó a Urantia. En cooperación con las potencias espirituales y las fuerzas suprafísicas, organizamos y dimos inicio a los modelos primigenios de vida de este mundo y los establecimos en las acogedoras aguas de su entorno. Toda la vida planetaria (aparte de los seres personales extraplanetarios) que tuvo lugar hasta los tiempos de Caligastia, o príncipe planetario, se produjo a partir de nuestras tres implantaciones de vida marina, primigenias, idénticas y simultáneas. Estas tres implantaciones de la vida se han designado de la siguiente manera: la \bibemph{central} o eurasiático\hyp{}africana, la \bibemph{oriental} o australasiática y la \bibemph{occidental,} que abarca Groenlandia y las Américas.
\vs p058 4:3 \pc Hace \bibemph{500\,000\,000} de años, la vida vegetal marina primitiva estaba bien establecida en Urantia. Groenlandia y la masa de tierra ártica, junto con América del Norte y del Sur, comenzaban su larga y lenta deriva al oeste. África se desplazaba ligeramente hacia el sur, creando una depresión de este a oeste, la cuenca del Mediterráneo, entre sí misma y la masa continental materna. La Antártida, Australia y la tierra que ocupan las islas del Pacífico se desprendieron por el sur y el este, y se han alejado bastante desde aquel momento.
\vs p058 4:4 Habíamos implantado la forma primitiva de vida marina en las bahías tropicales resguardadas de los mares centrales situados en la fisura este\hyp{}oeste de la masa de tierra continental en vías de rotura. Al hacer tres implantaciones de vida marina, nuestro propósito era asegurarnos de que cada una de estas grandes masas de tierra se llevara esta vida consigo en sus mares de aguas tibias, cuando el suelo posteriormente se separase. Preveíamos que en la era siguiente de la gradual aparición de la vida terrestre, grandes océanos de agua separarían a estas masas de tierra continentales que iban a la deriva.
\usection{5. LA DERIVA CONTINENTAL}
\vs p058 5:1 El desplazamiento de las masas continentales continuaba. El núcleo de la Tierra se había vuelto tan denso y rígido como el acero; estaba sometido a una presión de más de 3875 toneladas por centímetro cuadrado, y debido a la enorme presión de la gravedad, estaba y todavía está, muy recalentado en el interior de sus profundidades. La temperatura aumenta desde la superficie hacia el fondo hasta que, en el centro, está algo por encima de la temperatura de la superficie del Sol.
\vs p058 5:2 Los mil seiscientos kilómetros externos de la masa de la tierra están principalmente compuestos de distintas clases de roca. Por debajo están los elementos metálicos más densos y pesados. A lo largo de las eras primitivas y preatmosféricas, el mundo estaba tan próximo a ser fluido en su estado de fundición y tan altamente recalentado que los metales más pesados se hundieron en las profundidades del interior. Aquellos que hoy en día se encuentran cerca de la superficie son el resultado de la exudación de antiguos volcanes, de grandes flujos de lava posteriores y de depósitos meteóricos más recientes.
\vs p058 5:3 La corteza externa tenía un grosor de unos sesenta y cuatro kilómetros. Esta envoltura estaba sostenida por un mar de basalto derretido de espesor variable, sobre el que descansaba directamente; se trataba de una capa móvil de lava fundida que se mantenía a alta presión, pero que siempre tendía a fluir de un lado a otro, compensando las cambiantes presiones planetarias y contribuyendo, de este modo, a estabilizar la corteza terrestre.
\vs p058 5:4 Aún hoy, continúan los continentes flotando sobre este flexible mar de basalto derretido sin cristalizar. A no ser por esta forma de protección, el mundo estaría sacudido por virulentos terremotos que lo harían literalmente pedazos. La causa de los terremotos no son los volcanes, sino el deslizamiento y el desplazamiento de la sólida corteza externa.
\vs p058 5:5 \pc Las capas de lava de la corteza de la Tierra, al enfriarse, forman el granito. La densidad media de Urantia es algo superior a cinco veces y media la del agua; la del granito es algo inferior a tres veces la del agua. El núcleo de la Tierra es doce veces más denso que el agua.
\vs p058 5:6 Los fondos marinos son más densos que las masas de tierra, y esto es lo que mantiene a los continentes por encima del agua. Cuando los fondos marinos son impelidos hacia arriba, por encima del nivel del mar, se ha constatado que, en su mayor parte, están compuestos de basalto, esto es, de una forma de lava considerablemente más pesada que el granito de las masas de tierra. Además, si los continentes no fuesen más ligeros que los lechos oceánicos, la gravedad arrastraría las márgenes de los océanos hasta por encima del suelo; pero no se han apreciado tales fenómenos.
\vs p058 5:7 El peso de los océanos es también un factor que interviene en el aumento de la presión sobre los lechos marinos. Los lechos oceánicos más bajos, pero relativamente más pesados, más el peso del agua que los cubre, soportan una carga que se aproxima al peso de los continentes, que son más altos pero mucho más ligeros. Sin embargo, todos los continentes tienden a resbalar hacia el interior de los océanos. La presión continental en los niveles del fondo oceánico es sobre 1400 kilogramos por centímetro cuadrado. Es decir, esta sería la presión de una masa continental que se elevara a 4575 metros por encima del suelo del océano. La presión del agua sobre el suelo oceánico es solo unos 351 kilogramos por centímetro cuadrado. Estas diferencias de presiones tienden a provocar que los continentes se deslicen hacia los lechos oceánicos.
\vs p058 5:8 La depresión del fondo oceánico, durante las épocas previas a la vida, había empujado a una solitaria masa de tierra continental a tal altura que la presión lateral tendía a provocar que los bordes orientales, occidentales y australes descendieran, sobre los lechos subyacentes semiviscosos de lava, hacia las aguas circundantes del Océano Pacífico. Esto compensó de tal manera la presión continental, que no se produjo una gran rotura en la orilla oriental de este antiguo continente asiático, sino que, desde entonces, ese litoral oriental ha flotado sobre el abismo de las profundidades oceánicas adyacentes, amenazando con deslizarse a una sepultura líquida.
\usection{6. EL PERÍODO DE TRANSICIÓN}
\vs p058 6:1 Hace \bibemph{450\,000\,000} de años se produjo \bibemph{la transición de la vida vegetal a la animal}. Esta metamorfosis tuvo lugar en las aguas resguardadas y poco profundas de las bahías y lagunas tropicales de los extensos litorales de los continentes que se iban separando. Y este avance, totalmente inherente a los modelos originales de vida, sobrevino de forma progresiva. Hubo muchas etapas de transición entre las tempranas formas primitivas de la vida vegetal y los posteriores organismos animales bien definidos. Aun hoy persisten los mohos de limo de transición, que resultan de difícil catalogación ni como plantas ni como animales.
\vs p058 6:2 \pc Aunque se pueda rastrear la evolución de la vida vegetal a la vida animal, y aun cuando se haya encontrado una serie escalonada de plantas y animales que conducen progresivamente desde los organismos más simples a los más complejos y avanzados, no es posible encontrar vínculos de conexión entre las grandes divisiones del reino animal ni entre los tipos más superiores de animales prehumanos y los hombres de los albores de la raza humana. Estos “eslabones perdidos”, como se les llama, permanecerán perdidos para siempre, por la sencilla razón de que nunca existieron.
\vs p058 6:3 De era en era, surgen especies de vida animal radicalmente nuevas. No evolucionan como resultado de la acumulación gradual de pequeñas variaciones; aparecen como órdenes de vida nuevos y plenos, y lo hacen \bibemph{de repente.}
\vs p058 6:4 La aparición \bibemph{repentina} de especies nuevas y de órdenes diferenciados de organismos vivos es enteramente biológica, rigurosamente natural. No hay nada sobrenatural en estas mutaciones genéticas.
\vs p058 6:5 La vida animal evolucionó al alcanzarse el grado adecuado de salinidad en los océanos, y fue relativamente sencillo dejar que las aguas salobres circularan por los cuerpos de los animales marinos. No obstante, al contraerse los océanos y aumentar considerablemente el porcentaje de sal, estos mismos animales desarrollaron la facultad de reducir la salinidad de sus fluidos corporales, y, al igual que esos organismos que aprendieron a vivir en agua dulce, estos adquirieron la capacidad de mantener el grado apropiado de cloruro sódico en sus fluidos corporales por medio de ingeniosos modos de conservar la sal.
\vs p058 6:6 El estudio de los fósiles de la vida marina adheridos a la roca revela las tempranas luchas de estos organismos primitivos por adaptarse. Las plantas y los animales nunca cesan en sus procesos de adaptación. El medio ambiente está constantemente cambiando, y los organismos vivos siempre están pugnando por acomodarse a estas interminables fluctuaciones.
\vs p058 6:7 El equipamiento fisiológico y la estructura anatómica de todo orden nuevo de vida son una forma de respuesta a la actuación de las leyes físicas, pero la dotación posterior de la mente es un don de los espíritus asistentes de la mente otorgado en conformidad a la capacidad innata del cerebro. La mente, aunque no es el resultado de la evolución física, depende por completo de la capacidad del cerebro, que es puramente producto de la acción física y evolutiva.
\vs p058 6:8 A través de ciclos casi interminables de beneficio y pérdida, de adaptación y readaptación, todo organismo vivo oscila hacia atrás y hacia delante de era en era. Aquellos que alcanzan la unión cósmica perduran, mientras que los que no logran esta meta cesan de existir.
\usection{7. EL LIBRO DE LA HISTORIA GEOLÓGICA}
\vs p058 7:1 En la actualidad, el inmenso grupo de sistemas rocosos que constituían la corteza externa del mundo en los albores de la vida, o era proterozoica, no aparece en muchos puntos de la superficie terrestre. Y cuando, en efecto, emergen de debajo de todos los apilamientos de las eras siguientes, solo se hallan los restos fósiles de la vida vegetal y de la temprana vida animal primitiva. Algunas de estas rocas de mayor antigüedad, que se depositaron en el agua, se entremezclaron con capas posteriores y, a veces, contienen restos fósiles de algunas de las formas más tempranas de la vida vegetal, mientras que en las capas superiores se pueden encontrar, ocasionalmente, algunas de las formas más antiguas de los primeros organismos animales marinos. En muchos lugares, estas capas rocosas estratificadas de mayor antigüedad, portadoras de los fósiles de la primera vida marina, tanto animal como vegetal, pueden hallarse directamente sobre rocas indiferenciadas más antiguas.
\vs p058 7:2 Los fósiles de esta era contienen algas, plantas parecidas al coral, protozoos primitivos y organismos de transición que semejan esponjas. Pero la ausencia de tales fósiles en las primeras capas rocosas no demuestra necesariamente que los organismos vivos no existieran en algún otro lugar, en el momento de su sedimentación. La vida era escasa y estaba dispersa durante todos estos tiempos primitivos y solo lentamente se abrió paso sobre la faz de la Tierra.
\vs p058 7:3 \pc Las rocas de esta antigua era están actualmente en la superficie de la Tierra, o muy cerca de la superficie, sobre aproximadamente una octava parte de la presente superficie terrestre. El grosor medio de esta roca de transición, las capas de rocas estratificadas de más antigüedad, es de unos dos kilómetros y medio. En algunos puntos, estos antiguos sistemas rocosos llegan a alcanzar unos seis kilómetros y medio de espesor, pero muchas de las capas que se han atribuido a esta era pertenecen a períodos posteriores.
\vs p058 7:4 En América del Norte, esta capa rocosa antigua y primitiva, portadora de fósiles, aflora en las regiones orientales, centrales y septentrionales de Canadá. Existe, también, de este a oeste, una cadena montañosa de dicha capa que aparece a intervalos y que se extiende desde Pensilvania y los antiguos montes Adirondack hacia el oeste por Míchigan, Wisconsin y Minesota. Otras cadenas montañosas se despliegan desde Terranova a Alabama y desde Alaska a México.
\vs p058 7:5 En esta era, las rocas están al descubierto por doquier en todo el mundo, pero ninguna de ellas es tan fácil de interpretar como las del entorno del Lago Superior y del Gran Cañón del Río Colorado. Allí estas primitivas rocas, existentes en diversas capas, dan testimonio de los levantamientos y fluctuaciones de la superficie de esos remotos tiempos.
\vs p058 7:6 Esta capa rocosa, el estrato portador de fósiles más antiguo de la corteza terrestre, se ha desplomado, plegado y retorcido grotescamente, como resultado de perturbaciones sísmicas y de los primeros volcanes. Los flujos de lava de esta era transportaron hierro, cobre y plomo hasta cerca de la superficie planetaria.
\vs p058 7:7 Hay pocos lugares de la Tierra en los que tal actividad se muestre de forma más gráfica como en el Valle de St. Croix de Wisconsin. En esta región hubo ciento veintisiete flujos sucesivos de lava sobre la tierra con su consiguiente sumergimiento en agua y el consecuente depósito de roca. Aunque una gran parte de la sedimentación rocosa superior y del flujo de lava intermitente es inexistente hoy en día, y aunque la base de este cúmulo está profundamente enterrada en la Tierra, no obstante, alrededor de sesenta y cinco o setenta de estos registros estratificados de eras pasadas están hoy en día expuestos a la vista.
\vs p058 7:8 \pc En estas tempranas eras en las que una gran parte del suelo de la Tierra estaba cerca del nivel del mar, se produjeron muchos sumergimientos y emersiones consecutivos. La corteza terrestre comenzaba su último período de estabilización relativa. Las ondulaciones, elevaciones y depresiones de la primera deriva continental contribuyeron a la frecuencia de la sumersión periódica de las grandes masas de tierra.
\vs p058 7:9 Durante estos tiempos de la vida marina primitiva, grandes zonas continentales costeras se hundieron en los mares a una profundidad que oscilaba entre unos metros y ochocientos metros. Gran parte de la arenisca y de los conglomerados más antiguos constituyen la acumulación sedimentaria de estas ancestrales costas. Las rocas sedimentarias, pertenecientes a esta temprana estratificación, reposan directamente sobre aquellas capas cuya datación es muy anterior al origen de la vida, remontándose a la primera aparición del océano mundial.
\vs p058 7:10 Algunas de las capas superiores de estos depósitos rocosos de transición contienen pequeñas cantidades de esquisto o pizarra de colores oscuros; lo que indica la presencia de carbono orgánico y da testimonio de la existencia de los predecesores de aquellas formas de vida vegetal, que dominaron la Tierra durante el siguiente periodo Carbonífero, o del carbón. Gran parte del cobre presente en estas capas rocosas se ha depositado allí por efecto del agua. En las grietas de las rocas más antiguas, podemos encontrar alguna cantidad de cobre por la concentración de aguas pantanosas estancadas de algún antiguo litoral resguardado. Las minas de hierro de América del Norte y Europa están situadas en depósitos y protuberancias que yacen, en parte, en las rocas no estratificadas más antiguas y, en parte, en estas posteriores rocas estratificadas de los períodos de transición de la formación de la vida.
\vs p058 7:11 \pc En esta era se presencia la diseminación de la vida por todas las aguas del mundo; la vida marina está bien asentada en Urantia. Una vegetación que crece abundante y exuberantemente va progresivamente invadiendo los fondos de los extensos y poco profundos mares interiores, en tanto que las aguas de los litorales están repletas de las formas más simples de vida animal.
\vs p058 7:12 \pc Toda esta historia se relata gráficamente en las páginas fósiles del inmenso “libro de piedra” de los archivos del mundo. Y las páginas de este gigantesco registro bio\hyp{}geológico os dirán indefectiblemente la verdad con que tan solo adquiráis la destreza para poder interpretarlas. Muchos de estos antiguos lechos marinos se elevan ahora muy por encima del suelo terrestre, y sus depósitos, era tras era, narran la historia de la pugna por la vida de aquellos tempranos tiempos. Es literalmente cierto, como dijo vuestro poeta, que “El polvo que pisamos estuvo una vez vivo”.
\vsetoff
\vs p058 7:13 [Exposición de un miembro del colectivo de los portadores de vida de Urantia con residencia actual en el planeta.]
