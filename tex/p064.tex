\upaper{64}{Las razas evolutivas de color}
\author{Portador de vida}
\vs p064 0:1 Esta es la historia de las razas evolutivas de Urantia desde los días de Andón y Fonta, hace casi un millón de años, pasando por la época del príncipe planetario, hasta el final de la edad del hielo.
\vs p064 0:2 La raza humana tiene casi un millón de años de antigüedad, y la primera mitad de su historia corresponde aproximadamente a los tiempos anteriores al príncipe planetario de Urantia. La última mitad de la historia de la humanidad se inicia en el momento de la llegada del príncipe planetario y la aparición de las seis razas de color y, en líneas generales, engloba el período considerado normalmente como la antigua edad de piedra.
\usection{1. LOS ABORÍGENES ANDÓNICOS}
\vs p064 1:1 El hombre primitivo hizo su aparición evolutiva en la tierra hace algo menos de un millón de años, y tuvo una intensa experiencia. Instintivamente, trató de escapar del peligro de mezclarse con las tribus simias peor dotadas. Pero no pudo emigrar en dirección este debido a las áridas elevaciones de suelo del Tíbet, con 9000 metros sobre el nivel del mar; tampoco pudo ir hacia el sur o hacia el oeste por la expansión del Mar Mediterráneo, que se extendía entonces hacia el este hasta el Océano Índico; y cuando se dirigió hacia el norte, se encontró con el avance del hielo. Pero incluso cuando el hielo les impedía seguir adelante con su emigración, y aunque las tribus que se dispersaban se volvían cada vez más hostiles, los grupos más inteligentes nunca concibieron la idea de encaminarse al sur para vivir entre sus primos arborícolas peludos, de inferior intelecto.
\vs p064 1:2 Muchas de las emociones religiosas más tempranas del hombre fueron el resultado de su sensación de impotencia al sentirse encerrados en aquel entorno geográfico: montañas a la derecha, agua a la izquierda y el hielo delante. Sin embargo, estos avanzados andonitas se negaban a retroceder al sur y volver con sus peor dotados parientes arborícolas.
\vs p064 1:3 En contraste con los hábitos de sus parientes no humanos, estos andonitas evitaban los bosques. El hombre siempre se ha degradado en los bosques; la evolución humana solo ha avanzado en latitudes abiertas y más elevadas. El frío y el hambre, propios de las tierras abiertas, estimulan la acción, la invención y el ingenio. Y, mientras que de estas tribus andónicas surgieron los pioneros de la presente raza humana en medio de las adversidades y privaciones de estos rigurosos climas norteños, sus atrasados primos disfrutaban de los bosques tropicales sureños en unas tierras que testimoniaban su común origen primitivo.
\vs p064 1:4 \pc Estos hechos ocurrieron durante los tiempos del tercer glaciar, el primero de acuerdo a la estimación de los geólogos. Los dos primeros glaciares no fueron tan extensos en la Europa septentrional.
\vs p064 1:5 Durante la mayor parte de la edad del hielo, Inglaterra estuvo comunicada por tierra con Francia, mientras que más tarde África estuvo unida a Europa mediante el puente terrestre de Sicilia. En el momento de las emigraciones andónicas, existía una continua ruta terrestre desde Inglaterra en el oeste, que pasaba por Europa y Asia, hasta Java en el este; pero Australia quedó de nuevo aislada, lo que acentuó más el desarrollo de su propia y singular fauna.
\vs p064 1:6 \pc Hace \bibemph{950\,000} años, los descendientes de Andón y Fonta, en su movimiento migratorio, se adentraron hacia el este y el oeste. En su camino al oeste, pasaron por Europa, llegando a Francia e Inglaterra. En épocas posteriores, penetraron por el este hasta llegar a Java, donde tan recientemente se han hallado sus huesos ---el llamado hombre de Java---, para continuar luego su viaje hacia Tasmania.
\vs p064 1:7 Los grupos que iban hacia el oeste se contaminaron menos con los linajes atrasados, con los que compartían en su origen un común ancestro, que los que se desplazaron hacia el este, que se mezclaron sin restricción alguna con sus retrasados primos animales. Estos atrasados seres se encaminaron hacia el sur y se aparearon enseguida con las tribus peor dotadas. Luego, un creciente número de sus mestizos regresó al norte y se emparejó con los pueblos andónicos que se extendían con rapidez; estas desafortunadas uniones deterioraron indefectiblemente a los linajes mejor dotados. Cada vez quedaban menos asentamientos primitivos que conservaran la adoración al Dador del Aliento. En sus albores, esta temprana civilización estuvo al borde de la extinción.
\vs p064 1:8 Y siempre ha sido así en Urantia. Unas civilizaciones muy prometedoras se han deteriorado de forma consecutiva. y han acabado por extinguirse debido a la insensatez de permitir que los individuos más dotados procrearan libremente con los peor dotados.
\usection{2. LOS PUEBLOS DE FOXHALL}
\vs p064 2:1 Hace \bibemph{900\,000} años, las artes de Andón y Fonta y la cultura de Onagar estaban desapareciendo de la faz de la tierra; la cultura, la religión e incluso su trabajo del sílex se hallaban en sus niveles más bajos.
\vs p064 2:2 Fue en estos tiempos cuando un gran número de mestizos menos dotados llegó a Inglaterra desde el sur de Francia. Estas tribus estaban tan mezcladas con las criaturas simias de los bosques que apenas eran humanas. No tenían ninguna religión, pero eran rudimentarios obreros del sílex y poseían suficiente inteligencia como para encender el fuego.
\vs p064 2:3 Tras estas tribus, llegó a Europa un pueblo ligeramente mejor dotado y prolífico, cuyos descendientes se extendieron pronto por todo el continente desde los hielos del norte hasta los Alpes y el Mediterráneo en el sur. Estas tribus se denominan \bibemph{raza de Heidelberg}.
\vs p064 2:4 Durante este largo período de decadencia cultural, los pueblos de Foxhall en Inglaterra y las tribus de Badonán en el noroeste de la India continuaron aferrándose a algunas tradiciones de Andón y a determinados remanentes de la cultura de Onagar.
\vs p064 2:5 \pc Los pueblos de Foxhall estaban más situados al oeste y lograron conservar gran parte de la cultura andónica; también preservaron sus conocimientos sobre el trabajo del sílex, que trasmitieron a sus descendientes, los ancestrales antecesores de los esquimales.
\vs p064 2:6 Aunque los vestigios de los pueblos de Foxhall han sido los últimos en descubrirse en Inglaterra, estos andonitas fueron realmente los primeros seres humanos que vivieron en estas regiones. En ese momento, el puente terrestre todavía comunicaba a Francia con Inglaterra; y, puesto que la mayoría de los primeros asentamientos de los descendientes de Andón estaban ubicados a lo largo de los ríos y de las costas de aquellos tempranos días, ahora yacen bajo las aguas del Canal de la Mancha y del Mar del Norte, pero unos tres o cuatro siguen aún por encima del agua en la costa inglesa.
\vs p064 2:7 Muchos de los más inteligentes y espirituales pueblos de Foxhall mantuvieron su superioridad racial y perpetuaron sus costumbres religiosas primitivas. Estos, una vez que se mezclaron con linajes posteriores, viajaron desde Inglaterra hacia el oeste tras una siguiente aparición del hielo y han sobrevivido en sus descendientes los esquimales de hoy día.
\usection{3. LAS TRIBUS DE BADONÁN}
\vs p064 3:1 Además de los pueblos de Foxhall en el oeste, otro centro de cultura persistía en el este con dificultades. Este grupo estaba situado en las estribaciones de las altiplanicies del noroeste de la India, entre las tribus de Badonán, tataranieto de Andón. Ellos fueron los únicos descendientes de Andón que jamás llegarían a practicar sacrificios humanos.
\vs p064 3:2 Estos badonitas de las altiplanicies ocupaban una extensa meseta rodeada de bosques, atravesada por riachuelos y con abundante caza. Al igual que algunos de sus primos del Tíbet, vivían en toscas cabañas de piedra, en grutas en las laderas de las colinas y en pasajes semisubterráneos.
\vs p064 3:3 Mientras que las tribus del norte tenían cada vez más miedo al hielo, las que vivían cerca de su lugar de origen se volvieron sumamente temerosas del agua. Observaban que la península mesopotámica se sumergía en el océano de forma paulatina y, aunque había emergido varias veces, las tradiciones de estas razas primitivas se crearon en torno a los peligros del mar y al temor de un hundimiento periódico. Y este temor, unido a su experiencia respecto al desbordamiento de los ríos, explica por qué buscaban las altiplanicies como espacio seguro para vivir.
\vs p064 3:4 Al este de los pueblos de Badonán, en las colinas Siwalik del norte de la India, se pueden encontrar fósiles que más se aproximan a los ejemplos de transición entre el hombre y los diversos grupos prehumanos que en cualesquiera otros lugares de la tierra.
\vs p064 3:5 \pc \bibemph{Hace 850\,000} años, las tribus de Badonán, mejor dotadas, comenzaron una guerra de exterminio de sus vecinos peor dotados y de índole animal. En menos de mil años, la mayoría de los grupos animales de las zonas fronterizas de estas regiones habían sido eliminados o forzados a retroceder hasta los bosques del sur. Esta ofensiva para aniquilar estos seres trajo consigo un ligero mejoramiento de las tribus montañesas de aquella era. Y los descendientes mezclados de este mejorado linaje badonita aparecieron en acción como un pueblo aparentemente nuevo: la \bibemph{raza de Neandertal}.
\usection{4. LAS RAZAS DE NEANDERTAL}
\vs p064 4:1 Los neandertales eran excelentes luchadores y grandes viajeros. Se expandieron progresivamente desde sus núcleos de influencia en las altiplanicies del noroeste de la India hasta Francia en el oeste, China en el este, e incluso bajaron hasta el norte de África. Dominaron el mundo durante casi medio millón de años, hasta los tiempos de la emigración de las razas evolutivas de color.
\vs p064 4:2 \pc \bibemph{Hace 800\,000} años, la caza era abundante; muchas especies de ciervos, al igual que elefantes e hipopótamos, recorrían Europa. El ganado era numeroso; los caballos y los lobos estaban por doquier. Los hombres de Neandertal eran magníficos cazadores, y las tribus de Francia fueron las primeras en adoptar la costumbre de que los mejores cazadores eligieran a sus esposas de entre las mujeres de la tribu.
\vs p064 4:3 El reno era extremadamente provechoso para estos pueblos neandertales; les servía de alimento, de ropa y de utensilios, ya que hacían distintos usos de los cuernos y de los huesos. Eran poco cultivados, pero mejoraron de tal manera el trabajo del sílex que este casi llegó a alcanzar los niveles de los días de Andón. Volvieron a utilizarse grandes piedras de sílex atadas a unos mangos de madera que se usaban como hachas y piquetas.
\vs p064 4:4 \pc \bibemph{Hace 750\,000} años, la cuarta capa de hielo avanzó bastante hacia el sur. Con sus perfeccionadas herramientas, los neandertales hacían orificios en el hielo que cubría los ríos norteños para poder así arponear a los peces que acudían a estas aberturas. Estas tribus retrocedieron ante el avance del hielo, que en este momento realizaba su más extensa invasión de Europa.
\vs p064 4:5 En esos tiempos, el glaciar siberiano llegaba a su punto más meridional, obligando al hombre primitivo a desplazarse hacia el sur, de vuelta a las tierras en las que tuvieron su origen. Pero la especie humana se había diferenciado tanto, que el peligro de mezclarse de nuevo con sus involucionados parientes simios se había reducido en gran medida.
\vs p064 4:6 \pc \bibemph{Hace 700\,000} años, el cuarto glaciar, el más grande de todos los habidos en Europa, estaba en retroceso; los hombres y los animales comenzaron a regresar al norte. El clima era fresco y húmedo, y el hombre primitivo prosperó una vez más en Europa y en Asia occidental. Los bosques se extendieron paulatinamente hacia el norte sobre las tierras que el glaciar había últimamente cubierto.
\vs p064 4:7 El gran glaciar había transformado poco la vida de los mamíferos. Estos animales resistieron en la estrecha franja de tierra situada entre el hielo y los Alpes y, al retroceder el glaciar, se extendieron de nuevo rápidamente por toda Europa. Los elefantes de colmillos rectos, los rinocerontes de hocico ancho, las hienas y los leones africanos llegaron de África por el puente terrestre de Sicilia; y estos nuevos animales prácticamente exterminaron a los tigres con dientes de sable y a los hipopótamos.
\vs p064 4:8 \pc \bibemph{Hace 650\,000} años se observó que el clima continuaba templado. Hacia mediados del período interglaciar se había vuelto tan cálido que los Alpes estaban casi desprovistos de hielo y nieve.
\vs p064 4:9 \pc \bibemph{Hace 600\,000} años, el hielo había alcanzado entonces su punto de retroceso más septentrional y, tras una pausa de unos miles de años, inició de nuevo su quinto trayecto al sur. Pero, durante cincuenta mil años, el clima se modificó poco. También cambiaron poco los hombres y los animales de Europa. Se redujo la ligera aridez del período anterior y los glaciares alpinos descendieron hasta adentrarse en los valles fluviales.
\vs p064 4:10 \pc \bibemph{Hace 550\,000} años, el avance del glaciar empujó de nuevo a los hombres y a los animales hacia el sur. Pero esta vez, los hombres disponían de bastante espacio dentro de la ancha franja de tierra que se extendía hacia el nordeste de Asia, que estaba situada entre la capa de hielo y el Mar Negro, una extensión del Mediterráneo considerablemente expandida entonces.
\vs p064 4:11 Estos tiempos de los glaciares cuarto y quinto fueron testigos de un mayor progreso de la rudimentaria cultura de las razas neandertales. Pero se hicieron tan pocos avances que, en verdad, parecía que el intento de crear un tipo nuevo y modificado de vida inteligente en Urantia estaba a punto de fracasar. Durante casi un cuarto de millón de años, estos pueblos primitivos fueron a la deriva, cazando y peleando, mejorando a intervalos en algunos campos, pero, en términos generales, degradándose constantemente, en comparación con sus mejor dotados antepasados andónicos.
\vs p064 4:12 \pc Durante estas eras de oscuridad espiritual, la cultura de la superstición en la humanidad alcanzó sus niveles más bajos. Realmente, los neandertales no tenían más religión que una lamentable superstición. Sentían un fatal miedo a las nubes y aún más a la niebla y a la bruma. Paulatinamente, se desarrolló una religión primitiva enraizada en el temor a las fuerzas naturales, mientras que la adoración de los animales decrecía, a medida que la mejora de las herramientas, junto a la abundancia de la caza, posibilitó que estos pueblos vivieran con menos ansiedad respecto a la comida; las recompensas sexuales por la caza contribuyeron bastante a mejorar la destreza en esta práctica. Esta nueva religión del miedo condujo a ciertos intentos por aplacar las fuerzas invisibles ocultas tras los elementos naturales y llevó, más tarde, a los sacrificios humanos para apaciguar tales fuerzas físicas invisibles y desconocidas. Y esta terrible costumbre de los sacrificios humanos se perpetuó hasta el mismo siglo XX entre los pueblos más atrasados de Urantia.
\vs p064 4:13 Difícilmente se puede denominar a estos primeros neandertales como adoradores del sol. Vivían más bien temiendo la oscuridad; sentían un pavor mortal al anochecer. Mientras que la luna brillara levemente, se las arreglaban bien; pero cuando se oscurecía, les invadía el pánico y empezaban a sacrificar a sus mejores ejemplares de hombres y mujeres, en un intento por hacer que la luna brillase de nuevo. Pronto aprendieron a que el sol reaparecía regularmente, pero suponían que la luna solo lo hacía si sacrificaban a sus compañeros de tribu. Conforme la raza progresaba, el objeto y la finalidad de los sacrificios cambiaron de forma gradual, pero la ofrenda de sacrificios humanos, como parte de su ceremonial religioso, subsistió durante mucho tiempo.
\usection{5. EL ORIGEN DE LAS RAZAS DE COLOR}
\vs p064 5:1 \bibemph{Hace 500\,000} años, las tribus de Badonán, que habitaban en las altiplanicies del noroeste de la India, se vieron envueltas en otra gran contienda racial. Esta implacable guerra se libró durante más de cien años y, cuando la larga lucha terminó, solo quedaban unas cien familias. Pero estos supervivientes eran los más inteligentes y meritorios de todos los descendientes de Andón y Fonta que estaban vivos en aquel entonces.
\vs p064 5:2 En este momento, ocurrió algo nuevo y extraño entre estos badonitas de las altiplanicies. Un hombre y una mujer, moradores de la parte nordeste de la región de las altiplanicies, entonces habitadas, empezaron a procrear \bibemph{de repente} a una progenie de hijos inusualmente inteligentes. Se trataba de la \bibemph{familia sangik,} los ancestros de las seis razas de color de Urantia.
\vs p064 5:3 Estos hijos sangik, diecinueve en total, no solo eran más inteligentes que sus semejantes, sino que su piel manifestaba una singular tendencia a volverse de colores diferentes al exponerse a la luz solar. De estos diecinueve hijos, cinco eran rojos, dos naranjas, cuatro amarillos, dos verdes, cuatro azules y dos índigos. Estos colores se hacían más acentuados conforme los niños crecían, y cuando estos jóvenes se casaron después con otros miembros de su tribu, todos sus vástagos propendían al color de piel de su progenitor sangik.
\vs p064 5:4 Y ahora interrumpo esta narración cronológica, tras llamar vuestra atención sobre la llegada del príncipe planetario ocurrida alrededor de este tiempo, para examinar detenidamente y por separado las seis razas sangiks de Urantia.
\usection{6. LAS SEIS RAZAS SANGIK DE URANTIA}
\vs p064 6:1 En un planeta evolutivo promedio, las seis razas evolutivas de color aparecen una a una; el hombre rojo es el primero en evolucionar y, durante eras, recorre el mundo antes de que las siguientes razas de color hagan su aparición. El surgimiento simultáneo de las seis razas en Urantia, \bibemph{y en el seno de una sola familia,} fue algo extremadamente insólito.
\vs p064 6:2 La aparición de los primeros andonitas en Urantia fue también algo nuevo en Satania. En ningún otro mundo del sistema local, se había desarrollado tal raza de criaturas volitivas con anterioridad a las razas evolutivas de color.
\vs p064 6:3 \li{1.}\bibemph{El hombre rojo}. Estos pueblos eran excelentes ejemplares de la raza humana, mejor dotados que Andón y Fonta en muchos aspectos. Constituían un grupo con un alto grado de inteligencia y fueron los primeros de los hijos sangik en desarrollar una civilización y un gobierno tribales. Siempre fueron monógamos; incluso sus descendientes al mezclarse rara vez practicaron la poligamia.
\vs p064 6:4 Más adelante, tuvieron dificultades, serias y prolongadas en el tiempo, con sus hermanos amarillos de Asia. Les sirvió de ayuda el hecho de haber inventado tempranamente el arco y la flecha, pero desafortunadamente habían heredado la gran propensión de sus ancestros a luchar entre ellos, y esto los debilitó de tal forma que las tribus amarillas fueron capaces de expulsarlos del continente asiático.
\vs p064 6:5 Hace unos ochenta y cinco mil años, los supervivientes relativamente puros de la raza roja se trasladaron masivamente a América del Norte y, poco después, el istmo terrestre de Bering se hundió, quedando de este modo aislados. Ningún hombre rojo regresó jamás a Asia. Si bien, por toda Siberia, China, Asia central, la India y Europa dejaron tras ellos gran parte de su linaje mezclado con las otras razas de color.
\vs p064 6:6 Cuando el hombre rojo cruzó hasta América, se llevó consigo muchas enseñanzas y tradiciones de sus tempranos orígenes. Sus inmediatos ancestros habían estado implicados en las últimas iniciativas desarrolladas en la sede mundial del príncipe planetario. Pero poco tiempo después de haber llegado a las Américas, el hombre rojo empezó a desaprovechar estas enseñanzas y su cultura intelectual y espiritual sufrió un grave deterioro. Muy pronto, estos hombres empezaron a pelearse de nuevo entre ellos, tan encarnizadamente, que pareció que estas guerras tribales tendrían como resultado la pronta extinción de este remanente de la relativamente pura raza roja.
\vs p064 6:7 A causa de este gran retroceso, el hombre rojo parecía estar condenado a un destino adverso cuando, hace unos sesenta y cinco mil años, apareció Onamonalontón como su líder y libertador espiritual. Trajo una paz transitoria entre los hombres rojos americanos y reavivó la adoración del “Gran Espíritu”. Onamonalontón vivió hasta los noventa y seis años de edad y mantuvo su sede entre las grandes secoyas de California. Muchos de sus postreros descendientes han llegado hasta los tiempos modernos entre los indios Pies Negros.
\vs p064 6:8 A medida que trascurría el tiempo, las enseñanzas de Onamonalontón se convirtieron en borrosas tradiciones. Las guerras internas reanudaron y, tras los días de este gran maestro, ningún otro líder logró jamás traer una paz generalizada entre ellos. Cada vez más, las más inteligentes estirpes perecían en estas luchas tribales; en caso contrario, estos hombres rojos, capaces e inteligentes, hubieran desarrollado una gran civilización en el continente norteamericano.
\vs p064 6:9 Después de cruzar de China a América, el hombre rojo del norte jamás volvería de nuevo a recibir influencias de otras partes del mundo (excepto de los esquimales) hasta que más tarde el hombre blanco lo descubrió. Es muy lamentable que el hombre rojo perdiera, casi por completo, la oportunidad de elevar su naturaleza mezclándose con el posterior linaje de Adán. En aquel orden de cosas, el hombre rojo no podía prevalecer sobre el hombre blanco, y no estaba dispuesto a servirle de forma voluntaria. En estas circunstancias, si las dos razas no se mezclaban, una o la otra estaba condenada.
\vs p064 6:10 \li{2.}\bibemph{El hombre naranja}. La excepcional característica de esta raza era su particular afán de construir cualquier cosa, fuese lo que fuese, incluso apilaban enormes montículos de piedra solo para ver qué tribu podía construir el montículo más grande. Aunque no era un pueblo avanzado, sacaron bastante provecho de las escuelas del príncipe y enviaron allí a sus delegados para su formación.
\vs p064 6:11 La raza naranja fue la primera en seguir el litoral en dirección sur, hacia África, conforme el Mediterráneo se replegaba hacia el oeste. Pero nunca consiguieron establecerse con estabilidad en África y fueron aniquilados por la raza verde que llegaría más tarde.
\vs p064 6:12 Antes de que les llegara el fin, este pueblo había perdido gran parte de sus fundamentos culturales y espirituales. No obstante, experimentaron un gran renacimiento y una consecuente elevada forma de vida como resultado del inteligente liderazgo de Porshunta, el gran pensador de esta desafortunada raza, que los asistió cuando tenían su sede en Armagedón, hace unos trescientos mil años.
\vs p064 6:13 La última gran contienda entre los hombres naranjas y los verdes ocurrió en la región del bajo valle del Nilo, en Egipto. Esta interminable guerra se libró durante cerca de cien años y, cuando acabó, muy pocos miembros de la raza naranja quedaban con vida. Los devastados supervivientes de este pueblo fueron absorbidos por los hombres verdes y por los índigos, que llegarían más tarde. Pero como raza, el hombre naranja dejó de existir hace unos cien mil años.
\vs p064 6:14 \li{3.}\bibemph{El hombre amarillo}. Las tribus amarillas primitivas fueron las primeras en abandonar la caza, establecer comunidades estables y desarrollar una vida hogareña basada en la agricultura. Intelectualmente eran algo menos dotadas que el hombre rojo, pero, social y colectivamente, demostraron ser superiores a todos los pueblos sangiks en cuanto al fomento de la civilización racial. Al desarrollar un espíritu fraternal, las distintas tribus aprendieron a convivir en una paz relativa y, a medida que se extendían paulatinamente en Asia, fueron capaces de desplazar a la raza roja.
\vs p064 6:15 Se alejaron mucho de las influencias de la sede espiritual del mundo y se vieron envueltos en una gran oscuridad tras la apostasía de Caligastia; pero se produjo en este pueblo una brillante era hace unos cien mil años, cuando Singlangtón asumió el liderazgo de estas tribus y proclamó la adoración de la “Verdad Única”.
\vs p064 6:16 La supervivencia del número, relativamente grande, de miembros de la raza amarilla se debe a la paz intertribal que disfrutaban. Desde los días de Singlangtón hasta los tiempos de la China moderna, la raza amarilla se cuenta entre las naciones más pacíficas de Urantia. Esta raza recibió un legado pequeño pero poderoso del posterior linaje adánico importado.
\vs p064 6:17 \li{4.}\bibemph{El hombre verde}. La raza verde fue uno de los grupos menos capaces de hombres primitivos y quedaron muy debilitados a causa de sus extensas y multidireccionales emigraciones. Antes de dispersarse, estas tribus experimentaron un gran renacimiento cultural bajo el liderazgo de Fantad, hace unos trescientos cincuenta mil años.
\vs p064 6:18 La raza verde se escindió en tres divisiones principales: las tribus norteñas, que fueron sometidas, esclavizadas y absorbidas por la raza amarilla y la raza azul; el grupo oriental, que se mezcló con los pueblos de la India de aquellos días, y cuyos remanentes aún persisten entre ellos; y la nación sureña, que penetró en África, exterminando allí a sus primos naranjas casi tan poco dotados como ellos.
\vs p064 6:19 En muchos aspectos, en su enfrentamiento, ambos grupos estaban igualados, puesto que cada uno portaba estirpes del orden de los gigantes: muchos de sus líderes medían entre dos metros cuarenta y dos metros setenta de altura. Estos linajes gigantes del hombre verde se circunscribían mayormente a esta nación sureña o egipcia.
\vs p064 6:20 Los remanentes de los victoriosos hombres verdes fueron absorbidos posteriormente por la raza índigo, la última de las razas de color en desarrollarse y emigrar desde el centro originario Sangik desde el que las razas se dispersaron.
\vs p064 6:21 \li{5.}\bibemph{El hombre azul}. Los hombres azules fueron un gran pueblo. Inventaron pronto la lanza y elaboraron, más adelante, las bases de muchas de las artes de la civilización moderna. El hombre azul tenía la capacidad cerebral del hombre rojo junto con el alma y los sentimientos del hombre amarillo. Los descendientes adánicos los prefirieron sobre todas las otras razas de color que perdurarían después.
\vs p064 6:22 Los primeros hombres azules se mostraron receptivos a las ideas de los maestros de la comitiva del príncipe Caligastia, y se sumieron en una gran confusión con las posteriores y distorsionadas enseñanzas de estos líderes traidores. Como otras razas primitivas, nunca se recuperaron completamente de la convulsión creada por la traición de Caligastia ni tampoco llegaron a superar del todo su tendencia a luchar entre ellos.
\vs p064 6:23 Unos quinientos años tras la caída de Caligastia, se produjo un renacimiento generalizado del conocimiento y de la religión de un orden primitivo ---pero no por ello menos real y beneficioso---. Orlandof se convirtió en un gran maestro de la raza azul y recondujo a muchas de las tribus a la adoración del verdadero Dios bajo el nombre de “el Jefe Supremo”. Este fue el mayor avance del hombre azul hasta esos tiempos posteriores en los que su raza mejoró de forma considerable mediante el cruce con el linaje adánico.
\vs p064 6:24 Las investigaciones y exploraciones europeas sobre la antigua edad de piedra han tenido que ver mayormente con el desenterramiento de las herramientas, los huesos y las artesanías de estos ancestrales hombres azules, puesto que subsistieron en Europa hasta tiempos recientes. Las llamadas \bibemph{razas blancas} de Urantia descienden de ellos, modificados primero al mezclarse ligeramente con la raza amarilla y la roja y, después, mejorados enormemente al asimilar la mayor parte de la raza violeta.
\vs p064 6:25 \li{6.}\bibemph{La raza índigo}. Al igual que los hombres rojos fueron los más avanzados de todos los pueblos sangiks, los hombres negros fueron los menos desarrollados. Fueron los últimos en emigrar desde sus altiplanicies natales. Viajaron hasta África, tomaron posesión del continente y allí han permanecido desde entonces, excepto cuando, a través de los tiempos, se los han llevado a la fuerza como esclavos.
\vs p064 6:26 Aislados en África, los pueblos índigos, al igual que el hombre rojo, recibieron poca o ninguna de la elevación racial que podría haberse derivado de la infusión del linaje adánico. Estando sola en África, la raza índigo hizo pocos avances hasta los días de Orvonón, momento en el que estos pueblos experimentaron un gran despertar espiritual. Aunque más tarde casi llegaron a olvidarse totalmente del “Dios de Dioses”, proclamado por Orvonón, no perderían del todo el deseo de adorar al Desconocido; al menos mantuvieron una forma de culto hasta hace algunos pocos miles de años.
\vs p064 6:27 A pesar de su atraso, estos pueblos índigos tienen exactamente el mismo estatus ante los poderes celestiales que cualquier otra raza de la tierra.
\vs p064 6:28 \pc Fueron eras de intensas luchas entre las distintas razas, pero cerca de la sede del príncipe planetario, los grupos más cultivados e instruidos más recientemente convivían en una relativa armonía; si bien, hasta el momento de producirse las serias alteraciones de este régimen por el estallido de la rebelión de Lucifer, las razas del mundo no habían realizado ninguna gran conquista cultural.
\vs p064 6:29 \pc Cada cierto tiempo, todos estos diferentes pueblos experimentaron renacimientos culturales y espirituales. Mansant fue un gran maestro de los días posteriores al príncipe planetario. Pero solo mencionamos a esos excepcionales líderes y maestros que influyeron e inspiraron de manera notable a toda una raza. Con el trascurso del tiempo, aparecerían, en diferentes regiones, maestros de menor rango; y, en conjunto, todos contribuyeron sobremanera al cúmulo de influencias salvadoras que impedirían el total colapso de la civilización cultural, en particular durante las largas eras de oscuridad que mediaron entre la rebelión de Caligastia y la llegada de Adán.
\vs p064 6:30 \pc Existen muchas razones adecuadas y suficientes para planificar el desarrollo evolutivo de tres o de seis razas de color en los mundos del espacio. Aunque los mortales de Urantia quizás no estén en total disposición de entender todas estas razones, queremos llamar la atención sobre lo siguiente:
\vs p064 6:31 \li{1.}La variedad resulta indispensable para permitir que la selección natural opere ampliamente, esto es, que se dé de forma diferenciada la supervivencia de las estirpes mejor dotadas.
\vs p064 6:32 \li{2.}El cruce entre los distintos pueblos proporciona razas más fuertes y mejores, cuando esas razas diferentes son portadoras de factores hereditarios de orden superior. Y las razas de Urantia se hubieran beneficiado de tal temprano cruzamiento, si esta conjunción de pueblos hubiese podido ser posteriormente mejorado al mezclarse totalmente con el mejor dotado linaje adánico. En las condiciones raciales actuales, cualquier intento por llevar a cabo un experimento de este tipo en Urantia sería sumamente catastrófico
\vs p064 6:33 \li{3.}La diversificación de las razas incita a una competición sana.
\vs p064 6:34 \li{4.}Las diferencias de estatus entre las razas y entre grupos dentro de cada raza son esenciales para el desarrollo de la tolerancia y el altruismo humanos.
\vs p064 6:35 \li{5.}La homogeneidad de la raza humana no es deseable hasta que los pueblos de un mundo en evolución no logren unos niveles relativamente elevados de desarrollo espiritual.
\usection{7. DISPERSIÓN DE LAS RAZAS DE COLOR}
\vs p064 7:1 Cuando los descendientes de color de la familia sangik empezaron a multiplicarse y a buscar la ocasión para extenderse a territorios contiguos, el quinto glaciar, el tercero según el cómputo geológico, ya había avanzado bastante en su deriva hacia el sur sobre Europa y Asia. Estas tempranas razas de color sufrieron una extraordinaria prueba por los rigores y adversidades de la edad glacial en la que tuvieron su origen. Dicho glaciar fue tan extenso en Asia, que, durante miles de años, la emigración hacia Asia oriental quedó interrumpida. Y no fue hasta el posterior retroceso del Mar Mediterráneo, como consecuencia de la elevación de Arabia, cuando les fue posible llegar a África.
\vs p064 7:2 Y así fue como, durante casi cien mil años, estos pueblos sangiks se diseminaron en torno a las estribaciones de las montañas y se mezclaron en cierta medida entre ellos, a pesar de la aversión peculiar, pero natural, que se había manifestado tempranamente entre las diferentes razas.
\vs p064 7:3 Entre los tiempos del príncipe planetario y los de Adán, la India se convirtió en el lugar de residencia de la población más cosmopolita que se haya visto jamás sobre la faz de la tierra. Pero fue lamentable que esta mezcla contuviese tanta proporción de las razas verde, naranja e índigo. Estos pueblos sangiks secundarios consideraron que tendrían una existencia más fácil y placentera en las tierras del sur y muchos emigraron después a África. Los pueblos sangiks primarios, las razas más dotadas, evitaron los trópicos; el hombre rojo se desplazó al nordeste en dirección a Asia, seguido de cerca por el hombre amarillo, mientras que la raza azul se dirigió al noroeste penetrando en Europa.
\vs p064 7:4 Los hombres rojos empezaron pronto a emigrar hacia el nordeste, inmediatamente tras los hielos que retrocedían. Rodearon las altiplanicies de la India y ocuparon todo el nordeste de Asia. Les siguieron de cerca las tribus amarillas, que acabarían por empujarlos de Asía en dirección a América del Norte.
\vs p064 7:5 Cuando los remanentes, relativamente puros, de la raza roja abandonaron Asia, había once tribus y su número ascendía a algo más de siete mil hombres, mujeres y niños. Estas tribus estaban acompañadas de tres reducidos grupos cuyos ancestros estaban cruzados; el más grande de ellos era una suma de las razas naranja y azul. Estos tres grupos nunca confraternizaron del todo con los hombres rojos, y pronto viajaron al sur hasta México y América Central. Allí más tarde se incorporó un pequeño grupo de amarillos y rojos mezclados. Todos estos pueblos se casaron entre sí y fundaron una nueva raza mixta mucho menos belicosa que los hombres rojos de linaje puro. En el transcurso de cinco mil años, esta raza mestiza se dividió en tres grupos, los creadores de las civilizaciones respectivas de México, América Central y América del Sur. Los descendientes sudamericanos sí recibieron un leve trazo de la sangre de Adán.
\vs p064 7:6 Hasta cierto punto, los primeros hombres rojos y amarillos se mezclaron en Asia, y los vástagos de esta unión viajaron al este y a lo largo de la costa meridional; con el tiempo, la raza amarilla, que aumentaba velozmente, los empujó hacia las penínsulas y las islas cercanas. Estos son los hombres cobrizos de hoy día.
\vs p064 7:7 La raza amarilla continúa ocupando las regiones centrales de Asia oriental. De las seis razas de color, es la que ha sobrevivido en mayor número. Aunque los hombres amarillos se vieron inmersos ocasionalmente en guerras raciales, no se involucraron en las constantes e implacables guerras de exterminio de los hombres rojos, verdes y naranjas. Prácticamente, estas tres razas se destruyeron entre sí antes de ser, finalmente, casi erradicadas por sus enemigos de las otras razas.
\vs p064 7:8 Al no adentrarse tanto el quinto glaciar en el sur de Europa, estos pueblos sangiks tuvieron, parcialmente, vía abierta para emigrar hacia el noroeste; y cuando el hielo se retiró, los hombres azules, junto con otros reducidos grupos raciales, emigraron hacia el oeste por las antiguas rutas de las tribus de Andón. En oleadas sucesivas invadieron Europa, llegando a ocupar la mayor parte del continente.
\vs p064 7:9 En Europa, concurrieron pronto con los descendientes neandertales de Andón, su temprano y común ancestro. El glaciar había empujado a estos neandertales europeos más antiguos hacia el sur y el este, y estaban, por tanto, en situación de enfrentarse y absorber rápidamente a sus primos invasores de las tribus sangiks.
\vs p064 7:10 A grandes rasgos, y ante todo, las tribus sangiks eran más inteligentes, y mucho mejor dotadas en casi todos los aspectos, que los degradados descendientes de los tempranos hombres andónicos de las llanuras; y la mezcla de estas tribus sangiks con los pueblos neandertales derivó en la inmediata mejora de la raza más antigua. Fue dicha infusión de sangre sangik, más particularmente la del hombre azul, la que produjo en los pueblos neandertales esta notable mejora que se puso de manifiesto en las continuas oleadas de las tribus, cada vez más inteligentes, que barrieron Europa desde el este.
\vs p064 7:11 Durante el siguiente período interglaciar, esta nueva raza neandertal se extendió desde Inglaterra hasta la India. El remanente de la raza azul que había permanecido en la antigua península pérsica se cruzaría más tarde con algunas otras razas, principalmente con la amarilla; el consiguiente mestizaje, que de alguna manera fue más tarde mejorado por la raza violeta de Adán, ha perdurado en las tribus nómadas de piel morena de los árabes modernos.
\vs p064 7:12 \pc Todos las iniciativas que se realicen para identificar la ascendencia sangik de los pueblos modernos han de tener en cuenta la mejora experimentada, más adelante, por las estirpes raciales al mezclarse después con la sangre adánica.
\vs p064 7:13 \pc Las razas mejor dotadas buscaron los climas septentrionales o moderados, mientras que las razas naranja, verde e índigo tendieron a desplazarse sucesivamente hacia África por el puente terrestre recién emergido que separaba al Mediterráneo, en su repliegue hacia el oeste, del Océano Índico.
\vs p064 7:14 El hombre índigo fue el último de los pueblos sangik en emigrar desde su punto de origen racial. Aproximadamente, en el momento en el que el hombre verde exterminaba a la raza naranja en Egipto, y quedaba enormemente debilitado al hacerlo, comenzó el gran éxodo del hombre negro hacia el sur, a través de Palestina, a lo largo de la costa; y, después, cuando estos pueblos índigos, fuertes físicamente, invadieron Egipto, eliminaron al hombre verde con la fuerza abrumadora de su número. Estas razas índigos absorbieron a los remanentes del hombre naranja y a gran parte del linaje del hombre verde; gracias a este cruzamiento racial, algunas tribus índigos mejoraron de forma considerable.
\vs p064 7:15 Y tal como podemos observar, Egipto estuvo dominado primero por el hombre naranja, luego por el verde, seguido por el hombre índigo (o negro) y, aún más tarde, por una raza mixta de hombres índigos, azules y verdes, estos últimos modificados por cruzamiento. Pero mucho antes de la llegada de Adán, los hombres azules de Europa y las razas mezcladas de Arabia habían echado a la raza índigo de Egipto, y la habían empujado hacia el extremo sur del continente africano.
\vs p064 7:16 Al concluirse las emigraciones sangik, las razas verde y naranja han desaparecido; el hombre rojo domina América del Norte; el hombre amarillo, Asia oriental; el hombre azul, Europa; y la raza índigo se ha desplazado a África. La India alberga una mezcla de las razas sangik secundarias, y el hombre cobrizo, una mezcla del rojo y del amarillo, domina las islas frente a la costa asiática. Una raza mestiza, con un potencial de alguna manera superior, ocupa las altiplanicies de América del Sur. Los andonitas más puros viven en las regiones septentrionales extremas de Europa y en Islandia, Groenlandia y el nordeste de América del Norte.
\vs p064 7:17 \pc Durante los períodos de máximo avance del glaciar, las tribus andonitas más occidentales estuvieron a punto de ser desplazadas al mar. Durante años vivieron en una estrecha franja de tierra al sur de la actual isla de Inglaterra. Y fue la tradición acerca de estos repetidos avances glaciares la que les hizo hacerse a la mar cuando finalmente apareció el sexto y último glaciar. Fueron los primeros aventureros marinos. Construyeron barcos y partieron a la búsqueda de nuevas tierras con la esperanza de poder librarse de las aterradoras invasiones de hielo. Y algunos llegaron a Islandia, otros a Groenlandia, pero la inmensa mayoría pereció de hambre y de sed en alta mar.
\vs p064 7:18 Hace algo más de ochenta mil años, poco después de que el hombre rojo accediera al noroeste de América del Norte, la congelación de los mares del norte y el avance de los campos de hielo locales en Groenlandia llevaron a estos descendientes esquimales de los aborígenes de Urantia a buscar una tierra mejor, un nuevo hogar; y lo lograron al cruzar de forma segura los angostos estrechos que entonces separaban a Groenlandia de las masas terrestres del nordeste de América del Norte. Alcanzaron el continente unos dos mil cien años después de que el hombre rojo llegara a Alaska. Más tarde, algunos de los descendientes mixtos del hombre azul viajaron hacia el oeste y se cruzaron con los esquimales más recientes; esta unión resultó ser ligeramente beneficiosa para las tribus esquimales.
\vs p064 7:19 Hace unos cinco mil años, en las costas surorientales de la Bahía de Hudson, tuvo lugar un encuentro fortuito entre una tribu india y un grupo esquimal solitario. Estas dos tribus tuvieron dificultades para comunicarse entre sí, pero muy pronto se casaron entre ellos con el resultado de que los hombre rojos, más numerosos, acabaron por absorber a estos esquimales. Y este constituye el único contacto que tuvo el hombre rojo norteamericano con otro linaje humano hasta hace alrededor de mil años, cuando el hombre blanco desembarcó de forma accidental, por primera vez, en la costa atlántica.
\vs p064 7:20 \pc Las luchas de estas tempranas eras se caracterizaron por el coraje, la valentía e incluso el heroísmo. Y todos lamentamos que muchos de esos rasgos fuertes y admirables de vuestros antepasados primitivos se hayan perdido en las razas de los últimos tiempos. Aunque apreciamos el valor de muchos de los perfeccionamientos propios del progreso de la civilización, echamos de menos la magnífica tenacidad y la espléndida dedicación de vuestros primeros ancestros, que con frecuencia rayaban en la grandeza y en lo sublime.
\vsetoff
\vs p064 7:21 [Exposición de un portador de vida, residente en Urantia.]
