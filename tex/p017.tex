\upaper{17}{Los siete grupos de espíritus supremos}
\author{Consejero divino}
\vs p017 0:1 Los siete grupos de espíritus supremos son los directores universales que coordinan la administración de los siete segmentos del gran universo. Aunque todos forman parte de la familia operativa del Espíritu Infinito, a los tres grupos siguientes se les agrupa generalmente dentro de los hijos de la Trinidad del Paraíso:
\vs p017 0:2 \li{1.}Los siete espíritus mayores.
\vs p017 0:3 \li{2.}Los siete mandatarios supremos.
\vs p017 0:4 \li{3.}Los espíritus reflectores.
\vs p017 0:5 \pc Los cuatro grupos restantes deben su existencia a los actos creativos del Espíritu Infinito o a sus colaboradores de condición creativa:
\vs p017 0:6 \li{4.}Los auxiliares reflectores de imagen.
\vs p017 0:7 \li{5.}Los siete espíritus de las vías.
\vs p017 0:8 \li{6.}Los espíritus creativos del universo local.
\vs p017 0:9 \li{7.}Los espíritus asistentes de la mente.
\vs p017 0:10 \pc A estos siete órdenes se les conoce en Uversa como los siete grupos de espíritus supremos. Su campo de acción se extiende desde la presencia personal de los siete espíritus mayores en la periferia de la Isla eterna, a través de los siete satélites del Espíritu en el Paraíso, las vías planetarias de Havona, los gobiernos de los suprauniversos y la administración y supervisión de los universos locales, hasta incluir incluso el servicio más humilde de los espíritus asistentes, que se otorgan en los ámbitos de la mente evolutiva a los mundos del tiempo y del espacio.
\vs p017 0:11 Los siete espíritus mayores son los directores que coordinan este extenso ámbito administrativo. En algunos asuntos relativos a la regulación central de la potencia física organizada, de la energía mental y del ministerio impersonal del espíritu obran de forma personal y directa, mientras que, en otros, lo hacen a través de sus diversos colaboradores. En todos los asuntos de naturaleza ejecutiva ---resoluciones, reglamentaciones, modificaciones y decisiones administrativas--- los espíritus mayores actúan en las personas de los siete mandatarios supremos. En el universo central, los espíritus mayores pueden ejercer su labor a través de los siete espíritus de las vías de Havona; en las sedes centrales de los siete suprauniversos, se manifiestan mediante los espíritus reflectores y actúan a través de las personas de los ancianos de días, con quienes están en comunicación personal a través de los auxiliares reflectores de imagen.
\vs p017 0:12 Los siete espíritus mayores no participan de forma directa y personal en la administración del universo por debajo del nivel de los tribunales de los ancianos de días. Vuestro universo local está administrado por el espíritu mayor de Orvontón como parte integrante de nuestro suprauniverso, pero tal función respecto de los seres originarios de Nebadón enseguida la asume y dirige personalmente el espíritu materno creativo residente en Lugar de Salvación, sede central de vuestro universo local.
\usection{1. LOS SIETE MANDATARIOS SUPREMOS}
\vs p017 1:1 Las sedes rectoras de los espíritus mayores ocupan en el Paraíso los siete satélites del Espíritu Infinito, que giran alrededor de la Isla central entre las resplandecientes esferas del Hijo Eterno y la vía circulatoria más interior de Havona. Dichas esferas rectoras están bajo la dirección de los siete mandatarios supremos, un grupo de siete seres trinitizados por el Padre, el Hijo y el Espíritu de acuerdo con las instrucciones dadas por los siete espíritus mayores, que solicitaban la creación de unos seres que los pudieran representar de forma universal.
\vs p017 1:2 Los espíritus mayores se mantienen en contacto con las diferentes divisiones de los gobiernos del suprauniverso a través de estos siete mandatarios supremos. Son ellos quienes determinan, en gran medida, las tendencias fundamentales que caracterizan a los siete suprauniversos. Son coherentes y divinamente perfectos, pero también difieren en cuanto a su ser personal. No tienen quien los dirija de forma permanente; cada vez que se reúnen, eligen a uno entre ellos para que presida ese consejo conjunto. Viajan con periodicidad al Paraíso para reunirse en consejo con los siete espíritus mayores.
\vs p017 1:3 \pc Los siete mandatarios supremos tienen la labor de coordinar la administración del gran universo; se les podría denominar la junta de directores que gobierna la creación posterior a Havona. Aunque no se ocupan de los asuntos internos del Paraíso y dirigen unas limitadas parcelas de actividad de Havona a través de los siete espíritus de las vías, la magnitud de su supervisión, por lo demás, conoce pocos límites. Se encargan de la dirección de los asuntos de orden físico, intelectual y espiritual; todo lo ven, todo lo oyen, todo lo sienten, incluso saben todo lo que ocurre en los siete suprauniversos y en Havona.
\vs p017 1:4 Estos mandatarios no dictan normas ni modifican procedimientos relativos al universo; se ocupan de llevar a cabo los planes divinos promulgados por los siete espíritus mayores. Tampoco interfieren con el gobierno de los ancianos de días en los suprauniversos ni con la soberanía de los hijos creadores en los universos locales. Su labor es coordinar y dar cumplimiento al conjunto de normativas dictadas por los gobernantes del gran universo convenientemente nombrados.
\vs p017 1:5 Cada uno de estos mandatarios supremos, junto con la dotación de que dispone su esfera, se dedica eficientemente a la administración de un solo suprauniverso. El número uno, que obra en la esfera rectora número uno, está totalmente dedicado a los asuntos del suprauniverso número uno, y así sucesivamente hasta el número siete, que opera desde el séptimo satélite del Espíritu, en el Paraíso, y que dedica sus energías a la administración del séptimo suprauniverso. El nombre de la séptima esfera es Orvontón, porque los satélites del Paraíso se denominan igual que los suprauniversos con los que se relacionan; de hecho, los suprauniversos llevan su nombre.
\vs p017 1:6 Los encargados de mantener en orden los asuntos de Orvontón desde la esfera rectora del séptimo suprauniverso se cuentan en números que sobrepasan la comprensión humana e incluyen prácticamente a cualquier orden de inteligencia celestial. Todo el servicio de envío de seres personales que se realiza en el suprauniverso (con excepción de los espíritus inspirados de la Trinidad y de los modeladores del pensamiento), en sus viajes por el universo desde o hasta el Paraíso, pasa por uno de estos mundos rectores, y aquí se mantienen los archivos centrales para todos los seres personales creados por la Tercera Fuente y Centro que actúan en los suprauniversos. El sistema de archivos materiales, morontiales y espirituales existente en cada uno de estos mundos rectores del Espíritu causa asombro incluso a un ser del orden al que pertenezco.
\vs p017 1:7 Los subordinados directos de los mandatarios supremos están formados mayoritariamente por los hijos trinitizados de seres personales del Paraíso\hyp{}Havona y por la progenie trinitizada de los mortales glorificados que acabaron su larga formación siguiendo el método de ascensión del tiempo y del espacio. El jefe del consejo supremo del colectivo final del Paraíso es quien asigna a estos hijos trinitizados al servicio de dichos seres.
\vs p017 1:8 Cada mandatario dispone de dos órganos de consulta: en la sede central de cada suprauniverso, los hijos del Espíritu Infinito eligen, de entre ellos mismos, representantes que sirvan durante un milenio en el principal órgano de consulta de su respectivo mandatario supremo. Para todos los asuntos que afectan a los mortales ascendentes del tiempo existe un órgano de orden secundario formado por los mortales que han alcanzado el Paraíso y por los hijos trinitizados de mortales glorificados; este colectivo lo eligen los seres en proceso de perfección y ascensión, que moran de forma transitoria en las siete sedes centrales de los siete suprauniversos. Los mandatarios supremos nombran a todos los encargados de otros asuntos.
\vs p017 1:9 \pc Periódicamente tienen lugar grandes cónclaves en estos satélites del Espíritu que circundan el Paraíso. Los hijos trinitizados que se asignan a estos mundos, junto con los ascendentes que han llegado al Paraíso, se congregan con los seres personales espirituales de la Tercera Fuente y Centro para recordar las pugnas y triunfos por los que pasaron en su andadura de ascensión. Los mandatarios supremos siempre presiden estas reuniones fraternales.
\vs p017 1:10 Una vez cada milenio del Paraíso, estos dejan sus sedes de gobierno para ir al Paraíso donde celebran un cónclave milenario de salutación y de buenos deseos universales para las multitudes de seres inteligentes de la creación. Este momento tan trascendental tiene lugar ante la presencia directa de Majestón, jefe de todos los grupos de espíritus reflectores. Así, a través del singular funcionamiento de la reflectividad universal, se pueden comunicar de forma simultánea con todos sus colaboradores en el gran universo.
\usection{2. MAJESTÓN: JEFE DE LA REFLECTIVIDAD}
\vs p017 2:1 Los espíritus reflectores tienen su origen divino en la Trinidad. Existen cincuenta de estos singulares y un tanto misteriosos seres. Estos extraordinarios seres personales se crearon en grupos de siete, y este episodio creativo se efectuó mediante la conjunción de la Trinidad del Paraíso y uno de los siete espíritus mayores.
\vs p017 2:2 Tal trascendental acontecimiento, ocurrido en los albores del tiempo, significa el esfuerzo inicial de las personas creadoras supremas, representadas por los espíritus mayores, para obrar como cocreadores con la Trinidad del Paraíso. Esta unión del poder creativo de los creadores supremos con los potenciales creativos de la Trinidad constituye la fuente misma de la realidad del Ser Supremo. Por tanto, cuando el ciclo de creación de la reflectividad cumplió su curso, cuando cada uno de los siete espíritus mayores halló una perfecta sincronía en el terreno creativo con la Trinidad del Paraíso, cuando el espíritu reflector número cuarenta y nueve adquirió su ser personal, se dio en el Absoluto de la Deidad una reacción nueva de grandes proporciones que impartió nuevas prerrogativas personales al Ser Supremo y culminó en la manifestación personal de Majestón. Majestón es el jefe de la reflectividad y centro en el Paraíso de toda la labor de los cuarenta y nueve espíritus reflectores y de sus colaboradores en todo el universo de los universos.
\vs p017 2:3 Majestón es una persona auténtica; es el centro personal infalible de los fenómenos de la reflectividad en los siete suprauniversos del tiempo y del espacio. Él mantiene su sede central permanente en el Paraíso cerca del centro de todas las cosas, en el punto de encuentro de los siete espíritus mayores. Se ocupa exclusivamente de la coordinación y el mantenimiento del servicio de reflectividad para la extensa creación; no participa de ninguna otra manera en la administración de los asuntos del universo.
\vs p017 2:4 No se ha incluido a Majestón en nuestra relación de seres personales del Paraíso, porque es el único ser personal divino existente que haya sido creado por el Ser Supremo en acción conjunta con el Absoluto de la Deidad. Es una persona, pero se ocupa en exclusividad, y al parecer de forma característica, de este único aspecto de la eficaz organización del universo; hasta el presente no tiene competencia personal alguna con relación a otros órdenes (no reflectores) de seres personales del universo.
\vs p017 2:5 \pc La creación de Majestón significó la primera acción suprema creativa del Ser Supremo. Esto demostraba una voluntariedad de acción en el Ser Supremo, pero no se anticipaba esta formidable reacción del Absoluto de la Deidad. Desde la aparición de Havona en la eternidad, el universo no había presenciado tan extraordinaria efectuación de una alineación tan gigantesca e inmensa de la potencia y de la coordinación de la actividad espiritual operativa. La respuesta de la Deidad a las voluntades creativas del Ser Supremo y de sus colaboradores fue extremadamente más allá de su decidido intento y sobrepasó considerablemente sus previsiones conceptuales.
\vs p017 2:6 Nos sentimos sobrecogidos ante la posibilidad de lo que las eras futuras, en las que el Supremo y el Último puedan alcanzar nuevos niveles de divinidad y ascender a nuevas áreas de acción personal, puedan ser testigos, en los ámbitos de la deidización de otros seres incluso más imprevisibles y jamás soñados, que posean poderes nunca imaginados para mejorar la coordinación del universo. Parece no existir límite para el potencial de respuesta del Absoluto de la Deidad a tal unificación de las relaciones entre la Deidad experiencial y la Trinidad existencial del Paraíso.
\usection{3. LOS ESPÍRITUS REFLECTORES}
\vs p017 3:1 Los cuarenta y nueve espíritus reflectores tienen su origen en la Trinidad, pero cada uno de los siete momentos creativos que propiciaron su aparición produjo un tipo de ser con características parecidas a las del espíritu mayor con quien comparte su ancestro. Por tanto, reflejan de forma diferente la naturaleza y el carácter de las siete combinaciones posibles de las características divinas del Padre Universal, del Hijo Eterno y del Espíritu Infinito. Por esta razón, es necesario que haya siete de estos espíritus reflectores en las sedes centrales de cada suprauniverso. Se requiere la presencia de cada una de estas siete clases de seres, con el fin de conseguir reflejar perfectamente cualquier forma posible de manifestación de las tres Deidades del Paraíso, puesto que dichos fenómenos pueden ocurrir en cualquier parte de los siete suprauniversos. Se asignó, en consecuencia, a uno de cada tipo al servicio de cada uno de los suprauniversos. Estos grupos de siete espíritus reflectores diferentes mantienen sus sedes centrales en las capitales de los suprauniversos, en el foco reflector de cada zona, que no es el mismo que el centro de polaridad espiritual.
\vs p017 3:2 A los espíritus reflectores se les ha asignado un nombre, pero estos no se han revelado a los mundos del espacio, porque aluden a la naturaleza y al carácter de estos seres y forman parte de uno de los siete misterios universales de las esferas secretas del Paraíso.
\vs p017 3:3 El atributo de la reflectividad, el fenómeno relativo a los niveles mentales del Actor Conjunto, del Ser Supremo y de los espíritus mayores, se transmite a todos los seres que se ocupan de llevar a cabo este inmenso plan de información universal. Y he aquí un gran misterio: ni los espíritus mayores ni las Deidades del Paraíso, de forma individual o conjunta, muestran estos poderes correlacionados de la reflectividad universal tal como se manifiestan en estos cuarenta y nueve seres personales de enlace de Majestón y, sin embargo, son los creadores de todos estos seres tan maravillosamente dotados. La herencia divina a veces hace que se manifiesten en las criaturas ciertos atributos que no se perciben en su creador.
\vs p017 3:4 Con la excepción de Majestón y de los espíritus reflectores, el personal a cargo del servicio de la reflectividad son todas criaturas del Espíritu Infinito y de sus inmediatos colaboradores y subordinados. Los espíritus reflectores de cada suprauniverso son los creadores de sus auxiliares reflectores de imagen, sus portavoces personales ante los tribunales de los ancianos de días.
\vs p017 3:5 \pc Los espíritus reflectores no son meros instrumentos de transmisión; también son seres personales con facultades memorísticas. Sus vástagos, los seconafines, son igualmente seres personales con facultades memorísticas o archivísticas. Todo lo que tenga verdadero valor espiritual se archiva por duplicado, y su huella queda impresa como parte del conocimiento personal de algún miembro de los numerosos órdenes de seres personales secoráficos pertenecientes al gran número de ayudantes de los espíritus reflectores.
\vs p017 3:6 Los archivos regulares de los universos se transmiten a través de los ángeles archivistas, pero los verdaderos datos de índole espiritual se recogen mediante la reflectividad y se conservan en las mentes de seres personales pertenecientes a la familia del Espíritu Infinito aptos para tal fin. Estos son los documentos \bibemph{vivos} a diferencia de los regulares e \bibemph{inertes} del universo, y se conservan en perfección en la mente viva de los seres personales de facultades archivísticas del Espíritu Infinito.
\vs p017 3:7 El sistema de la reflectividad constituye también el procedimiento usado en toda la creación para la recolección de datos y la promulgación de decretos. Está en constante operación a diferencia de los distintos servicios de transmisión que se realizan de forma periódica.
\vs p017 3:8 Todo lo que suceda de importancia en la sede central de un universo local se refleja de forma natural en la capital de su suprauniverso. Y por el contrario, todo lo que tenga significación para el universo local se refleja en dirección a las capitales de los universos locales desde la sede central de sus suprauniversos. El servicio de reflectividad desde los universos del tiempo a los suprauniversos parece operar de forma automática o por sí mismo, pero en realidad no es así. Es todo muy personal e inteligente; su precisión es el resultado de la perfecta cooperación de seres personales y, por tanto, difícilmente podría atribuirse a la presencia\hyp{}actuación impersonal de los Absolutos.
\vs p017 3:9 Aunque los modeladores del pensamiento no participan en la operación del sistema universal de la reflectividad, tenemos motivos para creer que cada fracción del Padre tiene un conocimiento pleno de este hecho y es capaz de recurrir a la reflectividad para obtener información.
\vs p017 3:10 \pc Durante la actual era del universo, el alcance espacial del servicio de reflectividad realizado más allá del Paraíso parece limitarse a la periferia de los siete suprauniversos. Por otra parte, este servicio parece desempeñarse con independencia del tiempo y del espacio. Es al parecer independiente de todas las vías circulatorias conocidas del universo a nivel subabsoluto.
\vs p017 3:11 En la sede central de cada suprauniverso, el sistema de reflectividad sirve de unidad separada; pero en ciertas ocasiones especiales, bajo la dirección de Majestón, los siete pueden actuar al unísono universal, como ante la celebración de júbilo que sucede al asentamiento de todo un universo local en luz y vida, y en los momentos de las salutaciones milenarias de los siete mandatarios supremos.
\usection{4. LOS AUXILIARES REFLECTORES DE IMAGEN}
\vs p017 4:1 Los cuarenta y nueve auxiliares reflectores de imagen fueron creados por los espíritus reflectores. Hay exactamente siete de estos auxiliares en la sede central de cada suprauniverso. La primera acción creativa de los siete espíritus reflectores de Uversa fue dar origen a sus siete auxiliares de imagen. Cada espíritu reflector crea a su propio auxiliar. Los auxiliares de imagen son, con respecto a algunos de sus atributos y características, copias perfectas de sus espíritus reflectores maternos; son prácticamente duplicados menos en lo que respecta al atributo de la reflectividad. Son verdaderas imágenes y actúan constantemente como canal de comunicación entre los espíritus reflectores y las autoridades del suprauniverso. Los auxiliares de imagen no son simplemente ayudantes; son representaciones reales de los respectivos espíritus en donde tienen su ancestro; son \bibemph{imágenes,} tal como su nombre indica.
\vs p017 4:2 Los espíritus reflectores mismos son auténticos seres personales, pero de un orden que resulta incomprensible para los seres materiales. Incluso en la esfera sede del suprauniverso se requiere la asistencia de sus auxiliares de imagen para relacionarse de forma personal con los ancianos de días y con sus colaboradores. A veces, en los contactos entre los auxiliares de imagen y los ancianos de días, puede ser suficiente con un auxiliar, mientras que en otras ocasiones se requieren dos, tres, cuatro, e incluso los siete para poder transmitir de forma completa y apropiada la información que se les ha confiado. Del mismo modo, uno, dos, o los tres ancianos de días podrían recibir los mensajes de los auxiliares de imagen, según lo requiera el contenido de la información.
\vs p017 4:3 Los auxiliares de imagen sirven por siempre junto a los espíritus de los que descienden, y tienen a su disposición, como ayudantes, a una increíble multitud de seconafines. Los auxiliares de imagen no actúan directamente en relación con los mundos de formación de los mortales ascendentes. Están estrechamente vinculados con el servicio de información relativo al sistema universal usado para el progreso de los mortales, pero vosotros, cuando residáis en las escuelas de Uversa, no entraréis en contacto personal con ellos porque estos seres, aparentemente personales, no poseen voluntad; no tienen capacidad de decisión. Son auténticas imágenes que reflejan completamente el ser personal y la mente del espíritu al que deben su ancestro. Como clase, los mortales que ascienden no entran en contacto estrecho con la reflectividad. Siempre habrá algún ser de naturaleza reflectora que intervendrá entre vosotros y el servicio mismo que estéis realizando.
\usection{5. LOS SIETE ESPÍRITUS DE LAS VÍAS}
\vs p017 5:1 Los siete espíritus de las vías circulatorias de Havona constituyen la representación conjunta e impersonal del Espíritu Infinito y de los siete espíritus mayores en las vías que circundan el universo central. Son los servidores de los espíritus mayores, sus progenitores colectivos. Los espíritus mayores llevan a la administración de los siete suprauniversos su bien diferenciada individualidad. A través de estos homogéneos espíritus de las vías de Havona, son capaces de proporcionar al universo central, de forma individual, una supervisión espiritual unificada, homogénea y coordinada.
\vs p017 5:2 Cada uno de estos siete espíritus permea solamente una de las vías de Havona. No se ocupan de forma directa de los regímenes de los eternos de días, los gobernantes de cada uno de los mundos de Havona. No obstante, están en vinculación con los siete mandatarios supremos y en sincronía con la presencia en el universo central del Ser Supremo. Su labor se ciñe exclusivamente a Havona.
\vs p017 5:3 Estos espíritus de las vías se ponen en contacto con los que residen en Havona a través de su progenie de seres personales, los supernafines terciarios. Aunque los espíritus de las vías coexisten con los siete espíritus mayores, su actuación en la creación de los supernafines terciarios no cobró mayor importancia hasta que, en los días de Granfanda, los primeros peregrinos del tiempo llegaron a la vía planetaria más exterior de Havona.
\vs p017 5:4 Conoceréis a los espíritus de las vías a medida que avanzáis vía tras vía en Havona, pero no os podréis comunicar personalmente con ellos, aunque podáis gozar personalmente de su influencia espiritual y reconocer su presencia impersonal.
\vs p017 5:5 Los espíritus de las vías se relacionan con los habitantes originarios de Havona de la misma manera que los modeladores del pensamiento se relacionan con las criaturas mortales que habitan los mundos de los universos evolutivos. Como los modeladores del pensamiento, estos espíritus son impersonales, y acompañan a las mentes perfectas de los seres de Havona de la misma manera que los espíritus impersonales del Padre Universal moran en las mentes finitas de los hombres mortales. Pero los espíritus de las vías jamás se convierten en partes permanentes de los seres personales de Havona.
\usection{6. LOS ESPÍRITUS CREATIVOS DEL UNIVERSO LOCAL}
\vs p017 6:1 Mucho de lo que se refiere a la naturaleza y labor de los espíritus creativos del universo local pertenece más bien al relato de su relación con los hijos creadores respecto a la organización y dirección de las creaciones locales; pero existen muchos rasgos relativos a las experiencias de estos seres maravillosos, con anterioridad a la creación del universo local, que se pueden narrar como parte de la exposición que estamos realizando sobre los siete grupos de espíritus supremos.
\vs p017 6:2 \pc Conocemos seis etapas en la andadura del espíritu materno de un universo local, y nos hacemos muchas conjeturas sobre la probabilidad de un séptimo estadio de actividad, de la séptima. Estas diferentes etapas en cuanto a su existencia son:
\vs p017 6:3 \li{1.}\bibemph{Diferenciación inicial en el Paraíso}. Cuando un hijo creador se hace personal mediante la acción conjunta del Padre Universal y del Hijo Eterno, ocurre de forma simultánea en la persona del Espíritu Infinito lo que se conoce como “suprema reacción complementaria”. No comprendemos la naturaleza de esta reacción, pero entendemos que conlleva una modificación inherente a la capacidad de concesión del estado personal incluida en el potencial creativo del Creador Conjunto. El nacimiento de un hijo creador de igual rango señala a su vez el nacimiento en la persona del Espíritu Infinito del potencial de creación de una futura consorte de este hijo del Paraíso para el universo local. No tenemos conocimiento de que se establezca la identidad de este nuevo ser prepersonal, pero sabemos que este hecho se encuentra en los registros existentes en el Paraíso relativos a la andadura de dicho hijo creador.
\vs p017 6:4 \li{2.}\bibemph{La formación preliminar como creadora}. Durante el prolongado período de formación preliminar de un hijo miguel necesario para la organización y administración de los universos, su futura consorte experimenta un mayor desarrollo de su ser y se torna consciente de ser parte de un grupo destinado a un fin. No lo sabemos, pero sospechamos que, al hacerse consciente de ser parte de un grupo, ese ser empieza a tener conocimiento del espacio y comienza esa instrucción preliminar necesaria para adquirir una habilidad espiritual que la capacite en su futura labor como colaboradora de Miguel y complementaria en la creación y administración del universo.
\vs p017 6:5 \li{3.}\bibemph{La etapa de la creación física}. En el momento en que el Hijo Eterno dispone que el hijo miguel cumpla con su responsabilidad de crear, el espíritu mayor que rige el suprauniverso al que está destinado este nuevo hijo creador pronuncia la “plegaria del establecimiento de identidad” en presencia del Espíritu Infinito y, por primera vez, la identidad del futuro espíritu creativo aparece como diferenciada de la persona del Espíritu Infinito. Y dirigiéndose hacia la persona del espíritu mayor que la solicitó, este ser se pierde de forma inmediata de nuestra percepción, tornándose aparentemente parte de la persona de ese espíritu mayor. El espíritu creativo recién identificado permanece con el espíritu mayor hasta el momento de la partida del hijo creador hacia la aventura del espacio; en ese momento, este espíritu mayor confía el nuevo espíritu consorte a los cuidados del hijo creador, disponiendo al mismo tiempo que el espíritu consorte cumpla con su responsabilidad de ser eternamente fiel e ilimitadamente leal. Y luego ocurre uno de los momentos más profundamente emocionantes que pueden tener lugar en el Paraíso. El Padre Universal habla en reconocimiento de la unión eterna del hijo creador y del espíritu creativo y en confirmación del otorgamiento de determinados poderes gobernativos conjuntos de parte del espíritu mayor que tiene jurisdicción en ese suprauniverso.
\vs p017 6:6 El hijo creador y el espíritu creativo, unidos por el Padre, parten luego hacia su trepidante creación del universo. Y realizan su tarea así vinculados durante todo el largo y arduo período de la organización material de su universo.
\vs p017 6:7 \li{4.}\bibemph{La era de la creación de vida}. Cuando el hijo creador declara su intención de crear vida, comienzan en el Paraíso las “ceremonias de dotación del ser personal”, en las que toman parte los siete espíritus mayores, y que el espíritu mayor supervisor experimenta personalmente. Esto es una contribución de la Deidad del Paraíso al ser individual del espíritu consorte del hijo creador que se manifiesta al universo en el fenómeno de la “erupción primaria” en la persona del Espíritu Infinito. Con simultaneidad a este fenómeno que tiene lugar en el Paraíso, el espíritu consorte del hijo creador, que hasta ese momento había sido impersonal, se vuelve, en todos los sentidos, una auténtica persona. De ahí en adelante y, para siempre, este mismo espíritu materno del universo local será considerado persona y mantendrá relaciones personales con todo el gran número de seres personales de esa vida que se va a crear a continuación.
\vs p017 6:8 \li{5.}\bibemph{Las eras posteriores al ministerio de gracia.} Otro cambio importante ocurre en la andadura sin fin de un espíritu creativo cuando el hijo creador regresa a la sede central del universo después de completar su séptimo ministerio de gracia, tras haber adquirido la soberanía plena del universo. En esa ocasión, ante una asamblea de administradores de gobierno del universo, el triunfante hijo creador concede al espíritu materno del universo la cosoberanía y reconoce a este espíritu consorte como su igual.
\vs p017 6:9 \li{6.}\bibemph{Las eras de luz y vida}. Cuando se establece la era de luz y vida, la cosoberana de un universo local inicia la sexta etapa en su andadura como espíritu creativo. Pero no podemos describir la naturaleza de esta profunda experiencia. Estas cosas pertenecen a un estadio futuro en la evolución de Nebadón.
\vs p017 6:10 \li{7.}\bibemph{La andadura no revelada}. Sabemos de estas seis etapas en la andadura de un espíritu materno del universo local. Es inevitable que nos preguntemos si existe una séptima. Somos conscientes de que, cuando los finalizadores logran lo que parece ser el destino final de su ascensión como mortales, inician oficialmente su andadura como espíritus de la sexta etapa. Suponemos que todavía aguarda a los finalizadores, en su otra trayectoria no revelada, algún cometido que realizar en el universo. Es lógico esperar, pues, que los espíritus maternos del universo tengan ante ellos, de igual manera, una etapa no desvelada de sus trayectorias, un séptimo estadio en el que su experiencia personal se ponga al servicio del universo y en el que coopere lealmente con el orden de los migueles creadores.
\usection{7. LOS ESPÍRITUS ASISTENTES DE LA MENTE}
\vs p017 7:1 Estos espíritus asistentes constituyen el séptuplo don de la mente del espíritu materno de un universo local a las criaturas vivas de la creación conjunta de un hijo creador y de ese espíritu creativo. Este don se hace posible en el momento en que se concede a este espíritu su condición y prerrogativas como persona. El relato de la naturaleza y la actuación de los siete espíritus asistentes de la mente pertenecen, más bien, a la historia de vuestro universo local de Nebadón.
\usection{8. LA LABOR DE LOS ESPÍRITUS SUPREMOS}
\vs p017 8:1 Los siete grupos de espíritus supremos conforman el núcleo de la familia operativa de la Tercera Fuente y Centro tanto como Espíritu Infinito y como Actor Conjunto. El ámbito de los espíritus supremos se extiende desde la presencia de la Trinidad en el Paraíso hasta la actuación de la mente de orden evolutivo de los mortales de los planetas del espacio. De esta manera unifican los niveles descendentes de la administración y coordinan todas las diferentes actividades de los seres que trabajan en esta. Bien sea un grupo de espíritus reflectores en conjunción con los ancianos de días, un espíritu creativo actuando en conjunción con un hijo miguel o bien los siete espíritus mayores encauzados en la vía circulatoria en torno a la Trinidad del Paraíso, la actividad de los espíritus supremos se percibe por todos los lugares del universo central, de los suprauniversos y de los universos locales. Obran del mismo modo con los seres personales de la Trinidad del orden de “días” y con los seres personales del Paraíso del orden de los “hijos”.
\vs p017 8:2 \pc Junto con su Espíritu Materno Infinito, los grupos de espíritus supremos son los creadores directos de la inmensa familia de criaturas de la Tercera Fuente y Centro. Todos los órdenes de espíritus servidores surgen de esta vinculación. El Espíritu Infinito da origen a los supernafines primarios; los espíritus mayores crean a los seres secundarios de este orden; los siete espíritus de las vías, a los supernafines terciarios. Los espíritus reflectores, de forma conjunta, son los progenitores madre de un orden maravilloso de multitudes angélicas: los poderosos seconafines que sirven en el suprauniverso. El espíritu creativo es la madre de los órdenes angélicos de su creación local; estos servidores seráficos son exclusivos en cada universo local, aunque siguen el modelo del universo central. Todos los creadores de estos espíritus se benefician indirectamente del Espíritu Infinito ---la madre primigenia y eterna de todos los servidores angélicos---, que permanece personalmente en su morada central.
\vs p017 8:3 \pc Los siete grupos de espíritus supremos coordinan la creación habitada. En vinculación, sus dirigentes, los siete espíritus mayores, parecen coordinar la extensa actividad del Dios Séptuplo:
\vs p017 8:4 \li{1.}De forma colectiva, los espíritus mayores casi corresponden al nivel divino de la Trinidad de las Deidades del Paraíso.
\vs p017 8:5 \li{2.}De forma individual, agotan las posibilidades primarias de combinación de la Deidad trina.
\vs p017 8:6 \li{3.}Al representar de forma diversa al Actor Conjunto, se constituyen en depositarios de esa soberanía de espíritu\hyp{}mente\hyp{}potencia del Ser Supremo que todavía no ejerce de forma personal.
\vs p017 8:7 \li{4.}A través de los espíritus reflectores sincronizan los gobiernos del suprauniverso de los ancianos de días con Majestón, centro de la reflectividad universal en el Paraíso.
\vs p017 8:8 \li{5.}Al participar en la individualización de las benefactoras divinas del universo local, los espíritus mayores contribuyen al último nivel de Dios Séptuplo, a la unión hijo creador\hyp{}espíritu creativo de los universos locales.
\vs p017 8:9 \pc La unidad de acción, consustancial en el Actor Conjunto, se desvela a los universos en evolución en los siete espíritus mayores, sus seres personales primarios. Pero en los suprauniversos perfeccionados del futuro esta unidad será sin duda inseparable de la soberanía experiencial del Supremo.
\vsetoff
\vs p017 8:10 [Exposición de un consejero divino de Uversa.]
