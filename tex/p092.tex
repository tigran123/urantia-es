\upaper{92}{Evolución posterior de la religión}
\author{Melquisedec}
\vs p092 0:1 Mucho antes de que se efectuara en Urantia de forma sistemática cualquiera de las revelaciones, el hombre poseía una religión de origen \bibemph{natural} como parte de su experiencia evolutiva. Pero dicha religión era, en sí misma, el fruto de la dotación del hombre fuera del ámbito de lo animal. La religión evolutiva surgió lentamente durante los milenios de andadura experiencial de la humanidad, a través del ministerio de las siguientes influencias que obraban, e incidían, en el hombre salvaje, el primitivo y el civilizado:
\vs p092 0:2 \li{1.}\bibemph{El asistente de la adoración:} la aparición en la conciencia animal de potenciales supranimales para la percepción de la realidad; podría denominarse el instinto humano primigenio hacia la Deidad.
\vs p092 0:3 \li{2.}\bibemph{El asistente de la sabiduría:} la manifestación en la mente adoradora de la tendencia a dirigir su adoración en canales superiores de expresión y hacia crecientes conceptos de la realidad de la Deidad.
\vs p092 0:4 \li{3.}\bibemph{El espíritu santo:} se trata del don inicial de la supramente e, invariablemente, aparece en todos los seres personales genuinos. A través de su ministerio, se crea, en la mente que ansía la adoración y desea la sabiduría la capacidad de realizar en sí misma los principios fundamentales de la supraexperiencia humana, no solo en su concepción teológica sino en cuanto a la experiencia actual y efectiva del ser personal.
\vs p092 0:5 \pc La actuación correlacionada de estos tres ministerios divinos es más que suficiente para iniciar y perseverar en el crecimiento de la religión evolutiva. Estas influencias se incrementan, más tarde, mediante los modeladores del pensamiento, los serafines y el espíritu de la verdad, todos los cuales aceleran el grado de desarrollo religioso. Estas influencias han obrado desde hace mucho tiempo en Urantia, y continuarán aquí siempre que el planeta siga siendo una esfera habitada. Una gran parte del potencial de dichas influencias divinas nunca ha tenido la oportunidad de manifestarse; muchas cosas se revelarán en eras venideras conforme la religión humana ascienda, nivel a nivel, hacia las supremas alturas del valor morontial y de la verdad espiritual.
\usection{1. NATURALEZA EVOLUTIVA DE LA RELIGIÓN}
\vs p092 1:1 La evolución de la religión se remonta al temor primitivo y a los espectros, pasando por muchas etapas de desarrollo consecutivas, incluidos aquellos intentos primeramente por coaccionar y, luego, por halagar a los espíritus. Los fetiches de la tribu se convirtieron en tótems y en dioses tribales; las fórmulas mágicas se volvieron oraciones modernas. La circuncisión, que al principio era un sacrificio, se transformó en un procedimiento higiénico.
\vs p092 1:2 Durante toda la infancia salvaje de las razas, la religión progresó desde la adoración de la naturaleza hasta el fetichismo, pasando por la adoración de los espectros. Con el despertar de la civilización, la raza humana hizo suyas las creencias más místicas y simbólicas, mientras que ahora, acercándose a su madurez, la humanidad está creciendo en el reconocimiento de la verdadera religión, incluso hasta acercarse a la revelación de la verdad misma.
\vs p092 1:3 La religión surge como respuesta biológica de la mente a las creencias espirituales y al entorno; es lo último que perece o cambia en una raza. La religión es la adaptación de la sociedad, en cualquier época, a aquello que es misterioso. Como institución social consta de ritos, símbolos, sistemas de culto, escrituras, altares, santuarios y templos. El agua bendita, las reliquias, los fetiches, los amuletos, las vestimentas, las campanas, los tambores y el sacerdocio son comunes a todas las religiones. Y es imposible separar enteramente la religión puramente evolutiva de la magia o la brujería.
\vs p092 1:4 El misterio y el poder han suscitado, constantemente, sentimientos y temores religiosos, mientras que la emoción ha actuado siempre como un factor condicionante y poderoso en su desarrollo. De forma continua, el temor ha sido un estímulo religioso fundamental; ha dado forma a los dioses de la religión evolutiva y ha motivado el ritual religioso de los creyentes primitivos. A medida que la civilización avanza, el temor se modifica por la veneración, la admiración, el respeto y la comprensión y, entonces, se condiciona además por el remordimiento y el arrepentimiento.
\vs p092 1:5 Un pueblo asiático enseñaba que “Dios es un temor grande”; ese es el resultado de una religión puramente evolutiva. Jesús, que reveló el orden más elevado de vida religiosa, proclamó que “Dios es amor”
\usection{2. RELIGIÓN Y COSTUMBRES}
\vs p092 2:1 La religión es la más rígida e inflexible de todas las instituciones humanas, aunque es cierto que tardíamente se adapta a una sociedad cambiante. Con el tiempo, la religión evolutiva refleja, de hecho, el cambio de las costumbres que, a su vez, pueden haberse visto afectadas por la religión revelada. Lentamente, con seguridad, pero reticentemente, la religión (la adoración) sigue en verdad la estela de la sabiduría: el conocimiento dirigido por la razón experiencial e iluminado por la revelación divina.
\vs p092 2:2 La religión se aferra a las costumbres; aquello que \bibemph{era} es ancestral y presuntamente sagrado. Por esta razón y no por otra, los utensilios de piedra perduraron durante largo tiempo en la edad del bronce y del hierro. Hay constancia de esta afirmación: “Y si me haces un altar de piedras, no las labres de cantería, porque si alzas tus herramientas sobre él, lo profanarás”. Incluso hoy en día, los hindúes encienden los fuegos de su altar con utensilios de taladro manual. En el curso de la religión evolutiva, la novedad se ha considerado siempre como un sacrilegio. El sacramento debe consistir, no en un elaborado alimento nuevo, sino en la más primitiva de las viandas: “Carne asada al fuego y panes sin levadura; servidos con hierbas amargas”. Todos los tipos de usos sociales e incluso los procedimientos legales se aferran a las viejas formas.
\vs p092 2:3 Cuando el hombre moderno se extraña ante la presencia de tantas costumbres que podrían considerarse detestables en las escrituras de distintas religiones, debería considerar que las nuevas generaciones han tenido miedo de eliminar lo que sus ancestros consideraban santo y sagrado. Muchas cosas que una generación podría estimar detestables, las generaciones precedentes las han considerado como parte de sus costumbres aceptadas, incluso como rituales religiosos aceptados. Se ha originado una gran cantidad de controversia religiosa debido a los inacabables intentos por conciliar las prácticas antiguas, aunque censurables, con los nuevos planteamientos de una avanzada razón, por encontrar teorías plausibles que justifiquen la perpetuación en los credos de costumbres antiguas y desfasadas.
\vs p092 2:4 Pero es una insensatez intentar acelerar demasiado repentinamente el crecimiento religioso. Una raza o una nación pueden solamente asimilar de cualquier religión de orden superior aquello que sea razonablemente consistente y compatible con su estado evolutivo vigente, sumado a su capacidad de adaptación. Las condiciones sociales, climáticas, políticas y económicas influyen en el establecimiento del rumbo y el progreso de la evolución religiosa. La moral social no está determinada por la religión, esto es, por la religión evolutiva; es la moral racial la que dicta las formas de la religión.
\vs p092 2:5 Las razas de los hombres solo asumen una religión extraña y nueva de modo superficial; en realidad, la adaptan a sus costumbres y a sus viejas formas de creer. Esto queda bien ilustrado con el ejemplo de una tribu de Nueva Zelanda cuyos sacerdotes, tras aceptar supuestamente el cristianismo, manifestaron haber recibido revelaciones directas de Gabriel, en virtud de la cual esa misma tribu se habría convertido en el pueblo elegido de Dios, disponiendo que podían libremente entregarse a relaciones sexuales disolutas y a otras muchas de sus antiguas y censurables costumbres. Y, de inmediato, todos estos cristianos recién convertidos se pasaron a esta versión nueva y menos exigente del cristianismo.
\vs p092 2:6 La religión ha avalado alguna u otra vez toda clase de comportamientos contradictorios e incongruentes, ha permitido, en algún momento, prácticamente todo lo que ahora se considera inmoral o pecaminoso. Una conciencia, no instruida por la experiencia ni asistida por la razón, nunca ha sido, ni nunca podrá, ser una guía segura y certera para la conducta humana. La conciencia no es una voz divina que habla al alma humana. Es sencillamente la suma final del contenido moral y ético de las costumbres en cualquier etapa vigente de la existencia; simplemente representa la respuesta ideal, humanamente concebida, ante cualquier conjunto de circunstancias.
\usection{3. NATURALEZA DE LA RELIGIÓN EVOLUTIVA}
\vs p092 3:1 El estudio de la religión humana es el análisis de los estratos sociales portadores de fósiles de eras pasadas. Las costumbres de los dioses antropomórficos son un leal reflejo de la moral de los hombres que primeramente concibieron tales deidades. Las religiones ancestrales y la mitología describen fielmente las creencias y las tradiciones de pueblos perdidos en la oscuridad desde hace mucho tiempo. Estas antiguas formas de culto persisten junto con costumbres económicas nuevas y desarrollos sociales y, por supuesto, parecen manifiestamente incongruentes. Los vestigios de los sistemas de culto presentan una visión genuina de las religiones raciales del pasado. Recordad siempre que tales sistemas de culto se constituyen, no para descubrir la verdad, sino, más bien, para promulgar sus credos.
\vs p092 3:2 La religión ha sido siempre, en gran medida, una cuestión de ritos, rituales, celebraciones, ceremonias y dogmas. Habitualmente se ha contaminado con ese error persistente y dañino: la ilusión de ser el pueblo elegido. Todas las ideas religiosas capitales de encantamiento, inspiración, revelación, propiciación, arrepentimiento, expiación, intercesión, sacrificio, oración, confesión, adoración, supervivencia tras la muerte, sacramento, ritual, rescate, salvación, redención, pacto, impureza, purificación, profecía, pecado original se remontan a los tiempos primitivos del miedo ancestral a los espectros.
\vs p092 3:3 \pc La religión primitiva no es más ni menos que la lucha por la existencia material ampliada para abarcar la existencia más allá de la tumba. Las celebraciones de tal credo representaban la extensión de la lucha por el automantenimiento hasta el entorno de un mundo imaginario de los espíritus\hyp{}espectros. Pero cuando estéis tentados de criticar la religión evolutiva, tened cuidado. Recordad que eso es \bibemph{lo que pasó;} es un hecho histórico. Y rememorad además que el poder de cualquier idea yace, no en su certidumbre o verdad, sino más bien en la intensidad con la que apela al ser humano.
\vs p092 3:4 \pc La religión evolutiva no contempla cambios ni revisiones; a diferencia de la ciencia, no dispone de su propia y paulatina corrección. La religión evolucionada impone respeto porque sus seguidores creen que es \bibemph{La Verdad;} “la fe que ha sido una vez dada a los santos” debe, en teoría, ser tanto definitiva como infalible. El sistema de culto se resiste al desarrollo porque es seguro que el progreso real modificará o destruirá su conjunto de prácticas y creencias; es por ello por lo que su revisión siempre debe serle impuesta.
\vs p092 3:5 Solo hay dos factores que pueden influir en la modificación y elevación de los dogmas de la religión natural: la presión de las costumbres en su lento avance y la iluminación periódica gracias a las revelaciones de los tiempos. Y no es extraño que el progreso haya sido pausado; en los días antiguos, ser adelantado o inventivo conllevaba la muerte como brujo. El sistema de culto avanza en épocas generacionales y en ciclos perdurables. Si bien, continúa adelante. La creencia evolutiva en los espectros sentó las bases de una filosofía de la religión revelada que acabará por destruir la superstición que le dio su origen.
\vs p092 3:6 La religión ha dificultado el desarrollo social en muchos aspectos, pero sin religión no habría existido ni moral ni ética duraderas, ni civilización meritoria. La religión engendró mucha cultura no religiosa: la escultura se originó en la fabricación de ídolos; la arquitectura, en la construcción de templos; la poesía, en las invocaciones; la música, en los cánticos de adoración; el teatro, en las actuaciones llevadas a cabo para lograr la guía de los espíritus y, la danza, en los festivales estacionales de adoración.
\vs p092 3:7 Pero aunque hay que poner de relieve el hecho de que la religión fue esencial para el desarrollo y la preservación de la civilización, debe hacerse constar que la religión natural ha paralizado y obstaculizado, en buena medida, la misma civilización que, por otra parte, promovía y mantenía. La religión ha dificultado las actividades industriales y el desarrollo económico; ha desperdiciado el trabajo y malgastado el capital; no siempre ha sido de ayuda para la familia; y no ha fomentado adecuadamente la paz y la buena voluntad; a veces, ha desatendido la educación y retrasado la ciencia; ha empobrecido indebidamente la vida en favor de un pretendido enriquecimiento de la muerte. La religión evolutiva, la religión humana, sin duda, es culpable de todas estas y de otras muchas equivocaciones, errores y descuidos; sin embargo, ciertamente ha mantenido la ética cultural, la moral civilizada y la coherencia social, y ha hecho posible que la posterior religión revelada compensara estas múltiples deficiencias evolutivas.
\vs p092 3:8 \pc La religión evolutiva ha sido la más costosa institución del hombre, pero, a la vez, eficaz sin parangón. La religión humana se puede justificar solo en el contexto de la civilización evolutiva. Si el hombre no tuviese su ascendencia en la evolución animal, entonces esa trayectoria del desarrollo religioso quedaría sin fundamento.
\vs p092 3:9 \pc La religión facilitó la acumulación del capital; fomentó determinados tipos de trabajo; el tiempo libre de los sacerdotes promovió el arte y el conocimiento; la raza, en definitiva, ganó mucho como resultado de todos estos tempranos errores de su enfoque ético. Los chamanes, los honestos y los deshonestos, fueron sumamente costosos, pero valieron su precio. Las profesiones de estudios avanzados y la misma ciencia surgieron de sacerdocios parasitarios. La religión favoreció la civilización y proporcionó continuidad social; ha sido la fuerza policial de todos los tiempos. La religión proveyó esa disciplina humana y ese autocontrol que posibilitaron la \bibemph{sabiduría}. La religión es el eficaz azote de la evolución que empuja implacablemente a la humanidad indolente y doliente desde su estado natural de inercia intelectual, impulsándola hacia adelante y al alza, hasta los niveles superiores de la razón y de la sabiduría.
\vs p092 3:10 Y este patrimonio sagrado de procedencia animal, la religión evolutiva, debe estar por siempre perfeccionándose y ennobleciéndose mediante el dictamen continuo de la religión revelada y el fuego ardoroso de la auténtica ciencia.
\usection{4. EL DON DE LA REVELACIÓN}
\vs p092 4:1 La revelación es evolutiva pero siempre progresiva. En la historia del mundo, a través de las eras, las revelaciones de la religión son cada vez más extensivas y consecutivamente más iluminadoras. La misión de la revelación es poner en orden y hacer una revisión de las sucesivas religiones evolutivas. Pero si la revelación ha de dignificar y elevar dichas religiones, estas visitas divinas deben presentar enseñanzas que no estén demasiado alejadas ni del pensamiento ni de las reacciones que se den en la era en las que se exponen. Por ello, la revelación debe permanecer siempre en contacto con la evolución, como así hace. La religión revelada debe estar permanentemente limitada por la capacidad del hombre para recibirla.
\vs p092 4:2 Si bien, al margen de su aparente conexión o derivación, las religiones reveladas siempre se caracterizan por la creencia en una Deidad de valor último y en algún concepto de la supervivencia de la identidad de la persona tras la muerte.
\vs p092 4:3 La religión evolutiva es sentimental; no es lógica. Es la reacción del hombre a la creencia en un mundo hipotético de espíritus espectrales ---la reflexión\hyp{}creencia humana, estimulada por la toma de conciencia y el miedo a lo desconocido---. El mundo espiritual real afirma la religión revelada; es la respuesta del cosmos supraintelectual a la sed del mortal por creer, y confiar, en las Deidades universales. La religión evolutiva muestra los sinuosos tanteos de la humanidad en su búsqueda de la verdad; la religión revelada \bibemph{es} esa verdad misma.
\vs p092 4:4 \pc En el ámbito de la revelación religiosa, ha habido muchos acontecimientos, pero solo cinco marcaron época. Fueron los siguientes:
\vs p092 4:5 \li{1.}\bibemph{Las enseñanzas dalamatianas}. El verdadero concepto de la Primera Fuente y Centro se promulgó por primera vez en Urantia por los cien miembros corpóreos de la comitiva del príncipe Caligastia. Esta revelación en expansión de la Deidad se prolongó durante más de trescientos mil años hasta que terminó de repente por la secesión planetaria y la interrupción de su régimen docente. Salvo por la labor de Van, la influencia de la revelación dalamatiana estaba prácticamente perdida para el mundo entero. Incluso los noditas habían olvidado esta verdad en el momento de la llegada de Adán. De todos los que recibieron las enseñanzas de los cien, los hombres rojos fueron quienes las retuvieron durante más tiempo, pero la idea del Gran Espíritu no era sino un difuso concepto de la religión amerindia, cuando el contacto con el cristianismo lo clarificó y reforzó considerablemente.
\vs p092 4:6 \li{2.}\bibemph{Las enseñanzas edénicas}. Adán y Eva presentaron de nuevo el concepto del Padre de todos a los pueblos evolutivos. La disgregación del primer Edén detuvo el curso de la revelación adánica antes de que incluso hubiese comenzado del todo. Si bien, los sacerdotes setitas prosiguieron con las malogradas enseñanzas de Adán y algunas de estas verdades nunca se perdieron por completo para el mundo. Las enseñanzas de los setitas modificaron en su totalidad la tendencia de la evolución religiosa levantina. Pero hacia el año 2500 a. C., la humanidad se había alejado considerablemente de la revelación auspiciada en los días de Edén.
\vs p092 4:7 \li{3.}\bibemph{Melquisedec de Salem}. Este hijo de emergencia de Nebadón inauguró la tercera revelación de la verdad en Urantia. Los preceptos principales de sus enseñanzas fueron \bibemph{confianza} y \bibemph{fe}. Enseñó confianza en la beneficencia omnipotente de Dios y proclamó que la fe era el acto por el que los hombres obtenían el favor de Dios. Sus enseñanzas se mezclaron paulatinamente con las creencias y prácticas de distintas religiones evolutivas y, finalmente, se convirtieron en esos sistemas teológicos presentes en Urantia al inicio del primer milenio después de Cristo.
\vs p092 4:8 \li{4.}\bibemph{Jesús de Nazaret}. Cristo Miguel expuso por cuarta vez a Urantia el concepto de Dios como Padre Universal, y esta enseñanza se ha mantenido por lo general desde entonces. La esencia de su enseñanza era \bibemph{amor} y \bibemph{servicio,} la adoración amorosa que la criatura ofrece deliberadamente en reconocimiento y respuesta al ministerio amante de Dios su Padre; el servicio voluntario que estos hijos creados dispensan a sus hermanos, siendo gozosamente conscientes de que, con este servicio, están sirviendo igualmente a Dios Padre.
\vs p092 4:9 \li{5.}\bibemph{Los escritos de Urantia}. Los escritos, de los que este es uno de ellos, constituyen la más reciente exposición de la verdad a los mortales de Urantia. Estos escritos difieren de todas las revelaciones anteriores, porque no es tarea de un solo ser personal del universo sino la combinación de diferentes exposiciones de multitud de seres. Pero ninguna revelación es del todo completa hasta que no se consigue llegar al Padre Universal. Todos los demás ministerios celestiales no son sino parciales, transitorios y prácticamente adaptaciones a las condiciones locales en el tiempo y en el espacio. Aunque admitir estos hechos puede posiblemente restar valor a la fuerza y a la autoridad vigentes en todas las revelaciones, ha llegado el momento en Urantia en el que es aconsejable hacer estas honestas afirmaciones, aun a riesgo de debilitar la influencia futura y la autoridad de esta revelación de la verdad, la última de las realizadas para las razas mortales de Urantia.
\usection{5. GRANDES LÍDERES RELIGIOSOS}
\vs p092 5:1 En la religión evolutiva se cree que los dioses existen en semejanza de la imagen del hombre; en la religión revelada, se enseña a los hombres que son los hijos de Dios ---incluso que están formados conforme a la imagen finita de la divinidad---; en las creencias combinadas, compuestas por las enseñanzas de la revelación y los frutos de la evolución, el concepto de Dios es una mezcla de:
\vs p092 5:2 \li{1.}Las ideas preexistentes de los sistemas de culto evolutivos.
\vs p092 5:3 \li{2.}Los ideales sublimes de la religión revelada.
\vs p092 5:4 \li{3.}Los puntos de vista personales de los grandes líderes religiosos, los profetas y los maestros de la humanidad.
\vs p092 5:5 \pc La mayor parte de las grandes épocas religiosas se han inaugurado en función de la vida y enseñanzas de alguna persona sobresaliente; el liderazgo ha originado una mayoría de los meritorios movimientos morales de la historia. Y los hombres han tendido siempre a venerar al líder, aun a expensas de sus enseñanzas; a reverenciar su persona, incluso perdiendo de vista las verdades que proclamaba. Y no sin razón; existe un anhelo instintivo en el corazón del hombre evolutivo por recibir ayuda de arriba y del más allá. Este deseo tiene por objeto anticipar la aparición en la tierra del príncipe planetario y de los posteriores hijos materiales. En Urantia, se le ha privado al hombre de estos líderes y gobernantes sobrehumanos y, por consiguiente, trata constantemente de compensar esta pérdida, encubriendo a sus líderes humanos en leyendas de origen sobrenatural y en andaduras milagrosas.
\vs p092 5:6 Multitud de razas se han imaginado que sus líderes nacían de vírgenes; sus andaduras estaban generosamente rociadas con episodios milagrosos, y, en cada respectivo grupo, se espera siempre su vuelta. En Asia central, aún esperan el regreso de Gengis Kan; en el Tíbet, China y la India, de Buda; en el Islam, de Mahoma; entre los amerindios, de Hesunanín Onamonalontón; entre los hebreos, por lo general, de Adán como gobernante material. En Babilonia, el dios Marduc fue una perpetuación de la leyenda de Adán, la idea del hijo de Dios, el vínculo que conectaba al hombre con Dios. Después de la aparición de Adán en la tierra, los llamados hijos de Dios eran algo común entre las razas del mundo.
\vs p092 5:7 Pero, con independencia del respeto supersticioso en el que con frecuencia se les tenía, sigue siendo un hecho que estos maestros fueron, en sus personas, los puntos de apoyo en los que se articulaban las palancas de la verdad revelada, de la que dependía el avance de la moral, la filosofía y la religión de la humanidad.
\vs p092 5:8 En el millón de años de historia humana en Urantia, ha habido cientos y cientos de líderes religiosos, desde Onagar hasta Guru Nanak. Durante ese periodo de tiempo, se han dado muchas fluctuaciones en la marea de la verdad religiosa y de la fe espiritual, y, en el pasado, cada renacimiento de la religión urantiana se ha identificado con la vida y las enseñanzas de algún líder religioso. Al considerar los maestros habidos en tiempos recientes, quizás pueda resultar útil agruparlos en las siete grandes épocas religiosas de la Urantia posadánica:
\vs p092 5:9 \li{1.}\bibemph{El período setita}. Los sacerdotes setitas, conforme se regeneraron bajo el liderazgo de Amosad, se convirtieron en los grandes maestros posadánicos. Desempeñaron sus funciones en todas las tierras de los anditas, y su influencia persistió durante mayor tiempo entre los griegos, los sumerios y los hindúes. Entre estos últimos, han continuado hasta el tiempo presente en calidad de brahmanes de la fe hindú. Los setitas y sus seguidores nunca perdieron por completo el concepto de la Trinidad revelado por Adán.
\vs p092 5:10 \li{2.}\bibemph{La era de los misioneros de Melquisedec}. La religión de Urantia se revitalizó notablemente gracias a los esfuerzos de maestros designados por Maquiventa Melquisedec, que vivió e impartió sus enseñanzas en Salem casi dos mil años a. C. Estos misioneros proclamaron que la fe era el precio por conseguir el favor de Dios, y sus enseñanzas, aunque infructuosas en cuanto a la inmediata aparición de religiones, constituyeron, no obstante, las bases sobre las que venideros maestros de la verdad erigieron las religiones de Urantia.
\vs p092 5:11 \li{3.}\bibemph{La era posterior a Melquisedec}. Aunque Amenemope e Ikhnaton enseñaron en este período posterior a Melquisedec, el excepcional genio religioso de esta era fue el líder de un grupo de beduinos levantinos y el fundador de la religión hebrea: Moisés. Moisés impartió el conocimiento del monoteísmo. Él dijo: “Oye, Israel: el Señor, nuestro Dios, uno es”. “El Señor es Dios y no hay otro fuera de él”. Trató de forma persistente de erradicar de su pueblo los vestigios del sistema de culto a los espectros, llegando incluso a imponer la pena de muerte a quienes lo practicaran. El monoteísmo de Moisés se vio alterado por sus sucesores, pero en tiempos posteriores volvieron de hecho a muchas de sus enseñanzas. La grandeza de Moisés reside en su sabiduría y sagacidad. Otros hombres han tenido nociones superiores de Dios, pero ninguno tuvo nunca tanto éxito persuadiendo a un número tan grande de personas a que adoptasen creencias tan avanzadas.
\vs p092 5:12 \li{4.}\bibemph{El siglo VI a. C.} En este siglo, uno de los más grandes del despertar religioso jamás presenciado en Urantia, aparecieron muchos hombres para proclamar la verdad. Entre ellos, debe hacerse constar a Gautama, Confucio, Lao\hyp{}Tse, Zoroastro y a los maestros jainistas. Las enseñanzas de Gautama se difundieron ampliamente por Asia, y millones de personas lo veneran como el Buda. Confucio fue para la moral china lo que Platón para la filosofía griega y, aunque sus enseñanzas tuvieron repercusiones religiosas, estrictamente hablando, ninguno de los dos fue un maestro religioso; Lao\hyp{}Tse concibió más a Dios en el Tao de lo que hizo Confucio en el humanismo o Platón en el idealismo. Zoroastro, aunque muy influenciado por las nociones imperantes sobre el doble espiritualismo, esto es, espíritus buenos y espíritus malos, enalteció inequívocamente, al mismo tiempo, la idea de una Deidad eterna y de la victoria final de la luz sobre la oscuridad.
\vs p092 5:13 \li{5.}\bibemph{El siglo I d. C.} Como maestro religioso, Jesús de Nazaret partió del sistema de culto establecido por Juan el Bautista y avanzó, en la medida en la que pudo, dejando atrás ayunos y ceremoniales. Aparte de Jesús, Pablo de Tarso y Filón de Alejandría fueron los más grandes maestros de esta era. Sus conceptos de la religión han tenido un papel predominante en el desarrollo de esa fe que lleva el nombre de Cristo.
\vs p092 5:14 \li{6.}\bibemph{El siglo VI d. C}. Mahoma fundó una religión que fue superior a muchos de los credos de su tiempo. La suya fue una protesta contra las exigencias sociales de las creencias religiosas extranjeras y contra la incoherencia de la vida religiosa de su propio pueblo.
\vs p092 5:15 \li{7.}\bibemph{El siglo XV d. C.} En este período se presenciaron dos movimientos religiosos: la disgregación de la unidad del cristianismo en Occidente y la síntesis de una nueva religión en Oriente. En Europa, el cristianismo institucionalizado había alcanzado tal grado de rigidez que cualquier nuevo crecimiento que se realizara era incompatible con la unidad. En Oriente, las enseñanzas combinadas de Islam, el hinduismo y el budismo se sintetizaron por parte de Nanac y sus seguidores en el sijismo, una de las religiones más avanzadas de Asia.
\vs p092 5:16 \pc No cabe duda de que el futuro de Urantia se distinguirá por la aparición de maestros de la verdad religiosa ---la paternidad de Dios y la fraternidad de todas las criaturas---. Pero es de esperar que las iniciativas fervientes y sinceras de estos futuros profetas se dirijan menos hacia el fortalecimiento de las barreras interreligiosas y más hacia el aumento de la hermandad religiosa, en relación a la adoración espiritual entre los numerosos seguidores de las diversas teologías intelectuales que tanto caracterizan a Urantia de Satania.
\usection{6. RELIGIONES COMPUESTAS}
\vs p092 6:1 En Urantia, las religiones del siglo XX presentan un interesante análisis sobre la evolución social del impulso humano hacia la adoración. Muchos de los credos religiosos han avanzado muy poco desde los días del sistema de culto de los espectros. Los pigmeos de África no tienen actitudes religiosas como grupo, aunque algunos de ellos creen ligeramente en un entorno espiritual. Están hoy justo donde se hallaba el hombre primitivo cuando comenzó la evolución de la religión. La creencia fundamental de la religión primitiva era la supervivencia tras la muerte. La idea de adorar a un Dios personal indica un desarrollo evolutivo avanzado, e incluso la primera etapa de la revelación. Los dayacos han desarrollado solo las prácticas religiosas más primitivas. Los relativamente recientes esquimales y amerindios tenían nociones muy pobres sobre Dios; creían en los espectros y tenían una vaga idea de algún tipo de supervivencia después de la muerte. Los aborígenes australianos actuales solo tienen temor a los espectros, miedo de la oscuridad y una tosca veneración a los ancestros. Los zulúes están precisamente desarrollando una religión de sacrificios y de temor a los espectros. En su evolución religiosa, muchas tribus africanas, salvo donde hubo labor misionera de cristianos y mahometanos, no están aún más allá de la etapa fetichista de la evolución religiosa. Pero, desde hace mucho tiempo, algunos grupos se han aferrado a la idea del monoteísmo, como los antiguos tracios, que también creían en la inmortalidad.
\vs p092 6:2 \pc En Urantia, la religión evolutiva y la revelada están avanzando una al lado de la otra, mientras se mezclan y convergen en los variados sistemas teológicos establecidos en el mundo en el momento de la redacción de estos escritos. Estas religiones, las del siglo XX en Urantia, se pueden enumerar de la siguiente manera:
\vs p092 6:3 \li{1.}El hinduismo: la más antigua.
\vs p092 6:4 \li{2.}La religión hebrea.
\vs p092 6:5 \li{3.}El budismo.
\vs p092 6:6 \li{4.}Las enseñanzas de Confucio.
\vs p092 6:7 \li{5.}Las creencias taoístas.
\vs p092 6:8 \li{6.}El zoroastrismo.
\vs p092 6:9 \li{7.}El sintoísmo.
\vs p092 6:10 \li{8.}El jainismo.
\vs p092 6:11 \li{9.}El cristianismo.
\vs p092 6:12 \li{10.}El islam.
\vs p092 6:13 \li{11.}El sijismo: la más reciente.
\vs p092 6:14 \pc Las religiones más avanzadas de los tiempos ancestrales fueron el judaísmo y el hinduismo y, cada una, respectivamente, ha influido profundamente sobre el curso del desarrollo religioso en Oriente y Occidente. Tanto los hindúes como los hebreos creían que sus religiones eran inspiradas y reveladas, y que todas las demás eran manifestaciones decadentes de la única fe verdadera.
\vs p092 6:15 La India se distribuye entre hindúes, sijs, mahometanos y jainistas, cada cual con una idea de Dios, el hombre y el universo conforme a sus diferentes conceptos de ellos. China sigue las enseñanzas taoístas y confucionistas; en Japón se reverencia el sintoísmo.
\vs p092 6:16 Las grandes confesiones internacionales e interraciales son la hebraica, la budista, la cristiana y la islámica. El budismo se extiende desde Ceilán y Birmania, a través del Tíbet y de China, hasta el Japón. Ha demostrado una capacidad de adaptación a las costumbres de muchos pueblos que únicamente el cristianismo ha igualado.
\vs p092 6:17 La religión hebrea abarca la transición filosófica desde el politeísmo al monoteísmo; es un eslabón evolutivo entre las religiones de evolución y las religiones de revelación. Los hebreos fueron el único pueblo occidental que siguió a sus tempranos dioses evolutivos directamente hasta el Dios de la revelación. Pero esta verdad nunca se empezó a aceptar ampliamente hasta los días de Isaías, que nuevamente enseñó la idea compuesta de la deidad racial sumada a un Creador Universal: “Oh Señor de las huestes, Dios de Israel, tú eres Dios, aun tú solo. Tú hiciste el cielo y la tierra”. En determinado momento, la esperanza de la supervivencia de la civilización occidental residía en los sublimes conceptos hebraicos de la bondad y las avanzadas nociones helénicas de la belleza.
\vs p092 6:18 La religión cristiana es la religión sobre la vida y las enseñanzas de Cristo basada en la teología del judaísmo, modificada aún más mediante la asimilación de ciertas enseñanzas zoroástricas y la filosofía griega, y elaborada principalmente por tres personas: Filón, Pedro y Pablo. Ha pasado a través de múltiples etapas en su evolución desde los tiempos de Pablo y, como cabía esperar, se ha vuelto tan completamente occidentalizada que muchos pueblos no europeos consideran el cristianismo como una extraña revelación de un Dios extraño y para extraños.
\vs p092 6:19 El islam es el vínculo religioso\hyp{}cultural entre África del norte, el Levante y el sudeste de Asia. Fue la teología judía, en el marco de las posteriores enseñanzas cristianas, la que hizo que el islam fuese monoteísta. Los seguidores de Mahoma vacilaron ante las enseñanzas avanzadas sobre la Trinidad; no pudieron comprender la doctrina de las tres personas divinas y una Deidad. Resulta siempre difícil alentar a la mente evolutiva a que acepte \bibemph{de repente} una verdad revelada de carácter avanzado. El hombre es una criatura evolutiva y, por lo general, debe conseguir su religión mediante métodos evolutivos.
\vs p092 6:20 \pc En algún momento, la adoración de los ancestros constituyó un decidido avance en el desarrollo religioso, pero es a la vez asombroso y deplorable que esta práctica primitiva persista en China, en Japón y en la India en medio de tanta otra relativamente más avanzada, como el budismo y el hinduismo. En Occidente, la adoración de los ancestros se transformó en la veneración de los dioses nacionales y el respeto por los héroes raciales. En el siglo XX, esta religión nacionalista de veneración de los héroes hace su aparición en los distintos secularismos radicales y nacionalistas que caracterizan a muchas razas y naciones occidentales. Gran parte de esta misma actitud se encuentra igualmente en las grandes universidades y en las mayores comunidades industriales de los pueblos de habla inglesa. La idea de que la religión no es sino “una búsqueda compartida de la buena vida” no difiere mucho de estas consideraciones. Las “religiones nacionales” no son más que una vuelta a la primitiva adoración del emperador romano y al sintoísmo: la adoración del Estado personificado en la familia imperial.
\usection{7. EVOLUCIÓN POSTERIOR DE LA RELIGIÓN}
\vs p092 7:1 La religión nunca puede convertirse en un hecho científico. La filosofía puede, de hecho, basarse en un fundamento científico, pero la religión siempre seguirá siendo evolutiva o revelada, o una posible combinación de ambas, como lo es en el mundo de hoy en día.
\vs p092 7:2 No se pueden inventar nuevas religiones; pueden evolucionar, o bien \bibemph{revelarse de repente}. Todas las nuevas religiones evolutivas son sencillamente expresiones avanzadas de viejas creencias, nuevas adaptaciones y ajustes. Lo antiguo no deja de existir; está fusionado con lo nuevo, al igual que el sijismo brotó y floreció de las semillas y manifestaciones del hinduismo, del budismo, del islam y de otros sistemas de culto contemporáneos. La religión primitiva era muy plural; el salvaje era rápido en tomar prestado o prestar conceptos. Solo con la religión revelada apareció realmente el egocentrismo teológico autocrático e intolerante.
\vs p092 7:3 Las numerosas religiones de Urantia son todas buenas en la medida en que llevan al hombre a Dios y le aportan la comprensión del Padre. Todo grupo de devotos religiosos que piense que su credo es \bibemph{La Verdad} está cometiendo un error; esa actitud habla más de arrogancia teológica que de certidumbre de fe. No existe religión en Urantia que no pueda examinar provechosamente y asimilar lo mejor de las verdades contenidas en cada una de las otras confesiones, porque todas albergan verdades. Las personas religiosas harían mejor en pedir prestado lo mejor de la fe espiritual viva de sus vecinos en lugar de reprobar lo peor de sus persistentes supersticiones y ritos caducos.
\vs p092 7:4 Todas estas religiones han surgido como resultado de la respuesta intelectual y variable del hombre a una idéntica guía espiritual. Jamás se puede esperar lograr uniformidad de credos, dogmas y rituales ---estos son elementos intelectuales---; pero sí pueden, y algún día lo harán, conseguir la unidad en la auténtica adoración del Padre de todos, por su carácter espiritual, y es por siempre verdad que todos los hombres son iguales en el espíritu.
\vs p092 7:5 \pc En gran parte, la religión primitiva se basaba en la conciencia de los valores materiales, pero la civilización eleva los valores religiosos, porque la verdadera religión es la dedicación del yo al servicio de valores significativos y supremos. Conforme evoluciona la religión, la ética se convierte en la filosofía de los principios morales, y la moral pasa a ser la disciplina del yo según los parámetros de los contenidos más elevados y de los valores supremos ---de unos ideales divinos y espirituales---. Así, la religión se vuelve devoción espontánea y espléndida, la experiencia viva de la lealtad del amor.
\vs p092 7:6 El grado de excelencia de una religión se mide por:
\vs p092 7:7 \li{1.}El nivel de los valores: las lealtades.
\vs p092 7:8 \li{2.}La profundidad de los contenidos: la sensibilización de la persona al reconocimiento idealista de estos valores supremos.
\vs p092 7:9 \li{3.}La intensidad de la dedicación: el grado de devoción a estos valores divinos.
\vs p092 7:10 \li{4.}El irrestricto avance de la persona en este sendero cósmico de vida espiritual e idealista, de la comprensión de la filiación con Dios y de la interminable ciudadanía progresiva en el universo.
\vs p092 7:11 \pc Los contenidos religiosos avanzan en la conciencia de uno mismo cuando el niño transfiere sus ideas sobre la omnipotencia de sus padres a Dios. Y toda su experiencia religiosa depende, en buena parte, de si el temor o el amor ha dominado la relación padre\hyp{}hijo. Los esclavos han tenido siempre grandes dificultades en trasladar su temor al amo a los conceptos del amor a Dios. La civilización, la ciencia y las religiones avanzadas deben liberar a la humanidad de esos temores nacidos del miedo a los fenómenos naturales. Y, por consiguiente, una mayor lucidez debería liberar a los mortales educados de toda dependencia a los intermediarios en su comunión con la Deidad.
\vs p092 7:12 Estas etapas intermedias de indecisión fetichista en la transferencia de veneración desde lo humano y lo visible hasta lo divino e invisible son inevitables, pero se deberían reducir mediante la conciencia del ministerio favorecedor del espíritu divino interior. No obstante, el hombre se ha visto profundamente influenciado no solo por sus nociones sobre la Deidad, sino también por el carácter de los héroes a los que ha elegido honrar. Es muy de lamentar que aquellos que han dado el paso de venerar al divino Cristo resucitado hayan pasado por alto al hombre ---al héroe valiente y tenaz--- a Josué ben José.
\vs p092 7:13 \pc El hombre moderno posee una adecuada autoconciencia de la religión, pero sus costumbres respecto a la adoración son confusas y se han visto invalidadas por su acelerada metamorfosis social y sus desarrollos científicos sin precedentes. Hay hombres y mujeres reflexivos que quieren redefinir la religión, lo que obligará a esta a reevaluarse a sí misma.
\vs p092 7:14 El hombre moderno se enfrenta con la tarea de hacer más reajustes de los valores humanos en una sola generación de los que se han hecho en dos mil años. Y todo ello influye sobre la actitud social hacia la religión, porque la religión es una manera de vivir al igual que un modo de pensar.
\vs p092 7:15 \pc La verdadera religión debe ser siempre, y a la vez, el fundamento eterno y la estrella guía de toda perdurable civilización.
\vsetoff
\vs p092 7:16 [Exposición de un melquisedec de Nebadón.]
