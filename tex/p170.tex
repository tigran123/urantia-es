\upaper{170}{El reino de los cielos}
\author{Comisión de seres intermedios}
\vs p170 0:1 El sábado 11 de marzo, Jesús predicó su último sermón en Pella, uno de los más notables de su ministerio público. Abordó con gran detenimiento el tema del reino de los cielos. Era consciente de la confusión que existía en las mentes de sus apóstoles y discípulos sobre el significado y la importancia de los términos “reino de los cielos” y “reino de Dios”, que él usaba indistintamente para designar su misión de gracia. Aunque en sí mismo el término reino de los \bibemph{cielos} debería haber sido suficiente para distinguir lo que este representaba de los reinos \bibemph{terrenales} y de los gobiernos temporales, no era realmente así. La idea de un rey temporal estaba demasiado enraizada en la mente judía como para poder desprenderse de ella en una sola generación. Por ello, Jesús, en un principio, no se opuso abiertamente a tal concepto del reino por tanto tiempo gestado.
\vs p170 0:2 Ese \bibemph{sabbat} por la tarde el Maestro quiso aclarar sus enseñanzas sobre el reino de los cielos; examinó el tema desde todos los puntos de vista e intentó especificar con claridad los diferentes sentidos en los que había usado el término. En esta narrativa, ampliaremos su sermón, añadiendo numerosas afirmaciones realizadas anteriormente en otras ocasiones, incluyendo también algunas observaciones hechas solo a los apóstoles durante las charlas vespertinas de aquel mismo día. Haremos además ciertos comentarios respecto a la evolución posterior de la idea de reino en su relación con la Iglesia cristiana que estaba por venir.
\usection{1. CONCEPTOS DEL REINO DE LOS CIELOS}
\vs p170 1:1 En el marco de la exposición del sermón de Jesús, hay que señalar que, en todas las escrituras hebreas, existía una doble conceptualización del reino de los cielos. Los profetas planteaban el reino de Dios como:
\vs p170 1:2 \li{1.}Una realidad presente; y como:
\vs p170 1:3 \li{2.}Una esperanza futura ---el reino se realizaría en su plenitud cuando apareciera el Mesías---. Este es el concepto de reino que enseñaba Juan el Bautista.
\vs p170 1:4 Desde el primer momento, Jesús y los apóstoles impartieron estos dos conceptos. Había otras dos ideas del reino que deben tenerse en cuenta:
\vs p170 1:5 \li{3.}El concepto judío más tardío de un reino mundial y trascendental, de origen sobrenatural, que se inauguraría de forma milagrosa.
\vs p170 1:6 \li{4.}Las enseñanzas persas que definían la instauración, en el fin de mundo, de un reino divino como logro del triunfo del bien sobre el mal.
\vs p170 1:7 \pc Poco antes de la venida de Jesús a la tierra, los judíos mezclaban y confundían todas estas ideas del reino en su concepto apocalíptico de la llegada del Mesías para instaurar la era del triunfo judío, la era eterna del gobierno supremo de Dios sobre la tierra, el nuevo mundo, la era en la que toda la humanidad adoraría a Yahvé. Al optar por el empleo de este concepto de reino de los cielos, Jesús eligió hacer uso del legado más esencial y relevante tanto de la religión judía como de la persa.
\vs p170 1:8 El reino de los cielos, tal como se ha entendido y malentendido a lo largo de los siglos de la era cristiana, incluía cuatro grupos diferenciados de ideas:
\vs p170 1:9 \li{1.}El concepto judío.
\vs p170 1:10 \li{2.}El concepto persa.
\vs p170 1:11 \li{3.}El concepto de Jesús sobre la experiencia personal: “el reino de los cielos está en vosotros”.
\vs p170 1:12 \li{4.}Los conceptos entremezclados y confusos que inculcaron en el mundo quienes fundaron y difundieron el cristianismo.
\vs p170 1:13 \pc En diferentes momentos y en distintas circunstancias, da la impresión de que Jesús impartió numerosos conceptos del “reino” en su predicación pública, pero siempre enseñó a sus apóstoles que el reino consistía en la experiencia personal del hombre respecto a sus semejantes en la tierra y al Padre de los cielos. Referente al reino, sus últimas palabras siempre eran: “el reino está en vosotros”.
\vs p170 1:14 Los siglos de confusión existidos en cuanto al significado del término “reino de los cielos”, se debieron a tres factores:
\vs p170 1:15 \li{1.}Una confusión causada por el hecho de que la idea de “reino” tuvo progresivamente, en su formulación por parte de Jesús y sus apóstoles, distintas facetas.
\vs p170 1:16 \li{2.}Una confusión asociada inevitablemente al tránsito del cristianismo primitivo desde el suelo judío al gentil.
\vs p170 1:17 \li{3.}Una confusión implícita al hecho de que el cristianismo se convirtió en una religión que se organizó esencialmente sobre la idea de la persona de Jesús; cada vez más, el evangelio del reino se transformó en una religión \bibemph{sobre} él.
\usection{2. EL CONCEPTO DE JESÚS SOBRE EL REINO}
\vs p170 2:1 El Maestro expresó con claridad que el reino de los cielos debe empezar y centrarse sobre la base del doble concepto de la verdad de la paternidad de Dios y del hecho, correlacionado, de la hermandad del hombre. La aceptación de esta idea, afirmó Jesús, liberaría al hombre de largos años de atadura al miedo animal y enriquecería, al mismo tiempo, la vida humana con los siguientes dones que una nueva vida en libertad espiritual conlleva:
\vs p170 2:2 \li{1.}La posesión de una renovada valentía y de un mayor poder espiritual. El evangelio del reino iba a liberar al hombre y motivarlo valerosamente a tener esperanza en la vida eterna.
\vs p170 2:3 \li{2.}El evangelio ofrecía una confianza nueva en Dios y un verdadero consuelo para todos los hombres, incluido los pobres.
\vs p170 2:4 \li{3.}Contenía, en sí mismo, un nuevo criterio de valores morales, un nuevo rasero ético por el que medir la conducta humana. Representaba el ideal del nuevo orden de sociedad humana que resultaría de él.
\vs p170 2:5 \li{4.}Enseñaba la preeminencia de lo espiritual respecto a lo material; glorificaba las realidades espirituales y exaltaba los ideales sobrehumanos.
\vs p170 2:6 \li{5.}Para este nuevo evangelio, el logro espiritual era el verdadero objetivo a alcanzar en la vida. La vida humana se dotaba de nuevos valores morales y de dignidad divina.
\vs p170 2:7 \li{6.}Jesús enseñó que las realidades eternas eran el resultado (la recompensa) de haberse esforzado por vivir rectamente en la tierra. La estancia del hombre mortal en el mundo adquirió nuevos significados al tomar conciencia de su noble destino.
\vs p170 2:8 \li{7.}El nuevo evangelio afirmaba que la salvación humana es la manifestación de un propósito divino que se ha de cumplir y lograr en un destino futuro, y que comporta el ilimitado servicio de los hijos salvados de Dios.
\vs p170 2:9 \pc En estas enseñanzas, se incluye la idea del reino tal como Jesús la llegaría a desarrollar. Esta extraordinaria conceptualización apenas se dio en las enseñanzas impartidas por Juan sobre el reino, que eran confusas y de carácter elemental.
\vs p170 2:10 Los apóstoles no lograron captar el verdadero significado de las afirmaciones del Maestro sobre el reino. La distorsión de las enseñanzas de Jesús sobrevenida posteriormente, tal como se constata en el Nuevo Testamento, se debe al hecho de que los escritores de los evangelios desvirtuaron el concepto de reino al creer que Jesús se había marchado del mundo por un poco tiempo y que volvería pronto para instaurar el reino en poder y gloria ---exactamente la misma idea que habían tenido mientras él estaba con ellos en la carne---. Pero Jesús no vinculó la instauración del reino a la idea de su regreso a este mundo. El hecho de que hayan pasado siglos sin ninguna señal de la aparición de la “Nueva Era” no desdice en absoluto las enseñanzas de Jesús.
\vs p170 2:11 El gran afán de Jesús en este sermón fue convertir el concepto del reino de los cielos en el ideal de la idea de hacer la voluntad de Dios. Durante mucho tiempo, el Maestro había enseñado a sus seguidores a orar: “Venga tu reino; hágase tu voluntad”; y, en aquel momento, trató encarecidamente de instarlos a desistir del uso del término \bibemph{reino de Dios} en favor de otro equivalente y más práctico como el de \bibemph{la voluntad de Dios}. Pero no lo consiguió.
\vs p170 2:12 Jesús quiso sustituir la idea de reino, rey y súbditos, por la noción de familia celestial, Padre celestial e hijos liberados de Dios, comprometidos en el servicio gozoso y voluntario de sus semejantes y en la adoración sublime e inteligente a Dios Padre.
\vs p170 2:13 Hasta este momento, los apóstoles habían adquirido una doble perspectiva del reino. Lo consideraban:
\vs p170 2:14 \li{1.}Un asunto relacionado con la experiencia personal presente en ese momento en el corazón de los verdaderos creyentes, y
\vs p170 2:15 \li{2.}Algo que ocurriría a nivel racial o mundial; que el reino llegaría en un futuro, algo a lo que aspirar.
\vs p170 2:16 \pc Creían que la venida del reino en el corazón de los hombres tenía un carácter progresivo como la levadura en la masa o como el crecimiento de la semilla de mostaza. Pensaban de tal venida en un sentido racial o mundial y que sería repentina y espectacular a la vez. Jesús nunca se cansó de decirles que el reino de los cielos era su propia experiencia personal del logro de las cualidades superiores de la vida espiritual; que esas realidades de la experiencia espiritual se transformaban paulatinamente en niveles nuevos y superiores de certidumbre divina y de grandeza eterna.
\vs p170 2:17 Aquella tarde, el Maestro les enseñó con nitidez el nuevo concepto de la doble naturaleza del reino al mencionar las dos etapas siguientes:
\vs p170 2:18 “Primera. El reino de Dios en este mundo, el deseo supremo de hacer la voluntad de Dios, el amor desinteresado del hombre, que rinde los buenos frutos de una conducta ética y moral enaltecida.
\vs p170 2:19 “Segunda. El reino de Dios en el cielo, el objetivo de los creyentes mortales, la heredad en la que el amor por Dios se ha perfeccionado, y en la que se hace la voluntad de Dios de forma más divina”.
\vs p170 2:20 Jesús enseñó que, mediante la fe, el creyente entra en el reino \bibemph{ahora}. En distintas charlas, comentó que hay dos cosas esenciales para acceder por la fe al reino:
\vs p170 2:21 \li{1.}\bibemph{La fe, la sinceridad}. Venir como un niño pequeño, recibir de gracia el don de la filiación; sumisión sin preguntas al cumplimiento de la voluntad del Padre y convicción plena y genuina confianza en la sabiduría del Padre; venir al reino libre de prejuicios e ideas preconcebidas; tener la mente abierta y educable como un niño sin malcriar.
\vs p170 2:22 \li{2.}\bibemph{El hambre de la verdad}. La sed de rectitud, un cambio de mente, tener el propósito de ser como Dios y encontrar a Dios.
\vs p170 2:23 Jesús enseñó que el pecado no es hijo de una naturaleza deficiente, sino vástago de una mente consciente, dominada por una voluntad insumisa. En relación al pecado, Jesús dijo que Dios \bibemph{ha} perdonado; que dicho perdón se nos hace personalmente presente en el momento en el que nosotros perdonamos a nuestros semejantes. Cuando perdonas a tu hermano en la carne, creas, pues, en tu alma la capacidad para recibir la realidad del perdón de Dios por tus propias faltas.
\vs p170 2:24 Cuando el apóstol Juan comenzó a escribir la historia de la vida y las enseñanzas de Jesús, los cristianos primitivos habían tenido tantas dificultades con la idea del reino de Dios, al haber sido causa de persecuciones, que, en buena medida, habían dejado de utilizar tal término. Juan habla bastante de “vida eterna”. Jesús a menudo alude a él como el “reino de la vida”. Con frecuencia, habló también del “reino de Dios en vosotros”. Cierta vez, se refirió a dicha experiencia como “fraternidad en una familia con Dios como Padre”. Jesús trató de sustituir el término “reino” por algunos otros términos, pero siempre sin lograrlo. Entre ellos, empleó: la familia de Dios, la voluntad del Padre, los amigos de Dios, la fraternidad de los creyentes, la hermandad del hombre, el redil del Padre, los hijos de Dios, la fraternidad de los fieles, el servicio del Padre y los hijos liberados de Dios.
\vs p170 2:25 Pero no pudo eludir el uso de la idea del reino. No fue hasta más de cincuenta años después, tras la destrucción de Jerusalén por los ejércitos romanos, cuando este concepto del reino comenzó a transformarse en el sistema de culto de la vida eterna, conforme la Iglesia cristiana, que se expandía y cristalizaba rápidamente, asumió sus aspectos sociales e institucionales.
\usection{3. EN CUANTO A LA RECTITUD}
\vs p170 3:1 Jesús siempre trató de inculcar en sus apóstoles y discípulos la idea de que, mediante la fe, se había de adquirir una rectitud que superara la rectitud de las obras serviles de los escribas y los fariseos, y de las que con tanta vanagloria alardeaban ante el mundo.
\vs p170 3:2 Aunque Jesús enseñó que la fe, creer con la sencillez de un niño, es la llave de la puerta del reino, también dijo que, una vez entrado en él, el creyente debía ascender escalonadamente hacia la rectitud y crecer hasta lograr la estatura plena y robustez de los hijos de Dios.
\vs p170 3:3 En la manera de \bibemph{recibir} el perdón de Dios se revela el logro de la rectitud del reino. La fe es el precio a pagar por entrar en la familia de Dios; pero el perdón constituye el acto de Dios que acepta vuestra fe como precio para ser admitidos. Y, para el creyente del reino, recibir el perdón de Dios supone una experiencia personal concreta y real, y consiste en los siguientes cuatro pasos, los pasos a dar en el reino para alcanzar la rectitud interior:
\vs p170 3:4 \li{1.}El perdón de Dios se hace realmente presente en el hombre y lo experimenta personalmente justo en la misma medida en la que él perdona a sus semejantes.
\vs p170 3:5 \li{2.}El hombre no perdona en verdad a sus semejantes a no ser que los ame como a sí mismo.
\vs p170 3:6 \li{3.}Amar por tanto a tu prójimo como a ti mismo \bibemph{es} el más elevado principio ético.
\vs p170 3:7 \li{4.}La conducta moral, la verdadera rectitud, se convierte, entonces, en el estado natural de dicho amor.
\vs p170 3:8 \pc Es evidente, por tanto, que la verdadera religión interior del reino tiende a manifestarse, invariable y crecientemente, en formas prácticas de servir a la sociedad. Jesús enseñó una religión viva que llevaba a sus creyentes a dedicarse al servicio amoroso de los demás. Pero Jesús no dio preeminencia a la ética sobre la religión. Enseñó la religión como causa y la ética como resultado.
\vs p170 3:9 La rectitud de cualquier acto debe medirse por el motivo que lo impulsa; las formas más elevadas de la bondad son pues inconscientes. Jesús nunca se preocupó por la moral ni por la ética en sí mismas. Su mayor interés fue la fraternidad interior y espiritual con Dios Padre, que se manifiesta externamente y de manera indudable y directa en el servicio amoroso al hombre. Jesús enseñó que la religión del reino es una experiencia personal genuina que nadie puede contener en sí mismo; que la conciencia de ser miembro de la familia de los creyentes lleva inevitablemente a la práctica de los preceptos que rigen la conducta familiar, al servicio a los propios hermanos y hermanas en un esfuerzo por mejorar y extender la hermandad.
\vs p170 3:10 La religión del reino es personal, individual; sus frutos, los resultados, son de orden familiar, social. Jesús nunca dejó de exaltar la sacralidad del individuo en contraste con la comunidad. Pero también reconocía que el hombre desarrolla su carácter mediante el servicio desinteresado; que desvela su naturaleza moral en la relación afectiva con sus semejantes.
\vs p170 3:11 Al enseñar que el reino está en cada persona, al exaltar al individuo, Jesús dio un gran revés a la antigua sociedad, marcando el comienzo de la nueva dispensación de la verdadera rectitud social. El mundo ha conocido poco este nuevo orden social, porque se ha negado a practicar los principios del evangelio del reino de los cielos. Y, cuando este reino, espiritualmente preeminente, venga por fin a la tierra, no se manifestará meramente en mejores condiciones sociales y materiales, sino más bien en la gloria de esos valores espirituales enaltecidos y enriquecidos, característicos de una próxima era de mejores relaciones humanas y del avance de los logros espirituales.
\usection{4. ENSEÑANZAS DE JESÚS SOBRE EL REINO}
\vs p170 4:1 Jesús nunca dio una definición precisa del reino. Algunas veces comentaba cierta etapa de este, mientras que, otras, abordaba un aspecto diferente de la hermandad del reino de Dios en el corazón de los hombres. En el transcurso de este sermón del sábado por la tarde, Jesús señaló no menos de cinco etapas, o épocas, del reino, y que son las que siguen:
\vs p170 4:2 \li{1.}La experiencia personal e interior de la vida espiritual que implica la fraternidad que el creyente individual tiene con Dios Padre.
\vs p170 4:3 \li{2.}La creciente hermandad de los creyentes del evangelio, los aspectos sociales del fortalecimiento de la moral y de la ética como consecuencia del reinado del espíritu de Dios en el corazón de cada creyente.
\vs p170 4:4 \li{3.}La hermandad supramortal de seres espirituales invisibles que impera en la tierra y en el cielo, el reino sobrehumano de Dios.
\vs p170 4:5 \li{4.}La expectativa de un más perfecto cumplimiento de la voluntad de Dios, el progreso hasta el surgimiento de un nuevo orden social en correspondencia con una vida espiritual de mayor elevación ---la siguiente era del hombre---.
\vs p170 4:6 \li{5.}El reino en toda su plenitud, la futura era espiritual de luz y vida en la tierra.
\vs p170 4:7 \pc Por lo cual, siempre debemos examinar las enseñanzas del Maestro para comprobar a cuál de estas cinco etapas se refiere cuando usa el término “reino de los cielos”. A través de estos pasos mediante los que la voluntad va progresivamente modificándose, y que repercuten a su vez en las decisiones humanas, Miguel y sus compañeros están igualmente, de forma gradual pero cierta, cambiando por entero el curso de la evolución humana, tanto social como de otro orden.
\vs p170 4:8 En esta ocasión, el Maestro destacó los cinco puntos siguientes, que constituyen los rasgos fundamentales del evangelio del reino:
\vs p170 4:9 \li{1.}La preeminencia del individuo.
\vs p170 4:10 \li{2.}La voluntad como factor determinante en la experiencia del hombre.
\vs p170 4:11 \li{3.}La fraternidad espiritual con Dios Padre.
\vs p170 4:12 \li{4.}La suprema satisfacción de servir amorosamente al hombre.
\vs p170 4:13 \li{5.}La trascendencia de lo espiritual sobre lo material en la persona humana.
\vs p170 4:14 \pc Este mundo nunca ha intentado poner en práctica con seriedad o sinceridad u honestidad estas vigorizantes ideas, estos ideales divinos de la doctrina de Jesús sobre el reino de los cielos. Pero no debéis desalentaros por el aparentemente lento avance de la idea del reino en Urantia. Recordad que la secuencia de la evolución progresiva está sometida a cambios periódicos, súbitos e inesperados, tanto en el mundo material como en el espiritual. El ministerio de gracia de Jesús como Hijo encarnado fue, de hecho, un inusual e inesperado acontecimiento en la vida espiritual del mundo. Cuando tratéis de determinar la era en la que el reino está ahora manifestándose, no cometáis tampoco el fatal error de no lograr instaurarlo en vuestras propias almas.
\vs p170 4:15 Aunque Jesús se refirió a una etapa del reino que ocurriría en el futuro, y en numerosas ocasiones, insinuó que dicho acontecimiento podría aparecer como parte de una crisis mundial; y, aunque en distintas ocasiones ciertamente prometió que regresaría algún día a Urantia, es preciso hacer notar que nunca vinculó específicamente estas dos ideas entre sí. Prometió una nueva revelación del reino sobre la tierra y en algún tiempo futuro; también prometió alguna vez que volvería a este mundo en persona; pero no dijo que ambos acontecimientos fueran simultáneos. Hasta donde sabemos, estas promesas pueden aludir o no al mismo acontecimiento.
\vs p170 4:16 No obstante, sus apóstoles y discípulos sí vincularon ambas enseñanzas entre sí. Cuando el reino no llegó a materializarse tal como ellos habían esperado que ocurriera, acordándose de la enseñanza del Maestro sobre un reino futuro y recordando su promesa de volver, extrajeron la conclusión de que estas promesas hacían referencia a un mismo acontecimiento; y, por lo tanto, vivieron esperanzados en su inmediata segunda venida para instaurar el reino en su plenitud y con poder y gloria. Y así han vivido también las siguientes generaciones de creyentes en la tierra, al abrigo de la misma esperanza, alentadora, pero decepcionante.
\usection{5. IDEAS MÁS RECIENTES SOBRE EL REINO}
\vs p170 5:1 Habiendo resumido las enseñanzas de Jesús sobre el reino de los cielos, se nos permite exponer ciertas ideas más recientes, que quedaron añadidas al concepto del reino y proceder a predecir la posible evolución del reino en la próxima era.
\vs p170 5:2 A lo largo de los primeros siglos de diseminación del cristianismo, la idea del reino de los cielos se vio sumamente influenciada por las nociones del idealismo griego entonces en rápida difusión, por la idea de que lo natural era como la sombra de lo espiritual ---lo temporal, como la sombra en el tiempo de lo eterno---.
\vs p170 5:3 Pero el gran paso que señaló el tránsito de las enseñanzas de Jesús desde el suelo judío al gentil se dio en el momento en el que el Mesías del reino se convirtió en el Redentor de la Iglesia, una institución religiosa y social que creció de la labor de Pablo y de sus sucesores, y que se basó en las enseñanzas de Jesús ampliadas con las ideas de Filón y las doctrinas persas del bien y del mal.
\vs p170 5:4 Las ideas e ideales de Jesús, contenidos en las enseñanzas del evangelio del reino, casi no llegaron a realizarse porque sus seguidores fueron progresivamente distorsionando sus afirmaciones. El concepto del reino definido por el Maestro se vio considerablemente modificado debido a dos grandes tendencias:
\vs p170 5:5 \li{1.}Los creyentes judíos siguieron considerándolo el \bibemph{Mesías}. Creían que Jesús regresaría muy pronto para realmente instaurar el reino a escala mundial y más o menos de índole material.
\vs p170 5:6 \li{2.}Los cristianos gentiles comenzaron muy tempranamente a aceptar las doctrinas de Pablo que apuntaban, cada vez más, a la creencia generalizada de que Jesús era el \bibemph{Redentor} de los hijos de la Iglesia, el nuevo concepto institucional que sucedería al concepto de una hermandad del reino, puramente espiritual.
\vs p170 5:7 \pc La Iglesia, como consecuencia social del reino, hubiera sido totalmente natural e incluso deseable. El mal de la Iglesia no consistió en su existencia, sino más bien en el hecho de haber reemplazado casi por completo el concepto de reino que definió el Maestro. La Iglesia institucionalizada de Pablo se convirtió prácticamente en la sustituta del reino de los cielos proclamado por Jesús.
\vs p170 5:8 Pero no tengáis duda: este mismo reino de los cielos que el Maestro nos enseñó existe aún en el corazón del creyente, y se anunciará nuevamente a esta Iglesia cristiana, al igual que a todas las demás religiones, razas y naciones de la tierra ---e incluso a nivel personal---.
\vs p170 5:9 Este reino, el ideal espiritual de la rectitud personal y el concepto de la fraternidad divina del hombre con Dios, fue paulatinamente quedando inmerso en el concepto místico de la persona de Jesús como Redentor\hyp{}Creador y como la cabeza espiritual de una comunidad religiosa socializada. De este modo, una Iglesia ritualista e institucionalizada sustituyó a una hermandad del reino compuesta por creyentes bajo la guía individual del espíritu.
\vs p170 5:10 La Iglesia fue el resultado \bibemph{social} inevitable y útil de la vida y de las enseñanzas de Jesús; la tragedia consistió en el hecho de que esta respuesta social a las enseñanzas del reino llegara a desplazar tan enteramente el concepto espiritual del verdadero reino, tal como Jesús lo enseñó y vivió.
\vs p170 5:11 Para los judíos, el reino era la \bibemph{comunidad} israelita; para los gentiles, se convirtió en la \bibemph{Iglesia} cristiana. Para Jesús, el reino era la totalidad de aquellos que \bibemph{de forma individual} habían confesado su fe en la paternidad de Dios, afirmando con ello su dedicación incondicional a la realización de la voluntad de Dios, haciéndose, de este modo, miembros de la hermandad espiritual del hombre.
\vs p170 5:12 El Maestro era totalmente consciente de que en el mundo se producirían algunos resultados de orden social como consecuencia de la propagación del evangelio del reino; pero su intención era que todas estas válidas manifestaciones sociales aparecieran como efectos inconscientes e inevitables, o frutos naturales, de la experiencia personal interior del creyente individual, de esta fraternidad puramente espiritual y de la comunión con el espíritu divino que habita en todos estos creyentes y los motiva.
\vs p170 5:13 Jesús anticipaba que el avance del verdadero reino espiritual iría seguido de una institución de orden social, o Iglesia, y fue por ello por lo que nunca se opuso a que los apóstoles practicaran el rito del bautismo de Juan. Enseñó que, mediante la fe, se admite en el reino espiritual al alma que ama la verdad, al que tiene hambre y sed de rectitud por Dios; al mismo tiempo, los apóstoles impartieron la enseñanza de que a este creyente se le admite en la fraternidad social de los discípulos mediante el rito externo del bautismo.
\vs p170 5:14 Cuando los más directos seguidores de Jesús reconocieron su parcial fracaso en llevar a cabo la instauración del reino en los corazones de los hombres mediante el dominio y la guía del espíritu del creyente individual, se dispusieron a evitar que sus enseñanzas se perdieran por completo, al sustituir el ideal del reino del Maestro por una institución social visible, la Iglesia cristiana, que fueron paulatinamente creando. Tras llevar a cabo este proceso de sustitución, a fin de mantener la congruencia y asegurar el reconocimiento de las enseñanzas del Maestro respecto al hecho del reino, procedieron a concebir la idea del reino futuro. La Iglesia, en cuanto estuvo bien establecida, comenzó a enseñar que el reino aparecería en realidad al culminar la era cristiana, con la segunda venida de Cristo.
\vs p170 5:15 De esta manera, el reino se convirtió en el concepto de una nueva era, en la idea de una futura visita y en el ideal de la redención final de los santos del Altísimo. Generalmente, los primeros cristianos (y muchos de los que vinieron después) perdieron de vista la idea de la relación Padre\hyp{}hijo contenida en las enseñanzas de Jesús sobre el reino, sustituyéndola por la bien organizada fraternidad social de la iglesia. La Iglesia, pues, se convirtió principalmente en una hermandad \bibemph{social} que logró reemplazar el concepto y el ideal de Jesús de una hermandad \bibemph{espiritual}.
\vs p170 5:16 Este ideal de Jesús fracasó en gran medida; si bien, sobre los principios de la vida personal del Maestro y de sus enseñanzas, complementados con los conceptos griego y persa de la vida eterna y ampliados con la doctrina de Filón que separaba lo temporal de lo espiritual, Pablo emprendió la formación de una de las comunidades humanas más avanzadas que jamás había existido en Urantia.
\vs p170 5:17 El concepto que Jesús enseñó sobre el reino aún pervive en las religiones desarrolladas del mundo. La Iglesia cristiana de Pablo es la sombra socializada y humanizada de lo que Jesús quería que fuese el reino de los cielos ---y que, con toda seguridad, llegará a ser---. Pablo y sus sucesores transfirieron parcialmente las alusiones de Jesús sobre la vida eterna desde el individuo a la Iglesia. Cristo se convirtió, pues, en la cabeza de la Iglesia, en lugar del hermano mayor del creyente individual que cree en el reino y en la familia del Padre. Pablo y sus contemporáneos aplicaron todas las consecuencias espirituales de Jesús, referidas a él mismo y a estos creyentes particulares, a la \bibemph{Iglesia,} como grupo de creyentes; y al hacerlo así, desvirtuaron el concepto de Jesús sobre el reino divino que existe en el corazón de cada uno de los creyentes.
\vs p170 5:18 Y, por tanto, durante siglos, la Iglesia cristiana ha desempeñado su labor en circunstancias embarazosas, porque tuvo el atrevimiento de reivindicar para sí esos poderes y privilegios misteriosos del reino que son tan solo factibles de efectuar y experimentar entre Jesús y sus hermanos, los creyentes espirituales. Y, así pues, resulta evidente que membrecía en la Iglesia no significa necesariamente fraternidad en el reino; una es espiritual, la otra, principalmente social.
\vs p170 5:19 Antes o después ha de surgir otro, y más grande, Juan el Bautista, que anunciará que “el reino de Dios se ha acercado” ---esto es, la vuelta al elevado concepto espiritual del reino proclamado por Jesús y que él equiparó a la voluntad de su Padre celestial, dominante y transcendente en el corazón del creyente ---y hará todo ello sin referencia alguna ni a la Iglesia visible en la tierra ni a la esperada segunda venida de Cristo---. Las enseñanzas \bibemph{reales} de Jesús han de resurgir y reafirmarse de tal manera que deshagan el trabajo de sus primeros seguidores, que crearon un sistema socio\hyp{}filosófico de creencias centrado en el \bibemph{hecho} de la estancia de Miguel en la tierra. En poco tiempo, la enseñanza de esta historia \bibemph{sobre} Jesús casi suplantó la predicación del evangelio del reino. De esta forma, una religión de carácter histórico desplazó a esas enseñanzas en la que Jesús había unido las ideas morales y los ideales espirituales de mayor elevación del hombre a su más sublime esperanza para el futuro ---la vida eterna---. Y aquel era el evangelio del reino.
\vs p170 5:20 Es precisamente por el hecho de las muchas vertientes del evangelio de Jesús por lo que, en pocos siglos, los estudiantes de las enseñanzas suyas de las que han quedado constancia se dividieron en tantos sistemas de culto y confesiones. Estas lastimosas divisiones entre creyentes cristianos resultan de la incapacidad de distinguir, de entre las múltiples enseñanzas del Maestro, la unicidad divina de su inigualable vida. Pero llegará el día en el que los verdaderos creyentes de Jesús no estarán divididos espiritualmente así en cuanto a su actitud ante los no creyentes. Siempre podremos tener pluralidad de comprensión intelectual e interpretación, e incluso diversos grados de socialización, pero la falta de hermandad espiritual es inexcusable y reprensible.
\vs p170 5:21 ¡No os equivoquéis! En las enseñanzas de Jesús hay una cualidad eterna que no les permitirá permanecer infructuosas para siempre en los corazones de los hombres reflexivos. El reino tal como lo concibió Jesús se ha malogrado en la tierra en buena medida; por el momento, una Iglesia externa ha ocupado su lugar; pero debéis comprender que esta Iglesia es solo la etapa larvaria del frustrado reino espiritual, que lo llevará, a través de esta era material, hasta una dispensación más espiritual, en la que las enseñanzas del Maestro tendrán una mejor oportunidad para su desarrollo. De este modo, la llamada Iglesia cristiana se convierte en la crisálida en la que el concepto del reino de Jesús duerme ahora. El reino de la hermandad divina sigue vivo y ciertamente resurgirá de su largo letargo tan ciertamente como la mariposa que con el tiempo se despliega bellamente de su menos atrayente larva, a partir de la que se ha transformado.
