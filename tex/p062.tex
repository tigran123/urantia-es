\upaper{62}{Las razas precursoras del hombre primitivo}
\author{Portador de vida}
\vs p062 0:1 Hace alrededor de un millón de años, los ancestros inmediatos del género humano hicieron su aparición mediante tres mutaciones sucesivas y repentinas surgidas del linaje primitivo del tipo lémur de mamíferos placentarios. Los factores dominantes de estos primeros lémures procedían del grupo americano occidental o tardío de plasma vital que venía evolucionado. Pero antes de establecerse la línea directa de la ascendencia humana, esta estirpe se vio reforzada por las aportaciones de la implantación central de vida que había evolucionado en África. El grupo oriental de vida contribuyó poco o nada de hecho al origen de la especie humana.
\usection{1. LOS PRIMEROS TIPOS DE LÉMURES}
\vs p062 1:1 Los primeros lémures implicados en la ascendencia de la especie humana no estaban directamente emparentados con las tribus previamente existentes de gibones y monos que vivían entonces en Eurasia y África del norte, y cuya progenie ha sobrevivido hasta el presente. Tampoco eran descendientes del tipo moderno de lémur, aunque ambos provenían de un ancestro común extinto hacía ya mucho tiempo.
\vs p062 1:2 Mientras que estos primeros lémures evolucionaron en el hemisferio occidental, los mamíferos ascendentes directos de la humanidad se establecieron en el suroeste de Asia, en la zona original de la implantación central de vida, aunque en las fronteras de las regiones orientales. Hacía algunos millones de años que los lémures del tipo norteamericano habían emigrado en dirección oeste por el puente terrestre de Bering y se habían encaminado lentamente hacia el suroeste a lo largo de la costa asiática. Estas tribus migratorias acabaron por alcanzar la región salubre situada entre el entonces expandido Mediterráneo y las zonas montañosas, que seguían elevándose, de la península índica. En estas tierras del oeste de la India, se unieron con otras estirpes favorables, estableciendo así la ascendencia de la raza humana.
\vs p062 1:3 Con el paso del tiempo, el litoral de la India, situado al suroeste de las montañas, se sumergió paulatinamente, aislando por completo la vida de esta región. No había ninguna posible vía de aproximarse o de escapar de esta península mesopotámica o persa excepto por el norte, y esa quedaba cortada repetidas veces por las expansiones meridionales de los glaciares. En esta zona, casi paradisíaca entonces, y a partir de los descendientes superiores de este tipo de lémur mamífero, brotaron dos grandes grupos: las tribus simias de los tiempos modernos y la actual especie humana.
\usection{2. LOS MAMÍFEROS PRIMIGENIOS}
\vs p062 2:1 Hace algo más de un millón de años que aparecieron \bibemph{de repente} los mamíferos primigenios mesopotámicos, los descendientes directos del tipo de lémur norteamericano de mamíferos placentarios. Eran criaturas pequeñas y activas, medían casi un metro de altura; y aunque no caminaban habitualmente sobre las patas traseras, podían mantenerse fácilmente erguidas. Eran peludas y ágiles y parloteaban a la manera de los monos, pero eran carnívoras, contrariamente a las tribus simias. Tenían un pulgar oponible primitivo, así como un dedo gordo prensil en el pie, sumamente útil. A partir de este momento, las especies prehumanas desarrollaron de manera sucesiva el pulgar oponible, mientras que perdían progresivamente la facultad prensora del dedo gordo del pie. Las futuras tribus de monos conservarían el dedo gordo prensil del pie, pero no llegarían a desarrollar el tipo de pulgar humano.
\vs p062 2:2 Estos mamíferos primigenios alcanzaban su pleno desarrollo cuando tenían tres o cuatro años de edad; la duración potencial de sus vidas era de unos veinte años de media. Por lo general, tenían una sola descendencia a la vez, aunque de vez en cuando nacían gemelos.
\vs p062 2:3 En relación a su tamaño, los miembros de esta nueva especie tenían el cerebro más grande que cualquier otro animal que hubiera existido en la tierra con anterioridad. Experimentaban muchas de las emociones y compartían un gran número de los instintos que caracterizarían más tarde al hombre primitivo; eran sumamente curiosos y expresaban un enorme júbilo cuando tenían éxito en cualquier tarea. Su apetito por la comida y su deseo sexual estaban bien desarrollados, y manifestaban una precisa selección sexual en una forma tosca de cortejo y en la elección de la pareja. Luchaban ferozmente para defender a sus congéneres y eran bastante tiernos en sus relaciones familiares; poseían un sentido de la humillación de sí mismos que rozaba el sentimiento de vergüenza y remordimiento. Eran muy afectuosos y conmovedoramente fieles hacia su pareja, pero si las circunstancias los separaban, optaban por una nueva compañía.
\vs p062 2:4 Al ser pequeños de estatura y poseer mentes despiertas para darse cuenta de los peligros existentes en su hábitat forestal, desarrollaron un extraordinario temor que los llevó a tomar unas prudentes medidas de precaución y que contribuyeron, de manera muy importante, a su supervivencia; entre estas está la construcción de refugios rudimentarios en las copas altas de los árboles, lo que eliminaba muchos de los peligros de la vida en tierra. El comienzo de la tendencia al miedo que experimenta la humanidad data concretamente de estos días.
\vs p062 2:5 Estos mamíferos primigenios desarrollaron un espíritu tribal mayor que el que se había manifestado previamente. Eran de hecho sumamente gregarios, pero extremadamente belicosos, no obstante, cuando se sentían de alguna manera perturbados en el ajetreo normal de su vida rutinaria; su ira se encendía por completo y, entonces, mostraban un fiero temperamento. Su naturaleza belicosa, sin embargo, tenía un buen motivo; los grupos mejor dotados no vacilaban en declarar la guerra a sus vecinos peor dotados; y, de este modo, por medio de la supervivencia selectiva, la especie fue mejorando de forma progresiva. Muy pronto dominaron a las criaturas más pequeñas de esta región, y muy pocas de las más antiguas tribus simias no carnívoras lograron sobrevivir.
\vs p062 2:6 Estos animales, pequeños y agresivos, se multiplicaron y se repartieron por la península mesopotámica durante más de mil años, mejorando constantemente en sus condiciones físicas e intelectuales. Y justo setenta generaciones después de que esta nueva tribu se hubiese originado a partir del tipo superior de ancestros lémures, ocurrió el siguiente desarrollo que marcó época: la \bibemph{repentina} diferenciación de los ancestros de la posterior etapa vital en la evolución de los seres humanos de Urantia.
\usection{3. LOS MAMÍFEROS INTERMEDIOS}
\vs p062 3:1 Temprano en la trayectoria seguida por los mamíferos primigenios, nacieron en la copa de los árboles, habitáculo de una pareja de orden superior de estas ágiles criaturas, dos gemelos, un macho y una hembra. Comparadas con sus ancestros, eran unas criaturas pequeñas, realmente hermosas. Tenían poco pelo en el cuerpo, pero, al vivir en un clima cálido y uniforme, esto no representaba desventaja alguna.
\vs p062 3:2 Estas crías llegarían a medir algo más de un metro veinte de altura. En todos los sentidos, eran más grandes que sus padres; sus piernas eran más largas y sus brazos más cortos. Tenían unos pulgares oponibles prácticamente perfectos, casi tan bien adaptados como el pulgar de los humanos actuales para realizar las tareas más diversas. Caminaban erguidos; tenían unos pies casi tan aptos para andar como los de las futuras razas humanas.
\vs p062 3:3 Su cerebro era inferior al de los seres humanos, y más pequeño, pero superior al de sus ancestros, y relativamente más grande. Los gemelos demostraron enseguida que tenían una inteligencia superior y pronto los reconocieron como jefes de toda la tribu de los mamíferos primigenios, llegando realmente a instituir una forma primitiva de organización social y una rudimentaria división racional del trabajo. Este hermano y su hermana se aparearon y pronto disfrutaron de la compañía de veintiún hijos muy semejantes a ellos, todos con más de un metro veinte de altura y mejor dotados, en todos los sentidos, a las especies que los  antecedieron. Este nuevo grupo, formó el núcleo de los mamíferos intermedios.
\vs p062 3:4 Cuando este grupo, nuevo y superior, aumentó en número estalló la guerra, una guerra implacable; y cuando el terrible enfrentamiento terminó, no quedaba vivo ni un solo individuo de la raza preexistente y antecesora de mamíferos primigenios. Los descendientes de esa especie, menos numerosos pero más fuertes e inteligentes, habían sobrevivido en detrimento de sus ancestros.
\vs p062 3:5 A continuación, y durante casi quince mil años (seiscientas generaciones), estas criaturas se convirtieron en el terror de esta parte del mundo. Todos los animales grandes y fieros de otros tiempos habían perecido. Los grandes animales salvajes, nativos de estas regiones, no eran carnívoros, y las especies más grandes de la familia felina, los leones y los tigres, aún no habían irrumpido en aquel recoveco peculiarmente protegido de la superficie de la tierra. Por lo tanto, estos mamíferos intermedios se hicieron fuertes y sometieron enteramente a su rincón de la creación.
\vs p062 3:6 \pc Comparados con la especie que los antecedió, los mamíferos intermedios significaban una mejora en todos los aspectos. Incluso su esperanza potencial de vida, de unos veinticinco años, era mayor. En esta nueva especie, aparecieron algunos rasgos humanos elementales. Además de las propensiones innatas de sus ancestros, estos mamíferos intermedios eran capaces de mostrar disgusto ante ciertas situaciones repulsivas. Poseían, además, un instinto de acaparamiento bien definido; escondían la comida para poder consumirla más tarde y eran muy dados a reunir guijarros lisos y redondos y algunos tipos de piedras redondas que usaban como munición defensiva y ofensiva.
\vs p062 3:7 Estos mamíferos intermedios fueron los primeros en manifestar una inequívoca tendencia a la construcción, tal como lo demuestra su rivalidad en la edificación de sus hogares en las copas de los árboles al igual que en refugios subterráneos repletos de túneles; fue la primera especie de mamíferos en garantizarse su seguridad tanto en los refugios arbóreos como en los subterráneos. Mayormente, abandonaron su hábitat en los árboles, viviendo en el suelo durante el día y durmiendo por la noche en las copas de los árboles.
\vs p062 3:8 Con el paso del tiempo, el natural aumento del número de estos mamíferos acabó por provocar una seria competencia por el alimento y un grave antagonismo sexual, todo lo cual desembocó en una serie de guerras internas que casi extermina a toda la especie. Estas luchas continuaron hasta que solo un grupo de menos de cien individuos quedó vivo. Pero la paz imperó de nuevo, y esta única tribu superviviente construyó otra vez sus lugares de descanso en las copas de los árboles y reanudó nuevamente una existencia normal y semipacífica.
\vs p062 3:9 \pc Es difícil que os percatéis de la razón por la que, por tan estrecho margen, vuestros ancestros prehumanos eludieron la extinción cada cierto tiempo. Si la rana antecesora de toda la humanidad hubiera saltado en determinada ocasión cinco centímetros menos, todo el curso de la evolución hubiese cambiado significativamente. La progenitora directa, similar a los lémures, de la especie de los mamíferos primigenios escapó por muy poco de la muerte, no menos de cinco veces, antes de dar a luz al padre del nuevo orden de mamíferos superiores. Si bien, la mayor ocasión de peligro de todas se produjo cuando un rayo impactó en el árbol donde dormía la futura madre de los gemelos primates. Ambos padres, mamíferos intermedios, sufrieron una fuerte conmoción y tuvieron graves quemaduras; tres de sus siete hijos murieron por esta saeta de los cielos. Estos animales en evolución eran casi supersticiosos. Esta pareja, sobre cuyo refugio en la copa del árbol había caído el rayo, era en realidad la líder del grupo más avanzado de la especie de los mamíferos intermedios, y, siguiendo su ejemplo, más de la mitad de la tribu, que incluía a las familias más inteligentes, se trasladó a unos tres kilómetros de este sitio y empezó a construir unos nuevos habitáculos en la copa de los árboles y nuevos refugios subterráneos: sus lugares de retiro temporales en caso de peligro repentino.
\vs p062 3:10 Poco después de concluir su hogar, esta pareja, experimentada en tantas luchas, se convirtió en los orgullosos progenitores de gemelos: los animales más interesantes y destacados que habían nacido en el mundo hasta ese momento; se trataba de los primeros miembros de la nueva especie de los \bibemph{primates,} integrantes de la siguiente etapa vital de la evolución prehumana.
\vs p062 3:11 \pc Al mismo tiempo que nacían estos gemelos primates, otra pareja ---un macho y una hembra particularmente retrasados de la tribu de los mamíferos intermedios, una pareja mental y físicamente peor dotados--- también dio a luz a unos gemelos. Estos gemelos, un macho y una hembra, eran indiferentes a las conquistas; solo se preocupaban de conseguir comida y, puesto que no se alimentaban de carne, pronto perdieron todo interés en buscar presas. Los gemelos se convertirían pronto en los fundadores de las tribus simias modernas. Sus descendientes buscaron las regiones meridionales más cálidas con climas templados y abundancia de frutas tropicales. Allí han continuado de forma muy parecida desde aquellos días, exceptuando aquellas ramas que se aparearon con los tipos tempranos de gibones y monos, y que, por consiguiente, sufrieron un gran deterioro.
\vs p062 3:12 Y así, se puede apreciar fácilmente que el hombre y el mono solo se relacionan entre sí por el hecho de que ambos descienden de los mamíferos intermedios, de una tribu en la que se produjo el nacimiento simultáneo y la posterior separación de dos parejas de gemelos: la pareja peor dotada, destinada a procrear a los tipos modernos de monos, babuinos, chimpancés y gorilas, y la mejor dotada, destinada a continuar la línea ascendente de la que surgiría por evolución el hombre mismo.
\vs p062 3:13 En verdad, el hombre moderno y los simios provienen de la misma tribu y de la misma especie, pero no de los mismos progenitores. Los antepasados del hombre descendían de la estirpe mejor dotada del selecto remanente de esta tribu de mamíferos intermedios, mientras que los simios modernos (excepto algunos tipos preexistentes de lémures, gibones, monos y otras criaturas similares) son los descendientes de la pareja peor dotada de este grupo de mamíferos intermedios, una pareja que solo sobrevivió porque, en el último y encarnizado combate de su tribu, se escondieron durante más de dos semanas en un refugio subterráneo donde se almacenaba el alimento, y no salieron hasta que se terminaron por completo las hostilidades.
\usection{4. LOS PRIMATES}
\vs p062 4:1 Volvamos al nacimiento de los gemelos mejor dotados, un macho y una hembra, ambos miembros destacados de la tribu de los mamíferos intermedios. Eran crías de un orden inhabitual; tenían incluso menos pelo en el cuerpo que sus padres y, desde muy pequeños, se empeñaron en caminar erguidos. Sus ancestros habían aprendido a caminar siempre sobre sus patas traseras, pero estos gemelos primates permanecieron erguidos desde el principio. Llegaron a medir más de un metro y medio y, en relación con otros miembros de la tribu, sus cabezas eran más grandes. Aunque aprendieron pronto a comunicarse entre ellos por medio de signos y sonidos, nunca fueron capaces de hacer comprender a su pueblo estos nuevos símbolos.
\vs p062 4:2 Cuando tenían unos catorce años de edad, huyeron de la tribu, en dirección oeste para criar a su familia y constituir la nueva especie de los primates. A estas nuevas criaturas se las denomina, apropiadamente, \bibemph{primates,} puesto que fueron los ancestros animales directos e inmediatos de la mismísima familia humana.
\vs p062 4:3 Así fue como los primates llegaron a ocupar una zona de la costa oeste de la península mesopotámica, que entonces se adentraba en el mar del sur, mientras que las tribus menos inteligentes y estrechamente emparentadas vivían en torno al vértice de la península y por su costa oriental.
\vs p062 4:4 \pc Los primates eran más humanos y menos animales que sus predecesores, los mamíferos intermedios. Las proporciones del esqueleto de esta nueva especie eran muy similares a las de las razas humanas primitivas. El tipo humano de mano y de pie se había desarrollado por completo, y estas criaturas podían caminar e incluso correr igual que cualquiera de sus futuros descendientes humanos. Abandonaron prácticamente la vida en los árboles, aunque siguieron recurriendo por la noche a las copas de los árboles como medida de seguridad, ya que, como sus tempranos ancestros, estaban muy condicionados por el miedo. El mayor empleo de sus manos contribuyó bastante a desarrollar su inherente capacidad cerebral, pero todavía no poseían una mente a la que realmente se le pudiera denominar humana.
\vs p062 4:5 Aunque, en su naturaleza emocional, los primates diferían poco de sus antepasados, sí tendían a inclinaciones más humanas. De hecho, eran unos animales magníficos y superiores; alcanzaban la madurez hacia los diez años de edad y su esperanza de vida natural era de unos cuarenta años. Esto es, podrían haber vivido hasta esa edad de haber fallecido de muerte natural, pero en aquellos tempranos días, muy pocos animales morían así; la lucha por subsistir era del todo demasiado intensa.
\vs p062 4:6 Y ahora, después de casi novecientas generaciones de evolución, que abarcaban alrededor de veintiún mil años desde el origen de los mamíferos primigenios, los primates dieron nacimiento \bibemph{de repente} a dos excepcionales criaturas: a los primeros seres realmente humanos.
\vs p062 4:7 \pc Así fue como los mamíferos primigenios, surgidos del tipo norteamericano de lémur, dieron origen a los mamíferos intermedios y estos, a su vez, engendraron a los primates superiores, que se convirtieron en los ancestros directos de la raza humana primitiva. Las tribus primates constituyeron el último eslabón vital en la evolución del hombre, pero en menos de cinco mil años no quedó ni un solo miembro de estas extraordinarias tribus.
\usection{5. LOS PRIMEROS SERES HUMANOS}
\vs p062 5:1 El nacimiento de los dos primeros seres humanos tuvo lugar exactamente 993\,419 años antes del año 1934 d. C.
\vs p062 5:2 Estas dos extraordinarias criaturas eran verdaderos seres humanos. Sus pulgares eran perfectamente humanos, como los de muchos de sus ancestros, y tenían los pies tan perfectos como los de las razas humanas de hoy día. Caminaban y corrían, pero no eran trepadores; la función prensil del dedo gordo del pie había desaparecido por completo. Cuando el peligro los llevaba a escalar hasta las copas de los árboles, lo hacían como los actuales humanos. Subían por el tronco de los árboles como los osos y no como los chimpancés o los gorilas, que lo hacen balanceándose de rama en rama.
\vs p062 5:3 Estos primeros seres humanos (y sus descendientes) alcanzaban la plena madurez a la edad de doce años y su esperanza potencial de vida era de unos setenta y cinco años.
\vs p062 5:4 Muchas emociones nuevas harían pronto su aparición en estos gemelos humanos. Sentían admiración tanto por las cosas como por otros seres y daban muestras de una considerable vanidad. Si bien, dentro de su desarrollo emocional, el más destacable de los avances fue la aparición repentina de un nuevo conjunto de sentimientos realmente humanos: aquellos relacionados con la adoración, que incluían el sobrecogimiento, la veneración, la humildad e incluso una forma primitiva de gratitud. El miedo, junto a la ignorancia de los fenómenos naturales, estaba a punto de dar origen a la religión primitiva.
\vs p062 5:5 En estos seres primitivos no solo se manifestaban estos sentimientos humanos, sino que también estaban presentes, de forma elemental, muchos otros sentimientos altamente evolucionados. Conocían ligeramente la lástima, la vergüenza y el reproche, y eran muy conscientes del amor, del odio y de la venganza; eran además susceptibles de experimentar unos acusados sentimientos de celos.
\vs p062 5:6 Estos dos primeros humanos ---los gemelos--- representaron una dura prueba para sus padres primates. Eran tan curiosos e intrépidos que, antes de cumplir los ocho años, estuvieron a punto de perder la vida en múltiples ocasiones. En efecto, al cumplir los doce años estaban bastante llenos de cicatrices.
\vs p062 5:7 Muy pronto aprendieron a comunicarse verbalmente; a los diez años ya habían elaborado un mejor lenguaje de signos y palabras de casi medio centenar de ideas y habían perfeccionado y desarrollado de forma considerable el rudimentario método de comunicación de sus antepasados. Pero, por mucho que se esforzaran, solo pudieron enseñar a sus padres algunos pocos signos y símbolos nuevos.
\vs p062 5:8 Un luminoso día, cuando tenían unos nueve años de edad, viajaron río abajo y mantuvieron una reunión de gran trascendencia. Todas las inteligencias celestiales emplazadas en Urantia, incluido yo mismo, estábamos presentes como observadores del desarrollo de este encuentro, que tuvo lugar al mediodía. En este memorable día, llegaron al compromiso de vivir el uno con el otro y el uno para el otro, y este fue el primero de una serie de acuerdos que acabaron por desembocar en la decisión de huir de sus compañeros animales peor dotados y partir hacia el norte, sin saber que de este modo estaban fundando la raza humana.
\vs p062 5:9 Aunque todos teníamos una gran preocupación por lo que estos dos pequeños salvajes tenían previsto hacer, no estábamos capacitados para dirigir sus mentes, y no influimos arbitrariamente en sus decisiones ---no podíamos hacerlo---. Pero dentro de los límites permisibles de nuestros cometidos planetarios, nosotros, los portadores de vida, junto con nuestros acompañantes, cooperamos todos para orientar a los gemelos humanos hacia el norte, lejos de su pueblo de seres peludos, parcialmente arborícolas. Y así, en razón de su propia e inteligente elección, los gemelos por fin \bibemph{emigraron} y, a causa de nuestra supervisión, lo hicieron \bibemph{en dirección norte,} hacia una región apartada en la que eludieran la posibilidad de degradarse biológicamente al mezclarse con sus parientes peor dotados de las tribus de los primates.
\vs p062 5:10 Poco antes de su partida de aquellos bosques, que habían sido su hogar, perdieron a su madre durante un ataque de los gibones. Aunque no poseía su misma inteligencia, ella, como mamífero, sentía por su progenie un loable y elevado afecto, y dio su vida valientemente intentando salvar a aquella magnífica pareja. Su sacrificio no fue en vano, ya que contuvo al enemigo hasta que el padre llegó con refuerzos haciendo que los invasores huyeran.
\vs p062 5:11 Poco después de que esta joven pareja abandonara a sus compañeros para fundar la raza humana, su padre primate quedó desconsolado, descorazonado. Se negó a comer, incluso aunque sus otros hijos le llevaran el alimento. Al haber perdido a su brillante prole, no le merecía la pena vivir la vida entre sus mediocres congéneres; empezó a vagar entonces por el bosque hasta que unos hostiles gibones lo atacaron y mataron a golpes.
\usection{6. EVOLUCIÓN DE LA MENTE HUMANA}
\vs p062 6:1 Nosotros, los portadores de vida de Urantia, que habíamos pasado por la larga vigilia de una espera atenta desde el día en el que implantamos el plasma vital por primera vez en las aguas planetarias, la aparición de los primeros seres realmente inteligentes y volitivos nos produjo naturalmente un gran regocijo y una satisfacción suprema.
\vs p062 6:2 Habíamos estado vigilando el desarrollo mental de los gemelos mediante la observación de la actuación de los siete espíritus asistentes de la mente, asignados a Urantia en el momento de nuestra llegada al planeta. Durante todo el largo desarrollo evolutivo de la vida planetaria, estos incansables servidores de la mente siempre habían indicado sus posibilidades, cada vez mayores, de obrar en la capacidad cerebral de las criaturas animales, a medida que esta se expandía de forma gradual hacia niveles superiores.
\vs p062 6:3 En un principio solo el \bibemph{espíritu de intuición} pudo obrar sobre el comportamiento instintivo y reflejo de la vida animal primitiva. Al diferenciarse los tipos superiores de animales, el \bibemph{espíritu de entendimiento} fue capaz de dotar a estas criaturas con el don de la asociación espontánea de ideas. Luego observamos la actuación del \bibemph{espíritu de valentía;} los animales en evolución efectivamente desarrollaron una forma elemental de conciencia protectora de sí mismos. Tras la aparición de los grupos de mamíferos, vimos cómo el \bibemph{espíritu del conocimiento} se manifestaba en mayor grado. Y la evolución de los mamíferos superiores permitió la actuación del \bibemph{espíritu de consejo,} con el consiguiente incremento del instinto de manada y los comienzos de un desarrollo social primitivo.
\vs p062 6:4 Desde los tiempos de los mamíferos primigenios, de los mamíferos intermedios y de los primates, habíamos venido observando, cada vez más, el servicio en aumento de los cinco primeros asistentes de la mente, pero en el tipo de mente evolutiva de Urantia jamás habían podido actuar los dos restantes: los servidores superiores de la mente.
\vs p062 6:5 Imaginad nuestro gozo ese día ---los gemelos tenían unos diez años--- en el que el \bibemph{espíritu de adoración} realizó su primer contacto con la mente de la gemela y, poco después, con la del gemelo. Supimos que algo muy semejante a la mente humana se acercaba a su punto culminante; y cuando, aproximadamente al año, decidieron finalmente, como fruto de la reflexión y de una deliberada determinación, huir de su lugar de origen y viajar hacia el norte, el \bibemph{espíritu de sabiduría} empezó a obrar entonces en Urantia, en estas dos mentes humanas, que fueron reconocidas en aquel momento como tales.
\vs p062 6:6 Se produjo de inmediato un nuevo orden de activación para el servicio de los siete espíritus asistentes de la mente. Rebosábamos de expectación; nos dimos cuenta de que el tan esperado momento estaba cercano; supimos que estábamos a las puertas de hacer realidad nuestro empeño, tan dilatado en el tiempo, de desarrollar criaturas volitivas en Urantia.
\usection{7. RECONOCIMIENTO COMO MUNDO HABITADO}
\vs p062 7:1 No tuvimos que esperar mucho tiempo. Al mediodía, al siguiente día de la huida de los gemelos, se produjo, en el centro receptor planetario de Urantia, un primer test de conexión rápida con las señales de las vías circulatorias del universo. Naturalmente, todos estábamos muy emocionados porque nos percatábamos de que algo grande estaba a punto de suceder; pero al ser este mundo una base de vida experimental, no teníamos la más ligera idea de cómo seríamos informados del reconocimiento de vida inteligente en el planeta. Pero no permanecimos mucho tiempo en la incertidumbre. Al tercer día de la fuga de los gemelos, y antes de que partiera el colectivo de portadores de vida, llegó el arcángel de Nebadón encargado de establecer la vía circulatoria inicial del planeta.
\vs p062 7:2 Fue un día memorable en Urantia cuando nuestro pequeño grupo se reunió alrededor de la terminal planetaria de las comunicaciones espaciales, y recibió el primer mensaje de Lugar de Salvación por la recién establecida vía circulatoria planetaria de la mente. Este primer mensaje, dictado por el jefe del colectivo de arcángeles, decía:
\vs p062 7:3 “A los portadores de vida de Urantia: ¡Saludos! Trasmitimos testimoniando la gran satisfacción sentida en Lugar de Salvación, Edentia y Jerusem en honor del registro en la sede central de Nebadón indicando la existencia de una mente con dignidad volitiva en Urantia. Se ha observado que los gemelos determinaron deliberadamente huir hacia el norte y apartar a su prole de sus ancestros, pobremente dotados. Esta es la primera decisión que toma una mente ---una mente de tipo humano--- en Urantia, y establece de forma automática la vía de comunicación por el que este mensaje preliminar de reconocimiento se está transmitiendo”.
\vs p062 7:4 Luego, por esta nueva vía circulatoria, llegaron las felicitaciones de los Altísimos de Edentia con instrucciones para los portadores de vida residentes en las que se nos prohibía interferir en el modelo de vida que habíamos establecido. Se nos ordenó que no interviniésemos en los asuntos relativos al progreso humano. No se debe inferir que los portadores de vida interfieran de modo arbitrario y mecánico en el desarrollo natural de los planes evolutivos de un planeta, porque no lo hacemos. Pero, hasta ese momento, se nos había permitido actuar sobre el medio ambiente y proteger de manera especial el plasma vital; y, aunque completamente natural, esta supervisión había sido excepcional, y tenía que ser discontinuada.
\vs p062 7:5 Y en cuanto los Altísimos terminaron de hablar, el bello mensaje de Lucifer, entonces soberano del sistema de Satania, se dejó oír en el planeta. Los portadores de vida escucharon las palabras de acogida de su propio jefe y recibieron su permiso para regresar a Jerusem. Este mensaje de Lucifer contenía la aprobación oficial de la labor de los portadores de vida en Urantia, y nos absolvía de toda futura reprobación en relación a cualquiera de las tareas que habíamos realizado para mejorar los modelos de vida de Nebadón, de acuerdo a lo establecido en el sistema de Satania.
\vs p062 7:6 Estos mensajes de Lugar de Salvación, Edentia y Jerusem señalaban oficialmente el final de la supervisión multisecular del planeta por parte de los portadores de vida. Durante eras, habíamos estado de servicio, asistidos únicamente por los siete espíritus asistentes de la mente y por los controladores físicos mayores. Y, ahora, una vez que la voluntad, la facultad de elegir la adoración y la ascensión, había aparecido en las criaturas evolutivas del planeta, nos dimos cuenta de que nuestro trabajo había terminado, y nuestro grupo se dispuso a partir. Como Urantia era un mundo de modificación de la vida, se nos concedió permiso para dejar en el planeta a dos portadores de vida de superior rango con doce ayudantes; se me escogió como miembro de este grupo y, desde entonces, he permanecido en Urantia.
\vs p062 7:7 Hace exactamente 993\,408 años (desde el año 1934 d. C.) que se reconoció oficialmente a Urantia, perteneciente al Universo de Nebadón, como planeta con presencia humana. La evolución biológica había alcanzado de nuevo los niveles humanos de dignidad volitiva; el hombre estaba presente en el planeta 606 de Satania.
\vsetoff
\vs p062 7:8 [Auspiciado por un portador de vida de Nebadón, residente en Urantia.]
