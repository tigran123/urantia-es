\upaper{69}{Instituciones humanas primitivas}
\author{Melquisedec}
\vs p069 0:1 Emocionalmente, el hombre trasciende a sus ancestros animales por su capacidad de valorar el humor, el arte y la religión. Socialmente, el hombre manifiesta su superioridad por cuanto que es fabricante de herramientas, comunicador y creador de instituciones.
\vs p069 0:2 Cuando los seres humanos mantienen los grupos sociales por mucho tiempo, estas colectividades son siempre proclives a la formación de determinadas pautas que culminan en su institucionalización. La mayor parte de las instituciones del hombre han demostrado que ahorran trabajo a la vez que contribuyen a la mejora de la seguridad del grupo.
\vs p069 0:3 El hombre civilizado está muy orgulloso de la naturaleza, la estabilidad y la continuidad de sus instituciones establecidas, pero todas las instituciones humanas no son más que las costumbres acumuladas del pasado tal como han sido conservadas por los tabúes y dignificadas por la religión. Estos legados se convierten en tradiciones, y las tradiciones acaban por transformarse en convenciones.
\usection{1. INSTITUCIONES HUMANAS BÁSICAS}
\vs p069 1:1 Todas las instituciones humanas sirven a alguna necesidad social, pasada o actual, con independencia de que su desmesurado desarrollo menoscabe de forma indefectible la integridad y valía de las personas al eclipsarlas personalmente y reducir su iniciativa. El hombre debe regir sus instituciones, en lugar de dejarse dominar por estas creaciones que han nacido con el avance de la civilización.
\vs p069 1:2 \pc Las instituciones humanas son de tres clases generales:
\vs p069 1:3 \li{1.}\bibemph{Instituciones de autoconservación}. Estas instituciones engloban aquellas prácticas nacidas del hambre de comida y de instintos afines relacionados con la autopreservación. En ellas se incluyen la industria, la propiedad, la guerra para fines de lucro, y todos los mecanismos reguladores de la sociedad. Tarde o temprano, el instinto del miedo favorece el establecimiento de estas instituciones de supervivencia por medio del tabú, las convenciones y la aprobación religiosa. Si bien, el miedo, la ignorancia y la superstición han desempeñado un papel notable en los tempranos orígenes y en el desarrollo posterior de todas las instituciones humanas.
\vs p069 1:4 \li{2.}\bibemph{Instituciones de autoperpetuación}. Se trata de estructuras sociales que surgen del apetito sexual, del instinto materno y de los sentimientos tiernos y superiores de las razas. Abarcan la protección social del hogar y la escuela, de la vida familiar, la educación, la ética y la religión. Incluyen las costumbres matrimoniales, las guerras de defensa y el establecimiento del hogar.
\vs p069 1:5 \li{3.}\bibemph{Instituciones de autogratificación}. Se trata de prácticas que surgen de la propensión a la vanidad y de las emociones del orgullo; y engloban las costumbres en el vestir y en el adorno personal, los usos sociales, las guerras por conseguir gloria, el baile, la diversión, los juegos y otras facetas de gratificación sensual. Pero en la civilización jamás se han desarrollado peculiares instituciones de autogratificación.
\vs p069 1:6 \pc Estos tres grupos de prácticas sociales están íntimamente interrelacionados y son intensamente interdependientes unos de otros. En Urantia, constituyen una organización compleja que actúa como un solo mecanismo social.
\usection{2. COMIENZOS DE LA LABORIOSIDAD}
\vs p069 2:1 La laboriosidad del hombre primitivo se desarrolló lentamente en previsión al terror a las hambrunas. Al comienzo de su existencia, el hombre empezó a aprender de algunos animales, los cuales, durante una cosecha abundante, almacenaban comida para los días de escasez.
\vs p069 2:2 Antes de los primeros momentos de economización y de esta laboriosidad, las tribus vivían normalmente en una situación de penuria y de auténtico sufrimiento. Los primeros hombres tenían que competir con todo el reino animal para conseguir sus alimentos. El peso que conlleva esta competición arrastra siempre al hombre hacia el nivel animal; la escasez es un estado natural y opresivo. Los bienes no son dones naturales; son el resultado del trabajo, del conocimiento y de la organización.
\vs p069 2:3 El hombre primitivo no tardó demasiado en reconocer las ventajas de la asociación. Esta llevó a la organización, y la primera consecuencia de dicha organización fue la división del trabajo, con su inminente ahorro de tiempo y de materiales. La especialización del trabajo surgió como adecuación a unas demandas ---buscando el camino que ofreciera una menor resistencia---. Los salvajes primitivos nunca realizaban verdaderos trabajos ni gustosamente ni voluntariamente. En su caso, su disponibilidad se debía a la coerción que la necesidad ejercía sobre ellos.
\vs p069 2:4 El hombre primitivo detestaba el trabajo duro, y no tenía prisas a menos que se enfrentara a un grave peligro. El elemento tiempo propio de una labor, la idea de realizar una determinada tarea dentro de los límites de un cierto tiempo, es una noción enteramente moderna. Los antiguos nunca se apresuraban. La doble exigencia de la intensa lucha por la existencia y los niveles de vida, continuamente en avance, fue la que impulsó a estas razas, naturalmente inactivas, por los senderos de la laboriosidad.
\vs p069 2:5 El trabajo, las iniciativas planificadas, distingue al hombre de los animales, cuyos actos son en gran parte instintivos. La necesidad de trabajar es la principal bendición del hombre. Toda la comitiva del príncipe trabajó; hizo mucho por ennoblecer las labores físicas en Urantia. Adán fue horticultor; el Dios de los hebreos laboró ---fue el creador y sostenedor de todo---. Los hebreos fueron la primera tribu que primó supremamente la laboriosidad; fue el primer pueblo en decretar que “si alguno no quiere trabajar, tampoco coma”. Pero muchas de las religiones del mundo volvieron al temprano ideal de la ociosidad. Júpiter era aficionado a la diversión y Buda se convirtió en un devoto reflexivo del ocio.
\vs p069 2:6 Las tribus sangiks eran bastante trabajadoras cuando residían lejos de los trópicos. Si bien, hubo una muy dilatada lucha entre los perezosos devotos de la magia y los apóstoles del trabajo ---aquellos que ejercitaban la previsión---.
\vs p069 2:7 El primer acto de previsión del ser humano fue la conservación del fuego, del agua y de la comida. Pero el hombre primitivo era un jugador nato; siempre quería obtener algo a cambio de nada, y, muy a menudo, durante estos tempranos tiempos, los logros conseguidos gracias al ejercicio de la paciencia se atribuían a los encantamientos. La magia tardaría en dar paso a la previsión, a la abnegación y a la laboriosidad.
\usection{3. ESPECIALIZACIÓN DEL TRABAJO}
\vs p069 3:1 En la sociedad primitiva, la división del trabajo se determinaba primero por las circunstancias naturales y, después, por las sociales. El primer tipo de especialización laboral fue:
\vs p069 3:2 \li{1.}\bibemph{Especialización basada en el sexo}. La labor de la mujer se deriva de la presencia puntual de los hijos; las mujeres, por naturaleza, aman a los recién nacidos más que los hombres. Por ello, la mujer se convirtió en trabajadora habitual, mientras que el hombre se hizo cazador y luchador, disfrutando de marcados períodos de trabajo y descanso.
\vs p069 3:3 \pc A través de los tiempos, los tabúes sirvieron para mantener a la mujer rigurosamente en su propio sector laboral. El hombre, de forma bastante interesada, escogió el trabajo más agradable, dejando para la mujer el trabajo penoso y rutinario. El hombre siempre se ha avergonzado de hacer las tareas de la mujer, pero la mujer nunca ha mostrado reticencias en hacer las del hombre. Si bien, resulta extraño dejar constancia de que ambos siempre hayan trabajado juntos en la construcción y el equipamiento del hogar.
\vs p069 3:4 \li{2.}\bibemph{Modificación derivada de la edad y de las enfermedades}. Estas diferencias determinaron la siguiente división del trabajo. Pronto se puso a trabajar a los ancianos y a los lisiados fabricando herramientas y armas. Después se les asignó a la construcción de obras de riego.
\vs p069 3:5 \li{3.}\bibemph{Diferenciación basada en la religión}. Los curanderos fueron los primeros seres humanos en quedar exentos del duro trabajo físico; fueron los precursores de las clases profesionales. Los herreros formaban un reducido grupo que competía en calidad de magos con los curanderos. Su pericia en el trabajo con los metales hizo que la gente los temiera. Los llamados “herreros blancos” y los “herreros negros” dieron origen a las tempranas creencias sobre la magia blanca y la magia negra. Y estas creencias se relacionarían después con la superstición de los fantasmas buenos y malos, esto es, de los espíritus buenos y malos.
\vs p069 3:6 Los herreros protagonizaron el primer grupo no religioso en disfrutar de privilegios especiales. Se les consideraba neutrales durante las guerras, y este tiempo libre sobrante los llevó a convertirse, como clase, en los políticos de la sociedad primitiva. Sin embargo, a causa de las graves violaciones de sus privilegios, los herreros llegaron a ser odiados en todas partes; los curanderos, no perdieron tiempo en propiciar el odio hacia sus rivales. En esta primera disputa entre ciencia y religión, fue la religión (la superstición) la ganadora. Tras haber sido expulsados de las aldeas, los herreros establecieron las primeras posadas, o albergues públicos, en las periferias de los poblados.
\vs p069 3:7 \li{4.}\bibemph{Amos y esclavos}. La siguiente diferenciación del trabajo surgió a partir de la relación de los conquistadores hacia los conquistados, algo que marcó el principio de la esclavitud humana.
\vs p069 3:8 \li{5.}\bibemph{Diferenciación basada en las distintas dotes físicas y mentales}. Otras divisiones del trabajo se vieron favorecidas por las intrínsecas diferencias entre los hombres; no todos los seres humanos nacen iguales.
\vs p069 3:9 En el ámbito laboral, los primeros especialistas fueron los tallistas de pedernal y los canteros; a continuación vinieron los herreros. Más tarde, se desarrolló una especialización grupal; familias y clanes enteros se dedicaron a ciertos tipos de trabajo. El origen de una de las primeras castas de sacerdotes, aparte de los curanderos tribales, se debió al enaltecimiento supersticioso de una familia de expertos fabricantes de espadas.
\vs p069 3:10 \pc El primer grupo de especialistas en el sector laboral fueron los exportadores de sal gema y los alfareros. Las mujeres trabajaron la alfarería sencilla y, los hombres, la lujosa. En algunas tribus, eran las mujeres las que trabajaban la costura y la tejeduría; en otras, eran los hombres.
\vs p069 3:11 Las mujeres fueron las primeras en dedicarse al trueque; se las empleó como espías, llevando a cabo el intercambio de bienes como algo secundario. El trueque se extendió enseguida; las mujeres actuaban de intermediarias ---distribuidoras---. Entonces llegó la clase mercantil, que cobraba una comisión, unas ganancias, por sus servicios. El crecimiento del trueque en grupo se convirtió en el comercio; y, tras el intercambio de productos básicos, vino el intercambio de mano de obra competente.
\usection{4. INICIOS DEL COMERCIO}
\vs p069 4:1 Al igual que el matrimonio por contrato siguió al matrimonio por captura, el comercio mediante trueque siguió a la incautación por redadas. Pero hubo un prolongado período intermedio de piratería entre los tempranos trueques silenciosos y el posterior comercio por métodos modernos de intercambio.
\vs p069 4:2 El primer trueque se llevó a cabo de manos de comerciantes armados que dejaban sus mercancías en un lugar neutral. Las mujeres se ocupaban de los primitivos mercados; fueron las primeras comerciantes, algo que se debía al hecho de que eran ellas quienes portaban las cargas; los hombres eran guerreros. Muy pronto apareció el mostrador de intercambios, esto es, un muro lo suficientemente ancho como para evitar que los comerciantes se alcanzasen unos a otros con sus armas.
\vs p069 4:3 A objeto de estar de guardia sobre los depósitos de mercancías y facilitar el trueque silencioso, se utilizó un fetiche. Estos mercados eran seguros contra el robo; nada se podía retirar de allí excepto por trueque o compra; con un fetiche de guardia, los bienes estaban siempre protegidos. Los primeros comerciantes eran siempre escrupulosamente honrados dentro de sus propias tribus, pero creían que era lícito engañar a desconocidos que venían desde lugares distantes. Incluso los hebreos primitivos disponían de otro código de ética para hacer tratos con los gentiles.
\vs p069 4:4 El trueque silencioso continuó durante mucho tiempo antes de que los hombres se reuniesen, desarmados, en las sagrada plaza del mercado. Estas mismas plazas se convirtieron en los primeros lugares seguros y, en algunos países, se conocieron posteriormente como “ciudades de refugio”. Todo fugitivo que llegara a la plaza del mercado estaba a salvo y protegido de ataques.
\vs p069 4:5 \pc Los granos de trigo y de otros cereales se utilizaron como primeros pesos. El primer medio de intercambio fue un pescado o una cabra. Más tarde, la vaca se convertiría en una unidad de trueque.
\vs p069 4:6 La escritura moderna tuvo su origen en las primeras anotaciones comerciales; el primer texto escrito del hombre fue un documento de promoción comercial, una publicidad de la sal. Muchas de las tempranas guerras se libraron por la posesión de depósitos naturales, tales como los de pedernal, sal y metales. El primer tratado formal tuvo que ver con la explotación intertribal de un yacimiento de sal. Acudir a esos lugares para concertar los acuerdos brindó, a las distintas tribus, la oportunidad de intercambiar ideas y entremezclarse de modo amistoso y pacífico.
\vs p069 4:7 La escritura se desarrolló desde los estadios del “palo mensajero”, cuerdas anudadas, pictografías, jeroglíficos y cinturones de cuentas de concha, hasta los tempranos alfabetos simbólicos. El envío de mensajes ha evolucionado desde las primitivas señales de humo hasta los portados por corredores, jinetes, ferrocarriles y aviones, al igual que los que llegan a través del telégrafo, del teléfono y de las comunicaciones inalámbricas.
\vs p069 4:8 Los antiguos comerciantes llevaron nuevas ideas y métodos mejores a todo el mundo habitado. El comercio, que estaba ligado a la aventura, condujo a la exploración y al descubrimiento. Y todo esto dio nacimiento al transporte. El comercio ha sido el gran civilizador al favorecer el mutuo enriquecimiento cultural.
\usection{5. COMIENZOS DEL CAPITAL}
\vs p069 5:1 El capital es un trabajo empleado como renuncia del presente en favor del futuro. Los ahorros representan una forma de seguro para la manutención y la supervivencia. El acaparamiento de la comida hizo que se desarrollase el autocontrol y creó los primeros problemas del capital y del trabajo. El hombre que tenía comida, siempre que pudiese protegerla de los ladrones, disponía de una ventaja clara sobre el que no la tenía.
\vs p069 5:2 Los primeros banqueros eran los hombres más valiosos de la tribu. Tenían en depósito las riquezas del grupo, de modo que el clan entero defendía su choza en caso de ataque. De esta manera, la acumulación del capital individual y la riqueza colectiva llevaron de inmediato a la organización militar. Al principio, estas medidas de precaución estaban destinadas a defender la propiedad contra los saqueadores extranjeros, pero, más tarde, se convirtió en costumbre mantener esta formación militar bien ejercitada atacando la propiedad y los bienes de las tribus vecinas.
\vs p069 5:3 Hay algunos motivos esenciales que instaron a la acumulación del capital:
\vs p069 5:4 \li{1.}\bibemph{El hambre} ---\bibemph{relacionada con la previsión}---. El ahorro y la conservación de la comida significó poder y bienestar para aquellos que poseían la adecuada \bibemph{previsión} para prepararse así en vista a las necesidades futuras. El almacenamiento de la comida era suficiente seguro contra las hambrunas y las catástrofes. Todo el conjunto de costumbres primitivas estaba en realidad concebido para ayudar al hombre a supeditar el presente al futuro.
\vs p069 5:5 \li{2.}\bibemph{El amor a la familia} ---el deseo de atender sus carencias---. El capital representa el ahorro de bienes pese a la presión de las carencias del presente, con el fin de asegurarse contra las exigencias del futuro. Una parte de esta necesidad futura puede tener que ver con las propias generaciones futuras.
\vs p069 5:6 \li{3.}\bibemph{La vanidad} ---el deseo de mostrar la acumulación de las propias pertenencias---. Tener ropa de sobra fue uno de los distintivos de diferenciación. La ostentación apeló pronto al orgullo del hombre.
\vs p069 5:7 \li{4.}La \bibemph{posición social} ---el afán de ganar prestigio social y político---. Pronto surgió una nobleza comprada, la admisión a la cual dependía del desempeño de algún servicio especial a la realeza o se concedía simplemente a cambio de dinero.
\vs p069 5:8 \li{5.}\bibemph{El poder} ---el afán de ser los amos---. Prestar riquezas se vio como medio de esclavizar al ser la tasa del préstamo en esos tiempos ancestrales de un cien por ciento anual. Los prestamistas se erigían a sí mismos reyes mediante la creación de un ejército permanente de deudores. Los siervos esclavizados eran las primeras formas de bienes que se acumulaban y, en tiempos antiguos, la esclavitud por deudas se amplió incluso hasta el control del cuerpo después de la muerte.
\vs p069 5:9 \li{6.}\bibemph{El temor a los espíritus de los muertos} ---las tasas pagadas a los sacerdotes por protección---. El hombre comenzó pronto a hacer regalos a los sacerdotes con vistas a que se utilizara su patrimonio para facilitar su progreso en la próxima vida. De este modo, los sacerdotes se volvieron muy ricos; eran los más importantes de los capitalistas antiguos.
\vs p069 5:10 \li{7.}\bibemph{El impulso sexual} ---el deseo de comprar una o más esposas---. La primera forma de comercio del hombre fue el intercambio de mujeres, que precedió durante mucho tiempo al de los caballos. Pero el trueque de esclavas del sexo nunca hizo que avanzara la sociedad; esta trata fue, y sigue siendo, una desgracia racial, pues, a la vez, obstaculizó el desarrollo de la vida familiar y contaminó la aptitud biológica de los pueblos mejor dotados.
\vs p069 5:11 \li{8.}\bibemph{Las numerosas formas de autogratificación}. Algunos procuraron riquezas porque les otorgaban poder; otros pugnaron por las propiedades porque significaban bienestar. El hombre primitivo (y algunos otros en días posteriores) tenía tendencia a despilfarrar sus recursos en lujos. Las bebidas embriagantes y las drogas fascinaban a las razas primitivas.
\vs p069 5:12 \pc A medida que la civilización se desarrollaba, el hombre adquiría nuevos alicientes para ahorrar; rápidamente, se agregaron nuevas necesidades a la temprana hambre de comida. Se llegó a detestar tanto la pobreza que se pensaba que solo los ricos iban directamente al cielo cuando morían. Se llegó a valorar de tal manera la propiedad que quien diese un ostentoso banquete podría borrar la deshonra de su reputación.
\vs p069 5:13 Acaudalar riquezas pronto se convirtió en una insignia de distinción social. Los integrantes de algunas tribus acumulaban bienes durante años únicamente para causar impresión al quemarlas en algún día festivo o repartirlas de forma gratuita entre los miembros de su tribu. Esto los convertía en grandes hombres. Incluso los pueblos actuales se deleitan repartiendo generosamente regalos en Navidad, en tanto que los ricos dotan de fondos a grandes instituciones filantrópicas y formativas. Los modos del hombre varían, pero su temperamento permanece bastante inalterado.
\vs p069 5:14 No obstante, es justo indicar que muchos ricos de la antigüedad repartían gran parte de su fortuna por temor a morir a manos de quienes la codiciaban. Las personas acaudaladas habitualmente sacrificaban a docenas de esclavos para demostrar su desdén por las riquezas.
\vs p069 5:15 Aunque el capital ha contribuido a liberar al hombre, también ha complicado enormemente su organización social e industrial. El uso abusivo del capital por capitalistas deshonestos no desdice el hecho de que este sea la base de la sociedad industrial moderna. Por medio del capital y de la invención, la generación de hoy en día disfruta de un grado de libertad superior al que haya existido con anterioridad en la tierra. Esto lo hacemos constar como una realidad y no para justificar los muchos usos indebidos del capital por parte de depositarios desconsiderados y egoístas.
\usection{6. EL FUEGO EN RELACIÓN A LA CIVILIZACIÓN}
\vs p069 6:1 La sociedad primitiva con sus cuatro divisiones ---laboral, de regulación, religiosa y militar--- surgió de la contribución realizada por el fuego, los animales, los esclavos y la propiedad.
\vs p069 6:2 Por sí mismo, el hecho de encender el fuego separó para siempre al hombre del animal; fue un invento o un descubrimiento humano capital. Al temerlo todos los animales, el fuego permitió al hombre permanecer en el suelo por la noche. Al anochecer, alentó el trato social; y no solo protegía del frío y de las fieras, sino que también se empleaba como salvaguarda contra los espíritus. Al principio, se usó más para alumbrar que para calentar; muchas tribus atrasadas se niegan a dormir a menos que haya una llama ardiendo toda la noche.
\vs p069 6:3 El fuego fue un gran civilizador: proporcionó al hombre la primera forma de ser altruista sin pérdida alguna, al permitirle ofrecer a los vecinos brasas sin ninguna merma para él. El fuego hogareño, que la madre o la hija mayor cuidaban, fue el primer elemento educativo, pues requería vigilancia y fiabilidad. El hogar primitivo no era una construcción; la familia se reunía alrededor de la fogata, u hoguera familiar. Cuando un hijo fundaba un nuevo hogar, se llevaba una tea de esta hoguera.
\vs p069 6:4 \pc Aunque Andón, el descubridor del fuego, no lo veía como objeto de adoración, muchos de sus descendientes llegaron a percibir la llama como un fetiche o espíritu. No pudieron extraer los beneficios sanitarios del fuego porque se negaron a quemar los desechos. El hombre primitivo tenía miedo del fuego y siempre procuró mantenerlo bien humorado, de ahí que se esparciera incienso. Los antiguos, bajo ninguna circunstancia, escupían en el fuego, como tampoco pasaban entre una persona y el fuego ardiente. Los primeros seres humanos consideraban sagrados incluso las piritas de hierro y los pedernales usados para encender el fuego.
\vs p069 6:5 Era pecado extinguir una llama; si una choza se incendiaba, se dejaba que ardiese. Los fuegos de los templos y de los santuarios eran sagrados, y nunca se permitía que se apagaran, solo que era costumbre encender nuevas llamas anualmente o después de alguna calamidad. Se seleccionaba a las mujeres como sacerdotisas porque eran ellas las que custodiaban los fuegos caseros.
\vs p069 6:6 Los primeros mitos acerca de cómo el fuego había descendido de los dioses surgieron a partir de la observación de los incendios a causa de los rayos. Estas ideas sobre su origen sobrenatural resultó directamente en la adoración del fuego, y la adoración del fuego condujo a su vez a la costumbre de “pasar por el fuego”, una práctica que se llevó a cabo hasta los tiempos de Moisés. Y todavía persiste la idea de que se pasa por el fuego tras la muerte. El mito del fuego fue un gran nexo de unión en los tiempos primitivos y aún perdura en el simbolismo de los parsis.
\vs p069 6:7 \pc El fuego llevó al proceso de cocinar los alimentos, y el concepto “consumidores de lo crudo” se convirtió en una expresión de sorna. Y el cocinado de los alimentos disminuyó el gasto de la energía vital necesaria para la digestión de la comida e hizo al hombre primitivo contar con algo de fuerzas para la cultura social, al mismo tiempo que la cría de animales, al reducir el esfuerzo necesario para conseguir la comida, le daba tiempo para las actividades sociales.
\vs p069 6:8 Es preciso recordar que el fuego abrió las puertas de la metalistería y condujo al posterior descubrimiento de la energía del vapor y a la utilización actual de la electricidad.
\usection{7. UTILIZACIÓN DE LOS ANIMALES}
\vs p069 7:1 En un principio, todo el reino animal era enemigo del hombre; los seres humanos tuvieron que aprender a protegerse de las fieras. Al comienzo, el hombre se comía a los animales pero, después, aprendió a domesticarlos y a ponerlos a su servicio.
\vs p069 7:2 La domesticación de los animales ocurrió de forma accidental. El hombre salvaje cazaba las manadas de forma muy similar a como los indios norteamericanos cazaban al bisonte; rodeaban a la manada para poder tenerlos bajo control y así poder matarlos según su necesidad de comida. Luego, se construyeron los corrales y se capturaban manadas enteras.
\vs p069 7:3 Era fácil domar a algunos animales, pero, al igual que el elefante, muchos de ellos no se reproducían en cautividad. Algo más tarde, se descubrió que ciertas especies de animales se sometían a la presencia del hombre y se reproducían en cautividad. Se impulsó pues la domesticación de los animales mediante la cría selectiva, un arte que hizo grandes avances desde los días de Dalamatia.
\vs p069 7:4 El perro fue el primer animal en domesticarse, y esta difícil tarea comenzó cuando un cierto perro, tras haber seguido durante todo el día a un cazador, fue de hecho a su casa con él. Por mucho tiempo, se utilizaron a los perros como comida, caza, transporte y compañía. Al principio, los perros solamente aullaban, pero más tarde aprendieron a ladrar. Su agudo sentido del olfato llevó a la idea de que podían ver espíritus, y surgieron de este modo los cultos de los perros fetiche. El uso de perros guardianes hizo posible, por primera vez, que todo el clan pudiera dormir por la noche. Luego se convirtió en costumbre usar a los perros guardianes para proteger el hogar contra los espíritus a la vez que contra los enemigos carnales. Cuando un perro ladraba, se entendía que alguna persona o fiera se estaba acercando, pero, cuando aullaba, eran los espíritus los que estaban cerca. Todavía se cree hoy en día que el aullido de un perro por la noche es augurio de muerte.
\vs p069 7:5 Mientras fue cazador, el hombre era bastante afable con la mujer, pero tras la domesticación de los animales, a la que se sumó la confusión originada por Caligastia, muchas tribus trataban a sus mujeres de forma ignominiosa; las trataban tan mal como a sus animales. El trato brutal que el hombre infligía a la mujer constituye uno de los capítulos más oscuros de la historia de la humanidad.
\usection{8. LA ESCLAVITUD COMO FACTOR INFLUYENTE EN LA CIVILIZACIÓN}
\vs p069 8:1 El hombre primitivo jamás vaciló en esclavizar a sus semejantes. La mujer fue la primera esclava, una esclava familiar. El hombre pastoril esclavizaba a su mujer, a la que consideraba inferior y utilizaba como pareja sexual. Esta clase de esclavitud sexual surgió directamente del hecho del menor grado de dependencia que tenía el hombre hacia la mujer.
\vs p069 8:2 No hace mucho tiempo, la esclavitud era la suerte que corrían los prisioneros de guerra que se negaban a aceptar la religión del conquistador. En otros tiempos, se comía a los prisioneros, se torturaban hasta la muerte, se los hacía luchar entre sí, se sacrificaban a los espíritus o se los esclavizaba. La esclavitud fue un gran avance respecto a la masacre y al canibalismo.
\vs p069 8:3 La esclavitud significó un paso adelante respecto a la clemencia en el trato a los prisioneros de guerra. La emboscada de Hai, con la matanza masiva de hombres, mujeres y niños, de la que solo se salvó el rey para satisfacer la vanidad del vencedor, es una imagen fiel de las salvajes masacres que incluso los pueblos supuestamente civilizados llevaban a cabo. El ataque fulgurante a Og, rey de Basan, fue igual de brutal y efectivo. Los hebreos “destruyeron por completo” a sus enemigos y se apoderaban, como botín, de todos sus bienes. Imponían un tributo a todas las ciudades bajo pena de “matar a todos los varones”. Pero muchas de las tribus coetáneas, de menor egoísmo tribal, habían empezado a practicar, desde hacía mucho tiempo, la adopción de los cautivos mejor dotados.
\vs p069 8:4 Los cazadores, tal como los hombres rojos americanos, no esclavizaban a sus cautivos; o los adoptaban o los mataban. Entre los pueblos pastoriles, la esclavitud no era común, porque precisaban poca mano de obra. En la guerra, los pastores tenían como costumbre matar a todos los cautivos varones y tomar como esclavos solo a las mujeres y a los niños. El código mosaico contenía instrucciones específicas para convertir a estas mujeres en esposas. Se podían apartar a aquellas que no resultasen satisfactorias, pero no se permitía a los hebreos vender como esclavas a estas consortes rechazadas ---algo que representaba un avance para la civilización---. Aunque las normas sociales de los hebreos eran rudimentarias, sí estaban muy por encima de las tribus de los alrededores.
\vs p069 8:5 Los pastores fueron los primeros capitalistas; sus rebaños representaban su capital y vivían de los intereses ---del natural incremento de estos---. Y eran reacios a confiar esta riqueza al cuidado de los esclavos o a de las mujeres. Pero más tarde, hicieron prisioneros a varones y los forzaron a cultivar el suelo. Este es el temprano origen de la servidumbre ---el hombre apegado a la tierra---. Se podía enseñar fácilmente a los africanos a labrar la tierra; de ahí que se convirtieran en una relevante raza de esclavos.
\vs p069 8:6 \pc La esclavitud significó un eslabón imprescindible en la cadena de la civilización humana. Fue el puente por el que la sociedad pasó del caos y la indolencia al orden y a la actividad civilizada; obligó a los pueblos atrasados y perezosos a trabajar y a proporcionar con ello la riqueza y el ocio necesarios para el avance social de aquellos mejor dotados.
\vs p069 8:7 La institución de la esclavitud forzó al hombre a inventar el mecanismo regulador de la sociedad primitiva; dio origen a los comienzos del gobierno. La esclavitud exige estricta regulación y, durante la Edad Media en Europa, desapareció prácticamente porque los señores feudales no podían controlar a los esclavos. Las tribus atrasadas de tiempos ancestrales, al igual que los aborígenes australianos de hoy en día, nunca tuvieron esclavos.
\vs p069 8:8 Cierto, la esclavitud fue opresiva, pero fue en un entorno de opresión donde el hombre aprendió a ser laborioso. Los esclavos acabaron por compartir las ventajas de una sociedad superior, que con tan poca voluntariedad habían ayudado a crear. La esclavitud da origen a un sistema del logro social y cultural, pero pronto ataca de forma insidiosa a la sociedad internamente como el más grave y destructivo de todos los males sociales.
\vs p069 8:9 \pc Los inventos mecánicos modernos hicieron que los esclavos dejaran de tener utilidad. La esclavitud, como la poligamia, está desapareciendo porque no es rentable. Pero siempre resultó pernicioso liberar de repente a grandes cantidades de esclavos; se producen menos problemas cuando se les da la libertad de forma gradual.
\vs p069 8:10 \pc Hoy día, los hombres no son esclavos sociales, pero hay miles que permiten que la ambición los esclavice a las deudas. La esclavitud involuntaria ha dado paso a la servidumbre laboral, a una nueva y enmendada forma de esclavitud.
\vs p069 8:11 Aunque el ideal de la sociedad es la libertad universal, nunca se ha de tolerar la ociosidad. Se debería obligar a toda persona capacitada físicamente a realizar al menos alguna cantidad de trabajo para poder sustentarse a sí misma.
\vs p069 8:12 La sociedad moderna está dando marcha atrás. La esclavitud casi ha desaparecido; el empleo de los animales domésticos se está perdiendo. La civilización está volviendo al fuego ---el mundo inorgánico--- en busca de energía. El hombre salió de la barbarie mediante el fuego, los animales y la esclavitud; hoy se remonta al pasado, descartando la ayuda de los esclavos y la asistencia de los animales, mientras procura arrebatar nuevos secretos y fuentes de riqueza y energía a los recursos naturales básicos.
\usection{9. LA PROPIEDAD PRIVADA}
\vs p069 9:1 Aunque la sociedad primitiva era prácticamente comunal, el hombre primitivo no seguía los principios del comunismo moderno. El comunismo de esos tempranos tiempos no era una mera teoría o doctrina social, sino una sencilla y práctica adaptación que ocurría de forma natural. El comunismo impidió el pauperismo y las privaciones; la mendicidad y la prostitución eran algo casi desconocido entre estas tribus ancestrales.
\vs p069 9:2 \pc El comunismo primitivo no hizo particularmente iguales a los hombres ni enalteció al hombre ordinario, pero sí primó la inactividad y la pereza, puso freno a la laboriosidad y destruyó las aspiraciones. La promoción del comunismo fue el indispensable andamiaje para el desarrollo de la sociedad primitiva, aunque dio paso a la evolución de un orden social superior, porque iba en contra de cuatro intensas inclinaciones humanas:
\vs p069 9:3 \li{1.}\bibemph{La familia}. El hombre no solo anhela acumular patrimonio, sino que desea legar a su progenie los bienes generados. Pero, en la primera sociedad comunal, el capital de un hombre se usaba de inmediato o se repartía en el grupo a su muerte. No existía la herencia de la propiedad ---el impuesto sucesorio era del cien por cien---. Las costumbres que llegaron después, respecto a la acumulación de capital y a la herencia de la propiedad, fueron un manifiesto avance social. Y esto es cierto a pesar de los graves abusos que se dieron más adelante por el uso indebido del capital.
\vs p069 9:4 \li{2.}\bibemph{Las tendencias religiosas}. El hombre primitivo también quería conservar su patrimonio como premisa para empezar la vida en la próxima existencia. Esto explica por qué persistió durante tanto tiempo la costumbre de enterrar al difunto con sus pertenencias personales. Los antiguos creían que solo los ricos sobrevivían a la muerte con algún tipo de disfrute y dignidad inmediatos. Los maestros de la religión revelada, más específicamente los maestros cristianos, fueron los primeros en dar a conocer que los pobres pueden salvarse en igualdad de condiciones que los ricos.
\vs p069 9:5 \li{3.}\bibemph{El deseo de libertad y de ocio}. En los primeros días de la evolución social, la repartición en el grupo de los ingresos individuales era prácticamente una forma de esclavitud; el trabajador se convirtió en esclavo del holgazán. En ello radicaba la flaqueza autodestructiva del comunismo: el impróvido vivía habitualmente del ahorrador. Incluso en tiempos modernos, la persona imprevisora depende del estado (de los contribuyentes ahorradores) para que cuide de ellos. Los que no poseen capital todavía esperan su sustento de los que sí lo tienen.
\vs p069 9:6 \li{4.}\bibemph{El ansia de seguridad y poder}. El comunismo acabó por desaparecer por las astutas prácticas de personas avanzadas y prósperas que recurrieron a distintos subterfugios para tratar de escapar de la esclavitud de los ociosos y holgazanes de sus tribus. Pero, al principio, cualquier acaparamiento de bienes era clandestino; la inseguridad de esos tiempos primitivos evitaba la acumulación ostentosa de capital. E incluso más tarde resultaba bastante peligroso amasar un exceso de riqueza; el rey se aseguraría de inventar alguna acusación para confiscar los bienes del rico, y, cuando algún hombre acaudalado fallecía, se retrasaba el funeral hasta que la familia donase una cuantiosa suma para la asistencia pública o al rey, esto es, un impuesto sucesorio.
\vs p069 9:7 En tiempos remotos, las mujeres eran propiedad de la comunidad, y la madre tenía el papel dominante en la familia. Los primeros jefes eran los dueños de todas las tierras y los propietarios de todas las mujeres; contraer matrimonio requería el consentimiento del dirigente de la tribu. Con la desaparición del comunismo, las mujeres se convirtieron en posesión individual, y el padre, de forma gradual, fue asumiendo la dirección de la familia. Así fue como comenzó el hogar, y las costumbres polígamas imperantes se sustituyeron paulatinamente por la monogamia. (La poligamia es la supervivencia del componente de esclavitud femenina que existía en el matrimonio. La monogamia es el ideal, sin esclavitud, de la inigualable unión de un hombre y una mujer en la hermosa aventura de formar un hogar, criar a los hijos, culturizarse mutuamente y mejorar personalmente.)
\vs p069 9:8 En un principio, cualquier patrimonio, incluidas las herramientas y las armas, era posesión común de la tribu. Al comienzo, la propiedad privada consistía en todas las cosas que se tocasen personalmente. Si un extraño bebía en una taza, esta, a partir de ese momento, era suya. Luego, cualquier lugar en el que se derramase sangre se convertía en la propiedad de la persona o del grupo herido.
\vs p069 9:9 Inicialmente se respetaba así la propiedad privada, porque se pensaba que estaba impregnada de alguna parte de la persona de su propietario. La honradez respecto a la propiedad se mantuvo a salvo gracias a este tipo de superstición; no se necesitaba ningún agente de la autoridad que protegiera las pertenencias personales. No existía el robo en el grupo, aunque los hombres no dudaban en apropiarse de los bienes de otras tribus. El vínculo con las propiedades no acababa con la muerte; al principio, se quemaban todos los objetos personales, luego se enterraban con el difunto y, más tarde, los heredaba la familia superviviente o la tribu.
\vs p069 9:10 Las objetos personales de tipo ornamental se originaron con el uso de los amuletos. La vanidad, sumado al miedo a los espíritus, llevó al hombre primitivo a resistirse a cualquier intento por liberarle de sus amuletos preferidos; se valoraban estas posesiones por encima de las cosas necesarias.
\vs p069 9:11 \pc El espacio para dormir fue una de las tempranas posesiones del hombre. Más tarde, los jefes de las tribus asignaban los lugares de emplazamientos; ellos eran los depositarios de todos los bienes inmuebles del grupo. Al poco tiempo, el lugar donde se situaba la hoguera confería propiedad; y, más adelante, un pozo otorgaba la titularidad del terreno contiguo.
\vs p069 9:12 Las charcas y los pozos se contaban entre las primeras posesiones privadas. Se utilizó todo tipo de prácticas fetichistas para custodiar charcas, pozos, árboles, cosechas y miel. Al perderse la fe en los fetiches, las leyes evolucionaron para proteger las pertenencias privadas. Pero las leyes de la caza, el derecho a cazar, precedieron con mucha anterioridad a las leyes del suelo. El hombre rojo americano nunca entendió la idea de la propiedad privada del suelo; jamás pudo comprender el punto de vista del hombre blanco.
\vs p069 9:13 Pronto se marcó la propiedad privada con la insignia de la familia, y este es el origen primitivo de los blasones familiares. Los bienes inmuebles también se ponían bajo la vigilancia y el cuidado de los espíritus. Los sacerdotes “consagraban” parcelas de suelo, que luego quedaban bajo la protección de los tabúes mágicos que se erigirían sobre estas. Se decía que los propietarios de los mismos tenían “titularidad sacerdotal”. Los hebreos tenían un gran respeto por estos hitos familiares: “Maldito el que desplace el límite de su vecino”. Estas señales de piedra portaban las iniciales del sacerdote. Incluso los árboles, cuando llevaban iniciales, se convertían en propiedad privada.
\vs p069 9:14 En los tiempos primitivos, solo las cosechas eran privadas, pero la sucesión de cultivos confería la titularidad de las tierras; la agricultura fue pues el origen de la propiedad privada del suelo. En un principio, solo se concedía a las personas una tenencia vitalicia; al morir estas, las tierras retornaban a la tribu. Las primeras titularidades sobre el suelo que las tribus concedieron fueron las tumbas ---el cementerio familiar---. En tiempos posteriores, el suelo pertenecía a quienes lo cercara. Pero, en las ciudades, siempre se reservaban algunas tierras para pastizales públicos y casos de asedio; estos “bienes comunes” representan la supervivencia de la forma anterior de propiedad colectiva.
\vs p069 9:15 Con el tiempo, el Estado llegó a asignar la propiedad a las personas de manera individual, reservándose el derecho de tributación. Al haber conseguido la titularidad sobre los terrenos, los terratenientes podían recaudar alquileres, y el suelo se convirtió en fuente de ingresos ---en capital---. Por último, el suelo se volvió en algo verdaderamente negociable, por lo que fue objeto de venta, traspasos, hipotecas y embargos hipotecarios.
\vs p069 9:16 La propiedad privada trajo consigo mayor libertad y estabilidad; si bien, la propiedad privada del suelo recibió aprobación social solamente tras el fracaso del control y la dirección comunal, a lo que pronto siguió una sucesión de esclavos, siervos y clases sin tierras. No obstante, el perfeccionamiento de las máquinas está paulatinamente liberando al hombre del penoso trabajo servil.
\vs p069 9:17 El derecho a la propiedad no es absoluto; es puramente social. Pero cualquier gobierno, ley, orden, derecho civil, libertad social, convención, paz y felicidad, disfrute de los pueblos modernos, se ha desarrollado en torno a la posesión privada de propiedades.
\vs p069 9:18 El actual orden social no es necesariamente justo ---ni divino ni sagrado--- pero la humanidad haría bien en avanzar despacio al realizar cambios. El sistema social del que disponéis es inmensamente mejor que cualquiera de los que vuestros ancestros conocieron. Cercioraos de que cuando transforméis el orden social lo hagáis para mejor. No os sintáis persuadidos a experimentar con métodos que vuestros antecesores desestimaron. ¡Marchad adelante y no para atrás! ¡Que la evolución prosiga su curso! ¡No retrocedáis ni un solo paso!
\vsetoff
\vs p069 9:19 [Exposición de un melquisedec de Nebadón.]
