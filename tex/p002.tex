\upaper{2}{La naturaleza de Dios}
\author{Consejero divino}
\vs p002 0:1 Puesto que el concepto más elevado de Dios que el hombre pueda tener reside en la idea y el ideal humanos de un ser personal primordial e infinito, es admisible, y puede resultar de utilidad, examinar ciertos rasgos de la naturaleza divina constitutivos del carácter de la Deidad. La naturaleza de Dios se puede entender mejor mediante la revelación del Padre llevada a cabo por Miguel de Nebadón en sus múltiples enseñanzas y en su magnífica vida en carne mortal. El hombre puede también entender mejor la naturaleza divina si se considera a sí mismo como un hijo de Dios que ve al Padre del Paraíso como a su verdadero Padre espiritual.
\vs p002 0:2 Se puede examinar la naturaleza de Dios en la revelación de unas ideas supremas, se puede concebir el carácter divino como la representación de unos ideales celestiales, pero de todas las revelaciones de la naturaleza divina, la de mayor lucidez y edificación espiritual hay que encontrarla comprendiendo la vida religiosa de Jesús de Nazaret, tanto antes como después de que tomara plena conciencia de su divinidad. Si se toma la vida encarnada de Miguel como punto de partida para la revelación de Dios al hombre, podemos intentar expresar, mediante signos verbales humanos, algunas ideas e ideales concernientes a la naturaleza divina que tal vez aporten más claridad y unidad al concepto humano de la naturaleza y del carácter de la persona del Padre Universal.
\vs p002 0:3 A pesar de nuestro empeño por ampliar y espiritualizar el concepto humano de Dios, nos vemos tremendamente coartados por las facultades limitadas de la mente mortal. Pese a nuestro esfuerzo por describir valores divinos y exponer contenidos espirituales conforme a la mente finita y material del hombre, nos vemos de igual manera seriamente condicionados, en el cumplimiento de nuestro cometido, por las limitaciones del lenguaje y por la escasez de material en qué basar nuestras ilustraciones o comparaciones. Todo nuestro afán por ampliar el concepto humano de Dios sería prácticamente vano si no fuera por el hecho de que la mente humana está habitada por el modelador que el Padre Universal otorga e infundida del espíritu de la verdad del hijo creador. Contando, por consiguiente, con la presencia de estos espíritus divinos en el corazón del hombre para que lo asistan en la ampliación del concepto de Dios, emprendo, con alegría, el cumplimiento de mi mandato, intentando realizar una descripción ampliada de la naturaleza de Dios de acuerdo con la mente del hombre.
\usection{1. LA INFINITUD DE DIOS}
\vs p002 1:1 “Él es Infinito, el cual no alcanzamos. Las pisadas divinas no son halladas”. “Su entendimiento es infinito y su grandeza es insondable”. Es tal la luz cegadora de la presencia del Padre que para sus modestas criaturas parece que “habita en la espesa oscuridad”. No solo son sus pensamientos y planes inescrutables, sino que “hace cosas grandes y maravillas sin número”. “Dios es grande, y nosotros no lo comprendemos, ni se puede seguir el curso de sus años”. “¿Es verdad que Dios habitará sobre la tierra? He aquí que el cielo (el universo) y el cielo de los cielos (el universo de los universos) no lo pueden contener”. “¡Cuán insondables son sus juicios e inescrutables sus caminos!”.
\vs p002 1:2 “No hay sino un solo Dios, el Padre Infinito, que es además un fiel Creador”. “El Creador Divino es asimismo el Concertador Universal, la fuente y destino de las almas. Es el Alma Suprema, la Mente Primordial y el Espíritu Ilimitado de toda la creación”. “El Gran Rector no comete errores. Él resplandece en majestad y gloria”. “El Dios Creador está del todo libre de temor y enemistad. Él es inmortal, eterno, autoexistente, divino y munificente”. “¡Cuán puro y bello, cuán profundo e impenetrable es el Predecesor celestial de todas las cosas!”. “El Infinito es más excelente por el hecho de darse a los hombres. Él es el principio y el fin, el Padre de todo propósito bueno y perfecto”. “Con Dios todas las cosas son posibles. El Creador Eterno es la causa de las causas”.
\vs p002 1:3 \pc A pesar de las infinitas y asombrosas manifestaciones de la persona eterna y universal del Padre, él es incondicionalmente consciente tanto de su infinitud como de su eternidad; del mismo modo, conoce plenamente su perfección y su poder. Él es el único ser del universo, aparte de sus divinos iguales en rango, capaz de tener una apreciación de sí mismo de forma perfecta, adecuada y completa.
\vs p002 1:4 El Padre acude, constante e indefectiblemente, a la solicitud diferenciada que se hace de él, según esta cambia periódicamente en las varias secciones de su universo matriz. El gran Dios se conoce y entiende a sí mismo; es infinitamente consciente de todos sus atributos primordiales de perfección. Dios no es un accidente cósmico ni tampoco un experimentador del universo. Quizás los soberanos del universo emprendan alguna aventura, quizás los padres de las constelaciones experimenten, quizás los dirigentes de los sistemas practiquen, pero el Padre Universal ve el fin desde el principio, y su plan divino y su propósito eterno en realidad abarcan y comprenden todos los experimentos y aventuras de todos sus subordinados en cualquier mundo, sistema y constelación, en cualquiera de los universos de sus inmensos dominios.
\vs p002 1:5 Nada es nuevo para Dios, y ningún acontecimiento cósmico le causa extrañeza; él habita el círculo de la eternidad. Sus días no tienen principio ni fin. Para Dios no hay pasado, presente o futuro; todo tiempo, en cualquier momento dado, es presente. Él es el grande y el único YO SOY.
\vs p002 1:6 \pc El Padre Universal es, absolutamente y sin condición alguna, infinito en todos sus atributos; y este hecho, en sí mismo y por sí mismo, le impide ineludiblemente cualquier comunicación personal y directa con seres materiales y finitos, así como con otras modestas inteligencias creadas.
\vs p002 1:7 Y todo esto necesita de las medidas que se han tomado para entablar contacto y comunicación con sus múltiples criaturas, primero, en las personas de los Hijos de Dios del Paraíso, que, aunque perfectos en divinidad, con frecuencia participan además de la misma naturaleza de carne y huesos de las razas planetarias, haciéndose uno de vosotros y uno con vosotros; de este modo, por así decirlo, Dios se hace hombre, como aconteció durante el ministerio de gracia de Miguel, a quien se llamó igualmente Hijo de Dios e Hijo del Hombre. Segundo, están los seres personales del Espíritu Infinito, los varios órdenes de multitudes de serafines y de otras inteligencias celestiales que se acercan a los seres materiales de origen modesto y que de tantos modos los asisten y sirven. Y, tercero, están los impersonales mentores misteriosos, los modeladores del pensamiento, el don verdadero del mismo gran Dios, enviados para morar en seres humanos como los de Urantia, sin anuncio ni explicación. Descienden, en un sinfín de abundancia, desde las gloriosas alturas, para habitar en las modestas mentes de esos mortales que poseen la capacidad para tener conciencia de Dios, o el potencial para ello, y conferirles dignidad.
\vs p002 1:8 De esta manera y de muchas otras, de forma desconocida para vosotros y que escapan completamente a la comprensión finita, el Padre del Paraíso, amorosamente y por voluntad propia, disminuye, esto es, modifica, diluye y atenúa su infinitud para poder acercarse a las mentes finitas de sus hijos creaturales. Así pues, el Padre Infinito, al distribuir su persona en una sucesión cada vez menos absoluta, es capaz de gozar de un estrecho contacto con las distintas inteligencias de los numerosos mundos de su extenso universo.
\vs p002 1:9 Todo esto lo ha hecho y lo hace en la actualidad, y continuará haciéndolo por siempre, sin restar en lo más mínimo el hecho y la realidad de su infinitud, de su eternidad y de su primacía. Y todas estas cosas son absolutamente verdad, a pesar de ser difíciles de comprender, de estar rodeadas de misterio o de la imposibilidad de que criaturas como las que habitan en Urantia las puedan entender del todo.
\vs p002 1:10 \pc Al ser el Padre Primero infinito en sus planes y eterno en sus propósitos, es intrínsecamente imposible que los seres finitos puedan alguna vez captar o comprender estos planes divinos en su plenitud. Solamente de vez en cuando, aquí y allá, puede el hombre mortal vislumbrar los propósitos del Padre, conforme se revelan en relación al desarrollo del plan de ascensión de las criaturas en el universo, en sus continuados niveles de progreso. Aunque el hombre no pueda abarcar con su entendimiento lo que significa la infinitud, el Padre infinito, con toda certeza, sí comprende plenamente y acoge amorosamente toda la finitud de todos sus hijos en todos los universos.
\vs p002 1:11 El Padre comparte la divinidad y la eternidad con un gran número de seres superiores del Paraíso, pero nos preguntamos si la infinitud y la consiguiente primacía universal es plenamente compartida por quienes no sean sus colaboradores coiguales de la Trinidad del Paraíso. La infinitud del ser personal ha de incluir, forzosamente, la finitud existencial del ser personal; de aquí la verdad ---la verdad literal--- de la enseñanza que proclama que “en él vivimos y nos movemos y somos”. Esa fracción de la Deidad pura del Padre Universal que mora en el hombre mortal \bibemph{es} parte de la infinitud de la Primera Gran Fuente y Centro, el Padre de los Padres.
\usection{2. LA PERFECCIÓN ETERNA DEL PADRE}
\vs p002 2:1 Incluso vuestros antiguos profetas entendieron la naturaleza eterna, sin principio ni fin, la naturaleza circular del Padre Universal. Dios está real y eternamente presente en su universo de los universos. Él habita el momento presente con toda su absoluta majestad y eterna grandeza. “El Padre tiene vida en sí mismo, y esta vida es vida eterna”. Desde la eternidad de los tiempos, es el Padre quien “da vida a todo”. Hay infinita perfección en la integridad divina. “Yo soy el Señor; yo no cambio”. Nuestro conocimiento del universo de los universos no solo nos desvela que él es el Padre de las luces, sino que también en su forma de dirigir las cuestiones interplanetarias, “no hay mudanza ni sombra de variación”. Él “anuncia lo por venir desde el principio”. Él dice: “Mi consejo permanecerá, y haré todo lo que quiero” “conforme al propósito eterno que hice en mi Hijo”. Así pues los planes y los propósitos de la Primera Fuente y Centro son como él mismo: eternos, perfectos y por siempre invariables.
\vs p002 2:2 En los mandatos del Padre hay plena completud y perfecta repleción. “Todo lo que Dios hace será perpetuo; sobre aquello no se añadirá; ni de ello se disminuirá”. El Padre Universal no se arrepiente de sus propósitos primigenios de sabiduría y perfección. Sus planes son firmes, su consejo inmutable y sus acciones, al mismo tiempo, divinas e infalibles. “Mil años delante de sus ojos son como el día de ayer, que pasó, y como una de las vigilias de la noche”. La perfección de la divinidad y la magnitud de la eternidad estarán, para siempre, fuera de la total cognición de la circunscrita mente del hombre mortal.
\vs p002 2:3 \pc Según la actitud cambiante y la mente variable de los seres inteligentes de su creación, el modo de proceder del Dios inmutable, en el cumplimiento de su propósito eterno, quizás parezca variar; esto es, quizás varíe aparente y superficialmente, pero el propósito invariable, el plan perpetuo del Dios Eterno continúa estando vigente bajo la superficie y tras cualquier forma de manifestación externa.
\vs p002 2:4 Fuera, en los universos, la perfección necesariamente ha de ser un término relativo, pero en el universo central y, en especial, en el Paraíso, la perfección es pura e incluso absoluta en determinadas facetas. Al manifestarse la Trinidad, la forma de expresión de la perfección divina cambia pero no se atenúa.
\vs p002 2:5 \pc La perfección primordial de Dios no radica en su supuesta rectitud, sino en la inherente perfección de la bondad de su naturaleza divina. Él es final, completo y perfecto. Nada falta en la belleza y perfección de su recto carácter. Y todo el proyecto trazado para las existencias vivas de los mundos del espacio está basado en el propósito divino de elevar a todas las criaturas de voluntad hacia su elevado destino, hacia la experiencia de compartir la perfección del Padre del Paraíso. Dios no es egocéntrico ni autosuficiente; nunca cesa de darse de gracia a las criaturas conscientes de sí mismas del inmenso universo de los universos.
\vs p002 2:6 Al ser perfecto de una forma eterna e infinita, Dios no puede tener personalmente la experiencia de la imperfección; no obstante, sí comparte, con todos los hijos creadores del Paraíso, la conciencia que poseen de la experiencia de imperfección de todas las tenaces criaturas de los universos evolutivos. Este toque personal y liberador del Dios de perfección da cobijo a los corazones y encauza hacía sí la naturaleza de todas las criaturas mortales que han ascendido en el universo hasta el nivel del discernimiento moral. De esta manera, al igual que mediante la proximidad de la presencia divina, el Padre Universal en realidad participa de la experiencia \bibemph{con} la inmadurez e imperfección en la andadura evolutiva de cualquier ser moral del universo completo.
\vs p002 2:7 Las limitaciones humanas, el mal en potencia, no forman parte de la naturaleza divina, pero la experiencia de los mortales \bibemph{con} el mal y todas las relaciones del hombre al respecto sí forman ciertamente parte de la propia realización en constante expansión de Dios en los hijos del tiempo ---criaturas moralmente responsables creadas o evolucionadas por cualquier hijo creador que sale del Paraíso---.
\usection{3. JUSTICIA Y RECTITUD}
\vs p002 3:1 Dios es recto y, por lo tanto, justo. “Recto es el Señor en todos sus caminos”. “'No sin causa hice todo lo que he hecho', dice el Señor'”. “Los juicios del Señor son verdad, todos rectos”. Los actos y comportamientos de sus criaturas no pueden influir en la justicia del Padre Universal, “porque con el Señor nuestro Dios no hay iniquidad, ni acepción de personas, ni admisión de cohecho”.
\vs p002 3:2 \pc ¡Resulta tan inútil y pueril pedir a un Dios así que altere sus inmutables decretos para eludir las justas consecuencias de la aplicación de sus sabias leyes naturales y de sus rectos mandatos espirituales! “No os engañéis; Dios no puede ser burlado: pues todo lo que el hombre sembrare, eso también segará”. Pero es verdad que, incluso aunque sea justo cosechar los frutos de la maleficencia, la justicia divina siempre se templa con la misericordia. Es la sabiduría infinita la que se erige como árbitro eterno y la que determina qué proporción de justicia y misericordia ha de aplicarse en cada caso. El mayor castigo (en realidad una consecuencia inevitable) a la maleficencia y a la rebelión deliberada contra el gobierno de Dios consiste en dejar de existir como súbdito individual de ese gobierno. El resultado final del pecado intencionado es ser reducido a la nada. En última instancia, los seres que de tal manera se han identificado con el pecado se destruyen a sí mismos ya que, al abrazar la iniquidad, se vuelven totalmente irreales. Sin embargo, siempre se posterga la desaparición efectiva de tal criatura hasta que se hayan cumplido totalmente las condiciones exigidas por la justicia vigente en ese universo.
\vs p002 3:3 Se suele decretar el cese de la existencia en el juicio dispensacional o de época del mundo o de los mundos. En mundos como los de Urantia, esto sucede al final de una dispensación planetaria. Es posible decretar el cese de la existencia en momentos así mediante la corroboración de todos los tribunales de jurisdicción, que se extiende en dirección ascendente desde el consejo planetario, pasando por los tribunales del hijo creador, hasta los órganos judiciales de los ancianos de días. El mandato de disolución se origina en los tribunales superiores del suprauniverso, tras ratificarse la acusación que se origina en la esfera, lugar donde reside el infractor; y, entonces, una vez se ha confirmado en lo alto la sentencia de extinción, esta se ejecuta mediante la actuación directa de aquellos jueces que residen y operan en las sedes centrales del suprauniverso.
\vs p002 3:4 Cuando dicha sentencia finalmente se confirma, en un instante el ser que se identificó con el pecado será como si nunca hubiera sido. No hay resurrección posible de este sino; es perpetuo y eterno. Los componentes de la identidad de la energía viva se descomponen, mediante la transformación del tiempo y la metamorfosis del espacio, en los potenciales cósmicos de los que una vez emergieron. Por otro lado, al ser personal de la criatura en iniquidad se le priva de su continuo soporte vital, por haber elegido no tomar esas decisiones finales que le hubiesen asegurado la vida eterna. Y cuando la mente del ser personal abraza el pecado de forma continuada y acaba por identificarse completamente con la iniquidad, en la terminación de la vida, en la disolución cósmica, un ser así aislado se absorbe en la sobrealma de la creación y se convierte en parte de la experiencia evolutiva del Ser Supremo. Nunca aparece de nuevo como ser personal; su identidad desaparece como si jamás hubiera existido. Si es un ser personal en el que mora un modelador, los valores de las experiencias espirituales sobreviven en la realidad del modelador que continúa existiendo.
\vs p002 3:5 \pc Cuando se plantea en el universo un conflicto entre niveles efectivos de la realidad, el ser personal de nivel superior terminará por triunfar sobre el ser personal de nivel inferior. Este resultado inevitable de tal confrontación en el universo es consustancial al hecho de que la divinidad como atributo es igual al grado de realidad o actualidad de cualquier criatura volitiva. El puro mal, el extravío total, el pecado intencionado y la iniquidad implacable son inherentes y automáticamente suicidas. Actitudes así, cósmicamente irreales, perduran en el universo debido únicamente a la transitoria misericordia\hyp{}tolerancia que pende de la puesta en funcionamiento, por parte de los rectos tribunales del universo, de mecanismos que resuelvan con justicia y fallen con ecuanimidad.
\vs p002 3:6 En los universos locales, la soberanía de los hijos creadores consiste en la creación y la espiritualización. Estos hijos se dedican eficazmente al cumplimiento del plan del Paraíso en cuanto a la ascensión progresiva de los mortales, a la rehabilitación de aquellos en rebeldía y de pensamiento errado, pero cuando se rechazan finalmente y para siempre estos actos de amor, las fuerzas que actúan bajo la jurisdicción de los ancianos de días cumplen el decreto final de disolución.
\usection{4. LA MISERICORDIA DIVINA}
\vs p002 4:1 La misericordia es sencillamente la justicia atenuada por esa sabiduría que nace de la perfección del conocimiento y del pleno reconocimiento de la debilidad natural y de los impedimentos ambientales de las criaturas finitas. “Nuestro Dios es compasivo, clemente, lento para la ira y grande en misericordia”. Por tanto “todo aquel que invocare al Señor, será salvo”, “el cual será amplio en perdonar”. “La misericordia del Señor dura desde la eternidad hasta la eternidad”; sí, “es eterna su misericordia”. “Yo soy el Señor, que ejerzo misericordia, juicio y rectitud en la tierra, porque en estas cosas me complazco”. “No aflijo voluntariamente ni entristezco a los hijos de los hombres”, porque yo soy “Padre de misericordias y Dios de toda consolación”.
\vs p002 4:2 Dios es bondadoso en esencia, compasivo por naturaleza y misericordioso en perpetuidad. Y nunca es necesario influir en el Padre para suscitar su benevolencia. La necesidad de la criatura es en sí misma totalmente suficiente para asegurar el flujo pleno de su tierna misericordia y de su gracia salvadora. Puesto que Dios conoce todo acerca de sus hijos, le resulta fácil perdonar. Cuanto mejor entienda el hombre a su prójimo, tanto más fácil le resultará perdonarlo e incluso amarlo.
\vs p002 4:3 \pc Solamente la percepción de su sabiduría infinita facilita al Dios de rectitud proveer, al mismo tiempo y en cualquier situación que se presente en el universo, justicia y misericordia. El Padre celestial no se debate en actitudes opuestas hacia sus hijos del universo; Dios nunca experimenta antagonismos en su actitud. Dios, con su omnisciencia, pone en acción su libre voluntad, y elige de modo indefectible la manera de proceder en el universo que satisfaga de forma perfecta, simultánea y por igual las exigencias de todos sus atributos divinos y de las cualidades infinitas de su naturaleza eterna.
\vs p002 4:4 La misericordia es el resultado natural e inevitable de la bondad y del amor. Por su naturaleza bondadosa, el Padre amoroso jamás negaría el sabio ministerio de su misericordia a miembro alguno de cualquier grupo de sus hijos del universo. Juntas la justicia eterna y la misericordia divina constituyen lo que se denominaría \bibemph{ecuanimidad} en la experiencia humana.
\vs p002 4:5 La misericordia divina representa de por sí un modo ecuánime de realizar el ajuste entre los niveles universales de perfección e imperfección. La misericordia es la justicia de la Supremacía adaptada a las situaciones de lo finito en evolución; es la rectitud de la eternidad modificada para atender a los hijos del tiempo en sus más sublimes intereses y en su bienestar en el universo. La misericordia no contraviene a la justicia, sino que es más bien una interpretación benévola de las exigencias de la justicia suprema al aplicarse con equidad a los seres espirituales de menor rango y a las criaturas materiales de los universos evolutivos. La misericordia es la justicia que parte de la Trinidad del Paraíso en visitación, sabia y amorosa, a las múltiples inteligencias de las creaciones del tiempo y del espacio, y se origina en la sabiduría divina, y se determina en la mente omnisapiente y en la voluntad libre y soberana del Padre Universal y de los creadores a él vinculados.
\usection{5. EL AMOR DE DIOS}
\vs p002 5:1 “Dios es amor”; por tanto, su actitud, única y personal, hacia los asuntos del universo surge siempre como reflejo de su divino afecto. El Padre nos ama lo suficiente como para darnos su vida. “Él hace salir su sol sobre malos y buenos, y hace llover sobre justos e injustos”.
\vs p002 5:2 \pc Es erróneo pensar que el sacrificio de sus Hijos o la intercesión de criaturas suyas de menor rango puedan convencer a Dios para que ame a sus hijos, “pues el Padre mismo os ama”. Es en respuesta a este afecto paternal que Dios envía a los maravillosos modeladores para que moren en las mentes de los hombres. El amor de Dios es universal; “si alguno quiere venir que venga”. Él quiere “que todos los hombres sean salvos y vengan al conocimiento de la verdad”. Él no quiere “que ninguno perezca”.
\vs p002 5:3 Los mismos creadores son los primeros que tratan de salvar al hombre de las nefastas consecuencias de su insensata transgresión de las leyes divinas. Por naturaleza, el amor de Dios es de índole paternal; por tanto, a veces “nos disciplina para lo que nos es provechoso, para que participemos de su santidad”. Incluso durante vuestras pruebas más difíciles, recordad que “en toda nuestra angustia él es angustiado con nosotros”.
\vs p002 5:4 Dios es divinamente bondadoso con los pecadores. Cuando los rebeldes vuelven a la rectitud, se les recibe con misericordia, “pues nuestro Dios es amplio en perdonar”. “Yo soy el que borro tus rebeliones por amor de mí mismo, y no me acordaré de tus pecados”. “Mirad cuál amor nos ha dado el Padre, para que seamos llamados hijos de Dios”.
\vs p002 5:5 Al fin y al cabo, la más grande evidencia de la bondad de Dios y la suprema razón para amarlo lo constituye el don del Padre que mora en vosotros: el modelador que con tanta paciencia aguarda la hora en que los dos os hagáis uno para la eternidad. No encontraréis a Dios aunque lo busquéis, pero si os dejáis guiar por el espíritu interior, os sentiréis infaliblemente llevado paso tras paso, vida tras vida, universo tras universo y era tras era, hasta llegar finalmente a encontraros en la presencia de la persona del Padre Universal del Paraíso.
\vs p002 5:6 \pc Qué poco razonable es que no adoréis a Dios porque las limitaciones de la naturaleza humana y los impedimentos de vuestro origen material os imposibiliten verlo. Hay una distancia tremenda (un espacio físico) por cubrir entre Dios y vosotros. Existe una gran abismo de diferencia espiritual por salvar; mas a pesar de todo lo que os separa, ya sea físico o espiritual, de la presencia personal de Dios en el Paraíso, reflexionad por un momento en el hecho solemne de que Dios vive en vosotros; él, a su manera, ya ha salvado ese abismo. Él ha enviado algo de sí mismo, su espíritu, para que viva en vosotros y se esfuerce con vosotros cuando emprendéis vuestra eterna andadura en el universo.
\vs p002 5:7 Me es fácil y grato adorar a quien es tan grande y a quien está, al mismo tiempo, tan afectuosamente dedicado a la elevación espiritual de sus modestas criaturas. Amo de forma natural a quien con tanta potestad crea y rige su creación, y a quien, además, es tan perfecto en bondad y tan fiel en la amorosa benevolencia con la que constantemente nos da cobijo. Aunque Dios no fuera ni tan grande ni tan poderoso, lo seguiría amando con la misma intensidad por ser tan bueno y tan misericordioso. Todos amamos al Padre más en razón de su naturaleza que en reconocimiento de sus asombrosos atributos.
\vs p002 5:8 Cuando observo a los hijos creadores y a los administradores del gobierno de los universos bajo su potestad luchando, tan valientemente, contra las múltiples dificultades del tiempo propias de la evolución de los universos del espacio, me doy cuenta de que profeso un afecto grande y profundo por estos gobernantes menores. Al fin y al cabo, pienso que todos nosotros, incluyendo los mortales de los mundos, amamos al Padre Universal y a todos los demás seres, divinos o humanos, porque percibimos que estos seres personales en verdad nos aman a nosotros. La vivencia de amar es, en buena parte, una respuesta directa a la vivencia de ser amado. Sabiendo que Dios me ama, debo seguir amándolo en grado sumo, incluso si se despojara de todos sus atributos de supremacía, ultimidad y absolutidad.
\vs p002 5:9 El amor del Padre va con nosotros ahora y a lo largo del interminable círculo de la eternidad de los tiempos. Cuando se reflexiona sobre la naturaleza amorosa de Dios, solo hay hacia ella una respuesta lógica y natural de la persona: amar cada vez más al Hacedor; depositar en Dios un afecto semejante al que siente un niño por su padre terrenal; porque, como un padre, un padre verdadero, un auténtico padre ama a sus hijos, así nos ama el Padre Universal y por siempre procura el bien de los hijos e hijas de su creación.
\vs p002 5:10 Pero el amor de Dios constituye un afecto paterno inteligente y previsor. El amor divino obra en íntima vinculación con la sabiduría divina y con todas las demás características infinitas de la naturaleza perfecta del Padre Universal. Dios es amor, pero el amor no es Dios. La más grande manifestación del amor divino para los seres mortales se observa en la dádiva de los modeladores del pensamiento, pero la más grande revelación del amor del Padre hacia vosotros se observa en la vida de gracia de su hijo miguel al vivir en la tierra el ideal de la vida espiritual. Es el modelador interior el que hace que el amor de Dios se realice de forma individual en cada alma humana.
\vs p002 5:11 \pc A veces, casi me apena verme obligado a describir el afecto divino que el Padre celestial siente por sus hijos del universo mediante el empleo del símbolo verbal humano \bibemph{amor}. Este término, aunque en efecto connote el concepto más elevado del hombre sobre las relaciones de respeto y devoción entre los mortales, designa, con tanta frecuencia, tantas relaciones humanas totalmente innobles y completamente inadecuadas de conocerse con la misma palabra que se usa para indicar el inigualable afecto del Dios vivo por sus criaturas del universo. ¡Me resulta tan desafortunado no poder hacer uso de algún término excelso que transmitiera a la mente del hombre, con exclusividad, la auténtica naturaleza y la excelente y hermosa relevancia del afecto divino del Padre del Paraíso!
\vs p002 5:12 \pc Cuando el hombre pierde de vista el amor de un Dios personal, el reino de Dios se convierte meramente en el reino del bien. Pese a la unidad infinita de la naturaleza divina, el amor es la característica dominante de cualquier trato personal de Dios con sus criaturas.
\usection{6. LA BONDAD DE DIOS}
\vs p002 6:1 En el universo físico, podemos ver la belleza divina; en el mundo intelectual, podemos percibir la verdad eterna, pero la bondad de Dios solo se encuentra en el mundo espiritual de la experiencia religiosa y personal. En su verdadera esencia, la religión es una fe\hyp{}confianza en la bondad de Dios. Dios puede que sea grande y absoluto e incluso inteligente y personal de algún modo para la filosofía, pero, para la religión, Dios debe ser también moral; debe ser bueno. El hombre quizás sienta temor hacia un gran Dios, pero solo hacia un Dios bueno siente confianza y amor. Esta bondad de Dios forma parte de su persona, y su plena revelación se manifiesta tan solo en la experiencia religiosa y personal de los hijos creyentes de Dios.
\vs p002 6:2 La religión da a entender que el supramundo de la naturaleza espiritual tiene conocimiento de las necesidades fundamentales del mundo humano y es receptivo a ellas. La religión evolutiva puede llegar a ser ética, pero solo la religión revelada puede llegar a ser verdadera y espiritualmente moral. El antiguo concepto de Dios como Deidad caracterizada por su moralidad regia fue elevado por Jesús a ese nivel afectuosamente conmovedor, a la moral íntima y familiar de la relación padre\hyp{}hijo, de la que no existe otra tan tierna y bella en la vivencia de los mortales.
\vs p002 6:3 \pc La “riqueza de la benignidad de Dios guía al hombre descarriado al arrepentimiento”. “Toda buena dádiva y todo don perfecto desciende del Padre de las luces”. “Dios es bueno; él es el refugio eterno de las almas de los hombres”. “Misericordioso y piadoso es el Señor Dios; tardo para la ira, y grande en bondad y verdad”. “¡Gustad y ved que es bueno el Señor! Dichoso el hombre que confía en él”. “Clemente y compasivo es el Señor. Él es el Dios de salvación”. “Él sana a los quebrantados de corazón, y venda las heridas del alma. Él es el todopoderoso benefactor del hombre”.
\vs p002 6:4 \pc El concepto de Dios como rey\hyp{}juez, aunque alentó a un elevado nivel moral y dio lugar a un pueblo que, como grupo, era respetuoso de la ley, dejaba al creyente individual en una pesarosa situación de inseguridad respecto a su condición en el tiempo y en la eternidad. Los profetas hebreos posteriores proclamaron que Dios era el Padre de Israel; Jesús reveló a Dios como el Padre de todo ser humano. El concepto que los mortales tienen de Dios está, en su totalidad y de manera suprema, iluminado por la vida de Jesús. La entrega desinteresada es algo inherente al amor paterno. Dios no nos ama como si fuera un padre, sino \bibemph{como} el padre que es. Él es el Padre del Paraíso para todo ser personal del universo.
\vs p002 6:5 \pc La rectitud da a entender que Dios constituye la fuente de la ley moral del universo. La verdad nos muestra a Dios como revelador, como maestro. Pero el amor da y anhela afecto, busca coparticipación comprensiva como la que existe entre padre e hijo. La rectitud quizás sea el pensamiento divino, pero el amor es una actitud de padre. La suposición errónea de que la rectitud de Dios era irreconciliable con el amor desinteresado del Padre celestial presuponía una falta de unidad en la naturaleza de la Deidad, y condujo directamente a la elaboración de la doctrina de la expiación, que es un atentado filosófico contra la unidad y la libre voluntad de Dios.
\vs p002 6:6 El cariñoso Padre celestial, cuyo espíritu mora en sus hijos de la tierra, no es un ser personal dividido en uno de justicia y en otro de misericordia, ni tampoco se necesita de un mediador para conseguir el favor o el perdón del Padre. La rectitud divina no se caracteriza por una justicia severa y punitiva; Dios como padre trasciende a Dios como juez.
\vs p002 6:7 \pc Dios nunca es irascible, vengativo o colérico. Es muy cierto que con frecuencia la sabiduría refrena su amor, en tanto que la justicia condiciona su misericordia cuando se la rechaza. Su amor hacia la rectitud no puede sino manifestarse igualmente en odio hacia el pecado. El Padre no es una persona incongruente; la unidad divina es perfecta. En la Trinidad del Paraíso existe unidad absoluta pese a la identidad eterna de los homólogos de Dios.
\vs p002 6:8 \pc Dios ama al pecador y \bibemph{odia} el pecado: este enunciado es filosóficamente cierto, pero Dios es un ser personal supremo, y las personas tan solo pueden amar y odiar a otras personas. El pecado no es una persona. Dios ama al pecador porque este es una realidad personal (potencialmente eterna), mientras que hacia el pecado Dios no adopta ninguna actitud personal, porque el pecado no es una realidad espiritual; no es personal; por lo tanto, solamente la justicia de Dios tiene conocimiento de su existencia. El amor de Dios salva al pecador; la ley de Dios destruye el pecado. Esta actitud de la naturaleza divina al parecer cambiaría si el pecador acabara por identificarse totalmente con el pecado, de la misma manera que esa misma mente mortal puede también plenamente identificarse con el modelador espiritual que vive en su interior. Un mortal así identificado con el pecado se haría, pues, de una naturaleza totalmente carente de espiritualidad (y por lo tanto personalmente irreal), y experimentaría finalmente la extinción de su ser. En un universo que se hace progresivamente real y cada vez más espiritual, la irrealidad, tanto más la imperfección de la naturaleza de las criaturas, no puede existir por siempre.
\vs p002 6:9 \pc Frente al mundo del ser personal, Dios es manifiestamente una persona amorosa; frente al mundo espiritual, él es amor personal; en la experiencia religiosa, él es ambas cosas. El amor reconoce la decidida voluntad de Dios. La bondad de Dios descansa sobre la base de la libre voluntad divina: la tendencia universal a amar, a mostrar misericordia, a manifestar paciencia y a dispensar perdón.
\usection{7. VERDAD Y BELLEZA DIVINAS}
\vs p002 7:1 Todo conocimiento finito y todo entendimiento creatural son \bibemph{relativos}. La información y la inteligencia, incluso si se recogen de fuentes superiores, solo son relativamente completas, localmente exactas y personalmente válidas.
\vs p002 7:2 Los hechos físicos son bastante uniformes, pero la verdad es un factor vivo y adaptable en la filosofía del universo. Los seres personales en evolución, en sus actos comunicativos, solo son parcialmente acertados y relativamente verdaderos. Solamente pueden tener certeza dentro de los límites de su experiencia personal. Lo que al parecer puede ser totalmente verdadero en un lugar, puede ser tan solo relativamente cierto en otro segmento de la creación.
\vs p002 7:3 La verdad divina, la verdad final, es uniforme y universal, pero el relato de las cosas espirituales, tal como lo narran numerosos seres procedentes de diversas esferas, puede que a veces varíe en sus detalles debido a la propia relatividad del conocimiento total y de la repleción de la experiencia personal, así como de la amplitud y alcance de dicha experiencia. Aunque las leyes y los decretos, los pensamientos y las actitudes de la Primera Gran Fuente y Centro sean verdaderos de forma eterna, infinita y universal, al mismo tiempo, su aplicación y adaptación a cada uno de los universos, sistemas, mundos e inteligencias de la creación están de acuerdo con los planes y métodos de los hijos creadores cuando obran en sus respectivos universos, así como en armonía con los planes y procedimientos, para un determinado lugar, del Espíritu Infinito y de todos los otros seres personales y celestiales que colaboran en esta labor.
\vs p002 7:4 \pc La falsa ciencia del materialismo condenaría al hombre mortal a convertirse en un paria del universo. Un conocimiento parcial así es malo en potencia; es un conocimiento compuesto de bien y mal. La verdad es bella porque es plena y simétrica. Cuando el hombre busca la verdad, persigue lo divinamente real.
\vs p002 7:5 Los filósofos cometen su más grave error cuando se dejan inducir por la falacia de la abstracción, de la práctica de enfocar la atención sobre un aspecto de la realidad para luego afirmar que dicho aspecto aislado es la verdad total. El filósofo juicioso siempre buscará el diseño creativo preexistente que se halla tras todos los fenómenos del universo. El pensamiento creador precede invariablemente a la acción creativa.
\vs p002 7:6 La propia conciencia intelectual puede hallar la belleza de la verdad, su carácter espiritual, no solo por la coherencia filosófica de sus conceptos, sino, con mayor certeza y seguridad, por la respuesta infalible del siempre presente espíritu de la verdad. La felicidad resulta del reconocimiento de la verdad porque esta puede ser \bibemph{actuada;} puede ser vivida. La decepción y el pesar se producen por el error porque, no siendo una realidad, no pueden realizarse como experiencia. La verdad divina se conoce mejor por su \bibemph{sabor espiritual.}
\vs p002 7:7 \pc Se está en eterna búsqueda de un todo unificado, de una coherencia divina. El extenso universo adquiere coherencia en la Isla del Paraíso; el universo intelectual adquiere coherencia en el Dios de la mente, en el Actor Conjunto; el universo espiritual se hace coherente en el ser personal del Hijo Eterno. Pero el mortal aislado en el tiempo y el espacio adquiere coherencia en Dios Padre mediante la relación directa entre el modelador del pensamiento que mora en su interior y el Padre Universal. El modelador del hombre es una fracción de Dios que busca de forma perpetua la unificación divina; adquiere coherencia con y en la Deidad del Paraíso de la Primera Fuente y Centro.
\vs p002 7:8 \pc Reconocer la belleza suprema significa descubrir e integrar la realidad: discernir la bondad divina en la verdad eterna, que es la belleza última. Incluso el encanto del arte humano radica en la armonía de su unidad.
\vs p002 7:9 El gran error de la religión hebrea se debió a que no supo asociar la bondad de Dios con las verdades fehacientes de la ciencia y la belleza atrayente del arte. A medida que la civilización avanzaba y, puesto que la religión proseguía el mismo insensato camino de poner demasiado énfasis en la bondad de Dios con una relativa exclusión de la verdad y descuido de la belleza, se desarrolló una creciente tendencia, en cierto tipo de personas, a apartarse del concepto abstracto y disociado de la bondad aislada. La moralidad aislada y exagerada de la religión moderna, que no sabe mantener la devoción y la lealtad de muchos hombres del siglo XX, se rehabilitaría a sí misma si, además de sus mandatos morales, concediera la misma importancia a las verdades de la ciencia, de la filosofía y de la experiencia espiritual al igual que a la belleza de la creación física, al encanto del arte intelectual y a la grandeza del logro de un carácter genuino.
\vs p002 7:10 El reto religioso de esta era pertenece a aquellos hombres y mujeres de percepción espiritual con visión de futuro y amplias miras que se atrevan a construir una nueva y atrayente filosofía de la vida a partir de conceptos modernos, ampliados e integrados espléndidamente, de la verdad cósmica, de la belleza del universo y de la bondad divina. Esta visión nueva y justa de la moral atraerá todo lo bueno de la mente del hombre y estimulará lo mejor del alma humana. La verdad, la belleza y la bondad son realidades divinas y, a medida que el hombre asciende en la escala de la vida espiritual, estas cualidades supremas del Eterno se coordinan y unifican cada vez más en Dios, que es amor.
\vs p002 7:11 \pc Toda verdad ---material, filosófica o espiritual--- es a la vez bella y buena. Toda belleza real ---arte material o simetría espiritual--- es a la vez verdadera y buena. Toda bondad genuina ---ya sea moral personal, equidad social o ministerio divino--- es igualmente verdadera y bella. La salud, la sensatez y la felicidad se convierten en partes integrantes de la verdad, de la belleza y de la bondad a medida que se combinan en la experiencia humana. Estos niveles de vida eficiente se originan mediante la unificación de los sistemas de la energía, de los sistemas de las ideas y de los sistemas del espíritu.
\vs p002 7:12 La verdad es coherente; la belleza, atractiva; la bondad, estabilizadora. Y cuando estos valores de lo que es real se coordinan en la experiencia de la persona, el resultado es un elevado orden de amor condicionado por la sabiduría y determinado por la lealtad. El propósito real de toda educación en el universo consiste en llevar a cabo la mejor coordinación del hijo aislado de los mundos con las realidades más grandes de su experiencia en expansión. La realidad es finita en el nivel humano e infinita y eterna en los niveles superiores y divinos.
\vsetoff
\vs p002 7:13 [Exposición de un consejero divino actuando por autorización de los ancianos de días de Uversa]
