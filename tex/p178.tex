\upaper{178}{Último día en el campamento}
\author{Comisión de seres intermedios}
\vs p178 0:1 Jesús tenía planeado pasar aquel jueves, su último día de libertad en la tierra como Hijo divino encarnado, con sus apóstoles y algunos pocos discípulos leales y devotos. Aquella hermosa mañana, poco después del desayuno, el Maestro los condujo a una zona apartada, a corta distancia, montaña arriba, del campamento, y les enseñó allí numerosas verdades nuevas. Aunque dio otras charlas a los apóstoles durante las tempranas horas de la noche de aquel día, la del jueves por la mañana fue su sermón de despedida al grupo de apóstoles y discípulos elegidos, tanto judíos como gentiles, allí acampados. Excepto Judas, todos los apóstoles estaban presentes. Pedro y diversos apóstoles comentaron su ausencia, y algunos de ellos pensaron que Jesús lo había enviado a la ciudad para atender algún asunto, quizás para arreglar los detalles de la próxima celebración de la Pascua. Judas no volvió al campamento hasta media tarde, algo antes de que Jesús llevara a los doce a Jerusalén para compartir la Última Cena.
\usection{1. DISCURSO SOBRE LA FILIACIÓN Y LA CIUDADANÍA}
\vs p178 1:1 Jesús habló a unos cincuenta de sus seguidores de confianza durante casi dos horas y contestó a una veintena de preguntas sobre la relación del reino de los cielos con los reinos de este mundo, sobre la relación entre la filiación con Dios y la ciudadanía en los gobiernos terrenales. Esta charla, junto con sus respuestas a estas preguntas, puede resumirse en lenguaje moderno tal como sigue:
\vs p178 1:2 \pc Los reinos de este mundo, siendo materiales, pueden a menudo verse en la necesidad de emplear la fuerza física para hacer cumplir sus leyes y poder mantener el orden. En el reino de los cielos, los verdaderos creyentes no recurren a la fuerza física. El reino de los cielos, al ser una hermandad espiritual de los hijos de Dios nacidos del espíritu, solo se puede instaurar mediante el poder del espíritu. La diferencia entre ambos modos de establecer un reino, ya sea mediante la fuerza física de los ciudadanos de gobiernos seculares o mediante el poder espiritual de los creyentes en el reino de los cielos, no priva del derecho que tienen los grupos sociales de creyentes a mantener el orden entre los suyos y administrar disciplina a sus miembros rebeldes e indignos.
\vs p178 1:3 No existe ninguna incompatibilidad entre la filiación en el reino espiritual y la ciudadanía en un gobierno secular o civil. El deber del creyente es dar al césar lo que es del césar y a Dios lo que es de Dios. No puede haber ninguna discrepancia entre estos dos requerimientos, al ser uno material y el otro espiritual, a menos que algún césar decida usurpar las prerrogativas de Dios y exija que se le rindan a él tributos espirituales y suprema adoración. Si así es el caso, adoraréis solamente a Dios, al mismo tiempo que tratáis de arrojar luz sobre esos errados gobernantes terrenales y asimismo guiarlos al reconocimiento del Padre de los cielos. No le rendiréis adoración espiritual a ningún dirigente terrenal; ni tampoco debéis emplear la fuerza física de los gobiernos de la tierra, cuyos representantes puedan en algún momento convertirse en creyentes, para difundir la misión del reino espiritual.
\vs p178 1:4 Desde el punto de vista de una civilización en continuo avance, la filiación en el reino debería serviros de ayuda para que os convirtáis en los ciudadanos ideales de los reinos de este mundo, dado que la hermandad y el servicio son los pilares fundamentales del evangelio del reino. El llamamiento al amor del reino espiritual debería ser capaz de destruir eficientemente el impulso al odio de los ciudadanos incrédulos y beligerantes de los reinos terrestres. Pero estos hijos de mente materialista que residen en la oscuridad nunca van a conocer vuestra luz espiritual de la verdad a no ser que os acerquéis bastante a ellos, ofreciéndoles ese servicio social desinteresado, resultado natural de los frutos del espíritu, que cada creyente por si mismo manifiesta en su experiencia de vida.
\vs p178 1:5 En verdad, como hombres mortales y materiales, sois ciudadanos de los reinos terrenales, y deberíais ser buenos ciudadanos, incluso los mejores, por el hecho de haber renacido como hijos espirituales del reino celestial. Como hijos iluminados por la fe y liberados por el espíritu del reino de los cielos, os enfrentáis a la responsabilidad doble del deber al hombre y el deber a Dios, mientras que, voluntariamente, asumís una tercera y sagrada obligación: servir a la hermandad de los creyentes conocedores de Dios.
\vs p178 1:6 No debéis adorar a vuestros dirigentes temporales, y no debéis emplear el poder físico para extender el reino espiritual, aunque sí impartir, a creyentes y a no creyentes por igual, el ministerio ecuánime de vuestro amoroso servicio. En el evangelio del reino, está inherentemente presente el poderoso espíritu de la verdad y, en breve, yo derramaré ese mismo espíritu sobre toda carne. Los frutos del espíritu, esto es vuestro servicio sincero y amoroso, constituyen una potente palanca social que hace que las razas que permanecen en la oscuridad se eleven, y este espíritu de la verdad es el punto de apoyo que multiplicará vuestro poder espiritual.
\vs p178 1:7 En vuestro trato con dirigentes civiles descreídos, manifestad sensatez y astucia. Con tacto, mostraos expertos en allanar pequeños desacuerdos y aclarar nimios malentendidos. De cualquier forma posible ---en todo salvo en vuestra lealtad espiritual hacia los gobernantes del universo---, procurad vivir en paz con todos los hombres. Sed siempre astutos como serpientes pero inocentes como palomas.
\vs p178 1:8 Como resultado de convertiros en hijos espiritualmente iluminados del reino, debéis ser aún mejores ciudadanos de los gobiernos seculares; asimismo, como resultado de creer en este evangelio celestial, los dirigentes de estos gobiernos terrenales deben convertirse en incluso mejores gobernantes de los asuntos civiles. La actitud de servir desinteresadamente al hombre y de adorar de modo inteligente a Dios debería hacer que todos los creyentes del reino fueran incluso los mejores ciudadanos del mundo; mientras que la actitud que lleva a un ciudadano a ser honesto y su devoción sincera a los deberes temporales debería ayudar a que ese ciudadano se hiciera aún más fácilmente receptivo a la llamada espiritual de la filiación en el reino celestial.
\vs p178 1:9 Siempre que los dirigentes de los gobiernos terrenales traten de imponer su autoridad como dictadores religiosos, los que creáis en este evangelio solo podéis esperar que os sobrevengan problemas, persecuciones e incluso la muerte. Pero esa misma luz que exhibís ante el mundo y hasta el modo en el que padeceréis y moriréis por este evangelio del reino acabarán por iluminar, por sí mismos, al mundo entero y resultarán en la paulatina separación entre política y religión. La perseverante predicación de este evangelio del reino traerá algún día, a todas las naciones, una liberación nueva e inimaginable y una libertad tanto intelectual como religiosa.
\vs p178 1:10 Ante las inminentes persecuciones que sufriréis por parte de los que odian este evangelio de gozo y libertad, vosotros os creceréis y el reino prosperará. Pero correréis un grave peligro, en tiempos venideros, cuando la mayoría de los hombres hablen bien de los creyentes del reino y muchos en altos cargos acepten formalmente el evangelio del reino celestial. Aprended a ser fieles al reino incluso en tiempos de paz y prosperidad. No tentéis a los ángeles que os supervisan a que os lleven por caminos difíciles y os disciplinen amorosamente para salvar vuestras almas de la fácil deriva.
\vs p178 1:11 Recordad que se os ha designado para predicar este evangelio del reino ---el supremo deseo de hacer la voluntad del Padre aunado al gozo supremo del reconocimiento mediante la fe de la filiación con Dios--- y no debéis permitir que nada os desvíe de vuestra dedicación a este único deber. Que toda la humanidad se beneficie del derramamiento de vuestro amoroso ministerio espiritual, de vuestra instructiva comunicación intelectual, de vuestro estimulante servicio social; pero ninguna de estas tareas humanitarias deben ocupar el lugar de la proclamación del evangelio. Estos importantes servicios son los resultados sociales de ese otro ministerio aún más poderoso y sublime y de las transformaciones, que se realizan en el corazón del creyente por el Espíritu vivo de la Verdad y por el entendimiento personal de que la fe de un hombre nacido del espíritu confiere la seguridad de la fraternidad viva con el Dios eterno.
\vs p178 1:12 No intentéis promulgar la verdad ni instaurar la rectitud propugnando el poder de los gobiernos civiles ni adoptando leyes seculares. Siempre podéis afanaros por persuadir las mentes de los hombres, pero nunca os atreváis a obligarlos. No olvidéis la gran ley de la equidad humana que os he enseñado en su forma positiva: aquello que queráis que los hombres hagan con vosotros, así también haced vosotros con ellos.
\vs p178 1:13 Cuando un creyente del reino se siente llamado a servir al gobierno civil, que preste este servicio como ciudadano temporal de dicho gobierno, cuidando de mostrar en su servicio social todas las cualidades ordinarias de la ciudadanía, aunque enaltecidas por la iluminación espiritual, cuyo origen está en el ennoblecedor vínculo de la mente del hombre mortal con el espíritu morador proveniente del Dios eterno. Si resulta que un no creyente es superior a ti como servidor civil, deberías cuestionarte seriamente si es que las raíces de tu corazón no se han secado por la falta de esas aguas vivas, que son los frutos de la comunión espiritual junto con el servicio social. La conciencia de la filiación con Dios debe revitalizar enteramente la vida de servicio de cualquier hombre, mujer y niño, que posea tan gran estímulo para desarrollar todos los poderes inherentes a una persona humana.
\vs p178 1:14 No debéis ser místicos pasivos ni ascetas insulsos; no os convirtáis en soñadores ni en vagabundos que confían indolentemente en una Providencia ficticia que les proporcione incluso lo necesario para vivir. De hecho, debéis ser amables en vuestro trato con los mortales errados, pacientes en vuestras interrelaciones con los hombres ignorantes, tolerantes cuando os provoquen; pero también debéis ser valientes en la defensa de la rectitud, poderosos en la proclamación de la verdad y contundentes en la predicación de este evangelio del reino, llevándolo incluso a todos los rincones de la tierra.
\vs p178 1:15 Este evangelio del reino es una verdad viva. Os he dicho que es como la levadura en la masa, como el grano de la semilla de mostaza; y ahora os digo que es como la simiente de los seres vivos, la cual, de generación en generación, aunque sigue siendo la misma simiente viva, se desarrolla de forma inequívoca, tomando nuevas formas y creciendo satisfactoriamente hasta adaptarse de nuevo a las peculiares necesidades y condiciones de cada sucesiva generación. La revelación que yo os he brindado es una \bibemph{revelación viva,} y deseo que rinda idóneamente sus frutos en cada persona y generación, según las leyes del crecimiento espiritual, del incremento de servicio y del desarrollo adaptativo. De generación en generación, este evangelio debe mostrar cada vez mayor vitalidad y un poder espiritual más intenso. No puede permitirse que se convierta en un simple recuerdo sagrado, en una mera historia sobre mí y los tiempos que ahora vivimos.
\vs p178 1:16 Y no olvidéis: no hemos sido hostiles a las personas ni a la autoridad de los que se sientan en la cátedra de Moisés; solo les hemos ofrecido la nueva luz, que ellos tan rotundamente rechazaron. Sí nos hemos opuesto frontalmente a ellos, denunciado su deslealtad espiritual a esas mismas verdades que profesan enseñar y salvaguardar. Nos hemos enfrentado a esos líderes y dirigentes reconocidos solo cuando obstaculizaron deliberadamente nuestra predicación del evangelio del reino a los hijos de los hombres. E incluso ahora, no somos nosotros los que los hostigamos, sino que son ellos los que quieren acabar con nosotros. No os olvidéis de que tenéis el cometido de salir a predicar solo la buena nueva. No debéis atacar las antiguas maneras, sino, más bien, añadir, hábilmente, la levadura de la nueva verdad en las viejas creencias. Dejad que el espíritu de la verdad realice su propia labor. Que la controversia surja solamente cuando los que desprecian la verdad os obliguen a ella. Pero cuando el descreído, en su obstinamiento, os ataque, no dudéis en defender con contundencia la verdad que os ha salvado y santificado.
\vs p178 1:17 Ante las vicisitudes de la vida, recordad siempre que os debéis amar unos a otros. No contendáis con los hombres, ni siquiera con los incrédulos. Mostrad misericordia incluso con los que os ofenden y desprecian. Mostrad que sois ciudadanos leales, artesanos íntegros, encomiables vecinos, parientes entregados, padres comprensivos y creyentes sinceros en la hermandad del reino del Padre. Y mi espíritu estará con vosotros, ahora e incluso hasta el fin del mundo.
\vs p178 1:18 \pc Cuando Jesús terminó de impartir sus enseñanzas, era casi la una, y regresaron de inmediato al campamento, donde David y sus compañeros les tenían ya preparado el almuerzo.
\usection{2. TRAS EL ALMUERZO}
\vs p178 2:1 No muchos de los que oyeron al Maestro pudieron asimilar ni siquiera una parte de las palabras que les dirigió aquella mañana. De todos los asistentes, fueron los griegos los que más llegaron a comprenderlo. Incluso los once apóstoles se quedaron desconcertados por sus alusiones a futuros reinos políticos y a sucesivas generaciones de creyentes del reino. Los seguidores más fervientes de Jesús no podían compatibilizar el fin inminente de su ministerio terrenal con estas referencias a un dilatado futuro de actividades evangélicas. Algunos de estos creyentes judíos empezaban a tener la sensación de que estaba a punto de producirse la mayor tragedia habida sobre la tierra, pero no podían compaginar un desastre inmediato de tal envergadura con la actitud personal del Maestro de animosa indiferencia, con su charla de aquella mañana en la que aludió repetidas veces a los futuros asuntos del reino celestial, que al parecer se prolongaría muy extensamente en el tiempo y llegaría a abarcar interrelaciones con muchos y sucesivos reinos temporales de la tierra.
\vs p178 2:2 Hacia el mediodía de esa misma jornada, todos los apóstoles y discípulos sabían de la precipitada huida de Lázaro de Betania. Comenzaban a darse cuenta de la funesta determinación de los dirigentes judíos por matar a Jesús y poner fin a sus enseñanzas.
\vs p178 2:3 David Zebedeo, por medio del trabajo de sus agentes secretos en Jerusalén, estaba enteramente informado sobre los detalles del plan para arrestar y matar a Jesús. Conocía todo lo referente al papel de Judas en este complot, pero nunca desveló lo que sabía a los otros apóstoles ni a ninguno de los discípulos. Poco después del almuerzo, llevó de hecho a Jesús aparte y, atreviéndose, le preguntó a Jesús si sabía\ldots ---pero no pudo acabar de formular su pregunta---. El Maestro, alzando la mano, lo detuvo diciendo: “Sí, David, lo sé todo, y sé que tú lo sabes, pero cuídate de no contárselo a nadie. Solo te digo que no dudes en tu propio corazón de que la voluntad de Dios prevalecerá al final”.
\vs p178 2:4 Esta conversación con David se vio interrumpida por la llegada de un mensajero proveniente de Filadelfia con la noticia de que Abner se había enterado del complot para matar a Jesús y preguntaba si debía partir para Jerusalén. Este corredor salió a toda prisa para Filadelfia con estas palabras para él: “Continúa con tu labor. Si yo me alejo de ti en la carne, es únicamente para poder regresar en el espíritu. No te abandonaré. Estaré contigo hasta el fin”.
\vs p178 2:5 Sobre aquella hora, Felipe se acercó al Maestro y preguntó: “Maestro, viendo que se acerca el momento de la Pascua, ¿dónde quieres que preparemos para comer la cena?”. Y cuando Jesús oyó la pregunta de Felipe, respondió: “Ve y trae a Pedro y a Juan, y os daré instrucciones en cuanto a la cena que tomaremos juntos esta noche. En cuanto a la Pascua, eso deberéis considerarlo vosotros tras haber hecho esto primero”.
\vs p178 2:6 Cuando Judas vio al Maestro hablando con Felipe sobre estas cosas, se aproximó para oír lo que decían. Pero David Zebedeo, que se encontraba cerca, se adelantó e intercambió unas palabras con él, mientras Felipe, Pedro y Juan se fueron aparte para hablar con el Maestro.
\vs p178 2:7 Jesús dijo a los tres: “Id inmediatamente a Jerusalén y, al entrar por la puerta, hallaréis a un hombre que lleva un cántaro de agua. Él os hablará y, entonces, seguidlo. Cuando os lleve a una cierta casa, entrad detrás de él y decid al señor de esa casa: ‘¿Dónde está la sala donde el Maestro ha de comer la cena con sus apóstoles?’ Y cuando le hayáis preguntado tal cosa, este propietario os mostrará un gran aposento en el piso superior ya dispuesto y preparado para nosotros”.
\vs p178 2:8 Cuando los apóstoles llegaron a la ciudad, encontraron al hombre con el cántaro de agua, próximo a la puerta, y lo siguieron hasta la casa de Juan Marcos, donde el padre del muchacho los recibió y les mostró el aposento alto ya listo para la cena.
\vs p178 2:9 Y todo esto vino a pasar gracias al acuerdo alcanzado entre el Maestro y Juan Marcos la tarde del día anterior, estando ambos solos en las colinas. Jesús quería asegurarse de tener esta última cena con sus apóstoles sin ser molestado, y, suponiendo que si Judas conocía con antelación su lugar de encuentro podría planear con sus enemigos su prendimiento, llegó a este arreglo secreto con Juan Marcos. Por lo tanto, Judas no supo que se reunirían en aquel sitio hasta más tarde, cuando llegó allí en compañía de Jesús y de los demás apóstoles.
\vs p178 2:10 \pc David Zebedeo tenía muchos asuntos de tipo económico que tratar con Judas, por lo que le resultó fácil impedirle que siguiera a Pedro, Juan y Felipe, como este deseaba tanto hacer. Cuando Judas le dio a David una determinada suma de dinero para abastecimientos, David le dijo: “Judas ¿no sería aconsejable, dadas las circunstancias, que me proporcionaras algo de dinero por adelantado para cubrir las necesidades que se me presenten?”. Y tras reflexionar un momento, Judas respondió: “Sí, David, creo que sería prudente. De hecho, teniendo en cuenta la situación inquietante de Jerusalén, pienso que sería mejor para mí hacerte entrega de todo el dinero. Hay un complot contra el Maestro y, así, si me sucediera algo, no te verías obstaculizado en tu labor”.
\vs p178 2:11 Y de este modo fue como David recibió todos los fondos apostólicos en efectivo y los recibos del dinero que tenían en depósito. Los apóstoles no conocieron este hecho hasta las últimas horas de la tarde del día siguiente.
\vs p178 2:12 \pc Fue sobre las cuatro y media cuando los tres apóstoles regresaron e informaron a Jesús de que todo estaba listo para la cena. El Maestro se preparó de inmediato para llevar a sus doce apóstoles por el sendero que conducía a la carretera de Betania, la cual tomaron para ir a Jerusalén. Aquel fue el último viaje que Jesús haría con los doce.
\usection{3. DE CAMINO AL CENÁCULO}
\vs p178 3:1 Tratando de evitar de nuevo a las multitudes que atravesaban el valle del Cedrón, yendo y viniendo entre el parque de Getsemaní y Jerusalén, Jesús y los doce se encaminaron por la cumbre occidental del Monte de los Olivos para llegar a la carretera que bajaba desde Betania a la ciudad. Al aproximarse al lugar en el que Jesús se había detenido al final de la tarde anterior para conversar sobre la destrucción de Jerusalén, inconscientemente, hicieron una pausa y se quedaron allí mirando silenciosos la ciudad, desde lo alto. Como era algo temprano y, dado que no quería pasar por la ciudad antes de la caída del sol, Jesús dijo a sus acompañantes:
\vs p178 3:2 \pc “Sentaos y reposad mientras yo os hablo de lo que pronto sucederá. Todos estos años hemos vivido como hermanos, os he enseñado la verdad sobre el reino de los cielos y os he revelado sus misterios. Y, en verdad, mi Padre ha hecho muchas obras magníficas en relación a mi misión en la tierra. Habéis sido testigos de todo esto y habéis sido partícipes de la experiencia de trabajar en cooperación con Dios. Y os pongo por testigos de lo que os he venido advirtiendo durante algún tiempo, de que en breve debo regresar a la labor que el Padre ha dispuesto que haga; con claridad os he manifestado que debo dejaros en el mundo para llevar a cabo la obra del reino. Fue por este motivo por el que os escogí en las colinas de Cafarnaúm. Debéis prepararos ahora para compartir con otros la experiencia que habéis vivido conmigo. Como el Padre me envió a este mundo, así yo os enviaré a vosotros para que me representéis y acabéis la tarea que he comenzado.
\vs p178 3:3 “Con dolor, depositáis vuestra mirada sobre la ciudad que veis allá abajo, porque habéis oído mis palabras que os hablaban del fin de Jerusalén. Os lo he advertido con anticipación para que no perezcáis al ser destruida y se retrase así la proclamación del evangelio del reino. Asimismo, os advierto que miréis por vosotros mismos y que no os expongáis innecesariamente al peligro cuando vengan a llevarse al Hijo del Hombre. Debo irme, pero vosotros debéis quedaros y dar testimonio de este evangelio cuando yo haya partido, tal como le dije a Lázaro que huyera de la ira del hombre para que viviera y diera así a conocer la gloria de Dios. Si la voluntad del Padre es que yo me vaya, nada de lo que vosotros hagáis interferirá con el designio divino. Cuidaos de que no os maten también a vosotros. Que con valentía vuestras almas salvaguarden el evangelio por el poder del espíritu, pero no os induzcáis a error intentando imprudentemente defender al Hijo del Hombre. No necesito defensa de la mano del hombre; a la verdad, los ejércitos del cielo están en este momento muy cerca; pero estoy determinado a hacer la voluntad de mi Padre, y por ello debemos someternos a lo que pronto nos acaecerá.
\vs p178 3:4 “Cuando veáis esa ciudad destruida, no os olvidéis de que ya habéis entrado en una vida eterna de perdurable servicio en el reino de los cielos, siempre en avance, e incluso del cielo de los cielos. Debéis saber que en el universo de mi Padre y en el mío hay muchas estancias, y que allí aguarda a los hijos de la luz la revelación de ciudades cuyo constructor es Dios y mundos cuyo hábito de vida es la rectitud y el gozo de la verdad. He traído el reino de los cielos a vosotros, aquí a la tierra, pero os declaro, que todos aquellos de vosotros, que por la fe entréis en él y permanezcáis en él por medio del servicio vivo de la verdad, ciertamente ascenderéis a los mundos de lo alto y os sentaréis conmigo en el reino espiritual de nuestro Padre. Pero, primeramente, debéis aprestaos y completar la obra que habéis comenzado conmigo. Debéis padecer muchas tribulaciones y soportar muchos pesares ---y estas pruebas están ya ahora sobre nosotros--- y cuando hayáis culminado vuestra labor en la tierra, vendréis a mi gozo, así como yo he culminado en la tierra la obra de mi Padre, y estoy a punto de regresar para ser acogido por él”.
\vs p178 3:5 \pc Cuando el Maestro acabó de hablar, se levantó, y todos ellos lo siguieron, bajando el Monte de los Olivos con él hasta entrar en la ciudad. Salvo tres de los apóstoles, ninguno de ellos sabía adónde iban, mientras se abrían camino por las estrechas calles, próxima ya la oscuridad. Se vieron empujados por las multitudes, pero nadie los reconoció ni supo que el Hijo de Dios pasaba por allí, camino de su último encuentro como humano con los embajadores del reino escogidos por él. Y los apóstoles tampoco sabían que uno de entre ellos mismos ya había conspirado para traicionar al Maestro y ponerlo en manos de sus enemigos.
\vs p178 3:6 Juan Marcos los había seguido todo el camino hasta la ciudad, y tras entrar con ellos por la puerta, tomó a toda prisa otra calle para poder recibirlos en la casa de su padre cuando llegaran.
