\upaper{123}{Niñez temprana de Jesús}
\author{Comisión de seres intermedios}
\vs p123 0:1 Debido a las incertidumbres y ansiedades sufridas durante su estancia en Belén, María no destetó al niño hasta que llegaron sin ningún percance a Alejandría, lugar en el que la familia pudo establecerse y llevar una vida normal. Vivieron con unos parientes, y José estuvo en condiciones de mantener a su familia al conseguir trabajo poco después de su llegada. Durante varios meses, estuvo empleado como carpintero y luego fue ascendido al puesto de capataz a cargo de un gran grupo de obreros contratados para la construcción de un edificio público, que se encontraba en aquel momento en obras. Esta nueva experiencia le dio la idea de convertirse en contratista y constructor tras su regreso a Nazaret.
\vs p123 0:2 \pc Durante todos estos tempranos años de la infancia de Jesús en la que era un pequeño indefenso, María estuvo siempre vigilante para que no le sucediera nada a su hijo que lo pudiera poner en peligro o que pudiera interferir con su futura misión en la tierra; nunca madre alguna estuvo tan entregada a un hijo como lo estaba ella. En la casa en la que Jesús vivía había otros dos niños más o menos de la misma edad y, entre los vecinos de la cercanía, había otros seis cuyas edades se aproximaban lo suficiente a la suya como para ser unos compañeros de juego adecuados. Al principio, María trató de mantener a Jesús siempre cerca de ella. Temía que le pudiese ocurrir alguna cosa si se le permitía jugar en el jardín con los otros niños, pero José, con la ayuda de sus parientes, logró convencerla de que tal actitud privaría a Jesús de la provechosa experiencia de aprender a relacionarse con los niños de su misma edad. Y, María, comprendiendo que una medida así que lo protegiera de forma innecesaria e inusual tendería a convertir al niño en alguien cohibido y algo egocéntrico, acabó por acceder y actuar de modo que permitiera al niño de la promesa crecer como otro niño cualquiera. Si bien, aunque acató tal decisión, su preocupación personal era estar siempre pendiente de él mientras que los pequeños jugaban alrededor de la casa o en el jardín. Solo una madre amorosa es capaz de comprender la tensión que María tuvo que soportar en su corazón en bien de la seguridad de su hijo durante estos años de su infancia y primera niñez.
\vs p123 0:3 Durante los dos años de estancia en Alejandría, Jesús gozó de buena salud y continuó creciendo con normalidad. Exceptuando a unos pocos amigos y parientes, a nadie se le dijo que Jesús era un “niño de la promesa”. Uno de los parientes de José se lo reveló a unos amigos de Menfis, descendientes distantes de Akenatón y ellos, junto con un pequeño grupo de creyentes de Alejandría, se reunieron en la casa palacio del pariente y benefactor de José, poco antes de regresar a Palestina, para expresarles sus mejores deseos a la familia de Nazaret y presentar sus respetos al niño. Con este motivo, los amigos allí congregados le regalaron a Jesús un ejemplar completo de la traducción al griego de las escrituras hebreas. Pero este ejemplar de los textos sagrados judíos no estuvo a disposición de José hasta que él y María declinaron de manera definitiva la invitación de sus amigos de Menfis y de Alejandría de permanecer en Egipto. Estos creyentes afirmaban que el hijo de destino podría ejercer una influencia en el mundo mucho mayor si residía en Alejandría que si lo hacía en algún determinado lugar de Palestina. Estas opiniones retrasaron algún tiempo su partida a Palestina una vez que recibieron la noticia de la muerte de Herodes.
\vs p123 0:4 \pc José y María se despidieron finalmente de Alejandría partiendo en un barco propiedad de su amigo Esraeon con destino a Jope, puerto al que llegaron a finales de agosto del año 4 a. C. Desde allí se dirigieron directamente a Belén, donde pasaron todo el mes de septiembre en consulta con amigos y parientes para decidir si debían permanecer allí o regresar a Nazaret.
\vs p123 0:5 María nunca había renunciado del todo a la idea de que Jesús debía crecer en Belén, la ciudad de David. José no creía en realidad que su hijo fuese a convertirse en un rey libertador de Israel. Además, sabía que él mismo no era en verdad descendiente de David; el hecho de contarse entre los vástagos de David se debía a que uno de sus ancestros había sido adoptado por alguien de linaje davídico. María pensaba naturalmente que el lugar más apropiado para criar al nuevo candidato al trono de David era la ciudad de David, pero José prefería arriesgarse con Herodes Antipas antes que con su hermano Arquelao. Sentía un gran temor por la seguridad del niño en Belén o en cualquier otra ciudad de Judea; suponía que había más probabilidades de que Arquelao continuase con la política amenazadora de su padre, Herodes, a que lo hiciera Antipas en Galilea. Y aparte de todas estas razones, José era categórico respecto a su preferencia por Galilea como el mejor lugar para criar y educar al niño, pero necesitó tres semanas para vencer las objeciones de María.
\vs p123 0:6 Para el primer día de octubre, José había persuadido a María y a todos sus amigos de que era mejor para ellos regresar a Nazaret. Por consiguiente, a principios de octubre del año 4 a. C., partieron de Belén para Nazaret por el camino de Lida y Escitópolis. Salieron un domingo por la mañana temprano. María y el niño iban montados en el animal de carga que acababan de adquirir, mientras que José y los cinco parientes que los acompañaban lo hacían a pie; los parientes de José no estaban dispuestos a permitir que hicieran el viaje hasta Nazaret solos. Temían que fuesen a Galilea pasando por Jerusalén y el valle del Jordán, y las rutas occidentales no eran del todo seguras para dos viajeros solitarios con un niño de corta edad.
\usection{1. DE VUELTA A NAZARET}
\vs p123 1:1 Al cuarto día, los viajeros arribaron a su destino sanos y salvos. Sin previo aviso, llegaron a la casa de Nazaret que había estado ocupada desde hacía más de tres años por uno de los hermanos casados de José, y quien en efecto se sorprendió al verlos. Lo habían hecho todo con tanta discreción que ni la familia de José ni la de María sabían siquiera que habían dejado Alejandría. Al día siguiente, el hermano de José se mudó con su familia y, María, por primera vez desde el nacimiento de Jesús, se instaló allí con su pequeña familia para disfrutar de la vida en su propio hogar. En menos de una semana, José consiguió trabajo como carpintero, y fueron sumamente felices.
\vs p123 1:2 Jesús tenía unos tres años y dos meses cuando regresaron a Nazaret. Había soportado muy bien todos estos viajes y gozaba de una excelente salud. Tener una casa propia en la que correr y divertirse lo llenaba de alegría y de entusiasmo infantil. No obstante, echaba mucho de menos la relación con sus compañeros de juego de Alejandría.
\vs p123 1:3 Camino de Nazaret, José había convencido a María de lo imprudente que sería divulgar entre sus amigos y parientes galileos que Jesús era un niño de la promesa. Convinieron en evitar mencionar dicho asunto a nadie. Y ambos cumplieron muy fielmente con su palabra.
\vs p123 1:4 Todo el cuarto año de vida de Jesús fue un periodo de un desarrollo físico normal y de una actividad mental inusual. Entretanto, había forjado una estrecha relación con un niño vecino llamado Jacob, aproximadamente de su edad. Jesús y Jacob siempre se divertían jugando juntos, y se convirtieron en grandes amigos y leales compañeros.
\vs p123 1:5 El siguiente acontecimiento importante en la vida de esta familia de Nazaret fue el nacimiento del segundo hijo, Santiago, al amanecer del 2 de abril del año 3 a. C. A Jesús le encantaba tener un hermano menor, y durante horas permanecía cerca de él simplemente para observar los primeros movimientos del bebé.
\vs p123 1:6 Fue a mediados del verano de este mismo año cuando José construyó un pequeño taller cercano a la fuente de la localidad y del terreno donde se detenían las caravanas. A partir de entonces realizaba al día muy poca labor de carpintería. Asociados a él tenía a dos de sus hermanos y a varios otros artesanos, a quienes enviaba a trabajar mientras él permanecía en el taller fabricando arados y yugos o realizando otros trabajos en madera. También trabajó con el cuero, la soga y la lona. Y Jesús, a medida que crecía, cuando no estaba en la escuela, repartía su tiempo casi por igual entre ayudar a su madre en las tareas del hogar y observar a su padre trabajando en el taller, escuchando mientras tanto las conversaciones y los comentarios de los conductores y viajeros de las caravanas procedentes de todos los rincones de la tierra.
\vs p123 1:7 En julio de este año, un mes antes de que Jesús cumpliera los cuatro años, una epidemia maligna de problemas intestinales, causada por el contacto con los viajeros de las caravanas, se propagó por todo Nazaret. María se alarmó tanto por el peligro que suponía aquella enfermedad epidémica para Jesús, que dispuso a sus dos hijos para el trayecto y huyó a la casa de campo de su hermano, a varios kilómetros al sur de Nazaret en la carretera de Megido, cerca de Sarid. No volvieron a Nazaret hasta que trascurrieron más de dos meses; Jesús disfrutó bastante de la que sería su primera experiencia en una granja.
\usection{2. SU QUINTO AÑO (AÑO 2 A. C.)}
\vs p123 2:1 En algo más de un año tras su regreso a Nazaret, el niño Jesús llegó a la edad de su primera decisión moral personal e incondicional; fue entonces cuando vino a morar en él un modelador del pensamiento, un don divino del Padre del Paraíso, que había servido anteriormente con Maquiventa Melquisedec y había adquirido así la experiencia de actuar en relación con la encarnación de un ser supramortal que había vivido semejando un hombre mortal. Este importante hecho tuvo lugar el 11 de febrero del año 2 a. C. Jesús no tuvo más conciencia de la llegada del mentor divino que millones y millones de otros niños que, antes y desde este día, han recibido igualmente estos modeladores del pensamiento para residir en sus mentes y obrar para la postrera espiritualización de dichas mentes y la supervivencia eterna de sus almas inmortales en evolución.
\vs p123 2:2 En este día de febrero concluyó la supervisión directa y personal de los gobernantes del universo en referencia a la integridad de la encarnación como niño Miguel. Desde este momento y abarcando todo el desarrollo humano de su encarnación, la custodia de Jesús correspondió a este modelador interior y a los guardianes seráficos acompañantes, a la que se agregaba a veces el ministerio de las criaturas intermedias designadas para la realización de ciertas tareas concretas, de acuerdo con las instrucciones de sus superiores planetarios.
\vs p123 2:3 \pc Jesús cumplió cinco años en agosto de este año y, por ello, nos referiremos a este periodo como el quinto año (natural) de su vida. En dicho año, 2 a. C., poco más de un mes antes de su quinto cumpleaños, la llegada al mundo de su hermana Miriam, nacida en la noche del 11 de julio, le proporcionó una gran alegría. A última hora de la tarde del día siguiente, Jesús tuvo una larga conversación con su padre sobre el modo en el que los diversos grupos de seres vivos nacen en el mundo como seres diferentes. La parte más valiosa de la temprana educación de Jesús se la aportaban sus padres respondiendo a sus profundas y penetrantes preguntas. José no dejó nunca de cumplir al completo con su deber, esforzándose y encontrando el tiempo para contestar a las numerosas cuestiones que el niño les planteaba. Desde los cinco hasta los diez años, Jesús fue una continua interrogante. Aunque José y María no siempre podían dar respuesta a sus preguntas, nunca dejaron de debatirlas a fondo y, de cualquier manera posible, asistirle en su afán por hallar una solución satisfactoria a la cuestión que su mente despierta le había propuesto.
\vs p123 2:4 Desde su regreso a Nazaret, habían tenido una ajetreada vida doméstica, y José había estado excepcionalmente atareado con la construcción de su nuevo taller y el reinicio de su negocio. Tan ocupado había estado que no había tenido tiempo de hacer una cuna para Santiago, aunque pudo remediarlo mucho antes de que naciera Miriam, así que ella pudo contar con una cuna muy cómoda en la que se acurrucaba mientras que la familia la contemplaba con admiración. Y el niño Jesús se involucraba con entusiasmo en todas estas experiencias naturales y normales del hogar. Disfrutaba enormemente de su hermano pequeño y de su hermanita, y ayudaba mucho a María a atenderlos.
\vs p123 2:5 Existían pocos hogares en el mundo gentil de esos días que pudiesen proporcionar a un niño una mejor formación intelectual, moral y religiosa que la que se proporcionaba en los hogares judíos de Galilea. Estos judíos seguían un programa sistemático para criar y educar a sus hijos. Dividían la vida de los niños en siete etapas:
\vs p123 2:6 \li{1.}El niño recién nacido, desde el primero hasta el octavo día.
\vs p123 2:7 \li{2.}El niño de pecho.
\vs p123 2:8 \li{3.}El niño destetado.
\vs p123 2:9 \li{4.}El período de dependencia de la madre, hasta el final del quinto año.
\vs p123 2:10 \li{5.}El comienzo de la independencia del niño y, en el caso de los hijos varones, la asunción de la responsabilidad del padre en su educación.
\vs p123 2:11 \li{6.}El adolescente y la adolescente.
\vs p123 2:12 \li{7.}El hombre y la mujer joven.
\vs p123 2:13 \pc Era la costumbre entre los judíos de Galilea que la madre asumiera la responsabilidad de la educación de los hijos hasta que estos cumplieran los cinco años de edad y, si eran varones, era el padre el que se encargaba desde aquel momento de su educación. Aquel año, pues, Jesús entró en la quinta etapa de la andadura de un niño judío de Galilea; y, por lo tanto, el 21 de agosto del año 2 a. C., María lo entregó formalmente a José para que prosiguiera con su educación.
\vs p123 2:14 Si bien, aunque José tenía ahora la responsabilidad directa de la educación intelectual y religiosa de Jesús, su madre prestaba todavía atención a su formación doméstica. Le enseñó a conocer y a cuidar las parras y las flores que crecían en las tapias del jardín que rodeaban por completo la parcela de la vivienda. María también colocó en la azotea de la casa (el dormitorio de verano) unos cajones de arena poco profundos en los que Jesús elaboraba mapas y llegó a realizar gran parte de sus primeras prácticas de escritura en arameo, en griego y más tarde en hebreo, porque con el tiempo aprendió a leer, escribir y hablar con fluidez estas tres lenguas.
\vs p123 2:15 Jesús, físicamente, tenía la apariencia de un niño prácticamente perfecto, y mental y emocionalmente continuaba haciendo progresos de manera normal. Tuvo su primera enfermedad leve, un ligero malestar digestivo, a finales de este año, su quinto año (natural).
\vs p123 2:16 Aunque José y María hablaban a menudo del futuro de su hijo mayor, si hubieseis estado allí, solamente habríais observado el crecimiento de un niño normal de aquel tiempo y lugar, sano, despreocupado, pero extraordinariamente inquisitivo.
\usection{3. HECHOS RELEVANTES DE SU SEXTO AÑO (AÑO 1 A. C.)}
\vs p123 3:1 Con la ayuda de su madre, Jesús ya había dominado el dialecto galileo de la lengua aramea; y, ahora, su padre empezó a enseñarle el griego. María lo hablaba poco, pero José hablaba con fluidez el griego y el arameo. Para estudiar la lengua griega se usaba como libro de texto un ejemplar de las escrituras hebreas ---una versión completa de la ley y de los profetas, incluidos los salmos---, que les habían regalado a su partida de Egipto. En todo Nazaret, solo había dos ejemplares completos de las escrituras en griego, y el hecho de que la familia del carpintero poseyese uno de ellos hacía de la casa de José un lugar muy solicitado, algo que permitió a Jesús conocer, conforme crecía, una procesión casi interminable de estudiosos fervientes y de buscadores sinceros de la verdad. Antes de terminar este año, Jesús había asumido la custodia de este inestimable manuscrito; en su sexto cumpleaños, se le había dicho que el libro sagrado había sido un regalo de amigos y parientes de Alejandría. Y, en muy poco tiempo, Jesús pudo leerlo con facilidad.
\vs p123 3:2 \pc La primera gran conmoción en la joven vida de Jesús sucedió cuando aún no había cumplido los seis años. Al pequeño le parecía que su padre ---o al menos su padre y su madre juntos--- lo sabían todo. Imaginaos pues la sorpresa que se llevó este inquisitivo niño cuando preguntó a su padre por la causa de un leve terremoto que acababa de ocurrir, y oyó que José le decía: “Hijo mío, realmente no lo sé”. Así comenzó una larga y desconcertante sucesión de desilusiones en la que Jesús descubrió que sus padres terrenales no tenían la sabiduría ni el conocimiento de todas las cosas.
\vs p123 3:3 El primer pensamiento de José fue decirle a Jesús que Dios había causado el terremoto, pero un momento de reflexión lo hizo percatarse de que una respuesta de tal naturaleza provocaría de inmediato otras preguntas aún más embarazosas. Resultaba muy difícil contestar a las preguntas que formulaba Jesús, incluso a una edad muy temprana, sobre los fenómenos físicos o sociales diciéndole sin más que el responsable era Dios o el diablo. En consonancia con las creencias arraigadas en el pueblo judío, hacía tiempo que Jesús estaba dispuesto a aceptar la doctrina de los buenos y de los malos espíritus como una posible explicación de los fenómenos mentales y espirituales, pero muy pronto llegó a estar dudoso de que estas influencias invisibles fueran responsables de los sucesos físicos del mundo natural.
\vs p123 3:4 \pc Antes de que Jesús cumpliera los seis años de edad, a comienzos del verano del año 1 a. C., Zacarías, Isabel y su hijo Juan vinieron a visitar a la familia de Nazaret. Jesús y Juan disfrutaron mucho durante este encuentro, el primero del que tenían memoria. Aunque los visitantes solo podían quedarse unos días, los padres hablaron de muchas cosas, incluyendo los planes para el futuro de sus hijos. Mientras se ocupaban de esto, en la azotea de la casa los niños jugaban en la arena con trozos de madera y, como hacen los niños, se divertían de muchas distintas maneras.
\vs p123 3:5 \pc Tras conocer a Juan, que venía de las cercanías de Jerusalén, Jesús empezó a manifestar un excepcional interés por la historia de Israel y comenzó a querer informarse con gran detalle sobre el significado de los ritos del \bibemph{sabbat,} los sermones de la sinagoga y las fiestas conmemorativas periódicas. Su padre le explicó el significado de todas estas festividades estacionales. La primera era la fiesta de las Luminarias, a mediados del invierno, que duraba ocho días; la primera noche encendían una vela y cada noche sucesiva añadían una nueva. Con esto se conmemoraba la consagración del templo una vez que Judas Macabeo restaurara los oficios mosaicos. A continuación venía la celebración de Purim, a principios de la primavera, la fiesta de Esther y de la liberación de Israel gracias a ella. Le seguía la solemne Pascua, que los adultos celebraban en Jerusalén siempre que fuese posible, entretanto en el hogar los niños recordaban que no se podía comer pan con levadura en toda la semana. Luego venía la fiesta de los Primeros Frutos, la recogida de la cosecha; y, por último, la más solemne de todas, la fiesta del Año Nuevo, el día de la expiación. Aunque algunas de estas celebraciones y prácticas eran difíciles de comprender para la joven mente de Jesús; las examinaba seriamente y luego participaba plenamente y con alegría en la fiesta de los Tabernáculos, el período de las vacaciones anuales de todo el pueblo judío, la época en la que acampaban en casetas hechas de ramas y se entregaban al júbilo y al placer.
\vs p123 3:6 \pc Durante este año, José y María tuvieron dificultades con Jesús acerca de sus oraciones. Persistía en dirigirse a su Padre celestial como si estuviera hablando con José, su padre terrenal. Este alejamiento de las formas más solemnes y reverentes de comunicación con la Deidad resultaba algo desconcertante para sus padres, en especial para su madre, pero no podían persuadirlo de que cambiara; decía sus oraciones tal como le habían enseñado, tras lo cual insistía en tener “una pequeña charla con mi Padre que está en los cielos”.
\vs p123 3:7 En junio de este año, José cedió el taller de Nazaret a sus hermanos y comenzó formalmente a trabajar como constructor. Antes de finalizar el año, los ingresos de la familia se habían más que triplicado. Nunca hasta después de la muerte de José, conocería de nuevo la familia de Nazaret la acuciante pobreza. La familia creció cada vez más y gastaron mucho dinero en educación adicional y en viajes, pero los ingresos en aumento de José siempre se mantenían a la altura de los crecientes gastos.
\vs p123 3:8 Durante los años que siguieron, José hizo muchos trabajos en Caná, Belén (de Galilea), Magdala, Naín, Séforis, Cafarnaúm y Endor, al igual que muchas construcciones en Nazaret y sus alrededores. Como Santiago había crecido lo suficiente como para ayudar a su madre en los quehaceres domésticos y en el cuidado de los niños más pequeños, Jesús a menudo hacia viajes con su padre a estas ciudades y aldeas vecinas. Jesús era un observador penetrante y adquirió muchos conocimientos prácticos en estos viajes fuera de su hogar; acumulaba diligentemente conocimientos relacionados con el hombre y su modo de vivir en la tierra.
\vs p123 3:9 \pc Ese año, Jesús hizo grandes progresos para adaptar sus fuertes sentimientos e impulsos a las exigencias de la cooperación familiar y de la disciplina del hogar. María era una madre amorosa, pero bastante estricta en cuanto a la disciplina. En muchos aspectos, sin embargo, era José quien ejercía la mayor ascendencia sobre Jesús, porque solía sentarse con el muchacho para explicarle íntegramente los verdaderos y fundamentales motivos por los que era necesario disciplinar y coartar los deseos personales para contribuir al bienestar y la tranquilidad de toda la familia. Cuando se le explicaba la situación, Jesús siempre cooperaba inteligentemente y de buen grado con los deseos paternos y las reglas familiares.
\vs p123 3:10 \pc Cuando su madre no precisaba su ayuda en la casa, Jesús pasaba una gran parte de su tiempo libre en estudiar las flores y las plantas durante el día y las estrellas por la noche. Tenía la problemática afición de tenderse de espaldas y contemplar con admiración los cielos estrellados mucho después de la hora habitual de acostarse en esta bien organizada casa de Nazaret.
\usection{4. SU SÉPTIMO AÑO (AÑO 1 D. C.)}
\vs p123 4:1 Este fue, de hecho, un azaroso año en la vida de Jesús. A comienzos de enero, hubo una gran tormenta de nieve en Galilea. La nieve se acumuló hasta casi los setenta centímetros de espesor; fue la nevada más grande que Jesús había visto en toda su vida y una de las más importantes ocurridas en Nazaret en los últimos cien años.
\vs p123 4:2 Las distracciones de los niños judíos en los tiempos de Jesús eran bastante limitadas; demasiado a menudo, jugaban a las cosas más serias que observaban en sus mayores. Jugaban mucho a las bodas y a los funerales, ceremonias que tan frecuentemente veían y que resultaban tan espectaculares. Bailaban y cantaban, pero tenían pocos juegos organizados como los que tanto gustan a los niños de épocas posteriores.
\vs p123 4:3 A Jesús, acompañado primero de un niño vecino y después de su hermano Santiago, le encantaba jugar en el rincón más alejado del taller de carpintería de la familia, donde se divertían con el serrín y los trozos de madera. Jesús siempre tenía dificultades en comprender el daño que podrían causar ciertas clases de juegos para estar prohibidos en el \bibemph{sabbat,} pero nunca dejó de amoldarse a los deseos de sus padres. Poseía una habilidad para el humor y el juego que tenían pocas posibilidades de expresión en el entorno de su época y de su generación; pero hasta la edad de catorce años, la mayor parte del tiempo estaba contento y de buen humor.
\vs p123 4:4 María poseía un palomar encima del establo que estaba contiguo a la casa, y usaban los beneficios de la venta de las palomas como fondo especial de caridad, que Jesús administraba una vez que deducía el diezmo y se lo entregaba al empleado de la sinagoga.
\vs p123 4:5 \pc Hasta ese momento, el único verdadero accidente que tuvo Jesús fue una caída por las escaleras de piedra del patio trasero que conducían al dormitorio cubierto de lona. Ocurrió en julio, durante una inesperada tormenta de arena procedente del este. Los vientos cálidos que trasportaban ráfagas de arena fina soplaban generalmente durante la estación de las lluvias, particularmente en marzo y abril. Era algo extraordinario tener en el mes de julio una tormenta así. Cuando esta se desencadenó, Jesús estaba jugando, como solía hacer, en la azotea de la casa, porque durante una gran parte de la temporada seca, aquel era su lugar acostumbrado de juego. La arena lo cegó cuando bajaba las escaleras y cayó. Después de este accidente, José construyó una balaustrada a ambos lados de la escalera.
\vs p123 4:6 De ningún modo podía haberse prevenido este accidente. No fue ninguna negligencia achacable a las criaturas intermedias, sus custodios temporales, uno primario y otro secundario, asignados a la protección del muchacho; tampoco era imputable al serafín guardián. Sencillamente no se podía haber evitado. Pero este leve accidente, ocurrido mientras que José estaba ausente en Endor, provocó en la mente de María una ansiedad tan grande que trató, de forma desaconsejable, mantener a Jesús siempre cerca de ella durante varios meses.
\vs p123 4:7 Los seres personales celestiales no interfieren arbitrariamente en los accidentes materiales, esto es, en sucesos comunes y corrientes de naturaleza física. En circunstancias ordinarias, solamente las criaturas intermedias pueden intervenir en las condiciones materiales para salvaguardar a las personas, hombres y mujeres, de destino, e incluso, en situaciones especiales, estos seres solamente pueden actuar así en obediencia a los mandatos específicos de sus superiores.
\vs p123 4:8 Y aquel no fue más que uno de tantos accidentes menores que le sobrevinieron más tarde a este joven inquisitivo e intrépido. Si os figuraseis cómo sería la niñez y la juventud ordinaria de un muchacho resuelto, podríais haceros una buena idea de la trayectoria juvenil de Jesús, y casi podríais imaginar toda la ansiedad que causó a sus padres, particularmente a su madre.
\vs p123 4:9 \pc José, el cuarto hijo de la familia de Nazaret, nació la mañana del miércoles 16 de marzo del año 1 d. C.
\usection{5. SUS AÑOS ESCOLARES EN NAZARET}
\vs p123 5:1 Jesús tenía en este momento siete años, la edad en la que los niños judíos daban comienzo a su educación formal en las escuelas de la sinagoga. Por consiguiente, en agosto de este año comenzó su agitada vida escolar en Nazaret. El muchacho ya leía, escribía y hablaba con soltura dos idiomas, el arameo y el griego. Ahora tenía ante sí la tarea de aprender a leer, escribir y hablar la lengua hebrea. Y estaba realmente deseoso de empezar la nueva vida escolar que tenía por delante.
\vs p123 5:2 Durante tres años ---hasta que cumplió los diez años--- asistió a la escuela elemental de la sinagoga de Nazaret en la que estudió los rudimentos del Libro de la Ley, tal como estaba redactado en la lengua hebrea. Y durante los tres años siguientes acudió a la escuela superior, teniendo que memorizar, por el método de repetición en voz alta, las enseñanzas más profundas de la ley sagrada. Se graduó de esta escuela de la sinagoga durante su décimo tercer año, y los rectores de la sinagoga se lo entregaron a sus padres educado como un “hijo del mandamiento”. De ahí en adelante, sería un ciudadano responsable de la comunidad de Israel, lo que comportaba su asistencia a la Pascua en Jerusalén, como así hizo, participando ese año en su primera Pascua, en la compañía de su padre y su madre.
\vs p123 5:3 \pc En Nazaret, los alumnos se sentaban en el suelo formando un semicírculo, mientras que su profesor, el jazán, un funcionario de la sinagoga, se sentaba frente a ellos. Empezaban por el Libro del Levítico; pasaban luego al estudio de los demás libros de la ley, siguiéndoles el estudio de los Profetas y de los Salmos. La sinagoga de Nazaret disponía de un ejemplar completo de las Escrituras en hebreo. Antes de los doce años solo se estudiaban las Escrituras. En los meses de verano, las horas escolares se acortaban de forma considerable.
\vs p123 5:4 Jesús se convirtió pronto en un experto en hebreo y, siendo un hombre joven, cuando sucedía que ningún visitante de relieve se encontraba en Nazaret, se le pedía a él a menudo que leyera las escrituras hebreas a los fieles que se congregaban en la sinagoga para los servicios regulares del \bibemph{sabbat}.
\vs p123 5:5 Naturalmente, estas escuelas de la sinagoga carecían de libros de texto. Para enseñar, el jazán hacía comentarios que los alumnos repetían al unísono después de él. Cuando tenían acceso a los libros escritos de la ley, los estudiantes aprendían su lección leyendo en voz alta y repitiendo constantemente.
\vs p123 5:6 \pc Después, además de la escolarización formal, Jesús empezó a tomar contacto con la naturaleza humana de todos los rincones del mundo, ya que por la tienda de reparaciones de su padre pasaban hombres de numerosos lugares. Cuando era mayor se mezclaba libremente con las caravanas que se detenían cerca de la fuente para descansar y comer. Como hablaba el griego con fluidez, tenía pocos problemas para conversar con la mayoría de los viajeros y conductores de las caravanas.
\vs p123 5:7 Nazaret era una estación de paso de caravanas y un cruce de caminos; una gran parte de su población era gentil. Al mismo tiempo, Nazaret era bien conocida como centro de interpretación liberal de la ley tradicional judía. En Galilea, los judíos se mezclaban más libremente con los gentiles que lo hacían en Judea. De todas las ciudades de Galilea, los judíos de Nazaret eran los más liberales en cuanto a su interpretación de las restricciones sociales basadas en el miedo a contaminarse por estar en contacto con los gentiles. Y esta situación dio origen a un dicho común en Jerusalén: “¿De Nazaret puede salir algo bueno?”.
\vs p123 5:8 Jesús recibió su formación moral y su cultura espiritual principalmente en su propio hogar. La mayor parte de su educación intelectual y teológica la adquirió del jazán. Pero su verdadera educación ---esa dote de mente y corazón que lo capacitaba para la prueba real de enfrentarse a los difíciles problemas de la vida--- la consiguió mezclándose con sus semejantes. Fue esta estrecha relación con ellos ---jóvenes y viejos, judíos y gentiles---, la que le brindó la oportunidad de conocer a la raza humana. Jesús tenía una elevada formación en el sentido de que comprendía a fondo a los hombres y los amaba con devoción.
\vs p123 5:9 \pc Durante todos sus años en la sinagoga fue un estudiante brillante; tenía la gran ventaja de ser versado en tres idiomas. Con motivo de la terminación de los cursos de Jesús en su escuela, el jazán de Nazaret comentó a José que temía que él “había aprendido más de las preguntas inquisitivas de Jesús” que lo que él mismo había “sido capaz de enseñar al muchacho”.
\vs p123 5:10 Durante todo el curso de sus estudios, Jesús aprendió mucho y le sirvieron de gran inspiración los sermones regulares del \bibemph{sabbat} que se daban en la sinagoga. Era costumbre pedir a los visitantes distinguidos, que se detenían en Nazaret durante el \bibemph{sabbat,} que hablaran en la sinagoga. Conforme crecía, Jesús oía cómo exponían sus ideas muchos grandes pensadores de todo el mundo judío, al igual que otros judíos escasamente ortodoxos, ya que la sinagoga de Nazaret era un centro avanzado liberal del pensamiento y de la cultura hebrea.
\vs p123 5:11 Al ingresar en la escuela a los siete años (en aquella época los judíos acababan de promulgar una ley sobre la educación obligatoria), era costumbre que los alumnos escogieran su “texto de cumpleaños”, una especie de regla de oro que los guiaría a lo largo de sus estudios, y que a menudo tenían que desarrollar en el momento de graduarse a la edad de trece años. Jesús optó por un texto sacado del profeta Isaías: “El espíritu del Señor Dios está sobre mí, porque me ha ungido el Señor. Me ha enviado a predicar buenas noticias a los mansos, a vendar a los quebrantados de corazón, a proclamar libertad a los cautivos y a liberar a los prisioneros espirituales”.
\vs p123 5:12 \pc Nazaret era uno de los veinticuatro centros sacerdotales de la nación hebrea. Pero el clero de Galilea era más liberal en su interpretación de las leyes tradicionales que los escribas y rabinos de Judea. Y en Nazaret eran también más liberales en cuanto a la observancia del \bibemph{sabbat}. José tenía pues la costumbre de llevar a Jesús de paseo las tardes del \bibemph{sabbat;} una de sus excursiones preferidas era la subida a la alta colina cercana a su casa, desde la que podían disfrutar de una vista panorámica de toda Galilea. Al noroeste, en los días claros, podían ver la alargada cima del Monte Carmelo asomándose al mar; Jesús escuchó muchas veces a su padre contar la historia de Elías, uno de los primeros de la larga lista de profetas hebreos que reprobó a Acab y desenmascaró a los sacerdotes de Baal. Al norte, el Monte Hermón alzaba su pico nevado con un esplendor majestuoso, dominando el horizonte con sus más de 900 metros de laderas superiores que relucían con la blancura de las nieves perpetuas. A lo lejos, por el este, podían percibir el valle del Jordán y, más allá, las colinas rocosas de Moab. También hacia el sur y el este, cuando el sol brillaba sobre sus muros de mármol, podían ver las ciudades grecorromanas de la Decápolis, con sus anfiteatros y sus pretenciosos templos. Y cuando se quedaban para mirar la puesta de sol, podían divisar al oeste los barcos de vela en el lejano Mediterráneo.
\vs p123 5:13 Jesús podía observar las filas de caravanas que acudían y salían de Nazaret en cuatro direcciones y, hacia el sur, podía divisar la amplia y fértil llanura de Esdraelón, que se extendía hacia el Monte Gilboa y Samaria.
\vs p123 5:14 Cuando no escalaban las alturas para contemplar el paisaje lejano, paseaban por el campo y estudiaban la naturaleza en sus diferentes formas de manifestación de acuerdo a las estaciones. Aparte de la que había recibido en el hogar familiar, la formación más temprana de Jesús estaba relacionada con su contacto reverente y comprensivo con la naturaleza.
\vs p123 5:15 \pc Antes de cumplir los ocho años de edad, Jesús era conocido por todas las madres y mujeres jóvenes de Nazaret, que habían coincidido y hablado con él en la fuente, situada no lejos de su casa y que era uno de los centros sociales de encuentro y de comentarios de toda la ciudad. Este año, Jesús aprendió a ordeñar la vaca de la familia y a cuidar de los demás animales. Durante este año y el siguiente, también aprendió a hacer queso y a tejer. Cuando cumplió los diez años era un experto tejedor. Sobre esta época, Jesús y Jacob, el muchacho vecino, se convirtieron en grandes amigos del alfarero que trabajaba cerca de la fuente; y mientras observaban los hábiles dedos de Natán moldeando la arcilla en el torno, muchas veces ambos estaban determinados a ser alfareros cuando crecieran. Natán sentía un gran cariño hacia los muchachos y a menudo les daba arcilla para que jugaran, procurando estimular su imaginación creativa animándoles a que compitieran en la modelación de objetos y animales diversos.
\usection{6. SU OCTAVO AÑO (AÑO 2 D. C.)}
\vs p123 6:1 Este fue un año interesante en la escuela. Aunque Jesús no era un estudiante extraordinario, sí era un alumno perseverante y pertenecía al tercio más avanzado de la clase; realizaba tan bien sus tareas que durante una semana al mes se le eximía de la asistencia a la escuela. Dicha semana la pasaba por lo general con su tío el pescador en las orillas del mar de Galilea, cerca de Magdala, o en la granja de otro tío suyo (hermano de su madre) a unos ocho kilómetros al sur de Nazaret.
\vs p123 6:2 Aunque su madre se preocupaba excesivamente por su salud y su seguridad, paulatinamente se fue resignando a estos viajes fuera del hogar. Los tíos y las tías de Jesús le tenían mucho cariño, y se originó una viva rivalidad entre ellos para conseguir su compañía en estas visitas mensuales durante todo este año y los años siguientes. Su primera estancia de una semana (desde la infancia) en la granja de su tío fue en enero de este año; su primera semana de experiencia con la pesca en el mar de Galilea ocurrió en el mes de mayo.
\vs p123 6:3 Por esta época, Jesús conoció a un profesor de matemáticas de Damasco y, una vez que aprendió algunas nuevas técnicas numéricas, durante varios años le dedicó mucho tiempo a las matemáticas. Desarrolló un agudo sentido de los números, de las distancias y de las proporciones.
\vs p123 6:4 Jesús empezó a disfrutar mucho con su hermano Santiago y, al final de este año, había comenzado a enseñarle el alfabeto.
\vs p123 6:5 Este año Jesús tomó medidas para intercambiar productos lácteos por clases de arpa. Sentía una excepcional propensión hacia todo lo musical. Más adelante, contribuyó mucho a promover el interés por la música vocal entre sus jóvenes compañeros. A la edad de once años, ya era un arpista capaz, que disfrutaba mucho entreteniendo a la familia y a los amigos con sus extraordinarias interpretaciones y hábiles improvisaciones.
\vs p123 6:6 Aunque Jesús continuaba haciendo excelentes progresos en la escuela, no todo marchaba fácilmente para sus padres o sus maestros. Persistía en hacer muchas preguntas embarazosas tanto sobre la ciencia como sobre la religión, particularmente en relación a la geografía y a la astronomía. Hacía especialmente hincapié en averiguar por qué había una temporada seca y una temporada de lluvias en Palestina. Repetidas veces buscó la explicación de la gran diferencia entre las temperaturas de Nazaret y las del valle del Jordán. Sencillamente, nunca dejaba de hacer este tipo de preguntas, inteligentes pero desconcertantes.
\vs p123 6:7 \pc Su tercer hermano, Simón, nació la noche del viernes14 de abril de este año, el 2 d. C.
\vs p123 6:8 \pc En febrero vino a Nazaret, Nacor, uno de los maestros de una academia rabínica de Jerusalén, para observar a Jesús, tras haber llevado a cabo una misión similar en casa de Zacarías, cerca de Jerusalén. Lo hizo a instancias del padre de Juan. Aunque al principio le sorprendió algo la franqueza de Jesús y su forma nada convencional de relacionarse con las cosas religiosas, lo atribuyó a la lejanía de Galilea de los centros de enseñanza y de cultura hebreas, y recomendó a José y a María que le permitieran llevarse de vuelta con él a Jerusalén a Jesús. Allí se podría aprovechar de las ventajas de la educación y de la formación en el centro de la cultura judía. María estaba casi disuadida a dar su consentimiento; estaba convencida de que su hijo mayor iba a ser el Mesías, el libertador de los judíos. José dudaba; él también estaba convencido de que cuando Jesús creciera sería un hombre de destino, pero sentía una gran inseguridad en cuanto a lo que ese destino depararía. Si bien, nunca dudó realmente de que su hijo tenía una gran misión que cumplir en la tierra. Cuanto más pensaba en el consejo de Nacor, más cuestionaba la conveniencia de aquella propuesta de estancia en Jerusalén.
\vs p123 6:9 Debido a esta diferencia de opinión entre José y María, Nacor solicitó permiso para exponer todo el asunto a Jesús. Jesús lo escuchó con atención y habló con José, con María y con un vecino, Jacob el albañil, cuyo hijo era su compañero preferido de juegos. Y entonces, dos días más tarde, le informó que había tales diferencias de opinión entre sus padres y los consejeros y, puesto que no estaba capacitado para asumir la responsabilidad de tal decisión ni se sentía él mismo fuertemente inclinado en un sentido o en otro, había contemplado toda la situación y había decidido finalmente “hablar con mi Padre que está en el cielo”. Y, aunque no estaba completamente seguro de su respuesta, pensaba que debía más bien quedarse en casa “con mi padre y mi madre”, añadiendo: “los que tanto me quieren serán capaces de hacer más por mí y de guiarme con mayor seguridad que personas ajenas que solo pueden ver mi cuerpo y observar mi mente, pero que difícilmente pueden conocerme de verdad”. Todos se quedaron maravillados, y Nacor se puso en camino de regreso a Jerusalén. Y pasarían muchos años antes de volver a considerarse la posibilidad de que Jesús se ausentara de su hogar.
