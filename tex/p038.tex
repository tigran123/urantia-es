\upaper{38}{Los espíritus servidores del universo local}
\author{Melquisedec}
\vs p038 0:1 Existen tres órdenes diferentes de seres personales del Espíritu Infinito. El impetuoso apóstol lo entendió cuando escribió de Jesús: “Quien habiendo subido al cielo está a la diestra de Dios; y a él están sujetos ángeles y autoridades y potestades”. Los ángeles son los espíritus servidores del tiempo; las autoridades, las multitudes de mensajeros del espacio; las potestades, los seres personales superiores del Espíritu Infinito.
\vs p038 0:2 \pc Del mismo modo que los supernafines en el universo central y los seconafines en los suprauniversos, los serafines, con sus colaboradores los querubines y sanobines, constituyen el colectivo angélico de los universos locales.
\vs p038 0:3 Los serafines son bastante uniformes en cuanto a su configuración. De universo en universo, a través de los siete suprauniversos, manifiestan un mínimo de variaciones; de todas las otras clases espirituales de seres personales, son las que más se acercan a un modelo tipo. Sus diversos órdenes componen un colectivo de hábiles servidores, habituales de las creaciones locales.
\usection{1. EL ORIGEN DE LOS SERAFINES}
\vs p038 1:1 Los serafines son creación del espíritu materno del universo y se conciben conformando unidades ---41\,472 a la vez--- desde la creación de los “ángeles que son modelos originales” y de ciertos arquetipos angélicos en los primeros tiempos de Nebadón. El hijo creador y la representación del Espíritu Infinito en el universo colaboran en la creación de un gran número de hijos y de otros seres personales del universo. Una vez que se completa esta tarea conjunta, el hijo creador se ocupa seguidamente de la creación de los hijos materiales, las primeras de las criaturas sexuadas, mientras que el espíritu materno del universo, de forma simultánea, se encarga de iniciar en solitario su labor de reproducción espiritual. Así comienza la creación de las multitudes seráficas de un universo local.
\vs p038 1:2 Estos órdenes angélicos se conciben en el momento en el que se planifica la evolución de las criaturas volitivas mortales. La creación de los serafines data del tiempo en el que el espíritu materno logró un estado personal relativo, no el que tendría después como coigual del hijo mayor, sino como asistente creativo previo del hijo creador. Con anterioridad a este acontecimiento, los serafines de servicio en Nebadón habían sido cedidos temporalmente por un universo cercano.
\vs p038 1:3 Todavía se continúan creando serafines periódicamente. El universo de Nebadón aún está en proceso de formación y el espíritu materno del universo nunca cesa su actividad creativa en un universo que crece y se perfecciona.
\usection{2. NATURALEZAS ANGÉLICAS}
\vs p038 2:1 Los ángeles no tienen cuerpos materiales, pero son claramente seres individuales; son espíritus en su naturaleza y origen. Aunque invisibles a los mortales, ellos os perciben a vosotros tal como sois en la carne sin la ayuda de transformadores ni de traductores; comprenden intelectualmente el modo de vida de los mortales y comparten todas las emociones y sentimientos no sensuales del hombre. Aprecian y disfrutan enormemente de vuestros esfuerzos en el campo de la música, el arte y el auténtico humor. Conocen plenamente vuestras luchas morales y vuestras dificultades espirituales. Aman a los seres humanos y solo puede resultar algo bueno de vuestro tesón por comprenderlos y amarlos.
\vs p038 2:2 \pc Aunque los serafines son seres muy afectuosos y comprensivos, carecen de emociones sexuales. Estas criaturas son, en gran parte, como seréis vosotros en los mundos de las moradas, en los que “ni os casaréis ni seréis dados en casamiento sino que seréis como los ángeles que están en el cielo”. Puesto que todos los que “sean tenidos por dignos de alcanzar los mundos de las moradas ni se casan ni se dan en casamiento; porque ya no pueden morir, pues son iguales a los ángeles”. No obstante, al tratar con criaturas sexuadas como vosotros, es nuestra costumbre aludir a los seres que descienden más directamente del Padre y del Hijo como “Hijos de Dios” y referirnos a los vástagos del Espíritu como “hijas de Dios”. Por consiguiente, en planetas donde habitan estas criaturas, normalmente designamos a los ángeles con pronombres femeninos.
\vs p038 2:3 Los serafines se crean para servir tanto en el nivel espiritual como en el físico. Hay pocos aspectos de la actividad morontial o espiritual que no sean susceptibles a sus servicios. Aunque los ángeles, en cuanto a su estatus personal, no están tan lejos de los seres humanos, en el desempeño de ciertas funciones, los serafines los trascienden en mucho. Poseen muchas facultades que sobrepasan la comprensión humana. Por ejemplo: se os ha dicho que “hasta los cabellos de vuestra cabeza están todos contados”, y en verdad es así, pero un serafín no pasa el tiempo contándolos y actualizando las cantidades. Los ángeles poseen facultades consustanciales y automáticas (es decir, automáticas en cuanto a vuestra percepción) de saber estas cosas; no dudaríais en considerar a los serafines como un prodigio matemático. Por ello, los serafines desempeñan con gran facilidad numerosos cometidos que representarían un enorme esfuerzo para los mortales.
\vs p038 2:4 \pc Los ángeles son superiores a vosotros en estatus espiritual, pero no son vuestros jueces ni vuestros acusadores. Sean cuales fueren vuestras faltas, “los ángeles, aunque mayores en poder y fuerza, no presentan acusación alguna contra vosotros”. Si los ángeles no juzgan a la humanidad, tampoco deberían los mortales prejuzgar a sus semejantes.
\vs p038 2:5 \pc Hacéis bien en amarlos, pero no debéis adorarlos; los ángeles no son objetos de adoración. El gran serafín, Loyalatia, cuando vuestro profeta “se postró a los pies del ángel para adorarlo”, dijo: “No, cuidado; yo soy consiervo tuyo y de tu raza; todos tenemos que adorar a Dios”.
\vs p038 2:6 En cuanto a su naturaleza y al ser personal del que están dotados, los serafines están algo por delante de las razas mortales en la escala de la existencia creatural. De hecho, cuando os liberáis de la carne, os volvéis muy parecidos a ellos. En los mundos de las moradas empezaréis a valorar a los serafines, en las esferas de la constelación disfrutaréis de ellos, mientras que en Lugar de Salvación compartirán sus sitios de descanso y adoración con vosotros. A lo largo de todo vuestro ascenso morontial y, posteriormente, espiritual, vuestra fraternidad con los serafines será ideal; vuestro compañerismo será magnífico.
\usection{3. LOS ÁNGELES NO REVELADOS}
\vs p038 3:1 Hay numerosos órdenes de seres espirituales que, aunque ejercen su labor en todos los ámbitos del universo local, no se revelan a los mortales por no tener conexión alguna con el plan evolutivo diseñado para la ascensión al Paraíso. En este escrito, la palabra “ángel” se limita expresamente a designar a la progenie seráfica vinculada al espíritu materno del universo y que, en su mayor parte, está implicada en el funcionamiento de los planes de supervivencia de los mortales. Pero en el universo local realizan su servicio otros seis órdenes de seres afines; se trata de ángeles no revelados, sin relación concreta con dicho ámbito de actividad del universo respecto a la ascensión al Paraíso de los mortales evolutivos. A estos seis grupos de colaboradores angélicos nunca se les llama serafines; tampoco se hace referencia a ellos como espíritus servidores. En Nebadón, estos seres personales están totalmente dedicados a asuntos como los de tipo administrativo entre otros, que nada tienen que ver con el camino de ascenso progresivo espiritual del hombre y su logro de la perfección.
\usection{4. LOS MUNDOS SERÁFICOS}
\vs p038 4:1 El noveno grupo de las siete esferas primarias de la vía circulatoria de Lugar de Salvación forma los mundos de los serafines. Cada uno de ellos tiene seis satélites dependientes, en donde se encuentran las escuelas especiales dedicadas a todas las etapas de la formación seráfica. Aunque tienen acceso a los cuarenta y nueve mundos que componen este grupo de esferas de Lugar de Salvación, los serafines habitan exclusivamente solo el primero de los siete grupos de esferas. Los restantes seis están ocupados por los seis órdenes de colaboradores angélicos no revelados en Urantia; cada uno de estos grupos mantiene su sede central en uno de estos seis mundos primarios y desarrolla su especial labor en los seis satélites dependientes. Todos los órdenes angélicos gozan de libre acceso a la totalidad de mundos de estos siete grupos diferentes.
\vs p038 4:2 Estos mundos sedes se cuentan entre los magníficos entornos de Nebadón; las residencias seráficas se caracterizan tanto por su belleza como por su inmensidad. Aquí cada serafín tiene su verdadero hogar; “hogar” quiere decir que es el domicilio de dos serafines: viven en parejas.
\vs p038 4:3 \pc Los serafines no son ni masculinos ni femeninos como lo son los hijos materiales y las razas mortales, sino negativos y positivos. En la mayoría de las tareas que se les encargan, se requieren dos ángeles para realizarla. Cuando no permanecen conectados entre sí, pueden trabajar solos; tampoco necesitan a su ser complementario cuando están estacionarios. Por lo común, conservan a sus seres complementarios primigenios, pero no forzosamente. La funcionalidad, y no la emoción sexual, es la que los impulsa mayormente a vincularse, aunque sean sumamente personales y verdaderamente afectuosos.
\vs p038 4:4 Aparte de los hogares que se les asignan, los serafines también tienen sedes centrales de grupo, compañía, batallón y unidad. Se reúnen cada milenio y están todos presentes de acuerdo al momento en el que fueron creados. Si un serafín tiene responsabilidades que le impiden ausentarse de su puesto, alternará la asistencia con su complemento y se le reemplazará por un serafín nacido en otra fecha. De esta manera, cada compañero seráfico estará presente al menos en una reunión de cada dos.
\usection{5. FORMACIÓN DE LOS SERAFINES}
\vs p038 5:1 Los serafines pasan su primer milenio sirviendo como observadores sin nombramiento en Lugar de Salvación y en sus mundos escuela. El segundo milenio están en los mundos seráficos de la vía circulatoria de Lugar de Salvación. Su escuela central de formación está en este momento presidida por los primeros cien mil serafines de Nebadón y, a la cabeza, está el ángel primigenio o primogénito de este universo local. Un colectivo de mil serafines provenientes de Avalón formó al primer grupo de serafines que se creó en Nebadón; posteriormente, nuestros ángeles recibieron instrucción de parte de sus propios compañeros de mayor rango. Los melquisedecs también desempeñan un papel esencial en la educación y formación de todos los ángeles del universo local: los serafines, los querubines y los sanobines.
\vs p038 5:2 Al concluir este período de formación en los mundos seráficos de Lugar de Salvación, los serafines se movilizan en los grupos y unidades convencionales de la organización angélica y se les destina a una de las constelaciones. Todavía no se les nombra como espíritus servidores, aunque ya se encuentran en una etapa de instrucción que precede a dicho nombramiento.
\vs p038 5:3 Los serafines se inician como espíritus servidores actuando como observadores en los mundos evolutivos de inferior rango. Tras esta experiencia regresan a los mundos vinculados a la sede de la constelación en la que están destinados para comenzar estudios superiores y prepararse más concretamente para el servicio en algún sistema local determinado. Después de completar esta instrucción de tipo general, se les eleva de rango y entran al servicio de algún sistema local. En los mundos arquitectónicos asociados a la capital de algún sistema de Nebadón, nuestros serafines completan su formación y se les designa como espíritus servidores del tiempo.
\vs p038 5:4 Una vez que los serafines reciben su nombramiento, pueden recorrer todo Nebadón, e incluso Orvontón, en el cumplimiento de alguna misión. Su labor en el universo no tiene límites ni limitaciones; se relacionan estrechamente con las criaturas materiales de los mundos y están constantemente al servicio de los órdenes menores de seres personales espirituales, estableciendo contacto entre estos seres del mundo espiritual y los mortales de los reinos materiales.
\usection{6. ORGANIZACIÓN SERÁFICA}
\vs p038 6:1 Después del segundo milenio de permanencia en las sedes seráficas, los serafines se organizan en grupos de doce (12 pares, 24 serafines) bajo las órdenes de sus respectivos jefes, y doce de estos grupos constituyen una compañía (144 pares, 288 serafines), al mando de un líder. Doce compañías, bajo un comandante, constituyen un batallón (1728 pares o 3456 serafines), y doce batallones bajo la autoridad de un director equivalen a una unidad seráfica (20\,736 pares o 41\,472 serafines), mientras que doce unidades, sujetas a la autoridad de un supervisor, constituyen una legión compuesta de 248\,832 pares o 497\,664 serafines. Jesús aludió a un grupo de estos ángeles aquella noche en el jardín de Getsemaní cuando dijo: “Ahora mismo puedo pedírselo a mi Padre y enseguida me dará más de doce legiones de ángeles”.
\vs p038 6:2 Doce legiones de ángeles componen un cuerpo de ejército integrado por 2\,985\,984 pares o 5\,971\,968 serafines y doce de estos cuerpos (35\,831\,808 pares o 71\,663\,616 serafines) conforman la más grande estructura operativa de los serafines: un ejército angélico. Cada uno de los cuerpos seráficos está al mando de un arcángel o algún otro ser de coigual estatus, mientras que los ejércitos angélicos están bajo la dirección de las brillantes estrellas vespertinas o de lugartenientes directos de Gabriel. Y Gabriel es el “comandante supremo de los ejércitos del cielo”, el mandatario en jefe del soberano de Nebadón, “el Dios Señor de los ejércitos”.
\vs p038 6:3 Aunque sirven bajo la supervisión directa del Espíritu Infinito tal como se manifiesta personalmente en Lugar de Salvación, desde el ministerio de gracia de Miguel en Urantia, los serafines y todos los demás órdenes del universo local están bajo la soberanía del hijo mayor. Incluso cuando Miguel nació de la carne en Urantia, se emitió un comunicado desde el suprauniverso a todo Nebadón que anunciaba, “Adórenlo todos los ángeles de Dios”. Todas las jerarquías de ángeles están sujetas a su soberanía; son parte de ese grupo que se ha denominado “sus ángeles poderosos”.
\usection{7. LOS QUERUBINES Y LOS SANOBINES}
\vs p038 7:1 En todos sus atributos esenciales, los querubines y los sanobines son similares a los serafines. Comparten el mismo origen pero no siempre el mismo destino. Son asombrosamente inteligentes, maravillosamente eficaces, conmovedoramente afectuosos y casi humanos. Constituyen el orden más modesto de ángeles y son, por consiguiente, los parientes más próximos del grupo de seres humanos más progresivos de los mundos evolutivos.
\vs p038 7:2 Los querubines y sanobines están intrínsecamente relacionados, funcionalmente unidos. A nivel energético, uno de estos seres personales es positivo y, el otro, negativo. El deflector derecho, o ángel cargado positivamente, es el querubín ---el de mayor rango o el ser personal rector---. El deflector izquierdo, o ángel cargado negativamente, es el sanobín ---su ser complementario---. Cuando operan de forma solitaria, están muy limitados en cuanto a actividad; de ahí que presten sus servicios normalmente en parejas. Cuando sirven independientemente de sus directores seráficos, dependen más que nunca de su contacto mutuo y siempre actúan juntos.
\vs p038 7:3 \pc Los querubines y los sanobines son los ayudantes fieles y eficaces de los servidores seráficos, y todos los siete órdenes de serafines gozan de estos asistentes de menor rango. Los querubines y sanobines desempeñan estas funciones durante eras, pero no acompañan a los serafines cuando realizan misiones más allá de los confines del universo local.
\vs p038 7:4 Los querubines y sanobines son los laboradores espirituales rutinarios de los distintos mundos de los sistemas. En una misión no personal y en casos de emergencia, pueden servir en sustitución de una pareja seráfica, pero no pueden actuar, ni siquiera temporalmente, como ángeles acompañantes de los seres humanos; ese es un privilegio exclusivamente seráfico.
\vs p038 7:5 \pc Cuando se les destina a un planeta, los querubines realizan cursos locales de formación que incluyen un estudio de las costumbres y de los idiomas planetarios. Los espíritus servidores del tiempo son todos bilingües: hablan el idioma de su universo local de origen y el del suprauniverso nativo, si bien, al estudiar en estas escuelas de los mundos adquieren otras lenguas. Los querubines y los sanobines, como los serafines y todos los demás órdenes de seres espirituales, realizan esfuerzos continuos para mejorarse a sí mismos. Únicamente esos seres de menor rango que controlan la potencia y dirigen la energía se muestran incapaces de progresar. Todas las criaturas personales poseen una volición actual o potencial que les hace procurar nuevos logros.
\vs p038 7:6 \pc Los querubines y sanobines están por naturaleza muy cerca del nivel morontial de existencia y resultan ser sumamente eficaces en el trabajo que realizan en las zonas fronterizas entre los ámbitos físico, morontial y espiritual. Estos hijos del espíritu materno del universo local se caracterizan por ser “cuartas criaturas” como lo son los servitales de Havona y las comisiones conciliadoras. Cada cuarto querubín y cada cuarto sanobín son cuasi materiales; su existencia está claramente próxima a la del nivel morontial.
\vs p038 7:7 Estas cuartas criaturas angélicas representan una gran ayuda para los serafines en los aspectos más materiales de la actividad que estos llevan a cabo en los planetas y en el universo. Dichos querubines morontiales desempeñan igualmente muchas tareas limítrofes, indispensables en los mundos morontiales de formación, y, en gran número, se les asigna al servicio de los acompañantes morontiales. Son para las esferas morontiales prácticamente lo que las criaturas intermedias son para los planetas evolutivos. En los mundos habitados, estos querubines morontiales, con frecuencia, operan en conjunción con las criaturas intermedias. Los querubines y las criaturas intermedias son órdenes de seres claramente distintos; tienen orígenes diferentes, pero revelan una gran similitud en cuanto a su naturaleza y labor.
\usection{8. EVOLUCIÓN DE LOS QUERUBINES Y SANOBINES}
\vs p038 8:1 Hay numerosos caminos disponibles para que los querubines y sanobines puedan realizar servicios, de cada vez mayor prestancia, conducentes al enaltecimiento de su estatus, que puede incluso engrandecerse más por el acogimiento de la benefactora divina. Respecto a su potencial evolutivo, existen tres grandes clases de querubines y sanobines:
\vs p038 8:2 \li{1.}\bibemph{Los aspirantes a la ascensión}. Estos seres son por naturaleza candidatos al estatus seráfico. Los querubines y sanobines de este orden son brillantes, aunque por sus inherentes atributos no son iguales a los serafines; no obstante gracias a su dedicación y experiencia les es posible alcanzar la plena condición seráfica.
\vs p038 8:3 \li{2.}\bibemph{Los querubines de la fase intermedia}. No todos los querubines y sanobines son iguales en cuanto a su potencial de ascensión y, de las creaciones angélicas, estos son los que, por sus inherentes características, tienen límites en este sentido. La mayoría de ellos continuará siendo querubines y sanobines, aunque los más dotados podrán alcanzar un limitado grado seráfico de servicio.
\vs p038 8:4 \li{3.}\bibemph{Los querubines morontiales}. Estas “cuartas criaturas” de los órdenes angélicos mantendrán siempre sus características como seres cuasi materiales. Continuarán siendo querubines y sanobines, junto con una mayoría de sus hermanos de la fase intermedia, a la espera de la completa efectuación del Ser Supremo.
\vs p038 8:5 \pc Aunque los querubines y sanobines del segundo y del tercer grupo están algo limitados en cuanto a su potencial de crecimiento, los aspirantes a la ascensión sí pueden alcanzar las alturas del ministerio seráfico universal. De entre ellos, a los más experimentados, se les adscribe a los guardianes seráficos de destino y, de esta manera, cuando sus superiores, los serafines, han de partir sin ellos, pueden directamente avanzar hasta el estatus de maestros de los mundos de las moradas. Cuando sus pupilos mortales alcanzan la vida morontial, los guardianes del destino no tienen como ayudantes a los querubines y sanobines. Y cuando se concede autorización a otros tipos de serafines evolutivos para proseguir para Lugar de Serafines y el Paraíso, ellos, al salir de los confines de Nebadón, tienen que separarse de sus antiguos subordinados. El espíritu materno del universo acoge normalmente a estos querubines y sanobines que se han quedado atrás, pudiendo alcanzar así, en su camino hacia el estatus seráfico, un nivel equivalente al de maestro de los mundos de las moradas.
\vs p038 8:6 Cuando, como maestros de los mundos de las moradas, los querubines y sanobines, ya acogidos por el espíritu materno, han servido durante largo tiempo en las esferas morontiales, desde las de rango inferior a las de rango superior, y cuando su colectivo de Lugar de Salvación excede en su dotación de miembros, la brillante estrella de la mañana convoca a estos fieles servidores de las criaturas del tiempo para que comparezcan ante su presencia. Se les toma el juramento de transformación de su ser personal; y, acto seguido, en grupos de siete mil, el espíritu materno nuevamente acoge a estos solitarios querubines y sanobines de mayor avance y experiencia. De este segundo acogimiento, emergen como plenos serafines. En lo sucesivo, se abre a estos querubines y sanobines, que han nacido de nuevo, toda la completa andadura de un serafín, con todas las posibilidades de alcanzar el Paraíso. Estos ángeles pueden ser destinados como guardianes del destino de algún mortal y, si este pupilo consigue la supervivencia, cumplirían los requisitos para poder avanzar hasta Lugar de Serafines y lograr alcanzar los siete círculos, e incluso el Paraíso y el colectivo de los finalizadores.
\usection{9. LAS CRIATURAS INTERMEDIAS}
\vs p038 9:1 Las criaturas intermedias se clasifican en tres categorías: es adecuado clasificarlas junto con los hijos ascendentes de Dios; objetivamente se agrupan con los órdenes de ciudadanía permanente, mientras que, funcionalmente, se cuentan entre los espíritus servidores del tiempo debido a su estrecha y eficaz relación con las multitudes angélicas, en su labor de servir al hombre mortal en los distintos mundos del espacio.
\vs p038 9:2 Estas singulares criaturas aparecen en la mayoría de los mundos habitados y siempre se hallan en los planetas decimales o de experimentación de la vida, como es el caso de Urantia. Los seres intermedios son de dos tipos ---primarios y secundarios--- y tienen su origen de los siguientes modos:
\vs p038 9:3 \li{1.}\bibemph{Los seres intermedios primarios,} el grupo más espiritual, constituyen un orden de seres con características uniformes que se derivan consistentemente de mortales ascendentes modificados pertenecientes a las comitivas de los príncipes planetarios. El número de criaturas intermedias primarias es siempre cincuenta mil; ningún planeta que goce de su ministerio posee un grupo más numeroso.
\vs p038 9:4 \li{2.}\bibemph{Los seres intermedios secundarios,} el grupo más material de estas criaturas, varía de forma considerable en número según sea el mundo, aunque el promedio es de unos cincuenta mil. Se derivan distintamente de los mejoradores biológicos planetarios, los adanes y las evas, o de su progenie inmediata. En los mundos evolutivos del espacio existen no menos de veinticuatro modos diferentes de crear a estas criaturas intermedias secundarias. El grupo acaecido en Urantia tuvo un carácter extraordinario y poco común.
\vs p038 9:5 \pc Ninguno de estos grupos es un accidente evolutivo; ambos constituyen elementos fundamentales de la planificación determinada anticipadamente por los arquitectos del universo y, su aparición, en los mundos evolutivos, en la coyuntura apropiada, está conforme con los diseños originales y con los planes de desarrollo de los portadores de vida supervisores.
\vs p038 9:6 Los seres intermedios primarios consiguen su energía intelectual y espiritual del mismo modo que los ángeles y son uniformes en cuanto a su estatus intelectual. Los siete espíritus asistentes de la mente no entablan contacto con ellos; tan solo el sexto y el séptimo, el espíritu de adoración y el espíritu de sabiduría, se muestran capaces de prestar su ministerio al grupo de seres intermedios secundarios.
\vs p038 9:7 Los seres intermedios secundarios se energizan físicamente según el modo adánico, se conectan a las vías espirituales según el modo seráfico y se dotan intelectualmente con el tipo de mente de transición morontial. Se dividen en cuatro tipos físicos, en siete órdenes espirituales y en doce niveles pertinentes a su reacción intelectual al ministerio conjunto de los últimos dos espíritus asistentes y de la mente morontial. Esta diversidad determina su distinta labor y sus responsabilidades planetarias.
\vs p038 9:8 Los seres intermedios primarios se asemejan más a los ángeles que a los mortales; los órdenes secundarios son mucho más parecidos a los seres humanos. Cada orden presta una inestimable asistencia al otro en la ejecución de sus numerosos cometidos planetarios. Los servidores primarios pueden conseguir enlazar y cooperar tanto con los controladores de la energía morontial y de la energía espiritual como con los encargados de las vías circulatorias de la mente. El grupo secundario puede entablar relaciones de trabajo solamente con los controladores físicos y con los operadores de las vías circulatorias materiales. Pero, puesto que cada uno de estos órdenes de seres intermedios puede establecer una sincronía perfecta en su contacto con el otro, cada uno de los grupos es, por consiguiente, capaz de conseguir la utilización práctica de toda gama de energía que se extiende desde la potencia física bruta de los mundos materiales, pasando por las etapas de transición de las energías del universo, hasta las fuerzas superiores de la realidad espiritual de los ámbitos celestiales.
\vs p038 9:9 La brecha existente entre el mundo material y el mundo espiritual se salva espiritualmente mediante la vinculación sucesiva del hombre mortal, el ser intermedio secundario, el ser intermedio primario, el querubín morontial, el querubín de la fase intermedia y el serafín. En la experiencia personal de cualquier mortal estos distintos niveles están sin duda más o menos unificados y se hacen personalmente significativos mediante la acción inadvertida y misteriosa del divino modelador del pensamiento.
\vs p038 9:10 \pc En los mundos normales, los seres intermedios primarios mantienen su servicio como cuerpo de información y como anfitriones celestiales en nombre del príncipe planetario, mientras que los secundarios continúan su cooperación con el régimen adánico de impulsar la civilización planetaria progresiva. En caso de la deserción del príncipe planetario y del fracaso del hijo material, tal como ocurrió en Urantia, las criaturas intermedias se convierten en tutelados del soberano del sistema y realizan sus servicios bajo la dirección y guía del custodio en funciones del planeta. Si bien, solamente hay otros tres mundos en Satania donde estos seres actúan como un solo grupo bajo un mando unificado, como lo hacen los servidores intermedios unidos de Urantia.
\vs p038 9:11 En los numerosos mundos del universo, la labor planetaria tanto de los seres intermedios primarios como la de los secundarios es variada y diversa, pero en los planetas típicamente normales sus cometidos son muy diferentes a los que ocupan su tiempo en las esferas aisladas como Urantia.
\vs p038 9:12 Los seres intermedios primarios son los historiadores planetarios que, desde el momento de la llegada del príncipe planetario hasta la era asentada en luz y vida, elaboran las representaciones y diseñan las descripciones de la historia planetaria para la presentación de los planetas en los mundos sedes del sistema.
\vs p038 9:13 \pc Los seres intermedios permanecen durante largos períodos en un mundo habitado, pero, si son fieles a su misión, con toda seguridad, se les reconocerá su multisecular servicio en el sostenimiento de la soberanía del hijo creador; y se les compensará debidamente por su paciente ministerio a los mortales materiales en su mundo del tiempo y del espacio. Tarde o temprano, todas las criaturas intermedias autorizadas se incorporarán al grupo de hijos ascendentes de Dios, dando comienzo merecidamente a la larga aventura de la ascensión al Paraíso en compañía de aquellos mismos mortales de origen animal, sus hermanos terrestres, a quienes tan celosamente custodiaron y tan eficazmente sirvieron durante su larga estancia planetaria.
\vsetoff
\vs p038 9:14 [Exposición de un melquisedec que actúa a instancias del jefe de las multitudes seráficas de Nebadón.]
