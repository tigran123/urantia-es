\upaper{10}{La Trinidad del Paraíso}
\author{Censor universal}
\vs p010 0:1 La Trinidad de las eternas Deidades del Paraíso facilita al Padre su evasión de la absolutidad del ser personal. En la Trinidad, la expresión ilimitada de la infinita voluntad personal de Dios está vinculada de forma perfecta a la absolutidad de la Deidad. El Hijo Eterno y los diferentes hijos de origen divino, junto con el Actor Conjunto y sus hijos del universo, liberan en efecto al Padre de las limitaciones por otra parte inherentes a la primacía, perfección, inmutabilidad, eternidad, universalidad, absolutidad e infinitud.
\vs p010 0:2 La Trinidad del Paraíso proporciona, de hecho, la plena expresión y la revelación perfecta de la naturaleza eterna de la Deidad. Los hijos estacionarios de la Trinidad proporcionan, de la misma manera, una plena y perfecta revelación de la justicia divina. La Trinidad constituye la unidad de la Deidad, y esta unidad descansa eternamente sobre los pilares absolutos de la unicidad divina de los tres seres personales primigenios, homólogos y coexistentes: el Dios Padre, el Dios Hijo y el Dios Espíritu.
\vs p010 0:3 \pc A partir de la situación presente en el círculo de la eternidad, mirando atrás, hacia el ilimitado pasado, reconocemos en los asuntos relacionados con el universo una inevitabilidad ineludible, y esa es la Trinidad del Paraíso. Estimo que la Trinidad era inevitable. Según veo el pasado, el presente y el futuro del tiempo, considero que no hay otra cosa en todo el universo de los universos que lo haya sido. El universo matriz actual, visto en retrospectiva o en perspectiva, es impensable sin la Trinidad. Ya con la Trinidad del Paraíso, podemos proponer modos alternativos e incluso múltiples de hacer todas las cosas, pero sin la Trinidad de Padre, Hijo y Espíritu somos incapaces de concebir de qué manera pudo el Infinito lograr un equiparable estado personal triple frente a la unicidad absoluta de la Deidad. Ningún otro concepto de la creación está a la altura del criterio establecido por la Trinidad para la compleción de la absolutidad que es inherente a la unidad de la Deidad sumado a la repleción de la liberación volitiva, que es inherente a la manifestación personal triple de la Deidad.
\usection{1. AUTODISTRIBUCIÓN DE LA PRIMERA FUENTE Y CENTRO}
\vs p010 1:1 Parecería que el Padre, en el pasado eterno, inaugurara un modo de establecer una profunda distribución de sí mismo. Hay en la naturaleza altruista, amante y amable del Padre Universal algo que le hace conservar para sí mismo el ejercicio de solo esos poderes y esa autoridad que parece resultarle imposible delegar u otorgar.
\vs p010 1:2 El Padre Universal siempre se ha despojado de toda parte de sí mismo que pudiera otorgar a otro creador o criatura. Ha delegado en sus hijos divinos y en los seres de inteligencia a ellos vinculados todo el poder y toda la autoridad que pudiera delegarse. Realmente transfirió a sus hijos soberanos, en sus respectivos universos, cualquier prerrogativa de orden gobernativo que fuera transferible. En los asuntos de un universo local, ha hecho a cada hijo creador un soberano tan perfecto, competente y de tanta autoridad como el Hijo Eterno es en el universo primigenio central. Entregó, en realidad confirió, con la dignidad y santidad de su ser personal, todo de sí mismo y de sus atributos, de todo lo que podía despojarse, de todos los modos, en todas las eras, en todos los lugares y a todas las personas, y en todos los universos exceptuando el de su morada central.
\vs p010 1:3 \pc El ser personal divino no es egocéntrico; distribuirse a sí mismo y compartir su ser personal caracterizan la libertad de la voluntad del yo divino. Las criaturas anhelan la colaboración con otras criaturas personales; los Creadores se sienten impulsados a compartir su divinidad con sus hijos del universo; el ser personal del Infinito se desvela como Padre Universal, que comparte la realidad de su ser y la igualdad de su yo con dos seres personales del mismo rango: el Hijo Eterno y el Actor Conjunto.
\vs p010 1:4 \pc Para conocer el ser personal y los atributos divinos del Padre siempre tendremos que depender de las revelaciones del Hijo Eterno, porque cuando se efectuó el acto conjunto de la creación, cuando la Tercera Persona de la Deidad tuvo su existencia como ser personal y puso por obra los conceptos combinados de sus padres divinos, el Padre dejó de existir como ser personal incondicionado. Con la aparición del Actor Conjunto y la materialización del núcleo central de la creación, tuvieron lugar ciertos cambios eternos. Dios se dio a sí mismo como ser personal absoluto a su Hijo Eterno. Así, el Padre otorga el “ser personal infinito” a su Hijo Unigénito, mientras que ambos otorgan el “ser personal conjunto” de su unión eterna al Espíritu Infinito.
\vs p010 1:5 Por estas y otras razones más allá del concepto de la mente finita, es sumamente difícil para la criatura humana comprender el padre\hyp{}ser personal infinito de Dios, excepto tal como este se revela universalmente en el Hijo Eterno y, con el Hijo, se lleva a efecto de manera universal en el Espíritu Infinito.
\vs p010 1:6 Puesto que los Hijos de Dios del Paraíso visitan los mundos evolutivos y, a veces, incluso moran en ellos semejando la carne mortal, y puesto que estos ministerios de gracia hacen posible que el hombre mortal realmente conozca algo de la naturaleza y del carácter del ser personal divino, las criaturas de las esferas planetarias deben, por tanto, poner atención a dichos ministerios en los que los hijos del Paraíso se dan como don para así poder obtener información fidedigna y fiable con respecto al Padre, al Hijo y al Espíritu.
\usection{2. MANIFESTACIÓN PERSONAL DE LA DEIDAD}
\vs p010 2:1 Mediante la trinitización, el Padre se despoja de ese ser personal espiritual y sin límites que es el Hijo, pero al hacerlo, se constituye en Padre de este mismo Hijo y, por ello, se inviste de la ilimitada capacidad de ser el Padre divino de todos los tipos de criaturas de voluntad inteligente posteriormente creadas, devenidas o de otro modo hechas personales. Como \bibemph{ser personal absoluto e incondicionado,} el Padre puede obrar solamente como el Hijo y con el Hijo, pero como \bibemph{Padre personal} continúa otorgando el ser personal a diversas multitudes de criaturas volitivas e inteligentes de diferentes niveles y, por siempre, mantiene relaciones personales y vínculos amorosos con esa inmensa familia de hijos del universo.
\vs p010 2:2 Después de que el Padre ha otorgado a su Hijo la plenitud de sí mismo y, cuando este acto de darse a sí mismo es completo y perfecto, de la infinita potencia y naturaleza existentes de este modo en la unión Padre\hyp{}Hijo, los eternos compañeros otorgan conjuntamente aquellas cualidades y atributos, que constituyen otro ser más como ellos mismos; y este ser personal conjunto, el Espíritu Infinito, completa la manifestación personal existencial de la Deidad.
\vs p010 2:3 El Hijo es indispensable para la paternidad de Dios. El Espíritu es indispensable para la fraternidad de la Segunda y Tercera Personas. Tres personas constituyen la mínima forma social de agrupación, pero esta es la menos importante de todas las muchas razones para creer en la inevitabilidad del Actor Conjunto.
\vs p010 2:4 \pc La Primera Fuente y Centro es el \bibemph{padre\hyp{}ser personal} infinito, la fuente ilimitada del ser personal. El Hijo Eterno es el \bibemph{ser personal\hyp{}absoluto} sin límites, ese ser divino que permanece en el tiempo y en la eternidad como la revelación perfecta de la naturaleza personal de Dios. El Espíritu Infinito es el \bibemph{ser personal conjunto,} la única consecuencia personal de la perpetua unión Padre\hyp{}Hijo.
\vs p010 2:5 \pc El ser personal de la Primera Fuente y Centro es el ser personal infinito menos el ser personal absoluto del Hijo Eterno. El ser personal de la Tercera Fuente y Centro es la consecuencia sobreañadida de la unión del Padre\hyp{}ser personal liberado y del Hijo\hyp{}ser personal absoluto.
\vs p010 2:6 \pc El Padre Universal, el Hijo Eterno y el Espíritu Infinito son personas únicas; ninguno es un duplicado; cada uno es primigenio; todos están unidos.
\vs p010 2:7 \pc Solamente el Hijo Eterno experimenta la plenitud de la relación personal divina, consciente tanto de su filiación con el Padre como de su paternidad del Espíritu y de la igualdad divina con el Padre\hyp{}antecesor y con el Espíritu\hyp{}compañero. El Padre conoce la experiencia de tener un hijo que es su igual, pero no conoce ningún antecedente ancestral. El Hijo Eterno tiene la experiencia de la filiación, reconocimiento de ascendencia personal y, al mismo tiempo, el Hijo es consciente de ser el padre conjunto del Espíritu Infinito. El Espíritu Infinito es consciente de su doble ascendencia personal, pero no es progenitor de un ser personal de igual rango a la Deidad. Con el Espíritu, se completa el ciclo existencial de la manifestación personal de la Deidad; los seres personales primarios de la Tercera Fuente y Centro son experienciales y son siete en número.
\vs p010 2:8 Yo tengo origen en la Trinidad del Paraíso. Conozco la Trinidad como Deidad unificada; sé también que el Padre, el Hijo y el Espíritu existen y actúan en sus particulares capacidades personales. Sé positivamente que no solo actúan de manera personal y colectiva, sino que también coordinan sus acciones agrupados de formas diversas, de modo que al final obran en siete capacidades diferentes singulares y plurales. Y puesto que estas siete relaciones agotan las posibilidades de tales combinaciones divinas, es inevitable que las realidades del universo aparezcan en siete variaciones de valores, contenidos y ser personal.
\usection{3. LAS TRES PERSONAS DE LA DEIDAD}
\vs p010 3:1 A pesar de que existe una sola Deidad, hay tres manifestaciones personales definidas y divinas de la Deidad. Respecto a la dotación del hombre de los modeladores divinos, el Padre dijo: “Hagamos al hombre mortal a nuestra imagen”. En las escrituras de Urantia, existen repetidas referencias a los actos y hechos de la Deidad plural, mostrando con claridad el reconocimiento de la existencia y la labor de las tres Fuentes y Centros.
\vs p010 3:2 \pc Se nos enseña que el Hijo y el Espíritu en su vinculación con la Trinidad mantienen con el Padre relaciones idénticas. En la eternidad y como Deidades sin duda lo hacen, pero en el tiempo y como seres personales ciertamente desvelan relaciones de una naturaleza muy variada. Observando a los universos desde el Paraíso, estas relaciones parecen muy similares, pero cuando se las contempla desde los ámbitos del espacio, parecen bastante diferentes.
\vs p010 3:3 Los hijos divinos son ciertamente el “Verbo de Dios”, pero los hijos del Espíritu son en verdad la “Acción de Dios”. Dios habla a través del Hijo y, con el Hijo, actúa a través del Espíritu Infinito, mientras que, en toda actividad del universo, el Hijo y el Espíritu son hermosamente fraternales, obrando como dos hermanos iguales en admiración y amor hacia un Padre común honorable y divinamente respetado.
\vs p010 3:4 El Padre, el Hijo y el Espíritu son ciertamente iguales en naturaleza, equiparables en ser, pero hay diferencias inequívocas en sus actuaciones universales y, cuando actúan solos, cada persona de la Deidad está aparentemente limitada en absolutidad.
\vs p010 3:5 \pc El Padre Universal, antes de despojarse voluntariamente de su ser personal, de los poderes y de los atributos que constituyen el Hijo y el Espíritu, parece haber sido (filosóficamente considerado) una Deidad ilimitada, absoluta e infinita. Pero esa teórica Primera Fuente y Centro sin un Hijo no podría considerarse, en ningún sentido de la palabra, el \bibemph{Padre Universal;} la paternidad no es real sin filiación. Además, el Padre, para haber sido absoluto en un sentido total, en algún momento eternamente distante, debe haber existido a solas. Pero nunca tuvo tal existencia solitaria; el Hijo y el Espíritu son ambos coeternos con el Padre. La Primera Fuente y Centro siempre ha sido y siempre será el Padre Eterno del Hijo Primigenio y, con el Hijo, el Progenitor Eterno del Espíritu Infinito.
\vs p010 3:6 Observamos que el Padre se ha despojado de todas las manifestaciones directas de su absolutidad excepto de la paternidad absoluta y de la volición absoluta. No sabemos si la volición es un atributo inalienable del Padre; solo podemos observar que \bibemph{no} se despojó de la volición. Dicha voluntad infinita debe haber sido eternamente inherente a la Primera Fuente y Centro.
\vs p010 3:7 Al otorgar el ser personal absoluto al Hijo Eterno, el Padre Universal se evade de las ataduras a la absolutidad del ser personal, pero, al hacerlo así, da un paso que le imposibilita por siempre actuar por sí solo como ser personal\hyp{}absoluto. Y con la manifestación personal final de la Deidad coexistente ---el Actor Conjunto--- sobreviene la crítica interdependencia trinitaria de los tres seres personales divinos respecto a la acción absoluta de la Deidad total.
\vs p010 3:8 Dios es el Padre\hyp{}absoluto de todos los seres personales del universo de los universos. El Padre es personalmente absoluto en libertad de acción, pero en los universos del tiempo y del espacio, ya hechos, en proceso de hacerse y aún por hacerse, el Padre no es perceptiblemente absoluto como Deidad total salvo en la Trinidad del Paraíso.
\vs p010 3:9 \pc La Primera Fuente y Centro obra fuera de Havona en los universos fenoménicos de la manera siguiente:
\vs p010 3:10 \li{1.}Como creador, a través de los hijos creadores, sus nietos.
\vs p010 3:11 \li{2.}Como rector, a través del centro de gravedad del Paraíso.
\vs p010 3:12 \li{3.}Como espíritu, a través del Hijo Eterno.
\vs p010 3:13 \li{4.}Como mente, a través del Creador Conjunto.
\vs p010 3:14 \li{5.}Como Padre, mantiene contacto paterno con todas las criaturas a través de la vía circulatoria del ser personal.
\vs p010 3:15 \li{6.}Como persona, actúa \bibemph{directamente} en toda la creación por medio de las exclusivas fracciones de sí mismo ---en el hombre mortal, mediante los modeladores del pensamiento---.
\vs p010 3:16 \li{7.}Como Deidad total, obra tan solo en la Trinidad del Paraíso.
\vs p010 3:17 \pc Todas estas renuncias y delegaciones de jurisdicción del Padre Universal son por completo voluntarias y autoimpuestas. El Padre Todopoderoso asume de forma deliberada estas limitaciones de autoridad sobre los universos.
\vs p010 3:18 \pc El Hijo Eterno parece obrar como uno con el Padre en todos los aspectos espirituales, excepto en la dádiva de las fracciones de Dios y en alguna otra actividad prepersonal. Tampoco se identifica el Hijo estrechamente con la actividad intelectual de las criaturas materiales ni con la actividad energética de los universos materiales. Como absoluto, el Hijo obra como persona solamente en el ámbito del universo espiritual.
\vs p010 3:19 \pc El Espíritu Infinito es asombrosamente universal e increíblemente versátil en todas sus acciones. Actúa en las esferas de la mente, de la materia y del espíritu. El Actor Conjunto representa la vinculación Padre\hyp{}Hijo, pero obra también por sí mismo. No está directamente relacionado con la gravedad física, la gravedad espiritual o la vía de circulación del ser personal, pero participa en mayor o menor grado en cualquier otra actividad del universo. Aunque al parecer dependiente de un triple control gravitatorio existencial y absoluto, el Espíritu Infinito parece ejercer un triple extra control. Esta triple dotación se emplea en muchos sentidos para trascender y, al parecer, para neutralizar incluso las manifestaciones de fuerzas y energías primarias, incluyendo las fronteras supraúltimas de la absolutidad. En ciertas situaciones, este extra control trasciende por completo incluso las manifestaciones primordiales de la realidad cósmica.
\usection{4. UNIÓN TRINITARIA DE LA DEIDAD}
\vs p010 4:1 De todas las vinculaciones absolutas, la Trinidad del Paraíso (la primera triunidad) es única como agrupación exclusiva de la Deidad personal. Dios obra como Dios solo en relación a Dios y a los que pueden conocer a Dios, pero como Deidad absoluta solo en la Trinidad del Paraíso y en relación con la totalidad del universo.
\vs p010 4:2 \pc La Deidad eterna está perfectamente unificada; sin embargo, hay en la Deidad tres personas perfectamente individualizadas. La Trinidad del Paraíso hace posible la expresión simultánea de toda la diversidad de rasgos del carácter y poderes infinitos de la Primera Fuente y Centro y de sus eternos iguales en rango, y de toda la unidad divina de las capacidades universales de la Deidad indivisa.
\vs p010 4:3 La Trinidad constituye la vinculación de personas infinitas que obran en una capacidad no personal, pero sin contravenir el ser personal. La ilustración es burda, pero un padre, un hijo y un nieto podrían formar una entidad corporativa que sería no personal y, sin embargo, estaría sujeta a sus voluntades personales.
\vs p010 4:4 La Trinidad del Paraíso es \bibemph{real}. Existe en la unión como Deidades del Padre, el Hijo y el Espíritu. Sin embargo, el Padre, el Hijo o el Espíritu, o alguno de ellos dos, pueden obrar en relación con esta misma Trinidad del Paraíso. El Padre, el Hijo y el Espíritu pueden colaborar de una manera no trinitaria, pero no como tres Deidades. Como personas pueden colaborar como ellos elijan, pero la Trinidad no es eso.
\vs p010 4:5 \pc Recordad siempre que lo que hace el Espíritu Infinito es labor del Actor Conjunto. Tanto el Padre como el Hijo están obrando en él, y a través de él y como él. Pero sería inútil intentar dilucidar el misterio de la Trinidad: tres como uno y en uno, y uno como dos y actuando para dos.
\vs p010 4:6 \pc La Trinidad está tan relacionada con los asuntos del universo total que debe tomarse en consideración en nuestros intentos por explicar la totalidad de cualquier suceso cósmico aislado o de relación personal. La Trinidad obra en todos los niveles del cosmos, y el hombre mortal está limitado al nivel de lo finito; por consiguiente, el hombre debe contentarse con un concepto finito de la Trinidad como Trinidad.
\vs p010 4:7 Como mortal en la carne debes contemplar la Trinidad de acuerdo con tu lucidez individual y en armonía con las reacciones de tu mente y de tu alma. Puedes saber muy poco del carácter absoluto de la Trinidad, pero según asciendas hacia el Paraíso, te asombrarás muchas veces de las revelaciones sucesivas y de los descubrimientos inesperados de la supremacía y de la ultimidad, e incluso de la absolutidad de la Trinidad.
\usection{5. FUNCIONES DE LA TRINIDAD}
\vs p010 5:1 Las Deidades personales tienen atributos, pero, para ser coherentes y hablar con mayor propiedad, la conjunción de seres divinos en la Trinidad más que atributos desempeñan \bibemph{funciones,} tales como administrar la justicia, actitudes de totalidad, acción equiparada y acción directiva cósmica. Dichas funciones son activamente supremas, últimas y (dentro de los límites de la Deidad) absolutas en lo que concierne a todas las realidades vivas de valor personal.
\vs p010 5:2 Las funciones de la Trinidad del Paraíso no son simplemente la suma de la dotación aparente de divinidad del Padre más aquellos atributos particulares que son únicos en la existencia personal del Hijo y el Espíritu. La vinculación en la Trinidad de las tres Deidades del Paraíso da como resultado la evolución, el acontecimiento y la deidización de nuevos contenidos, valores, potencias y facultades para la revelación, la acción y la administración universales. Los grupos de seres humanos, las familias humanas, los grupos sociales o la Trinidad del Paraíso no crecen por mera adición aritmética. La potencialidad del grupo excede siempre en mucho la simple suma de los atributos de sus componentes individuales.
\vs p010 5:3 \pc La Trinidad mantiene una actitud única como Trinidad hacia todo el universo del pasado, presente y futuro. Y su labor se puede apreciar mejor en relación con sus actitudes hacia el universo. Tales actitudes son simultáneas y pueden ser múltiples respecto de cualquier situación o acontecimiento aislado:
\vs p010 5:4 \li{1.}\bibemph{Actitud hacia lo finito}. La autolimitación máxima de la Trinidad es su actitud hacia lo finito. La Trinidad no es una persona ni el Ser Supremo es una manifestación personal exclusiva de la Trinidad, pero el Supremo es el que más se acerca al punto de convergencia de la potencia\hyp{}ser personal de la Trinidad comprensible para las criaturas finitas. De ahí que a veces se habla de la Trinidad en relación con lo finito como la Trinidad de la Supremacía.
\vs p010 5:5 \li{2.}\bibemph{Actitud hacia lo absonito}. La Trinidad del Paraíso se ocupa de aquellos niveles existenciales que son más que finitos pero menos que absolutos, y esta relación se denomina a veces la Trinidad de Ultimidad. Ni el Último ni el Supremo representan totalmente a la Trinidad del Paraíso pero, en un sentido condicionado y a sus respectivos niveles, cada uno parece representar a la Trinidad durante las eras prepersonales del desarrollo de la experiencia\hyp{}potencia.
\vs p010 5:6 \li{3.}\bibemph{La actitud absoluta} de la Trinidad del Paraíso está en relación con las existencias absolutas y culmina en la acción de la Deidad total.
\vs p010 5:7 \pc La Trinidad Infinita conlleva la acción correlacionada de todas las relaciones en triunidad de la Primera Fuente y Centro ---tanto deificada como no deificada--- y, en consecuencia, es muy difícil que los seres personales lleguen a comprenderlas. En la contemplación de la Trinidad como infinita, no olvidéis las siete triunidades; así se podrán evitar ciertas dificultades de entendimiento y resolverse parcialmente algunas paradojas.
\vs p010 5:8 \pc Pero no encuentro términos que me permitan transmitir a la mente humana limitada la plenitud de la verdad y la significación eterna de la Trinidad del Paraíso y la naturaleza de la vinculación mutua e interminable de los tres seres de perfección infinita.
\usection{6. LOS HIJOS ESTACIONARIOS DE LA TRINIDAD}
\vs p010 6:1 Toda ley se origina en la Primera Fuente y Centro, que \bibemph{es la ley.} La administración de la ley espiritual es intrínseca a la Segunda Fuente y Centro. La revelación de la ley, la promulgación e interpretación de los estatutos divinos, es el cometido de la Tercera Fuente y Centro. La aplicación de la ley, la justicia, entra dentro de la competencia de la Trinidad del Paraíso y se lleva a cabo por ciertos hijos de la Trinidad.
\vs p010 6:2 \pc La \bibemph{justicia} es inherente a la soberanía universal de la Trinidad del Paraíso, pero la bondad, la misericordia y la verdad representan el ministerio universal de los seres personales divinos, cuya unión como Deidad constituye la Trinidad. La justicia no es la actitud del Padre, el Hijo o el Espíritu. La justicia es la actitud trinitaria de estos tres seres personales de amor, misericordia y servicio. Ninguna de las Deidades del Paraíso promueve la administración de la justicia. La justicia no es nunca una actitud personal; es siempre una labor plural.
\vs p010 6:3 \pc La \bibemph{evidencia,} la base de la ecuanimidad (la justicia en armonía con la misericordia), la suministran los seres personales de la Tercera Fuente y Centro, el representante conjunto del Padre y del Hijo en todos los mundos y en la mente de todos los seres inteligentes de toda la creación.
\vs p010 6:4 \pc \bibemph{La decisión judicial,} la aplicación final de la justicia de acuerdo con la evidencia remitida por los seres personales del Espíritu Infinito es la tarea de los hijos estacionarios de la Trinidad, seres que comparten la naturaleza trinitaria del Padre, del Hijo y del Espíritu unidos.
\vs p010 6:5 \pc Este grupo de hijos de la Trinidad comprende los siguientes seres personales:
\vs p010 6:6 \li{1.}Los seres secretos trinitizados de supremacía.
\vs p010 6:7 \li{2.}Los eternos de días.
\vs p010 6:8 \li{3.}Los ancianos de días.
\vs p010 6:9 \li{4.}Los perfectos de días.
\vs p010 6:10 \li{5.}Los recientes de días.
\vs p010 6:11 \li{6.}Los uniones de días.
\vs p010 6:12 \li{7.}Los fieles de días.
\vs p010 6:13 \li{8.}Los perfeccionadores de la sabiduría.
\vs p010 6:14 \li{9.}Los consejeros divinos.
\vs p010 6:15 \li{10.}Los censores universales.
\vs p010 6:16 \pc Nosotros somos hijos de las tres Deidades del Paraíso en su atribución como Trinidad; sucede que pertenezco al décimo orden de este grupo, los censores universales. Estos órdenes no representan la actitud de la Trinidad en un sentido universal; representan dicha actitud, colectiva, de la Deidad solamente en los ámbitos del juicio ejecutivo: la justicia. La Trinidad los concibió específicamente para la tarea precisa que se les asigna, y representan a la Trinidad tan solo en esas funciones para las que se hicieron personales.
\vs p010 6:17 Los ancianos de días y sus colaboradores de origen en la Trinidad imparten un juicio justo y supremamente ecuánime en los siete suprauniversos. En el universo central tales funciones existen solamente en teoría; allí la ecuanimidad es patente en la perfección, y la perfección de Havona excluye toda posibilidad de desarmonía.
\vs p010 6:18 La justicia es el pensamiento colectivo de la rectitud; la misericordia es su expresión personal. La misericordia es la actitud del amor; la precisión caracteriza la operación de la ley; el juicio divino es el alma de la ecuanimidad, siempre conformándose a la justicia de la Trinidad, siempre cumpliendo el amor divino de Dios. Cuando se les percibe plenamente y se les comprende por completo, la justicia recta de la Trinidad y el amor misericordioso del Padre Universal coinciden. Pero el hombre no posee ese pleno entendimiento de la justicia divina. Así pues, en la Trinidad, tal como el hombre lo percibiría, las personas del Padre, del Hijo y del Espíritu se adaptan al ministerio equiparado del amor y de la ley en los universos experienciales del tiempo.
\usection{7. LA ACCIÓN DIRECTIVA DE LA SUPREMACÍA}
\vs p010 7:1 La Primera, Segunda y Tercera Personas de la Deidad son iguales entre sí, y son una. “El Señor nuestro Dios es un solo Dios”. Existe un propósito perfecto y una unicidad de ejecución en la Trinidad divina de las Deidades eternas. El Padre, el Hijo y el Actor Conjunto son en verdad y en divinidad uno. Ciertamente está escrito: “Yo soy el primero y yo soy el último, y fuera de mí no hay ningún Dios”.
\vs p010 7:2 \pc Según las cosas parecen para los mortales en el nivel finito, la Trinidad del Paraíso, al igual que el Ser Supremo, se ocupa solamente de lo total: del planeta total, del universo total, del suprauniverso total, del gran universo total. Esta actitud de totalidad existe porque la Trinidad es el total de la Deidad además de por otras muchas razones.
\vs p010 7:3 El Ser Supremo es algo menos que la Trinidad y algo distinto a ella, que obra en los universos finitos; pero dentro de ciertos límites y durante la era presente de la potencia\hyp{}manifestación personal incompletos, esta Deidad evolutiva parece reflejar la actitud de la Trinidad de la Supremacía. El Padre, el Hijo y el Espíritu no obran de forma personal con el Ser Supremo, pero durante la presente era del universo colaboran con él como Trinidad. Entendemos que tienen una relación similar con el Último. A menudo hacemos suposiciones sobre cuál será la relación personal entre las Deidades del Paraíso y el Dios Supremo cuando este haya finalmente evolucionado, pero realmente no lo sabemos.
\vs p010 7:4 \pc No creemos que la acción directiva de la Supremacía sea por completo previsible. Además, esta imprevisibilidad parece caracterizarse por un desarrollo incompleto, sin duda un signo de la incompletitud del Supremo y de lo inacabado de la reacción finita a la Trinidad del Paraíso.
\vs p010 7:5 La mente mortal puede pensar inmediatamente en mil y una cosas ---catástrofes físicas, accidentes espantosos, desastres horribles, enfermedades dolorosas y calamidades mundiales--- y preguntarse si tales sucesos están correlacionados con los desconocidos designios de esta probable acción del Ser Supremo. Francamente, no lo sabemos; no estamos realmente seguros. Pero sí observamos que, según transcurre el tiempo, todas estas situaciones difíciles y más o menos misteriosas tienen \bibemph{siempre} como resultado el bien y el progreso de los universos. Puede ser que las circunstancias de la existencia y las inexplicables vicisitudes de la vida se entrelacen formando un patrón significativo y de elevado valor mediante la actividad del Supremo y la acción directiva de la Trinidad.
\vs p010 7:6 Como hijo de Dios, puedes percibir la actitud personal de amor en todos los actos del Dios Padre. Pero no siempre podrás comprender cuántos de los actos de la Trinidad del Paraíso en el universo redundan individualmente en bien de los mortales de los mundos evolutivos del espacio. En el progreso de la eternidad, los actos de la Trinidad se revelarán como completamente significativos y benevolentes, pero no siempre los perciben así las criaturas del tiempo.
\usection{8. LA TRINIDAD MÁS ALLÁ DE LO FINITO}
\vs p010 8:1 Muchas verdades y hechos relativos a la Trinidad del Paraíso solo pueden comprenderse, aunque parcialmente, mediante el reconocimiento de una función que trascienda lo finito.
\vs p010 8:2 No sería aconsejable hablar de las funciones de la Trinidad de Ultimidad, pero puede revelarse que el Dios Último es la manifestación trinitaria tal como la conciben los trascendentales. Somos propensos a creer que la unificación del universo matriz es resultado de la acción del Último y con probabilidad refleja ciertas facetas, no todas, de la acción directiva absonita de la Trinidad del Paraíso. El Último constituye una manifestación condicionada de la Trinidad en relación con lo absonito solo en el sentido en que el Supremo representa, pues, parcialmente a la Trinidad en relación con lo finito.
\vs p010 8:3 \pc El Padre Universal, el Hijo Eterno y el Espíritu Infinito son en cierto modo los seres personales que constituyen la Deidad total. Su unión en la Trinidad del Paraíso y la capacidad de actuación absoluta de la Trinidad equivalen a la capacidad de actuación de la Deidad total. Y dicha compleción de la Deidad trasciende tanto lo finito como lo absonito.
\vs p010 8:4 Si bien, ninguna de las personas de las Deidades del Paraíso colma realmente todo el potencial de la Deidad, colectivamente las tres lo hacen. Tres personas infinitas parecen ser el mínimo de seres que se requiere para activar el potencial prepersonal y existencial de la Deidad total ---el Absoluto de la Deidad---.
\vs p010 8:5 Conocemos al Padre Universal, al Hijo Eterno y al Espíritu Infinito como \bibemph{personas,} pero no conozco personalmente al Absoluto de la Deidad. Amo y adoro al Dios Padre; respeto y honro al Absoluto de la Deidad.
\vs p010 8:6 \pc Una vez estuve en un universo donde cierto grupo de seres impartían la enseñanza de que los finalizadores, en la eternidad, acabarían por llegar a ser los hijos del Absoluto de la Deidad. Pero soy reticente a aceptar tal solución al misterio que envuelve el futuro de los finalizadores.
\vs p010 8:7 El colectivo final incluye, entre otros, a aquellos mortales del tiempo y del espacio que han alcanzado la perfección en todo lo que se refiere a la voluntad de Dios. Como criaturas y dentro de los límites de su capacidad creatural, conocen plena y verdaderamente a Dios. Habiendo encontrado así a Dios como Padre de todas las criaturas, estos finalizadores deben comenzar, en algún momento, la búsqueda del Padre suprafinito. Pero esta búsqueda conlleva la aprehensión de la naturaleza absonita de los atributos y del carácter últimos del Padre del Paraíso. La eternidad desvelará si alcanzar tal cosa es posible, pero estamos convencidos, incluso si los finalizadores realmente alcanzan esta ultimidad de la divinidad, de que serán probablemente incapaces de alcanzar los niveles supraúltimos de la Deidad Absoluta.
\vs p010 8:8 Puede ser posible que los finalizadores alcancen parcialmente el Absoluto de la Deidad, pero incluso si lo hicieran, en la eternidad de eternidades el problema del Absoluto Universal todavía continuaría intrigando, desconcertando, confundiendo y desafiando a estos en su camino de ascenso y progreso, porque percibimos que la insondabilidad de las relaciones cósmicas del Absoluto Universal tenderá a crecer proporcionalmente conforme los universos materiales y su gobierno espiritual continúen su expansión.
\vs p010 8:9 \pc Solo la infinitud puede desvelar al Padre\hyp{}Infinito.
\vsetoff
\vs p010 8:10 [Auspiciado por un censor universal bajo la autoridad de los ancianos de días residentes en Uversa.]
