\upaper{139}{Los doce apóstoles}
\author{Comisión de seres intermedios}
\vs p139 0:1 Un testimonio elocuente del carisma y de la rectitud de la vida de Jesús mientras estaba en la tierra es el hecho de que, aunque hizo añicos las esperanzas de sus apóstoles y cada una de sus pretensiones de exaltación personal, solo uno de ellos lo abandonó.
\vs p139 0:2 Los apóstoles aprendieron de Jesús sobre el reino de los cielos, y Jesús aprendió mucho de ellos sobre el reino de los hombres, sobre la naturaleza humana tal como se vive en Urantia y en otros mundos evolutivos del tiempo y el espacio. Estos doce hombres representaban tipos diferentes de temperamentos humanos, y no se habían vuelto \bibemph{iguales} por su instrucción escolar. Muchos de estos pescadores galileos portaban importantes linajes de sangre gentil como resultado de la conversión forzosa de la población gentil de Galilea, cien años antes.
\vs p139 0:3 \pc No cometáis el error de creer que los apóstoles eran totalmente ignorantes e iletrados. Todos ellos, salvo los gemelos Alfeo, eran graduados de las escuelas de la sinagoga, se habían formado concienzudamente en el estudio de las escrituras hebreas y habían adquirido una gran parte del conocimiento imperante en aquellos días. Siete de ellos se habían graduado en las escuelas de la sinagoga de Cafarnaúm, los mejores centros de instrucción judíos de toda Galilea.
\vs p139 0:4 Cuando vuestros escritos aluden a estos mensajeros del reino como seres “sin letras y del vulgo”, querían transmitir la idea de laicos, iletrados en las tradiciones de los rabinos e inexpertos en los métodos de la interpretación rabínica de las Escrituras. Ellos carecían de la denominada educación superior. Sin duda, en tiempos modernos se les hubiese considerado personas sin formación y, en algunos círculos sociales, incluso incultos. Algo es realmente cierto: no todos se habían sometido al mismo programa de estudios rígido y estereotipado. A partir de su adolescencia, habían gozado de experiencias por separado en su aprendizaje de la vida.
\usection{1. ANDRÉS, EL PRIMER ELEGIDO}
\vs p139 1:1 Andrés, presidente del cuerpo apostólico del reino, nació en Cafarnaúm. Era el hijo mayor de una familia de cinco ---él, su hermano Simón y tres hermanas---. Su padre, ya fallecido, había sido socio de Zebedeo en el negocio de secado de pescado en Betsaida, el puerto pesquero de Cafarnaúm. Cuando se convirtió en apóstol, Andrés era soltero pero vivía con su hermano Simón Pedro, que estaba casado. Los dos eran pescadores y socios de Santiago y Juan, los hijos de Zebedeo.
\vs p139 1:2 En el año 26 d. C., el año en el que se le eligió apóstol, Andrés tenía 33 años; era un año completo mayor que Jesús y el mayor de los apóstoles. Procedía de una excelente línea de ancestros y era el hombre más capaz de los doce. Excepto en oratoria, estaba a la par de sus compañeros casi en cualquier imaginable habilidad. Jesús no le puso ningún sobrenombre a Andrés, ningún apelativo fraternal. Pero así como los apóstoles empezaron pronto a llamar Maestro a Jesús, lo harían también con Andrés usando un término equivalente a “jefe”.
\vs p139 1:3 \pc Andrés tenía buenas dotes para organizar, pero incluso era mejor en la gestión de los asuntos apostólicos. Era uno de los cuatro apóstoles del círculo más cercano a Jesús, pero cuando él lo nombró jefe del grupo apostólico, se sintió obligado a estar de servicio con sus hermanos, mientras los otros tres disfrutaban de muy estrecha comunión con el Maestro. Hasta el último momento, Andrés continuó ejerciendo como decano del cuerpo apostólico.
\vs p139 1:4 Aunque Andrés nunca fue un predicador idóneo, sí era un eficiente trabajador a nivel personal; fue el misionero precursor del reino ya que, al ser el primero en ser elegido, llevó de inmediato a Jesús a su hermano Simón, que llegaría a convertirse en uno de los más grandes predicadores del reino. Andrés fue el principal defensor de la política de Jesús respecto a fomentar la labor personal para formar a los doce como mensajeros del reino.
\vs p139 1:5 Aunque Jesús enseñara privadamente a los apóstoles o predicara a las multitudes, Andrés era por lo general conocedor de lo que ocurría; era un jefe comprensivo y un eficiente gestor. Tomaba decisiones rápidas en cualquier asunto que se le planteara, a no ser que estimara que el problema estaba más allá del ámbito de su autoridad, en cuyo caso lo consultaba directamente con Jesús.
\vs p139 1:6 \pc Andrés y Pedro eran muy diferentes en cuanto a carácter y temperamento, pero hay que constatar sempiternamente en su favor que se llevaban espléndidamente bien entre ellos. Andrés jamás tuvo celos de la facultad oratoria de Pedro. Es poco frecuente que un hermano mayor, como lo era Andrés de Pedro, ejerciera tan profunda influencia sobre un hermano más joven y talentoso. Andrés y Pedro nunca parecían sentir el más mínimo celo de las habilidades o logros del otro. Avanzada la noche del día de Pentecostés, cuando, en gran parte debido a la predicación vigorosa y edificante de Pedro, se añadieron al reino dos mil almas, Andrés le dijo a su hermano: “Yo no hubiese podido hacerlo, pero me congratulo de tener un hermano que sí ha podido”. A lo que Pedro contestó: “Y si no fuese porque me hubieses llevado al Maestro y por tu determinación en \bibemph{mantenerme} a su lado, jamás habría estado aquí para lograrlo”. Andrés y Pedro eran la excepción a la regla, demostrando que incluso los hermanos pueden convivir en calma y trabajar juntos con eficacia.
\vs p139 1:7 Tras Pentecostés, Pedro se hizo famoso, pero nunca irritó a Andrés, su hermano mayor, que durante el resto de su vida se aludiese a él como “el hermano de Simón Pedro”.
\vs p139 1:8 \pc De todos los apóstoles, Andrés era quien mejor sabía juzgar a las personas. Se percató de que en el corazón de Judas Iscariote se estaban gestando problemas incluso cuando nadie más sospechara de que algo no iba bien con el tesorero; pero no comentó a nadie sus temores. El gran servicio que Andrés prestó al reino fue aconsejar a Pedro, Santiago y Juan respecto a la elección de los primeros misioneros enviados a proclamar el evangelio, y también en orientar a estos primeros líderes sobre la organización y funcionamiento del reino. Andrés tenía un gran don para descubrir en la gente joven sus recursos ocultos y sus talentos latentes.
\vs p139 1:9 Muy poco después de que Jesús hubiese ascendido a las alturas, Andrés comenzó a hacer anotaciones personales de muchos de los dichos y hechos de su Maestro. Tras la muerte de Andrés, se hicieron otras copias de estos apuntes privados, las cuales circularon libremente entre los primeros maestros de la Iglesia cristiana. Posteriormente, estas notas informales de Andrés se revisaron, enmendaron, alteraron y aumentaron hasta conformar una narrativa, de carácter bastante cronológico, sobre la vida del Maestro en la tierra. La última de estas copias, alteradas y enmendadas, se destruyó en un incendio ocurrido en Alejandría, unos cien años después de que el primer apóstol elegido de los doce redactara el original.
\vs p139 1:10 Andrés era un hombre de percepción clara, pensamiento lógico y firme decisión, cuya gran fuerza de carácter consistía en su formidable estabilidad. El mayor defecto de su temperamento era su falta de entusiasmo; muchas veces no conseguía alentar a sus compañeros con encomios sensatos. Y esta reticencia a alabar los logros merecidos de sus amigos provenía de su aversión al elogio y la insinceridad. Andrés era un hombre íntegro, sereno, hecho a sí mismo y modestamente exitoso.
\vs p139 1:11 \pc Todos los apóstoles amaban a Jesús, pero es cierto que cada uno de los doce se sentía atraído hacia él por algún determinado rasgo de su persona que a ese apóstol le llamara particularmente la atención. Andrés admiraba a Jesús por su constante franqueza, su dignidad sin afectación. Una vez que los hombres conocían a Jesús, sentían el impulso de compartir esta experiencia con todos sus amigos; deseaban intensamente que todo el mundo lo conociera.
\vs p139 1:12 \pc Cuando las últimas persecuciones acabaron por dispersar a los apóstoles y a alejarlos de Jerusalén, Andrés viajó por Armenia, Asia Menor y Macedonia y, después de conducir al reino a muchos miles de almas, fue finalmente apresado y crucificado en Patras, en Acaya. Este robusto hombre estuvo dos días enteros crucificado antes de expirar y, en esas horas trágicas, tuvo fuerzas para continuar proclamando la buena nueva de la salvación del reino de los cielos.
\usection{2. SIMÓN PEDRO}
\vs p139 2:1 Cuando Simón se unió a los apóstoles, tenía treinta años. Estaba casado, tenía tres hijos y vivía en Betsaida, cerca de Cafarnaúm. Andrés, su hermano, y su suegra vivían con él. Pedro y Andrés eran pescadores y socios de los hijos de Zebedeo.
\vs p139 2:2 El Maestro había conocido a Simón hacía algún tiempo antes de que Andrés se lo presentara como el segundo de los apóstoles. Cuando Jesús le puso a Simón el nombre de Pedro, lo hizo con una sonrisa; era una especie de sobrenombre. Simón era bien conocido entre sus amigos como un hombre imprevisible e impulsivo. Bien es verdad que, más tarde, Jesús atribuyó un sentido nuevo y significativo a este sobrenombre, puesto a la ligera.
\vs p139 2:3 \pc Simón Pedro era un hombre impulsivo, un optimista. Había crecido sintiéndose con la libertad de expresar sus fuertes sentimientos; tropezaba constantemente con dificultades por su insistencia en hablar sin pensar. Este tipo de irreflexión también originaba constantes problemas a todos sus amigos y conocidos, y fue la causa de que su Maestro le hiciera a menudo leves reprimendas. La única razón por la que Pedro no se metió en más problemas por su forma irreflexiva de hablar se debió a que, desde muy temprano, aprendió a compartir con Andrés, su hermano, la mayoría de sus planes y proyectos, antes de aventurarse a hacer públicos sus planteamientos.
\vs p139 2:4 Pedro hablaba con fluidez, elocuencia y emotividad. Era también un líder natural e inspirador, un pensador veloz, aunque no un razonador profundo. Él, por sí solo, hacía más preguntas que todos los apóstoles juntos, y aunque la mayoría de ellas eran apropiadas y relevantes, muchas eran irreflexivas e ingenuas. Pedro no tenía una mente profunda, pero se conocía a sí mismo bastante bien. Era, por lo tanto, un hombre de decisiones rápidas y prontitud de acción. Mientras otros hablaban sorprendidos al ver a Jesús en la playa, Pedro saltaba al agua y nadaba hacia la costa para encontrarse con el Maestro.
\vs p139 2:5 \pc El rasgo que Pedro más admiraba de Jesús era su formidable dulzura. Pedro nunca se cansaba de apreciar la indulgencia de Jesús. Jamás olvidó la lección que impartió de perdonar al injusto no siete sino setenta veces siete. Meditó mucho sobre sus sentimientos acerca del carácter misericordioso del Maestro en esos días sombríos y tristes que siguieron de inmediato a su irreflexiva e involuntaria negación de Jesús en el patio del sumo sacerdote.
\vs p139 2:6 \pc Simón Pedro era penosamente inconstante; de repente pasaba de un extremo al otro. Primero se negó a que Jesús le lavara los pies, y luego, al oír la réplica de su Maestro, le rogó que le lavara todo el cuerpo. Pero, con todo, Jesús sabía que los defectos de Pedro eran más de cabeza que de corazón. Pedro personificaba una de las mezclas más inexplicables de valor y cobardía jamás presenciada en la tierra. La gran fortaleza de su carácter era la lealtad y la amistad. Pedro amaba real y verdaderamente a Jesús. Y, sin embargo, a pesar de la fuerza impresionante de su devoción, era tan inestable y variable que una joven sirviente pudo con sus burlas hacer que negara a su Maestro y Señor. Pedro podía soportar la persecución y cualquier otra forma de agresión directa a su persona, pero quedaba paralizado y se empequeñecía ante el ridículo. Era un valiente soldado ante un ataque frontal, pero un cobarde temeroso y sumiso cuando se veía sorprendido por un ataque por detrás.
\vs p139 2:7 Pedro fue el primero de los apóstoles de Jesús en dar un paso adelante a favor de la labor de Felipe entre los samaritanos y de la de Pablo entre los gentiles; sin embargo, posteriormente en Antioquía, se retractó ante unos judaizantes que lo ridiculizaban, alejándose temporalmente de los gentiles, para luego afrontar las consecuencias de la valiente denuncia de Pablo.
\vs p139 2:8 Él fue el primero de los apóstoles en hacer una confesión incondicional sobre la humanidad y la divinidad combinadas de Jesús y el primero ---salvo Judas--- en negarlo. Pedro no era un gran soñador, pero le disgustaba descender de las nubes del éxtasis y el entusiasmo de la subyugante complacencia al mundo llano y práctico de la realidad.
\vs p139 2:9 Al seguir a Jesús, de forma literal y figurada, a veces él lideraba la marcha o, de lo contrario, iba a la zaga ---”lo seguía de lejos”---. Pero él era el predicador más prominente de los doce; contribuyó más que cualquier otro hombre, aparte de Pablo, a la instauración del reino y al envío de sus mensajeros a todas las partes del mundo en una sola generación.
\vs p139 2:10 Tras sus impulsivas negaciones del Maestro, se encontró a sí mismo y, con la guía compasiva y comprensiva de Andrés, lideró el camino de vuelta a las redes de pesca de los apóstoles, mientras aguardaban lo que estaba por llegar tras la crucifixión. Cuando estuvo totalmente seguro de que Jesús lo había perdonado y supo que había sido acogido de nuevo en el redil del Maestro, las llamas del reino ardieron tan intensamente en su alma que llegó a ser una luz, grande y salvífica, para esos millares de seres que moraban en las tinieblas.
\vs p139 2:11 \pc Después de partir de Jerusalén y antes de que Pablo se convirtiera en el espíritu impulsor de las iglesias cristianas gentiles, Pedro viajó extensamente, visitando todas las iglesias desde Babilonia hasta Corinto. Incluso llegó a visitar e impartir su ministerio a muchas de las iglesias fundadas por Pablo. Aunque Pedro y Pablo eran muy diferentes en temperamento y educación, en incluso en sus conceptos teológicos, trabajaron juntos armoniosamente para el desarrollo de las iglesias durante sus últimos años.
\vs p139 2:12 \pc En los sermones parcialmente documentados por Lucas y en el Evangelio de Marcos, se muestra algo del estilo y de la forma de enseñar de Pedro. Su estilo enérgico se percibe mejor en su carta conocida como la Primera Epístola de Pedro; al menos esto era verdad antes de que fuese posteriormente alterada por un discípulo de Pablo.
\vs p139 2:13 Pero Pedro continuó cometiendo el error de tratar de convencer a los judíos de que, al fin y al cabo, Jesús era real y verdaderamente el Mesías judío. Hasta el día de su muerte, en la mente de Simón Pedro persistía la confusión entre los conceptos de Jesús como el Mesías judío, Cristo como redentor del mundo y el Hijo del Hombre como la revelación de Dios, el Padre amoroso de toda la humanidad.
\vs p139 2:14 \pc La esposa de Pedro era una mujer muy capaz. Trabajó durante muchos años satisfactoriamente como miembro del cuerpo de mujeres y, cuando Pedro fue expulsado de Jerusalén, ella lo acompañó en todos sus viajes a las iglesias y en todas sus visitas misioneras. Y el día en que su insigne marido entregó su vida, ella fue arrojada a las bestias salvajes en la arena de Roma.
\vs p139 2:15 \pc Y, así, este hombre, Pedro, íntimo amigo de Jesús, de su círculo más cercano, se alejó de Jerusalén y proclamó la buena nueva del reino con poder y gloria hasta cumplir íntegramente su ministerio; y consideró que sus captores le concedían un gran honor cuando le comunicaron que moriría de la misma manera que había muerto su Maestro ---en la cruz---.Y de este modo fue Simón Pedro crucificado en Roma.
\usection{3. SANTIAGO ZEBEDEO}
\vs p139 3:1 Santiago, el mayor de los dos hijos apóstoles de Zebedeo, a quienes Jesús les dio el sobrenombre de “los hijos del trueno”, tenía treinta años cuando se convirtió en apóstol. Estaba casado, tenía cuatro hijos y vivía cerca de sus padres, en la periferia de Cafarnaúm, en Betsaida. Era pescador; desempeñaba su actividad en compañía de Juan, su hermano menor y conjuntamente con Andrés y Simón. Santiago y su hermano Juan gozaban de la ventaja de haber conocido a Jesús con bastante anterioridad a todos los demás apóstoles.
\vs p139 3:2 \pc Este capaz apóstol tenía un temperamento contradictorio; realmente parecía poseer una doble naturaleza, ambas motivadas por fuertes sentimientos. Era particularmente vehemente cuando se le provocaba lo suficiente y, cuando la tormenta había pasado, acostumbraba siempre a excusar su ira bajo el pretexto de que aquello no había sido sino una manera de expresar su justa indignación. Excepto por estos estallidos periódicos de ira, la personalidad de Santiago era muy parecida a la de Andrés. Carecía de la discreción y del conocimiento de la naturaleza humana de Andrés, pero era mucho mejor hablando en público. Después de Pedro, y quizás Mateo, Santiago era el mejor orador de entre los doce
\vs p139 3:3 Aunque Santiago no era voluble en absoluto, un día se le veía silencioso y taciturno y, al otro, se mostraba muy hablador y narrador de historias y anécdotas. A menudo hablaba abiertamente con Jesús, pero, entre los doce, era el que permanecía callado durante varios días seguidos. Su único gran defecto eran estos intervalos de inexplicables silencios.
\vs p139 3:4 El rasgo más sobresaliente de la personalidad de Santiago era su capacidad para ver todos los aspectos de cualquier afirmación. De los doce, resultó ser el que más se aproximó a comprender la verdadera importancia y significado de las enseñanzas de Jesús. Él, también, fue lento al principio en captar lo que el Maestro quería decir, pero, antes de finalizar su formación, había conseguido adquirir una idea superior del mensaje de Jesús. Santiago era capaz de tener un amplio entendimiento de la naturaleza humana. Se llevaba bien con el versátil Andrés, con el impetuoso Pedro y con su taciturno hermano Juan.
\vs p139 3:5 Aunque Santiago y Juan tenían sus propios problemas cuando intentaban trabajar juntos, era inspirador ver lo bien que se llevaban. Entre ellos no consiguieron tener una relación tan buena como la de Andrés y Pedro, pero congeniaban mucho mejor de lo que normalmente se espera de dos hermanos, en especial de dos hermanos tan tozudos y resueltos. Pero, aunque parezca raro, estos dos hijos de Zebedeo eran mucho más tolerantes el uno con el otro de lo que lo eran con los extraños. Se profesaban un gran afecto; siempre habían sido buenos compañeros de juego. Fueron estos “hijos del trueno” los que querían mandar que descendiera el fuego del cielo para consumir a los samaritanos que habían osado mostrarse irrespetuosos con su Maestro. Pero la muerte prematura de Santiago cambió sumamente el temperamento vehemente de Juan, su hermano menor.
\vs p139 3:6 \pc La característica que Santiago más admiraba del Maestro era su cariñosa comprensión. La consideración y el entendimiento que Jesús demostraba por el pequeño y el grande, por el rico y el pobre, tenían un gran poder de atracción sobre él.
\vs p139 3:7 \pc Santiago Zebedeo era un pensador bien equilibrado a quien le gustaba planificar con antelación. Junto con Andrés, era uno de los más juiciosos del grupo apostólico. Santiago era una criatura vigorosa pero nunca tenía prisa. Era un excelente contrapunto de Pedro.
\vs p139 3:8 Era humilde y nada pretensioso; prestaba diariamente su servicio, trabajaba sin pretensiones, sin buscar recompensas particulares una vez que logró comprender el verdadero significado del reino. E incluso en lo sucedido en torno a la madre de Santiago y de Juan, que pidió que se les concediese a sus hijos colocarse a la diestra y a la siniestra de Jesús, conviene recordar que fue la madre quien hizo esta petición. Y cuando indicaron que estaban listos para asumir tales responsabilidades, se debe reconocer que eran conscientes de los peligros que conllevaba la presunta rebelión del Maestro contra el poder de Roma, y que estaban igualmente dispuestos a pagar el precio por ello. Cuando Jesús preguntó si estaban listos para beber de la copa, respondieron que lo estaban. Y en lo que respecta a Santiago, fue literalmente cierto: bebió de la copa con el Maestro al ser el primero de los apóstoles en sufrir el martirio; murió a espada por orden de Herodes Agripa. Santiago fue, pues, el primero de los doce en sacrificar su vida en el nuevo frente de combate del reino. Herodes Agripa tenía más temor de Santiago que de todos los demás apóstoles. Efectivamente, a menudo era tranquilo y silencioso, pero demostró ser valiente y decidido cuando se instigaban y cuestionaban sus convicciones.
\vs p139 3:9 \pc Santiago vivió su vida al máximo, y cuando le llegó el fin, se comportó con tal gracia y fortaleza que incluso su acusador e informante, que estuvo presente en su juicio y ejecución, se conmovió tanto que se alejó corriendo de la escena de la muerte de Santiago para unirse a los discípulos de Jesús.
\usection{4. JUAN ZEBEDEO}
\vs p139 4:1 Cuando Juan se convirtió en apóstol, tenía veinticuatro años y era el más joven de los doce. Estaba soltero y vivía con sus padres en Betsaida; era pescador y trabajaba con su hermano Santiago en colaboración con Andrés y Pedro. Tanto antes como después de convertirse en apóstol, Juan actuó como representante personal de Jesús en las relaciones con la familia del Maestro, y continuó asumiendo esta responsabilidad el tiempo que María, la madre de Jesús, vivió.
\vs p139 4:2 Al ser el más joven y estar tan estrechamente unido a Jesús en relación a su familia, Juan era muy querido por el maestro, pero en verdad no se puede afirmar que él era “el discípulo al cual Jesús amaba”. Difícilmente se podría llegar a pensar que a una persona tan magnánima como Jesús se le pudiese acusar de mostrar favoritismos, de amar a un apóstol más que a los otros. El hecho de que Juan fuese uno de los tres asistentes personales de Jesús hizo reforzar esta idea equivocada, sin dejar de mencionar que Juan, junto con su hermano Santiago, sabía de Jesús desde hacía mucho más tiempo que los demás.
\vs p139 4:3 \pc Al poco tiempo de convertirse en apóstoles, a Pedro, Santiago y Juan se les encomendó como ayudantes personales de Jesús. Algo después de la elección de los doce y, en el momento en el que Jesús designó a Andrés director del grupo, él le dijo: “Y ahora deseo que asignes a dos o tres de tus compañeros para que estén conmigo y permanezcan a mi lado, me conforten y atiendan mis necesidades diarias”. Y Andrés pensó que lo mejor era escoger para esta especial responsabilidad a los siguientes tres apóstoles en ser seleccionados. A él le hubiese gustado ofrecerse como voluntario para tal bienaventurado servicio, pero el Maestro ya le había asignado su cometido; así pues, dispuso de inmediato destinar a esta tarea a Pedro, Santiago y Juan.
\vs p139 4:4 \pc Juan Zebedeo tenía muchos hermosos rasgos de carácter, pero uno que no era tan hermoso era su desmedida arrogancia, habitualmente bien disimulada. Su larga relación con Jesús hizo que se produjeran en su carácter numerosos y grandes cambios. Esta arrogancia disminuyó en gran medida, pero, al envejecer y volverse más o menos infantil, su orgullo reapareció hasta cierto punto, de forma que, cuando emprendió la tarea de asistir a Natán en la redacción del evangelio que ahora lleva su nombre, el anciano apóstol no vaciló en referirse a sí mismo repetidas veces como el “discípulo al cual Jesús amaba”. Teniendo en cuenta que Juan llegó a ser, más que cualquier otro mortal terrestre, el buen amigo de Jesús, y de que se le eligió como representante personal suyo en tantos asuntos, no es de extrañar que llegara a considerarse a sí mismo como tal discípulo amado, ya que sabía de cierto que él era el discípulo en quien Jesús, con tanta frecuencia, depositaba su confianza.
\vs p139 4:5 El rasgo más distintivo del carácter de Juan era su fiabilidad; era diligente y valeroso, fiel y dedicado. Su gran punto débil era su característica arrogancia. Era el miembro más joven de la familia de su padre, el más joven del grupo apostólico. Tal vez era algo consentido; quizás se había sido casi demasiado condescendiente con él. Pero el Juan de años posteriores fue un tipo de persona diferente al joven engreído y caprichoso que formó parte de los apóstoles de Jesús a los veinticuatro años de edad.
\vs p139 4:6 \pc Las características que Juan más apreciaba de Jesús eran el amor y la generosidad del Maestro; estos rasgos le impresionaron tanto que todo el resto de su vida estuvo dominada por el sentimiento de amor y de devoción fraternal. Habló del amor y escribió del amor. Este “hijo del trueno” se convirtió en el “apóstol del amor”; en Éfeso, cuando el anciano obispo no podía ya mantenerse de pie en el púlpito para predicar sino que tenían que llevarle a la Iglesia en una silla, y cuando al terminar del servicio le pedían que dijera algunas palabras a los creyentes, durante años, sus únicas palabras fueron: “Hijos míos, amaos unos a otros”.
\vs p139 4:7 \pc Juan era hombre de pocas palabras, excepto cuando su temperamento se exaltaba. Pensaba mucho pero decía poco. Con la edad, su carácter se volvió más moderado, más contenido, pero nunca logró vencer su poca disposición a hablar; nunca se sobrepuso del todo a tal reticencia. Si bien, estaba dotado de una excepcional imaginación creativa.
\vs p139 4:8 \pc Había otra faceta de Juan que nadie esperaría encontrar en una persona callada e introspectiva. Tenía ciertos prejuicios y era excesivamente intolerante. En este sentido, él y Santiago eran muy parecidos ---ambos querían mandar descender el fuego desde el cielo sobre las cabezas de los irrespetuosos samaritanos---. Cuando Juan se encontraba con desconocidos que enseñaban en nombre de Jesús, rápidamente lo prohibía. Pero, él no era el único de los doce aquejado con este tipo de orgullo y de conciencia de superioridad.
\vs p139 4:9 Ver como Jesús iba por ahí sin hogar tuvo una gran influencia en la vida de Juan, sabiendo con la lealtad con la que él había provisto para el cuidado de su madre y de su familia. Juan también simpatizaba profundamente con Jesús debido a la falta de comprensión que sufría por parte de su familia, dándose cuenta de que se estaban paulatinamente distanciando de él. Toda esta situación, junto al hecho de que Jesús siempre postergaba su más mínimo deseo a la voluntad del Padre de los cielos y su vivir diario de absoluta confianza, tocaron tan hondamente a Juan que produjo en él marcados y permanentes cambios de carácter, cambios que se manifestarían durante el resto de su vida.
\vs p139 4:10 Juan tenía una serena valentía y arrojo, algo que solo unos pocos apóstoles poseían. Fue el único apóstol que se mantuvo con Cristo la noche de su arresto y se atrevió a acompañar a su Maestro hasta las garras mismas de la muerte. Estuvo presente y cercano a él hasta su última hora en la tierra y demostró su lealtad al responsabilizarse de la madre de Jesús y estar dispuesto a seguir cualquier instrucción que se le diera en los últimos momentos de la existencia mortal del Maestro. Hay algo que está claro: Juan era enteramente digno de confianza. Se solía sentar a la diestra de Jesús cuando los doce comían juntos. Fue el primero de los doce en creer, verdadera y plenamente, en la resurrección, y el primero en reconocer al Maestro cuando se dirigió hacia ellos por la playa tras su resurrección.
\vs p139 4:11 Este hijo de Zebedeo se relacionó muy estrechamente con Pedro en las primeras actividades del movimiento cristiano, llegando a convertirse en uno de los principales defensores de la Iglesia de Jerusalén. Y fue su mano derecha y sostén el día de Pentecostés.
\vs p139 4:12 Algunos años después del martirio de Santiago, Juan se casó con la viuda de su hermano. Durante los últimos veinte años de su vida quedó al cuidado de una cariñosa nieta.
\vs p139 4:13 Juan estuvo varias veces en prisión y, durante cuatro años, estuvo desterrado en la isla de Patmos hasta que otro emperador subió al poder en Roma. Si Juan no hubiese sido tan diplomático y sensato, sin duda hubiese perdido la vida como su hermano Santiago, que se expresaba de forma más directa. Conforme pasaron los años, Juan, junto con Santiago, el hermano del Señor, aprendió a ser más sensato y conciliador cuando comparecían ante los magistrados civiles. Se había dado cuenta de que una “respuesta suave aplacaba la ira”. Aprendieron también a definir la Iglesia como una “hermandad espiritual dedicada al servicio social de la humanidad” más que como “el reino de los cielos”. Enseñaron el servicio amoroso más que un poder dominante ---reino y el rey---.
\vs p139 4:14 Durante su exilio temporal en Patmos, Juan escribió el libro del Apocalipsis, que tenéis ahora enormemente abreviado y distorsionado. Este libro contiene fragmentos restantes de una gran revelación, ya que muchos pasajes se perdieron y otros se suprimieron después de que Juan lo redactara. Se conserva solamente de forma parcial y adulterada.
\vs p139 4:15 Juan viajó mucho, realizó una labor incesante y, después de convertirse en obispo de las iglesias de Asia, se estableció en Éfeso. Cuando tenía noventa y nueve años de edad, asistió a Natán, su colaborador, en la redacción del denominado “Evangelio según Juan”, en Éfeso. De los doce apóstoles, Juan Zebedeo llegaría a ser el teólogo más destacado. Murió de muerte natural en Éfeso en el año 103 d. C., a los ciento un años de edad.
\usection{5. FELIPE EL CURIOSO}
\vs p139 5:1 Felipe fue el quinto apóstol en ser elegido; recibió la llamada cuando Jesús y sus primeros cuatro apóstoles se dirigían desde el emplazamiento de Juan, en el Jordán, hacia Caná de Galilea. Al vivir en Betsaida, Felipe hacía algún tiempo que conocía a Jesús, pero no se le había ocurrido pensar que era verdaderamente un gran hombre hasta ese día en el valle del Jordán en el que él le dijo “Sígueme”. Felipe también se vio, en cierta manera, influenciado por el hecho de que Andrés, Pedro, Santiago y Juan hubiesen ya aceptado a Jesús como el Libertador.
\vs p139 5:2 Felipe tenía veintisiete años cuando se unió a los apóstoles; hacía poco que se había casado, pero no tenía hijos en ese momento. Los apóstoles le dieron un sobrenombre que significaba “curiosidad”. Felipe siempre quería que le enseñaran. No parecía percibir más allá de lo que había delante de él. No era necesariamente torpe, pero carecía de imaginación, y esa falta de imaginación era el mayor punto débil de su carácter. Era una persona común y simple.
\vs p139 5:3 \pc Cuando los apóstoles se organizaron para el servicio, a Felipe se le nombró encargado de abastecimientos; su obligación era vigilar que, en todo momento, estuviesen todos bien abastecidos de provisiones. Y realizó bien su cargo. Su característica más destacada era su ordenada minuciosidad; era a la vez bueno con los números y sistemático.
\vs p139 5:4 Felipe procedía de una familia de siete hermanos, tres niños y cuatro niñas. Él era el segundo y, después de la resurrección, bautizó a toda su familia para su entrada en el reino. La familia de Felipe eran pescadores. Su padre era un hombre muy capaz, un profundo pensador, pero su madre venía de una familia muy común. No se podía esperar de Felipe que hiciera grandes cosas, pero era un hombre que podía hacer pequeñas cosas de una manera grande, hacerlas bien y de manera satisfactoria. Pocas veces en cuatro años dejó de disponer de comida para satisfacer las necesidades de todos. Incluso las apremiantes exigencias que les surgían en la vida que llevaban raramente le hallaban desprevenido. El servicio de abastecimiento de la familia apostólica se gestionaba con inteligencia y eficiencia.
\vs p139 5:5 El punto fuerte de Felipe era su metódica fiabilidad; el punto débil de su carácter era su absoluta falta de imaginación, le faltaba capacidad para pensar de forma creativa. Era bueno con las matemáticas en el sentido abstracto, pero no tenía una imaginación constructiva. Carecía casi enteramente de ciertos tipos de imaginación. Era el típico hombre común, simple y llano. Había muchos hombres y mujeres como él entre las multitudes que acudían para escuchar las enseñanzas y predicaciones de Jesús y, a todos ellos, les consolaba observar cómo alguien semejante a ellos había sido elevado a un puesto de honor en los consejos del Maestro; les daba ánimo comprobar que alguien como ellos ya hubiese encontrado un lugar alto en los asuntos del reino. Y Jesús aprendió mucho de la forma de actuar de algunas mentes humanas cuando escuchaba con paciencia las preguntas sin sentido de Felipe y, tantas veces, se atenía al ruego de su encargado de abastecimientos de “ser enseñado”.
\vs p139 5:6 La cualidad de Jesús que Felipe siempre admiraba era la inquebrantable generosidad del Maestro. Felipe nunca encontró en Jesús nada de exigüidad, tacañería o avaricia, y él adoraba esta permanente e indefectible magnanimidad.
\vs p139 5:7 \pc No había mucho de la persona de Felipe que causara impresión. Con frecuencia se hablaba de él como “Felipe de Betsaida, el pueblo en el que Andrés y Pedro viven”. Al carecer prácticamente de un claro discernimiento, era incapaz de captar las posibilidades nuevas de una determinada situación. No era pesimista; era simplemente prosaico. También, en gran medida, carecía de percepción espiritual. No vacilaba en interrumpir a Jesús en medio de algunos de sus más profundos discursos del Maestro para formularle alguna pregunta manifiestamente sin sentido. Pero Jesús nunca lo reprendió por tal inconsciencia; fue paciente con él y respetuoso de su incapacidad para entender los significados más profundos de su enseñanza. Jesús sabía muy bien que, una vez que le reprochara a Felipe por sus inoportunas preguntas, no solo hubiese herido a esta honesta alma, sino que la reprimenda hubiese lastimado tanto a Felipe que jamás se habría sentido libre para volver a hacer preguntas de nuevo. Jesús era consciente de que en sus mundos del espacio había incontables miles de millones de mortales de similar tardo pensamiento, y quería alentarlos a que le pidieran ayuda y que se sintieran con la libertad de venir a él con sus cuestiones y dificultades. Al fin y al cabo, a Jesús realmente le importaban más las preguntas vanas de Felipe que el sermón que pudiera estar predicando. Estaba supremamente interesado en los \bibemph{hombres,} en todos los tipos de hombres.
\vs p139 5:8 El apóstol encargado de los abastecimientos no hablaba bien en público, pero en su labor personal era muy persuasivo y acertado. No se desalentaba fácilmente; era perseverante y tenaz en todo lo que llevaba a cabo. Tenía el magnífico y singular don de decir “ven”. Cuando Nataniel, su primer converso, quiso discutir acerca de los méritos y deméritos de Jesús y de Nazaret, la eficaz respuesta de Felipe fue “ven y ve”. No era un predicador dogmático que exhortara a sus oyentes a que “fueran” ---haced esto y haced aquello---; afrontaba todas las situaciones que apareciesen en su trabajo con un “ven; ven conmigo; te mostraré el camino”. Y ese es siempre el método más efectivo de cualquier forma o faceta de la enseñanza. Incluso los padres pueden aprender de Felipe un modo mejor de hablarle a sus hijos ---\bibemph{no} decirle “id a hacer esto y aquello”, sino, en su lugar, “venid con nosotros que os mostraremos y compartiremos con vosotros el camino mejor”---.
\vs p139 5:9 La falta de capacidad de Felipe de adaptarse a una situación nueva se demostró claramente cuando, en Jerusalén, los griegos acudieron a él, diciéndole: “Señor, deseamos ver a Jesús”. De inmediato, a cualquier judío que hubiese hecho tal pregunta, Felipe le habría dicho: “ven”. Pero estos hombres eran extranjeros y él no recordaba ninguna instrucción de sus superiores respecto a aquello; por lo tanto, lo único que pensó en hacer fue consultar al jefe, a Andrés, y luego ambos escoltaron a estos griegos hasta Jesús. Del mismo modo, cuando fue a Samaria para predicar y bautizar a los creyentes, conforme el Maestro le había indicado, se contuvo de imponer sus manos sobre los conversos como símbolo de que habían recibido el espíritu de la verdad. Esto lo hacían Pedro y Juan, llegados, al poco tiempo, a Jerusalén para observar su labor en nombre de la Iglesia madre.
\vs p139 5:10 Felipe vivió los momentos difíciles de la muerte del Maestro, participó en la reorganización de los doce y fue el primero en salir a ganar almas para el reino al margen del inmediato colectivo judío, obteniendo grandes resultados en su labor con los samaritanos y en toda su actividad posterior por el evangelio.
\vs p139 5:11 \pc La esposa de Felipe, que era una eficiente miembro del cuerpo de mujeres, se convirtió en una activa colaboradora de su marido en la tarea evangélica cuando huyeron de Jerusalén a raíz de las persecuciones. Su esposa era una mujer valiente. Permaneció al pie de la cruz de Felipe, alentándolo a que proclamara la buena nueva incluso a sus asesinos y, cuando le fallaron las fuerzas, ella empezó a narrar la historia de la salvación mediante la fe en Jesús, y solo se la silenció cuando los airados judíos arremetieron contra ella y la apedrearon hasta la muerte. Lea, su hija mayor, continuó la labor de ambos, convirtiéndose después en la reconocida profetiza de Hierápolis.
\vs p139 5:12 \pc Felipe, el encargado de abastecimientos de los doce, fue un gran hombre del reino, ganando almas dondequiera que iba; hasta que, finalmente, lo crucificaron por su fe y lo enterraron en Hierápolis.
\usection{6. EL HONESTO NATANAEL}
\vs p139 6:1 A Natanael, el sexto y último de los apóstoles elegido por el propio Maestro, lo llevó su amigo Felipe. Había colaborado en diferentes empresas comerciales con Felipe y, con él, se encaminaba para ver a Juan el Bautista cuando se encontraron con Jesús.
\vs p139 6:2 Natanael tenía veinticinco años cuando se unió a los apóstoles; era el segundo más joven del grupo. Era el hijo menor de una familia de siete, estaba soltero y constituía el único sostén de sus padres, ancianos y enfermos, con los que vivía en Caná; sus hermanos y su hermana estaban casados o habían fallecido, y ninguno vivía allí. Natanael y Judas Iscariote eran, de los doce, los que mayor nivel educativo tenían. Natanael había considerado convertirse en comerciante.
\vs p139 6:3 \pc El mismo Jesús no le puso ningún sobrenombre a Natanael, pero los doce empezaron pronto a hablar de él en términos que significaban honestidad, sinceridad. En él “no había engaño”. Y esta era su gran virtud; era a la vez honesto y sincero. El punto débil de su carácter era su orgullo; estaba muy orgulloso de su familia, su ciudad, su reputación y su país, todo lo cual es encomiable si no se lleva demasiado lejos. Pero Natanael tendía a irse a los extremos en sus prejuicios personales. Era propenso a prejuzgar a las personas de acuerdo con sus propias opiniones personales. Incluso antes de conocer a Jesús, no tardó en preguntar: “¿Puede venir algo bueno de Nazaret?”. Pero, a pesar de ser orgulloso, Natanael no era obstinado. Se apresuró a dar marcha atrás una vez que vio el rostro de Jesús.
\vs p139 6:4 En múltiples aspectos, Natanael era el genio singular de los doce. Era el filósofo y el soñador apostólico, pero un soñador bastante práctico. Alternaba entre temporadas de profunda filosofía y períodos de un humor inusual e ingenioso; con el talante adecuado, era probablemente el mejor narrador de los doce. A Jesús le gustaba mucho oír a Natanael disertar sobre cosas tanto serias como frívolas. Natanael fue paulatinamente tomando, con mayor seriedad, a Jesús y el reino, pero en realidad nunca se tomó, a sí mismo, demasiado en serio.
\vs p139 6:5 Todos los apóstoles amaban y respetaban a Natanael, y él se llevaba espléndidamente bien con todos ellos, excepto con Judas Iscariote. Judas pensaba que Natanael no se tomaba su apostolado con la seriedad suficiente y, cierta vez, tuvo la temeridad de ir a Jesús, a escondidas, y presentar sus quejas contra él. Jesús le dijo: “Judas, observa con cuidado tus pasos; no te excedas en tu mandato. ¿Quién de entre nosotros es apto para juzgar a su hermano? No es voluntad del Padre que sus hijos deban tomar parte solo de las cosas serias de la vida. Te lo repetiré: he venido para que mis hermanos en la carne puedan tener más gozo, alegría y abundancia de vida. Vete pues, Judas, y haz bien lo que se te ha encomendado, pero deja que Natanael, tu hermano, dé cuenta de sí mismo ante Dios”. Y el recuerdo de este hecho, junto con el de otras experiencias similares, vivió durante mucho tiempo en el ilusorio corazón de Judas Iscariote.
\vs p139 6:6 Muchas veces, cuando Jesús estaba en la montaña con Pedro, Santiago y Juan, y las cosas se ponían tensas y complicadas para los apóstoles, cuando incluso Andrés dudaba sobre qué decir a sus inconsolables hermanos, Natanael aliviaba la tensión con un poco de filosofía o un toque de humor; de buen humor, también.
\vs p139 6:7 Natanael tenía el deber de cuidar del bienestar de las familias de los doce. Con frecuencia estaba ausente de los consejos apostólicos, porque, cuando se enteraba de que la enfermedad o algo fuera de lo común que le había ocurrido a alguna de las personas bajo su cargo, no perdía tiempo en ir a ese hogar. Los doce se sentían tranquilos sabiendo que, en manos de Natanael, el bienestar de sus familias estaba asegurado.
\vs p139 6:8 \pc Lo que Natanael veneraba más de Jesús era su tolerancia. No se cansaba de contemplar su mentalidad abierta y la generosa empatía del Hijo del Hombre.
\vs p139 6:9 \pc El padre de Natanael (Bartolomé) murió poco tiempo después de Pentecostés, tras lo cual, este apóstol se adentró en Mesopotamia y en la India para proclamar la buena nueva del reino y bautizar a los creyentes. Sus hermanos apóstoles jamás llegaron a saber qué le había ocurrido al que había sido su filósofo, poeta y humorista. Si bien, él también fue un gran hombre del reino e hizo mucho por difundir las enseñanzas de su Maestro, incluso aunque no participara en la organización de la futura Iglesia cristiana. Natanael falleció en la India.
\usection{7. MATEO LEVÍ}
\vs p139 7:1 Mateo, elegido por Andrés, fue el séptimo apóstol. Mateo pertenecía a una familia de cobradores de impuestos, o publicanos, pero él mismo era un recaudador de aduanas en Cafarnaúm, donde vivía. Tenía treinta y un años, era casado con cuatro hijos. Poseía una riqueza modesta; de los miembros del cuerpo apostólico, era el único que disponía de medios. Era un buen negociante, muy sociable y tenía el don de hacer amigos y de relacionarse sin inconvenientes con una gran variedad de personas.
\vs p139 7:2 \pc Andrés nombró a Mateo coordinador de la economía de los apóstoles. En cierto sentido, era el recolector de recursos y el portavoz público de la organización apostólica. Era un juez perspicaz de la naturaleza humana y un eficaz promotor. Era una persona difícil de prever, pero era un ferviente discípulo con una fe creciente en la misión de Jesús y en la certitud del reino. Jesús no le dio a Leví ningún sobrenombre, pero, por lo común, sus hermanos apóstoles se referían a él como el “captador de fondos”.
\vs p139 7:3 El punto fuerte de Leví era su devoción incondicional a la causa. El hecho de que Jesús y sus apóstoles le hubiesen aceptado a él, un publicano hacía que este antiguo cobrador de impuestos sintiese una inmensa gratitud. Sin embargo, fue necesario cierto tiempo hasta que el resto de los apóstoles, particularmente Simón el Zelote y Judas Iscariote, se resignasen a tener entre ellos a un publicano. El punto débil de Mateo era su perspectiva de la vida, corta de miras y materialista. Pero, en todas estas cuestiones, hizo grandes progresos conforme pasaban los meses. Al tener que mantener las arcas con fondos suficientes, él, como era lógico, tuvo que estar ausente de muchos de los más preciados periodos de instrucción.
\vs p139 7:4 Lo que Mateo más valoraba del Maestro era su predisposición al perdón. No cesaba de decir que hallar a Dios era únicamente un tema de fe. Le gustaba hablar siempre del reino como “este asunto de encontrar a Dios”.
\vs p139 7:5 \pc Aunque Mateo era un hombre con un pasado, hizo una excelente labor y, con el paso del tiempo, sus compañeros se enorgullecieron de la forma de proceder del publicano. Fue uno de los apóstoles que tomaban extensas notas de los dichos de Jesús. Estas notas se utilizarían como base para la posterior narrativa de Isador de los dichos y hechos de Jesús, y que ha llegado a conocerse como el Evangelio según Mateo.
\vs p139 7:6 Por medio de la vida, grandiosa y útil, de Mateo, el hombre de negocios y recaudador de aduanas de Cafarnaúm, muchos miles de otros hombres de negocios, funcionarios públicos y políticos, a través de las eras, se han sentido llevados a escuchar también la cautivadora voz del Maestro que dice “sígueme”. En verdad, Mateo era un astuto político, pero fue sumamente leal a Jesús y se consagró preeminentemente a la tarea de velar para que los mensajeros del reino venidero contaran con suficientes recursos económicos.
\vs p139 7:7 La presencia de Mateo entre los doce fue la forma de mantener las puertas del reino bien abiertas para multitudes de almas abatidas y marginadas que se consideraban a sí mismas, desde hacía mucho tiempo, fuera de la religión. Muchos hombres y mujeres, desesperados y sintiéndose rechazados, se congregaban para escuchar a Jesús, y él jamás le dio la espalda a ninguno de ellos.
\vs p139 7:8 \pc Mateo recibía las ofrendas voluntarias de parte de los discípulos creyentes y de quienes se acercaban a oír las enseñanzas del Maestro, pero nunca, expresamente, solicitaba fondos de las multitudes. Hizo toda su labor económica de manera reservada y personal, recaudando la mayor parte del dinero entre la clase más acaudaladas de los creyentes receptivos. Entregó prácticamente toda su modesta fortuna a la obra del Maestro y a sus apóstoles, pero ellos nunca tuvieron conocimiento de este acto de generosidad, salvo Jesús, que sí era consciente de aquello. Mateo dudaba si contribuir manifiestamente a los fondos apostólicos temeroso de que Jesús y sus compañeros pudiesen pensar que su dinero estaba manchado; así pues, aportaba bastante en nombre de otros creyentes. Durante los primeros meses, sabiendo que su presencia entre ellos les causaba, en cierto modo, un sufrimiento, estuvo tentado repetidas veces de decirles que era su dinero con frecuencia el que les proveía su pan diario, pero no cedió a la tentación. Cuando era patente el desdén hacia el publicano, Leví ardía en deseos de revelarles su generosidad, pero siempre se las arregló para mantenerse callado.
\vs p139 7:9 Cuando escaseaban los fondos para afrontar las necesidades semanales previstas, muchas veces, Leví recurría a sus propios recursos personales y realizaba generosas aportaciones. También, algunas veces, cuando las enseñanzas de Jesús le despertaban un gran interés, prefería quedarse para oírlas, sabiendo incluso que iba a tener que compensar personalmente por no haber recaudado los fondos que se precisaban. ¡Pero cuánto deseaba Leví que Jesús supiera que una gran parte del dinero provenía de su propio bolsillo! Poco podía imaginarse que el Maestro estaba ya al corriente. Todos los apóstoles murieron sin saber que Mateo había sido su benefactor, hasta tal punto que, cuando salió a proclamar el evangelio del reino, tras el comienzo de las persecuciones, Mateo estaba prácticamente en la ruina.
\vs p139 7:10 \pc Cuando estas persecuciones llevaron a los creyentes a abandonar Jerusalén, Mateo viajó al norte para predicar el evangelio del reino y bautizar a los creyentes. Sus antiguos compañeros apostólicos no supieron nada más de él, pero continuó predicando y bautizando en Siria, Capadocia, Galacia, Bitinia y Tracia. Y fue en Tracia, en Lisimaquia, donde ciertos incrédulos judíos conspiraron con los soldados romanos para tramar su muerte. Y este publicano regenerado murió triunfante en la fe de la salvación que él ciertamente había aprendido de las enseñanzas del Maestro durante su reciente estancia en la tierra con él.
\usection{8. TOMÁS DÍDIMO}
\vs p139 8:1 Tomas, elegido por Felipe, fue el octavo apóstol. Más tarde se le conocería como “Tomás el incrédulo”, pero sus hermanos apóstoles no lo veían como un incrédulo crónico. Es cierto que tenía un tipo de mente lógica y escéptica, pero era tan valerosamente leal que aquellos que lo conocieron de cerca no podían considerarlo como un escéptico trivial.
\vs p139 8:2 Cuando se unió a los apóstoles, Tomás tenía veintinueve años, estaba casado y tenía cuatro hijos. Con anterioridad había sido carpintero y albañil, pero más tarde se hizo pescador y residió en Tariquea, situada en la orilla occidental del Jordán, por donde fluía el río desde el mar de Galilea, y se le tenía como el más prominente ciudadano de esta pequeña aldea. Tenía poco nivel educativo, pero poseía una mente aguda y razonadora, y era hijo de excelentes progenitores, habitantes de Tiberias. Su mente era auténticamente analítica, lo que le distinguía de los doce; era el verdadero científico del cuerpo apostólico.
\vs p139 8:3 Los primeros años de la vida familiar de Tomás habían sido lamentables; en su vida conyugal, sus padres no habían sido muy felices, y esto tuvo repercusión en su vida adulta. Creció teniendo un carácter desagradable y pendenciero. Hasta su esposa se alegró de ver cómo se unía a los apóstoles; le aliviaba pensar que su pesimista marido estaría lejos de su casa la mayoría del tiempo. Tomás también tendía a la suspicacia, lo que hacía muy difícil la convivencia con él de modo pacífico. Al comienzo, a Pedro le disgustaba mucho Tomás y se quejaba a su hermano Andrés de él diciéndole que era un hombre “rudo, feo y siempre desconfiado”. Pero a medida que sus compañeros lo conocían mejor, más les gustaba. Se dieron cuenta de su extraordinaria honestidad e inquebrantable lealtad. Era absolutamente sincero e indiscutiblemente veraz, aunque también un crítico innato, que se había convertido en un verdadero pesimista. Su mente analítica era muy recelosa. Estaba perdiendo rápidamente la fe en sus semejantes cuando se vinculó a los doce y percibió el noble carácter de Jesús. Esta relación con el Maestro comenzó, de inmediato, a transformar su actitud y propició significativos cambios en sus reacciones mentales hacia los demás.
\vs p139 8:4 La gran virtud de Tomás era su formidable mente analítica junto a su inquebrantable arrojo ---una vez que tomaba una decisión---. Su peor punto débil era su incredulidad y desconfianza, que nunca llegaría a vencer del todo durante su vida en la carne.
\vs p139 8:5 Dentro de la organización de los doce, a Tomás se le asignó la concertación y dirección de los desplazamientos, y dirigió competentemente la labor y los movimientos del cuerpo apostólico. Era un buen gestor, un excelente hombre de negocios, pero tenía el inconveniente de ser muy variable en sus estados de ánimo; no era la misma persona de un día para otro. Cuando Tomás se unió a los apóstoles era propenso a hundirse en la melancolía, pero el contacto con Jesús y los apóstoles le curó, en buena medida, su enfermiza tendencia a la introspección.
\vs p139 8:6 Jesús disfrutaba bastante de la compañía de Tomás y mantuvo muchas largas charlas personales con él. Su presencia entre los apóstoles significó un gran consuelo para cualquier incrédulo honesto y animó a muchas mentes angustiadas a entrar en el reino, aunque no pudieran llegar a entender todos los aspectos espirituales y filosóficos de las enseñanzas de Jesús. El hecho de que Tomás fuese miembro de los doce era una manifestación constante de que Jesús amaba incluso a aquellos que eran escépticos sinceros.
\vs p139 8:7 \pc Los otros apóstoles veneraban a Jesús por algún rasgo particular y sobresaliente de su excepcional persona, pero Tomás veneraba a su Maestro por su carácter, magníficamente equilibrado. Tomás admiraba y respetaba cada vez más a aquel por ser tan amorosamente misericordioso y, sin embargo, tan inflexiblemente justo y equitativo; tan firme pero nunca obstinado; tan sereno, pero nunca indiferente; dispuesto a ser de ayuda y compasivo, pero nunca entrometido ni dictatorial; tan fuerte y, al mismo tiempo, tan cariñoso; tan positivo, pero nunca tosco ni rudo; tan tierno pero nunca vacilante; tan puro e inocente pero, a la vez, tan viril, firme y dinámico; tan verdaderamente valeroso, pero nunca impulsivo ni temerario; tan amante de la naturaleza pero tan alejado de la tendencia a reverenciarla; tan divertido y jovial, pero tan exento de veleidad y frivolidad. Era esta inigualable armonía de su persona la que tanto fascinaba a Tomás. De entre los doce, probablemente era él quien disfrutaba de un mayor entendimiento intelectual de Jesús y poseía una más elevada conciencia de su persona.
\vs p139 8:8 \pc En los consejos de los doce, Tomás siempre era cauto, abogando primeramente por una política de seguridad, pero, si su actitud conservadora se rechazaba por votación o se desestimaba, solía ser el primero en proceder sin miedo a llevar a cabo el plan aprobado. Repetidas veces rechazaba algún determinado proyecto por considerarlo temerario y pretencioso; debatía hasta sus últimas consecuencias, pero cuando Andrés sometía la propuesta a votación, y cuando los doce elegían hacer algo a lo que él se había opuesto tan denodadamente, Tomás era el primero en decir: “¡Vamos!”. Era un buen perdedor. No guardaba rencor ni albergaba resentimientos. En reiteradas ocasiones, se opuso a que Jesús se expusiera al peligro, pero cuando el Maestro decidía correr tales riesgos, era él siempre el que movilizaba a los apóstoles con sus valientes palabras: “Vamos, compañeros, muramos por él”.
\vs p139 8:9 Tomás, en ciertos aspectos, era como Felipe; quería “que le enseñaran”, pero sus manifestaciones de duda estaban basadas en procesos intelectuales enteramente distintos. Tomás era analítico, no meramente escéptico. En cuanto a su particular coraje físico, era uno de los más valientes de los doce.
\vs p139 8:10 \pc Tomás tenía algunos días bastante malos; en ocasiones, estaba deprimido y abatido. La pérdida de su hermana gemela, cuanto él tenía nueve años, le había ocasionado una gran tristeza durante su juventud y había contribuido a los trastornos de conducta que experimentaría en su vida posterior. Siempre que Tomás estaba deprimido, algunas veces era Nataniel quien le ayudaba a sobreponerse; otras, Pedro; y, no infrecuentemente, también uno de los gemelos Alfeo. En los momentos de mayor depresión, desafortunadamente, Tomás trataba de evitar entrar en contacto directo con Jesús. Pero el Maestro era consciente de ello, comprendía a su apóstol y sentía compasión por él cuando padecía estas crisis y las dudas lo acosaban.
\vs p139 8:11 Algunas veces, Tomás lograba permiso de Andrés para marcharse a solas uno o dos días. Pero pronto aprendió que esta forma de proceder no era aconsejable; tempranamente se dio cuenta de que, cuando estaba desanimado, era mejor ceñirse a su trabajo y permanecer cerca de sus compañeros. Si bien, con independencia de lo que sucediera en su vida emocional, se mantuvo firme como apóstol. Cuando llegaba realmente el momento de seguir adelante, siempre era Tomás quien decía: “¡Vamos!”.
\vs p139 8:12 Tomás constituye un magnífico ejemplo del ser humano que tiene dudas, se enfrenta a ellas y las vence. Tenía una gran mente; no era un crítico que se quejase constantemente. Era un pensador lógico; fue la prueba de fuego para Jesús y sus hermanos apóstoles. Si Jesús y su labor no hubiesen sido genuinos, no habrían podido conservar a un hombre como Tomás desde el principio hasta el fin. Tenía un sentido agudo y claro de los \bibemph{hechos}. Al primer síntoma de fraude o de engaño, Tomás los habría abandonado. Los científicos pueden que no entiendan del todo a Jesús y su labor en la tierra, pero, con el Maestro y sus acompañantes humanos, vivió y trabajó un hombre cuya mente era la de un verdadero científico ---Tomás Dídimo--- y él creyó en Jesús de Nazaret.
\vs p139 8:13 \pc Tomás atravesó momentos de dificultad durante los días del juicio y la crucifixión. Estuvo por algún tiempo en lo más profundo de la desesperación, pero hizo acopio de su coraje, permaneció firme con los apóstoles y estuvo presente con ellos para recibir a Jesús en el mar de Galilea. Cayó durante cierto tiempo en la depresión y en la duda, pero acabó por mostrar su fe y valor. Impartió sensatos consejos a los apóstoles tras Pentecostés y, cuando la persecución dispersó a los creyentes, se marchó a Chipre, a Creta, a la costa norteafricana y a Sicilia, predicando la buena nueva del reino y bautizando a los creyentes. Y Tomás continuó predicando y bautizando hasta que los agentes del gobierno romano lo arrestaron y ejecutaron en Malta. Solo pocas semanas antes de su muerte había empezado a escribir la vida y las enseñanzas de Jesús.
\usection{9 Y 10. SANTIAGO Y JUDAS ALFEO}
\vs p139 9:1 Santiago y Judas, los hijos de Alfeo, los pescadores gemelos que vivían cerca de Queresa, fueron el noveno y el décimo apóstol. Los eligieron Santiago y Juan Zebedeo. Tenían veintiséis años y estaban casados; Santiago tenía tres hijos y Judas dos.
\vs p139 9:2 \pc No hay mucho que comentar acerca de estos dos pescadores corrientes. Amaban a su Maestro y Jesús los amaba a ellos. Nunca interrumpían sus discursos con preguntas; entendían muy escasamente las argumentaciones filosóficas o los debates teológicos de sus compañeros apóstoles, pero se regocijaban de contarse entre los miembros de este grupo de impresionantes hombres. Eran casi idénticos en su apariencia personal, características mentales y grado de percepción espiritual. Lo que se diga de uno de ellos puede atribuirse al otro.
\vs p139 9:3 Andrés les asignó la labor de velar por el orden público en las multitudes. Eran los jefes de los ujieres durante las horas de la predicación y, de hecho, los asistentes y los recaderos de los doce. Ayudaban a Felipe con las provisiones, llevaban a las familias el dinero en nombre de Natanael y siempre estaban dispuestos a echar una mano a cualquiera de los apóstoles.
\vs p139 9:4 A las multitudes de gente corriente les animaba sobremanera ver cómo a estas dos personas semejantes a ellos se les honraba con un lugar entre los apóstoles. El mismo hecho de que a estos dos comunes gemelos se les aceptara como apóstoles sirvió para atraer a una gran cantidad de tímidos creyentes. Además, la gente ordinaria seguía con mayor amabilidad la guía y las instrucciones de ujieres autorizados, que tanto se les parecían.
\vs p139 9:5 Santiago y Judas, también llamados Tadeo y Lebeo, no tenían ni punto fuerte ni débil. Los sobrenombres que los discípulos les dieron eran apelativos afables que reflejaban su normalidad como personas. Eran “los más pequeños de todos los apóstoles”; lo sabían y se sentían contentos de serlo.
\vs p139 9:6 \pc Santiago Alfeo amaba a Jesús particularmente por la sencillez del Maestro. Los gemelos no podían comprender la mente de Jesús, pero sentían que había un vínculo de compresión entre ellos y el corazón de su Maestro. Sus mentes no eran de un elevado orden; hasta se les podría calificar, con reverencia, de tener muy poco entendimiento, pero, habían tenido auténticas experiencias espirituales. Creían en Jesús; eran los hijos de Dios y miembros del reino.
\vs p139 9:7 Judas Alfeo se sentía atraído hacia Jesús por la discreta humildad del Maestro. Dicha humildad, unida a tan alto grado de dignidad personal, apelaba intensamente a Judas. El hecho de que Jesús siempre requería de ellos que no mencionaran sus insólitos actos impresionaba sobremanera a este simple hijo de la naturaleza.
\vs p139 9:8 \pc Los gemelos eran unos asistentes de buen carácter y simples de mente, y todos los amaban. Jesús acogió a estos jóvenes, de un talento, en puestos de honor de su comitiva personal en el reino, porque hay incontables millones de otras almas del mismo modo, simples y temerosas, en los mundos del espacio a quienes desea igualmente darles la bienvenida a una fraternidad dinámica de creyentes con él y el espíritu de la verdad derramado. Jesús no menosprecia la pequeñez, sino que detesta el mal y el pecado. Santiago y Judas eran \bibemph{pequeños,} pero a la vez eran \bibemph{leales}. Eran simples y faltos de cultura, pero, a la vez, de gran corazón, benevolentes y generosos.
\vs p139 9:9 Y cuánto orgullo y gratitud sintieron estas personas humildes el día en el que el Maestro se negó a aceptar como evangelista a un hombre rico, a menos que vendiera sus bienes y ayudara al pobre. Cuando las multitudes oyeron esto y veían a los gemelos entre sus instructores, se convencieron de que Jesús no hacía acepción de personas. ¡Pero solo una institución divina ---el reino de los cielos--- podía construirse sobre unos pilares humanos tan simples!
\vs p139 9:10 Los gemelos, durante todo el tiempo relacionándose con Jesús, solamente una o dos veces se aventuraron a hacer preguntas en público. Judas se sintió cierta vez intrigado lo suficiente como para realizarla cuando el Maestro habló de revelarse a sí mismo abiertamente al mundo. Él se sintió algo decepcionado de que no quedaran más secretos entre los doce y se atrevió a preguntar: “Pero, Maestro, cuando te declares, pues, a ti mismo al mundo ¿cómo nos favorecerás a nosotros con manifestaciones especiales de tu bondad?”.
\vs p139 9:11 \pc Los gemelos prestaron con fidelidad sus servicios hasta el final, hasta los sombríos días del juicio, la crucifixión y la desesperación. En su corazón, nunca perdieron la fe en Jesús y (salvo Juan) fueron los primeros en creer en su resurrección. Si bien, no podían comprender la instauración del reino. Poco tiempo después de la crucifixión de su Maestro, volvieron a sus familias y a sus redes; su labor había concluido. Carecían de la capacidad para seguir adelante luchando en las batallas más complejas del reino. Pero vivieron y murieron conscientes de haber sido honrados y bendecidos con cuatro años de estrecha relación personal con el Hijo de Dios, el hacedor soberano de un universo.
\usection{11. SIMÓN EL ZELOTE}
\vs p139 11:1 A Simón Zelotes, el undécimo apóstol, lo eligió Simón Pedro. Era un hombre capaz, de buenos ancestros, que vivía con su familia en Cafarnaúm. Cuando se unió a los apóstoles, tenía veintiocho años. Era un apasionado agitador al igual que un hombre que hablaba mucho sin pensar lo que decía. Había sido comerciante en Cafarnaúm antes de prestar toda su atención a la organización patriótica de los zelotes.
\vs p139 11:2 \pc A Simón Zelotes se le puso a cargo de la diversión y el esparcimiento del grupo apostólico y supo organizar con eficacia las distracciones y las actividades de ocio de los doce.
\vs p139 11:3 El punto fuerte de Simón estaba en su estimulante lealtad. Cuando los apóstoles se encontraban con un hombre o una mujer indecisos respecto a su entrada en el reino, enviaban por Simón. Normalmente, en solo quince minutos, este entusiasta defensor de la salvación mediante la fe en Dios resolvía cualquier duda y hacía desaparecer las indecisiones, y veía el nacimiento de una nueva alma a la “libertad de la fe y el gozo de la salvación”.
\vs p139 11:4 El gran punto débil de Simón era su mentalidad materialista. No podía, de manera rápida, mudar de judío nacionalista a un internacionalista de mentalidad espiritual. Cuatro años fue un periodo muy corto para lograr tal transformación intelectual y emocional, pero Jesús fue siempre paciente con él.
\vs p139 11:5 \pc Lo que Simón más admiraba de Jesús era la calma del Maestro, su seguridad en sí mismo, su dignidad e inexplicable aplomo.
\vs p139 11:6 \pc Aunque Simón era un revolucionario desenfrenado, un intrépido instigador de la agitación, paulatinamente moderó su apasionada naturaleza hasta convertirse en un predicador imponente y eficaz de la “paz en la tierra y la buena voluntad para con los hombres”. Simón era un gran polemista; le gustaba discutir. Y a la hora de tratar con las mentes legalistas de los judíos eruditos o con las nimias críticas intelectuales de los griegos, era a Simón a quien siempre se le asignaba esta tarea.
\vs p139 11:7 Era rebelde por naturaleza y un iconoclasta por formación, pero Jesús consiguió su lealtad en los conceptos superiores del reino de los cielos. Siempre se había identificado con los defensores de las protestas, pero ahora se unía a los defensores del progreso, del progreso ilimitado y eterno del espíritu y la verdad. Simón era hombre de intensas lealtades y de fervientes devociones personales, y de cierto amaba profundamente a Jesús.
\vs p139 11:8 \pc Jesús no temía relacionarse con hombres de negocios, obreros, optimistas, pesimistas, filósofos, escépticos, publicanos, políticos y patriotas.
\vs p139 11:9 \pc El Maestro sostuvo muchas charlas con Simón, pero nunca conseguiría del todo convertir en internacionalista a este vehemente nacionalista judío. Jesús solía decirle a Simón que era legítimo querer la mejora del orden social, económico y político, pero siempre añadía: “Eso no es asunto del reino de los cielos. Debemos dedicarnos a hacer la voluntad del Padre. Nuestra labor es ser embajadores de un gobierno espiritual de lo alto, y no debemos preocuparnos en este momento por cosa alguna que no sea la representación de la voluntad y del carácter del Padre divino que preside el gobierno cuya autoridad representamos”. A Simón, todo aquello le resultaba difícil de comprender, pero gradualmente comenzaría a captar algo del significado de las enseñanzas del Maestro.
\vs p139 11:10 \pc Tras la dispersión por las persecuciones de Jerusalén, Simón se retiró temporalmente. Estaba literalmente destrozado. Como patriota nacionalista, había renunciado a sus ideales por respeto a las enseñanzas de Jesús; ahora, todo estaba perdido. Cayó en la desesperación, pero en unos pocos años recobró sus esperanzas y salió a proclamar el evangelio del reino.
\vs p139 11:11 Fue a Alejandría, y después de remontar el Nilo, se adentró en el corazón de África, predicando por todas partes el evangelio de Jesús y bautizando a los creyentes. Así trabajó hasta hacerse viejo y débil. Y murió y fue enterrado en el corazón de África.
\usection{12. JUDAS ISCARIOTE}
\vs p139 12:1 A Judas Iscariote, el duodécimo apóstol, lo eligió Natanael. Había nacido en Queriot, una pequeña ciudad al sur de Judea. Cuando era un muchacho, sus padres se trasladaron a Jericó, donde vivió y trabajó en las distintas empresas de su padre hasta que comenzó a interesarse por la predicación y la labor de Juan el Bautista. Los padres de Judas eran saduceos y, cuando su hijo se unió a los discípulos de Juan, lo repudiaron.
\vs p139 12:2 \pc Cuando Natanael lo encontró en Tariquea, Judas estaba buscando empleo en una empresa de secado de pescado en el extremo sur del mar de Galilea. Al unirse a los apóstoles, tenía treinta años y estaba soltero. De los doce, quizás fuese el que mayor nivel educativo tenía y, de la familia apostólica del Maestro, el único originario de Judea. Judas no poseía ningún punto fuerte que le destacase personalmente, aunque tenía muchas conductas aprendidas y rasgos externos que denotaban su cultura. Era un buen pensador, aunque no siempre un pensador verdaderamente \bibemph{honesto}. En realidad, Judas no se entendía a sí mismo; no era muy sincero consigo mismo.
\vs p139 12:3 Andrés nombró a Judas como tesorero del grupo, puesto para el que era excepcionalmente apto y, hasta el momento en el que traicionó a su Maestro, desempeñó las responsabilidades de su cargo con honradez, lealtad y gran eficacia.
\vs p139 12:4 \pc No había particularmente ningún rasgo que Judas admirara de Jesús, más allá del atractivo habitual y la admirable fascinación que la persona del Maestro ejercía. Judas nunca fue capaz de estar por encima de sus prejuicios contra sus compañeros galileos, algo propio de la gente de Judea; incluso, en su mente, criticaba muchas cosas del mismo Jesús. En su corazón, este ser engreído, oriundo de Judea, se atrevía a censurar a quien once de los apóstoles veían como el hombre perfecto, como “alguien enteramente adorable, señalado entre diez mil”. De hecho, albergaba la idea de que Jesús era un pusilánime y de que, en cierta medida, sentía temor a afirmar su propio poder y autoridad.
\vs p139 12:5 \pc Judas era un buen hombre de negocios. Se requería tacto, habilidad y paciencia, al igual que una rigurosa dedicación, para gestionar la cuentas de un idealista como Jesús, por no hablar de su pugna con los métodos de negocios caóticos de algunos apóstoles. Judas era un buen administrador de bienes, un ecónomo capaz y previsor y un perfeccionista en temas de organización. Jamás ninguno de los doce criticó a Judas. Hasta donde podían observar, Judas Iscariote era un inigualable tesorero, un hombre culto, un apóstol leal (aunque crítico a veces) y, en toda la extensión de la palabra, una persona muy exitosa. Los apóstoles amaban a Judas; era realmente uno de ellos. Debe haber \bibemph{creído} en Jesús, pero dudamos de que \bibemph{amara} verdaderamente al Maestro con todo su corazón. El caso de Judas ejemplifica la verdad de aquel proverbio que reza así: “Hay camino que al hombre le parece derecho, pero es camino que lleva a la muerte”. Es del todo posible caer víctima del apacible engaño que conduce a adaptarse plácidamente a las sendas del pecado y de la muerte. Tened la certeza de que, en los aspectos monetarios, Judas siempre fue leal a su Maestro y a sus compañeros apóstoles. El dinero no podía haber sido nunca el motivo que le indujo a traicionar al Maestro.
\vs p139 12:6 Judas era el hijo único de padres poco sensatos. Cuando era muy pequeño, lo consintieron y mimaron; fue un niño malcriado. Creció con un desmesurado concepto de su propia importancia. No era buen perdedor. Albergaba ideas vagas y tergiversadas sobre la equidad; era dado a dejarse llevar por el odio y la desconfianza. Era experto en malinterpretar las palabras y los actos de sus amigos. Durante toda su vida, Judas había desarrollado el hábito de vengarse de quienes, según imaginaba, le habían maltratado. Tenía un deficiente sentido de los valores y de las lealtades.
\vs p139 12:7 \pc Para Jesús, Judas era una aventura de la fe. Desde el principio el Maestro fue totalmente consciente de los puntos débiles de este apóstol y se dio perfecta cuenta de los peligros que conllevaba admitirlo en la comunidad. Pero es innato en la naturaleza de los Hijos de Dios dar a todo ser creado una plena y misma oportunidad de salvación y supervivencia. Jesús deseaba que no solo los mortales de este mundo sino los observadores de innumerables otros mundos supieran que, cuando existen dudas sobre la sinceridad y la dedicación incondicional de una criatura en cuanto a su devoción al reino, los jueces de los hombres acostumbran invariablemente y en todos los sentidos a aceptar al candidato dudoso. La puerta de la vida eterna está abierta de par en par para todos; “Si alguno quiere venir, que venga”; no hay restricciones ni condiciones salvo la \bibemph{fe} de aquel que viene.
\vs p139 12:8 Sencillamente, esta fue la razón por la que Jesús permitió que Judas continuara hasta el final, haciendo siempre cuanto le fuera posible por transformar y salvar a este apóstol débil y confundido. Pero, cuando la luz no se recibe ni se vive con honestidad, tiende a convertirse en oscuridad dentro del alma. Judas creció intelectualmente respecto a las enseñanzas de Jesús sobre el reino, pero no hizo avances en la adquisición de un carácter espiritual, como les sucedió a los otros apóstoles. En su experiencia espiritual, no logró realizar personalmente el adecuado progreso.
\vs p139 12:9 \pc Judas elucubraba cada vez más sobre sus decepciones personales, y finalmente se convirtió en víctima del resentimiento. Le habían lastimado tantas veces en sus sentimientos, que se volvió anormalmente desconfiado de sus mejores amigos, incluso del Maestro. En poco tiempo, llegó a obsesionarse con la idea de ajustar cuentas, de hacer cualquier cosa para vengarse, sí, hasta incluso traicionando a sus compañeros y a su Maestro.
\vs p139 12:10 Pero estas ideas perversas y dañinas no tomaron forma definitiva hasta el día en que una mujer agradecida quebró un frasco de incienso de gran valor a los pies de Jesús. Aquello le pareció a Judas un desperdicio y, cuando manifestó su enojo públicamente, y Jesús de forma tan aplastante lo desautorizó allí mismo, a oídas de todos, se sintió sobrepasado. Aquel suceso fue el desencadenante del odio, el agravio, el rencor, el prejuicio, los celos y la venganza que tenía acumulados durante toda su vida, y decidió tomar revancha incluso sin saber de quién; pero concentró toda la maldad de su naturaleza en Jesús, la \bibemph{única} persona inocente de toda la sórdida historia de su lamentable vida. Y solo porque Jesús resultó ser el actor principal en el incidente que marcó su paso desde el reino progresivo de la luz hasta el dominio de las sombras, por él mismo elegido.
\vs p139 12:11 Muchas veces, el Maestro había avisado a Judas, tanto privada como públicamente, que iba mal encaminado, pero las advertencias divinas suelen resultar inútiles cuando se trata con una naturaleza humana rencorosa. Jesús hizo todo lo posible, en consonancia con la libertad moral del hombre, para evitar que Judas optara por el camino errado. La gran prueba finalmente llegó y este hijo del resentimiento no la superó. Se rindió a los dictados agrios y sórdidos de una mente arrogante, orgullosa y vengativa, que con celeridad lo sumió en la confusión, la desesperación y la depravación.
\vs p139 12:12 Judas se involucró en la trama, abyecta e ignominiosa, de traicionar a su Señor y Maestro y, con rapidez, llevo a efecto su nefasta maquinación. Durante el desarrollo de sus furibundos planes de deslealtad y traición, pasó por momentos de arrepentimiento y vergüenza y, en estos intervalos de lucidez, cobardemente concibió, como un mecanismo de defensa de su mente, la idea de que Jesús, en el último momento, haría probablemente uso de su poder y se liberaría a sí mismo.
\vs p139 12:13 Cuando la sórdida y pecaminosa traición había acabado, este mortal apóstata, que tan despreocupadamente había vendido a su amigo por treinta piezas de plata para satisfacer un deseo de venganza, tanto tiempo albergado, salió de forma precipitada y acometió el trágico acto final de huir de las realidades de la existencia humana: el suicidio.
\vs p139 12:14 Los once apóstoles estaban horrorizados, atónitos. Jesús solo consideró al traidor con piedad. A los mundos les resulta difícil perdonar a Judas y se ha desdeñado su nombre en todo un extenso universo.
