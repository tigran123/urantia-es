\upaper{195}{Tras Pentecostés}
\author{Comisión de seres intermedios}
\vs p195 0:1 La predicación de Pedro el día de Pentecostés marcó el curso de las futuras directrices y la planificación que la mayoría de los apóstoles seguiría respecto a la proclamación del evangelio del reino. Pedro fue el verdadero fundador de la Iglesia cristiana; él llevó el mensaje cristiano a los gentiles, y los creyentes griegos lo llevaron a su vez a todo el Imperio romano.
\vs p195 0:2 Aunque los hebreos, por sus ataduras a la tradición y por el dominio que sobre ellos ejercían los sacerdotes, se negaron como pueblo a aceptar el evangelio de Jesús sobre la paternidad de Dios y la hermandad del hombre o la proclamación de Pedro y Pablo sobre la resurrección y ascensión de Cristo (a la que se llamaría cristianismo), el resto del Imperio romano fue receptivo a las enseñanzas cristianas que se estaban expandiendo. En esta época, la civilización occidental era intelectualista, estaba hastiada de guerras y era sumamente escéptica de todas las religiones y las filosofías existentes sobre el universo. Los pueblos del mundo occidental, los beneficiarios de la cultura griega, veneraban las tradiciones de su magnífico pasado. Tenían ante sí el legado de grandes logros en filosofía, arte, literatura y política. Pero, a pesar de estos logros, carecían de una religión que les fuera satisfactoria para sus almas. No hallaban complacencia para sus anhelos espirituales.
\vs p195 0:3 En esta etapa del desarrollo de la sociedad humana, las enseñanzas de Jesús, comprendidas en el mensaje cristiano, se aceptaron con rapidez. Se presentaba, pues, ante los corazones hambrientos de estos pueblos occidentales, un nuevo orden de vida. Y se creó una situación que causó un inmediato conflicto entre las antiguas prácticas religiosas y la nueva versión, cristianizada, del mensaje de Jesús que se impartía al mundo. Un conflicto de tal índole se resuelve o bien con la victoria de lo antiguo o de lo nuevo, o bien alcanzando algún grado de \bibemph{acuerdo}. La historia muestra que esta pugna acabó en consenso. En realidad, lo que el cristianismo propugnaba era excesivo para que cualquier pueblo pudiera asimilarlo en una o dos generaciones. Tal como Jesús lo había presentado a las almas de los hombres, no se trataba de un simple llamamiento espiritual; ya tempranamente adoptó una decidida actitud sobre ritos religiosos, educación, magia, medicina, arte, literatura, ley, gobierno, moralidad, regulación del sexo, poligamia y, hasta en cierta medida, incluso sobre la esclavitud. El cristianismo no vino meramente como una nueva religión ---algo que todo el Imperio romano y el Oriente estaban esperando--- sino como un \bibemph{nuevo orden de sociedad humana}. Y, estas pretensiones del cristianismo desencadenaron rápidamente el enfrentamiento socio\hyp{}moral mayor de todas las eras. Los ideales de Jesús, tal como la filosofía griega los reinterpretó y el cristianismo los socializó, pusieron en tela de juicio las tradiciones de la raza humana que se habían incorporado en la ética, la moral y las religiones de la civilización occidental.
\vs p195 0:4 \pc En un principio, el cristianismo ganó conversos únicamente en los estratos sociales y económicos más humildes. Si bien, hacia el comienzo del siglo II, lo mejor de la cultura grecorromana se fue crecientemente inclinando a este nuevo orden doctrinal cristiano, hacia esta nueva idea del propósito de la vida y de la meta de la existencia.
\vs p195 0:5 ¿Cómo pudo este nuevo mensaje de origen judío, prácticamente fallido en la tierra en la que se había originado, captar tan pronto y con tanta efectividad las mejores mentes del Imperio romano? El triunfo del cristianismo sobre las religiones filosóficas y los cultos de misterio se debió a las siguientes razones:
\vs p195 0:6 \li{1.}A la organización: Pablo fue un gran organizador y sus sucesores siguieron el paso por él marcado.
\vs p195 0:7 \li{2.}El cristianismo estaba ampliamente helenizado; integraba lo mejor de la filosofía griega, junto a lo más excelente de la teología hebrea.
\vs p195 0:8 \li{3.}Contenía, sobre todo, un \bibemph{ideal,} nuevo y grande, eco de la vida de gracia de Jesús y reflejo de su mensaje de salvación para toda la humanidad.
\vs p195 0:9 \li{4.}Los líderes cristianos accedieron a hacer tales concesiones al mitraísmo, que lograron que la mitad de los mejores adeptos de este culto se adhiriera al sistema religioso de Antioquía.
\vs p195 0:10 \li{5.}Y, además, la siguiente generación de líderes cristianos y las venideras hicieron otras nuevas concesiones al paganismo, consiguiendo incluso granjearse al emperador romano Constantino para la nueva religión.
\vs p195 0:11 \pc Pero los cristianos fueron astutos respecto a sus concesiones a los paganos en el sentido de que toleraron la inclusión de su pompa ritualista, requiriendo a cambio su aceptación de la versión helenizada del cristianismo paulino. Con los paganos llegaron incluso a una mejor alianza que con el culto mitraico; si bien, en su temprano consenso con este último, los cristianos fueron más que vencedores, porque consiguieron la eliminación de las flagrantes inmoralidades y de otras reprobables prácticas de los cultos mistéricos persas.
\vs p195 0:12 Con sensatez o no, estos primeros líderes del cristianismo renunciaron a los \bibemph{ideales} de Jesús en un intento por salvar y promover muchas de sus \bibemph{ideas;} y tuvieron un notable éxito. ¡Pero no os equivoquéis! Estos ideales del Maestro que se vieron comprometidos siguen latentes en su evangelio, y acabarán haciendo valer todo su poder en el mundo.
\vs p195 0:13 Debido a esta paganización del cristianismo, el viejo orden logró muchas victorias menores de naturaleza ritualista, pero los cristianos consiguieron supremacía en cuanto que:
\vs p195 0:14 \li{1.}Establecieron una mayor consideración de la moral humana.
\vs p195 0:15 \li{2.}Impartieron al mundo un concepto de Dios nuevo y significativamente ampliado.
\vs p195 0:16 \li{3.}En esta religión reconocida, se aseguraba ahora la anterior esperanza en la inmortalidad.
\vs p195 0:17 \li{4.}Ofrecían a Jesús de Nazaret al alma hambrienta del hombre.
\vs p195 0:18 \pc Muchas de las grandes verdades impartidas por Jesús casi se perdieron en tales tempranos consensos con otras religiones, pero estas persisten todavía, aunque aletargadas, en esta religión del cristianismo paganizado, que resultó ser la versión paulina de la vida y enseñanzas del Hijo del Hombre. Y el cristianismo, incluso antes de ser paganizado, fue, antes, helenizado por completo. El cristianismo debe mucho, muchísimo, a los griegos. Fue un griego, de Egipto, quien, en Nicea, se puso de pie con arrojo ante los miembros del concilio que allí se celebraba y defendió su punto de vista con tal valentía que la asamblea no se atrevió a ensombrecer el concepto de la naturaleza de Jesús. Esto hubiera puesto en peligro la auténtica verdad de su ministerio de gracia, que se habría perdido para el mundo. El nombre de este griego era Atanasio, y de no ser por la elocuencia y la lógica de este creyente, los persuasivos criterios de Ario habrían triunfado.
\usection{1. LA INFLUENCIA GRIEGA}
\vs p195 1:1 La helenización del cristianismo comenzó realmente ese crucial día en el que el apóstol Pablo, puesto en pie ante el Areópago de Atenas, habló a los atenienses acerca del “Dios no conocido”. Allí, bajo la sombra de la Acrópolis, este ciudadano romano anunció a los griegos su versión de la nueva religión, cuyos orígenes estaban en la tierra judía de Galilea. Y había un cierto extraño parecido entre la filosofía griega y muchas de las enseñanzas de Jesús. Tenían un objetivo común: ambas tenían como fin el \bibemph{auge del individuo}. Los griegos, aludían a un auge social y político; Jesús, a un auge moral y espiritual. Los griegos enseñaban un liberalismo intelectual conducente a la libertad política; Jesús, un liberalismo espiritual conducente a la libertad religiosa. Puestas juntas, estas dos ideas constituían un nuevo y extraordinario capítulo para la libertad humana; auguraban la libertad social, política y espiritual del hombre.
\vs p195 1:2 El cristianismo tuvo su origen y triunfó sobre todas las otras religiones rivales debido primordialmente a dos factores:
\vs p195 1:3 \li{1.}La mente griega se prestaba a tomar ideas nuevas y válidas incluso de los judíos.
\vs p195 1:4 \li{2.}Pablo y sus sucesores eran proclives a la conciliación aunque de forma astuta y sagaz; eran hábiles negociadores teológicos.
\vs p195 1:5 \pc Cuando Pablo, de pie, predicó “Cristo y este crucificado” en Atenas, los griegos estaban espiritualmente hambrientos; eran inquisitivos, estaban interesados y verdaderamente buscaban la verdad espiritual. No olvidéis nunca que, en un principio, los romanos lucharon contra el cristianismo, mientras que los griegos lo acogieron, siendo ellos los que literalmente forzaron más tarde a los romanos a aceptar esta nueva religión, tal como se había modificado y formaba parte de la cultura griega.
\vs p195 1:6 Los griegos veneraban la belleza, los judíos, la santidad, pero ambos pueblos amaban la verdad. Durante siglos, los griegos habían reflexionado con detenimiento y debatido encarecidamente acerca de todos los problemas humanos ---sociales, económicos, políticos y filosóficos--- salvo de la religión. Pocos griegos habían prestado gran atención a la religión, ni siquiera tomaban su propia religión demasiado en serio. A lo largo de los siglos, los judíos habían ignorado esos otros campos del pensamiento, y se habían centrado en la religión. Tomaban su religión con gran seriedad, con demasiada seriedad. El resultado conjunto de siglos de pensamientos de estos dos pueblos, iluminado gracias al contenido del mensaje de Jesús, se convertía ahora en la fuerza impulsora de un nuevo orden de sociedad humana y, en cierta medida, de un nuevo orden de creencias y prácticas religiosas humanas.
\vs p195 1:7 \pc La influencia de la cultura griega ya había penetrado en las tierras del Mediterráneo occidental cuando Alejandro difundió la civilización helénica por el mundo del Oriente Próximo. Los griegos gestionaban bien su religión y su política mientras vivían en pequeñas ciudades\hyp{}estado; pero cuando el rey macedonio se determinó a expandir Grecia hasta convertirla en un Imperio que se extendía desde el Adriático hasta el Indo, comenzaron las dificultades. El arte y la filosofía de Grecia se propagaron en paralelo a la expansión imperial, pero no fue así en el caso de la administración política ni de su religión. Una vez que las ciudades\hyp{}estado de Grecia se convirtieron en un Imperio, sus dioses, bastante regionales, resultaron algo extraños. En realidad, los griegos estaban buscando \bibemph{un solo Dios,} un Dios más grande y excelente, cuando les llegó la versión cristianizada de la más antigua religión judía.
\vs p195 1:8 El Imperio helénico, como tal, no podía perdurar. Su influencia cultural continuó, pero subsistió solo hasta después de adquirir del oeste el genio político de los romanos para la administración del Imperio y de obtener del este una religión cuyo único Dios poseía dignidad imperial.
\vs p195 1:9 En el siglo I d. C., la cultura helénica ya había alcanzado su más alto nivel de desarrollo y había comenzado su regresión; el conocimiento avanzaba, pero su genialidad decaía. Fue en este mismo momento cuando las ideas y los ideales de Jesús, parcialmente enunciados en el cristianismo, se convirtieron en parte de la salvación de la cultura y el conocimiento de los griegos.
\vs p195 1:10 Alejandro tomó Oriente llevando con él el don cultural de la civilización griega; Pablo invadió Occidente con la versión cristiana del evangelio de Jesús. Y dondequiera que la cultura griega predominara a lo largo de Occidente, allí echaba raíces el cristianismo helenizado.
\vs p195 1:11 \pc La versión oriental del mensaje de Jesús, pese a que permaneció más fiel a sus enseñanzas, siguió los pasos de la inflexible actitud de Abner. Nunca llegaría a avanzar como lo hizo la versión helenizada del mensaje de Jesús, y llegó incluso a desaparecer finalmente en el movimiento islámico.
\usection{2. LA INFLUENCIA ROMANA}
\vs p195 2:1 Los romanos asumieron totalmente la cultura griega, instaurando un gobierno representativo en el lugar de un gobierno por sorteo. Y en poco tiempo, este cambio favoreció al cristianismo, ya que Roma introdujo en todo el mundo occidental una nueva tolerancia hacia las lenguas, hacia los pueblos e incluso hacia las religiones extranjeras.
\vs p195 2:2 La mayoría de las primeras persecuciones de los cristianos en Roma se debió exclusivamente al desafortunado uso de la palabra “reino” en sus predicaciones. Los romanos eran tolerantes con todas y cada una de las religiones, pero les irritaba cualquier cosa que pareciera rivalidad política. Y, así, cuando estas primeras persecuciones, causadas principalmente por un equívoco, llegaron a su fin, quedó por completo despejado el camino para la difusión de las ideas religiosas. A los romanos les interesaba particularmente la gestión de los asuntos políticos; el arte y la religión les importaba bien poco, pero eran inusualmente tolerantes hacia los dos.
\vs p195 2:3 La ley oriental era severa y arbitraria; la griega, fluida y artística; la ley romana era honorable e inculcaba respeto. La educación romana generaba una lealtad inaudita e inconmovible. Los primeros romanos estaban profundamente dedicados a los valores políticos. Eran honestos, entusiastas y fieles a sus ideales, pero no tenían una religión que mereciera tal nombre. No es sorprendente que sus maestros griegos fueran capaces de convencerlos para que aceptaran el cristianismo de Pablo.
\vs p195 2:4 Y estos romanos eran un gran pueblo. Podían gobernar el Occidente, porque se gobernaban a sí mismos. Dicha inigualable honestidad, dedicación e inquebrantable autocontrol significaron el terreno ideal para la acogida favorable y el crecimiento del cristianismo.
\vs p195 2:5 A estos grecorromanos les resultó fácil dedicarse espiritualmente a una Iglesia institucional tal como lo estaban, políticamente, dedicados al Estado. Los romanos combatieron contra la Iglesia solamente cuando temieron que rivalizaran con ellos por el Estado. Roma, casi desprovista de filosofía nacional y de cultura autóctona, asumió la cultura griega como suya propia y adoptó valientemente a Cristo como su filosofía moral. El cristianismo se convirtió en la cultura moral de esta nación, pero no realmente en su religión en el sentido de ser una experiencia individual de crecimiento en el espíritu, de aquellos que habían acogido tan masivamente la nueva religión. Es cierto, no obstante, que hubo muchas personas que en efecto, a nivel individual, consiguieron penetrar más allá de la superficie de esta religión estatal, encontrando, como sustento para sus almas, los valores reales bajo los significados ocultos escondidos en las verdades latentes del cristianismo helenizado y paganizado.
\vs p195 2:6 \pc Los estoicos, con su fuerte llamamiento a “la naturaleza y a la conciencia”, habían dispuesto óptimamente a Roma para recibir a Cristo, al menos desde una perspectiva intelectual. El romano era por naturaleza y formación un intérprete de la ley; veneraba incluso las leyes de la naturaleza. Ahora, en el cristianismo, percibía las leyes de Dios en las leyes de la naturaleza. Un pueblo que pudo dar origen a Cicerón y a Virgilio estaba maduro para recibir el cristianismo helenizado de Pablo.
\vs p195 2:7 Y, así, estos griegos romanizados forzaron a judíos y a cristianos a filosofar su religión, a coordinar sus ideas y sistematizar sus ideales para adaptar las prácticas religiosas a la vida cotidiana vigente. Y todo esto se vio en gran manera favorecido por la traducción de las escrituras hebreas al griego y, luego, por la redacción del Nuevo Testamento en la lengua griega.
\vs p195 2:8 Los griegos, a diferencia de los judíos y de numerosos otros pueblos, habían creído durante mucho tiempo en algún tipo de inmortalidad, en alguna clase de supervivencia después de la muerte, y, puesto que esto representaba el corazón mismo de las enseñanzas de Jesús, sin lugar a dudas, el cristianismo llegó a ejercer un gran atractivo sobre ellos.
\vs p195 2:9 La sucesión de victorias de la cultura griega y de la política romana había consolidado las tierras mediterráneas en un solo Imperio, con una sola lengua y una sola cultura, y había preparado al mundo occidental para un único Dios. El judaísmo proporcionaba este Dios, pero, para estos griegos romanizados, el judaísmo no representaba una opción religiosa. Filón ayudó a algunas personas a reducir sus objeciones, pero el cristianismo les reveló un concepto incluso mejor de un solo Dios, y lo acogieron con entusiasmo.
\usection{3. BAJO EL IMPERIO ROMANO}
\vs p195 3:1 Tras la consolidación del gobierno político romano y después de la diseminación del cristianismo, los cristianos se encontraron con un Dios único, un gran concepto religioso, pero sin un Imperio. Los grecorromanos se encontraron con un gran Imperio, pero sin un concepto religioso de Dios apto para el culto de adoración del Imperio y la unificación espiritual. Los cristianos aceptaron el Imperio; el Imperio adoptó el cristianismo. Los romanos aportaron unidad de gobierno político; los griegos, unidad de cultura y conocimiento; el cristianismo, unidad de pensamiento y de práctica religiosa.
\vs p195 3:2 Roma superó su tradicional nacionalismo gracias al universalismo imperial y, por primera vez en la historia, y al menos de una manera nominal, resultó posible que distintas razas y naciones aceptaran una única religión.
\vs p195 3:3 El cristianismo se vio favorecido en Roma en un momento en el que existían grandes controversias entre las enérgicas enseñanzas de los estoicos y las promesas de salvación de los cultos de misterio. El cristianismo llegaba alentando el consuelo y aportando fuerza liberadora a un pueblo espiritualmente hambriento, cuya lengua no contenía la palabra “altruismo”.
\vs p195 3:4 \pc Lo que otorgó al cristianismo su mayor poder fue la forma en la que sus creyentes vivían sus vidas dedicadas al servicio e incluso la manera en la que morían por su fe durante los primeros tiempos de las brutales persecuciones.
\vs p195 3:5 \pc La enseñanza de Cristo en cuanto al amor por los niños puso pronto fin a la práctica generalizada de exponer a los niños no deseados a la muerte, particularmente a las niñas.
\vs p195 3:6 \pc La primera forma de planificación del culto de adoración cristiana se tomó en gran medida de la sinagoga judía, modificado con aportaciones del rito mitraico; más tarde, se le agregó mucho de la pomposidad pagana. Los griegos cristianizados, prosélitos del judaísmo, constituyeron los pilares de esta Iglesia cristiana primitiva.
\vs p195 3:7 \pc De toda la historia del mundo, el siglo II d. C. representó el momento más propicio para el avance en el mundo occidental de una excelente religión. Durante el siglo I, el cristianismo se había preparado, mediante luchas y concesiones, para afianzarse y difundirse con celeridad. El cristianismo adoptó al emperador; más tarde, él adoptó el cristianismo. Fue una gran era para la diseminación de una religión nueva. Existía libertad religiosa; los viajes se habían generalizado y no se ponían trabas al pensamiento.
\vs p195 3:8 El ímpetu espiritual de la aceptación nominal del cristianismo helenizado llegó a Roma demasiado tarde para poder evitar el declive moral, ya bien avanzado para entonces, o para subsanar el deterioro racial, bastante asentado y en incremento. Culturalmente, la Roma imperial necesitaba esta nueva religión, pero fue muy de lamentar que no se convirtiera, en un sentido más amplio, en instrumento de salvación espiritual.
\vs p195 3:9 Ni siquiera una excepcional religión podía salvar a un gran Imperio de las consecuencias originadas por la falta de participación individual en los asuntos del gobierno, del excesivo paternalismo, de la sobrecarga de impuestos y de los flagrantes abusos de su recaudación, de un desequilibrado balance comercial con el Levante que mermaba el oro, de la locura por la diversión, de la perniciosa uniformidad romana, de la degradación de la mujer, de la esclavitud y la decadencia racial, de plagas físicas y de una Iglesia estatal que se había institucionalizado hasta casi el punto de la esterilidad espiritual.
\vs p195 3:10 Sin embargo, Alejandría no se encontraba en estas pésimas condiciones. Las primeras escuelas continuaron manteniendo muchas de las enseñanzas de Jesús, sin menoscabarlas. Pantaenus enseñó a Clemente, y luego siguió los pasos de Natanael, proclamando a Cristo en la India. Aunque, en la formación del cristianismo, se sacrificaran algunos de los ideales de Jesús, es de justicia dejar constancia de que, a fines del siglo II, prácticamente todas las grandes mentes del mundo grecorromano se habían vuelto cristianas. El triunfo estaba próximo a culminarse.
\vs p195 3:11 Y este Imperio romano duró lo suficiente como para asegurar la supervivencia del cristianismo incluso después que el Imperio colapsara. Si bien, con frecuencia, hemos hecho algunas conjeturas sobre qué habría sucedido en Roma y en el mundo si se hubiera aceptado el evangelio del reino en lugar del cristianismo griego.
\usection{4. LAS EDADES OSCURAS DE EUROPA}
\vs p195 4:1 La Iglesia, como integrante de la sociedad y aliada de la política, estaba destinada a compartir el declive intelectual y espiritual de las llamadas “edades oscuras” de Europa. En esta época, se volvió crecientemente más monástica, ascética y oficialista. En un sentido espiritual, el cristianismo estaba en hibernación. Durante todo este período, existió, junto a esta religión aletargada y secularizada, un flujo continuo de misticismo, una vivencia espiritual imaginaria que rozaba la irrealidad y que, filosóficamente, era afín al panteísmo.
\vs p195 4:2 Durante estos siglos de oscuridad y desesperanza, la religión prácticamente se institucionalizó. La persona se sentía prácticamente perdida, anulada, ante la autoridad, la tradición y los dictados de la Iglesia. Y surgió una nueva amenaza espiritual con la creación de una constelación de “santos”, que supuestamente poseían alguna influencia especial en los tribunales divinos y que, por lo tanto, si se recurría a ellos, convenientemente, podrían interceder ante los Dioses a favor del hombre.
\vs p195 4:3 Aunque se vio impotente para detener la llegada de las edades oscuras, el cristianismo estaba, no obstante, suficientemente socializado y paganizado como para estar bien preparado y poder soportar este largo período de oscuridad moral y de estancamiento espiritual. Y, ciertamente, subsistió durante la larga noche de la civilización occidental y aún ejercía su influencia moral en el mundo cuando comenzó el renacimiento. La rehabilitación del cristianismo, tras el trascurso de estas tenebrosas eras, trajo consigo la aparición de numerosas denominaciones basadas en las enseñanzas cristianas, cuyas creencias se adecuaban a distintos tipos particulares de seres humanos: intelectuales, emocionales y espirituales. Y muchos de estos grupos de cristianos especiales, o familias religiosas, todavía perviven en el momento de hacer esta exposición.
\vs p195 4:4 \pc El cristianismo se despliega históricamente a partir de la transformación no deliberada de la religión de Jesús en una religión sobre Jesús. Sufrió, además, un proceso de helenización, paganización, secularización, institucionalización, deterioro intelectual, decadencia espiritual, hibernación moral, amenaza de extinción, revitalización posterior, fragmentación y, más recientemente, una rehabilitación relativa. Dicho legado muestra su intrínseca vitalidad y su posesión de inmensos recursos de recuperación. Y este mismo cristianismo está ahora presente en el mundo civilizado de los pueblos occidentales y se enfrenta a una lucha por su existencia, que es incluso más ominosa que esas importantes crisis que marcaron sus anteriores contiendas por lograr la supremacía.
\vs p195 4:5 \pc En este momento, la religión afronta el reto de una nueva era en la que prevalecen las mentes científicas y las tendencias materialistas. En esta imponente pugna entre lo secular y lo espiritual, la religión de Jesús acabará por triunfar.
\usection{5. EL PROBLEMA MODERNO}
\vs p195 5:1 El siglo XX ha traído nuevos problemas que tanto el cristianismo como las demás otras religiones han de resolver. Cuanto más alto escala una civilización, más necesario es el deber de “buscar primero las realidades del cielo” ante cualquier intento del hombre por estabilizar la sociedad y facilitar la solución de sus problemas materiales.
\vs p195 5:2 A menudo, la verdad se vuelve incierta e incluso engañosa cuando se desmiembra, segrega, aísla y se la analiza en extremo. La verdad viva enseña correctamente a quien busca la verdad cuando se la acoge en su totalidad y como realidad espiritual viva, no como un hecho de la ciencia material o por mediación de la inspiración artística.
\vs p195 5:3 Para el hombre, la religión constituye la revelación de su destino divino y eterno. La religión es una vivencia puramente personal y espiritual, y debe siempre distinguirse de cualesquiera otras formas elevadas de pensamientos del hombre, tales como:
\vs p195 5:4 \li{1.}Su actitud lógica hacia los elementos de la realidad material.
\vs p195 5:5 \li{2.}Su estética de la belleza en contraste con la fealdad.
\vs p195 5:6 \li{3.}Su reconocimiento ético de las obligaciones sociales y del deber político.
\vs p195 5:7 \li{4.}Su sentido de la moral humana que ni siquiera es, en sí o por sí mismo, religioso.
\vs p195 5:8 \pc Por medio de la religión, se encuentran esos valores del universo que llaman a la fe, a la confianza y a la certidumbre; la religión tiene su punto culminante en la adoración. La religión descubre para el alma esos valores supremos, que son contrapuestos a los valores relativos que la mente descubre. Solo a través de la genuina experiencia religiosa se puede acceder a la percepción de esos valores que trascienden lo humano.
\vs p195 5:9 Sin una moral basada en las realidades espirituales, ningún sistema social podría perdurar más de lo que lo haría el propio sistema solar sin la gravedad.
\vs p195 5:10 No tratéis de satisfacer vuestra curiosidad ni de complacer todos esos latentes deseos de aventura que brotan del alma durante vuestra corta vida en la carne. ¡Sed pacientes! No caigáis en la tentación de lanzaros de forma ingobernable a aventuras deleznables y sórdidas. Colocad arneses a vuestras energías y riendas a vuestras pasiones; estad en calma mientras aguardáis a que vuestra interminable andadura se despliegue majestuosa en una aventura, siempre hacia adelante, de emocionante descubrimiento.
\vs p195 5:11 \pc En la confusión acerca del origen del hombre, no perdáis de vista su destino eterno. No olvidéis que Jesús amó incluso a los niños pequeños, y que dejó claro, para siempre, el gran valor de la persona humana.
\vs p195 5:12 \pc Al contemplar el mundo, recordad que lo que veis son parches negros de maldad sobre un fondo blanco de bondad última, y no meros parches blancos de bondad que destacaran deplorablemente sobre un trasfondo negro de maldad.
\vs p195 5:13 Cuando hay tanta hermosa verdad que anunciar y proclamar, ¿por qué han de insistir tanto los hombres en la maldad del mundo simplemente porque parezca ser un hecho? Las bellezas de los valores espirituales de la verdad son más gratas y alentadoras que lo es el fenómeno del mal.
\vs p195 5:14 \pc En la religión, Jesús promovió y siguió una forma de proceder basada en la experiencia, de un modo incluso igual al que la ciencia moderna aplica sus técnicas experimentales. Hallamos a Dios mediante las directrices de nuestra percepción espiritual, pero nos aproximamos a esta percepción del alma mediante el amor por lo bello, la búsqueda de la verdad, la lealtad al deber y la adoración de la bondad divina. Si bien, de todos estos valores, el amor constituye la verdadera guía que lleva hacia una auténtica percepción espiritual.
\usection{6. EL MATERIALISMO}
\vs p195 6:1 Involuntariamente, los científicos, debido a su materialismo, han desatado el pánico en la humanidad, que por ello ha arremetido insensatamente contra las reservas morales de las eras. Pero estas reservas de la experiencia humana tienen inmensos recursos espirituales y pueden soportar cualquier acoso al que se las someta. Solo el hombre irreflexivo puede llenarse de pánico por la pérdida de los activos espirituales de la raza humana. Cuando se acabe este pánico provocado por el secularismo y el materialismo, la religión de Jesús no estará en quiebra. El depósito espiritual del reino de los cielos dotará con fe, esperanza y seguridad moral a todos los que acudan a él “en Su nombre”.
\vs p195 6:2 Sea cual fuere el conflicto aparentemente existente entre el materialismo y las enseñanzas de Jesús, podéis tener la seguridad de que en las próximas eras, las enseñanzas del Maestro triunfarán por completo. En realidad, la verdadera religión no puede involucrarse en polémicas con la ciencia; no le conciernen en absoluto las cosas materiales. Sencillamente, la religión es indiferente, aunque comprensiva, con la ciencia, al tiempo que se preocupa supremamente por el \bibemph{científico}.
\vs p195 6:3 La búsqueda de meros hechos sin la interpretación anexa de la sabiduría y de la visión espiritual de la experiencia religiosa conduce, a la larga, al pesimismo y a la desesperanza humana. Un exiguo conocimiento es en verdad desconcertante.
\vs p195 6:4 En el momento de la redacción de estos escritos, lo peor de esta era materialista ha concluido; comienza a clarear el día de un mejor entendimiento. Las mentes más elevadas del mundo científico ya no son completamente materialistas en sus filosofías, pero, en general, la gente todavía se inclina en esa dirección como resultado de postulados previos. Pero esta era de realismo físico es solo un episodio pasajero de la vida del hombre en la tierra. La ciencia moderna ha dejado intacta a la verdadera religión ---a las enseñanzas de Jesús tal como se traducen en la vida de sus creyentes---. Todo lo que la ciencia ha hecho es acabar con las infantiles ilusiones de una interpretación equivocada de la vida.
\vs p195 6:5 La ciencia es una experiencia de orden cuantitativo; la religión, una experiencia de orden cualitativo en lo que se refiere a la vida del hombre en la tierra. La ciencia aborda los fenómenos; la religión, los orígenes, los valores y las metas. Asignar \bibemph{causas} a la explicación de los fenómenos físicos es declararse ignorante de los motivos últimos y, al final, solo lleva al científico directamente de vuelta a la primera gran causa: al Padre Universal del Paraíso.
\vs p195 6:6 El fuerte viraje de una era de los milagros a una era de las máquinas ha demostrado ser muy inquietante para el hombre. La astucia y la destreza de las falsas filosofías mecanicistas contradicen sus propias argumentaciones. La propia aptitud mental, fatalista, de un defensor de esta tendencia refuta por siempre su afirmación de que el universo es un fenómeno energético ciego y sin un propósito.
\vs p195 6:7 El naturalismo mecanicista de algunos hombres, supuestamente formados, de la misma manera que el secularismo irreflexivo del hombre de la calle se preocupan exclusivamente de las \bibemph{cosas;} están yermos de verdaderos valores, de la aprobación y las gratificaciones de naturaleza espiritual, de la misma manera que están faltos de fe, de esperanza y de seguridad en lo eterno. Uno de los grandes problemas de la vida moderna es que el hombre piensa que está demasiado ocupado y sin tiempo para la meditación espiritual y la adoración religiosa.
\vs p195 6:8 El materialismo reduce al hombre a un autómata, sin alma, y lo convierte en un simple símbolo aritmético que encuentra un desvalido lugar en la fórmula matemática de un universo mecanicista y poco romántico. Pero, ¿de dónde viene todo este inmenso universo matemático sin un Maestro Matemático? La ciencia puede dar explicaciones detalladas sobre la conservación de la materia, pero la religión valida la conservación del alma de los hombres ---se preocupa de sus experiencias con las realidades espirituales y de los valores eternos---.
\vs p195 6:9 El sociólogo materialista de hoy en día observa una comunidad, hace un informe de ella y deja a las personas tal como las encontró. Mil novecientos años atrás, unos iletrados galileos observaron a Jesús dando su vida para contribuir espiritualmente a la experiencia interior del hombre y, entonces, ellos salieron y trastocaron el Imperio romano por completo.
\vs p195 6:10 Pero los líderes religiosos cometen una grave equivocación cuando tratan de llamar al hombre moderno a la lucha espiritual con los toques de trompetas de la Edad Media. La religión debe proveerse de consignas nuevas y actualizadas. Ni la democracia ni ninguna otra panacea política reemplazarán al progreso espiritual. Las falsas religiones pueden suponer una evasión de la realidad, pero Jesús, en su evangelio, instauró al hombre en la entrada misma de una realidad eterna que entraña progreso espiritual.
\vs p195 6:11 Decir que la mente “surgió” de la materia no explica nada. Si el universo fuera meramente un mecanismo y la mente fuera indisociable de la materia, jamás tendríamos dos interpretaciones diferentes de cualquier fenómeno que se sometiese a la observación. Los conceptos de la verdad, la belleza y la bondad no son intrínsecos a la física ni a la química. Una máquina no puede \bibemph{saber,} mucho menos conocer la verdad, tener hambre de rectitud y valorar la bondad.
\vs p195 6:12 La ciencia puede ser física, pero la mente del científico que percibe la verdad es, al mismo tiempo, supramaterial. La materia no conoce la verdad, tampoco puede amar la misericordia ni deleitarse en las realidades espirituales. Las convicciones morales basadas en la lucidez espiritual y enraizadas en la experiencia humana son tan reales y ciertas como las conclusiones matemáticas basadas en las observaciones físicas, si bien, se hallan en otro nivel, más elevado.
\vs p195 6:13 Si los hombres fueran solamente máquinas, responderían más o menos de manera uniforme al universo material. No existiría la individualidad y, mucho menos, el ser personal.
\vs p195 6:14 \pc El hecho del mecanismo absoluto del Paraíso en el centro del universo de los universos, en la presencia de la volición incondicionada de la Segunda Fuente y Centro, deja para siempre claro que los determinantes no constituyen la exclusiva ley del cosmos. El materialismo está allí, pero no la única cosa; el mecanismo está allí, pero no es irrestricto; el determinismo está allí, pero no está solo.
\vs p195 6:15 El universo finito de la materia acabaría siendo uniforme y determinista si no fuera por la presencia conjunta de la mente y del espíritu. Constantemente, la influencia de la mente cósmica inyecta espontaneidad incluso en los mundos materiales.
\vs p195 6:16 En cualquier ámbito de la existencia, la libertad o la iniciativa es directamente proporcional al grado de influencia espiritual y de dominio de la mente cósmica; o sea, en la experiencia humana, esto significa el grado en el que se hace realmente “la voluntad del Padre”. Y, así, el hecho de haber emprendido la búsqueda de Dios constituirá la prueba concluyente de que Dios ya os ha encontrado a vosotros.
\vs p195 6:17 La ferviente búsqueda de la bondad, la belleza y la verdad conduce a Dios. Y cualquier descubrimiento científico demuestra la existencia a la vez de libertad y de uniformidad en el universo. El descubridor realizó libremente su descubrimiento. La cosa descubierta es real y aparentemente uniforme, de lo contrario, no hubiera podido conocerse como una \bibemph{cosa}.
\usection{7. LA CUESTIONABILIDAD DEL MATERIALISMO}
\vs p195 7:1 ¡Qué necedad la del hombre de mentalidad materialista que permite que teorías tan cuestionables como las del universo mecanicista lo despoje de los inmensos recursos espirituales de la experiencia personal que la verdadera religión le proporciona! Los hechos jamás entran en disputa con la auténtica fe espiritual; las teorías pueden que sí. Sería preferible que la ciencia se dedicara a erradicar la superstición, en lugar de tratar de derribar la fe religiosa ---la creencia humana en las realidades espirituales y en los valores divinos---.
\vs p195 7:2 Materialmente, la ciencia debe hacer por el hombre lo que la religión, espiritualmente, hace por él: extender su horizonte de vida y engrandecer su persona. La verdadera ciencia no puede mantener por mucho tiempo esta disputa con la verdadera religión. El “método científico” es simplemente un rasero intelectual con el que se exploran las observaciones materiales y los logros físicos. Pero, ser material y totalmente intelectual no es en absoluto de utilidad para evaluar las realidades espirituales y las experiencias religiosas.
\vs p195 7:3 La incoherencia del mecanicista moderno es como sigue: si este universo fuera meramente material y el hombre solo una máquina, tal hombre sería completamente incapaz de reconocerse a sí mismo como máquina e, igualmente, dicho hombre\hyp{}máquina sería absolutamente inconsciente del hecho de la existencia de este universo material. En su consternación y desesperación, el científico mecanicista no ha logrado reconocer el hecho de que en su propia mente habita el espíritu, esa misma mente que posee suprapercepción espiritual para formular estos \bibemph{conceptos} equivocados y contradictorios de un universo materialista.
\vs p195 7:4 En el Paraíso, los valores de la eternidad y de la infinitud, de la verdad, la belleza y la bondad, se ocultan en los hechos fenoménicos de los universos del tiempo y el espacio. Pero, para detectar y percibir estos valores espirituales, se precisa la visión de la fe del mortal nacido del espíritu.
\vs p195 7:5 Las realidades y los valores que son parte de vuestro progreso espiritual no son una “proyección psicológica” ---un mero ensueño glorificado de la mente material---. Tales cualidades son previsiones espirituales del modelador interior, del espíritu de Dios que habita en la mente del hombre. Y no permitáis que vuestros devaneos con los hallazgos de la “relatividad”, tenuemente vislumbrados, perturben vuestra concepción de la eternidad e infinitud de Dios. Y en todas vuestras peticiones en cuanto a la necesidad de \bibemph{autoexpresión,} no cometáis la equivocación de no facilitar la \bibemph{expresión del modelador,} la manifestación de vuestro real y mejor yo.
\vs p195 7:6 Si este universo fuera únicamente material, jamás le sería posible al hombre conceptualizar el carácter mecanicista de dicha existencia, tan exclusivamente material. Este mismo \bibemph{concepto mecanicista} es, en sí mismo, un fenómeno inmaterial de la mente y toda mente es inmaterial en su origen, al margen de cómo pueda parecer estar por entero materialmente condicionada y controlada mecánicamente.
\vs p195 7:7 El mecanismo mental del hombre mortal, parcialmente evolucionado, no está dotado de sobreabundancia de coherencia ni de sabiduría. La arrogancia del hombre a menudo supera su razón y elude su lógica.
\vs p195 7:8 El mismo pesimismo del materialista más pesimista evidencia suficientemente, en sí y por sí mismo, que el universo del pesimista no es plenamente material. Tanto el optimismo como el pesimismo constituyen respuestas de índole conceptual de una mente consciente de los \bibemph{valores} al igual que de los \bibemph{hechos}. Si el universo fuera verdaderamente lo que el materialista considera que es, el hombre, como máquina humana, estaría desprovisto de cualquier reconocimiento consciente de ese mismo \bibemph{hecho}. Sin la conciencia del concepto de los \bibemph{valores} contenido en la mente nacida del espíritu, el hecho del materialismo del universo y de los fenómenos mecanicistas que operan en él serían irreconocibles por parte del hombre. Una máquina no puede ser consciente ni de la naturaleza ni del valor de otra máquina.
\vs p195 7:9 Una filosofía mecanicista de la vida y del universo no puede ser científica, porque la ciencia reconoce y aborda solamente materia y hechos. Inevitablemente, la filosofía es supracientífica. El hombre es un hecho material de la naturaleza, pero su \bibemph{vida} es un fenómeno que trasciende los niveles materiales de dicha naturaleza, debido a que manifiesta los atributos dominantes de la mente y las cualidades creativas del espíritu.
\vs p195 7:10 El deliberado intento del hombre por convertirse en un mecanicista es un trágico suceso y un fútil esfuerzo de su parte por cometer un suicidio intelectual y moral. Pero no puede hacerlo.
\vs p195 7:11 Si el universo fuese solamente material y el hombre solamente una máquina, no habría ciencia que alentase al científico a postular esta mecanización del universo. Las máquinas no pueden medirse, clasificarse ni evaluarse a sí mismas. Dicha labor científica solo puede llevarse a cabo por una entidad con el estatus de una supramáquina.
\vs p195 7:12 Si la realidad del universo es únicamente una inmensa máquina, entonces el hombre debería estar fuera del universo y aparte de él, a fin de reconocer este \bibemph{hecho} y tomar conciencia de la \bibemph{percepción} de dicha \bibemph{evaluación}.
\vs p195 7:13 \pc Si el hombre es tan solo una máquina, ¿mediante qué método puede este hombre llegar a \bibemph{creer} o a afirmar su \bibemph{conocimiento} de que no es sino una máquina? La experiencia de evaluar la propia conciencia de su yo no es nunca atribuible a una simple máquina. Un mecanicista declarado y autoconsciente es posiblemente la mejor respuesta al mecanicismo. Si el materialismo fuera un hecho, no podría haber mecanicistas conscientes de sí mismos. Es también cierto que se debe ser primeramente una persona moral antes de poder realizar actos inmorales.
\vs p195 7:14 \pc La aseveración del concepto mismo del materialismo supone una conciencia supramaterial de la mente que se atreva a afirmar tales dogmas. Un mecanismo puede deteriorarse, pero jamás progresar. Las máquinas no piensan, no crean, no sueñan, no tienen aspiraciones, no idealizan, no tienen hambre de verdad ni sed de rectitud. Sus idas no están motivadas con la pasión de servir a otras máquinas ni de elegir como la meta de su progreso eterno la sublime tarea de encontrar a Dios y de anhelar ser como él. Las máquinas nunca son intelectuales, afectivas, estéticas, éticas, morales ni espirituales.
\vs p195 7:15 El arte demuestra que el hombre no es mecanicista, pero no demuestra que es espiritualmente inmortal. El arte es la morontia humana, es el ámbito intermedio existente entre el hombre, el material, y el hombre, el espiritual. La poesía es un intento por escapar de las realidades materiales y reemplazarlas por los valores espirituales.
\vs p195 7:16 En una civilización elevada, el arte humaniza la ciencia, mientras que, a su vez, la verdadera religión ---la percepción de los valores eternos y espirituales--- espiritualiza al arte. El arte representa la valoración humana y espacio\hyp{}temporal de la realidad. La religión \bibemph{constituye} la aceptación divina de los valores cósmicos y conlleva el progreso eterno por medio de la ascensión eterna y la expansión espiritual. El arte del tiempo es peligroso solo cuando se ciega ante los principios espirituales establecidos por los modelos divinos, que la eternidad refleja como sombras temporales de la realidad. El verdadero arte actúa eficientemente sobre las cosas materiales de la vida; la religión transforma y ennoblece los hechos materiales de la vida, y jamás cesa en su valoración espiritual del arte.
\vs p195 7:17 \pc ¡Qué insensatez pretender que un autómata pueda idear una filosofía del automatismo, y qué ridículo que este mismo autómata pueda llegar a concebir que sus propios semejantes sean también autómatas!
\vs p195 7:18 \pc Cualquier interpretación científica del universo material es deficiente, salvo que brinde al \bibemph{científico} su debido reconocimiento. Ninguna apreciación del arte es genuina, salvo que se le conceda reconocimiento al artista. Ninguna valoración de la moral es digna, salvo que incluya al \bibemph{moralista}. Ningún reconocimiento de la filosofía es edificante si ignora al \bibemph{filósofo,} y la religión no puede existir sin la experiencia real del \bibemph{devoto religioso} que, en esta misma experiencia y a través de ella, trata de encontrar a Dios y conocerlo. De igual manera, el universo de los universos carece de sentido aparte del YO SOY, el Dios Infinito que lo hizo y e incesantemente lo gobierna.
\vs p195 7:19 \pc Los mecanicistas ---los humanistas--- se dejan llevar por las corrientes materiales. Los idealistas y los espiritualistas \bibemph{se atreven} a usar sus remos con inteligencia y vigor para modificar el curso, en apariencias, puramente material, de los flujos de la energía.
\vs p195 7:20 \pc La ciencia vive como resultado de las matemáticas de la mente; la música expresa el compás de las emociones. La religión es el ritmo espiritual del alma en armonía, en el espacio y el tiempo, con el compás melódico más elevado y eterno de la Infinitud. En la vida humana, la experiencia religiosa es algo verdaderamente supramatemático.
\vs p195 7:21 Mientras que el alfabeto representa el mecanismo material del lenguaje, las palabras, reveladoras del significado de mil pensamientos, grandes ideas, y nobles ideales ---amor y odio, cobardía y valor---, representan la actuación de la mente dentro del ámbito definido tanto por la ley material como por la espiritual, regido por la expresión de la voluntad de la persona y limitado por las circunstancias que la afectan en ese momento.
\vs p195 7:22 El universo no es comparable a las leyes, los mecanismos y la uniformidad que el científico descubre, y que llega a considerar como ciencia, sino, más bien lo es al \bibemph{científico} curioso, reflexivo, selectivo, integrador y juicioso que observa los fenómenos del universo y clasifica los hechos matemáticos intrínsecos a las facetas mecanicistas del lado material de la creación. Tampoco el universo es como el arte del artista, sino, más bien, como el \bibemph{artista} que se afana, sueña, aspira y avanza, buscando trascender el mundo de las cosas materiales en un intento por alcanzar una meta espiritual.
\vs p195 7:23 Es el científico, no la ciencia, quien percibe la realidad de un universo de energía y materia que evoluciona y avanza. Es el artista, no el arte, quien demuestra la existencia de un mundo morontial transitorio que media entre la existencia material y la libertad espiritual. Es el creyente religioso, no la religión, quien prueba la existencia de las realidades espirituales y de los valores divinos que hallaréis en vuestro progreso a la eternidad.
\usection{8. EL TOTALITARISMO SECULAR}
\vs p195 8:1 Pero incluso después de que el materialismo y el mecanicismo hayan sido más o menos vencidos, la influencia devastadora del secularismo del siglo XX seguirá asolando la experiencia espiritual de millones de confiadas almas.
\vs p195 8:2 A nivel mundial, fueron dos las causas que fomentaron el secularismo moderno. El antecesor del secularismo fue el carácter, falto de Dios y estrecho de miras, de la llamada ciencia ---la ciencia atea--- de los siglos XIX y XX. Su antecesora fue la totalitaria Iglesia cristiana medieval. El secularismo se inició como un movimiento de protesta contra el dominio que la Iglesia cristiana institucionalizada ejercía, casi por completo, sobre la civilización occidental.
\vs p195 8:3 En el momento de esta revelación, el clima intelectual y filosófico predominante tanto en la vida europea como en la americana es claramente secular ---humanista---. Durante trescientos años, el pensamiento occidental se ha ido secularizando progresivamente. La influencia de la religión es cada vez más de índole nominal, ritualista en gran medida. La mayoría de los cristianos profesos de la civilización occidental son en verdad, de forma inconsciente, secularistas.
\vs p195 8:4 Se necesitó un gran poder, una causa extraordinaria, para liberar el pensamiento y la vida de los pueblos occidentales de la debilitante opresión del totalitarismo eclesiástico. En efecto, el secularismo rompió el yugo de la controladora Iglesia, si bien, ahora en cambio, este secularismo amenaza con instaurar un nuevo despotismo, estas vez irreligioso, en el corazón y en la mente del hombre moderno. El Estado político tiránico y dictatorial es vástago directo del materialismo científico y del secularismo filosófico. El secularismo, en cuanto que libera al hombre de la avasallante actitud de la Iglesia institucionalizada, lo enajena a la servil atadura del Estado totalitario. El secularismo exime al hombre de la esclavitud eclesiástica tan solo para traicionarlo, entregándolo a una tiránica esclavitud política y económica.
\vs p195 8:5 \pc El materialismo niega a Dios, el secularismo simplemente lo ignora; al menos así fue su primera actitud. Más recientemente, el secularismo ha tomado una actitud más militante, pretendiendo ocupar el lugar de la religión a cuyo totalitario yugo se opuso en otro tiempo. El secularismo del siglo XX tiende a afirmar que el hombre no necesita a Dios. Pero, ¡tened cuidado! Esta filosofía sin Dios de la sociedad humana solo llevará a la agitación, a la animosidad, a la infelicidad, a la guerra y a desastres a escala mundial.
\vs p195 8:6 \pc El secularismo jamás podrá traer la paz a la humanidad. Nada puede ocupar el lugar de Dios en la sociedad humana. ¡Pero, prestad mucha atención! No os apresuréis a renunciar a los beneficios de la revuelta secular contra el totalitarismo eclesiástico. La civilización occidental disfruta hoy de muchas libertades y satisfacciones gracias a esta rebelión. El gran error cometido por el secularismo fue el siguiente: al rebelarse contra el casi completo control que ejercía la Iglesia sobre la vida, y, una vez lograda la liberación de tal tiranía eclesiástica, los secularistas continuaron para instituir su rebelión contra Dios mismo, a veces tácitamente y, a veces, abiertamente.
\vs p195 8:7 A la rebelión secularista debéis la formidable creatividad del industrialismo americano y el inaudito progreso material de la civilización occidental. Y, dado que dicha revuelta fue demasiado lejos y perdió de vista a Dios y a la religión \bibemph{verdadera,} también le siguió una inesperada cosecha de guerras mundiales y de inestabilidad internacional.
\vs p195 8:8 No es necesario sacrificar la fe en Dios para poder disfrutar de las bendiciones del levantamiento secularista moderno: tolerancia, servicio social, gobierno democrático y libertades civiles. No era preciso que los secularistas se opusieran a la religión verdadera para promover la ciencia y hacer avanzar la educación.
\vs p195 8:9 Pero el secularismo no es el único antecesor de todos estos recientes logros en la mejora de la vida. Detrás de los avances del siglo XX, no solo está la ciencia y el secularismo, sino también la acción espiritual, no reconocida e inadvertida, de la vida y las enseñanzas de Jesús de Nazaret.
\vs p195 8:10 Sin Dios, sin religión, el secularismo científico no puede coordinar sus fuerzas, armonizar sus intereses, razas y nacionalismos discrepantes y rivales. Pese a sus inigualables logros materialistas, esta secular sociedad humana se está disgregando lentamente. La mayor fuerza cohesiva que se opone a esta antagónica desintegración de la sociedad secular es el nacionalismo. Aunque el nacionalismo sea la principal barrera a la paz mundial.
\vs p195 8:11 La intrínseca debilidad del secularismo consiste en que descarta la ética y la religión y se coloca al lado de la política y del poder. Simplemente, no se puede instaurar la hermandad de los hombres ignorando o negando la paternidad de Dios.
\vs p195 8:12 El optimismo secular en el terreno social y político es una ilusión. Sin Dios, ni la liberación interior ni la libertad, ni la propiedad ni la riqueza conducirán a la paz.
\vs p195 8:13 La plena secularización de la ciencia, la educación, la industria y la sociedad solo puede llevar al desastre. Durante el primer tercio del siglo XX, se mataron a más seres humanos que durante toda la era cristiana hasta ese momento. Y esto es tan solo el comienzo de la atroz cosecha del materialismo y el secularismo, y aún están por venir devastaciones más terribles.
\usection{9. EL PROBLEMA DEL CRISTIANISMO}
\vs p195 9:1 No paséis por alto el valor de vuestro legado espiritual: el río de la verdad que fluye a lo largo de los siglos hasta regar incluso los tiempos inhóspitos de una era materialista y secular. En vuestro meritorio afán por deshaceros de los credos supersticiosos de épocas pasadas, aseguraos de aferraros fuerte a la verdad eterna. Pero, ¡sed pacientes! Cuando haya terminado la presente revuelta contra la superstición, las verdades del evangelio de Jesús subsistirán gloriosamente para iluminar un camino nuevo y mejor.
\vs p195 9:2 Si bien, el cristianismo paganizado y socializado precisa de un nuevo vínculo con las enseñanzas inalteradas de Jesús; languidece por falta de una nueva visión de la vida del Maestro en la tierra. La revelación, nueva y más plena, de la religión de Jesús está destinada a vencer al imperio del secularismo materialista y hacer caer el predominio que el naturalismo mecanicista ejerce sobre el mundo. En este momento, Urantia se encuentra, vibrante, en el umbral mismo de una de sus épocas más asombrosas y emocionantes de readaptación social, de avivamiento moral y de iluminación espiritual.
\vs p195 9:3 Las enseñanzas de Jesús, aunque considerablemente modificadas, sobrevivieron a los cultos de misterio imperantes en el momento de su nacimiento, a la ignorancia y a la superstición de la eras oscuras, e incluso está ahora triunfando, lentamente, sobre el materialismo, el mecanicismo y el secularismo del siglo XX. Y estos tiempos de grandes pruebas y de amenazantes derrotas son siempre tiempos de grandes revelaciones.
\vs p195 9:4 \pc Ciertamente, la religión precisa de nuevos líderes, hombres y mujeres espirituales que se atrevan a apoyarse únicamente en Jesús y en sus incomparables enseñanzas. Si el cristianismo insiste en dejar de lado su misión espiritual, mientras continúa ocupado con los problemas sociales y materiales, el renacimiento espiritual deberá aguardar la venida de estos nuevos maestros de la religión de Jesús, que se dedicarán de manera exclusiva a la regeneración espiritual de los hombres. Entonces, esas almas, nacidas del espíritu, proporcionarán con rapidez el liderazgo y la inspiración imprescindibles para reorganizar social, moral, económica y políticamente el mundo.
\vs p195 9:5 La era moderna se negará a aceptar una religión que sea incompatible con los hechos y que esté en desarmonía con los conceptos más elevados de la verdad, la belleza y la bondad. Se acerca la hora de redescubrir los verdaderos y primigenios pilares del cristianismo de hoy en día, tan distorsionado y menoscabado, de descubrir la verdadera vida y enseñanzas de Jesús.
\vs p195 9:6 \pc El hombre primitivo vivió una vida de sometimiento supersticioso al temor religioso. Los hombres modernos y civilizados sienten pavor a la idea de caer bajo el dominio de rígidas convicciones religiosas. El hombre reflexivo siempre tuvo miedo de verse \bibemph{sometido} a alguna religión. Cuando una religión magnífica y viva amenaza con tener algún predominio sobre él, trata, invariablemente, de racionalizar, imbuir tradiciones e institucionalizar dicha religión con la esperanza de conseguir controlarla. Con tal forma de proceder incluso una religión revelada se convierte en una religión a hechura del hombre y bajo su dominio. En esta época moderna, los hombres y las mujeres de inteligencia rehúyen la religión de Jesús, por temor a lo que esta les haría \bibemph{a} ellos ---y haría \bibemph{con} ellos---. Y todos estos temores están bien fundados. En efecto, la religión de Jesús domina y transforma a sus creyentes, exigiéndoles que dediquen sus vidas a la búsqueda del conocimiento de la voluntad del Padre de los cielos y les requiere que consagren sus energías de vida al servicio desinteresado de la hermandad del hombre.
\vs p195 9:7 En su egoísmo, los hombres y las mujeres egoístas sencillamente se niegan a pagar tal precio ni siquiera a cambio de conseguir el tesoro espiritual más grande que jamás se haya ofrecido al hombre mortal. Solo cuando el hombre se haya decepcionado lo suficiente por el doloroso desencanto que conlleva la insensata e ilusoria búsqueda del egoísmo y haya, después, descubierto la esterilidad de la religión establecida, estará dispuesto a recurrir, de todo corazón, al evangelio del reino, a la religión de Jesús de Nazaret.
\vs p195 9:8 El mundo necesita más religión vivida de manera personal. Incluso el cristianismo ---la mejor de las religiones del siglo XX--- no es solamente una religión \bibemph{sobre} Jesús, sino que también es, en gran medida, una religión que los hombres viven ajena a ellos. Toman su religión exactamente tal como se la entregan sus maestros religiosos autorizados. ¡Qué gran despertar experimentaría el mundo con tan solo ver a Jesús tal como él realmente vivió en la tierra y conocer, de primera mano, sus enseñanzas dadoras de vida! Las palabras con las que se describen las cosas bellas no pueden emocionar tanto como la visión de estas, como tampoco pueden las palabras de los credos inspirar las almas de los hombres tanto como la experiencia misma de conocer la presencia de Dios. Pero la expectante fe siempre mantendrá abierta la puerta de la esperanza del alma del hombre, para que deje entrar las realidades espirituales eternas que manifiestan los valores divinos de los mundos del más allá.
\vs p195 9:9 \pc El cristianismo se atrevió a rebajar sus ideales ante el reto de la codicia humana, la locura de la guerra y la ambición de poder; pero la religión de Jesús se erige como apelación espiritual, impoluta y suprema, que llama a lo mejor que hay en el hombre, para que se eleve por encima de todos estos legados, consecuencia de la evolución animal, y, por la gracia, logre alcanzar las alturas morales del verdadero destino humano.
\vs p195 9:10 El cristianismo está en peligro de desaparecer lentamente por el ritualismo, por la excesiva organización, por el intelectualismo y por otras tendencias no espirituales. La Iglesia cristiana moderna no es esa hermandad de dinámicos creyentes a los que Jesús comisionó para que continuamente propiciaran la transformación espiritual de las generaciones por venir de la humanidad.
\vs p195 9:11 El denominado cristianismo se ha convertido en un movimiento social y cultural, al igual que en un sistema de creencias y de prácticas religiosas. En la corriente del cristianismo moderno drenan sus aguas muchas antiguas ciénagas paganas y muchos primitivos lodazales; en esta corriente cultural de hoy día, también drenan sus aguas muchas antiguas cuencas culturales, incluyendo las provenientes de las altiplanicies galileas, que se supone constituyen su exclusiva fuente.
\usection{10. EL FUTURO}
\vs p195 10:1 De hecho, el cristianismo ha prestado un gran servicio a este mundo, pero ahora a quien más se necesita es a Jesús. El mundo precisa ver a Jesús viviendo de nuevo en la tierra en las vivencias de los mortales nacidos del espíritu, que realmente revelen al Maestro a todos los hombres. Es inútil hablar de un renacimiento del cristianismo primitivo; debéis avanzar desde donde os halláis. La cultura moderna ha de bautizarse espiritualmente con una nueva revelación de la vida de Jesús e iluminarse con un nuevo entendimiento de su evangelio de la salvación eterna. Y cuando se enaltezca así a Jesús, él atraerá a todos hacia él. Más que conquistadores, los discípulos de Jesús deberían ser fuentes desbordantes de inspiración y de vida realzada para todos los hombres. La religión es tan solo un humanismo excelso hasta que se diviniza por medio del descubrimiento de la realidad de la presencia de Dios en la experiencia personal.
\vs p195 10:2 La belleza y la sublimidad, la humanidad y la divinidad, la sencillez y la singularidad de la vida de Jesús en la tierra presentan una visión tan impresionante y atrayente de la salvación del hombre y de la revelación de Dios, que los teólogos y filósofos de todos los tiempos deberían saber contenerse antes de atreverse a elaborar credos o crear sistemas teológicos que lleven a la servidumbre espiritual, sirviéndose de tal trascendental dádiva de Dios en forma de hombre. Con Jesús, el universo dio nacimiento a un hombre mortal en quien el espíritu del amor triunfó sobre los impedimentos materiales del tiempo y superó el hecho de su origen físico.
\vs p195 10:3 \pc Tened presente siempre: Dios y el hombre se necesitan uno al otro. Son mutuamente necesarios para que el ser personal logre su expresión plena y final en la eternidad, en el destino divino último del universo.
\vs p195 10:4 “El reino de Dios está en vosotros” fue probablemente el más formidable pronunciamiento jamás antes realizado por Jesús, tras haber afirmado que su Padre es un espíritu vivo y amoroso.
\vs p195 10:5 \pc Al ganar almas para el Maestro, no será acompañando al hombre una milla, la milla de la coerción, el deber o las convenciones, la que lo transformará a él y a su mundo, sino, más bien, acompañándolo una \bibemph{segunda} milla, la milla del servicio generoso y de la devoción amorosa a la libertad, característica de los seguidores de Jesús, que salen para acoger a su hermano en el amor y llevarlo, bajo la guía espiritual, hacia la meta más elevada y divina de la existencia humana. El cristianismo está ahora voluntariosamente caminando con el hombre esa \bibemph{primera} milla, pero la humanidad languidece y se tambalea en las tinieblas morales, porque son pocos los auténticos caminantes que recorran esa segunda milla: son escasos los seguidores profesos de Jesús que realmente vivan y amen, como él enseñó a sus discípulos a vivir, a amar y a servir.
\vs p195 10:6 El llamamiento a la aventura de construir una sociedad humana, nueva y transformada, gracias al renacimiento espiritual de la hermandad del reino anunciado por Jesús, debería suscitar en todos los que creen en él una emoción no sentida por los hombres desde los días en los que caminaban con él en la tierra como sus compañeros en la carne.
\vs p195 10:7 Ningún sistema social ni régimen político que niegue la realidad de Dios puede contribuir de forma constructiva y perdurable al avance de la civilización humana. Pero la actual subdivisión y secularización actual del cristianismo constituyen sus mayores obstáculos para poder avanzar, algo particularmente cierto en lo que respecta a Oriente.
\vs p195 10:8 \pc El tradicionalismo eclesiástico es, de una vez y para siempre, incompatible con esa fe viva, con el crecimiento en el espíritu y con la inmediatez de la vivencia de los camaradas de Jesús, por la fe, en la hermandad del hombre, unidos espiritualmente como partícipes del reino de los cielos. El loable deseo de preservar las tradiciones consolidadas en el pasado lleva frecuentemente a la defensa de sistemas de adoración ya superados. El deseo bien intencionado de fomentar antiguos sistemas de pensamiento impide, en efecto, la adopción de medios y métodos, nuevos y adecuados, que sean pertinentes para satisfacer los anhelos espirituales de las mentes del hombre moderno en su expansión y desarrollo. Asimismo, aunque totalmente inconscientes, las Iglesias cristianas del siglo XX significan un gran obstáculo para que el verdadero evangelio ---las enseñanzas de Jesús de Nazaret--- logre avanzar de forma inmediata.
\vs p195 10:9 Muchas personas fervorosas, que felizmente rendirían su lealtad al Cristo del evangelio, encuentran muy difícil apoyar entusiásticamente a una Iglesia que tan pobremente manifiesta el espíritu de su vida y de sus enseñanzas, y a cuyos fieles, erróneamente, se les ha enseñado que él la fundó. Jesús no fundó la denominada Iglesia cristiana, aunque él, de la manera más consecuente con su naturaleza, la ha \bibemph{impulsado} como el mejor exponente actual de su labor de vida en la tierra.
\vs p195 10:10 Si la Iglesia cristiana se atreviera a impulsar las enseñanzas del Maestro, miles de jóvenes, aparentemente indiferentes, emprenderían apresurados tal iniciativa espiritual, y no vacilarían en llegar hasta el final de esta gran aventura.
\vs p195 10:11 El cristianismo se enfrenta seriamente con la fatalidad plasmada en una de sus propias consignas: “Una casa dividida contra sí misma no puede permanecer”. El mundo no cristiano no capitulará ante una cristiandad dividida en distintas denominaciones religiosas. El Jesús vivo es la única esperanza de alcanzar la posible unificación del cristianismo. La verdadera Iglesia ---la hermandad de Jesús--- es invisible, espiritual y se caracteriza por la \bibemph{unidad,} y no siempre por la \bibemph{uniformidad}. La uniformidad es el rasgo característico del mundo físico, de naturaleza mecanicista. La unidad espiritual es el fruto de la unión del creyente por la fe con el Jesús vivo. La Iglesia visible no debería seguir obstaculizando el progreso de la hermandad invisible y espiritual del reino de Dios. Y esta hermandad está destinada a convertirse en un \bibemph{organismo vivo,} que claramente es contraria a una organización social institucionalizada. Puede perfectamente usar tales estructuras sociales, pero nunca ser suplantada por ellas.
\vs p195 10:12 Pero ni incluso el cristianismo del siglo XX debe menospreciarse. Es el resultado del genio moral de hombres conocedores de Dios, de muchas razas, durante muchas eras y, en verdad, ha sido una de las mayores fuerzas benefactoras de la tierra. Nadie puede, por tanto, considerarlo con liviandad, pese a sus defectos tanto intrínsecos como adquiridos. El cristianismo aún consigue suscitar magníficas emociones morales en las mentes de personas reflexivas.
\vs p195 10:13 Pero no hay excusa para la implicación de la Iglesia en el comercio y en la política; este tipo de alianzas profanas constituyen una traición flagrante del Maestro. Y los auténticos amantes de la verdad tardarán en olvidar que esta poderosa Iglesia institucionalizada se ha atrevido con frecuencia a reprimir la fe naciente en las personas y a perseguir a los portadores de la verdad que aparecieran casualmente ataviados con vestiduras heterodoxas.
\vs p195 10:14 Es también muy cierto que esta Iglesia no habría sobrevivido de no haber sido por la existencia de personas en el mundo que prefirieron tal sistema de culto. Hay muchas almas, espiritualmente indolentes, que ansían una religión de rituales y de sagradas tradiciones, una religión antigua y basada en autoridades. La evolución humana y el progreso espiritual no son suficientes como para facultar a todos los hombres a prescindir de la autoridad religiosa. Y la invisible hermandad del reino podría perfectamente incluir a estos grupos afines de distintas clases sociales y actitudes con tan solo estar dispuestos a convertirse en los hijos de Dios verdaderamente llevados por la guía del espíritu. Pero en esta hermandad de Jesús no hay lugar para antagonismos sectarios, rencores entre grupos ni alegaciones de superioridad moral e infalibilidad espiritual.
\vs p195 10:15 Y es verdad que estas distintas agrupaciones cristianas pueden servir para dar cabida a numerosas clases diferentes de futuros creyentes entre los diversos pueblos de la civilización occidental, si bien, esta división de la cristiandad tiene un gran punto débil cuando intenta llevar el evangelio de Jesús a los pueblos orientales. Estas razas todavía no entienden que hay una \bibemph{religión de Jesús} que es diferente y, de alguna manera, aparte del cristianismo, el cual se ha convertido más y más en una \bibemph{religión sobre Jesús}.
\vs p195 10:16 La gran esperanza de Urantia consiste en la posibilidad de una nueva revelación de Jesús que presentara su mensaje salvífico de una manera nueva y ampliada, lo que haría unirse espiritualmente en amor y servicio a las numerosas familias de los que actualmente profesan ser sus seguidores.
\vs p195 10:17 Inclusive la educación secular podría ayudar en este gran renacimiento espiritual si prestara más atención a la labor de enseñar a los jóvenes a planificar sus vidas y a desarrollar su carácter. El propósito de la educación debería ser fomentar y profundizar en el propósito supremo de la vida, desarrollar en ellos un ser personal espléndido y bien equilibrado. Existe una gran necesidad de que se enseñe disciplina moral, en lugar de tanta gratificación personal. Sobre tales pilares, la religión podría contribuir con su incentivo espiritual a engrandecer y enriquecer la vida mortal, incluyendo la certeza y el exaltamiento de la vida eterna.
\vs p195 10:18 El cristianismo es una religión que nació de improviso y, por lo tanto, procederá a un ritmo lento y con escasos logros. Para poder actuar espiritualmente con mayor celeridad y mayores logros, se debe aguardar la nueva revelación y la aceptación más generalizada de la auténtica religión de Jesús. Pero el cristianismo es una formidable religión, si consideramos que los discípulos de un carpintero crucificado, siendo personas corrientes, pusieron en marcha esas enseñanzas que conquistaron el mundo romano en trescientos años y que, más tarde, siguieron adelante hasta triunfar sobre los bárbaros que hicieron caer a Roma. Este mismo cristianismo también venció ---absorbió y exaltó--- toda la corriente de teología hebrea y de filosofía griega. Luego, tras un estado comatoso de más de mil años, como consecuencia de una sobredosificación de misterios y paganismo, resucitó y prácticamente reconquistó todo el mundo occidental. El cristianismo contiene suficientes elementos de las enseñanzas de Jesús como para convertirse en inmortal.
\vs p195 10:19 Si el cristianismo incorporara más enseñanzas de Jesús, podría hacer mucho más para ayudar al hombre moderno a resolver sus problemas nuevos y crecientemente complejos.
\vs p195 10:20 El cristianismo adolece de una gran dificultad porque, en las mentes de todo el mundo, se ha identificado como parte del sistema social, de la vida industrial y de los principios morales de la civilización occidental; y, de este modo, el cristianismo, sin saberlo, pareció auspiciar una sociedad que se tambalea bajo la culpa de tolerar una ciencia sin idealismo, una política sin principios, una riqueza sin trabajo, una gratificación sin trabas, un conocimiento sin carácter, un poder sin conciencia y una industria sin moral.
\vs p195 10:21 La esperanza del cristianismo pasa por dejar de auspiciar los sistemas sociales y las políticas industriales de la civilización occidental, al tiempo que se inclina humildemente ante esa cruz que tan valientemente ensalza, para aprender ahí, nuevamente, de Jesús de Nazaret, las verdades más grandes que el hombre mortal pueda oír jamás: el evangelio vivo de la paternidad de Dios y de la hermandad del hombre.
