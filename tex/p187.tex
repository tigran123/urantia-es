\upaper{187}{La crucifixión}
\author{Comisión de seres intermedios}
\vs p187 0:1 Una vez que estuvieron preparados los dos bandidos, los soldados, comandados por un centurión, salieron hacia el lugar de la crucifixión. El centurión a cargo de estos doce soldados era el mismo capitán que la noche anterior había liderado a los soldados romanos a Getsemaní para arrestar a Jesús. Los romanos acostumbraban asignar a cuatro soldados a cada una de las personas que iban a ser crucificadas. Tal como las circunstancias exigían, azotaron a los bandidos antes de llevárselos para crucificarlos, si bien, Jesús no recibió ningún otro castigo; el capitán sin duda pensó que ya se le había azotado suficientemente, antes incluso de ser condenado a muerte.
\vs p187 0:2 Los dos ladrones crucificados con Jesús eran cómplices de Barrabás y se les hubiera dado muerte más tarde junto a su líder si Pilato no lo hubiera soltado con motivo del perdón que se concedía en la Pascua. Así pues, Jesús fue crucificado en lugar de Barrabás.
\vs p187 0:3 Lo que Jesús está a punto de hacer, sometiéndose a morir en la cruz, lo hace por su propia y libre voluntad. Cuando predijo este hecho, él dijo: “El Padre me ama y me sostiene porque estoy dispuesto a dar mi vida. Pero la volveré a tomar. Nadie me la puede quitar ---por mí mismo la doy---. Tengo autoridad para darla y tengo autoridad para tomarla. Este mandamiento recibí de mi Padre”.
\vs p187 0:4 Era justo antes de las nueve de esta mañana cuando los soldados salieron del pretorio con Jesús camino del Gólgota. Muchos de los que los siguieron apoyaban secretamente a Jesús, pero la mayoría de los que iban en el grupo, unos doscientos o más, eran o bien sus enemigos o bien personas ociosas llevadas por la curiosidad, que meramente buscaban gozar con el impacto emocional de presenciar las crucifixiones. Solo algunos pocos de los líderes judíos fueron a ver la muerte de Jesús en la cruz. Los demás, sabiendo que Pilato lo había entregado a los soldados romanos y que estaba condenado a muerte, se ocuparon de su reunión en el templo, en la que trataron respecto a lo que se debería hacer con los seguidores de Jesús.
\usection{1. DE CAMINO AL GÓLGOTA}
\vs p187 1:1 Antes de salir del patio del pretorio, los soldados pusieron el travesaño de la cruz sobre los hombros de Jesús. Era habitual que el condenado tuviera que cargar con él hasta el sitio de la crucifixión. No portaba toda la cruz, sino solo este madero más corto. Las piezas de madera más largas y verticales de las tres cruces ya se habían transportado al Gólgota y, en el momento en el que los soldados llegaron con sus presos, estas estaban ya hendidas firmemente en tierra.
\vs p187 1:2 Conforme a la costumbre, el capitán lideraba la procesión, llevando pequeñas tablillas blancas en las que se habían escrito con carboncillo los nombres de los delincuentes y la naturaleza de los hechos delictivos cometidos. Para los dos ladrones, el centurión tenía inscripciones con sus nombres, y debajo de estos había una sola palabra escrita: “Bandido”. Tras clavar a la víctima al travesaño e izarlo hasta su lugar adecuado en la estaca vertical, era costumbre clavar igualmente esta inscripción en la parte superior de la cruz, justo por encima de la cabeza del malhechor. Esto se hacía para que todos los testigos pudieran conocer el delito por el que se crucificaba al condenado. El título que el centurión portaba para colocarlo en la cruz de Jesús lo había escrito Pilato mismo en latín, griego y arameo, y decía: “Jesús de Nazaret, Rey de los judíos”.
\vs p187 1:3 Algunas de las autoridades judías, aún presentes cuando Pilato escribió este título, protestaron enérgicamente para que no se llamara a Jesús “Rey de los judíos”. Pero Pilato les recordó que dicha acusación era parte de los cargos que habían dado pie a su condena. Cuando los judíos vieron que no podían convencer a Pilato de que cambiara de opinión, abogaron porque lo modificara al menos para que en ella se leyera: “Él dijo: ‘yo soy el rey de los judíos’”. Pero Pilato se mostró inflexible y no quiso cambiar lo que había escrito. Ante todos sus demás ruegos, él se limitaba a responder: “Lo que he escrito, he escrito”.
\vs p187 1:4 Era común desplazarse al Gólgota yendo por el trayecto más largo para que un gran número de personas pudieran ver al condenado, pero aquel día tomaron la ruta más directa hasta la puerta de Damasco, la salida de la ciudad hacia el norte, y, siguiendo esta carretera, llegaron pronto al Gólgota, el sitio oficialmente designado en Jerusalén para las crucifixiones. Más allá del Gólgota estaban las villas de las personas adineradas y, al otro lado de la carretera, estaban las tumbas de muchos judíos acomodados.
\vs p187 1:5 \pc La crucifixión no era una forma de castigo propiamente judía. Tanto los griegos como los romanos habían aprendido este modo de ejecución de los fenicios. Incluso Herodes, con toda su crueldad, nunca llegó a recurrir a la crucifixión. Los romanos nunca crucificaban a los ciudadanos romanos; solo los esclavos y los pueblos sometidos eran víctimas de esta deshonrosa manera de morir. Durante el asedio de Jerusalén, tan solo cuarenta años después de la crucifixión de Jesús, todo el Gólgota se cubrió de miles y miles de cruces en las que, día tras día, perecía lo mejor de la raza judía. De hecho, fue una terrible cosecha por la siembra de aquel día.
\vs p187 1:6 \pc Conforme la procesión de la muerte recorría las estrechas calles de Jerusalén, una gran cantidad de mujeres judías, tiernas de corazón, que habían oído las palabras de ánimo y de compasión de Jesús, y que conocían el ministerio amoroso de su vida, no pudieron contener el llanto cuando vieron cómo lo llevaban a una muerte tan innoble. Al pasar cerca de ellas, muchas de estas mujeres lloraban y hacían lamentación por él. Y, cuando algunas de estas se atrevieron incluso a seguirlo y caminar a su lado, el Maestro, volviéndose hacia ellas, les dijo: “Hijas de Jerusalén, no lloréis por mí, sino llorad por vosotras mismas y por vuestros hijos. Mi labor está al acabar ---pronto iré a mi Padre--- pero se aproximan momentos de terrible aflicción para Jerusalén. He aquí que vendrán días en los que diréis: bienaventuradas sean las estériles y aquellas cuyos pechos nunca dieron de mamar a sus pequeños. Entonces, pediréis que las rocas de las colinas caigan sobre vosotras para poder libraros de los terrores de vuestras tribulaciones”.
\vs p187 1:7 Estas mujeres de Jerusalén eran realmente valientes al solidarizarse con Jesús, ya que estaba estrictamente prohibido por la ley dar muestras de cualquier sentimiento amigable hacia quienes llevaban a su crucifixión. Se permitía a la turba abuchear, burlar o ridiculizar al condenado, pero no expresar ninguna conmiseración. Aunque Jesús agradecía tales expresiones de comprensión en aquella hora oscura en la que sus amigos estaban escondidos, no quería que estas bondadosas mujeres sufrieran la reprobación de las autoridades por atreverse a mostrar misericordia hacia él. Incluso en aquel momento, Jesús pensaba poco en sí mismo, sino más bien en los terribles y trágicos días que sobrevendrían a Jerusalén y a toda la nación judía.
\vs p187 1:8 Camino de su crucifixión, el Maestro andaba con gran dificultad y se sentía agotado, prácticamente exhausto. No había ingerido ni comida ni bebida desde la última cena en la casa de Elías Marcos; ni tampoco, ni por un momento, le habían permitido dormir. Además, había tenido que soportar una comparecencia tras otra hasta la hora en la que se le condenó, sin mencionar los vejatorios azotes junto con el sufrimiento físico recibido y la pérdida de sangre. A todo esto había que añadir su extremada angustia mental, su intensa tensión espiritual y un terrible sentimiento de soledad humana.
\vs p187 1:9 Poco después de cruzar la puerta de salida de la ciudad, al tambalearse Jesús yendo con el travesaño a cuestas, sus fuerzas físicas desfallecieron de momento y cayó bajo el gran peso de su carga. Los soldados le gritaron y patearon, pero no podía levantarse. Cuando el capitán vio aquello, sabiendo todo lo que Jesús llevaba soportado, mandó a sus soldados a que parasen. Luego ordenó a alguien que pasaba por allí, a un cierto Simón de Cirene, que tomara el travesaño de los hombros de Jesús y lo forzó a llevarlo el resto del camino hasta llegar al Gólgota.
\vs p187 1:10 \pc Este hombre, Simón, había venido desde el norte de África, de Cirene, para asistir a la Pascua. Estaba parando junto con otros cirineos fuera de los muros de la ciudad e iba de camino al templo para asistir a sus servicios en la ciudad, cuando el capitán romano le mandó que portara el madero de la cruz de Jesús. Simón se quedó allí todo el tiempo que el Maestro tardó en morir en la cruz, llegando a conversar con muchos de sus amigos y con sus enemigos. Tras la resurrección y antes de dejar Jerusalén, Simón se convirtió con valentía en creyente del evangelio del reino y, cuando volvió a su hogar, guió a toda su familia al reino celestial. Sus dos hijos, Alejandro y Rufo, fueron competentes maestros del nuevo evangelio en África. Si bien, Simón nunca supo que Jesús, cuya carga llevó, y el tutor judío que cierta vez había entablado amistad con su hijo herido, eran la misma persona.
\vs p187 1:11 \pc Fue algo después de las nueve cuando esta procesión de la muerte llegó al Gólgota, y los soldados romanos se dispusieron a la tarea de clavar a los dos bandidos y al Hijo del Hombre en sus respectivas cruces.
\usection{2. LA CRUCIFIXIÓN}
\vs p187 2:1 Primeramente, los soldados ataron con cuerdas los brazos del Maestro al travesaño de madera y, luego, clavaron allí sus manos. Cuando habían izado el travesaño hasta arriba, en la estaca, y haberlo clavado firmemente a este palo vertical de la cruz, ataron sus pies y los clavaron al madero utilizando un clavo largo para penetrarle ambos pies. La estaca vertical llevaba encajado, a una altura adecuada, un gran taco, que servía de asiento para soportar el peso del cuerpo. La cruz no era alta, los pies del Maestro se encontraban a menos de un metro del suelo. De ahí que él pudiera oír todo lo que se decía en su afrenta y pudo ver con claridad la expresión de los rostros de todos los que tan desconsideradamente se mofaban de él. Y, de igual manera, los allí presentes pudieron oír con facilidad todo lo que dijo Jesús durante aquellas horas de persistente tortura y de muerte lenta.
\vs p187 2:2 Se acostumbraba a despojar de toda su vestimenta a los que iban a ser crucificados, si bien, dado que los judíos ponían muchas objeciones a que se mostrara en público la desnudez del cuerpo humano, los romanos siempre facilitaban un paño adecuado como para cubrir los genitales de los crucificados en Jerusalén. Por consiguiente, después de que despojaran a Jesús de su ropa y, antes de colocarlo en la cruz, lo vistieron de esa manera.
\vs p187 2:3 Se recurría a la crucifixión cuando se quería infligir un castigo cruel y dilatado en el tiempo, ya que a veces la víctima no moría hasta pasados varios días. En Jerusalén había fuertes sentimientos en contra de la crucifixión, y existía una agrupación de mujeres judías, que siempre enviaba a una representante de ellas a las crucifixiones con el objeto de darle a la víctima un vino adormecedor que le aliviara su sufrimiento. Pero cuando Jesús probó esta bebida embriagante, se negó a beberla a pesar de estar sediento. El Maestro decidió conservar su conciencia humana hasta el último momento. Deseaba afrontar la muerte, incluso de esta manera tan cruel e inhumana, y conquistarla, sometiéndose de forma voluntaria a experimentar la existencia humana en su totalidad.
\vs p187 2:4 Antes de que colocaran a Jesús en la cruz, ya estaban en sus respectivas cruces los dos bandidos, los cuales, en todo momento, maldijeron y escupieron a sus verdugos. Las únicas palabras que Jesús pronunció, cuando lo clavaban en el travesaño, fueron “Padre, perdónalos porque no saben lo que hacen”. Jesús no podía haber intercedidos de forma tan misericordiosa y amorosa por sus propios verdugos, si tales pensamientos de cariñosa devoción a los seres humanos no hubieran sido la fuente misma de una vida, la suya, por entero dedicada al servicio desinteresado de los demás. Las ideas, los motivos y los anhelos de toda una vida se desvelan manifiestamente en los momentos de crisis.
\vs p187 2:5 Una vez que izaron al Maestro en la cruz, el capitán clavó el título por encima de su cabeza, en el que se podía leer en tres lenguas: “Jesús de Nazaret, Rey de los Judíos”. Los judíos estaban furiosos por lo que ellos creían un insulto. Pero Pilato estaba irritado por sus irrespetuosas maneras; sentía que se le había intimidado y humillado, y eligió esta trivial venganza. Podía haber escrito “Jesús, un rebelde”. Pero él sabía bien que estos judíos de Jerusalén detestaban el mero nombre de Nazaret, y estaba decidido a humillarlos así. Sabía también que se sentirían muy ofendidos al ver cómo a este galileo, que había sido ejecutado, se lo llamaba “Rey de los judíos”.
\vs p187 2:6 Muchos de los líderes judíos, cuando conocieron cómo Pilato había tratado de insultarlos poniendo aquella inscripción en la cruz de Jesús, se dirigieron de prisa al Gólgota, si bien, al llegar, no se atrevieron a quitarla porque los soldados romanos estaban de guardia. Al no poder retirar el título de la cruz, estos líderes se entremezclaron con la multitud e hicieron todo lo posible para incitar a la sorna y al ridículo, no fuera que la inscripción se tomara en serio.
\vs p187 2:7 El apóstol Juan, con María la madre de Jesús, Ruth y Judá, llegaron al lugar de la crucifixión poco después de que alzaran a Jesús en la cruz hasta su posición correspondiente, y justo cuando el capitán estaba clavando el título por encima de la cabeza del Maestro. Juan fue el único de los once apóstoles que fue testigo de la crucifixión, e incluso él tampoco pudo estar todo el tiempo presente, ya que corrió a Jerusalén para traer de vuelta con él a su madre y a sus amigas, inmediatamente después de llevar hasta allí a la madre de Jesús.
\vs p187 2:8 Cuando Jesús vio a su madre junto a Juan, su hermano y su hermana sonrió, pero no dijo nada. Entretanto, los cuatro soldados asignados a la crucifixión del Maestro, como era costumbre, se habían repartido, sus vestidos entre ellos: uno se llevó las sandalias, otro el turbante, otro el cinto y, el cuarto, el manto. Tan solo quedó la túnica, la cual era sin costura y llegaba hasta cerca de las rodillas, y de la que harían cuatro trozos, pero cuando vieron que se trataba de una prenda de vestir tan inusual, decidieron echar suertes sobre ella. Jesús los miraba desde arriba mientras se dividían sus vestimentas, al mismo tiempo que la desconsiderada multitud vociferaba contra él.
\vs p187 2:9 \pc Vino bien que los soldados romanos se apropiaran de los vestidos del Maestro. De otra manera, si sus seguidores se hubieran hecho con ellos, habrían estado tentados de hacer un uso supersticioso de tales reliquias rindiéndoles culto. El Maestro deseaba que sus seguidores no tuvieran nada material que guardara relación con su vida en la tierra. Quería dejar para la humanidad únicamente el recuerdo de una vida humana consagrada al elevado ideal espiritual de hacer la voluntad del Padre.
\usection{3. LOS TESTIGOS DE LA CRUCIFIXIÓN}
\vs p187 3:1 Sobre las nueve y media de la mañana de ese viernes, colgaron a Jesús en la cruz. Antes de las once, más de mil personas se habían congregado para poder presenciar aquel acontecimiento de la crucifixión del Hijo del Hombre. Durante aquellas terribles horas, las multitudes invisibles del universo guardaron silencio ante la visión de aquella extraordinaria circunstancia del creador muriendo como una criatura y padeciendo la más innoble muerte por la que se pudiera condenar a un delincuente.
\vs p187 3:2 En un momento u otro, estaban junto a la cruz María, Ruth, Judá, Juan, Salomé (la madre de Juan) y un grupo de honestas creyentes, entre ellas se contaba María, mujer de Cleofas y hermana de la madre de Jesús, María Magdalena y Rebeca, antigua residente de Séforis. Estos y otros amigos de Jesús mantuvieron silencio mientras eran testigos de su gran paciencia y fortaleza y de su intenso sufrimiento.
\vs p187 3:3 Muchos de los que pasaban lo insultaban moviendo la cabeza y diciendo: “Tú el que derribarías el templo y en tres días lo reedificarías, sálvate a ti mismo. Si eres el Hijo de Dios, ¿por qué no desciendes de la cruz?”. De esta manera, algunos de los líderes de los judíos se mofaban de él diciendo: “A otros salvó, pero a sí mismo no se puede salvar”. Otros decían: “Si eres el rey de los judíos, baja de la cruz y creeremos en ti”. Y más tarde se mofaron aún más, diciendo: “Confió en que Dios lo libraría. Afirmó incluso que era el Hijo de Dios, pero miradlo ahora ahí: crucificado entre dos ladrones”. Hasta los dos ladrones lo injuriaban y le hacían reproches.
\vs p187 3:4 187:3.4Debido a que no quiso responder a sus provocaciones y, dado que se aproximaba el mediodía de aquel día tan especial de la preparación, la mayor parte de aquella muchedumbre mordaz y vociferante se había marchado hacia las once y media, quedando allí menos de cincuenta personas. Los soldados se dispusieron ahora a comer y beber su vino barato y amargo, mientras se acomodaban para vigilar durante largo rato a los condenados hasta su muerte. Mientras compartían su vino, brindaron sarcásticamente por Jesús, diciendo: “¡Salud y buena fortuna al rey de los judíos!”. Y se quedaron sorprendidos de la actitud tolerante del Maestro ante sus desprecios y burlas.
\vs p187 3:5 Al verlos Jesús comer y beber, mirándolos dijo: “Tengo sed”. Cuando el capitán de los guardias oyó a Jesús decir “tengo sed”, tomó algo de vino de su botella y, empapando su tapón esponjoso, lo colocó en la punta de una jabalina y lo alzó para que Jesús pudiera humedecer sus labios resecos.
\vs p187 3:6 Jesús se había propuesto vivir sin recurrir a sus poderes sobrenaturales y eligió asimismo morir en la cruz como un ordinario mortal. Había vivido como hombre y quiso morir como hombre: haciendo la voluntad del Padre.
\usection{4. EL LADRÓN, DESDE SU CRUZ}
\vs p187 4:1 Uno de los bandidos insultaba a Jesús, diciéndole: “Si tú eres el Hijo de Dios, ¿por qué no te salvas a ti mismo y a nosotros?”. Pero cuando había recriminado a Jesús, el otro ladrón, que había oído muchas veces las enseñanzas del Maestro, dijo: “¿Es que no temes ni siquiera a Dios? ¿No ves que nosotros padecemos justamente por nuestros hechos, pero este hombre padece injustamente? Sería mejor que buscáramos perdón por nuestros pecados y salvación para nuestras almas”. Cuando Jesús oyó al ladrón decir esto, volvió la cara hacia él y sonrió con aprobación. Al ver el malhechor que el rostro de Jesús se había girado para mirarlo, se llenó de valor, avivó la parpadeante llama de su fe, y dijo: “Señor, acuérdate de mí cuando vengas en tu reino”. Entonces Jesús le dijo: “De cierto, de cierto, te digo hoy, que algún día estarás conmigo en el Paraíso”.
\vs p187 4:2 En medio de los dolores de la muerte humana, el Maestro tuvo tiempo para escuchar la confesión de fe del fervoroso bandido. Cuando este ladrón buscó la salvación, encontró la liberación del mal. Anteriormente, se había visto muchas veces movido a creer en Jesús, pero únicamente en aquellas últimas horas de conciencia se convirtió con todo su corazón a las enseñanzas del Maestro. Cuando vio el modo en el que Jesús afrontaba la muerte en la cruz, este ladrón no pudo resistirse más al convencimiento de que este Hijo del Hombre era en verdad el Hijo de Dios.
\vs p187 4:3 \pc Durante este hecho de la conversión y de la aceptación por Jesús del ladrón en el reino, el apóstol Juan no se encontraba allí porque había ido a la ciudad a recoger a su madre y a sus amigas para llevarlas al lugar de la crucifixión. El capitán de la guardia, convertido al evangelio, contaría más tarde esta historia a Lucas.
\vs p187 4:4 El apóstol Juan se refirió a la crucifixión tal como él la recordaba dos tercios de siglo después de que ocurriera. Los otros textos se basan en el relato del centurión romano de turno, que, debido a lo que vio y oyó, creyó después en Jesús y entraría en la hermandad plena del reino de los cielos en la tierra.
\vs p187 4:5 \pc Aquel joven, aquel bandido arrepentido, se había visto alentado a llevar una vida de violencia y fechorías por quienes alababan esta trayectoria de asaltos como una contundente forma de protesta patriótica contra la opresión política y la injusticia social. Y este tipo de doctrina, junto con el impulso a la aventura, animaba a muchos jóvenes, por otro lado, bien intencionados, a enrolarse en estas atrevidas incursiones de robo. Este joven había considerado a Barrabás como un héroe. Ahora, se daba cuenta de que había estado equivocado. Allí, en la cruz, a su lado, veía realmente a un gran hombre, a un verdadero héroe. Allí había un héroe que enardecía su fervor e inspiraba en él los más elevados pensamientos de dignidad moral y vivificaba todos sus ideales de valentía, hombría y arrojo. Al contemplar a Jesús, emanó de su corazón un arrollador sentimiento de amor, lealtad y genuina grandeza.
\vs p187 4:6 Si otra persona cualquiera de la vociferante muchedumbre hubiera experimentado el nacimiento en su alma de la fe y hubiera apelado a la misericordia de Jesús, él la habría acogido con la misma amorosa consideración que mostró hacia el bandido que había creído en él.
\vs p187 4:7 \pc Juan regresó de la ciudad justo después de que el ladrón arrepentido oyera la promesa del Maestro de que algún día se encontrarían en el Paraíso. Venía acompañado de su madre y de un grupo de casi doce mujeres creyentes. Juan se colocó cerca de María la madre de Jesús para sostenerla. Su hijo Judá estaba en el otro lado. Al mediodía, cuando Jesús los vio allí, le dijo a su madre: “¡Mujer, he ahí a tu hijo!”. Y, entonces, dirigiéndose a Juan, le dijo: “¡Hijo mío, he ahí a tu madre!”. Y, luego, hablándoles a ambos, dijo: “Deseo que os marchéis de aquí”. Y, así pues, Juan y Judá se llevaron a María del Gólgota. Juan se dirigió con la madre de Jesús al sitio donde se quedaba en Jerusalén y, después, se apresuró de vuelta al punto de la crucifixión. Tras la Pascua, María volvió a Betsaida. Allí viviría en la casa de Juan durante el resto de su vida natural. María no vivió ni un año entero tras la muerte de Jesús.
\vs p187 4:8 Una vez que María se fue, las otras mujeres se retiraron a escasa distancia y se quedaron con Jesús hasta que expiró en la cruz, y seguían aún allí cuando bajaron el cuerpo del Maestro y se lo llevaron para darle sepultura.
\usection{5. ÚLTIMAS HORAS EN LA CRUZ}
\vs p187 5:1 Aunque un fenómeno así no ocurría tan tempranamente en aquella estación, poco después de las doce del día, el cielo se oscureció por la arena fina en el aire. La gente de Jerusalén sabía que esto significaba la llegada de una tormenta de arena que venía acompañada de aire caliente del desierto árabe. Antes de la una, el cielo se ensombreció tanto que ocultó al sol, y el resto de la multitud se apresuró a volver a la ciudad. Algo más tarde, cuando el Maestro rindió su vida, quedaban allí menos de treinta personas. Solo permanecían los trece soldados romanos y un grupo de unos quince creyentes, todas mujeres excepto dos, Judá, el hermano de Jesús, y Juan Zebedeo, que regresó justo antes de que el Maestro expirara.
\vs p187 5:2 Algo después de la una, en medio de la creciente tiniebla provocada por la feroz tormenta de arena, a Jesús comenzó a fallarle la conciencia humana. Había expresado sus últimas palabras de misericordia, perdón y consejo. Había manifestado su último deseo, referido al cuidado de su madre. Durante esta hora próxima a la muerte, la mente humana de Jesús recurrió a la repetición de numerosos pasajes de las escrituras hebreas, particularmente de los salmos. El último pensamiento consciente del Jesús humano fue esta recitación mental de una parte del Libro de los salmos, ahora conocida como el salmo veinte, el veintiuno y el veintidós. Aunque sus labios se movían frecuentemente, se encontraba demasiado débil como para pronunciar las palabras conforme estos pasajes, que tan bien conocía de memoria, le pasaban por la mente. Solo varias veces pudieron captar algunas de estas, quienes estaban junto a él, y que fueron: “Conozco que el Señor salvará a su ungido”, “Alcanzará tu mano a todos mis enemigos” y “Dios mío, Dios mío, ¿por qué me has desamparado?”. Jesús, en ningún momento, albergó la mínima duda de que había vivido según la voluntad del Padre; y nunca cuestionó que en aquel instante él renunciaba a su vida en la carne de acuerdo con la voluntad del Padre. No sintió que el Padre lo había desamparado; estaba simplemente recitando en este momento en el que su conciencia se desvanecía muchos versos de las escrituras entre los que estaba este salmo veintidós, que comienza con “Dios mío, Dios mío, ¿por qué me has desamparado?”. Y resultó que este fue uno de los pasajes que pudo decir con la suficiente claridad como para que lo oyeran aquellos próximos a él.
\vs p187 5:3 \pc El último deseo que el Jesús mortal solicitó de sus semejantes fue sobre la una y media cuando, por segunda vez, dijo: “Tengo sed”, y el mismo capitán de la guardia le humedeció una vez más los labios con la misma esponja mojada en el vino amargo, comúnmente llamado vinagre en aquellos días.
\vs p187 5:4 \pc La tormenta de arena creció en intensidad y los cielos se fueron oscureciendo cada vez más. Todavía quedaban allí los soldados y el pequeño grupo de creyentes. Los soldados estaban agachados cerca de la cruz; se habían acurrucado todos juntos para protegerse de la cortante arena. La madre de Juan y otras personas miraban estas cosas de lejos, cobijadas de alguna manera bajo un saliente rocoso. Cuando el Maestro finalmente expiró, a los pies de su cruz se hallaban Juan Zebedeo, su hermano Judá, su hermana Ruth, María Magdalena y Rebeca, anteriormente de Séforis.
\vs p187 5:5 Fue justo antes de las tres cuando Jesús, clamando a gran voz, dijo: “¡Consumado es! Padre, en tus manos encomiendo mi espíritu”. Y, habiendo dicho esto, inclinó la cabeza y dejó de luchar por la vida. Cuando el centurión romano vio como Jesús había muerto, se golpeó el pecho y dijo: “Realmente este hombre era justo; en verdad debe haber sido un Hijo de Dios”. Y, desde aquella hora, empezó a creer en Jesús.
\vs p187 5:6 \pc Jesús murió con majestuosidad; tal como había vivido. Admitió abiertamente su soberanía y permaneció dueño de la situación durante todo aquel trágico día. Por propia voluntad fue a una muerte ignominiosa, tras haber proporcionado seguridad a sus apóstoles elegidos. Con sabiduría, refrenó la problemática violencia de Pedro y dispuso que Juan estuviera cerca de él hasta el fin de su existencia mortal. Reveló su verdadera naturaleza al homicida sanedrín y recordó a Pilato la procedencia de su autoridad soberana como un Hijo de Dios. Marchó al Gólgota con el travesaño de su propia cruz a cuestas y acabó su amoroso ministerio de gracia entregando al Padre del Paraíso su espíritu, que había adquirido siendo mortal. Después de una vida así ---y de una muerte así--- el Maestro verdaderamente podía decir: “Consumado es”.
\vs p187 5:7 \pc Por cuanto era a la vez el día de la preparación para la Pascua y el día del \bibemph{sabbat,} los judíos no quisieron que estos cuerpos estuvieran expuestos a la vista en el Gólgota. Así pues, fueron ante Pilato pidiéndole que quebrara las piernas de estos tres hombres para acabar rápidamente con ellos y poder bajarlos de sus cruces y arrojarlos, antes de ponerse el sol, a las fosas de enterramiento para delincuentes comunes. Cuando Pilato oyó esta petición, sin dilación mandó a tres soldados a que les rompieran las piernas y fueran quitados de allí Jesús y los dos bandidos.
\vs p187 5:8 Cuando estos soldados llegaron al Gólgota, obraron en consecuencia con los dos ladrones, pero vieron, para su gran sorpresa, que Jesús ya estaba muerto. Sin embargo, queriendo estar seguros de su muerte, uno de los soldados le atravesó el costado izquierdo con su lanza. Aunque era habitual que las víctimas de la crucifixión permanecieran con vida en la cruz hasta incluso dos o tres días, la sobrecogedora agonía emocional y la intensa angustia espiritual de Jesús pusieron fin a su vida en la carne como mortal en algo menos de cinco horas y media.
\usection{6. DESPUÉS DE LA CRUCIFIXIÓN}
\vs p187 6:1 En medio de la oscuridad de la tormenta de arena, sobre las tres y media de la tarde, David Zebedeo envió a los últimos de los mensajeros a llevar la noticia de la muerte del Maestro. Mandó al último de sus corredores a la casa de Marta y María en Betania, donde supuso que paraba la madre de Jesús con el resto de la familia.
\vs p187 6:2 Tras la muerte del Maestro, Juan envió a las mujeres, bajo el cuidado de Judá, a la casa de Elías Marcos, donde se quedaron durante el día del \bibemph{sabbat}. Respecto a Juan, en aquel momento ya bien conocido por el centurión romano, permaneció en el Gólgota hasta que llegaron al sitio José y Nicodemo con la orden de Pilato autorizándolos a hacerse con los restos de Jesús.
\vs p187 6:3 De esa manera acabó un día trágico y doloroso para un inmenso universo, cuyas miríadas de inteligencias se habían estremecido ante la terrible visión de la crucifixión de la encarnación humana de su amado Soberano. Estaban conmocionados ante esta demostración de la depravación y la perversidad humana.
