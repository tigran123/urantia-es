\upaper{120}{El ministerio de gracia de Miguel en Urantia}
\author{Mantutia Melquisedec}
\vs p120 0:1 Soy el melquisedec designado por Gabriel para supervisar la nueva exposición de la vida de Miguel durante su estancia en Urantia semejando un hombre mortal y el director de la comisión reveladora encargada de realizar esta tarea. Estoy autorizado para presentar el relato de ciertos acontecimientos inmediatamente anteriores a la llegada del hijo creador a Urantia con el fin de emprender la fase última en su experiencia de darse de gracia en el universo. Vivir unas vidas idénticas a las que él mismo dispone para los seres inteligentes de su propia creación, otorgándose en semejanza de sus distintos órdenes de seres creados, es parte del precio que todo hijo creador ha de pagar con objeto de lograr la soberanía plena y suprema de su universo de seres y cosas por él mismo creados.
\vs p120 0:2 Antes de los acontecimientos que estoy a punto de describir, Miguel de Nebadón se había dado seis veces en la similitud de seis órdenes diferentes de seres inteligentes de su diversa creación. Tras ello, se preparó para descender a Urantia como hombre mortal, el orden de seres inteligentes de menor rango de sus criaturas inteligentes y de voluntad y, como ser humano de un entorno material, acometer el acto final: adquirir la soberanía sobre su universo de acuerdo con el mandato de los Gobernantes Divinos del Paraíso del universo de los universos.
\vs p120 0:3 En el curso de cada uno de sus ministerios de gracia anteriores, Miguel no solo llegó a adquirir la experiencia finita de un grupo de sus seres creados, sino también una experiencia esencial en cooperación con el Paraíso que, en sí misma y de sí misma, contribuiría, además, a convertirle en el soberano del universo creado por él. En cualquier momento, durante todo el tiempo pasado del universo local, Miguel podría haber hecho valer su soberanía personal como hijo creador y, como tal, podría haber gobernado su universo según su propio arbitrio. En tal caso, Emanuel y los hijos del Paraíso que le prestaban su colaboración hubiesen abandonado el universo. Pero Miguel no deseaba gobernar Nebadón amparado simplemente en su propio y solitario derecho como hijo creador. Deseaba ascender adquiriendo una genuina experiencia en subordinación y cooperación con la Trinidad del Paraíso y lograr ese elevado estatus en el universo, que le facultara poder gobernar su universo y administrar los asuntos de este con la perfecta comprensión y la lúcida consecución que algún día caracterizarían el gobierno excelso del Ser Supremo. No aspiraba a un gobierno de perfección como hijo creador, sino a una supremacía de orden administrativo, que englobara la sabiduría en los asuntos del universo y la experiencia divina del Ser Supremo.
\vs p120 0:4 Miguel, por lo tanto, al realizar estos siete ministerios en los que se otorgaba a los distintos órdenes de criaturas del universo, albergaba un doble propósito. En primer lugar, completaba la experiencia necesaria para entender a las criaturas, algo exigido a todos los hijos creadores antes de asumir total soberanía. Los hijos creadores pueden gobernar su universo por derecho propio en cualquier momento, pero únicamente pueden hacerlo como representantes supremos de la Trinidad del Paraíso tras pasar por dichos siete ministerios. En segundo lugar, aspiraba a tener el privilegio de representar la máxima autoridad de la Trinidad del Paraíso que puede ejercerse en la administración directa y personal de un universo local. En consecuencia, durante el transcurso de cada uno de sus ministerios de gracia en el universo, Miguel, se subordinó voluntariamente, y de modo propicio y perfecto, a las voluntades distintamente combinadas de las diferentes relaciones de las personas de la Trinidad del Paraíso. Esto es, en el primer ministerio, se subordinó a la voluntad combinada del Padre, el Hijo y el Espíritu; en el segundo, a la voluntad del Padre y del Hijo; en el tercero, a la voluntad del Padre y del Espíritu; en el cuarto, a la voluntad del Hijo y del Espíritu; en el quinto, a la voluntad del Espíritu Infinito; en el sexto, a la voluntad del Hijo Eterno; y, en el séptimo y último, en Urantia, a la voluntad del Padre Universal.
\vs p120 0:5 Como consecuencia, Miguel, en su soberanía personal, combina la voluntad divina de las facetas séptuplas de los creadores universales con el entendimiento de sus criaturas del universo local. De este modo, su administración se convierte en la representante del más elevado poder y autoridad posibles, aunque despojado de cualquier arrogación arbitraria. Su poder es ilimitado porque resulta de una vinculación experimentada con las Deidades del Paraíso; su autoridad es incuestionable porque se adquirió mediante una experiencia genuina al semejarse a las criaturas del universo; su soberanía es suprema porque engloba a la vez la perspectiva séptupla de la Deidad del Paraíso junto con la perspectiva de las criaturas del tiempo y del espacio.
\vs p120 0:6 \pc Tras haber determinado el momento de su ministerio de gracia final y haber elegido el planeta en el que tendría lugar este extraordinario acontecimiento, como era habitual, Miguel, antes de este ministerio, conferenció con Gabriel y luego se presentó ante Emanuel, su hermano mayor y consejero del Paraíso. Miguel asignó entonces a la custodia de Emanuel todos aquellos poderes pertinentes a la administración del universo que no habían sido conferidos a Gabriel. Y justo antes de la partida de Miguel para su encarnación en Urantia, Emanuel, aceptando la custodia del universo durante el curso de dicho ministerio en Urantia, procedió a impartir los consejos que le servirían de guía para su encarnación cuando al poco tiempo creciera en Urantia como un mortal más.
\vs p120 0:7 \pc A este respecto, debe tenerse en cuenta que Miguel había optado por darse de gracia con la semejanza de un hombre mortal, en subordinación a la voluntad del Padre del Paraíso. El hijo creador no necesitaba instrucciones de nadie para llevar a cabo esta encarnación si su único fin hubiese sido conseguir la soberanía del universo; pero había emprendido un plan de revelación del Supremo que implicaba obrar de manera colaborativa con las distintas voluntades de las Deidades del Paraíso. De este modo, su soberanía, cuando finalmente y de manera personal la adquiriese, abarcaría de hecho la séptupla voluntad de la Deidad tal como tiene su culminación en el Supremo. Por lo tanto, había previamente recibido instrucciones seis veces de los representantes personales de las diferentes Deidades del Paraíso y de estas en conjunción; y, en dicha ocasión, las instrucciones venían del unión de días, el embajador de la Trinidad del Paraíso ante el universo local de Nebadón, que actuaba en nombre del Padre Universal.
\vs p120 0:8 \pc La buena disposición con la que este poderoso hijo creador se subordinaba de nuevo a la voluntad de las Deidades del Paraíso ---esta vez a la del Padre Universal--- conllevaría ventajas inmediatas y grandes compensaciones. Al decidir llevar a cabo dicha subordinación y vinculación, Miguel experimentaría, en esta encarnación, no solamente la naturaleza del hombre mortal, sino también la voluntad del Padre de todos, del Paraíso. Y, además, podía emprender este excepcional ministerio de gracia con la completa seguridad de que no solo Emanuel ejercería la plena autoridad del Padre del Paraíso en la administración de su universo durante su ausencia al otorgarse a Urantia, sino también con el reconfortante conocimiento de que los ancianos de días del suprauniverso habían dispuesto que sus dominios estuvieran seguros durante la totalidad de dicho ministerio.
\vs p120 0:9 \pc Y este era el marco de la trascendental ocasión en la que Emanuel expuso el cometido del séptimo ministerio de gracia. Y estoy autorizado a presentar los siguientes fragmentos de la encomienda dada por Emanuel al gobernante del universo en Urantia, que se convertiría con posterioridad en Jesús de Nazaret (Cristo Miguel):
\usection{1. COMETIDO DEL SÉPTIMO MINISTERIO DE GRACIA}
\vs p120 1:1 “Mi hermano creador, estoy a punto de presenciar tu séptimo y último ministerio en el que te otorgas al universo. Con gran fidelidad y perfección has llevado a cabo las seis misiones anteriores, y no albergamos duda alguna de que saldrás igualmente triunfante de esta, la última, que te conducirá hasta la soberanía. Hasta ahora, apareciste en las esferas receptoras de tu ministerio como un ser totalmente desarrollado del orden por el que optaste. Ahora estás a punto de aparecer en Urantia, el desordenado y agitado planeta que has elegido, no como mortal totalmente desarrollado, sino como un indefenso bebé recién nacido. Esto, compañero mío, va a ser para ti una nueva experiencia, no probada. Estás a punto de cumplir el requisito de darte de gracia y de tener conciencia completa de lo que conlleva como creador encarnarse con la semejanza de las criaturas.
\vs p120 1:2 “En cada uno de tus ministerios anteriores optaste voluntariamente por subordinarte a la voluntad de las tres Deidades del Paraíso y sus interrelaciones divinas. En estos, te has subordinado a todas las siete facetas de la voluntad del Supremo salvo a la voluntad personal de tu Padre del Paraíso. Ahora, que has decidido subordinarte por completo a la voluntad de tu Padre en este, tu séptimo ministerio de gracia, yo, como representante personal de nuestro Padre, asumo la jurisdicción incondicionada de tu universo durante el período de tu encarnación.
\vs p120 1:3 “Al emprender este ministerio en Urantia, te has despojado de manera voluntaria de todo el apoyo extraplanetario y de la ayuda especial que podría prestarte cualquiera de las criaturas de tu propia creación. Al igual que tus hijos de Nebadón por ti creados dependen por completo de tu guía y cuidado durante sus andaduras en el universo, debes tú ahora depender enteramente y sin reservas de tu Padre del Paraíso para conducirte bajo su cuidado en medio de las vicisitudes no desveladas de tu próxima andadura como mortal. Y cuando hayas terminado y experimentado este ministerio, llegarás a conocer, en toda verdad, el pleno sentido y el valioso significado de esa fe\hyp{}confianza que tú con tanta constancia requieres que tus criaturas aprendan como parte de su relación personal contigo, como Padre y creador del universo local.
\vs p120 1:4 “Durante tu ministerio en Urantia solo necesitarás preocuparte de algo: que haya una comunicación ininterrumpida entre tú y tu Padre del Paraíso; y será esa perfecta relación la que hará que el mundo en el que te vas a otorgar, e incluso todo el universo por ti creado, contemple una revelación nueva y más comprensible de tu Padre y de mi Padre, del Padre Universal de todos. Tu preocupación, por lo tanto, será solo en cuanto a tu vida personal en Urantia. Yo me responsabilizaré, total y eficientemente, de la seguridad y de la administración inquebrantable de tu universo, desde el momento en el que renuncies voluntariamente a tu autoridad sobre él hasta tu regreso a nosotros en calidad de soberano del universo, confirmado por el Paraíso, y recibas nuevamente de mis manos, ya no la autoridad de vicerregente que ahora me entregas, sino, en su lugar, el poder supremo y la jurisdicción sobre tu universo.
\vs p120 1:5 “Y para que tengas la seguridad de que estoy facultado para llevar a cabo todo lo que te estoy prometiendo ahora (a sabiendas de que yo soy la garantía de todo el Paraíso para el fiel cumplimiento de mi palabra), te anuncio que me acaban de comunicar un mandato de los ancianos de días de Uversa que prevendrá a Nebadón de cualquier peligro espiritual durante todo el período de tu ministerio de gracia voluntario. Desde el momento en el que ceses tu conciencia, al comienzo de tu encarnación como mortal, hasta que regreses a nosotros en calidad de soberano supremo e incondicional de este universo que tú mismo has creado y organizado, nada grave podrá suceder en todo Nebadón. En el transcurso de tu encarnación, tengo en mi posesión las órdenes de los ancianos de días que establecen, de forma incondicionada, la extinción instantánea y automática de cualquier ser culpable de rebelión o que ose instigar la insurrección en el universo de Nebadón mientras estés ausente durante este ministerio. Hermano mío, en vista de la autoridad del Paraíso inherente en mi presencia, reforzada por el mandato judicial de Uversa, tu universo y todas tus criaturas leales estarán a salvo durante el período de dicho ministerio. Lleva a cabo tu misión con un único pensamiento: la revelación mejorada de nuestro Padre para los seres inteligentes de tu universo.
\vs p120 1:6 “Como en cada uno de tus ministerios anteriores, quisiera recordarte que soy el destinatario de la jurisdicción de tu universo en calidad de hermano y fiduciario y que ejerceré toda autoridad y poder en tu nombre. Obraré como lo haría nuestro Padre del Paraíso y acorde con tu petición expresa de que actúe así en tu nombre. Y teniendo esto en cuenta, toda esta autoridad delegada en mí será tuya de nuevo en el momento en el que consideres conveniente solicitar que vuelva a ti. Tu ministerio de gracia es, en su totalidad, completamente voluntario. En el mundo, como mortal encarnado, no tendrás atributos celestiales, pero el poder al que has renunciado puede ser nuevamente tuyo en el momento en el que decidas volver a investirte de autoridad sobre el universo. Recuerda que si decides restablecer tu poder y autoridad, será por razones enteramente \bibemph{personales,} puesto que yo soy la promesa viva y suprema cuya presencia y compromiso garantizan la administración segura de tu universo, en conformidad con la voluntad de nuestro Padre. Durante tu ausencia de Lugar de Salvación debido a tu ministerio en Urantia, no hay posibilidad de rebelión, como ya sucediera tres veces en Nebadón. Para el período correspondiente a dicho ministerio, los ancianos de días han decretado que cualquier rebelión en Nebadón ha de contener el germen de su propia e indefectible erradicación.
\vs p120 1:7 “Mientras que en tu último y extraordinario ministerio estés ausente, me comprometo (con la cooperación de Gabriel) a la administración fiel de tu universo; y, al encargarte que emprendas este ministerio de revelación divina y pases por esta experiencia de llegar a un completo entendimiento de los seres humanos, actúo en nombre de mi Padre y tu Padre y te ofrezco los siguientes consejos, que han de guiarte en tu vida terrenal conforme tomes paulatinamente conciencia de la misión divina implícita a tu estancia continuada en la carne:
\usection{2. LIMITACIONES DEL MINISTERIO DE GRACIA}
\vs p120 2:1 “1. De acuerdo a las costumbres y en conformidad con el método seguido en Lugar del Hijo ---en cumplimiento de los mandatos del Hijo Eterno del Paraíso---, lo he previsto todo para que, de forma inmediata, emprendas este ministerio de gracia como mortal en sintonía con los planes formulados por ti y entregados por Gabriel para mi custodia. Crecerás en Urantia como hijo del mundo, completarás tu educación humana ---en todo momento subordinado a la voluntad de tu Padre del Paraíso--- y vivirás tu vida en Urantia tal como decidiste, terminarás tu estancia planetaria, te prepararás para la ascensión a tu Padre y recibirás de él la soberanía suprema de tu universo.
\vs p120 2:2 \pc “2. Además de tu misión en la tierra y de tu revelación al universo, pero relacionado con ambas, te aconsejo que, una vez que seas suficientemente consciente de tu identidad divina, asumas la tarea de poner oficialmente fin a la rebelión de Lucifer en el sistema de Satania, y hacerlo todo como \bibemph{Hijo del Hombre;} de este modo, como una criatura mortal del mundo, en su debilidad hecha poderosa por la fe y la subordinación a la voluntad de tu Padre, te sugiero que lleves a cabo con condescendencia la acción que de modo repetido y discrecionalmente declinaste realizar mediante el uso del poder y la fuerza, cuando estabas dotado de ellos al inicio de esta rebelión pecaminosa e injustificada. Consideraría un digno colofón a tu ministerio como mortal si regresaras a nosotros como el Hijo del Hombre, Príncipe Planetario de Urantia, al igual que como el Hijo de Dios y soberano supremo de tu universo. Como hombre mortal, la criatura inteligente de menor rango de todas las criaturas de Nebadón, haz frente y arbitra las blasfemas pretensiones de Caligastia y Lucifer y, en esa humilde condición que has asumido, acaba para siempre con las vergonzosas tergiversaciones de estos hijos de luz caídos. Habiendo sido firme en tu rechazo de desacreditar a estos rebeldes mediante el ejercicio de tus prerrogativas como creador, sería conveniente que ahora, semejando las criaturas de menor rango de tu creación, lo hicieras arrebatando la autoridad de las manos de estos hijos caídos; y, así, la totalidad de tu universo local podrá, con toda equidad, reconocer claramente y para siempre la justicia de tus actos al desempeñar, en tu función de criatura de carne mortal, aquellas cosas que la misericordia te aconsejó que no hicieras con el poder de una autoridad soberana. Y, de esta manera, habiendo establecido con tu ministerio de gracia la posibilidad de la soberanía del Supremo en Nebadón, habrás, en efecto, resuelto para siempre los asuntos no dictaminados de todas las insurrecciones anteriores, pese al mayor o menor lapso de tiempo que te lleve realizar estas tareas. Mediante este acto, se disolverán sustancialmente todas las disensiones que aguardan resolución en tu universo. Y, al ser dotado seguidamente de la soberanía suprema sobre tu universo, en ninguna parte de tu gran creación personal se volverán a repetir similares desafíos a tu autoridad.
\vs p120 2:3 \pc “3. Cuando hayas conseguido acabar con la secesión en Urantia, como sin duda lo harás, te aconsejo que aceptes que Gabriel te confiera el título de ‘Príncipe Planetario de Urantia’ como reconocimiento eterno de parte de tu universo por llevar a cabo este último ministerio de gracia; y que, además, hagas todo lo posible, consecuente con la intención de tu ministerio, por reparar la aflicción y la confusión que la traición de Caligastia y el posterior incumplimiento adánico ha causado en Urantia.
\vs p120 2:4 \pc “4. De acuerdo con tu petición, Gabriel y todos los que les compete este asunto cooperarán contigo en cumplir tu deseo expreso de que tu ministerio de gracia en Urantia culmine con el pronunciamiento de un juicio dispensacional del mundo, acompañado del fin de una era, de la resurrección de supervivientes mortales dormidos y del establecimiento de la dispensación de la dádiva del espíritu de la verdad.
\vs p120 2:5 \pc “5. En cuanto al planeta de tu ministerio y a la generación más inmediata de hombres que vivirán allí en el momento de tu estancia como mortal, te aconsejo que desempeñes en gran medida el papel de maestro. Presta atención, primeramente, a la liberación e inspiración de la naturaleza espiritual del hombre. Luego, ilumina el intelecto ensombrecido del ser humano, sana las almas de los hombres y libera sus mentes de miedos ancestrales. Y después, según tu sabiduría humana, sirve al bienestar físico y a la comodidad material de tus hermanos en la carne. Vive una vida religiosa ideal que sirva de inspiración y edificación a todo tu universo.
\vs p120 2:6 \pc “6. En Urantia, el planeta de tu ministerio, libera espiritualmente al hombre aislado por la rebelión. Haz una nueva aportación a la soberanía del Supremo, extendiendo de este modo el establecimiento de esta soberanía por todos los amplios dominios de tu creación personal. En este, tu ministerio de gracia material en la forma de un hombre mortal, estás a punto de alcanzar finalmente, como creador espacio\hyp{}temporal, el entendimiento de tener la experiencia de obrar dentro de la naturaleza del hombre siguiendo, al mismo tiempo, la voluntad de tu Padre del Paraíso. En tu vida temporal, la voluntad de la criatura finita y la voluntad del Creador Infinito han de llegar a ser una sola, de la misma manera que estas se unen en la Deidad evolutiva del Ser Supremo. Derrama sobre el planeta al que te otorgas el espíritu de la verdad para que todos los mortales ordinarios de esa aislada esfera puedan acceder, de forma inmediata y por completo, al ministerio de la presencia individualizada de nuestro Padre del Paraíso, de los modeladores del pensamiento.
\vs p120 2:7 \pc “7. En todo lo que hagas en el mundo de tu ministerio, ten siempre presente que estás viviendo una vida para la instrucción y la edificación de todo tu universo. Estás \bibemph{otorgando} a Urantia esta vida en la que te encarnas como mortal, pero has de \bibemph{vivirla} para la inspiración espiritual de toda inteligencia humana y sobrehumana que vivió, existe ahora o está aún por vivir en cada uno de los mundos habitados que ha formado, forme ahora o esté aún por formar parte de la inmensa galaxia de tu ámbito gobernativo. Durante los días en los que residas en la tierra, no vivirás tu vida terrenal como hombre con el fin de ser un \bibemph{ejemplo} para los mortales de Urantia, ni tampoco para cualquier generación posterior de seres humanos de Urantia o de cualquier otro mundo. Más bien, en Urantia, tu vida en la carne será la \bibemph{inspiración} para todas las vidas de todos los mundos de Nebadón, a través de todas las generaciones en los tiempos venideros.
\vs p120 2:8 \pc “8. La gran misión que has de realizar y experimentar en tu encarnación como mortal es parte de la decisión que tomaste de vivir una vida incondicionalmente motivada a hacer la voluntad de tu Padre del Paraíso, esto es, \bibemph{revelar} a Dios, tu Padre, en la carne y particularmente a las criaturas de carne. Al mismo tiempo, también \bibemph{interpretarás,} con un nuevo realce, a nuestro Padre para los seres supramortales de todo Nebadón. Igualmente, con este ministerio, en el que impartirás una nueva revelación y una interpretación acrecentada del Padre del Paraíso para las mentes humanas y sobrehumanas, obrarás además realizando una nueva revelación del hombre a Dios. En tu corta vida en la carne, ilustra, como nunca antes se haya visto en todo Nebadón, las extraordinarias posibilidades que tiene ante sí el ser humano que conoce a Dios durante la breve andadura de su existencia como mortal, y haz una \bibemph{interpretación} nueva y esclarecedora, y para todos los tiempos, del hombre, y de las vicisitudes de su vida planetaria para todas las inteligencias sobrehumanas de Nebadón. Vas a descender a Urantia con la forma de un hombre mortal y, al vivir como un hombre de tu tiempo y generación, actuarás de manera que puedas mostrar a todo tu universo el modo ideal y perfeccionado de asumir el supremo compromiso respecto a las relaciones que se entablan en tu inmensa creación: el logro de Dios que busca al hombre y lo encuentra y el hecho extraordinario y prodigioso del hombre que busca a Dios y lo encuentra; haciendo todo esto para satisfacción mutua y durante una corta vida en la carne.
\vs p120 2:9 \pc “9. Te advierto que has de tener siempre presente que, aunque en efecto llegarás a ser un hombre ordinario del mundo, potencialmente seguirás siendo un hijo creador del Padre del Paraíso. Durante esta encarnación, aunque vivirás y actuarás como Hijo del Hombre, los atributos creativos de tu divinidad personal te seguirán desde Lugar de Salvación hasta Urantia. Tras la llegada del modelador del pensamiento, con la fuerza de tu voluntad podrás optar siempre por terminar tu encarnación. Antes de la llegada y recepción del modelador seré el garante de la integridad de tu persona. Pero después de su llegada y, en simultaneidad con tu paulatino reconocimiento de la naturaleza y la trascendencia de tu misión de gracia, deberás abstenerte de usar la vía sobrehumana para realizar algún deseo, conseguir algún logro o alcanzar algún poder, teniendo en cuenta el hecho de que tus prerrogativas creadoras permanecerán vinculadas a tu persona mortal debido a la inseparabilidad de estos atributos de tu presencia personal. Pero, al margen de la voluntad del Padre del Paraíso, no habrá ninguna influencia de orden sobrehumano en tu andadura, a menos que, mediante un acto consciente y deliberado de tu voluntad, tomes una decisión indivisa que comporte la elección de todo tu ser personal.
\usection{3. OTROS CONSEJOS Y RECOMENDACIONES}
\vs p120 3:1 “Y ahora, hermano mío, al despedirme de ti al disponerte a partir para Urantia y, tras haberte aconsejado sobre el plan general de tu ministerio de gracia, permíteme que te dé algunas recomendaciones, resultado de una consulta con Gabriel, y que se refieren a ciertos aspectos menores de tu vida mortal. Te sugerimos pues lo siguiente:
\vs p120 3:2 “1. Que, en la búsqueda del ideal de tu vida mortal terrenal, debes también prestar atención a la realización y ejemplificación de algunas cosas prácticas y directamente provechosas para tus hermanos mortales.
\vs p120 3:3 \pc “2. En cuanto a las relaciones familiares, da prioridad a las costumbres aceptadas de la vida familiar tal como están establecidas en el momento y en la generación en la que realizas tu ministerio. Vive tu vida familiar y comunitaria de acuerdo con las prácticas de las personas entre las que has elegido aparecer.
\vs p120 3:4 \pc “3. En tus relaciones con el orden social, te aconsejamos que, en gran parte, restrinjas tu labor a la regeneración espiritual y a la emancipación intelectual. Evita toda implicación con la estructura económica y con los compromisos políticos de tu tiempo. Dedícate sobre todo a vivir el ideal de la vida religiosa en Urantia.
\vs p120 3:5 \pc “4. Bajo ningún concepto y ni siquiera en lo más mínimo, debes interferir con la evolución normal, ordenada y progresiva de las razas de Urantia. Sin embargo, no se debe interpretar esta prohibición como una restricción a tu labor de dejar detrás de ti, en Urantia, un sistema perdurable y mejorado de \bibemph{ética religiosa positiva}. Como hijo, árbitro de una dispensación, se te han otorgado ciertas prerrogativas relacionadas con el avance del estatus \bibemph{espiritual} y \bibemph{religioso} de los pueblos del mundo.
\vs p120 3:6 \pc “5. Si lo estimas conveniente, podrás identificarte con movimientos religiosos y espirituales existentes en Urantia, pero evita por todos los medios establecer formalmente un sistema religioso organizado, una religión cristalizada o algún grupo ético separado de seres mortales. Tu vida y tus enseñanzas han de llegar a ser el patrimonio común de todas las religiones y de todos los pueblos.
\vs p120 3:7 \pc “6. Con el fin de que no contribuyas de forma innecesaria a la posterior creación de sistemas estereotipados de creencias religiosas en Urantia o a la formación de otros tipos de devociones religiosas estancadas, te aconsejamos además: no dejes escritos tras de ti en el planeta. Abstente de escribir usando materiales permanentes; pide encarecidamente a tus colaboradores que no hagan imágenes ni cualquier otra semejanza física de ti. Cuida de que nada potencialmente idólatra quede en el planeta en el momento de tu partida.
\vs p120 3:8 \pc “7. Aunque vivas la vida social normal y corriente del planeta, y seas una criatura normal del sexo masculino, es probable que no mantengas relaciones matrimoniales, que serían totalmente loables y compatibles con tu ministerio de gracia; pero debo recordarte que uno de los mandatos de Lugar del Hijo relativos a la encarnación prohíbe que un hijo con origen en el Paraíso deje descendencia humana en cualquiera de los planetas en los que se otorgue.
\vs p120 3:9 \pc “8. En todos los demás aspectos de tu inminente ministerio de gracia, te encomendamos a la dirección de tu modelador interior, a las enseñanzas del espíritu divino, siempre presente, que guía a los seres humanos y a la razón\hyp{}juicio de tu mente humana en expansión, que es tuya por heredad. Esta conjunción de atributos como criatura y creador te permitirá vivir para nosotros la vida perfecta del hombre en las esferas planetarias; no necesariamente perfecta tal como la consideraría cualquier hombre de cualquier generación y de cualquier mundo (y menos aun en Urantia), sino completa y supremamente plena tal como se valora en los mundos más altamente perfeccionados y en vía de perfección de tu inmenso universo.
\vs p120 3:10 “Y ahora, que tu Padre y mi Padre, cuyo apoyo siempre hemos tenido en todos nuestros actos pasados, te guíe, te apoye y esté contigo desde el momento en el que nos dejes y realices el cese de la conciencia de tu ser personal, durante tu retorno gradual al reconocimiento de tu identidad divina encarnada en forma humana y, después, durante toda tu experiencia de gracia en Urantia hasta tu liberación de la carne y tu ascensión a la mano derecha de soberanía de nuestro Padre. Cuando te vea de nuevo en Lugar de Salvación, te daremos la bienvenida por tu regreso a nosotros en calidad de soberano supremo e incondicional de este universo creado por ti, al que has servido y del que has adquirido un completo entendimiento.
\vs p120 3:11 “Ahora reino en tu lugar. Asumo la jurisdicción de todo Nebadón como soberano en funciones durante el transcurso de tu séptimo ministerio de gracia, en el que te otorgas a Urantia como mortal. Y a ti, Gabriel, te encomiendo la custodia del que está a punto de convertirse en Hijo del Hombre hasta el momento en que regrese a mí en poder y gloria como Hijo del Hombre e Hijo de Dios. Y, Gabriel, yo seré pues tu soberano hasta que vuelva Miguel”.
\separatorline
\vs p120 3:12 Entonces, de inmediato, ante la presencia de todo Lugar de Salvación congregado, Miguel se alejó de entre nosotros, y no volvimos a verlo en su lugar de costumbre hasta su regreso como gobernante supremo y personal del universo, una vez terminada su andadura de gracia en Urantia.
\usection{4. LA ENCARNACIÓN: HACER UNO DE DOS}
\vs p120 4:1 Y, así pues, ciertos hijos indignos de Miguel, que habían acusado a su padre\hyp{}creador de buscar el poder egoístamente y se habían permitido insinuar que el hijo creador se mantenía en el poder de forma arbitraria y autocrática debido a la lealtad irreflexiva de un universo de criaturas engañadas y serviles, serían acallados para siempre y quedarían frustrados y desconcertados al contemplar la vida de servicio desinteresado que el Hijo de Dios comenzaba a vivir como Hijo del Hombre ---en todo momento sometiéndose a “la voluntad del Padre del Paraíso”---.
\vs p120 4:2 \pc Pero no os equivoquéis: aunque Cristo Miguel tuvo en verdad un origen doble, su ser personal no era doble. No era Dios en conjunción \bibemph{con} el hombre, sino más bien Dios \bibemph{encarnado} en el hombre. Y siempre fue precisamente ese ser coligado. Y el único factor gradual en esa relación incomprensible era entender y reconocer de forma consciente y paulatina (por parte de la mente humana) este hecho de ser Dios y hombre.
\vs p120 4:3 Cristo Miguel no se convirtió en Dios de forma gradual. Dios no se convirtió en hombre en algún momento crucial de la vida terrena de Jesús. Jesús era Dios \bibemph{y} hombre ---por siempre y para siempre---. Y este Dios y este hombre eran, y ahora son, \bibemph{uno,} tal como la Trinidad del Paraíso, la unión de tres seres, es en realidad \bibemph{una} Deidad.
\vs p120 4:4 Nunca perdáis de vista el hecho de que el propósito espiritual y supremo de la encarnación de Miguel era engrandecer la \bibemph{revelación de Dios}.
\vs p120 4:5 \pc Los mortales de Urantia albergan nociones diferentes respecto al sentido de lo milagroso, pero para nosotros que vivimos como ciudadanos del universo local existen pocos milagros y, entre ellos, con diferencia, los más fascinantes son los que se refieren a los ministerios de gracia y encarnación de los hijos del Paraíso. La aparición de un hijo divino en vuestro mundo, mediante procesos aparentemente naturales, es un milagro para nosotros: son leyes universales cuya actuación sobrepasan nuestra capacidad de comprensión. Jesús de Nazaret fue una persona milagrosa.
\vs p120 4:6 A través de toda esta extraordinaria experiencia, el Dios Padre optó por manifestarse como siempre lo hace ---\bibemph{de la manera habitual}---, procediendo en el modo normal, natural y fidedigno de la acción divina.
