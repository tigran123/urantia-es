\upaper{86}{Evolución temprana de la religión}
\author{Brillante estrella vespertina}
\vs p086 0:1 La evolución de la religión, que partió del impulso previo y primitivo a la adoración, no depende de la revelación. El normal desenvolvimiento de la mente humana bajo las influencias rectoras del sexto y del séptimo asistente de la mente, unos espíritus que se otorgan de forma generalizada, resulta totalmente suficiente para garantizar tal desarrollo.
\vs p086 0:2 El más temprano temor prerreligioso del hombre a las fuerzas de la naturaleza se volvió progresivamente religioso conforme la naturaleza se personificó, espiritualizó y, finalmente, se deificó en la conciencia humana. La religión de tipo primitivo fue, por lo tanto, una consecuencia biológica y natural de la inercia psicológica de la mente animal en evolución, después de que dicha mente hubiese llegado a percibir el concepto de lo sobrenatural.
\usection{1. EL AZAR: BUENA SUERTE Y MALA SUERTE}
\vs p086 1:1 Al margen del impulso natural a la adoración, la religión evolutiva primitiva ahonda sus orígenes en las experiencias humanas del azar: la llamada suerte, o sucesos ordinarios. El hombre primitivo cazaba para alimentarse. Los resultados de la caza variaban siempre necesariamente y esto, indefectiblemente, dio origen a esas experiencias que el hombre interpreta como \bibemph{buena suerte} y \bibemph{mala suerte}. El infortunio era un importante elemento en la vida de los hombres y mujeres que vivían constantemente al borde de una existencia precaria y de acoso.
\vs p086 1:2 El limitado horizonte intelectual del salvaje fijaba tanto su atención en el azar que la suerte se convirtió en una constante en su vida. Los urantianos primitivos luchaban por la existencia, no por niveles de bienestar; llevaban una vida peligrosa en la que el azar jugaba un papel importante. El miedo permanente a calamidades desconocidas e invisibles se cernía desesperadamente sobre ellos como una nube que eclipsaba infaliblemente cualquier tipo de gozo; vivían con un miedo constante a hacer algo que les trajera mala suerte. Los supersticiosos salvajes le tenían siempre miedo a las rachas de buena suerte; consideraban la buena fortuna como cierto presagio de calamidades.
\vs p086 1:3 Este omnipresente miedo a la mala suerte era paralizante. ¿Para qué trabajar duro y cosechar mala suerte ---nada a cambio de algo--- cuando era posible deambular por la vida y encontrarse con la buena suerte ---algo a cambio de nada---? El hombre irreflexivo se olvida de la buena suerte ---la da por sentado---, pero recuerda dolorosamente la mala suerte.
\vs p086 1:4 El hombre primitivo vivía en la incertidumbre y en el miedo permanente al azar ---a la mala suerte---. La vida era un emocionante juego de azar; la existencia, una arriesgada apuesta. No es de extrañar que los pueblos parcialmente civilizados aún crean en la suerte y manifiesten una constante predisposición por los juegos de azar. El hombre primitivo alternaba entre dos poderosos intereses: la pasión de obtener algo por nada y el temor de obtener nada por algo. Y esta apuesta de la existencia era el principal aliciente y la suprema fascinación de la mente salvaje primitiva.
\vs p086 1:5 Más adelante, los pastores mantuvieron los mismos puntos de vista sobre el azar y la suerte, mientras que, aún más tarde, los agricultores fueron cada vez más conscientes de que había muchas cosas sobre las que el hombre tenía poco o ningún control, que influían de forma directa sobre las cosechas. El agricultor se vio víctima de las sequías, las inundaciones, el granizo, las tormentas, las plagas y las enfermedades que afectaban a las plantas, así como del calor y el frío. Y puesto que todos estos factores naturales repercutían en su prosperidad personal, los consideraba como buena o mala suerte.
\vs p086 1:6 Esta noción del azar y la suerte impregnó firmemente la filosofía de todos los pueblos ancestrales. Incluso en tiempos recientes, en la Sabiduría de Salomón está escrito: “Me volví y vi que ni es de los ligeros la carrera, ni la guerra de los fuertes, ni aun de los sabios el pan, ni de los prudentes las riquezas, ni de los hombres de destreza el favor; sino que tiempo y ocasión acontecen a todos. Porque el hombre tampoco conoce su tiempo; como los peces que son presos en la mala red, y como las aves que se enredan en lazo, así son enlazados los hijos de los hombres en el tiempo malo, cuando cae de repente sobre ellos”.
\usection{2. PERSONIFICACIÓN DEL AZAR}
\vs p086 2:1 La ansiedad era el estado natural de la mente del salvaje. Cuando los hombres y las mujeres son víctimas de una excesiva ansiedad, están simplemente volviendo al estado natural de sus remotos antepasados; y, cuando llega a ser realmente dolorosa, la ansiedad inhibe la actividad y genera ineludibles cambios evolutivos y adaptaciones biológicas. El dolor y el sufrimiento son esenciales para el avance de la evolución.
\vs p086 2:2 La lucha por la vida es tan dolorosa que ciertas tribus subdesarrolladas aún gritan y se lamentan ante cada nuevo amanecer. El hombre primitivo incesantemente se preguntaba: “¿Quién me atormenta?”. Al no hallar el origen material de sus miserias, se resolvieron por una explicación espiritual. Y así nació la religión del temor a lo misterioso, del asombro ante lo invisible y del miedo a lo desconocido. El temor a la naturaleza se convirtió de este modo en determinante de la lucha por la existencia, primero debido al azar y, después, al misterio.
\vs p086 2:3 \pc La mente primitiva era lógica pero albergaba pocas ideas para que poder asociarlas de manera inteligente; la mente del salvaje no estaba cultivada, era muy simple. Si un suceso seguía a otro, el salvaje lo percibía como causa y efecto. Lo que el hombre civilizado considera superstición era sencillamente ignorancia en el salvaje. La humanidad ha tardado en aprender que no hay necesariamente relación alguna entre propósitos y resultados. Los seres humanos están apenas empezando a darse cuenta de que las reacciones de la existencia aparecen entre los actos y sus consecuencias. El salvaje se esfuerza por dar un carácter personal a todo lo que sea intangible y abstracto y, así, tanto la naturaleza como el azar se personifican como espectros ---espíritus--- y, más adelante, como dioses.
\vs p086 2:4 \pc Por naturaleza, el hombre tiende a creer en lo que él estima que es lo mejor para él, en lo que es de su interés, inmediato o remoto; el interés propio oscurece en gran medida la lógica. La diferencia entre las mentes del salvaje y del hombre civilizado es más de contenido que de naturaleza, de grado más que de calidad.
\vs p086 2:5 Pero continuar atribuyendo a causas sobrenaturales las cosas difíciles de comprender no es más que una manera ociosa y cómoda de evitar cualquier forma de dificultoso trabajo intelectual. La suerte es meramente un término acuñado para englobar lo inexplicable en cualquier periodo de la existencia humana; designa a aquellos fenómenos que el hombre es incapaz o no está dispuesto a captar. El azar es una palabra que significa que el hombre es demasiado ignorante o demasiado indolente como para determinar las causas. Los hombres consideran los sucesos naturales como accidentes o mala suerte solo cuando están desprovistos de curiosidad e imaginación, cuando las razas carecen de iniciativa y de sentido de la aventura. La exploración de los fenómenos de la vida destruye, más tarde o más temprano, la creencia del hombre en el azar, la suerte y los denominados accidentes, sustituyéndola por un universo de ley y de orden en el que las causas precisas preceden a los efectos. En consecuencia, el temor a la existencia se sustituye por el gozo de vivir.
\vs p086 2:6 El salvaje imaginaba que toda la naturaleza estaba viva, como poseída por algo. El hombre civilizado todavía da puntapiés y maldice aquellos objetos inanimados que se interponen en su camino y lo hacen tropezar. El hombre primitivo nunca consideraba nada como accidental; todo era siempre intencional. Para él, el entorno del destino, la función de la suerte, el mundo de los espíritus, era precisamente tan desorganizada y azarosa como lo era la sociedad primitiva misma. La suerte se veía como la respuesta caprichosa y temperamental de este mundo de los espíritus; más adelante, se vio como el humor de los dioses.
\vs p086 2:7 Pero no todas las religiones evolucionaron del animismo. Hubo otras nociones contemporáneas de lo sobrenatural, y tales creencias condujeron igualmente a la adoración. El naturalismo no es una religión ---es vástago de la religión---.
\usection{3. LA MUERTE: LO INEXPLICABLE}
\vs p086 3:1 La muerte significaba para el hombre evolutivo la conmoción absoluta, la combinación más desconcertante de azar y misterio. No fue la santidad de la vida sino la conmoción de la muerte la que les suscitó temor y, así, de forma efectiva, se fomentó la religión. Por lo general, entre los pueblos salvajes, la muerte era producto de la violencia, de modo que la muerte no violenta se volvió cada vez más misteriosa. La noción de la muerte como la terminación natural y esperada de la vida no resultaba clara en la conciencia de los pueblos primitivos, y tuvieron que pasar muchas eras antes de que el hombre se percatara de su inevitabilidad.
\vs p086 3:2 \pc El hombre primitivo aceptaba la vida como un hecho, mientras que consideraba la muerte como alguna especie de castigo. Todas las razas tienen sus leyendas sobre hombres que no morían, tradiciones residuales de una temprana actitud hacia la muerte. Ya existía en la mente humana el nebuloso concepto de un mundo espiritual confuso y desorganizado, un entorno del que procedía todo lo que era inexplicable en la vida humana, y la muerte se añadió a esta larga lista de fenómenos inexplicables.
\vs p086 3:3 Se creía en un principio que cualquier enfermedad humana al igual que la muerte natural se debían a la influencia de los espíritus. Incluso en los tiempos presentes, algunas razas civilizadas piensan que la enfermedad la ocasiona “el enemigo” y confían en ceremonias religiosas para lograr su curación. Hay sistemas filosóficos posteriores más complejos que aún atribuyen la muerte a la acción del mundo de los espíritus, todo lo cual ha llevado a doctrinas como el pecado original y la caída del hombre.
\vs p086 3:4 Comprender su impotencia ante la poderosa fuerza de la naturaleza, junto con el reconocimiento de la debilidad humana ante las calamidades de la enfermedad y de la muerte, impulsó al salvaje a buscar ayuda en el mundo supramaterial, que vagamente percibía como la causa de estas vicisitudes misteriosas de la vida.
\usection{4. CONCEPTO DE LA SUPERVIVENCIA TRAS LA MUERTE}
\vs p086 4:1 El concepto de la existencia de una etapa supramaterial de la persona mortal nació de la asociación inconsciente y puramente accidental de los sucesos de la vida sumada al hecho de soñar con los espectros. El sueño simultáneo de varios miembros de una tribu con su jefe fallecido parecía constituir una prueba convincente de que el anciano jefe había realmente regresado bajo alguna forma. Todo resultaba muy real para el salvaje, que despertaba de estos sueños empapado en sudor, temblando y gritando.
\vs p086 4:2 El origen onírico de la convicción de una existencia futura explica la propensión a imaginar constantemente cosas invisibles en términos de las cosas visibles. Y, enseguida, este nuevo concepto de vida futura surgido del sueño espectral empezó a ser, en conjunción con el instinto biológico de la autoconservación, un remedio eficaz, contra el temor a la muerte.
\vs p086 4:3 El hombre primitivo también se preocupaba mucho por su respiración, especialmente en los climas fríos, en los que, al exhalar, aparecía una especie de nube. El \bibemph{aliento de la vida} se consideraba un fenómeno que diferenciaba a los vivos de los muertos. Sabía que el aliento podía abandonar al cuerpo y sus sueños, en los que hacía toda clase de cosas extrañas mientras dormía, lo convencieron de que había algo inmaterial en el ser humano. La idea más primitiva del alma humana, el espectro, se derivó del sistema conceptual formado a partir de la relación aliento\hyp{}sueños.
\vs p086 4:4 Con el tiempo, el salvaje llegó a verse a sí mismo como un ser doble: cuerpo y aliento. El aliento menos el cuerpo equivalía al espíritu, al espectro. Aunque tenían un origen humano muy concreto, los espectros, o espíritus, se consideraban sobrehumanos. Y esta creencia en la existencia de espíritus desencarnados parecía dar una explicación a la aparición de lo inusual, lo extraordinario, lo infrecuente y lo inexplicable.
\vs p086 4:5 \pc La doctrina primitiva de la supervivencia tras la muerte no significaba necesariamente que se creyera en la inmortalidad. Estos seres, que no sabían contar más de veinte, difícilmente podían concebir la infinitud y la eternidad; pensaban más bien en encarnaciones recurrentes.
\vs p086 4:6 La raza naranja era particularmente dada a creer en la transmigración y en la reencarnación. Esta idea de la reencarnación tuvo su origen en la observación del parecido hereditario y de rasgos entre descendientes y ancestros. La costumbre de llamar a los hijos con el nombre de los abuelos y de otros antepasados se debía a esta creencia. Más tarde, algunas razas consideraban que el hombre moría entre tres y siete veces. Esta convicción (vestigio de las enseñanzas de Adán sobre los mundos de estancia), y muchos otros remanentes de la religión revelada, se pueden encontrar entre las doctrinas, por otra parte absurdas, de personas poco cultivadas del siglo XX.
\vs p086 4:7 \pc El hombre primitivo no albergaba idea alguna sobre el infierno o los castigos futuros. El salvaje percibía la vida futura exactamente como esta misma, menos toda la mala suerte. Más adelante, se concibió un destino separado para los buenos y los malos espectros ---el cielo y el infierno---. Pero, puesto que muchas razas primitivas creían que el hombre entraba a la vida próxima tal como dejaba esta, no les entusiasmaba la idea de volverse viejos y decrépitos. Las personas de edad preferían que los mataran antes de llegar a estar demasiado enfermos.
\vs p086 4:8 Casi todos los grupos tenían una idea diferente respecto al destino del alma espectro. Los griegos creían que los hombres débiles debían tener almas débiles; así que inventaron el Hades como el lugar más adecuado para acoger a dichas almas anémicas; también se suponía que estos individuos poco fuertes tenían sombras más reducidas. Los primeros anditas creían que sus espectros volvían a sus tierras ancestrales. En antaño, los chinos y los egipcios pensaban que el alma y el cuerpo permanecían juntos. Entre los egipcios, esto llevó a la construcción de esmeradas tumbas y a hacer denodados intentos por preservar el cuerpo. Incluso los pueblos modernos tratan de detener la descomposición causada por la muerte. Los hebreos concebían que una réplica fantasmal de la persona bajaba al Seol; no podía regresar a la tierra de los vivos. Realmente, lograron ese avance importante en la doctrina de la evolución del alma.
\usection{5. CONCEPTO DEL ALMA ESPECTRO}
\vs p086 5:1 La parte no material del hombre se ha denominado indistintamente como espectro, espíritu, sombra, fantasma, aparición y, últimamente, \bibemph{alma}. En los sueños del hombre primitivo, el alma era su doble; en todos los sentidos, era exactamente igual al mortal mismo excepto que no era sensible al tacto. La creencia en los dobles oníricos llevó directamente a la idea de que todas las cosas animadas e inanimadas tenían un alma, tal como la tenían los hombres. Este concepto tendió a perpetuar durante mucho tiempo la creencia en los espíritus de la naturaleza. Los esquimales aún tienen la convicción de que todo en la naturaleza tiene un espíritu.
\vs p086 5:2 El alma espectro podía verse y oírse, pero no podía tocarse. Paulatinamente, la vida onírica de la raza desarrolló y amplió tanto la actividad de este mundo evolutivo de los espíritus que la muerte llegó a definirse como “exhalar el espíritu”. Todas las tribus primitivas, salvo aquellas que estaban poco por encima de los animales, han desarrollado algún concepto del alma. Conforme la civilización avanza, este supersticioso concepto del alma queda invalidado, y el hombre depende enteramente de la revelación y de la experiencia religiosa personal para su nueva idea del alma, como creación conjunta de la mente mortal conocedora de Dios y de su divino espíritu morador: el modelador del pensamiento.
\vs p086 5:3 Por lo general, los mortales primitivos no sabían diferenciar entre espíritu morador y alma de naturaleza evolutiva. El salvaje estaba muy confundido sobre si el alma espectro nacía del cuerpo o era un organismo externo en posesión del cuerpo. La ausencia de pensamiento razonado ante la perplejidad explica las graves incongruencias de la opinión del salvaje sobre las almas, los espectros y los espíritus.
\vs p086 5:4 Se pensaba que el alma era al cuerpo lo que el perfume a la flor. Los antiguos creían que el alma podía abandonar al cuerpo de distintas maneras, tales como:
\vs p086 5:5 \li{1.}El desvanecimiento ordinario y transitorio.
\vs p086 5:6 \li{2.}Durmiendo, durante el sueño natural.
\vs p086 5:7 \li{3.}El coma y la pérdida del conocimiento, relacionados con la enfermedad y los accidentes.
\vs p086 5:8 \li{4.}La muerte, la partida definitiva.
\vs p086 5:9 \pc El salvaje consideraba el estornudo como un intento fallido del alma por escapar del cuerpo. Estando despierto y en guardia, el cuerpo podía frustrar el intento de fuga del alma. Más adelante, el estornudo venía siempre acompañado de alguna expresión religiosa como “¡Que Dios te bendiga!”.
\vs p086 5:10 \pc Pronto en la evolución, el sueño se consideraba como una prueba de que el alma espectro podía estar ausente del cuerpo, y se creía que se podía hacer que volviese diciendo o gritando el nombre del durmiente. En otras formas de pérdida del conocimiento, se creía que el alma se distanciaba aún más, tal vez tratando de escapar para siempre ---o muerte inminente---. Se pensaba que los sueños eran experiencias del alma, temporalmente ausente del cuerpo, mientras se dormía. El salvaje cree que sus sueños son tan reales como cualquier otra experiencia estando despierto. Los antiguos adoptaron la práctica de despertar a los durmientes de forma gradual para que el alma tuviese tiempo de retornar al cuerpo.
\vs p086 5:11 A lo largo de los tiempos, los hombres se han sentido abrumados por las apariciones nocturnas, y los hebreos no fueron una excepción. Verdaderamente creían que Dios les hablaba en sueños, a pesar de los mandatos de Moisés en contra de esta idea. Y Moisés tenía razón, porque los seres personales del mundo espiritual no hacen uso de los sueños ordinarios cuando tratan de comunicarse con los seres materiales.
\vs p086 5:12 Los antiguos creían que las almas podían entrar en los animales o incluso en los objetos inanimados. Esto dio lugar a la idea del hombre lobo y de la identificación del hombre con los animales. Era posible que el alma de un ciudadano respetuoso con la ley durante el día, podía meterse, al quedarse dormido, en un lobo o en cualquier otro animal y merodear cometiendo expolios nocturnos.
\vs p086 5:13 Los hombres primitivos creían que el alma estaba relacionada con la respiración, y que sus cualidades podían impartirse o transferirse a través del aliento. El valiente jefe soplaba sobre el recién nacido, con lo que le transmitía valor. Entre los primeros cristianos, la ceremonia del otorgamiento del espíritu santo se acompañaba del soplido sobre los candidatos. Dijo el salmista: “Por la palabra del Señor fueron hechos los cielos; y todo el ejército de ellos, por el aliento de su boca”. Durante mucho tiempo se acostumbraba a que el hijo mayor tratara de atrapar el último aliento de su moribundo padre.
\vs p086 5:14 Más adelante, se llegó a temer y a reverenciarse la sombra de forma similar al aliento. El reflejo de uno mismo en el agua también se consideraba a veces como prueba del doble ser, y los espejos se consideraban con un respeto supersticioso. Incluso ahora muchas personas civilizadas vuelven el espejo hacia la pared en caso de fallecimiento. Algunas tribus subdesarrolladas todavía creen que hacer retratos, dibujos, modelos o imágenes saca toda el alma, o parte de ella, del cuerpo; por consiguiente, tales cosas están prohibidas.
\vs p086 5:15 Se pensaba generalmente que el alma se identificaba con el aliento, pero diferentes pueblos también la localizaban en la cabeza, el cabello, el corazón, el hígado, la sangre y la grasa. La “voz de la sangre de Abel clama desde la tierra” expresa la antigua creencia de la presencia del espectro en la sangre. Los semitas enseñaban que el alma residía en la grasa corporal y, para muchos, comer grasa de animal era tabú. Cazar cabezas era un método de capturar el alma del enemigo, como lo era arrancarle la cabellera. En tiempos recientes, los ojos se han considerado las ventanas del alma.
\vs p086 5:16 Aquellos que defendían la doctrina de las tres o cuatro almas creían que la pérdida de una de ellas significaba malestar; de dos de ellas, enfermedad; de tres, muerte. Un alma vivía en el aliento, otra en la cabeza, otra más en el cabello y la última en el corazón. Se aconsejaba a los enfermos que pasearan al aire libre con la esperanza de recuperar sus almas extraviadas. Se suponía que los mejores curanderos intercambiaban el alma enferma de una persona enferma por un alma nueva, el “nuevo nacimiento”.
\vs p086 5:17 Los hijos de Badonán desarrollaron la creencia en dos almas: el aliento y la sombra. Las primeras razas noditas consideraban que el hombre constaba de dos personas: el alma y el cuerpo. Esta filosofía de la existencia humana se reflejó más tarde en la perspectiva griega. Los griegos creían en tres almas: el alma vegetativa residía en el estómago, la animal en el corazón, la intelectual en la cabeza. Los esquimales creen que el hombre está compuesto de tres partes: el cuerpo, el alma y el nombre.
\usection{6. ENTORNO DEL ESPÍRITU ESPECTRO}
\vs p086 6:1 El hombre heredó un medio ambiente natural, adquirió un contexto social e imaginó un entorno espectral. El Estado es la reacción del hombre a su medio ambiente natural; el hogar, a su contexto social; la Iglesia a su ilusorio entorno espectral.
\vs p086 6:2 Muy pronto en la historia de la humanidad, las realidades del mundo imaginario de espectros y espíritus se convirtieron en creencias generalizadas, y este nuevo mundo imaginado de los espíritus se hizo poderoso en la sociedad primitiva. La vida mental y moral de toda la humanidad se vio modificada para siempre por la aparición de este nuevo elemento en el pensamiento y la actuación de los seres humanos.
\vs p086 6:3 Dentro de esta primordial premisa de ilusión e ignorancia, el temor humano ha incorporado todas las supersticiones y religiones posteriores de los pueblos primitivos. Esta fue la única religión del hombre hasta los tiempos de la revelación, y, hoy en día, muchas de las razas del mundo tienen únicamente esta rudimentaria religión evolutiva.
\vs p086 6:4 A medida que progresaba la evolución, la buena suerte empezó a relacionarse con los espíritus buenos y, la mala suerte, con los malos. El malestar por adaptarse forzosamente a un entorno cambiante se consideraba como mala suerte, o el descontento de los espectros espíritus. Lentamente, el hombre primitivo desarrolló la religión, partiendo de su impulso innato a la adoración y de su errónea interpretación del azar. El hombre civilizado dispone de mecanismos que le aseguran la superación de los sucesos del azar; la ciencia moderna asesora con cálculos matemáticos sobre los riesgos, sustituyendo así a los espíritus ficticios y a los dioses caprichosos.
\vs p086 6:5 Cada nueva generación se sonríe ante las insensatas supersticiones de sus ancestros, pero sigue albergando esos errores de pensamiento y adoración que motivarán más adelante la sonrisa por parte de una generación más informada.
\vs p086 6:6 \pc Pero, por fin, la mente del hombre primitivo se ocupó de pensamientos que transcendían todos sus impulsos biológicos inherentes; por fin, el hombre estaba a punto de desarrollar un arte de vivir basado en algo más que la respuesta a estímulos materiales. Empezaban a emerger los comienzos de una primitiva orientación filosófica de la vida. Estaba a punto de aparecer una calidad de vida espiritual porque, si el espectro espíritu cuando está airado hace descender la mala suerte y, cuando está complacido, la buena fortuna, entonces el comportamiento humano debe regularse en consonancia. Al fin, había aflorado la noción del bien y del mal; y, todo esto, mucho tiempo antes de que se produjera cualquier revelación en la tierra.
\vs p086 6:7 Con la gradual aparición de estas nociones, se inició la larga e inútil lucha por apaciguar a los espíritus siempre descontentos, el sometimiento servil al temor religioso evolutivo, ese prolongado malgasto de esfuerzo humano en tumbas, templos, sacrificios y sacerdocios. El precio que se tuvo que pagar fue tremendo y terrible, pero su coste mereció la pena, porque con ello el hombre adquirió una conciencia natural del bien y del mal relativos; ¡había nacido la ética humana!
\usection{7. LA FUNCIÓN DE LA RELIGIÓN PRIMITIVA}
\vs p086 7:1 El salvaje sentía la necesidad de protección y, por lo tanto, pagaba voluntariamente gravosas bonificaciones de temor, superstición, terror y obsequios a los sacerdotes para conseguir su póliza de seguro mágico contra la mala suerte. La religión primitiva consistía simplemente en el pago de dichas primas que lo aseguraban contra los peligros del bosque; el hombre civilizado paga bonificaciones materiales para asegurarse contra los accidentes de la industria y las exigencias de los modos de vida modernos.
\vs p086 7:2 La sociedad actual está trasladando la acción aseguradora desde el ámbito sacerdotal y religioso al de la economía. La religión se preocupa cada vez más por el seguro de la vida más allá de la tumba. Los hombres modernos, al menos aquellos que son reflexivos, ya no pagan primas dispendiosas para controlar la suerte. La religión está lentamente ascendiendo a niveles filosóficos superiores, contrariamente a su antigua labor como plan de seguro contra la mala suerte.
\vs p086 7:3 Pero estas ancestrales ideas religiosas evitaron que los hombres se volvieran fatalistas e irremediablemente pesimistas; los hombres creyeron que podían al menos hacer algo para influir en su suerte. La religión del temor a los espectros les impuso el deber de \bibemph{regular su comportamiento,} de que existía un mundo supramaterial que regía el destino humano.
\vs p086 7:4 Las razas civilizadas modernas están comenzando a salir del temor a los espectros como explicación de la suerte y de las desigualdades ordinarias de la existencia. La humanidad está logrando emanciparse de la servidumbre de tener que justificar la mala suerte acudiendo a los espíritus espectrales. Si bien, aunque los hombres están abandonando la doctrina errónea de que las vicisitudes de la vida tienen una causa espiritual, muestran una sorprendente disposición a aceptar unas enseñanzas, casi igualmente falaces, que los incitan a atribuir todas las desigualdades humanas a la disfunción política, a la injusticia social y a la competencia industrial. Pero las nuevas legislaciones, la creciente filantropía y la mayor reorganización industrial, aunque sean buenas en sí mismas, no remediarán las realidades del nacimiento ni los accidentes de la vida. Solo la comprensión de los hechos y la inteligente actuación dentro de las leyes de la naturaleza permitirán que el hombre consiga lo que quiere y evite lo que no quiere. El conocimiento científico, conducente a la acción científica, es el único antídoto posible para los llamados males accidentales.
\vs p086 7:5 \pc La industria, la guerra, la esclavitud y el gobierno civil surgieron en respuesta a la evolución social del hombre en su medio natural; la religión apareció igualmente como respuesta al entorno ilusorio del mundo espectral imaginario. La religión fue un desarrollo evolutivo del automantenimiento, y ha dado resultado, a pesar de que era, en un principio, conceptualmente errónea y completamente ilógica.
\vs p086 7:6 Mediante la fuerza poderosa e impresionante del falso temor, la religión primitiva preparó el terreno a la mente humana para la dádiva de una genuina fuerza espiritual de origen sobrenatural: el modelador del pensamiento. Y los modeladores divinos han obrado desde entonces para transmutar el temor a Dios en amor a Dios. La evolución puede ser lenta, pero es inequívocamente efectiva.
\vsetoff
\vs p086 7:7 [Exposición de una estrella vespertina de Nebadón.]
