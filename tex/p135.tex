\upaper{135}{Juan el Bautista}
\author{Comisión de seres intermedios}
\vs p135 0:1 Juan el Bautista nació el 25 de marzo del año 7 a. C., en cumplimiento de la promesa que Gabriel le hizo a Elisabet en junio del año anterior. Durante cinco meses, Elisabet guardó en secreto la visita de Gabriel; cuando se lo contó a su marido, Zacarías, él se preocupó mucho y no acabó por creer su relato hasta que tuvo un sueño extraño, unas seis semanas antes del nacimiento de Juan. Salvo esta visitación de Gabriel a Elisabet y del sueño de Zacarías, no hubo nada fuera de lo común o sobrenatural en relación al nacimiento de Juan el Bautista.
\vs p135 0:2 Al octavo día, Juan fue circuncidado con arreglo a la costumbre judía. Día tras día y año tras año, creció como un niño ordinario en la pequeña aldea conocida en esos días como la ciudad de Judá, situada a unos seis kilómetros al oeste de Jerusalén.
\vs p135 0:3 El suceso más memorable de la infancia temprana de Juan fue la visita que hizo, acompañado de sus padres, a Jesús y a la familia de Nazaret. Esta visita tuvo lugar en el mes de junio del año 1 a. C., cuando él tenía poco más de seis años de edad.
\vs p135 0:4 Tras su regreso de Nazaret, los mismos padres de Juan comenzaron a educar al muchacho de forma regular. En aquella pequeña aldea, no había escuela en la sinagoga. Sin embargo, Zacarías, siendo sacerdote, era un hombre bastante bien formado y Elisabet poseía mucha mayor cultura que cualquier mujer corriente de Judea; ella era de linaje sacerdotal al ser descendiente de las “hijas de Aarón”. Dado que Juan era hijo único, podían emplear bastante tiempo en su instrucción mental y espiritual. Zacarías servía en el templo de Jerusalén por breves períodos de tiempo, por lo que podía dedicar gran parte de su tiempo a enseñar a su hijo.
\vs p135 0:5 Zacarías y Elisabet poseían una pequeña granja en la que criaban ovejas. Esta tierra apenas les daba lo suficiente para vivir, pero Zacarías percibía una asignación periódica de los fondos del templo destinados al sacerdocio.
\usection{1. JUAN SE CONVIERTE EN NAZAREO}
\vs p135 1:1 Juan no disponía de escuela en la que pudiera graduarse a la edad de catorce años, pero sus padres habían elegido este año como el más conveniente para que tomara los votos formales del nazareato. Consiguientemente, Zacarías y Elisabet llevaron a su hijo a En\hyp{}gadi, cerca del Mar Muerto. En este lugar se encontraba la sede sureña de la hermandad nazarea, y allí se admitió de por vida al joven en esta orden, con la solemnidad debida. Tras estas ceremonias y tomar los votos para abstenerse de bebidas embriagantes, dejarse crecer el pelo y evitar tocar a los muertos, la familia se encaminó a Jerusalén, donde Juan, frente al templo, dio cumplimiento a las ofrendas exigidas a quienes hacían estos votos.
\vs p135 1:2 Juan tomó los mismos votos perpetuos que se habían dispensado a sus ilustres predecesores: Sansón y el profeta Samuel. Al nazareo de por vida se le consideraba una persona consagrada y santa. Los judíos sentían por los nazareos casi el mismo respecto y veneración que profesaban por el sumo sacerdote, y esto no era de extrañar, puesto que los nazareos que se consagraban para toda la vida eran las únicas personas, salvo los sumos sacerdotes, a las que se les permitía entrar en el lugar santísimo del templo.
\vs p135 1:3 \pc Desde Jerusalén, Juan regresó a su casa para ocuparse del rebaño de ovejas de su padre; creció hasta convertirse en un hombre fuerte y de carácter noble.
\vs p135 1:4 A los dieciséis años, Juan, a raíz de su lectura de Elías, quedó muy impresionado con el profeta del monte Carmelo y decidió adoptar su forma de vestir. Desde aquel día, Juan siempre llevaba una indumentaria de pelo con un cinto de cuero. A los dieciséis años, medía ya más de un metro ochenta de altura y estaba casi completamente desarrollado. Su cabello suelto y su manera peculiar de vestir le hacían ciertamente un joven pintoresco. Y sus padres esperaban grandes cosas de él, su único hijo, un hijo de la promesa y nazareo perpetuo.
\usection{2. LA MUERTE DE ZACARÍAS}
\vs p135 2:1 Tras una enfermedad que duró varios meses, Zacarías murió en julio del año 12 d. C., justo después de que Juan cumpliera los dieciocho años. Aquel fue un momento de gran desconcierto para Juan puesto que el voto nazareo le prohibía cualquier contacto con los muertos, incluso con los de su propia familia. Aunque Juan procuró dar cumplimiento a las restricciones debidas a su voto en relación a la contaminación de los muertos, albergaba dudas respecto a su completa obediencia de las exigencias de su orden; así pues, después del entierro de su padre, fue a Jerusalén, donde, en el rincón nazareo del atrio de las mujeres, hizo ofrenda de los sacrificios necesarios para su purificación.
\vs p135 2:2 \pc En septiembre de este año, Elisabet y Juan viajaron a Nazaret para visitar a María y a Jesús. Juan casi había tomado la decisión de emprender su labor de vida, pero no solo las palabras de Jesús sino su ejemplo le advirtieron de que debía regresar a su casa, dedicarse a cuidar a su madre y esperar “a que llegara la hora del Padre”. Tras despedirse de Jesús y de María, al término de esta grata visita, Juan no volvería a ver a Jesús hasta el momento de su bautismo en el Jordán.
\vs p135 2:3 Juan y Elisabet regresaron a su hogar y empezaron a hacer planes para el futuro. Dado que Juan se negó a aceptar la asignación sacerdotal que le correspondía de los fondos del templo, al cabo de dos años habían prácticamente perdido su casa; así pues, decidieron ir al sur con su rebaño de ovejas. Por lo tanto, en el verano en el que Juan cumplió sus veinte años, se produjo su traslado a Hebrón. En el llamado “desierto de Judea”, Juan cuidaba de sus ovejas junto a un arroyo, afluente de un torrente mayor que se adentraba en el Mar Muerto a la altura de En\hyp{}gadi. El asentamiento de En\hyp{}gadi estaba integrado no solo por nazareos que se habían consagrado de por vida o de forma transitoria, sino que había también otros numerosos pastores ascéticos, que se congregaban en esta región con sus rebaños y confraternizaban con la hermandad nazarea. Se sostenían a sí mismos con la cría de las ovejas y con los donativos que los judíos ricos daban a la orden.
\vs p135 2:4 Conforme transcurría el tiempo, Juan regresaba a Hebrón cada vez con menor frecuencia, mientras que frecuentaba En\hyp{}gadi con mayor asiduidad. Era tan completamente diferente a la mayoría de los nazareos que le resultaba muy difícil confraternizar del todo con la hermandad. Si bien, tenía en gran estima a Abner, el reconocido líder y jefe del asentamiento.
\usection{3. SU VIDA COMO PASTOR}
\vs p135 3:1 A lo largo del valle de este arroyuelo, Juan construyó no menos de una docena de refugios de piedra y corrales nocturnos, consistentes en piedras apiladas, desde donde vigilaba y protegía a sus rebaños de ovejas y de cabras. Su vida como pastor le permitía tener bastante tiempo para pensar. Charlaba mucho con Ezda, un niño huérfano de Bet\hyp{}sur, a quien de alguna manera había adoptado, y que cuidaba de los rebaños cuando Juan iba a Hebrón a ver a su madre y vender ovejas, al igual que cuando bajaba a En\hyp{}gadi para asistir al \bibemph{sabbat}. Juan y el muchacho vivían de manera muy simple; subsistían a base de carne de cordero, leche de cabras, miel silvestre y de las langostas comestibles de esta región. Ocasionalmente, complementaban su dieta habitual con provisiones traídas de Hebrón y de En\hyp{}gadi.
\vs p135 3:2 \pc Elisabet mantenía a Juan informado sobre los asuntos de Palestina y del mundo, y estaba cada vez más profundamente convencido de que se acercaba rápidamente el fin del viejo orden y de que él se convertiría en el heraldo de la llegada de una nueva era, “el reino de los cielos”. Este rudo pastor era un gran lector de los escritos del profeta Daniel. Había leído cientos de veces su descripción de la gran imagen, que, según le había comentado Zacarías, representaba la historia de los grandes imperios del mundo, comenzando con Babilonia, luego Persia, Grecia y, finalmente, Roma. Juan se daba cuenta de que Roma estaba ya compuesta por tal heterogeneidad de pueblos y razas que nunca podría llegar a afianzarse ni a consolidarse firmemente como imperio. Creía que Roma estaba, incluso entonces, repartida entre Siria, Egipto, Palestina y otras provincias; y entonces leyó que “en los días de estos reyes, el Dios de los cielos levantará un reino que no será jamás destruido ni será este reino dejado a otro pueblo; desmenuzará y consumirá a todos estos reinos, pero él permanecerá para siempre”. “Y le fue dado dominio, gloria y reino, para que todos los pueblos, naciones y lenguas le sirvieran; su dominio es un lugar de eterna soberanía, que nunca pasará, y su reino uno que nunca será destruido”. “Y que el reino, el dominio y la majestad de los reinos debajo de todo el cielo sean dados al pueblo de los santos del Altísimo, cuyo reino es reino eterno, y todos los dominios lo servirán y obedecerán”.
\vs p135 3:3 \pc Juan nunca lograría sobreponerse por completo a la confusión que le había creado lo que había oído decir a sus padres sobre Jesús y estos pasajes de las escrituras. En el libro de Daniel leía: “Miraba yo en la visión de la noche, y vi que con las nubes del cielo venía uno como un hijo de hombre. Y le fue dado dominio, gloria y reino”. Pero estas palabras del profeta no coincidían con lo que sus padres le habían enseñado. Como tampoco lo que había hablado con Jesús, al tiempo de su visita cuando contaba dieciocho años, se correspondía con lo que las escrituras decían. A pesar de esta confusión, en medio de todo este sentimiento de perplejidad, su madre le aseguraba que su primo lejano, Jesús de Nazaret, era el verdadero Mesías, que había venido para ocupar el trono de David, y que él (Juan) se convertiría en el heraldo que iría por delante y en su principal apoyo.
\vs p135 3:4 Todo lo que había oído sobre el vicio y la iniquidad de Roma y el libertinaje y la estéril moral del Imperio, y lo que sabía acerca de los malvados actos de Herodes Antipas y de los gobernadores de Judea, llevaron a Juan a pensar que el fin de la era estaba próximo. A este rudo y noble hijo de la naturaleza le dio la impresión de que el mundo estaba listo para el fin de la era del hombre y para los albores de una era nueva y divina ---el reino de los cielos---. En el corazón de Juan creció la sensación de que él era el último de los antiguos profetas y el primero de los nuevos. Y vibró intensamente con un deseo que le empujaba a salir a proclamar a todos los hombres: “¡Arrepentíos! ¡Estad bien con Dios! Preparaos para el fin; preparaos para la aparición de un orden de los asuntos del mundo nuevo y eterno: el reino de los cielos”.
\usection{4. LA MUERTE DE ELISABET}
\vs p135 4:1 El 17 de agosto del año 22 d. C., cuando Juan contaba con veintiocho años de edad, su madre falleció de repente. Los amigos de Elisabet, conociendo las restricciones nazareas respecto al contacto con los muertos, incluso aquellos de la propia familia, hicieron todos los preparativos para el entierro de Elisabet antes de mandar a llamar a Juan. Al recibir noticias de la muerte de su madre, Juan indicó a Ezda que llevara sus rebaños a En\hyp{}gadi y partió para Hebrón.
\vs p135 4:2 A su regreso a En\hyp{}gadi, tras el funeral de su madre, entregó sus rebaños a la hermandad y durante una temporada se separó del resto del mundo para ayunar y orar. Juan conocía únicamente los viejos métodos de acercarse a la divinidad; solo conocía los textos escritos de Elías, Samuel y Daniel. Elías era su ideal de profeta. Él fue el primero de los maestros de Israel en tener la consideración de profeta; Juan creía realmente que él sería el último de este largo e ilustre linaje de mensajeros del cielo.
\vs p135 4:3 Juan vivió en En\hyp{}gadi durante dos años y medio y convenció a la mayor parte de la hermandad de que “el fin de la era ha llegado”, que “el reino de los cielos estaba a punto de aparecer”. Y sus primeras enseñanzas se basaban en la idea y el concepto judíos habituales del Mesías prometido, que liberaría a la nación judía de la dominación de sus gobernantes gentiles.
\vs p135 4:4 A lo largo de todo este período, Juan leyó con bastante frecuencia los escritos sagrados que encontró en el emplazamiento nazareo de En\hyp{}gadi. Particularmente le impresionaron Isaías y Malaquías, los últimos de los profetas hasta aquel momento. Leyó y releyó los últimos cinco capítulos de Isaías, y creyó en estas profecías. Luego leía en Malaquías: “He aquí que yo os envío al profeta Elías antes que venga el día de Jehová, grande y terrible. Él hará volver el corazón de los padres hacia los hijos, y el corazón de los hijos hacia los padres, no sea que yo venga y castigue la tierra con maldición”. Y fue solo esta promesa de Malaquías del regreso de Elías la que lo hizo desistir de salir a predicar la llegada del reino y exhortar a sus compatriotas judíos a que escaparan de la ira venidera. Juan estaba preparado para proclamar el mensaje de la llegada del reino, pero su expectativa de la vuelta de Elías lo detuvo durante más de dos años. Él sabía que no era Elías. ¿Qué quería decir Malaquías? ¿Era literal o figurada esta profecía? ¿Cómo podría él saber la verdad? Finalmente, se atrevió a pensar que, puesto que el primero de los profetas se llamaba Elías, el último llegaría a ser igualmente conocido por el mismo nombre. A pesar de todo, albergaba dudas, suficientes dudas como para impedirle llamarse a sí mismo, alguna vez, Elías.
\vs p135 4:5 Fue la influencia de Elías la que le hizo adoptar sus métodos de ataque directo y contundente contra los pecados y vicios de sus contemporáneos. Juan procuró vestir y hablar como Elías; en cada detalle de su aspecto exterior, era como el profeta de antiguo. Era un hijo de la naturaleza robusto y pintoresco, un predicador valeroso y audaz de la rectitud. Juan no era analfabeto, conocía bastante bien las sagradas escrituras judías, pero no era culto. Era un pensador lúcido, un orador poderoso y un acusador feroz. Difícilmente se podría considerar un ejemplo para su época, sino más bien, en sí mismo, un elocuente reproche.
\vs p135 4:6 Por último, ideó un método de proclamar la nueva era, el reino de Dios; resolvió que iba a convertirse en el heraldo del Mesías; dejó de lado todas sus dudas y partió de En\hyp{}gadi un día de marzo del año 25 d. C. para comenzar su breve pero brillante andadura como predicador público.
\usection{5. EL REINO DE DIOS}
\vs p135 5:1 Para entender el mensaje de Juan, habría que tener en cuenta las condiciones en las que se encontraba el pueblo judío en el momento de su aparición en escena. Durante casi cien años, todo Israel había estado en una disyuntiva; no sabía cómo explicar su continuado sometimiento a unos gentiles tiranos. ¿No había enseñado Moisés que siempre se recompensaba la rectitud con la prosperidad y el poder? ¿Es que no eran ellos el pueblo elegido de Dios? ¿Por qué estaba el trono de David asolado y vacante? A la luz de las doctrinas mosaicas y de los preceptos de los profetas, les resultaba difícil a los judíos dar una explicación a su larga desolación nacional.
\vs p135 5:2 Unos cien años antes de los días de Jesús y de Juan, había aparecido en Palestina una nueva escuela de maestros religiosos: los apocalípticos. Estos nuevos maestros desarrollaron un sistema de creencias para dar una explicación a los sufrimientos y a la humillación de los judíos, basándose en que eran el pago por los pecados de la nación. Recurrieron a razones históricamente bien conocidas, destinadas a justificar el cautiverio en Babilonia al igual que otros cautiverios de tiempos pasados. Pero, tal como los apocalípticos impartían, Israel debía cobrar ánimos; los días de su aflicción estaban llegando a su fin; el correctivo impuesto al pueblo elegido de Dios estaba casi terminando; la paciencia de Dios con los extranjeros gentiles estaba a punto de agotarse. El fin del gobierno romano era sinónimo del fin de la era y, en cierto modo, del fin del mundo. Estos nuevos maestros se apoyaban, en gran medida, en las predicciones de Daniel y enseñaban constantemente que la creación estaba próxima a entrar en su última etapa; los reinos de este mundo estaban muy cerca de convertirse en el reino de Dios; para la mentalidad judía de aquel tiempo, este era el significado de la expresión “el reino de los cielos” presente en las enseñanzas de Juan y de Jesús. Para los judíos de Palestina, dicha expresión tenía solamente un sentido: una condición de absoluta justicia en la que Dios (el Mesías) gobernaría las naciones de la tierra con el perfecto poder con el que gobernaba en los cielos: “Hágase tu voluntad, como en el cielo, así también en la tierra”.
\vs p135 5:3 En los días de Juan, los judíos se preguntaban expectantes: “¿Cuánto se demoraría la llegada del reino?”. Había un sentimiento general de que el fin del dominio de las naciones gentiles estaba cerca. En todo el mundo judío, se albergaba la esperanza viva y la intensa expectativa de que en el transcurso de aquella generación se culminaría su deseo de los tiempos.
\vs p135 5:4 Aunque entre los judíos había una gran diversidad de opiniones en cuanto a la naturaleza del reino venidero, todos ellos coincidían en creer que aquel acontecimiento era inminente, que estaba por llegar e incluso ya a las puertas. Muchos de quienes leían el Antiguo Testamento literalmente anticipaban con expectación la llegada del nuevo rey de Palestina, una nación judía regenerada que sería liberada de sus enemigos y gobernada por el sucesor del rey David, el Mesías que rápidamente sería reconocido como el soberano, legítimo y recto, del todo el mundo. Otro grupo más reducido de devotos judíos mantenían una opinión enteramente distinta acerca de este reino de Dios. Impartían la enseñanza de que el reino venidero no era de este mundo, que el mundo se estaba acercando a su fin y que “un nuevo cielo y una nueva tierra” marcarían la instauración del reino de Dios; que este reino dominaría sempiternamente, que el pecado acabaría y que los ciudadanos del nuevo reino se convertirían en inmortales y disfrutarían de esta dicha sin fin.
\vs p135 5:5 Todos estaban de acuerdo en que una drástica depuración, que un purificante correctivo precedería necesariamente a la instauración del nuevo reino en la tierra. Los literalistas enseñaban que se produciría una guerra a escala mundial que destruiría a todos los no creyentes, mientras que los creyentes arrasarían hasta lograr rápidamente una victoria universal y eterna. Los espiritualistas enseñaban que el reino se constituiría gracias un gran juicio por parte de Dios, que relegaría a los inicuos a un bien merecido juicio que los castigara y finalmente los aniquilara, elevando, al mismo tiempo, a los santos creyentes del pueblo elegido a los altos tronos de honor y autoridad con el Hijo del Hombre, que reinaría sobre las naciones redimidas en nombre de Dios. Y en este último grupo se creía incluso que muchos devotos gentiles podrían ser admitidos a la fraternidad del nuevo reino.
\vs p135 5:6 Algunos judíos sostenían que quizás Dios pudiese establecer este nuevo reino por intervención directa y divina, pero la inmensa mayoría creía que lo haría por mediación de alguien que lo representara, el Mesías. Y aquella era la única acepción posible que el término “Mesías” podía albergar en las mentes de los judíos de la generación de Juan y de Jesús. Era imposible que la palabra \bibemph{Mesías} pudiese hacer referencia solo a una persona que únicamente enseñara la voluntad de Dios o proclamara la necesidad de vivir una vida recta. A estas personas santas, los judíos les denominaban \bibemph{profetas}. El Mesías tendría que ser más que un profeta; el Mesías sería quien trajera la instauración del nuevo reino, el reino de Dios. Nadie que no lograra llevar esto a cabo podría ser el Mesías, en el sentido tradicional judío.
\vs p135 5:7 ¿Quién sería este Mesías? De nuevo, los maestros judíos diferían en la respuesta. Los más viejos se aferraban a la doctrina del hijo de David. Los más jóvenes enseñaban que, dado que el nuevo reino era un reino celestial, también el nuevo soberano podría ser una persona divina, alguien que por mucho tiempo se había sentado a la derecha del Dios de los cielos. Y, por extraño que parezca, quienes concebían así al soberano del nuevo reino no lo veían como un Mesías humano, no como un mero \bibemph{hombre,} sino como “el Hijo del Hombre” ---el Hijo de Dios---, un Príncipe celestial, mantenido en espera por mucho tiempo para asumir el gobierno de la tierra renovada. Este era el contexto religioso del mundo judío cuando Juan salió a proclamar: “¡Arrepentíos, porque el reino de los cielos se ha acercado!”.
\vs p135 5:8 Se pone de manifiesto, por tanto, que el anuncio de Juan sobre la llegada del reino tenía al menos media docena de lecturas diferentes por parte de las mentes de quienes oían su enardecida predicación. Pero, al margen del significado que los oyentes atribuyesen a las palabras de Juan, cada uno de estos distintos grupos de judíos expectantes del reino se sentía fascinado por las proclamas de este predicador, sincero, entusiasta y tosco pero vigoroso, de la rectitud y del arrepentimiento, que con tanta solemnidad exhortaba a sus oyentes a “huir de la ira venidera”.
\usection{6. JUAN COMIENZA SU PREDICACIÓN}
\vs p135 6:1 A comienzos del mes de marzo del año 25 d.C., Juan viajó por la costa occidental del Mar Muerto y se dirigió río arriba por el Jordán hasta llegar a la altura de Jericó, al antiguo vado por el que Josué y los hijos de Israel pasaron cuando entraron en la tierra prometida por primera vez; y cruzando al otro lado del río, se instaló cerca de la entrada del vado y comenzó a predicar a la gente que atravesaba el río en ambas direcciones. De todos los cruces del Jordán, este era el más frecuentado.
\vs p135 6:2 Para todos los que lo oían, resultaba evidente que Juan era más que un predicador. La gran mayoría de los que escuchaban a este hombre extraño venido del desierto de Judea se marchaba creyendo que aquella era la voz de un profeta. No era de extrañar que el alma de estos cansados y expectantes judíos se conmoviese intensamente ante tal extraordinaria circunstancia. Nunca, en toda la historia judía, habían anhelado tanto los devotos hijos de Abraham “la consolación de Israel” ni tan fervientemente esperado “la restauración del reino”. Jamás, en toda la historia judía, hubiese podido el mensaje de Juan, “el reino de los cielos se ha acercado”, producir un llamamiento tan profundo y generalizado como el que tuvo en aquel mismo momento en el que apareció tan misteriosamente en la orilla de este cruce meridional del río Jordán.
\vs p135 6:3 Juan procedía de pastores como Amós. Vestía como el antiguo Elías, y clamaba con fuerza sus amonestaciones y vertía sus advertencias en el “espíritu y el poder de Elías”. No es sorprendente que este extraño predicador creara una intensa agitación en toda Palestina conforme los viajeros difundían fuera de allí la noticia de su predicación junto al río Jordán.
\vs p135 6:4 Pero aún había otro rasgo más \bibemph{nuevo} respecto a la labor de este predicador nazareo: bautizaba a cada uno de sus creyentes en el Jordán “para la remisión de los pecados”. Aunque el bautismo no era una ceremonia nueva entre los judíos, nunca lo habían visto empleado de la manera en la que él lo usaba. Durante mucho tiempo se había tenido la costumbre de bautizar, en el atrio exterior del templo, a los prosélitos gentiles en el momento de su ingreso en la fraternidad judía, pero jamás se había pedido a los mismos judíos que se sometieran al bautismo del arrepentimiento. Solamente transcurrieron quince meses entre el momento en el que Juan comenzó a predicar y a bautizar y su arresto y encarcelación por mandato de Herodes Antipas, pero en este breve espacio de tiempo bautizó a muchos más de cien mil penitentes.
\vs p135 6:5 Juan predicó durante cuatro meses en el vado de Betania, antes de dirigirse al norte remontando el Jordán. Acudían decenas de miles de personas para oírle. Venían de todas partes de Judea, Perea y Samaria, e incluso algunas de Galilea. Había quienes asistían por curiosidad, pero muchas otras lo hacían con fervor y seriedad.
\vs p135 6:6 En mayo de este año, estando aún en el vado de Betania, los sacerdotes y levitas enviaron a una delegación para preguntarle si afirmaba ser el Mesías y con qué autoridad predicaba. Juan respondió a quienes le interrogaban diciendo: “Id y decid a quienes os mandan que habéis oído ‘la voz del que clama en el desierto’ tal como manifestó el profeta, diciendo: ‘Preparad un camino al Señor, enderezad las sendas para nuestro Dios. Todo valle se rellenará, y se bajará todo monte y collado; los caminos torcidos serán enderezados mientras que los caminos ásperos se convertirán en un valle allanado; y verá toda carne la salvación de Dios’”.
\vs p135 6:7 Juan era un predicador heroico, pero poco diplomático. Un día, cuando estaba predicando y bautizando en la margen occidental del Jordán, un grupo de fariseos y un cierto número de saduceos se llegaron hasta él y se presentaron para ser bautizados. Antes de conducirlos al agua, Juan dirigiéndose a ellos les dijo: “¿Quién os enseñó a huir, como víboras ante el fuego, de la ira venidera? Yo os bautizaré, pero os aviso que debéis traer frutos meritorios de arrepentimiento sincero, si queréis recibir la remisión de vuestros pecados. No me digáis que Abraham es vuestro padre. Os declaro que Dios puede levantar hijos dignos a Abraham aun de estas doce piedras. Además, el hacha ya está puesta a la raíz de los árboles; por tanto, todo árbol que no da buen fruto es cortado y echado al fuego”. (Las doce piedras a las que se refería eran las conocidas piedras conmemorativas, levantadas por Josué para rememorar el cruce de “las doce tribus” en este mismo punto, cuando entraron por primera vez en la tierra prometida).
\vs p135 6:8 Juan impartía clases a sus discípulos, durante las cuales los instruía en los detalles de su nueva vida y se esforzaba por contestar a sus numerosas preguntas. Aconsejaba a los maestros que educaran en el espíritu al igual que en la letra de la ley. Enseñaba a los ricos que dieran de comer a los pobres; a los cobradores de impuestos les decía: “No exijáis más de lo que os está ordenado”. A los soldados decía: “No hagáis extorsión a nadie, ni calumniéis; y contentaos con vuestro salario”. Asimismo aconsejaba a todos: “Estad preparados para el fin de la era ---el reino de los cielos se ha acercado---”.
\usection{7. JUAN VIAJA AL NORTE}
\vs p135 7:1 Juan albergaba aún ideas encontradas respecto a la venida del reino y a su rey. Mientras más predicaba, más confundido estaba, pero este desconcierto intelectual respecto a la naturaleza del reino venidero no hizo decrecer, en lo más mínimo, su convencimiento sobre su indudable aparición inmediata. En su mente, Juan podía estar confundido, pero nunca en su espíritu. No tenía dudas acerca de la llegada del reino, pero estaba lejos de estar seguro de si Jesús iba a ser o no el soberano de ese reino. Siempre y cuando Juan sostuviese la idea de la restauración del trono de David, las enseñanzas de sus padres de que Jesús, nacido en la ciudad de David, sería el tan largamente esperado libertador, le parecían consecuentes; pero, en esos momentos en los que era más proclive a la doctrina de un reino espiritual y al fin de la era temporal en la tierra, tenía la duda acuciante sobre el papel que desempeñaría Jesús en tales eventos. A veces lo cuestionaba todo, aunque no por mucho tiempo. Le hubiese gustado realmente haber podido hablar de todo esto con su primo, pero aquello era contrario a su acuerdo explícito.
\vs p135 7:2 \pc Conforme viajaba hacia el norte, Juan pensaba mucho en Jesús. Remontando el Jordán, se detuvo en más de una docena de lugares. Fue en Adam donde por primera vez hizo referencia a “el que viene tras de mí” en respuesta a la franca pregunta que sus discípulos le hicieron: “¿Eres tú el Mesías?”. Siguió diciendo: “Viene tras de mí el que es más poderoso que yo, ante quien no soy digno de agacharme y desatar la correa de su calzado. Yo os bautizo con agua, pero él os bautizará con espíritu santo. En su mano lleva la pala y va a limpiar cuidadosamente su era; recogerá su trigo en su granero, pero la paja la quemará con el fuego del juicio”.
\vs p135 7:3 En respuesta a las preguntas de sus discípulos, Juan siguió ampliando sus enseñanzas, añadiendo día a día principios de más ayuda y consuelo en comparación con su críptico primer mensaje: “Arrepentíos y sed bautizados”. Sobre esta época, de Galilea y de la Decápolis llegaban multitudes. Un gran número de fervorosos creyentes se quedaban con su adorado maestro día tras día.
\usection{8. EL ENCUENTRO DE JESÚS Y JUAN}
\vs p135 8:1 Hacia diciembre del año 25 d.C., cuando Juan llegó a los alrededores de Pella en su viaje río arriba del Jordán, su fama ya se había extendido por toda Palestina, y su labor se había convertido en el principal asunto de conversación de todas las poblaciones próximas al lago de Galilea. Jesús había hablado favorablemente del mensaje de Juan, y esto había hecho que muchas personas de Cafarnaúm se unieran al sistema de culto de Juan de arrepentimiento y el bautismo. Los pescadores, Santiago y Juan, hijos de Zebedeo, habían ido en diciembre, poco tiempo después de que Juan se estableciera cerca de Pella para predicar, y se habían ofrecido para ser bautizados. Acudían a ver a Juan una vez por semana y le llevaban a Jesús noticias, nuevas y de primera mano, de la labor del evangelista.
\vs p135 8:2 Santiago y Judá, los hermanos de Jesús, habían hablado de la posibilidad de ir a ver a Juan para ser bautizados; y ahora que Judá había llegado a Cafarnaúm para los servicios del \bibemph{sabbat,} él y Santiago, tras oír el discurso de Jesús en la sinagoga, decidieron que él los aconsejara respecto a estos planes. Esto ocurrió el sábado por la noche del 12 de enero del año 26 d. C. Jesús les pidió que aplazaran la conversación hasta el día siguiente, momento en el que les daría su respuesta. Jesús durmió muy poco aquella noche por estar en íntima comunión con el Padre celestial. Había convenido almorzar con sus hermanos al mediodía y aconsejarles en cuanto a su bautismo por Juan. Aquel domingo por la mañana, Jesús estaba trabajando, como era habitual, en la factoría de barcos. Santiago y Judá habían llegado con el almuerzo y le esperaban en el almacén de maderas, ya que aún no había llegado la hora del descanso del mediodía, y sabían que Jesús era muy metódico en tales cuestiones.
\vs p135 8:3 Poco antes de este descanso, Jesús dejó sus herramientas, se quitó su delantal de trabajo y, con sencillez, anunció a los tres trabajadores que estaban en el cuarto con él: “Ha llegado mi hora”. Salió en busca de sus hermanos Santiago y Judá, repitiendo: “Ha llegado mi hora ---vamos a donde está Juan---”. Y de inmediato marcharon para Pella, almorzando en el camino. Esto sucedió el domingo 13 de enero. Pernoctaron en el valle del Jordán y llegaron al lugar donde se encontraba Juan bautizando, al mediodía del día siguiente.
\vs p135 8:4 \pc Juan acababa de comenzar a bautizar a los candidatos de aquel día. Veintenas de penitentes esperaban en fila su turno cuando Jesús y sus dos hermanos ocuparon su lugar en esta hilera de fervientes hombres y mujeres que se habían convertido en creyentes de la predicación de Juan sobre el reino venidero. Juan había preguntado a los hijos de Zebedeo por Jesús. Había oído sus comentarios sobre su labor, y día y tras día esperaba verlo aparecer por allí, aunque no recibirlo en la fila de los candidatos al bautismo.
\vs p135 8:5 Estando absorto con los pormenores de bautizar con celeridad a un número tan grande de conversos, Juan no levantó la mirada ni vio a Jesús hasta que el Hijo del Hombre no estaba en su presencia inmediata. Cuando Juan reconoció a Jesús, interrumpió por un momento aquella ceremonia bautismal mientras que saludaba a su primo carnal y le preguntaba: “Pero ¿por qué has bajado hasta el agua para saludarme?”. Y Jesús respondió: “Me presento ante ti para que me bautices”. Y Juan contestó: “Pero soy yo quien necesita ser bautizado por ti. ¿Por qué acudes a mí?”. Y Jesús susurró a Juan: “Sé paciente conmigo ahora, porque conviene dar este ejemplo a mis hermanos que están aquí junto a mí, y para que la gente pueda saber que ha llegado mi hora”.
\vs p135 8:6 \pc Había un tono de determinación y autoridad en la voz de Jesús. Estremecido de emoción, Juan se preparó para bautizar a Jesús de Nazaret en el Jordán. Era el mediodía del 14 de enero del año 26 d. C. Así fue, por tanto, como Juan bautizó a Jesús y a Santiago y Judá, sus dos hermanos. Y cuando había bautizado a los tres, Juan despidió a los demás por el resto del día, indicando que reanudaría los bautismos al mediodía del día siguiente. Conforme la gente se marchaba, los cuatro hombres, todavía de pie en el agua, oyeron un sonido extraño y, pronto, se manifestó, durante un momento, una aparición directamente sobre la cabeza de Jesús, y oyeron una voz que decía: “Este es mi Hijo amado, en quien tengo complacencia”. Se produjo un gran cambio en el semblante de Jesús, que saliendo del agua en silencio se alejó de ellos, dirigiéndose hacia el este, hacia las colinas. Y nadie volvió a verlo en cuarenta días.
\vs p135 8:7 Juan siguió a Jesús la suficiente distancia como para narrarle la historia de la visitación de Gabriel a su madre antes de que ninguno de los dos hubiera nacido, tal como él la había escuchado tantas veces de labios de su madre. Dejó que Jesús continuara su camino después de decir: “Ahora sé de cierto que tú eres el Libertador”. Pero Jesús no le respondió.
\usection{9. LOS CUARENTA DÍAS DE PREDICACIÓN}
\vs p135 9:1 Cuando Juan volvió al lado de sus discípulos (tenía ya en ese momento unos veinticinco o treinta que lo acompañaban permanentemente), los encontró reunidos comentando seriamente lo que acababa de ocurrir en relación al bautismo de Jesús. Se sorprendieron aún más al oír cómo Juan les contaba la historia de la visitación de Gabriel a María antes de que Jesús naciera, y también debido al hecho de que Jesús no dijera nada, incluso tras haberle referido Juan este acontecimiento. Aquella noche no llovió, y los treinta hablaron durante largo rato bajo la noche estrellada. Se preguntaban dónde habría ido Jesús, y cuándo volverían a verlo.
\vs p135 9:2 \pc Tras lo acontecido ese día, la predicación de Juan cobró un nuevo y definitivo carácter de proclamación con respecto al reino venidero y al Mesías esperado. Estos cuarenta días de espera, aguardando el regreso de Jesús, les hizo vivir momentos de tensión. Pero Juan continuó predicando con gran fortaleza, y sus discípulos comenzaron a predicar, sobre esta fecha, a la desbordada multitud que se congregaba en torno a Juan, junto al Jordán.
\vs p135 9:3 En el trascurso de estos cuarenta días de espera, se difundieron muchos rumores por las zonas rurales, que llegaron incluso a Tiberias y a Jerusalén. Acudieron por miles al campamento de Juan para ser testigos de la nueva atracción, el reputado Mesías, pero allí no estaba Jesús para que lo pudieran ver. Cuando los discípulos de Juan refirieron que el extraño hombre de Dios se había ido a las colinas, hubo bastantes personas que dudaron de toda la historia.
\vs p135 9:4 Unas tres semanas después de la marcha de Jesús, una nueva delegación de los sacerdotes y fariseos de Jerusalén se acercó a Pella. Le preguntaron a Juan directamente si él era Elías o el profeta que Moisés había prometido; y al contestar Juan, “No soy yo”, se atrevieron a inquirir, “¿Eres tú el Mesías?”, y Juan respondió: “No soy yo”. Entonces estos hombres de Jerusalén dijeron: “Si tú no eres Elías ni el profeta ni el Mesías, ¿por qué entonces bautizas a la gente y creas tanto revuelo?”. Y Juan le dijo: “Son aquellos que me han oído y han recibido mi bautismo los que deberían deciros quién soy yo, pero os declaro que, mientras yo bautizo con agua, ha habido entre nosotros alguien que volverá para bautizaros con el espíritu santo”.
\vs p135 9:5 Aquel período de cuarenta días fue difícil para Juan y sus discípulos. ¿Cuál iba a ser la relación entre Juan y Jesús? Se planteaban cientos de interrogantes. Empezaron a ponerse de manifiesto políticas a seguir y preferencias egoístas. Se originaban intensos debates en torno a las distintas ideas y conceptos del Mesías. ¿Se convertiría en un líder militar y en un rey davídico? ¿Aniquilaría a los ejércitos romanos como había hecho Josué con los cananeos? ¿O establecería un reino espiritual? Juan llegó más bien a la conclusión, junto a una minoría, de que Jesús había venido para instituir el reino de los cielos, aunque en su mente no tenía del todo claro lo que esta misión exactamente podría conllevar.
\vs p135 9:6 Fueron días extenuantes para Juan; oraba para que Jesús volviera. Algunos de sus discípulos se organizaron en grupos para ir a buscar a Jesús, pero él lo prohibió, diciendo: “Nuestro tiempo está en las manos del Dios de los cielos; él es quien guiará a su Hijo elegido”.
\vs p135 9:7 \pc Temprano, en la mañana del 23 de febrero, día del \bibemph{sabbat,} estando Juan y sus acompañantes tomando su comida matinal, al mirar al norte, vieron como Jesús venía hacia ellos. Conforme se aproximaba, Juan se colocó en una gran roca y, levantando su voz resonante, dijo: “¡Contemplad al Hijo de Dios, al libertador del mundo! Este es de quien yo dije: ‘Después de mí viene un hombre que es antes de mí, porque era primero que yo’. Por esta causa he salido del desierto para predicar el arrepentimiento y bautizar con agua, proclamando que el reino de los cielos se ha acercado. Y ahora viene aquel que os bautizará con el espíritu santo. Yo he contemplado al espíritu divino descender sobre este hombre, y he oído la voz de Dios declarar: ‘Éste es mi Hijo amado, en quien tengo complacencia’”.
\vs p135 9:8 Jesús les rogó que continuasen con su comida mientras se sentaba a comer con Juan; sus hermanos, Santiago y Judá, ya habían regresado a Cafarnaúm.
\vs p135 9:9 \pc Al día siguiente, por la mañana temprano, Jesús se despidió de Juan y de sus discípulos y se encaminó de regreso a Galilea. No les mencionó cuándo lo verían de nuevo. A las preguntas de Juan sobre su propia predicación y misión, Jesús solo dijo: “Mi Padre te guiará ahora y en el futuro como lo ha hecho en el pasado”. Y así se separaron estos dos grandes hombres aquella mañana a orillas del Jordán, para no volver a verse nunca más el uno al otro en la carne.
\usection{10. JUAN VIAJA AL SUR}
\vs p135 10:1 Al haber ido Jesús al norte, a Galilea, Juan se sintió llamado a volver sobre sus pasos en dirección sur. Por ello, el domingo por la mañana del 3 de marzo, Juan y el resto de sus discípulos emprendieron viaje al sur. Entretanto, alrededor de una cuarta parte de los seguidores más directos de Juan habían partido hacia Galilea en busca de Jesús. La tristeza y la confusión se adueñaron de Juan. Nunca más volvió a predicar como lo había hecho antes de bautizar a Jesús. De alguna manera, tuvo la impresión de que sobre sus hombros ya no recaía la responsabilidad del reino venidero. Notó que su labor había acabado prácticamente; estaba desconsolado y se sentía solo. Pero, predicó, bautizó y continuó su viaje al sur.
\vs p135 10:2 Juan se detuvo varias semanas cerca de la aldea de Adam, y fue aquí donde lanzó su memorable ataque contra Herodes Antipas por haber tomado ilícitamente a la mujer de otro hombre. En junio de este año (26 d.C.), Juan se encontraba de nuevo en el vado del Jordán en Betania, lugar en el que había empezado su predicación del reino venidero más de un año antes. En las semanas siguientes al bautismo de Jesús, la naturaleza de la predicación de Juan fue paulatinamente modificándose hasta convertirse en una proclamación de misericordia para la gente común, en tanto que denunciaba, con renovada vehemencia, a dirigentes políticos y religiosos corruptos.
\vs p135 10:3 Herodes Antipas, en cuyo territorio había estado predicando Juan, empezó a alarmarse, no fuese que él y sus discípulos comenzaran una rebelión. Herodes estaba también resentido por la denuncia pública que Juan hacía de sus asuntos domésticos. Ante todo esto, Herodes tomó la decisión de meterlo en la cárcel. Así pues, muy temprano, en la mañana del 12 de junio, antes de que llegara la multitud para escuchar la predicación de Juan y presenciar los bautismos, los agentes de Herodes lo pusieron bajo arresto. Como las semanas pasaban sin que quedase en libertad, sus discípulos se repartieron por toda Palestina, y muchos de ellos fueron a Galilea para unirse a los seguidores de Jesús.
\usection{11. JUAN EN LA CÁRCEL}
\vs p135 11:1 En la cárcel, Juan experimentó soledad y cierta amargura. A pocos de sus seguidores se les dio permiso para visitarlo. Ansiaba ver a Jesús, pero tuvo que conformarse con enterarse de su labor a través de discípulos suyos que se habían convertido en creyentes del Hijo del Hombre. Con frecuencia, se veía tentado a dudar de Jesús y de su misión divina. Si Jesús era el Mesías, ¿por qué no hacía nada por liberarlo de esta insoportable reclusión? Durante más de año y medio, este robusto hombre de la naturaleza languideció en aquella deleznable prisión. Y esta encarcelación fue una gran prueba de su fe en Jesús y de su lealtad hacia él; de hecho, significó una gran prueba de su fe en Dios incluso. Muchas veces tuvo la tentación de dudar además de la autenticidad de su propia misión y experiencias.
\vs p135 11:2 \pc Tras haber estado en prisión durante varios meses, un grupo de sus discípulos se llegó hasta él y, después de informarle acerca de la actividad pública de Jesús, le dijeron: “Como verás, Maestro, aquel que estuvo contigo en el alto del Jordán prospera y recibe a todos los que van a él. Incluso festeja con publicanos y pecadores. Tú diste un valiente testimonio de él y, sin embargo, él no hace nada por liberarte”. Pero Juan contestó a sus amigos: “Este hombre no puede hacer nada a menos que le sea dado por su Padre celestial. Recordad bien que os dije, ‘Yo no soy el Mesías sino que se me ha enviado delante de él para prepararle el camino’. El que tiene a la esposa es el esposo; pero el amigo del esposo, el que está a su lado y lo oye, se goza grandemente de la voz del esposo. Por eso, mi gozo está completo. Es necesario que él crezca, y que yo disminuya. Yo soy de esta tierra y he proclamado mi mensaje. Jesús de Nazaret bajó a la tierra desde el cielo y está por encima de todos. El Hijo del Hombre ha descendido de Dios y habla las palabras de Dios. Porque el Padre celestial no da el espíritu por medida a su propio Hijo. El Padre ama a su Hijo y ha entregado todas las cosas en sus manos. Aquel que cree en el Hijo tiene vida eterna. Y estas palabras que yo hablo son veraces y permanecerán”.
\vs p135 11:3 \pc Estos discípulos se sorprendieron tanto de las aseveraciones de Juan que partieron en silencio. Juan también se sentía muy alterado, porque observó que acababa de pronunciar una profecía. Jamás volvió a dudar enteramente de la misión y divinidad de Jesús. Pero, para Juan, fue una dolorosa decepción que Jesús no le enviara palabra alguna ni fuese a visitarle ni hiciera uso de su gran poder para liberarle de la prisión. Pero Jesús era consciente de todo aquello. Le profesaba un gran amor, si bien, siendo en ese momento conocedor de su naturaleza divina y, sabiendo las grandes cosas que le estaban preparadas a Juan cuando dejara este mundo así como que su labor en la tierra había acabado, se vio obligado a no interferir con el desarrollo natural de la andadura del gran predicador\hyp{}profeta.
\vs p135 11:4 \pc La larga incertidumbre que experimentó en la cárcel era humanamente insufrible. Muy pocos días antes de su muerte, Juan envió de nuevo mensajeros de su confianza a Jesús para que le preguntaran: “¿Ha concluido mi labor? ¿Por qué estoy padeciendo en la cárcel? ¿Eres tú en verdad el Mesías que había de venir o esperaremos a otro?”. Y cuando estos dos discípulos le transmitieron el mensaje a Jesús, el Hijo del Hombre respondió: “Volved a Juan, decidle que yo no lo he olvidado, pero que resista esto por mí, porque es conveniente que cumplamos toda justicia. Haced saber a Juan lo que habéis visto y oído, que a los pobres es anunciado la buena nueva, y, por último, decidle al amado heraldo de mi misión terrenal que será bendecido abundantemente en la era venidera si no busca pretextos para dudar de mí ni halle tropiezo en mí”. Y estas fueron las últimas palabras que Juan recibiría de Jesús. Encontró mucho consuelo en ellas y se esforzó en desarrollar una fe estable y prepararse para el trágico fin de su vida en la carne que pronto seguiría de cerca a esta memorable ocasión.
\usection{12. LA MUERTE DE JUAN EL BAUTISTA}
\vs p135 12:1 Como Juan predicaba y enseñaba en el sur de Perea, al ser arrestado, se le condujo de inmediato a la prisión de la fortaleza de Maqueronte, donde estaría encarcelado hasta su ejecución. Herodes gobernaba en Perea a la vez que en Galilea y, en este momento, residía en Perea, tanto en Julias como en Maqueronte. En Galilea, su residencia oficial se había trasladado desde Séforis hasta Tiberias, la nueva capital.
\vs p135 12:2 Herodes temía liberar a Juan no fuese que instigase una rebelión. Tenía miedo de darle muerte no fuese que la multitud organizase alguna revuelta; había miles de habitantes de Perea que creían que Juan era un hombre santo, un profeta. Por ello, Herodes mantuvo al predicador nazareo en la cárcel sin saber qué hacer con él. Juan había comparecido repetidas veces ante Herodes, pero no accedió a abandonar el territorio de Herodes ni a renunciar a su actividad pública aunque fuese puesto en libertad. Y la nueva y creciente agitación formada en torno a Jesús advirtió a Herodes que aquel no era el momento de soltarlo. Además, Juan era asimismo víctima del encarnizado e intenso odio que Herodías, la mujer ilegítima de Herodes, le profesaba.
\vs p135 12:3 En numerosas ocasiones, Herodes habló con Juan sobre el reino de los cielos y, aunque este se sentía a veces profundamente impresionado con sus palabras, tenía miedo de excarcelarlo.
\vs p135 12:4 Dado que en Tiberias aún se seguían construyendo bastantes edificios, Herodes pasaba un tiempo considerable en sus residencias de Perea, y tenía predilección por la fortaleza de Maqueronte. Pasarían algunos años antes de que todos los edificios públicos y la residencia oficial de Tiberias se acabaran de construir.
\vs p135 12:5 \pc Para celebrar su cumpleaños, Herodes organizó una gran fiesta en el palacio de Maqueronte para sus oficiales principales y para otros hombres de alto rango de los consejos del gobierno de Galilea y Perea. Como Herodías había fracasado en su intento de lograr la muerte de Juan apelando de forma directa a Herodes, se dispuso entonces astutamente a la tarea de llevar a Juan a la muerte.
\vs p135 12:6 Por la noche, en el transcurso de las fiestas y la diversión, Herodías presentó a su hija para que bailara ante los comensales. A Herodes le complació mucho la actuación de la joven y, llamándola ante él, le dijo: “Eres adorable. Estoy muy contento contigo. Pídeme en este mi cumpleaños lo que desees, y yo te lo daré, hasta incluso la mitad de mi reino”. Y Herodes se comportó así porque estaba bajo los efectos del mucho vino que había bebido. La muchacha se apartó y le preguntó a su madre qué debería requerir de Herodes. Herodías le dijo: “Ve a Herodes y pídele la cabeza de Juan el Bautista”. Y la muchacha, regresando a la mesa del banquete, dijo a Herodes: “Quiero que ahora mismo me des en un plato la cabeza de Juan el Bautista”.
\vs p135 12:7 A Herodes le invadió el temor y la tristeza, pero a causa del juramento y de los que estaban con él a la mesa, no quiso desairarla. Y Herodes Antipas, envió a un soldado, mandándole que trajese la cabeza de Juan. Juan fue pues decapitado aquella noche en la prisión; el soldado trajo la cabeza del profeta en un plato y se lo dieron a la muchacha, que se encontraba en el fondo del salón del banquete. Y ella le entregó el plato a su madre. Cuando oyeron esto sus discípulos, fueron a la prisión a por el cuerpo de Juan y, tras ponerlo en un sepulcro, dieron noticia de ello a Jesús.
