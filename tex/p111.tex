\upaper{111}{El modelador y el alma}
\author{Mensajero solitario}
\vs p111 0:1 La presencia del modelador divino en la mente humana hace para siempre imposible que, a través de la ciencia o de la filosofía, se llegue a un entendimiento satisfactorio del alma evolutiva de la persona humana. El alma morontial es hija del universo y únicamente se le puede conocer realmente por medio de la percepción cósmica y del descubrimiento espiritual.
\vs p111 0:2 \pc El concepto del alma y del espíritu morador no es nuevo en Urantia; con frecuencia, ha aparecido en los diferentes sistemas de creencias planetarias. Muchas de las fes orientales, al igual que algunas de las occidentales, han llegado a la conclusión de que el hombre es de linaje divino a la vez que humano. La idea de la presencia interior así como de la omnipresencia exterior de la Deidad ha formado parte, durante mucho tiempo, de numerosas religiones urantianas. Los hombres llevan tiempo creyendo que hay algo que crece en el seno de la naturaleza humana, algo vital que está llamado a perdurar más allá del corto período de la vida temporal.
\vs p111 0:3 Antes de que el hombre se percatara de que quien engendró su alma evolutiva fue un espíritu divino, se creía que esta residía en diferentes órganos físicos: el ojo, el hígado, el riñón, el corazón y, más tarde, el cerebro. El salvaje relacionaba el alma con la sangre, el aliento, las sombras y con el reflejo de sí mismo en el agua.
\vs p111 0:4 Al concebir el \bibemph{atman,} los maestros hindúes se acercaron realmente a la aceptación de la naturaleza y de la presencia del modelador, pero no supieron distinguir la presencia simultánea del alma evolutiva, potencialmente inmortal. Los chinos, sin embargo, reconocieron dos aspectos del ser humano, el \bibemph{yang} y el \bibemph{yin,} el alma y el espíritu. Los egipcios y muchas tribus africanas también creían en dos elementos, el \bibemph{ka} y el \bibemph{ba;} no se solía pensar que el alma era preexistente, sino solo el espíritu.
\vs p111 0:5 Los habitantes del valle del Nilo creían que a cualquier persona privilegiada se le había otorgado como don, en el momento de su nacimiento, o poco después, un espíritu protector que llamaban el ka. Impartían la enseñanza de que este espíritu guardián permanecía con el mortal durante toda su vida y pasaba antes que él a una futura heredad. En los muros de un templo de Lúxor, donde se representa el nacimiento de Amenhotep III, el pequeño príncipe figura en los brazos del dios del Nilo y, cerca de él, hay otro niño, de apariencia idéntica al príncipe, que constituye un símbolo de esa entidad que los egipcios denominaban el ka. Esta escultura se construyó en el siglo XV a. C.
\vs p111 0:6 Se creía que el ka era un genio espiritual de orden superior que deseaba guiar al alma del mortal al que acompañaba por los mejores senderos de la vida temporal pero, sobre todo, influir en la suerte del sujeto humano en el más allá. Cuando un egipcio de este período moría, se preveía que su ka lo estaría esperando en el otro lado del Gran Río. En un principio, se suponía que solo los reyes tenían ka, pero pronto se llegó a la creencia de que todos los hombres justos lo poseían. Un gobernante egipcio, al hablar del ka que estaba en su corazón, dijo: “No ignoré sus palabras; temía transgredir su guía. Con lo que prosperé sobremanera; así pues, tuve éxito en razón de lo que se me llevó a hacer; se me dignificó con su guía”. Muchos pensaban que el ka era “un oráculo de Dios presente en todos”, que “pasarían la eternidad con el corazón alegre a favor del Dios que hay en vosotros”.
\vs p111 0:7 Cada una de las razas de mortales evolutivas de Urantia tiene una palabra equivalente al concepto del alma. Muchos pueblos primitivos consideraban que el alma se asomaba al mundo a través de los ojos humanos; por consiguiente, sentían un temor tan lamentable por el malévolo mal de ojo. Por mucho tiempo han creído que “Lámpara del Señor es el espíritu del hombre”. Dice el Rig\hyp{}Veda: “Mi mente habla a mi corazón”.
\usection{1. LA MENTE COMO ESCENARIO DE LAS DECISIONES}
\vs p111 1:1 Aunque su labor es de naturaleza espiritual, los modeladores deben necesariamente llevarla a cabo sobre unos pilares intelectuales. La mente es el terreno humano en el que el mentor espiritual ha de hacer evolucionar al alma morontial con la cooperación de la persona en la que mora.
\vs p111 1:2 Hay una unidad cósmica en los diversos niveles mentales del universo de los universos. Los yos intelectuales tienen su origen en la mente cósmica tal como las nebulosas lo tienen en las energías cósmicas del espacio del universo. En el nivel humano (y personal por consiguiente) de los yos intelectuales, el potencial de la evolución espiritual se vuelve preponderante, con el asentimiento de la mente mortal, a causa de las dotes espirituales respecto del ser personal humano en conjunción con la presencia creativa de una entidad\hyp{}presencia de valor absoluto en dichos yos humanos. Pero tal predominio del espíritu sobre la mente material depende de dos circunstancias: esta mente debe haber evolucionado mediante el ministerio de los siete espíritus ayudantes de la mente, y el yo material (personal) debe optar por cooperar con el modelador interior en la creación y avance del yo morontial, esto es, el alma, evolutiva y potencialmente inmortal.
\vs p111 1:3 \pc La mente material es el escenario en el que vive el ser personal humano, que tiene autoconciencia, toma decisiones, opta por Dios o lo abandona, se eterniza o se destruye a sí mismo.
\vs p111 1:4 \pc La evolución material os ha proporcionado un mecanismo vivo, vuestro cuerpo; el Padre mismo os ha dotado de la realidad espiritual más pura conocida en el universo, vuestro modelador del pensamiento. Pero en vuestras manos, sometidas a vuestras decisiones, se os ha puesto la mente, y es por medio de esta que vivís o morís. En el seno de la mente, y con ella, tomáis esas decisiones morales que os posibilitan lograr semejaros a vuestro modelador, lo que es semejaros a Dios.
\vs p111 1:5 La mente mortal es un mecanismo, temporal e intelectual, prestado a los seres humanos para usarlo durante toda la vida material y, según la empleen, estarán aceptando o rechazando el potencial de la existencia eterna. La mente es prácticamente todo lo que tenéis, en el universo, de realidad sujeta a vuestra voluntad, y el alma ---el yo morontial--- reflejará fielmente la cosecha de las decisiones temporales que el yo mortal vaya realizando. La conciencia humana reposa suavemente sobre el mecanismo electroquímico que está por debajo, y toca delicadamente el sistema de energía espíritu\hyp{}morontial que está por arriba. Durante su vida mortal, el ser humano nunca es completamente consciente de ninguno de estos dos sistemas; por lo tanto, debe actuar pensando en lo que es consciente. Y no se trata tanto de lo que la mente comprenda, sino de lo que desea comprender lo que le asegura la supervivencia; no se trata tanto de cómo es la mente, sino de lo que trata de ser lo que constituye la identificación espiritual. No se trata tanto de que el hombre sea consciente de Dios, sino de su anhelo de Dios lo que trae consigo su ascenso en el universo. Lo que sois hoy no es tan importante como lo que estáis llegando a ser día a día y en la eternidad.
\vs p111 1:6 La mente es el instrumento cósmico con el cual la voluntad humana puede ejecutar las discordancias de la destrucción o puede interpretar con esta misma voluntad las excelentes melodías de su identificación con Dios y la consiguiente supervivencia eterna. El modelador, que se otorga al hombre es, en última instancia, inmune al mal e incapaz de pecar, pero la mente mortal puede, en efecto, retorcerse, distorsionarse, hacerse maligna y llenarse de fealdad por las maquinaciones pecaminosas de una voluntad humana perversa e interesada. De igual manera, esta mente puede hacerse noble, bella, verdadera y buena ---realmente grande--- en conformidad con la voluntad espiritualmente iluminada del ser humano que conoce a Dios.
\vs p111 1:7 \pc La mente evolutiva es totalmente estable y digna de confianza solo cuando se manifiesta incluyendo los dos extremos de la intelectualidad cósmica: el completamente mecanizado y el enteramente espiritualizado. Entre los extremos intelectuales del puro control mecánico y el de la verdadera naturaleza espiritual media ese inmenso grupo de mentes en evolución y ascenso, cuya estabilidad y tranquilidad dependen de sus opciones personales y de su identificación con el espíritu.
\vs p111 1:8 Pero el hombre no entrega su voluntad al modelador pasiva y sumisamente. Más bien, opta, de forma activa, positiva y cooperativa, por seguir la guía del modelador cuando y en la medida en la que dicha guía se diferencia conscientemente de los deseos e impulsos de la mente mortal natural. Los modeladores actúan sobre la mente del hombre, pero nunca la dominan en contra de su voluntad; para los modeladores, la voluntad humana es suprema. Y por eso, la tienen en consideración y la respetan mientras tratan de lograr sus objetivos espirituales de modelar el pensamiento y de transformar el carácter, en el casi ilimitado escenario del intelecto humano en evolución.
\vs p111 1:9 \pc La mente es vuestro navío, el modelador es vuestro piloto, la voluntad humana es el capitán. El dueño de la nave humana debería tener la sabiduría de confiar en el piloto divino para guiar a su alma en su ascenso a los puertos morontiales de la supervivencia eterna. Solo mediante el egoísmo, la indolencia y la pecaminosidad puede la voluntad del hombre rechazar la guía de un piloto tan amoroso y acabar su andadura mortal por sucumbir en los malévolos acantilados del rechazo a la misericordia y contra las rocas del pecado asumido. Con vuestro consentimiento, este fiel piloto os conducirá, con seguridad, a través de las barreras del tiempo y los impedimentos del espacio, hacia la fuente misma de la mente divina y más allá, incluso al Paraíso, hasta el Padre de los modeladores.
\usection{2. NATURALEZA DEL ALMA}
\vs p111 2:1 En todas las funciones mentales de la inteligencia cósmica, la totalidad de la mente predomina sobre las partes de la función intelectual. La mente es, en esencia, una unidad de carácter operativo; por consiguiente, la mente nunca deja de manifestar esta unidad fundamental a ella, incluso cuando se ve obstaculizada e impedida por las acciones y opciones insensatas de un yo equivocado. Y esta unidad de la mente busca indefectiblemente armonizarse con el espíritu en todos los niveles en los que dicha mente se relaciona con unos yos que gozan de dignidad volitiva y de las prerrogativas de la ascensión.
\vs p111 2:2 La mente material del hombre mortal es el telar cósmico que porta el tejido morontial sobre el que el modelador del pensamiento interior teje los patrones espirituales de un carácter, pertinente al universo, de valores perdurables y de contenidos divinos ---un alma superviviente con un destino último y una interminable andadura: un finalizador potencial---.
\vs p111 2:3 El ser personal humano se identifica con la mente y el espíritu, que se mantienen unidos en una relación de naturaleza operativa mediante la vida existente en un cuerpo material. Esta relación operante de tal mente y espíritu no da lugar a una combinación de las cualidades o atributos de la mente y del espíritu, sino más bien a un valor universal enteramente nuevo, primigenio y único, de una duración potencialmente eterna: el \bibemph{alma}.
\vs p111 2:4 \pc Son tres, y no dos, los factores involucrados en la creación evolutiva de tal alma inmortal. Estos tres antecedentes del alma morontial humana son:
\vs p111 2:5 \li{1.}\bibemph{La mente humana} y todas las influencias cósmicas que la anteceden e inciden en ella.
\vs p111 2:6 \li{2.}\bibemph{El espíritu divino} que mora en esta mente humana y todos los potenciales intrínsecos a tal fracción de espiritualidad absoluta, junto con todas las influencias y factores espirituales acompañantes que existen en la vida humana.
\vs p111 2:7 \li{3.}\bibemph{La relación entre la mente material y el espíritu divino,} que supone un valor y comporta un significado que no se halla en ninguno de los factores que contribuyen a tal vinculación. La realidad de esta singular relación no es ni material ni espiritual, sino morontial. Es el alma.
\vs p111 2:8 \pc Desde hace tiempo, las criaturas intermedias han denominado a esta alma evolutiva del hombre como mente intermedia a diferencia de la mente material o menor y de la mente cósmica o superior. Esta mente intermedia es, de hecho, un fenómeno morontial dado que existe en un entorno situado entre lo material y lo espiritual. El potencial de tal evolución morontial es intrínseco a los dos impulsos universales de la mente: el impulso de la mente finita de la criatura por conocer a Dios y lograr la divinidad del Creador, y el impulso de la mente infinita del Creador por conocer al hombre y adquirir la \bibemph{experiencia} de la criatura.
\vs p111 2:9 Esta suprema empresa de hacer desarrollar el alma es posible porque la mente mortal es, primero, personal y, segundo, porque está en contacto con realidades supraanimales; posee una dotación supramaterial gracias al ministerio cósmico que asegura la evolución de una naturaleza moral capaz de tomar decisiones morales, efectuándose con ello un genuino contacto creativo con sus servidores espirituales acompañantes y con el modelador del pensamiento interior.
\vs p111 2:10 El inevitable resultado de dicha espiritualización, mediante este contacto de la mente humana, constituye el nacimiento del alma, la progenie conjunta de una mente encauzada en los asistentes dominada por una voluntad humana que anhela conocer a Dios, obrando en vinculación con las fuerzas espirituales del universo, que están bajo la acción directiva de una fracción auténtica del mismo Dios de toda la creación: el mentor misterioso. Y, de este modo, la realidad material y mortal del yo trasciende las limitaciones temporales de la maquinaria de la vida física, logrando una nueva expresión y una nueva identificación en el vehículo evolutivo, que da continuidad al propio yo: el alma morontial e inmortal.
\usection{3. EL ALMA EVOLUTIVA}
\vs p111 3:1 Las equivocaciones de la mente mortal y los errores del comportamiento humano pueden retrasar considerablemente la evolución del alma, aunque no pueden inhibir este fenómeno morontial, una vez que el modelador interior lo haya iniciado con el consentimiento de la voluntad de la criatura. Pero, en cualquier momento, con anterioridad a la muerte física, esta misma voluntad material y humana está facultada para revocar tal elección y rechazar la supervivencia. Incluso después de sobrevivir, el mortal ascendente retiene todavía la prerrogativa de decidir rechazar la vida eterna; en cualquier momento, antes de la fusión con el modelador, la criatura en evolución y en ascenso puede optar por renunciar a la voluntad del Padre del Paraíso. La fusión con el modelador indica el hecho de que el mortal ascendente ha elegido eterna e incondicionalmente hacer la voluntad del Padre.
\vs p111 3:2 Durante la vida en la carne, el alma evolutiva está capacitada para consolidar las decisiones supramateriales de la mente mortal. El alma, siendo supramaterial, no obra de por sí en el nivel material de la experiencia humana. Ni puede esta alma subespiritual, sin la colaboración de algún espíritu de la Deidad, tal como el modelador, obrar por encima del nivel morontial. Tampoco adopta el alma decisiones definitivas hasta que la muerte o la traslación al cielo la separan de su vinculación material con la mente mortal, salvo cuando y en la medida en la que la mente material delega esta autoridad, libre y voluntariamente, a tal alma morontial en su actuación conjunta. Durante la vida, la voluntad mortal, el poder personal de la decisión\hyp{}elección, reside en las vías circulatorias de la mente material; mientras prosigue el crecimiento del mortal en la tierra, este yo, con sus invaluables facultades de elección, se vuelve cada vez más identificado con la entidad emergente del alma morontial. Tras la muerte y la resurrección en los mundos de las moradas, el ser personal humano está completamente identificado con el yo morontial. El alma es, por ello, el embrión del futuro vehículo morontial de la identidad personal.
\vs p111 3:3 Esta alma inmortal es, en un principio, de naturaleza totalmente morontial, pero posee tal capacidad de desarrollo que indefectiblemente asciende a los verdaderos niveles espirituales de valor en los que puede ocurrir la fusión con los espíritus de la Deidad, normalmente con el mismo espíritu del Padre Universal que inició este fenómeno creativo en la mente de la criatura.
\vs p111 3:4 Tanto la mente humana como el modelador divino son conscientes de la presencia y la naturaleza diferenciada de la evolución del alma ---el modelador lo es plenamente, la mente de forma parcial---. El alma se vuelve cada vez más consciente de la mente y del modelador como identidades acompañantes en proporción a su propio crecimiento evolutivo. El alma participa de las cualidades tanto de la mente humana como del espíritu divino, pero persistentemente evoluciona hacia el incremento de la dirección espiritual y de la preponderancia divina, facilitando una actividad de la mente que genere contenidos y busque coordinarse con el verdadero valor espiritual.
\vs p111 3:5 La andadura del mortal, la evolución del alma, no es tanto un período de prueba como educativo. La fe en la supervivencia de los valores supremos es el núcleo de la religión; la experiencia religiosa genuina consiste en la unión de los valores supremos y de los contenidos cósmicos, lo que resulta en una toma de conciencia de la realidad universal.
\vs p111 3:6 La mente conoce la cantidad, la realidad y los contenidos. Pero la cualidad ---los valores--- es algo que \bibemph{se siente}. Lo que se siente es creación mutua de la mente, que sabe, y del espíritu acompañante, que realiza la realidad.
\vs p111 3:7 En la medida en la que el alma morontial evolutiva del hombre se infunde de verdad, belleza y bondad, de tal conciencia de los valores de la conciencia de Dios, resulta un ser que se vuelve indestructible. Si no hay supervivencia de los valores eternos en el alma evolutiva del hombre, en ese caso, la existencia humana es un sin sentido y la propia vida es una trágica ilusión. Pero es para siempre verdad: lo que empezáis en el tiempo, de cierto lo terminaréis en la eternidad ---si es que merece la pena terminarlo---.
\usection{4. LA VIDA INTERIOR}
\vs p111 4:1 El reconocimiento es un proceso intelectual consistente en adaptar las impresiones sensoriales que se reciben del mundo exterior a los patrones memorísticos de la persona. Entender supone que esas impresiones sensoriales reconocidas y los patrones de memoria que lo acompañan se han integrado u organizado en una red dinámica de criterios.
\vs p111 4:2 Los contenidos dimanan de una combinación de reconocimiento y entendimiento. Los contenidos son inexistentes en un mundo enteramente material o sensorial. Los contenidos y los valores solo se perciben en las esferas interiores o supramateriales de la experiencia humana.
\vs p111 4:3 \pc Todo avance de la verdadera civilización nace en este mundo interior de la humanidad. Solo la vida interior es verdaderamente creativa. La civilización puede difícilmente progresar cuando la mayoría de los jóvenes de cualquier generación consagran sus intereses y energías a las actividades materialistas del mundo sensorial o exterior.
\vs p111 4:4 Los mundos interior y exterior poseen un conjunto diferente de valores. Cualquier civilización está en riesgo cuando las tres cuartas partes de su juventud se incorporan a profesiones materialistas y se entregan a la realización de las actividades sensoriales del mundo exterior. La civilización corre peligro cuando los jóvenes dejan de interesarse por la ética, la sociología, la eugenesia, la filosofía, las bellas artes, la religión y la cosmología.
\vs p111 4:5 Solo en los niveles más elevados de la mente supraconsciente, conforme esta incide en el mundo espiritual de la experiencia humana, podéis encontrar esos conceptos superiores en conjunción con unos válidos modelos matrices que contribuirán a la construcción de una civilización mejor y más perdurable. El ser personal es inherentemente creativo, pero únicamente obra así en la vida interior de la persona.
\vs p111 4:6 \pc Los cristales de nieve son siempre de forma hexagonal, pero no hay dos iguales. Los niños se corresponden con ciertos tipos de seres, pero no hay dos exactamente iguales, incluso en el caso de los gemelos. El ser personal se ajusta a unos tipos, pero es siempre único.
\vs p111 4:7 \pc La felicidad y el júbilo tienen su origen en la vida interior. No podéis experimentar un verdadero júbilo por vosotros mismos. Una vida solitaria es fatal para la felicidad. Incluso las familias y las naciones disfrutarían más de la vida si la compartiesen con otros.
\vs p111 4:8 \pc No podéis tener un completo dominio del mundo exterior ---del entorno---. Es la creatividad del mundo interior la que está más sujeta a vuestra dirección, porque allí vuestro ser personal está, en tan gran medida, liberado de las ataduras a las leyes de la causalidad antecedente. El ser personal lleva aparejado una limitada soberanía de la voluntad.
\vs p111 4:9 Dado que esta vida interior del hombre es verdaderamente creativa, recae en cada persona la responsabilidad de decidir si esta creatividad será espontánea y enteramente fortuita o regulada, dirigida y constructiva. ¿Cómo puede una imaginación creativa producir frutos meritorios si el escenario sobre el que obra está ya inmerso en prejuicios, odio, temores, resentimiento, venganza e intolerancias?
\vs p111 4:10 Las ideas pueden tener su origen en los estímulos del mundo exterior, pero los ideales nacen únicamente en los ámbitos creativos del mundo interior. Hoy en día, las naciones del mundo se dirigen por hombres que tienen sobreabundancia de ideas, pero que están afectados por carencias de ideales. Esto explica la pobreza, el divorcio, la guerra y los odios raciales.
\vs p111 4:11 Este es el problema: si el hombre, de libre voluntad, está dotado, en su ser interior, de poderes creativos, debemos reconocer, entonces, que el ejercicio libre de la creatividad encierra el potencial mismo de la destructividad voluntaria. Y cuando la creatividad se convierte en destructividad, os encontráis cara a cara con la devastación del mal y del pecado ---opresión, guerra y destrucción---. El mal es creatividad parcial, que tiende hacia su propia desintegración y posterior destrucción. Cualquier conflicto es malévolo en cuanto que inhibe la función creativa de la vida interior ---es una especie de guerra civil en la persona---.
\vs p111 4:12 \pc La creatividad interior contribuye al ennoblecimiento del carácter por medio de la integración del ser personal y la unificación del yo. Es para siempre verdad: el pasado es inmutable; solo se puede cambiar el futuro mediante la actuación de la creatividad del yo interior en el momento presente.
\usection{5. CONSAGRACIÓN DE LA ELECCIÓN}
\vs p111 5:1 Hacer la voluntad de Dios no es nada más ni nada menos que una muestra de la disponibilidad de la criatura para compartir su vida interior con Dios ---con el mismo Dios que ha posibilitado que esa vida creatural posea significado\hyp{}valor interior---. Compartir es semejarse a Dios ---es divino---. Dios lo comparte todo con el Hijo Eterno y el Espíritu Infinito, mientras ellos, a su vez comparten todas las cosas con los hijos divinos y las hijas espirituales de los universos.
\vs p111 5:2 La imitación de Dios es la clave de la perfección; hacer su voluntad es el secreto de la supervivencia y de la perfección en la supervivencia.
\vs p111 5:3 Los mortales viven en Dios, y es por lo que Dios ha querido vivir en los mortales. Al igual que los hombres se confían a él, de la misma manera ---y él primero--- ha confiado una parte de sí mismo para que esté con los hombres; ha elegido vivir en los hombres y hacerlos su morada, sujeto a la voluntad humana.
\vs p111 5:4 La paz en esta vida, la supervivencia en la muerte, la perfección en la próxima vida, el servicio en la eternidad ---todo esto se consigue \bibemph{ahora} (en espíritu) cuando el ser personal creatural consiente (elige) someter la voluntad de la criatura a la voluntad del Padre---. Y el Padre ya ha elegido hacer que una fracción de sí mismo se someta a la voluntad de este ser personal creatural.
\vs p111 5:5 Esta decisión de la criatura no significa entregar su voluntad. Constituye una consagración, expansión, glorificación y perfeccionamiento de la voluntad; y tal opción eleva la voluntad de la criatura desde el nivel de los contenidos temporales hasta ese estatus más elevado en el que la persona del hijo creatural está en comunión con la persona del Padre espíritu.
\vs p111 5:6 Esta elección de la voluntad del Padre es el descubrimiento espiritual del Padre espíritu por el hombre mortal, aunque deba trascurrir una era antes de que el hijo creatural pueda, realmente, estar ante la presencia real de Dios en el Paraíso. Dicha elección no consiste tanto en la negación de la voluntad de la criatura ---“Que no se haga mi voluntad, sino la tuya”--- como en la afirmación positiva de esta criatura: “es \bibemph{mi} voluntad que se haga \bibemph{tu} voluntad”. Y si se toma esta decisión, tarde o temprano, el hijo que optó por Dios encontrará una unión interior (fusión) con la fracción de Dios que mora en su interior, mientras que este mismo hijo, al ir perfeccionándose, hallará una suprema satisfacción personal en la comunión reverencial de la persona del hombre con la persona de su Hacedor, dos seres personales cuyos atributos creativos se unen eternamente en la mutua expresión de la voluntad creatural y la voluntad de Dios: el nacimiento de otra alianza eterna de la voluntad del hombre con la voluntad de Dios.
\usection{6. LA PARADOJA HUMANA}
\vs p111 6:1 Muchas de las dificultades temporales del hombre mortal nacen de su doble relación con el cosmos. El hombre es parte de la naturaleza ---existe en la naturaleza--- y, sin embargo, es capaz de trascenderla. El hombre es finito, pero dentro de él mora una chispa de la infinitud. Esta doble circunstancia no solo proporciona el potencial para el mal, sino que también engendra muchas situaciones sociales y morales plagadas de considerable incertidumbre y de no poca ansiedad.
\vs p111 6:2 La valentía necesaria para llevar a cabo la conquista de la naturaleza y trascenderse a sí mismo puede sucumbir ante las tentaciones de la soberbia. El mortal que puede trascenderse a sí mismo puede caer en la tentación de deificar su propia autoconciencia. Su dilema consiste en el doble hecho de que el hombre es cautivo de la naturaleza mientras que, al mismo tiempo, posee una libertad única: la libertad de elegir y actuar de forma espiritual. En los niveles materiales, el hombre se encuentra supeditado a la naturaleza, mientras que en los niveles espirituales triunfa sobre esta y sobre todas las cosas temporales y finitas. Esta paradoja es inseparable de la tentación, del mal potencial, de los errores decisorios, y, cuando el yo se convierte en soberbio y arrogante, se puede desarrollar el pecado.
\vs p111 6:3 \pc El problema del pecado no existe por sí mismo en el mundo finito. La finitud en sí no es mala ni pecaminosa. Un Creador infinito hizo el mundo finito ---por obra de sus hijos divinos--- y, por consiguiente, debe ser \bibemph{bueno}. Es el mal uso, la distorsión y la perversión de lo finito lo que da origen al mal y al pecado.
\vs p111 6:4 \pc El espíritu puede tener dominio sobre la mente; al igual que la mente puede manejar la energía. Pero la mente solo puede hacerlo por medio de su propia manipulación inteligente de los potenciales metamórficos, intrínsecos en el nivel matemático de las causas y efectos de los ámbitos físicos. La mente de la criatura carece inherentemente de potestad sobre la energía; esto es prerrogativa de la Deidad. Pero la mente de las criaturas puede actuar sobre ella, como de hecho lo hace, justo en la medida en la que adquiere competencia en los secretos energéticos del mundo físico.
\vs p111 6:5 Cuando el hombre desea modificar la realidad física, ya se trate de él mismo o de su entorno, lo logra hasta el punto en el que haya descubierto el camino y los medios de manejar la materia y de dirigir la energía. Sin asistencia, la mente es impotente para actuar sobre cualquier cosa material que no sea su propio mecanismo físico, con el que está inevitablemente vinculada. Pero mediante el uso inteligente del mecanismo corporal, el hombre puede crear otros mecanismos, incluso relaciones energéticas y relaciones vivas, mediante cuya utilización, la mente puede manipular el nivel físico del universo e incluso llegar a dominarlo.
\vs p111 6:6 La ciencia es la fuente de los hechos, y la mente no puede operar sin ellos. Estos constituyen los pilares sobre los que se edifica la sabiduría y se consolidan mediante las experiencias de la vida. El hombre puede encontrar el amor de Dios sin los hechos y puede descubrir las leyes de Dios sin el amor, pero el hombre jamás podrá comenzar a apreciar la simetría infinita, la armonía suprema, la excelente repleción de la naturaleza todo incluyente de la Primera Fuente y Centro hasta que no haya encontrado la ley divina y el amor divino, y los haya unificado experiencialmente en su propia filosofía cósmica en evolución.
\vs p111 6:7 La expansión del conocimiento material permite una mayor apreciación intelectual del significado de las ideas y de los valores de los ideales. Un ser humano puede encontrar la verdad en su experiencia interior, pero precisa un conocimiento claro de los hechos, a fin de aplicar su descubrimiento personal de la verdad a las inflexibles exigencias prácticas de la vida diaria.
\vs p111 6:8 \pc Es perfectamente natural que el hombre mortal esté hostigado por sentimientos de inseguridad cuando se ve inextricablemente ligado a la naturaleza, mientras posee atribuciones espirituales que transcienden por completo todas las cosas temporales y finitas. Solo la confianza religiosa ---la fe viva--- puede sostener al hombre en medio de estos problemas tan difíciles y desconcertantes.
\vs p111 6:9 \pc De todos los peligros que acosan la naturaleza mortal del hombre y hacen peligrar su integridad espiritual, el peor de ellos es la soberbia. La valentía es valiosa, pero el egotismo es jactancioso y suicida. La confianza razonable en uno mismo no es de lamentar. La habilidad que posee el hombre de transcenderse a sí mismo es lo que lo distingue del reino animal.
\vs p111 6:10 \pc La soberbia es engañosa, emponzoñadora y el ancestro del pecado, ya radique en una persona, un grupo, una raza o una nación. Es literalmente cierto: “Antes de la caída es la soberbia”.
\usection{7. EL PROBLEMA DEL MODELADOR}
\vs p111 7:1 La incertidumbre unida a la seguridad es la esencia de la aventura que se vive en la senda al Paraíso: incertidumbre en el tiempo y en la mente, incertidumbre en cuanto al despliegue de acontecimientos en el ascenso al Paraíso; seguridad en el espíritu y en la eternidad, seguridad en la confianza incondicionada del hijo creatural en la compasión divina y en el infinito amor del Padre Universal; incertidumbre como ciudadano no experimentado del universo; seguridad como hijo ascendente, en las moradas del universo, de un Padre todopoderoso, omnisapiente y omniamante.
\vs p111 7:2 \pc ¿Puedo aconsejaros que atendáis al eco distante de la llamada fiel que hace el modelador a vuestra alma? El modelador interior no puede detener ni incluso alterar materialmente vuestra esforzada andadura en el tiempo; el modelador no puede reducir las penurias de la vida en vuestro penoso viaje por este mundo. Este morador divino tan solo puede, con paciencia, mantenerse al margen mientras que vosotros libráis la batalla de la vida tal como se vive en vuestro planeta; pero podríais, si accedierais ---mientras trabajáis y os preocupáis, mientras lucháis y os afanáis---, dejar que el valiente modelador luche con vosotros y para vosotros. Os podríais, pues, sentir muy reconfortados e inspirados, muy cautivados y fascinados, si tan solo permitierais que el modelador os mostrara constantemente las imágenes del verdadero motivo, del objetivo final y del propósito eterno de toda esta difícil pugna cuesta arriba con los problemas ordinarios de vuestro presente mundo material.
\vs p111 7:3 ¿Por qué no ayudáis al modelador en su tarea de mostraros el equivalente espiritual de todos estos extenuantes esfuerzos materiales? ¿Por qué no permitís que el modelador os fortalezca con las verdades espirituales del poder cósmico mientras lucháis en vuestra existencia con las dificultades temporales propias de las criaturas? ¿Por qué no alentáis a este asistente celestial a que os reconforte con una clara visión y una perspectiva eterna de la vida universal, conforme vuestra mirada perpleja se tiende sobre los problemas de cada hora que pasa? ¿Por qué os negáis a sentiros iluminados e inspirados por el punto de vista del universo, mientras os esforzáis en medio de las dificultades del tiempo y os perdéis en el laberinto de incertidumbres que acosan vuestro viaje en la vida mortal? ¿Por qué no aceptáis que el modelador espiritualice vuestro pensamiento, aunque vuestros pies deban pisar las rutas materiales de vuestros empeños terrenales?
\vs p111 7:4 Las razas humanas mejor dotadas de Urantia son una mezcla compleja; son una combinación de numerosas razas y linajes de diferente origen. Esta naturaleza compuesta hace que les resulte extremadamente difícil a los mentores obrar con eficacia durante la vida y contribuye, sin duda, a los problemas del modelador y del serafín guardián tras la muerte. No hace mucho tiempo, estando en Lugar de Salvación, escuché a un guardián del destino formular una declaración formal como atenuante de las dificultades que conllevaba servir a su tutelado. Este serafín dijo:
\vs p111 7:5 \pc “Una gran parte de mis dificultades se debían al conflicto interminable entre las dos naturalezas de mi tutelado: el impulso de sus ambiciones en oposición a su indolencia animal; producto de los ideales de un pueblo superiormente dotado cruzado con los instintos de una raza pobremente dotada; los objetivos elevados de una gran mente enfrentados al impulso de una herencia primitiva; la visión de largo alcance de un mentor previsor contrarrestada por la miopía de una criatura del tiempo; los planes progresivos de un ser ascendente modificados por los deseos y anhelos de una naturaleza material; los destellos de una inteligencia del universo anulados por los mandatos energético\hyp{}químicos de una raza en evolución; el impulso de los ángeles en antagonismo con las emociones animales; la formación de un intelecto anulado por las tendencias del instinto; las experiencias individuales en oposición a las propensiones acumuladas de la raza; la meta de lo mejor relegada por la deriva de lo peor; el vuelo de la genialidad neutralizado por la gravedad de lo mediocre; el progreso de lo bueno retrasado por la inercia de lo malo; el arte de lo hermoso mancillado por la presencia del mal; el vigor de la salud neutralizado por la debilidad de la enfermedad; la fuente de la fe contaminada por los venenos del miedo; el manantial de la alegría afectado por las aguas de la tristeza; el gozo de la anticipación desilusionado por la amargura de la realidad sobrevenida; las alegrías de la vida siempre amenazadas por las tristezas de la muerte. ¡Qué vida y qué planeta! Y, sin embargo, gracias a la ayuda y al impulso siempre presentes del modelador del pensamiento, esta alma ciertamente alcanzó un buen grado de felicidad y de éxito, e incluso ahora ha ascendido a las salas de juicio de los mundos de las moradas”.
\vsetoff
\vs p111 7:6 [Exposición de un mensajero solitario de Orvontón.]
