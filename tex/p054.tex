\upaper{54}{Los problemas de la rebelión de Lucifer}
\author{Mensajero poderoso}
\vs p054 0:1 Al hombre evolutivo le resulta difícil comprender del todo el significado y el alcance del mal, el error, el pecado y la iniquidad. El hombre es lento en percibir que el contraste entre perfección e imperfección ocasiona el mal en potencia; que la oposición entre verdad y falsedad crea error y confusión; que el don divino de la libre voluntad resulta en los dos polos divergentes del pecado y la rectitud; que la búsqueda perseverante de la divinidad conduce al reino de Dios a diferencia de su continuo rechazo, que conduce a los dominios de la iniquidad.
\vs p054 0:2 Los Dioses no crean el mal y no permiten ni el pecado ni la rebelión. El mal en potencia existe en el tiempo, en un universo que acoge diferentes niveles de contenidos y de valores de la perfección. El pecado está en potencia en todos los ámbitos en los que los seres imperfectos gozan de la facultad de elegir entre el bien y el mal. La misma presencia contradictoria de la verdad y la mentira, de lo real y lo falso, constituye la potencialidad del error. La elección deliberada del mal constituye el pecado; el rechazo intencionado de la verdad es error; la obstinada búsqueda del pecado y del error es iniquidad.
\usection{1. LA VERDADERA LIBERTAD Y LA FALSA LIBERTAD}
\vs p054 1:1 De todos los desconcertantes problemas que se derivaron de la rebelión de Lucifer, ninguno ha causado tanta adversidad como la que surge de la deficiente capacidad de los mortales evolutivos inmaduros en diferenciar entre la verdadera y la falsa libertad.
\vs p054 1:2 La verdadera libertad es la búsqueda de los siglos y la recompensa del progreso evolutivo. La falsa libertad es el sutil engaño del error y el mal que el tiempo y el espacio respectivamente conllevan. La libertad perdurable se basa en la realidad de la justicia ---la inteligencia, la madurez, la fraternidad y la ecuanimidad---.
\vs p054 1:3 La libertad, cuando se mueve por motivos poco inteligentes y sin restricción ni control, destruye la propia existencia cósmica. La verdadera libertad se relaciona progresivamente con la realidad y está siempre atenta a la ecuanimidad social, a la justicia cósmica, a la fraternidad del universo y a las obligaciones divinas.
\vs p054 1:4 La libertad resulta suicida cuando se desconecta de la justicia material, de la ecuanimidad intelectual, de la indulgencia social, del deber moral y de los valores espirituales. La libertad es inexistente aparte de la realidad cósmica, y toda realidad personal es proporcional a sus relaciones con la divinidad.
\vs p054 1:5 La voluntad propia desenfrenada y la expresión descontrolada de uno mismo equivalen al egoísmo absoluto, a la suma impiedad. Si no viene acompañada de una creciente conquista del yo, la libertad es creación de una egocéntrica imaginación humana. La libertad inducida por nuestro propio yo es una ilusión conceptual, un atroz engaño. El libertinaje que se enmascara bajo la apariencia de libertad preludia una esclavitud deplorable.
\vs p054 1:6 LA VERDADERA LIBERTAD es compañera del genuino respeto de uno mismo; la falsa libertad es el adlátere de la admiración de sí mismo. La verdadera libertad es el fruto del autocontrol; la falsa libertad, la asunción de la autoafirmación. El autocontrol conduce al servicio altruista; la admiración propia tiende a la explotación de los demás para el engrandecimiento egoísta del ser errado, que está dispuesto a renunciar al fruto honesto de su esfuerzo en pro de un injusto poder sobre sus semejantes.
\vs p054 1:7 \pc Incluso la sabiduría es divina y honorable solamente cuando tiene dimensiones cósmicas y está espiritualmente motivada.
\vs p054 1:8 \pc No hay error más grande que esa especie de autoengaño que lleva a los seres inteligentes a ansiar ejercer su poder sobre otras personas como medio de privarlas de sus libertades naturales. La regla de oro de la ecuanimidad humana clama contra tal impostura, arbitrariedad, egoísmo e injusticia. Únicamente la verdadera y auténtica libertad es compatible con el reino del amor y con el ministerio de la misericordia.
\vs p054 1:9 ¡Cómo osa la pertinaz criatura vulnerar los derechos de sus semejantes en nombre de la libertad personal cuando los gobernantes supremos del universo se mantienen, con misericordioso respeto, al margen de estas prerrogativas de la voluntad y de estos potenciales del ser personal! Ningún ser tiene el derecho de privar a otros, en el ejercicio de su supuesta libertad personal, de esos privilegios de vida conferidos por los creadores y debidamente respetados por todos sus leales colaboradores, subordinados y ciudadanos regulares.
\vs p054 1:10 El hombre evolutivo puede que tenga que luchar por sus libertades materiales contra tiranos y opresores, en un mundo de pecado e iniquidad o en las épocas tempranas de una esfera primitiva en evolución, pero esto no es así en los mundos morontiales ni en las esferas espirituales. La guerra es la herencia del hombre evolutivo primitivo; si bien, hace tiempo que en esos mundos donde la civilización avanza de manera normal, se considera una infamia el enfrentamiento físico como método de solventar conflictos raciales.
\usection{2. EL ROBO DE LA LIBERTAD}
\vs p054 2:1 Junto con el Hijo y el Espíritu, Dios proyectó la eterna Havona y, desde ese momento, se estableció el modelo eterno de participación igualitaria en la creación ---el compartir---. Este modelo de compartición es el diseño magistral para cada uno de los hijos e hijas de Dios que salen al espacio con el empeño de reproducir, en el tiempo, el universo central de perfección eterna.
\vs p054 2:2 Cualquier criatura de cualquier universo en evolución que aspira a hacer la voluntad del Padre está destinada a convertirse en la acompañante de los creadores espacio\hyp{}temporales en esta magnífica aventura de lograr la perfección de manera experiencial. Si esto no fuera cierto, el Padre no habría dotado a estas criaturas de una libre voluntad creativa, ni tampoco habitaría en ellas ni llegaría realmente a hacerse su compañero por medio de su propio espíritu.
\vs p054 2:3 \pc La locura de Lucifer consistió en intentar hacer lo que no era factible: eludir el tiempo en un universo experiencial. El delito de Lucifer fue intentar privar a todos los seres personales de Satania de sus derechos creativos: el menoscabo subrepticio de la participación personal de las criaturas ---participación por propia voluntad--- en la larga lucha evolutiva por alcanzar la condición de luz y vida tanto de forma individual como colectiva. Al hacer esto, este antiguo soberano de vuestro sistema colocaba el propósito temporal de su propia voluntad directamente en oposición al propósito eterno de la voluntad de Dios, tal como se revela en el don de la libre voluntad otorgada a las criaturas personales. De este modo, la rebelión de Lucifer amenazaba con vulnerar, en todo lo posible, el poder de libre elección de los ascendentes y servidores del sistema de Satania, esto es, amenazaba con privar para siempre, a cada uno de estos seres, de la apasionante experiencia de contribuir con algo personal y único al monumento que lentamente se erige a la sabiduría experiencial, y que algún día existirá en la forma del sistema perfeccionado de Satania. Así pues, el manifiesto de Lucifer, disfrazado bajo los ropajes de la libertad, representa a la clara luz de la razón, una imponente amenaza que buscaba perpetrar el robo de la libertad personal y llevarlo a cabo a una escala evidenciada únicamente dos veces en toda la historia de Nebadón.
\vs p054 2:4 En resumen, Lucifer habría despojado a hombres y ángeles de lo que Dios les había dado, esto es, del privilegio divino de participar en la creación de sus propios destinos y del destino de este sistema local de mundos habitados.
\vs p054 2:5 \pc No hay en todo el universo ser alguno cuya libertad le dé legitimidad para privar a otros seres de la verdadera libertad, del derecho de amar y ser amado, del privilegio de adorar a Dios y de servir a sus semejantes.
\usection{3. LA DILACIÓN DE LA JUSTICIA}
\vs p054 3:1 A las criaturas morales y de voluntad de los mundos evolutivos siempre les ha preocupado irreflexivamente poder dar respuesta a la pregunta de por qué los omnisapientes creadores permiten el mal y el pecado. No alcanzan a comprender que ambos son inevitables si la criatura ha de ser realmente libre. La libertad de la voluntad del hombre en evolución o del magnífico ángel no es un mero concepto filosófico, un ideal simbólico. La facultad del hombre para optar por el bien o el mal es una realidad del universo. Esta libertad para elegir por sí mismos es un don de los gobernantes supremos, y no permitirán que ningún ser o grupo de seres despoje a ninguna persona del amplio universo de esta libertad divinamente otorgada ---incluso si estos seres errados e ignorantes gozan de esta mal llamada libertad personal---.
\vs p054 3:2 Aunque la identificación consciente e incondicionada con el mal (o pecado) equivale a la no existencia (reducción a la nada), entre el momento de dicha identificación personal con el pecado y la ejecución del castigo ---el resultado natural de tal deliberada acogida del mal--- siempre se ha de dejar transcurrir un período de tiempo lo suficientemente amplio como para permitir que la sentencia, en cuanto al estatus en el universo de dicho ser, resulte enteramente satisfactoria para todos los seres personales del universo implicados, y que sea tan ecuánime y justa como para que tenga la aprobación del pecador mismo.
\vs p054 3:3 Pero si este rebelde del universo, que se opone a la realidad de la verdad y de la bondad, se niega a aprobar el veredicto, y si el culpable conoce en su corazón la justicia de su condena pero rehúsa a hacer tal confesión, entonces la ejecución de la sentencia debe demorarse en conformidad con el criterio de los ancianos de días. Y los ancianos de días se niegan a la disolución de ningún ser hasta que todos los valores morales y todas las realidades espirituales no hayan dejado de existir tanto en el malhechor como en todos sus adeptos y en los posibles partidarios.
\usection{4. LA MISERICORDIOSA DILACIÓN}
\vs p054 4:1 Otro problema algo difícil de explicar en la constelación de Norlatiadec está relacionado con las razones que permitieron a Lucifer, Satanás y a los príncipes caídos obrar el mal durante tanto tiempo antes de ser detenidos, internados y juzgados.
\vs p054 4:2 Aquellos que son padres, que han tenido y criado hijos, están mejor capacitados para comprender por qué Miguel, un padre\hyp{}creador, puede ser lento en condenar y poner fin a sus propios hijos. La historia del hijo pródigo que Jesús narró ilustra bien cómo un padre amoroso es capaz de esperar durante largo tiempo el arrepentimiento de su errado hijo.
\vs p054 4:3 El mero hecho de que la malvada criatura pueda realmente escoger hacer el mal ---cometer el pecado--- establece el hecho en sí de la libre voluntad y justifica plenamente la tardanza en la ejecución de la justicia, siempre que la misericordia que se brinda pueda conducir al arrepentimiento y a la rehabilitación.
\vs p054 4:4 \pc Lucifer gozaba ya de la mayor parte de las libertades que buscaba y recibiría las demás en el futuro. Perdió todos estos preciosos dones por ceder a la impaciencia y rendirse al deseo de poseer lo que se ansía en el momento, y poseerlo haciendo caso omiso de cualquier obligación de respeto de los derechos y libertades de todos los otros integrantes del universo de los universos. Las obligaciones éticas son innatas, divinas y universales.
\vs p054 4:5 \pc Conocemos muchas razones por las que los gobernantes supremos no dieron fin o internaron de inmediato a los líderes de la rebelión de Lucifer. Pero no hay duda de que existen todavía otras razones, posiblemente mejores, que nos son desconocidas. Miguel de Nebadón personalmente demoró, en su misericordia, la ejecución de la justicia. De no haber sido por el afecto que sentía este padre\hyp{}creador hacia sus errados hijos, la justicia suprema del suprauniverso habría obrado. Si hubiese ocurrido en Nebadón un incidente similar al de la rebelión de Lucifer mientras Miguel estaba encarnado en Urantia, los instigadores de tal mal podrían haber sido ineludiblemente reducidos a la nada de forma instantánea.
\vs p054 4:6 Sin la contención de la misericordia divina, la justicia suprema puede obrar instantáneamente. Pero el ministerio de la misericordia que se dispensa a los hijos del tiempo y del espacio siempre facilita esta dilación, este intervalo salvador entre la siembra y la cosecha. Si la siembra es buena, tal intervalo proporciona la verificación y la edificación del carácter; si la siembra es mala, esta misericordiosa demora proporciona tiempo para el arrepentimiento y la rectificación. Este aplazamiento de la sentencia y de la terminación de la existencia de los malvados es connatural al ministerio misericordioso de los siete suprauniversos. Dicha contención de la justicia por la misericordia demuestra que Dios es amor, y que este Dios de amor gobierna los universos y rige con misericordia el destino y el juicio de todas sus criaturas.
\vs p054 4:7 Las demoras que se originan en la misericordia son por mandato de la voluntad libre de los creadores. De la paciencia en el tratamiento de los pecaminosos rebeldes, se puede obtener un bien para el universo. Aunque es del todo muy cierto que el bien no puede venir del mal para aquel que concibe y hace el mal, es igualmente cierto que todas las cosas (incluyendo el mal, potencial o manifiesto) cooperan para el bien de todos los seres que conocen a Dios, desean hacer su voluntad y ascienden al Paraíso de acuerdo con su plan eterno y su propósito divino.
\vs p054 4:8 Si bien, estas misericordiosas demoras no son interminables. A pesar del largo retraso existente (tal como se calcula el tiempo en Urantia) en la sentencia respecto a la rebelión de Lucifer, podemos dar crédito que durante el período en el que se efectuaba esta revelación, se celebró la primera vista en el caso pendiente en Uversa de Gabriel contra Lucifer y, poco después, se emitió el mandato de los ancianos de días ordenando que desde ese momento se confinara a Satanás en el mundo prisión junto con Lucifer. Esto ponía fin a la posibilidad de que Satanás realizara otras visitas a cualquiera de los mundos caídos de Satania. La justicia en un universo que se rige por la misericordia puede ser lenta, pero de cierto llega.
\usection{5. LA SABIDURÍA DE LA DEMORA}
\vs p054 5:1 De las muchas razones que me son conocidas por las que no se internó ni sentenció antes a Lucifer y a sus aliados, se me permite enumerar las siguientes:
\vs p054 5:2 \li{1.}La misericordia exige que todo infractor tenga tiempo suficiente para adoptar plenamente una actitud consciente en relación a sus malévolos pensamientos y actos pecaminosos.
\vs p054 5:3 \li{2.}La justicia suprema se rige por el amor de un Padre; por lo tanto, la justicia nunca pondrá fin a lo que la misericordia puede salvar. A todo malhechor se le otorga tiempo para aceptar la salvación.
\vs p054 5:4 \li{3.}Ningún padre cariñoso se precipita a imponer un castigo a algún miembro errado de su familia. La paciencia no puede obrar con independencia del tiempo.
\vs p054 5:5 \li{4.}Aunque la maleficencia es siempre perjudicial para una familia, la sabiduría y el amor aconsejan a los hijos honestos que tengan paciencia con el hermano errado durante el tiempo que su cariñoso padre le otorga, para que el pecador pueda dilucidar su equivocado camino y aceptar la salvación.
\vs p054 5:6 \li{5.}Con independencia de la actitud de Miguel hacia Lucifer, a pesar de ser el padre\hyp{}creador de Lucifer, no era competencia del hijo creador ejercer jurisdicción sumaria contra el apóstata soberano del sistema, porque en aquel momento aún no había completado su andadura de gracia por la que llegaría a alcanzar la soberanía incondicional sobre Nebadón.
\vs p054 5:7 \li{6.}Los ancianos de días podían haber reducido a la nada de inmediato a estos rebeldes, pero rara vez terminan con la existencia de los infractores sin una audiencia completa. En esta ocasión, declinaron desautorizar las decisiones tomadas por Miguel.
\vs p054 5:8 \li{7.}Es evidente que Emanuel aconsejó a Miguel que se mantuviese al margen de los rebeldes y permitiese que la rebelión prosiguiese su curso natural hasta que desapareciera por sí misma. Y la sabiduría del unión de días es el reflejo en el tiempo de la sabiduría unida de la Trinidad del Paraíso.
\vs p054 5:9 \li{8.}El fiel de días de Edentia aconsejó a los padres de la constelación que permitieran a los rebeldes seguir libremente su curso, a fin de que toda comprensión hacia estos malhechores se erradicase cuanto antes de los corazones de todos los ciudadanos, presentes y futuros, de Norlatiadec ---de toda criatura mortal, morontial o espiritual---.
\vs p054 5:10 \li{9.}En Jerusem, el representante personal del mandatario supremo de Orvontón aconsejó a Gabriel que impulsara cualquier posible oportunidad para que toda criatura viva pudiese madurar en conciencia una decisión respecto a aquellas cuestiones relacionadas con la Declaración Libertaria de Lucifer. Al haberse planteado el problema de la rebelión, el asesor en el Paraíso de Gabriel para casos de emergencia indicó que, si no se ofrecía plena e incondicionadamente esta posibilidad a todas las criaturas de Norlatiadec, entonces, en nombre de la autoprotección, se extendería la cuarentena determinada por el Paraíso a todas estas criaturas posiblemente tibias y dubitativas de toda la constelación. Para mantener abiertas las puertas de la ascensión al Paraíso a los seres de Norlatiadec, era necesario facilitar el desarrollo de la rebelión en su totalidad y garantizar que los seres, de alguna manera implicados en ella, pudieran determinar enteramente su actitud.
\vs p054 5:11 \li{10.}La benefactora divina de Lugar de Salvación promulgó un mandato, que constituía su tercera proclamación independiente, indicando que no se hiciera nada para paliar a medias, reprimir pusilánimamente u ocultar de alguna otra manera el horrible semblante de los rebeldes y de la rebelión. Se ordenó a las multitudes angélicas que dieran cabida plena y posibilidad sin límites a la manifestación del pecado, ya que sería el modo más rápido de lograr la absoluta y definitiva erradicación de la plaga del mal y del pecado.
\vs p054 5:12 \li{11.}Se organizó en Jerusem un consejo de emergencia de ex\hyp{}mortales integrado por mensajeros poderosos, esto es, mortales glorificados que habían tenido experiencia personal en situaciones similares, junto con sus compañeros. Asesoraron a Gabriel en el sentido de que si para reprimir la rebelión se aplicaban métodos de represión de la rebelión discrecionales o sumarios, al menos un número tres veces mayor de seres podrían descarriarse. Todo el colectivo de consejeros de Uversa acordó recomendar a Gabriel que permitiese que dicha rebelión siguiera por completo su curso natural, aunque fuese preciso un millón de años para poner fin a sus consecuencias.
\vs p054 5:13 \li{12.}El tiempo, incluso en un universo temporal, es relativo: si un mortal de Urantia con una vida de duración media cometiese un delito que provocara una conmoción a escala mundial y si se le detuviera, se le juzgara y se le sentenciara a la pena capital a los dos o tres días de perpetrar dicho delito, ¿os parecería a vosotros mucho tiempo? Y, sin embargo, estos dos o tres días serían la comparación más aproximada al tiempo restante de la vida de Lucifer, incluso si su juicio, ya iniciado, no se terminara en cien mil años de Urantia. Desde el punto de vista de Uversa, en donde el litigio está pendiente, este lapso de tiempo relativo significaría que el delito de Lucifer se sometió a juicio a los dos segundos y medio de haberse cometido. Desde el punto de vista del Paraíso, el dictado de la sentencia es simultáneo al acto delictivo.
\vs p054 5:14 \pc Hay un número equivalente de razones para no haber puesto fin sumariamente a la rebelión de Lucifer que os resultarían parcialmente comprensibles, pero que no se me permite exponer. Puedo informaros de que en Uversa se enseñan cuarenta y ocho razones que justifican que el mal siga su curso completo de su propia quiebra moral y extinción espiritual. No dudo de que exista otro igual número de razones desconocidas para mí.
\usection{6. EL TRIUNFO DEL AMOR}
\vs p054 6:1 Cualesquiera que sean las dificultades que los mortales evolutivos pudieran tener al intentar comprender la rebelión de Lucifer, debería quedar claro a todo pensador reflexivo que el modo de tratar a los rebeldes es una confirmación del amor divino. La misericordia amorosa que se ofrece a los rebeldes parece haber implicado a muchos seres inocentes en dificultades y aflicciones, pero todos estos apesadumbrados seres personales deben tener la seguridad de que los omnisapientes Jueces juzgarán sus destinos con misericordia al igual que con justicia.
\vs p054 6:2 En todas sus relaciones con los seres inteligentes, tanto el hijo creador como su Padre del Paraíso se gobiernan por el amor. Es imposible comprender muchas facetas de la actitud de los gobernantes del universo hacia los rebeldes y las rebeliones ---hacia el pecado y los pecadores---, a menos que se recuerde que Dios como Padre prevalece sobre todas las demás manifestaciones de la Deidad en todas sus relaciones divinas con la humanidad. También se debería recordar que la misericordia mueve a todos los hijos creadores del Paraíso.
\vs p054 6:3 \pc Si el padre afectuoso de una familia grande decide mostrar misericordia hacia uno de sus hijos culpable de graves maleficencias, es muy posible que el ofrecimiento de la misericordia que se hace a este hijo en su mal comportamiento pueda ocasionar adversidades temporales en todos los demás hijos que tienen un buen comportamiento. Esta posible circunstancia es inevitable; tal riesgo es inseparable del hecho en sí de tener un padre amoroso y de ser parte de una familia. Cada uno de los miembros de la familia se beneficia de la recta conducta de todos los demás miembros; asimismo, cada cual ha de sufrir las consecuencias temporales inmediatas de la mala conducta de cualquier otro miembro. Las familias al igual que los grupos, las naciones, las razas, los mundos, los sistemas, las constelaciones y los universos conllevan relaciones vinculantes que poseen individualidad y, por lo tanto, todo miembro de algún grupo, grande o pequeño, cosecha los beneficios y sufre las consecuencias del buen hacer y de la maleficencia de todos los otros miembros del grupo en cuestión.
\vs p054 6:4 Pero hay algo que debería quedar claro: si habéis de sufrir las funestas consecuencias del pecado de un miembro de vuestra familia, de algún conciudadano o semejante mortal, incluso de la rebelión en el sistema o en algún otro sitio ---sea lo que sea que tengáis que soportar debido a la maleficencia de vuestros colaboradores, semejantes o superiores--- podéis contar con la eterna certeza de que dichas tribulaciones son aflicciones pasajeras. Ninguna de estas consecuencias concomitantes con el mal comportamiento de algún integrante del grupo puede jamás poner en peligro vuestras expectativas eternas ni privaros, en lo más mínimo, de vuestro derecho divino a ascender al Paraíso y llegar a Dios.
\vs p054 6:5 Y hay una compensación para estas vicisitudes, demoras y decepciones que acompañan de forma invariable al pecado de la rebelión. Entre las muchas repercusiones importantes de la rebelión de Lucifer que se puedan reseñar, quiero prestar atención especial a la magnífica trayectoria de esos mortales ascendentes, ciudadanos de Jerusem que, por resistirse a los sofismas del pecado, se abrieron camino para convertirse en futuros mensajeros poderosos, en seres de mi propio orden. Todo ser que pudo soportar la prueba que representó este funesto episodio de inmediato avanzó en su estatus administrativo y acentuó sus méritos espirituales.
\vs p054 6:6 \pc En un principio, la sublevación de Lucifer pareció ser un absoluto desastre para el sistema y para el universo. Paulatinamente, empezaron a sucederse sus beneficios. Trascurridos veinticinco mil años de tiempo del sistema (veinte mil años de tiempo de Urantia), los melquisedecs comenzaron a impartir la enseñanza de que la bondad, como resultado de la locura de Lucifer, había comenzado a equipararse al mal causado. En su totalidad, el mal había llegado casi a estabilizarse en ese momento; solo continuaba creciendo en ciertos mundos aislados, mientras que los efectos beneficiosos continuaban multiplicándose y extendiéndose por todo el universo y el suprauniverso, incluso hasta Havona. Los melquisedecs imparten ahora la enseñanza de que el bien resultante de la rebelión de Satania es más de mil veces la suma de todo el mal.
\vs p054 6:7 Pero lograr tan extraordinaria y beneficiosa cosecha de la maleficencia solamente podría ocurrir gracias a la actitud sabia, divina y misericordiosa de todos los superiores de Lucifer, desde los padres de la constelación de Edentia hasta el Padre Universal del Paraíso. El paso del tiempo ha incrementado el bien conseguido a partir de la locura de Lucifer y, puesto que el mal punible se había desarrollado por completo en un relativamente corto período de tiempo, resultaba evidente que los gobernantes del universo, omnisapientes y previsores, claramente prolongarían el período de tiempo para cosechar resultados cada vez más beneficiosos. Pese a otras muchas razones que explican la demora en la detención y sentencia de los rebeldes de Satania, en sí mismo, este bien que se ha recibido hubiese sido suficiente para explicar por qué no se internó a estos pecadores antes, y por qué no se les ha sentenciado y dado fin.
\vs p054 6:8 Las mentes mortales, de poca visión de futuro y confinadas en el tiempo, no deberían criticar las demoras temporales de los previsores y omnisapientes administradores de los asuntos del universo.
\vs p054 6:9 Uno de los errores del pensamiento humano en lo que concierne a estos problemas consiste en la idea de que todos los mortales evolutivos de un planeta en evolución habrían optado por emprender la andadura hacia el Paraíso si su mundo no hubiese sido maldecido por el pecado. La facultad de rechazar la supervivencia no data de los tiempos de la rebelión de Lucifer. En cuanto a su andadura hacia el Paraíso, el hombre mortal siempre ha gozado de la facultad de la libre elección.
\vs p054 6:10 \pc A medida que ascendéis y experimentáis la supervivencia, ampliaréis vuestros conceptos sobre el universo y extenderéis vuestro horizonte en cuanto a contenidos y valores; y, seréis de este modo capaces de tener una mejor comprensión de por qué a seres como Lucifer y Satanás se les permite continuar en su rebeldía. También comprenderéis mejor de qué manera se puede obtener el bien último (si no inmediato) del mal limitado en el tiempo. Tras lograr el Paraíso, realmente estaréis iluminados y os sentiréis confortados cuando oigáis a los filósofos superáficos analizar y explicar estas profundas cuestiones que llevarán a ajustes en el universo. Pero incluso entonces, dudo de que en vuestra propia mente encontréis una plena satisfacción. Por lo menos, yo no la encontré ni siquiera cuando llegué a alcanzar así la cumbre de la filosofía del universo. No conseguí una completa comprensión de estas complejidades hasta después de que se me hubiese destinado a cometidos de orden administrativo en el suprauniverso, en donde, en base a la experiencia real, adquirí la competencia conceptual adecuada para comprender los múltiples aspectos de tales cuestiones, en cuanto a la equidad cósmica y a la filosofía espiritual. A medida que ascendéis en dirección al Paraíso, entenderéis cada vez más que muchos aspectos problemáticos de la administración del universo solamente se pueden comprender con posterioridad a vuestra adquisición de una mayor capacidad experiencial y al logro de una mejorada percepción espiritual. La sabiduría cósmica es fundamental para comprender las situaciones cósmicas.
\vsetoff
\vs p054 6:11 [Exposición de un mensajero poderoso, actualmente adscrito al gobierno del suprauniverso de Orvontón, que experimentó la supervivencia en la primera rebelión de un sistema de los universos del tiempo; interviene en esta materia a petición de Gabriel de Lugar de Salvación.]
