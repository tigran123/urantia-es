\upaper{105}{Deidad y realidad}
\author{Melquisedec}
\vs p105 0:1 Incluso para los elevados órdenes de inteligencias del universo, la infinitud solo es parcialmente comprensible y la completud de la realidad solo es relativamente inteligible. La mente humana, cuando trata de penetrar en la eternidad\hyp{}misterio del origen y el destino de todo lo que se llama \bibemph{real,} puede abordar convenientemente el problema concibiendo la eternidad\hyp{}infinitud como una elipse casi ilimitada, que se genera por una única causa absoluta, y que actúa por todo este círculo universal de diversificación sin fin, buscando siempre algún potencial de destino absoluto e infinito.
\vs p105 0:2 Cuando el intelecto humano intenta comprender el concepto de la totalidad de la realidad, dicha mente finita se enfrenta con la infinitud\hyp{}realidad; la totalidad de la realidad es infinitud y, por lo tanto, una mente cuya capacidad nocional es subinfinita jamás puede llegar a captarla por completo.
\vs p105 0:3 La mente humana difícilmente puede formarse una idea adecuada de las existencias eternas y, sin tal comprensión, nos resulta imposible incluso exponer nuestros conceptos de la totalidad de la realidad. No obstante, trataremos de hacer esta exposición siendo completamente conscientes de que tales conceptos, en el proceso de traducirlos y modificarlos para adaptarlos al nivel de comprensión de la mente mortal, deben someterse a una profunda alteración.
\usection{1. EL CONCEPTO FILOSÓFICO DEL YO SOY}
\vs p105 1:1 Los filósofos de los universos atribuyen la causalidad primordial absoluta de la infinitud al Padre Universal, que obra como el YO SOY infinito, eterno y absoluto.
\vs p105 1:2 Exponer esta idea de un infinito YO SOY al intelecto humano entraña muchos elementos problemáticos, puesto que se trata de un concepto tan alejado de su entendimiento experiencial que ocasiona serias alteraciones de los contenidos e interpretaciones erróneas de los valores. No obstante, el concepto filosófico del YO SOY aporta ciertamente a los seres finitos una base con la que intentar acercarse a una parcial comprensión de los orígenes absolutos y de los destinos infinitos. Pero, en todos nuestros intentos por dilucidar la génesis y la consecución de la realidad, es conveniente dejar claro que este concepto del YO SOY, en todos sus contenidos y valores de lo personal, es sinónimo de la Primera Persona de la Deidad, del Padre Universal de todos los seres personales. Si bien, el postulado del YO SOY no es identificable con tanta claridad en los ámbitos no deificados de la realidad universal.
\vs p105 1:3 \pc \bibemph{El YO SOY es el Infinito; el YO SOY es también la infinitud}. Desde un punto de vista secuencial, temporal, toda realidad tiene su origen en el infinito YO SOY, cuya solitaria existencia en la eternidad infinita del pasado debe ser el principal postulado filosófico de la criatura finita. El concepto del YO SOY connota \bibemph{infinitud incondicionada,} la realidad no diferenciada de todo aquello que podría llegar a ser alguna vez en toda la eternidad infinita.
\vs p105 1:4 Como concepto existencial, el YO SOY no es ni deificado ni no deificado ni actual ni potencial ni personal ni impersonal ni estático ni dinámico. Ningún calificativo es aplicable al Infinito salvo afirmar que el YO SOY es. El postulado filosófico del YO SOY es un concepto universal algo más difícil de entender que el del Absoluto Indeterminado.
\vs p105 1:5 Para la mente finita ha de haber sencillamente un principio y, aunque la realidad nunca lo ha tenido realmente, existen, no obstante, ciertas relaciones originarias que la realidad manifiesta en la infinitud. La situación eterna, primordial, previa a la realidad, podría pensarse de esta manera: en algún momento infinitamente distante, hipotético, de la eternidad pasada, el YO SOY se puede concebir como algo y no algo, como causa y efecto, como volición y respuesta. En dicho momento hipotético de la eternidad, no hay diferenciación en toda la infinitud. La infinitud está ocupada por el Infinito; el Infinito abarca la infinitud. Este es el momento hipotético y estático de la eternidad; los actuales están aún contenidos en sus potenciales, y los potenciales no han aparecido aún en la infinitud del YO SOY. Pero incluso en esta hipotética situación debemos asumir que existe la posibilidad de una voluntad propia.
\vs p105 1:6 \pc Recordad siempre que, para el hombre, comprender al Padre Universal constituye una experiencia personal. Dios, como vuestro Padre espiritual, es comprensible para vosotros y para todos los demás mortales; pero \bibemph{vuestro concepto experiencial y devocional del Padre Universal ha de ser siempre menos que vuestro postulado filosófico de la infinitud de la Primera Fuente y Centro, del YO SOY}. Cuando hablamos del Padre, queremos decir Dios tal como las criaturas tanto de mayor como de menor rango lo entienden, pero hay mucho más de la Deidad que no es comprensible para las criaturas del universo. Dios, vuestro Padre y mi Padre, es esa faceta del Infinito que percibimos en nuestro ser personal como una realidad experiencial real, pero el YO SOY sigue siempre siendo nuestra hipótesis de todo lo que nosotros creemos que es incognoscible de la Primera Fuente y Centro. E incluso tal hipótesis dista probablemente mucho de abarcar la insondable infinitud de la realidad primigenia.
\vs p105 1:7 El universo de los universos, con la innumerable multitud de seres personales que lo habitan, es un organismo inmenso y complejo, pero la Primera Fuente y Centro es infinitamente más compleja que los universos y seres personales que se han hecho reales en respuesta a sus mandatos volitivos. Cuando os maravilláis de la magnitud del universo matriz, considerad, por un momento, que incluso esta inconcebible creación no puede ser más que una revelación parcial del Infinito.
\vs p105 1:8 De hecho, la infinitud remotamente puede comprenderse partiendo del nivel experiencial de los mortales, pero incluso en esta era, en Urantia, vuestros conceptos de la infinitud están creciendo, y continuarán haciéndolo durante vuestras andaduras sin fin, conforme avanzan y se extienden hacia la eternidad futura. La infinitud incondicionada carece de sentido para la criatura finita, pero la infinitud es susceptible de limitarse a sí misma y de expresar la realidad en todos los niveles existenciales del universo. Y el rostro que el Infinito muestra a todos los seres personales del universo es el rostro de un Padre, del Padre Universal del amor.
\usection{2. EL YO SOY COMO TRINO Y SÉPTUPLO}
\vs p105 2:1 Al considerar la génesis de la realidad, tened siempre en cuenta que toda realidad absoluta proviene de la eternidad y que su existencia no tiene principio. Cuando hacemos alusión a la realidad absoluta, nos referimos a las tres personas existenciales de la Deidad, a la Isla del Paraíso y a los tres Absolutos. Estas siete realidades son igualitariamente eternas, a pesar de que, al exponer sus orígenes secuenciales a los seres humanos, recurramos al lenguaje espacio\hyp{}temporal.
\vs p105 2:2 \pc Siguiendo la descripción cronológica de los orígenes de la realidad, debe haber un supuesto momento teórico en el que haya una “primera” expresión volitiva y una “primera” reacción resultante en el YO SOY. En nuestros intentos por representar la génesis y la generación de la realidad, se puede concebir esta etapa como aquella en la que \bibemph{El Uno Infinito} se diferencia a sí mismo de \bibemph{La Infinidad,} pero el postulado de esta doble relación debe siempre ampliarse para incluir una concepción trina mediante el reconocimiento del continuo eterno de \bibemph{La Infinitud:} el YO SOY.
\vs p105 2:3 Esta autometamorfosis del YO SOY culmina en la diferenciación múltiple de la realidad deificada y de la realidad no deificada, de la realidad potencial y de la actual y de ciertas otras realidades que difícilmente se pueden clasificar de esta manera. Estas diferenciaciones del YO SOY teóricamente monista se integran eternamente por medio de relaciones simultáneas que emanan del mismo YO SOY ---la prerrealidad prepotencial, preactual, prepersonal, monotética que, aunque infinita, se revela como absoluta en presencia de la Primera Fuente y Centro y como ser personal en el amor ilimitado del Padre Universal---.
\vs p105 2:4 A través de estas metamorfosis internas, el YO SOY está estableciendo la base para una relación séptupla en sí mismo. El concepto filosófico (en el tiempo) del YO SOY solitario y el concepto transitorio (en el tiempo) del YO SOY como trino se pueden ampliar ahora hasta englobar el YO SOY como séptuplo. Esta naturaleza séptupla ---o de siete facetas--- se puede presentar mejor en relación con los Siete Absolutos de la Infinitud:
\vs p105 2:5 \li{1.}\bibemph{El Padre Universal}. El YO SOY padre del Hijo Eterno. Es la relación primordial personal de lo manifestado. El ser personal absoluto del Hijo convierte en absoluto el hecho de la paternidad de Dios y determina la filiación potencial de todos los seres personales. Esta relación establece el ser personal del Infinito y consuma su revelación espiritual en el ser personal del Hijo Primigenio. Incluso los mortales, aunque estén todavía en la carne, al adorar a nuestro Padre, pueden, a niveles espirituales, experimentar parcialmente esta faceta del YO SOY.
\vs p105 2:6 \li{2.}\bibemph{El Rector Universal}. El YO SOY causa del Paraíso eterno. Se trata de la relación primordial impersonal de lo manifestado, la vinculación primigenia no espiritual. El Padre Universal es Dios como amor; el Rector Universal es Dios como modelo. Esta relación establece el potencial de la forma ---o configuración--- y determina el modelo matriz de las relaciones impersonales y no espirituales ---el modelo matriz del que se hacen todas las copias---\ldots
\vs p105 2:7 \li{3.}\bibemph{El Creador Universal}. El YO SOY uno con el Hijo Eterno. Esta unión del Padre y del Hijo (en presencia del Paraíso) inicia el ciclo creador, que consuma en la aparición del ser personal conjunto y del universo eterno. Desde la perspectiva de un mortal finito, la realidad tiene sus verdaderos comienzos con la aparición, en la eternidad, de la creación de Havona. Este acto creador de la Deidad se produce por medio del Dios de Acción, que es en esencia la unidad del Padre\hyp{}Hijo manifestada en todos y para todos los niveles de lo actual. Así pues, la creatividad divina se caracteriza indefectiblemente por la unidad, y esta unidad es el reflejo externo de la unicidad absoluta de la dualidad de Padre\hyp{}Hijo y de la Trinidad de Padre\hyp{}Hijo\hyp{}Espíritu.
\vs p105 2:8 \li{4.}\bibemph{El Sostenedor Infinito}. El YO SOY autorelacional. Se trata de la agrupación primordial de las facetas estáticas y potenciales de la realidad. En esta relación, se compensa todo lo condicionado y lo incondicionado. Esta faceta del YO SOY se comprende mejor como el Absoluto Universal ---que unifica el Absoluto de la Deidad y el Absoluto Indeterminado---.
\vs p105 2:9 \li{5.}\bibemph{El Potencial Infinito}. El YO SOY autocondicionado. Es el parámetro de referencia infinito, que da testimonio eterno de la propia limitación volitiva del YO SOY en virtud de la cual alcanzó la expresión y la revelación triple de sí mismo. Esta faceta del YO SOY se entiende normalmente como el Absoluto de la Deidad.
\vs p105 2:10 \li{6.}La \bibemph{Capacidad Infinita}. El YO SOY estático\hyp{}reactivo. Esta es la matriz ilimitada, la posibilidad de cualquier expansión cósmica futura. Tal vez se entienda mejor esta faceta del YO SOY si se concibe como la presencia supragravitatoria del Absoluto Indeterminado.
\vs p105 2:11 \li{7.}\bibemph{El Uno Universal de la Infinitud}. El YO SOY como YO SOY. Es el estatismo o autorrealización de la Infinitud, el hecho eterno de la infinitud\hyp{}realidad y de la verdad universal de la realidad\hyp{}infinitud. En la medida en la que esta relación sea perceptible como ser personal, se revela a los universos en el Padre divino de todo ser personal ---incluso del ser personal absoluto---. Conforme esta relación se exprese de forma impersonal, el universo toma contacto con ella como la coherencia absoluta de la energía pura y del espíritu puro en la presencia del Padre Universal. Conforme esta relación se conciba como un absoluto, se revela en la primacía de la Primera Fuente y Centro; en ella todos vivimos y nos movemos y tenemos nuestro ser, desde las criaturas del espacio hasta los ciudadanos del Paraíso; y esto es tan cierto para el universo matriz como para el ultimatón infinitesimal, tan cierto para lo que va a ser como para lo que es y lo que ha sido.
\usection{3. LOS SIETE ABSOLUTOS DE LA INFINITUD}
\vs p105 3:1 Las siete relaciones primordiales existentes en el YO SOY se eternizan como los siete Absolutos de la Infinitud. Pero aunque podamos narrar los orígenes de la realidad y la diferenciación de la infinitud secuencialmente, de hecho los siete absolutos son eternos de forma incondicionada e igualitaria. Tal vez sea necesario que la mente mortal conciba sus inicios, pero este concepto debería eclipsarse con la comprensión de que los siete Absolutos no tuvieron comienzo; son eternos y así lo han sido siempre. Los siete Absolutos son la premisa de la realidad. En estos escritos se han descrito de la siguiente manera:
\vs p105 3:2 \li{1.}\bibemph{La Primera Fuente y Centro}. La Primera Persona de la Deidad y el modelo primordial en cuanto no deidad, Dios, el Padre Universal, creador, rector y sustentador; amor universal, espíritu eterno y energía infinita; potencial de \bibemph{todos} los potenciales y fuente de todos los actuales; la estabilidad de todo lo estático y el dinamismo de todo cambio; fuente del modelo y Padre de las personas. Colectivamente, los siete Absolutos equivalen a la infinitud, pero el mismo Padre Universal es realmente infinito.
\vs p105 3:3 \li{2.}\bibemph{La Segunda Fuente y Centro}. La Segunda Persona de la Deidad, el Hijo Eterno y Primigenio; las realidades personales absolutas del YO SOY y la base para la realización\hyp{}revelación del “YO SOY ser personal”. Ninguna persona puede albergar la esperanza de llegar al Padre Universal salvo a través de su Hijo Eterno; ninguna persona puede tampoco alcanzar niveles de existencia espirituales, aparte de la acción y ayuda de este modelo absoluto de todos los seres personales. En la Segunda Fuente y Centro, el espíritu es incondicionado mientras que el ser personal es absoluto.
\vs p105 3:4 \li{3.}\bibemph{La Fuente y Centro Paradisíaca}. El segundo modelo en cuanto no deidad, la eterna Isla del Paraíso; la base para la realización\hyp{}revelación del “YO SOY fuerza” y el pilar que establece el control gravitatorio en todos los universos. Con respecto a toda la realidad actualizada, no espiritual, impersonal y no volitiva, el Paraíso es el absoluto de los modelos. Al igual que la energía espiritual se relaciona con el Padre Universal a través del ser personal absoluto de la Madre\hyp{}Hijo, asimismo toda la energía cósmica está sujeta al control gravitatorio de la primera Fuente y Centro mediante el modelo absoluto de la Isla del Paraíso. El Paraíso no está en el espacio; el espacio existe en relación al Paraíso, y la cronicidad del movimiento se determina a través de su relación con el Paraíso. La Isla Eterna está absolutamente en reposo; toda otra energía organizada y en proceso de organización está en eterno movimiento; en todo el espacio, solo la presencia del Absoluto Indeterminado es quiescente y el Indeterminado es coincidente con el Paraíso. El Paraíso existe en el punto central del espacio, el Indeterminado lo infunde y toda existencia relativa tiene su ser en este ámbito.
\vs p105 3:5 \li{4.}\bibemph{La Tercera Fuente y Centro}. La Tercera Persona de la Deidad, el Actor Conjunto; el integrador infinito de las energías cósmicas del Paraíso con las energías espirituales del Hijo Eterno; el coordinador perfecto de las motivaciones de la voluntad y de la mecánica de la fuerza; el unificador de toda realidad actual y en proceso de actualización. Por medio del ministerio de sus múltiples hijos, el Espíritu Infinito revela la misericordia del Hijo Eterno mientras que obra al mismo tiempo como actuante infinito, entrelazando por siempre el modelo del Paraíso en las energías del espacio. Este mismo Actor Conjunto, este Dios de Acción, es la expresión perfecta de los ilimitados planes y propósitos del Padre\hyp{}Hijo, mientras que él mismo actúa como fuente de la mente y donador del intelecto a las criaturas del inmenso cosmos.
\vs p105 3:6 \li{5.}\bibemph{El Absoluto de la Deidad}. Las posibilidades causativas y potencialmente personales de la realidad universal, la totalidad de todo el potencial de la Deidad. El Absoluto de la Deidad es el delimitador intencional de las realidades incondicionadas, absolutas y en cuanto no deidad. El Absoluto de la Deidad es el que determina el absoluto y absolutiza lo condicionado ---el que origina el destino---.
\vs p105 3:7 \li{6.}\bibemph{El Absoluto Indeterminado}. Estático, reactivo y latente; la infinitud cósmica no revelada del YO SOY; la totalidad de la realidad en cuanto no deidad y la completud de todo el potencial no personal. El espacio limita la acción del Indeterminado, pero su presencia no tiene límites, es infinita. En el universo matriz existe una noción de periferia, pero la presencia del Indeterminado es ilimitada; ni siquiera la eternidad puede agotar la inconmensurable quiescencia de este Absoluto en cuanto no deidad.
\vs p105 3:8 \li{7.}\bibemph{El Absoluto Universal}. Unifica lo deificado y lo no deificado; correlaciona lo absoluto y lo relativo. El Absoluto Universal (siendo estático, potencial y vinculado) compensa la tensión entre lo siempre existente y lo incompleto.
\vs p105 3:9 \pc Los Siete Absolutos de la Infinitud constituyen los orígenes de la realidad. De la manera en la que la mente mortal lo contemplaría, la Primera Fuente y Centro parecería ser anterior a todos los absolutos. Pero tal postulado, aunque de utilidad, queda invalidado por la coexistencia en la eternidad del Hijo, el Espíritu, los tres Absolutos y la Isla del Paraíso.
\vs p105 3:10 Es una \bibemph{verdad} que los Absolutos son manifestaciones del YO SOY\hyp{}Primera Fuente y Centro; es un \bibemph{hecho} que estos Absolutos nunca tuvieron un principio, sino que son coeternos con la Primera Fuente y Centro. En el lenguaje temporal y en los modelos conceptuales del espacio, no siempre se pueden exponer las relaciones de los absolutos en la eternidad sin que se produzcan paradojas. Pero con independencia de cualquier confusión respecto al origen de los Siete Absolutos de la Infinitud, es un hecho, al igual que una verdad, que toda realidad se basa en su existencia eterna y en sus relaciones infinitas.
\usection{4. UNIDAD, DUALIDAD Y TRIUNIDAD}
\vs p105 4:1 Los filósofos del universo postulan la existencia en la eternidad del YO SOY como fuente primordial de toda la realidad. Y, conjuntamente con esto, postulan la propia segmentación del YO SOY en realizaciones primarias de sí mismo: las siete facetas de la infinitud. Y simultáneamente con este supuesto está el tercer postulado: la aparición en la eternidad de los Siete Absolutos de la Infinitud y la eternización de la relación en dualidad de las siete facetas del YO SOY y de estos siete Absolutos.
\vs p105 4:2 La autorrevelación del YO SOY se extiende, pues, desde el yo estático, a través de su autosegmentación y autorrelación, hasta las relaciones absolutas, esto es, relaciones con los Absolutos autoderivados. La dualidad llega, de este modo, a tener su existencia en la agrupación eterna de los Siete Absolutos de la Infinitud con la infinitud séptupla de las facetas autosegmentadas de la autorrevelación del YO SOY. Estas relaciones dobles, eternizadas en los universos como los siete Absolutos, eternizan a su vez los pilares básicos de toda la realidad universal.
\vs p105 4:3 Se ha señalado alguna vez que la unidad engendra la dualidad, que la dualidad engendra la triunidad y que la triunidad es el ancestro eterno de todas las cosas. Existen, de hecho, tres grandes clases de relaciones primordiales, que son las siguientes:
\vs p105 4:4 \li{1.}\bibemph{Relaciones en la unidad}. Relaciones existentes en el YO SOY, al concebirse esta unidad propia como diferenciación triple y, luego, séptupla de sí mismo.
\vs p105 4:5 \li{2.}\bibemph{Relaciones en dualidad}. Las relaciones existentes entre el YO SOY como séptuplo y los Siete Absolutos de la Infinitud.
\vs p105 4:6 \li{3.}\bibemph{Relaciones en triunidad}. Se trata de agrupaciones de carácter operativo de los Siete Absolutos de la Infinitud.
\vs p105 4:7 \pc Las relaciones en triunidad surgen de fundamentos dualistas debido a la inevitabilidad de la correlación entre los Absolutos. Tales agrupaciones trinas eternizan el potencial de toda la realidad; engloban tanto la realidad deificada como la no deificada.
\vs p105 4:8 El YO SOY es infinitud incondicionada como \bibemph{unidad}. Las dualidades perpetúan para siempre los \bibemph{fundamentos} de la realidad. Las triunidades devienen de la realización de la infinitud como \bibemph{función} universal.
\vs p105 4:9 Los preexistenciales se convierten en existenciales en los siete Absolutos, y los existenciales se convierten en funcionales en las triunidades, la agrupación esencial de los Absolutos. Y, conjuntamente con la eternización de las triunidades, el escenario universal está dispuesto ---los potenciales existen y los actuales están presentes--- y la plenitud de la eternidad da testimonio de la diversificación de la energía cósmica, de la expansión del espíritu del Paraíso y de la dotación de la mente junto con la dádiva del ser personal, por virtud de los cuales todos estos derivados de la Deidad y del Paraíso se unifican experiencialmente en el nivel creatural y, por otros métodos, en el nivel de las supracriaturas.
\usection{5. PROMULGACIÓN DE LA REALIDAD FINITA}
\vs p105 5:1 Al igual que la diversificación primigenia del YO SOY debe atribuirse a su voluntad inherente y autónoma, del mismo modo, el inicio de la promulgación de la realidad finita debe adscribirse a los actos volitivos de la Deidad del Paraíso y a las modificaciones consecuentes a las triunidades en su carácter operativo.
\vs p105 5:2 Antes de la deidización de lo finito, parecería que toda la diversificación de la realidad había tenido lugar en los niveles absolutos; pero la acción volitiva que promulga la realidad finita supone delimitar la absolutidad e implica la aparición de relatividades.
\vs p105 5:3 \pc Aunque exponemos esta narración de manera secuencial y describimos la aparición histórica de lo finito como un derivado directo de lo absoluto, debe tenerse en cuenta que los trascendentales precedieron a la vez que siguieron a todo lo que es finito. Los trascendentales últimos son, en relación con lo finito, tanto causales como culminativos.
\vs p105 5:4 \pc La posibilidad finita es inherente al Infinito, pero la transmutación de la posibilidad a la probabilidad y a la inevitabilidad debe atribuirse a la voluntad libre, existente por sí misma, de la Primera Fuente y Centro, que activa todas las relaciones de la triunidad. Solo la infinitud de la voluntad del Padre podría haber delimitado el nivel absoluto de la existencia para devenir la realidad última o crear la realidad finita.
\vs p105 5:5 Con la aparición de la realidad relativa y delimitada comienza a existir un nuevo ciclo de la realidad ---el ciclo del crecimiento--- un majestuoso movimiento de descenso desde las alturas de la infinitud hacia el ámbito de lo finito, que se balancea perpetuamente hacia dentro, hacia el Paraíso y la Deidad, siempre buscando esos altos destinos conmesurables con su fuente infinita.
\vs p105 5:6 Estas impensables interacciones señalan el comienzo de la historia del universo, señalan la existencia del tiempo mismo. Para las criaturas, el principio de lo finito \bibemph{es} la génesis de la realidad; desde la visión de la mente creatural, no hay actualidad concebible con anterioridad a lo finito. Esta realidad finita, recién aparecida, ocurre en dos facetas primigenias:
\vs p105 5:7 \li{1.}\bibemph{Los máximos primarios,} la realidad supremamente perfecta, el tipo de universo y criaturas de Havona.
\vs p105 5:8 \li{2.}\bibemph{Los máximos secundarios,} la realidad supremamente perfeccionada, el tipo de criaturas y creación de los suprauniversos.
\vs p105 5:9 \pc Estas son, pues, las dos manifestaciones primigenias: la constitutivamente perfecta y la evolutivamente perfeccionada. Las dos son coiguales en sus relaciones con la eternidad, pero dentro de los límites del tiempo son supuestamente diferentes. El factor tiempo implica crecimiento para aquello que crece; los finitos secundarios crecen; de ahí que aquellos que están creciendo deben aparecer como incompletos en el tiempo. Pero estas diferencias, que tan importantes son en este lado del Paraíso, son inexistentes en la eternidad.
\vs p105 5:10 Hablamos de lo perfecto y de lo perfeccionado como máximos primarios y secundarios, pero aún existe otra clase: se producen relaciones trinitizadas y de otro tipo entre los primarios y los secundarios, que dan lugar a la aparición de \bibemph{máximos terciarios} ---cosas, contenidos y valores que no son perfectos ni perfeccionados, aunque son correlatos de los dos factores ancestrales---.
\usection{6. REPERCUSIONES DE LA REALIDAD FINITA}
\vs p105 6:1 La completa promulgación de las existencias finitas representa una transferencia de los potenciales a los actuales en el seno de las relaciones absolutas de la infinitud operativa. De entre las múltiples consecuencias de la actualización creativa de lo finito, se pueden citar las siguientes:
\vs p105 6:2 \li{1.}\bibemph{La respuesta en cuanto a la deidad,} la aparición de los tres niveles de la supremacía experiencial: la realidad de la supremacía de la persona\hyp{}espíritu en Havona, el potencial para la supremacía de lo personal y la potencia en el venidero gran universo y la capacidad para alguna actuación desconocida de la mente experiencial en algún nivel de supremacía del futuro universo matriz.
\vs p105 6:3 \li{2.}\bibemph{La respuesta en cuanto al universo} conllevó la activación de los planes arquitectónicos para el nivel espacial del suprauniverso, y esta evolución sigue todavía adelante en toda la organización física de los siete suprauniversos.
\vs p105 6:4 \li{3.}\bibemph{La repercusión en cuanto a las criaturas} de la promulgación de la realidad finita resultó en la aparición de seres perfectos del orden de los habitantes eternos de Havona y de los ascendentes evolutivos perfeccionados de los siete suprauniversos. Pero alcanzar la perfección como experiencia evolutiva (en el tiempo y de manera creativa) suponía algo distinto a la perfección como punto de partida. Como consecuencia, surge la imperfección en las creaciones evolutivas. Y este es el origen del mal potencial. La falta de adaptación, la desarmonía y el conflicto, circunstancias en su totalidad intrínsecas al crecimiento evolutivo que ocurre en los universos físicos y en las criaturas personales.
\vs p105 6:5 \li{4.}\bibemph{La respuesta en cuanto divinidad} a la imperfección intrínseca en el lapso de tiempo de la evolución se desvela en la presencia compensadora del Dios Séptuplo, por cuya acción lo que está perfeccionándose se integra con lo perfecto y con lo perfeccionado. Este intervalo temporal es inseparable de la evolución, que es creatividad en el tiempo. Debido a esto, al igual que a otras razones, la potencia todopoderosa del Supremo se fundamenta en los logros divinos del Dios Séptuplo. Este lapso de tiempo hace posible la participación de la criatura en la creación divina, permitiendo que los seres personales creaturales se conviertan en compañeros de la Deidad en la consecución de su máximo desarrollo. Incluso la mente material de la criatura mortal se vuelve, de este modo, compañera del modelador divino en la creación, de parte de ambos, del alma inmortal. El Dios Séptuplo también aporta formas de compensar las limitaciones experienciales de la perfección inherente al igual que las limitaciones previas al ascenso desde la imperfección.
\usection{7. EL DEVENIR DE LOS TRASCENDENTALES}
\vs p105 7:1 Los trascendentales son subinfinitos y subabsolutos, pero suprafinitos y supracreaturales. Los trascendentales devienen en un nivel que integra y correlaciona los supravalores de los absolutos con los valores máximos de los finitos. Desde la perspectiva creatural, aquello que es trascendental parece haber devenido como consecuencia de lo finito; desde la perspectiva de la eternidad, como anticipación de lo finito; y hay quienes lo consideran como “pre\hyp{}eco” de lo finito.
\vs p105 7:2 Lo trascendental no supone necesariamente que no se desarrolle, pero es supraevolutivo en el sentido finito; tampoco es no experiencial, pero es supraexperiencial en lo que ello pueda tener de significado para las criaturas. Quizás, la mejor ilustración de dicha paradoja sea el universo central de perfección: es difícilmente absoluto ---solo la Isla del Paraíso es verdaderamente absoluta en el sentido “materializado”---. Tampoco es una creación evolutiva finita, como lo son los siete suprauniversos. Havona es eterno pero no inmutable en el sentido de ser un universo sin crecimiento. Está habitado por criaturas (nativos de Havona) que nunca fueron realmente creadas, porque existen eternamente. Havona, pues, es un ejemplo de algo que no es exactamente finito ni incluso absoluto. Havona actúa además como amortiguación entre el Paraíso absoluto y las creaciones finitas, poniendo incluso más de manifiesto el propósito de los trascendentales. Pero la misma Havona no es trascendental ---es Havona---.
\vs p105 7:3 Del mismo modo que el Supremo se relaciona con los finitos, el Último se identifica con los trascendentales. Si bien, aunque comparemos de este modo el Supremo con el Último, ambos difieren en algo más que en una cuestión de grado; la diferencia entre ellos es una cuestión de cualidad. El Último es algo más que un supra\hyp{}Supremo proyectado en el nivel trascendental. El Último es todo eso, pero incluso más; el Último es el devenir de nuevas realidades de la Deidad, la delimitación de nuevas facetas de lo que hasta entonces era incondicionado.
\vs p105 7:4 \pc Entre esas realidades vinculadas con nivel trascendental, se hallan las siguientes:
\vs p105 7:5 \li{1.}La presencia de Deidad del Último.
\vs p105 7:6 \li{2.}El concepto del universo matriz.
\vs p105 7:7 \li{3.}Los arquitectos del universo matriz.
\vs p105 7:8 \li{4.}Los dos órdenes de organizadores de la fuerza del Paraíso.
\vs p105 7:9 \li{5.}Ciertas modificaciones en la potencia del espacio.
\vs p105 7:10 \li{6.}Ciertos valores del espíritu.
\vs p105 7:11 \li{7.}Ciertos contenidos de la mente.
\vs p105 7:12 \li{8.}Las cualidades y realidades absonitas.
\vs p105 7:13 \li{9.}La omnipotencia, la omnisciencia y la omnipresencia.
\vs p105 7:14 \li{10.}El espacio.
\vs p105 7:15 \pc Se puede pensar que el universo en el que ahora vivimos existe en los niveles finito, trascendental y absoluto. Este es el escenario cósmico en el que se desarrolla la inacabable historia de la actuación del ser personal y de la metamorfosis de la energía.
\vs p105 7:16 Y todas estas realidades múltiples están unificadas \bibemph{absolutamente} por las varias triunidades, \bibemph{operativamente} por los arquitectos del universo matriz y \bibemph{relativamente} por los siete espíritus mayores, coordinadores subsupremos de la divinidad del Dios Séptuplo.
\vs p105 7:17 El Dios Séptuplo constituye la revelación del ser personal y de la divinidad del Padre Universal a las criaturas de estatus tanto máximo como submáximo, pero hay otras relaciones séptuplas de la Primera Fuente y Centro que no se corresponden con la manifestación del ministerio espiritual y divino de Dios que es espíritu.
\vs p105 7:18 \pc En la eternidad del pasado, las fuerzas de los Absolutos, los espíritus de las Deidades y las personas de los Dioses empezaron a actuar en respuesta a la propia voluntad primordial de su propia y autoexistente voluntad. En esta era del universo, todos estamos asistiendo a las formidables repercusiones que acontecen en el inmenso panorama cósmico en cuanto a las manifestaciones subabsolutas de los potenciales ilimitados de todas estas realidades. Y es del todo posible que la continuada diversificación de la realidad primigenia de la Primera Fuente y Centro pueda proseguir adelante y hacia el exterior durante eras y eras, más y más, hasta las extensiones lejanas e inimaginables de la infinitud absoluta.
\vsetoff
\vs p105 7:19 [Exposición de un melquisedec de Nebadón.]
