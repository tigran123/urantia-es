\upaper{136}{El bautismo y los cuarenta días}
\author{Comisión de seres intermedios}
\vs p136 0:1 Jesús comenzó su ministerio público en el punto álgido del interés popular por la predicación de Juan y en un momento en el que el pueblo judío de Palestina esperaba con impaciencia la aparición del Mesías. Había un marcado contraste entre Juan y Jesús. Juan realizaba su labor con vehemencia y seriedad, mientras que Jesús la llevaba a cabo con calma y alegría; muy pocas veces en toda su vida se le vio con prisas. Jesús era un reconfortante consuelo para el mundo y, en cierto modo, un ejemplo; Juan apenas ofrecía consuelo o era un ejemplo. Predicaba el reino de los cielos, pero escasamente experimentaba su gozo. Aunque Jesús hablaba de él como el más grande de los profetas del antiguo orden, también dijo que el menor de los que vieron la gran luz del nuevo camino y la cruzó para entrar el reino de los cielos sería de hecho más grande que Juan.
\vs p136 0:2 Cuando Juan predicaba la llegada del reino, el núcleo de su mensaje era: ¡Arrepentíos! ¡Huid de la ira venidera! Cuando Jesús comenzó a predicar, aunque siguió instando al arrepentimiento, a su mensaje le seguía el evangelio, la buena nueva del gozo y la libertad del nuevo reino.
\usection{1. CONCEPTOS SOBRE EL MESÍAS ESPERADO}
\vs p136 1:1 Los judíos albergaban numerosas ideas sobre el libertador esperado, y cada una de estas diferentes escuelas de enseñanza mesiánica podía acogerse a afirmaciones de las escrituras hebreas para probar su opinión. En términos generales, los judíos pensaban que su historia nacional comenzaba con Abraham y culminaba en el Mesías y en la nueva era del reino de Dios. En tiempos previos, habían concebido a este libertador como “el siervo del Señor”; después, como “el Hijo del Hombre”; en tanto que, últimamente, algunos habían ido tan lejos como para referirse al Mesías como al “Hijo de Dios”. Pero, al margen de que se le llamara “la simiente de Abraham” o “el hijo de David”, todos coincidían en que él habría de ser el Mesías, el “ungido”. De este modo, el concepto evolucionó desde el “siervo del Señor” hasta el “hijo de David”, el “Hijo del Hombre” y el “Hijo de Dios”.
\vs p136 1:2 En los días de Juan y Jesús, los judíos más eruditos habían desarrollado la idea de que el Mesías venidero sería un israelita, característico y que habría logrado la perfección, reuniendo en su persona, como “siervo del Señor”, el papel triple de profeta, sacerdote y rey.
\vs p136 1:3 Los judíos creían fervientemente que, al igual que Moisés había liberado a sus ancestros de la esclavitud egipcia mediante milagrosos portentos, del mismo modo el Mesías venidero liberaría al pueblo judío de la dominación romana, mediante más grandes milagros y prodigiosos triunfos raciales. Los rabinos habían reunido casi quinientos pasajes de las Escrituras aseverando, a pesar de sus evidentes contradicciones, que profetizaban la llegada del Mesías. Y, dando detalles de cuándo vendría el Mesías y cuál sería su proceder y actividad, perdieron totalmente de vista el hecho de la \bibemph{persona} del Mesías prometido. Buscaban la restauración de la gloria nacional judía ---la exaltación temporal de Israel--- en lugar de la salvación del mundo. Se hacía por tanto patente que Jesús de Nazaret nunca podría haber satisfecho este concepto mesiánico materialista de la mente judía. Si hubiesen visto estas supuestas afirmaciones proféticas bajo una luz diferente, muchas de sus predicciones mesiánicas les habrían preparado la mente, con naturalidad, para poder reconocer en Jesús a aquel que terminaba una era e inauguraba una dispensación, nueva y mejor, de misericordia y salvación para todas las naciones.
\vs p136 1:4 \pc Los judíos se habían educado en la creencia de la doctrina de la \bibemph{Shekinah}. Pero este presunto símbolo de la Presencia Divina no podía ser visto en el templo. Creían que la venida del Mesías llevaría a efecto su restablecimiento. Tenían ideas encontradas respecto al pecado racial y a la supuesta naturaleza pecaminosa del hombre. Algunos enseñaban que el pecado de Adán había maldecido a la raza humana, y que el Mesías suprimiría esa maldición y restituiría al hombre en el favor divino. Otros instruían que Dios, al crear al hombre, había puesto en su ser naturalezas tanto de orden bueno como pecaminoso; que cuando observó las consecuencias de esta disposición, se decepcionó grandemente y “se arrepintió de haber creado al hombre así”. Y aquellos que impartían esta enseñanza creían que el Mesías vendría para redimir al hombre de esta innata naturaleza pecaminosa.
\vs p136 1:5 La mayoría de los judíos creía que ellos seguían padeciendo bajo el poder romano debido a sus pecados nacionales y a la impasibilidad de los gentiles prosélitos. La nación judía no se había \bibemph{arrepentido} de manera incondicional; por ello, el Mesías había retrasado su llegada. Se hablaba mucho del arrepentimiento; de ahí el atractivo, poderoso y apremiante, de la predicación de Juan: “Arrepentíos y sed bautizados porque el reino de los cielos se ha acercado”. Y el reino de los cielos tenía un único significado para cualquier fervoroso judío devoto: la venida del Mesías.
\vs p136 1:6 Existía una característica en el ministerio de gracia de Miguel que era totalmente ajena al concepto judío del Mesías, y se trataba de la \bibemph{unión} de las dos naturalezas: la humana y la divina. Los judíos habían concebido al Mesías de distintas maneras ya fuese como humano que había conseguido la perfección, sobrehumano e incluso divino, pero jamás habían contemplado la idea de la \bibemph{unión} de lo humano y lo divino. Y este fue el gran escollo con el que se encontraron los primeros discípulos de Jesús. Captaban el concepto humano del Mesías como el hijo de David, tal como lo habían expuesto los antiguos profetas; como el Hijo del Hombre, la idea sobrehumana de Daniel y de algunos de los últimos profetas; e incluso como el Hijo de Dios, tal como lo describía el autor del libro de Enoc y algunos de sus contemporáneos; pero nunca captaron, ni por un instante, el verdadero concepto de la unión, en una persona terrenal, de las dos naturalezas, la humana y la divina. La encarnación del creador en forma de criatura no se había revelado previamente. Solo se reveló en la persona de Jesús; el mundo no sabía nada de estas cosas hasta que el hijo creador se hizo carne y habitó entre los mortales del mundo.
\usection{2. EL BAUTISMO DE JESÚS}
\vs p136 2:1 Jesús se bautizó en el momento culminante de la predicación de Juan, cuando Palestina estaba enfervorizada con la expectativa de su mensaje ---“El reino de Dios se ha acercado”---, cuando todo el pueblo judío estaba inmerso en un riguroso y solemne examen de sí mismo. Los judíos poseían un profundo sentido de la solidaridad racial. No solo creían que los pecados del padre podían aquejar a sus hijos, sino que creían asimismo, y con firmeza, que el pecado de una sola criatura podía resultar en maldición para todo el pueblo. Por ello, no todos los que se presentaban al bautismo de Juan se consideraban culpables de los pecados concretos que él denunciaba. Muchas almas fervorosas iban a bautizarse por el bien de Israel. Temían que su ignorancia de algún pecado propio pudiese demorar la llegada del Mesías. Sentían que pertenecían a una nación culpable y maldecida por el pecado, y se prestaban al bautismo para que, al hacerlo, así se manifestaran los frutos de la penitencia de la raza. Es, por lo tanto, evidente que Jesús de ninguna manera recibió el bautismo de manos de Juan como un rito de arrepentimiento ni para la remisión de sus pecados. Al aceptarlo, Jesús solo seguía el ejemplo de muchos devotos israelitas.
\vs p136 2:2 \pc Cuando Jesús de Nazaret bajó al Jordán para bautizarse, era un mortal del mundo que había alcanzado la cúspide de la ascensión evolutiva humana, en todo lo relacionado con la conquista de la mente y la identificación de sí mismo con el espíritu. Aquel día, de pie en el Jordán, era un mortal de los mundos evolutivos del tiempo y del espacio que había logrado la perfección. Entre la mente mortal de Jesús y su modelador espiritual interior, el don divino de su Padre del Paraíso, se había establecido una perfecta sintonía y una completa comunicación. Y un modelador así mora en todos los seres normales que viven en Urantia desde la ascensión de Miguel a la jefatura de su universo, con la excepción de que el modelador de Jesús se había preparado previamente para esta misión especial, al haber igualmente habitado en Maquiventa Melquisedec, otro ser sobrehumano encarnado semejando un hombre mortal.
\vs p136 2:3 Habitualmente, cuando un mortal llega a tan altos niveles de perfección personal, se producen, a medida que se eleva espiritualmente, esos fenómenos previos que, en última instancia, culminan en la fusión del alma desarrollada del mortal con el modelador divino al que está vinculado. Y tal cambio se produjo aparentemente en la experiencia personal de Jesús de Nazaret aquel mismo día en el que descendió al Jordán con sus dos hermanos para que Juan les bautizase. Esta ceremonia fue el acto final de su vida puramente humana en Urantia, y muchos observadores sobrehumanos esperaban presenciar la fusión del modelador con la mente en la que habitaba, pero estaban todos llamados a sufrir una decepción. Sucedió algo nuevo e incluso más grande. Al poner Juan sus manos sobre Jesús para bautizarlo, el modelador interior se despidió definitivamente del alma humana perfeccionada de Josué ben José. Y en unos instantes, esta entidad divina regresó de Lugar de la Divinidad como modelador personificado y jefe, en todo el universo de Nebadón, de los de su clase. De este modo, Jesús pudo ciertamente ver a su propio antiguo espíritu divino descendiendo sobre él, a su vuelta, ya personificado. Y entonces oyó a este mismo espíritu, procedente del Paraíso, hablar, diciendo: “Este es mi Hijo amado en quien tengo complacencia”. Juan, junto a los dos hermanos de Jesús, también oyó estas palabras. Los discípulos de Juan, que estaban en la orilla, no las oyeron como tampoco presenciaron la aparición del modelador personificado. Solo los ojos de Jesús lo contemplaron.
\vs p136 2:4 \pc Cuando el modelador personificado, a su vuelta, ya enaltecido, había hablado, todo quedó en silencio. Mientras que los cuatro permanecían en el agua, Jesús, levantó la mirada hacia el modelador, que se encontraba cerca de él, y oró: “Padre mío que reinas en el cielo, santificado sea tu nombre. ¡Venga tu reino! Que se haga tu voluntad en la tierra, así como se hace en el cielo”. Cuando había orado, “se abrieron los cielos”, y el Hijo del Hombre contempló la visión de sí mismo, presentada ante él por el modelador ya personificado, como Hijo de Dios, tal como lo era antes de venir a la tierra, con la semejanza de un hombre mortal, y como lo sería cuando terminara su vida encarnada. Solo Jesús tuvo esta visión celestial.
\vs p136 2:5 Lo que Juan y Jesús oyeron fue la voz del modelador personificado, que hablaba en nombre del Padre Universal, porque el modelador es del Padre del Paraíso y como él. Este modelador estuvo vinculado a Jesús, en toda su labor en la tierra, durante el resto de su vida terrenal; Jesús se mantuvo en constante comunión con este modelador enaltecido.
\vs p136 2:6 \pc Cuando Jesús se bautizó, no se arrepintió de ninguna mala acción ni hizo ninguna confesión de pecados. El suyo fue un bautismo de consagración al cumplimiento de la voluntad del Padre celestial. Durante este, él oyó el inequívoco llamamiento de su Padre, su petición última para que se ocupara de los asuntos de su Padre, y se marchó para vivir en soledad durante cuarenta días, y reflexionar sobre estos múltiples problemas. En este retiro alejado de todo contacto personal activo con sus allegados terrenales, Jesús, continuando con su forma humana en Urantia, seguía el mismo modo de proceder que imperaba en los mundos morontiales, siempre que un mortal ascendente se fusiona con la presencia interior del Padre Universal.
\vs p136 2:7 Este día de bautismo se puso fin a la vida exclusivamente humana de Jesús. El hijo divino ha encontrado a su Padre, el Padre Universal ha encontrado a su Hijo encarnado, y hablan el uno con el otro.
\vs p136 2:8 \pc (Jesús se bautizó cuando casi tenía la edad de treinta y un años. Aunque Lucas dice que este bautismo tuvo lugar en el año decimoquinto del imperio de Tiberio César, que sería en el año 29 d. C. puesto que Augusto murió en el año 14 d.C., convendría recordar que Tiberio fue coemperador con Augusto durante dos años y medio antes de la muerte de Augusto, habiéndose acuñado monedas en su honor en octubre del año 11 d. C. El año decimoquinto de su Imperio fue, por lo tanto, este mismo año 26 d.C., año en el que Jesús se bautizó. Y fue también este el año en el que Poncio Pilato empezó a ejercer el cargo de gobernador de Judea.)
\usection{3. LOS CUARENTA DÍAS}
\vs p136 3:1 Jesús había soportado la gran tentación de su ministerio de gracia como mortal antes de su bautismo; en este lugar, durante seis semanas se había dejado empapar por el rocío del monte Hermón. Allí, en el monte Hermón, como un ser humano del mundo y sin ayuda, se había enfrentado y derrotado a Caligastia, aspirante al poder de Urantia, el príncipe de este mundo. Aquel día memorable, según consta en los archivos del universo, Jesús de Nazaret se convirtió en el príncipe planetario de Urantia. Y este príncipe de Urantia, a quien muy pronto se le proclamaría como soberano supremo de Nebadón, se retiró durante cuarenta días a un lugar apartado a fin de elaborar planes y determinar el método a utilizar para proclamar el nuevo reino de Dios en el corazón de los hombres.
\vs p136 3:2 Tras su bautismo, Jesús emprendió cuarenta días de adaptación al cambio de relaciones del mundo y del universo ocurridas a causa de la personificación de su modelador. Durante este aislamiento en las colinas de Perea, Jesús determinó las directrices a seguir y los métodos que emplearía en esta faceta, nueva y distinta, de la vida terrenal que estaba a punto de emprender.
\vs p136 3:3 Jesús no se retiró con el propósito de ayunar ni por la aflicción de su alma. No era un asceta, y había venido para abolir para siempre tales nociones sobre el acercamiento a Dios. Las razones que lo llevaron a procurar este retiro eran completamente diferentes a las que habían instado a Moisés y a Elías, e incluso a Juan el Bautista. Jesús era entonces totalmente consciente de su relación con el universo por él creado al igual que con el universo de los universos, bajo la supervisión del Padre del Paraíso, su Padre celestial. Ya recordaba por completo el cometido de su ministerio de gracia y las instrucciones que Emanuel, su hermano mayor, le había impartido antes de comenzar su encarnación en Urantia. Ya comprendía clara y enteramente todas estas extensas relaciones, y deseaba alejarse, por un tiempo, para dedicarse a la meditación y a la quietud, a fin de trazar esos planes y decidir su forma de proceder, en cuanto a la consecución de su ministerio público para el beneficio de este mundo y de todos los demás mundos de su universo local.
\vs p136 3:4 \pc Cuando Jesús caminaba por las colinas buscando un refugio que le fuera propicio, tuvo un encuentro con Gabriel, la brillante estrella de la mañana de Nebadón, mandatario en jefe de su universo. Gabriel restablecía en este momento su comunicación personal con el hijo creador del universo; se reunieron por primera vez desde que Miguel se despidió en Lugar de Salvación de sus compañeros para ir a Edentia, en preparación al inicio de su ministerio de gracia en Urantia. Gabriel, por instrucción de Emanuel y el mandato de los ancianos de días de Uversa, notificaba ahora a Jesús que su misión de gracia en Urantia estaba prácticamente acabada, en cuanto a la consumación de la soberanía sobre su universo y a la terminación de la rebelión de Lucifer. Lo primero lo consiguió el día de su bautismo, cuando la personificación de su modelador demostró la perfección y la culminación de su ministerio de gracia con la semejanza de un hombre mortal y, lo segundo, un hecho histórico, aquel día en el que bajó del monte Hermón para reunirse con el joven Tiglat, que lo esperaba. Se informaba ahora a Jesús, desde la más alta autoridad del universo local y del suprauniverso, de que su labor de gracia había finalizado, en la medida en la que afectaba a su estatus personal con respecto a su soberanía y a la rebelión. Ya había recibido tal constatación directamente del Paraíso en su visión bautismal y en el fenómeno de la personificación de su modelador del pensamiento interior.
\vs p136 3:5 Mientras charlaba en la montaña con Gabriel, el Padre de la constelación de Edentia apareció ante ellos en persona y dijo: “Ha quedado total constancia. La soberanía de Miguel número 611\,121 sobre su universo de Nebadón se ha llevado a cabo y reside en la diestra del Padre Universal”. Yo te traigo de Emanuel, tu hermano y auspiciador de tu encarnación en Urantia, la exención del ministerio de gracia. Ahora o en cualquier otro momento, en el modo que elijas, podrás poner fin al don de tu encarnación, ascender a la diestra de tu Padre, recibir la soberanía y asumir el bien merecido gobierno incondicional de todo Nebadón. También doy fe, por autorización de los ancianos de los días, de la completitud de los archivos del suprauniverso sobre la terminación de las rebeliones en tu universo por causa del pecado, y de que se te ha concedido autoridad plena e ilimitada para afrontar todas y cada una de tales posibles sublevaciones en el futuro. Oficialmente, tu labor en Urantia y tu encarnación como criatura mortal, ha terminado. El rumbo que tomes de ahora en adelante dependerá de tu propia decisión”.
\vs p136 3:6 Cuando el Padre Altísimo de Edentia se despidió, Jesús conversó durante largo rato con Gabriel sobre el bienestar del universo y, al enviar saludos a Emanuel, le aseguró de que, en la labor que estaba a punto de emprender en Urantia, siempre tendría presente las instrucciones recibidas en Lugar de Salvación con anterioridad a la asignación de su ministerio de gracia.
\vs p136 3:7 \pc Durante estos cuarenta días de retiro, Santiago y Juan, los hijos de Zebedeo, se dedicaron a buscar a Jesús. Muchas veces no estuvieron lejos del lugar en el que se cobijaba, pero nunca lo pudieron encontrar.
\usection{4. PLANES PARA SU MINISTERIO PÚBLICO}
\vs p136 4:1 Día tras día, en las colinas, Jesús definía los planes que seguiría durante el resto de su ministerio de gracia en Urantia. Primeramente, decidió que sus enseñanzas no serían simultáneas con las de Juan. Optó por estar relativamente alejado hasta que la labor de Juan lograra su propósito, o bien hasta que la encarcelación detuviese a Juan. Jesús sabía bien que la predicación de Juan, temeraria y sin tacto, suscitaría pronto el temor y la hostilidad de los gobernantes civiles. Dada la situación precaria de Juan, Jesús dio comienzo definitivamente a planificar el curso de acción a seguir en su tarea pública en beneficio de su pueblo y del mundo, y por el bien de todos los mundos habitados de su inmenso universo. El ministerio de gracia de Miguel como mortal tuvo lugar \bibemph{en} Urantia pero \bibemph{para} todos los mundos de Nebadón.
\vs p136 4:2 Lo primero que hizo Jesús, tras examinar detenidamente cómo coordinaría su plan general de acción con el movimiento de Juan, fue revisar mentalmente las indicaciones de Emanuel. Con atención, reflexionó sobre los consejos de su hermano mayor en cuanto a sus métodos de trabajo y al hecho de que no dejara escritos permanentes en el planeta. Jesús nunca escribiría en ningún lugar que no fuera la arena. En su siguiente visita a Nazaret, pese a la gran tristeza de su hermano José, Jesús destruyó todos los escritos suyos que se conservaban en tablillas en la carpintería y que colgaban de las paredes de la vieja casa. Y Jesús recapacitó bastante sobre los consejos de Emanuel en cuanto a su acercamiento al mundo, tal como él lo encontraría, en el terreno económico, social y político.
\vs p136 4:3 \pc Jesús no ayunó durante los cuarenta días de su retiro. El período más largo en el que se abstuvo de alimentarse fue durante los primeros dos días en las colinas; estaba tan absorto en sus pensamientos que se olvidó de comer. Si bien, al tercer día, salió a buscar alimentos. En este momento, tampoco fue \bibemph{tentado} por espíritus malignos ni por seres rebeldes emplazados en este mundo o venidos de cualquier otro mundo.
\vs p136 4:4 \pc Estos cuarenta días sirvieron para que se llegase a entablar el último diálogo entre las mentes divina y humana o, más bien, la primera actuación real de estas dos mentes hechas ahora una sola. El resultado de este trascendental período de meditación demostró, de forma concluyente, que la mente divina ya dominaba triunfante y espiritualmente sobre el intelecto humano. Desde este momento, la mente del hombre se ha convertido en la mente de Dios y, aunque la consciencia de la mente del hombre está constantemente presente, siempre esta mente humana espiritualizada dice: “No se haga mi voluntad, sino la tuya”.
\vs p136 4:5 Las acontecimientos de este memorable momento no fueron las visiones fantásticas de una mente hambrienta y debilitada ni tampoco los simbolismos desconcertantes y pueriles que más tarde se reflejarían como las “tentaciones de Jesús en el desierto”. Se trató más bien de un periodo de reflexión sobre toda la significativa y diversa andadura de su ministerio de gracia en Urantia y para la detenida preparación de esos proyectos en vista a que su posterior ministerio fuese de la mayor utilidad a este mundo y contribuyeran asimismo, en algo, al mejoramiento de todas las otras esferas aisladas por la rebelión. Jesús recapacitó sobre todo el lapso de tiempo de la vida humana en Urantia, desde los días de Andón y Fonta, pasando por la transgresión de Adán, hasta al ministerio de Melquisedec de Salem.
\vs p136 4:6 Gabriel le había recordado a Jesús que había dos formas en las que se podía manifestar al mundo en caso de que decidiera permanecer en Urantia por algún tiempo. Y también se le indicó con claridad que su opción en esta cuestión no guardaría relación ni con su soberanía del universo ni con la terminación de la rebelión de Lucifer. Estos dos modos de impartir su ministerio al mundo eran:
\vs p136 4:7 \li{1.}Su propio camino: el camino que pudiese parecerle más grato y provechoso desde la perspectiva de las necesidades inmediatas de este mundo y el avance presente de su propio universo.
\vs p136 4:8 \li{2.}El camino del Padre: la ejemplificación de un ideal a largo plazo de vida creatural, como lo conciben los elevados seres personales del Paraíso que administran el universo de los universos.
\vs p136 4:9 Por lo tanto, a Jesús se le puso de manifestó claramente que había dos formas de organizar la vida que le restaba en la tierra. Cada una de ellas tenía algo a su favor, considerándola a la luz de la situación inmediata. El Hijo del Hombre vio lúcidamente que su elección entre estos dos modos de conducirse no tenía relación con el hecho de haber recibido la soberanía del universo; que aquella era una cuestión ya dispuesta y sellada en los archivos del universo de los universos, y que solamente aguardaba su solicitud en persona. Pero se le señaló que sería de gran complacencia a Emanuel, su hermano del Paraíso, si él, Jesús, tuviese a bien completar su andadura de encarnación en la tierra con la nobleza con la que la había iniciado, siempre sujeto a la voluntad del Padre. Al tercer día de su retiro, Jesús se prometió a sí mismo que volvería al mundo para concluir su andadura terrenal y que, en cualquier situación que supusiera tomar uno de los dos caminos, siempre optaría por seguir la voluntad del Padre. Y así vivió el resto de su vida en la tierra, siendo siempre fiel a esa resolución. Incluso hasta en su amargo final, él indefectiblemente subordinó su voluntad soberana a la de su Padre celestial.
\vs p136 4:10 \pc Los cuarenta días de soledad que pasó en el desierto montañoso no significaron un período de grandes tentaciones sino, más bien, el período de las \bibemph{grandes decisiones} del Maestro. Durante estos días de solitaria comunión consigo mismo y con la presencia cercana de su Padre ---el modelador personificado (ya no tenía un guardián serafín personal)--- adoptó, una a una, las grandes decisiones que regirían la directrices y la conducta a seguir durante el resto de su andadura en la tierra. Con posterioridad, y en conexión con estos días de retiro, apareció la tradición acerca de una gran tentación, que se debió a la confusión creada por los fragmentados relatos de la lucha habida en el monte Hermón y, además, porque era habitual que todos los grandes profetas y líderes humanos comenzaran su andadura pública sometiéndose a supuestas temporadas de ayuno y oración. Cuando tenía que hacer frente a decisiones nuevas o importantes, Jesús siempre acostumbraba a retirarse para estar en comunión con su propio espíritu, buscando así conocer la voluntad de Dios.
\vs p136 4:11 \pc En toda su planificación para lo que le quedaba de vida en la tierra, el corazón humano de Jesús siempre se debatía entre dos formas contrapuestas de proceder:
\vs p136 4:12 \li{1.}Sentía un fuerte deseo por lograr que su pueblo ---y el mundo entero--- creyera en él y aceptara su nuevo reino espiritual. Y conocía bien sus ideas acerca del Mesías venidero.
\vs p136 4:13 \li{2.}Vivir y obrar de la manera en la que él sabía que iba a recibir la aprobación del Padre, desempeñar su labor en beneficio de otros mundos necesitados y continuar, a medida que instauraba el reino, revelando al Padre y mostrando su carácter amoroso y divino.
\vs p136 4:14 \pc Durante estos cruciales días, Jesús vivió en una antigua caverna rocosa, un refugio, en la ladera de las colinas, próximo a una aldea llamada, en otro tiempo, Beit Adis. Bebía del pequeño manantial que venía desde el costado de la colina hasta cerca de este cobijo de roca.
\usection{5. SU PRIMERA GRAN DECISIÓN}
\vs p136 5:1 Tres días después de comenzar este diálogo consigo mismo y con su modelador personificado, a Jesús se le concedió tener la visión de las multitudes celestiales de Nebadón reunidas y enviadas por sus comandantes a la espera de la voluntad de su amado soberano. Esta poderosa multitud estaba compuesta por doce legiones de serafines y un número proporcional de cada uno de los órdenes de inteligencia del universo. Y la primera gran decisión de Jesús en su retiro consistió en el hecho de hacer o no uso de estos poderosos seres en cuanto al plan a seguir en su venidera obra pública en Urantia.
\vs p136 5:2 Jesús optó por \bibemph{no} utilizar a nadie de esta inmensa asamblea, a no ser que resultara evidente que se trataba de la \bibemph{voluntad de su Padre}. A pesar de esta decisión general, esta extensa multitud permaneció con él durante el resto de su vida en la tierra, siempre dispuesta a obedecer la más mínima manifestación de la voluntad de su soberano. Aunque Jesús no veía continuamente a estos acompañantes seres personales con sus ojos humanos, el modelador personificado, con el que estaba vinculado, sí los veía constantemente, y podía comunicarse con ellos en cualquier momento.
\vs p136 5:3 \pc Antes de bajar de su retiro de cuarenta días en las montañas, Jesús encomendó a su modelador, recientemente personificado, el mando directo de esta multitud de vigilantes seres del universo y, durante más de cuatro años en tiempo de Urantia, estos seres, elegidos de entre todas las categorías de inteligencias del universo, obediente y respetuosamente, actuaron bajo la acertada dirección de este enaltecido y experimentado mentor misterioso. Al asumir el mando de esta poderosa asamblea, el modelador, siendo al mismo tiempo parte y esencia del Padre del Paraíso, aseguró a Jesús de que, en ningún caso, se permitiría a estas instancias intermedias servir o actuar en el contexto o en el beneficio de su andadura en la tierra, a no ser que tal intervención se debiese a la voluntad del Padre. Así pues, mediante la toma de esta importante decisión, Jesús se privó a sí mismo, voluntariamente, de cualquier cooperación de carácter sobrehumano en todo lo que tuviera que ver con el resto de su andadura como mortal, a no ser que el Padre eligiese libremente participar en algún acto o en alguna determinada circunstancia de la labor de su Hijo en el mundo.
\vs p136 5:4 Al aceptar el mando de las multitudes del universo que asistían a Cristo Miguel, el modelador personificado puso sumo cuidado en señalar a Jesús que, aunque sería posible limitar en el \bibemph{espacio} la actividad de tal asamblea de criaturas del universo debido a la autoridad a él delegada, dichas limitaciones quedaban sin vigor en cuanto a su labor en el \bibemph{tiempo}. Y esta limitación estaba supeditada al hecho de que los modeladores, una vez personificados, eran seres atemporales. Consiguientemente, se le advirtió a Jesús que, aunque la dirección del modelador sobre las inteligencias vivas bajo su mando sería plena y perfecta en todo lo que afectara al \bibemph{espacio,} no podrían imponerse limitaciones igualmente perfectas en aquello relacionado con el \bibemph{tiempo}. El modelador dijo: “Tal como me has instruido que haga, impediré cualquier forma de intervención de estas multitudes acompañantes de inteligencias del universo en relación con tu andadura terrenal, salvo en aquellos casos en los que el Padre del Paraíso me mande eximir a tales instancias intermedias de la prohibición para que se cumpla su voluntad divina, tal como tú has elegido, y en esos casos en los que tú puedas proceder a alguna elección o acción por parte de tu voluntad divina\hyp{}humana, que solo entrañara desviación del orden natural del mundo en relación al \bibemph{tiempo}. En tales circunstancias, me siento impotente, y tus criaturas que están aquí congregadas en poder y dispuestas a actuar en perfección se sienten igualmente impotentes. Si tus naturalezas unidas albergan alguna vez dichos deseos, estos mandatos que habrás elegido se ejecutarán sin dilación. Tu deseo en todas estas cuestiones supondrá la reducción del tiempo, y la cosa deseada \bibemph{se hará} realidad en ese momento. Bajo mis órdenes, esto constituye la máxima limitación posible que pueda imponerse a tu potencial soberanía. En mi propia consciencia, el tiempo es inexistente, y, por lo tanto, no puedo limitar a tus criaturas en ninguna cuestión relacionada con este”.
\vs p136 5:5 \pc De esta manera se le informó a Jesús sobre cómo se concretaría su decisión de continuar viviendo como un hombre entre los hombres. Mediante una única decisión, había evitado que todas sus vigilantes multitudes del universo de los distintos órdenes de inteligencias participaran en su ministerio público venidero, salvo en cuestiones solamente referidas al \bibemph{tiempo}. Resulta, por lo tanto evidente, que cualquier otro acontecimiento referido al ministerio de Jesús, de carácter sobrenatural o supuestamente sobrehumano, correspondía enteramente a la eliminación del tiempo, excepto cuando el Padre celestial dictaminara expresamente otra cosa. Ningún milagro, ministerio de misericordia ni cualquier otro posible acontecimiento que aconteciese posiblemente respecto a lo que le restaba a Jesús de su labor en la tierra podría ser de una naturaleza o carácter que trascendiese las leyes naturales establecidas y normalmente operativas en los asuntos del hombre tal como él vivía en Urantia, \bibemph{salvo} en la cuestión del tiempo, expresamente señalada. Evidentemente, no se podía imponer ningún límite a la manifestación de “la voluntad del Padre”. Cuando este soberano potencial de un universo expresaba su deseo de que ocurriera algún hecho, la eliminación del tiempo en conexión con este suceso no podía evitarse a menos que este Dios\hyp{}hombre expresamente afirmara su \bibemph{voluntad,} disponiendo pues que el tiempo no \bibemph{se acortase ni eliminase}. Para prevenir la aparición de supuestos \bibemph{milagros del tiempo,} era necesario que Jesús se mantuviese continuamente consciente de él. Cualquier lapso de la conciencia del tiempo por su parte, con respecto a la expresión de un claro deseo, equivaldría a que lo concebido en la mente de este hijo creador se llevase a efecto, y sin la intervención del tiempo.
\vs p136 5:6 Por medio de la supervisión y dirección del modelador personificado al que estaba vinculado, a Miguel le fue perfectamente posible poner un límite a su actividad personal en la tierra en lo referente al espacio, pero no le fue posible al Hijo del Hombre limitar, de la misma manera, su nueva situación en la tierra como soberano potencial de Nebadón en cuanto al \bibemph{tiempo}. Y este era el estatus real de Jesús de Nazaret cuando emprendió su ministerio público en Urantia.
\usection{6. SU SEGUNDA DECISIÓN}
\vs p136 6:1 Habiendo establecido ciertas normas respecto a todos los seres personales de todas las clases de inteligencias de su creación, al menos en cuanto a lo que podía decidir, al existir potenciales inherentes a su nuevo estatus divino que no podía determinar en ese momento. ¿Qué podría hacer él, siendo ahora el creador plenamente consciente de todas las cosas y de todos los seres existentes de este universo, con estas prerrogativas creadoras en las reiteradas situaciones de la vida, a las que se enfrentaría de forma inmediata al regresar a Galilea para reanudar su labor entre los hombres? De hecho, ya, justo donde se encontraba, en estas colinas solitarias, se vio forzosamente ante la necesidad de tener que obtener comida. Hacia el tercer día de sus meditaciones solitarias, su cuerpo humano sintió hambre. ¿Debería ir a buscarla, como haría cualquier mortal corriente, o debería simplemente ejercer sus habituales poderes creativos y producir, convenientemente, alimentos para nutrir su cuerpo? Y esta gran decisión del Maestro os ha llegado descrita a vosotros como una tentación ---como un desafío de supuestos enemigos retándolo: “día esta piedra que se convierta en pan”---.
\vs p136 6:2 Jesús, pues, fijó otras normas con carácter regular para el resto de su labor en la tierra. En lo que concernía a sus necesidades personales y, en general, incluso a sus relaciones con otras personas, eligió, entonces, deliberadamente, seguir el camino de la existencia ordinaria en el mundo; optó definitivamente por no seguir una norma que trascendiera, violara o atentara contra las leyes naturales implantadas por él mismo. Pero no podía prometerse a sí mismo, como ya su modelador personificado le había advertido, que estas leyes naturales no pudieran, en determinadas posibles circunstancias, \bibemph{acelerarse} notablemente. En un principio, Jesús decidió que su labor de vida se estructuraría y conduciría de acuerdo con la ley natural y en armonía con la organización social existente. El Maestro escogió, por tanto, un plan de vida que rechazaba los milagros y los prodigios. Una vez más se pronunció en favor de “la voluntad del Padre” y, nuevamente, dejó todas las cosas en manos de su Padre del Paraíso.
\vs p136 6:3 La naturaleza humana de Jesús le dictaba que su primer cometido era la preservación de sí mismo; tal es la actitud normal del hombre natural en los mundos del tiempo y el espacio, y es, por tanto, una reacción legítima de cualquier mortal de Urantia. Pero Jesús no estaba preocupado simplemente por este mundo y sus criaturas, sino que estaba viviendo una vida destinada a la instrucción e inspiración de las numerosas criaturas de un extenso universo.
\vs p136 6:4 Antes de su revelador bautismo, Jesús había vivido en perfecta sumisión a la voluntad y guía de su Padre celestial. Con rotundidad, decidió continuar con esa misma incondicional dependencia humana a la voluntad del Padre. Resolvió seguir un rumbo innatural ---optó por no procurar la preservación de sí mismo---. Eligió persistir en su norma de negarse a defenderse a sí mismo. Formuló sus conclusiones en torno a las palabras de las Escrituras que le resultaban familiares a su mente humana: “No solo de pan vivirá el hombre, sino de toda palabra de Dios”. Al llegar a esta decisión sobre el apetito de la naturaleza física expresado en el hambre de alimento, el Hijo del Hombre se manifestó definitivamente sobre todos los otros estímulos de la carne y sobre los impulsos naturales de la naturaleza humana.
\vs p136 6:5 Tal vez pudiese emplear sus poderes sobre humanos en beneficio de los demás, pero en su propio beneficio, jamás. Y se atuvo invariablemente a esta norma hasta el último momento, cuando se burlaban de él y decían: “A otros salvó, pero a sí mismo no se puede salvar” ---porque él no lo quiso así---.
\vs p136 6:6 Los judíos esperaban a un Mesías que hiciera incluso mayores prodigios que Moisés, a quien se le atribuía haber hecho manar agua de una roca de un lugar desértico y también haber alimentado en el desierto con maná a sus predecesores. Jesús conocía la clase de Mesías que sus compatriotas esperaban, y poseía todo el poder y las prerrogativas para estar a la altura de sus más entusiastas expectativas, pero eligió proceder en contra de tal grandioso plan de acción de poder y gloria. Veía esta forma de actuar con alardes de esos esperados milagros como un retroceso a los viejos tiempos en los que imperaba la ignorancia de la magia y las degradadas prácticas de los primitivos curanderos. Posiblemente, para salvar a sus criaturas, acelerara la ley natural, pero, lo que no haría sería trascender sus propias leyes, ya fuese en su propio bien o para impresionar a sus semejantes. Y la decisión del Maestro era irrevocable.
\vs p136 6:7 Jesús sentía pena por su gente; entendía perfectamente cómo se las había llevado a esperar al Mesías venidero, al momento cuando “la tierra dará su fruto diez mil veces más, sobre cada vid habrá mil ramas, y cada rama producirá mil racimos, y cada racimo producirá mil uvas, y cada uva producirá un galón de vino”. Los judíos creían que el Mesías inauguraría una era de abundancia de milagros. A los hebreos, durante mucho tiempo, se les había educado en tradiciones portentosas y leyendas de prodigios.
\vs p136 6:8 Jesús no era un Mesías llegado para multiplicar el pan y el vino. No venía para atender solamente las necesidades temporales, sino para revelar el Padre celestial a sus hijos de la tierra, tratando, al mismo tiempo, de que estos hijos se unieran a él y se esforzasen sinceramente por vivir acorde con la voluntad del Padre de los cielos.
\vs p136 6:9 \pc (1519.2) Al tomar esta decisión, Jesús de Nazaret ilustraba para un expectante universo la insensatez y el pecado de prostituir los talentos divinos y las capacidades otorgadas por Dios para el engrandecimiento personal o para el lucro y la glorificación puramente egoístas. Aquel fue el pecado de Lucifer y Caligastia.
\vs p136 6:10 Esta importante determinación expresa manifiestamente la verdad de que la propia satisfacción egoísta y la gratificación de los sentidos, solas y por sí mismas, no pueden proporcionar la felicidad a los seres humanos evolutivos. En la existencia humana, hay valores superiores ---competencia intelectual y logro espiritual--- que trascienden en mucho la necesaria gratificación de los apetitos e impulsos puramente físicos del hombre. La dotación natural de talento y capacidad del hombre deberían dedicarse fundamentalmente al desarrollo y al ennoblecimiento de sus más elevadas facultades de mente y espíritu.
\vs p136 6:11 De este modo, Jesús revelaba a las criaturas de su universo el modo de seguir el camino nuevo y mejor, los valores morales superiores de la vida y las satisfacciones espirituales más profundas de la existencia humana evolutiva de los mundos del espacio.
\usection{7. SU TERCERA DECISIÓN}
\vs p136 7:1 Habiendo adoptado sus decisiones sobre las cuestiones relacionadas con el alimento y con la atención física de las necesidades de su cuerpo material, el cuidado de su salud y el de la salud de los demás, todavía quedaban otros problemas por resolver. ¿Cuál sería su actitud en el caso de tener que enfrentarse a situaciones de peligro personal? Él optó por mantener una vigilancia normal respecto a su seguridad humana y tomar precauciones razonables para evitar la terminación prematura de su andadura en la carne, pero resolvió abstenerse de cualquier intervención sobrehumana cuando llegase el punto crítico de su vida en la carne. Cuando se planteaba esta cuestión, Jesús estaba sentado a la sombra de un árbol sobre un saliente rocoso, justo ante un precipicio. Era totalmente consciente de que podía arrojarse de la cornisa y saltar al espacio, y de que no sucedería nada que le dañase siempre que renunciara a su primera gran decisión de no recurrir a la intervención de sus inteligencias celestiales en la continuación de su labor de vida en Urantia, y siempre que derogara su segunda decisión respecto a su actitud hacia la preservación de sí mismo.
\vs p136 7:2 Jesús sabía que sus compatriotas aguardaban un Mesías que estuviera por encima de la ley natural. En la Escritura se le había impartido bien que: “No te sobrevendrá mal ni plaga tocará tu morada, pues a sus ángeles mandará acerca de ti, que te guarden en todos tus caminos. En las manos te llevarán para que tu pie no tropiece en piedra”. ¿Estaría justificado este tipo de arrogancia, este reto a las leyes de la gravedad de su Padre, para protegerse de cualquier daño posible o acaso para ganarse la confianza de su pueblo, mal instruido y conturbado? Pero tal proceder, por muy alentador que pudiese resultar para los judíos anhelantes de alguna señal, no sería una revelación de su Padre, sino un cuestionable jugueteo con las leyes establecidas del universo de los universos.
\vs p136 7:3 \pc Entendiendo todo esto y sabiendo que el Maestro se negaba a actuar en desafío de las leyes de la naturaleza establecidas por él en relación a su conducta personal, tendréis la certeza de que Jesús nunca caminó sobre las aguas ni hizo cosa alguna que constituyera un ultraje a su forma material de regir el mundo; siempre, por supuesto, tomando en consideración que, hasta ese momento, no estaba por completo exento de falta de control sobre el elemento tiempo en esos asuntos que eran competencia del modelador personificado.
\vs p136 7:4 Durante toda su vida en la tierra, Jesús fue invariablemente fiel a esta decisión. Aunque los fariseos se mofaban de él pidiéndole un signo o los que estaban en el Calvario le retaran a que descendiese de la cruz, él cumplió firmemente la decisión tomada en esta hora en la montaña.
\usection{8. SU CUARTA DECISIÓN}
\vs p136 8:1 El siguiente gran problema con el que este Dios\hyp{}hombre tuvo que enfrentarse, y que pronto resolvió conforme a la voluntad del Padre celestial, se refería a la cuestión de si debía o no emplear alguno de sus facultades sobrehumanas para atraer la atención y ganarse la adhesión de sus compatriotas. ¿Debía él, de alguna manera, ofrecer los poderes del universo a él otorgados para complacer el ansia de los judíos por lo espectacular y lo portentoso? Decidió que no lo haría. Optó por seguir una norma de actuación que prescindía de todos esos métodos de dar a conocer su misión a los hombres. Y, permanentemente, vivió a la altura de esta gran decisión. Incluso cuando, al impartir misericordia, permitió el acortamiento del tiempo, casi indefectiblemente advertía a los destinatarios de su labor curativa que no le hablaran a nadie del bien recibido. Y siempre rechazó la mofa de sus enemigos que le desafiaban a que “les mostrara un signo” como prueba y demostración de su divinidad.
\vs p136 8:2 Jesús, con gran acierto, previó que la realización de milagros y prodigios atraería, al abrumar a la mente material, solo una lealtad aparente; tales espectáculos no revelarían a Dios ni salvarían a los hombres. Se negó a ser un mero hacedor de milagros. Tomó la resolución de que se dedicaría a una sola tarea: la instauración del reino de los cielos.
\vs p136 8:3 \pc Durante todo este memorable diálogo de Jesús en comunión consigo mismo, el elemento humano que le hacía preguntarse y casi dudar estaba presente, porque Jesús era hombre al igual que Dios. Era evidente que los judíos nunca lo recibirían como el Mesías si no realizaba algún prodigio. Además, si él accedía a hacer solo una cosa que fuera innatural, la mente humana sabría con certeza que lo hacía en sumisión a una mente verdaderamente divina. ¿Sería coherente con “la voluntad del Padre” que la mente divina transigiera con la naturaleza dubitativa de la mente humana? Jesús descartó que lo fuera, y se refirió a la presencia del modelador personificado como prueba suficiente de la divinidad en alianza con la humanidad.
\vs p136 8:4 \pc Jesús había viajado mucho; recordaba Roma, Alejandría y Damasco. Conocía las maneras del mundo ---cómo el hombre conseguía sus fines en la política y en el comercio mediante el compromiso y la diplomacia---. ¿Utilizaría él este conocimiento en el avance de su misión en la tierra? ¡No! Decidió igualmente estar en contra de hacer concesiones a la sabiduría del mundo y a la influencia de las riquezas en la instauración del reino. De nuevo optó por depender exclusivamente de la voluntad del Padre.
\vs p136 8:5 Jesús era plenamente consciente de los atajos que alguien con sus poderes podía tomar. Conocía muchas formas para que la atención de la nación y del mundo entero recayera de inmediato sobre él. Pronto se celebraría la Pascua en Jerusalén; la ciudad se abarrotaría de visitantes. Podía ascender al pináculo del templo y, ante la desconcertada multitud, caminar por el aire; aquel sería el tipo de Mesías que buscaban. Pero, posteriormente, los decepcionaría, porque no había venido para restablecer el trono de David. Y conocía la inutilidad del método de Caligastia de tratar de ir por delante de la forma natural, lenta y segura, de cumplir el propósito divino. De nuevo, el Hijo del Hombre se inclinó obedientemente ante la voluntad del Padre, ante el camino del Padre.
\vs p136 8:6 Jesús optó por instaurar el reino de los cielos en el corazón de la humanidad de una manera natural, corriente, dificultosa y esforzada, justo la misma que sus hijos de la tierra tendrían que seguir luego en su labor de engrandecer y extender ese reino celestial. Porque el Hijo del Hombre sabía bien que sería “a través de muchas tribulaciones como muchos de los hijos de todas las eras entrarían en el reino”. Jesús pasaba ahora por la gran prueba del hombre civilizado: tener el poder y negarse rotundamente a usarlo para fines puramente egoístas o personales.
\vs p136 8:7 \pc Al examinar la vida y las experiencias del Hijo del Hombre, conviene tener presente que el Hijo de Dios estaba encarnado en la mente de un ser humano del siglo I y no en la mente de un mortal del siglo XX ni de cualquier otro siglo. Con ello, queremos transmitir la idea de que Jesús había adquirido sus cualidades humanas de forma natural. Era el resultado de los factores hereditarios y ambientales de su época, más la influencia de su formación y educación. Poseía una humanidad auténtica, natural, enteramente derivada, y propiciada, a partir de los antecedentes del estatus intelectual real y de las condiciones económicas y sociales de aquel día y generación. Aunque, en la experiencia de este Dios\hyp{}hombre, existía siempre la posibilidad de que la mente divina trascendiera el intelecto humano, no obstante, cuando su mente humana obraba, lo hacía como lo haría una verdadera mente mortal bajo los condicionamientos del entorno humano de aquella época.
\vs p136 8:8 \pc Jesús ilustró para todos los mundos de su inmenso universo la insensatez de crear situaciones artificiales con el fin de mostrar una autoridad arbitraria o de permitirse un poder excepcional para enaltecer los valores morales o acelerar el avance espiritual. Jesús decidió que no ofrecería su misión en la tierra para que se repitiera la decepción del reinado de los macabeos. Se negó a envilecer sus atributos divinos con el propósito de adquirir una popularidad inmerecida o conseguir prestigio político. No toleraría que la energía divina y creativa se trasmutara en poderío nacional o en prestigio internacional. Jesús de Nazaret rehusó hacer concesiones al \bibemph{mal,} y mucho menos confraternizar con el pecado. Triunfalmente, el Maestro puso la lealtad a la voluntad de su Padre por encima de cualquier otra consideración terrenal y temporal.
\usection{9. SU QUINTA DECISIÓN}
\vs p136 9:1 Habiendo establecido todo lo referente a sus normas sobre su relación individual hacia la ley natural y al poder espiritual, se centró en la elección del método que emplearía para proclamar e instaurar el reino de Dios. Juan ya había comenzado esta labor; ¿cómo podría continuar su mensaje? ¿Cómo podría hacerse cargo de la misión de Juan? ¿Cómo debería organizar a sus propios seguidores ser efectivos y lograr una inteligente cooperación entre estos y los de Juan? Jesús ya estaba llegando a una decisión definitiva que le impediría considerarse además el Mesías judío, al menos el Mesías, tal como se concebía popularmente en aquellos días.
\vs p136 9:2 Los judíos imaginaban a un libertador que vendría revestido de poder milagroso para abatir a los enemigos de Israel y establecer a los judíos como los gobernantes del mundo, sin carencias y libres de la opresión. Jesús sabía que nunca se realizaría tan esperanza. Era consciente de que el reino de los cielos guardaba relación con el derrocamiento del mal del corazón de los hombres, y que era estrictamente una preocupación de carácter espiritual. Meditó sobre la conveniencia de inaugurar el reino espiritual efectuando un despliegue, brillante e impresionante, de poder ---y tal proceder era permisible por estar enteramente dentro de la jurisdicción de Miguel--- pero decidió ponerse en contra de tal plan. No quería transigir con los métodos revolucionarios de Caligastia. Había ganado el mundo potencialmente al someterse a la voluntad del Padre, y se proponía acabar su labor como la había comenzado, y como el Hijo del Hombre.
\vs p136 9:3 ¡No os podéis imaginar qué habría sucedido en Urantia si este Dios\hyp{}hombre, poseedor potencial ahora de todo el poder del cielo y de la tierra, hubiera decidido alguna vez enarbolar el estandarte de la soberanía, llamar a formación militar a sus batallones de hacedores de portentos! Pero no transigió ante esto. No serviría al mal para que el culto de Dios presumiblemente se derivara de ahí. Se atendría a la voluntad del Padre. Proclamaría a un expectante universo: “Al Señor tu Dios adorarás y solo a él servirás”.
\vs p136 9:4 Conforme pasaban los días, Jesús percibía, cada vez con mayor claridad, qué clase de revelador de la verdad llegaría a ser. Comprendía que el camino de Dios no iba a ser un camino fácil. Empezó a darse cuenta de que la copa de la que bebería el resto de su experiencia humana sería amarga, pero estaba decidido a beber de ella.
\vs p136 9:5 Incluso su mente humana se despide del trono de David y sigue, paso a paso, la senda de lo divino. Esta mente humana sigue formulando preguntas, pero acepta infaliblemente las respuestas divinas como resoluciones definitivas para su vida, una vida que combina la existencia en el mundo como hombre mientras se somete continua e incondicionalmente a hacer la voluntad eterna y divina del Padre.
\vs p136 9:6 Roma era la señora del mundo occidental. El Hijo del Hombre, ahora en su aislamiento, y tomando estas cruciales decisiones, con las multitudes del cielo bajo su mando, significaba la última oportunidad que tenía el pueblo judío para dominar el mundo; pero este judío humano, poseedor de sabiduría y poder tan grandes, se negó a emplear sus dotes como soberano del universo ni para su propio engrandecimiento ni para la entronización de su pueblo. El veía, por así decirlo, “los reinos de este mundo” y disponía del poder para enseñorearse de ellos. Los altísimos de Edentia habían puesto estos poderes en sus manos, pero no los quería. Los reinos de la tierra eran algo irrisorio para suscitar el interés del creador y gobernante de un universo. Tenía un solo objetivo: hacer avanzar la revelación de Dios al hombre, instaurar el reino, la soberanía del Padre celestial en el corazón de la humanidad.
\vs p136 9:7 A Jesús le repugnaba la idea de las batallas, las contiendas y las matanzas; no quería saber nada de eso. Aparecería en la tierra como el Príncipe de Paz para revelar al Dios del amor. Antes de su bautismo, había rechazado de nuevo el ofrecimiento de los zelotes para liderarlos en su rebelión contra los opresores romanos. Y, ahora, tomaba, pues, su decisión sobre esas citas de las Escrituras que su madre le había enseñado, como la que decía: “El Señor me ha dicho: ‘Mi Hijo eres tú; yo te engendré hoy. Pídeme, y te daré por herencia las naciones y como posesión tuya los confines de la tierra. Los quebrantarás con una vara de hierro; como vasija de alfarero los desmenuzarás’”.
\vs p136 9:8 Jesús de Nazaret llegó a la conclusión de que estos pasajes no aludían a él. Por último, y finalmente, la mente humana del Hijo del Hombre se desprendió de todos estos inconvenientes y contradicciones mesiánicas ---escrituras hebreas, formación paterna, enseñanzas del jazán, expectativas de los judíos y ambiciosas aspiraciones, y, de una vez y para siempre, determinó su curso de acción---. Regresaría a Galilea y empezaría discretamente la proclamación del reino, y confiaría en su Padre (el modelador personificado) en cuanto a los detalles de su proceder diario.
\vs p136 9:9 \pc Mediante esta toma de decisiones, Jesús sentó un meritorio ejemplo para todas las personas de todos los mundos de un inmenso universo cuando se negó a aplicar pruebas materiales para demostrar problemas espirituales, cuando se negó a desafiar arrogantemente las leyes naturales. Y ofreció un ejemplo inspirador de lealtad al universo y de nobleza moral cuando rehusó hacerse con el poder temporal como preludio a la gloria espiritual.
\vs p136 9:10 \pc Si el Hijo del Hombre tenía alguna duda sobre su misión y sobre la naturaleza de esta al subir las colinas tras su bautismo, no la tenía cuando volvió con sus compatriotas, tras pasar cuarenta días de aislamiento y de decisiones.
\vs p136 9:11 Jesús ha trazado un plan para la instauración del reino del Padre. No satisfará los deseos materiales de la gente. No repartirá pan a las multitudes como tan recientemente había visto en Roma. No atraerá la atención hacia sí mismo obrando prodigios, a pesar de que los judíos estaban esperando precisamente a ese tipo de libertador. Tampoco intentará que se acepten sus mensajes espirituales con alarde de autoridad política o de poder temporal.
\vs p136 9:12 Al rechazar estos métodos de enaltecimiento del reino venidero ante los ojos de los expectantes judíos, Jesús se aseguraba de que estos mismos judíos rechazarían, de forma cierta e irrevocable, todas sus reivindicaciones de autoridad y divinidad. Consciente de todo esto, Jesús procuró evitar, durante mucho tiempo, que sus primeros seguidores aludieran a él como el Mesías.
\vs p136 9:13 Durante todo su ministerio público, Jesús tuvo necesidad de hacer frente a tres situaciones constantemente recurrentes: el clamor a ser alimentados, el empeño en los milagros y la petición última de permitir que sus seguidores le nombraran rey. Pero Jesús nunca se apartó de las decisiones tomadas durante estos días de aislamiento en las colinas de Perea.
\usection{10. SU SEXTA DECISIÓN}
\vs p136 10:1 En el último día de su memorable retiro, antes de comenzar a bajar del monte para reunirse con Juan y sus discípulos, el Hijo del Hombre adoptó su última decisión. Y se la comunicó al modelador personificado con estas palabras: “Y en todas las demás cuestiones, así como en estas decisiones ya constatadas, te prometo que me someteré a la voluntad de mi Padre”. Y, tras haber dicho esto, descendió de la montaña. Y su rostro brillaba con la gloria de la victoria espiritual y del logro moral.
