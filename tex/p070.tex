\upaper{70}{Evolución del gobierno humano}
\author{Melquisedec}
\vs p070 0:1 En cuanto el hombre solucionó parcialmente, el problema de ganarse el sustento se enfrentó a la tarea de regular las relaciones humanas. El desarrollo de la actividad laboral exigía leyes, orden y adaptación social; la propiedad privada requería gobierno.
\vs p070 0:2 En un mundo evolutivo, los antagonismos son naturales; la paz se alcanza únicamente por medio de algún tipo de sistema regulativo social. La regulación social es inseparable de la organización social; la asociación entraña alguna autoridad regidora. El gobierno impulsa la coordinación de los antagonismos de las tribus, los clanes, las familias y los individuos.
\vs p070 0:3 El gobierno es un desarrollo involuntario; evoluciona a base de ensayo y error. Tiene rasgos de continuidad; por ello, se convierte en tradicional. La anarquía aumentaba la miseria; de ahí, que lentamente surgió o está surgiendo el gobierno, un orden público relativo. Las exigencias coercitivas propias de la lucha por la existencia llevaron a la raza humana por el camino progresivo de la civilización.
\usection{1. GÉNESIS DE LA GUERRA}
\vs p070 1:1 La guerra es el estado y el legado natural del hombre evolutivo; la paz es el indicador social que mide el progreso de la civilización. Antes de la socialización parcial de las razas en el curso de su avance, el hombre era sumamente individualista, extremadamente receloso e increíblemente pendenciero. La violencia es la ley de la naturaleza; la hostilidad, la respuesta automática de los hijos de la naturaleza; mientras que la guerra no es sino esto mismo realizado de forma colectiva. Y donde y cuando quiera que el entramado de la civilización se tensione debido a las complicaciones del avance de la sociedad, siempre se produce un retorno inmediato y pernicioso a estos primitivos métodos de resolución violenta de la irritabilidad que se origina en las interrelaciones humanas.
\vs p070 1:2 La guerra es una respuesta irracional ante desencuentros y sentimientos airados; la paz se presenta como la solución civilizada a todos estos problemas y dificultades. Las razas sangiks, junto con los posteriores y degradados adanitas y noditas, eran pueblos beligerantes. A los andonitas pronto se les enseñó la regla de oro e, incluso hoy día, sus descendientes esquimales viven en buena medida bajo ese código; entre ellos, las costumbres están bien arraigadas, y se encuentran bastante libres de antagonismos violentos.
\vs p070 1:3 Andón enseñó a sus hijos a dirimir las disputas golpeando cada cual un árbol con un palo, mientras maldecían al árbol; el vencedor era aquel cuyo palo se rompía primero. Los posteriores andonitas solían resolver sus conflictos celebrando un acto público en el que los litigantes se mofaban unos de otros y se ridiculizaban mutuamente, en tanto que el público elegía al ganador con sus aplausos.
\vs p070 1:4 Pero no podía darse un fenómeno como el de la guerra hasta que la sociedad no evolucionara lo suficiente como para realmente experimentar períodos de paz y admitir las prácticas bélicas. El mismo concepto de guerra supone un cierto grado de organización.
\vs p070 1:5 Con la paulatina aparición de los grupos sociales, la irritabilidad individual comenzó a quedar inmersa en los sentimientos del grupo, y esto promovió la calma dentro de la tribu, aunque a expensas de la paz entre las distintas tribus. Por consiguiente, la paz se disfrutó primeramente dentro del grupo, o tribu, que siempre detestaba y odiaba al grupo de fuera, a los foráneos. El hombre primitivo consideraba una virtud derramar sangre extranjera.
\vs p070 1:6 Si bien, en un principio, ni siquiera esto dio resultado. Cuando los primeros jefes intentaban resolver los desencuentros, a menudo creyeron necesario permitir las peleas de piedras tribales, al menos una vez al año. El clan se dividía en dos grupos y libraban una batalla durante todo el día, sin otra razón que por pura diversión; no cabe duda de que disfrutaban peleando.
\vs p070 1:7 \pc La guerra persiste porque el hombre es humano, evolucionó de un animal, y todos los animales son belicosos. Entre las tempranas causas de la guerra estaban las siguientes:
\vs p070 1:8 \li{1.}\bibemph{El hambre,} que dio lugar al saqueo de alimentos. La escasez de tierras siempre llevó a la guerra y, durante estas luchas, las tribus pacíficas primitivas fueron prácticamente exterminadas.
\vs p070 1:9 \li{2.}\bibemph{La escasez de mujeres:} el esfuerzo para mitigar la escasez de ayuda doméstica. El rapto de mujeres siempre ha sido causa de guerra.
\vs p070 1:10 \li{3.}\bibemph{La vanidad:} el deseo de demostrar valentía tribal. Los grupos más dotados peleaban para imponer su modo de vida a los pueblos menos dotados.
\vs p070 1:11 \li{4.}\bibemph{Los esclavos:} la necesidad de incorporar mano de obra.
\vs p070 1:12 \li{5.}\bibemph{La venganza} era motivo de guerra cuando alguna tribu creía que otra tribu vecina había provocado la muerte de uno de los suyos. Se guardaba luto hasta que se traía una cabeza a la tribu. Las guerras de venganza estuvieron bien consideradas hasta tiempos relativamente modernos.
\vs p070 1:13 \li{6.}\bibemph{El esparcimiento:} los jóvenes de estos tiempos primitivos veían la guerra como una forma de entretenimiento. Si no se producía ningún pretexto razonable y suficiente para la guerra, cuando la paz se convertía en algo opresivo, entre las tribus colindantes se acostumbraba a salir a combatir de forma semi\hyp{}amistosa; realizaban incursiones de carácter festivo y disfrutaban de una batalla simulada.
\vs p070 1:14 \li{7.}\bibemph{La religión:} el deseo de conseguir adeptos. Todas las religiones primitivas autorizaban la guerra. Solo en épocas recientes comenzó la religión a desaprobar la guerra. Desafortunadamente, el sacerdocio primitivo se aliaba por lo general con el poder militar. Una de las grandes iniciativas de los tiempos a favor de la paz ha sido el intento de separar la Iglesia del Estado.
\vs p070 1:15 \pc Estas tribus ancestrales siempre hacían la guerra cumpliendo el mandato de sus dioses, por orden de sus jefes o curanderos. Los hebreos creían en un “Dios de las batallas”; y la narración de su redada a los madianitas es el típico relato de la atroz crueldad de las antiguas guerras tribales; este ataque, con la masacre de todos los varones y más tarde la matanza de todos los niños varones y mujeres que no eran vírgenes, hubiese hecho honor a las costumbres de un cacique tribal de doscientos mil años antes. Y todo se llevó a cabo en el “nombre del Señor Dios de Israel”.
\vs p070 1:16 Este es el relato de la evolución de la sociedad ---consecuencia natural de los problemas de las razas---, del hombre forjando su propio destino en la tierra. La Deidad no instiga estas atrocidades, a pesar de la propensión del hombre a atribuir la responsabilidad a sus dioses.
\vs p070 1:17 \pc La clemencia de las fuerzas militares ha tardado en llegar a la humanidad. Incluso cuando una mujer, Débora, gobernaba a los hebreos, continuó la misma crueldad generalizada. Su general, cuando venció a los gentiles, hizo que “todo el ejército cayera a filo de espada, hasta no quedar ni uno”.
\vs p070 1:18 En la historia de la raza humana muy pronto se hizo uso de armas envenenadas. Y se llevaban a cabo todo tipo de mutilaciones. Saúl no vaciló en exigir a David que le entregara cien prepucios filisteos como dote de su hija Mical.
\vs p070 1:19 Las guerras primitivas se libraban entre tribus completas; pero, en tiempos posteriores, cuando dos integrantes de tribus diferentes tenían una disputa, en lugar de luchar las dos tribus, los dos contendientes se batían en duelo. También se convirtió en costumbre que dos ejércitos lo apostaran todo al resultado del combate entre representantes elegidos de cada lado, como en el caso de David y Goliat.
\vs p070 1:20 La primera mejora de la guerra consistió en la toma de prisioneros. Tras ello, se eximió a las mujeres de las hostilidades y, más tarde, llegó el reconocimiento de los no combatientes. Rápidamente, para adecuarse a la creciente complejidad de los combates, se desarrollaron las castas militares y los ejércitos permanentes. A estos guerreros se les prohibió pronto que se relacionaran con mujeres, y hace tiempo que las mujeres dejaron de combatir, aunque siempre han alimentado y asistido a los soldados y los han instado a acudir a la batalla.
\vs p070 1:21 La práctica de declarar la guerra representó un gran avance. Estas declaraciones sobre la intención de ir a combate significaban la llegada de un sentido de la justicia, y a esto siguió el paulatino desarrollo de las reglas de la guerra “civilizada”. Muy pronto se convirtió en costumbre no luchar cerca de sitios religiosos y, más adelante, no hacerlo en ciertos días sagrados. A continuación, vino el reconocimiento general del derecho de asilo; los fugitivos políticos recibirían protección.
\vs p070 1:22 Así, progresivamente, evolucionó la guerra, desde la caza primitiva del hombre hasta el sistema algo más estructurado de las naciones “civilizadas” de épocas posteriores. Si bien, solo lentamente una actitud social de amistad sustituye a otra de enemistad.
\usection{2. VALOR SOCIAL DE LA GUERRA}
\vs p070 2:1 En tiempos pasados, una cruenta guerra instituía cambios sociales y facilitaba la adopción de nuevas ideas, algo que no hubiese ocurrido de forma natural en diez mil años. El terrible precio pagado por estas ciertas ventajas que se obtenían de la guerra fue que la sociedad se vio de nuevo arrojada temporalmente a la barbarie, renunciándose a la lógica de la civilización. La guerra es un poderoso remedio, muy costoso y de lo más peligroso; aunque cura a menudo algunos trastornos sociales, a veces mata al paciente, destruye la sociedad.
\vs p070 2:2 La necesidad constante de la defensa nacional produce numerosos ajustes sociales nuevos y avanzados. Hoy día, la sociedad disfruta del beneficio de una larga lista de valiosas innovaciones que eran, en un principio, enteramente militares; a la guerra se le debe incluso la danza, una de cuyas primitivas formas era un ejercicio militar.
\vs p070 2:3 \pc La guerra ha sido de utilidad social para las civilizaciones anteriores debido a que:
\vs p070 2:4 \li{1.}Imponía disciplina, forzaba a la cooperación.
\vs p070 2:5 \li{2.}Primaba la fortaleza y el valor.
\vs p070 2:6 \li{3.}Fomentaba y consolidaba el nacionalismo.
\vs p070 2:7 \li{4.}Exterminaba a los pueblos débiles e inaptos.
\vs p070 2:8 \li{5.}Anulaba la ilusión de la igualdad primitiva y estratificaba a la sociedad de forma selectiva.
\vs p070 2:9 \pc La guerra ha tenido cierto valor evolutivo y selectivo, pero, como la esclavitud, deberá abandonarse algún día, a medida que la civilización vaya lentamente progresando. Las guerras antiguas fomentaban los viajes y los intercambios culturales; estos objetivos se alcanzan ahora mejor mediante los métodos modernos de transporte y comunicación. Las guerras antiguas fortalecían a las naciones, pero las luchas modernas trastornan la cultura civilizada. Las guerras ancestrales daban lugar a la aniquilación de los pueblos peor dotados; el resultado final de los conflictos modernos es la destrucción selectiva de los mejores linajes humanos. Las guerras primitivas favorecían la organización y la eficiencia, pero estos elementos se han convertido ahora en la meta de la industria moderna. En tiempos pasados, la guerra era un fermento social que daba impulso a la civilización; este efecto se consigue ahora mejor mediante las aspiraciones y la invención. Las guerras ancestrales apoyaban la idea de un Dios de las batallas, pero al hombre moderno se le ha dicho que Dios es amor. La guerra ha servido para lograr muchos fines valiosos en el pasado; ha sido un andamiaje imprescindible en la edificación de la civilización, pero está llegando rápidamente a la insolvencia cultural ---incapaz de producir réditos en términos de beneficio social, de alguna forma acordes con las terribles pérdidas que la acompañan cuando se le invoca---.
\vs p070 2:10 Hubo un tiempo en el que los médicos creían que la sangría era la cura para muchas enfermedades; pero, desde entonces, han descubierto mejores remedios para la mayor parte de estas afecciones. Y, por ello, el derramamiento de sangre internacional de la guerra debe por supuesto dar lugar al descubrimiento de mejores métodos para curar los males de las naciones.
\vs p070 2:11 Las naciones de Urantia ya han iniciado la gigantesca lucha entre el militarismo nacionalista y el industrialismo y, en gran medida, este conflicto es similar a la lucha secular entre el pastor\hyp{}cazador y el agricultor. Pero si el industrialismo ha de triunfar sobre el militarismo, debe evitar los peligros a los que se enfrenta. Los peligros de la industria incipiente en Urantia son:
\vs p070 2:12 \li{1.}La fuerte tendencia al materialismo, la ceguera espiritual.
\vs p070 2:13 \li{2.}El culto al poder y las riquezas, la distorsión de los valores.
\vs p070 2:14 \li{3.}Los vicios del lujo, la inmadurez cultural.
\vs p070 2:15 \li{4.}Los crecientes peligros de la indolencia, la falta de sensibilidad al servicio.
\vs p070 2:16 \li{5.}El desarrollo de una indeseable debilidad racial, el deterioro biológico.
\vs p070 2:17 \li{6.}La amenaza de la esclavitud industrial estandarizada, el estancamiento personal. El trabajo ennoblece, pero el trabajo tedioso es paralizante.
\vs p070 2:18 \pc El militarismo es autocrático y cruel ---salvaje---. Fomenta la organización social entre los conquistadores, pero abate a los vencidos. El industrialismo es más civilizado y debería proceder de tal modo que facilite la iniciativa y aliente el individualismo. La sociedad debe propiciar la originalidad de todas las formas posibles.
\vs p070 2:19 No cometáis el error de glorificar la guerra; percibid más bien lo que ha hecho por la sociedad de modo que podáis, de manera más precisa, prever lo que sus alternativas deben aportar para continuar el progreso de la civilización. Y si tales alternativas no son las adecuadas, podéis estar entonces seguros de que las guerras seguirán durante mucho tiempo.
\vs p070 2:20 El hombre jamás aceptará la paz como modo normal de vida hasta que no se haya convencido concienzuda y repetidamente de que la paz es lo más conveniente para su bienestar material, y hasta que la sociedad no proporcione con sensatez alternativas pacíficas para satisfacer esa tendencia, inherente y periódica, a dar rienda suelta al impulso colectivo tendente a liberar aquellas emociones y energías, que constantemente se acumulan y que pertenecen a las reacciones de autopreservación de las especies humanas.
\vs p070 2:21 Pero, aunque sea de paso, la guerra debe ser reconocida como la escuela de la experiencia que forzó a una raza de arrogantes individualistas a someterse a una autoridad muy concentrada ---a un jefe del ejecutivo---. En las guerras antiguas se elegía como líderes a hombres innatamente grandes; pero esto no se hace ya en las guerras modernas. Para descubrir a sus líderes, la sociedad debe recurrir ahora a las conquistas de la paz: a la industria, a la ciencia y al logro social.
\usection{3. PRIMERAS ASOCIACIONES HUMANAS}
\vs p070 3:1 En la sociedad más primitiva, la \bibemph{horda} lo es todo; incluso los niños son sus bienes comunales. La familia evolutiva desplazó a la horda en cuanto a la crianza de los hijos, mientras que los clanes y las tribus que iban surgiendo la reemplazaron como unidad social.
\vs p070 3:2 El apetito sexual y el amor maternal instauran la familia. Si bien, el verdadero gobierno no aparece hasta que no se empiezan a formar grupos más extensos que los familiares. En los días anteriores a la familia de la horda, el liderazgo lo componían personas que se elegían de forma informal. Los bosquimanos africanos nunca avanzaron más allá de este estadio primitivo; no hay jefes en su horda.
\vs p070 3:3 \pc Las familias se unieron por lazos de sangre en clanes, agrupaciones de parientes; y estos evolucionaron posteriormente hasta convertirse en tribus, en comunidades territoriales. La guerra y la presión externa forzaron a los clanes de parientes a organizarse en tribus, pero el comercio y el trueque mantuvieron a estos primeros grupos primitivos unidos con cierto grado de paz interna.
\vs p070 3:4 Más que todos los sofismas sentimentales de una planificación visionaria de la paz, serán las organizaciones comerciales internacionales las que fomentarán la paz en Urantia. Las relaciones comerciales se han visto favorecidas por el desarrollo del lenguaje y por medios de comunicación más perfectos, al igual que por un mejor transporte.
\vs p070 3:5 La falta de un lenguaje común siempre ha obstaculizado el desarrollo de grupos pacíficos, pero el dinero se ha convertido en el lenguaje universal del comercio moderno. La sociedad moderna se mantiene unida mayormente gracias al mercado laboral. El móvil de obtener beneficios es un poderoso civilizador cuando se le refuerza con el deseo de servir.
\vs p070 3:6 \pc En los tiempos primitivos, cada tribu se rodeaba de un perímetro territorial de miedo y suspicacia crecientes; de ahí la costumbre que existió alguna vez de matar a todos los extraños y, más adelante, de esclavizarlos. La antigua idea de la amistad significaba la adopción dentro del clan; y se creía que la pertenencia al clan sobrevivía a la muerte ---uno de los más tempranos conceptos de la vida eterna---.
\vs p070 3:7 La ceremonia de adopción consistía en beber cada cual la sangre del otro. En algunos grupos, se intercambiaba saliva en lugar de beber sangre; este fue el ancestral origen de la costumbre de besarse socialmente. Y todas las ceremonias de asociación, ya fuese de matrimonio o de adopción, siempre terminaban en banquete.
\vs p070 3:8 En tiempos posteriores, se usó sangre diluida con vino tinto y, con el tiempo, solo se bebió el vino para sellar la ceremonia de adopción, que se anunciaba chocando las dos copas de vino y se consumaba tragando el brebaje. Los hebreos empleaban una forma modificada de esta ceremonia de adopción. Sus ancestros árabes utilizaban un juramento que se prestaba mientras la mano del candidato descansaba sobre el órgano reproductor del nativo de la tribu. Los hebreos trataban a los extranjeros que habían adoptado amable y fraternalmente. “Como a uno de vosotros trataréis al extranjero que habite entre vosotros, y lo amaréis como a vosotros mismos”.
\vs p070 3:9 “La amistad con los huéspedes” definía una relación de hospitalidad que se entablaba de forma temporal. Cuando los visitantes se marchaban, se partía un plato por la mitad, y se le daba un trozo al amigo que partía para que sirviera de posible presentación a un tercero que más adelante pudiese venir de visita. Era habitual que los huéspedes asumieran sus gastos contando historias de sus viajes y aventuras. Los narradores de los tiempos antiguos se volvieron tan populares que las costumbres tradicionales dictaban que se prohibiera su actividad durante las estaciones de caza o recolecta.
\vs p070 3:10 Los primeros tratados de paz eran los “vínculos de sangre”. Los embajadores de la paz de dos tribus en guerra se reunían, presentaban sus respetos y, luego, procedían a pincharse la piel hasta que sangrara; tras lo cual, chupaban la sangre uno del otro y declaraban la paz.
\vs p070 3:11 Las primeras misiones de paz estaban formadas por delegaciones de hombres que llevaban a sus doncellas preferidas a sus antiguos enemigos para gratificarles sexualmente; el apetito sexual se utilizaba como forma de combatir el impulso bélico. La tribu a la que se honraba de esta manera devolvía la visita y le hacía a su vez su ofrecimiento de doncellas; con lo que se instauraba firmemente la paz. Y, al poco tiempo, se autorizaron los matrimonios entre las familias de los jefes.
\usection{4. CLANES Y TRIBUS}
\vs p070 4:1 El primer grupo pacífico fue la familia, luego el clan, la tribu y, más tarde, la nación, que acabaría por convertirse en el moderno Estado territorial. Resulta alentador que los actuales grupos pacíficos, desde hace mucho tiempo, se hayan expandido más allá de los lazos de sangre hasta abarcar a las naciones, a pesar de que en Urantia estas siguen gastando inmensas sumas de dinero en preparativos de guerra.
\vs p070 4:2 Los clanes eran grupos unidos por vínculos de sangre pertenecientes a la misma tribu, y debían su existencia a ciertos intereses conjuntos, entre los que se cuentan:
\vs p070 4:3 \li{1.}Su origen, que se remontaba a un ancestro común.
\vs p070 4:4 \li{2.}Su lealtad a un tótem religioso común.
\vs p070 4:5 \li{3.}Hablar el mismo dialecto.
\vs p070 4:6 \li{4.}Compartir un lugar de residencia común.
\vs p070 4:7 \li{5.}Temer a los mismos enemigos.
\vs p070 4:8 \li{6.}Haber tenido una experiencia militar común.
\vs p070 4:9 \pc Los cabecillas de los clanes estaban siempre subordinados al jefe de la tribu; los primeros gobiernos tribales eran una confederación poco compacta de clanes. Los aborígenes australianos nunca desarrollaron una forma tribal de gobierno.
\vs p070 4:10 Los jefes pacíficos de los clanes normalmente gobernaban por línea materna; los jefes guerreros de la tribu implantaron la línea paterna. La corte de los jefes tribales y reyes primitivos la componían los cabecillas de los clanes, a quienes se acostumbraba a invitar a la presencia del rey varias veces al año. Esto le permitía vigilarlos y procurarse mejor su cooperación. Los clanes cumplían un valioso servicio en el autogobierno local, pero retrasaron enormemente el desarrollo de naciones grandes y fuertes.
\usection{5. LOS COMIENZOS DEL GOBIERNO}
\vs p070 5:1 Toda institución humana tuvo un principio, y el gobierno civil es el resultado de una evolución progresiva, al igual que el matrimonio, la industria y la religión. A partir de los primeros clanes y de las tribus primitivas se desarrollaron, gradualmente, las clases sucesivas de gobiernos humanos que han surgido y desaparecido hasta llegar a las formas de regulación civil y social que caracterizan al segundo tercio del siglo XX.
\vs p070 5:2 Con la paulatina aparición de los núcleos familiares, se establecieron las bases del gobierno en la organización del clan, en la agrupación de familias consanguíneas. El primer órgano gubernamental verdadero fue el \bibemph{consejo de los ancianos}. Este grupo regidor se componía de ancianos que se habían distinguido de alguna manera por su eficiencia. Incluso el hombre salvaje apreció tempranamente la sabiduría y la experiencia, y siguió a continuación un largo período en el que el poder estaba en manos de los ancianos. Este reinado de la oligarquía de la edad se convirtió gradualmente en la idea patriarcal.
\vs p070 5:3 En el primitivo consejo de ancianos se hallaba el potencial de todas las funciones gubernamentales: la ejecutiva, la legislativa y la judicial. Cuando el consejo interpretaba las costumbres vigentes, era un tribunal; cuando establecía las nuevas formas de los usos sociales, era el órgano legislativo; en la medida en la que hacía cumplir estos decretos y normativas, era el ejecutivo. El presidente del consejo fue uno de los predecesores del futuro jefe tribal.
\vs p070 5:4 Algunas tribus tenían consejos de mujeres y, ocasionalmente, muchas tribus tenían mujeres como dirigentes. Ciertas tribus del hombre rojo preservaron las enseñanzas de Onamonalontón al seguir la regla de la unanimidad del “consejo de los siete”.
\vs p070 5:5 \pc A la humanidad le ha sido difícil aprender que un círculo de debates no puede gestionar ni la paz ni la guerra. Las primitivas “palabrerías” rara vez fueron de utilidad. La raza humana pronto supo que un ejército comandado por un grupo de cabecillas de clanes no podía hacer nada contra un ejército fuerte bajo el mando de un solo hombre. La guerra siempre ha sido una hacedora de reyes.
\vs p070 5:6 \pc En un primer momento, se elegían a jefes guerreros solo para actuar militarmente; estos renunciaban a parte de su autoridad durante los períodos de paz, cuando sus deberes eran de carácter más social. Si bien, paulatinamente, comenzaron a intervenir en los intervalos de paz, tratando de continuar estar al mando entre guerras. A menudo se aseguraban de que las guerras no tardaran demasiado en seguirse unas a otras. Estos primitivos señores de la guerra no eran partidarios de la paz.
\vs p070 5:7 En tiempos posteriores, se elegían a los jefes para tareas que no fuesen militares; se escogían por su excepcional constitución física o extraordinarias capacidades personales. Con frecuencia, los hombres rojos tenían dos grupos de jefes: los \bibemph{sachems,} o jefes de paz, y los jefes de guerra hereditarios. Los dirigentes de paz eran además jueces y maestros.
\vs p070 5:8 Algunas comunidades primitivas estaban gobernadas por curanderos, que a menudo ejercían como jefes. Un solo hombre actuaba en calidad de sacerdote, médico y jefe del ejecutivo. Con bastante frecuencia, las primeras insignias reales habían sido originariamente los símbolos o emblemas de las vestiduras sacerdotales.
\vs p070 5:9 Y, de manera escalonada, la rama ejecutiva del gobierno se fue constituyendo gradualmente. Los consejos del clan y de la tribu continuaron, aunque con carácter consultivo y como predecesores de las ramas legislativa y judicial, que más tarde harían su aparición. En África, hoy día, todas estas formas de gobierno primitivo están realmente presentes en las diferentes tribus.
\usection{6. EL GOBIERNO MONÁRQUICO}
\vs p070 6:1 El régimen estatal apareció de forma operativa solo cuando el jefe adquirió plenos poderes ejecutivos. El hombre descubrió que solo se lograba un gobierno efectivo cuando se le confería el poder a una persona, y no apoyando una idea.
\vs p070 6:2 La soberanía nació a partir de la idea de la autoridad o de la riqueza de la familia. Cuando un reyezuelo patriarcal se convertía en un verdadero rey, se le llamaba a veces “padre de su pueblo”. Más tarde, se pensó que los reyes procedían de los héroes. E incluso, más adelante, la soberanía llegó a ser hereditaria, debido a la creencia del origen divino de los reyes.
\vs p070 6:3 Con la monarquía hereditaria, se evitó la anarquía que tantos estragos había previamente originado entre la muerte de un rey y la elección de su sucesor. La familia tenía un cabecilla biológico; el clan, un jefe natural elegido; la tribu y, posteriormente, el Estado no tenían ningún líder natural, lo que justificó que se erigieran con carácter hereditario a los jefes\hyp{}reyes. La idea de las familias reales y de la aristocracia también se basó en las costumbres de la “propiedad del nombre” en los clanes.
\vs p070 6:4 Con el tiempo, la sucesión de los reyes llegó a considerarse como algo sobrenatural; se creía que la sangre real se remontaba a los tiempos de la comitiva materializada del príncipe Caligastia. Por ello, los reyes se convirtieron en personas fetiche y se les temió de manera desmesurada, hasta se adoptó una forma particular de hablar para su uso en la corte. Incluso en los últimos tiempos se creía que tocar a un rey curaba las enfermedades, y algunos pueblos de Urantia todavía piensan que sus gobernantes tienen un origen divino.
\vs p070 6:5 El rey fetiche primitivo se mantenía a menudo recluido; se le consideraba demasiado sagrado para ser contemplado a no ser en los días de fiesta y en los días sagrados. Normalmente, se elegía a un representante para hacerse pasar por él, y de ahí el origen de los primeros ministros. El primer jefe del gabinete era el responsable de los alimentos; otros, en breve lo siguieron. Los dirigentes pronto nombraron representantes para hacerse cargo del comercio y la religión; y el desarrollo del gabinete fue un paso que llevó directamente hasta la despersonalización de la autoridad ejecutiva. Estos asistentes de los primeros reyes se convirtieron en la nobleza reconocida, y la esposa del rey ascendió de forma gradual a la dignidad de reina, conforme las mujeres gozaban de mayor estima.
\vs p070 6:6 \pc Algunos dirigentes poco escrupulosos consiguieron un gran poder gracias al descubrimiento del veneno. La magia de las cortes primitivas era diabólica; los enemigos del rey morían pronto. Pero incluso el tirano más despótico estaba supeditado a algunas restricciones; al menos, el miedo constante a ser asesinado lo reprimía. Los curanderos, los hechiceros y los sacerdotes siempre fueron una poderosa sujeción para los reyes. Más adelante, los terratenientes, la aristocracia, tuvieron igualmente un efecto restrictivo. Y, ocasionalmente, los clanes y las tribus simplemente se sublevaban y derrocaban a sus déspotas y tiranos. A los soberanos depuestos, cuando eran condenados a muerte, se les concedía con frecuencia la posibilidad de suicidarse, lo que dio origen a la ancestral moda social de suicidarse en determinadas circunstancias.
\usection{7. CLUBES PRIMITIVOS Y SOCIEDADES SECRETAS}
\vs p070 7:1 El parentesco de sangre estableció los primeros grupos sociales; la asociación amplió el clan de parentesco. Los matrimonios mixtos fueron el siguiente paso en la expansión de los grupos, y la compleja tribu resultante fue el primer órgano político verdadero. El siguiente avance en el desarrollo social fue la evolución de las entidades religiosas y los clubes políticos. Aparecieron primeramente con carácter de sociedades secretas y, originariamente, tenían un carácter enteramente religioso; luego, adquirieron un carácter regulador. En un principio, fueron clubes de hombres; más tarde, aparecieron clubes de mujeres. Al poco tiempo, se dividieron en dos clases: sociopolítico y religioso\hyp{}místico.
\vs p070 7:2 \pc Había muchas razones para el secretismo de estas sociedades, entre las que se cuentan:
\vs p070 7:3 \li{1.}El temor a incurrir en el descontento de los dirigentes por haber violado algún tabú.
\vs p070 7:4 \li{2.}Practicar ritos religiosos minoritarios.
\vs p070 7:5 \li{3.}Preservar valiosa información secreta proveniente de los “espíritus”.
\vs p070 7:6 \li{4.}Disfrutar de algún amuleto o magia especiales.
\vs p070 7:7 \pc El secretismo de estas sociedades otorgaba a todos sus miembros el poder del misterio sobre el resto de la tribu. El secretismo también apelaba a la vanidad; los iniciados eran la aristocracia social de su tiempo. Tras la iniciación, los muchachos podían cazar con los hombres; mientras que antes recogían verduras con las mujeres. Y era una humillación suprema, una deshonra ante la tribu, no lograr pasar las pruebas de la pubertad y verse así obligado a permanecer fuera de la morada de los hombres, junto con las mujeres y los niños, y ser considerado afeminado. Además, a los no iniciados no se les permitía contraer matrimonio.
\vs p070 7:8 \pc Los pueblos primitivos enseñaron muy pronto a sus jóvenes adolescentes a controlar su instinto sexual. Se estableció la costumbre de separar a los muchachos de sus padres desde la pubertad hasta el matrimonio, confiando su educación y formación a las sociedades secretas masculinas. Una de las funciones principales de estos clubes era mantener bajo control a estos jóvenes para así evitar hijos ilegítimos.
\vs p070 7:9 La prostitución comercializada comenzó cuando estos clubes masculinos comenzaron a pagar con dinero la utilización de mujeres de otras tribus. Pero los grupos más primitivos estaban singularmente exentos de laxitud sexual.
\vs p070 7:10 La ceremonia de iniciación de la pubertad solía prolongarse por un período de cinco años. Estas ceremonias incluían bastante tortura autoinfligida y dolorosas incisiones. La circuncisión se practicó primeramente como rito de iniciación en algunas de estas cofradías secretas. Como parte de la iniciación de la pubertad, se grababan en el cuerpo las marcas tribales con incisiones; así se originó el tatuaje como distintivo de membrecía. Estas torturas, junto con muchas privaciones, estaban concebidas para endurecer a estos jóvenes, para imprimir en su mente la realidad de la vida y sus inevitables adversidades. Dicho propósito se logra mejor mediante los juegos atléticos y las competiciones físicas que más tarde aparecerían.
\vs p070 7:11 Pero las sociedades secretas intentaban de hecho mejorar la moralidad de los adolescentes; uno de los propósitos principales de las ceremonias de la pubertad era infundir en el muchacho la idea de que debía dejar en paz a las esposas ajenas.
\vs p070 7:12 Tras estos años de rigurosa disciplina y formación, y justo antes de contraer matrimonio, se solía liberar a los jóvenes para que gozaran de un corto período de ocio y libertad, tras el cual debían regresar para casarse y quedar subordinados durante toda su vida a los tabúes tribales. Y esta costumbre ancestral ha continuado hasta los tiempos modernos en la insensata idea de “echar canas al aire”.
\vs p070 7:13 \pc Muchas tribus posteriores autorizaron la formación de clubes secretos femeninos, cuyo propósito era preparar a las muchachas adolescentes para ser esposas y madres. Tras la iniciación, las jóvenes podían optar por el matrimonio y se les permitía asistir a la “presentación de la novia”, la puesta de largo de aquellos días. Pronto surgieron órdenes de mujeres que hacían promesas contra el matrimonio.
\vs p070 7:14 Al poco tiempo, cuando grupos de hombres solteros y de mujeres no comprometidas formaron sus propias organizaciones, hicieron su aparición los clubes no secretos. Estas asociaciones fueron realmente las primeras escuelas. Y aunque los clubes femeninos y masculinos eran a menudo dados a perseguirse unos a otros, algunas tribus avanzadas, una vez que contactaron con los maestros de Dalamatia, probaron la coeducación, instituyendo escuelas de internado para ambos sexos.
\vs p070 7:15 \pc Las sociedades secretas contribuyeron a la formación de las castas sociales principalmente a causa del carácter misterioso de sus iniciaciones. Los miembros de estas sociedades utilizaron primeramente máscaras para ahuyentar a los curiosos de sus ritos de duelo ---o culto a los ancestros---. Más tarde, este rito se convirtió en una supuesta sesión espiritualista en la cual se decía que aparecían espíritus. Las sociedades antiguas del “nuevo nacimiento” usaban signos y empleaban un lenguaje secreto especial; también renunciaban a determinados alimentos y bebidas. Actuaban como policía nocturna, además de ejercer una amplia variedad de actividades sociales.
\vs p070 7:16 Todas las asociaciones secretas imponían un juramento, ordenaban la confidencialidad y enseñaban a guardar secretos. Estas órdenes atemorizaban y controlaban a las muchedumbres; actuaban, además, como sociedades de vigilancia, aplicando, pues, la ley del linchamiento. Disponían de los primeros espías cuando las tribus estaban en guerra y de los primeros policías secretos en tiempos de paz. Lo más favorable de ellas era que mantenían a los reyes poco escrupulosos en un estado de ansiedad respecto a su futuro. Para contrarrestarlas, los reyes promovieron su propia policía secreta.
\vs p070 7:17 Estas sociedades dieron lugar a los primeros partidos políticos. El primer gobierno de partido fue el de “los fuertes” contra “los débiles”. En tiempos antiguos, solo se cambiaba de gobierno después de una guerra civil, sobrada prueba de que los débiles se habían convertido en fuertes.
\vs p070 7:18 Los comerciantes se servían de estos clubes para cobrar deudas y, los dirigentes, para recaudar impuestos. La tributación conllevó una larga lucha; una de sus primeras formas fue el diezmo o una décima parte de la caza o del botín. Originariamente, se cobraban los impuestos para mantener la casa del rey, pero se descubrió que se recaudaban con mayor facilidad si se enmascaraban bajo la forma de ofrendas para apoyar el servicio del templo.
\vs p070 7:19 En breve plazo, estas asociaciones secretas se convirtieron en las primeras organizaciones caritativas, para luego evolucionar y llegar a ser las primeras sociedades religiosas ---las predecesoras de las iglesias---. Por último, algunas de estas sociedades se hicieron intertribales, pasando a ser las primeras cofradías internacionales.
\usection{8. LAS CLASES SOCIALES}
\vs p070 8:1 La desigualdad mental y física de los seres humanos da pie a la aparición de las clases sociales. Los únicos mundos sin estratos sociales son los más primitivos y los más avanzados. En una civilización incipiente, la diferenciación de los niveles sociales no es todavía un hecho, mientras que en un mundo asentado en luz y vida se han borrado mayormente estas divisiones de la humanidad, que tan características son de todas las etapas evolutivas intermedias.
\vs p070 8:2 Conforme la sociedad iba del estado salvaje al de barbarie, sus integrantes humanos solían agruparse en clases debido a las siguientes razones de índole general:
\vs p070 8:3 \li{1.}\bibemph{Naturales:} Contacto, parentesco y matrimonio; las primeras distinciones sociales se basaron en el sexo, la edad y la sangre ---o parentesco con el jefe---.
\vs p070 8:4 \li{2.}\bibemph{Personales:} El reconocimiento de las capacidades, la resistencia, la destreza y la entereza; pronto seguido por el reconocimiento del dominio del lenguaje, el conocimiento y la inteligencia general.
\vs p070 8:5 \li{3.}\bibemph{Circunstanciales:} La guerra y la emigración dieron como resultado la separación de los grupos humanos. La evolución de las clases se vio fuertemente condicionada por las conquistas, la relación de los vencedores con los vencidos, a la vez que la esclavitud trajo consigo la primera división general de la sociedad en libres y cautivos.
\vs p070 8:6 \li{4.}\bibemph{Económicas:} Los ricos y los pobres. La riqueza y la posesión de esclavos fue la base hereditaria para una de las clases de la sociedad.
\vs p070 8:7 \li{5.}\bibemph{Geográficas:} Algunas clases surgieron como consecuencia de asentamientos urbanos o rurales. La ciudad y el campo, respectivamente, han contribuido a la diferenciación entre el pastor\hyp{}agricultor y el comerciante\hyp{}industrial, con sus distintos puntos de vista y reacciones.
\vs p070 8:8 \li{6.}\bibemph{Sociales:} Algunas clases se fueron formando gradualmente siguiendo la estimación popular del valor social de diferentes grupos. Entre las primeras divisiones de este tipo estaban las distinciones entre sacerdotes\hyp{}maestros, dirigentes\hyp{}guerreros, capitalistas\hyp{}comerciantes, obreros comunes y esclavos. El esclavo nunca podía convertirse en capitalista, aunque algunas veces el asalariado podía optar por unirse a las filas capitalistas.
\vs p070 8:9 \li{7.}\bibemph{Vocacionales:} A medida que las vocaciones se multiplicaban, se tendía a establecer castas y gremios. Los trabajadores se dividieron en tres grupos: las clases profesionales, incluyendo a los curanderos, luego los obreros especializados, seguidos por los obreros no especializados.
\vs p070 8:10 \li{8.}\bibemph{Religiosas:} Los primeros clubes de tipo religioso generaron sus propias clases dentro de los clanes y las tribus, y la piedad y el misticismo de los sacerdotes las perpetuaron durante mucho tiempo como un grupo social separado.
\vs p070 8:11 \li{9.}\bibemph{Raciales:} La presencia de dos o más razas dentro de una nación o unidad territorial determinada da origen a las castas de color. El sistema original de castas de la India se basaba en el color, tal como sucedió en el antiguo Egipto.
\vs p070 8:12 \li{10.}\bibemph{Edad:} La juventud y la madurez. Entre las tribus, el niño permanecía bajo la protección de su padre mientras que este estuviese vivo; si bien, la niña quedaba al cuidado de su madre hasta que contrajera matrimonio.
\vs p070 8:13 \pc Es imprescindible para una sociedad en desarrollo que las clases sociales sean flexibles y cambiantes; pero cuando la \bibemph{clase} se convierte en \bibemph{casta,} cuando los niveles sociales se anquilosan, la mejora de la estabilidad social se adquiere mermando la iniciativa personal. La casta social resuelve el problema de hallar cada cual su lugar en el sector laboral, pero también restringe severamente el desarrollo individual y prácticamente impide la cooperación social.
\vs p070 8:14 Habiéndose formado naturalmente, las clases sociales persistirán hasta que el hombre logre paulatinamente su desaparición de forma evolutiva mediante la actuación inteligente sobre las fuentes biológicas, intelectuales y espirituales de una civilización en progreso, de las siguientes formas:
\vs p070 8:15 \li{1.}La renovación biológica de los linajes raciales: la exclusión selectiva de las variedades humanas menos dotadas. Esto tenderá a erradicar muchas desigualdades humanas.
\vs p070 8:16 \li{2.}La formación educativa de la mayor capacidad cerebral que surgirá de esta mejora biológica.
\vs p070 8:17 \li{3.}El avivamiento religioso de los sentimientos de parentesco y hermandad humanos.
\vs p070 8:18 \pc Si bien, estas medidas solo pueden dar sus verdaderos frutos en los remotos milenios del futuro, aunque resultará una gran e inmediata mejora social si se actúa de forma inteligente, prudente y \bibemph{paciente} sobre estos elementos aceleradores del progreso cultural. La religión es la palanca poderosa que alza a la civilización por encima del caos, pero que no tiene ningún efecto sin el punto de apoyo de una mente sana y normal, que repose firmemente sobre una herencia sana y normal.
\usection{9. LOS DERECHOS HUMANOS}
\vs p070 9:1 La naturaleza no confiere al hombre ningún derecho, sino solamente la vida y un mundo en el que vivirla. La naturaleza no confiere ni siquiera el derecho a vivir, tal como se puede desprender de lo que probablemente le sucedería a un hombre desarmado que se encontrara frente a frente con un tigre hambriento en un bosque primitivo. La sociedad otorga al hombre un don fundamental: la seguridad.
\vs p070 9:2 \pc Paulatinamente la sociedad ha hecho valer sus derechos, que, en la actualidad, son los siguientes:
\vs p070 9:3 \li{1.}Garantía en el suministro de alimentos.
\vs p070 9:4 \li{2.}Defensa militar: seguridad mediante el estado de preparación.
\vs p070 9:5 \li{3.}Mantenimiento de la paz interna: prevención de la violencia personal y el desorden social.
\vs p070 9:6 \li{4.}Regulación sexual: matrimonio, la institución de la familia.
\vs p070 9:7 \li{5.}Propiedad: derecho a poseer.
\vs p070 9:8 \li{6.}Fomento de la competitividad individual y grupal.
\vs p070 9:9 \li{7.}Medidas para la educación y la formación de la juventud.
\vs p070 9:10 \li{8.}Promoción del intercambio y del comercio: desarrollo industrial.
\vs p070 9:11 \li{9.}Mejora de las condiciones y las remuneraciones laborales.
\vs p070 9:12 \li{10.}Garantía de la libertad de las prácticas religiosas con el fin de que todas estas otras actividades sociales se enaltezcan al estar espiritualmente motivadas.
\vs p070 9:13 \pc Cuando los derechos son tan antiguos que se desconocen sus orígenes, a menudo se les llama \bibemph{derechos naturales}. Pero, en realidad, los derechos humanos no son naturales; son enteramente sociales. Son relativos y en constante cambio; no son sino parte de las reglas del juego: adaptaciones aceptadas de las relaciones que rigen los siempre cambiantes fenómenos de la competitividad humana.
\vs p070 9:14 Lo que se puede considerar como un derecho en una era, puede que no lo sea en otra. La supervivencia de un gran número de personas con deficiencias y en declive degenerativo no se debe a que tengan el derecho natural de dificultar la civilización del siglo XX, sino simplemente porque la sociedad de la época, las costumbres, así lo decreta.
\vs p070 9:15 En la Europa de la Edad Media, se reconocían pocos derechos humanos; entonces, cualquier hombre pertenecía a otro, y los derechos no eran sino privilegios o favores concedidos por la Iglesia o el Estado. Y rebelarse contra este error fue igualmente erróneo al dar lugar a la convicción de que todos los hombres nacen iguales.
\vs p070 9:16 El débil y el peor dotado siempre han luchado por la igualdad de derechos; siempre han insistido en que el Estado obligue al fuerte y al mejor dotado a satisfacer sus carencias y compensar además aquellas deficiencias, que con demasiada frecuencia son el resultado natural de su propia indiferencia e indolencia.
\vs p070 9:17 Pero este ideal de igualdad es el fruto de la civilización; no se halla en la naturaleza. Incluso la cultura misma demuestra, de forma concluyente, la desigualdad innata a los hombres mediante sus muy desiguales capacidades para esta. La consecución repentina y no evolutiva de la supuesta igualdad natural volvería a precipitar rápidamente al hombre civilizado a los usos rudimentarios de las eras primitivas. La sociedad no puede ofrecer los mismos derechos a todos, pero sí puede comprometerse a administrar los distintos derechos de cada cual con justicia y equidad. Es competencia y deber de la sociedad proporcionar al hijo de la naturaleza una oportunidad justa y pacífica de buscar su propio sostenimiento, de participar en la perpetuación de sí mismo, mientras disfruta al mismo tiempo de cierto grado de autocomplacencia; la suma de estos tres factores constituye la felicidad humana.
\usection{10. EVOLUCIÓN DE LA JUSTICIA}
\vs p070 10:1 La justicia natural es una teoría formulada por el hombre; no es una realidad. En la naturaleza, la justicia es puramente teórica, una completa ficción. La naturaleza no ofrece más que un tipo de justicia: inevitable atribución de los resultados a las causas.
\vs p070 10:2 La justicia, tal como la concibe el hombre, significa conseguir los derechos que le son propios y es, por tanto, una cuestión de evolución progresiva. El concepto de justicia podría ser muy bien esencial en una mente dotada de espíritu, pero no aparece con toda su plenitud en los mundos del espacio.
\vs p070 10:3 El hombre primitivo atribuía todos los fenómenos a una persona. En caso de muerte, el salvaje no se preguntaba \bibemph{qué} lo había matado, sino \bibemph{quién}. El asesinato accidental no estaba, por tanto, reconocido y, cuando se castigaba un delito, se desestimaba completamente el móvil del delincuente; se dictaba la sentencia de acuerdo con el daño ocasionado.
\vs p070 10:4 \pc En la sociedad más primitiva, la opinión pública actuaba directamente; no se necesitaban agentes de la ley. En la vida primitiva no había privacidad. Los vecinos de un hombre eran responsables por su conducta; de ahí el derecho que tenían a entrometerse en sus asuntos personales. La sociedad se regía por el principio de que los miembros del grupo debían interesarse, y tener algún grado de control, respecto al comportamiento de cada cual.
\vs p070 10:5 Muy pronto se creyó que los espíritus administraban la justicia por medio de los curanderos y sacerdotes; lo que hizo que estos colectivos se erigieran como los primeros investigadores de delitos y agentes de la ley. Sus métodos primitivos de detección de los delitos consistían en efectuar ordalías de veneno, fuego y dolor. Estas despiadadas pruebas no eran más que rudos métodos de arbitraje; no resolvían necesariamente un litigio con justicia. Por ejemplo, cuando se le administraba veneno a un acusado, si este vomitaba, es que era inocente.
\vs p070 10:6 En el Antiguo Testamento hay constancia de una de estas ordalías en relación a una prueba de culpabilidad conyugal: si un hombre sospechaba que su esposa lo engañaba, la llevaba ante el sacerdote y expresaba sus sospechas, tras lo cual el sacerdote preparaba un brebaje que consistía en agua bendita y barreduras del suelo del templo. Después de la debida ceremonia, que incluía maldiciones amenazantes, a la acusada se le hacía beber la desagradable poción. Si era culpable, que “el agua que da maldición entre en ella y se vuelva amarga y haga hinchar su vientre y pudrirse sus muslos, y sea objeto de maldición en medio de su pueblo”. Si resultara que alguna mujer bebiese este repugnante brebaje y no manifestase síntomas de enfermedad física, se la absolvía de las acusaciones vertidas por su celoso esposo.
\vs p070 10:7 Estos atroces instrumentos de detección de los delitos se practicaron en algún momento u otro en todas las tribus en su curso de evolución. Batirse en duelo fue la continuidad moderna del juicio por medio de las ordalías.
\vs p070 10:8 No es de extrañar que los hebreos y otras tribus semi\hyp{}civilizadas practicaran hace tres mil años estos métodos tan primitivos de administrar la justicia, pero es muy sorprendente que hubiese hombres juiciosos que llegaran a retener tal reliquia de la barbarie en las páginas de una recopilación de escrituras sagradas. El pensar reflexivo debe dejar claro que ningún ser divino le dio jamás al hombre instrucciones tan injustas para la detección y enjuiciamiento de supuestas infidelidades matrimoniales.
\vs p070 10:9 \pc Pronto, la sociedad adoptó la actitud de tomar represalias: ojo por ojo, vida por vida. Todas las tribus en vías de evolución reconocían este derecho a la venganza de sangre, y esta se convirtió en el objetivo de la vida primitiva; no obstante, desde entonces, la religión ha modificado significativamente estas prácticas tribales primitivas. Los maestros de la religión revelada siempre han proclamado: “'Mía es la venganza', dice el Señor’”. Las muertes por venganza de los tiempos primitivos no eran muy distintas a los asesinatos que se cometen hoy en día bajo el pretexto de la ley no escrita.
\vs p070 10:10 El suicidio era una forma común de represalia. Si alguien no podía vengarse en vida, moría albergando la creencia de que, bajo la forma de un espíritu, podría regresar y desatar su ira sobre el enemigo. Y puesto que esta creencia era de carácter muy general, la amenaza de suicidarse ante el domicilio de un enemigo era suficiente para hacerle llegar a un acuerdo. El hombre primitivo no tenía la vida en alta estima; el suicidio a causa de nimiedades era algo común, pero las enseñanzas de los dalamatianos redujo esta costumbre de forma considerable; además, en tiempos más recientes, se sumaron el ocio, las comodidades, la religión y la filosofía para endulzar la vida y hacerla más gratificante. No obstante, las huelgas de hambre son, en épocas modernas, un equivalente de este antiguo método de represalia.
\vs p070 10:11 Uno de los más tempranos pronunciamientos de la desarrollada ley tribal tenía que ver con la asunción de las reyertas familiares como asuntos tribales. Pero resulta extraño que, incluso en aquel momento, un hombre podía matar a su esposa sin ser castigado siempre que hubiese pagado totalmente por ella. Los esquimales de hoy, sin embargo, todavía permiten que sea la familia agraviada la que dicte y administre el castigo por el delito, incluso en caso de un asesinato.
\vs p070 10:12 Otro avance fue la imposición de multas por violar los tabúes, la estipulación de penalizaciones. Estas multas representaron los primeros ingresos públicos. La costumbre del “precio de la sangre” como sustituto de la venganza de sangre se convirtió en una práctica habitual. Estos daños y perjuicios se pagaban por lo general en mujeres o ganado; pasaría mucho tiempo antes de que las verdaderas multas, la indemnización monetaria, se consideraran como castigo por los delitos. Y puesto que el castigo era algo esencialmente retributivo, todas las cosas, incluyendo la vida humana, acabaron por tener un precio a pagar por los daños causados. Los hebreos fueron los primeros en abolir la práctica de pagar el precio de la sangre. Moisés impartió la enseñanza de que no debían “aceptar rescate por la vida del homicida, porque está condenado a muerte: indefectiblemente morirá”.
\vs p070 10:13 \pc Así pues, la familia impartió primeramente la justicia, luego fue el clan y, más tarde, la tribu. La administración de la verdadera justicia se remonta al momento en el que se desposeyó la venganza de las manos de grupos privados y consanguíneos y se depositó en las del grupo social, en las del Estado.
\vs p070 10:14 \pc En cierto momento, fue una práctica común quemar vivo a alguien en la hoguera como castigo. Muchos dirigentes de la antigüedad aprobaban esta práctica, incluidos Hammurabi y Moisés, que prescribió que muchos delitos, en particular los de carácter sexual grave, se castigaran quemando al culpable en la hoguera. Si “la hija del sacerdote” o de otro destacado ciudadano se daba a la prostitución, la costumbre hebrea era “quemarla al fuego”.
\vs p070 10:15 La traición ---el “venderse” o la traición a los propios compañeros de tribu--- fue el primer delito capital. El robo de ganado se castigaba de manera generalizada con la muerte sumaria y, hasta hace poco, se ha castigado el hurto de caballos de forma semejante. Pero, a medida que pasaba el tiempo, se comprobó que la severidad del castigo no era un elemento disuasorio en la comisión de delitos tan efectivo como lo era su certeza y prontitud.
\vs p070 10:16 Cuando una sociedad no logra castigar los delitos, la indignación colectiva suele imponerse cometiendo linchamientos; la creación de refugios sirvió como medio para escapar de esta repentina cólera comunitaria. El linchamiento y el duelo representan la falta de disposición del agraviado para renunciar a su resarcimiento personal en favor del Estado.
\usection{11. LEYES Y TRIBUNALES}
\vs p070 11:1 Resulta tan difícil establecer una clara distinción entre costumbres y leyes como lo es indicar con exactitud en qué momento del alba a la noche le sucede el día. Las costumbres son leyes y normas policiales en potencia. Cuando llevan mucho tiempo instituidas, las costumbres no escritas tienden a cristalizarse en leyes precisas, normas concretas y convenciones sociales bien definidas.
\vs p070 11:2 En un principio, la ley siempre es negativa y prohibitiva; en las civilizaciones en avance se hace cada vez más positiva y directiva. La sociedad primitiva se regulaba de manera negativa; otorgaba a las personas el derecho a vivir imponiendo sobre todos los demás el mandamiento, “no matarás”. Toda concesión de derechos o libertades de forma individual entraña el recorte de libertades a todos los demás, y esto se lleva a cabo mediante el tabú, la ley primitiva. Toda la idea del tabú es en sí misma negativa, porque la organización de la sociedad fue enteramente negativa, y la temprana administración de la justicia estribaba en la observancia de los tabúes. Si bien, inicialmente estas leyes se aplicaban solamente a los miembros de la tribu, tal como se ilustra en los hebreos de los últimos días, que tenían un diferente código de ética para tratar con los gentiles.
\vs p070 11:3 El juramento se originó en los tiempos de Dalamatia con la intención de hacer que los testimonios fuesen más veraces. Tales juramentos consistían en pronunciar una maldición contra uno mismo. Antiguamente, nadie quería testificar en contra de su grupo nativo.
\vs p070 11:4 \pc La delincuencia era un ataque a las costumbres tribales, el pecado era transgresión de aquellos tabúes que gozaban de la aprobación de los espíritus, y se dio una prolongada confusión por el error de no haber separado delincuencia y pecado.
\vs p070 11:5 El interés personal estableció el tabú del asesinato, la sociedad lo legitimó como costumbre tradicional, mientras que la religión consagró esta costumbre como ley moral; y, de este modo, los tres contribuyeron a hacer la vida humana más segura y sagrada. La sociedad no se hubiese podido mantener unida durante estos tempranos tiempos si los derechos no hubiesen tenido la aprobación de la religión; la superstición fue el órgano policial moral y social de las largas eras evolutivas. Todos los antiguos aseguraban que sus ancestros habían recibido las viejas leyes, los tabúes, de mano de los dioses.
\vs p070 11:6 La ley es un relato codificado de la dilatada experiencia humana, de la opinión pública cristalizada y legalizada. Las costumbres eran la materia prima de la experiencia acumulada, de la cual las mentes dirigentes formulaban las leyes escritas. Los jueces antiguos no disponían de leyes. Cuando emitían una decisión, simplemente decían: “Es la costumbre”.
\vs p070 11:7 La referencia a la jurisprudencia en las decisiones judiciales representa el esfuerzo de los jueces para adaptar las leyes escritas a las condiciones cambiantes de la sociedad. Esto prevé una adaptación progresiva a tales variables condiciones junto al efecto de las continuadas tradiciones.
\vs p070 11:8 \pc Los litigios sobre la propiedad se trataban de muchas maneras, entre las que están las siguientes:
\vs p070 11:9 \li{1.}Destruyendo la propiedad en disputa.
\vs p070 11:10 \li{2.}Mediante la fuerza: los litigantes luchaban entre sí.
\vs p070 11:11 \li{3.}Mediante el arbitraje: un tercero decidía.
\vs p070 11:12 \li{4.}Mediante la apelación a los ancianos ---posteriormente se llevarían a los tribunales---.
\vs p070 11:13 \pc Los primeros tribunales venían a ser enfrentamientos con los puños de manera regulada; los jueces eran simplemente evaluadores o árbitros. Se ocupaban de que la lucha se llevara a cabo de acuerdo con unas normas aprobadas. Al iniciar la pelea ante los jueces, cada parte tenía que dejar una fianza al juez para pagar los gastos y la multa tras haber vencido uno al otro. “La fuerza aún llevaba la razón”. Más tarde, los pleitos verbales sustituyeron a los golpes físicos.
\vs p070 11:14 Toda la idea de la justicia primitiva no era tanto ser justo como solucionar la disputa y evitar así el desorden público y la violencia privada. Pero el hombre primitivo no se indignaba tanto ante lo que ahora se consideraría como una injusticia; se daba por hecho que los que tenían el poder lo usarían de manera interesada. No obstante, se puede determinar, con bastante precisión, la condición de cualquier civilización mediante el rigor y la equidad de sus tribunales y mediante la integridad de sus jueces.
\usection{12. DISTRIBUCIÓN DE LA AUTORIDAD CIVIL}
\vs p070 12:1 La gran pugna que ha existido en torno a la evolución del gobierno ha sido a causa de la concentración del poder. Los administradores del universo han aprendido por experiencia que los pueblos evolutivos en los mundos habitados se rigen mejor siguiendo el sistema representativo de gobierno civil cuando hay un adecuado equilibrio de poderes entre las bien coordinadas ramas ejecutiva, legislativa y judicial.
\vs p070 12:2 \pc Aunque la autoridad primitiva se basaba en la fuerza, en el poder físico, el gobierno ideal es el sistema representativo en el que el liderazgo se basa en la capacidad; si bien, en los días de barbarie había, en todos los aspectos, demasiadas guerras como para posibilitar la instauración de un gobierno representativo efectivo. En la larga lucha entre la división de la autoridad y la unidad de mando, ganó el dictador. Los tempranos y difusos poderes de los primeros consejos de ancianos se fueron concentrando en la persona de un monarca absoluto. Tras la llegada de los verdaderos reyes, los grupos de ancianos persistieron en calidad de órganos asesores cuasi legislativos y judiciales; más adelante, hicieron su aparición las cámaras legislativas de carácter igualitario y, con el tiempo, se establecieron, separados de las cámaras legislativas, los tribunales supremos.
\vs p070 12:3 El rey hacía respetar las costumbres, la ley original o no escrita. Luego hizo cumplir las medidas legislativas, cristalización de la opinión pública. Aunque lenta en hacer su aparición, la asamblea popular como expresión de la opinión pública significó un gran avance social.
\vs p070 12:4 Los primeros reyes se encontraron considerablemente limitados por las costumbres ---por la tradición o la opinión pública---. En los últimos tiempos, algunas naciones de Urantia han codificado estas costumbres en unas bases documentales de gobierno.
\vs p070 12:5 \pc Los mortales de Urantia tienen derecho a la libertad; deben crear sus sistemas de gobierno; deben adoptar sus constituciones u otros estatutos relativos a la autoridad civil y a los procedimientos administrativos. Y, habiendo hecho esto, deben seleccionar entre ellos a los más competentes y dignos como jefes del ejecutivo. Como representantes de la rama legislativa, deben elegir solo aquellos que estén intelectual y moralmente capacitados para desempeñar estas sagradas responsabilidades. Para jueces de sus tribunales superiores y supremos, solamente se deben optar por aquellos dotados de habilidad natural y cuya sabiduría esté basada en una profunda experiencia.
\vs p070 12:6 Para que los hombres mantengan su libertad, una vez que han escogido sus estatutos de libertad, deben asegurarse de que se interpreten de forma sensata, inteligente y valiente con el fin de impedir:
\vs p070 12:7 \li{1.}La usurpación de poder injustificado por parte de las ramas ejecutiva o legislativa.
\vs p070 12:8 \li{2.}Las maquinaciones de agitadores ignorantes y supersticiosos.
\vs p070 12:9 \li{3.}El retraso del progreso científico.
\vs p070 12:10 \li{4.}El estancamiento por el dominio de la mediocridad.
\vs p070 12:11 \li{5.}El dominio por parte de minorías despiadadas.
\vs p070 12:12 \li{6.}El control por parte de ambiciosos y astutos dictadores en potencia.
\vs p070 12:13 \li{7.}El catastrófico trastorno del pánico.
\vs p070 12:14 \li{8.}La explotación por parte de personas sin escrúpulos.
\vs p070 12:15 \li{9.}La esclavitud tributaria de la ciudadanía por parte del Estado.
\vs p070 12:16 \li{10.}La falta de justicia social y económica.
\vs p070 12:17 \li{11.}La unión de la Iglesia y el Estado.
\vs p070 12:18 \li{12.}La pérdida de la libertad personal.
\vs p070 12:19 \pc Estos son los objetivos y fines de los tribunales constitucionales que actúan en los mundos evolutivos como reguladores de la maquinaria del gobierno representativo.
\vs p070 12:20 La lucha de la humanidad para perfeccionar el gobierno en Urantia guarda relación con la optimización de los cauces administrativos, con la adaptación de estos a las necesidades presentes siempre cambiantes, con el mejoramiento de la distribución del poder dentro del gobierno y, finalmente, con la selección de unos líderes en el ámbito de la administración, que sean verdaderamente sabios. Aunque exista una forma divina e ideal de gobierno, esta no puede ser revelada sino que han de descubrirla, lenta y laboriosamente, los hombres y mujeres de cada planeta de todos los universos del tiempo y el espacio.
\vsetoff
\vs p070 12:21 [Exposición de un melquisedec de Nebadón.]
