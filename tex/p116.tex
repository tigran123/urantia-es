\upaper{116}{El Todopoderoso Supremo}
\author{Mensajero poderoso}
\vs p116 0:1 Si el hombre reconociera que sus creadores ---sus inmediatos supervisores--- aunque divinos son también finitos, y que el Dios del tiempo y del espacio es una Deidad evolutiva y no absoluta, las incoherencias de las desigualdades temporales cesarían de ser profundas paradojas religiosas. La fe religiosa dejaría de ser degradada con el fomento de la petulancia social de los afortunados, mientras sirve solo para propiciar la resignación estoica en las desafortunadas víctimas de las privaciones sociales.
\vs p116 0:2 Cuando se observan las esferas espléndidamente perfectas de Havona, es a la vez razonable y lógico creer que las hizo un Creador perfecto, infinito y absoluto. Pero esa misma razón y lógica llevaría a cualquier ser honesto a concluir, al percibir la confusión, las imperfecciones e injusticias de Urantia, que vuestro mundo se hizo, y se gestionó, por creadores subabsolutos, preinfinitos e imperfectos.
\vs p116 0:3 \pc El crecimiento experiencial supone una alianza criatura\hyp{}Creador ---Dios y hombre en vinculación---. El crecimiento es la marca distintiva de la Deidad experiencial: Havona no creció; Havona es y siempre ha sido; es existencial como los sempiternos Dioses de los que procede. Si bien, el crecimiento es característico del gran universo.
\vs p116 0:4 El Todopoderoso Supremo es una Deidad viva y evolutiva de potencia y ser personal. Su ámbito actual, el gran universo, es también un dominio de crecimiento de la potencia y del ser personal. Su destino es la perfección, pero su experiencia presente engloba elementos de desarrollo y de estatus incompleto.
\vs p116 0:5 \pc El Ser Supremo obra primariamente en el universo central como espíritu\hyp{}persona; secundariamente, en el gran universo, como Dios Todopoderoso, un ser personal de potencia. La función terciaria del Supremo en el universo matriz está ahora latente, existiendo solamente como potencial mental desconocido. Nadie sabe exactamente qué desvelará este tercer desarrollo del Ser Supremo. Algunos creen que, cuando se asienten los universos en luz y vida, el Supremo se volverá operativo desde Uversa y actuará como el soberano todopoderoso y experiencial del gran universo, expandiéndose, al mismo tiempo, en potencia como el excelso todopoderoso de los universos exteriores. Otros especulan que la tercera etapa de la Supremacía entrañará el tercer nivel de manifestación de la Deidad. Pero ninguno de nosotros lo sabe realmente.
\usection{1. LA MENTE SUPREMA}
\vs p116 1:1 La experiencia de cualquier ser personal creatural en evolución es una faceta de la experiencia del Todopoderoso Supremo. El sometimiento inteligente de cualquier segmento físico de los suprauniversos forma parte de su creciente dominio. La síntesis creativa de la potencia y del ser personal es, a la vez, parte del impulso creativo de la Mente Suprema y constituye la esencia misma del crecimiento evolutivo de la unidad en el Ser Supremo.
\vs p116 1:2 La unión de los atributos de la potencia y del ser personal de la Supremacía es labor de la Mente Suprema; y la completa evolución del Todopoderoso Supremo dará como resultado una Deidad unificada y personal ---no una agrupación laxamente coordinada de atributos divinos---. Desde una perspectiva más amplia, no habrá Todopoderoso al margen del Supremo, ni Supremo al margen del Todopoderoso.
\vs p116 1:3 A través de las eras evolutivas, el potencial de la potencia física del Supremo recae en los siete directores supremos de la potencia, y el potencial mental reside en los siete espíritus mayores. La Mente Infinita es labor del Espíritu Infinito; la mente cósmica, el ministerio de los siete espíritus mayores; la Mente Suprema está en proceso de actualizarse en la coordinación del gran universo y en conjunción operativa con la revelación y la consecución del Dios Séptuplo.
\vs p116 1:4 \pc La mente espacio\hyp{}temporal, la mente cósmica, actúa de forma diferente en los siete suprauniversos, pero se coordina en el Ser Supremo mediante algún modo de vinculación desconocido. La acción directiva del Todopoderoso sobre el gran universo no es exclusivamente física y espiritual. En los siete suprauniversos es primordialmente material y espiritual, pero se dan igualmente fenómenos del Supremo que son a la vez intelectuales y espirituales.
\vs p116 1:5 En realidad, sabemos menos de la mente de la Supremacía que de cualquier otro aspecto de esta Deidad evolutiva. Está incuestionablemente activa en todo el gran universo, y se cree que tiene un destino potencial de carácter operativo en el universo matriz, que reviste una gran envergadura. Pero hay algo que sí sabemos: mientras que lo físico puede alcanzar un crecimiento completo, y mientras que el espíritu puede conseguir un desarrollo perfecto, la mente jamás deja de progresar ---es el método experiencial del progreso sin fin---. El Supremo es una Deidad experiencial y, en consecuencia, nunca llega a realizarse mentalmente por completo.
\usection{2. EL TODOPODEROSO Y EL DIOS SÉPTUPLO}
\vs p116 2:1 La aparición de la presencia del Todopoderoso en el universo viene acompañada por la aparición, en el escenario de la acción cósmica, de los elevados creadores y rectores de los suprauniversos evolutivos.
\vs p116 2:2 El Dios Supremo deriva sus atributos espirituales y personales de la Trinidad del Paraíso, pero está actualizando su potencia en las acciones de los hijos creadores, de los ancianos de días y de los espíritus mayores, cuyos actos colectivos son la fuente de su creciente potencia como soberano todopoderoso para y en los siete suprauniversos.
\vs p116 2:3 \pc La Deidad incondicionada del Paraíso es incomprensible para las criaturas evolutivas del tiempo y del espacio. La eternidad y la infinitud designan un nivel de realidad en cuanto deidad que las criaturas espacio\hyp{}temporales no pueden entender. La infinitud de la deidad y la absolutidad de la soberanía son intrínsecas a la Trinidad del Paraíso, y la Trinidad es una realidad que se halla de alguna manera más allá de la comprensión del hombre mortal. Las criaturas espacio\hyp{}temporales deben tener orígenes, relatividades y destinos para captar las relaciones en el ámbito del universo y comprender los valores significativos de la divinidad. Por consiguiente, la Deidad del Paraíso atenúa y delimita por otra parte las manifestaciones personales de la divinidad fuera del Paraíso, trayendo con ello a la existencia a los creadores supremos y a sus colaboradores, que constantemente llevan la luz de la vida cada vez más lejos de su fuente en el Paraíso, hasta que encuentra su expresión más distante y bella en las vidas terrenales de los hijos de gracia de los mundos evolutivos.
\vs p116 2:4 Y este es el origen del Dios Séptuplo, cuyos sucesivos niveles el hombre mortal encuentra en el siguiente orden:
\vs p116 2:5 \li{1.}Los hijos creadores (y los espíritus creativos).
\vs p116 2:6 \li{2.}Los ancianos de días.
\vs p116 2:7 \li{3.}Los siete espíritus mayores.
\vs p116 2:8 \li{4.}El Ser Supremo.
\vs p116 2:9 \li{5.}El Actor Conjunto.
\vs p116 2:10 \li{6.}El Hijo Eterno.
\vs p116 2:11 \li{7.}El Padre Universal.
\vs p116 2:12 \pc Los tres primeros niveles son los creadores supremos; los últimos tres, las Deidades del Paraíso. El Supremo interviene continuamente como manifestación personal espiritual y experiencial de la Trinidad del Paraíso y como centro de convergencia experiencial de la todopoderosa potencia evolutiva de los hijos creadores de las Deidades del Paraíso. El Ser Supremo es la revelación máxima de la Deidad a los siete suprauniversos y para la era presente del universo.
\vs p116 2:13 Mediante la lógica de los mortales, se podría inferir que la reunificación experiencial de los actos colectivos de los primeros tres niveles del Dios Séptuplo equivaldrían al nivel de la Deidad del Paraíso, pero esto no es así. La Deidad del Paraíso es Deidad \bibemph{existencial}. Los creadores supremos, en su unidad divina de potencia y ser personal, constituyen y expresan un nuevo potencial de la potencia como Deidad \bibemph{experiencial}. Y este potencial de origen experiencial encuentra inevitable e ineludible su unión con la Deidad experiencial de origen Trinitario: el Ser Supremo.
\vs p116 2:14 El Dios Supremo no es la Trinidad del Paraíso, tampoco es ninguno ni todos esos creadores de los suprauniversos, con cuya capacidad de acción realmente sintetizan su todopoderosa potencia evolutiva. El Dios Supremo, aunque se origine en la Trinidad, se pone de manifiesto a las criaturas evolutivas como ser personal de potencia solo a través de la actuación coordinada de los primeros tres niveles del Dios Séptuplo. El Todopoderoso Supremo se hace efectivo ahora en el tiempo y en el espacio por medio de la actividad de los seres personales creadores supremos, tal como, en la eternidad, el Actor Conjunto se convirtió de modo instantáneo en un ser por la voluntad del Padre Universal y del Hijo Eterno. Estos seres de los primeros tres niveles del Dios Séptuplo son la naturaleza misma y la fuente de la potencia del Todopoderoso Supremo; por lo tanto, deben siempre acompañar y respaldar sus actos administrativos.
\usection{3. EL TODOPODEROSO Y LA DEIDAD DEL PARAÍSO}
\vs p116 3:1 Las Deidades del Paraíso no solo actúan de forma directa en sus vías circulatorias de la gravedad por todo el gran universo, sino que también desempeñan su actividad a través de distintas instancias intermedias y de otras manifestaciones, tales como:
\vs p116 3:2 \li{1.}\bibemph{Los centros de convergencias mentales de la Tercera Fuente y Centro}. Los ámbitos finitos de la energía y del espíritu se mantienen unidos literalmente mediante las presencias mentales del Actor Conjunto. Esto es verdad desde el espíritu creativo del universo local hasta los espíritus mayores del gran universo, pasando por los espíritus reflectores del suprauniverso. Las vías circulatorias mentales que emanan de estos diversos centros de convergencias de la inteligencia representan el entorno cósmico de la elección creatural. La mente es la realidad flexible sobre la cual criaturas y creadores pueden actuar con facilidad; es el eslabón vital que conecta la materia y el espíritu. La dádiva de la mente por parte de la Tercera Fuente y Centro unifica el espíritu\hyp{}persona del Dios Supremo con la potencia experiencial del Todopoderoso evolutivo.
\vs p116 3:3 \li{2.}\bibemph{Las revelaciones personales de la Segunda Fuente y Centro}. Las presencias mentales del Actor Conjunto unifican el espíritu de la divinidad con el modelo energético. Las encarnaciones de gracia del Hijo Eterno y de sus hijos del Paraíso unifican, en realidad fusionan, la naturaleza divina de un creador con la naturaleza evolutiva de una criatura. El Supremo es a la vez criatura y creador; la posibilidad de ser así se revela en los actos de gracia del Hijo Eterno y de sus hijos homólogos y de menor rango de estos. Los órdenes de filiación de gracia ---los migueles y los avonales--- engrandecen de hecho sus naturalezas divinas con genuinas naturalezas creaturales, que se han convertido en suyas al vivir la vida real de las criaturas de los mundos evolutivos. Cuando la divinidad se vuelve humanidad, consustancial a esta relación, existe la posibilidad de que la humanidad pueda volverse divina.
\vs p116 3:4 \li{3.}\bibemph{Las presencias interiores de la Primera Fuente y Centro}. La mente unifica las causalidades espirituales con las reacciones energéticas; el ministerio de gracia unifica el descenso de la divinidad con el ascenso de las criaturas; y las fracciones interiores del Padre Universal ciertamente unifican la criatura evolutiva con el Dios del Paraíso. Hay muchas presencias similares del Padre que moran en numerosos órdenes de seres personales y, en los hombres mortales, estas fracciones divinas de Dios son los modeladores del pensamiento. Los mentores misteriosos son para los seres humanos lo que la Trinidad del Paraíso es para el Ser Supremo. Los modeladores son unos cimientos de carácter absoluto y, sobre tales cimientos, la libre elección puede propiciar la evolución de la realidad divina de una naturaleza eternizada, la naturaleza de un finalizador en el caso del hombre, de una naturaleza en cuanto Deidad en el Dios Supremo.
\vs p116 3:5 \pc Los ministerios de gracia creaturales de los órdenes de filiación del Paraíso posibilitan que estos hijos divinos enriquezcan sus personas al adquirir la naturaleza real de las criaturas del universo, mientras que dichos ministerios indefectiblemente revelan a las criaturas mismas la ruta del Paraíso que las llevará a lograr la divinidad. La dádiva de los modeladores por parte del Padre Universal le permite atraer hacia sí el ser personal de las criaturas volitivas. Y, a lo largo de todas estas relaciones que se dan en los universos finitos, el Actor Conjunto es la fuente por siempre presente del ministerio mental en virtud de la que tiene lugar esta actividad.
\vs p116 3:6 De esta y de muchas otras formas, las Deidades del Paraíso participan en las evoluciones del tiempo a medida que estas se despliegan en los planetas giratorios del espacio, y a medida que culminan en la aparición del ser personal del Supremo, consecuencia de toda evolución.
\usection{4. EL TODOPODEROSO Y LOS CREADORES SUPREMOS}
\vs p116 4:1 La unidad del Todo Supremo depende de la unificación creciente de sus partes finitas; la actualización del Supremo es producto, al igual que productor, de estas mismas unificaciones de los componentes de la supremacía: los creadores, las criaturas, las inteligencias y las energías de los universos.
\vs p116 4:2 \pc Durante esas eras en las que la soberanía de la Supremacía está experimentando su desarrollo en el tiempo, la potencia todopoderosa del Supremo depende de los actos divinos del Dios Séptuplo, si bien, parece existir una relación particularmente estrecha entre el Ser Supremo y el Actor Conjunto, unido a sus seres personales primordiales, los siete espíritus mayores. El Espíritu Infinito, como Actor Conjunto, obra de muchas formas que compensan la incompletitud de la Deidad evolutiva y mantiene relaciones muy próximas con el Supremo. Todos los espíritus mayores comparten en parte esta proximidad, pero en especial el espíritu mayor número siete, que habla en nombre del Supremo. Dicho espíritu mayor conoce al Supremo ---está en contacto personal con él---.
\vs p116 4:3 En los comienzos del diseño del plan de creación del suprauniverso, los espíritus mayores se unieron con la Trinidad ancestral en la cocreación de los cuarenta y nueve espíritus reflectores y, paralelamente, el Ser Supremo, obrando de forma creativa, llevó a su culminación los actos conjuntos de la Trinidad del Paraíso y de los hijos creativos de la Deidad del Paraíso. Majestón apareció y, desde entonces, hace converger la presencia cósmica de la Mente Suprema, en tanto que los espíritus mayores continúan como fuentes\hyp{}centros del extenso ministerio de la mente cósmica.
\vs p116 4:4 Pero los espíritus mayores siguen con la supervisión de los espíritus reflectores. El séptimo espíritu mayor (en su supervisión integral de Orvontón desde el universo central) está en contacto personal con los siete espíritus reflectores situados en Uversa (y ejerce una acción directiva sobre ellos). En su ejercicio de dirección y administración entre suprauniversos y dentro de ellos, está en contacto mediante la reflectividad con los espíritus reflectores de su propio tipo, que se localizan en cada una de las capitales de los suprauniversos.
\vs p116 4:5 Estos espíritus mayores no solamente apoyan y engrandecen la soberanía de la Supremacía, sino que, a su vez, se ven influidos por los propósitos creativos del Supremo. Por lo general, las creaciones colectivas de los espíritus mayores son de orden casi material (directores de la potencia, etc.), mientras que sus creaciones individuales son de orden espiritual (supernafines, etc.). Pero cuando los espíritus mayores crearon \bibemph{colectivamente} a los siete espíritus de las vías circulatorias en respuesta a la voluntad y el propósito del Ser Supremo, conviene señalar que los vástagos de este acto creativo son espirituales, no son ni materiales ni casi materiales.
\vs p116 4:6 \pc Y como sucede con los espíritus mayores de los suprauniversos, así ocurre con los gobernantes trinos de estas supracreaciones ---los ancianos de días---. Estas manifestaciones personales de la justicia\hyp{}juicio de la Trinidad en el tiempo y en el espacio actúan como palancas para la movilización de la todopoderosa potencia del Supremo, sirviendo como puntos séptuplos de convergencia para la evolución de la soberanía trinitaria en los dominios del tiempo y del espacio. Desde su posición privilegiada, a medio camino entre el Paraíso y los mundos evolutivos, estos soberanos de origen en la Trinidad ven, conocen y coordinan ambas partes.
\vs p116 4:7 \pc Pero los universos locales son los auténticos laboratorios en los que se elaboran los experimentos mentales, las aventuras galácticas, los despliegues de la divinidad y las progresiones del ser personal que, cuando se totalizan cósmicamente, constituyen los verdaderos cimientos sobre los que el Supremo está logrando su evolución en cuanto deidad en y mediante la experiencia.
\vs p116 4:8 En los universos locales, evolucionan incluso los creadores: la presencia del Actor Conjunto evoluciona desde su centro vivo de potencia hasta el estatus personal de un espíritu materno del universo; el hijo creador evoluciona desde la naturaleza de una divinidad existencial del Paraíso a la naturaleza experiencial de la soberanía suprema. Los universos locales son los puntos de partida de la verdadera evolución, el lugar de propagación de auténticos seres personales imperfectos dotados de libre voluntad para elegir convertirse en cocreadores de sí mismos, tal como llegarán a ser.
\vs p116 4:9 Los hijos magistrados al darse en sus ministerios de gracia a los mundos evolutivos llegan, en algún momento, a adquirir naturalezas que expresan la divinidad del Paraíso en unificación experiencial con los valores espirituales de mayor elevación de la naturaleza humana material. Y a través de estos y de otros ministerios de gracia, los creadores del orden de los migueles adquieren igualmente la naturaleza y la perspectiva cósmica de sus auténticos hijos del universo local. Estos hijos creadores mayores se aproximan a la compleción de la experiencia subsuprema; y cuando su soberanía del universo local se amplía hasta abarcar a sus colaboradores, los espíritus creativos, se puede decir que se acercan a los límites de la supremacía dentro de los potenciales actuales del gran universo evolutivo.
\vs p116 4:10 Cuando los hijos de gracia revelan al hombre nuevos caminos para encontrar a Dios, no están creando estos senderos a través de los que logra la divinidad; más bien, están iluminando las vías sempiternas de progreso que llevan, por medio de la presencia del Supremo, a la persona del Padre del Paraíso.
\vs p116 4:11 El universo local es el punto de partida para aquellos seres personales que están más distantes de Dios y que, por ello, pueden experimentar el más alto grado de ascenso espiritual en el universo, pueden alcanzar el máximo de participación experiencial en la cocreación de sí mismos. De igual manera, estos mismos universos locales proporcionan a los seres personales descendentes el más profundo grado de experiencia, con lo que logran algo que para ellos es tan significativo como la ascensión al Paraíso lo es para una criatura evolutiva.
\vs p116 4:12 \pc El hombre mortal parece ser necesario para la plena actividad del Dios Séptuplo, puesto que esta agrupación divina llega a su culminación al actualizarse el Supremo. Existen muchos órdenes de seres personales del universo que son igualmente necesarios para la evolución de la todopoderosa potencia del Supremo, pero esta descripción se expone para la edificación de los seres humanos, por lo que se limita en gran parte a esos elementos que llevan a cabo la evolución del Dios Séptuplo y que se relacionan con el hombre mortal.
\usection{5. EL TODOPODEROSO Y LOS RECTORES SÉPTUPLOS}
\vs p116 5:1 Se os ha instruido en la relación del Dios Séptuplo con el Ser Supremo, y debéis ser conscientes ahora de que el Séptuplo abarca a los rectores al igual que a los creadores del gran universo. En estos rectores séptuplos del gran universo se incluyen los siguientes:
\vs p116 5:2 \li{1.}Los controladores físicos mayores.
\vs p116 5:3 \li{2.}Los centros supremos de la potencia.
\vs p116 5:4 \li{3.}Los directores supremos de la potencia.
\vs p116 5:5 \li{4.}El Todopoderoso Supremo.
\vs p116 5:6 \li{5.}El Dios de Acción ---el Espíritu Infinito---.
\vs p116 5:7 \li{6.}La Isla del Paraíso.
\vs p116 5:8 \li{7.}La Fuente del Paraíso ---el Padre Universal---.
\vs p116 5:9 Estos siete grupos son operativamente inseparables del Dios Séptuplo y constituyen el nivel de control físico de esta conjunción de la Deidad.
\vs p116 5:10 \pc La bifurcación de la energía y el espíritu (dimanante de la presencia conjunta del Hijo Eterno y de la Isla del Paraíso) se simbolizó, en cuanto a los suprauniversos, cuando los siete espíritus mayores emprendieron unidos su primer acto de creación colectiva. En este acontecimiento, se presenció la aparición de los siete directores supremos de la potencia. Junto con ello, las vías circulatorias espirituales de los espíritus mayores se diferenciaron individualmente de la actividad física de supervisión de los directores de la potencia y, de inmediato, apareció la mente cósmica como un elemento nuevo, coordinando la materia y el espíritu.
\vs p116 5:11 El Todopoderoso Supremo está evolucionando como regidor de la potencia física del gran universo. En la presente era del universo, este potencial de la potencia física parece centrarse en los siete directores supremos de la potencia, que operan a través de los emplazamientos fijos de los centros de la potencia y mediante las presencias móviles de los controladores físicos.
\vs p116 5:12 \pc Los universos temporales no son perfectos; su destino es la perfección. La pugna por la perfección no solamente concierne a los niveles intelectual y espiritual, sino también al nivel físico de la energía y de la masa. El asentamiento de los siete suprauniversos en luz y vida implica la consecución de su estabilidad física. Y se postula que esta consecución final del equilibrio material indicará que se habrá completado la evolución del control físico del Todopoderoso.
\vs p116 5:13 Durante los primeros días de la creación del universo, incluso los creadores del Paraíso se ocupan principalmente del equilibrio material. El patrón de un universo local se concreta no solo como resultado de la actividad de los centros de la potencia, sino también gracias a la presencia espacial del espíritu creativo. A lo largo de estos tempranos tiempos de la formación del universo local, el hijo creador pone de manifiesto un atributo, poco comprendido, relativo al control material, y no deja su planeta capital hasta que no se ha establecido el total equilibrio del universo local.
\vs p116 5:14 \pc En última instancia, toda la energía responde a la mente, y los controladores físicos son los hijos del Dios de la mente, que es quien activa el modelo del Paraíso. La inteligencia de los directores de la potencia está incesantemente dedicada a la tarea de lograr el control material. Su pugna por la dominación física sobre las relaciones de la energía y los movimientos de la masa jamás cesa hasta que no consiguen la victoria finita sobre las energías y la masa, que constituyen los ámbitos perpetuos de su actividad.
\vs p116 5:15 Las pugnas espirituales del tiempo y del espacio guardan relación con la evolución del dominio del espíritu sobre la materia por mediación de la mente (personal); la evolución física (no personal) de los universos guarda relación con la armonización de la energía cósmica respecto a los conceptos de equilibrio de la mente sujetos a la acción directiva del espíritu. La evolución total de todo el gran universo es cuestión de la unificación personal de la mente regidora de la energía con el intelecto en coordinación con el espíritu, y se revelará en la plena aparición de la potencia todopoderosa del Supremo.
\vs p116 5:16 La dificultad para alcanzar un estado de equilibrio dinámico es consustancial al hecho del creciente cosmos. Las vías circulatorias establecidas de la creación física están continuamente en riesgo por la aparición de una nueva energía y una nueva masa. Un universo en crecimiento es un universo inestable; en consecuencia, ninguna parte de la totalidad cósmica puede encontrar una estabilidad real hasta que no se presencie, en la plenitud de los tiempos, la compleción material de los siete suprauniversos.
\vs p116 5:17 En los universos asentados en luz y vida, no se dan acontecimientos físicos imprevistos de gran envergadura. Se ha alcanzado un control, relativamente completo, sobre la creación material; no obstante, los problemas de la relación de los universos estables con los universos en evolución continúan planteando un reto a la destreza de los directores de la potencia del universo. Si bien, dichos problemas irán paulatinamente desapareciendo al disminuir toda nueva actividad creativa, a medida que el gran universo se aproxime a la culminación de su expresión evolutiva.
\usection{6. EL DOMINIO DEL ESPÍRITU}
\vs p116 6:1 En los suprauniversos evolutivos la energía\hyp{}materia es dominante excepto en el ser personal, donde el espíritu, a través de la mediación de la mente, pugna por la supremacía. La meta de los universos evolutivos es el sometimiento de la energía\hyp{}materia por la mente, es la coordinación de la mente con el espíritu y, todo ello, gracias a la presencia creativa y unificadora del ser personal. De este modo, en relación con el ser personal, los sistemas físicos se subordinan; los sistemas mentales, se armonizan; y los sistemas espirituales, se convierten en directivos.
\vs p116 6:2 Esta unión de potencia y ser personal se expresa en los niveles propios de la Deidad en y como el Supremo. Pero la evolución real del dominio del espíritu trata de un crecimiento basado en los actos voluntarios de los creadores y de las criaturas del gran universo.
\vs p116 6:3 \pc En los niveles absolutos, la energía y el espíritu son una misma cosa. Pero, desde el momento en el que se parte de estos niveles absolutos, aparece la diferencia, y a medida que la energía y el espíritu avanzan hacia el espacio desde el Paraíso, la brecha entre ellos se agranda hasta que, en los universos locales, divergen considerablemente. Dejan de ser idénticos, tampoco similares, y la mente debe intervenir para interrelacionarlos.
\vs p116 6:4 \pc El hecho de que la energía pueda llegar a orientarse gracias a la intervención de seres personales rectores desvela la receptividad de la energía a la acción de la mente. El hecho de que la masa pueda estabilizarse por la actuación de estas mismas entidades rectoras indica esa receptividad de la masa a la presencia de tal mente generadora de orden. Y el hecho de que el espíritu mismo, en el ser personal volitivo, pueda tender, a través de la mente, a dominar la energía\hyp{}materia desvela la unidad potencial de toda la creación finita.
\vs p116 6:5 Por todo el universo de los universos, existe una interdependencia de todas las fuerzas y seres personales. En la organización del universo, los hijos creadores y los espíritus creativos dependen de la labor cooperativa de los centros de la potencia y de los controladores físicos; los directores supremos de la potencia están incompletos sin el extra control que ejercen sobre ellos los espíritus mayores. En el ser humano, el mecanismo de la vida física es receptivo, en parte, a los dictados de la mente (personal). Esta misma mente puede, a su vez, verse dominada por la guía intencional del espíritu, y el resultado de dicho desarrollo evolutivo es la creación de un nuevo hijo del Supremo, una nueva unificación personal de los varios tipos de realidades cósmicas.
\vs p116 6:6 Tal como ocurre con las partes, ocurre con el todo; el espíritu\hyp{}persona de la Supremacía requiere la potencia evolutiva del Todopoderoso para completarse como Deidad y alcanzar su destino en vinculación con la Trinidad. Son los seres personales del tiempo y del espacio los que realizan el esfuerzo, pero la culminación y consumación de dicho esfuerzo es labor del Todopoderoso Supremo. Y mientras que el crecimiento del todo es pues la adición del crecimiento colectivo de las partes, se desprende igualmente que la evolución de las partes es un reflejo segmentado del crecimiento intencional del todo.
\vs p116 6:7 En el Paraíso, monota y espíritu son una misma cosa ---indiferenciables salvo por sus nombres---. En Havona, materia y espíritu, aunque perceptiblemente diferentes, son a la vez intrínsecamente armoniosos. En los siete suprauniversos, sin embargo, existe una enorme divergencia entre ambos: hay un gran abismo entre la energía cósmica y el espíritu divino; así pues, se produce un mayor potencial experiencial para la acción de la mente en la armonización y la eventual unificación del modelo físico con los propósitos espirituales. En los universos evolutivos del tiempo y del espacio, existe mayor atenuación de la divinidad, cuestiones más difíciles por resolver y más oportunidades para adquirir experiencia al solucionarlos. Toda esta situación en el ámbito del suprauniverso da origen a un más amplio marco de la existencia evolutiva en la que la posibilidad de la experiencia cósmica está disponible, por igual, tanto para la criatura como para el creador ---incluso más para la Deidad Suprema---.
\vs p116 6:8 La dominación del espíritu, existencial en los niveles absolutos, se convierte en una experiencia evolutiva en los niveles finitos y en los siete suprauniversos. Y todos comparten esta experiencia por igual, desde el hombre mortal hasta el Ser Supremo. Todos pugnan, pugnan personalmente, por tal logro; todos participan, participan personalmente, en el destino.
\usection{7. EL GRAN UNIVERSO: UN ORGANISMO VIVO}
\vs p116 7:1 El gran universo no es solo una creación material de grandeza física, sublimidad espiritual y magnitud intelectual; es también un organismo vivo magnífico y sensible. Hay una vida real que palpita por todo el mecanismo de la inmensa creación del vibrante cosmos. La realidad física de los universos simboliza la realidad perceptible del Todopoderoso Supremo; y este organismo, material y vivo, está penetrado por vías circulatorias de la inteligencia, al igual que el cuerpo humano está atravesado por una red de vías neurales sensitivas. Este universo físico está permeado por líneas de energía que activan eficientemente la creación material, al igual que el cuerpo humano se nutre y energiza gracias a la distribución circulatoria de productos energéticos nutritivos y asimilables. El inmenso universo no está exento de esos centros de coordinación que ejercen sobre él una magnífica acción directiva comparable con el delicado sistema de control químico del mecanismo humano. Pero si tan solo supierais algo sobre la constitución física de un centro de la potencia, podríamos deciros, por analogía, mucho más sobre el universo físico.
\vs p116 7:2 Al igual que los mortales recurren a la energía solar para su mantenimiento vital, del mismo modo el gran universo depende de las energías inagotables que emanan del Paraíso inferior para sustentar su actividad material y los movimientos cósmicos del espacio.
\vs p116 7:3 Se ha otorgado la mente a los mortales para que con ella puedan llegar a ser autoconscientes de su identidad y de su ser personal; y la mente ---incluso una Mente Suprema--- se ha concedido a la totalidad de lo finito, mediante la cual el espíritu de este ser personal emergente del cosmos pugna siempre por dominar la energía\hyp{}materia.
\vs p116 7:4 El hombre mortal responde a la guía del espíritu, así como el gran universo responde a la extensa atracción de la gravedad espiritual del Hijo Eterno, que cohesiona de forma supramaterial y universal los valores espirituales eternos de todas las creaciones del cosmos finito del tiempo y del espacio.
\vs p116 7:5 Es factible para los seres humanos identificarse a sí mismos, sempiterna y totalmente, con la realidad indestructible del universo ---al fusionarse con el modelador del pensamiento interior---. Asimismo, el Supremo depende perpetuamente de la estabilidad absoluta de la Deidad Primigenia, de la Trinidad del Paraíso.
\vs p116 7:6 El impulso del hombre por alcanzar la perfección del Paraíso, su esfuerzo por conseguir llegar a Dios, crea, en el cosmos vivo, una genuina tensión divina que únicamente puede resolverse mediante la evolución de un alma inmortal; esto es lo que sucede en la experiencia de la criatura mortal a nivel individual. Pero cuando todas las criaturas y todos los creadores del gran universo se esfuerzan por la consecución de Dios y el logro de la perfección divina, se establece una profunda tensión cósmica que solo puede encontrar solución en la síntesis divina de la potencia todopoderosa con el espíritu\hyp{}persona del Dios evolutivo de todas las criaturas, el Ser Supremo.
\vsetoff
\vs p116 7:7 [Auspiciado por un mensajero poderoso con residencia temporal en Urantia]
