\upaper{56}{La unidad universal}
\author{Mensajero poderoso y Maquiventa Melquisedec}
\vs p056 0:1 Dios es unidad. La Deidad está universalmente coordinada. El universo de universos constituye un inmenso mecanismo, un todo integrado regido absolutamente por una mente infinita. Los ámbitos físicos, intelectuales y espirituales de la creación universal están correlacionados de forma divina. Lo perfecto y lo imperfecto están realmente interrelacionados y es por ello por lo que la criatura evolutiva finita puede ascender al Paraíso en obediencia al mandato del Padre Universal: “Sed perfectos, como yo soy perfecto”.
\vs p056 0:2 Los distintos niveles de la creación están todos unificados en los planes y en la administración de los arquitectos del universo matriz. Para la limitada mente de los mortales del espacio y del tiempo, en el universo se pueden presentar problemas y situaciones que podrían aparentemente ser indicativos de disonancia y de ausencia de una eficaz coordinación; pero aquellos de nosotros que podemos observar una extensión más amplia de los fenómenos universales, y que hemos adquirido mayor experiencia en esta destreza de identificar la unidad fundamental subyacente a la diversidad creativa y descubrir la unicidad divina encubierta en el funcionamiento de la pluralidad, percibimos mejor el propósito divino y único presente en todas estas múltiples manifestaciones de la energía creativa universal.
\usection{1. LA COORDINACIÓN FÍSICA}
\vs p056 1:1 La creación física o material no es infinita, pero está perfectamente coordinada. Existen la fuerza, la energía y la potencia, pero, en origen, todas son una misma cosa. En su constitución, los siete suprauniversos son aparentemente binarios; el universo central, trino; pero el Paraíso tiene una constitución única. Y el Paraíso es la fuente real de todos los universos materiales ---pasados, presentes y futuros---. Si bien, esta derivación cósmica es un acontecimiento de la \bibemph{eternidad;} en ningún \bibemph{momento} ---pasado, presente o futuro---, surgen el espacio o el cosmos material de la Isla Nuclear de Luz. Como fuente cósmica, el Paraíso precede en su acción al espacio y al tiempo; por consiguiente, sus derivaciones parecerían estar huérfanas en el tiempo y en el espacio si no emergieran a través del Absoluto Indeterminado, su depositario último en el espacio y su revelador y regulador en el tiempo.
\vs p056 1:2 \pc El Absoluto Indeterminado sostiene el universo físico, mientras que el Absoluto de Deidad impele la extraordinaria acción directiva de toda la realidad material; y ambos Absolutos se unifican operativamente mediante el Absoluto Universal. Todos los seres personales ---materiales, morontiales, absonitas o espirituales--- entienden mejor esta correlación cohesiva del universo material observando cómo la respuesta gravitacional de toda genuina realidad material a la gravedad se centra en el Paraíso inferior.
\vs p056 1:3 La unificación gravitacional es universal e invariable; la respuesta de la energía pura es igualmente universal e ineludible. La energía pura (fuerza primordial) y el espíritu puro son enteramente pre\hyp{}sensibles a la gravedad. El Padre Universal rige personalmente estas fuerzas primigenias, inherentes en los Absolutos; de ahí que toda la gravedad se centre en la presencia personal del Padre del Paraíso, energía pura y espíritu puro, y en su morada supramaterial.
\vs p056 1:4 \pc La energía pura es la antecesora de todas las realidades operativas no espirituales y relativas, mientras que el espíritu puro es el potencial de la acción divina que dirige todos los sistemas energéticos básicos. Y estas realidades, en sus distintas manifestaciones por todo el espacio y tal como se observa en los movimientos del tiempo, están ambas centradas en la persona del Padre del Paraíso. En él son una sola ---deben unificarse--- porque Dios es uno. El ser personal del Padre está absolutamente unificado.
\vs p056 1:5 En la naturaleza infinita de Dios Padre no podría haber una realidad binaria, como la física y la espiritual; pero en el instante en que desviamos nuestra mirada de los niveles infinitos y de la realidad absoluta de los valores personales del Padre del Paraíso, observamos la existencia de estas dos realidades y reconocemos que son enteramente sensibles a su presencia personal; todas las cosas en él subsisten.
\vs p056 1:6 En el momento en el que os alejáis del concepto incondicionado del ser personal infinito del Padre del Paraíso, debéis asumir que la MENTE es el ineludible sistema que unifica la divergencia creciente en el universo de estas manifestaciones binarias del ser personal monotético y primigenio del Creador, la Primera Fuente y Centro ---el YO SOY---.
\usection{2. LA UNIDAD INTELECTUAL}
\vs p056 2:1 El Padre\hyp{}Pensamiento realiza su expresión espiritual en el Hijo\hyp{}Verbo y, a través del Paraíso, logra expandir la realidad en los extensos universos materiales. Las expresiones espirituales del Hijo Eterno se correlacionan con los niveles materiales de la creación mediante la actuación del Espíritu Infinito, por cuyo ministerio mental sensible al espíritu, y en cuyos actos mentales, directivos de lo físico, las realidades espirituales de la Deidad y las repercusiones materiales de la Deidad se correlacionan entre sí.
\vs p056 2:2 La mente es el don de carácter operativo del Espíritu Infinito, de ahí que sea infinita en cuanto a su potencial y universal en cuanto a su otorgamiento. El pensamiento primordial del Padre Universal se eterniza y expresa de manera doble: en la Isla del Paraíso y en el Hijo Eterno y Espiritual, su igual en la Deidad. Esta dualidad de la realidad eterna convierte al Dios de la mente, el Espíritu Infinito, en una inevitabilidad. La mente es la indispensable vía de comunicación entre las realidades espirituales y las materiales. Tan solo mediante el ministerio de la mente puede la mente material y evolutiva concebir y comprender al espíritu morador.
\vs p056 2:3 Esta mente infinita y universal ejerce su ministerio en los universos del tiempo y del espacio en la forma de mente cósmica; y, aunque se extiende desde el temprano ministerio de los espíritus asistentes hasta la espléndida mente del mandatario en jefe de un universo, esta mente cósmica está adecuadamente unificada en la supervisión de los siete espíritus mayores, los cuales están, a su vez, coordinados con la Mente Suprema del tiempo y del espacio y perfectamente correlacionados con la todo incluyente mente del Espíritu Infinito.
\usection{3. LA UNIFICACIÓN ESPIRITUAL}
\vs p056 3:1 Al igual que la gravedad mental universal está centrada en la presencia personal y paradisíaca del Espíritu Infinito, del mismo modo, la gravedad espiritual universal se centra en la presencia personal y paradisíaca del Hijo Eterno. El Padre Universal es uno, pero, en el tiempo y en el espacio, se revela en el doble fenómeno de la energía pura y del espíritu puro.
\vs p056 3:2 Las realidades espirituales del Paraíso son igualmente una sola, pero en todas las situaciones y relaciones espacio\hyp{}temporales este espíritu único se revela en el fenómeno doble de los seres personales y emanaciones espirituales del Hijo Eterno y de los seres personales e influencias espirituales del Espíritu Infinito y de las creaciones a él vinculadas; y todavía existe un tercer fenómeno ---el fraccionamiento del espíritu puro---: la dádiva del Padre de los modeladores del pensamiento y de otras entidades espirituales prepersonales.
\vs p056 3:3 \pc Al margen de los niveles de actividad del universo en los que podáis encontrar fenómenos espirituales o entrar en contacto con seres espirituales, debéis saber que todos ellos provienen de Dios que es espíritu mediante el ministerio del Hijo Espíritu y del Espíritu Mente Infinito. Y este extenso espíritu opera manifestándose en los mundos evolutivos del tiempo, según se dirige desde las sedes de los universos locales. Desde estas capitales, el espíritu santo y el espíritu de la verdad, junto con el ministerio de los espíritus asistentes de la mente, acuden a los niveles de orden inferior y evolutivos de las mentes materiales.
\vs p056 3:4 Aunque la mente está más unificada en el nivel de los espíritus mayores en vinculación con el Ser Supremo y como mente cósmica en subordinación a la Mente Absoluta, el ministerio espiritual que se dispensa a los mundos en evolución está más directamente unificado en los seres personales residentes en las sedes de los universos locales y en las personas de las benefactoras divinas que los presiden, las cuales, a su vez, se correlacionan casi perfectamente con la vía circulatoria de la gravedad del Paraíso del Hijo Eterno, en donde se produce la unificación final de todas las manifestaciones espirituales del tiempo y del espacio.
\vs p056 3:5 \pc Las criaturas pueden alcanzar, mantener y eternizar una existencia perfeccionada mediante la fusión de su mente autoconsciente con una fracción de la dote espiritual pre\hyp{}trinitaria de alguna de las personas de la Trinidad del Paraíso. La mente mortal es la creación de los hijos e hijas del Hijo Eterno y del Espíritu Infinito y, cuando se fusiona con el modelador del pensamiento procedente del Padre, comparte la dotación espiritual triple de los reinos evolutivos. Pero estas tres expresiones espirituales se llegan a unificar en perfección en los finalizadores, tal como lo estaban en la eternidad en el YO SOY universal antes de convertirse en el Padre Universal del Hijo Eterno y del Espíritu Infinito.
\vs p056 3:6 El espíritu debe siempre y en última instancia ser expresión triple y unificarse, en su realización final, en la Trinidad. El espíritu tiene su origen en una sola fuente a través de una expresión triple; y, en su completud, debe alcanzar, como así lo hace, su plena realización en esa unificación divina que se experimenta al encontrar a Dios ---la unicidad con la divinidad--- en la eternidad, y por medio del ministerio de la mente cósmica que deriva de la expresión infinita del verbo eterno del pensamiento universal del Padre.
\usection{4. LA UNIFICACIÓN DEL SER PERSONAL}
\vs p056 4:1 El Padre Universal es un ser personal divinamente unificado; de ahí, que todos sus hijos ascendentes que salieron del Paraíso para habitar en los mortales materiales en obediencia al mandato del Padre, sean, por el impulso de recuperación ejercido por los modeladores del pensamiento, seres personales plenamente unificados antes de llegar a Havona.
\vs p056 4:2 Por naturaleza, el ser personal trata de unificar todas las realidades que lo constituyen. El ser personal infinito de la Primera Fuente y Centro, el Padre Universal, unifica los siete Absolutos constitutivos de la Infinitud; y el ser personal del hombre mortal, al ser la dádiva exclusiva y directa del Padre Universal, posee igualmente el potencial de unificar los factores constitutivos de la criatura mortal. Dicha creatividad unificadora de todo ser personal creatural es el distintivo de nacimiento de su alta y única fuente y es indicativa, además, de su contacto ininterrumpido con esta misma fuente a través de la vía circulatoria del ser personal, mediante la que dicho ser personal creatural mantiene un contacto directo y sostenido con el Padre de todos los seres personales del Paraíso.
\vs p056 4:3 Pese a que Dios se manifiesta desde los dominios del Séptuplo, a través de la supremacía y la ultimidad, hasta el Dios Absoluto, la vía circulatoria del ser personal, centrada en el Paraíso y en la persona de Dios Padre, facilita la unificación completa y perfecta de todas estas distintas expresiones de ser personal divino, en lo que respecta a todos los seres personales creaturales de todos los niveles de existencia inteligente y de todos los ámbitos de los universos perfectos, perfeccionados y en camino de perfección.
\vs p056 4:4 \pc Aunque Dios es para los universos, y en los universos, todo lo que hemos mostrado, no obstante, para vosotros y para todas las demás criaturas conocedoras de Dios, él es uno, vuestro Padre y su Padre. Para un ser personal, Dios no puede ser plural. Dios es el Padre de cada una de sus criaturas y es literalmente imposible que un hijo pueda tener más de un Padre.
\vs p056 4:5 Filosóficamente, cósmicamente y en referencia a los niveles y lugares diferenciados donde se manifiesten, podéis y necesariamente debéis concebir la acción de Deidades plurales y presuponer la existencia de Trinidades plurales; si bien, por todo el universo matriz, en la experiencia y en la comunión personal de cada persona en adoración, Dios es uno; y esa Deidad unificada y personal es nuestra progenitora del Paraíso, el Dios Padre, el otorgador, preservador y Padre de todos los seres personales desde el hombre mortal de los mundos habitados hasta el Hijo Eterno de la Isla Central de la Luz.
\usection{5. LA UNIDAD DE LA DEIDAD}
\vs p056 5:1 La unicidad, la indivisibilidad, de la Deidad del Paraíso es existencial y absoluta. Hay tres manifestaciones personales eternas de la Deidad ---el Padre Universal, el Hijo Eterno y el Espíritu Infinito--- pero, en la Trinidad del Paraíso, constituyen \bibemph{en realidad} una sola Deidad, indivisa e indivisible.
\vs p056 5:2 \pc A partir del nivel primigenio de la realidad existencial del Paraíso\hyp{}Havona, se diversificaron dos niveles subabsolutos y, acto seguido, el Padre, el Hijo y el Espíritu se ocuparon de la creación de numerosos colaboradores y subordinados de índole personal. Y aunque sea inadecuado en este contexto proceder a considerar la unificación de la deidad absonita en los niveles trascendentales de ultimidad, es factible abordar algunas características de la acción unificadora de los diferentes estados personales de la Deidad en los que la divinidad se manifiesta operativamente a los distintos sectores de la creación y a los diferentes órdenes de seres inteligentes.
\vs p056 5:3 En cuanto a su acción presente en los suprauniversos, la divinidad se manifiesta activamente en las actuaciones de los creadores supremos ---los hijos y espíritus creadores de los universos locales, los ancianos de días de los suprauniversos y los siete espíritus mayores del Paraíso---. Estos seres constituyen los primeros tres niveles del Dios Séptuplo que conducen hacia el interior, hacia el Padre Universal, y todo este ámbito del Dios Séptuplo está coordinando, en el primer nivel de la deidad experiencial, la evolución del Ser Supremo.
\vs p056 5:4 \pc En el Paraíso y en el universo central, la unidad de la Deidad es una realidad. En todos los universos evolutivos del tiempo y del espacio, la unidad de la Deidad es un logro.
\usection{6. LA UNIFICACIÓN DE LA DEIDAD EVOLUTIVA}
\vs p056 6:1 Cuando las tres personas eternas de la Deidad actúan como Deidad indivisa en la Trinidad del Paraíso consiguen una unidad perfecta; igualmente, cuando crean, ya sea en vinculación o por separado, en su progenie del Paraíso se pone de manifiesto la unidad característica de lo divino. Y esta divinidad de propósito, manifestada por los creadores y por los gobernantes supremos de los dominios del espacio\hyp{}tiempo, deviene en el potencial de la potencia unificadora de la soberanía de la supremacía experiencial que, en presencia de la unidad energética impersonal del universo, constituye una tensión de la realidad, que únicamente puede resolverse por medio de su adecuada unificación con las realidades personales experienciales de la Deidad experiencial.
\vs p056 6:2 Las realidades personales del Ser Supremo provienen de las Deidades del Paraíso y, en el mundo piloto de la vía circulatoria exterior de Havona, se unifican con las prerrogativas sobre la potencia del Supremo Todopoderoso procedentes de las divinidades creadoras del gran universo. El Dios Supremo, como persona, existía en Havona antes de la creación de los siete suprauniversos, pero solo actuaba en niveles espirituales. La evolución de la potencia del Todopoderoso de Supremacía, mediante la síntesis de la diversa divinidad que tiene lugar en los universos evolutivos, devino en una nueva presencia de la potencia de la Deidad, que se coordinó con la persona espiritual del Supremo en Havona por medio de la Mente Suprema, que, al mismo tiempo, se trasladó desde el potencial residente en la mente infinita del Espíritu Infinito a la mente operativamente activa del Ser Supremo.
\vs p056 6:3 \pc Las criaturas de mente material de los mundos evolutivos de los siete suprauniversos solo pueden llegar a comprender la unidad de la Deidad conforme dicha unidad se desarrolla en esta síntesis de la potencia y del ser personal del Ser Supremo. En cualquier nivel de existencia, Dios no puede exceder la capacidad conceptual de los seres que habitan dicho nivel. El hombre mortal debe, por medio del reconocimiento de la verdad, de la apreciación de la belleza y de la adoración de la bondad, evolucionar en su reconocimiento de un Dios de amor y, luego, progresar por los niveles ascendentes en cuanto deidad hasta la comprensión del Supremo. Cuando se ha alcanzado así a comprender la Deidad, como unificada en la potencia, puede entonces manifestarse personalmente en espíritu para el entendimiento y la consecución creatural.
\vs p056 6:4 Aunque los mortales ascendentes logran comprender la potencia del Todopoderoso en las capitales de los suprauniversos y el ser personal del Supremo en las vías circulatorias exteriores de Havona, no encuentran de hecho al Ser Supremo de la misma manera que están destinados a encontrar a las Deidades del Paraíso. Ni incluso los finalizadores, los espíritus de la sexta etapa, han encontrado al Ser Supremo, ni lo encontrarán probablemente hasta que no hayan alcanzado el estatus de espíritus de la séptima etapa, y hasta que el Supremo realmente no actúe operativamente en la actividad múltiple que se desarrolla en los universos exteriores futuros.
\vs p056 6:5 Pero cuando los seres ascendentes encuentran al Padre Universal, correspondiente al séptimo nivel del Dios Séptuplo, han logrado llegar al ser personal de la Primera Persona de \bibemph{todos} los niveles en cuanto deidad que tienen relaciones personales con las criaturas del universo.
\usection{7. REPERCUSIONES EVOLUTIVAS UNIVERSALES}
\vs p056 7:1 El constante avance de la evolución en los universos del espacio\hyp{}tiempo viene acompañado de revelaciones cada vez más amplias de la Deidad para todas las criaturas inteligentes. Lograr la cima del progreso evolutivo en un mundo, sistema, constelación, universo, suprauniverso o gran universo señala una correspondiente ampliación de la actuación en cuanto deidad en y para estas unidades de la creación en vías de progreso. Y cada aumento a nivel local de la cognición de la divinidad viene aparejado por ciertas repercusiones, bien definidas, de una dilatada manifestación en cuanto deidad respecto a todos los demás sectores de la creación. Al extenderse al exterior, desde el Paraíso, cada nuevo ámbito de cognición y de logro de la evolución constituyen una nueva y creciente revelación de la Deidad experiencial para el universo de los universos.
\vs p056 7:2 A medida que los componentes que integran un universo local se asientan progresivamente en luz y vida, más se acrecienta la manifestación del Dios Séptuplo. En un planeta, la evolución espacio\hyp{}temporal comienza con la primera expresión del Dios Séptuplo, bajo el dominio del hijo creador y el espíritu creativo en vinculación. Con el asentamiento de un sistema en luz, esta conjunción hijo\hyp{}espíritu realiza su labor en plenitud; y, cuando una constelación entera se asienta igualmente en luz, la segunda faceta del Dios Séptuplo se hace más activa en todo este ámbito espacial. La culminación de la evolución a niveles administrativos de un universo local viene acompañada de la impartición de ministerios, nuevos y más directos, de los espíritus mayores del suprauniverso; y, en este momento, también comienza la revelación y la cognición crecientes del Dios Supremo, que culmina en la comprensión del Ser Supremo por parte del ascendente, a su paso por los mundos de la sexta vía circulatoria de Havona.
\vs p056 7:3 \pc El Padre Universal, el Hijo Eterno y el Espíritu Infinito son manifestaciones existenciales en cuanto deidad para las criaturas inteligentes y, por consiguiente, no se expanden del mismo modo en sus relaciones personales con las criaturas de mente y espíritu de toda la creación.
\vs p056 7:4 \pc Cabe observar que los mortales ascendentes pueden experimentar la presencia impersonal de los niveles sucesivos de la Deidad mucho antes de volverse suficientemente espirituales y estar adecuadamente formados como para conseguir reconocer personalmente y de forma experiencial a estas Deidades como seres personales y establecer contacto con ellas.
\vs p056 7:5 Cada nuevo logro evolutivo en un sector de la creación, al igual que cada nueva ocupación del espacio por manifestaciones de la divinidad, viene acompañado de expansiones simultáneas de la revelación operativa de la Deidad en las unidades de toda la creación, en ese momento existentes y previamente organizadas. Esta nueva ocupación de la labor administrativa de los universos y de las unidades que los componen puede que no siempre parezca que se realice exactamente del modo hasta aquí descrito. Esto resulta así porque es habitual enviar por adelantado a unos grupos de administradores, a fin de preparar el camino para las eras siguientes y sucesivas en las que se establecerá una nueva dirección de los asuntos administrativos. Incluso el Dios Último prefigura su trascendental acción directiva sobre los universos durante las postreras etapas del asentamiento en luz y vida de un universo local.
\vs p056 7:6 Es un hecho que, a medida que el estatus evolutivo de las creaciones del tiempo y del espacio va progresivamente asentándose, se observa en el Dios Supremo el ejercicio de una actuación nueva y más plena, en conjunción con la correspondiente retirada de las primeras tres manifestaciones del Dios Séptuplo. Si y cuando el gran universo llegara a establecerse en luz y vida, ¿cuál será entonces la labor futura de las manifestaciones del creador y del espíritu creativo del Dios Séptuplo si el Dios Supremo asume el control directo de estas creaciones del espacio\hyp{}tiempo? ¿Es que estos organizadores y pioneros de los universos del tiempo y del espacio serán liberados de sus cometidos para poder desarrollar una actividad similar en el espacio exterior? No lo sabemos, pero hacemos muchas conjeturas sobre estas y otras cuestiones afines.
\vs p056 7:7 \pc Conforme las fronteras de la Deidad experiencial se extienden hacia el exterior, hacia los ámbitos del Absoluto Indeterminado, vislumbramos la actividad del Dios Séptuplo durante las tempranas épocas evolutivas de estas creaciones del futuro. No estamos todos de acuerdo en cuanto al estatus futuro de los ancianos de días ni el de los espíritus mayores de los suprauniversos. Tampoco sabemos si el Ser Supremo desempeñará su función en estas creaciones como en los siete suprauniversos. Si bien, todos presuponemos que los migueles, los hijos creadores, están destinados a actuar en estos universos exteriores. Algunos sostienen que las eras futuras serán testigos de alguna forma de unión más estrecha que la ya existente entre los hijos creadores y las benefactoras divinas; es incluso posible que dicha unión creadora pueda devenir en alguna nueva expresión de identidad de la conjunción de estos creadores, de naturaleza última. Pero, en realidad, no sabemos nada de estos posibles hechos del futuro no revelado.
\vs p056 7:8 Sabemos, sin embargo, que en los universos del tiempo y del espacio, el Dios Séptuplo facilita una aproximación progresiva al Padre Universal, y que esta aproximación evolutiva se unifica experiencialmente en el Dios Supremo. Puede conjeturarse que un plan así debería prevalecer en los universos exteriores; por otra parte, los nuevos órdenes de seres que alguna vez puedan habitar estos universos podrían ser capaces de aproximarse a la Deidad en sus últimos niveles y mediante métodos absonitos. En resumen, no tenemos la más mínima noción de qué modo de acercamiento a la Deidad puede resultar efectivo en los universos futuros del espacio exterior
\vs p056 7:9 Sin embargo, consideramos que, de alguna manera, los suprauniversos perfeccionados se convertirán en parte de las andaduras de ascensión hacia el Paraíso de aquellos seres que habiten las creaciones exteriores. Es muy posible que en esa era futura podamos presenciar cómo estos seres del espacio exterior se acercan a Havona a través de los siete suprauniversos, cuya administración será responsabilidad del Dios Supremo con o sin la colaboración de los siete espíritus mayores.
\usection{8. EL UNIFICADOR SUPREMO}
\vs p056 8:1 El Ser Supremo desempeña una labor triple en la experiencia del hombre mortal: en primer lugar, es el unificador de la divinidad espacio temporal o Dios Séptuplo; en segundo lugar, constituye el máximo concepto sobre la Deidad realmente comprensible para las criaturas finitas; en tercer lugar, es la única vía que el hombre mortal tiene para hacer su aproximación a la experiencia trascendental de relacionarse con la mente absonita, el espíritu eterno y el ser personal paradisíaco.
\vs p056 8:2 Al haber nacido en los universos locales, al haberse criado en los suprauniversos y formado en el universo central, los finalizadores ascendentes abarcan, en sus experiencias personales, todo el potencial posible para llegar a comprender la divinidad espacio temporal del Dios Séptuplo que se unifica en el Supremo. Los finalizadores sirven sucesivamente en suprauniversos distintos a los de su nacimiento, superponiendo así experiencias tras experiencias hasta englobar toda la diversidad séptupla que las criaturas puedan llegar a experimentar. El ministerio de los modeladores interiores posibilita \bibemph{hallar} al Padre Universal, pero la experiencia constituye el modo de llegar realmente a \bibemph{conocer} al Ser Supremo. Los finalizadores están destinados a servir y a \bibemph{revelar} a esta Deidad Suprema en y para los universos futuros del espacio exterior.
\vs p056 8:3 Tened presente que todo lo que el Dios Padre y sus hijos del Paraíso hacen por nosotros, nosotros a nuestra vez y en espíritu tenemos la posibilidad de hacerlo por y en el Ser Supremo emergente. En el universo, la experiencia del amor, la felicidad y el servicio es mutua. Dios Padre no necesita que sus hijos le devuelvan todo lo que él les da de gracia, pero ellos a su vez dan (o pueden dar) de gracia todo esto a sus semejantes y al Ser Supremo en evolución.
\vs p056 8:4 Todos los fenómenos relacionados con la creación reflejan una actividad creadora y espiritual precedente. Jesús dijo, y es literalmente verdad, que “el Hijo solo hace aquellas cosas que ve hacer al Padre”. Con el tiempo, vosotros los mortales comenzaréis a revelar el Supremo a vuestros semejantes e incrementaréis progresivamente esta revelación a medida que ascendáis hacia el Paraíso. Es posible que se os permita en la eternidad, como finalizadores de la séptima etapa, realizar revelaciones crecientes de este Dios a las criaturas evolutivas de los niveles supremos ---e incluso últimos---.
\usection{9. LA UNIDAD DEL ABSOLUTO UNIVERSAL}
\vs p056 9:1 El Absoluto Indeterminado y el Absoluto de la Deidad se unifican en el Absoluto Universal. Los Absolutos se coordinan en el Último, se condicionan en el Supremo y se modifican, en el espacio\hyp{}tiempo, en el Dios Séptuplo. En los niveles subinfinitos hay \bibemph{tres} Absolutos, pero en la infinitud parecen ser \bibemph{uno}. En el Paraíso hay tres manifestaciones personales de la Deidad, pero en la Trinidad \bibemph{son} una sola.
\vs p056 9:2 \pc El principal postulado filosófico del universo matriz es el siguiente: ¿Existía el Absoluto (los tres Absolutos como uno solo en la infinitud) antes que la Trinidad? y ¿es el Absoluto el ancestro de la Trinidad? o ¿es la Trinidad la antecesora del Absoluto?
\vs p056 9:3 ¿Es el Absoluto Indeterminado una presencia de fuerza independiente de la Trinidad? ¿Implica la presencia del Absoluto de la Deidad una ilimitada acción de la Trinidad? ¿Es el Absoluto Universal la acción final de la Trinidad, e incluso de una Trinidad de Trinidades?
\vs p056 9:4 A simple vista, el concepto del Absoluto como el ancestro de todas las cosas ---incluso de la Trinidad--- parece, por su gratificante consistencia y la unificación que conlleva a niveles filosóficos, producir una satisfacción pasajera; si bien, la realidad de la eternidad de la Trinidad del Paraíso invalida este tipo de conclusiones. Se nos enseña, y nosotros así lo creemos, que el Padre Universal y sus colaboradores de la Trinidad son, en cuanto a su naturaleza y existencia, eternos. Por lo tanto, no hay sino una conclusión filosófica que tenga consistencia y es esta: el Absoluto es, para todas las inteligencias del universo, la respuesta impersonal y coordinada de la Trinidad (o Trinidades) a todas las situaciones espaciales fundamentales y primarias, intrauniversales y extrauniversales. Para todas las inteligencias personales del gran universo, la Trinidad del Paraíso por siempre permanece en completud, eternidad, supremacía y ultimidad y, para todos los efectos prácticos respecto a la comprensión personal y a la conciencia creatural, como absoluta.
\vs p056 9:5 Al examinar esta cuestión, la mente de las criaturas se ve inclinada a realizar una premisa final del YO SOY universal como causa primigenia y fuente incondicionada a la vez de la Trinidad y del Absoluto. Así pues, cuando aspiramos a albergar un concepto personal del Absoluto, volvemos a nuestras ideas e ideales del Padre del Paraíso. Cuando deseamos mejorar nuestra comprensión o aumentar nuestra conciencia de este Absoluto, impersonal, volvemos al hecho de que el Padre Universal es el Padre existencial del ser personal absoluto; el Hijo Eterno es la Persona Absoluta, el Absoluto en estado personal, aunque no en un sentido experiencial. Y entonces continuamos imaginando que las Trinidades experienciales culminan en la manifestación experiencial y personal del Absoluto de la Deidad, mientras concebimos al Absoluto Universal como constitutivo de los fenómenos universales y extrauniversales, concernientes a la presencia manifiesta de la actividad impersonal de las conjunciones unificadas y coordinadas de la Deidad de supremacía, de ultimidad y de infinitud ---la Trinidad de Trinidades---.
\vs p056 9:6 \pc Dios Padre es perceptible en todos los niveles desde el finito al infinito y, aunque sus criaturas, desde el Paraíso hasta los mundos evolutivos, lo han percibido de distintas maneras, solo el Hijo Eterno y el Espíritu Infinito lo conocen como infinitud.
\vs p056 9:7 El ser personal espiritual es absoluto solo en el Paraíso, y el concepto del Absoluto es incondicionado solo en la infinitud. La presencia de la Deidad es absoluta solo en el Paraíso, y la revelación de Dios siempre debe ser parcial, relativa y progresiva hasta que su potencia se vuelva experiencialmente infinita en la potencia espacial del Absoluto Indeterminado, mientras que la manifestación de su persona se vuelva experiencialmente infinita en la presencia manifiesta del Absoluto de la Deidad y mientras estos dos potenciales de la infinitud se vuelvan una realidad unificada en el Absoluto Universal.
\vs p056 9:8 Pero, más allá de los niveles subinfinitos, los tres Absolutos \bibemph{son} uno solo, y así la Deidad engloba la infinitud, con independencia de que cualquier otro orden de existencia tome conciencia propia, alguna vez, de esta infinitud.
\vs p056 9:9 En la eternidad, el estatus existencial supone autoconciencia existencial de la infinitud, aunque se precise de otra eternidad para experimentar la autorrealización de las potencialidades experienciales inherentes a una eternidad infinita ---a una infinitud eterna---.
\vs p056 9:10 \pc Y Dios Padre es la fuente personal de todas las manifestaciones de la Deidad y de la realidad para todas las criaturas inteligentes y los seres espirituales de todo el universo de los universos. Como seres personales, ahora o en vuestra continuada experiencia durante el futuro eterno en el universo, no importa si conseguís alcanzar al Dios Séptuplo, comprender al Dios Supremo, hallar al Dios Último o intentáis aprehender el concepto del Dios Absoluto, ya que descubriréis, para vuestra satisfacción eterna, que al culminar cada una de las aventuras en las que participéis, en vuestros nuevos niveles experienciales, redescubrís al Dios eterno ---al Padre del Paraíso de todos los seres personales del universo---.
\vs p056 9:11 En el Padre Universal radica la unidad universal tal como debe realizarse de manera suprema, e incluso última, en la unidad posúltima de los valores y contenidos absolutos ---la Realidad incondicionada---.
\vs p056 9:12 Los organizadores mayores de la fuerza salen al espacio y movilizan sus energías para hacerlas sensibles a la gravedad, a la atracción del Paraíso del Padre Universal. Con posterioridad, llegan los hijos creadores, que organizan estas fuerzas sensibles a la gravedad en los universos habitados y ahí crean de forma evolutiva a criaturas inteligentes que reciben dentro de sí al espíritu del Padre del Paraíso, y que luego ascienden al Padre para volverse como él en todos los posibles atributos pertinentes a divinidad.
\vs p056 9:13 La marcha incesante y en expansión de las fuerzas creativas del Paraíso a través del espacio parece predecir el ámbito en constante despliegue de la atracción gravitatoria del Padre Universal y la inacabable multiplicación de los distintos tipos de criaturas inteligentes capaces de amar a Dios y ser amadas por él y que, al llegar a conocer a Dios, pueden elegir ser como él, pueden optar por alcanzar el Paraíso y hallar a Dios.
\vs p056 9:14 El universo de los universos está unificado en su totalidad. Dios es uno en potencia y en ser personal. Hay coordinación en todos los niveles de la energía y en todas las facetas del ser personal. Filosófica y experiencialmente, en concepto y en realidad, todas las cosas y los seres tienen su centro en el Padre del Paraíso. Dios lo es todo y está en todos, y sin él no existe ni cosa ni ser alguno.
\usection{10. VERDAD, BELLEZA Y BONDAD}
\vs p056 10:1 A medida que los mundos asentados en luz y vida progresan desde su etapa inicial hasta la séptima época, tratan continuadamente de alcanzar la cognición de la realidad del Dios Séptuplo, que se extiende desde la adoración del hijo creador hasta la adoración de su Padre del Paraíso. Durante el curso de esta séptima etapa de la historia de estos mundos, los mortales en constante progreso crecen en el conocimiento del Dios Supremo, mientras que perciben levemente la realidad del preponderante ministerio de Dios Último.
\vs p056 10:2 Durante esta gloriosa era, la principal actividad de los mortales en su constante avance es la búsqueda de un mejor entendimiento y de una mayor conciencia de los elementos comprensibles de la Deidad: la verdad, la belleza y la bondad. Esto significa el esfuerzo del hombre por percibir a Dios en la mente, en la materia y en el espíritu. Y conforme el mortal prosigue esta búsqueda, más se ve absorto en el estudio experiencial de la filosofía, de la cosmología y de la divinidad.
\vs p056 10:3 \pc De alguna manera, alcanzáis a entender la filosofía, y comprendéis la divinidad en la adoración, en el servicio social y en vuestra experiencia espiritual personal, pero con demasiada frecuencia reducís la belleza ---la cosmología--- al estudio de los toscos esfuerzos artísticos del hombre. La belleza, el arte, es, en su mayor parte, una cuestión de unificación de contrastes. La variedad es esencial para el concepto de la belleza. La belleza suprema, la cima del arte finito, es el escenario de la unificación de la inmensidad de los extremos cósmicos de Creador y criatura. El hombre que encuentra a Dios y Dios que encuentra al hombre ---la criatura que se vuelve perfecta como lo es su Creador---; tal es el logro sublime de lo supremamente bello, la consecución de la cúspide del arte cósmico.
\vs p056 10:4 De ahí que el materialismo, el ateísmo, sea el culmen de la fealdad, el apogeo de la antítesis finita de lo bello. La belleza más elevada reside en la unificación de las variaciones nacidas de una realidad armoniosa y preexistente.
\vs p056 10:5 Alcanzar los niveles cosmológicos del pensamiento implica:
\vs p056 10:6 \li{1.}\bibemph{Curiosidad}. Sed de armonía y belleza. Perseverancia en el intento de descubrir nuevos niveles de relaciones cósmicas armoniosas.
\vs p056 10:7 \li{2.}\bibemph{Apreciación estética}. Amor por lo bello y aprecio siempre en avance del toque artístico existente en todas las manifestaciones creativas de todos los niveles de la realidad.
\vs p056 10:8 \li{3.}\bibemph{Sensibilidad ética}. Mediante la cognición de la verdad, la apreciación de la belleza lleva al sentido de la idoneidad eterna de aquellas cosas que inciden en el reconocimiento de la verdad divina en las relaciones de la Deidad con todos los seres; y, así, incluso la cosmología lleva a la búsqueda de los valores divinos de la realidad ---a la conciencia de Dios---.
\vs p056 10:9 \pc Los mundos asentados en luz y vida están tan completamente involucrados en la comprensión de la verdad, la belleza y la bondad, porque estos valores cualitativos abarcan la revelación de la Deidad destinada a los mundos del tiempo y del espacio. Los contenidos de la verdad eterna producen, en combinación, una atracción sobre las naturalezas intelectual y espiritual del hombre mortal. En la belleza universal se enmarcan las relaciones armoniosas y los ritmos de la creación cósmica; esto claramente conforma la atracción de índole espiritual y lleva hacia la comprensión unificada y sincrónica del universo material. La bondad divina representa la revelación de los valores infinitos para la mente finita, para que ahí se perciban y eleven al umbral mismo del nivel espiritual de la comprensión humana.
\vs p056 10:10 La verdad es la base de la ciencia y de la filosofía, y da a la religión su fundamento intelectual. La belleza auspicia el arte, la música y los ritmos significativos de toda experiencia humana. La bondad abarca el sentido de la ética, de la moral y de la religión ---la sed de perfección vivencial---.
\vs p056 10:11 La existencia de la belleza supone la presencia de una mente creatural apreciativa; esto es tan cierto como el hecho de que la evolución progresiva es indicativa de la dominación de la Mente Suprema. La belleza constituye el reconocimiento intelectual de la síntesis armoniosa en el espacio\hyp{}tiempo de la amplia diversificación de la realidad fenoménica, todo proveniente de una unicidad preexistente y eterna.
\vs p056 10:12 La bondad constituye el reconocimiento mental de los valores relativos de los distintos niveles de la perfección divina. El reconocimiento de la bondad supone una mente de índole moral, una mente personal con capacidad para distinguir entre el bien y el mal. Pero la posesión de la bondad, la grandeza, es la medida del verdadero logro de la divinidad.
\vs p056 10:13 \pc El reconocimiento de \bibemph{relaciones verdaderas} supone una mente capacitada para distinguir entre la verdad y el error. El don del espíritu de la verdad, del que se revisten las mentes humanas de Urantia, responde infaliblemente a la verdad ---a la relación espiritual viva de todas las cosas y de todos los seres a medida que se coordinan en el ascenso eterno a Dios---.
\vs p056 10:14 Cualquier impulso de cualquier electrón, pensamiento o espíritu es una unidad que actúa en todo el universo. En los niveles mentales y espirituales solo el pecado está aislado y el mal es resistente a la gravedad. El universo es un todo; ninguna cosa o ser existen o viven en aislamiento. La realización de uno mismo resulta potencialmente perniciosa si es antisocial. Es literalmente cierto: “Ningún hombre vive para sí”. La socialización cósmica constituye la forma más elevada de unificación del ser personal. Dijo Jesús: “Si alguno de vosotros quisiera hacerse grande entre vosotros, será vuestro servidor”.
\vs p056 10:15 Incluso la verdad, la belleza y la bondad ---el acercamiento intelectual del hombre al universo de la mente, la materia y el espíritu--- deben combinarse en un concepto unificado de un \bibemph{ideal} divino y supremo. Al igual que el ser personal mortal unifica la experiencia humana con la materia, la mente y el espíritu, así este ideal divino y supremo se convierte en potencia unificada en la Supremacía y luego se manifiesta personalmente como un Dios de amor paternal.
\vs p056 10:16 Cualquier entendimiento de las relaciones de las partes respecto a un todo determinado precisa llegar a una comprensión de la relación de todas las partes con ese todo; y, en el universo, esto conduce a la relación de las partes creadas con el Todo Creativo. La Deidad se convierte, por tanto, en la meta trascendental, incluso infinita, del logro universal y eterno.
\vs p056 10:17 \pc La belleza universal es el reconocimiento del reflejo de la Isla del Paraíso en la creación material, mientras que la verdad eterna es el ministerio especial de los hijos del Paraíso que no solo se dan de gracia a las razas humanas, sino que además derraman su espíritu de la verdad sobre todos los pueblos. La mejor manera de ilustrar la bondad divina es refiriéndonos al ministerio amoroso de los múltiples seres personales del Espíritu Infinito. Pero el amor, la suma total de estas tres cualidades, es la percepción que tiene el hombre de Dios como su Padre espiritual.
\vs p056 10:18 La materia física es la sombra espacio\hyp{}temporal de la resplandeciente energía paradisíaca de las Deidades absolutas. Los contenidos de la verdad son las repercusiones en el intelecto humano del verbo eterno de la Deidad ---la comprensión espacio\hyp{}temporal de los conceptos supremos---. Los valores de la bondad de la divinidad son los ministerios misericordiosos de las personas espirituales del Universal, del Eterno y del Infinito hacia las criaturas espacio temporales finitas de las esferas evolutivas.
\vs p056 10:19 Estos valores significativos de la realidad de la divinidad se combinan en la relación del Padre con cada una de sus criaturas personales como amor divino. Se coordinan en el Hijo y en sus hijos como misericordia divina. Manifiestan sus cualidades mediante el Espíritu y sus hijos espirituales como ministerio divino, la imagen de la misericordia amorosa hacia los hijos del tiempo. Estas tres divinidades se manifiestan principalmente por el Ser Supremo como síntesis de la potencia\hyp{}ser personal. Y se muestran de distintos modos por el Dios Séptuplo en siete relaciones diferentes de contenidos y valores divinos en siete niveles ascendentes.
\vs p056 10:20 \pc Para el hombre finito la verdad, la belleza, y la bondad abarcan por completo la revelación de la realidad de la divinidad. A medida que esta comprensión\hyp{}amor de la Deidad encuentra su expresión espiritual en la vida de los mortales que conocen a Dios, se producen los frutos de la divinidad: la paz intelectual, el progreso social, la satisfacción moral, el gozo espiritual y la sabiduría cósmica. En un mundo en su séptima etapa de luz y vida, sus avanzados mortales han aprendido que el amor es lo más grande que hay en el universo ---y saben que Dios es amor---.
\vs p056 10:21 \pc El amor es el deseo de hacer el bien a los demás.
\vs p056 10:22 \pc [Exposición de un mensajero poderoso de visita en Urantia, por petición del colectivo de reveladores de Nebadón y en colaboración con cierto melquisedec, vicerregente del príncipe planetario de Urantia.]
\separatorline
\vsetoff
\vs p056 10:23 \pc Este escrito sobre la Unidad Universal es el vigésimo quinto de una serie de presentaciones realizadas por diversos autores, que han sido auspiciadas como grupo por una comisión de doce seres personales de Nebadón actuando bajo la dirección de Mantutia Melquisedec. Dictamos estas narrativas y las expresamos en el idioma inglés, mediante un sistema autorizado por nuestros superiores, en el año 1934 del tiempo de Urantia.
