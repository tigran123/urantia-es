\upaper{150}{El tercer viaje de predicación}
\author{Comisión de seres intermedios}
\vs p150 0:1 El domingo, 16 de enero del año 29 d. C., a última hora de la tarde, Abner, llegó a Betsaida con los apóstoles de Juan para mantener, al día siguiente, un encuentro conjunto con Andrés y los apóstoles de Jesús. Abner y sus acompañantes habían establecido su sede en Hebrón y tenían la costumbre de ir a Betsaida con periodicidad para asistir a estas reuniones.
\vs p150 0:2 Entre las numerosas cuestiones que se trataron en este encuentro, se trató de la práctica de ungir a los enfermos con algún tipo de aceite en el momento de orar por su curación. De nuevo, Jesús declinó participar en las deliberaciones o expresar su opinión sobre las conclusiones a las que llegaran. Los apóstoles de Juan siempre habían hecho uso del aceite de unción en su ministerio hacia los enfermos y afligidos, e intentaron establecer esta práctica de forma generalizada para ambos grupos, pero los apóstoles de Jesús se negaron a asumir dicha norma.
\vs p150 0:3 \pc El martes 18 de enero, en la casa de Zebedeo en Betsaida, se unieron a los veinte y cuatro, los evangelistas ya probados, unos setenta y cinco en total, en anticipación a ser enviados al tercer viaje de predicación por Galilea. Esta tercera misión tuvo una duración de siete semanas.
\vs p150 0:4 Se enviaron a los evangelistas en grupos de cinco, mientras que Jesús y los doce viajaron juntos la mayoría del tiempo; los apóstoles iban de dos en dos para bautizar a los creyentes, cuando la ocasión lo requería. Durante un período de casi tres semanas, Abner y sus compañeros trabajaron igualmente con los grupos de evangelistas, aconsejándolos y bautizando a los creyentes. Visitaron Magdala, Tiberias, Nazaret y todas las principales ciudades y aldeas del centro y el sur de Galilea, lugares todos previamente visitados y muchos otros más. Sería su último mensaje destinado a Galilea, exceptuando las zonas del norte.
\usection{1. EL COLECTIVO DE MUJERES EVANGELISTAS}
\vs p150 1:1 De todos los valientes actos que realizó Jesús en su andadura terrenal, el más sorprendente de todos fue cuando, de repente, la noche del 16 de enero, anunció: “Mañana elegiremos a diez mujeres para trabajar en el ministerio del reino”. Al comienzo del período de dos semanas, durante el que los apóstoles y los evangelistas estarían ausentes de Betsaida para hacer un receso, Jesús pidió a David que llamara a sus padres para que volvieran a casa y que enviara mensajeros convocando a Betsaida a diez mujeres devotas que hubiesen prestado servicios en el antiguo campamento y en el hospital de tiendas. Todas estas mujeres habían escuchado las enseñanzas dadas a los jóvenes evangelistas, pero jamás se les había ocurrido ni a ellas ni a sus maestros que Jesús se aventurara a darles la misión de enseñar el evangelio del reino y cuidar de los enfermos. Estas diez mujeres, a las que Jesús escogió y encomendó, fueron: Susana, la hija del jazán de la sinagoga de Nazaret; Juana, esposa de Chuza, el mayordomo de Herodes Antipas; Elisabet, hija de un judío rico de Tiberias y Séforis; Marta, hermana mayor de Andrés y Pedro; Raquel, cuñada de Judá, el hermano carnal del Maestro; Nasanta, hija de Elman, el médico sirio; Milca, prima del apóstol Tomás; Ruth, la hija mayor de Mateo Leví; Celta, hija de un centurión romano; y Agaman, una viuda de Damasco. Jesús añadiría luego, a este grupo, a otras dos mujeres: María Magdalena y Rebeca, hija de José de Arimatea.
\vs p150 1:2 Jesús dio permiso a estas mujeres para que se organizaran por sí misma se indicó a Judas que les proporcionara fondos para su equipamiento y para la adquisición de animales de carga. Las diez escogieron a Susana como jefa y a Juana como tesorera. Desde ese momento, consiguieron sus propios fondos, y nunca más recurrirían a Judas para su sostenimiento.
\vs p150 1:3 Resultaba de lo más extraordinario en aquellos días, en los que no se permitía a las mujeres ni siquiera acceder a la planta principal de la sinagoga (confinándolas a la galería de mujeres), que se les reconociera y autorizara como maestras del nuevo evangelio del reino. El encargo que Jesús hizo a estas diez mujeres, al escogerlas para enseñar el evangelio y ejercer su ministerio, fue proclamar, para todos los tiempos, la emancipación y liberación de la mujer; nunca más miraría el hombre a la mujer como espiritualmente inferior. Esto produjo una notable conmoción incluso entre los doce apóstoles. Pese a que habían oído muchas veces al Maestro decir que “en el reino de los cielos no hay ni ricos ni pobres, ni libres ni esclavos, ni hombres ni mujeres, todos son por igual los hijos e hijas de Dios”, estaban literalmente estupefactos cuando él propuso formalmente nombrar a estas diez mujeres como maestras religiosas, permitiéndoles, además, viajar con ellos. El país completo se agitó ante este proceder, y los enemigos de Jesús sacaron un gran provecho de esta medida, pero, en todas partes, las mujeres creyentes en la buena nueva apoyaron con firmeza a estas hermanas suyas, designadas para tal tarea, y dieron su clara aprobación a este reconocimiento tardío del lugar de la mujer en la labor religiosa. Y los apóstoles, tras la partida del Maestro, llevaron a efecto de inmediato esta liberación de la mujer, a las que se les reconocía merecidamente; no obstante, se volvió a las antiguas costumbres en las generaciones siguientes. Durante los tempranos días de la Iglesia cristiana, se llamaban “\bibemph{diaconisas}” a las mujeres maestras y a las que desempeñaban el ministerio religioso, y se les profesaba un general reconocimiento. Pero Pablo, a pesar de admitirlo en principio, esto nunca tuvo cabida en su propia actitud y, personalmente, le resultó difícil llevarlo a la práctica.
\usection{2. LA PARADA EN MAGDALA}
\vs p150 2:1 Cuando el grupo apostólico abandonó Betsaida, las mujeres viajaban en la parte de atrás. Durante las charlas, estas siempre se sentaban en grupo enfrente y a la derecha del conferencista. El número de mujeres que se convertían en creyentes del evangelio del reino había crecido, y esto había causado muchas dificultades e innumerables situaciones embarazosas cuando querían conversar personalmente con Jesús o con algún apóstol. Ahora, todo esto había cambiado. Si una de las mujeres creyentes deseaba ver al Maestro o consultar con los apóstoles, acudía a Susana, y acompañada de alguna de las doces mujeres evangelistas, ambas iban de inmediato ante la presencia del Maestro o de alguno de sus apóstoles.
\vs p150 2:2 Fue en Magdala donde las mujeres demostraron por vez primera su valía y justificaron el acierto de haberlas elegido. Andrés había impuesto unas estrictas reglas a sus compañeros en cuanto al trabajo personal con las mujeres, en especial con las de dudosa reputación. Pero al llegar el grupo a Magdala, estas diez mujeres evangelistas tuvieron libertad para entrar en los antros del mal y predicar directamente la buena nueva a sus moradoras. Y, cuando visitaban a las enfermas, estas mujeres se mostraban muy cercanas a sus hermanas afligidas. Como resultado del ministerio realizado por estas diez mujeres (más adelante conocidas como las doce mujeres) en este lugar, se ganó a María Magdalena para el reino. A través de una serie de infortunios y como consecuencia de la actitud de una sociedad respetable hacia las mujeres que cometen tales errores de juicio, esta mujer se encontraba en uno de los nefastos burdeles de Magdala. Fueron Marta y Raquel quienes le expresaron a María que las puertas del reino estaban abiertas incluso para personas como ella. María creyó en la buena nueva, y Pedro la bautizó al día siguiente.
\vs p150 2:3 De todo este grupo de doce mujeres evangelistas, María Magdalena se convirtió en la más eficaz de las maestras del evangelio. Junto con Rebeca, se la eligió para este servicio en Jotapata, unas cuatro semanas después de su conversión. María y Rebeca, con otras de este grupo, continuaron, con lealtad y eficiencia, instruyendo y elevando a sus hermanas oprimidas durante el resto de la vida terrenal de Jesús; y cuando tuvo lugar el último y trágico episodio del relato de la vida de Jesús, pese a que todos los apóstoles excepto uno huyeron, todas estas mujeres se quedaron con él; ni una sola de ellas lo negó o traicionó.
\usection{3. EL \bibemph{SABBAT} EN TIBERIAS}
\vs p150 3:1 Por instrucciones de Jesús, Andrés había puesto los servicios del \bibemph{sabbat} del grupo apostólico en manos de las mujeres. Esto significaba, evidentemente, que no se podían celebrar en la nueva sinagoga. Las mujeres escogieron a Juana para que se responsabilizara de los preparativos, y la reunión ocurrió en la sala de banquetes del nuevo palacio de Herodes, al estar este fuera en su residencia de Julias de Perea. Juana leyó en las Escrituras acerca de la labor de la mujer en la vida religiosa de Israel, haciendo referencia, entre otras, a Miriam, a Débora y a Ester
\vs p150 3:2 \pc Más tarde aquella noche, Jesús impartió al grupo allí congregado una memorable charla sobre “La magia y la superstición”. En aquellos días, la aparición de una estrella, brillante y supuestamente nueva, se consideraba como la señal de que un gran hombre había nacido en la tierra. Así, al haberse observado recientemente una estrella semejante, Andrés preguntó a Jesús si estas creencias estaban bien fundadas. En la larga respuesta que le dio a Andrés, el Maestro abordó, con detenimiento, todo el tema de la superstición humana. Las palabras que pronunció Jesús en aquel momento se pueden resumir en terminología moderna de la siguiente manera:
\vs p150 3:3 \li{1.}El curso de las estrellas en los cielos no guarda ninguna relación con los acontecimientos de la vida humana en la tierra. La astronomía es el legítimo propósito de la ciencia, pero la astrología es un cúmulo de ideas supersticiosas que no tiene cabida en el evangelio del reino.
\vs p150 3:4 \li{2.}El examen de los órganos internos de un animal recientemente sacrificado no puede revelar nada del tiempo atmosférico, de sucesos futuros ni del resultado de actos humanos.
\vs p150 3:5 \li{3.}Los espíritus de los muertos no vuelven para comunicarse con sus familiares ni con sus antiguos amigos que todavía estén vivos.
\vs p150 3:6 \li{4.}Los amuletos y las reliquias son impotentes para curar enfermedades, no evitan los desastres ni influyen en los espíritus malignos; creer en estos medios materiales e intentar así tener poder sobre el mundo espiritual no es sino absoluta superstición.
\vs p150 3:7 \li{5.}Echar a suertes, aunque quizás pueda resultar de ayuda para la resolución de muchas pequeñas dificultades, no es un método encaminado a desvelar la voluntad divina. Los resultados que se desprenden de esta actividad obedecen a una pura casuística de orden físico. El único medio de comunión con el mundo espiritual se funda en el don espiritual del que la humanidad está dotada, en el espíritu morador otorgado por el Padre, junto con el espíritu derramado proveniente del hijo del Paraíso y la omnipresente influencia del Espíritu Infinito.
\vs p150 3:8 \li{6.}La adivinación, la hechicería y la brujería son supersticiones propias de mentes ignorantes, al igual que la engañosa ilusión de la magia. Creer en los números mágicos, en augurios de buena suerte y en presagios de mala suerte es superstición pura e infundada.
\vs p150 3:9 \li{7.}La interpretación de los sueños es, en gran medida, un sistema, supersticioso y sin fundamento, de especulación vana y fantasiosa. El evangelio del reino no debe tener nada en común con los sacerdotes adivinadores de las religiones primitivas.
\vs p150 3:10 \li{8.}Los espíritus del bien o del mal no pueden morar en símbolos físicos hechos de arcilla, madera o metal; los ídolos no consisten en nada más que en el propio material que se usa para fabricarlos.
\vs p150 3:11 \li{9.}Las prácticas de los encantadores, magos, hechiceros y brujos vienen de las supersticiones de los egipcios, los asirios, los babilonios y los antiguos cananeos. Los amuletos y todo tipo de encantamientos resultan inútiles tanto para granjearse la protección de los buenos espíritus como para evitar a los supuestos espíritus malignos.
\vs p150 3:12 \li{10.}Jesús puso al descubierto y denunció sus creencias en encantamientos, ordalías, hechicerías, maldiciones, signos, mandrágoras, cuerdas anudadas y cualquier otra forma de superstición estéril y esclavizante.
\usection{4. JESÚS ENVÍA A LOS APÓSTOLES DE DOS EN DOS}
\vs p150 4:1 La siguiente noche, habiendo reunido a los doce apóstoles, a los apóstoles de Juan y al recién nombrado grupo de mujeres, Jesús dijo: “Podéis ver por vosotros mismos que la mies es mucha, pero los obreros pocos; por tanto, rogad al Señor de la mies que envíe más obreros a su mies. Mientras me quedo aquí para confortar e instruir a los maestros más jóvenes, yo enviaré a los mayores de dos en dos para que recorran rápidamente toda Galilea predicando el evangelio del reino, siempre que sea todavía conveniente y haya sosiego”. Luego designó a las parejas de apóstoles, tal como él deseaba que salieran, y que fueron: Andrés y Pedro, Santiago y Juan Zebedeo, Felipe y Natanael, Tomás y Mateo, Santiago y Judas Alfeo, Simón Zelotes y Judas Iscariote.
\vs p150 4:2 Jesús fijó la fecha de su encuentro en Nazaret con los doce, y sus palabras de despedida fueron: “En esta misión, no vayáis a ninguna de las ciudades de los gentiles ni tampoco a Samaria; id en su lugar adonde están las ovejas extraviadas de la casa de Israel. Predicad el evangelio del reino y proclamad la verdad salvadora de que el hombre es un hijo de Dios. Recordad que el discípulo no es más que su maestro ni el siervo más que su señor. Bástale al discípulo ser como su maestro y al siervo como su señor. Si al padre de familia llamaron amigo de Beelzebú, ¡cuánto más a los de su casa! Así que no temáis a estos enemigos faltos de fe. Os hago saber que nada hay encubierto que no haya de ser descubierto; ni oculto que no haya de saberse. Lo que os enseñado en privado, predicadlo con sabiduría a plena luz; y lo que os he revelado al oído, proclamadlo a su tiempo desde las azoteas. Y yo os digo, amigos y discípulos míos, no temáis a los que matan el cuerpo pero el alma no pueden matar; temed más bien a aquel que puede destruir el alma; depositad mejor vuestra confianza en Aquel que puede sustentar el cuerpo y salvar el alma.
\vs p150 4:3 “¿No se venden dos pajarillos y por un cuarto? Con todo, ni uno de ellos está olvidado delante de Dios. ¿No sabéis que incluso los cabellos de vuestras cabezas están todos contados. No temáis, pues; más valéis vosotros que muchos pajarillos. No os avergoncéis de mi enseñanza; salid y proclamad paz y buena voluntad, pero no os engañéis ---vuestra predicación no estará siempre acompañada de paz---. He venido a traer paz a la tierra, pero cuando los hombres rechazan mi dádiva, se produce división y alboroto. Cuando todos los miembros de una familia reciben el evangelio del reino, verdaderamente la paz mora en esa casa; pero si algunos de la familia entran en el reino y otros rechazan el evangelio, esa división solo trae dolor y tristeza. Laborad fervientemente para salvar a toda la familia y que no sea el enemigo del hombre de su propia casa. Pero, cuando hayáis hecho lo posible por todos los de la familia, yo os declaro que el que ama a padre o madre más que este evangelio no es digno del reino”.
\vs p150 4:4 Cuando los doce oyeron estas palabras, se prepararon para partir. Y no volverían a estar juntos de nuevo hasta el momento en el que acudieron a Nazaret para reunirse con Jesús y con los otros discípulos, tal como el Maestro había dispuesto.
\usection{5. ¿QUÉ DEBO HACER PARA SER SALVA?}
\vs p150 5:1 Una noche en Sunem, una vez que los apóstoles de Juan habían regresado a Hebrón y a los apóstoles de Jesús se les había enviado de dos en dos, al encontrarse el Maestro enseñando a un grupo de doce de los evangelistas más jóvenes que realizaban su labor bajo la dirección de Jacob, junto con las doce mujeres, Raquel le hizo a Jesús esta pregunta: “Maestro, qué debemos responder cuando las mujeres nos preguntan: ¿Qué debo hacer para ser salva?”. Al oír a Raquel, él respondió:
\vs p150 5:2 \pc “Cuando los hombres y las mujeres os pregunten qué deben hacer para ser salvos, vosotras contestaréis: Creed en este evangelio del reino; recibid el perdón divino. Reconoced por la fe al espíritu morador de Dios, cuya aceptación os hace hijos de Dios. Es que no habéis leído en las Escrituras donde se afirma: ‘En el Señor tengo yo la justicia y la fuerza’. También, allí donde el Padre dice: ‘Muy cerca está mi justicia, ya ha salido mi salvación y mis brazos abrazarán a mi pueblo’. ‘Mi alma se gozará en el amor de mi Dios, porque me ha vestido con vestiduras de salvación y me rodeó del manto de su justicia’. Es que no habéis leído asimismo del Padre que ‘lo llamarán Señor, justicia nuestra’. ‘Quitadle esos harapos viles de hipocresía y vestid a mi hijo con el manto de la justicia divina y de la salvación eterna’. Es por siempre verdad que, ‘por su fe vivirá el justo’. La entrada en el reino del Padre es totalmente gratuita, pero avanzar ---crecer en la gracia--- es esencial para continuar allí.
\vs p150 5:3 “La salvación es el don del Padre, que sus hijos divinos revelan. Su aceptación, por vuestra parte, mediante la fe os hace partícipes de la naturaleza divina, en un hijo o hija de Dios. Por la fe estáis justificados; por la fe sois salvos; y por esta misma fe avanzaréis eternamente en un camino continuado de perfección divina. Por la fe fue Abraham justificado y las enseñanzas de Melquisedec le hicieron tomar conciencia de la salvación. A lo largo de todas las eras, esta misma fe ha salvado a los hijos del hombre, pero ahora ha venido del Padre un Hijo suyo para hacer más real y aceptable dicha salvación”.
\vs p150 5:4 \pc Cuando Jesús acabó de hablar, hubo un gran regocijo entre quienes habían oído sus afables palabras y, en los días siguientes, todos se pusieron a proclamar el evangelio del reino con un ímpetu nuevo y con renovadas fuerzas y entusiasmo. Y las mujeres se alegraron aún más al saber que se las incluía en estos planes para instaurar el reino en la tierra.
\vs p150 5:5 Resumiendo sus últimas palabras, Jesús dijo: “No podéis comprar la salvación ni podéis ganar la rectitud. La salvación es un don de Dios y, la rectitud ocurre naturalmente como resultado de la vida nacida del espíritu por vuestra filiación en el reino. No seréis salvos porque viváis una vida recta; sino, más bien, viviréis una vida recta porque ya habéis sido salvos, habéis reconocido la filiación como un don de Dios y el servicio en el reino como el gozo supremo de la vida en la tierra. Cuando los hombres creen en este evangelio, que es una revelación de la bondad de Dios, se verán llevados al arrepentimiento voluntario de todos los pecados conocidos. Ser consciente de la filiación es incompatible con el deseo de pecar. El que cree en el reino tiene sed de rectitud y de perfección divina”.
\usection{6. LECCIONES NOCTURNAS}
\vs p150 6:1 Durante las charlas que tenían por la noche, Jesús trató muchos temas. Durante el resto de este viaje ---antes de que volvieran a reunirse en Nazaret---, analizó “el amor de Dios”, “los sueños y las visiones”, “la malicia”, “la humildad y la mansedumbre”, “la valentía y la lealtad”, “la música y la adoración”, “el servicio y la obediencia”, “el orgullo y la arrogancia”, “el perdón en relación con el arrepentimiento”, “la paz y la perfección”, “la maledicencia y la envidia”, “el mal, el pecado y la tentación”, “las dudas y la increencia”, “la sabiduría y la adoración”. Al estar ausentes los apóstoles, de más edad, estos grupos más jóvenes de hombres se sentían con mayor libertad para participar en dichas charlas con el Maestro.
\vs p150 6:2 Tras pasar dos o tres días con algún grupo de doce evangelistas, Jesús se trasladaba para unirse a otro de ellos; los mensajeros de David le informaban del paradero y movimientos de todos estos trabajadores del reino. Siendo este su primer viaje, las mujeres se mantenían gran parte del tiempo con Jesús. Por medio del servicio de mensajería, cada uno de estos grupos estaba perfectamente informado sobre la marcha del viaje, y la llegada de noticias de los otros grupos servía siempre de estímulo a estos trabajadores, que se hallaban distribuidos por distintas partes y separados entre sí.
\vs p150 6:3 Antes de separarse, se había acordado que los doce apóstoles, junto con los evangelistas y el colectivo de mujeres, se congregarían en Nazaret para encontrarse con el Maestro el 4 de marzo. Así pues, por esta fecha, desde todas las partes de Galilea central y del sur, comenzaron a dirigirse a Galilea los distintos grupos de apóstoles y evangelistas. Hacia la media tarde, Andrés y Pedro, los últimos en llegar, arribaron al campamento, ya preparado por los primeros en llegar y emplazado en las colinas situadas al norte de la ciudad. Y esta fue la primera vez que Jesús iba a Nazaret desde el comienzo de su ministerio público.
\usection{7. ESTANCIA EN NAZARET}
\vs p150 7:1 Ese viernes por la tarde, Jesús caminó por Nazaret, absolutamente inadvertido y sin ser reconocido. Pasó por el hogar de su niñez y por el taller de carpintería, y estuvo media hora en la colina en la que tanto había disfrutado de joven. Desde el día en el que Juan lo bautizó en el Jordán, el Hijo del Hombre no había sentido tal caudal de emociones humanas en el alma. Al descender del monte, oyó los sonidos tan familiares del toque de trompeta que anunciaba la puesta del sol, tal como multitud de veces los había oído en Nazaret cuando era un muchacho. Antes de regresar al campamento, paseó por la sinagoga donde había ido a la escuela y dejó que su mente le trajera muchos recuerdos infantiles. Temprano ese día, Jesús había enviado a Tomás a organizar con el jefe de la sinagoga su predicación en el servicio matutino del \bibemph{sabbat}.
\vs p150 7:2 Nunca se había conocido a la gente de Nazaret por su piedad ni por su vida de rectitud. Con los años, la aldea se había ido contaminando crecientemente de los bajos principios morales de Séforis, una ciudad cercana. Durante la juventud de Jesús y su temprana edad adulta, había en Nazaret división de opiniones sobre él; su traslado a Cafarnaúm le granjeó mucho resentimiento. Aunque los habitantes de Nazaret habían oído hablar mucho de la labor de su antiguo carpintero, se sentían ofendidos por no haber incluido nunca, en ninguno de sus anteriores viajes de predicación, a su aldea natal. Les era ciertamente conocida la notoriedad de Jesús, pero la mayoría de la población estaba airada con él porque no había llegado a hacer ninguna de sus grandes obras en la ciudad de su juventud. Durante meses, la gente de Nazaret había hablado mucho de Jesús, pero las opiniones hacía él eran desfavorables por lo general.
\vs p150 7:3 Así pues, el Maestro se encontró en medio, no de una halagüeña bienvenida, sino de una atmósfera marcadamente hostil y muy crítica. Pero esto no era todo. Sus enemigos, sabiendo que pasaría este día del \bibemph{sabbat} en Nazaret y, suponiendo que hablaría en la sinagoga, habían contratado a un gran número de hombres toscos y burdos para hostigarlo y provocar el mayor alboroto posible.
\vs p150 7:4 La mayoría de los antiguos amigos de Jesús, incluyendo el dedicado maestro jazán, habían muerto o habían abandonado Nazaret, y la generación más joven tendía a sentirse molesta y muy celosa de la fama de Jesús. Se olvidaban de su temprana entrega a la familia de su padre, y lo criticaban duramente por su dejadez al no ir a ver a su hermano y a sus hermanas casadas, que vivían en Nazaret. La actitud de la familia de Jesús hacia él hizo aumentar la animosidad de la gente. Los judíos ortodoxos incluso llegaron a criticarlo porque andaba demasiado rápido de camino a la sinagoga en aquel \bibemph{sabbat} mañanero.
\usection{8. EL SERVICIO DEL \bibemph{SABBAT}}
\vs p150 8:1 Aquel \bibemph{sabbat} lucía un día hermoso, y todo Nazaret, amigos y enemigos, acudió a la sinagoga para escuchar el discurso de su antiguo conciudadano. Muchos miembros de la comitiva apostólica tuvieron que quedarse fuera; no había espacio suficiente para todos los que habían venido a oírlo. De joven, Jesús había hablado a menudo en este lugar de culto y, aquella mañana, al entregarle el jefe de la sinagoga el rollo de los escritos sagrados, de los que leería el pasaje del día en la Escritura, ninguno de los presentes pareció recordar que se trataba del mismo manuscrito que él había dado como ofrenda a la sinagoga.
\vs p150 8:2 Los servicios de ese día se celebraron de la misma manera que Jesús los había presenciado cuando era un muchacho. Subió a la tarima de las lecturas con el jefe de la sinagoga y comenzó el servicio con dos oraciones: “Bendito sea el Señor, Rey del mundo, que ha formado la luz y ha creado la oscuridad; que hace la paz y lo ha creado todo; que, en su misericordia, da luz a la tierra y a los que en ella habitan; que en su bondad, día tras día y cada día, renueva las obras de la creación. Bendito sea el Señor nuestro Dios por la gloria de su labor y por las luces que dan luz, que él ha hecho para su alabanza. \bibemph{Selah}. Bendito sea el Señor nuestro Dios, que ha formado la luz”.
\vs p150 8:3 Tras un momento de pausa, oraron de nuevo: “Con gran amor el Señor nuestro Dios nos ha amado y con desbordante piedad se ha compadecido de nosotros, nuestro Padre y nuestro Rey, por el bien de nuestros padres que confiaron en él. Tú les enseñaste los estatutos de la vida; ten misericordia de nosotros y enséñanos. Ilumina nuestros ojos en la ley; haz que nuestros corazones se aferren a tus mandamientos; une nuestros corazones para que te amemos y temamos tu nombre, y no seamos avergonzados, ahora y siempre. Porque tú eres el Dios de salvación, y nos escogiste entre todas las naciones y lenguas y, en verdad, nos has acercado a tu gran nombre ---\bibemph{Selah}--- para que podamos alabar amorosamente tu unidad. Bendito sea el Señor, que en amor eligió a su pueblo de Israel”.
\vs p150 8:4 La congregación recitó, entonces, el \bibemph{Shema,} del credo de fe judío. Este ritual consistía en repetir un gran número de pasajes de la ley y mostraba que los fieles aceptaban el yugo del reino de los cielos, al igual que el yugo del mandamiento de leerlo día y noche.
\vs p150 8:5 Y luego continuaron con la tercera oración: “Es verdad que tú eres Yahvé, nuestro Dios y el Dios de nuestros padres, nuestro Rey y el Rey de nuestros padres; nuestro Salvador y el Salvador de nuestros padres; nuestro Creador y la roca de nuestra salvación; nuestra ayuda y nuestro libertador. Tu nombre es desde lo eterno, y no hay otro Dios fuera de ti. Los que fueron liberados cantaron un nuevo cántico a tu nombre en la orilla del mar; juntos te alabaron y reconocieron como Rey y dijeron: Yahvé reinará, ahora y siempre. Bendito sea el Señor que salva a Israel”.
\vs p150 8:6 El jefe de la sinagoga ocupó su lugar ante el arca, o cofre, que contenía las escrituras sagradas y comenzó a recitar las diecinueve oraciones de alabanza, o bendiciones. Pero, en este momento, era aconsejable abreviar el servicio a fin de que el distinguido invitado pudiera tener más tiempo para su discurso; así pues, solo se recitaron la primera y la última de las bendiciones. La primera fue: “Bendito es el Señor nuestro Dios, y el Dios de nuestros padres, el Dios de Abraham, y el Dios de Isaac, y el Dios de Jacob; el magno, el poderoso y el Dios terrible, que muestra misericordia y bondad, que crea todas las cosas, que recuerda sus benevolentes promesas a los padres y que trae un salvador a los hijos de los hijos por causa de su propio nombre, en amor. ¡Oh Rey, auxilio, salvador y escudo! Bendito eres tú, oh Yahvé, escudo de Abraham”.
\vs p150 8:7 Luego siguió la última bendición: “¡Oh, concede a tu pueblo Israel por siempre gran paz, porque tú eres Rey y Señor de toda paz! Y es bueno en tus ojos bendecir a Israel en todo momento y en cada hora con la paz. Bendito eres tú, Yahvé, que bendices a tu pueblo Israel con paz”. La congregación no miraba al jefe de la sinagoga mientras él recitaba las bendiciones. Tras las bendiciones, ofreció una oración informal, apropiada para la ocasión, y cuando esto concluyó, toda la congregación se sumó para decir amén.
\vs p150 8:8 Luego el jazán fue al arca y sacó un rollo que entregó a Jesús para que leyera en la Escritura el pasaje del día. Existía la costumbre de llamar a siete personas para que leyeran no menos de tres versos de la ley, pero, en esta oportunidad, se omitió esta práctica para que el visitante pudiera leer el pasaje que él escogiese. Jesús, tomando el rollo, se puso de pie y comenzó a leer del Deuteronomio: “Porque este mandamiento que yo te ordeno hoy no está oculto de ti, ni está lejos de ti. No está en el cielo, para que digas: ¿Quién subirá por nosotros al cielo, nos lo traerá y nos lo hará oír para que lo cumplamos? Ni está al otro lado del mar, para que digas: ¿Quién pasará por nosotros el mar, para que nos lo traiga y nos lo haga oír, a fin de que lo cumplamos? Pues muy cerca de ti está la palabra de vida, en tu presencia y en tu corazón, para que la conozcas y la obedezcas”.
\vs p150 8:9 Y cuando terminó de leer de la ley, empezó a leer a Isaías: “El Espíritu del Señor está sobre mí, por cuanto me ha ungido para dar buenas nuevas a los pobres; me ha enviado a sanar a los quebrantados de corazón, a pregonar libertad a los cautivos y vista a los ciegos, a poner en libertad a los oprimidos y a predicar el año agradable del Señor”.
\vs p150 8:10 Jesús cerró el libro y, una vez que se lo devolvió al jefe de la sinagoga, se sentó y comenzó a hablarle a la gente: “Hoy se han cumplido estas Escrituras”. Y, luego, durante casi quince minutos, Jesús disertó sobre “Los hijos e hijas de Dios”. Muchos de los presentes se sintieron complacidos con el discurso, y se maravillaron de su gracia y sabiduría.
\vs p150 8:11 Tras concluir los servicios oficiales, era costumbre en la sinagoga que el orador permaneciera allí para que los que lo desearan pudieran hacerle preguntas. Por ello, esa mañana del \bibemph{sabbat,} Jesús descendió hasta donde estaba la multitud, que se amontonaba para acercarse a él. En este grupo, había muchos alborotadores que planeaban provocar disturbios y, alrededor de la multitud, se movían de un lado para otro aquellos hombres perversos, contratados para causarle problemas a Jesús. Muchos de los discípulos y evangelistas que se habían quedado fuera se abrieron camino para entrar en la sinagoga y no tardaron en darse cuenta de que se estaban gestando disturbios. Intentaron alejar de allí al Maestro, pero él se negó a irse con ellos.
\usection{9. EL RECHAZO DE NAZARET}
\vs p150 9:1 En la sinagoga, Jesús se vio rodeado de una gran aglomeración de enemigos y un puñado de sus propios seguidores, y en respuesta a las burdas preguntas y siniestras mofas, comentó, medio bromeando: “Sí, yo soy el hijo de José; soy el carpintero, y no me sorprende de que me recordéis el dicho ‘médico, cúrate a ti mismo’, y que me retéis a que haga en Nazaret lo que habéis oído que hice en Cafarnaúm; pero yo os llamo a que seáis testigos de que incluso en las Escrituras se afirma: ‘No hay profeta sin honra sino en su propia tierra, y entre su propio pueblo’”.
\vs p150 9:2 Pero lo empujaron y, señalándolo con el dedo acusador, le dijeron: “Tú crees que eres mejor que la gente de Nazaret; te fuiste de nosotros, pero tu hermano es un obrero ordinario y tus hermanas todavía viven entre nosotros. Conocemos a tu madre, María. ¿Donde están ellos hoy? Oímos grandes cosas de ti, pero vemos que cuando vuelves no obras ningún portento”. Jesús les contestó: “Amo a la gente que vive en la ciudad en la que crecí, y me sería de gozo veros a todos entrar en el reino de los cielos, pero no tengo decisión sobre los actos de Dios. Las transformaciones de la gracia suceden en respuesta a la fe viva de quienes se benefician de estas.
\vs p150 9:3 Con su dulce temperamento, Jesús hubiera podido contener a la multitud y logrado desarmar incluso a los más violentos, algo que hubiera sucedido de no haber sido por el proceder equivocado de uno de sus propios apóstoles, de Simón Zelotes. Este, con la ayuda de Nacor, uno de los evangelistas más jóvenes, había reunido, entretanto, a un grupo de amigos de Jesús de entre la muchedumbre y, adoptando una actitud beligerante, querían advertir a los enemigos del Maestro que se fueran de allí. Desde hacía mucho tiempo, Jesús había enseñado a sus apóstoles que una respuesta suave aplaca la ira, pero sus seguidores no estaban habituados a ver a su amado maestro, a quien tan voluntariosamente lo llamaban Maestro, tratado con tanto desprecio y falta de respeto. Resultó ser demasiado para ellos, y se vieron exteriorizando, con apasionamiento y vehemencia, una indignación que solo tendió a suscitar el furor colectivo de esta asamblea vergonzosa y tosca. Y, así, siguiendo el liderazgo de los matones a sueldo, estos rufianes prendieron a Jesús y lo sacaron precipitadamente de la sinagoga, llevándolo hasta la cumbre de un monte empinado con la intención de despeñarlo. Pero justo cuando estaban a punto de empujarlo, ya en el borde del acantilado, Jesús se volvió de repente hacia sus captores y, mirándoles de frente, se cruzó de brazos tranquilamente. No dijo nada, pero sus amigos se quedaron atónitos cuando, al empezar a caminar hacia adelante, la multitud se apartó y le dejó pasar sin molestarle.
\vs p150 9:4 Seguido de sus discípulos, Jesús se encaminó a su campamento, donde se habló de nuevo de todo lo sucedido. Y aquella noche se prepararon para volver a Cafarnaúm temprano al día siguiente, tal como Jesús había dispuesto. Este turbulento final del tercer viaje de predicación pública tuvo un efecto aleccionador sobre todos los seguidores de Jesús; empezaron a percatarse del significado de algunas de las enseñanzas del Maestro; estaban tomando conciencia de que el reino solo vendría tras mucho sufrimiento y amargas decepciones.
\vs p150 9:5 Aquel domingo por la mañana dejaron Nazaret y, tomando diferentes rutas, se congregaron finalmente en Betsaida hacia el mediodía del jueves, 10 de marzo. Al encontrarse, este grupo de predicadores del evangelio de la verdad estaba sombrío y serio; no era una tropa entusiasta y victoriosa de cruzados triunfantes.
