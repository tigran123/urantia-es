\upaper{196}{La fe de Jesús}
\author{Comisión de seres intermedios}
\vs p196 0:1 Jesús gozaba de una fe en Dios sublime e incondicional. Padeció los altibajos normales de la existencia humana, pero, en el terreno religioso, jamás dudó de la certeza del cuidado y de la guía de Dios. Su fe era consecuencia de una percepción espiritual nacida de la acción de la divina presencia, de su modelador interior. Su fe no era convencional ni meramente intelectual, sino totalmente personal y puramente espiritual.
\vs p196 0:2 El Jesús humano veía a Dios como santo, justo y grande, al igual que como verdadero, bello y bueno. En su mente, él encauzó todos estos atributos de la divinidad como “la voluntad del Padre de los cielos”. El Dios de Jesús era al mismo tiempo “el Santo de Israel” y “el Padre vivo y amoroso de los cielos”. El concepto de Dios como Padre no era privativo de Jesús, sino que él exaltó y elevó esta idea convirtiéndola en una experiencia sublime, al realizar una nueva revelación de Dios y al proclamar que toda criatura mortal es linaje de este Padre de amor, es un hijo de Dios.
\vs p196 0:3 Jesús no se aferró a la fe en Dios como lo haría una esforzada alma en guerra con el universo, que perseverara desesperadamente en su fe mientras se enfrenta a un mundo hostil y pecaminoso; no recurrió a la fe por el mero hecho de encontrar consuelo cuando estaba en medio de las dificultades o alivio ante la desesperación; su fe no era simplemente una forma de contrarrestar ilusoriamente las ingratas realidades y los pesares de la vida. Frente a todas estas adversidades naturales y a las contradicciones temporales de la existencia mortal, él experimentaba la tranquilidad de confiar suprema e inequívocamente en Dios y sintió la magnífica emoción de vivir, por la fe, en la presencia misma del Padre celestial. Y esta fe triunfante era el reflejo de una experiencia viva, el fruto de unos logros espirituales reales. La gran contribución de Jesús a los valores de la experiencia humana no fue su revelación de tantas nuevas ideas sobre el Padre de los cielos, sino más bien su demostración, de un modo tan magnífico y humano, de un nuevo y más elevado orden de \bibemph{fe viva en Dios}. Nunca, en ninguno de los mundos de este universo, en la vida de ningún mortal, llegó Dios a ser una \bibemph{realidad tan viva} como lo fue en la experiencia humana de Jesús de Nazaret.
\vs p196 0:4 En la vida del Maestro en Urantia, este y todos los demás mundos de la creación local descubren un nuevo orden, más elevado, de religión, de una religión basada en una relación personal con el Padre Universal, plenamente validada por la convicción suprema de su genuina experiencia personal. Esta fe viva de Jesús era algo más que reflexión intelectual, y no era en absoluto meditación de índole místico.
\vs p196 0:5 La teología puede fijar, formular, definir y dogmatizar la fe, si bien, en la vida humana de Jesús, la fe fue personal, viva, auténtica, espontánea y netamente espiritual. Esta fe no consistía en venerar la tradición ni en una mera creencia intelectual a la que se ciñera como a un credo sagrado, sino, que más bien fue una vivencia sublime y una profunda convicción que \bibemph{lo sustentaban con firmeza}. Su fe era tan real y totalizadora que alejó absolutamente de él cualquier duda espiritual y ciertamente acabó con cualquier deseo desacorde. Nada pudo hacerlo separar del anclaje espiritual de esta fe ferviente, sublime e inmutable. Incluso frente a la aparente derrota o en medio de decepciones y de la acechante angustia, se mantuvo calmado en la presencia divina, sin temor y enteramente consciente de ser espiritualmente invencible. Jesús gozó de la vivificante certeza de poseer una fe inquebrantable, y en cada una de las situaciones críticas de la vida demostró indefectiblemente una incuestionable lealtad hacia la voluntad del Padre. Y esta formidable fe permaneció irrefrenable incluso ante la cruel y aplastante amenaza de una muerte ignominiosa.
\vs p196 0:6 En personas de excepcional carácter religioso, sucede muy a menudo que una profunda fe espiritual puede llevarlas directamente a un funesto fanatismo, a la exageración del ego religioso, pero este no fue el caso de Jesús. Su vida cotidiana no se vio afectada desfavorablemente ni por su fe extraordinaria ni por sus logros espirituales, porque esta glorificación espiritual no era sino una expresión, plenamente inconsciente y espontánea, del alma como fruto de su experiencia personal con Dios.
\vs p196 0:7 La ardiente e irreductible fe espiritual de Jesús jamás declinó al fanatismo; su fe siempre permaneció apegada a su bien equilibrada actitud en cuanto a los valores relativos de las situaciones sociales, económicas y morales cotidianas y normales de la vida. La persona humana del Hijo del Hombre estaba espléndidamente unificada; era un ser divino dotado de perfección; asimismo, conciliaba magníficamente su doble naturaleza humana y divina, obrando en la tierra como una sola persona. El Maestro siempre supo compatibilizar la fe de su alma con unas razonables valoraciones, fruto de la madurez de la experiencia adquirida. En Jesús, la fe personal, la esperanza espiritual y la devoción a los valores morales estaban siempre correlacionados en una inigualable unidad religiosa, que conjuntaba armónicamente con su inteligente reconocimiento de la realidad y de la sacralidad de todas las lealtades humanas ---el honor personal, el amor familiar, el compromiso religioso, el deber social y la provisión de las necesidades económicas de la vida---.
\vs p196 0:8 A través de su fe, Jesús reconocía que todos los valores espirituales se hallaban en el reino de Dios; así pues, dijo: “Buscad primeramente el reino de los cielos”. En la idea de una futura fraternidad ideal, Jesús vio la consecución y el cumplimiento de la “voluntad de Dios”. El corazón mismo de la oración que enseñó a sus discípulos fue “venga tu reino; hágase tu voluntad”. Y habiendo, por tanto, concebido que el reino constituía la voluntad de Dios, se consagró a hacerlo realidad con una abnegación excepcional y un entusiasmo sin límites. Pero nunca, en toda su intensa misión ni a lo largo de su excepcional vida hubo el más mínimo asomo de airado fanatismo ni del superficial extremismo del ególatra religioso.
\vs p196 0:9 Esta fe viva, esta sublime experiencia religiosa, condicionó siempre toda su vida. Esta actitud espiritual imperaba completamente en sus pensamientos y sentimientos, en sus creencias y oración, en sus enseñanzas y predicación. Esta fe personal de un hijo en la certeza y la seguridad de la guía y protección del Padre celestial dotó a su extraordinaria vida de un profundo don de la conciencia de la realidad espiritual. Sin embargo, a pesar de tan gran profunda conciencia de su estrecha relación con la divinidad, este galileo, este Galileo de Dios, cuando se refirieron a él como “Maestro bueno”, respondió al instante: “¿Por qué me llamas bueno?”. Cuando abordamos esta espléndida abnegación, empezamos a entender cómo pudo el Padre Universal manifestarse en él con tal plenitud y revelarse por medio de él a los mortales de los mundos.
\vs p196 0:10 Como un hombre del mundo, Jesús hizo a Dios la más grande de las ofrendas: la consagración y la dedicación de su propia voluntad al majestuoso servicio del cumplimiento de la voluntad divina. Siempre y consecuentemente, Jesús interpretó la religión enteramente en su conformidad a la voluntad del Padre. Cuando os aproximéis a la andadura del Maestro en la tierra, en lo que se refiere a la oración o a cualquier otro rasgo de su vida religiosa, buscad no tanto qué fue lo que enseñó cómo qué fue lo que hizo. Jesús jamás oró porque se tratara de una obligación religiosa. Para él la oración expresaba fervientemente las actitudes espirituales, declaraba las lealtades del alma, describía la devoción personal, manifestaba gratitud, evitaba las tensiones emocionales, prevenía los conflictos, enaltecía el intelecto, ennoblecía los deseos, confirmaba las decisiones morales, enriquecía el pensamiento, vigorizaba las inclinaciones más elevadas, consagraba las motivaciones espirituales, aclaraba los puntos de vista, afirmaba la fe, conllevaba una suprema rendición de la voluntad, reafirmaba sublimemente la actitud de confianza, revelaba valentía, proclamaba el descubrimiento de la verdad, significaba la confesión de una devoción suprema, corroboraba la consagración a los valores, brindaba un modo de adaptarse a las dificultades y movilizaba inmensamente la acción conjunta de los poderes del alma para resistir la propensión humana hacia el egoísmo, el mal y el pecado. Jesús vivió una vida de oración consagrada a hacer la voluntad de su Padre y acabó su vida triunfalmente con este tipo de oración. El secreto de su inigualable religión fue esta conciencia de la presencia de Dios; y la alcanzó por medio de la oración inteligente y la adoración honesta ---la comunión ininterrumpida con Dios--- y no por medio de influencias paranormales, voces, visiones, apariciones o extrañas prácticas religiosas.
\vs p196 0:11 En la vida terrenal de Jesús, la religión significaba una experiencia viva, conllevaba el movimiento expreso y personal desde una actitud de veneración hasta el ejercicio práctico de la rectitud. La fe de Jesús rindió los frutos supremos del espíritu divino. Su fe no era inmadura ni crédula como la de un niño, aunque, en muchos aspectos recordaba de hecho la ingenua confianza de la mente de un niño; Jesús confiaba en Dios como el niño confía en su padre. Poseía una profunda confianza en el universo ---como la que tiene el niño en el entorno parental---. La incondicionada fe de Jesús en la bondad fundamental del universo era muy parecida a la confianza del niño en la seguridad de sus inmediaciones terrenales. Dependía del Padre celestial al igual que el niño busca el apoyo de su padre terrenal y, ni por un solo momento, su ferviente fe lo hizo dudar de la certeza de los amorosos cuidados del Padre celestial. Ni los temores ni las dudas ni el escepticismo le llegaron jamás a perturbar seriamente. Jamás la increencia dificultó la expresión libre y singular de su vida. Supo combinar el inquebrantable e inteligente arrojo de un hombre adulto con el optimismo franco y confiado de un niño crédulo. Su fe creció hasta niveles tan elevados que carecía de temor.
\vs p196 0:12 La fe de Jesús alcanzó la pureza de la confianza de un niño. Su fe era tan absoluta e incontestable que le deleitaban el contacto con sus semejantes y las maravillas del universo. Su sentido de dependencia hacia lo divino era tan absoluto y tan convencido que proporcionaba felicidad y la certidumbre de una absoluta seguridad personal. Nunca fue indeciso ni deshonesto en su experiencia religiosa. En este gran intelecto del hombre adulto, reinaba suprema la fe de un niño en todo lo referente a la conciencia religiosa. No es de extrañar que dijera alguna vez: “Si no os hacéis como niños, no entraréis en el reino”. Aunque la fe de Jesús era cándida \bibemph{como la de un niño,} no era \bibemph{infantil} en ningún sentido.
\vs p196 0:13 Jesús no requiere a sus discípulos que crean en él sino que crean \bibemph{con} él, que crean en la realidad del amor de Dios y que, con absoluta confianza, acepten la seguridad de la certeza de su filiación con el Padre celestial. El Maestro desea que todos sus seguidores participen totalmente de su fe suprema. Del modo más entrañable, Jesús retó a sus seguidores no solo a que creyeran lo \bibemph{que} él creía, sino también a que creyeran \bibemph{como} él creía. Este es el sentido pleno de su supremo mandato: “Sígueme”.
\vs p196 0:14 La vida terrenal de Jesús estuvo dedicada a un solo gran propósito: hacer la voluntad del Padre, a vivir su vida humana religiosamente y por medio de la fe. La fe de Jesús era confiada, como la de un niño, pero libre de toda presunción. Tomó decisiones difíciles y atrevidas, se enfrentó con arrojo a muchas decepciones, superó con resolución excepcionales obstáculos, hizo frente inquebrantablemente a los severos requisitos del deber. Se precisaba una voluntad fuerte y una indefectible confianza para creer lo que Jesús creía, y \bibemph{como} él lo creía.
\usection{1. JESÚS: EL HOMBRE}
\vs p196 1:1 La devoción de Jesús a la voluntad del Padre y al servicio del hombre fue incluso más que una decisión y una determinación de orden humano; fue una consagración de sí mismo a tal dádiva incondicional de amor. Con independencia de la magnitud de la soberanía de Miguel, no separéis al Jesús humano de los hombres. El Maestro ascendió a lo alto como hombre, al igual que como Dios; él pertenece a los hombres; los hombres le pertenecen. ¡Qué lamentable es que la religión se haya malinterpretado tanto como para arrebatarle el Jesús humano a los tenaces mortales! Que los estudios sobre la humanidad o la divinidad de Cristo no oscurezcan la verdad salvífica de que Jesús de Nazaret fue un hombre religioso que, por medio de la fe, consiguió conocer y hacer la voluntad de Dios; fue el hombre más verdaderamente religioso que jamás ha vivido en Urantia.
\vs p196 1:2 Ha llegado la hora de ser testigos de la resurrección figurada del Jesús humano tras estar sepultado, durante diecinueve siglos, en las tradiciones teológicas y en los dogmas religiosos. No debe sacrificarse durante más tiempo a Jesús de Nazaret ni siquiera en pro del magnífico concepto del Cristo glorificado. ¡Qué servicio tan extraordinario se prestaría si, mediante esta revelación, se rescatara al Hijo del Hombre de la tumba de la teología tradicional y se presentara a la Iglesia que lleva su nombre, y a todas las demás religiones, como el Jesús vivo! Seguramente, la fraternidad cristiana de creyentes no dudará en realizar esas transformaciones de su fe y prácticas de vida que le permitan “ir en pos” del Maestro, haciendo muestras de la vida real de Jesús y de su devoción religiosa a la voluntad del Padre y de consagración al servicio desinteresado del hombre. ¿Es que los cristianos que se declaran como tales temen que se desvele que constituyen una fraternidad arrogante y sin dedicación, con una fachada de respetabilidad social y con unos egoístas intereses económicos? ¿Es que el cristianismo oficial teme que la autoridad eclesiástica tradicional esté en posible riesgo, o sea incluso derrocada, si Jesús de Galilea se restituye en las mentes y en las almas de los hombres mortales como el ideal de la vida religiosa personal? De hecho, los reajustes sociales, las transformaciones económicas, la revigorización moral y la reconstrucción religiosa de la civilización cristiana serían drásticas y revolucionarias si la religión viva de Jesús reemplazara de repente a la religión teológica sobre Jesús.
\vs p196 1:3 \pc “Ir en pos de Jesús” supone compartir personalmente su fe religiosa y adentrarse en el espíritu de la vida del Maestro, que estaba dedicada al servicio desinteresado del hombre. Una de las cosas más importantes de la vida humana es descubrir qué creía Jesús, averiguar cuáles eran sus ideales y aspirar a lograr ese sublime propósito de vida. De todo el saber humano, el de mayor valor es el conocimiento de la vida religiosa de Jesús y la forma en la que él la vivió.
\vs p196 1:4 La gente común oía a Jesús placenteramente, y de nuevo responderá a la declaración de su honesta vida humana, consagrada y religiosamente motivada, si se proclaman estas verdades una vez más al mundo. La gente lo oía de buena gana porque era uno de ellos, un modesto laico; el más grande maestro religioso del mundo fue, en efecto, un laico.
\vs p196 1:5 La meta de los creyentes del mundo no debería ser imitar literalmente la vida diaria de Jesús en la carne, sino más bien compartir su fe; confiar en Dios como él confió en Dios y creer en los hombres como él creyó en ellos. Jesús nunca debatió ni sobre la paternidad de Dios ni sobre la hermandad de los hombres; él constituía un ejemplo vivo de lo uno y una profunda demostración de lo otro.
\vs p196 1:6 Al igual que los hombres deben progresar desde la conciencia de lo humano hasta el reconocimiento de lo divino, de igual manera ascendió Jesús desde la naturaleza del hombre a la conciencia de la naturaleza de Dios. Y el Maestro realizó este gran ascenso desde lo humano a lo divino gracias a los logros de la fe de su intelecto mortal junto a la acción de su modelador interior. La comprensión del hecho de haber adquirido la divinidad total (siendo al mismo tiempo tan completamente consciente de la realidad de su humanidad) se produjo en siete etapas, en las que, mediante la fe, se fue haciendo consciente de su creciente divinidad:
\vs p196 1:7 \li{1.}La llegada del modelador del pensamiento.
\vs p196 1:8 \li{2.}El mensajero de Emanuel que se le apareció en Jerusalén cuando tenía unos doce años.
\vs p196 1:9 \li{3.}Las manifestaciones que acompañaron a su bautismo.
\vs p196 1:10 \li{4.}Las experiencias tenidas en el monte de la transfiguración.
\vs p196 1:11 \li{5.}La resurrección morontial.
\vs p196 1:12 \li{6.}La ascensión como espíritu.
\vs p196 1:13 \li{7.}La acogida final del Padre del Paraíso, que le confirió la soberanía ilimitada sobre su universo.
\usection{2. LA RELIGIÓN DE JESÚS}
\vs p196 2:1 Tal vez algún día haya una reforma de la Iglesia cristiana lo suficientemente profunda como para lograr el retorno a las enseñanzas religiosas, impolutas, de Jesús, el autor y consumador de nuestra fe. Podéis \bibemph{predicar} una religión \bibemph{sobre} Jesús, pero, forzosamente, debéis \bibemph{vivir} la religión \bibemph{de} Jesús. En el entusiasmo de Pentecostés, Pedro involuntariamente inauguró una nueva religión, la religión del Cristo resucitado y glorificado. Más tarde, el apóstol Pablo transformó este nuevo evangelio en el cristianismo, en una religión que incorporaba sus propias opiniones teológicas y narraba su \bibemph{experiencia personal} con el Jesús del camino de Damasco. El evangelio del reino se basa en la experiencia religiosa personal de Jesús de Galilea; en cambio, el cristianismo se basa casi exclusivamente en la experiencia religiosa personal del Apóstol Pablo. Prácticamente, el Nuevo Testamento, más que estar dedicado a la descripción de la trascendental y alentadora vida religiosa de Jesús, lo está al relato de la experiencia religiosa de Pablo y a la exposición de sus convicciones religiosas personales. Las únicas notorias excepciones en este sentido son, además de ciertos pasajes de Mateo, Marcos y Lucas, el libro de los Hebreos y la Epístola de Santiago. El mismo Pedro solo una vez retomó en sus escritos la vida personal religiosa de su Maestro. El Nuevo Testamento es un formidable documento cristiano, aunque solo sea jesusiano escasamente.
\vs p196 2:2 La vida de Jesús en la carne muestra un desarrollo religioso supremo desde las tempranas ideas del primitivo temor reverencial humano, pasando por los años de comunión espiritual personal, hasta llegar finalmente a ese estatus avanzado y sublime en el que él es consciente de su unicidad con el Padre. Y, de este modo, Jesús, en una sola breve vida, experimentó todo el crecimiento religioso que el hombre, tras comenzar en la tierra, no consigue ordinariamente hasta que no concluye su largo viaje por las escuelas de formación espiritual durante su andadura, a través de los sucesivos niveles con antelación a su llegada al Paraíso. Jesús avanzó desde una conciencia, puramente humana, empezando a sentir las certezas de su propia experiencia religiosa personal, hasta llegar a las sublimes alturas espirituales, cuando reconoció definitivamente su naturaleza divina y tomó conciencia de su estrecha relación con el Padre Universal en el gobierno de un universo. Progresó desde la humilde condición de dependencia humana, que lo impulsó espontáneamente a decir al que lo llamó Maestro bueno, “¿Por qué me llamas bueno? Nadie es bueno sino Dios”, hasta esa conciencia sublime de haber alcanzado la divinidad, que lo llevaría a exclamar: “¿Quién de vosotros puede acusarme de pecado?”. Este ascenso gradual de lo humano a lo divino fue exclusivamente un logro suyo como humano. Y cuando alcanzó, pues, la divinidad, él seguía siendo el mismo Jesús humano, el Hijo del Hombre al igual que el Hijo de Dios.
\vs p196 2:3 Marcos, Mateo y Lucas conservan en sus textos la visión de Jesús comprometido en su formidable esfuerzo por descubrir la voluntad divina y llevarla a cabo. Juan presenta la imagen de un Jesús triunfante, caminando por la tierra totalmente consciente de su divinidad. La gran equivocación cometida por los estudiosos de su vida es hay quienes han concebido a Jesús como enteramente humano, mientras que otros lo han visto como únicamente divino. Durante toda su existencia material, él fue en verdad tanto humano como divino, tal como aún lo es.
\vs p196 2:4 Pero el error más grande se cometió cuando, pese a reconocerse que el Jesús humano \bibemph{tenía} una religión, el Jesús divino (Cristo), casi de la noche a la mañana, se convirtió en sí mismo en una religión. El cristianismo de Pablo logró que se adorara al Cristo divino, pero perdió casi por completo de vista al Jesús humano, al Jesús esforzado y valiente de Galilea, el cual, por la determinación de su fe personal religiosa y el heroísmo de su modelador interior, ascendió desde unos humildes niveles de humanidad hasta hacerse uno con la divinidad, convirtiéndose, pues, en el camino, nuevo y vivo, por el que todos los mortales puedan ascender desde la humanidad a la divinidad. En todas sus etapas de espiritualidad y en todos los mundos, los mortales pueden encontrar, en la vida personal de Jesús, aquello que los fortalecerá e inspirará conforme avanzan desde los más modestos niveles espirituales hasta los más elevados valores divinos, desde el comienzo de toda su experiencia religiosa personal hasta el fin.
\vs p196 2:5 En el momento de la redacción del Nuevo Testamento, sus autores no solo creían muy profundamente en la divinidad del Cristo resucitado, sino que también creían ferviente y sinceramente en su inmediato regreso a la tierra para consumar el reino celestial. Esta sólida fe en la vuelta inminente del Señor tuvo mucho que ver con la inclinación a no dejar constancia de las experiencias y las cualidades puramente humanas del Maestro. Todo el movimiento cristiano tendió a alejarse de la imagen humana del Jesús de Nazaret, favoreciendo la exaltación del Cristo resucitado, el Señor Jesucristo glorificado que pronto volvería.
\vs p196 2:6 \pc Jesús fundó la religión de la experiencia personal del cumplimiento de la voluntad de Dios a través de la hermandad de los hombres; Pablo fundó una religión en la que Jesús glorificado se convirtió en objeto de veneración y la hermandad constaba de compañeros creyentes en el Cristo divino. En el ministerio de gracia de Jesús, estos dos conceptos estaban en potencia en su vida humana y divina y es, de hecho, lamentable que estos seguidores no lograran crear una religión unificada que hubiera reconocido debidamente las dos naturalezas del Maestro, la humana y la divina, ambas inseparablemente enlazadas en su vida terrenal y tan gloriosamente manifestadas en el evangelio primigenio del reino.
\vs p196 2:7 No os dejaríais consternar ni inquietar por la firmeza de algunas de las aseveraciones de Jesús, si recordarais que él fue el creyente religioso más incondicional y entregado del mundo. Fue un mortal totalmente consagrado, dedicado sin reserva a hacer la voluntad de su Padre. Muchos de sus dichos aparentemente severos fueron más una confesión personal de fe y de su propio compromiso de devoción que mandatos dirigidos a sus seguidores. Y este mismo propósito único junto a su desinteresada devoción le posibilitó realizar, en una única breve vida, un avance de tal extraordinaria naturaleza en la conquista de su mente humana. Muchas de sus afirmaciones deben considerarse más como una confesión de lo que él se exigía a sí mismo que requisitos que él exigiera a sus seguidores. En su devoción a la causa del reino, Jesús quemó puentes tras de sí; renunció a todo lo que representara para él un obstáculo al cumplimiento de la voluntad de su Padre.
\vs p196 2:8 Jesús bendijo a los pobres porque eran generalmente francos y píos; censuraba a los ricos porque eran normalmente disolutos e irreligiosos. De igual manera, censuraría al pobre que fuera irreligioso y elogiaría al hombre rico consagrado y reverente.
\vs p196 2:9 Jesús ayudaba a los hombres a que se sintieran en el mundo como en su propio hogar; los liberaba de la esclavitud de los tabúes y les enseñaba que el mundo no era esencialmente malo. Jesús no deseaba escapar de la vida terrenal; mientras estaba en la carne, aprendió el modo de hacer razonablemente la voluntad del Padre. Llevó una vida religiosa idealista en medio mismo de un mundo realista. Jesús no compartía la opinión pesimista de Pablo sobre la humanidad. El Maestro consideraba a los hombres como hijos de Dios y preveía un futuro magnífico y eterno para quienes optaran por sobrevivir. No era un escéptico moral; contemplaba al hombre de forma positiva, no negativa. Veía a la mayoría de los hombres más como personas débiles que malvadas, más como personas desconsoladas que depravadas. Pero al margen del estatus que tuvieran, todos eran hijos de Dios y hermanos suyos.
\vs p196 2:10 196:2.10 (2093.4) Enseñó a los hombres a que se dieran a sí mismo un alto valor en el tiempo y en la eternidad. Debido a esta gran valía que atribuía al hombre, Jesús estuvo dispuesto a emplearse a sí mismo en el empeño de servir incansablemente a la humanidad. Y fue esta infinita valoración de lo finito la que hizo que la regla de oro fuera un componente vital de su religión. ¿Qué mortal no podría sentirse inspirado por la fe extraordinaria que Jesús tiene en él?
\vs p196 2:11 Jesús no dictó normas para el progreso social; su misión era religiosa, y la religión es una experiencia exclusivamente individual. La meta última del más avanzado logro de la sociedad nunca podrá trascender la hermandad de los hombres anunciada por Jesús y basada en el reconocimiento de la paternidad de Dios. El ideal de cualquier logro social solo puede alcanzarse con la llegada de este reino divino.
\usection{3. LA SUPREMACÍA DE LA RELIGIÓN}
\vs p196 3:1 La experiencia religiosa espiritual de índole personal disuelve satisfactoriamente la mayoría de las dificultades mortales; organiza, evalúa y resuelve eficazmente cualquier problema humano. La religión no quita ni elimina los inconvenientes humanos, pero ciertamente los disuelve, los absorbe, los ilumina y los trasciende. La verdadera religión unifica a la persona para que pueda adecuarse con éxito a todas las exigencias humanas. La fe religiosa ---la beneficiosa guía de la presencia divina interior--- indefectiblemente capacita al hombre conocedor de Dios para salvar el abismo existente entre la lógica intelectual que reconoce la Primera Causa Universal como \bibemph{Ello} y las incontestables afirmaciones del alma que declaran que la Primera Causa es \bibemph{Él,} el Padre celestial del evangelio de Jesús, el Dios personal de la salvación humana.
\vs p196 3:2 En la realidad universal existen simplemente tres elementos: hecho, idea y relación. La conciencia religiosa identifica estas realidades como ciencia, filosofía y verdad. La filosofía se inclina a considerar estos factores como razón, sabiduría y fe ---realidad física, realidad intelectual y realidad espiritual---. Nosotros solemos denominar estas realidades como cosa, significado y valor.
\vs p196 3:3 Comprender progresivamente la realidad equivale a acercarse a Dios. Encontrar a Dios, la conciencia de identidad con la realidad, equivale a experimentar la compleción del yo ---la integridad del yo, la totalidad del yo---. La experiencia de la realidad total constituye la comprensión plena de Dios, la completud de la experiencia de conocer a Dios.
\vs p196 3:4 La suma total de la vida humana consiste en el conocimiento de que el hombre se educa mediante los hechos, se ennoblece mediante la sabiduría y se salva ---se justifica--- mediante la fe religiosa.
\vs p196 3:5 La certidumbre física reside en la lógica de la ciencia; la certidumbre moral, en la sabiduría de la filosofía; la certidumbre espiritual, en la verdad de una auténtica experiencia religiosa.
\vs p196 3:6 La mente del hombre puede lograr altos niveles de percepción espiritual y sus esferas correspondientes de valores divinos, porque no es completamente material. Hay en la mente del hombre un núcleo espiritual ---el modelador, que es la presencia divina del Padre---. Hay tres distintas evidencias que apoyan la presencia de este espíritu morador de la mente humana:
\vs p196 3:7 \li{1.}La fraternidad humanitaria: el amor. La mente puramente animal puede ser gregaria porque busca su autoprotección, pero solo el intelecto inhabitado por el espíritu es desinteresadamente altruista e incondicionalmente amoroso.
\vs p196 3:8 \li{2.}La interpretación del universo: la sabiduría. Solo la mente inhabitada por el espíritu puede comprender que el universo es amigable hacia la persona.
\vs p196 3:9 \li{3.}La evaluación espiritual de la vida: la adoración. Solo el hombre inhabitado por el espíritu puede darse cuenta de la presencia divina y buscar la consecución de una experiencia más plena en este y con este preludio de la divinidad.
\vs p196 3:10 \pc La mente humana no crea valores verdaderos; la experiencia humana no produce la percepción del universo. Con respecto a la percepción ---el reconocimiento de los valores morales y el discernimiento de los significados espirituales---, todo lo que la mente humana puede hacer es descubrir, reconocer, interpretar y \bibemph{tomar una opción}.
\vs p196 3:11 Los valores morales del universo se convierten en posesiones intelectuales mediante el ejercicio de los siguientes tres juicios básicos, u opciones, que la mente mortal toma:
\vs p196 3:12 \li{1.}Juicio ante sí mismo: opción moral.
\vs p196 3:13 \li{2.}Juicio ante lo social: opción ética.
\vs p196 3:14 \li{3.}Juicio ante Dios: opción religiosa.
\vs p196 3:15 \pc Por lo tanto, parece que todo progreso humano se lleva a efecto, conjuntamente, tanto por la \bibemph{evolución} como por la \bibemph{revelación}.
\vs p196 3:16 A no ser que viviera en él un amante divino, el hombre no podría amar desinteresada ni espiritualmente. A no ser que residiera en su mente un intérprete, no podría ser verdaderamente consciente de la unidad del universo. A no ser que habitará en él un evaluador, al hombre no le sería posible apreciar los valores morales y reconocer los significados espirituales. Y este amante proviene de la fuente misma del amor infinito; este intérprete es parte de la Unidad Universal; este evaluador es vástago del Centro y Fuente de todos los valores absolutos de la realidad divina y eterna.
\vs p196 3:17 La evaluación moral con un significado religioso ---la percepción espiritual--- implica optar individualmente entre el bien y el mal, la verdad y el error, lo material y lo espiritual, lo humano y lo divino, el tiempo y la eternidad. La supervivencia humana depende, en gran medida, de que la voluntad humana se consagre a optar por aquellos valores elegidos por este organizador de valores espirituales ---el intérprete y unificador interior---. La experiencia religiosa personal consta de dos fases: descubrimiento en la mente humana y revelación por el espíritu divino interior. Por una innecesaria complejidad o como resultado de la conducta irreligiosa de creyentes profesos, un hombre, o incluso toda una generación de hombres, pueden optar por suspender sus intentos de descubrir al Dios que habita en ellos; pueden desatender, y no lograr pues, avanzar en la revelación divina. Pero estas actitudes de paralización del progreso espiritual no pueden persistir por mucho tiempo a causa de la presencia e influencia de los modeladores interiores del pensamiento.
\vs p196 3:18 Esta profunda experiencia de la realidad de la inhabitación divina trasciende para siempre el tosco método materialista de las ciencias físicas. No podéis colocar el gozo espiritual bajo un microscopio; no podéis pesar el amor en una balanza; no podéis medir los valores morales; ni tampoco podéis hacer un cálculo cualitativo de la adoración espiritual.
\vs p196 3:19 Los hebreos tenían una religión moralmente sublime; los griegos desarrollaron una religión de la belleza; Pablo y sus seguidores fundaron una religión de fe, esperanza y caridad. Jesús reveló y ejemplificó una religión de amor: seguridad en el amor del Padre, junto con el gozo y la satisfacción de compartir este amor en el servicio a la hermandad de los hombres.
\vs p196 3:20 Cada vez que toma una meditada opción moral, el hombre siente que su alma se invade nuevamente de divinidad. Tal elección moral establece a la religión como el motivo que impulsa una respuesta interior a las condiciones exteriores. Pero esta religión es real y no una experiencia puramente subjetiva. Implica, a nivel individual, que la subjetividad total responde de manera significativa e inteligente a la objetividad total ---al universo y su Hacedor---.
\vs p196 3:21 La espléndida y suprema experiencia de amar y ser amado no es simplemente una ilusión psíquica solo porque sea tan puramente subjetiva. La única realidad verdaderamente divina y objetiva que está vinculada a los seres mortales, el modelador del pensamiento, actúa, aparentemente, ante la observación humana, como un fenómeno exclusivamente subjetivo. El contacto del hombre con la más elevada realidad objetiva, Dios, solo se realiza mediante la experiencia puramente subjetiva de conocerlo, adorarlo y lograr la filiación con él.
\vs p196 3:22 La verdadera adoración religiosa no es un vano e ilusorio monólogo. La adoración entraña comunión personal con lo que es divinamente real, con aquello que constituye la fuente misma de la realidad. Mediante la adoración, el hombre aspira a ser mejor y, por consiguiente, acaba consiguiendo lo \bibemph{mejor}.
\vs p196 3:23 Idealizar y tratar de servir la verdad, la belleza y la bondad no sustituyen a la genuina experiencia religiosa ---a la realidad espiritual---. La psicología y el idealismo no equivalen a la realidad religiosa. Las extrapolaciones del intelecto humano pueden, de hecho, dar origen a dioses falsos ---dioses a imagen del hombre--- pero la verdadera conciencia de Dios no tiene ese origen. La conciencia de Dios reside en el espíritu interior. Muchos de los sistemas religiosos del hombre son elaboraciones del intelecto humano, pero ciertamente la conciencia de Dios no forma parte de estos grotescos sistemas religiosos esclavistas.
\vs p196 3:24 Dios no es en absoluto una invención del idealismo del hombre; él es la fuente misma de dichas percepciones y de los valores supraanimales. Dios no es una hipótesis formulada para unificar los conceptos humanos de la verdad, la belleza y la bondad; él es el ser personal de amor de quien derivan todas estas manifestaciones del universo. La verdad, la belleza y la bondad del mundo del hombre se unifican gracias a la creciente espiritualidad que los mortales experimentan, conforme ascienden hacia las realidades del Paraíso. La unidad de la verdad, la belleza y la bondad solo puede llevarse a cabo en la experiencia espiritual de la persona conocedora de Dios.
\vs p196 3:25 La moral es la base esencial preexistente a la conciencia personal de Dios, a la toma de conciencia personal de la presencia interior del modelador, pero esta moral no es la fuente de la experiencia religiosa ni de la consecuente percepción espiritual. La naturaleza moral es supraanimal aunque subespiritual. La moral equivale a reconocer el deber, a ser conscientes de la existencia del bien y del mal. La zona de lo moral media entre el orden de mente animal y el humano, al igual que la morontia actúa entre los ámbitos materiales y espirituales como consecuciones que son del ser personal.
\vs p196 3:26 La mente evolutiva es capaz de descubrir leyes, principios morales y ética; pero el espíritu otorgado de gracia, el modelador interior, revela a la mente evolutiva humana al legislador, al Padre, que es la fuente de todo lo que es verdad, bello y bueno; y este hombre, así iluminado, posee una religión, que lo equipa espiritualmente para comenzar la larga y trepidante búsqueda de Dios.
\vs p196 3:27 La moral no es necesariamente espiritual; puede ser completa y puramente humana. Si bien, la verdadera religión realza todos los valores morales, los hace más significativos. La moral sin religión no logra revelar la bondad última, ni tampoco posibilita la supervivencia de ni siquiera sus propios valores morales. La religión produce el enaltecimiento, la glorificación y la supervivencia segura de todo lo que la moral reconoce y aprueba.
\vs p196 3:28 La religión está por encima de la ciencia, el arte, la filosofía, la ética y la moral, pero no es independiente de ellos. Todos están indisolublemente interrelacionados en la experiencia humana, personal y social. La religión constituye la experiencia suprema del hombre mortal; pero el lenguaje finito hace por siempre imposible que la teología pueda alguna vez describir adecuadamente la experiencia religiosa real.
\vs p196 3:29 \pc La percepción religiosa posee el poder de convertir la derrota en deseos de orden superior y en nuevos propósitos y metas. En su ascenso en el universo, el amor es la más alta motivación que pueda mover al hombre. Pero, cuando al amor se le despoja de verdad, belleza y bondad, no es sino un sentimiento, una tergiversación filosófica, una ilusión psíquica, una decepción espiritual. El amor ha de redefinirse constantemente, a medida que sucesivamente se progresa por los distintos niveles morontiales y espirituales.
\vs p196 3:30 \pc El arte resulta del intento del hombre por escapar de la falta de belleza en su entorno material; es un gesto tendido al nivel morontial. La ciencia representa el esfuerzo del hombre por resolver los aparentes enigmas del universo material. La filosofía constituye el empeño del hombre por unificar la experiencia humana. La religión supone un gesto supremo y magnífico del hombre desplegado al alcance de la realidad final, su determinación de encontrar a Dios y de ser como él es.
\vs p196 3:31 \pc En el ámbito de la experiencia religiosa, la posibilidad espiritual entraña realidad potencial. El impulso espiritual que incita al hombre a seguir no es una ilusión psíquica. Es posible que el romance del hombre con el universo no constituya un hecho, pero tiene mucho, muchísimo de verdad.
\vs p196 3:32 La vida de algunos hombres es demasiado magnífica y noble como para descender al humilde nivel del mero éxito material. El animal debe adaptarse a su entorno, pero el hombre religioso transciende su propio entorno y se escapa de las imperantes limitaciones del mundo material a través de su percepción del amor divino. Este concepto de amor genera en el alma del hombre ese afán supraanimal que lo lleva a buscar la verdad, la belleza y la bondad; y, cuando de hecho las encuentra, se glorifica al aceptarlas, y le consume el deseo de vivirlas, de obrar con rectitud.
\vs p196 3:33 No os desaniméis; la evolución humana continúa su avance, y la revelación de Dios al mundo, en Jesús y a través de Jesús, no fracasará.
\vs p196 3:34 El gran reto del hombre moderno es lograr una mejor comunicación con el mentor divino que habita en la mente humana. La aventura más grande del hombre en la carne consiste en su empeño, bien equilibrado y sensato, de avanzar más allá de las lindes de su autoconciencia, a través de los difusos ámbitos de la conciencia embrionaria del alma, en un ferviente esfuerzo por alcanzar la zona fronteriza de la conciencia espiritual ---el contacto con la presencia divina---. Esta experiencia constituye la conciencia de Dios, una experiencia que confirma grandiosamente la preexistente verdad de la vivencia religiosa de conocer a Dios. Esta conciencia del espíritu equivale al conocimiento de la realidad de la filiación con Dios y, por otro lado, la certeza de la filiación es la experiencia de la fe.
\vs p196 3:35 Y la conciencia de Dios equivale a la integración del yo con el universo, y en sus más elevados niveles de la realidad espiritual. Solo el contenido espiritual de cualquier valor es imperecedero. Incluso aquello que es verdadero, bello y bueno no perece en la experiencia humana. Si el hombre no opta por sobrevivir, entonces, el modelador, que sí sobrevive, conservará esas realidades nacidas del amor y nutridas en el servicio. Y todas estas cosas son parte del Padre Universal. El Padre es amor vivo, y esta vida del Padre está en sus Hijos. Y el espíritu del Padre está en los hijos de sus Hijos ---en los hombres mortales---. Cuando todo está dicho y hecho, la idea del Padre sigue siendo todavía el más excelso concepto humano de Dios.
\separatorline
