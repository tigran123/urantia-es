\upaper{148}{Formación de los evangelistas en Betsaida}
\author{Comisión de seres intermedios}
\vs p148 0:1 Desde el 3 de mayo hasta el 3 de octubre del año 28 d. C., Jesús y el cuerpo apostólico vivieron en Betsaida, en la casa de Zebedeo. Durante todo este período de cinco meses de la estación seca, se mantuvo un gran campamento junto al mar, cerca de la residencia de Zebedeo, que se amplió considerablemente para dar cabida a la creciente familia de Jesús. Lo ocupaba una población, en constante cambio, de buscadores de la verdad, aspirantes a curaciones y adeptos a fenómenos inusuales, que sumaban entre quinientas y mil quinientas personas. Esta ciudad de tiendas estaba bajo la supervisión general de David Zebedeo, con la asistencia de los gemelos Alfeo. El campamento era un modelo de orden y salubridad, al igual que en cuanto a la gestión a nivel general. Los enfermos con distintas enfermedades estaban separados y bajo la supervisión de un médico creyente: el sirio Elman.
\vs p148 0:2 Durante todo este tiempo, los apóstoles iban de pesca al menos un día a la semana y vendían su captura a David para el consumo de los habitantes del campamento costero. Los fondos recibidos por este medio se entregaban al tesorero del grupo. Cada mes se permitía a los doce pasar una semana con sus familias o amigos.
\vs p148 0:3 Aunque Andrés seguía encargándose en general de todas las actividades apostólicas, Pedro tenía la escuela de los evangelistas bajo su completa responsabilidad. Cada mañana, todos los apóstoles participaban enseñando a los grupos de evangelistas y, por la tarde, tanto instructores como alumnos enseñaban a su vez a la gente. Tras la cena, cinco noches a la semana, los apóstoles, en beneficio de los evangelistas, impartían clases para responder a las preguntas que tuviesen. Una vez por semana, Jesús presidía esta sesión, contestando a las preguntas pendientes de sesiones previas.
\vs p148 0:4 En cinco meses, varios miles de personas pasaron por este campamento. Acudían con asiduidad personas interesadas en las enseñanzas que allí se impartían. Procedían de todas las partes del Imperio romano y de las tierras del este del Éufrates. Fue el período de mayor duración y mejor organización ocurrido en la labor formativa del Maestro. La familia directa de Jesús estuvo gran parte de este tiempo en Nazaret o Caná.
\vs p148 0:5 El campamento no estaba organizado a la manera de una comunidad con intereses comunes, como lo era la familia apostólica. David Zebedeo dirigía esta gran ciudad de tiendas para que fuera autosuficiente, pero nunca se rechazó a nadie. Este campamento, en continuo cambio, fue un elemento indispensable en la escuela de formación evangélica de Pedro.
\usection{1. LA NUEVA ESCUELA DE LOS PROFETAS}
\vs p148 1:1 Pedro, Santiago y Andrés eran los miembros de la comisión nombrada por Jesús y que se encargaba de dar su aprobación a los aspirantes que solicitaban ser admitidos en la escuela de evangelistas. Entre los estudiantes de esta nueva escuela de los profetas, había representantes de todas las razas y nacionalidades del mundo romano y del Oriente, hasta incluso de la lejana India. El programa de esta escuela consistía en aprendizaje y práctica. Lo que los estudiantes aprendían por la mañana lo impartían por la tarde a las multitudes que se congregaban en el litoral. Tras la cena, comentaban a la vez, de manera informal, tanto lo aprendido por la mañana como lo enseñado por la tarde.
\vs p148 1:2 Cada uno de los instructores apostólicos enseñaba su propia concepción del evangelio del reino. No se esforzaban por enseñar exactamente igual; no se hacía un enunciado estandarizado o dogmático de las doctrinas teológicas. Aunque todos impartían la \bibemph{misma verdad,} cada apóstol exponía su propia y personal interpretación de las enseñanzas del Maestro. Y Jesús respaldaba esta presentación plural de experiencias personales en los asuntos del reino, invariablemente armonizando y coordinando, durante las horas semanales de preguntas que él presidía, estas perspectivas, múltiples y divergentes, del evangelio. Pese a este gran grado de libertad personal en temas docentes, la teología de Simón Pedro tendía a prevalecer en la escuela evangelista. Después de Pedro, Santiago Zebedeo era el que mayor influencia personal ejercía.
\vs p148 1:3 De los más de cien evangelistas, formados al lado del mar durante estos cinco meses, y exceptuando a Abner y a los apóstoles de Juan, se elegirían los posteriores setenta instructores y predicadores del evangelio. En la escuela de evangelistas no existía el mismo grado de comunidad de bienes que había en los doce.
\vs p148 1:4 Estos evangelistas, aunque enseñaban y predicaban el evangelio, no bautizaban a los creyentes hasta después de que Jesús los ordenara y los nombrara como miembros de los setenta mensajeros del reino. Del gran número de personas que habían resultado sanadas, en aquel lugar, en aquella curación a la puesta de sol, solo siete de ellos se contaban entre estos estudiantes evangelistas. El hijo del noble de Cafarnaúm fue uno de los que se formaron en la escuela de Pedro para servir al evangelio.
\usection{2. EL HOSPITAL DE BETSAIDA}
\vs p148 2:1 En conjunción con el campamento costero, Elman, el médico sirio, con la ayuda de un colectivo de veinticinco mujeres jóvenes y doce hombres, organizó y dirigió, durante cuatro meses, lo que se podría considerar como el primer hospital del reino. En este centro médico, situado a escasa distancia del sur de la principal ciudad de tiendas, se trataba a los enfermos con arreglo a métodos físicos conocidos, al igual que mediante la práctica espiritual de la oración y el fomento de la fe. No menos de tres veces a la semana, Jesús visitaba a los enfermos de este campamento y se ponía en contacto personal con cada uno de ellos. Hasta donde sabemos, entre las mil personas afligidas y dolientes que salieron mejoradas o sanadas de esta clínica no se produjo ninguno de los llamados milagros de curación sobrenatural. Sin embargo, la inmensa mayoría de las personas que resultaron beneficiadas no dejaron de proclamar que Jesús los había curado.
\vs p148 2:2 Muchas de las curas realizadas por Jesús, en el ministerio que él dispensaba a los pacientes de Elman, se semejaban, de hecho, a milagros, pero se nos informó de que se trataban únicamente de transformaciones de mente y espíritu que pueden darse en la vivencia de personas esperanzadas y regidas por la fe, cuando se encuentran bajo la influencia, directa y vivificante, de una persona, optimista y benefactora, cuyos cuidados desvanecen el miedo y hacen desaparecer la ansiedad.
\vs p148 2:3 Elman y sus compañeros se esforzaron por enseñar a estos enfermos la verdad sobre la “posesión de los espíritus malignos”, pero tuvieron poco éxito. Se creía, de manera bastante generalizada, que las enfermedades físicas y los trastornos mentales se debían a alguno de los llamados espíritus impuros, que moraban en la mente o en el cuerpo de la persona afligida.
\vs p148 2:4 En todos sus contactos con los enfermos y afligidos, cuando se trataba de algún método de tratamiento o de revelar las causas desconocidas de las enfermedades, Jesús no desatendió las instrucciones recibidas de Emanuel, su hermano del Paraíso, antes de emprender la aventura de su encarnación en Urantia. Pese a ello, los cuidadores de estos enfermos aprendieron muchas valiosas lecciones al observar el modo en el que él estimulaba la fe y la confianza de enfermos y dolientes.
\vs p148 2:5 El campamento se desmanteló poco tiempo antes de que se acercara la estación de los catarros y las fiebres.
\usection{3. LOS ASUNTOS DEL PADRE}
\vs p148 3:1 Durante todo este período, Jesús celebró servicios públicos en el campamento menos de una docena de veces y solo una vez habló en la sinagoga de Cafarnaúm, en ocasión del segundo \bibemph{sabbat,} antes de partir con los evangelistas recién formados en su segundo viaje público de predicación por Galilea.
\vs p148 3:2 Desde su bautismo, el Maestro nunca había tenido tanto tiempo para él mismo como durante este período de formación de los evangelistas en el campamento de Betsaida. Siempre que alguno de los apóstoles se aventuraba a preguntarle a Jesús por qué se ausentaba tanto de ellos, él, de forma invariable, contestaba que estaba “en los asuntos del Padre”.
\vs p148 3:3 Durante estos períodos de ausencia, solo dos apóstoles lo acompañaban. Jesús había liberado temporalmente a Pedro, Santiago y Juan del cometido como acompañantes personales suyos, para que pudieran también participar en la labor de formación de los nuevos aspirantes a evangelistas, cuyo número ascendía a más de cien. Cuando el Maestro deseaba ir a las colinas para ocuparse de los asuntos de su Padre, llamaba para que lo acompañaran a los dos apóstoles que se encontrasen libres de tareas. De este modo, cada uno de los doce disfrutó de la oportunidad de una estrecha relación y de un contacto cercano con Jesús.
\vs p148 3:4 A efectos de este relato, no se nos ha revelado nada al respecto, pero hemos deducido que el Maestro, durante muchos de estos momentos de soledad pasados en las colinas, estuvo en relación directa y de potestad con muchos de sus principales directores a cargo de los asuntos del universo. Desde el momento de su bautismo, este Soberano de nuestro universo, encarnado, había desempeñado un papel cada vez más conscientemente activo en la dirección de ciertos aspectos administrativos del universo. Y hemos sido siempre de la opinión de que, de alguna manera no revelada a sus acompañantes más próximos, durante estas semanas de menor participación en las cuestiones terrenales, se dedicó a dirigir a esas elevadas inteligencias espirituales encargadas de la administración de su inmenso universo, y el Jesús humano decidió llamar a tales actividades de su parte como “los asuntos de su Padre”.
\vs p148 3:5 Muchas veces, cuando Jesús pasaba horas a solas, aunque sus dos apóstoles estuvieran cerca, estos notaron que su semblante experimentaba un sinnúmero de cambios súbitos, aunque no le oían decir palabra alguna. Tampoco observaron manifestación visible alguna de seres celestiales que pudieran haber estado en comunicación con su Maestro, tal como algunos de ellos ciertamente presenciaron en una posterior ocasión.
\usection{4. EL MAL, EL PECADO Y LA INIQUIDAD}
\vs p148 4:1 Dos días a la semana por la noche, en un rincón apartado y recubierto del jardín de Zebedeo, Jesús acostumbraba a tener conversaciones particulares con aquellas personas que deseaban charlar con él. En una de estas conversaciones, Tomás formuló al Maestro la siguiente pregunta: “¿Por qué es necesario nacer del espíritu para entrar al reino? ¿Es que se necesita volver a nacer para poder librarse de las garras del diablo? Maestro, ¿qué es el mal?”. Al oír estas preguntas, Jesús dijo a Tomás:
\vs p148 4:2 \pc “No cometas el error de confundir el \bibemph{mal} con el \bibemph{diablo,} más acertadamente llamado el \bibemph{inicuo}. Aquel a quien llamáis diablo es hijo del amor a sí mismo; él fue ese alto administrador de los mundos que, con conocimiento de causa, se declaró deliberadamente en rebeldía contra el gobierno de mi Padre y de sus leales hijos. Pero yo ya he derrotado a estos rebeldes pecaminosos. Aclara tu mente en cuanto a estas actitudes diferentes respecto al Padre y a su universo. Nunca os olvidéis de estas leyes relacionadas con la voluntad del Padre:
\vs p148 4:3 “El mal es la transgresión inconsciente o no intencionada de la ley divina, de la voluntad del Padre. Asimismo, el mal mide la imperfección de la obediencia de la persona a la voluntad del Padre.
\vs p148 4:4 “El pecado es la transgresión consciente, sabedora y deliberada de la ley divina, de la voluntad del Padre. El pecado es la medida en la que la persona se resiste a aceptar la guía divina y la dirección espiritual.
\vs p148 4:5 “La iniquidad es la transgresión obstinada, decidida y pertinaz de la ley divina, de la voluntad del Padre. La iniquidad mide el rechazo continuado al plan amoroso del Padre establecido para la supervivencia del ser personal y el ministerio misericordioso de salvación de los hijos divinos.
\vs p148 4:6 “Por naturaleza, antes de renacer del espíritu, el hombre mortal está sujeto a tendencias innatas al mal, pero tal imperfección de comportamiento de índole natural no son ni pecado ni iniquidad. El hombre mortal acaba de comenzar su largo ascenso para alcanzar la perfección del Padre del Paraíso. Ser imperfecto o incompletos en cuanto a dones naturales no es pecaminoso. El hombre está ciertamente sujeto al mal, pero no es, de ningún modo, hijo del diablo, a menos que haya elegido con conocimiento de causa y deliberadamente las sendas del pecado y una vida de iniquidad. El mal es intrínseco al orden natural de este mundo, pero el pecado es una actitud de rebelión consciente traída a este mundo por quienes cayeron de la luz espiritual a la densa oscuridad.
\vs p148 4:7 “Tomás, estás confundido por las doctrinas de los griegos y los errores de los persas. No entiendes la diferencia entre el mal y el pecado porque piensas que la humanidad comenzó en la tierra con un Adán perfecto, que se degradó rápidamente, por el pecado, hasta llegar a la actual y deplorable condición del hombre. Pero, ¿por qué te niegas a comprender el significado de la historia que da a conocer cómo Caín, el hijo de Adán, fue a la tierra de Nod y encontró allí esposa? Y ¿por qué te niegas a interpretar el significado del relato que narra cómo los hijos de Dios encontraron esposas entre las hijas de los hombres?
\vs p148 4:8 “Ciertamente, los hombres son malos por naturaleza, pero no necesariamente pecadores. El nuevo nacimiento ---el bautismo del espíritu--- resulta esencial para liberarse del mal y necesario para entrar al reino de los cielos, pero nada de esto resta valor al hecho de que el hombre es hijo de Dios. Esta presencia inherente del mal potencial no significa que el hombre, de algún modo misterioso, esté distanciado del Padre de los cielos y que deba, como si fuese un extraño, un extranjero o un hijastro, tratar, de alguna manera, que el Padre lo adopte legalmente. Todas estas ideas nacieron, primeramente, de vuestra incomprensión del Padre y, en segundo lugar, de vuestra ignorancia del origen, naturaleza y destino del hombre.
\vs p148 4:9 Los griegos y algunos otros os han enseñado que el hombre está descendiendo inexorablemente de la perfección divina hacia el olvido o a la destrucción; yo he venido para mostraros que el hombre, mediante su entrada en el reino, asciende a Dios y a la perfección divina de forma indudable y cierta. Cualquier ser que, de algún modo, no esté a la altura de los ideales divinos y espirituales de la voluntad del Padre eterno es potencialmente malo, pero estos seres no son, en ningún sentido, pecadores, ni mucho menos inicuos.
\vs p148 4:10 “Tomas, es que no leíste en las Escrituras, donde está escrito: ‘Vosotros sois los hijos del Señor vuestro Dios’. ‘Yo seré Padre para él, y él será hijo para mí’. ‘Lo he escogido por hijo, y yo seré su Padre’. ‘Trae de lejos a mis hijos, y a mis hijas de los confines de la tierra, a todos los llamados de mi nombre, que para gloria mía los he creado’. ‘Sois los hijos del Dios viviente’. ‘Los que tienen el espíritu de Dios son de cierto hijos de Dios’. Al igual que hay una parte material del padre humano en el hijo biológico, hay una parte espiritual del Padre celestial en cada uno de los hijos del reino por la fe”.
\vs p148 4:11 \pc Jesús le dijo a Tomás todo esto y mucho más, y el apóstol lo comprendió en gran parte, aunque Jesús le advirtió: “No hables con los demás sobre estos asuntos hasta que yo haya regresado al Padre”. Y Tomás no mencionó esta conversación hasta después de que el Maestro hubiera partido de este mundo.
\usection{5. EL PROPÓSITO DE LA AFLICCIÓN}
\vs p148 5:1 En otra de estas conversaciones privadas en el jardín, Natanael le preguntó a Jesús: “Maestro, aunque empiezo a comprender por qué te niegas a obrar curaciones de forma indiscriminada, todavía me desconcierta por qué el Padre amoroso de los cielos permite que tantos hijos suyos de la tierra padezcan tantas aflicciones”. El Maestro respondió a Natanael, diciendo:
\vs p148 5:2 \pc “Natanael, tú y muchos otros os sentís perplejos porque no comprendéis cómo la pecaminosa osadía de algunos traidores que se rebelaron contra la voluntad del Padre ha podido perturbar tantas veces el orden natural de este mundo. Y yo he llegado para empezar a poner orden en estas cosas. Pero se precisarán muchas eras para devolver esta parte del universo a su senda anterior y liberar, pues, a los hijos del hombre de la carga añadida del pecado y la rebelión. Solo se necesita la presencia del mal como prueba suficiente para que el hombre avance; no es esencial que, para su supervivencia, experimente las consecuencias del pecado.
\vs p148 5:3 “Pero, hijo mío, debes saber que el Padre no aflige a sus hijos intencionadamente. El hombre acarrea sobre sí mismo una innecesaria aflicción fruto de su insistente rechazo a caminar por los mejores senderos de la voluntad divina. La aflicción existe potencialmente en el mal, pero gran parte de esta se ha debido al pecado y la iniquidad. En este mundo han ocurrido muchos hechos inusuales, y no es raro que el hombre reflexione y se preocupe ante la visión del sufrimiento y la aflicción. Pero puedes estar seguro de algo: el Padre no envía arbitrariamente la aflicción como castigo por la maldad. Las imperfecciones y los impedimentos del mal son inherentes; el castigo por el pecado es inevitable; las devastadoras consecuencias de la iniquidad son inexorables. El hombre no debe culpar a Dios por esas aflicciones que son el resultado natural de la vida que él elige vivir; tampoco debe lamentarse por esas experiencias que son parte de la vida tal como se vive en este mundo. Es la voluntad del Padre que el hombre mortal trabaje con perseverancia y constancia hacia la mejora de su condición en la tierra. Su dedicación inteligente a este propósito hará que el hombre venza en buena parte su miseria en la tierra.
\vs p148 5:4 “Natanael, nuestra misión consiste en ayudar a los hombres a solucionar sus problemas espirituales y, de este modo, avivar sus mentes a fin de que se encuentren mejor preparados e inspirados para resolver sus numerosas dificultades materiales. Sé de tu confusión porque has leído las Escrituras. Con excesiva frecuencia se ha tendido predominantemente a responsabilizar a Dios por todo lo que el hombre, en su ignorancia, es incapaz de comprender. El Padre no es responsable, personalmente, de todo lo que no podéis entender. No dudes del amor del Padre porque transgrediste de forma inocente o deliberada alguna de sus sabias y justas leyes y puedas estar sufriendo las consecuencias.
\vs p148 5:5 “Pero, Natanael, hay mucho en las Escrituras que te hubiese servido de instrucción si la hubieras leído con criterio. ¿Es que no recuerdas lo que está escrito: ‘No menosprecies, hijo mío, el castigo del Señor, no te canses de que él te corrija, porque el Señor al que ama castiga como el padre al hijo a quien quiere’. ‘El Señor no se complace en afligir a los hijos de los hombres’. Antes que fuera yo afligido, descarriado andaba; pero ahora guardo tu ley. La aflicción me fue buena para que yo pudiera así aprender los estatutos divinos’. ‘Conozco tus angustias. El eterno Dios es tu refugio y sus brazos eternos son tu apoyo. El Señor es también refugio del oprimido, un cobijo para el tiempo de angustia. El Señor lo sostendrá en el lecho del dolor; el Señor no se olvidará del enfermo. ‘Como el padre se compadece de los hijos, se compadece el Señor de los que lo temen, porque él conoce vuestra condición; se acuerda de que sois polvo. Él sana a los quebrantados de corazón y venda sus heridas’. ‘Él es esperanza para el pobre, fortaleza para el necesitado en su aflicción, refugio contra la tormenta, y sombra contra el calor sofocante”. ‘Él da esfuerzo al cansado y acrecienta las fuerzas al que no tiene ningunas’. ‘No quebrará la caña cascada ni apagará el pábilo que se extingue’. Cuando pases por las aguas de la aflicción, yo estaré contigo; y cuando los ríos de la adversidad te aneguen, no te dejaré.’ ‘Me ha enviado a vendar a los quebrantados de corazón, a publicar libertad a los cautivos y a consolar a todos los que están de luto.’ ‘Hay corrección en el sufrimiento; la aflicción no sale del polvo’”.
\usection{6. CONCEPTO EQUIVOCADO DEL SUFRIMIENTO.\\CHARLA SOBRE JOB}
\vs p148 6:1 En Betsaida, esa misma noche, Juan también preguntó a Jesús por qué tantas personas, aparentemente inocentes, sufrían tantas enfermedades y padecían tantas aflicciones. El Maestro, al responder a las preguntas de Juan le dijo, entre otras muchas cosas, lo siguiente:
\vs p148 6:2 \pc “Hijo mío, no comprendes el significado de la adversidad ni la misión del sufrimiento. ¿Es que no has leído en las Escrituras esa obra maestra de la literatura semita, la historia sobre las aflicciones de Job? ¿Es que no te acuerdas cómo esta maravillosa parábola comienza con el relato de la prosperidad material del siervo del Señor? Si recuerdas bien, Job fue bendecido con hijos, riqueza, dignidad, posición, salud y todo lo demás que los hombres valoran en esta vida temporal. Según las antiguas enseñanzas de los hijos de Abraham, esta prosperidad material era prueba manifiesta del favor divino. Pero dichas posesiones materiales y la prosperidad temporal no demuestran el favor de Dios. Mi Padre de los cielos ama tanto a los pobres como a los ricos; él no hace acepción de personas.
\vs p148 6:3 “Aunque a la transgresión de la ley divina le sigue, tarde o temprano, la siega del pecado, aunque los hombres ciertamente acaban por cosechar lo que sembraron, deberías saber, de todas formas, que el sufrimiento humano no es siempre el castigo a algún pecado cometido. Job y sus amigos no pudieron encontrar una verdadera respuesta a sus perplejidades. Y a la luz que ahora disfrutas, nunca te creerías que Satanás o Dios actuarían de la forma que se dice en esta singular parábola. Aunque Job, a través del sufrimiento, no pudo dilucidar sus problemas intelectuales ni resolver sus dificultades filosóficas, sí logró grandes victorias; e incluso ante el desmoronamiento de sus defensas teológicas, ascendió a esas alturas espirituales en las que él pudo expresar con sinceridad: ‘yo me aborrezco’; entonces se le concedió la salvación mediante una \bibemph{visión de Dios}. Por lo que, incluso con una noción equivocada del sufrimiento, Job se elevó a un plano sobrehumano, consiguiendo comprensión moral y percepción espiritual. Cuando en su sufrimiento el siervo tiene una visión de Dios, tras ella viene una paz del alma que sobrepasa todo entendimiento humano.
\vs p148 6:4 “Elifaz, el primero de los amigos de Job, exhortó al sufriente a que demostrara en sus aflicciones la misma fortaleza que había estipulado a los demás durante los días de su prosperidad. Este falso consolador dijo: ‘Confía en tu religión, Job; recuerda que son los malvados los que sufren, no los rectos. Debes merecer este castigo o, de lo contrario, no estarías afligido. Bien sabes que, ante los ojos del Señor, ningún hombre puede ser recto. Sabes que los malvados nunca realmente prosperan. En cualquier caso, parece que el hombre está predestinado a la aflicción, y quizás solo te castiga por tu propio bien’. No era de extrañar que el pobre Job no obtuviera mucho consuelo ante esta interpretación del problema del sufrimiento humano.
\vs p148 6:5 “Pero el consejo de Bildad, su segundo amigo, fue incluso más desalentador a pesar de su validez desde la perspectiva de la teología aceptada en aquel momento. Bildad dijo: ‘Dios no puede ser injusto. Tus hijos deben haber sido pecadores, puesto que perecieron; tú debes estar en el error o, de otra manera, no sufrirías tanto. Y, si en verdad eres recto, Dios de cierto te librará de tus aflicciones. Deberías aprender de la historia sobre la forma que tiene Dios de relacionarse con el hombre y ver que el Todopoderoso solo derriba a los perversos’.
\vs p148 6:6 “Y recuerdas entonces cómo respondió Job a sus amigos, diciendo: ‘Yo sé bien que Dios no oye mi grito de ayuda. ¿Cómo puede Dios ser justo y, al mismo tiempo, ignorar absolutamente mi inocencia? Estoy aprendiendo que no hallo satisfacción si clamo al Todopoderoso. ¿Es que no percibís que Dios tolera que los perversos persigan a los buenos? Y, siendo el hombre tan débil, ¿qué posibilidades tiene de que un Dios omnipotente lo tenga en consideración? Dios me ha hecho como soy, y, cuando él se torna, pues, contra mí, estoy indefenso. Y ¿es que me ha creado Dios solamente para sufrir desdichas?’
\vs p148 6:7 “¿Y quién puede cuestionar la actitud de Job a la luz del consejo de sus amigos y de las ideas erróneas sobre Dios que ocupaban su propia mente? ¿Es que no ves que Job ansiaba un Dios \bibemph{humano,} que anhelaba comunicarse con un Ser divino que conociera la condición mortal del hombre y que entendiera que el justo, a menudo, debe sufrir, siendo inocente, como parte de esta primera vida en su largo ascenso al Paraíso? Es por lo que el Hijo del Hombre ha venido del Padre, para vivir una vida en la carne y ser capaz de confortar y socorrer a aquellos que serán llamados en adelante a soportar las aflicciones de Job.
\vs p148 6:8 “Zofar, el tercer amigo de Job, pronunció, luego, palabras aún menos consoladoras cuando afirmó: ‘Es una insensatez declarar que eres recto, viéndote, así, afligido. Pero reconozco que los caminos de Dios son imposibles de comprender. Quizás haya en tus desdichas algún recóndito propósito’. Y cuando terminó de escuchar a sus tres amigos, Job clamó directamente a Dios para que le ayudara alegando el hecho de que ‘El hombre, nacido de mujer, es corto de días y de muchos sinsabores’.
\vs p148 6:9 “Entonces comenzó la segunda conversación con sus amigos. Elifaz se volvió más estricto, acusador y sarcástico. Bilbad se indignó por el menosprecio de Job hacia sus amigos. Zofar reiteró su pesimista consejo. Job, para entonces, se había disgustado con sus amigos y clamó de nuevo a Dios, apelando ahora a un Dios justo frente al Dios de la injusticia enunciado en la filosofía de sus amigos e incluso valoradas desde su propia actitud religiosa. Después, Job buscó refugio en el consuelo de una vida futura, en la que las inequidades de la existencia mortal pudieran rectificarse con mayor justicia. El hecho de no lograr recibir ayuda de parte del hombre lo lleva a Dios. Luego, se entabla una gran lucha en su corazón entre la fe y las dudas. Finalmente, el humano, en su sufrimiento, comienza a ver la luz de la vida; su alma torturada asciende a las nuevas alturas de la esperanza y la valentía; puede que sufra e incluso que muera, pero su alma, iluminada, lanza ahora el grito de triunfo: ‘¡Mi Defensor vive!’.
\vs p148 6:10 “Job tenía toda la razón cuando puso en cuestión la doctrina de que Dios aflige a los hijos con el fin de castigar a sus padres. Job estaba siempre dispuesto a admitir la rectitud de Dios, pero anhelaba una revelación de la naturaleza personal del Eterno que satisficiera su alma. Y esa es nuestra misión en la tierra. Nunca más se le negará al sufriente mortal el consuelo de conocer el amor de Dios y de reconocer la misericordia del Padre de los cielos. Aunque el hecho de que Dios le hable desde el torbellino fue un concepto majestuoso para el día en el que se enunció, tú ya has aprendido que el Padre no se revela a sí mismo de ese modo, sino que habla más bien dentro del corazón humano con la voz apacible y delicada, diciendo: ‘éste es el camino; andad por él’. ¿Es que no comprendes que Dios vive en ti?, ¡que se ha convertido en lo que tú eres para que tú puedas hacerte como él es!”.
\vs p148 6:11 Entonces, Jesús afirmó finalmente: “El Padre de los cielos no se complace en afligir a los hijos de los hombres. El hombre sufre, primeramente, por los accidentes del tiempo y por las imperfecciones del mal, característicos de una existencia física inmadura. Más tarde, sufre las consecuencias inexorables del pecado ---la transgresión de las leyes de la vida y de la luz---. Y, por último, el hombre recoge la cosecha de su propia e inicua insistencia en la rebelión contra el recto gobierno del cielo en la tierra. Pero las desdichas no son un acto \bibemph{personal} de castigo decretado por juicio divino. El hombre puede hacer, y hará, muchas cosas para reducir sus sufrimientos temporales. Pero, de una vez y para siempre, libérate de la superstición de que Dios aflige al hombre a instancias del diablo. Estudia el libro de Job y descubrirás cuántas ideas equivocadas sobre Dios pueden incluso los hombres buenos albergar honradamente; y, a continuación, constata cómo Job, en su dolorosa aflicción, halló al Dios del consuelo y de la salvación, pese a estas erróneas enseñanzas. Al final, su fe traspasó las nubes del sufrimiento para percibir la luz de la vida que se derramaba desde el Padre como misericordia sanadora y sempiterna rectitud”.
\vs p148 6:12 En su corazón, Juan meditó estas palabras muchos días. Tras esta conversación, mantenida con el Maestro en el jardín, su vida restante sufriría un importante cambio y, con posterioridad, Juan influiría mucho en los otros apóstoles para que cambiaran sus perspectivas respecto a la fuente, naturaleza y propósito de las aflicciones humanas corrientes. Pero Juan nunca mencionó esta conversación hasta después de que el Maestro partiera.
\usection{7. EL HOMBRE DE LA MANO SECA}
\vs p148 7:1 El segundo \bibemph{sabbat,} antes de que los apóstoles y el nuevo colectivo de evangelistas partieran en su segundo viaje de predicación por Galilea, Jesús habló en la sinagoga de Cafarnaúm sobre “Los gozos de la vida en rectitud”. Cuando acabo de hablar, a su alrededor, buscando sanarse, se agolpó un gran grupo de mancos, cojos, enfermos y afligidos. En este grupo estaban también los apóstoles, muchos de los nuevos evangelistas y los espías fariseos de Jerusalén. Dondequiera que Jesús fuese (salvo cuando iba a las colinas para ocuparse de los asuntos de su Padre), iba ciertamente seguido de los seis espías de Jerusalén.
\vs p148 7:2 Mientras Jesús hablaba con la gente, el líder de los espías fariseos instó a un hombre con una mano seca a que se acercara a él y le preguntara si era lícito curar el día del \bibemph{sabbat} o debía pedir su ayuda otro día. Cuando Jesús vio al hombre, oyó sus palabras y se percató de que los fariseos lo habían enviado, dijo: “Ven a mí; te voy a hacer una pregunta. Si tuvieras una oveja y se cayera en un hoyo en \bibemph{sabbat,} ¿no le echarías una mano y la sacarías? ¿Es lícito hacer tal cosa en \bibemph{sabbat?}”. Y el hombre le respondió: “Sí, Maestro, estaría permitido hacer el bien ese día”. Entonces Jesús, mirándoles, les dijo: “Sé por qué habéis puesto a este hombre ante mí. ¿Queréis encontrar algún motivo de agravio a la ley en mí tentándome a mostrar misericordia en \bibemph{sabbat?} Vuestro silencio demuestra que dais vuestra aprobación a sacar a una infortunada oveja de un hoyo, incluso en \bibemph{sabbat,} y yo os tomo por testigo de que está permitido mostrar misericordia ese día no solo a los animales, sino también a los hombres. ¡Cuánto más vale un hombre que una oveja! Por consiguiente, es lícito hacer el bien en \bibemph{sabbat}. Y mientras estaban todos ellos allí en silencio delante de él, Jesús, dirigiéndose al hombre de la mano seca, le dijo: “Levántate y ponte a mi lado para que todos puedan verte. Y, ahora, para que sepáis que es voluntad de mi Padre hacer el bien en \bibemph{sabbat,} si tienes fe en tu curación, yo te mando que extiendas la mano”.
\vs p148 7:3 Y al extender el hombre la mano, le fue restaurada sana. La gente empezó a volverse en contra de los fariseos, pero Jesús les ordenó que se calmaran, diciendo: “Acabo de deciros que es lícito hacer el bien en \bibemph{sabbat,} salvar una vida, pero no que hagáis daño ni que os dejéis arrebatar por el deseo de matar”. Los fariseos, enfurecidos, se alejaron y, pese a ser el día del \bibemph{sabbat,} se apresuraron seguidamente a Tiberias para pedir el asesoramiento de Herodes, haciendo todo lo posible para soliviantarlo y lograr que los herodianos se confabularan con ellos contra Jesús. Pero Herodes se negó a tomar ninguna acción contra Jesús, aconsejándoles que llevaran sus denuncias a Jerusalén.
\vs p148 7:4 Este es el primer caso de un milagro que obró Jesús como respuesta a un reto planteado por sus enemigos. El Maestro realizó este denominado milagro, no como demostración de su poder de sanación, sino como una impresionante protesta de su parte en contra de convertir el descanso religioso del \bibemph{sabbat} en el yugo a unas vanas restricciones para la humanidad. Este hombre volvió a su trabajo de albañil, resultando que su curación fue seguida de una vida de acción de gracia y rectitud.
\usection{8. SU ÚLTIMA SEMANA EN BETSAIDA}
\vs p148 8:1 La última semana que pasaron en Betsaida, los espías de Jerusalén se dividieron en cuanto a su actitud hacia Jesús y sus enseñanzas. Tres de estos fariseos estaban profundamente impresionados por lo que habían visto y oído. Entretanto, en Jerusalén, Abraham, un joven e influyente miembro del sanedrín, abrazó públicamente las enseñanzas de Jesús y se bautizó de manos de Abner en el estanque de Siloam. Todo Jerusalén se conmocionó por este suceso y, de inmediato, se enviaron mensajeros a Betsaida para mandar a llamar a los seis espías fariseos.
\vs p148 8:2 \pc El filósofo griego que se había sumado al reino durante el viaje anterior por Galilea regresó con algunos judíos ricos de Alejandría y, nuevamente, instaron a Jesús a ir a su ciudad y establecer allí una escuela conjunta de filosofía y religión, al igual que un hospital para los enfermos. Pero Jesús, amablemente, declinó su ofrecimiento.
\vs p148 8:3 \pc Por este tiempo, llegó al campamento de Betsaida un tal Kirmeth, de Bagdad, que realizaba profecías en trance. Este supuesto profeta tenía peculiares visiones cuando entraba en este estado y grotescos sueños cuando se le alteraba el sueño. Originó un considerable alboroto en el campamento, y Simón Zelotes era partidario de tratar con cierta dureza a este iluso impostor, pero Jesús intervino y le concedió, durante unos días, completa libertad de acción. Quienes oyeron su predicación se percataron enseguida de que sus enseñanzas eran irracionales si se las juzgaba conforme al evangelio del reino. Poco después, Kirmeth regresó a Bagdad, llevándose con él solo a una media docena de almas inestables y erráticas. Pero antes de que Jesús intercediera por el profeta de Bagdad, David Zebedeo, asistido por una autodesignada comisión, llevó a Kirmeth al lago y, tras sumergirlo repetidamente en el agua, le habían aconsejado que se marchara de allí, que organizara y levantara su propio campamento.
\vs p148 8:4 \pc Ese mismo día, Bet\hyp{}Marión, una mujer fenicia, entró en tal fanatismo que perdió la cabeza, y, tras casi ahogarse al tratar de caminar sobre el agua, sus amigos la persuadieron para que se fuera de allí.
\vs p148 8:5 \pc Abraham el fariseo, el nuevo converso de Jerusalén, dio todos sus bienes materiales terrenales al tesoro apostólico, y su contribución ayudó sobremanera a que los cien evangelistas, recién formados, pudiesen partir de inmediato. Andrés ya había anunciado el cierre del campamento, y todos se prepararon ya fuese para volver a sus casas o seguir a los evangelistas al interior de Galilea.
\usection{9. LA CURACIÓN DEL PARALÍTICO}
\vs p148 9:1 El viernes 1 de octubre por la tarde, Jesús mantuvo su última reunión con los apóstoles, evangelistas y otros líderes del campamento, ya en desmovilización, y con los seis fariseos de Jerusalén sentados en la primera fila de esta asamblea. Fue allí, en la espaciosa habitación delantera de la casa de Zebedeo, ampliada por motivo de espacio, donde tuvo lugar uno de los más extraños y excepcionales episodios de toda la vida de Jesús en la tierra. En ese momento, el Maestro estaba de pie hablando en esta gran habitación, construida para acoger estas reuniones durante la temporada de lluvias. La casa estaba enteramente rodeada por una enorme multitud, que agudizaba sus oídos para lograr entender algunas partes del discurso de Jesús.
\vs p148 9:2 Mientras la casa estaba, pues, atestada de gente y rodeada por completo de personas ansiosas por oír a Jesús, desde Cafarnaúm, en una pequeña camilla, trajeron, a un hombre afligido, durante largo tiempo, de parálisis. Este paralítico había oído que Jesús estaba a punto de dejar Betsaida y, tras haber hablado con Aarón, el albañil que había resultado recientemente sanado, decidió que lo trasladasen hasta su presencia para poder curarse. Sus amigos que lo habían llevado hasta allí trataron de acceder a la casa de Zebedeo tanto por la puerta delantera como por la trasera, pero había demasiada gente agolpada. No obstante, el paralítico no se dio por vencido y mandó a sus amigos a por unas escaleras por las que lo subieron al tejado de la habitación en la que Jesús estaba hablando y, después de desprender las tejas, bajaron al enfermo en la camilla con cuerdas hasta colocarlo en el suelo justo delante del Maestro. Cuando Jesús vio lo que habían hecho, paró de hablar, a la vez que todos los que estaban con él se quedaron maravillados por la tenacidad de este enfermo y de sus acompañantes. El paralítico dijo: “Maestro, no deseo interrumpir tus enseñanzas, pero estoy determinado a que se me restaure la salud. Yo no soy como aquellos que se curaron y al instante se olvidaron de tus enseñanzas. Quisiera curarme para poder servir en el reino de los cielos”. Entonces, a pesar de que este hombre había traído sobre él la aflicción por su propia malgastada vida, Jesús, al ver su fe, le dijo: “Hijo, no temas; tus pecados te son perdonados. Tu fe te salvará”.
\vs p148 9:3 Cuando los fariseos de Jerusalén, junto con otros escribas e intérpretes de la ley que estaban sentados con ellos, escucharon este pronunciamiento de Jesús, comenzaron a decir entre ellos: “¿Cómo se atreve este hombre a hablar así? ¿Es que no se da cuenta de que sus palabras son blasfemias? ¿Quién puede perdonar los pecados sino solo Dios?”. Jesús, conociendo en espíritu sus pensamientos y lo que comentaban entre sí, les habló diciéndoles: “¿Qué pensáis en vuestros corazones? ¿Quiénes sois vosotros para juzgarme? ¿Qué es más fácil, decir a este paralítico: “Tus pecados te son perdonados”, o decir: “Levántate, toma tu camilla y anda”? Pero, para que vosotros, que sois testigos de todo esto, podáis por fin saber que el Hijo del Hombre tiene autoridad y potestad en la tierra para perdonar pecados, diré a este hombre afligido: ¡Levántate, toma tu camilla y vete a tu casa!”. Y cuando Jesús había hablado así, el paralítico se levantó y, mientras se le abría paso, caminó por delante de todos ellos. Y quienes vieron estas cosas estaban sobrecogidos de asombro. Pedro despidió al grupo allí congregado, mientras que muchos oraban y glorificaban a Dios, confesando que jamás habían visto antes unos sucesos tan extraños.
\vs p148 9:4 \pc Y, sobre este momento, los mensajeros del sanedrín llegaron para ordenar a los seis espías que volvieran a Jerusalén. Cuando oyeron este mensaje, se entabló entre ellos una seria polémica; y una vez que acabaron de discutir, el líder y dos de sus compañeros regresaron a Jerusalén con los mensajeros, mientras que tres de los espías fariseos confesaron su fe en Jesús y, dirigiéndose de inmediato al lago, se bautizaron de manos de Pedro para ser luego acogidos por los apóstoles como hijos del reino.
