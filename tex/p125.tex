\upaper{125}{Jesús en Jerusalén}
\author{Comisión de seres intermedios}
\vs p125 0:1 De toda la intensa andadura terrenal de Jesús, ningún acontecimiento resultó más fascinante, más emocionante a nivel humano, que esta visita a Jerusalén, la primera que recordaba. La experiencia de asistir solo a las ponencias del templo fue especialmente estimulante y la albergó en su memoria durante mucho tiempo como el hecho más relevante de su infancia tardía y de su primera juventud. Esta fue su primera oportunidad para disfrutar de unos pocos días de vida independiente y sentir la euforia de ir y venir sin trabas ni condicionamientos. Este breve período viviendo a su manera, durante la semana siguiente a la Pascua, fue el primero totalmente libre de obligaciones que jamás había gozado. Y tuvieron que pasar muchos años antes de que volviera a disfrutar, incluso por un breve espacio de tiempo, de un período parecido, libre de cualquier sentido de responsabilidad.
\vs p125 0:2 \pc Las mujeres raramente acudían a la fiesta de la Pascua en Jerusalén; no se requería su presencia. Jesús, sin embargo, prácticamente se negó a ir a menos que su madre los acompañara. Y cuando ella se decidió a hacerlo, muchas otras mujeres de Nazaret se sintieron movidas a realizar también el viaje, de manera que aquel grupo de personas contenía, en proporción con los hombres, el mayor número de mujeres que había partido nunca de Nazaret para asistir a la Pascua. En ocasiones, en el camino de Jerusalén, los viajeros cantaron el salmo ciento treinta.
\vs p125 0:3 Desde el momento en el que dejaron Nazaret hasta que llegaron a la cumbre del Monte de los Olivos, Jesús experimentó un continuo estado de tensión ante aquella ilusionante expectativa. Durante toda su feliz niñez, había oído hablar de Jerusalén y de su templo con reverencia; ahora estaba pronto a contemplarlos en la realidad. Desde el Monte de los Olivos y desde el exterior, visto más de cerca, el templo era todo lo que Jesús había previsto y más; pero, una vez que cruzó sus puertas sagradas, comenzó su gran desilusión.
\vs p125 0:4 En compañía de sus padres, Jesús atravesó los recintos del templo para unirse al grupo de los nuevos hijos de la ley que estaban a punto de ser consagrados como ciudadanos de Israel. Se sintió algo decepcionado por el comportamiento generalizado de las multitudes que habían acudido al templo, pero la primera gran conmoción del día se produjo cuando su madre se despidió de ellos para dirigirse a la galería de las mujeres. A Jesús nunca se le había ocurrido pensar que su madre no lo acompañara a las ceremonias de la consagración, y le indignó enormemente el hecho de que tuviera que soportar una discriminación tan injusta. Aunque aquello lo molestó bastante, aparte de unas palabras de protesta a su padre, no dijo nada. No obstante, meditó sobre ello, y lo hizo en profundidad, como sus preguntas a los escribas y maestros, una semana después, lo demostraron.
\vs p125 0:5 Pasó por los rituales de la consagración, pero se sintió decepcionado por su naturaleza superficial y rutinaria. Echaba de menos aquel interés personal característico de las ceremonias de la sinagoga de Nazaret. Jesús regresó entonces para saludar a su madre y se preparó para acompañar a su padre en su primer recorrido por el templo y sus diferentes patios, galerías y corredores. Los recintos del templo podían dar cabida a más de doscientos mil fieles a la vez y, aunque la extensión de estos edificios ---en comparación con otros que hubiese visto antes--- le produjo una gran impresión, le fascinaba más reflexionar sobre el significado espiritual de las ceremonias del templo y sus cultos de adoración.
\vs p125 0:6 Aunque muchos rituales del templo conmocionaron su sentido de la belleza y de lo simbólico, se sentía continuamente desilusionado por las aclaraciones que sus padres, en respuesta a sus muchas inquisitivas cuestiones, le daban respecto al auténtico significado de estas ceremonias. Jesús sencillamente no podía admitir unas explicaciones sobre el culto y la devoción religiosa basadas en la creencia de la ira de Dios o de la cólera del Todopoderoso. Tras concluir la visita al templo, conversaron en mayor amplitud sobre estas cuestiones, y su padre levemente le insistió que reconociera y aceptara las creencias ortodoxas judías; Jesús se volvió repentinamente hacia sus padres y, mirando a su padre a los ojos de manera implorante, le dijo: “Padre mío, no puede ser verdad; el Padre del cielo no puede relacionarse así con sus hijos descarriados de la tierra. El Padre que está en los cielos no puede amar a sus hijos menos de lo que tú me amas. Y, por muy insensatas que sean las cosas que yo haga, sé muy bien que nunca derramarías tu ira sobre mí ni descargarías tu cólera contra mí. Si tú, mi padre terrenal, posees tales reflejos humanos de lo Divino, cuánto más deberá estar el Padre celestial lleno de bondad y de desbordante misericordia. Me niego a creer que mi Padre celestial me ame menos que mi padre de la tierra”.
\vs p125 0:7 Cuando José y María oyeron estas palabras de su hijo primogénito, permanecieron callados. Y nunca más trataron de hacerlo cambiar de opinión sobre el amor de Dios y la misericordia del Padre de los cielos.
\usection{1. JESÚS VISITA EL TEMPLO}
\vs p125 1:1 A Jesús lo consternó y disgustó el espíritu de irreverencia que observó en todos los patios del templo. Consideraba que el comportamiento de las multitudes era incompatible con la presencia de estas en “la casa de su Padre”. Pero, cuando su padre lo condujo al patio de los gentiles, recibió la mayor impresión de su joven vida; allí la jerga ruidosa, las voces y las maldiciones se mezclaban indiscriminadamente con el balido de las ovejas y el parloteo alborotado que delataba la presencia de los cambistas y de los vendedores de animales para los sacrificios y de diversas otras mercancías.
\vs p125 1:2 Pero, por encima de todo, por su sentido de la decencia, le escandalizó ver a las frívolas cortesanas exhibiéndose dentro de este recinto del templo, tal como la mujeres maquilladas que tan recientemente había visto en su visita a Séforis. Esta profanación del templo suscitó en él toda su indignación juvenil y no vaciló en expresárselo abiertamente a José.
\vs p125 1:3 Jesús admiraba el entorno de sacralidad y los oficios del templo, pero le disgustaba la fealdad espiritual que veía en los rostros de tantos fieles irreflexivos.
\vs p125 1:4 Descendieron entonces al patio de los sacerdotes situado bajo la cornisa rocosa que estaba delante del templo, donde se hallaba el altar, para observar el sacrificio de las manadas de animales y el lavado en la fuente de bronce, de la sangre de las manos de los sacerdotes que lo oficiaban. El suelo manchado de sangre, las manos ensangrentadas de los sacerdotes y los sonidos que emitían los animales agonizantes eran más de lo que este muchacho amante de la naturaleza podía soportar. El terrible espectáculo descompuso a este joven de Nazaret; se agarró al brazo de su padre y le rogó que se lo llevara de allí. Regresaron cruzando el patio de los gentiles, e incluso las carcajadas groseras y las bromas profanas que oyó allí les resultaron de alivio tras lo que acababa de contemplar.
\vs p125 1:5 José vio cómo se había visto afectado su hijo por los ritos del templo y, prudentemente, lo llevó a ver “la puerta de la Hermosa”, la artística puerta hecha de bronce corintio. Pero Jesús ya había tenido suficiente de esta primera visita al templo. Regresaron al patio superior en busca de María y, por una hora, caminaron al aire libre lejos del gentío, viendo el palacio asmoneo, la majestuosa residencia de Herodes y la torre de los guardias romanos. Durante este paseo, José explicó a Jesús que solo a los habitantes de Jerusalén les era permitido asistir a los sacrificios diarios del templo, y que los residentes de Galilea venían únicamente tres veces al año para participar en el culto del templo: en la Pascua, en la fiesta de Pentecostés (siete semanas después de la Pascua) y en la fiesta de los Tabernáculos en octubre. Estas fiestas habían sido establecidas por Moisés. Hablaron entonces de las dos últimas fiestas que se habían constituido, la de la dedicación y la de Purim. Luego regresaron a su alojamiento y se prepararon para la celebración de la Pascua.
\usection{2. JESÚS Y LA PASCUA}
\vs p125 2:1 Cinco familias de Nazaret eran huéspedes, o acompañantes, de la familia de Simón de Betania en la celebración de la Pascua. Simón había comprado el cordero pascual para el grupo. Fue el sacrificio de un número tan enorme de estos corderos lo que había conmovido a Jesús en su visita al templo. El plan era comer la Pascua con los parientes de María, pero Jesús convenció a sus padres para que aceptaran la invitación de ir a Betania.
\vs p125 2:2 Esa noche se congregaron para participar en los ritos pascuales, comiendo carne asada con pan ázimo y hierbas amargas. Siendo Jesús un nuevo hijo de la alianza, se le pidió que relatara el origen de la Pascua, y lo hizo bien, pero desconcertó de alguna manera a sus padres al añadir numerosos comentarios que reflejaban de manera afable las impresiones que le habían dejado en su mente joven pero reflexiva las cosas que había visto y oído tan recientemente. Este fue el comienzo de los siete días de ceremonias de la fiesta pascual.
\vs p125 2:3 Incluso en esta fecha temprana, aunque no dijo nada a sus padres en relación a estos temas, Jesús había empezado a darle vueltas a la idea de la idoneidad de celebrar la Pascua sin el sacrificio del cordero. Sentía en su mente la seguridad de que este espectáculo de las ofrendas sacrificiales no complacían al Padre celestial y, con el paso de los años, cada vez estuvo más determinado a establecer algún día la celebración de una Pascua sin derramamiento de sangre.
\vs p125 2:4 Jesús durmió muy poco esa noche. Su descanso se vio tremendamente alterado por terribles pesadillas de sacrificios y sufrimientos. Su mente estaba angustiada y su corazón desgarrado por las contradicciones y absurdidad de la teología de todo aquel sistema ceremonial judío. Sus padres también durmieron poco. Estaban muy desconcertados por los acontecimientos del día que acababa de concluir. Tenían el corazón totalmente trastornado por la actitud del muchacho, en su opinión, extraña y decidida. María fue presa de una agitación nerviosa durante la primera parte de la noche, pero José permaneció en calma, aunque igualmente perplejo. Los dos temían hablar francamente con el joven sobre estas cuestiones, aunque Jesús, gustosamente, habría conversado con sus padres si se hubiesen atrevido a alentarlo a que lo hiciera.
\vs p125 2:5 Los oficios del templo del día siguiente fueron más aceptables para Jesús e hicieron mucho para mitigar los recuerdos desagradables del día anterior. A la mañana siguiente, el joven Lázaro se hizo cargo de Jesús y comenzaron a explorar de forma organizada Jerusalén y sus alrededores. Antes de finalizar el día, Jesús había descubierto los distintos lugares alrededor del templo donde se daban conferencias educativas en las que los participantes podían plantear cuestiones; aparte de algunas visitas al santo de los santos para preguntarse maravillado qué había realmente detrás del velo de separación, pasó la mayor parte del tiempo en torno al templo en estas conferencias.
\vs p125 2:6 Durante toda la semana de la Pascua, Jesús ocupó su lugar entre los nuevos hijos del mandamiento, lo que significaba que tenía que sentarse fuera de la barrera que separaba a todas las personas que no eran ciudadanos plenos de Israel. Siendo por ello consciente de su juventud, se abstuvo de hacer todas las preguntas que se agolpaban en su mente; al menos se contuvo hasta que concluyó la celebración de la Pascua y se levantaron las restricciones que se habían impuesto a los jóvenes recién consagrados.
\vs p125 2:7 El miércoles de la semana de la Pascua, a Jesús se le permitió ir a casa de Lázaro para pasar la noche en Betania. A última hora de la tarde de aquel día, Lázaro, Marta y María lo oyeron hablar sobre las cosas temporales y eternas, humanas y divinas, y, desde aquel momento, los tres lo amaron como si se tratase de su propio hermano.
\vs p125 2:8 Al final de la semana, Jesús vio a Lázaro menos porque el joven no era ni siquiera apto para ser admitido en el círculo exterior de las ponencias del templo, aunque sí asistió a algunas ponencias públicas que se impartían en los patios exteriores. Lázaro tenía la misma edad que Jesús, pero en Jerusalén, a los jóvenes raramente se les admitía a la consagración de los hijos de la ley hasta que no cumplieran los trece años de edad.
\vs p125 2:9 Una y otra vez, durante la semana pascual, los padres de Jesús hallaron a su hijo sentado a solas con su joven cabeza entre las manos, profundamente pensativo. Nunca lo habían visto comportarse de este modo. No conocían el grado de confusión de su mente ni el estado de turbación de su espíritu debido a la experiencia por la que estaba atravesando y se sintieron intensamente desconcertados; no sabían qué hacer. Les alegraba que los días de esta semana fueran transcurriendo y anhelaban ver a su hijo, que actuaba de manera extraña, de regreso a Nazaret indemne.
\vs p125 2:10 Día a día, Jesús sopesaba con detenimiento las dificultades por las que estaba pasando. Al terminar la semana, ya había efectuado muchos cambios en su comportamiento que lo ayudaban a afrontarlas. Cuando llegó la hora de volver a Nazaret, su joven mente era aún un enjambre de pensamientos confusos acuciada por un gran número de preguntas sin respuesta y de problemas sin resolver.
\vs p125 2:11 Antes de salir de Jerusalén, José y María, en compañía del maestro de Jesús en Nazaret, hicieron planes concretos para que regresara a Jerusalén cuando cumpliese los quince años y comenzara un prolongado plan de estudios en una de las academias rabínicas más conocidas. Jesús acompañó a sus padres y a su maestro en sus visitas a la escuela, pero los tres estaban en estado de consternación al comprobar su aparente indiferencia ante todo lo que hacían y decían. María estaba profundamente apenada por sus reacciones a la visita de Jerusalén y José extremadamente desconcertado por los extraños comentarios y el comportamiento insólito del muchacho.
\vs p125 2:12 Al fin y al cabo, la semana de la Pascua había sido un gran acontecimiento en la vida de Jesús. Había disfrutado de la oportunidad de conocer a decenas de muchachos de su misma edad, compañeros aspirantes como él a la consagración, que le sirvieron de ayuda para conocer cómo se vivía en Mesopotamia, Turquestán y Partia, al igual que en las provincias más occidentales de Roma. Ya estaba bastante familiarizado con el modo de vida de los jóvenes de Egipto y de otras regiones cercanas a Palestina. Había miles de jóvenes en Jerusalén en aquel momento y el muchacho de Nazaret conoció personalmente y entrevistó, más o menos a fondo, a más de ciento cincuenta. Estaba particularmente interesado en aquellos que procedían del Lejano Oriente y de remotos países de Occidente. Como resultado de estas relaciones, el joven empezó a sentir el deseo de viajar por el mundo para aprender de qué modo se ganaban el sustento los distintos grupos de sus semejantes.
\usection{3. LA PARTIDA DE JOSÉ Y MARÍA}
\vs p125 3:1 Se había acordado que el grupo de Nazaret se encontrase en la zona del templo a media mañana del primer día de la semana tras la terminación de la fiesta de la Pascua. Así lo hicieron y se dispusieron a emprender su viaje de regreso a Nazaret. Jesús había entrado en el templo para escuchar las ponencias mientras que sus padres aguardaban para reunirse con sus compañeros de viaje. Todos se prepararon para partir enseguida; los hombres irían en un grupo y las mujeres en otro como era costumbre en los viajes de ida y vuelta a las fiestas de Jerusalén. Jesús había ido a Jerusalén en compañía de su madre y de las mujeres; si bien, siendo ahora un joven consagrado, debía hacer el viaje de vuelta a Nazaret con su padre y los hombres. Pero, cuando todo el grupo de Nazaret se dirigía a Betania, a Jesús, completamente absorto en una charla sobre los ángeles, en el templo, se le olvidó por completo que el tiempo pasaba y que sus padres tenían que partir. Y no se dio cuenta de que se había quedado atrás hasta el receso del mediodía en las conferencias del templo.
\vs p125 3:2 Los viajeros de Nazaret no echaron de menos a Jesús porque María suponía que viajaba con los hombres, mientras que José pensaba que lo hacía con las mujeres; Jesús había ido a Jerusalén con ellas conduciendo el asno de María. No descubrieron su ausencia hasta que alcanzaron Jericó y se prepararon para pasar la noche. Tras preguntar a los últimos del grupo que iban llegando a esta ciudad y haberse enterado de que ninguno de ellos había visto a su hijo, pasaron la noche en vela, rondándoles en la cabeza qué podría haberle pasado a su hijo y recordando muchas de sus insólitas reacciones ante los acontecimientos de la semana pascual. Hubo algún leve reproche entre ellos por no haber comprobado que se hallaba en el grupo antes de salir de Jerusalén.
\usection{4. EL PRIMER Y EL SEGUNDO DÍA EN EL TEMPLO}
\vs p125 4:1 Jesús, entretanto, había permanecido en el templo durante las primeras horas de la tarde, escuchando las ponencias y disfrutando de un entorno más tranquilo y decoroso al haber casi desaparecido el enorme gentío de la semana pascual. Cuando concluyeron las charlas de la tarde, en las que no participó, Jesús se dirigió a Betania, adonde llegó justo cuando la familia de Simón se disponía a cenar. Los tres jóvenes se alborozaron al ver a Jesús, y él se quedó a pasar la noche en su casa. Departió muy poco con ellos durante esa noche; pasó la mayor parte del tiempo meditando a solas en el jardín.
\vs p125 4:2 Jesús se levantó a primera hora del día siguiente y se dirigió hacia el templo. Se detuvo en la cumbre del Monte de los Olivos y lloró por el espectáculo que contemplaban sus ojos: un pueblo espiritualmente empobrecido, atado a las tradiciones y viviendo bajo la vigilancia de las legiones romanas. Temprano en la mañana ya se encontraba en el templo determinado a participar en las ponencias. Mientras tanto, José y María también se habían levantado al alba con la intención de volver sobre sus pasos a Jerusalén. Primeramente, se apresuraron hasta la casa de sus parientes, donde se habían alojado en familia durante la semana de Pascua, pero pudieron comprobar que nadie había visto a Jesús. Después de buscar todo el día sin encontrar rastro de él, volvieron a la casa de estos parientes para pasar la noche.
\vs p125 4:3 En la segunda conferencia, Jesús se había atrevido a hacer preguntas y, de una manera muy sorprendente, participó en las discusiones del templo, pero siempre conforme a su juventud. Algunas veces, sus incisivas preguntas ponían en algún aprieto a los doctos maestros de la ley judía, pero demostraba tal espíritu de candidez y honestidad, unidas a una manifiesta sed de conocimientos, que la mayoría de los maestros del templo resolvió tratarlo con toda consideración. Pero cuando osó cuestionar la justicia de condenar a muerte a un gentil embriagado que se hubiese aventurado fuera del patio de los gentiles y hubiese entrado, de forma inconsciente, en los recintos prohibidos y presuntamente sagrados del templo, uno de los maestros más intolerantes, impacientándose por la implícita crítica del muchacho, lo miró desde su posición de altura de forma amenazadora y le preguntó cuántos años tenía. Jesús respondió: “Me falta muy poco más de cuatro meses para cumplir los trece años”. “Entonces”, añadió el maestro ahora furioso, “¿por qué estás aquí si no tienes edad para ser un hijo de la ley?”. Y cuando Jesús explicó que había sido consagrado durante la Pascua y que había completado sus estudios en las escuelas de Nazaret, los maestros contestaron unánimemente con sorna: “Deberíamos haberlo sabido; es de Nazaret”. Pero el líder de ellos afirmó que Jesús no era culpable de que los rectores de la sinagoga de Nazaret lo hubiesen graduado formalmente a los doce años en lugar de a los trece; y, a pesar de que algunos de sus detractores se levantaron y se fueron, se resolvió que el muchacho podía continuar con normalidad como alumno en las ponencias del templo.
\vs p125 4:4 Cuando esta segunda jornada en el templo concluyó, Jesús fue de nuevo a Betania para pasar la noche. Y salió otra vez al jardín para meditar y orar. Era manifiesto que su mente estaba ocupada reflexionando sobre cuestiones de gran magnitud.
\usection{5. EL TERCER DÍA EN EL TEMPLO}
\vs p125 5:1 Al tercer día de Jesús en el templo debatiendo con los escribas y los maestros, se congregaron numerosos espectadores que, habiendo oído hablar de este joven de Galilea, acudían para disfrutar de la experiencia de observar a un muchacho confundir a los sabios de la ley. Simón también vino desde Betania para ver lo que Jesús hacía. Durante todo el día, José y María continuaron con su ansiosa búsqueda de Jesús e incluso entraron varias veces en el templo, pero nunca se les ocurrió inspeccionar los varios grupos de discusión; si bien, en una ocasión se hallaron a una distancia casi suficiente como para oír su fascinante voz.
\vs p125 5:2 Antes de terminar el día, toda la atención del principal grupo de discusión del templo se había centrado en las numerosas preguntas de Jesús, entre las que estaban las que siguen:
\vs p125 5:3 \li{1.}¿Qué hay realmente en el santo de los santos, tras el velo?
\vs p125 5:4 \li{2.}¿Por qué las madres de Israel deben estar separadas en el templo de los devotos varones?
\vs p125 5:5 \li{3.}Si Dios es un padre que ama a sus hijos, ¿por qué todo este sacrificio de animales para obtener el favor divino? ¿Se han malinterpretado las enseñanzas de Moisés?
\vs p125 5:6 \li{4.}Puesto que el templo está consagrado al culto del Padre de los cielos, ¿es congruente permitir la presencia de quienes se dedican profanamente al trueque y al comercio?
\vs p125 5:7 \li{5.}¿Será el Mesías esperado un príncipe temporal que se sentará en el trono de David o que obrará como la luz de la vida en el establecimiento de un reino espiritual?
\vs p125 5:8 \pc Y a lo largo de todo el día, quienes escuchaban se maravillaban con estas preguntas, y ninguno estaba más sorprendido que Simón. Durante más de cuatro horas, este joven de Nazaret se dirigió insistentemente a aquellos maestros judíos con cuestiones que invitaban a la reflexión y escudriñaban el corazón. Hacía pocos comentarios a las observaciones de sus mayores. Trasmitía sus enseñanzas con las preguntas que formulaba y, mediante el hábil y sutil planteamiento de ellas, lograba a la vez disputar sus enseñanzas y sugerir las suyas propias. En su manera de preguntar, existía una fascinante combinación de sagacidad y humor que le granjeaban el cariño de incluso aquellos que llegaban más o menos a sentirse ofendidos por su juventud. Al plantear estas penetrantes preguntas, siempre era sumamente justo y considerado. Aquella memorable tarde en el templo, Jesús mostró la misma reticencia a aprovecharse de su oponente de forma desleal que caracterizaría todo su posterior ministerio público. Como joven, y más tarde como hombre, parecía estar absolutamente libre de cualquier pretensión egoísta de ganar una discusión por el simple placer de su lógico triunfo sobre sus adversarios; una sola cosa le interesaba de manera suprema: proclamar la verdad eterna y llevar a cabo por tanto una revelación más completa del Dios eterno.
\vs p125 5:9 \pc Al acabar el día, Simón y Jesús se encaminaron de vuelta a Betania. Durante la mayor parte del trayecto, tanto el hombre como el muchacho guardaron silencio. Jesús se detuvo de nuevo en la cima del Monte de los Olivos, pero al contemplar la ciudad y su templo no lloró; solamente inclinó la cabeza en un gesto de devoción silenciosa.
\vs p125 5:10 Tras cenar en Betania, declinó de nuevo unirse a la alegre reunión; en lugar de ello, se fue al jardín, donde permaneció hasta bien entrada la noche, procurando, infructuosamente, diseñar un plan concreto para abordar el dilema de su misión de vida y decidir el mejor modo de revelar a sus compatriotas, ciegos espiritualmente, un concepto más hermoso del Padre celestial y liberarlos así de su terrible servidumbre a la ley, a los rituales, a las ceremonias y a las rancias tradiciones. Pero ninguna luz acudió a iluminar a este joven buscador de la verdad.
\usection{6. EL CUARTO DÍA EN EL TEMPLO}
\vs p125 6:1 Extrañamente, Jesús se había olvidado de sus padres terrenales; incluso en el desayuno, cuando la madre de Lázaro comentó que sus padres debían estar cerca de su casa en aquel momento, Jesús parecía no entender que estarían bastante preocupados por haberse él quedado atrás.
\vs p125 6:2 De nuevo se dirigió al templo, pero no se detuvo en la cumbre del Monte de los Olivos para meditar. En el trascurso de las ponencias de la mañana, se dedicó mucho tiempo a la ley y a los profetas, y los maestros estaban sorprendidos de que Jesús estuviese tan familiarizado con las Escrituras, tanto en hebreo como en griego. Pero no estaban tan asombrados por su conocimiento de la verdad como por su juventud.
\vs p125 6:3 En la conferencia vespertina, apenas habían empezado a responder a su pregunta sobre el propósito de la oración cuando el líder de ellos invitó al muchacho a que se aproximara y, sentándose a su lado, le solicitó que expusiera su propio punto de vista respecto a la oración y la adoración.
\vs p125 6:4 \pc La tarde anterior, los padres de Jesús habían oído hablar de un extraño joven que debatía con mucha habilidad con los intérpretes de la ley, pero no se les había ocurrido que este muchacho pudiera ser su hijo. Habían prácticamente resuelto dirigirse a la casa de Zacarías, pues creían que Jesús podría haber ido allí para ver a Isabel y a Juan. Pensando que Zacarías pudiera quizás encontrarse en el templo, de camino a la Ciudad de Judá se detuvieron en el templo y, cuando paseaban por sus patios, imaginad cuál sería su sorpresa y asombro cuando reconocieron la voz del joven extraviado y lo vieron sentado entre los maestros del templo.
\vs p125 6:5 José se quedó sin habla, pero María dio rienda suelta a sus temores y a la ansiedad por largo tiempo reprimidos y, corriendo apresurada hacia el joven, que se había levantado en ese momento para saludar a sus sorprendidos padres, le dijo: “Hijo mío, ¿por qué te has portado así con nosotros? Hace ya más de tres días que tu padre y yo te buscamos afligidos. ¿Qué te ha llevado a abandonarnos?”. Fue un momento tenso. Todas las miradas se volvieron hacia Jesús para oír lo que iba a decir. Su padre lo miraba reprobándolo, pero no dijo nada.
\vs p125 6:6 \pc Es preciso recordar que se suponía que Jesús era un hombre joven. Había completado la escolarización normal de un niño, había sido reconocido como hijo de la ley y había recibido la consagración como ciudadano de Israel. Y, sin embargo, su madre lo reprendía severamente delante de toda la gente allí congregada, justo en el medio de la iniciativa más seria y sublime de su joven vida, poniendo así fin de forma lamentable a una de las mayores oportunidades que jamás había tenido de enseñar la verdad, de predicar la rectitud y de revelar el carácter amoroso de su Padre de los cielos.
\vs p125 6:7 Pero el joven estuvo a la altura de la ocasión. Si consideráis de manera imparcial todos los factores que se combinaron para dar origen a esta situación, estaréis mejor preparados para comprender la sensatez de la respuesta del muchacho a la reprimenda no intencionada de su madre. Tras un momento de reflexión, Jesús le dijo: “¿Por qué me habéis buscado tanto tiempo? ¿No os imaginabais que me hallaríais en la casa de mi Padre al haber llegado el momento de que me ocupe de los asuntos de mi Padre?”.
\vs p125 6:8 Todos se quedaron atónitos ante la manera de hablar del muchacho y, en silencio, se retiraron dejándolo a solas con sus padres. El joven mitigó inmediatamente la embarazosa situación que se había creado entre los tres diciendo apaciblemente: “Vamos, padres míos, nadie ha hecho nada sin pensar que hacía lo mejor. Nuestro Padre de los cielos ha ordenado estas cosas; marchemos a casa”.
\vs p125 6:9 Partieron en silencio y llegaron a Jericó por la noche. Solo se detuvieron una vez y fue en la cima del Monte de los Olivos, donde el joven levantó su báculo hacia el cielo y, temblando de los pies a la cabeza bajo el arrebato de una intensa emoción, dijo: “¡Oh Jerusalén, Jerusalén, y su gente, ¡cuán esclavos sois, sometidos al yugo romano y víctimas de vuestras propias tradiciones, pero volveré para purificar el templo y librar a mi pueblo de esta servidumbre!”.
\vs p125 6:10 Durante los tres días de viaje hasta Nazaret, Jesús habló poco; tampoco sus padres hablaron mucho en su presencia. Estaban realmente desconcertados ante la conducta de su hijo primogénito, pero sí atesoraban sus palabras en su corazón, aunque no pudieran comprender del todo su significado.
\vs p125 6:11 Al llegar a casa, Jesús se dirigió brevemente a sus padres, confirmándoles su afecto y dándoles a entender que no tenían nada que temer, puesto que no volvería a darles ningún motivo para que se sintieran apenados por su comportamiento. Concluyó esta crucial afirmación diciendo: “Aunque debo hacer la voluntad de mi Padre de los cielos, también obedeceré a mi padre terrenal. Aguardaré a que llegue mi hora”.
\vs p125 6:12 \pc Aunque Jesús, en su mente, se negaba muchas veces a \bibemph{dar su aprobación} al bien intencionado pero errado empeño de sus padres por dictarle el curso de sus pensamientos o por establecer el plan de su tarea en la tierra, no obstante, de cualquier modo que fuese consecuente con su compromiso de hacer la voluntad de su Padre del Paraíso, \bibemph{se atenía} airosamente a los deseos de su padre terrenal y a las costumbres de su familia carnal. Incluso cuando no podía darles su aprobación, hacía todo lo posible por avenirse a ellos. Era un experto en la cuestión de adaptar el compromiso al que se debía con sus obligaciones de lealtad a la familia y de servicio social.
\vs p125 6:13 \pc José estaba perplejo, pero María, una vez que reflexionó sobre lo sucedido, se sintió reconfortada, concluyendo que las palabras de Jesús en el Monte de los Olivos eran proféticas de la misión mesiánica de su hijo como libertador de Israel. Se aprestó con renovadas fuerzas a modelar los pensamientos de Jesús de acuerdo a los cauces nacionalistas y patrióticos, y recurrió a la ayuda de su hermano, el tío favorito de Jesús; e, igualmente, en todo lo posible, la madre de Jesús emprendió la tarea de preparar a su hijo primogénito para que asumiera el liderazgo de quienes querían restaurar el trono de David y deshacerse así para siempre del yugo político al que los gentiles los sometían.
