\upaper{19}{Seres correlacionados de origen en la Trinidad}
\author{Consejero divino}
\vs p019 0:1 Este grupo de seres del Paraíso, denominados Seres Correlacionados de Origen en la Trinidad, incluye a los hijos preceptores de la Trinidad, también considerados Hijos de Dios del Paraíso, a tres grupos de elevados administradores del suprauniverso y a la categoría un tanto impersonal de los espíritus inspirados de la Trinidad. Incluso los nativos de Havona pueden apropiadamente incluirse en esta clasificación de seres personales de la Trinidad junto con numerosos grupos de seres que residen en el Paraíso. Los seres de origen en la Trinidad que tratamos en este estudio son:
\vs p019 0:2 \li{1.}Los hijos preceptores de la Trinidad.
\vs p019 0:3 \li{2.}Los perfeccionadores de la sabiduría.
\vs p019 0:4 \li{3.}Los consejeros divinos.
\vs p019 0:5 \li{4.}Los censores universales.
\vs p019 0:6 \li{5.}Los espíritus inspirados de la Trinidad.
\vs p019 0:7 \li{6.}Los nativos de Havona.
\vs p019 0:8 \li{7.}Los ciudadanos del Paraíso.
\vs p019 0:9 \pc Exceptuando a los hijos preceptores de la Trinidad y tal vez a los espíritus inspirados de la Trinidad, estos grupos están compuestos por un número definitivo de seres; su creación es un hecho pasado y concluido.
\usection{1. LOS HIJOS PRECEPTORES DE LA TRINIDAD}
\vs p019 1:1 De todos los seres personales celestiales de órdenes elevados que se os han revelado, solo los hijos preceptores de la Trinidad poseen una competencia doble. Por ser de una naturaleza de origen en la Trinidad, están casi totalmente dedicados al ministerio de la filiación divina. Son seres de enlace que salvan la brecha existente en el universo entre los seres personales de origen en la Trinidad y los que tienen un doble origen.
\vs p019 1:2 Mientras que el número de hijos estacionarios de la Trinidad está completo, el número de hijos preceptores está constantemente en aumento. No se sabe cuál será su número final. Puedo deciros, sin embargo, que de acuerdo con el último de los informes periódicos presentados en Uversa, en los archivos del Paraíso constaba que había 21\,001\,624\,821 de estos hijos en servicio.
\vs p019 1:3 Estos seres constituyen el único grupo de Hijos de Dios con origen en la Trinidad del Paraíso que se os ha revelado. Recorren el universo central y los suprauniversos, y hay un numeroso colectivo asignado a cada universo local. También sirven en los distintos planetas tal como lo hacen otros Hijos de Dios del Paraíso. Puesto que el plan diseñado para el gran universo no está totalmente desarrollado, hay un gran número de hijos preceptores que se mantiene en reserva en el Paraíso, y se ofrecen como voluntarios en casos de emergencia para realizar servicios excepcionales en todos los sectores del gran universo, en los mundos solitarios del espacio, en los universos locales, en los suprauniversos y en los mundos de Havona. También desempeñan su labor en el Paraíso, pero creemos conveniente posponer una consideración detallada de dicha labor hasta el momento en que tratemos de los Hijos de Dios del Paraíso.
\vs p019 1:4 Sin embargo, se puede reseñar en este respecto que los hijos preceptores son seres personales originados en la Trinidad con capacidad para actuar como coordinadores supremos. En un universo de los universos tan extenso, siempre existe el grave peligro de sucumbir al error de tener un punto de vista circunscrito al mal inherente en una concepción fragmentaria de la realidad y de la divinidad.
\vs p019 1:5 Por ejemplo: la mente humana por lo general anhela acercarse a la filosofía cósmica que se describe en estas revelaciones procediendo de lo simple y finito a lo complejo e infinito, de los orígenes humanos a los destinos divinos. Pero esa senda no conduce a la \bibemph{sabiduría espiritual.} Tal proceder constituye la senda más fácil para alcanzar cierta forma de \bibemph{conocimiento genético,} que en el mejor de los casos tan solo puede revelar el origen del hombre, pero poco o nada revela sobre su destino divino.
\vs p019 1:6 Incluso en el estudio de la evolución biológica del hombre en Urantia, se plantean graves objeciones a un enfoque exclusivamente histórico aplicado a su estatus presente y a sus problemas actuales. Solo es posible obtener una perspectiva verdadera de cualquier problema de la realidad ---humano o divino, terrestre o cósmico--- mediante el estudio y la correlación, completa y sin prejuicios, de las tres facetas de la realidad del universo: origen, historia y destino. El adecuado entendimiento de estas tres realidades experienciales ofrece la base para poder evaluar con sensatez la situación actual.
\vs p019 1:7 \pc Cuando la mente humana aplica el método filosófico consistente en comenzar desde lo más bajo para alcanzar lo más alto, ya sea en biología o en teología, corre siempre el peligro de cometer estos cuatro errores de razonamiento:
\vs p019 1:8 \li{1.}Es posible que fracase totalmente en la percepción de la meta evolutiva final y completa en cuanto a la realización personal o al destino cósmico.
\vs p019 1:9 \li{2.}Es posible que cometa el supremo error filosófico de simplificar en exceso la realidad cósmica evolutiva (experiencial), lo que conlleva una distorsión de los hechos, una perversión de la verdad y una noción errónea de los destinos.
\vs p019 1:10 \li{3.}Estudiar la causalidad es leer la historia con detenimiento. Pero conocer \bibemph{cómo} un ser se transforma no proporciona una comprensión inteligente del estatus actual de ese ser ni de su verdadero carácter.
\vs p019 1:11 \li{4.}La historia por sí misma no es capaz de revelar de forma satisfactoria el despliegue del futuro: el destino. Conocer los orígenes finitos tiene su utilidad, pero solo las causas divinas revelan los efectos finales. Los fines eternos no se manifiestan en los comienzos del tiempo. El presente solamente puede interpretarse a la luz de su correlación con el pasado y el futuro.
\vs p019 1:12 \pc Por tanto, por este motivo e incluso por otras razones, el método que utilizamos para acercarnos al hombre y a sus problemas planetarios es embarcarnos en un viaje en el espacio\hyp{}tiempo partiendo del Paraíso, desde la Fuente y Centro divina, infinita y eterna; fuente y centro de toda la realidad personal y de toda la existencia cósmica.
\usection{2. LOS PERFECCIONADORES DE LA SABIDURÍA}
\vs p019 2:1 Los perfeccionadores de la sabiduría son creación de la Trinidad cuyo propósito específico es personificar la sabiduría de la divinidad en los suprauniversos. Existen exactamente siete mil millones de estos seres, mil millones de los cuales están asignados a cada uno de los siete suprauniversos.
\vs p019 2:2 Al igual que sus iguales, los consejeros divinos y los censores universales, los perfeccionadores de la sabiduría experimentan la sabiduría del Paraíso, de Havona y, exceptuando a Lugar de la Divinidad, de las esferas del Padre en el Paraíso. Después de estas experiencias, se les asignó permanentemente al servicio de los ancianos de días. No prestan servicio ni en el Paraíso ni en los mundos de la vía circulatoria del Paraíso\hyp{}Havona, sino que se dedican exclusivamente a la administración de los gobiernos de los suprauniversos.
\vs p019 2:3 \pc Donde y cuando quiera que obre un perfeccionador de la sabiduría, allí y en ese momento obra la sabiduría divina. Hay una presencia real y una manifestación perfecta en el conocimiento y la sabiduría expresados en las acciones de estos seres personales poderosos y majestuosos. No \bibemph{reflejan} la sabiduría de la Trinidad del Paraíso sino que \bibemph{son} esa sabiduría. Son los recursos de la sabiduría para todos los maestros que enseñan la aplicación del conocimiento del universo. Son las fuentes de juicio maduro y los manantiales del juicio crítico para las instituciones de enseñanza y el discernimiento en todos los universos.
\vs p019 2:4 La sabiduría tiene un doble origen al provenir de la perfecta percepción divina consustancial a los seres perfectos y de la experiencia personal adquirida por las criaturas evolutivas. Los perfeccionadores de la sabiduría \bibemph{son} la sabiduría divina del Paraíso procedente de la perfecta percepción de la Deidad. En Uversa, sus colaboradores en la administración, los mensajeros poderosos, los sin nombre ni número y los elevados en autoridad, cuando actúan en conjunto, \bibemph{son} la sabiduría universal procedente de la experiencia. Un ser divino puede poseer un perfecto conocimiento divino; un mortal evolutivo puede algún día llegar a alcanzar un conocimiento perfecto como ascendente, pero ninguno de los dos por sí solo agota las potencialidades de toda la sabiduría posible. Por consiguiente, siempre que, para dirigir un suprauniverso, se requiera sabiduría de tipo administrativo, estos perfeccionadores de la sabiduría, que gozan de percepción divina, se vinculan siempre con aquellos seres personales ascendentes que han llegado a tener grandes responsabilidades y autoridad en el suprauniverso a través de las tribulaciones propias de sus experiencias en su camino de progreso evolutivo.
\vs p019 2:5 Los perfeccionadores de la sabiduría siempre necesitarán de la sabiduría complementaria de la experiencia para poder completar su sagacidad administrativa. Pero se supone que posiblemente los finalizadores del Paraíso, \bibemph{tras} ser en algún momento iniciados en la séptima etapa de la existencia espiritual, logran un elevado nivel de sabiduría no alcanzado hasta entonces. Si esta conclusión es correcta, entonces estos perfeccionados seres que han ascendido de forma evolutiva llegarían, sin duda, a ser los administradores del universo más eficaces jamás conocidos en toda la creación. Creo que es este el elevado destino de los finalizadores.
\vs p019 2:6 \pc La versatilidad de los perfeccionadores de la sabiduría les permite participar prácticamente en todo el servicio celestial que se presta a las criaturas ascendentes. Los perfeccionadores de la sabiduría y el orden de seres personales al que pertenezco, los consejeros divinos, junto con los censores universales, constituyen los órdenes más elevados de seres que pueden ocuparse, como de hecho hacen, de la labor de revelar la verdad a los planetas y sistemas individuales, tanto en sus épocas primitivas como cuando se asientan en luz y vida. Periódicamente, todos nosotros tomamos parte del servicio que se realiza para los mortales ascendentes, desde la primera vida en un planeta, pasando por un universo local hasta llegar a un suprauniverso, particularmente en este último.
\usection{3. LOS CONSEJEROS DIVINOS}
\vs p019 3:1 Estos seres de origen en la Trinidad son el consejo de la Deidad para los ámbitos de los siete suprauniversos. No \bibemph{reflejan} el consejo divino de la Trinidad; \bibemph{son} ese consejo. Hay veintiún mil millones de consejeros en servicio; tres mil millones se asignan a cada suprauniverso.
\vs p019 3:2 Los consejeros divinos son colaboradores de igual rango que los censores universales y que los perfeccionadores de la sabiduría; hay entre uno y siete consejeros vinculados a cada uno de estos seres personales mencionados. Estos tres órdenes participan en el gobierno de los ancianos de días, incluyendo los sectores mayores y menores, los universos locales y las constelaciones, y en los consejos de los soberanos de los sistemas locales.
\vs p019 3:3 Servimos de forma individual, tal como yo lo hago al redactar este escrito, pero también actuamos en grupos de tres cuando las circunstancias lo exigen. Cuando actuamos con capacidad ejecutiva, siempre lo hacemos en vinculación con un perfeccionador de la sabiduría, con un censor universal y con un número de consejeros divinos que oscila entre uno y siete.
\vs p019 3:4 \pc Un perfeccionador de la sabiduría, siete consejeros divinos y un censor universal constituyen un tribunal de la divinidad trinitaria, el más elevado órgano asesor con movilidad de los universos del tiempo y del espacio. Estos grupos de nueve se conocen como los tribunales que recaban información o que revelan la verdad, y cuando emiten un juicio sobre una determinada causa y pronuncian una resolución, es como si un anciano de días la hubiese tomado, porque en todos los anales de los suprauniversos jamás ha ocurrido que los ancianos de días hayan revocado tal veredicto.
\vs p019 3:5 Cuando los tres ancianos de días obran, la Trinidad del Paraíso obra. Cuando el tribunal de los nueve llega a una resolución tras unirse a deliberar, a efectos prácticos es como si hubieran dictaminado los ancianos de días. Y esta es la forma en que los gobernantes del Paraíso se ponen en contacto personal en cuanto a asuntos de tipo administrativo y disposiciones gubernamentales con los mundos, sistemas y universos individuales.
\vs p019 3:6 \pc Los consejeros divinos son la perfección del consejo divino de la Trinidad del Paraíso. Nosotros representamos, de hecho \bibemph{somos,} el consejo de perfección. Cuando se nos complementa con el consejo experiencial de nuestros colaboradores, de los seres de ascensión evolutiva perfeccionados y acogidos por la Trinidad, las conclusiones a las que llegamos en combinación no solo tienen carácter acabado sino pleno. Una vez que un censor universal ha incorporado, arbitrado, confirmado y promulgado nuestro consejo unificado, es muy probable que se acerque al umbral mismo de la totalidad universal. Estos veredictos representan la máxima aproximación posible a la actitud absoluta de la Deidad, dentro de los límites del espacio\hyp{}tiempo, con relación a la situación aludida y a la causa pertinente.
\vs p019 3:7 Siete consejeros divinos, en conjunción con un trío trinitizado evolutivo ---un mensajero poderoso, un elevado en autoridad y un sin nombre ni número--- representan la máxima aproximación del suprauniverso a la unión del punto de vista humano con la actitud divina a niveles casi paradisíacos respecto a significación espiritual y a valores de la realidad. Una aproximación tan grande a la unión de las actitudes cósmicas de criatura y Creador solo la sobrepasan los hijos de gracia del Paraíso, que son, en cada etapa de su experiencia personal, Dios y hombre.
\usection{4. LOS CENSORES UNIVERSALES}
\vs p019 4:1 Existen exactamente ocho mil millones de censores universales. Estos singulares seres \bibemph{son} el juicio de la Deidad. No reflejan simplemente resoluciones perfectas, sino que \bibemph{son} el juicio de la Trinidad del Paraíso. Ni siquiera los ancianos de días se constituyen en tribunal a menos que sea en colaboración con los censores universales.
\vs p019 4:2 Se nombra a un censor para cada uno de los mil millones de mundos del universo central; dicho censor se adscribe a la administración planetaria del eterno de días que allí reside. Los perfeccionadores de la sabiduría y los consejeros divinos no se adscriben de forma permanente a la administración de Havona; tampoco entendemos del todo por qué los censores universales se emplazan en el universo central. Sus actividades actuales difícilmente justifican su asignación a Havona, por eso sospechamos que están allí en anticipación de las necesidades de alguna futura era del universo en la que la población de Havona experimente algún cambio parcial.
\vs p019 4:3 Se asignan mil millones de censores a cada uno de los siete suprauniversos. Tanto de manera individual como en colaboración con los perfeccionadores de la sabiduría y los consejeros divinos, los censores operan en todos los sectores de los siete suprauniversos. Así pues, actúan en todos los niveles del gran universo, desde los mundos perfectos de Havona hasta los consejos de los soberanos de los sistemas, y son parte integrante de todos los juicios dispensacionales de los mundos evolutivos.
\vs p019 4:4 \pc Cuando y donde quiera que haya un censor universal, será allí y entonces el juicio de la Deidad. Puesto que los censores siempre pronuncian su veredicto en conjunción con los perfeccionadores de la sabiduría y los consejeros divinos, tales resoluciones comprenden la unión de la sabiduría, del consejo y del juicio de la Trinidad del Paraíso. En este trío jurídico, el perfeccionador de la sabiduría sería el “yo era”; el consejero divino, el “yo seré” y el censor universal, siempre, el “yo soy”.
\vs p019 4:5 \pc Los censores son los seres personales totalizadores del universo. Cuando un millar ---o un millón--- de testigos hayan dado su testimonio, cuando haya hablado la voz de la sabiduría y el consejo de la divinidad haya tomado nota, cuando se haya añadido el testimonio de los ascendentes progresivamente perfectos, entonces actuará el censor, y se revelará, de inmediato, una infalible y divina totalidad de todo lo que ha sucedido, y dicha declaración representa la conclusión divina, la síntesis de una determinación perfecta y final. Así pues, cuando se haya pronunciado un censor, ya nadie más podrá hablar, porque el censor ha presentado el total, verdadero e inequívoco, de cuanto ha acaecido. Cuando él se pronuncia, no hay apelación.
\vs p019 4:6 Realmente, entiendo cómo opera la mente de los perfeccionadores de la sabiduría, pero de cierto no comprendo del todo el funcionamiento de la mente judicativa de los censores universales. Me parece que los censores definen significaciones nuevas y originan nuevos valores relacionando los hechos, las verdades y los hallazgos que se exponen ante ellos en el curso de una investigación de los asuntos relativos al universo. Parece probable que los censores universales elaboren interpretaciones originales sobre la base de la combinación de la perfecta percepción del Creador con la experiencia de las criaturas perfeccionadas. Esta conjunción de la perfección del Paraíso y la experiencia experimentada en el universo deviene sin duda en un nuevo y último valor.
\vs p019 4:7 Pero no terminan aquí nuestras dificultades de comprensión con relación al funcionamiento de las mentes de los censores universales. Tomando debida cuenta de todo lo que sabemos o creemos saber acerca de la actuación de los censores en situaciones que se dan en el universo, no somos capaces de predecir sus resoluciones ni de predecir sus veredictos. Podemos determinar con precisión el resultado probable, producto de la vinculación de la actitud del Creador con la experiencia de la criatura, pero dichas conclusiones no siempre constituyen un pronóstico preciso de las revelaciones del censor. Es posible que los censores estén de alguna manera en vinculación con el Absoluto de la Deidad; de ningún otro modo podríamos explicar muchas de sus decisiones y dictámenes.
\vs p019 4:8 \pc Los perfeccionadores de la sabiduría, los consejeros divinos y los censores universales, junto con los siete órdenes de seres personales supremos de la Trinidad, constituyen esos diez grupos que a veces se han denominado \bibemph{hijos estacionarios de la Trinidad}. Juntos componen el magno colectivo de administradores, gobernantes, mandatarios, asesores, consejeros y jueces de la Trinidad. Su número excede ligeramente los treinta y siete mil millones. Dos mil setenta millones están emplazados en el universo central y algo más de cinco mil millones en cada suprauniverso.
\vs p019 4:9 Es muy difícil describir los límites operativos de los hijos estacionarios de la Trinidad. Sería incorrecto decir que sus acciones se limitan a lo finito porque en el suprauniverso se han documentado actuaciones que indican lo contrario. Actúan en cualquier nivel administrativo o judicial del universo, según lo requieran las condiciones espacio\hyp{}temporales en relación con la evolución pasada, presente y futura del universo matriz.
\usection{5. ESPÍRITUS INSPIRADOS DE LA TRINIDAD}
\vs p019 5:1 Muy poco podré deciros respecto a los espíritus inspirados de la Trinidad porque pertenecen a uno de los pocos órdenes existentes que son completamente secretos; secretos sin duda, porque les es imposible desvelarse enteramente a sí mismos ni siquiera ante aquellos de nosotros cuyo origen es tan cercano a la fuente de su propia creación. Deben su ser a la acción de la Trinidad del Paraíso y pueden valerse de ellos una o dos Deidades al igual que tres de ellas. No sabemos si el número de estos espíritus está completo o si está constantemente incrementando, pero nos inclinamos a pensar que no hay una cifra limitada.
\vs p019 5:2 No entendemos del todo ni la naturaleza ni el proceder de los espíritus inspirados. Tal vez pertenezcan a la categoría de espíritus suprapersonales. Parecen operar en todas las vías circulatorias conocidas y casi con independencia del tiempo y del espacio. Pero poco sabemos acerca de ellos, a excepción de lo que deducimos de su carácter a partir de la naturaleza de su actividad, cuyo resultado ciertamente observamos por doquier en los universos.
\vs p019 5:3 Bajo ciertas condiciones, estos espíritus inspirados pueden individualizarse a sí mismos lo suficiente como para que los seres de origen en la Trinidad puedan reconocerlos. Yo los he visto; pero los órdenes más modestos de seres celestiales jamás podrían reconocer a ninguno de ellos. Periódicamente, también se dan ciertas circunstancias, respecto a la dirección de los universos en evolución, en las que un ser de origen en la Trinidad puede valerse directamente de estos espíritus para que lo apoyen en sus tareas. Sabemos, pues, que existen, y que bajo ciertas circunstancias podemos solicitarles que vengan para asistirnos, y que a veces podemos percatarnos de su presencia. Pero no forman parte de la estructura organizativa, manifiesta y definidamente revelada, planificada para los universos del tiempo y del espacio antes de que estas creaciones materiales se asienten en luz y vida. No ocupan un lugar claramente perceptible dentro de la actual organización o administración de los siete suprauniversos evolutivos. Son un secreto de la Trinidad del Paraíso.
\vs p019 5:4 Los melquisedecs de Nebadón imparten la enseñanza de que los espíritus inspirados de la Trinidad están destinados, en algún momento del futuro eterno, a ocupar el lugar de los mensajeros solitarios que va quedando vacante, de forma lenta pero firme, conforme que se les asigna como colaboradores de ciertos tipos de hijos trinitizados.
\vs p019 5:5 \pc Los espíritus inspirados son los espíritus solitarios del universo de los universos. Como espíritus tienen bastante semejanza con los mensajeros solitarios, aunque estos últimos son seres personales diferenciados. Nuestro conocimiento de los espíritus inspirados proviene en gran parte de los mensajeros solitarios, que detectan su cercanía en virtud de una inherente sensibilidad a su presencia que actúa tan indefectiblemente como una aguja imantada que señalara un polo magnético. Cuando un mensajero solitario está cerca de un espíritu inspirado de la Trinidad, percibe una indicación de tipo cualitativo de tal presencia divina y también una muy clara señal de orden cuantitativo que le permite, en efecto, conocer la clasificación o el número de la presencia o presencias espirituales de estos.
\vs p019 5:6 Podría relatar otro hecho de interés: cuando un mensajero solitario se encuentra en un planeta como Urantia, en cuyos habitantes moran modeladores del pensamiento, él, por esa sensibilidad que lo hace ser receptivo a la presencia espiritual, percibe un estímulo de tipo cualitativo. En estos casos no hay estímulo de tipo cuantitativo, sino tan solo una sensación de tipo cualitativo. Cuando se encuentra en un planeta no visitado por los modeladores, su contacto con los nativos de tal planeta no le produce ningún tipo de efecto en este sentido. Esto indica que los modeladores del pensamiento están de alguna manera relacionados o unidos con los espíritus inspirados de la Trinidad del Paraíso. Es posible que estén de alguna forma vinculados en ciertas etapas de su respectiva labor; aunque en realidad no lo sabemos. Ambos se originan cerca del centro y fuente de todas las cosas, pero no pertenecen al mismo orden de seres. Los modeladores del pensamiento provienen del Padre solo; los espíritus inspirados son vástagos de la Trinidad del Paraíso.
\vs p019 5:7 Los espíritus inspirados no parecen pertenecer al esquema evolutivo de los planetas ni de los universos y, sin embargo, parecen estar casi en todas partes. Ahora mismo, mientras estoy realizando esta exposición, la sensibilidad personal del mensajero solitario que me acompaña a la presencia de este orden de ser indica que se halla con nosotros, en este mismo momento, a no más de ocho metros de distancia, un espíritu del orden de los inspirados y con un tercer volumen de presencia expresada en potencia, que nos sugiere que es probable que haya tres de ellos actuando en conjunción.
\vs p019 5:8 \pc De los más de doce órdenes de seres que me acompañan en este momento, el mensajero solitario es el único que percibe la presencia de estas misteriosas entidades de la Trinidad. Y, además, aunque estemos informados respecto de la cercanía de estos espíritus divinos, todos nosotros igualmente ignoramos su misión. En realidad, no sabemos si observan lo que hacemos por simple interés o si, de alguna manera que nos es desconocida, contribuyen realmente al éxito de nuestra tarea.
\vs p019 5:9 Sabemos que los hijos preceptores de la Trinidad se dedican a la instrucción \bibemph{consciente} de las criaturas del universo. He llegado a la conclusión de que los espíritus inspirados de la Trinidad, mediante procedimientos \bibemph{supraconscientes,} también actúan en los mundos como preceptores. Estoy convencido de que existe un inmenso volumen de conocimiento espiritual de carácter esencial, de verdad indispensable para la elevada realización espiritual, que no se puede aprehender de forma consciente, puesto que ser consciente de ella pondría de hecho en riesgo la garantía de su recepción. Si tenemos razón en esta idea, y todos los seres del orden al que pertenezco la comparten, es posible que la misión de estos espíritus inspirados sea la de vencer este obstáculo, la de tender un puente en el plan universal entre la lucidez moral y el avance espiritual. Pensamos que estos dos tipos de preceptores de origen en la Trinidad están vinculados en cuanto a su actividad, aunque no lo sabemos realmente.
\vs p019 5:10 En los mundos de formación de los suprauniversos y en las eternas vías planetarias de Havona, he confraternizado con los mortales en su camino de perfección ---con almas espiritualizadas que ascienden desde los ámbitos evolutivos---, pero nunca son conscientes de la presencia de los espíritus inspirados, que una y otra vez las facultades detectoras de los mensajeros solitarios nos indicaban que estaban muy próximos a nosotros. He conversado abiertamente con todos los órdenes de Hijos de Dios, tanto de mayor como de menor rango, y ninguno de ellos ha sido consciente de las recomendaciones de los espíritus inspirados de la Trinidad. Estos pueden y de hecho examinan sus experiencias pasadas y relatan sucesos que resultarían difíciles de explicar a no ser por la acción de tales espíritus. Pero exceptuando a los mensajeros solitarios y, a veces, a los seres de origen en la Trinidad, ningún miembro de la familia celestial ha sido nunca consciente de la cercanía de los espíritus inspirados.
\vs p019 5:11 No creo que los espíritus inspirados de la Trinidad estén jugando al escondite conmigo. Es probable que pugnen por manifestarse ante mí como yo por comunicarme con ellos. Debemos compartir dificultades y limitaciones naturales. Estoy convencido de que no existen secretos arbitrariamente guardados en el universo y, por tanto, nunca cejaré en mi empeño de resolver el misterio del aislamiento de estos espíritus pertenecientes a mi orden de creación.
\vs p019 5:12 Y por todo esto, vosotros los mortales, que apenas estáis dando ahora vuestros primeros pasos hacia el viaje eterno, os dais cuenta de que debéis avanzar un largo trecho antes de progresar con certeza “visual” y “material”. Tendréis que valeros de la fe durante mucho tiempo y depender de la revelación si esperáis progresar con rapidez y seguridad.
\usection{6. LOS NATIVOS DE HAVONA}
\vs p019 6:1 Los nativos de Havona se crean por la acción directa de la Trinidad del Paraíso, y su número se escapa a la comprensión de vuestras limitadas mentes. Tampoco es posible para los habitantes de Urantia concebir los atributos naturales de criaturas tan divinamente perfectas como estas razas de origen en la Trinidad del universo eterno. Nunca podréis formaros una imagen verdadera de estas criaturas gloriosas. Debéis aguardar hasta el momento de vuestra llegada a Havona para poder saludarles como vuestros camaradas espirituales.
\vs p019 6:2 Durante vuestra larga estancia en los mil millones de mundos de Havona para aprender su cultura, desarrollaréis una amistad eterna con estos magníficos seres. ¡Y cuán profunda es la amistad que florece entre la más modesta criatura personal de los mundos del espacio y estos elevados seres personales nacidos en las esferas perfectas del universo central! Los mortales ascendentes, en su larga y tierna vinculación con los nativos de Havona, hacen mucho para compensar las carencias espirituales de las etapas primitivas en las que progresaban como mortales. Al mismo tiempo, gracias a sus contactos con los peregrinos ascendentes, los nativos de Havona adquieren unas vivencias que reducen, de forma considerable, sus limitaciones experienciales por haber vivido siempre una vida de perfección divina. Son grandes y mutuos los beneficios tanto para los mortales ascendentes como para los nativos de Havona.
\vs p019 6:3 \pc Los nativos de Havona, como todos los seres personales de origen en la Trinidad, se conciben en perfección divina y pueden, al igual que otros seres personales de origen en la Trinidad, incrementar sus atributos experienciales con el paso del tiempo. Pero, a diferencia de los hijos estacionarios de la Trinidad, los nativos de Havona pueden evolucionar en estatus y tener un futuro destino no revelado en la eternidad. Esto se ilustra en esos nativos de Havona que, a través de sus servicios, se han capacitado para fusionarse con una fracción del Padre diferente del modelador y reunir los requisitos para ser miembros del colectivo final de los mortales. Y hay otros colectivos de finalizadores accesibles a estos nativos del universo central.
\vs p019 6:4 \pc La evolución en el estatus de los nativos de Havona ha provocado muchas especulaciones en Uversa. Puesto que van penetrando constantemente en los distintos colectivos finales del Paraíso y, puesto que no se crean más, es evidente que el número de nativos que queda en Havona disminuye constantemente. Jamás se nos han revelado las últimas consecuencias de estos acontecimientos, pero no creemos que Havona quede completamente vacía de nativos. Albergamos la teoría de que los nativos de Havona posiblemente dejen de integrarse en los colectivos de finalizadores en algún momento durante las eras de creaciones consecutivas en los niveles exteriores del espacio. También abrigamos la idea de que, en estas eras posteriores del universo, el universo central podría poblarse de un grupo mixto de seres que residieran allí, una ciudadanía consistente solo en parte de nativos originarios de Havona. No sabemos qué orden o tipo de criatura puede estar destinada a tener su residencia en Havona en el futuro, pero hemos pensado en:
\vs p019 6:5 \li{1.}Los univitatias, que son actualmente los ciudadanos permanentes de las constelaciones de los universos locales.
\vs p019 6:6 \li{2.}Futuros tipos de mortales que puedan nacer en las esferas habitadas de los suprauniversos durante el florecimiento de las eras de luz y vida.
\vs p019 6:7 \li{3.}La nobleza espiritual que llegara de los consecutivos niveles de los universos exteriores.
\vs p019 6:8 Sabemos que la Havona de la era anterior del universo era un tanto diferente de la Havona de la era presente. No consideramos sino razonable suponer que estamos presenciando ahora esos lentos cambios del universo central que anticipan a las eras por venir. Hay algo cierto: el universo no es estático; solamente Dios es invariable.
\usection{7. LOS CIUDADANOS DEL PARAÍSO}
\vs p019 7:1 En el Paraíso residen numerosos grupos de seres magníficos: los ciudadanos del Paraíso. Puesto que no se ocupan directamente del plan de perfeccionamiento de las criaturas ascendentes de voluntad, no se revelan del todo a los mortales de Urantia. Existen más de tres mil órdenes de estas inteligencias excelsas. El último grupo adquirió su ser personal de forma simultánea al mandato de la Trinidad que promulgó el plan de creación de los siete suprauniversos del tiempo y del espacio.
\vs p019 7:2 \pc A los ciudadanos del Paraíso y a los nativos de Havona se les denomina a veces de forma conjunta \bibemph{seres personales del Paraíso\hyp{}Havona.}
\vs p019 7:3 \pc Esto completa la historia de los seres que tienen su existencia gracias a la Trinidad del Paraíso. Estos seres jamás se han descarriado, a pesar de estar todos dotados de libre voluntad en su máxima expresión.
\vs p019 7:4 Los seres de origen en la Trinidad poseen unas prerrogativas para transitar que los hacen independientes de los seres personales transportadores, como los serafines. Todos nosotros poseemos la facultad de movernos con libertad y rapidez en el universo de los universos. Exceptuando a los espíritus inspirados de la Trinidad, no podemos alcanzar la casi increíble velocidad de los mensajeros solitarios, pero podemos hacer uso de la totalidad de los servicios de transporte disponibles en el espacio para llegar a cualquier punto de un suprauniverso, desde su sede central, en menos de un año en tiempo de Urantia. Me llevó 109 días de vuestro tiempo viajar de Uversa a Urantia.
\vs p019 7:5 Por los mismos medios, nos es posible comunicarnos de forma instantánea entre nosotros. Nuestro orden de creación está en contacto con todos los seres de cualquiera de las categorías de hijos de la Trinidad del Paraíso, con la única excepción de los espíritus inspirados.
\vsetoff
\vs p019 7:6 [Exposición de un consejero divino de Uversa.]
