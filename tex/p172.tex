\upaper{172}{Entrada en Jerusalén}
\author{Comisión de seres intermedios}
\vs p172 0:1 Jesús y los apóstoles llegaron a Betania poco después de las cuatro de la tarde del viernes 31 de marzo, del año 30 d. C. Lázaro, sus hermanas y sus amigos los estaban esperando. Y como todos los días acudía tanta gente para hablar con Lázaro de su resurrección, le comunicaron a Jesús que se habían hecho arreglos para su alojamiento con un vecino creyente, un tal Simón, el más prominente ciudadano de esa pequeña villa desde la muerte del padre de Lázaro.
\vs p172 0:2 A última hora de aquella tarde, Jesús recibió a muchos visitantes, y la gente corriente de Betania y Betfagé hizo cuanto pudo para que se sintiera bien entre ellos. Aunque muchos creían que Jesús iba en aquel momento a Jerusalén para proclamarse rey de los judíos, desafiando manifiestamente el decreto de muerte del sanedrín que pesaba sobre él, la familia de Betania ---Lázaro, Marta y María--- era enteramente consciente de que el Maestro no era esa clase de rey; no obstante, vagamente tenían la sensación de que aquella podría ser su última visita a Jerusalén y Betania.
\vs p172 0:3 Los sumos sacerdotes estaban informados de que Jesús se hospedaba en Betania, pero, pensaron que sería mejor no prenderlo estando entre sus amigos y decidieron esperar a que llegara a Jerusalén. Jesús sabía todo esto, pero estaba majestuosamente tranquilo; sus amigos no lo habían visto nunca tan sereno y cordial; incluso los apóstoles estaban asombrados de que estuviera tan despreocupado, cuando el sanedrín había hecho un llamamiento a todos los judíos para que lo entregaran. Mientras el Maestro dormía aquella noche, los apóstoles lo vigilaban de dos en dos, y muchos de ellos portaban espadas. Temprano, a la mañana siguiente, incluso siendo el día del \bibemph{sabbat,} cientos de peregrinos procedentes de Jerusalén los despertaron, para ver a Jesús y a Lázaro, a quien él había resucitado de entre los muertos.
\usection{1. SÁBADO EN BETANIA}
\vs p172 1:1 Los peregrinos provenientes de fuera de Judea, al igual que las autoridades judías, se preguntaban unos a otros: “¿Qué pensáis? ¿Vendrá Jesús a la fiesta?”. Por lo tanto, cuando la gente oyó que Jesús estaba en Betania, se alegró, pero los sumos sacerdotes y los fariseos se quedaron algo perplejos. Les contentaba tenerlo bajo su jurisdicción, aunque estaban un poco desconcertados por su temeridad; recordaban que en su visita anterior a Betania, había resucitado a Lázaro de entre los muertos, y Lázaro se estaba convirtiendo en un grave problema para los enemigos de Jesús.
\vs p172 1:2 Seis días antes de la Pascua, al anochecer tras el \bibemph{sabbat,} todo Betania y Betfagé se reunió para celebrar la llegada de Jesús con un banquete público en la casa de Simón. Era una cena en honor de Jesús y de Lázaro, que se ofreció contraviniendo al sanedrín. Marta dirigía el servicio de los alimentos; su hermana María se hallaba entre las mujeres espectadoras, ya que iba en contra de las costumbres judías que las mujeres participaran en un banquete público. Los agentes del sanedrín estaban presentes, pero temían prender a Jesús en medio de sus amigos.
\vs p172 1:3 Jesús habló con Simón sobre el Josué de la antigüedad, cuyo nombre era homónimo del suyo, y relató la historia de cómo Josué y los israelitas habían llegado a Jerusalén a través de Jericó. Al comentar la leyenda del derrumbamiento de los muros de Jericó, Jesús dijo: “No me preocupan esos muros de ladrillos y piedras; pero provocaré el desmoronamiento de los muros del prejuicio, la hipocresía y el odio con esta predicación del amor del Padre por todos los hombres”.
\vs p172 1:4 El banquete se desarrolló, como era normal, de manera animosa, exceptuando a todos los apóstoles, que estaban inusitadamente serios. Jesús estaba excepcionalmente alegre y estuvo jugando con los niños hasta el momento de dirigirse a la mesa.
\vs p172 1:5 \pc No sucedió nada fuera de lo ordinario hasta cerca del fin de la comida, cuando María, la hermana de Lázaro, se adelantó del grupo de las mujeres espectadoras y, acercándose hasta donde Jesús estaba reclinado como invitado de honor, abrió un gran vaso de alabastro con un perfume muy exquisito y costoso; y tras ungir la cabeza del Maestro, empezó a derramarlo sobre los pies de Jesús, mientras se soltaba el cabello y le secaba los pies con él. Toda la casa se llenó del olor del perfume, y todos los presentes quedaron asombrados de lo que María había hecho. Lázaro no dijo nada, pero cuando algunas personas murmuraron, mostrando su indignación de que se hubiera usado un perfume tan caro de aquella manera, Judas Iscariote se dirigió hasta donde estaba Andrés reclinado y dijo: “¿Por qué no se vendió este perfume y se usó el dinero para dar de comer a los pobres?”. Deberías hablar con el Maestro para que desapruebe este desperdicio”.
\vs p172 1:6 Sabiendo lo que pensaban y oyendo lo que decían, Jesús puso su mano sobre la cabeza de María, que estaba arrodillada a su lado, y, con una expresión amable en su rostro, dijo: “Dejadla, ¿por qué la molestáis cuando ha hecho buena obra en su corazón? A vosotros, los que murmuráis y decís que este perfume ha debido venderse para darle el dinero a los pobres, os digo que siempre tendréis a los pobres con vosotros, y cuando queráis les podréis hacer bien, pero a mí no siempre me tendréis; pues pronto iré a mi Padre. Esta mujer lleva tiempo guardando este perfume para verterlo en mi cuerpo en la sepultura; y ahora que le ha parecido bien ungirlo sobre mí porque se ha anticipado a mi muerte, tal satisfacción no le será denegada. Con su acto, María os ha desautorizado a todos vosotros, porque manifiesta su fe en lo que yo he dicho respecto a mi muerte y ascensión a mi Padre de los cielos. A ella no se le reprochará nada por lo que ha hecho esta noche; más bien os digo que en las eras por venir, dondequiera que se predique, en todo el mundo, se contará lo que esta mujer ha hecho, para memoria de ella”.
\vs p172 1:7 Fue este reproche, que Judas Iscariote interpretó como una reprobación personal, el que lo llevó a tomar finalmente la decisión de buscar venganza para sus sentimientos heridos. Muchas veces había albergado en su subconsciente estas ideas, pero fue en aquel momento cuando se atrevió a pensar, clara y conscientemente, con tanta perversidad. Y muchos otros lo alentaron en aquella actitud, puesto que el coste de este perfume equivalía a lo que ganaba un hombre en todo un año ---suficiente para proporcionar pan a cinco mil personas---. Pero María amaba a Jesús; ella se había hecho de este valioso perfume para ungir su cuerpo en la hora de su muerte, porque creía en sus palabras cuando él los previno de que habría de morir, y nadie podría privarla de aquello si ella cambiaba de opinión y había decidido hacerle esta ofrenda al Maestro estando él aún vivo.
\vs p172 1:8 Tanto Lázaro como Marta sabían que María llevaba mucho tiempo ahorrando para poder comprar aquel vaso de perfume de nardo puro, y aprobaban enteramente lo que había hecho pues había seguido los deseos de su corazón, y porque eran personas acomodadas y podían fácilmente permitirse dicha ofrenda.
\vs p172 1:9 Cuando los sumos sacerdotes se enteraron de esta cena, dada en Betania para agasajar a Jesús y Lázaro, comenzaron a consultarse entre ellos y planear qué debían hacer con Lázaro. Y rápidamente decidieron que Lázaro también tenía que morir. Concluyeron por lógica que resultaría inútil dar muerte a Jesús si permitían que Lázaro, a quien Jesús había resucitado de entre los muertos, viviera.
\usection{2. DOMINGO POR LA MAÑANA CON LOS APÓSTOLES}
\vs p172 2:1 Ese domingo por la mañana, en el hermoso jardín de Simón, el Maestro convocó a sus doce apóstoles y les dio unas últimas instrucciones, previas a su entrada en Jerusalén. Les dijo que probablemente daría muchas charlas e impartiría numerosos preceptos antes de volver al Padre, pero aconsejó a los apóstoles que no realizaran ninguna labor pública mientras asistieran a la Pascua de Jerusalén. Les mandó que permanecieran cerca de él y que “vigilaran y oraran”. Jesús sabía que muchos de sus apóstoles y seguidores más cercanos llevaban con ellos incluso entonces espadas escondidas, pero no hizo ninguna alusión a aquello.
\vs p172 2:2 En las enseñanzas matutinas, el Maestro hizo un examen del ministerio que los apóstoles habían realizado desde el día de su ordenación, cerca de Cafarnaúm, hasta ese momento en el que se preparaban para entrar a Jerusalén. Los apóstoles escucharon en silencio; no hicieron ninguna pregunta.
\vs p172 2:3 Temprano aquella mañana, David Zebedeo entregó a Judas los fondos conseguidos por la venta del equipamiento del campamento de Pella y, Judas, por su parte, depositó la mayor parte de ese dinero en las manos de Simón, su anfitrión, para mayor seguridad y en previsión de las necesidades que pudieran surgirles al entrar en Jerusalén.
\vs p172 2:4 Tras su charla con los apóstoles, Jesús conversó con Lázaro y le dijo que evitara sacrificar su vida para aplacar las ansias de venganza del sanedrín. Obedeciendo esta advertencia, Lázaro huyó a Filadelfia pocos días después, cuando los oficiales del sanedrín mandaron a arrestarlo.
\vs p172 2:5 De alguna manera, todos los seguidores de Jesús presentían la crisis que se avecinaba, pero la alegría inusual y el excepcional buen humor del Maestro les impedía realmente darse cuenta de la gravedad de la situación.
\usection{3. SALIDA HACIA JERUSALÉN}
\vs p172 3:1 Betania distaba unos tres kilómetros del templo, y era la una y media de la tarde de ese domingo cuando Jesús se dispuso a salir para Jerusalén. Sentía un profundo afecto por Betania y su gente sencilla. Nazaret, Cafarnaúm y Jerusalén lo habían rechazado, pero Betania lo había aceptado, había creído en él. Y fue en esta pequeña aldea, en la que prácticamente todo hombre, mujer y niño eran creyentes, donde decidió llevar a cabo la obra más grande de su ministerio de gracia en la tierra: la resurrección de Lázaro. No hizo resucitar a Lázaro para que los aldeanos pudieran creer, sino más bien porque ya creían.
\vs p172 3:2 Durante toda la mañana, Jesús había estado pensando sobre su entrada en Jerusalén. Hasta entonces, su intención había sido la de impedir que se le aclamara públicamente como el Mesías, pero ahora era diferente; se acercaba el fin de su andadura en la carne, el sanedrín había decretado su muerte, y ningún mal podría venirles por permitir que sus discípulos dieran rienda suelta a sus emociones, algo que ocurriría si él optaba por entrar en la ciudad de forma oficial y pública.
\vs p172 3:3 Jesús no decidió hacer esta entrada pública a Jerusalén como una última tentativa de ganarse el favor de la gente ni para apropiarse definitivamente del poder. Tampoco lo hizo para satisfacer totalmente los anhelos humanos de sus discípulos y apóstoles. Jesús no se dejaba llevar por las ilusorias expectativas de un soñador fantasioso; sabía bien cuál sería el resultado de esta visita.
\vs p172 3:4 Al haber decidido hacer su entrada en Jerusalén de manera notoria, el Maestro se vio en la necesidad de elegir la manera apropiada de llevar a efecto su propósito. Jesús recapacitó sobre todas las llamadas profecías mesiánicas, más o menos contradictorias, pero solo una de ellas le pareció adecuada para lo que pretendía. La mayoría de estas manifestaciones proféticas retrataban a un rey, al hijo y sucesor de David, a un libertador terrenal, valeroso y beligerante, que liberaría a Israel del yugo de la dominación extranjera. Pero había un pasaje en las Escrituras, relacionado a veces con el Mesías por parte de aquellos que tenían una idea más espiritual de su misión, que Jesús consideró como el más apto para poder guiarse por él en su entrada prevista en Jerusalén. Este pasaje estaba en Zacarías y decía: “Alégrate mucho, hija de Sion; da voces de júbilo, hija de Jerusalén. Mira que tu rey vendrá a ti, justo y salvador. Vendrá humildemente, cabalgando sobre un asno, sobre un pollino hijo de asna”.
\vs p172 3:5 \pc Un rey guerrero entraba en una ciudad montado a caballo; un rey en misión de paz y amistad siempre entraba cabalgando sobre un asno. Jesús no quería entrar en Jerusalén a lomos de un caballo, aunque sí en son de paz y buena voluntad, como el Hijo del Hombre, sentado en un asno.
\vs p172 3:6 \pc Durante mucho tiempo, Jesús había tratado claramente, mediante sus enseñanzas, de inculcar en sus apóstoles y discípulos la idea de que su reino no era de este mundo, que era puramente espiritual; pero sus esfuerzos habían resultado baldíos. Ahora, lo que no había podido conseguir enseñándoles con franqueza y de manera personal, lo intentaría hacer mediante un acto simbólico. Por ello, justo después del almuerzo, Jesús llamó a Pedro y Juan y, tras decirles que fueran a Betfagé, una aldea vecina, un poco retirada de la carretera principal y a escasa distancia al noroeste de Betania, añadió: “Id a Betfagé y cuando lleguéis al cruce de los caminos encontraréis al pollino de una asna atado. Desatadlo y traedlo. Si alguien os pregunta por qué hacéis esto, decid simplemente ‘el Maestro lo necesita’”. Y, cuando los dos apóstoles fueron a Betfagé tal como el Maestro les había mandado, hallaron al pollino atado junto a su madre en la calle, junto a una casa de esquina. Al comenzar Pedro a desatar al pollino, vino el dueño y les preguntó por qué hacían aquello, y cuando Pedro le respondió tal como Jesús les había dicho, el hombre dijo: “Si vuestro Maestro es Jesús de Galilea, que tenga el pollino”. Y así regresaron trayendo al pollino con ellos.
\vs p172 3:7 Para ese momento, varios cientos de peregrinos se habían congregado alrededor de Jesús y de sus apóstoles. Desde media mañana, los visitantes que pasaban camino de la Pascua se habían ido deteniendo allí. Mientras David Zebedeo y algunos de sus antiguos compañeros mensajeros se dirigieron de prisa a Jerusalén y se encargaron acertadamente de difundir la noticia, entre las muchedumbres de peregrinos que habían acudido al templo, de que Jesús de Nazaret haría una entrada triunfal en la ciudad. Como consecuencia, varios miles de estos visitantes fueron en masa a saludar a este profeta y obrador de prodigios del que tanto se hablaba, y a quien algunos consideraban el Mesías. Al salir esta multitud de Jerusalén, se encontró con Jesús y con la muchedumbre que iba a la ciudad justo tras haber atravesado la cima del Monte de los Olivos y había comenzado la bajada hacia la ciudad.
\vs p172 3:8 Cuando la comitiva salió de Betania, se respiraba un gran entusiasmo de parte del festivo grupo de discípulos, creyentes y peregrinos visitantes, muchos procedentes de Galilea y Perea. Justo antes de partir, las doce integrantes del primer colectivo de mujeres, junto con algunas de sus compañeras, llegaron al lugar y se unieron a esta extraordinaria comitiva que avanzaba con alborozo hacia la ciudad.
\vs p172 3:9 Antes de salir, los gemelos Alfeo habían echado sus túnicas sobre el asno y lo habían sujetado mientras que el Maestro se subía encima. Y conforme la comitiva avanzaba hacia la cima del Monte de los Olivos, la jubilosa multitud arrojaba sus mantos al suelo y tomaban ramas de los árboles cercanos para hacer una alfombra al paso del asno que llevaba al Hijo real, al Mesías prometido. Mientras que esta alegre multitud continuaba su camino hacia Jerusalén, comenzaron a cantar o, más bien, a gritar al unísono, el salmo, “Hosanna al hijo de David; bendito el que viene en el nombre del Señor. Hosanna en las alturas. Bendito sea el reino que baja del cielo”.
\vs p172 3:10 Jesús estuvo despreocupado y alegre todo el camino hasta llegar a la cima del Monte de los Olivos, desde donde había una vista completa de la ciudad y de las torres del templo; allí el Maestro detuvo la comitiva, y todos se sintieron envueltos en un gran silencio al verle llorar. Bajando los ojos hacia la inmensa multitud que venía de la ciudad para recibirlo, el Maestro, lleno de emoción y entre lágrimas, dijo: “¡Oh, Jerusalén, si sencillamente hubieras conocido, incluso tú, al menos en este tu día, lo que es para tu paz y que podrías haber tenido con tanta abundancia! Pero ahora estas glorias están a punto de estar encubiertas a tus ojos. Vas a rechazar al Hijo de la Paz y dar la espalda al evangelio de la salvación. Pronto vendrán días sobre ti cuando tus enemigos te rodearán con cerca, te sitiarán y por todas partes te estrecharán; te derribarán a tierra, y no dejarán en ti piedra sobre piedra, por cuanto no conociste el tiempo de tu divina visitación. Vas a rechazar el don de Dios, y todos los hombres te rechazarán a ti”.
\vs p172 3:11 Cuando Jesús acabó de hablar, comenzaron la bajada del Monte de los Olivos y, al poco tiempo, se les unió la multitud de visitantes que venía de Jerusalén agitando ramas de palmeras, gritando hosannas, expresando de muchas maneras su regocijo y buena fraternidad. No estaba en los planes del Maestro que esta muchedumbre saliera de Jerusalén a su encuentro; aquello fue tarea de otros. Él nunca premeditaba nada que fuese espectacular.
\vs p172 3:12 Además de la multitud que afluía para darle la bienvenida al Maestro, venían también muchos fariseos y sus otros enemigos. Estaban tan alterados por este súbito e inesperado estallido de aclamación popular que tuvieron miedo de arrestarlo no fuese que tal medida precipitara una flagrante revuelta pública. Temían sobremanera la actitud del gran número de visitantes que habían oído hablar tanto de Jesús, y en el que, muchos de ellos, creían.
\vs p172 3:13 Al acercarse a Jerusalén, la multitud se volvió más efusiva, tanto que algunos de los fariseos se abrieron paso hasta el lado de Jesús y le dijeron: “Maestro, debes reprender a tus discípulos para que se comporten de manera más decorosa”. Jesús respondió: “Es natural que estos hijos míos le den la bienvenida al Hijo de la Paz, a quien los sumos sacerdotes han rechazado. Si se les callara, estas piedras del camino clamarían”.
\vs p172 3:14 Los fariseos se apresuraron para adelantarse a la comitiva y regresar al sanedrín, que estaba entonces reunido en el templo. Allí informaron a sus compañeros, diciéndoles: “Oíd, lo que hacemos no sirve de nada; estamos desconcertados ante este galileo. La gente ha perdido el juicio con él; si no paramos a estos ignorantes, todo el mundo lo seguirá”.
\vs p172 3:15 Realmente, no se puede asignar ningún significado de importancia a aquel estallido, superficial y espontáneo, de entusiasmo popular. Esta bienvenida, aunque festiva y honesta, no indicaba que los corazones de esta jubilosa multitud albergasen convicciones de fe reales ni profundamente afianzadas. Aquella misma multitudes tuvo asimismo dispuesta a rechazar rápidamente a Jesús más tarde, en esa misma semana, cuando el sanedrín adoptó decididamente una postura firme contra él, y cuando se sintieron defraudados, al darse cuenta de que Jesús no iba a instaurar el reino, conforme a unas expectativas que por tanto tiempo habían acariciado.
\vs p172 3:16 Pero toda la ciudad estaba fuertemente agitada hasta tal punto que todos preguntaban: “¿Quién es este hombre?”. Y la muchedumbre respondía: “Este es Jesús de Nazaret, el profeta de Galilea”.
\usection{4. LA VISITA AL TEMPLO}
\vs p172 4:1 Mientras los gemelos Alfeo devolvían el asno a su dueño, Jesús y los diez apóstoles se separaron de sus acompañantes más allegados y dieron un paseo por el templo, al tiempo que observaban los preparativos para la Pascua. No hubo ningún intento de importunar a Jesús, dado que el sanedrín sentía miedo de la gente, y aquella era, en definitiva, una de las razones por las que Jesús había permitido que se le aclamara de tal manera. Los apóstoles no llegaban a entender que aquel era el único modo humano posible de evitar el arresto de Jesús nada más entrar en la ciudad. El Maestro deseaba dar a los habitantes de Jerusalén, de cualquier condición, al igual como a las decenas de millares de visitantes de la Pascua, esta otra última oportunidad de oír el evangelio y de recibir, si así lo querían, al Hijo de la Paz.
\vs p172 4:2 Y, entonces, al caer la tarde y al ir la gente en busca de alimentos, Jesús y sus seguidores más cercanos se quedaron solos. ¡Aquel había sido un día extraño! Los apóstoles estaban pensativos, pero sin poder articular palabras. Nunca, en todos sus años en compañía de Jesús, habían vivido un día como este. Por un momento, se sentaron junto a la tesorería, mirando como el pueblo echaba sus aportaciones: los ricos echaban mucho en las arcas de las ofrendas, con forma de trompeta, y todos daban algo según sus posibilidades. Por último, vino una pobre viuda, pobremente ataviada, y vieron que echó dos blancas (dos monedas pequeñas de cobre) en una de las cajas del tesoro. Entonces, llamando la atención de los apóstoles sobre la viuda, Jesús les dijo: “Fijaos en lo que acabáis de ver. Esta viuda pobre echó más que todos los demás, porque todos, de lo que les sobraba, han echado una minucia como ofrenda, pero esta mujer pobre, aunque necesitada, echó todo lo que tenía, todo su sustento”.
\vs p172 4:3 Conforme anochecía, caminaron por los patios del templo en silencio y, después de que Jesús examinase aquellas escenas familiares, recordando sus sensaciones de anteriores visitas, sin exceptuar las primeras que realizó, dijo: “Vayamos a Betania para descansar”. Jesús, con Pedro y Juan, fueron a la casa de Simón, mientras que los otros apóstoles se alojaron en Betania y Betfagé con sus amigos.
\usection{5. LA ACTITUD DE LOS APÓSTOLES}
\vs p172 5:1 Aquel domingo por la noche, de regreso a Betania, Jesús caminó delante de los apóstoles. No dijeron ni una sola palabra hasta que se separaron al llegar a casa de Simón. Nunca doce seres humanos habían experimentado emociones tan distintas e inexplicables como las que afloraban a las mentes y a las almas de estos embajadores del reino. Estos robustos galileos estaban confusos y desconcertados; no sabían qué era lo que les esperaba; se encontraban tan extremadamente sorprendidos como para sentirse muy atemorizados. No sabían nada de los planes del Maestro para el día siguiente, y no hicieron ninguna pregunta. Se fueron a sus alojamientos, aunque no lograron dormir mucho, salvo los gemelos. Pero no mantuvieron a Jesús bajo vigilancia armada en casa de Simón.
\vs p172 5:2 Andrés estaba completamente abrumado, casi confundido. Era el único apóstol que no había reflexionado seriamente sobre aquel estallido de aclamación popular. Había estado extremadamente ocupado atendiendo sus deberes como jefe del cuerpo apostólico, y no pudo dar ninguna especial atención al significado o a la importancia de los sonoros hosannas de la multitud. Andrés se había dedicado a vigilar a algunos de sus compañeros por temor a que se dejaran llevar por sus emociones en aquellos momentos de agitación, en particular a Pedro, a Santiago, a Juan y a Simón Zelotes. Durante todo ese día, y los que siguieron después, Andrés se sintió inquieto y con serias dudas, pero no expresó ninguno de sus recelos a sus compañeros apostólicos. Le preocupaba la actitud de algunos de los doce, porque sabía que iban armados con espadas; pero ignoraba que su propio hermano, Pedro, también portaba una. Fue por ello por lo que la marcha de la comitiva hacia Jerusalén causó en Andrés una impresión relativamente superficial; estaba demasiado atareado cumpliendo con las responsabilidades de su cargo como para que pudiese ser de otra manera.
\vs p172 5:3 En un principio, Simón Pedro estaba desbordado de alegría ante aquella ante aquella entusiasta manifestación de la gente; pero al regresar aquella noche a Betania estaba mucho más serio. Sencillamente, no podía figurarse las intenciones del Maestro. Estaba terriblemente desilusionado de que Jesús no hubiera respondido a aquella oleada de fervor popular con algún tipo de pronunciamiento. No podía entender por qué Jesús no había hablado a la multitud cuando llegó al templo, o haber permitido al menos que alguno de los discípulos lo hiciera. Pedro era un gran predicador, y le disgustaba ver cómo se había desaprovechado una concurrencia tan grande, receptiva y entusiasta de personas. Ciertamente, le habría gustado predicar el evangelio del reino a aquella muchedumbre allí, en el templo; si bien, el Maestro les había encargado expresamente que no enseñaran ni predicaran en Jerusalén durante aquella semana de Pascua. Así pues, la reacción de Simón Pedro hacia aquella espectacular comitiva que se adentró en la ciudad fue, después de un primer momento, desoladora; por la noche, estaba apesadumbrado e inexplicablemente triste.
\vs p172 5:4 Para Santiago Zebedeo, aquel domingo había sido un día de perplejidad y gran confusión; no le veía sentido a lo que estaba ocurriendo; no podía comprender el propósito del Maestro permitiendo esa exaltada aclamación pública y luego negarse a hablarle a la gente al llegar al templo. Conforme la comitiva bajaba desde el Monte de los Olivos a Jerusalén, en particular cuando se encontraron con los miles de peregrinos que afluían para encontrarse con el Maestro, Santiago se había sentido fatalmente dividido por emociones contradictorias de entusiasmo y de satisfacción ante aquel escenario, al lado de un intenso sentimiento de temor por lo que podría suceder cuando accedieran al templo. Y, entonces, se sintió devastado y sobrepasado por la decepción cuando Jesús bajó del asno y se puso a caminar con calma por los patios del templo. Santiago no podía entender que se malgastara tal magnífica oportunidad para proclamar el reino. Por la noche, su mente se vio fuertemente sometida a una terrible y angustiosa incertidumbre.
\vs p172 5:5 De alguna manera, Juan Zebedeo casi logró entender la razón del proceder de Jesús; comprendía, al menos en parte, el significado espiritual de aquella supuesta entrada triunfal en Jerusalén. A medida que la multitud avanzaba hacia el templo, y que Juan veía a su Maestro sentado a lomo de un pollino, recordó haber oído a Jesús citar alguna vez un pasaje de la Escritura en el que Zacarías describía la venida del Mesías como hombre de paz, cabalgando hacia Jerusalén en un asno. Al reflexionar sobre este pasaje, Juan comenzó a darle sentido al simbolismo implícito en la procesión del domingo por la tarde. Captó, al menos, de manera suficiente, su significado como para permitirle disfrutar en cierto modo de aquel hecho, evitando, pues, sentirse excesivamente deprimido por el aparente despropósito de la comitiva triunfal. Juan tenía un tipo de mente propensa de forma natural a pensar y a sentir en símbolos.
\vs p172 5:6 \pc Felipe estaba absolutamente inquieto por el carácter repentino y espontáneo de aquella expresión de entusiasmo. Mientras bajaban del Monte de los Olivos no pudo ordenar sus pensamientos ni llegar a ninguna idea firme respecto al significado de tal marcha. De alguna manera, disfrutó de dicha manifestación pública porque su Maestro estaba siendo honesto. Cuando llegaron al templo, le turbaba pensar que Jesús quizás le pidiera que alimentara a la multitud, de modo que el proceder de Jesús de dar la espalda despreocupadamente a la multitud, algo que defraudó dolorosamente a la mayoría de los apóstoles, fue para Felipe un gran alivio. Las multitudes habían resultado ser una difícil prueba para el encargado de abastecimientos de los doce. Tras sentirse aliviado de estos temores personales en cuanto a las necesidades materiales de las multitudes, Felipe se unió a Pedro, manifestando su decepción al no haberse hecho nada por enseñar a la gente. Esa noche Felipe pensó sobre estos acontecimientos y estuvo tentado a dudar de toda la idea del reino; honestamente se preguntaba qué significado podría tener todo aquello, pero no comunicó sus dudas a nadie; amaba inmensamente a Jesús. Tenía una gran fe personal en el Maestro.
\vs p172 5:7 \pc Al margen de Juan, que apreció los aspectos simbólicos y proféticos del hecho, Natanael fue el que más próximo estuvo de entender la razón del Maestro por procurar el apoyo popular de los peregrinos asistentes a la Pascua. Llegó a la conclusión, antes de que llegaran al templo, de que sin tal entusiasta entrada en Jerusalén, los oficiales del sanedrín hubieran arrestado y encarcelado a Jesús en el momento en el que hubiera intentado entrar en la ciudad. Por consiguiente, no se extrañó en lo más mínimo cuando el Maestro dejó de valerse de las enardecidas multitudes una vez que franqueó los muros de la ciudad y de impresionar a los líderes judíos para evitar ser inmediatamente detenido. Al entender la verdadera razón de la entrada del Maestro en la ciudad de aquella manera, Natanael simplemente se comportó con más naturalidad y se sintió menos turbado y defraudado por el proceder de Jesús que los otros apóstoles. Natanael tenía una gran confianza en cómo Jesús entendía a los hombres al igual que en su sagacidad y astucia para resolver situaciones difíciles.
\vs p172 5:8 \pc En un principio, Mateo se quedó atónito ante aquella festiva manifestación pública. No captaba el sentido de lo que sus ojos percibían hasta que también recordó el pasaje de Zacarías, en el que el profeta aludía al júbilo de Jerusalén cuando su rey había venido trayendo salvación y cabalgando sobre un pollino de una asna. Conforme la comitiva se acercaba a la ciudad y, luego, se dirigía al templo, Mateo estaba exultante; estaba seguro de que algo extraordinario acontecería en el momento de que el Maestro llegara al templo a la cabeza de la clamorosa multitud. Cuando uno de los fariseos se burló de Jesús, diciendo: “¡Mirad, todos, ved quién viene aquí: el rey de los judíos cabalgando sobre un asno!”. Mateo se contuvo con dificultad para no ponerle las manos encima. A aquellas horas de la noche, en el camino de vuelta a Betania, ninguno de los doce se sintió más deprimido que él. Después de Simón Pedro y de Simón Zelotes, Mateo fue quien sufrió la mayor alteración nerviosa y se sintió extenuado por la noche. Si bien, por la mañana, Mateo estaba mucho más alegre; en el fondo, sabía perder.
\vs p172 5:9 \pc De los doce, Tomás fue quien más agitado y desorientado se sintió. La mayor parte del tiempo sencillamente se limitó a seguirlos, contemplando aquel espectáculo y preguntándose con toda honestidad cuál sería el motivo del Maestro para ser partícipe de aquella peculiar marcha. En lo más profundo de su corazón, percibía toda aquella exhibición popular como algo pueril, hasta como una completa necedad. Nunca había visto a Jesús hacer nada semejante y no podía encontrar ninguna explicación para su extraña conducta en aquel domingo por la tarde. Si bien, cuando llegaron al templo, Tomás había llegado a la conclusión de que el propósito de aquella peculiar manifestación de popularidad era atemorizar al sanedrín para que no se atreviese a arrestar al Maestro de inmediato. De regreso a Betania, Tomás reflexionó mucho sobre lo sucedido, pero no dijo nada. A la hora de acostarse, había comenzado a ver con cierto humor el ingenio demostrado por el Maestro para planificar la tumultuosa entrada en Jerusalén, y se sintió mucho mejor.
\vs p172 5:10 Ese domingo había comenzado siendo un gran día para Simón Zelotes. Imaginó que ocurrirían cosas portentosas en Jerusalén durante los próximos días y en eso llevaba razón, pero Simón soñaba con la instauración de un nuevo gobierno nacional de los judíos, con Jesús ocupando el trono de David. Simón visualizaba cómo los nacionalistas entrarían en acción en cuanto el reino se anunciara, y se veía a sí mismo ostentando el mando supremo de las fuerzas militares congregadas para el nuevo reino. Al bajar del Monte de los Olivos, incluso vislumbró que el sanedrín y todos sus adeptos morirían antes de que se pusiera el sol aquel día. Creía realmente que algo grande sucedería. Era la persona más ruidosa de toda la multitud. Hacia las cinco de esa tarde, no obstante, era un apóstol silencioso, devastado y desilusionado. Nunca llegaría a recuperarse del todo de la depresión que le sobrevino como resultado de la conmoción recibida aquel día; al menos no hasta bastante después de la resurrección de su Maestro.
\vs p172 5:11 Para los gemelos Alfeo, aquel había sido un día perfecto. Habían disfrutado verdaderamente de todo, de principio a fin, y al no estar presentes durante la tranquila visita al templo, prácticamente se libraron del declive en la agitación popular. No podían comprender de ninguna manera el abatimiento de los apóstoles cuando volvían a Betania aquella noche. En el recuerdo de los gemelos, aquel sería siempre el día en el que sintieron que el cielo se había acercado más a la tierra. Ese día fue el gratificante punto álgido de toda su andadura como apóstoles. Y la memoria del júbilo de ese domingo por la tarde los sostendría durante la tragedia de esa azarosa semana, justo hasta la hora de la crucifixión. Para los gemelos era la entrada más digna que se podía esperar para un rey; gozaron cada momento de la procesión. Aprobaban enteramente todo lo que habían visto y lo atesoraron en su memoria por mucho tiempo.
\vs p172 5:12 De todos los apóstoles, fue Judas Iscariote en quien repercutió de manera más negativa la entrada procesional en Jerusalén. Su mente bullía de irritación por la reprimenda del Maestro del día anterior con motivo de la unción de María durante la comida en la casa de Simón. A Judas le indignaba aquel espectáculo. Le parecía infantil, e incluso, de hecho, ridículo. Tal como este vengativo apóstol percibía los acontecimientos de aquel domingo por la tarde, Jesús le pareció más un payaso que un rey. Aquella escena lo enfureció por completo. Compartía la visión de los griegos y de los romanos, que menospreciaban a todo aquel que accediera a montarse en un asno o en el pollino de una asna. En el momento en el que la procesión triunfal entró en la ciudad, Judas había prácticamente tomado la decisión de renunciar a toda la idea de tal reino; estaba casi resuelto a abandonar todos aquellos absurdos intentos por instaurar el reino de los cielos. Y, entonces, pensó en la resurrección de Lázaro y en muchas otras cosas, y decidió permanecer con los doce, al menos por otro día. Además, llevaba la bolsa con los fondos apostólicos, y no quería desertar estando con estos en su poder. Esa noche, volviendo a Betania, su conducta no le resultó extraña a nadie puesto que todos los apóstoles estaban asimismo abatidos y silenciosos.
\vs p172 5:13 Judas estaba enormemente afectado por el ridículo sufrido ante sus amigos saduceos. No hubo ninguna otra circunstancia que ejerciera una influencia tan poderosa sobre él en su decisión definitiva de abandonar a Jesús y a sus apóstoles, como la que le produjo un episodio ocurrido justo cuando Jesús llegaba a la puerta de la ciudad: un prominente saduceo (amigo de la familia de Judas) se apresuró hacia él y en tono de sorna y burla, dándole una palmada en la espalda, le dijo: “¿Por qué tienes ese semblante de preocupación, mi buen amigo? Alégrate y únete a nosotros para aclamar a este Jesús de Nazaret, el rey de los judíos, que cruza cabalgando la puerta de Jerusalén sentado en un asno”. Judas no se había echado atrás ante las persecuciones, pero era incapaz de soportar aquel tipo de mofa. Al sentimiento de venganza, largamente gestado, se le sumó entonces un fatal temor a quedar en ridículo, un sentimiento terrible y tremendo de avergonzarse de su Maestro y de sus compañeros apóstoles. En su corazón, este embajador ordenado del reino ya era un desertor; tan solo le quedaba encontrar una justificación verosímil para romper de forma definitiva con el Maestro.
