\upaper{153}{La crisis de Cafarnaúm}
\author{Comisión de seres intermedios}
\vs p153 0:1 El viernes por la noche, el mismo día de su llegada a Betsaida, y el sábado por la mañana, los apóstoles observaron que a Jesús le preocupaba alguna cuestión que revestía gran gravedad; notaron que la atención del Maestro se centraba inusualmente en algún asunto de gran importancia. No tomó el desayuno; solo comió algo al mediodía. Durante todo el \bibemph{sabbat} por la mañana y la noche anterior, los doce y sus acompañantes estuvieron reunidos en pequeños grupos alrededor de la casa, en el jardín y en la orilla del mar. Entre ellos reinaba una tensa incertidumbre y una ansiosa inquietud. Desde que salieron de Jerusalén, Jesús apena les había dirigido la palabra.
\vs p153 0:2 Hacía meses que no habían visto al Maestro tan preocupado y tan poco comunicativo. Incluso Simón Pedro se encontraba deprimido, por no decir abatido. Andrés estaba perdido, sin saber qué hacer por sus desconsolados compañeros. Natanael comentó que estaban en medio de una calma que antecedía a la tormenta. Tomás era de la opinión de que algo fuera de lo ordinario estaba a punto de suceder. Felipe aconsejó a David Zebedeo que se olvidara de sus planes de alimentar y albergar a la multitud hasta que conociesen los pensamientos del Maestro. Mateo redoblaba sus esfuerzos para que las arcas estuvieran repletas. Santiago y Juan conversaban sobre el próximo sermón que se daría en la sinagoga, haciendo conjeturas respecto a su posible tema y alcance. Simón Zelotes creía, en realidad era más una esperanza, que el Padre en los cielos iba a intervenir, de alguna forma inesperada, para defender y apoyara su Hijo, mientras que Judas Iscariote hasta se atrevió a pensar que a Jesús le embargaba el remordimiento de no haber tenido el coraje y la audacia de permitir que los cinco mil lo proclamaran rey de los judíos.
\vs p153 0:3 Y, en medio de este grupo de seguidores deprimidos y desconsolados, Jesús se dirigió con determinación a aquella hermosa tarde de \bibemph{sabbat} para predicar en la sinagoga de Cafarnaúm un sermón que sería crucial. De todos sus más próximos seguidores, las únicas palabras animosas y de buenos deseos fueron las de uno de los confiados gemelos Alfeo, quien, al salir Jesús de la casa camino de la sinagoga, lo saludó de buen humor diciendo: “Oramos para que el Padre te ayude, y tengamos las mayores multitudes que jamás hayamos tenido”.
\usection{1. SE PREPARA EL MARCO DE LA ACCIÓN}
\vs p153 1:1 A las tres de la tarde de este espléndido \bibemph{sabbat,} una distinguida congregación recibió a Jesús en la nueva sinagoga de Cafarnaúm. Jairo presidía la celebración y entregó a Jesús las Escrituras para que leyese. El día antes, habían llegado de Jerusalén cincuenta y tres fariseos y saduceos; estaban también presentes más de treinta de los líderes y jefes de las sinagogas vecinas. Estos líderes religiosos judíos actuaban bajo las órdenes directas del sanedrín de Jerusalén; eran la avanzadilla ortodoxa que acudía para propugnar abiertamente la guerra contra Jesús y sus discípulos. Sentados al lado de dichos líderes judíos, en los asientos de honor de la sinagoga, estaban los observadores oficiales de Herodes Antipas, mandados por él para averiguar la verdad sobre los preocupantes rumores de que la multitud había intentado proclamar a Jesús como rey de los judíos, en los dominios de su hermano Felipe.
\vs p153 1:2 Jesús comprendía que se enfrentaba a una inminente declaración de guerra, jurada y manifiesta, por parte de sus enemigos, cada vez más numerosos, y decidió con arrojo afrontar la ofensiva. Al alimentar a los cinco mil, había puesto en cuestión sus ideas sobre el Mesías de un mundo material; ahora, de nuevo optó por atacar públicamente su concepto de lo que el libertador judío era. Esta crisis, que comenzó con la alimentación de los cinco mil y finalizó con este sermón de la tarde del \bibemph{sabbat,} señaló un giro patente en la oleada de fama y aclamación popular. En lo sucesivo, el trabajo del reino estaría crecientemente centrado en la más importante tarea de ganar conversos espirituales de carácter definitivo para la verdadera hermandad religiosa de la humanidad. Este sermón supone la transición de la crisis desde su período de debate, controversia y decisión hasta el de guerra expresa y la aceptación o rechazo final de Jesús.
\vs p153 1:3 El Maestro sabía muy bien que muchos de sus seguidores estaban lenta pero inexorablemente predisponiéndose a rechazarlo definitivamente. Era igualmente consciente de que muchos de sus discípulos estaban, de forma paulatina pero cierta, adquiriendo esa formación de mente y disciplina del alma que les permitirían triunfar sobre la duda y afirmar con valentía su plena fe en el evangelio del reino. Jesús era totalmente consciente de cómo se preparan los hombres para tomar decisiones ante una crisis y se determinan a realizar actos inmediatos y valerosos, optando de forma reiterada y gradual entre las situaciones de bien y mal, que repetidamente encuentran ante sí. Expuso a sus mensajeros elegidos a decepciones constantes y los puso en situaciones frecuentes de dificultad para que escogieran entre el modo, justo o equivocado, de hacer frente a las pruebas espirituales. Sabía que podía confiar en sus seguidores, cuando afrontaran el test final, que tomarían sus decisiones en la vida conforme a sus habituales actitudes mentales y reacciones espirituales.
\vs p153 1:4 \pc Esta crisis en la vida de Jesús en la tierra se inició cuando dio de comer a los cinco mil y tuvo su fin con este sermón en la sinagoga; la crisis en la vida de los apóstoles se inició con este sermón y continuó durante todo un año, no concluyendo hasta el juicio y crucifixión del Maestro.
\vs p153 1:5 \pc Aquella tarde, mientras estaban sentados allí en la sinagoga, antes de que Jesús tomara la palabra, en las mentes de todos no rondaba sino un gran misterio, una suprema pregunta. Amigos y enemigos compartían un solo pensamiento: “¿Por qué rechazó Jesús con tal resolución y contundencia la oleada de entusiasmo popular?”. Y fue justo antes y justo después de este sermón cuando las dudas y la decepción de sus contrariados seguidores se convirtieron en una oposición inconsciente, que acabaría por desembocar en un auténtico odio. Fue tras este sermón en la sinagoga cuando Judas Iscariote tuvo conscientemente, y por primera vez, la idea de desertar. Pero fue capaz en aquel momento de dominar tal deseo.
\vs p153 1:6 Todos estaban desconcertados. Jesús los había dejado estupefactos y confundidos. Últimamente había efectuado la más grande demostración de poder sobrenatural de su andadura. Y de toda su vida en la tierra, este hecho de alimentar a los cinco mil fue el que mayor apeló al concepto judío del Mesías esperado. Pero esta extraordinaria posición de favor quedó contrarrestada de inmediato e inexplicablemente al negarse tajantemente y sin dilación que se le nombrase rey.
\vs p153 1:7 El viernes por la noche, y de nuevo el sábado por la mañana, los líderes de Jerusalén, habían instado largamente a Jairo para que evitara que Jesús hablara en la sinagoga, pero fue en vano. La única respuesta que recibieron de Jairo a todos sus ruegos fue: “He dado mi consentimiento a esta petición, y no romperé mi palabra”.
\usection{2. EL TRASCENDENTAL SERMÓN}
\vs p153 2:1 Jesús inició este sermón leyendo un pasaje del Deuteronomio de la ley: “Pero acontecerá que, si el pueblo no oye atentamente la voz de Dios, de cierto sobrevendrá sobre ellos las maldiciones de la transgresión. El Señor te entregará derrotado delante de tus enemigos; serás el espanto de todos los reinos de la tierra. Y el Señor te llevará a ti y al rey que hayas puesto sobre ti a una nación extranjera. Serás motivo de horror, y servirás de refrán y de burla en todos los pueblos. Tus hijos e hijas irán en cautiverio. El extranjero que estará en medio de ti se elevará muy alto sobre ti, y tú descenderás muy abajo. Y todas estas cosas serán sobre ti y tu semilla para siempre por no haber querido oír la palabra del Señor. Servirás, por tanto, a tus enemigos que vendrán contra ti. Padecerás hambre y sed y llevarás este yugo extranjero de hierro. El Señor traerá contra ti una nación venida de lejos, de los confines de la tierra, una nación cuya lengua no entiendas, gente fiera de rostro, nación que tendrá poco respeto por ti. Pondrá sitio a todas tus ciudades hasta que caigan en tierra tus muros altos y fortificados en que tú confías; y toda la tierra caerá en sus manos. Y acontecerá que serás llevado a comer el fruto de tu vientre, la carne de tus hijos e hijas, en medio del sitio y en el apuro con que te angustiará tu enemigo”.
\vs p153 2:2 Y cuando Jesús terminó de hacer esta lectura, pasó a los Profetas, y leyó de Jeremías: “‘Si no atendéis a las palabras de mis siervos los profetas, que os he enviado, yo trataré a esta casa como Silo, y esta ciudad la pondré por maldición ante todas las naciones de la tierra’. Y los sacerdotes y los maestros oyeron a Jeremías hablar estas palabras en la casa del Señor. Y aconteció, que cuando Jeremías terminó de decir lo que el Señor le había ordenado que hablara a todo el pueblo, los sacerdotes y los maestros le echaron mano, diciendo: ‘De cierto morirás’. Y todo el pueblo se reunió contra Jeremías en la casa del Señor. Y cuando los príncipes de Judá oyeron estas cosas, se hicieron juicio contra él. Entonces hablaron los sacerdotes y maestros a los príncipes y a todo el pueblo, diciendo: ‘¡Este hombre ha incurrido en pena de muerte, porque ha profetizado contra nuestra ciudad, como vosotros habéis oído con vuestros propios oídos!’. Entonces habló Jeremías a los príncipes y a todo el pueblo: ‘El Señor me envió a profetizar contra esta casa y contra esta ciudad todas las palabras que habéis oído. Ahora, por tanto, mejorad vuestros caminos y vuestras obras, y escuchad la voz del Señor, vuestro Dios, para que podáis escapar del mal que se ha hablado contra vosotros. En lo que a mí toca, he aquí que estoy en vuestras manos. Haced conmigo lo que os parezca bueno y recto ante vuestros ojos. Pero sabed de cierto que, si me matáis, sangre inocente echaréis sobre vosotros y sobre este pueblo, porque en verdad el Señor me envió para que dijera todas estas palabras en vuestros oídos’.
\vs p153 2:3 “Los sacerdotes y maestros de ese día quisieron dar muerte a Jeremías, pero los jueces no accedieron, aunque, por sus palabras de advertencia, lo bajaron con sogas a una mazmorra inmunda, hundiéndolo en el barro hasta las axilas. Eso fue lo que esta gente le hizo al profeta Jeremías cuando él, obedeciendo el mandato del Señor, previno a sus hermanos sobre el inminente colapso de la soberanía nacional. Hoy quiero preguntaros: ¿Qué harán los sumos sacerdotes y los líderes religiosos de este pueblo con aquél que se atreve a advertirles sobre el día de su caída espiritual? ¿Trataréis de dar muerte al maestro que se atreve a proclamar la palabra del Señor, y que no teme mostraros la manera en la que rechazáis caminar en el camino de la luz que lleva hasta la entrada del reino de los cielos?
\vs p153 2:4 “¿Qué es lo que buscáis que os sirva como prueba de mi misión en la tierra? No os hemos perturbado en vuestras posiciones de influencia y poder. No hemos sido hostiles a aquello que reverenciáis, sino que hemos proclamado una nueva libertad para el alma atemorizada del hombre. He venido al mundo para revelar a mi Padre y para instaurar sobre la tierra la hermandad espiritual de los hijos de Dios, el reino de los cielos. Y pese a que os he recordado muchas veces que mi reino no es de este mundo, mi Padre, no obstante, os ha concedido numerosas manifestaciones de prodigios materiales junto a una más manifiesta transformación y regeneración espiritual.
\vs p153 2:5 “¿Qué nueva señal pedís de mis manos? Os declaro que ya tenéis pruebas suficientes para poder tomar vuestra decisión. En verdad, en verdad, os digo a muchos de los que estáis sentados ante mí este día, que tenéis el reto y la necesidad de elegir el camino que habréis de seguir; y yo os digo, tal como Josué le habló a vuestros ancestros: ‘escogeos hoy a quien sirváis’. Hoy, muchos de vosotros os encontráis ante una encrucijada.
\vs p153 2:6 “Algunos de entre vosotros, cuando no pudisteis encontrarme tras alimentar abundantemente a la multitud en la otra orilla, contratasteis para ir en mi búsqueda a la flota de pesca de Tiberias, que una semana antes había buscado refugio ahí cerca durante una tormenta, y ¿para qué? ¡No para buscar la verdad y la rectitud ni para poder aprender a servir y atender mejor a vuestros semejantes! No, sino, en su lugar, para tener más pan sin haberos afanado por él. No buscabais llenar vuestras almas con la palabra viva, sino saciar vuestras barrigas con un pan conseguido sin esfuerzo. Durante mucho tiempo, se os ha enseñado que el Mesías, cuando llegara, obraría esos portentos que harían la vida agradable y fácil a todo el pueblo elegido. No es de extrañar, pues, que al haber sido educados así, estéis tan deseosos de panes y peces. Pero yo os manifiesto que esa no es la misión del Hijo del Hombre. Yo he venido para proclamar la libertad espiritual, enseñar la verdad eterna y fomentar la fe viva.
\vs p153 2:7 “Hermanos míos, ansiad no la comida que perece, sino la comida espiritual que permanece para vida eterna; y este es el pan de vida el cual dará el Hijo a todos los que lo acepten y coman, porque el Padre ha dado esta vida a su Hijo sin medida. Y cuando vosotros me preguntasteis: ‘¿Qué debemos hacer para poner en práctica las obras de Dios?’, yo os dije claramente: ‘ésta es la obra de Dios, que creáis en aquel que él ha enviado’”.
\vs p153 2:8 Y entonces dijo Jesús, señalando a la imagen de la vasija de maná que decoraba el dintel de esta nueva sinagoga, y que estaba embellecida con racimos de uva, dijo: “Habéis creído que vuestros ancestros comieron maná ---el pan del cielo--- en el desierto, pero yo os digo que ese era el pan de la tierra. Moisés no dio a vuestros padres el pan del cielo, pero mi Padre os ofrece ahora el verdadero pan de vida, porque el pan del cielo es aquel que desciende de Dios y da vida eterna a los hombres del mundo. Y cuando vosotros me digáis, danos este pan vivo, yo os contestaré: Yo soy el pan de vida. El que a mí viene nunca tendrá hambre, y el que cree en mí jamás tendrá sed. Me habéis visto, habéis vivido conmigo, habéis contemplado mis obras, y sin embargo no creéis que haya venido del Padre. Pero a los que sí lo creen, no temáis. Todo lo que el Padre me da, vendrá a mí, y al que a mí viene, no lo echo fuera.
\vs p153 2:9 “Y ahora os haré saber, de una vez y para siempre, que he bajado a la tierra, no para hacer mi propia voluntad, sino la voluntad de Aquel que me envió. Y la absoluta voluntad del que me envió es que yo no pierda nada de todo lo que él me da. Y la voluntad del Padre es que todo el que ve al Hijo y cree en él tenga vida eterna. No hace mucho alimenté vuestros cuerpos con pan; hoy os ofrezco el pan de vida para vuestras almas hambrientas. ¿Tomaréis ahora el pan del espíritu de tan buena gana como comisteis entonces el pan de este mundo?”.
\vs p153 2:10 \pc Al detenerse Jesús un momento para mirar a la congregación, uno de los maestros de Jerusalén (miembro del sanedrín) se levantó y preguntó: “¿Debo entender de lo que dices que eres el pan que ha descendido del cielo, y que el maná que Moisés le dio a nuestros padres en el desierto no lo era?”. Y Jesús respondió al fariseo: “Lo has entendido bien”. Entonces dijo el fariseo: “Pero, ¿no eres tú Jesús de Nazaret, el hijo de José, el carpintero? ¿Es que no son tu padre y tu madre, como también tus hermanos y hermanas, bien conocidos para muchos de nosotros? ¿Cómo es que te presentas aquí en la casa de Dios afirmando que has bajado del cielo?”.
\vs p153 2:11 En aquel momento se produjo un gran murmullo, y se avecinaba tal alboroto que Jesús se puso de pie y dijo: “Seamos pacientes; la verdad no se resiente ante un examen honesto. Yo soy todo lo que tú dices pero incluso más. El Padre y yo somos uno; el Hijo hace solo lo que el Padre le enseña, y todos los que el Padre da al Hijo, el Hijo los recibirá para sí. Habéis leído lo que está escrito en los Profetas, ‘todos serán enseñados por Dios’, y que ‘aquellos a los que el Padre enseña oirán también a su Hijo’. Todo el que se rinde a las enseñanzas del espíritu morador del Padre acabará por venir a mí. Ningún hombre ha visto al Padre, pero el espíritu del Padre vive en el hombre. Y el Hijo que descendió del cielo de cierto ha visto al Padre. Y quienes verdaderamente creen en este Hijo, ya tienen vida eterna.
\vs p153 2:12 “Yo soy el pan de vida. Vuestros padres comieron el maná en el desierto, y aun así murieron. Pero este pan viene de Dios, si alguien lo come, nunca morirá en espíritu. Repito: Yo soy este pan vivo, y cualquier alma que logre un verdadero entendimiento de que Jesús es Dios en la carne vivirá para siempre. Y este pan de vida que daré a todos los que lo reciban es mi propia vida en unión al Padre. El Padre en el Hijo y el Hijo uno con el Padre: esa es mi revelación que da vida al mundo y mi regalo de salvación a todas las naciones”.
\vs p153 2:13 Cuando Jesús acabó de hablar, el jefe de la sinagoga despidió a la congregación, pero nadie se marchó. Se agolparon alrededor de Jesús para hacerle más preguntas a la vez que otros murmuraban y discutían entre sí. Y tal estado de cosas continuó durante más de tres horas. Eran más de las siete de la tarde cuando los asistentes acabaron por dispersarse.
\usection{3. TRAS LOS OFICIOS}
\vs p153 3:1 Tras los oficios, se le hicieron a Jesús un gran número de preguntas. Algunas fueron sus mismos perplejos discípulos quienes se las formularon, pero la mayoría las hicieron unos incrédulos alborotadores que solo buscaban humillar y tender una trampa a Jesús.
\vs p153 3:2 Uno de los fariseos que habían acudido como visitante, montándose en un candelabro, vociferó esta pregunta: “Nos dices que eres el pan de vida. ¿Cómo es que puedes darnos tu carne para comer y tu sangre para beber? ¿De qué sirven tus enseñanzas si no pueden llevarse a cabo?”. Y Jesús respondió diciendo: “Yo no os enseñé que mi carne es el pan ni que mi sangre es el agua de la vida. Pero de cierto os dije que mi vida de gracia en la carne es el pan del cielo. El hecho de que el Verbo de Dios se hiciera carne y la circunstancia de que el Hijo del Hombre se sometiese a la voluntad de Dios constituyen una experiencia real que equivale al sustento divino. No podéis comer mi carne ni beber mi sangre, pero podéis en espíritu convertiros en uno conmigo, tal como yo soy uno en espíritu con el Padre. Podéis alimentaros de la palabra eterna de Dios, que es realmente el pan de vida, y que se ha otorgado semejando un hombre mortal; vuestra alma puede regarse con el espíritu divino, que es en verdad el agua de la vida. El Padre me ha enviado al mundo para mostraros la manera en la que desea habitar en todos los hombres y guiarlos; y he vivido así esta vida en la carne para inspirar a todos los hombres a que, siempre y de igual manera, procuren conocer y hacer la voluntad del Padre celestial que habita en ellos”.
\vs p153 3:3 Entonces uno de los espías de Jerusalén que había estado observando a Jesús y a sus apóstoles, dijo: “Nos hemos fijado que ni tú ni tus apóstoles os laváis las manos convenientemente antes de comer pan. Debéis saber bien que comer con las manos impuras y sin lavar transgrede la ley de los ancianos. Tampoco laváis correctamente vuestros vasos de beber y vuestros jarros de comer. ¿Por qué os mostráis tan irreverentes con las tradiciones de los padres y las leyes de nuestros ancianos?”. Y cuando Jesús lo oyó hablar, respondió: “¿Cómo es que transgredís los mandamientos de Dios con las leyes de la tradición? El mandamiento dice, ‘honra a tu padre y a tu madre’, y os ordena que compartáis con ellos vuestros bienes de ser necesario; pero vosotros propugnáis una ley que permite que hijos irrespetuosos digan que el dinero con el que pudiera haber ayudado a sus padres es una ‘ofrenda a Dios’. De esta manera, la ley de los ancianos libera a estos taimados hijos de sus responsabilidades, pese a que usarán luego este dinero para su propio contento. ¿Por qué invalidáis así el mandamiento con vuestra tradición? Hipócritas, bien profetizó de vosotros Isaías, cuando dijo: ‘Este pueblo de labios me honra, mas su corazón está lejos de mí, pues en vano me honran, enseñando como doctrinas mandamientos de hombres’.
\vs p153 3:4 “¿Veis cómo ignoráis ese mandamiento mientras os aferráis a las tradiciones de los hombres? Estáis completamente dispuestos a rechazar la palabra de Dios, mientras conserváis vuestras propias tradiciones. Y son otras muchas las maneras en las que os atrevéis a colocar vuestras propias enseñanzas por encima de la ley y de los profetas”.
\vs p153 3:5 Entonces Jesús se dirigió a todos los presentes, diciendo: “Oídme, todos vosotros. Nada hay fuera del hombre que entre en él, que lo pueda contaminar espiritualmente, sino lo que sale de la boca y de su corazón, esto contamina al hombre”. Pero ni siquiera los apóstoles llegaron comprender del todo sus palabras, porque Simón Pedro también le preguntó: “Para que algunos de los asistentes no se sientan ofendidos sin necesidad, ¿puedes explicarnos qué quieres decir?”. Entonces le dijo Jesús a Pedro: “¿También tú estás falto de entendimiento? ¿No sabes que toda planta que no plantó mi Padre celestial será desarraigada. Centrad vuestra atención a aquellos que desean saber la verdad. No puedes forzar a los hombres a que amen la verdad. Muchos de estos maestros son ciegos guías de ciegos; y sabes que si un ciego guía a otro ciego, ambos caerán en el hoyo. Pero, óyeme cuando te digo la verdad sobre las cosas que corrompen moralmente al hombre y lo contaminan. Te digo que no es lo que entra en el cuerpo por la boca o penetra en la mente por los ojos y los oídos, lo que contamina al hombre. El hombre solo se contamina por el mal que engendra en su corazón, y que se expresa en las palabras y en los actos de estas impías personas. ¿Es que no sabéis que del corazón del hombre salen los malos pensamientos, los malévolos planes de asesinato, los hurtos y los adulterios, junto con los celos, la soberbia, la ira, la venganza, las blasfemias y los falsos testimonios? Y estas son justo las cosas que contaminan al hombre, pero el comer con las manos sin lavar, contraviniendo las prácticas ceremoniales, no contamina al hombre”.
\vs p153 3:6 Ya los fariseos, delegados del sanedrín de Jerusalén, estaban prácticamente convencidos de que había que detener a Jesús con el cargo de blasfemia o desprecio de la ley sagrada de los judíos; de ahí, sus intentos de enredarlo en comentarios, y posibles ataques, contra algunas de las tradiciones de los ancianos, o las llamadas tradiciones orales del país. Sin importarles la escasez de agua que hubiese, estos judíos, esclavos de las tradiciones, jamás dejaban de cumplir el exigido lavado ceremonial de las manos antes de cada comida. Creían que “es mejor morir, que quebrantar los mandamientos de los ancianos”. Los espías le hicieron esta pregunta porque se habían informado de que Jesús había dicho “la salvación es cuestión de corazón limpio más que de manos limpias”. Pero estas creencias, una vez que se convierten en parte de la religión de cada cual, resultan difíciles de erradicar. Incluso muchos años después de este día, el apóstol Pedro seguía sintiendo ese temor esclavo hacia muchas de estas tradiciones sobre las cosas limpias y sucias, pero acabó por liberarse de él tras tener un sueño vívido y extraordinario. Se puede entender mejor todo si se recuerda que estos judíos consideraban comer sin lavarse las manos bajo la misma luz que intimar con una prostituta, y ambos actos se castigaban igualmente con la excomunión.
\vs p153 3:7 En consecuencia, el Maestro optó por comentar y dar a conocer la necedad de todo el sistema rabínico de normas y reglas, que eran parte de la ley oral ---las tradiciones de los ancianos, las cuales, para los judíos, eran más sagradas y de obligado cumplimiento que incluso las enseñanzas de las Escrituras---. Y Jesús habló más abiertamente, porque sabía que había llegado la hora en la que no podía hacer nada más por evitar el claro rompimiento de relaciones con estos líderes religiosos.
\usection{4. ÚLTIMAS PALABRAS EN LA SINAGOGA}
\vs p153 4:1 Tras los servicios, en medio de las consultas que le hacían a Jesús, uno de los fariseos de Jerusalén trajo hasta él a un joven trastornado que estaba poseído por un espíritu rebelde e ingobernable. Una vez que estuvo este muchacho demente en su presencia, el fariseo le dijo: “¿Qué puedes hacer ante una aflicción así? ¿Eres capaz de echar fuera a los diablos?”. Y cuando el Maestro miró al joven, se sintió movido por la compasión y, haciéndole señas para que se acercara, lo tomó de la mano y le dijo: “Tú sabes quién soy yo; sal de él; ¡y mando a uno de tus compañeros leales que se asegure de que no vuelvas!”. De inmediato, el joven se sintió restablecido y en su sano juicio. Este fue el primer caso en el que Jesús realmente expulsaba a un “espíritu maligno” de un ser humano. En ninguno de los casos anteriores, se había producido esa supuesta posesión demoniaca; este sí era un caso auténtico, algo que sucedía ocasionalmente en aquellos días y que duraría hasta el día de Pentecostés, momento en el que el espíritu del Maestro se derramó sobre toda carne, imposibilitando para siempre que estos pocos rebeldes celestiales se aprovecharan de ciertas clases de seres humanos con desequilibrios.
\vs p153 4:2 Al ver que la gente se quedó maravillada, uno de los fariseos se puso de pie y acusó a Jesús de que podía hacer estas cosas porque estaba coaligado con los diablos; que, por las palabras que había empleado para echar fuera a este diablo, él mismo admitía que se conocían entre sí; y agregó que los maestros y líderes religiosos de Jerusalén habían llegado a la conclusión de que Jesús realizaba todos sus supuestos milagros por el poder de Beelzebú, el príncipe de los diablos. El fariseo dijo: “No tengáis nada que ver con este hombre; está en alianza con Satanás”.
\vs p153 4:3 Entonces dijo Jesús: “¿Cómo puede Satanás echar fuera a Satanás? Un reino dividido contra sí mismo no puede permanecer; si una casa está dividida contra sí misma, pronto es asolada. ¿Puede una ciudad resistir el asedio si no está unida? Si Satanás echa fuera a Satanás, contra sí mismo está dividido; ¿Cómo entonces permanecerá su reino? Pues, cómo puede alguno entrar en la casa del hombre fuerte y saquear sus bienes, si primero no lo domina y ata. Y, así pues, si decís que yo echo fuera a los demonios por el poder de Beelzebú, ¿por quién los echan vuestros propios discípulos? Por tanto, ellos mismos serán vuestros jueces. Pero si yo, por el espíritu de Dios, echo fuera a los diablos, ciertamente ha llegado a vosotros el reino de Dios. Si no estuvieseis cegados por los prejuicios e inducidos al error por el temor y la soberbia, percibiríais fácilmente que en medio de vosotros hay alguien que es más grande que los diablos. Me obligáis a deciros claramente que el que no está conmigo, está contra mí, que el que conmigo no recoge, desparrama. ¡Os daré esta grave advertencia a vosotros que presumís, con los ojos abiertos y con premeditada maldad, de atribuir conscientemente las obras de Dios a las actos de los diablos! En verdad, en verdad, os digo, que todos vuestros pecados serán perdonados, incluso todas vuestras blasfemias, pero cualquiera que blasfeme contra Dios con deliberación e intenciones perversas no será perdonado. Puesto que esos que persistentemente se afanan por la iniquidad nunca buscarán ni recibirán perdón, son culpables del pecado de rechazar eternamente el perdón divino.
\vs p153 4:4 “Muchos de vosotros estáis hoy en una encrucijada; llegáis al comienzo de una inevitable toma de decisión entre la voluntad del Padre y los caminos libremente elegidos de la oscuridad. Y de la manera en la que ahora elegís, así llegaréis a ser. Debéis hacer bueno el árbol y bueno su fruto o, de no ser así, el árbol estará podrido y podrido su fruto. Os hago saber que en el reino eterno de mi Padre, se conoce el árbol por sus frutos. Pero algunos de vosotros sois como víboras; habiendo elegido el mal ¿cómo podéis dar buenos frutos? Porque de la abundancia de mal que albergáis en vuestros corazones hablan vuestras bocas”.
\vs p153 4:5 Entonces, se puso de pie otro fariseo, y dijo: “Maestro, deseamos ver de ti alguna señal que todos la prefijáramos para determinar si posees autoridad y derecho a enseñar. ¿Estarías conforme?”. Y cuando Jesús oyó esto, dijo: “Esta generación descreída y obcecada en las señales demanda una prueba, pero no se os dará ninguna otra señal de que la ya tenéis, y la que veréis cuando el Hijo del Hombre se vaya de vosotros”.
\vs p153 4:6 Y cuando acabó de hablar, sus apóstoles lo rodearon y lo llevaron fuera de la sinagoga. En silencio, se encaminaron con él a su casa de Betsaida. Todos estaban asombrados y algo temerosos por el repentino cambio en la táctica de enseñanza del Maestro. En absoluto estaban acostumbrados a verlo proceder de manera tan combativa.
\usection{5. EL SÁBADO POR LA NOCHE}
\vs p153 5:1 Una y otra vez, Jesús había roto las esperanzas de sus apóstoles; en reiteradas ocasiones, él había hecho que sus más preciadas expectativas llegaran a desvanecerse, pero ningún otro momento de decepción o sufrimiento había sido nunca comparable con la que les había sobrevenido ahora. Además, esta vez, se añadía a su abatimiento el temor real a su propia seguridad. Estaban todos alarmantemente sorprendidos por la súbita y total deserción de la gente. También estaban algo temerosos y desconcertados por el inesperado denuedo y la firme determinación demostrada por los fariseos llegados de Jerusalén. Pero, sobre todo, estaban perplejos ante ese repentino cambio de táctica de Jesús. En circunstancias normales, habrían acogido con satisfacción esta actitud más combativa, pero habiéndose producido de aquel modo, junto a otras tantas cosas inesperadas, los sobresaltó.
\vs p153 5:2 Y ahora, por si fuera poco, cuando llegaron a casa, Jesús se negó a comer. Se aisló durante horas en una de las habitaciones de la planta superior. Era casi la medianoche cuando Joab, el líder de los evangelistas, regresó y dio la noticia de que un tercio de sus compañeros había abandonado la causa. A lo largo de la noche, los discípulos leales ya habían ido y venido informando de que el cambio de actitud hacia el Maestro en Cafarnaúm era generalizado. Los líderes de Jerusalén no tardaron en alentar este sentimiento de descontento de todas las formas e incitar colectivamente al alejamiento de Jesús y de sus enseñanzas. Durante estas horas difíciles, las doce mujeres se habían congregado en la casa de Pedro. Estaban muy afectadas, pero ninguna de ellas desertó.
\vs p153 5:3 Poco después de la medianoche, Jesús bajó de la habitación y, de pie, entre los doce y sus acompañantes, unos treinta en total, dijo: “Entiendo que la criba que se ha dado en el reino os angustie, pero es inevitable. Si bien, tras toda la formación que habéis recibido, ¿había alguna buena razón para que mis palabras os causaran tropiezo? ¿Por qué estáis tan amedrantados y consternados cuando veis que el reino se está despojando de multitudes tibias y de discípulos poco convencidos? ¿Por qué os afligís cuando amanece un nuevo día que resplandecerá en la nueva gloria de las enseñanzas espirituales del reino de los cielos? Si os resulta arduo sobrellevar esta prueba, ¿qué haréis, entonces, cuando el Hijo del Hombre tenga que regresar al Padre? ¿Cuándo y cómo estaréis preparados para el momento en que yo ascienda al lugar desde el que vine a este mundo?
\vs p153 5:4 “Mis bien amados, debéis recordar que es el espíritu el que vivifica; que la carne y todo lo que le atañe es de poco provecho. Las palabras que yo os he hablado son espíritu y vida. ¡Tened buen ánimo! No os he abandonado. Muchos se sentirán ofendidos por la claridad con la que me he expresado estos días. Ya habéis oído que muchos de mis discípulos me han dado la espalda; ya no caminan conmigo. Desde el comienzo, yo ya sabía que estos creyentes tan poco entusiastas se quedarían por el camino. ¿Es que no os elegí a vosotros doce y os aparté como embajadores del reino? Y, ahora, en momentos como estos, ¿queréis desertar vosotros también? Que cada uno de vosotros mire hacía si, hacia su propia fe, porque uno de vosotros se encuentra en grave peligro”. Y cuando Jesús acabó de hablar, Simón Pedro dijo: “Sí, Señor, estamos tristes y desconcertados, pero jamás te abandonaremos. Tú nos has enseñado las palabras de la vida eterna. Hemos creído en ti y te hemos seguido todo este tiempo. No nos volveremos atrás, porque sabemos que Dios te ha enviado”. Y cuando Pedro calló, todos ellos, de común acuerdo, asintieron en señal de aprobación.
\vs p153 5:5 Entonces, Jesús dijo: “Id a descansar, porque se avecinan momentos azarosos; se acercan días de agitación”.
