\upaper{88}{Fetiches, amuletos y magia}
\author{Brillante estrella vespertina}
\vs p088 0:1 La idea de un espíritu que entra en un objeto inanimado, en un animal o en un ser humano es una creencia muy antigua y respetable, que ha predominado desde el comienzo de la evolución de la religión. Esta doctrina de la posesión de los espíritus no es más ni menos que \bibemph{fetichismo}. El salvaje no adora necesariamente al fetiche, sino que rinde culto y venera con mucha lógica al espíritu que reside en él.
\vs p088 0:2 Al principio, se creía que el espíritu de un fetiche era el espectro de un fallecido; más tarde, se pensó que en los fetiches habitaban espíritus superiores. Y, así, con el tiempo, el sistema de culto basado en los fetiches llegó a incorporar todas las ideas primitivas de los espectros, las almas, los espíritus y la posesión demoníaca.
\usection{1. CREENCIA EN LOS FETICHES}
\vs p088 1:1 El hombre primitivo quería siempre hacer de cualquier cosa extraordinaria un fetiche; el azar dio en consecuencia origen a muchos fetiches. Un hombre está enfermo, ocurre algo y se pone bien. Lo mismo es cierto en cuanto a la notoriedad de muchos medicamentos y métodos fortuitos de tratar las enfermedades. Era probable que los objetos presentes en los sueños se convirtiesen en fetiches. Los volcanes, pero no las montañas, se volvieron fetiches; al igual que los cometas, aunque no las estrellas. Para el hombre primitivo, las estrellas fugaces y los meteoros indicaban la llegada a la tierra de espíritus visitantes de carácter especial.
\vs p088 1:2 Los primeros fetiches fueron guijarros con marcas extrañas y, desde entonces, el hombre ha buscado “piedras sagradas”; un collar de cuentas fue en algún momento un conjunto de piedras sagradas, una serie de amuletos. Muchas tribus poseían piedras fetiches, pero pocas han sobrevivido como lo han hecho la Kaaba y la Piedra de Scone. El fuego y el agua también estaban entre los primeros fetiches, y la adoración del fuego, junto con la creencia en el agua bendita, aún persiste.
\vs p088 1:3 Los árboles fetiches se desarrollaron con posterioridad, pero, entre algunas tribus, la continuidad de la adoración a la naturaleza llevó a la creencia en amuletos poseídos por algún tipo de espíritu de la naturaleza. Cuando las plantas y las frutas se convertían en fetiches, era tabú comerlas. La manzana fue entre las primeras en pertenecer en esta categoría; los pueblos levantinos no la comían jamás.
\vs p088 1:4 Si un animal comía carne humana, se volvía un fetiche. De esta manera, el perro llegó a ser un animal sagrado para los parsis. Si el fetiche es un animal y el espectro reside permanentemente en él, el fetichismo podía rozar la reencarnación. En cierto modo, los salvajes envidiaban a los animales; no se consideraban superiores a ellos y, a menudo, tomaban el nombre de sus fieras favoritas.
\vs p088 1:5 Cuando los animales se convirtieron en fetiches, se suscitaron los tabúes sobre el consumo de la carne del animal fetiche. Los simios y los monos, a causa de su parecido con el hombre, se volvieron pronto fetiches; más tarde, las serpientes, las aves y los cerdos tuvieron el mismo tratamiento. Durante algún tiempo, la vaca era un fetiche y su leche tabú, mientras sus excrementos eran muy apreciados. La serpiente se veneraba en Palestina, en especial por los fenicios que, junto con los judíos, la consideraban el vocero de los espíritus malignos. Hay todavía muchas personas modernas que creen en los poderes de encantamiento de los reptiles. Se ha reverenciado a la serpiente desde Arabia, pasando por la India, hasta la tribu moqui de los hombres rojos, con su danza de la serpiente.
\vs p088 1:6 Determinados días de la semana eran fetiches. Durante mucho tiempo, el viernes se ha considerado un día que traía mala suerte y el trece como un número maléfico. Los números de la buena suerte, el tres y el siete, provinieron de revelaciones más tardías; el cuatro era el número de la suerte del hombre primitivo y se derivó de la temprana identificación de los cuatro puntos cardinales. Se sostenía que era mala suerte contar el ganado u otras posesiones; los antiguos siempre se opusieron a hacer un censo, a “contar al pueblo”.
\vs p088 1:7 El hombre primitivo no hizo del sexo un innecesario fetiche; la función reproductora recibía solamente una atención limitada. El salvaje era de mentalidad natural, ni obscena ni lasciva.
\vs p088 1:8 La saliva era un fetiche poderoso; se podía expulsar a los diablos escupiendo sobre la persona. Se percibía como el más alto elogio que un anciano o una persona de rango superior escupiera sobre alguien. Algunas partes del cuerpo humano se veían como fetiches potenciales, en particular el pelo y las uñas. Las uñas largas de los jefes se tenían en gran estima, y sus cortaduras un poderoso fetiche. La creencia en las calaveras fetiches explica, en gran parte, la actividad más tardía de los cazadores de cabezas. El cordón umbilical era un fetiche muy apreciado; incluso se le considera así hoy en África. El primer juguete de la humanidad fue un cordón umbilical preservado; montado en perlas, como se hacía a menudo, constituyó el primer collar del hombre.
\vs p088 1:9 Los niños jorobados y tullidos se consideraban fetiches; se creía que los lunáticos estaban afectados por la luna. El hombre primitivo no sabía distinguir entre genio y locura; a las personas con retraso mental o se les golpeaba hasta darles muerte o se les veneraba como personas fetiches. La histeria confirmó, cada vez más, la creencia popular en la brujería; los epilépticos eran a menudo sacerdotes y curanderos. La embriaguez se veía como una forma de posesión de los espíritus. Cuando un salvaje estaba ebrio, se colocaba una hoja en el pelo con el objeto de no asumir responsabilidades por sus actos. Los venenos y las sustancias intoxicantes se convirtieron en fetiches; se pensaba que estaban poseídos.
\vs p088 1:10 Mucha gente creía que los genios eran personas fetiches que estaban poseídas por un espíritu sabio. Y estos seres humanos de talento aprendieron pronto a recurrir al fraude y al engaño para promocionar sus propios intereses egoístas. Se pensaba que un hombre fetiche era más que humano; era divino, e incluso infalible. De este modo, los jefes, los reyes, los sacerdotes, los profetas y los líderes eclesiásticos lograron, con el tiempo, un gran poder y ejercieron una autoridad ilimitada.
\usection{2. EVOLUCIÓN DEL FETICHE}
\vs p088 2:1 Se suponía que los espectros preferían habitar en un objeto que les hubiese pertenecido cuando vivían en la carne. Esta creencia explica la efectividad que se piensa tienen muchas reliquias modernas. Los antiguos siempre veneraban los huesos de sus líderes, y los restos óseos de los santos y de los héroes aún se consideran con un respeto supersticioso. Incluso hoy en día se hacen peregrinajes a las tumbas de grandes hombres.
\vs p088 2:2 La creencia en las reliquias es una extensión del antiguo sistema de culto de los fetiches. Las reliquias de las religiones modernas suponen un intento por racionalizar el fetiche del salvaje y elevarlos de esta manera a una posición de dignidad y respetabilidad en los modernos sistemas religiosos. Es paganismo creer en los fetiches y en la magia, pero presuntamente correcto aceptar las reliquias y los milagros.
\vs p088 2:3 El hogar ---la chimenea--- se convirtió más o menos en un fetiche, en un lugar sagrado. Los santuarios y los templos fueron al principio lugares fetiches porque los muertos estaban enterrados allí. Moisés elevó la choza fetiche de los hebreos al lugar donde se acogía a un fetiche excelso, el concepto por entonces existente de la ley de Dios. Pero los israelíes no renunciaron a la singular creencia cananea en la piedra del altar: “Y esta piedra que he puesto por columna, será casa de Dios”. Creían firmemente que el espíritu de su Dios habitaba en estos altares de piedra, que eran en realidad fetiches.
\vs p088 2:4 \pc Las primeras imágenes se hicieron para preservar la apariencia y la memoria de los muertos ilustres; eran, en realidad, monumentos. Los ídolos fueron un perfeccionamiento del fetichismo. Los primitivos creían que una ceremonia de consagración hacía que el espíritu entrara en la imagen; asimismo, cuando se bendecían ciertos objetos, estos se volvían amuletos.
\vs p088 2:5 Moisés, al añadir el segundo mandamiento al antiguo código moral dalamatiano, procuró controlar la adoración de los fetiches entre los hebreos. Con firmeza ordenó que no se hiciera ninguna clase de imágenes que se consagrara como fetiche. Lo expresó claramente: “No te harás imagen ni ninguna semejanza de lo que esté arriba en el cielo, ni abajo en la tierra, ni en las aguas debajo de la tierra”. Aunque este mandamiento retrasó mucho el arte entre los judíos, ciertamente disminuyó la adoración a los fetiches. Pero Moisés era demasiado sabio como para intentar desplazar de repente a los antiguos fetiches, así pues, accedió a colocar ciertas reliquias al lado de la ley, en el arca, que era una mezcla de altar de guerra y santuario religioso.
\vs p088 2:6 \pc Las palabras acabaron convirtiéndose en fetiches, más particularmente aquellas que se consideraban palabras de Dios; de este modo, los libros sagrados de muchas religiones se han vuelto prisiones fetichistas que encarcelan la imaginación espiritual del hombre. El esfuerzo mismo de Moisés por contrarrestar los fetiches se convirtió en un supremo fetiche; se utilizó su mandamiento más adelante para paralizar el arte y frenar el gozo y la adoración de lo bello.
\vs p088 2:7 En los tiempos antiguos, la palabra fetiche de autoridad era una \bibemph{doctrina} que inspiraba temor, el más terrible de todos los tiranos que esclavizan al hombre. Un fetiche doctrinal lleva al hombre mortal a traicionarse a sí mismo hasta caer en las garras del prejuicio, el fanatismo, la superstición, la intolerancia y en la más atroz y salvaje de las crueldades. El respeto moderno por la sabiduría y la verdad no es sino una forma reciente de huir de la propensión a crear fetiches y llegar a los más elevados niveles de pensamiento y razonamiento. En cuanto al cúmulo de los escritos fetiches que muchas personas religiosas suponen \bibemph{libros sagrados,} no solo se llega a creer que es verdad lo que está escrito, sino también que estos contienen toda la verdad. Si sucede que uno de estos libros habla de que la tierra es plana, entonces, durante largas generaciones, los hombres y mujeres, a pesar de su cordura, se negarán a aceptar las pruebas positivas de que el planeta es redondo.
\vs p088 2:8 La práctica de abrir uno de estos libros sagrados, leer al azar un pasaje y seguirlo para poder tomar importantes decisiones o hacer proyectos de vida no es más ni menos que un notorio fetichismo. Prestar juramento sobre “un libro sagrado” o jurar por algún objeto venerado supremamente es una forma de fetichismo refinado.
\vs p088 2:9 Pero, en verdad, hay que considerar como progreso evolutivo el avance desde el temor fetichista de las cortaduras de las uñas del jefe salvaje a la adoración de una espléndida colección de cartas, leyes, leyendas, alegorías, mitos, poemas y crónicas que, al fin y al cabo, reflejan la selección de la sabiduría moral de muchos siglos, al menos hasta el momento y acontecimiento en el que se reúnen como “libro sagrado”.
\vs p088 2:10 Para volverse fetiches, las palabras deben considerarse inspiradas, y la invocación de los escritos supuestamente inspirados de forma divina llevó directamente al establecimiento de la \bibemph{autoridad} de la Iglesia, al igual que la evolución de los procedimientos civiles llevó a la realización de la \bibemph{autoridad} del Estado.
\usection{3. EL TOTEMISMO}
\vs p088 3:1 El fetichismo se practicó en todos los sistemas de culto primitivos desde la creencia más temprana en las piedras sagradas, pasando por la idolatría, el canibalismo y la adoración de la naturaleza, hasta el totemismo.
\vs p088 3:2 El totemismo es una combinación de prácticas sociales y religiosas. Inicialmente, se pensaba que el respeto por el animal totémico de un supuesto origen biológico aseguraba el abastecimiento de alimentos. Los tótems eran al mismo tiempo símbolos del grupo y de su dios. Dicho dios era el clan personificado. El totemismo significó una etapa dentro del intento por socializar una religión, que era antes individual. Con el tiempo, el tótem evolucionó hasta convertirse en la bandera, o símbolo nacional, de los distintos pueblos modernos.
\vs p088 3:3 Una bolsa fetiche, o bolsa de medicina, era un saquito que contenía una reputada variedad de objetos impregnados por los espectros, y el curandero de la antigüedad jamás permitía que dicha bolsa, símbolo de su poder, tocara el suelo. Los pueblos civilizados del siglo XX se aseguran, igualmente, de que sus banderas, emblemas de la conciencia nacional, tampoco lo toquen.
\vs p088 3:4 Las insignias del cargo sacerdotal y real acabaron por considerarse como fetiches, y el fetiche del Estado supremo ha pasado por muchas etapas de desarrollo, desde los clanes a las tribus, desde el protectorado a la soberanía, desde los tótems a las banderas. Los reyes fetiches han gobernado por “derecho divino”, y se han establecido muchas otras formas de gobierno. El hombre también ha hecho un fetiche de la democracia, o exaltación y adoración de las ideas del hombre común cuando se las llama de manera colectiva “opinión pública”. La opinión de un hombre, por sí misma, no se la considera de demasiado valor, pero cuando muchos hombres actúan colectivamente como democracia, este mismo criterio de calidad media se erige como el árbitro de la justicia y el estándar de la rectitud.
\usection{4. LA MAGIA}
\vs p088 4:1 El hombre civilizado aborda los problemas de un entorno real mediante la ciencia; el salvaje intentaba solucionar los problemas reales de un entorno espectral ilusorio mediante la magia. La magia era la forma de actuar sobre el entorno imaginario de los espíritus, cuyas maquinaciones daban continuamente explicación a lo inexplicable; era el arte de obtener la cooperación voluntaria de los espíritus y de forzar su ayuda involuntaria por medio del uso de fetiches o de otros espíritus más poderosos.
\vs p088 4:2 El objeto de la magia, la brujería y la necromancia era doble:
\vs p088 4:3 \li{1.}Garantizar información sobre el futuro.
\vs p088 4:4 \li{2.}Influir favorablemente en el entorno.
\vs p088 4:5 \pc Los objetivos de la ciencia son idénticos a los de la magia. La humanidad está progresando de la magia a la ciencia, no mediante la meditación y la razón, sino más bien, de forma gradual y penosa, a través de una larga experiencia. El hombre está paulatinamente respaldando la verdad, comenzando desde el error, progresando en el error y, finalmente, alcanzando el umbral de la verdad. Solo miró adelante con la llegada del método científico. Pero el hombre primitivo tenía que experimentar o perecer.
\vs p088 4:6 La fascinación experimentada hacia las primeras supersticiones fue la madre de la venidera curiosidad científica. Había una emoción dinámica en avance ---temor más curiosidad--- respecto a estas supersticiones primitivas; había una continua fuerza impulsora en la antigua magia. Estas supersticiones suponían la gradual aparición del deseo humano por conocer y dominar el entorno planetario.
\vs p088 4:7 La magia influyó tanto en el salvaje porque no podía entender la idea de la muerte natural. La noción posterior del pecado original ayudó a debilitar el control que ejercía la magia sobre la raza porque daba una explicación sobre la muerte natural. Hubo un tiempo en el que no era infrecuente que se ajusticiasen a diez personas inocentes porque se les suponía responsables de alguna muerte natural. Este es uno de los motivos por los que los pueblos antiguos no crecieron más rápidamente, y todavía esto es cierto de algunas tribus africanas. La persona acusada normalmente se declaraba culpable, incluso cuando se enfrentaba a la muerte.
\vs p088 4:8 La magia es natural para el salvaje. Cree que un enemigo puede morir al practicársele brujería sobre un mechón de su cabello o sobre las cortaduras de sus uñas. La muerte por mordedura de serpiente se atribuía a la magia del brujo. La dificultad para combatir la magia radica en el hecho de que el miedo puede matar. Los pueblos primitivos temían tanto a la magia que esta realmente los mataba, y tal hecho era suficiente para justificar esa errónea creencia. En caso de fracaso, siempre había alguna posible explicación; la cura para una magia defectuosa era más magia.
\usection{5. AMULETOS MÁGICOS}
\vs p088 5:1 Puesto que todo lo referido al cuerpo podía convertirse en fetiche, la magia más primitiva estaba relacionada con el cabello y las uñas. El secretismo que caracteriza la excreción corporal nació del temor a que un enemigo se apoderase de algo derivado del cuerpo y lo empleara mágicamente para hacer daño; por lo tanto, se enterraban cuidadosamente todos los excrementos del cuerpo. Se evitaba escupir en público por miedo a que la saliva se usase de forma mágica para hacer daño; el esputo siempre se tapaba. Incluso las sobras de comida, la ropa y los ornamentos podían emplearse en la magia. El salvaje nunca dejaba restos de su comida sobre la mesa. Y todo esto se hacía por miedo a que los enemigos pudiesen utilizarlos en sus ritos mágicos, no por el reconocimiento del valor higiénico de dichas prácticas.
\vs p088 5:2 Se confeccionaban amuletos mágicos de una gran variedad de cosas: carne humana, garras del tigre, dientes del cocodrilo, semillas de plantas venenosas, veneno de serpiente y cabello humano. Los huesos de los muertos se consideraban extremadamente mágicos. Incluso el polvo de las pisadas se podía usar en la magia. Los antiguos eran grandes creyentes en los amuletos de amor. La sangre y otras formas de secreciones corporales podían ejercer un mágico influjo sobre el amor.
\vs p088 5:3 Se creía en la eficacia mágica de las imágenes. Se hacían efigies, y si se las trataba bien o mal, se suponía que estos mismos efectos recaían sobre la persona real. Cuando se hacían compras, los supersticiosos masticaban un trozo de madera dura para ablandar el corazón del vendedor.
\vs p088 5:4 La leche de una vaca negra era sumamente mágica; al igual que los gatos negros. El báculo o la vara eran mágicos, como también los tambores, las campanas y los nudos. Todos los objetos antiguos eran amuletos mágicos. Las prácticas de una civilización nueva o superior se veían de forma desfavorable al suponérseles de una naturaleza mágica maligna. Durante mucho tiempo, así se consideró la escritura, la imprenta y las imágenes.
\vs p088 5:5 El hombre primitivo creía que los nombres debían tratarse con respeto, particularmente los nombres de los dioses. El nombre se concebía como una entidad, con un poder distinto al de la persona física; se le estimaba en igualdad de condiciones que el alma y la sombra. Los nombres se empeñaban para conseguir préstamos; un hombre no podía usar su nombre hasta que lo hubiese rescatado mediante el pago del préstamo. Hoy en día, se firma el nombre en un pagaré. Pronto el nombre de una persona llegó a ser importante en la magia. El salvaje tenía dos nombres; el nombre principal se consideraba demasiado sagrado como para utilizarlo en ocasiones ordinarias, de ahí el segundo de ellos o el de todos los días: el sobrenombre. El salvaje nunca decía su nombre verdadero a los extraños. Cualquier experiencia de naturaleza inusual le hacía cambiarse de nombre; algunas veces era en un intento por curar una enfermedad o parar la mala suerte. El salvaje podía conseguir un nuevo nombre comprándoselo al jefe tribal; todavía invierten los hombres en títulos y diplomas. Sin bien, entre las tribus más primitivas, como los bosquimanos africanos, no existen nombres individuales.
\usection{6. LA PRÁCTICA DE LA MAGIA}
\vs p088 6:1 La magia se practicó haciendo uso de varas, rituales “medicinales”, y encantamientos, y era costumbre que quien la practicara lo hiciese sin ropa. Entre los magos primitivos, las mujeres eran más numerosas que los hombres. En magia, la palabra “medicina” significa misterio, no tratamiento. El salvaje nunca se curaba a sí mismo; jamás utilizaba medicamentos salvo por indicación de los profesionales de la magia. Los médicos vudús del siglo XX son similares a los magos antiguos.
\vs p088 6:2 La magia tenía una faceta pública y otra privada. La que el curandero, el chamán o el sacerdote practicaban era supuestamente para el bien de toda la tribu. Los brujos, hechiceros y magos impartían magia privada, personal y egoísta que se empleaba de forma coercitiva para hacer el mal a los enemigos. El concepto de doble espiritualismo, espíritus buenos y malos, dio origen a las posteriores creencias en la magia blanca y negra. Y, a medida que la religión evolucionaba, la magia fue el término que se aplicaba a las acciones de los espíritus fuera del propio sistema de culto, así como a creencias más antiguas en los espectros.
\vs p088 6:3 Las combinaciones de palabras, el ritual de cánticos e invocaciones, eran sumamente mágicas. Algunas invocaciones primitivas evolucionaron hasta convertirse en oraciones. Pronto se practicó la magia imitativa; las oraciones se representaban; las danzas mágicas no eran sino oraciones teatralizadas. Paulatinamente, la oración fue desplazando a la magia y se vinculó a los sacrificios.
\vs p088 6:4 Al ser más antiguos que el habla, los gestos eran más sagrados y mágicos, y se creía que la mímica tenía un fuerte poder mágico. A menudo, los hombres rojos escenificaban la danza del búfalo, en la que uno de ellos desempeñaba el papel de búfalo y, al dejarse capturar, aseguraba el éxito de la futura caza. Las celebraciones sexuales del Primero de Mayo eran simplemente magia imitativa, una sugerente apelación a las pasiones sexuales del mundo vegetal. Las muñecas se emplearon en un principio como talismanes mágicos por parte de las esposas estériles.
\vs p088 6:5 \pc La magia fue la rama nacida del árbol religioso evolutivo que, con el tiempo, hizo florecer la era científica. La creencia en la astrología llevó al desarrollo de la astronomía; la creencia en la piedra filosofal llevó al dominio de los metales, mientras que la creencia en los números mágicos fundó la ciencia de las matemáticas.
\vs p088 6:6 \pc Pero un mundo tan lleno de encantamientos contribuyó bastante a acabar con la iniciativa y la ambición personal. Los frutos del trabajo añadido o del esfuerzo diligente eran considerados mágicos. Si un hombre cosechaba en su campo más cereales que su vecino se le podía forzar a comparecer ante el jefe acusado de atraer el cereal extra del campo de su indolente vecino. De hecho, en los días de la barbarie era peligroso saber demasiado; siempre existía la posibilidad de ser ejecutado como artífice de magia negra.
\vs p088 6:7 La ciencia está paulatinamente suprimiendo de la vida el factor juego de azar. Pero si los modernos métodos educativos fracasaran, se produciría casi de inmediato una vuelta a las creencias primitivas en la magia. Estas supersticiones aún persisten en las mentes de muchas personas denominadas civilizadas. La lengua contiene muchos vestigios que dan testimonio de que la raza humana ha estado durante mucho tiempo inmersa en la superstición mágica; es el caso de palabras como hechizado, desventurado, poseso, inspirado, como por arte de magia, ingenio, cautivador, estupefacto y atónito. Hay seres humanos inteligentes que todavía creen en la buena suerte, en el mal de ojo y en la astrología.
\vs p088 6:8 La magia antigua fue la crisálida de la ciencia moderna, indispensable en su tiempo pero de ninguna utilidad ahora. Y así, los espectros, consecuencias de la superstición y la ignorancia, agitaron las mentes primitivas de los hombres hasta que pudieron producirse los conceptos de la ciencia. Hoy en día, Urantia está en la zona claroscura de su evolución intelectual. La mitad del mundo se aferra ansiosamente a la luz de la verdad y de los hechos de los descubrimientos científicos, mientras que la otra mitad languidece en los brazos de supersticiones ancestrales y de una magia levemente encubierta.
\vsetoff
\vs p088 6:9 [Exposición de una brillante estrella vespertina de Nebadón.]
