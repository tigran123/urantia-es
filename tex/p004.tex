\upaper{4}{La relación de Dios con el universo}
\author{Consejero divino}
\vs p004 0:1 El Padre Universal tiene un propósito eterno en relación con los fenómenos materiales, intelectuales y espirituales del universo de los universos, que lleva a efecto en todo momento. Dios creó los universos en un acto de su libre y soberana voluntad, y los creó de acuerdo con un designio eterno y pleno en sabiduría. Existen dudas de si existe alguien, con excepción de las Deidades del Paraíso y de sus más sublimes colaboradores, que pudiera tener realmente un gran conocimiento sobre dicho designio. Incluso entre los excelsos ciudadanos del Paraíso hay diversidad de opiniones sobre la naturaleza de tal propósito eterno de las Deidades.
\vs p004 0:2 Es fácil deducir que el propósito de la creación del perfecto universo central de Havona fue pura satisfacción de la naturaleza divina. Havona quizás sirva como el modelo de creación para todos los demás universos y como la escuela final para los peregrinos del tiempo en su camino al Paraíso; sin embargo, una creación tan magnífica tiene que existir ante todo para el gozo y la satisfacción de sus perfectos e infinitos Creadores.
\vs p004 0:3 El asombroso plan de perfeccionar a los mortales evolutivos y, tras alcanzar el Paraíso y el colectivo final, proporcionarles formación adicional para alguna tarea futura no desvelada, parece ser, en este momento, uno de los asuntos principales de los siete suprauniversos y de sus muchas subdivisiones; pero este proyecto de ascensión para la espiritualización y formación de los mortales del tiempo y del espacio no es, de ningún modo, la ocupación exclusiva de las inteligencias del universo. Existen, en efecto, muchas tareas que ocupan el tiempo y atraen las energías de las multitudes de seres celestiales.
\usection{1. LA ACTITUD UNIVERSAL DEL PADRE}
\vs p004 1:1 A lo largo de los tiempos, los habitantes de Urantia han malinterpretado la providencia de Dios. Hay una providencia divina que obra en vuestro mundo, pero su ministerio no es del tipo material, arbitrario y pueril que muchos mortales imaginan. Dicha providencia consiste en la actuación entrelazada de los seres celestiales y de los espíritus divinos que, de acuerdo con las leyes cósmicas, llevan incesantemente a cabo para honrar a Dios y hacer avanzar espiritualmente a sus hijos del universo.
\vs p004 1:2 ¿Es que no podéis avanzar en vuestro concepto del comportamiento de Dios hacia el hombre hasta el punto de reconocer que la consigna del universo es \bibemph{progreso?} A través de largos periodos de tiempo, la raza humana ha luchado por alcanzar su estado actual. Durante todos estos milenios, la Providencia ha estado elaborando el plan de la evolución progresiva. No hay oposición en estas dos ideas en la práctica, sino solo en las nociones equivocadas del hombre. La providencia divina jamás se despliega en oposición al verdadero progreso humano ya sea temporal o espiritual; es siempre coherente con la naturaleza invariable y perfecta del Legislador Supremo.
\vs p004 1:3 “Fiel es Dios” y “todos sus mandamientos son justicia”. “Su fidelidad se afirma en los mismos cielos”. “Para siempre, oh Señor, permanece tu palabra en los cielos. De generación en generación es tu fidelidad; tú afirmaste la tierra, y subsiste”. “Él es fiel Creador”.
\vs p004 1:4 No hay restricción en cuanto a las fuerzas y a los seres personales que el Padre puede utilizar para mantener su propósito y sustentar a sus criaturas. “El eterno Dios es nuestro refugio, y aquí abajo están los brazos eternos”. “El que habita al abrigo del Altísimo morará bajo la sombra del Omnipotente”. “He aquí, no se adormecerá ni dormirá el que nos guarda”. “Sabemos que a los que aman a Dios todas las cosas los ayudan a bien”, “porque los ojos del Señor están sobre los justos y sus oídos atentos a sus oraciones”.
\vs p004 1:5 Dios sostiene “todas las cosas con la palabra de su poder”. Y cuando nacen nuevos mundos, “envía a sus Hijos y son creados”. Dios no solo crea, sino que “los guarda a todos”. Dios constantemente sostiene todas las cosas materiales y a todos los seres espirituales. Los universos son eternamente estables. Existe estabilidad en medio de una inestabilidad aparente. Existe un orden y una seguridad subyacentes en medio de las convulsiones de la energía y de los cataclismos físicos de los reinos estelares.
\vs p004 1:6 El Padre Universal no se ha retirado de la dirección de los universos; él no es una Deidad inactiva. Si Dios dejara de ser el sostenedor presente de toda la creación, ocurriría de inmediato un desmoronamiento universal. A no ser por Dios, no existiría la \bibemph{realidad}. En este mismo momento, al igual que durante los tiempos remotos del pasado y en el futuro eterno, Dios continúa su sostenimiento. El dominio divino se extiende por el círculo de la eternidad. No se da cuerda al universo como a un reloj para que solamente ande durante cierto tiempo y luego cese de funcionar; todas las cosas se renuevan constantemente. El Padre derrama de forma incesante energía, luz y vida. La obra de Dios es real así como también espiritual. “Él extiende el norte sobre el espacio vacío y cuelga la tierra sobre la nada”.
\vs p004 1:7 \pc Un ser del orden al que pertenezco puede descubrir la armonía última y detectar una coordinación profunda y de gran alcance en los asuntos rutinarios de la administración del universo. Mucho de lo que parece inconexo y casual en la mente mortal, se muestra ordenado y constructivo en mi entendimiento. Pero en los universos ocurren muchas cosas que no comprendo del todo. Durante mucho tiempo he estudiado y soy más o menos conocedor de las fuerzas, energías, mentes, morontias, espíritus y seres personales reconocidos de los universos locales y de los suprauniversos. De modo general, entiendo cómo operan estas instancias intermedias y seres personales, y estoy muy familiarizado con la actividad de las inteligencias espirituales reconocidas del gran universo. Pese a mi conocimiento de los fenómenos de los universos, se me presentan constantemente reacciones cósmicas que no puedo desentrañar del todo. Me encuentro constantemente con coincidencias, en apariencia fortuitas, de la correlación entre fuerzas, energías, intelectos y espíritus, a las que no puedo dar una explicación satisfactoria.
\vs p004 1:8 Estoy por entero capacitado para trazar y analizar el funcionamiento de todos los fenómenos que resultan directamente de la acción del Padre Universal, del Hijo Eterno, del Espíritu Infinito y, en gran parte, de la Isla del Paraíso. Lo que suscita mi perplejidad es encontrarme con lo que parece ser la actuación de sus misteriosos iguales en rango, los tres Absolutos de potencialidad. Estos Absolutos parecen reemplazar la materia, trascender la mente y estar aparte del espíritu. Me siento constantemente confundido y a menudo perplejo, por mi incapacidad para comprender estas complejas interacciones que atribuyo a las presencias y acciones del Absoluto Indeterminado, del Absoluto de la Deidad y del Absoluto Universal.
\vs p004 1:9 Estos Absolutos deben ser las presencias no enteramente reveladas existentes fuera en el universo que, en los fenómenos de la potencia del espacio y en la actividad de otros supraúltimos, imposibilitan a los físicos, a los filósofos e incluso a los creyentes religiosos predecir con certeza cómo los principios primordiales de la fuerza, de los conceptos o del espíritu responden a las exigencias hechas en una situación compleja de la realidad, que suponga modificaciones supremas y valores últimos.
\vs p004 1:10 \pc Hay también una unidad orgánica en los universos del tiempo y del espacio que parece subyacer en todo el tejido de los acontecimientos cósmicos. Esta presencia viva del Ser Supremo en evolución, esta Inmanencia del Incompleto Proyectado, se manifiesta de forma inexplicable, periódicamente, mediante lo que parece ser una asombrosa y fortuita coordinación de sucesos que acontecen en el universo aparentemente sin relación entre sí. Esta debe ser la acción de la Providencia ---el ámbito del Ser Supremo y del Actor Conjunto---.
\vs p004 1:11 Me inclino a creer que esta extensa potestad, generalmente irreconocible, sobre la coordinación y correlación de todas las facetas y formas de la actividad del universo, es la que ocasiona que una combinación tan variada y, en apariencia, tan irremediablemente confusa de fenómenos físicos, mentales, morales y espirituales obre, de forma tan inequívoca, para la gloria de Dios y para el bien de hombres y ángeles.
\vs p004 1:12 Pero, en su sentido más amplio, los aparentes “accidentes” del cosmos forman parte, sin duda, de la acción finita de la aventura, en el espacio\hyp{}tiempo, del Infinito en su acción eterna sobre los absolutos.
\usection{2. DIOS Y LA NATURALEZA}
\vs p004 2:1 La naturaleza es, en un sentido limitado, la vestimenta física de Dios. La conducta o acción de Dios se condiciona y se modifica, de manera provisional, a partir de los planes experimentales y los modelos evolutivos de un universo local, de una constelación, de un sistema o de un planeta. Dios actúa de acuerdo con una ley bien definida, invariable e inmutable por todo el inmenso y creciente universo matriz, pero modifica sus pautas de acción a fin de contribuir a la dirección armonizada y equilibrada de cada universo, constelación, sistema, planeta y ser personal de acuerdo con los objetivos, propósitos y planes locales de los finitos designios del despliegue evolutivo.
\vs p004 2:2 Por tanto, la naturaleza, tal como la entiende el hombre mortal, constituye la base subyacente y el trasfondo fundamental de una Deidad invariable y de sus leyes inmutables, que se modifican, fluctúan y trastornan a causa del funcionamiento de los planes, propósitos, pautas y condiciones locales que el universo, la constelación, el sistema y las fuerzas y seres personales planetarios locales han establecido y están llevando a cabo. Por ejemplo, así como se han decretado las leyes de Dios en Nebadón, estas se modifican por los planes establecidos por el hijo creador y el espíritu creativo de este universo local; además de todo esto, en el ejercicio de estas leyes también han influido los errores, incumplimientos e insurrecciones de algunos seres residentes en vuestro planeta y pertenecientes a vuestro inmediato sistema planetario de Satania.
\vs p004 2:3 \pc La naturaleza es la resultante en el espacio\hyp{}tiempo de dos factores cósmicos: primero, la inmutabilidad, perfección y rectitud de la Deidad del Paraíso y, segundo, los planes experimentales, los desatinos gobernativos, los errores de rebeldía, el desarrollo incompleto y la imperfección de la sabiduría de las criaturas extraparadisíacas, desde las más elevadas hasta las más modestas. La naturaleza, por tanto, traza un hilo de perfección maravilloso, majestuoso, invariable y uniforme desde el círculo de la eternidad; pero en cada universo, en cada planeta y en cada vida individual esta naturaleza se modifica, se condiciona y tal vez se deforma con los actos, las equivocaciones y las deslealtades de las criaturas de los sistemas y universos evolutivos; y, por tanto, por siempre ha de ser una naturaleza cambiante, caprichosa, aunque de fondo estable, y variada de acuerdo con los procedimientos de carácter operativo de un universo local.
\vs p004 2:4 La naturaleza es la perfección del Paraíso dividida por la incompletitud, la maldad y el pecado de los universos inacabados. Este cociente expresa, pues, lo perfecto y lo parcial, lo eterno y lo temporal. La evolución continuada modifica la naturaleza al aumentar el contenido de perfección del Paraíso y disminuir el contenido del mal, del error y de la desarmonía de la realidad relativa.
\vs p004 2:5 \pc Dios no está personalmente presente en la naturaleza ni en cualesquiera de las fuerzas de la naturaleza, porque el fenómeno de la naturaleza consiste en la superposición de las imperfecciones de la evolución progresiva y, a veces, de las consecuencias de rebeliones por insurrección sobre las bases paradisíacas de la ley universal de Dios. Tal como aparece en un mundo como Urantia, la naturaleza no puede ser nunca la expresión adecuada ni la representación verdadera ni el fiel retrato de un Dios pleno en sabiduría e infinito.
\vs p004 2:6 La naturaleza, en vuestro mundo, consiste en el condicionamiento de las leyes de la perfección por los planes evolutivos del universo local. ¡Qué farsa adorar a la naturaleza porque en un sentido limitado, condicionado, esté infundida por Dios y sea una faceta del poder universal y por lo tanto divino! La naturaleza también es una manifestación de las inacabadas, incompletas, imperfectas elaboraciones del desarrollo, crecimiento y progreso de un experimento de evolución cósmica en el universo.
\vs p004 2:7 Los defectos aparentes del mundo natural no indican defecto alguno que tenga correspondencia en el carácter de Dios. Más bien, las imperfecciones que se observan son meramente las inevitables y momentáneas paradas en un carrete, siempre en movimiento, que estuviera proyectando imágenes infinitas. Son estas mismas interrupciones\hyp{}defectos de la continuidad\hyp{}perfección las que hacen posible que la mente finita del hombre material capte, en el tiempo y en el espacio, un atisbo fugaz de la realidad divina. Las manifestaciones materiales de la divinidad parecen defectuosas en la mente evolutiva del hombre solamente porque el hombre mortal persiste en ver los fenómenos de la naturaleza a través de los ojos naturales, una visión humana sin ayuda de la mota morontial o de la revelación, su sustituto compensatorio en los mundos del tiempo.
\vs p004 2:8 Y la naturaleza está deformada, su hermoso rostro marcado, sus rasgos abrasados por la rebelión, la mala conducta, los malos pensamientos de innumerables criaturas que son parte de la naturaleza, pero que han contribuido a su desfiguración en el tiempo. La naturaleza no es Dios. La naturaleza no es objeto de adoración.
\usection{3. EL CARÁCTER INVARIABLE DE DIOS}
\vs p004 3:1 Durante demasiado largo tiempo el hombre ha creído que Dios era semejante a él. Dios no es celoso, no lo ha sido nunca, ni lo será jamás ni del hombre ni de ningún otro ser del universo de los universos. Sabiendo que el hijo creador pretendía que el hombre fuera la obra maestra de la creación planetaria, que gobernara toda la tierra, la visión de su ser dominado por sus pasiones más bajas, el espectáculo de su sumisión ante ídolos de madera, piedra y oro, y su ambición egoísta son unas sórdidas escenas que mueven a Dios y a sus hijos del Paraíso a ser celosos \bibemph{por} el hombre, pero nunca de él.
\vs p004 3:2 El Dios Eterno es incapaz de experimentar cólera o ira en el sentido humano de estas emociones y tal como el hombre entiende esas reacciones. Estos sentimientos son mezquinos y despreciables; son indignos de ser llamados humanos, mucho menos divinos; y tales actitudes son completamente ajenas a la naturaleza perfecta y al carácter clemente del Padre Universal.
\vs p004 3:3 \pc Mucha, muchísima de la dificultad que tienen los mortales de Urantia para entender a Dios se debe al gran efecto de las consecuencias de la rebelión de Lucifer y a la traición de Caligastia. En los mundos sin aislar por el pecado, las razas evolutivas son capaces de desarrollar ideas más adecuadas sobre el Padre Universal; se resienten menos por la confusión, la distorsión y la deformación conceptual.
\vs p004 3:4 \pc Dios no se arrepiente de nada de lo que ha hecho, de lo que hace o de lo que hará. Él es plenamente sabio a la vez que omnipotente. La sabiduría del hombre surge del ensayo y error de la experiencia humana; la sabiduría de Dios consiste en la perfección incondicional de su infinita percepción del universo, y esta precognición divina dirige con eficacia la libre voluntad creativa.
\vs p004 3:5 El Padre Universal nunca hace nada que cause dolor o pesar; sin embargo, las criaturas volitivas que los seres personales creadores planifican y crean en los remotos universos, a veces por decisiones desafortunadas, sí ocasionan sentimientos de dolor divino en las personas de sus padres creadores. Pero, aunque el Padre no comete errores ni alberga pesar ni experimenta dolor, es cierto que es un ser con afecto de padre y su corazón, sin duda, se entristece cuando sus hijos no consiguen alcanzar los niveles espirituales que son capaces de llevar a cabo, con la asistencia que tan profusamente les ha proporcionado, los planes de desarrollo espiritual y los procedimientos de ascensión de los mortales de los universos.
\vs p004 3:6 La bondad infinita del Padre está fuera de la cognición de la mente finita del tiempo; de aquí que siempre haya que contrastarla con el mal (no el pecado) para poner de manifiesto, de forma efectiva, todas las facetas de la bondad relativa. Los mortales, con su imperfecta percepción, pueden discernir la perfección de la bondad divina solamente si la comparan con la imperfección relativa propia de las relaciones del tiempo y la materia, en las mociones del espacio.
\vs p004 3:7 El carácter de Dios es infinitamente sobrenatural; por consiguiente, esta naturaleza divina debe hacerse personal, como lo hace en los hijos divinos, antes de que la mente finita del hombre pueda siquiera alcanzar a comprenderla por la fe.
\usection{4. LA REALIDAD DE DIOS}
\vs p004 4:1 Dios es el único ser estacionario contenido en sí mismo e invariable de todo el universo de los universos, que no tiene exterior ni más allá ni pasado ni futuro. Dios es energía intencional (espíritu creativo) y voluntad absoluta, y estas son existentes por sí mismas y universales.
\vs p004 4:2 Puesto que Dios existe por sí mismo, es absolutamente independiente. La identidad misma de Dios es adversa al cambio. “Yo, el Señor, no cambio”. Dios es inmutable; pero hasta que no hayáis logrado el estatus del Paraíso, no podréis ni siquiera comenzar a entender cómo Dios puede pasar de la simplicidad a la complejidad, de la identidad a la variación, de la quiescencia al movimiento, de la infinitud a la finitud, de lo divino a lo humano y de la unidad a la dualidad y a la triunidad. Y Dios puede, de este modo, modificar las manifestaciones de su absolutidad porque la inmutabilidad divina no implica inmovilidad; Dios tiene voluntad: él \bibemph{es} la voluntad.
\vs p004 4:3 Dios es poseedor de una absoluta determinación propia; no hay límites en cuanto a sus respuestas al universo, salvo aquellas que él mismo se impone, y los actos de su libre voluntad están condicionados solo por las cualidades divinas y los atributos perfectos que caracterizan intrínsecamente su naturaleza eterna. Así pues, Dios se relaciona con el universo como un ser cuya bondad es final y su libre voluntad infinitamente creativa.
\vs p004 4:4 El Padre\hyp{}Absoluto es el creador del universo central y perfecto y el Padre de todos los otros creadores. Dios comparte con el hombre y otros seres el ser personal, la bondad y otras numerosas características, pero la infinitud de la voluntad es solo propia de él. Dios está limitado en sus actos creativos solo por los sentimientos de su naturaleza eterna y por los dictados de su infinita sabiduría. Dios personalmente elige solo lo que es infinitamente perfecto, de aquí la perfección excelsa del universo central; y aunque los hijos creadores comparten plenamente su divinidad, incluso facetas de su absolutidad, estos no están totalmente limitados por esa completud de sabiduría que dirige la infinitud de la voluntad del Padre. Por consiguiente, en el orden de filiación de Miguel, la libre voluntad creativa se hace todavía más activa, totalmente divina y casi última, si es que no es absoluta. El Padre es infinito y eterno, pero negar la posibilidad de la limitación volitiva de sí mismo equivaldría a la negación del concepto mismo de la absolutidad de su volición.
\vs p004 4:5 \pc La absolutidad de Dios se difunde por los siete niveles de la realidad del universo. Y la totalidad de esta naturaleza absoluta está sujeta a la relación del Creador con su familia de criaturas en el universo. La precisión puede caracterizar la justicia trinitaria en el universo de los universos, pero en todas sus inmensas relaciones parentales con las criaturas del tiempo, el Dios de los universos se rige por un \bibemph{sentimiento divino}. En todos los aspectos ---eternamente---, el Dios infinito es un \bibemph{Padre}. De todos los títulos posibles por los que podría ser conocido con propiedad, he sido instruido para describir al Dios de toda la creación como el Padre Universal.
\vs p004 4:6 En el Dios Padre, la actuación de su libre voluntad no se rige por el poder, ni se orienta únicamente por el intelecto; la persona divina se define por ser espíritu y por manifestarse a los universos como amor. Por tanto, en todas sus relaciones personales con los seres personales creaturales de los universos, la Primera Fuente y Centro es siempre e invariablemente un Padre amoroso. Dios es un Padre en el más elevado significado del término. Está eternamente motivado por el perfecto idealismo del amor divino y esa tierna naturaleza encuentra su más sólida expresión y su más grande satisfacción al amar y ser amado.
\vs p004 4:7 \pc En la ciencia, Dios es la primera causa; en la religión, el amoroso Padre Universal; en la filosofía es el único ser que existe por sí mismo, que no depende de ningún otro ser para existir, sino que graciosamente otorga existencia real a todas las cosas y a todos los otros seres. Pero se requiere la revelación para demostrar que la primera causa de la ciencia y la Unidad autoexistente de la filosofía son el Dios de la religión, pleno de misericordia y bondad, y comprometido en llevar a efecto la supervivencia eterna de sus hijos de la tierra.
\vs p004 4:8 Anhelamos el concepto del Infinito, pero adoramos la experiencia\hyp{}idea de Dios, nuestra capacidad para alcanzar a comprender, en cualquier momento y lugar, los componentes del ser personal y de la divinidad de nuestra más elevada noción de la Deidad.
\vs p004 4:9 La conciencia de una vida humana victoriosa en la tierra nace de esa fe de la criatura que se atreve a desafiar persistentes episodios de la existencia, en los que se enfrenta con el horrible espectáculo de las limitaciones humanas, con inquebrantable declaración: aunque yo no pueda hacer esto, en mí vive alguien que puede y que lo hará, una parte del Padre\hyp{}Absoluto del universo de los universos. Y esta es “la victoria que ha vencido al mundo, vuestra fe”.
\usection{5. IDEAS ERRÓNEAS SOBRE DIOS}
\vs p004 5:1 La tradición religiosa constituye el historial, imperfectamente conservado, de las experiencias de hombres de tiempos pasados conocedores de Dios y, como tal, no resulta fidedigno para servir de guía respecto a la vida religiosa ni de fuente de una información verdadera acerca del Padre Universal. Estas creencias religiosas antiguas se han alterado inevitablemente por el hecho de que el hombre primitivo era un hacedor de mitos.
\vs p004 5:2 Una de las fuentes más importantes de confusión en Urantia relativa a la naturaleza de Dios surge del error de vuestros libros sagrados al no distinguir con claridad entre las personas de la Trinidad del Paraíso y entre la Deidad del Paraíso y los creadores y administradores del universo local. Durante las pasadas dispensaciones, solo parcialmente comprendidas, vuestros sacerdotes y profetas no supieron distinguir entre los príncipes planetarios, los soberanos de los sistemas, los padres de las constelaciones, los hijos creadores, los gobernantes de los suprauniversos, el Ser Supremo y el Padre Universal. Muchos de los mensajes de seres personales de menor rango, como los portadores de vida y los distintos órdenes de ángeles, aparecen en vuestros documentos como procedentes de Dios mismo. En el pensamiento religioso de Urantia, aún se confunden las personas compañeras de la Deidad con el propio Padre Universal, de manera que se incluyen todos bajo un solo apelativo.
\vs p004 5:3 \pc Los habitantes de Urantia continúan padeciendo el efecto de los conceptos primitivos sobre Dios. Dioses que arrasan en la tormenta; que hacen temblar la tierra en su cólera y destruyen a los hombres en su ira; que ocasionan agravios con su indignación en tiempos de escasez y de inundaciones: estos son los dioses de la religión primitiva, no son los dioses que viven y gobiernan los universos. Estos conceptos son reliquias de los tiempos en que los hombres creían que el universo estaba bajo la guía y el dominio de estos caprichosos e imaginarios dioses. Pero el hombre mortal está comenzando a darse cuenta de que vive en un mundo de ley y orden relativos en lo que concierne a la política gobernativa y al proceder de los creadores supremos y de los controladores supremos.
\vs p004 5:4 \pc La descabellada idea de apaciguar a un Dios airado, de propiciar a un Señor ofendido, de ganar el favor de la Deidad mediante sacrificios y penitencias e incluso derramamiento de sangre representa una religión enteramente pueril y primitiva, una filosofía indigna de una iluminada era de ciencia y verdad. Tales creencias producen una total repulsión en los seres celestiales y en los gobernantes divinos que sirven y reinan en los universos. Es una afrenta hacia Dios creer, sostener o enseñar que debe derramarse sangre inocente para ganar su favor o eludir una ficticia ira divina.
\vs p004 5:5 Los hebreos creían que “sin derramamiento de sangre no se hacía remisión de los pecados”. No se habían liberado de la idea antigua y pagana de que los dioses no se podían apaciguar a no ser por el espectáculo de la sangre, aunque Moisés consiguió un gran avance al prohibir los sacrificios humanos y sustituirlos, en la mente primitiva de los pueriles beduinos que lo seguían, por el sacrificio ceremonial de animales.
\vs p004 5:6 La dádiva de un hijo del Paraíso a vuestro mundo era de esperar por la situación de clausura de una era planetaria; era ineludible y no era preciso para ganar el favor de Dios. Dicha dádiva resultó ser también, en sí misma, el último acto personal de un hijo creador en su larga aventura para adquirir la soberanía experiencial de su universo. ¡Qué deformación del carácter infinito de Dios!, ¡qué doctrina la de que su corazón paterno en toda su fría y dura severidad permaneciera tan inconmovible ante los infortunios y penas de sus criaturas, que su tierna misericordia no se derramó hasta que vio a su hijo inocente sangrante y moribundo en la cruz del Calvario!
\vs p004 5:7 Pero los habitantes de Urantia han de llegar a liberarse de estos antiguos errores y de estas supersticiones paganas respecto a la naturaleza del Padre Universal. La revelación de la verdad acerca de Dios está manifestándose, y la raza humana está destinada a conocer al Padre Universal en toda esa belleza de carácter y hermosura de atributos que con tanta magnificencia describió el hijo creador, habitante de Urantia como el Hijo del Hombre y el Hijo de Dios.
\vsetoff
\vs p004 5:8 [Exposición de un consejero divino de Uversa.]
