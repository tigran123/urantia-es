\upaper{9}{La relación del Espíritu Infinito con el universo}
\author{Consejero divino}
\vs p009 0:1 Algo extraño ocurrió cuando, ante la presencia del Paraíso, el Padre Universal y el Hijo Eterno se unieron para hacerse personales. Nada en estas circunstancias de la eternidad presagiaba que el Actor Conjunto se haría igualmente personal adquiriendo una espiritualidad ilimitada coordinada con la mente absoluta y dotada, además, de singulares prerrogativas para la actuación sobre la energía. Su aparición completaba la liberación del Padre de sus ataduras a la perfección centralizada y a la absolutidad del ser personal. Y esta liberación se revelaba en el sorprendente poder del Creador Conjunto para crear seres bien adaptados que obraran como espíritus servidores, que incluso atendiesen a las criaturas materiales de los universos aún por evolucionar.
\vs p009 0:2 \pc El Padre es infinito en amor y volición, en pensamiento y propósito espiritual; es el sostenedor universal. El Hijo es infinito en sabiduría y verdad, en expresión y elucidación espiritual; es el revelador universal. El Paraíso es infinito en su potencial para la dotación de fuerza y en su capacidad para dominar la energía; es el estabilizador universal. El Actor Conjunto posee prerrogativas únicas de síntesis, capacidad infinita para coordinar todas las energías existentes en el universo, todos los espíritus existentes en el universo y todos los auténticos intelectos del universo. La Tercera Fuente y Centro unifica de forma universal las múltiples energías y las múltiples creaciones que han aparecido como consecuencia del plan divino y del eterno propósito del Padre Universal.
\vs p009 0:3 El Espíritu Infinito, el Creador Conjunto, es benefactor universal y divino. El Espíritu continuamente ejerce el ministerio de la misericordia del Hijo y del amor del Padre, en armonía además con la justicia estable, invariable y recta de la Trinidad del Paraíso. Su influencia y los seres personales creados por él están cerca de vosotros; realmente os conocen y verdaderamente os entienden.
\vs p009 0:4 En todos los universos, las instancias intermedias del Actor Conjunto actúan sin cesar sobre las fuerzas y las energías de todo el espacio. Al igual que la Primera Fuente y Centro, la Tercera responde tanto a lo material como a lo espiritual. El Actor Conjunto es la revelación de la unidad de Dios, en quien todas las cosas consisten: las cosas, los contenidos y los valores; las energías, las mentes y los espíritus.
\vs p009 0:5 \pc El Espíritu Infinito se difunde por todo el espacio; habita el círculo de la eternidad; y el Espíritu, al igual que el Padre y el Hijo, es perfecto e inmutable: absoluto.
\usection{1. ATRIBUTOS DE LA TERCERA FUENTE Y CENTRO}
\vs p009 1:1 A la Tercera Fuente y Centro se le conoce por muchos nombres; todos indican relación y reconocen su labor. Como Dios Espíritu, es el ser personal del mismo rango y el igual divino del Dios Hijo y del Dios Padre. Como Espíritu Infinito, es una influencia espiritual omnipresente. Como Actuante Universal, antecede a las criaturas que dirigen la potencia y activa las fuerzas cósmicas del espacio. Como Actor Conjunto, es el representante conjunto y el compañero mandatario del Padre\hyp{}Hijo. Como Mente Absoluta, es la fuente de la dotación intelectual de todos los universos. Como Dios de Acción, es el aparente antecesor del movimiento, del cambio y de la relación.
\vs p009 1:2 Algunos de los atributos de la Tercera Fuente y Centro se derivan del Padre, algunos del Hijo; otros, sin embargo, no parecen estar presentes de forma activa y personal ni en el Padre ni en el Hijo: estos atributos, que difícilmente pueden explicarse excepto asumiendo la unión Padre\hyp{}Hijo que eterniza a la Tercera Fuente y Centro, obran uniformemente respecto, y en reconocimiento, del hecho eterno de que el Paraíso es absoluto. El Creador Conjunto incorpora la plenitud de los conceptos infinitos y combinados de la Primera y de la Segunda Persona de la Deidad.
\vs p009 1:3 \pc Cuando imaginéis al Padre como creador primigenio y al Hijo como administrador espiritual, debéis pensar en la Tercera Fuente y Centro como coordinador universal, como facilitador ilimitado. El Actor Conjunto correlaciona toda realidad actual; es el depositario divino del pensamiento del Padre y del verbo del Hijo y, en acción, es eternamente respetuoso con la absolutidad material de la Isla central. La Trinidad del Paraíso ha establecido el orden universal del \bibemph{progreso,} y la providencia de Dios es el ámbito del Creador Conjunto y del Ser Supremo en evolución. Ninguna realidad actual o en actualización puede escapar a su relación final con la Tercera Fuente y Centro.
\vs p009 1:4 \pc El Padre Universal preside sobre los reinos de la preenergía, del preespíritu y del ser personal; el Hijo Eterno domina las esferas de la actividad espiritual; la presencia de la Isla del Paraíso unifica el dominio de la energía física y la potencia en materialización; el Actor Conjunto opera no solo como un espíritu infinito que representa al Hijo sino también como actuante universal sobre las fuerzas y energías del Paraíso, dando así existencia a la mente universal y absoluta. El Actor Conjunto obra en todo el gran universo como un ser personal inequívoco y bien diferenciado, especialmente en las esferas superiores de los valores espirituales, de las relaciones de la energía física y de los verdaderos contenidos intelectuales. Él obra de forma específica donde y cuando la energía y el espíritu se vinculan e interactúan; domina todas las reacciones de la mente, tiene un gran poder en el mundo espiritual y ejerce una poderosa influencia sobre la energía y la materia. En todo momento, la Tercera Fuente expresa la naturaleza de la Primera Fuente y Centro.
\vs p009 1:5 \pc La Tercera Fuente y Centro comparte de forma perfecta y sin condicionamiento la omnipresencia de la Primera Fuente y Centro, siendo llamado a veces el Espíritu Omnipresente. De manera peculiar y muy personal, el Dios de la mente participa de la omnisciencia del Padre Universal y de su Hijo Eterno; el conocimiento que tiene el Espíritu es profundo y completo. El Creador Conjunto manifiesta ciertas fases de la omnipotencia del Padre Universal, pero solo es realmente omnipotente en el ámbito de la mente. La Tercera Persona de la Deidad es el centro intelectual y el gobernante universal de los ámbitos de la mente; en estos él es absoluto; su soberanía es incondicionada.
\vs p009 1:6 La unión Padre\hyp{}Hijo parece impulsar al Actor Conjunto, pero en todas sus acciones parece reconocer la relación Padre\hyp{}Paraíso. A veces y en ciertas tareas parece compensar el desarrollo incompleto de las Deidades experienciales: el Dios Supremo y el Dios Último.
\vs p009 1:7 \pc Y en esto hay un misterio insondable: que el Infinito reveló de forma simultánea su infinitud en el Hijo y en la acción del Paraíso, y entonces surge como existencia un ser igual a Dios en divinidad que refleja la naturaleza espiritual del Hijo y es capaz de activar el modelo original del Paraíso, un ser provisionalmente de menor grado de soberanía pero, al parecer, de muchas maneras, el más versátil en cuanto a la \bibemph{acción}. Esta aparente superioridad en la acción se revela como atributo de la Tercera Fuente y Centro que es superior incluso a la gravedad física, a la manifestación universal de la Isla del Paraíso.
\vs p009 1:8 Además de este extra control sobre la energía y las cosas físicas, el Espíritu Infinito está espléndidamente dotado de esos atributos de paciencia, misericordia y amor que se revelan de forma tan excelente en su ministerio espiritual. El Espíritu está supremamente cualificado para dispensar amor y matizar la justicia con la misericordia. El Dios Espíritu posee toda la bondad excelsa y el afecto misericordioso del Hijo Primigenio y Eterno. El universo donde tenéis vuestro origen se está forjando con el yunque y el martillo, con la justicia y el sufrimiento; pero los que empuñan el martillo son los hijos de la misericordia, los vástagos espirituales del Espíritu Infinito.
\usection{2. EL ESPÍRITU OMNIPRESENTE}
\vs p009 2:1 Dios es espíritu en un sentido triple: él es espíritu, en su Hijo aparece como espíritu sin condicionamiento y, en el Actor Conjunto, como espíritu ligado a la mente. Y además de estas realidades espirituales, creemos distinguir niveles en los fenómenos espirituales experienciales ---los espíritus del Ser Supremo, de la Deidad Última y del Absoluto de la Deidad---.
\vs p009 2:2 El Espíritu Infinito complementa tanto al Hijo Eterno como el Hijo complementa al Padre Universal. El Hijo Eterno es una manifestación personal espiritualizada del Padre; el Espíritu Infinito constituye una espiritualización en estado personal del Hijo Eterno y del Padre Universal.
\vs p009 2:3 Hay muchas líneas no restringidas de fuerza espiritual y fuentes de potencia supramaterial que enlazan a la población de Urantia directamente con las Deidades del Paraíso. Existe una comunicación directa de los modeladores del pensamiento con el Padre Universal, una extensa influencia del impulso de la gravedad espiritual del Hijo Eterno y una presencia espiritual del Creador Conjunto. Hay diferencia de capacidad de actuación entre el espíritu del Hijo y el espíritu del Espíritu. La Tercera Persona puede obrar en su ministerio espiritual como mente más espíritu o como espíritu solo.
\vs p009 2:4 Además de estas presencias del Paraíso, los habitantes de Urantia se benefician de las influencias y actividad espiritual del universo local y del suprauniverso, con su casi interminable grupo de amorosos seres personales, que por siempre llevan a aquel de verdadero propósito y de corazón honesto, en sentido ascendente e interior, hacia los ideales de divinidad y hacia la meta de la perfección suprema.
\vs p009 2:5 \bibemph{Conocemos} la presencia del espíritu universal del Hijo Eterno: podemos reconocerla de forma inequívoca. Incluso el hombre mortal puede conocer la presencia del Espíritu Infinito, la Tercera Persona de la Deidad, porque las criaturas materiales pueden experimentar realmente la beneficencia de esta influencia divina que obra como el espíritu santo del universo local, que se otorga a las razas de la humanidad. Los seres humanos también pueden, en alguna medida, llegar a tener conciencia del modelador, de la presencia impersonal del Padre Universal. Todos estos espíritus divinos que laboran para la elevación y espiritualización del hombre actúan al unísono y en cooperación perfecta. Son como uno en la tarea espiritual de llevar a cabo los planes de ascensión y de perfección de los mortales.
\usection{3. EL ACTUANTE UNIVERSAL}
\vs p009 3:1 La Isla del Paraíso es la fuente y la sustancia de la gravedad física; y eso debería ser suficiente para informaros de que la gravedad es una de las cosas más \bibemph{reales} y eternamente constantes de todo el universo de universos físicos. La gravedad no se puede modificar ni anular excepto mediante las fuerzas y energías promovidas conjuntamente por el Padre y el Hijo, que se han confiado y vinculado operativamente a la persona de la Tercera Fuente y Centro.
\vs p009 3:2 \pc El Espíritu Infinito posee un poder asombroso y singular: \bibemph{la antigravedad}. Este poder no está operativamente (de manera observable) presente ni en el Padre ni en el Hijo. Esta capacidad para resistir la atracción de la gravedad material, consustancial a la Tercera Fuente, se revela en las reacciones personales del Actor Conjunto respecto a ciertas facetas de sus relaciones con el universo. Este atributo singular es transmisible a algunos de los seres personales superiores del Espíritu Infinito.
\vs p009 3:3 \pc La antigravedad puede anular la gravedad dentro de un perímetro local; lo hace ejerciendo una presencia equivalente de fuerza. Opera solo con referencia a la gravedad material y no es la acción de la mente. El fenómeno de resistencia a la gravedad de un giroscopio es una ilustración clara del \bibemph{efecto} de la antigravedad, pero no sirve para ilustrar la \bibemph{causa} de la antigravedad.
\vs p009 3:4 Aún más, el Actor Conjunto despliega atributos que pueden trascender la fuerza y neutralizar la energía. Tales atributos operan mediante la aminoración de la velocidad de la energía hasta el punto de la materialización y mediante otros métodos desconocidos para vosotros.
\vs p009 3:5 \pc El Creador Conjunto no es energía, ni la fuente de la energía, ni el destino de la energía; es el \bibemph{actuante} de la energía. El Creador Conjunto es acción: movimiento, cambio, modificación, coordinación, estabilización y equilibrio. Las energías sujetas al control directo o indirecto del Paraíso por naturaleza responden a los actos de la Tercera Fuente y Centro y de sus múltiples instancias intermedias.
\vs p009 3:6 El universo de los universos alberga criaturas de la Tercera Fuente y Centro que rigen la potencia: los controladores físicos, los directores de potencia, los centros de la potencia y otros que representan al Dios de Acción, que guardan relación con la regulación y la estabilización de las energías físicas. Todas estas criaturas cuya labor es física varían en sus atributos en cuanto al control de la potencia, tal como el de la antigravedad, que utilizan en su empeño por establecer el equilibrio físico de la materia y de las energías del gran universo.
\vs p009 3:7 Toda esta actividad material del Dios de Acción parece relacionar su labor con la Isla del Paraíso y, en verdad, las instancias intermedias de la potencia están atentas al carácter absoluto de la Isla eterna e incluso dependen de esta. Pero el Actor Conjunto no actúa para el Paraíso ni en respuesta al mismo. Actúa, de forma personal, para el Padre y el Hijo. El Paraíso no es una persona. Las acciones de la Tercera Fuente y Centro no personales, impersonales y de alguna manera no personales son, en su totalidad, actos volitivos del Actor Conjunto mismo; no son reflejos, derivaciones ni consecuencias de nada ni de nadie.
\vs p009 3:8 El Paraíso constituye el modelo primigenio de la infinitud; el Dios de Acción es quien inicia ese modelo. El Paraíso es el punto de apoyo material de la infinitud; las instancias intermedias de la Tercera Fuente y Centro son las palancas inteligentes que impulsan el nivel material e infunden espontaneidad en el mecanismo de la creación física.
\usection{4. LA MENTE ABSOLUTA}
\vs p009 4:1 Existe una naturaleza intelectual en la Tercera Fuente y Centro que es distinta de sus atributos físicos y espirituales. Dicha naturaleza es difícilmente accesible, pero sí es relacionable de forma intelectual aunque no de forma personal. Es posible distinguirla de los atributos físicos y del carácter espiritual de la Tercera Persona en los niveles mentales en los que obra, pero, para la percepción de los seres personales, esta naturaleza nunca obra con independencia de manifestaciones físicas o espirituales.
\vs p009 4:2 La mente absoluta es la mente de la Tercera Persona; es inseparable del ser personal del Dios Espíritu. La mente, en los seres con capacidad de acción, no está separada de la energía ni del espíritu ni de ambos. La mente no es inherente a la energía; la energía es receptiva y responde a la mente; la mente puede imponerse sobre la energía, pero la conciencia no es inherente al nivel puramente material. La mente no tiene que agregarse al espíritu puro, porque el espíritu es, intrínsecamente, consciente e identificativo. El espíritu es siempre inteligente, posee \bibemph{mentalidad} en alguna medida. Puede ser esa mente o aquella, puede ser premente o supramente, incluso mente espiritual, pero realiza el equivalente a pensar y a conocer. La percepción del espíritu trasciende, ocurre y teóricamente antecede a la conciencia de la mente.
\vs p009 4:3 \pc El Creador Conjunto es absoluto tan solo en el dominio de la mente, en los ámbitos de la inteligencia universal. La mente de la Tercera Fuente y Centro es infinita; trasciende totalmente las vías activas y operativas circulatorias de la mente del universo de los universos. En los siete suprauniversos, la dotación de la mente se deriva de los siete espíritus mayores, de los seres personales primarios del Creador Conjunto. Estos espíritus mayores distribuyen la mente en el gran universo como mente cósmica, y la variante de mente cósmica de Nebadón del tipo de Orvontón se difunde por vuestro universo local.
\vs p009 4:4 La mente infinita ignora el tiempo, la mente última trasciende el tiempo, la mente cósmica está condicionada por el tiempo. Y así ocurre con el espacio: la Mente Infinita es independiente del espacio, pero según se desciende desde los niveles infinitos hasta los niveles de actuación de los asistentes de la mente, el intelecto debe tomar cada vez más en cuenta el hecho del espacio y sus limitaciones.
\vs p009 4:5 \pc La fuerza cósmica responde a la mente así como la mente cósmica responde al espíritu. El espíritu es propósito divino y la mente espiritual es propósito divino en acción. La energía es objeto, la mente es contenido, el espíritu es valor. Incluso en el tiempo y en el espacio, la mente establece esas relaciones relativas entre la energía y el espíritu que indican su mutua afinidad en la eternidad.
\vs p009 4:6 La mente trasmuta los valores del espíritu en los contenidos del intelecto; la volición tiene el poder de hacer fructificar los contenidos de la mente en los dominios materiales y espirituales. El ascenso al Paraíso entraña un crecimiento relativo y diferenciado en espíritu, mente y energía, componentes de la individualidad experiencial que el ser personal unifica.
\usection{5. EL MINISTERIO DE LA MENTE}
\vs p009 5:1 La Tercera Fuente y Centro es infinita en cuanto a mente. Si el universo creciera hasta la infinitud, su potencial mental sería todavía apta para dotar de una adecuada mente y de otros imperativos del intelecto a un ilimitado número de criaturas.
\vs p009 5:2 En el ámbito de la \bibemph{mente creada,} la Tercera Persona, con sus colaboradores homólogos y de menor rango, gobierna en supremacía. Los reinos de la mente creatural se originan exclusivamente en la Tercera Fuente y Centro: que es quien otorga la mente. Incluso a las fracciones del Padre les resulta imposible morar en la mente de los hombres hasta que el camino no se haya preparado de forma adecuada mediante la acción mental y la labor espiritual del Espíritu Infinito.
\vs p009 5:3 El rasgo singular de la mente es que puede otorgarse a una amplia variedad de vida. A través de sus colaboradores creaturales y creativos, la Tercera Fuente y Centro sirve a todas las mentes de todas las esferas. Atiende a intelectos humanos y subhumanos a través de los asistentes de la mente de los universos locales y, por mediación de los controladores físicos, auxilia incluso a las entidades más inferiores y no experienciales de los tipos más primitivos de seres vivos. La dirección de la mente es siempre un ministerio de seres personales de mente\hyp{}espíritu o de mente\hyp{}energía.
\vs p009 5:4 \pc Puesto que la tercera persona de la Deidad es el origen de la mente, es muy natural que a las criaturas volitivas evolutivas les resulte más fácil formar conceptos coherentes acerca del Espíritu Infinito que del Hijo Eterno o del Padre Universal. La realidad del Creador Conjunto se revela de forma imperfecta en la existencia misma de la mente humana. El Creador Conjunto es el antecesor de la mente cósmica, y la mente del hombre es una vía individualizada, una porción impersonal por donde circula esa mente cósmica que una hija creativa de la Tercera Fuente y Centro otorga a un universo local.
\vs p009 5:5 \pc Porque la Tercera Persona es la fuente de la mente, no supongáis que todos los fenómenos de la mente son divinos. El intelecto humano está enraizado en el origen material de las razas animales. La inteligencia del universo no es sino una verdadera revelación de Dios que es mente, al igual que la naturaleza física no es sino la verdadera revelación de la belleza y armonía del Paraíso. La perfección está en la naturaleza, pero la naturaleza no es perfecta. El Creador Conjunto es la fuente de la mente, pero la mente no es el Creador Conjunto.
\vs p009 5:6 La mente, en Urantia, es una concertación entre la esencia de perfección del pensamiento y la mentalidad en evolución de vuestra naturaleza humana e inmadura. El plan para vuestra evolución intelectual es, en efecto, de perfección sublime, pero distáis mucho de esa meta divina mientras obráis en el tabernáculo de la carne. La mente tiene en verdad origen divino y ciertamente tiene un destino divino, pero vuestras mentes mortales aún no han alcanzado dignidad divina.
\vs p009 5:7 Con mucha frecuencia, con demasiada frecuencia, desfiguráis vuestras mentes con la insinceridad y las marchitáis con la falta de rectitud, las sometéis al miedo animal y las distorsionáis con ansiedades inútiles. Por ello, aunque la fuente de la mente sea divina, la mente tal como la conocéis en vuestro mundo de ascensión, difícilmente puede llegar a ser objeto de gran admiración, ni mucho menos de adoración o de culto. La contemplación del intelecto humano, inmaduro e inactivo, debería tan solo suscitar humildad.
\usection{6. LA VÍA DE CIRCULACIÓN DE LA GRAVEDAD MENTAL}
\vs p009 6:1 La Tercera Fuente y Centro, la inteligencia universal, es personalmente consciente de cada \bibemph{mente,} de cada intelecto, en toda la creación, y mantiene un contacto personal y perfecto en los inmensos universos con todas estas criaturas físicas, morontiales y espirituales dotadas de mente. Toda esta actividad mental se percibe en la absoluta vía circulatoria de la gravedad mental, que converge en la Tercera Fuente y Centro y forma parte de la conciencia personal del Espíritu Infinito.
\vs p009 6:2 Así como el Padre atrae a todo ser personal hacia él y el Hijo atrae toda la realidad espiritual, así ejerce el Actor Conjunto un poder de atracción sobre todas las mentes; de forma ilimitada, domina y rige la vía universal y absoluta de circulación de la mente. Todos los valores intelectuales verdaderos y genuinos, todos los pensamientos divinos y las ideas perfectas, son infaliblemente atraídos hacia dicha vía.
\vs p009 6:3 \pc La gravedad mental puede operar con independencia de la gravedad material o de la gravedad espiritual, pero la gravedad mental siempre obra dondequiera y cuando quiera que las dos anteriores se superpongan. Cuando las tres se vinculan, la gravedad del ser personal puede abarcar a la criatura humana, física o morontial, finita o absonita. Pero con independencia de esto, incluso en los seres impersonales, el estar dotado de mente les da capacidad para pensar y los dota de conciencia, pese a la ausencia total de ser personal.
\vs p009 6:4 \pc El yo de dignidad personal, humano o divino, inmortal o potencialmente inmortal, no se origina en espíritu, mente o materia alguna; es el don del Padre Universal. Tampoco es la acción recíproca de la gravedad espiritual, mental y material un imperativo para la aparición de la gravedad del ser personal. La vía circulatoria del Padre puede incluir a un ser material dotado de mente que no responda a la gravedad espiritual o a un ser espiritual dotado de mente que no responda a la gravedad material. La utilización de la gravedad del ser personal es siempre un acto volitivo del Padre Universal.
\vs p009 6:5 Aunque la mente está vinculada a la energía en los seres puramente materiales y al espíritu en los seres personales puramente espirituales, hay innumerables órdenes de seres personales, incluyendo a los humanos, que poseen mentes vinculadas tanto a la energía como al espíritu. Los aspectos espirituales de la mente de las criaturas responden infaliblemente a la influencia de la gravedad espiritual del Hijo Eterno; los rasgos materiales responden al impulso de la gravedad del universo material.
\vs p009 6:6 \pc Cuando la mente cósmica no está vinculada ni a la energía ni al espíritu, tampoco está sujeta a las exigencias de la gravedad de las vías de circulación de lo material o de lo espiritual. La mente pura solo está sujeta al dominio de la gravedad del Actor Conjunto. La mente pura es bastante afín a la mente infinita y la mente infinita (el teorético coigual de los absolutos del espíritu y de la energía) es en apariencia una ley en sí misma.
\vs p009 6:7 Cuanto mayor sea la divergencia entre el espíritu y la energía, más se podrá apreciar la labor de la mente; cuanto menor sea la disparidad entre la energía y el espíritu, menos se podrá apreciar tal labor. Al parecer, la capacidad de actuación máxima de la mente se realiza en los universos temporales del espacio. Aquí la mente parece obrar en una zona intermedia entre la energía y el espíritu, pero esto no es cierto con respecto a los niveles más elevados de la mente; en el Paraíso, la energía y el espíritu son uno en esencia.
\vs p009 6:8 \pc La vía circulatoria de la gravedad mental es constante. Emana de la Tercera Persona de la Deidad del Paraíso, pero no se puede prever del todo la labor discernible de la mente. Por toda la creación conocida, en paralelismo con dicha vía, hay una presencia poco comprendida cuyo cometido no se puede prever. Creemos que esta imprevisibilidad se atribuye en parte a la acción del Absoluto Universal. No sabemos en qué consiste tal capacidad de actuación; solo podemos conjeturar lo que la hace actuar; y en cuanto a su relación con las criaturas, solamente podemos especular.
\vs p009 6:9 \pc Ciertas facetas de la imprevisibilidad de la mente finita pueden deberse a la incompletitud del Ser Supremo, y hay una inmensa zona de actividad en donde el Actor Conjunto y el Absoluto Universal puedan posiblemente ser tangenciales. Se desconoce mucho acerca de la mente, pero estamos seguros de esto: el Espíritu Infinito es la expresión perfecta de la mente del Creador para todas las criaturas y el Ser Supremo es la expresión en evolución de la mente de todas las criaturas para su Creador.
\usection{7. LA REFLECTIVIDAD DEL UNIVERSO}
\vs p009 7:1 El Actor Conjunto es capaz de coordinar todos los niveles de la realidad del universo de tal forma que posibilita la percepción simultánea de la mente, la materia y el espíritu. Este es el fenómeno de la \bibemph{reflectividad} del universo, ese poder singular e inexplicable para ver, oír, sentir y saber todas las cosas según ocurren por todo un suprauniverso y enfocar, mediante la reflectividad, toda esta información y conocimiento a cualquier punto deseado. La acción de la reflectividad se muestra a la perfección en cada uno de los mundos sedes de los siete suprauniversos. Opera también a través de todos los sectores de los suprauniversos y dentro de los confines de los universos locales. La reflectividad converge finalmente en el Paraíso.
\vs p009 7:2 El fenómeno de la reflectividad, como se desvela en las asombrosas actuaciones de los seres personales reflectores emplazados en los mundos sedes de los suprauniversos, representa la más compleja reciprocidad entre todas las facetas de la existencia que se encuentran en toda la creación. Las líneas del espíritu pueden trazarse de vuelta al Hijo; la energía física, al Paraíso; y la mente, a la Tercera Fuente. Pero en el fenómeno extraordinario de la reflectividad del universo se produce una unificación excepcional y singular de los tres que, así vinculados, permiten a los soberanos del universo, simultáneamente a su acontecimiento, conocer de forma instantánea circunstancias remotas.
\vs p009 7:3 Comprendemos bien la técnica de la reflectividad, pero muchas de sus facetas nos desconciertan verdaderamente. Sabemos que el Actor Conjunto es el centro universal de la vía circulatoria de la mente, que es el antecesor de la mente cósmica y que la mente cósmica opera bajo el dominio de la gravedad mental absoluta de la Tercera Fuente y Centro. Sabemos, además, que las vías de la mente cósmica influyen en los niveles intelectuales de toda existencia conocida; contienen los informes universales del espacio, al igual que ciertamente se centran en los siete espíritus mayores y convergen en la Tercera Fuente y Centro.
\vs p009 7:4 \pc La relación entre la mente cósmica finita y la mente absoluta divina parece estar evolucionando en la mente experiencial del Supremo. Se nos ha enseñado que, en los albores del tiempo, el Espíritu Infinito otorgó esta mente con capacidad experiencial al Supremo y suponemos que ciertos rasgos del fenómeno de la reflectividad pueden explicarse tan solo presuponiendo la actividad de la Mente Suprema. Si el Supremo no se ocupa de la reflectividad, no podemos explicar los complicados procesos e infalibles efectuaciones de esta conciencia del cosmos.
\vs p009 7:5 La reflectividad parece ser omnisciencia dentro de los límites de lo finito experiencial y puede representar la gradual aparición de la presencia\hyp{}conciencia del Ser Supremo. Si esta suposición es cierta, entonces la utilización de la reflectividad en cualquiera de sus facetas equivale a un contacto parcial con la conciencia del Supremo.
\usection{8. LOS SERES PERSONALES DEL ESPÍRITU INFINITO}
\vs p009 8:1 El Espíritu Infinito posee plena facultad para transmitir muchos de sus poderes y prerrogativas a sus seres personales colaboradores y de menor rango y a sus instancias intermedias.
\vs p009 8:2 El primer acto creativo del Espíritu Infinito como Deidad, obrando aparte de la Trinidad pero con algún vínculo no revelado con el Padre y el Hijo, se hizo personal al dar existencia a los siete espíritus mayores del Paraíso, los que distribuyen el Espíritu Infinito a los universos.
\vs p009 8:3 No hay ningún representante directo de la Tercera Fuente y Centro en la sede de un suprauniverso. Cada una de estas siete creaciones depende de uno de los siete espíritus mayores del Paraíso, que actúa a través de los siete espíritus reflectores localizados en la capital del suprauniverso.
\vs p009 8:4 El siguiente y continuado acto creativo del Espíritu Infinito se revela, ocasionalmente, al dar origen a los espíritus creativos. Cada vez que el Padre Universal y el Hijo Eterno se erigen padres de un hijo creador, el Espíritu Infinito se convierte en el progenitor de un espíritu creativo de un universo local, que se vuelve el más íntimo colaborador de ese hijo creador durante toda su posterior experiencia en el universo.
\vs p009 8:5 Así como es necesario distinguir entre el Hijo Eterno y los hijos creadores, del mismo modo hay que diferenciar entre el Espíritu Infinito y los espíritus creativos, los iguales en rango a los hijos creadores en el universo local. Lo que el Espíritu Infinito es para toda la creación, un espíritu creativo es para un universo local.
\vs p009 8:6 \pc La Tercera Fuente y Centro está representada en el gran universo por un inmenso conjunto de espíritus servidores, mensajeros, maestros, jueces, ayudantes y asesores, así como también de supervisores de ciertas vías circulatorias de naturaleza física, morontial y espiritual. No todos estos seres son personales en el sentido estricto del término. En las criaturas finitas, el ser personal se caracteriza por:
\vs p009 8:7 \li{1.}Conciencia subjetiva de sí mismo.
\vs p009 8:8 \li{2.}Respuesta objetiva a la vía del Padre por donde circula el ser personal.
\vs p009 8:9 \pc Existen seres personales creadores y seres personales creaturales, y además de estos dos tipos fundamentales existen \bibemph{seres personales de la Tercera Fuente y Centro,} seres que son personales para el Espíritu Infinito, pero no absolutamente personales para los seres creaturales. Estos seres personales de la Tercera Fuente no forman parte de la vía circulatoria del ser personal del Padre. El ser personal de la Primera Fuente y el ser personal de la Tercera Fuente pueden comunicarse entre sí; todo ser personal puede contactarse.
\vs p009 8:10 \pc El Padre otorga el ser personal según su voluntad libre y personal. Solo podemos conjeturar por qué lo hace; no sabemos cómo lo hace. Tampoco sabemos por qué la Tercera Fuente otorga a su vez un ser personal no relacionado con el que el Padre otorga, pero el Espíritu Infinito lo hace en su propio nombre, en conjunción creativa con el Hijo Eterno y de muchas maneras desconocidas para vosotros. El Espíritu Infinito también puede actuar para el Padre en la concesión del ser personal que hace la Primera Fuente.
\vs p009 8:11 \pc Existen numerosos tipos de seres personales de la Tercera Fuente. El Espíritu Infinito otorga el ser personal de la Tercera Fuente a numerosos grupos que no están incluidos en la vía del ser personal del Padre, tales como ciertos directores de la potencia. Asimismo el Espíritu Infinito considera como seres personales a numerosos grupos de seres, como los espíritus creativos, que componen una clase por sí mismos en sus relaciones con las criaturas encauzadas al Padre.
\vs p009 8:12 Tanto los seres personales de la Primera Fuente como los de la Tercera Fuente están dotados de todo e incluso más de lo que el hombre asocia con el concepto de ser personal: tienen mentes que incluyen memoria, razón, juicio, imaginación creativa, asociación de ideas, decisión, elección y numerosas otras facultades intelectuales por completo desconocidas para los mortales. Con pocas excepciones, las clases que se os han revelado poseen forma e individualidad definidas; son seres reales. Una mayoría de ellos son visibles para todos los órdenes de existencia espiritual.
\vs p009 8:13 Incluso podréis ver a vuestros acompañantes espirituales de los órdenes menores tan pronto como se os libere de los suprauniversos.
\vs p009 8:14 \pc \bibemph{La familia efectiva de la Tercera Fuente y Centro,} tal como está revelada en estas narraciones, se divide en tres grandes grupos:
\vs p009 8:15 \pc I. \bibemph{Los espíritus supremos}. Un grupo de origen compuesto que abarca, entre otros, a los órdenes siguientes:
\vs p009 8:16 \li{1.}Los siete espíritus mayores del Paraíso.
\vs p009 8:17 \li{2.}Los espíritus reflectores de los suprauniversos.
\vs p009 8:18 \li{3.}Los espíritus creativos de los universos locales.
\vs p009 8:19 \pc II. \bibemph{Los directores de la potencia}. Un grupo de criaturas e instancias intermedias que actúan en todo el espacio organizado.
\vs p009 8:20 \pc III. \bibemph{Los seres personales del Espíritu Infinito}. Esta designación no implica necesariamente que estos seres personales procedan de la Tercera Fuente, aunque algunos de ellos son únicos entre las criaturas volitivas. Normalmente se agrupan en tres clasificaciones principales:
\vs p009 8:21 \li{1.}Los seres personales superiores del Espíritu Infinito.
\vs p009 8:22 \li{2.}Las multitudes de mensajeros del espacio.
\vs p009 8:23 \li{3.}Los espíritus servidores del tiempo.
\vs p009 8:24 Estos grupos sirven en el Paraíso, en el universo central o de residencia, en los suprauniversos y abarcan órdenes que obran en los universos locales, llegando incluso hasta las constelaciones, sistemas y planetas.
\vs p009 8:25 Los seres personales espirituales de la inmensa familia del Espíritu Divino e Infinito están dedicados, por siempre, al servicio del ministerio del amor de Dios y de la misericordia del Hijo para con todas las criaturas inteligentes de los mundos evolutivos del tiempo y del espacio. Estos seres espirituales constituyen la escala viva por la cual el hombre mortal puede ascender del caos a la gloria.
\vsetoff
\vs p009 8:26 [Revelado en Urantia por un consejero divino de Uversa a quien los ancianos de días le han encargado que describa la naturaleza y la obra del Espíritu Infinito.]
