\upaper{193}{Últimas apariciones y ascensión}
\author{Comisión de seres intermedios}
\vs p193 0:1 La decimosexta manifestación morontial de Jesús se produjo el viernes, 5 de mayo, sobre las nueve de la noche, en el patio de Nicodemo. Esa noche, los creyentes de Jerusalén hicieron un primer intento por reunirse desde la resurrección. En aquel momento, estaban allí congregados los once apóstoles, el colectivo de mujeres y sus acompañantes, y en torno a otros cincuenta destacados seguidores del Maestro, incluyendo a varios griegos. Este grupo de creyentes llevaba más de media hora conversando entre ellos de manera informal cuando, de pronto, el Maestro morontial apareció por completo ante su vista y, de inmediato, comenzó a instruirlos, diciéndoles:
\vs p193 0:2 \pc “La paz sea con vosotros. Sois el grupo más representativo de creyentes ---apóstoles y discípulos, tanto hombres como mujeres--- al que yo me he aparecido desde el momento de mi liberación de la carne. Os hago ahora este llamamiento para que seáis testigos de que, como os dije por anticipado, mi estancia entre vosotros habría de llegar a su fin y os comuniqué que en algún momento debía volver al Padre. Y luego os dije con claridad cómo los principales sacerdotes y los líderes de los judíos me entregarían para darme muerte, y que resucitaría de la sepultura. ¿Por qué, entonces, estáis tan desconcertados ante todo lo que ha acontecido? ¿Por qué os sorprendisteis tanto cuando me levanté de la tumba al tercer día? No llegasteis a creer en mí, porque oísteis mis palabras sin entender su significado.
\vs p193 0:3 “Y ahora debéis prestar oído a mis palabras, no sea que cometáis de nuevo el error de oír mis enseñanzas con la mente, mientras que, en vuestros corazones, no lográis comprender su significado. Desde que comencé mi estancia como uno de vosotros, os enseñé que mi único propósito era revelar a mi Padre de los cielos a sus hijos de la tierra. Os he revelado a Dios en mi ministerio para que podáis percibirlo en vuestras propias vidas. He revelado a Dios, como vuestro Padre de los cielos; os he revelado que sois hijos de Dios en la tierra. Es un hecho que Dios os ama a vosotros, sus hijos. Por medio de la fe en mi palabra, este hecho se convierte en una verdad eterna y viva en vuestros corazones. Cuando, por medio de la fe viva, os hacéis divinamente conscientes de Dios, nacéis entonces del espíritu como hijos de la luz y de la vida, la vida eterna con la que ascenderéis al universo de los universos y tendréis la experiencia de encontrar a Dios Padre en el Paraíso.
\vs p193 0:4 “Os insto a que recordéis siempre que vuestra misión entre los hombres es proclamar el evangelio del reino ---la realidad de la paternidad de Dios y la verdad de la filiación del hombre---. Proclamad la plena verdad de la buena nueva y no solo una parte del evangelio salvífico. Vuestro mensaje no cambia por el hecho de mi resurrección. La filiación con Dios por la fe sigue siendo la verdad salvadora del evangelio del reino. Debéis salir a predicar el amor de Dios y el servicio al hombre. Sobre todo, lo que el mundo necesita saber es que los hombres son hijos de Dios y que, mediante la fe, pueden realmente ser conscientes de esta verdad ennoblecedora y vivirla diariamente. Mi ministerio de gracia debería ayudar a todos los hombres a saber en efecto que son hijos de Dios, pero dicho conocimiento no basta si no logran comprender, personalmente, por la fe, la verdad salvadora, de que son los hijos espirituales vivos del Padre eterno. El evangelio del reino expresa el amor del Padre y el servicio a sus hijos de la tierra.
\vs p193 0:5 “Todos vosotros aquí conocéis que yo he resucitado de entre los muertos, pero eso no es nada extraño. Tengo el poder de dar mi vida y de tomarla de nuevo; el Padre concede este poder a sus Hijos del Paraíso. Vuestros corazones deberían conmoverse más al saber que los muertos de una era emprendieron el ascenso eterno poco después de que yo dejara la tumba nueva de José. Viví mi vida en la carne para mostraros cómo vosotros, mediante el servicio amoroso, podéis revelar a Dios a vuestros semejantes, al igual que yo, al amaros y serviros, os he revelado a Dios a vosotros. He vivido entre vosotros como el Hijo del Hombre para que vosotros, y todos los demás hombres, podáis conocer que sois ciertamente hijos de Dios. Así pues, id ahora al mundo y predicad este evangelio del reino de los cielos a los hombres. Amad a todos los hombres tal como yo os he amado; servid a vuestros semejantes mortales tal como yo os he servido. De gracia habéis recibido, dad de gracia. Quedaos aquí en Jerusalén mientras que yo voy al Padre y os envíe el espíritu de la verdad. Él os guiará a una verdad engrandecida, y yo iré con vosotros por todo el mundo. Siempre estoy con vosotros, y mi paz os dejo”.
\vs p193 0:6 \pc Y, tras decir estas cosas, el Maestro desapareció de su vista. Estaba a punto de amanecer cuando estos creyentes se dispersaron; habían permanecido toda la noche juntos, comentando encarecidamente los mandatos del Maestro y reflexionando sobre todo lo que les había sobrevenido. Santiago Zebedeo y algunos otros de los apóstoles también contaron sus experiencias con el Maestro morontial en Galilea y relataron cómo se les había aparecido tres veces.
\usection{1. LA APARICIÓN EN SICAR}
\vs p193 1:1 En torno a las cuatro de la tarde del sábado, 13 de mayo, el Maestro se apareció a Nalda y a unos setenta y cinco creyentes samaritanos junto al pozo de Jacob en Sicar. Los creyentes acostumbraban a reunirse allí, próximo al lugar donde Jesús le había hablado a Nalda del agua viva. Este día, justo en el momento en el que estaban acabando de hablar de la resurrección, Jesús se apareció de repente a ellos y les dijo:
\vs p193 1:2 \pc “La paz esté con vosotros. Os alegra saber que yo soy la resurrección y la vida, pero este conocimiento no os servirá de nada a no ser que nazcáis primeramente del espíritu eterno, y poseáis, pues, por medio de la fe, el don de la vida eterna. Si sois hijos de mi Padre por la fe, no moriréis jamás, no pereceréis. El evangelio del reino os ha enseñando que todos los hombres son hijos de Dios. Y esta buena nueva sobre el amor del Padre celestial por sus criaturas en la tierra ha de llevarse a todo el mundo. Ha llegado el momento de que no adoréis a Dios en Gerizim ni en Jerusalén, sino donde estéis, tal como sois, en espíritu y en verdad. Es vuestra fe la que salva vuestras almas. La salvación es la dádiva de Dios para todos los que creen que son sus hijos. Pero no os engañéis; aunque la salvación sea un don gratuito de Dios y se confiera a los que la aceptan por la fe, a esto le sigue el hecho de rendir los frutos de esta vida espiritual tal como se vive en la carne. La aceptación de la doctrina de la paternidad de Dios supone que también aceptáis incondicionalmente la verdad afín de la hermandad del hombre. Si el hombre es vuestro hermano, él es incluso más que vuestro prójimo, a quien el Padre requiere que améis como a vosotros mismos. A vuestro hermano, siendo de vuestra propia familia, no solo lo amaréis con un afecto familiar, sino que también lo serviréis como yo os serví a vosotros mismos. Y amaréis y serviréis de ese modo a vuestro hermano, porque yo, siendo vosotros mis hermanos, os he amado y servido a vosotros de igual manera. Id, entonces, por todo el mundo, presentando esta buena nueva a todas las criaturas de cualquier raza, tribu y nación. Mi espíritu irá delante, y yo estaré siempre con vosotros”.
\vs p193 1:3 \pc Estos samaritanos se llenaron de asombro ante la aparición del Maestro, y se apresuraron a ir a los pueblos y aldeas cercanas, publicando a toda la gente la noticia de que habían visto a Jesús, y de que les había hablado. Y aquella fue la decimoséptima aparición morontial del Maestro.
\usection{2. APARICIÓN EN FENICIA}
\vs p193 2:1 La decimoctava aparición morontial del Maestro aconteció en Tiro, el martes 16 de mayo, algo antes de las nueve de la noche. De nuevo, se apareció al concluir una reunión de creyentes cuando estaban ya a punto de retirarse, y les dijo:
\vs p193 2:2 \pc “La paz esté con vosotros. Os alegra saber que el Hijo del Hombre ha resucitado de entre los muertos, porque así sabéis que vosotros y vuestros hermanos sobreviviréis también a la muerte humana. Pero esa supervivencia depende de que hayáis nacido primeramente del espíritu, y esto conlleva vuestra búsqueda de la verdad y vuestro descubrimiento de Dios. El pan y el agua de la vida tan solo se dan a quienes tienen hambre de verdad y sed de rectitud ---de Dios---. El hecho de que los muertos resuciten no constituye el evangelio del reino. Estas grandes verdades y hechos del universo guardan relación con este evangelio, en cuanto que son el resultado de creer en la buena nueva y forman parte de la posterior experiencia de aquellos que, por la fe, se convierten, en efecto y verdad, en hijos permanentes del Dios eterno. Mi Padre me envió a este mundo para proclamar a todos los hombres esta salvación por medio de la filiación. Y, por tanto, os envío para que la prediquéis. La salvación es la dádiva de Dios, pero los que nacen del espíritu empiezan de inmediato a mostrar los frutos del espíritu en su amoroso servicio a su semejante. Y los frutos del espíritu divino que se manifiestan en la vida de los mortales nacidos del espíritu y que conocen a Dios son: servicio amoroso, devoción desinteresada, lealtad valiente, equidad genuina, enaltecida honestidad, esperanza imperecedera, confianza implícita, ministerio misericordioso, inquebrantable bondad, tolerancia indulgente y perdurable paz. Si quienes profesan ser creyentes no rinden estos frutos del espíritu divino en su vida, están muertos; el espíritu de la verdad no está en ellos; son los pámpanos inservibles de una vid viva y pronto serán quitados. Mi Padre demanda de todos los hijos de la fe que rindan abundantes frutos del espíritu. Si, por consiguiente, no los rendís, él cavará alrededor de vuestras raíces y podará vuestras infructíferas ramas. Crecientemente, conforme progresáis en dirección al cielo en el reino de Dios, debéis rendir los frutos del espíritu. Podéis entrar en el reino como un niño, pero el Padre requiere de vosotros que crezcáis, mediante la gracia, a la estatura plena de un adulto espiritual. Y cuando vayáis por todos lados contando a las naciones la buena nueva de este evangelio, yo iré delante de vosotros, y mi espíritu de la verdad habitará en vuestros corazones. Mi paz os dejo”.
\vs p193 2:3 \pc Entonces, el Maestro desapareció de la vista de todos. Al día siguiente, partieron de Tiro quienes llevarían esta historia a Sidón e incluso a Antioquía y a Damasco. Jesús había estado con estos creyentes cuando vivía en la carne, y rápidamente lo reconocieron en cuanto empezó a impartirles sus enseñanzas. Aunque sus amigos no podían reconocer con facilidad su forma morontial cuando esta se hacía visible, sí identificaban su persona en cuanto les hablaba.
\usection{3. ÚLTIMA APARICIÓN EN JERUSALÉN}
\vs p193 3:1 El jueves, 18 de mayo, por la mañana temprano, Jesús hizo su última aparición en la tierra como persona morontial. Estando los once apóstoles a punto de sentarse para desayunar en el aposento alto de la casa de María Marcos, Jesús se les apareció y les dijo:
\vs p193 3:2 \pc “La paz esté con vosotros. Os he pedido que os quedéis aquí en Jerusalén hasta que yo ascienda al Padre e incluso hasta que yo os envíe el espíritu de la verdad, que pronto será derramado sobre toda carne y quien os investirá de poder desde lo alto”. Simón Zelotes interrumpió a Jesús, para preguntarle: “Entonces, Maestro, ¿restaurarás el reino, y veremos la gloria de Dios manifestada en la tierra?”. Cuando Jesús escuchó la pregunta de Simón, le respondió: “Simón, todavía te aferras a tus antiguas ideas sobre el Mesías judío y el reino material. Pero recibirás poder espiritual cuando el espíritu haya descendido sobre vosotros, y vayáis enseguida a todo el mundo a predicar este evangelio del reino. Tal como el Padre me envió al mundo, así os envío yo a vosotros. Y deseo que os améis unos a otros y que confiéis unos en otros. Judas ya no está con vosotros porque su amor languideció y porque se negó a confiar en vosotros, sus leales hermanos. ¿Es que no habéis leído en las Escrituras donde dice: ‘No es bueno que el hombre esté solo. Nadie vive para sí’? Y también allí donde está escrito: “‘el que quiere tener amigos debe mostrarse amistoso’”. Y ¿acaso no os envié a enseñar de dos en dos, para que no estuvierais solos y no cayerais en las maldades y en las penurias del aislamiento? Sabéis bien también que, cuando estaba en la carne, no me permití a mí mismo estar a solas durante largos períodos de tiempo. Desde el principio mismo de nuestra relación, tuve siempre a dos o tres de vosotros constantemente a mi lado si no muy cerca de mí, incluso cuando estaba en comunión con el Padre. Tened confianza, por tanto, unos en otros. Y esto es, sobre todo, lo más necesario, dado que en este día yo os dejo solos en el mundo. La hora ha llegado; estoy a punto de ir al Padre”.
\vs p193 3:3 \pc Cuando había dicho estas cosas, les hizo señas para que se fueran con él, y los llevó hasta el Monte de los Olivos, donde se despidió de ellos en preparación a su partida de Urantia. El trayecto al Monte de los Olivos estuvo lleno de solemnidad. Nadie dijo una sola palabra desde el momento en que salieron del aposento alto hasta que Jesús se detuvo con ellos en el Monte de los Olivos.
\usection{4. CAUSAS DE LA CAÍDA DE JUDAS}
\vs p193 4:1 En la primera parte de su mensaje de despedida a sus apóstoles, el Maestro aludió a la pérdida de Judas y mencionó la trágica suerte corrida por el traidor, antiguo compañero de ellos en la labor del reino, para advertirles seriamente de los peligros del aislamiento social y fraternal. Puede resultar de utilidad para los creyentes, de esta era y de las eras por venir, examinar brevemente las causas de la caída de Judas a la luz de los comentarios del Maestro y teniendo en cuenta la información acumulada a lo largo de todos los siglos que siguieron.
\vs p193 4:2 Al reflexionar sobre esta tragedia, creemos que Judas tomó el camino equivocado, principalmente, porque era una persona solitaria que evitaba cualquier contacto social de tipo ordinario. Insistentemente, se negaba a confiar en los apóstoles, sus compañeros, y a confraternizar de buen grado con ellos. Pero el hecho de ser un tipo de persona propensa al aislamiento no habría causado, por sí mismo, tanta malicia en Judas, si no hubiera sido porque tampoco logró crecer en el amor y en la gracia espiritual. Y, entonces, para un mayor empeoramiento de las cosas, constantemente albergaba resentimientos y le embargaban tendencias psicológicas dañinas como la venganza y un intenso y generalizado deseo de tomar “revancha” contra alguien por todas sus decepciones.
\vs p193 4:3 Esta lamentable combinación de peculiaridades individuales y tendencias mentales contribuyó a destruir a un hombre bien intencionado, que fue incapaz de dominar estas maldades por medio del amor, la fe y la confianza. El hecho de que el camino erróneo tomado por Judas no era inevitable se demuestra por los casos de Tomás y de Natanael, que también estaban embargados por esta misma clase de sentimientos de desconfianza y por un desarrollo excesivo de tendencias individualistas. Incluso Andrés y Mateo tenían muchas inclinaciones en este sentido; si bien, todos estos hombres, a medida que el tiempo pasaba, crecieron y no flaquearon en su amor por Jesús y por sus hermanos apóstoles. Maduraron en la gracia y en el conocimiento de la verdad. La confianza de estos en sus hermanos fue en incremento y, paulatinamente, desarrollaron la capacidad de depositar dicha confianza en sus compañeros. Judas se negaba insistentemente a confiar en ellos. Cuando se sentía forzado, al acumularse sus conflictos emocionales, de desahogarse, comunicándose con alguna persona, invariablemente buscaba el consejo, y recibía el insensato consuelo, de parientes suyos no espirituales o de personas poco conocidas, que eran indiferentes, o realmente hostiles, al bien y al avance de las realidades espirituales del reino celestial, del que él era uno de los doce consagrados embajadores en la tierra.
\vs p193 4:4 Judas cayó derrotado en las contiendas a las que se enfrentó en la tierra debido a las siguientes tendencias personales y debilidades de carácter:
\vs p193 4:5 \li{1.}Era una clase de ser humano que tendía al aislamiento. Era individualista en extremo y optó por convertirse en una persona empedernidamente “encerrada en sí misma” e insociable.
\vs p193 4:6 \li{2.}De niño, había tenido una vida excesivamente fácil. La frustración lo irritaba terriblemente. Siempre esperaba ganar; era un pésimo perdedor.
\vs p193 4:7 \li{3.}No supo enfrentarse a las decepciones con filosofía. En lugar de aceptarlas como un rasgo común y habitual de la existencia humana, indefectiblemente culpaba a alguien en particular, o a sus compañeros como grupo, de todas sus propias dificultades y desencantos personales.
\vs p193 4:8 \li{4.}Era dado a guardar rencor; siempre consideraba la idea de la venganza.
\vs p193 4:9 \li{5.}No le gustaba enfrentarse a los hechos con franqueza; era deshonesto en su actitud hacia las situaciones de la vida.
\vs p193 4:10 \li{6.}Detestaba comentar sus problemas personales con sus compañeros más cercanos; se negaba a conversar de sus dificultades con sus verdaderos amigos y con quienes realmente lo amaban. En todos los años en los que se relacionó con el Maestro, ni una sola vez acudió a él para exponerle algún problema de orden puramente personal.
\vs p193 4:11 \li{7.}No aprendió nunca que las verdaderas recompensas de llevar una vida noble consisten, al fin y al cabo, en unos premios espirituales, que no siempre se distribuyen durante esta corta vida en la carne.
\vs p193 4:12 \pc Como consecuencia del continuo aislamiento de su persona, su aflicción se multiplicó, su pesar creció, su ansiedad aumentó y su desesperación se hizo más profunda hasta casi llegar a convertirse en algo insoportable.
\vs p193 4:13 Aunque este apóstol egocéntrico y excesivamente individualista tenía muchos problemas psíquicos, emocionales y espirituales, sus dificultades principales eran: como persona, su aislamiento; en cuanto a actitud mental, su desconfianza y carácter vengativo; en cuanto a temperamento, era huraño y rencoroso; emocionalmente, estaba carente de amor y era inclemente; socialmente, desconfiaba de todos y era casi por completo autosuficiente; en espíritu, se volvió arrogante y egoístamente ambicioso; en la vida, ignoró a los que lo amaban y, en la muerte, no tuvo amigos.
\vs p193 4:14 Estas son, entonces, las características mentales y las malas influencias a las que estaba Judas expuesto, las cuales, tomadas todas juntas, explican por qué alguien que en antaño había sido un creyente de Jesús, bien intencionado y sincero, a pesar de haberse llevado años en estrecha relación con la persona transformadora de Jesús, abandonó a sus hermanos, repudió una causa sagrada, renunció a un llamamiento sagrado y traicionó a su divino Maestro.
\usection{5. LA ASCENSIÓN DEL MAESTRO}
\vs p193 5:1 Eran casi las siete y media de la mañana de ese jueves, 18 de mayo, cuando Jesús llegó a la ladera occidental del Monte de los Olivos con sus once apóstoles, que iban silenciosos y sintiéndose algo desconcertados. Desde allí, habiendo recorrido ya dos tercios del camino hasta ascender a la cumbre de la montaña, podían divisar Jerusalén y, mirando hacia abajo, Getsemaní. Jesús se preparaba ahora para decirles su último adiós a los apóstoles, antes de partir de Urantia. Estando de pie ante ellos, y sin que él se los pidiera, todos se arrodillaron formando un círculo en torno a él. Entonces, el Maestro dijo:
\vs p193 5:2 \pc “Os pedí que os quedarais en Jerusalén hasta que se os dotara de poder de lo alto. Mi despedida está cerca; estoy a punto de ascender a mi Padre, y pronto, muy pronto, enviaremos al espíritu de la verdad a este mundo en el que he vivido; y cuando él haya llegado, emprenderéis la nueva proclamación del evangelio del reino, primero en Jerusalén y, luego, hasta lo último de la tierra. Amad a los hombres con el amor con el que yo os he amado a vosotros y servid a vuestros semejantes mortales tal como yo os he servido. Manifestando los frutos espirituales en vuestras vidas, alentaréis a las almas a creer en la verdad de que el hombre es un hijo de Dios, y de que todos los hombres son hermanos. Recordad todo lo que yo os he enseñado y la vida que he vivido entre vosotros. Mi amor os cubre con su sombra, mi espíritu habitará en vosotros y mi paz permanecerá en vosotros. Adiós”.
\vs p193 5:3 \pc Y habiendo dicho estas cosas, el Maestro morontial desapareció de su vista. Esta llamada ascensión de Jesús no fue, de manera alguna, diferente de sus otras desapariciones ante la mirada humana acaecidas durante los cuarenta días que duró su andadura morontial en Urantia.
\vs p193 5:4 El Maestro fue a Edentia vía Jerusem, donde los altísimos, bajo la observación del Hijo del Paraíso, liberaron a Jesús de Nazaret del estado morontial y, por medio de los canales espirituales de la ascensión, lo retornaron al estatus de Hijo del Paraíso y a la soberanía suprema de Lugar de Salvación.
\vs p193 5:5 Eran sobre las siete y cuarenta y cinco de esa misma mañana cuando el Jesús morontial desapareció de la mirada de los once apóstoles y empezó su ascensión a la diestra de su Padre, para recibir formalmente la confirmación de haber conseguido en su plenitud la soberanía del universo de Nebadón.
\usection{6. PEDRO CONVOCA A UNA REUNIÓN}
\vs p193 6:1 Siguiendo las instrucciones de Pedro, Juan Marcos y algunos otros salieron y convocaron a los discípulos más destacados a una reunión que tendría lugar en la casa de María Marcos. Hacia las diez y media, ciento veinte de los principales discípulos de Jesús, residentes en Jerusalén, se habían congregado para oír el informe respecto al mensaje de despedida del Maestro y saber de su ascensión. En este grupo, estaba María, la madre de Jesús. María había regresado a Jerusalén con Juan Zebedeo, cuando los apóstoles volvieron de su reciente estancia en Galilea. Poco después de Pentecostés, ella regresó a la casa de Salomé en Betsaida. Santiago, el hermano de Jesús, también asistió a lo que sería el primer encuentro que se celebraba entre los discípulos del Maestro tras el fin de su andadura planetaria.
\vs p193 6:2 Simón Pedro se encargó de hablar en nombre de sus compañeros apóstoles, y realizó un emotivo informe de la última vez que los once habían estado todos juntos con su Maestro, describiendo de la manera más entrañable su adiós definitivo y su desaparición y ascenso. Aquella fue una reunión jamás ocurrida antes en este mundo. Esta parte de ella, no duró más de una hora. Luego, Pedro explicó que habían decidido elegir a un sucesor de Judas Iscariote, y que habría un receso para que los apóstoles optaran por uno de los dos hombres propuestos para el cargo: Matías y Justo.
\vs p193 6:3 Los once apóstoles entonces fueron abajo, donde acordaron echar suertes sobre cuál de estos hombres se convertiría en apóstol y serviría en el lugar de Judas. La suerte cayó en Matías, y fue contado como nuevo apóstol; se le incorporó debidamente al cargo y, luego, se le nombró tesorero. Pero fue escasa la participación de Matías en la actividad venidera de los apóstoles.
\vs p193 6:4 \pc Poco después de Pentecostés, los gemelos volvieron a sus casas en Galilea. Simón Zelotes se retiró durante algún tiempo antes de salir a predicar el evangelio. Tomás continuó con sus preocupaciones por un período de tiempo más breve y, entonces, prosiguió con sus enseñanzas. Las discrepancias entre Natanael y Pedro se hicieron más profundas respecto al hecho de predicar sobre Jesús en lugar de proclamar el auténtico evangelio del reino. Este desacuerdo se agravó tanto a mediados del siguiente mes, que Natanael se retiró y se marchó a Filadelfia para ver a Abner y Lázaro. Tras permanecer allí durante más de un año, se encaminó hasta las tierras más allá de Mesopotamia para predicar el evangelio tal como él lo entendía.
\vs p193 6:5 Así pues, de los originarios doce apóstoles, solo seis se convertirían en los actores de la temprana proclamación del evangelio en Jerusalén: Pedro, Andrés, Santiago, Juan, Felipe y Mateo.
\vs p193 6:6 \pc En torno al mediodía, los apóstoles regresaron con sus hermanos al aposento alto y anunciaron la elección de Matías como nuevo apóstol. Después, Pedro llamó a los creyentes a la oración, y oraron a fin de estar convenientemente preparados para recibir el don del espíritu que el Maestro había prometido enviar.
