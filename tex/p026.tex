\upaper{26}{Los espíritus servidores del universo central}
\author{Perfeccionador de la sabiduría}
\vs p026 0:1 Los supernafines son los espíritus servidores del Paraíso y del universo central. Conforman el orden más elevado del grupo más modesto de los hijos del Espíritu Infinito ---las multitudes angélicas---. Estos espíritus servidores se hallan desde la Isla del Paraíso hasta los mundos del espacio y del tiempo. No hay ninguna parte importante de la creación organizada y habitada que esté privada de sus servicios.
\usection{1. LOS ESPÍRITUS SERVIDORES}
\vs p026 1:1 Los ángeles son los allegados espirituales y los servidores de las criaturas de voluntad evolutivas y ascendentes de todo el espacio. Son también los compañeros y colaboradores de las multitudes superiores de seres personales divinos de las esferas. Los ángeles de todos los órdenes son seres personales diferentes y altamente individualizados. Todos tienen una gran capacidad para apreciar la ayuda de los directores de reversión. Junto con las multitudes de mensajeros del espacio, estos espíritus servidores disfrutan de temporadas de descanso y de cambio. Son de naturaleza muy sociable y poseen una capacidad de relacionarse con otros seres que sobrepasa con creces la de los seres humanos.
\vs p026 1:2 \pc Los espíritus servidores del gran universo se clasifican de la siguiente manera:
\vs p026 1:3 \li{1.}Supernafines.
\vs p026 1:4 \li{2.}Seconafines.
\vs p026 1:5 \li{3.}Terciafines.
\vs p026 1:6 \li{4.}Omniafines.
\vs p026 1:7 \li{5.}Serafines.
\vs p026 1:8 \li{6.}Querubines y sanobines.
\vs p026 1:9 \li{7.}Criaturas intermedias.
\vs p026 1:10 \pc Individualmente, los miembros de los órdenes angélicos no permanecen del todo sin cambio en lo que respecta a su estatus personal en el universo. Los ángeles de ciertos órdenes se pueden convertir en acompañantes del Paraíso durante un período de tiempo; algunos se convierten en archivistas celestiales; otros ascienden hasta la posición de asesores técnicos. Algunos querubines pueden aspirar a la condición y al destino seráfico, mientras que los serafines evolutivos pueden alcanzar los niveles espirituales de los hijos ascendentes de Dios.
\vs p026 1:11 \pc Los siete órdenes de espíritus servidores, según se revelan, se presentan agrupados de acuerdo con la importancia de las tareas que realizan con las criaturas ascendentes:
\vs p026 1:12 \li{1.}\bibemph{Los espíritus servidores del universo central}. Los tres órdenes de \bibemph{supernafines} sirven en el sistema Paraíso\hyp{}Havona. Los serafines del Paraíso o primarios son creación del Espíritu Infinito, mientras que los de orden secundario y terciario, que sirven en Havona, son respectivamente vástagos de los espíritus mayores y de los espíritus de las vías circulatorias.
\vs p026 1:13 \li{2.}\bibemph{Los espíritus servidores de los suprauniversos:} los seconafines, los terciafines y los omniafines. Los \bibemph{seconafines,} hijos de los espíritus reflectores, desempeñan su actividad en los siete suprauniversos de forma diversa. Los \bibemph{terciafines,} de origen en el Espíritu Infinito, se dedican con el tiempo a servir de enlace entre los hijos creadores y los ancianos de días. Los \bibemph{omniafines,} creación conjunta del Espíritu Infinito y de los siete mandatarios supremos, sirven exclusivamente a estos últimos. Estos tres órdenes se tratarán a continuación siguiendo el orden de esta narrativa.
\vs p026 1:14 \li{3.}\bibemph{Los espíritus servidores de los universos locales} engloban a los \bibemph{serafines} y a sus asistentes, los \bibemph{querubines.} En su andadura como ascendentes, los mortales se encuentran primeramente con estos vástagos del espíritu materno del universo. Las \bibemph{criaturas intermedias,} de origen en los mundos habitados, no son en realidad órdenes angélicas propiamente dichas; si bien, con frecuencia, se les agrupa, por motivos prácticos, con los espíritus servidores. El relato de estos, junto con la narrativa sobre los serafines y los querubines, se presenta en los escritos que tratan de vuestro universo local.
\vs p026 1:15 \pc Todos los órdenes de multitudes angélicas se dedican a la realización de las diversas tareas del universo y asisten, de una manera u otra, a órdenes más elevados de seres celestiales; no obstante, son los supernafines, los seconafines y los serafines los que, en gran número, se encargan de impulsar el plan de acción diseñado para que los hijos del tiempo asciendan y progresen en el camino de la perfección. Estos ángeles obran en el universo central, en los suprauniversos y los universos locales; forman una cadena ininterrumpida de servidores espirituales que el Espíritu Infinito proporciona para ayudar y guiar a todos los que tratan de llegar al Padre Universal a través del Hijo Eterno.
\vs p026 1:16 Los supernafines están limitados en “polaridad espiritual” con respecto solamente a un aspecto de su curso de acción: aquel relacionado con el Padre Universal. Pueden trabajar solos excepto cuando emplean de forma directa las vías circulatorias exclusivas del Padre. Cuando reciben poder por ministerio directo del Padre, los supernafines deben unirse voluntariamente en parejas para poder actuar. Los seconafines están igualmente limitados y deben, además, trabajar en parejas con el fin de sincronizarse con las vías circulatorias del Hijo Eterno. Los serafines pueden trabajar solos como seres personales individuales y localizados, pero tan solo pueden encauzarse en las vías circulatorias cuando están polarizados como parejas de enlace. Cuando estos seres espirituales se vinculan en parejas, se entiende que el uno es complementario del otro. Estas relaciones de complementariedad pueden ser transitorias; no tienen por qué ser necesariamente permanentes.
\vs p026 1:17 Estas brillantes criaturas de luz se sustentan directamente mediante la ingestión de la energía espiritual de las vías primarias del universo. Los mortales de Urantia han de obtener energía\hyp{}luz por medio de sustancias vegetativas, pero las multitudes angélicas tienen acceso a las vías circulatorias; tienen “una comida que comer, que vosotros no sabéis”. También son partícipes de las enseñanzas en circulación de los maravillosos hijos preceptores de la Trinidad; adquieren conocimiento y absorben sabiduría de forma muy semejante a como asimilan las energías vitales.
\usection{2. LOS PODEROSOS SUPERNAFINES}
\vs p026 2:1 Los supernafines son diestros servidores de todos los tipos de seres que tienen su residencia en el Paraíso y en el universo central. Estos elevados ángeles se crean según sus tres órdenes principales: orden primario, secundario y terciario.
\vs p026 2:2 \pc \bibemph{Los supernafines primarios} son vástagos exclusivos del Creador Conjunto. Dividen su ministerio de forma prácticamente equivalente entre ciertos grupos de ciudadanos del Paraíso y el colectivo, en constante aumento, de peregrinos ascendentes. Estos ángeles de la Isla eterna son muy eficaces a la hora de impulsar la formación esencial de ambos grupos de habitantes del Paraíso. Hacen una gran aportación al entendimiento mutuo de estos dos órdenes singulares de criaturas del universo: el uno del tipo más elevado de criatura de voluntad divina y perfecta y, el otro, el que proviene de la evolución perfeccionada, del tipo más modesto de criatura de voluntad de todo el universo de los universos.
\vs p026 2:3 \pc La labor de los supernafines primarios es tan excepcional y característica que se estudiará por separado en la narrativa siguiente.
\vs p026 2:4 \pc \bibemph{Los supernafines secundarios} están a cargo de los asuntos de los seres ascendentes de las siete vías planetarias de Havona. Se preocupan igualmente de asistir en la formación educativa de los numerosos órdenes de ciudadanos del Paraíso que residen durante largos períodos en las vías circulatorias de los mundos de la creación central, pero no podemos comentar esta faceta de su servicio.
\vs p026 2:5 \pc Hay siete tipos de estos elevados ángeles. Cada uno de ellos tiene su origen en uno de los siete espíritus mayores y, en consecuencia, su naturaleza se conforma a la de estos. Los siete espíritus mayores crean colectivamente muchos grupos diferentes de seres y de entidades singulares y los miembros individuales de cada uno de estos órdenes son de naturaleza relativamente uniforme. No obstante, cuando estos mismos siete espíritus crean de forma individual, los órdenes resultantes son siempre de naturaleza séptupla; los hijos de cada espíritu mayor son partícipes de la naturaleza de su creador y, por consiguiente, son distintos de los demás. Tal es el origen de los serafines secundarios, y los ángeles de todos estos siete tipos así creados actúan en todos los ámbitos de actividad accesibles a la totalidad de su orden, principalmente en las siete vías circulatorias del universo central y divino.
\vs p026 2:6 \pc Cada una de las siete vías planetarias de Havona está bajo la supervisión directa de uno de los siete espíritus de las vías, siendo ellos mismos creación colectiva ---y por consiguiente uniforme--- de los siete espíritus mayores. Aunque son partícipes de la naturaleza de la Tercera Fuente y Centro, estos siete espíritus auxiliares de Havona no eran parte del universo en su modelo original. Comenzaron a ejercer su actividad tras la creación primigenia (eterna) pero mucho antes de los tiempos de Granfanda. No hay duda de que su aparición se debió a una respuesta creativa por parte de los espíritus mayores al propósito del Ser Supremo, y se descubrió su actividad en el momento de la organización del gran universo. El Espíritu Infinito y todos sus colaboradores creativos, como coordinadores universales, parecen estar abundante y capazmente dotados para dar respuestas creativas apropiadas al despliegue simultáneo de las Deidades experienciales y de los universos en evolución.
\vs p026 2:7 \pc \bibemph{Los supernafines terciarios} tienen su origen en estos siete espíritus de las vías circulatorias. El Espíritu Infinito los ha investido de poder para crear, en las diferentes vías circulatorias de Havona, un número suficiente de elevados servidores superáficos del orden terciario con el fin de satisfacer las necesidades del universo central. Aunque los espíritus de las vías generaron un número relativamente pequeño de estos servidores angélicos antes de la llegada a Havona de los peregrinos del tiempo, los siete espíritus mayores ni siquiera comenzaron la creación de los supernafines secundarios hasta que Granfanda arribó. Así pues, al ser los supernafines terciarios los más antiguos de los dos órdenes, les damos prioridad en nuestros comentarios.
\usection{3. LOS SUPERNAFINES TERCIARIOS}
\vs p026 3:1 Estos servidores de los siete espíritus mayores son los especialistas angélicos de las distintas vías circulatorias de Havona, y su ministerio se hace extensible tanto a los peregrinos ascendentes del tiempo como a los peregrinos descendentes de la eternidad. En los mil millones de mundos de estudio de la perfecta creación central, vuestros colaboradores superáficos de todos los órdenes distintos os resultarán completamente visibles. Allí todos os relacionaréis, en el sentido más elevado, como seres mutuamente fraternales y comprensivos. También reconoceréis totalmente a los ciudadanos del Paraíso y fraternizaréis perfectamente con ellos. Estos peregrinos descendentes recorren las vías circulatorias desde el interior hacia el exterior, accediendo a Havona a través del mundo piloto de la primera vía y avanzando hacia el exterior, hasta la séptima.
\vs p026 3:2 Los peregrinos ascendentes de los siete suprauniversos pasan por Havona en el sentido opuesto: acceden por el mundo piloto de la vía séptima y avanzan hacia el interior. No hay límite de tiempo establecido que rija el progreso de las criaturas ascendentes de mundo en mundo y de vía en vía, así como tampoco se establece de forma arbitraria ningún periodo de tiempo fijo para residir en los mundos morontiales. Si bien, aunque algunos seres que han evolucionado de forma satisfactoria pueden quedar exentos de morar en uno o más de los mundos de formación del universo local, ningún peregrino podrá evitar pasar por las siete vías de Havona en las que ha de avanzar espiritualmente.
\vs p026 3:3 \pc Ese colectivo de supernafines terciarios asignado mayormente al servicio de los peregrinos del tiempo se clasifica de la manera siguiente:
\vs p026 3:4 \li{1.}\bibemph{Los supervisores de la armonía}. Resulta evidente que algún tipo de coordinación se hace necesaria, incluso en la perfecta Havona, para mantener un orden y asegurar la armonía en toda la labor de preparación de los peregrinos del tiempo para sus posteriores logros en el Paraíso. Esta es la verdadera misión de los supervisores de la armonía: cuidar de que todo marche sin complicaciones y a buen ritmo. Estos supernafines se originan en la primera vía y sirven en todo Havona. Su presencia en estas vías circulatorias indica que no es posible que pueda ir algo mal. Su gran habilidad para coordinar las diferentes tareas en las que se implican seres personales de órdenes distintos, e incluso de múltiples niveles, les facilita poder dar asistencia donde y cuando quiera que sea necesaria. Contribuyen grandemente al entendimiento mutuo de los peregrinos del tiempo y de los peregrinos de la eternidad.
\vs p026 3:5 \li{2.}\bibemph{Los archivistas jefes}. Estos ángeles se crean en la segunda vía pero operan en la totalidad del universo central. Registran la información por triplicado, preparándola para su inclusión en los archivos literales de Havona, en los archivos espirituales de su orden y en los archivos oficiales del Paraíso. Además transmiten de forma automática los acontecimientos que son de relevancia para el verdadero conocimiento a las bibliotecas vivas del Paraíso, los custodios del conocimiento del orden primario de los supernafines.
\vs p026 3:6 \li{3.}\bibemph{Los informadores}. Los hijos del tercer espíritu de las vías desempeñan su labor en todo Havona, aunque su estación oficial está localizada en el planeta número setenta, en el círculo más exterior. Estos diestros técnicos son receptores y emisores de las transmisiones de la creación central al igual que directores de los informes espaciales de todos los fenómenos relacionados con la Deidad del Paraíso. Pueden operar en todas las vías circulatorias fundamentales del espacio.
\vs p026 3:7 \li{4.}\bibemph{Los transmisores}. Tienen su origen en la vía número cuatro. Recorren el sistema Paraíso\hyp{}Havona como portadores de todos los mensajes que precisen transmisión personal. Sirven a sus semejantes, a los seres personales celestiales, a los peregrinos del Paraíso e incluso a las almas ascendentes del tiempo.
\vs p026 3:8 \li{5.}\bibemph{Los coordinadores de la información}. Estos supernafines terciarios, hijos sabios y compasivos del quinto espíritu de las vías, siempre impulsan las relaciones fraternales entre los peregrinos ascendentes y los descendentes. Sirven a todos los habitantes de Havona y, en especial, a los ascendentes, manteniéndoles informados de inmediato de los asuntos del universo de los universos. Debido a su contacto personal con los informadores y los reflectores, estos “periódicos vivos” de Havona conocen al instante cualquier información que pase por las inmensas vías circulatorias de noticias del universo central. La obtienen por medio del método gráfico de Havona, el cual les permite, de forma automática, absorber en una hora de tiempo de Urantia una información que tardaríais mil años en procesar mediante vuestra técnica telegráfica más rápida.
\vs p026 3:9 \li{6.}\bibemph{Los seres personales transportadores}. Estos seres, que tienen su origen en la vía número seis, generalmente operan desde el planeta número cuarenta de la vía más exterior. Ellos son los que se llevan a los decepcionados candidatos que fracasan de forma transitoria en la aventura de la Deidad. Están listos para asistir a todos los que deben ir y venir para prestar sus servicios en Havona, y que no son surcadores del espacio.
\vs p026 3:10 \li{7.}\bibemph{El colectivo de reserva}. Las fluctuaciones existentes en la labor con los seres ascendentes, los peregrinos del Paraíso y otros órdenes de seres que residen en Havona, obligan a mantener estas reservas de supernafines en el mundo piloto del séptimo círculo, en el que tienen su origen. Estos seres se crean sin ningún plan especial y están capacitados para prestar su servicio en las facetas menos exigentes de cualquiera de los cometidos de sus colaboradores superáficos del orden terciario.
\usection{4. LOS SUPERNAFINES SECUNDARIOS}
\vs p026 4:1 Los supernafines secundarios prestan su ayuda en las siete vías planetarias del universo central. Una parte de ellos se dedica al servicio de los peregrinos del tiempo, y la mitad de la totalidad de este orden se destina a la formación de los peregrinos del Paraíso y de la eternidad. A estos ciudadanos del Paraíso, en su peregrinaje por las vías de Havona, también los atienden los voluntarios del colectivo de los mortales finalizadores, como se dispuso y se mantiene desde que se formó el primer grupo de finalizadores.
\vs p026 4:2 \pc Según sea su asignación periódica al cuidado de los peregrinos ascendentes, los supernafines secundarios trabajan en los siete grupos siguientes:
\vs p026 4:3 \li{1.}Los ayudantes de los peregrinos.
\vs p026 4:4 \li{2.}Los guías de la supremacía.
\vs p026 4:5 \li{3.}Los guías de la Trinidad.
\vs p026 4:6 \li{4.}Los descubridores del Hijo.
\vs p026 4:7 \li{5.}Los guías del Padre.
\vs p026 4:8 \li{6.}Los consejeros y asesores.
\vs p026 4:9 \li{7.}Los acompañadores del descanso.
\vs p026 4:10 \pc En cada uno de estos grupos de trabajo se incluyen ángeles de todos los siete tipos creados, y al peregrino del espacio siempre lo instruye un supernafín secundario originado en el espíritu mayor que preside el suprauniverso en el que nació. Cuando vosotros los mortales de Urantia logréis llegar a Havona, os guiarán sin duda supernafines cuya naturaleza ---como vuestra propia naturaleza evolutiva--- se deriva del espíritu mayor de Orvontón. Y puesto que vuestros tutores surgen del espíritu mayor de vuestro propio suprauniverso, están particularmente capacitados para comprenderos, confortaros y asistiros en todos vuestros afanes por alcanzar la perfección del Paraíso.
\vs p026 4:11 Los seres personales transportadores del orden primario de los seconafines, operando desde las sedes de los siete suprauniversos, trasladan a los peregrinos del tiempo más allá de los cuerpos oscuros de gravedad de Havona hasta la vía planetaria exterior. Aunque no todos, la mayoría de los serafines que prestan sus servicio en los planetas y en los universos locales y que disponen de autorización para la ascensión al Paraíso se separarán de sus compañeros mortales antes del largo vuelo a Havona y comenzarán enseguida un largo período de intensa formación con el fin de conseguir un excelso destino, esperando lograr, como serafines, perfección de vida y supremacía de servicio. Y esto lo hacen con la esperanza de reunirse con los peregrinos del tiempo, para contarse entre aquellos que siguen por siempre el camino de esos mortales que han logrado llegar al Padre Universal y cuyo destino es el servicio no revelado del colectivo final.
\vs p026 4:12 El peregrino llega al planeta receptor de Havona, al mundo piloto de la séptima vía, con un único don de perfección: perfección de propósito. El Padre Universal ha decretado: “Sed vosotros perfectos, como yo soy perfecto”. Y este asombroso mandato\hyp{}invitación se transmite a los hijos finitos de los mundos del espacio. La promulgación de dicho requerimiento ha puesto en movimiento a toda la creación en la labor conjunta de los seres celestiales por ayudar al cumplimiento y a la realización de tal extraordinario mandato de la Primera Gran Fuente y Centro.
\vs p026 4:13 Cuando, y mediante el ministerio de todas las multitudes de ayudantes involucrados en el proyecto universal de supervivencia, se os deposita finalmente en el mundo de recepción de Havona, llegáis con un solo tipo de perfección: \bibemph{perfección de propósito}. Vuestro propósito se ha verificado por completo; vuestra fe se ha comprobado. Se sabe que estáis a prueba de decepciones. Ni siquiera el hecho de no poder percibir al Padre Universal podrá sacudir la fe ni perturbar seriamente la confianza del mortal ascendente, que haya pasado por la experiencia que todos deben tener para lograr las esferas perfectas de Havona. Cuando alcancéis Havona, vuestra autenticidad se habrá vuelto sublime. La perfección de vuestro propósito y la divinidad de vuestro deseo, junto a vuestra firmeza de fe, os han asegurado vuestra entrada en las estables moradas de la eternidad; os habéis liberado plenamente y por completo de las incertidumbres del tiempo; y ahora debéis enfrentaros a los problemas que hallaréis en Havona y en las inmensidades del Paraíso, para encontraros con aquello para lo cual os habéis estado formando durante tanto tiempo en las épocas experienciales del tiempo en los mundos escuela del espacio.
\vs p026 4:14 La fe ha hecho ganar al peregrino ascendente el propósito de perfección que concede entrada a la eternidad a los hijos del tiempo. Ahora los ayudantes de los peregrinos deben comenzar su labor para desarrollar esa perfección de entendimiento y ese modo de comprensión que tan imprescindibles son para la perfección personal del Paraíso.
\vs p026 4:15 \bibemph{La capacidad de comprensión es el pase del mortal para acceder al Paraíso}. La voluntad de creer es el secreto de la entrada en  Havona. La aceptación de la filiación, la cooperación con el modelador interior, es el precio de la supervivencia evolutiva.
\usection{5. LOS AYUDANTES DE LOS PEREGRINOS}
\vs p026 5:1 El primero de los siete grupos de supernafines secundarios con los que os encontraréis es el de los ayudantes de los peregrinos, esos seres de rápido entendimiento y de gran comprensión que dan la bienvenida a los mundos estabilizados y a la organización eficiente y estable del universo central a los ascendentes del espacio, que tanto han viajado. Al mismo tiempo, estos nobles servidores comienzan su labor con los peregrinos del Paraíso y de la eternidad, el primero de los cuales llegó al mundo piloto de la vía circulatoria interior de Havona, coincidiendo con la llegada de Granfanda al mundo piloto de la vía exterior. En aquellos remotos días, los peregrinos del Paraíso y los peregrinos del tiempo se encuentran por primera vez en el mundo receptor de la vía número cuatro.
\vs p026 5:2 Estos ayudantes de los peregrinos, que actúan en el séptimo círculo de los mundos de Havona, efectúan su labor de instrucción de los mortales ascendentes atendiendo a tres divisiones principales: la primera, entendimiento supremo de la Trinidad del Paraíso; la segunda, comprensión espiritual de la vinculación Padre\hyp{}Hijo; y, la tercera, percepción intelectual del Espíritu Infinito. Cada una de estas etapas de instrucción se divide en siete ramas de doce secciones menores de setenta grupos secundarios, y cada uno de estos grupos de instrucción consta a su vez de mil clasificaciones. En los círculos posteriores se imparte una instrucción más pormenorizada, pero los ayudantes de los peregrinos imparten un esbozo de cada uno de los requisitos del Paraíso.
\vs p026 5:3 Ese es, pues, el curso básico o elemental que tienen que afrontar los peregrinos del espacio que tanto han viajado y cuya fe se ha puesto a prueba. Pero mucho antes de llegar a Havona, esos hijos ascendentes del tiempo han aprendido a celebrar la incertidumbre, a nutrirse de la decepción, a entusiasmarse ante la derrota manifiesta, a tomar fuerzas ante las dificultades, a demostrar un valor indómito ante la inmensidad y a ejercer una fe invencible ante los retos de lo inexplicable. Por mucho tiempo, el grito de batalla de estos peregrinos ha sido: “Junto a Dios, nada, absolutamente nada, es imposible”.
\vs p026 5:4 Los peregrinos del tiempo tienen un requisito claro que cumplir en cada uno de los círculos de Havona, y, aunque cada cual continúa bajo la tutela de los supernafines que, por naturaleza, se han adaptado para asistir a este tipo específico de criatura ascendente, la preparación que se debe adquirir ha de ser bastante uniforme para todos los ascendentes que llegan al universo central. Esta preparación conlleva logros de tipo cuantitativo, cualitativo y experiencial ---intelectual, espiritual y supremo---.
\vs p026 5:5 El tiempo tiene poca trascendencia en los círculos de Havona. Interviene, de forma limitada, en las posibilidades de avance, pero la prueba final y suprema es el logro conseguido. En el momento mismo en el que vuestro colaborador superáfico considere que estáis capacitados para pasar hacia el interior, al siguiente círculo, se os llevará ante los doce asistentes del séptimo espíritu de la vía circulatoria. Aquí se os pedirá que superéis las pruebas del círculo dispuestas en el suprauniverso del que sois originarios y en el sistema en el que habéis nacido. El logro divino de este círculo tiene lugar en el mundo piloto y consiste en el reconocimiento y la conciencia espirituales del espíritu mayor del suprauniverso del que procede el peregrino ascendente.
\vs p026 5:6 Cuando el trabajo del círculo exterior de Havona se completa y se domina el curso planteado, los ayudantes de los peregrinos llevan a sus tutorados al mundo piloto del círculo siguiente y los confían al cuidado de los guías de la supremacía. Estos ayudantes siempre permanecen allí durante un tiempo para asegurarse de que el traslado sea agradable y provechoso a la vez.
\usection{6. LOS GUÍAS DE LA SUPREMACÍA}
\vs p026 6:1 A los ascendentes del espacio se les denomina “graduados espirituales” cuando se les traslada del séptimo al sexto círculo y se les coloca bajo la supervisión inmediata de los guías de la supremacía. No se debe confundir a estos guías con los guías de los graduados ---pertenecientes a los seres personales superiores del Espíritu Infinito--- y que, con sus colaboradores servitales, prestan asistencia en todos los círculos de Havona tanto a los peregrinos ascendentes como a los descendentes. Los guías de la supremacía actúan tan solo en el sexto círculo del universo central.
\vs p026 6:2 Es en este círculo donde los ascendentes consiguen una nueva comprensión de la Divinidad Suprema. A lo largo de su larga andadura en los universos evolutivos, los peregrinos del tiempo han estado experimentando una creciente conciencia de la realidad de una todopoderosa potestad de las creaciones espacio\hyp{}temporales. Aquí, en esta vía de Havona, están próximos a encontrarse con la fuente de la unidad espacio\hyp{}tiempo del universo central: la realidad espiritual del Dios Supremo.
\vs p026 6:3 No sé muy bien cómo explicar lo que ocurre en este círculo. No hay ninguna presencia personal de la Supremacía que los ascendentes puedan percibir. En cierto sentido, la nueva relación con el séptimo espíritu mayor compensa esta falta de contacto con el Ser Supremo. Si bien, a pesar de nuestra incapacidad para captar lo que ocurre, las criaturas ascendentes parecen experimentar un crecimiento que los transforma, una nueva integración de su conciencia, una nueva espiritualización de su propósito, una nueva sensibilidad hacia la divinidad, que difícilmente se puede explicar de manera satisfactoria sin presuponer la actividad no revelada del Ser Supremo. Para aquellos de nosotros que han observado estos misteriosos acontecimientos, parece que el Dios Supremo estuviese confiriendo afectuosamente a sus hijos experienciales, hasta los límites mismos de la capacidad vivencial de estos, este enaltecimiento de la comprensión intelectual, de la percepción espiritual y del alcance del ser personal que tanto necesitarán en todos sus afanes por penetrar el nivel de divinidad de la Trinidad de la Supremacía, para lograr llegar a las Deidades eternas y existenciales del Paraíso.
\vs p026 6:4 Cuando los guías de la supremacía consideran que sus pupilos están maduros para avanzar, los conducen ante la comisión de los setenta, un grupo mixto que actúa como examinador en el mundo piloto de la vía número seis. Tras demostrar satisfactoriamente ante esta comisión su comprensión del Ser Supremo y de la Trinidad de la Supremacía, los peregrinos reciben autorización para trasladarse a la quinta vía.
\usection{7. LOS GUÍAS DE LA TRINIDAD}
\vs p026 7:1 Los guías de la Trinidad son los incansables servidores del quinto círculo de formación en Havona de los peregrinos del tiempo y del espacio en su camino de avance. Los graduados espirituales se denominan aquí “candidatos para la aventura de la Deidad” al ser en dicho círculo, bajo la dirección de estos guías, donde los peregrinos reciben instrucción superior sobre la Trinidad divina como preparación para poder lograr el reconocimiento de la persona del Espíritu Infinito. Aquí los peregrinos ascendentes descubren lo que significa el verdadero estudio y el auténtico esfuerzo mental cuando comienzan a vislumbrar la naturaleza del todavía más imperioso y mucho más arduo esfuerzo espiritual necesario para satisfacer las exigencias del noble objetivo que tienen que alcanzar en los mundos de esta vía.
\vs p026 7:2 Los guías de la Trinidad son fieles y eficaces en grado sumo, y cada peregrino recibe toda la atención, y disfruta de todo el afecto, de un supernafín secundario perteneciente a este orden. Jamás hallaría el peregrino del tiempo a la primera persona accesible de la Trinidad del Paraíso, si no fuese por la ayuda y asistencia de estos guías y de la multitud de otros seres espirituales que se ocupan de instruir a los seres ascendentes con respecto a la naturaleza y al modo de proceder de su próxima aventura en búsqueda de la Deidad.
\vs p026 7:3 Tras terminar el curso formativo en esta vía, los guías de la Trinidad llevan a sus pupilos a su mundo piloto y los presentan ante una de las muchas comisiones trinas que examinan y autorizan a los candidatos para la aventura de la Deidad. Estas comisiones están compuestas por un miembro de los finalizadores, uno de los directores de la conducta del orden de los supernafines primarios y un mensajero solitario del espacio o un hijo trinitizado del Paraíso.
\vs p026 7:4 Cuando el alma ascendente se pone realmente en camino hacia el Paraíso, va solamente acompañada de un trío de tránsito: el colaborador superáfico del círculo, el guía de los graduados y el siempre presente colaborador servital de este último. Estas excursiones desde los círculos de Havona al Paraíso son viajes de prueba; los ascendentes todavía no han alcanzado estatus en el Paraíso. No pueden ser residentes del Paraíso hasta no haber pasado por el descanso final del tiempo, tras su consecución del Padre Universal y haber obtenido la autorización final de las vías de Havona. No son partícipes de la “esencia de la divinidad” y del “espíritu de la supremacía” hasta el término de este descanso divino, comenzando así verdaderamente a actuar en el círculo de la eternidad y en presencia de la Trinidad.
\vs p026 7:5 \pc A los miembros del trío de tránsito que acompañan al ascendente no se les solicita que le posibiliten localizar la presencia geográfica de la luminosidad espiritual de la Trinidad, sino más bien que ofrezcan al peregrino toda la asistencia necesaria en su difícil tarea de reconocer, percibir y comprender al Espíritu Infinito, lo suficiente como para que conlleve el reconocimiento de su persona. Todo peregrino ascendente del Paraíso puede reconocer la presencia geográfica o situacional de la Trinidad, la gran mayoría puede ponerse en contacto con la realidad intelectual de las Deidades, especialmente de la Tercera Persona, pero no todos pueden percibir o siquiera comprender parcialmente la realidad de la presencia espiritual del Padre y del Hijo. Y todavía resulta más difícil alcanzar el mínimo de comprensión espiritual del Padre Universal.
\vs p026 7:6 Es raro que la búsqueda del Espíritu Infinito no llegue a consumarse, y cuando sus tutorados han tenido éxito en esta etapa de la aventura de la Deidad, los guías de la Trinidad se preparan para trasladarlos al cuidado de los descubridores del Hijo, en el cuarto círculo de Havona.
\usection{8. LOS DESCUBRIDORES DEL HIJO}
\vs p026 8:1 A veces, a la cuarta vía circulatoria de Havona se la denomina la “vía circulatorias de los hijos”. Desde los mundos de esta vía, los peregrinos ascendentes se dirigen al Paraíso para lograr aproximarse y empezar a comprender al Hijo Eterno, mientras que en los mundos de esta misma vía los peregrinos descendentes logran un nuevo entendimiento de la naturaleza y misión de los hijos creadores del tiempo y del espacio. Existen siete mundos en dicha vía en los que los colectivos de reserva de los migueles del Paraíso mantienen escuelas especiales que sirven de ayuda mutua tanto para asistir a los peregrinos ascendentes como a los descendentes; y es en estos mundos de los hijos migueles donde los peregrinos del tiempo y los peregrinos de la eternidad alcanzan por vez primera una verdadera comprensión mutua. En muchos aspectos, las vivencias que se experimentan en esta vía son las más fascinantes de toda la estancia en Havona.
\vs p026 8:2 Los descubridores del Hijo son los servidores superáficos de los mortales ascendentes de la cuarta vía. Además de la labor general de preparar a sus candidatos para percibir la relación del Hijo Eterno en la Trinidad, estos descubridores deben proporcionar a sus tutorados una instrucción tan completa como para que puedan conseguir un éxito total, en primer lugar, comprendiendo espiritualmente, de forma adecuada, al Hijo; en segundo lugar, sabiendo apreciar, de forma satisfactoria, el ser personal del Hijo; y, en tercer lugar, pudiendo diferenciar, de forma apropiada, al Hijo respecto de la persona del Espíritu Infinito.
\vs p026 8:3 No se realizan más exámenes tras llegar al Espíritu Infinito. En los círculos interiores las pruebas son en sí las actuaciones de los candidatos peregrinos cuando se envuelven en el acogimiento de las Deidades. Es la espiritualidad de la persona la que sencillamente determina su avance y nadie, a excepción de los Dioses, se atreve a juzgar lo conseguido. En caso de fracaso, no se determinan razones ni tampoco se reprende ni se critica ni a los candidatos mismos ni a sus diferentes tutores y guías. En el Paraíso, la decepción nunca se considera una derrota, el aplazamiento nunca se considera una desgracia, los fracasos aparentes del tiempo nunca se confunden con los retrasos significativos de la eternidad.
\vs p026 8:4 \pc No hay muchos peregrinos que experimenten, por un aparente fracaso, un retraso en la aventura de la Deidad. Casi todos llegan al Espíritu Infinito, aunque ocasionalmente algún peregrino del suprauniverso número uno no lo consiga al primer intento. Los peregrinos que logran alcanzar al Espíritu raras veces fracasan en hallar al Hijo. Casi todos los peregrinos que no consiguen llegar al Padre en la primera aventura provienen de los suprauniversos tres y cinco. Y una gran mayoría de los que no logran alcanzar al Padre en la primera aventura, tras hallar al Espíritu y al Hijo, provienen del suprauniverso número seis, si bien, algunos procedentes de los suprauniversos números dos y tres tampoco tienen éxito. Todo esto parece indicar claramente que existe alguna explicación razonable para estos aparentes fracasos; en realidad, sencillamente se trata de retrasos inevitables.
\vs p026 8:5 A los candidatos que han fracasado en la aventura de la Deidad se les coloca bajo la jurisdicción de los jefes por designación, un grupo de supernafines primarios, y se les envía de vuelta al trabajo de los mundos del espacio por un período de no menos de un milenio. Nunca regresan al suprauniverso en el que nacieron, sino siempre a la supracreación que les sea más propicia para poder reeducarse y prepararse para la segunda aventura de la Deidad. Tras este servicio, por iniciativa propia, regresan al círculo exterior de Havona, se les conduce de inmediato a la vía circulatoria en la que se interrumpió su andadura y reanudan enseguida sus preparativos para la aventura de la Deidad. Los supernafines secundarios nunca dejan de pilotar con éxito a sus tutorados en la segunda tentativa, y los mismos servidores superáficos, al igual que otros guías, siempre se ocupan de estos candidatos durante esta segunda aventura.
\usection{9. LOS GUÍAS DEL PADRE}
\vs p026 9:1 Cuando el alma peregrina alcanza el tercer círculo de Havona, se acoge bajo la tutela de los guías del Padre, los servidores superáficos de más antigüedad, mayor destreza y más experimentados. En los mundos de esta vía, los guías del Padre mantienen escuelas de sabiduría y facultades técnicas en las que sirven como maestros todos los seres que habitan el universo central. No se desatiende nada que pueda ser de utilidad a la criatura del tiempo en esta aventura suprema de conseguir la eternidad.
\vs p026 9:2 Llegar hasta el Padre Universal constituye el pase para la eternidad, a pesar de las vías que aún puedan quedar por recorrer. Por ello, representa un gran acontecimiento para el mundo piloto del círculo número tres cuando el trío de tránsito anuncia que la última aventura del tiempo está por comenzar, que otra criatura del espacio aspira a entrar en el Paraíso por las puertas de la eternidad.
\vs p026 9:3 \pc La prueba del tiempo casi ha concluido; la carrera por la eternidad está a punto de finalizar. Los días de incertidumbre se están terminando; la tentación de la duda se desvanece; el mandato a ser \bibemph{perfecto} se ha obedecido. Desde el fondo mismo de la existencia inteligente, la criatura del tiempo y del ser personal material ha escalado las esferas evolutivas del espacio, confirmando así la viabilidad del plan de ascensión, demostrando, al mismo tiempo y para siempre, la justicia y rectitud del mandato dado por el Padre Universal a sus humildes criaturas de los mundos: “Sed vosotros perfectos, como yo soy perfecto”.
\vs p026 9:4 Paso a paso, vida a vida, mundo a mundo, se ha superado la andadura ascendente, se ha alcanzado la meta de la Deidad. La supervivencia es completa en perfección y la perfección abunda en la supremacía de la divinidad. El tiempo se pierde en la eternidad; el espacio se absorbe en identidad y armonía reverente con el Padre Universal. Las transmisiones de Havona emiten al espacio mensajes de gloria, dando la buena nueva de que en verdad las diligentes criaturas de naturaleza animal y de origen material se han convertido real y eternamente, como resultado de su ascensión evolutiva, en hijos perfeccionados de Dios.
\usection{10. LOS CONSEJEROS Y ASESORES}
\vs p026 10:1 Los consejeros y asesores superáficos del segundo círculo son los instructores de los hijos del tiempo en cuanto a su andadura en la eternidad. Lograr el Paraíso conlleva responsabilidades de un orden nuevo y más elevado, y la estancia en el segundo círculo ofrece una gran oportunidad para recibir el beneficioso consejo de estos dedicados supernafines.
\vs p026 10:2 \pc A los que no tienen éxito en su primer intento por llegar a la Deidad se les eleva directamente desde el círculo en el que fracasaron al segundo círculo antes de que se les envíe de nuevo al servicio del suprauniverso. De este modo, los consejeros y los asesores también sirven como orientadores y consoladores de estos desalentados peregrinos. Estos acaban de tropezar con su mayor decepción, que no difiere de manera alguna, salvo por su magnitud, de la larga lista de experiencias similares por las que han ascendido, como en una escalera, del caos a la gloria. Estos seres son aquellos que han apurado hasta el poso la copa de la experiencia. Yo he observado que regresan temporalmente al servicio de los suprauniversos y son los más afectuosos servidores de los hijos del tiempo en sus decepciones temporales.
\vs p026 10:3 Tras una larga estancia en la vía número dos, los consejos de perfección que intervienen en el mundo piloto de este círculo examinan a los decepcionados peregrinos y certifican que han superado la prueba de Havona; y esto, en cuanto a lo que respecta a su estatus no espiritual, les otorga la misma posición en los universos del tiempo que si hubiesen tenido realmente éxito en la aventura de la Deidad. La actitud de estos candidatos era totalmente aceptable y su fracaso se debía a algún aspecto de su modo de acercamiento a la Deidad o a alguna parte de su trasfondo experiencial.
\vs p026 10:4 Luego, los consejeros del círculo los llevan ante los jefes de nombramientos del Paraíso y se les envía de vuelta al servicio del tiempo en los mundos del espacio, y hacia allí parten con gozo y alegría para hacerse cargo de las tareas que realizaban en días y eras anteriores. En otro momento volverán al círculo en el que experimentaron su mayor decepción e intentarán nuevamente la aventura de la Deidad.
\vs p026 10:5 Para los peregrinos que tuvieron éxito en la segunda vía se ha acabado el estímulo de la incertidumbre experimentado en la andadura evolutiva, pero la aventura de la misión eterna está por comenzar, y aunque la estancia en este círculo sea totalmente placentera y muy fructífera, esta carece del entusiasmo y la ilusión de los círculos anteriores. Son muchos los peregrinos que, en tales momentos, recuerdan, con una placentera nostalgia, la larguísima lucha por la que pasaron, deseando realmente poder regresar, de algún modo, a los mundos del tiempo y comenzar todo de nuevo, de la misma manera que vosotros los mortales, al acercaros a una edad avanzada, recordáis las dificultades con las que os enfrentasteis en vuestra juventud y en los primeros años de vuestra vida, y verdaderamente desearíais vivir vuestras vidas de nuevo.
\vs p026 10:6 Pero la travesía del círculo más interior está ante ellos y, un poco más adelante, llegará a su fin el último sueño de tránsito y comenzará su nueva aventura en la andadura eterna. Los consejeros y asesores del segundo círculo comienzan a preparar a sus tutorados para este gran descanso final, el ineludible sueño que siempre sobreviene entre las etapas decisivas de la andadura ascendente.
\vs p026 10:7 Cuando los peregrinos ascendentes que han logrado llegar al Padre Universal completan su experiencia en el segundo círculo, los guías de los graduados a ellos asignados, siempre en su compañía, dictan la orden que los admite al círculo final. Estos guías conducen personalmente a sus tutorados hasta el círculo interior y los confían allí a la custodia de los acompañadores del descanso, el último de los órdenes de supernafines secundarios encargados de asistir a los peregrinos del tiempo en las vías circulatorias de los mundos de Havona.
\usection{11. LOS ACOMPAÑADORES DEL DESCANSO}
\vs p026 11:1 La mayor parte del tiempo que pasan lo seres ascendentes en la última vía se dedica a continuar el estudio de problemas inminentes relacionados con la residencia en el Paraíso. Una inmensa y diversa multitud de seres, la mayoría no revelada, reside de forma permanente o transitoria en este anillo interior de los mundos de Havona. La combinación de estos múltiples tipos de seres proporciona a los acompañadores superáficos del descanso un entorno rico en situaciones que saben utilizar con eficacia para favorecer la instrucción de los peregrinos ascendentes, especialmente con respecto a los problemas de adaptación a los muchos grupos de seres que pronto encontrarán en el Paraíso.
\vs p026 11:2 \pc Entre aquellos que moran en este círculo interior están los hijos trinitizados por criaturas. Los supernafines primarios y secundarios son los custodios generales de los colectivos conjuntos de estos hijos, incluidos los vástagos trinitizados por los finalizadores mortales y la progenie similar trinitizada por los ciudadanos del Paraíso. Algunos de estos hijos están acogidos por la Trinidad y tienen como encargo servir en los suprauniversos, otros tienen destinos diversos, pero la gran mayoría se reúne en un colectivo conjunto en los mundos perfectos del círculo interior de Havona. Aquí, bajo la supervisión de los supernafines, un colectivo especial no identificado de elevados ciudadanos del Paraíso los prepara para su futura labor. Estos elevados ciudadanos fueron, antes de los tiempos de Granfanda, los primeros mandatarios adjuntos de los eternos de días. Existen muchas razones para conjeturar que estos dos grupos singulares de seres trinitizados trabajarán juntos en un futuro remoto, y una de estas razones es su destino común en las reservas del colectivo de finalizadores trinitizados del Paraíso.
\vs p026 11:3 En esta vía interior, los peregrinos ascendentes y descendentes fraternizan entre sí al mismo tiempo que con los hijos trinitizados por criaturas. Al igual que sus padres, estos hijos obtienen un gran beneficio de su relación mutua; de hecho, la misión especial de los supernafines es facilitar y asegurar la confraternización entre los hijos trinitizados por los finalizadores mortales y los hijos trinitizados por los ciudadanos del Paraíso. Los acompañadores del descanso no se ocupan tanto de su formación como del fomento de la relación entre los diversos grupos basada en la comprensión.
\vs p026 11:4 Los mortales han recibido un mandato desde el Paraíso: “Sed vosotros perfectos, así como vuestro Padre del Paraíso es perfecto”. Los supernafines supervisores no cesan de proclamar a estos hijos trinitizados del colectivo conjunto en el que se integran: “Sed comprensivos con vuestros hermanos ascendentes tal como los hijos creadores del Paraíso los conocen y aman”.
\vs p026 11:5 \pc La criatura mortal debe encontrar a Dios. El hijo creador no se detiene hasta que halla al hombre, a la más modesta criatura de voluntad. Sin duda, los hijos creadores y sus hijos mortales se preparan para algún futuro servicio desconocido en el universo. Ambos cruzan todo el espectro del universo experiencial y así se instruyen y capacitan para su misión eterna. En todos los universos está ocurriendo una singular combinación de lo humano y lo divino, la integración de criatura y Creador. Algunos mortales irreflexivos se han referido a la manifestación de la misericordia y la indulgencia divina, en especial a la experimentada hacia los débiles y a favor de los necesitados, como indicativo de un Dios antropomorfo. ¡Qué error! Tales manifestaciones de misericordia y abnegación hacia los seres humanos se deberían considerar más bien como prueba de que el hombre mortal está habitado por el espíritu del Dios vivo, que la divinidad, después de todo, motiva la acción de las criaturas.
\vs p026 11:6 \pc Hacia el fin de su estancia en el primer círculo, los peregrinos ascendentes se encuentran por primera vez con los habilitadores del sueño, pertenecientes al orden primario de los supernafines. Estos son los ángeles del Paraíso que acuden para dar la bienvenida a aquellos que se hallan en el umbral de la eternidad y para terminar de prepararlos para el sueño de transición de la última resurrección. No eres realmente hijo del Paraíso hasta que no hayas recorrido el círculo interior y hayas experimentado la resurrección de la eternidad tras el último sueño del tiempo. Los perfeccionados peregrinos comienzan este descanso, se duermen, en el primer círculo de Havona, pero despiertan en las orillas del Paraíso. De todos aquellos que ascienden a la Isla eterna, solo los que llegan de esta forma son hijos de la eternidad; los demás van como visitantes, como huéspedes sin condición de residentes.
\vs p026 11:7 Y ahora, vosotros los mortales, en la culminación de la andadura de Havona, al dormiros en el mundo piloto del círculo interior, no vais solos a vuestro descanso como lo hicisteis en los mundos de los que sois originarios cuando cerrasteis los ojos en el sueño natural de la muerte, ni como lo hicisteis cuando entrasteis en el largo trance de tránsito en preparación para el viaje a Havona. Ahora, al prepararos para el descanso de consecución, se coloca a vuestro lado vuestro colaborador de tantos años del primer círculo, el majestuoso acompañador del descanso, que se prepara para emprender tal descanso junto con vosotros, como promesa de Havona de que vuestra transición será completa, y de que aguardáis tan solo los toques finales de la perfección.
\vs p026 11:8 Tu primera transición fue, de hecho, la muerte; la segunda, un sueño ideal y, ahora, la tercera metamorfosis es el verdadero descanso, el reposo de los tiempos.
\vsetoff
\vs p026 11:9 [Exposición de un perfeccionador de la sabiduría de Uversa.]
