\upaper{151}{Estancia y enseñanzas junto al mar}
\author{Comisión de seres intermedios}
\vs p151 0:1 Para el 10 de marzo, todos los grupos de predicadores y enseñantes se habían reunido en Betsaida. El jueves por la noche y el viernes muchos de ellos salieron de pesca, si bien, el día del \bibemph{sabbat} asistieron a la sinagoga para oír el discurso de un judío de Damasco, de avanzada edad, sobre la gloria del padre Abraham. Jesús pasó la mayor parte de este \bibemph{sabbat} a solas en las colinas. Ese sábado por la noche, el Maestro habló más de una hora a los grupos allí congregados sobre “La misión de las adversidades y el valor espiritual de las decepciones”. Aquel fue una inolvidable charla y sus oyentes jamás olvidarían la lección que les impartió.
\vs p151 0:2 Jesús aún no se había recuperado por completo del dolor que le había producido aquel reciente rechazo de Nazaret; los apóstoles notaron en él una tristeza muy especial que se entremezclaba con su comportamiento, normalmente alegre. Santiago y Juan estaban gran parte del tiempo con él, ya que Pedro estaba muy ocupado con las numerosas responsabilidades contraídas al tener que atender y dirigir el nuevo colectivo de evangelistas. Las mujeres pasaron este tiempo de espera, antes de dirigirse a Jerusalén para la Pascua, yendo de casa en casa, enseñando el evangelio y prestando sus cuidados a los enfermos en Cafarnaúm y en las ciudades y aldeas de los alrededores.
\usection{1. LA PARÁBOLA DEL SEMBRADOR}
\vs p151 1:1 Hacia esta época empezó Jesús, por primera vez, a usar la parábola como método para enseñar a las multitudes que tan asiduamente se reunían a su alrededor. Al haber hablado Jesús con los apóstoles y con los demás casi hasta bien avanzada la madrugada, aquel domingo por la mañana, muy pocos del grupo se habían levantado para el desayuno; así que se dirigió a la orilla del mar y se sentó solo en una barca, la antigua barca de pesca de Andrés y Pedro, que se mantenía siempre a su disposición. Allí meditó sobre el siguiente paso a dar en la tarea de expandir el reino. Pero el Maestro no permanecería a solas por mucho tiempo. Muy pronto, empezó a llegar gente de Cafarnaúm y de las aldeas cercanas y, hacia las diez de la mañana, casi mil personas se habían congregado en la orilla, cerca de la barca donde él estaba, requiriendo su atención. Pedro ya se había levantado y, encaminándose a la barca, dijo a Jesús: “Maestro, ¿debería decirles algo?”. Pero Jesús respondió: “No, Pedro, yo les contaré una historia”. Y, entonces, Jesús comenzó el relato de la parábola del sembrador, una de las primeras de una larga serie de ellas, que contaría a las muchedumbres que le seguían. Esta barca tenía un asiento elevado en el que él se sentó (porque era costumbre sentarse para enseñar) mientras hablaba a la multitud que se había congregado a lo largo de la orilla. Una vez que Pedro les dirigió unas palabras, Jesús dijo:
\vs p151 1:2 \pc “Un sembrador salió a sembrar y, mientras sembraba, parte de la semilla cayó junto al camino y vinieron las aves y la pisotearon y se la comieron con avidez. Parte cayó en pedregales, donde no había mucha tierra y brotó pronto porque no tenía profundidad de tierra; cuando salió el sol, se quemó y, como no tenía raíz, se secó. Parte cayó entre espinos, y los espinos crecieron y la ahogaron. Pero parte cayó en buena tierra y creció, y dio fruto a ciento, a sesenta y a treinta por uno. Y cuando terminó de hablar en parábola, dijo a la multitud: El que tiene oídos para oír, oiga”.
\vs p151 1:3 \pc Cuando oyeron a Jesús enseñar a la gente de esta manera, los apóstoles y quienes estaban con ellos se quedaron bastante perplejos; y tras mucho hablar entre ellos, esa noche, en el jardín de Zebedeo, Mateo dijo a Jesús: “Maestro, ¿cuál es el significado de los dificultosos dichos que diriges a la multitud? ¿Por qué les hablas en parábolas a los que buscan la verdad?”. Y Jesús, respondiendo, les dijo:
\vs p151 1:4 “Os he instruido con paciencia todo este tiempo. A vosotros os es dado saber los misterios del reino de los cielos, pero a las multitudes irreflexivas y a aquellos que buscan nuestra destrucción, se les darán los misterios del reino en parábolas. Y lo haremos así para que los que realmente desean entrar al reino puedan comprender el significado de la enseñanza y, de este modo, encontrar la salvación, mientras que los que escuchan con la intención de atraparnos queden aún más confundidos, porque viendo no verán y oyendo no oirán. Hijos míos, acaso no percibís la ley del espíritu que decreta que a cualquiera que tiene, se le dará y tendrá más; pero al que no tiene, aun lo que tiene le será quitado. Por lo tanto, en lo sucesivo, utilizaré con frecuencia las parábolas para que nuestros amigos y los que desean conocer la verdad puedan encontrar lo que buscan, mientras que nuestros enemigos y los que no aman esta verdad puedan oír sin entender. Mucha de esta gente no sigue el camino de la verdad. El profeta ciertamente describió todas estas almas de poco discernimiento cuando dijo: ‘Porque el corazón de este pueblo se ha entorpecido, y con los oídos oyen pesadamente, y han cerrado sus ojos para evitar reconocer la verdad y entenderla en sus corazones’”.
\vs p151 1:5 Los apóstoles no comprendieron del todo el significado de las palabras del Maestro. Mientras Andrés y Tomás siguieron conversando con Jesús, Pedro y los demás apóstoles se retiraron a otra parte del jardín, y allí estuvieron comentando aquella cuestión detenida y largamente.
\usection{2. INTERPRETACIÓN DE LA PARÁBOLA}
\vs p151 2:1 Pedro y el grupo que tenía junto a él llegaron a la conclusión de que la parábola del sembrador era una alegoría, que cada uno de los rasgos tenía algún significado oculto, así que decidieron ir a Jesús y pedirle que se la explicara. Así pues, Pedro se aproximó al Maestro, diciendo: “No podemos descifrar el significado de esta parábola y desearíamos que nos las aclararas ya que dices que se nos da para conocer los misterios del reino”. Al oír esto, Jesús le dijo a Pedro: “Hijo mío, no deseo esconder nada de ti, pero ¿por qué no me cuentas primero lo que habéis estado hablando? ¿Cuál es tu interpretación de la parábola?”
\vs p151 2:2 Tras un momento de silencio, Pedro dijo: “Maestro, hemos hablado mucho sobre la parábola, y esta es la interpretación a la que he llegado: el sembrador es el predicador del evangelio; la semilla es la palabra de Dios. La semilla que cayó al lado del camino representa a aquellos que no entienden las enseñanzas del evangelio. Los pájaros que arrebataron la semilla que cayó sobre el terreno endurecido representan a Satanás, o al maligno, que quita lo que se ha sembrado en el corazón de estos seres sin conocimiento. La semilla que cayó sobre los pedregales y que brotó tan rápidamente representa a esas personas superficiales e irreflexivas que oyen la buena nueva y la reciben con gozo; pero como la verdad no echa raíces en su interior, no se mantienen firmes ante la tribulación y la persecución. Tropiezan en cuanto se topan con las dificultades; sucumben a la tentación. La semilla que cayó sobre los espinos representa a los que oyen la palabra con mucha voluntad, pero permiten que los afanes del mundo y el engaño de las riquezas ahoguen la palabra de la verdad y la hacen infructuosa. Pero la semilla que se sembró en buena tierra y brotó a treinta, a sesenta y a cien por uno representa a los que cuando oyen la verdad, la reciben en diferente medida ---dependiendo de sus diversas dotes intelectuales--- y así manifiestan estos distintos grados de experiencia religiosa”.
\vs p151 2:3 Una vez que Jesús escuchó la interpretación que hizo Pedro de la parábola, les preguntó a los otros apóstoles si no tenían también alguna sugerencia que hacer al respecto. Solo Natanael respondió a su ofrecimiento. Él dijo: “Maestro, aunque reconozco muchas cosas positivas de la interpretación de Pedro de la parábola, no estoy completamente de acuerdo con él. Mi idea de esta parábola sería esta: la semilla representa el evangelio del reino, mientras que el sembrador se refiere a los mensajeros del reino. La semilla que cayó al lado del camino sobre la tierra endurecida representa a aquellos que saben poco del evangelio, al igual que a aquellos que son indiferentes al mensaje y a los que han endurecido sus corazones. Los pájaros del cielo que arrebatan la semilla que cayó al borde del camino representan nuestros propios hábitos de vida, la tentación del mal y los deseos de la carne. La semilla que cayó sobre los pedregales alude a esas almas de fuertes sentimientos que reciben rápidamente la nueva enseñanza y son igualmente rápidas en renunciar a la verdad cuando se enfrentan a las dificultades y a las realidades de tener que vivir de acuerdo con dicha verdad; les falta percepción espiritual. La semilla que cayó entre los espinos representa a los que se sienten atraídos por las verdades del evangelio; están dispuestos a seguir sus enseñanzas, pero la vanagloria de la vida, los celos, la envidia y las ansiedades de la existencia humana se lo impiden. La semilla que cayó en buena tierra da fruto, y produce a ciento, a sesenta y a treinta por uno, representa los grados variables de habilidad natural por los que los hombres y mujeres comprenden la verdad y responden diferentemente a las enseñanzas espirituales.
\vs p151 2:4 Cuando Natanael acabó de hablar, los apóstoles y sus acompañantes se ensalzaron seriamente en comentarios y debates sobre todo esto; algunos argumentaban el acierto de la interpretación de Pedro, mientras que, casi el mismo número, intentaba defender la explicación de la parábola por parte de Natanael. Entretanto, Pedro y Natanael se habían retirado a la casa, donde ambos trataron por todos los medios de convencer decididamente al otro para que cambiara de opinión.
\vs p151 2:5 El Maestro permitió que estos momentos de confusión alcanzaran su máxima expresión; entonces, dio unas palmadas y los llamó a su lado. Cuando todos estaban reunidos una vez más a su alrededor, dijo: “Antes de que os hable de esta parábola, ¿tiene alguien algo más que decir?”. Tras un momento de silencio, Tomás habló: “Sí, Maestro, querría decir unas palabras. Recuerdo que cierta vez nos dijiste que desconfiáramos precisamente de eso mismo. Nos indicaste que, cuando, al predicar, diéramos ejemplos, que estos fueran historias verdaderas, no fábulas, y que debíamos elegir la historia más pertinente a la ilustración de la verdad central y vital que deseamos enseñar a la gente y que, una vez que empleáramos dicha historia, no debemos intentar aplicar espiritualmente todos los detalles menores contenidos en el relato de la historia. Pienso que las explicaciones dadas por Pedro y Natanael están equivocadas. Admiro su talento en estas cosas, pero estoy igualmente seguro de que tratar de hacer que una parábola sobre la naturaleza refleje, en todos sus elementos, analogías espirituales solo puede crear confusión y una idea muy equivocada de su verdadero propósito. Me da la razón el hecho de que, si hace una hora todos coincidíamos, ahora estamos divididos en dos grupos separados con diferentes opiniones sobre la parábola, y nos aferramos a estas opiniones tan encarecidamente hasta incluso interferir, en mi opinión, con nuestra propia aptitud para captar completamente la gran verdad que tenías en mente al contar esta parábola a la multitud y luego pedirnos nuestro comentario”.
\vs p151 2:6 Las palabras de Tomás tuvieron un efecto tranquilizador sobre todos ellos. Les hizo recordar lo que Jesús les había enseñado en otras ocasiones y, antes de que Jesús continuara hablando, Andrés se levantó y dijo: “Estoy convencido de que Tomás tiene razón, y me gustaría que él nos dijera qué significado atribuye a la parábola del sembrador”. Una vez que Jesús le hizo señas a Tomás para que hablara, él comentó: “Hermanos míos, no era mi intención prolongar esta conversación, pero si lo deseáis, os diré que pienso que esta parábola se relató para enseñarnos una gran verdad. Y esta es que nuestra enseñanza del evangelio del reino, al margen de que llevemos a cabo nuestro encargo divino con lealtad y efectividad, se verá seguida por diferente grado de éxito; y que toda esta variedad de resultados está sujeta a circunstancias y condicionamientos de nuestro ministerio sobre los que tenemos poco o ningún control”.
\vs p151 2:7 Cuando Tomás terminó de hablar, la mayoría de sus compañeros predicadores estaba a punto de mostrarle su conformidad con lo que había expresado, incluso Pedro y Natanael iban a su encuentro para hablar con él, entonces Jesús se levantó y dijo: “Bien hecho, Tomás; has percibido el verdadero significado de las parábolas; pero tanto Pedro como Natanael os han hecho un mismo bien a todos, al haber mostrado acertadamente el peligro que existe cuando se pretende hacer una alegoría de mis parábolas. En vuestros corazones, podría resultar beneficioso dar rienda suelta a estas especulaciones de vuestra imaginación, pero cometeréis un error al querer integrar estas conclusiones en vuestras enseñanza”.
\vs p151 2:8 Una vez desaparecida la tensión acumulada, Pedro y Natanael se felicitaron mutuamente por sus sendas interpretaciones y, con la excepción de los gemelos Alfeo, cada uno de los apóstoles se aventuró a dar su interpretación de la parábola del sembrador antes de retirarse por la noche. Incluso la ofrecida por Judas Iscariote resultó muy plausible. Muchas veces, los doce intentarían, entre ellos, interpretar las parábolas del Maestro como si fuese una alegoría, pero nunca más consideraron esas conjeturas con seriedad. Aquella fue una charla muy provechosa para los apóstoles y sus acompañantes, muy especialmente porque desde aquel momento Jesús llegaría a incluir cada vez más parábolas en sus enseñanzas públicas.
\usection{3. MÁS SOBRE LAS PARÁBOLAS}
\vs p151 3:1 Los apóstoles se sentían atraídos hacia las parábolas, hasta tal punto que toda la noche siguiente se dedicó a tratar más extensamente el tema de las parábolas. Jesús comenzó su charla diciendo: “Amados míos, al enseñar debéis hacerlo siempre tratando de adaptar vuestra exposición de la verdad a las mentes y a los corazones de quienes están ante vosotros. Cuando os halléis frente a una multitud cuyos intelectos y temperamentos difieran, no podéis hablar de forma distinta a cada clase de oyente, pero sí podéis contar una historia que trasmita vuestra enseñanza, y, así, cada grupo, incluso cada persona, hará su propia interpretación de vuestra parábola, de acuerdo con sus propias capacidades intelectuales y espirituales. Debéis hacer que vuestra luz brille, pero hacedlo con sabiduría y discreción. No se enciende una luz y se pone debajo de una vasija o debajo de la cama, sino sobre el candelero para que alumbre a todos los que estén allí. Pues bien, nada hay oculto en el reino de los cielos que no haya de ser manifestado ni escondido que no haya de salir a luz. Pensad no solo en las multitudes y cómo oyen la verdad; mirad, pues, cómo la oís vosotros. Recordad que muchas veces os he dicho: a cualquiera que tiene, se le dará y tendrá más; pero al que no tiene, aun lo que tiene le será quitado.
\vs p151 3:2 \pc La plática sobre las parábolas, que se alargó durante algún tiempo, y las instrucciones dadas en cuanto a su interpretación pueden resumirse y expresarse en términos modernos de la siguiente manera:
\vs p151 3:3 \li{1.}Jesús desaconsejó el empleo de fábulas o alegorías en la enseñanza de las verdades del evangelio. Sí recomendó que se usaran con libertad las parábolas, en especial aquellas que tuvieran que ver con la naturaleza. Hizo hincapié en el valor del uso de la \bibemph{analogía} existente entre el mundo natural y el espiritual, como medio para enseñar la verdad. En repetidas ocasiones se refirió a lo natural como “la sombra irreal y fugaz de las realidades del espíritu”.
\vs p151 3:4 \li{2.}Jesús narró tres o cuatro parábolas de las escrituras hebreas, destacando el hecho de que este método de enseñanza no era enteramente nuevo, aunque sí lo fue, prácticamente y desde este momento en adelante, de la manera en la que él lo llevó a cabo.
\vs p151 3:5 \li{3.}Al instruir a los apóstoles en el valor de las parábolas, Jesús les llamó la atención sobre los siguientes puntos:
\vs p151 3:6 La parábola facilita que se apele, de manera simultánea, a muy distintos niveles mentales y espirituales. La parábola estimula la imaginación, reta el discernimiento y suscita el pensamiento crítico; contribuye a la empatía y no al antagonismo.
\vs p151 3:7 La parábola parte de las cosas conocidas a la percepción de lo desconocido. La parábola utiliza lo material y lo natural como medio para presentar lo espiritual y lo supramaterial.
\vs p151 3:8 Las parábolas favorecen la adopción de decisiones morales imparciales. La parábola evita mucho prejuicio y deposita grácilmente una nueva verdad en la mente, y lo hace todo suscitando poco sentido de justificación de sí mismo o de indignación personal.
\vs p151 3:9 Rechazar la verdad que la parábola establece a través de la analogía exige un acto intelectual consciente que redunda en el absoluto menosprecio del juicio honesto y de la decisión imparcial de cada cual. La parábola da impulso al pensamiento mediante el sentido del oído.
\vs p151 3:10 El uso de la forma parabólica para impartir enseñanzas posibilita al maestro exponer verdades nuevas e incluso sorprendentes, evitando al mismo tiempo, y en gran medida, cualquier controversia y enfrentamiento con la tradición y la autoridad establecida.
\vs p151 3:11 La parábola también tiene la ventaja de estimular la memoria de la verdad impartida, cuando hay un encuentro con unas escenas que resultan ya familiares.
\vs p151 3:12 \pc De este modo, Jesús procuró dar a conocer a sus seguidores muchas de las razones por las que él hacía un uso cada vez mayor de las parábolas en su enseñanza pública.
\vs p151 3:13 \pc Hacia el final de esta sesión nocturna, Jesús hizo su primer comentario sobre la parábola del sembrador. Dijo que la parábola hacía referencia a dos cosas: la primera, era un repaso de su propio ministerio hasta ese momento y una predicción de lo que aguardaba durante el resto de su vida en la tierra; y, la segunda, un anuncio a los apóstoles y otros mensajeros del reino de lo que cabría esperar en su ministerio de generación en generación, conforme trascurriera el tiempo.
\vs p151 3:14 Jesús recurrió asimismo al uso de las parábolas para refutar, de la mejor manera posible, el premeditado afán de los líderes religiosos de Jerusalén por inculcar la idea de que él hacía toda su labor con la ayuda de los demonios y del príncipe de los diablos. El recurso a la naturaleza contradecía tal pretensión, dado que la gente de esos días creía que cualquier fenómeno natural y sobrenatural era consecuencia directa de los actos de seres espirituales y de fuerzas sobrenaturales. También se decidió por ese método de enseñanza, porque le permitía proclamar verdades esenciales a los que deseaban conocer el mejor camino, dándoles a sus enemigos, al mismo tiempo, menos oportunidades para que encontraran motivo de ofensa y acusación.
\vs p151 3:15 Por la noche, antes de despedir al grupo, Jesús dijo: “Ahora os contaré la última de las parábolas del sembrador. Quiero probaros para saber cómo interpretaréis esta: Así es el reino de Dios, como cuando un hombre echa una buena semilla en la tierra. Y ya duerma y vele, de noche y de día, la semilla brota y crece sin que él sepa cómo, porque de por sí lleva fruto la tierra: primero hierba, luego espiga, después grano lleno en la espiga; y cuando el fruto está maduro, en seguida se mete la hoz, porque la siega ha llegado. El que tiene oído para oír, oiga”.
\vs p151 3:16 Los apóstoles reflexionaron mucho sobre esta parábola, pero el Maestro jamás volvería a referirse a esta ampliación de la parábola del sembrador.
\usection{4. OTRAS PARÁBOLAS JUNTO AL MAR}
\vs p151 4:1 Al día siguiente Jesús de nuevo enseñó a la gente desde la barca, diciendo: “El reino de los cielos es semejante a un hombre que sembró buena semilla en su campo; pero mientras dormía, vino su enemigo y sembró cizaña entre el trigo, y huyó de prisa. Y cuando brotó la hierba y fue a dar su fruto, entonces apareció también la cizaña. Fueron entonces los siervos del padre de familia y le dijeron: “Señor, ¿no sembraste buena semilla en tu campo? ¿Cómo, pues, tiene cizaña?”. Él les dijo: “Un enemigo ha hecho esto”. Y los siervos le preguntaron: “¿Quieres, entonces, que vayamos y la arranquemos?”. Él les contestó, diciéndoles: “No, no sea que al arrancar la cizaña arranquéis también con ella el trigo. Dejad crecer juntamente lo uno y lo otro hasta la siega, y al tiempo de la siega yo diré a los segadores: ‘Recoged primero la cizaña y atadla en manojos para quemarla; pero recoged el trigo en mi granero’”.
\vs p151 4:2 \pc Una vez que la gente le hizo algunas preguntas, Jesús les refirió otra parábola: “El reino de los cielos es semejante al grano de mostaza que un hombre sembró en su campo. Esta es a la verdad la más pequeña de todas las semillas, pero cuando ha crecido es la mayor de las hortalizas y se hace árbol, de tal manera que vienen las aves del cielo y se posan en sus ramas”.
\vs p151 4:3 \pc “El reino de los cielos es semejante a la levadura que tomó una mujer y escondió en tres medidas de harina, hasta que todo quedó leudado”.
\vs p151 4:4 \pc “Además el reino de los cielos es semejante a un tesoro escondido en un campo, el cual un hombre halla y gozoso por ello va y vende todo lo que tiene y compra aquel campo”.
\vs p151 4:5 \pc “También el reino de los cielos es semejante a un comerciante que busca buenas perlas y, al hallar una perla preciosa, fue y vendió todo lo que tenía y compró aquella perla extraordinaria”.
\vs p151 4:6 \pc “Asimismo el reino de los cielos es semejante a una red barredera que, echada al mar, recoge toda clase de peces. Cuando está llena, la sacan a la orilla, se sientan y recogen lo bueno en cestas y echan fuera lo malo.
\vs p151 4:7 \pc Jesús contó otras muchas parábolas a las multitudes. De hecho, desde ese momento en adelante, raras fueron las veces en las que no usara este método de enseñanza. Tras dirigirse públicamente en parábolas a la gente, durante las clases nocturnas exponía de forma más pormenorizada y explícita sus enseñanzas a los apóstoles y a los evangelistas.
\usection{5. VISITA A QUERESA}
\vs p151 5:1 La multitud siguió aumentando a lo largo de esa semana. El día del \bibemph{sabbat} Jesús se dio prisa para ir a las colinas, pero, el domingo por la mañana, las multitudes volvieron. Jesús les habló temprano por la tarde tras la predicación de Pedro y, cuando terminó, dijo a sus apóstoles: “Estoy cansado de las muchedumbres; pasemos al otro lado y descansamos un día”.
\vs p151 5:2 Al cruzar el lago, se levantó de repente una gran tempestad, característica del mar de Galilea, particularmente en esa temporada del año. Esta masa de agua se halla a más de doscientos metros por debajo del nivel del mar y está rodeada por altos acantilados, especialmente por el oeste. Hay empinados desfiladeros que van desde el lago a las colinas y, al ascender la bolsa de aire caliente sobre el lago durante el día, tras la caída del sol, el aire frio de los desfiladeros tiende a enfriarse y a precipitarse sobre este. Estos vendavales aparecen rápidamente y, muchas veces, desaparecen tan repentinamente como llegaron.
\vs p151 5:3 Y, aquel domingo, a última hora de la tarde, fue uno de esos vendavales el que alcanzó de lleno a la barca que llevaba a Jesús a la otra orilla. Tras ella iban otras tres barcas con algunos de los evangelistas más jóvenes. Estaban en medio de una violenta tempestad, a pesar de darse solamente en aquella zona del lago; no había señales de tormenta en la orilla occidental. El viento era tan fuerte que echaba las olas en la barca, de tal manera que ya se anegaba. El fuerte viento había desgarrado la vela antes de que los apóstoles pudieran recogerla, y dependían ahora enteramente de sus remos, que batían con fuerzas, para lograr alcanzar la costa, a más de dos kilómetros de distancia.
\vs p151 5:4 Entretanto, Jesús dormía en la popa resguardado bajo un saliente. El Maestro se encontraba agotado cuando salió de Betsaida y, para poder descansar, les había ordenado zarpar y dirigirse al otro lado del lago. Estos antiguos pescadores eran fuertes y experimentados remeros, pero aquel era uno de los peores vendavales con los que jamás habían tenido que enfrentarse. Aunque el viento y las olas sacudían la barca como si se tratase de un barco de juguete, Jesús dormía imperturbable. Pedro remaba cerca de la popa, a la derecha. Y, cuando la barca empezó a inundarse, soltó su remo y, corriendo hacia él, lo sacudió enérgicamente para despertarlo. Al levantarse Jesús, Pedro le dijo: “Maestro, ¿es que no ves que estamos ante una fuerte tormenta? ¡Sálvanos que pereceremos todos”.
\vs p151 5:5 Al salir de donde estaba, ya bajo la lluvia, Jesús miró primeramente a Pedro, luego divisando en la oscuridad a los tenaces remeros y dirigiendo de nuevo su mirada a Simón Pedro, que, en su agitación, no había vuelto a su remo, dijo: “¿Por qué estáis así amedrentados? ¿Cómo no tenéis fe? Ten paz y tranquilízate. Apenas había terminado Jesús de hacer este reproche a Pedro y a los demás apóstoles, y haber mandado a Pedro que se apaciguara, que aquietara su perturbada alma, la convulsa atmósfera se estabilizó, se hizo una gran calma. Las olas enfurecidas amainaron casi momentáneamente, mientras que los negros nubarrones, habiéndose descargado por la breve lluvia, desaparecieron, y las estrellas del cielo brillaron en lo alto. Hasta donde podemos discernir, todo esto fue una mera coincidencia; pero los apóstoles, Simón Pedro en particular, jamás dejarían de considerar aquel hecho como un milagro de la naturaleza. A los hombres de aquella época, les resultaba especialmente fácil creer en este tipo de los milagros, dado que estaban completamente convencidos de que las fuerzas espirituales y los seres sobrenaturales regían directamente sobre los fenómenos naturales.
\vs p151 5:6 Jesús explicó claramente a los doce que sus palabras iban destinadas a sus consternadas almas y a sus mentes agitadas por el miedo, que él no había ordenado a los elementos que obedecieran su palabra, pero todo fue en vano. Los seguidores del Maestro se obstinaron en hacer siempre su propia interpretación de todos estos sucesos fortuitos. Desde este día en adelante, se empeñaron en pensar que el Maestro poseía poder absoluto sobre los elementos naturales. Pedro no se cansaría nunca de contar cómo “aun el viento y las olas lo obedecen”.
\vs p151 5:7 Cuando Jesús y sus acompañantes alcanzaron la orilla ya avanzada la noche que al final resultó tranquila y hermosa, y todos pudieron descansar en las barcas. No bajaron a tierra hasta el día siguiente, poco después de que amaneciera. Cuando se reunieron, unos cuarenta en total, Jesús dijo: “Vámonos a aquellas colinas lejanas y quedémonos allí algunos días mientras meditamos sobre los problemas del reino del Padre”.
\usection{6. EL LUNÁTICO DE QUERESA}
\vs p151 6:1 Aunque gran parte de la cercana orilla oriental del lago tenía suaves pendientes que subían hasta las más alejadas elevaciones del terreno, en ese sitio en concreto había una ladera empinada que, en algunos puntos, descendía escarpadamente hasta el lago. Señalando a la ladera de la colina cercana, Jesús dijo: “Escalémosla y desayunemos, y descansemos y conversemos bajo algún cobijo”.
\vs p151 6:2 Toda esta pendiente estaba llena de cavernas excavadas en la roca. Muchos de estos nichos eran antiguos sepulcros. Hacia medio camino de la ladera, en una pequeña y relativa planicie estaba el cementerio de la pequeña aldea de Queresa. Cuando Jesús y sus acompañantes pasaron cerca de este cementerio, un lunático que vivía en estas cavernas corrió hacia ellos. Este hombre demente era bien conocido en esas regiones; en otro tiempo, había estado atado con grilletes y cadenas y recluido en una de aquellas grutas. Hacía mucho tiempo que había roto sus ataduras y ahora vagaba a voluntad entre las tumbas y los sepulcros abandonados.
\vs p151 6:3 Este hombre, cuyo nombre era Amós, estaba afligido de brotes periódicos de locura. Tenía largas etapas en las que se hacía de ropa y se portaba moderadamente bien con sus semejantes. Durante uno de estos intervalos de lucidez, había ido a Betsaida, donde había escuchado la predicación de Jesús y de los apóstoles y, en aquel momento, se había convertido en creyente, aunque poco convencido, del evangelio del reino. Pero de pronto había tenido una fuerte recaída y se había ido a las tumbas, donde gemía, andaba gritando y se comportaba de un modo que aterrorizaba a todos los que casualmente se encontraban con él.
\vs p151 6:4 Cuando Amós reconoció a Jesús, cayó a sus pies y exclamó: “Yo te conozco, Jesús, pero estoy poseído por muchos diablos, y te suplico que no me atormentes”. Este hombre creía verdaderamente que su aflicción mental cíclica se debía al hecho de que, en tales momentos, entraban en él los espíritus malignos o impuros y dominaban su cuerpo y su mente. Sus trastornos eran mayormente emocionales ---su enfermedad mental no revestía gran gravedad---.
\vs p151 6:5 Jesús, bajando la mirada hacia aquel hombre encorvado en el suelo a sus pies como un animal, se inclinó y, tomándolo de la mano, lo levantó y le dijo: “Amós, tú no estás poseído por ningún diablo; ya has oído la buena nueva de que eres hijo de Dios. Te ordeno que salgas de ese estado mental en el que estás”. Y, cuando Amós oyó a Jesús decir estas palabras, se produjo tal transformación en su mente que, de inmediato, recobró la salud mental y el dominio normal de sus emociones. Para entonces, se había congregado una gran multitud llegada de la aldea cercana, que junto a un grupo de porqueros que venían de las colinas, se quedó atónita al ver al lunático sentado junto a Jesús y a sus seguidores, ya en posesión de su juicio cabal y conversando abiertamente con ellos.
\vs p151 6:6 Mientras los porqueros corrieron a la aldea para difundir la noticia de que el lunático había sido amansado, los perros atacaron a un pequeño hato de unos treinta cerdos que habían quedado desatendidos y empujaron a la mayoría de ellos hasta que cayeron al mar por un precipicio. Y este hecho fortuito, con motivo de la presencia de Jesús y la curación, presuntamente milagrosa del lunático, dio origen a la leyenda de que Jesús había sanado a Amós. Se pensaba que Jesús había echado de él a una legión de demonios y que estos se habían metido en un hato de cerdos, haciendo que corrieran a toda prisa y se despeñaran abajo en el mar. Antes de haber terminado el día, los porqueros habían dado aviso por todas partes y toda la aldea lo creyó. Amós, sin lugar a dudas, daba credibilidad a esta historia; él había visto caer a los cerdos por la cresta de la colina poco después de que su perturbada mente se aquietase, y siempre pensó que los cerdos se habían llevado con ellos a los mismos espíritus malos que por tanto tiempo lo habían atormentado y afligido. Y esto tuvo bastante que ver con el hecho de que su cura fuera permanente. No es menos cierto que todos los apóstoles de Jesús (salvo Tomás) creyeron que el incidente de los cerdos estaba directamente relacionado con la curación de Amós.
\vs p151 6:7 \pc Jesús no encontró el deseado descanso. La mayor parte de ese día se vio asediado por los que acudieron allí en respuesta a la noticia de que Amós se había curado, y atraídos por la historia de que los demonios se habían ido del lunático y habían entrado en los cerdos. Y, así, tras solo una noche de descanso, el martes por la mañana temprano, una delegación de aquellos gentiles criadores de cerdos despertó a Jesús y a sus amigos; venían a instarles a que se fueran de entre ellos. Su portavoz les dijo a Pedro y Andrés: “Pescadores de Galilea, marchaos de aquí y llevaos a vuestro profeta con vosotros. Sabemos que es un santo varón, pero los dioses de nuestro país no lo conocen, y estamos en peligro de perder muchos cerdos. Tenemos gran temor de vosotros y por ello os pedimos que os vayáis”. Y cuando Jesús los oyó, le dijo a Andrés: “Regresemos a casa”.
\vs p151 6:8 Cuando estaban a punto de partir, Amós rogó a Jesús que le permitiese volver con ellos, pero el Maestro no accedió. Jesús le dijo a Amós: “No olvides que eres hijo de Dios. Regresa a los tuyos y cuenta cuán grandes cosas ha hecho Dios contigo”. Y Amós anduvo publicando que Jesús había echado a una legión de diablos de su atormentada alma, y que estos espíritus malignos habían entrado en un hato de cerdos, llevándolos a perecer rápidamente. Y no paró hasta no haber ido por todas las ciudades de la Decápolis, proclamando las grandes cosas que Jesús había hecho por él.
