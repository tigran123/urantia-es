\upaper{84}{Matrimonio y vida familiar}
\author{Jefe de los serafines}
\vs p084 0:1 La necesidad material llevó a fundar el matrimonio, el apetito sexual lo embelleció, la religión lo aprobó y lo ensalzó, el Estado lo reclamó y lo reguló, mientras que, en tiempos más recientes, el amor en su despliegue empieza a justificar y a glorificar el matrimonio como el ancestro y creador de la institución más útil y sublime de la civilización: el hogar. Y la constitución del hogar debe ser el centro y la esencia de toda labor educativa.
\vs p084 0:2 El emparejamiento es puramente un acto de autoperpetuación, relacionado con distintos grados de autogratificación; el matrimonio, la constitución del hogar, es en gran parte una cuestión que implica automantenimiento y entraña la evolución de la sociedad. La sociedad misma está integrada por unidades familiares. Como factor planetario, el individuo es muy transitorio ---solo las familias son órganos permanentes dentro de la evolución social---. La familia es el canal por el que el río de la cultura y del conocimiento fluye de una generación a la otra.
\vs p084 0:3 El hogar es fundamentalmente una institución sociológica. El matrimonio surgió de la cooperación en el automantenimiento y de la colaboración en la autoperpetuación; el elemento de la autogratificación fue en gran medida circunstancial. No obstante, el hogar de hecho incluye las tres funciones esenciales de la existencia humana, mientras que la propagación de la vida lo convierte en la institución humana fundamental y el sexo lo separa de todas otras actividades sociales.
\usection{1. LAS PRIMITIVAS RELACIONES DE PAREJA}
\vs p084 1:1 El matrimonio no se fundamentó en las relaciones sexuales; estas eran accesorias. El hombre primitivo no tenía necesidad del matrimonio; complacía sus apetitos sexuales con libertad sin tener que ocuparse de las responsabilidades de esposa, hijos u hogar.
\vs p084 1:2 La mujer, debido a su apego físico y emocional a su progenie, depende de la cooperación del varón, y esto la insta a buscar refugio y protección en el matrimonio. Pero no existe ningún impulso biológico directo que lleve al hombre al matrimonio ---y mucho menos que lo retenga ahí---. No fue el amor lo que hizo atractivo el matrimonio para el hombre, sino que fue el hambre el que, primeramente, empujó al salvaje a la mujer y al refugio primitivo que compartía con sus hijos.
\vs p084 1:3 \pc El matrimonio ni siquiera se llevó a cabo a raíz de la toma de conciencia de las obligaciones que las relaciones sexuales conllevaban. El hombre primitivo no percibía ninguna relación entre gratificación sexual y el nacimiento posterior de un niño. Hace tiempo se tenía la creencia generalizada de que una virgen podía quedarse embarazada. Pronto, el salvaje concibió la idea de que los bebés se hacían en la tierra de los espíritus; se consideraba que el embarazo se producía cuando el espíritu, un espectro en evolución, entraba en la mujer. También se pensaba que tanto la dieta como el mal de ojo podían ser los causantes del embarazo de una virgen o de una mujer no casada; más adelante, se vinculó el comienzo de la vida con el aliento y con la luz del sol.
\vs p084 1:4 Muchos pueblos primitivos relacionaban los espectros con el mar; de ahí que se restringiera el baño a las vírgenes; las jóvenes tenían mucho más miedo de bañarse en el mar con la marea alta que de mantener relaciones sexuales. Se pensaba que los niños nacían deformes o prematuros, porque alguna cría de animal se había introducido en el cuerpo de una mujer como resultado de un baño imprudente o por la acción malévola de un espíritu. Los salvajes, por supuesto, no le daban importancia al hecho de estrangular a tales vástagos al nacer.
\vs p084 1:5 El primer paso hacia la lucidez se produjo con la creencia de que las relaciones sexuales abrían el camino para que el espectro inseminador entrara en la mujer. Desde entonces, el hombre descubrió que el padre y la madre contribuyen por igual a los factores hereditarios vivos que dan origen a su progenie. Pero, incluso en el siglo XX, hay muchos padres que aún procuran mantener a sus hijos en una mayor o menor ignorancia sobre el origen de la vida humana.
\vs p084 1:6 \pc El hecho de que la función reproductora supusiese la relación entre madre e hijo daba por sentado la existencia de algún tipo simple de familia. El amor de madre es instintivo; no nació de las costumbres tal como sucedió con el matrimonio. El amor maternal de los mamíferos es el don inherente de los espíritus asistentes de la mente del universo local, y su fuerza y devoción son siempre directamente proporcionales al tiempo que dure la indefensión en la infancia de las especies.
\vs p084 1:7 La relación entre madre e hijo es natural, fuerte e instintiva, y obligó, por lo tanto, a las mujeres primitivas a someterse a muchas condiciones extrañas y a soportar penurias indecibles. Este apremiante amor maternal es un sentimiento que siempre ha obstaculizado a la mujer y la ha colocado en una desventaja tan enorme en todos sus enfrentamientos con el hombre. Pero incluso así, el instinto maternal de la especie humana no es tan abrumador; puede verse frustrado por la ambición, el egoísmo y las convicciones religiosas.
\vs p084 1:8 Aunque la relación madre\hyp{}hijo no constituye ni matrimonio ni hogar fue el núcleo del que ambos surgieron. El gran avance en la evolución del emparejamiento vino cuando estos vínculos temporales duraron lo suficiente como para criar a la consiguiente descendencia, ya que eso significaba que el hogar se estaba formando.
\vs p084 1:9 Con independencia de los antagonismos que existían en estas primeras parejas, al margen de la laxitud de los vínculos creados, las probabilidades de sobrevivir mejoraron considerablemente gracias a estas uniones entre varones y mujeres. Un hombre y una mujer, que cooperan, incluso aparte de la familia y de sus vástagos, son inmensamente superiores en muchos sentidos a dos hombres o dos mujeres. Este emparejamiento de los sexos aumentaba la supervivencia y significó en sí el comienzo de la sociedad humana. La división del trabajo por sexo también contribuyó al confort e incrementó la felicidad.
\usection{2. LAS PRIMERAS FAMILIAS MATRIARCALES}
\vs p084 2:1 Las hemorragias periódicas de la mujer y su nueva pérdida de sangre en el parto eran en un principio indicios de que la sangre era la creadora del hijo (incluso se le consideró la base del alma) y dieron origen al concepto del vínculo de sangre en las relaciones humanas. En los tiempos primitivos, todos los ancestros se calculaban siguiendo la línea femenina, puesto que era la única parte de la herencia completamente segura.
\vs p084 2:2 La familia primitiva, que nació del vínculo sanguíneo biológico instintivo entre la madre y el hijo, era inevitablemente matriarcal; y muchas tribus mantuvieron este sistema durante bastante tiempo. El matriarcado era la única transición posible desde la etapa del matrimonio colectivo en la horda a la mejor vida hogareña, que vendría después, de las familias patriarcales polígamas y monógamas. La familia matriarcal era natural y biológica; la familia patriarcal es social, económica y política. La persistencia de la familia matriarcal entre los hombres rojos de América del Norte es una de las razones principales por la que los iroqueses, de orden avanzado por lo demás, nunca llegaron a constituir un verdadero Estado.
\vs p084 2:3 Bajo las costumbres de la familia matriarcal, la madre de la esposa disfrutaba de una autoridad prácticamente suprema en el hogar; incluso los hermanos de la esposa y sus hijos participaban más activamente en la dirección de la familia que el propio marido. A menudo, los padres adoptaban el nombre de sus propios hijos.
\vs p084 2:4 Las razas más primitivas daban poco mérito al padre; consideraban que el niño provenía por completo de la madre. Creían que los hijos se parecían al padre como consecuencia de la relación entre ellos o que estaban “marcados”, de esta manera, porque la madre deseaba que se pareciesen al padre. Más tarde, cuando se produjo el cambio de la familia matriarcal a la patriarcal, el padre se atribuyó todo el mérito por el hijo y muchos de los tabúes sobre las mujeres embarazadas se hicieron después extensivos para incluir al marido. El futuro padre dejaba de trabajar a medida que se acercaba el momento del parto y, cuando este tenía lugar, se acostaba, junto con su mujer, permaneciendo en reposo entre tres y ocho días. La esposa podía levantarse al día siguiente y dedicarse al trabajo duro, pero el marido se quedaba en cama para recibir las felicitaciones; todo esto era parte de las costumbres primitivas, diseñadas para establecer el derecho del padre sobre el hijo.
\vs p084 2:5 Al principio, era costumbre que el hombre se fuera a vivir con la familia de su esposa, pero, en tiempos posteriores, una vez que el hombre hubiese pagado el precio que la novia valía o había trabajado para conseguirla, podía llevarse a su esposa y a los hijos a su propia familia. La transición de la familia matriarcal a la patriarcal explica la prohibición, por otra parte sin sentido, de algunos tipos de matrimonios entre primos, mientras se aprobaban otros de parentesco similar.
\vs p084 2:6 Con la desaparición de las costumbres de los cazadores, cuando el pastoreo dio al hombre el control sobre el principal suministro de alimentos, la familia matriarcal llegó a un rápido final. Fracasó sencillamente porque no podía competir con éxito con la nueva familia patriarcal. El poder que recaía en los parientes masculinos de la madre no podía rivalizar con el poder acumulado por el marido\hyp{}padre. La mujer no estaba en disposición de conciliar las tareas de criar hijos y ejercer una autoridad continua y una creciente potestad sobre los asuntos domésticos. La aparición del robo de esposas y, más adelante, el de su compra aceleró el fin de la familia matriarcal.
\vs p084 2:7 El extraordinario cambio de la familia matriarcal a la patriarcal es una de las adaptaciones más radicales y completas que jamás la raza humana había experimentado. Este cambio en la dirección opuesta llevó de inmediato a una mayor expresión social y a una creciente aventura familiar.
\usection{3. LA FAMILIA BAJO EL DOMINIO DEL PADRE}
\vs p084 3:1 Puede ser que el instinto maternal llevara a la mujer al matrimonio, pero fue la fuerza superior del hombre, junto con la influencia de las costumbres, la que prácticamente la forzó a permanecer en él. La vida pastoril tendió a crear un nuevo sistema de costumbres: el tipo patriarcal de vida familiar; y la base de la unidad familiar bajo las primitivas costumbres pastoriles y agrícolas fue la autoridad incuestionable y arbitraria del padre. Toda la sociedad, tanto la nacional como la familiar, pasó por la etapa de la autoridad autocrática de índole patriarcal.
\vs p084 3:2 La escasa deferencia prestada a las mujeres durante la época del Antiguo Testamento es un fiel reflejo de las costumbres de los pastores. Todos los patriarcas hebreos eran pastores, tal como se demuestra por la expresión, “El Señor es mi pastor”.
\vs p084 3:3 Pero el hombre no fue más culpable de su mal concepto de la mujer durante las eras pasadas que ella misma. La mujer no consiguió ganarse el reconocimiento social durante los tiempos primitivos, porque no supo actuar en situaciones de emergencia; no fue la impresionante heroína que hiciera frente a los momentos de crisis. La maternidad fue un claro impedimento en la lucha por la existencia; el amor maternal dificultó a la mujer participar en la defensa de la tribu.
\vs p084 3:4 De manera involuntaria, las mujeres primitivas también crearon su dependencia del varón por la admiración y el encomio que manifestaban hacia su combatividad y virilidad. Esta exaltación del guerrero elevó el ego masculino hundiendo asimismo el de la mujer y haciéndola más dependiente; un uniforme militar sigue removiendo poderosamente las emociones femeninas.
\vs p084 3:5 Entre las razas más adelantadas, las mujeres no son tan grandes ni tan fuertes como los hombres. Al ser la mujer la más débil, se volvió pues más discreta; aprendió pronto a utilizar sus encantos sexuales. Se hizo más vigilante y conservadora que el hombre, aunque algo menos profunda. El hombre era superior a la mujer en el campo de batalla y en la caza; pero, en la gestión del hogar, la mujer ha superado generalmente en liderazgo incluso al más primitivo de los hombres.
\vs p084 3:6 \pc El pastor buscaba el sustento en su rebaño, pero, durante estas épocas pastoriles, la mujer debía seguir proveyendo los alimentos de origen vegetal. El hombre primitivo despreciaba la tierra; era una labor demasiado pacífica y exenta de aventura. También existía una antigua superstición que decía que las mujeres podían hacer crecer mejor las plantas, como madres que eran. En muchas tribus atrasadas de hoy día, el hombre cocina la carne; la mujer, las verduras, y, cuando las tribus primitivas de Australia se ponen en marcha, las mujeres nunca cazan, mientras que un hombre no se rebajaría a desenterrar una raíz.
\vs p084 3:7 Las mujeres siempre han tenido que trabajar; al menos, hasta los tiempos modernos, la mujer siempre ha sido una auténtica trabajadora. El hombre habitualmente ha optado por el camino más fácil, y esta desigualdad ha existido a lo largo de la historia de la raza humana. La mujer ha sido siempre quien ha soportado las cargas; ha tenido que ocuparse de las propiedades de la familia y de atender a los hijos, dejando de esta manera las manos libres al hombre para luchar o cazar.
\vs p084 3:8 La primera liberación de la mujer ocurrió cuando el hombre accedió a labrar la tierra, consintió en hacer lo que hasta entonces se había considerado como un trabajo de mujer. Se dio un gran paso adelante cuando no se mataba a los prisioneros varones, sino que se les esclavizaba para que trabajaran como agricultores. Esto trajo consigo la liberación de la mujer de tal manera que pudo dedicar más tiempo a las tareas del hogar y a la formación de los hijos.
\vs p084 3:9 El suministro de leche a los pequeños llevó a un destete más temprano de los bebés y, en consecuencia, a tener más hijos, porque de esta manera las madres quedaban liberadas de su esterilidad temporal, en tanto que el empleo de la leche de vaca y de cabra redujo notablemente la mortalidad infantil. Antes de la etapa pastoril de la sociedad, las madres acostumbraban a amamantar a sus hijos hasta los cuatro o cinco años de edad.
\vs p084 3:10 La disminución de las guerras primitivas redujo de forma significativa las disparidades existentes en la división del trabajo basada en el sexo. Pero las mujeres aún tenían que hacer el verdadero trabajo mientras que los hombres hacían labores de vigilancia. No se podía dejar a un campamento o aldea sin vigilar ni de día ni de noche, pero incluso esta tarea se alivió con la domesticación del perro. En general, la aparición de la agricultura aumentó el prestigio y la posición social de la mujer; al menos, así ocurrió hasta el momento en el que el hombre mismo se convirtió en agricultor. Y en cuanto el hombre se dedicó a cultivar la tierra, hubo, de inmediato, una gran mejora en los métodos agrícolas, que se transmitió a las sucesivas generaciones. En la caza y en la guerra, el hombre había aprendido el valor de la organización e introdujo estas técnicas en el ámbito laboral, y, más tarde, cuando asumió una gran parte del trabajo de la mujer, mejoró en buena parte los métodos poco rígidos de ella.
\usection{4. EL ESTATUS DE LA MUJER EN LA SOCIEDAD PRIMITIVA}
\vs p084 4:1 En líneas generales, durante cualquier época, el estatus de la mujer es un criterio justo para medir el proceso evolutivo del matrimonio como institución social, mientras que el progreso del matrimonio mismo es un indicador razonablemente exacto de los avances de la civilización humana.
\vs p084 4:2 \pc El estatus de la mujer ha sido siempre una paradoja social; ha dirigido continuamente al hombre con astucia; ha aprovechado constantemente el impulso sexual más fuerte de este en su propio interés y beneficio. Utilizando sutilmente sus encantos sexuales, ha sido muchas veces capaz de ejercer un poder dominante sobre el hombre, incluso cuando este la esclavizaba de forma abyecta.
\vs p084 4:3 La mujer primitiva no era para el hombre una amiga, ni una amante ni una compañera, sino más bien una propiedad, una sierva o una esclava y, más adelante, una socia económica, un juguete y una procreadora. No obstante, las relaciones sexuales para que fuesen adecuadas y satisfactorias siempre han conllevado el elemento de elección y cooperación por parte de la mujer, algo que ha otorgado siempre a las mujeres inteligentes una influencia notable en cuanto a su situación personal inmediata, pese a la posición social de su sexo. Pero, el hecho de que la mujer se vio en todo tiempo obligada a recurrir a la astucia para aliviar su sometimiento agravó la desconfianza y la sospecha del hombre.
\vs p084 4:4 \pc Los sexos han tenido grandes dificultades para entenderse entre ellos. El hombre encontraba difícil comprender a la mujer; la miraba con una extraña mezcla de ignorante recelo y fascinación temerosa, cuando no con sospecha y desprecio. Muchas tradiciones tribales y raciales inculpaban de todos los problemas a Eva, a Pandora o a alguna representante del género femenino. Estos relatos siempre se tergiversaron a fin de que pareciera que la mujer traía el mal sobre el hombre; y todo ello muestra que el sentimiento de desconfianza hacia la mujer fue algo generalizado. Entre las razones que se citaban en apoyo del sacerdocio célibe, el principal era la vileza de la mujer. El hecho de que la mayoría de las supuestas brujas eran mujeres no mejoró la antigua reputación de este sexo.
\vs p084 4:5 Los hombres han considerado durante mucho tiempo que las mujeres eran peculiares, incluso anormales. Hasta han llegado a creer que las mujeres no tenían alma, algo por lo que se les negaba el nombre. Durante los tiempos primitivos existía un gran temor a la primera relación sexual con una mujer; de ahí que se volviera costumbre que el sacerdote fuese el primero en copular con una virgen. Inclusive se pensaba que hasta la sombra de una mujer era peligrosa.
\vs p084 4:6 Alguna vez se creyó que la maternidad volvía a la mujer peligrosa e impura. Y muchas costumbres tribales prescribían que la madre debía someterse a largas ceremonias de purificación tras el nacimiento de un hijo. Salvo entre aquellos grupos en los que el marido participaba en el nacimiento, se rechazaba a la mujer embarazada, se la dejaba sola. Los antiguos incluso evitaban que el niño naciese dentro de la casa. Finalmente, se permitió a las mujeres mayores atender a la madre durante el parto, y esta práctica dio origen a la profesión de partera. Durante el parto, se decían y hacían decenas de insensateces con el fin de facilitar el alumbramiento. Era costumbre rociar al recién nacido con agua bendita para impedir la injerencia de los espectros.
\vs p084 4:7 Entre las tribus no mezcladas, el parto era relativamente fácil, necesitándose solamente dos o tres horas; raras veces era tan sencillo entre las razas mezcladas. Si una mujer moría de parto, especialmente durante el alumbramiento de mellizos, se creía que había sido culpable de adulterio espiritual. Más adelante, las tribus más avanzadas pensaban que la muerte durante el parto era voluntad del cielo; se consideraba que tales madres habían muerto por una causa noble.
\vs p084 4:8 La llamada modestia de las mujeres respecto a la ropa y a la exposición de su persona a los demás surgieron del temor fatal de ser observada durante el período menstrual. Ser así vistas constituía un pecado grave, la violación de un tabú. Bajo las costumbres de los tiempos antiguos, cualquier mujer, desde la adolescencia hasta el fin de su vida fértil, estaba sometida a una cuarentena familiar y social total durante una semana completa cada mes. Todo lo que ella tocaba, donde se sentaba o yacía era “impuro”. Durante mucho tiempo era costumbre golpear brutalmente a una joven después de su ciclo mensual en un intento por expulsar al espíritu maligno de su cuerpo. Si bien, cuando la mujer pasaba más allá de la edad fecunda, se la trataba normalmente con más consideración y contaba con más derechos y privilegios. Ante todo esto, no era extraño que se menospreciara a las mujeres. Incluso los griegos mantenían que una mujer con menstruación era una de las tres grandes causas de impureza; las otras eran la carne de cerdo y el ajo.
\vs p084 4:9 Por muy insensatas que fuesen estas antiguas nociones, tuvieron algún efecto positivo, puesto que se concedía a las mujeres, sobrecargadas de trabajo, al menos cuando eran jóvenes, un bienvenido descanso y una meditación provechosa, una semana al mes. De esta manera, podían agudizar el ingenio para tratar con sus compañeros masculinos el resto del tiempo. Esta cuarentena de las mujeres también protegía a los hombres de los excesos sexuales, con lo que se contribuía indirectamente a restringir la población y al mejoramiento del autocontrol.
\vs p084 4:10 \pc Se hizo un gran avance cuando se le negó al hombre el derecho de matar a su mujer a voluntad. De igual manera, se dio un paso adelante cuando la mujer pudo ser la dueña de los obsequios de boda. Más tarde, obtuvo el derecho legal de poseer, tener el control y hasta deshacerse de propiedades, pero, durante mucho tiempo, se le privó del derecho a ocupar puestos en la Iglesia o en el Estado. La mujer ha sido tratada siempre prácticamente como una propiedad, hasta el siglo XX d. C., y en dicho siglo también. Aun no ha conseguido liberarse, a nivel mundial, de la reclusión que el hombre le ha impuesto. Incluso entre los pueblos avanzados, el intento del hombre por proteger a la mujer siempre ha sido una afirmación tácita de superioridad.
\vs p084 4:11 Pero las mujeres primitivas no se compadecían de ellas mismas, como sus hermanas, más recientemente liberadas, suelen hacer. Al fin y al cabo, se sentían bastante felices y contentas. No se atrevían a imaginar una forma de existencia diferente o mejor.
\usection{5. LA MUJER BAJO LAS COSTUMBRES EN DESARROLLO}
\vs p084 5:1 En su autoperpetuación, la mujer es igual al hombre, si bien, en la cooperación para el automantenimiento, ella sufre una clara desventaja, y este inconveniente de la maternidad forzada tan solo puede compensarse por costumbres avanzadas en una civilización en evolución y por la adquisición, por parte del hombre, de un mayor sentido de la equidad.
\vs p084 5:2 Conforme evolucionó la sociedad, las normas sexuales se promovieron más entre las mujeres, porque ellas sufrían, en mayor grado, las consecuencias de la transgresión de las costumbres sexuales. Las normas sexuales del hombre están solo mejorando poco a poco como resultado del puro sentido de esa equidad que demanda la civilización. La naturaleza no sabe nada de equidad ---permite que la mujer sufra sola los dolores del parto---.
\vs p084 5:3 La idea moderna de la igualdad de los sexos es bella y digna de una civilización en expansión, pero no se halla en la naturaleza. Cuando la fuerza es la ley, el hombre se enseñorea sobre la mujer; cuando la justicia, la paz y la equidad son más predominantes, la mujer emerge paulatinamente de la esclavitud y de la oscuridad. Por lo general, la posición social de la mujer ha variado de manera inversa al grado de militarismo de cualquier nación o época.
\vs p084 5:4 Pero el hombre no se apoderó de forma consciente ni intencionada de los derechos de la mujer para luego devolvérselos de forma gradual y a regañadientes; todo esto fue un episodio accidental y no planificado de la evolución social. Cuando llegó realmente el momento de que la mujer disfrutara de mayores derechos, los obtuvo, y todo ello muy a pesar de la actitud consciente del hombre. Lentamente pero de forma segura, las costumbres cambian para proporcionar esas adaptaciones sociales que son parte de la constante evolución de la civilización. En su avance, las costumbres facilitaron con lentitud un mejor tratamiento de las mujeres; las tribus que insistieron en su crueldad hacia ellas, no sobrevivieron.
\vs p084 5:5 \pc Los adanitas y los noditas dieron un mayor reconocimiento a las mujeres, y los grupos sobre los que influyeron los anditas en sus migraciones tendieron a dejarse influir por las enseñanzas edénicas respecto a la posición de la mujer en la sociedad.
\vs p084 5:6 Los antiguos chinos y los griegos trataban a las mujeres mejor que lo hizo la mayoría de los pueblos aledaños. Pero los hebreos tenían una gran desconfianza hacia ellas. En occidente, la mujer ha tenido que escalar socialmente con dificultades debido a las doctrinas paulinas que se adhirieron al cristianismo, aunque el cristianismo, al imponer obligaciones sexuales más rigurosas a los hombres, hizo avanzar las costumbres. El estado de la mujer es poco menos que desesperanzador ante la particular degradación que sufre bajo el mahometismo, y es todavía peor bajo las enseñanzas de otras distintas religiones orientales.
\vs p084 5:7 \pc La ciencia, no la religión, emancipó realmente a la mujer; fue la fábrica moderna la que en buena medida la liberó de los confines del hogar. La capacidad física del hombre dejó de ser un elemento indispensable en el nuevo mecanismo de mantenimiento; la ciencia cambió de tal manera las condiciones de vida que la mano de obra del hombre ya no era tan superior a la de la mujer.
\vs p084 5:8 Estos cambios han conducido a la liberación de la mujer de la esclavitud doméstica, y han traído consigo tal modificación de su estatus que disfruta ahora de un grado de libertad personal y de determinación sexual que prácticamente iguala a la del hombre. Alguna vez, el valor de la mujer consistía en su habilidad para proporcionar alimentos, pero la invención y la riqueza le han permitido crear un nuevo mundo en el que actuar ---entornos de gracia y encanto---. De este modo, la industria ha ganado su lucha inconsciente e involuntaria por la emancipación social y económica de la mujer. Y, nuevamente, la evolución ha logrado hacer lo que ni incluso la revelación ha podido.
\vs p084 5:9 \pc La reacción de los pueblos evolucionados hacia las costumbres injustas que regían el lugar de la mujer en la sociedad ha sido, de hecho, como un péndulo en su movimiento de un extremo a otro. Entre las razas industrializadas, se le ha otorgado casi todos los derechos y disfruta de la exención de muchas obligaciones, tales como el servicio militar. Cada mitigación de la lucha por la existencia ha redundado en la liberación de la mujer, que se ha beneficiado directamente de cualquier avance hacia la monogamia. En la evolución progresiva de la sociedad, los más débiles siempre consiguen beneficios desproporcionados con cada adaptación de las costumbres.
\vs p084 5:10 En los ideales del matrimonio en pareja, la mujer ha obtenido finalmente reconocimiento, dignidad, independencia, igualdad y educación; pero, ¿demostrará que es digna de todos estos logros nuevos y sin precedentes? ¿Responderá la mujer moderna a este gran logro de su liberación social con ociosidad, indiferencia, infertilidad e infidelidad? ¡Hoy, en día, en el siglo XX, la mujer está pasando por la prueba crucial de su larga existencia en el mundo!
\vs p084 5:11 La mujer es la compañera en paridad del hombre en la reproducción de la raza, de ahí que sea igualmente importante en el despliegue de la evolución racial; por lo tanto, la evolución ha contribuido de forma creciente a la consecución de los derechos de la mujer. Pero sus derechos no son de ninguna manera los del hombre. La mujer no puede prosperar con los derechos del hombre, como tampoco puede hacerlo el hombre con los derechos de la mujer.
\vs p084 5:12 Cada sexo tiene su propio y distintivo entorno en el que se desenvuelve su existencia, junto con sus propios derechos dentro de dicho entorno. Si la mujer aspira literalmente a disfrutar de todos los derechos del hombre, entonces, antes o después, se reemplazará la caballerosidad y la consideración especial de la que muchas mujeres disfrutan hoy en día, y que tan recientemente han ganado de los hombres, por una competitividad implacable y fría.
\vs p084 5:13 La civilización nunca podrá borrar el abismo que media en el comportamiento de los sexos. Era tras era las costumbres cambian, pero el instinto nunca lo hace. El afecto innato del instinto maternal nunca permitirá a la mujer emancipada que se convierta en una seria rival del hombre en el mundo laboral. Por siempre cada sexo se mantendrá supremo en su propio dominio, en esos dominios determinados por la diferenciación biológica y la disimilitud mental.
\vs p084 5:14 Cada sexo tendrá siempre su propia esfera especial, aunque por siempre y para siempre se solapen. Hombres y mujeres solo competirán en igualdad de condiciones en el ámbito social.
\usection{6. COOPERACIÓN ENTRE HOMBRE Y MUJER}
\vs p084 6:1 El impulso reproductor une, indefectiblemente, a hombres y mujeres para la perpetuación de sí mismos, pero, por sí solo, este impulso no garantiza que permanezcan juntos en mutua cooperación ---en la fundación de un hogar---.
\vs p084 6:2 Cualquier institución humana exitosa entraña la existencia de unos antagonismos de intereses personales que se han coordinado para actuar de manera práctica y armónica, y la creación del hogar no es una excepción. El matrimonio, base de la formación del hogar, es la manifestación más elevada de esa cooperación antagónica que con tanta frecuencia caracteriza los contactos entre la naturaleza y la sociedad. El conflicto es inevitable. El emparejamiento es inherente al ser humano; es natural. Pero el matrimonio no es biológico; es sociológico. La pasión asegura la unión entre el hombre y la mujer, pero el instinto parental, más débil, y las costumbres sociales son los que los mantienen juntos.
\vs p084 6:3 \pc Desde un punto de vista práctico, el hombre y la mujer son dos variedades distintas de la misma especie que viven en relación íntima y estrecha. Sus perspectivas y reacciones hacia la vida son esencialmente diferentes; son totalmente incapaces de comprenderse entre ellos de una manera plena y auténtica. El entendimiento completo entre los sexos es inalcanzable.
\vs p084 6:4 Las mujeres parecen tener más intuición que los hombres, pero también parecen ser algo menos lógicas. La mujer, sin embargo, siempre ha sido la adalid moral y la líder espiritual de la humanidad. La mano que mece la cuna todavía confraterniza con el destino.
\vs p084 6:5 \pc La diferencia de naturaleza, reacción, punto de vista y pensamiento entre hombres y mujeres, lejos de causar preocupación, debería considerarse sumamente beneficiosa para la humanidad, tanto de manera individual como colectiva. Muchos órdenes de criaturas del universo se crean con una doble manifestación de su ser personal. Entre los mortales, los hijos materiales y los midsonitas esta diferencia se describe como masculina y femenina; entre los serafines, querubines y acompañantes morontiales se ha denominado positiva o enérgica y negativa o reservada. Estas correlaciones dobles multiplican, sobremanera, la versatilidad y vencen las limitaciones intrínsecas existentes, como ocurre con ciertas vinculaciones trinas en el sistema Paraíso\hyp{}Havona.
\vs p084 6:6 El hombre y la mujer se necesitan el uno al otro en sus andaduras morontiales y espirituales, al igual que en las mortales. Las diferencias de puntos de vista entre ambos sexos perduran incluso más allá de la primera vida y a todo lo largo de su ascensión en el universo local y en el suprauniverso. E incluso en Havona, los peregrinos que alguna vez fueron hombres y mujeres seguirán ayudándose unos a otros en el ascenso al Paraíso. Nunca, ni siquiera en el colectivo de los finalizadores, será tan grande la metamorfosis de la criatura como para borrar las tendencias del ser personal que los humanos llaman masculino y femenino; estas dos variaciones elementales de la humanidad siempre continuarán interesándose, dándose aliento, inspirándose valor y asistiéndose mutuamente; siempre dependerán de su común cooperación para solucionar los desconcertantes problemas del universo y para superar las múltiples dificultades cósmicas.
\vs p084 6:7 \pc Aunque los sexos nunca podrán abrigar la esperanza de comprenderse del todo el uno al otro, son eficazmente complementarios, y aunque su cooperación sea a menudo más a menos personalmente antagónica, es capaz de mantener y reproducir la sociedad. El matrimonio es una institución concebida para integrar las diferencias sexuales, a la vez que lleva a cabo la continuación de la civilización y garantiza la reproducción de la raza.
\vs p084 6:8 El matrimonio es la madre de todas las instituciones humanas, puesto que lleva directamente a la fundación y a la preservación del hogar, que es la base estructural de la sociedad. La familia está vitalmente ligada al mecanismo del automantenimiento; es la única posibilidad de perpetuación de la raza bajo las costumbres de la civilización, mientras que, al mismo tiempo, proporciona eficazmente ciertas formas altamente satisfactorias de autogratificación. La familia es el principal logro puramente humano del hombre, al combinar, como de hecho hace, la evolución de las relaciones biológicas entre ambos sexos con las relaciones sociales entre marido y mujer.
\usection{7. LOS IDEALES DE LA VIDA FAMILIAR}
\vs p084 7:1 El emparejamiento sexual es instintivo, los hijos son el resultado natural y la familia, por consiguiente, empieza a existir de manera automática. Como lo son las familias de una raza o de una nación, así es su sociedad. Si las familias son buenas, la sociedad es igualmente buena. La gran estabilidad cultural de los pueblos judío y chino radica en la fuerza de sus grupos familiares.
\vs p084 7:2 El instinto femenino de amar y cuidar de los hijos contribuyó a hacer de ella la parte interesada en el fomento del matrimonio y de la vida familiar primitiva. Solo la presión de las posteriores costumbres y convenciones sociales forzaron al hombre a formar un hogar; tardó en interesarse por la instauración del matrimonio y del hogar, porque el acto sexual no impone sobre él consecuencias biológicas.
\vs p084 7:3 La relación sexual es natural, pero el matrimonio es social y siempre ha estado regulado por las costumbres. Las costumbres (religiosas, morales y éticas), junto con la propiedad, el orgullo y la caballerosidad, estabilizan las instituciones del matrimonio y la familia. Siempre que fluctúan, hay fluctuación en la estabilidad de estas instituciones. En estos momentos, el matrimonio está abandonando la etapa de la propiedad para adentrarse en la era de lo personal. Anteriormente, el hombre protegía a la mujer porque ella era su propiedad, y ella obedecía por la misma razón. Al margen de sus méritos, este sistema realmente proporcionaba estabilidad. Ahora, la mujer ha dejado de considerarse una propiedad, y están surgiendo nuevas costumbres que tienen por objeto estabilizar la institución matrimonio\hyp{}hogar:
\vs p084 7:4 \li{1.}El nuevo papel de la religión: la enseñanza de que la experiencia parental es fundamental, la idea de procrear ciudadanos cósmicos, el entendimiento ampliado de los privilegios de la procreación, esto es, dar hijos al Padre.
\vs p084 7:5 \li{2.}El nuevo papel de la ciencia: la procreación se está volviendo cada vez más voluntaria, sujeta al control del hombre. En los tiempos antiguos la falta de entendimiento garantizaba la aparición de hijos en ausencia del deseo de tenerlos.
\vs p084 7:6 \li{3.}La nueva función de la seducción del placer: esto introduce un nuevo factor en la supervivencia racial; el hombre antiguo sometía a los hijos no deseados a la muerte; los modernos se niegan a tenerlos.
\vs p084 7:7 \li{4.}El realce del instinto paterno: ahora cada generación tiende a separar, de la corriente reproductora de la raza, a aquellos individuos en los que el instinto paterno no es lo suficientemente fuerte como para garantizar la procreación de hijos, para ser los futuros padres de la próxima generación.
\vs p084 7:8 \pc Pero el hogar es una institución, una asociación entre un hombre y una mujer, que data más concretamente de los días de Dalamatia, hace alrededor de medio millón de años, cuando se habían abandonado con bastante antelación las costumbres monógamas de Andón y de sus descendientes inmediatos. La vida familiar, sin embargo, no era para enorgullecerse antes de los días de los noditas y de los posteriores adanitas. Adán y Eva ejercieron una influencia duradera sobre toda la humanidad; por primera vez en la historia del mundo, se pudo observar a hombres y mujeres trabajando codo a codo en el Jardín. El ideal edénico, toda la familia dedicada a la horticultura, era una idea nueva en Urantia.
\vs p084 7:9 La familia primitiva constaba de un grupo de trabajo conexo, incluido los esclavos, todos viviendo en una sola morada. El matrimonio y la vida familiar no han sido siempre idénticos, pero, por necesidad, han estado siempre estrechamente relacionados. La mujer siempre ha querido tener su propia familia individual y, con el tiempo, consiguió lo que se propuso.
\vs p084 7:10 \pc El amor a los vástagos es casi universal y posee un claro valor para la supervivencia. Los antiguos siempre sacrificaban los intereses de la madre por el bienestar del niño; una madre esquimal, incluso todavía, lame a su niño en lugar de bañarlo. Pero las madres primitivas solo alimentaban y cuidaban de sus hijos cuando estos eran muy pequeños; como los animales, los desatendían en cuanto crecían. Las relaciones humanas perdurables y continuadas nunca se han fundado en el afecto biológico solamente. Los animales aman a sus crías; el hombre ---el hombre civilizado--- ama a los hijos de sus hijos. Cuanto más elevada sea la civilización, mayor será el gozo de los padres por el avance y éxito de sus hijos; así es como se da comienzo a una nueva y mayor toma de conciencia del orgullo \bibemph{por el nombre}.
\vs p084 7:11 Entre los pueblos ancestrales, las familias grandes no eran necesariamente afectivas. Se deseaba tener muchos hijos porque:
\vs p084 7:12 \li{1.}Eran valiosos como trabajadores.
\vs p084 7:13 \li{2.}Eran un seguro para la vejez.
\vs p084 7:14 \li{3.}Las hijas se podían vender.
\vs p084 7:15 \li{4.}El orgullo familiar requería la prolongación del nombre.
\vs p084 7:16 \li{5.}Los hijos proporcionaban protección y defensa.
\vs p084 7:17 \li{6.}El miedo a los espectros creó el temor a estar solos.
\vs p084 7:18 \li{7.}Algunas religiones exigían vástagos.
\vs p084 7:19 \pc Los adoradores de los ancestros creían que no tener hijos era la absoluta desgracia de todos los tiempos y de la eternidad. Por encima de todo, deseaban tenerlos para que ellos oficiaran sus festivales post mortem y ofrecieran los sacrificios precisos para que el espectro avanzase en el mundo de los espíritus.
\vs p084 7:20 Entre los salvajes antiguos, la disciplina de los hijos se iniciaba muy tempranamente; y el niño pronto se daba cuenta de que la desobediencia significaba su perdición y e incluso la muerte, tal como sucedía en el caso de los animales. Es la protección que la civilización ofrece al niño contra las consecuencias naturales de su conducta insensata la que tanto contribuye a su insubordinación en la época moderna.
\vs p084 7:21 Los niños esquimales crecen con tan poca disciplina y enmienda, porque sencillamente son, por naturaleza, animalitos dóciles; los hijos de los hombres rojos y de los amarillos son casi igualmente tratables. Pero, en las razas que contienen la herencia andita, los niños no son tan apacibles; estos jóvenes, más imaginativos e intrépidos, requieren una mayor formación y disciplina. Los problemas modernos de la educación de los niños se vuelven cada vez más difíciles debido a:
\vs p084 7:22 \li{1.}El alto grado de mezcla racial.
\vs p084 7:23 \li{2.}La educación artificial y superficial.
\vs p084 7:24 \li{3.}La imposibilidad de que el niño adquiera cultura imitando a sus padres: estos se encuentran ausentes del ámbito familiar durante gran parte del tiempo.
\vs p084 7:25 \pc Las antiguas ideas sobre la disciplina familiar eran biológicas; nacían de la creencia de que los padres eran los creadores del hijo. El avance de los ideales en cuanto a la vida familiar llevan al concepto de que traer un hijo al mundo, en lugar de conferir algunos derechos parentales, entraña adquirir la suprema responsabilidad de la existencia humana.
\vs p084 7:26 La civilización considera que los padres asumen todos los deberes, mientras que el hijo posee todos los derechos. El respeto del hijo hacia su padre no brota del conocimiento de la obligación implícita en la procreación parental, sino que se origina, de manera natural, como resultado del cuidado, formación y afecto que manifiestan cariñosamente sus padres al ayudarle a ganar la batalla de la vida. El verdadero padre participa en un continuo servicio\hyp{}ministerio que su sensato hijo llega a reconocer y apreciar.
\vs p084 7:27 \pc En la actual era industrial y urbana, la institución del matrimonio se está desarrollando en nuevos frentes económicos. La vida familiar se ha vuelto más y más costosa, mientras que los hijos, que solían ser un activo, se han convertido en un pasivo económico. Pero la seguridad de la civilización misma todavía radica en la creciente disposición de una generación de invertir en el bienestar de las generaciones próximas y futuras. Y cualquier intento que se haga por transferir la responsabilidad parental al Estado o a la Iglesia resultará suicida para el bienestar y el avance de la civilización.
\vs p084 7:28 \pc El matrimonio, con los hijos y con la consiguiente vida familiar, estimula los potenciales más elevados de la naturaleza humana y proporciona, de manera simultánea, una vía idónea para la expresión de esos atributos vivificados de la persona de los mortales. La familia otorga la perpetuación biológica de la especie humana. El hogar es el escenario social natural en el que los niños al crecer pueden llegar a comprender la ética de la hermandad de la sangre. La familia es la unidad fundamental de la fraternidad en la que padres e hijos aprenden esas lecciones de paciencia, altruismo, indulgencia y benevolencia, que son tan esenciales para la realización de la hermandad entre los hombres.
\vs p084 7:29 La sociedad humana mejoraría de manera considerable si las razas civilizadas volviesen a la tradición de los consejos de familia de los anditas. Estos no mantenían una forma patriarcal o autocrática de gobierno familiar. Eran muy fraternales y asociativos; discutían con franqueza y libertad todas las propuestas y regulaciones de naturaleza familiar. Eran idóneamente fraternales en cualquier gobierno familiar. En una familia ideal, el afecto filial y el afecto parental aumentan por medio de la devoción fraternal.
\vs p084 7:30 La vida familiar es la progenitora de la verdadera moralidad, el ancestro de la conciencia de la lealtad al deber. Las interrelaciones forzosas de la vida familiar estabilizan la persona y estimulan su crecimiento mediante la obligatoriedad de una necesaria adaptación a otras personas distintas. Pero incluso más, la verdadera familia ---una buena familia--- revela a los procreadores parentales la actitud del Creador hacia sus hijos, mientras que, al mismo tiempo, estos verdaderos padres ilustran para sus hijos la primera de una larga serie de revelaciones ascendentes del amor del Padre del Paraíso hacia todos sus hijos del universo.
\usection{8. LOS PELIGROS DE LA AUTOGRATIFICACIÓN}
\vs p084 8:1 El gran peligro que pesa sobre la vida familiar es la amenazante y creciente marejada de autogratificación: la manía moderna del placer. El principal incentivo al matrimonio solía ser económico; la atracción sexual era secundaria. El matrimonio, fundado en el automantenimiento, llevaba a la autoperpetuación y conjuntamente proporcionaba una de las formas más deseables de autogratificación. Es la única institución de la sociedad humana que incluye los tres grandes incentivos de la vida.
\vs p084 8:2 Originariamente, la propiedad privada era la institución básica del automantenimiento, mientras que el matrimonio obraba como la única institución de la autoperpetuación. Aunque la satisfacción alimentaria, el esparcimiento y el humor, junto con la gratificación sexual periódica, eran modos de autogratificación, sigue siendo cierto que las costumbres en su evolución no han logrado crear una institución específica para la autogratificación. Y es debido a este fracaso en el desarrollo de métodos particulares para el disfrute del placer que todas las instituciones humanas estén inmersas en su búsqueda. La acumulación de la propiedad se está volviendo un instrumento para aumentar todas las formas de autogratificación, mientras que el matrimonio, con frecuencia, se considera solamente un medio de placer. Esta satisfacción excesiva, esta manía tan extendida de encontrar placer, constituye en estos momentos la mayor amenaza que jamás antes se haya dirigido contra la institución evolutiva social de la vida familiar: el hogar.
\vs p084 8:3 La raza violeta introdujo un rasgo nuevo y solo imperfectamente comprendido en la experiencia de la humanidad: el instinto del esparcimiento aunado al sentido del humor. Ya existía, en cierta medida, entre los sangiks y los andonitas, pero la estirpe adánica elevó esta tendencia primitiva hasta convertirse en un \bibemph{potencial de placer,} en una forma nueva y glorificada de autogratificación. El tipo elemental de autogratificación, aparte del apaciguamiento del hambre de alimentos, es la gratificación sexual, y esta forma de placer sensual se agudizó enormemente por la mezcla de los sangiks y los anditas.
\vs p084 8:4 Existe un verdadero peligro en la combinación de la inquietud, la curiosidad, la aventura y el abandono al placer, característico de las razas posteriores a los anditas. El hambre del alma no puede satisfacerse con los placeres físicos; el amor del hogar y de los hijos no aumenta con la búsqueda irreflexiva del placer. Aunque agotéis los recursos del arte, el color, el sonido, el ritmo, la música y los ornamentos de la persona, no podéis pretender elevar así el alma ni nutrir el espíritu. La vanidad y la moda no pueden ayudar a la formación del hogar ni a la educación de los niños: el orgullo y la rivalidad son impotentes para mejorar las cualidades de supervivencia de las generaciones venideras.
\vs p084 8:5 En su progreso, todos los seres celestiales disfrutan del descanso y del ministerio de los directores de reversión. Es acertado esforzarse por lograr una diversión saludable y participar en juegos edificantes; merece la pena disfrutar del sueño reparador, del descanso, del entretenimiento y de todos los pasatiempos que impiden el aburrimiento por monotonía. Los juegos competitivos, las narraciones e incluso el sabor de la buena comida pueden servir como formas de autogratificación. (Cuando empleáis la sal para sazonar el alimento, considerad con detenimiento que, durante casi un millón de años, el hombre tan solo podía conseguir sal mojándolo en las cenizas.)
\vs p084 8:6 \pc Dejad que el hombre disfrute; dejad que la raza humana encuentre el placer de mil y una maneras; dejad que la humanidad evolutiva explore todas las formas legítimas de autogratificación, frutos de su larga lucha biológica hacia adelante. El hombre tiene bien merecido algunos de sus actuales gozos y placeres. ¡Pero mirad bien hacia la meta de vuestro destino! Los placeres son de hecho suicidas si logran destruir la propiedad privada, que se ha convertido en la institución del automantenimiento; y la autogratificación acarrea en efecto un fatídico precio fatal si lleva aparejado el colapso del matrimonio, la decadencia de la vida familiar y la destrucción del hogar ---la adquisición evolutiva suprema del hombre y la única esperanza de supervivencia de la civilización---.
\vsetoff
\vs p084 8:7 [Exposición del jefe de serafines emplazados en Urantia.]
