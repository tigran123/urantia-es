\upaper{11}{La Isla eterna del Paraíso}
\author{Perfeccionador de la sabiduría}
\vs p011 0:1 El Paraíso es el centro eterno del universo de los universos y la morada del Padre Universal, del Hijo Eterno, del Espíritu Infinito y de sus divinos iguales en rango entre sí y colaboradores. Esta isla central es la masa organizada de realidad cósmica más gigantesca de todo el universo matriz. El Paraíso es una esfera material al igual que una morada espiritual. Toda la creación inteligente del Padre Universal reside en moradas materiales; por consiguiente, el centro rector absoluto debe ser también material, real. Y de nuevo debe reiterarse que las cosas y los seres espirituales son \bibemph{reales}.
\vs p011 0:2 La belleza material del Paraíso consiste en la magnificencia de su perfección física; la grandiosidad de la Isla de Dios se manifiesta en los excelentes logros intelectuales y en el desarrollo mental de sus habitantes; la gloria de la Isla central se muestra en la infinita dotación del ser personal espiritual y divino, de la luz de la vida. Pero la intensidad de la belleza espiritual y las maravillas de este conjunto magnífico sobrepasan por completo la comprensión de la mente finita de las criaturas materiales. La gloria y el esplendor espiritual de la morada divina son inimaginables para los mortales. Y el Paraíso existe desde la eternidad; no existe constancia ni tradición alguna en cuanto al origen de esta isla nuclear de luz y vida.
\usection{1. EL LUGAR DE RESIDENCIA DIVINO}
\vs p011 1:1 El Paraíso tiene muchos cometidos en relación con la administración de los dominios universales, pero para los seres creados existe esencialmente como morada de la Deidad. La presencia personal del Padre Universal reside en el centro mismo de la superficie superior de esta morada de las Deidades, cuya forma es casi circular, pero no esférica. Esta presencia del Padre Universal en el Paraíso está rodeada de cerca por la presencia personal del Hijo Eterno, mientras que ambos están investidos de la inefable gloria del Espíritu Infinito.
\vs p011 1:2 Dios habita, ha habitado y habitará por siempre en esta misma morada central y eterna. Siempre lo hemos hallado allí y siempre allí lo hallaremos. El Padre Universal converge cósmicamente, se hace personal espiritualmente y reside geográficamente en este centro del universo de los universos.
\vs p011 1:3 \pc Todos conocemos la ruta que hemos de seguir para encontrar al Padre Universal. Vosotros no podéis comprender mucho acerca de la morada divina debido a la remota distancia que está de vosotros y a la inmensidad del espacio que os separa de ella, pero los que pueden entender el significado de estas distancias enormes conocen la ubicación y la morada de Dios con tanta certeza y exactitud como vosotros conocéis la ubicación de Nueva York, Londres, Roma o Singapur, ciudades claramente situadas en la geografía de Urantia. Si fueras un hábil navegante y estuvieras equipado de nave, mapas y brújula, podrías encontrar estas ciudades con facilidad. Asimismo, si tuvieras el tiempo y el modo de viajar, estuvieras capacitado espiritualmente y contaras con la guía necesaria, podrías pilotar de universo en universo y circular vía tras vía, viajando siempre hacia el interior a través de los dominios estelares, hasta que por fin te hallaras ante el resplandor central de la gloria espiritual del Padre Universal. Provisto de todo lo necesario para el viaje, resulta tan factible encontrar la presencia personal de Dios en el centro de todas las cosas como lo es encontrar ciudades distantes en vuestro propio planeta. El hecho de que no hayas visitado estos sitios no niega en modo alguno su realidad ni su existencia como tal. El hecho de que tan pocas criaturas del universo hayan encontrado a Dios en el Paraíso niega, en modo alguno, ni la realidad de su existencia ni la de su persona espiritual en el centro de todas las cosas.
\vs p011 1:4 Siempre se encontrará al Padre localizado en este centro. Si él se trasladara, se desencadenaría el caos universal, porque en él convergen, en este centro de residencia, las líneas universales de la gravedad desde los confines de la creación. Ya sea que remontemos la vía circulatoria del ser personal a través de los universos o sigamos a los seres personales que ascienden en dirección interna en su viaje hacia el Padre, ya sea que remontemos las líneas de la gravedad material hasta el Paraíso inferior o sigamos los ciclos pulsantes de la fuerza cósmica, ya sea que remontemos las líneas de la gravedad espiritual hasta el Hijo Eterno o sigamos la marcha en dirección interna de los Hijos de Dios del Paraíso, ya sea que remontemos las vías circulatorias de la mente o sigamos los billones de billones de seres celestiales que surgen del Espíritu Infinito, mediante cualquiera o todas estas observaciones se nos conducirá directamente hasta la presencia del Padre, en su morada central. Aquí Dios está personal, auténtica y realmente presente, y desde su ser infinito fluye el caudal de la vida, de la energía y del ser personal para todos los universos.
\usection{2. NATURALEZA DE LA ISLA ETERNA}
\vs p011 2:1 Puesto que estáis comenzando a vislumbrar la enormidad del universo material perceptible incluso desde vuestra ubicación astronómica, desde vuestra posición espacial en los sistemas estelares, debería ser evidente para vosotros que un universo material tan colosal ha de contar con una capital adecuada y digna, con una sede central a la altura de la dignidad e infinidad del Gobernante universal de toda esa extensa creación de dominios materiales y de seres vivos.
\vs p011 2:2 \pc El Paraíso, en su forma, se diferencia de los cuerpos espaciales habitados por no ser esférico. Es claramente elipsoide, siendo un sexto más largo en su diámetro norte\hyp{}sur que en su diámetro este\hyp{}oeste. La Isla central es esencialmente plana, y la distancia desde la superficie superior hasta la superficie inferior es un décimo del diámetro este\hyp{}oeste.
\vs p011 2:3 Estas diferencias en dimensiones, junto con su condición estacionaria y la mayor presión exterior de la fuerza\hyp{}energía en el extremo norte de la Isla, permiten establecer su absoluta orientación en el universo matriz.
\vs p011 2:4 \pc La Isla central se divide geográficamente en tres ámbitos de actividad:
\vs p011 2:5 \li{1.}El Paraíso superior.
\vs p011 2:6 \li{2.}El Paraíso periférico.
\vs p011 2:7 \li{3.}El Paraíso inferior.
\vs p011 2:8 \pc La superficie del Paraíso, el lugar de actividad de los seres personales, se define como la zona superior y, la superficie opuesta, como la zona inferior. La periferia del Paraíso acomoda una actividad que no es, en sentido estricto, ni personal ni no personal. La Trinidad parece dominar el plano personal o superior, el Absoluto Indeterminado, el plano inferior o impersonal. No creemos de ninguna manera que el Absoluto Indeterminado sea una persona, pero sí consideramos que la presencia operativa en el espacio de este Absoluto tiene su punto de convergencia en el Paraíso inferior.
\vs p011 2:9 \pc La Isla eterna está formada de una sola materia ---sistemas estacionarios de realidad---. Esta sustancia real del Paraíso, organizada homogéneamente a partir de la potencia espacial, no se encuentra en ninguna otra parte del amplio universo de los universos. Ha recibido muchos nombres en los diferentes universos, y los melquisedecs de Nebadón, desde hace mucho tiempo, la han denominado \bibemph{absolutum}. Esta materia originaria del Paraíso no está ni muerta ni viva; es la expresión primigenia y no espiritual de la Primera Fuente y Centro; es \bibemph{Paraíso,} y el Paraíso no tiene duplicado.
\vs p011 2:10 Parece que la Primera Fuente y Centro ha concentrado todo el potencial absoluto de la realidad cósmica en el Paraíso como parte de su modo de liberarse a sí misma de las limitaciones de la infinitud, como medio para posibilitar la creación subinfinita e incluso la espacio\hyp{}temporal. Pero del hecho de que el universo de los universos exhiba estas cualidades no se desprende que el Paraíso esté limitado por el tiempo y el espacio. El Paraíso existe sin tiempo y no tiene ubicación en el espacio.
\vs p011 2:11 A grandes rasgos, el espacio al parecer se origina justo debajo del Paraíso inferior; el tiempo, justo encima del Paraíso superior. El tiempo, tal como vosotros lo entendéis, no caracteriza la existencia en el Paraíso, aunque los habitantes de la Isla central son plenamente conscientes de la secuencia temporal de los acontecimientos. El movimiento no es inherente al Paraíso; es volitivo. Pero el concepto de distancia, incluso de distancia absoluta, tiene gran significado al poderse aplicar a ubicaciones relativas en el Paraíso. El Paraíso no es espacial; por consiguiente sus áreas son absolutas y, por lo tanto, utilizables en modos que sobrepasan la noción de la mente mortal.
\usection{3. EL PARAÍSO SUPERIOR}
\vs p011 3:1 En el Paraíso superior se dan tres grandes ámbitos de actividad: la \bibemph{presencia de la Deidad,} el \bibemph{Ámbito Santísimo} y el \bibemph{Área Santa}. La inmensa región que rodea inmediatamente la presencia de las Deidades, el Ámbito Santísimo, se dedica a la labor de adoración, de trinitización y de elevado logro espiritual. En esta zona, no hay construcciones materiales ni creaciones puramente intelectuales; no podrían existir allí. Me resulta inútil intentar describir para la mente humana la naturaleza divina y la esplendorosa magnitud del Ámbito Santísimo del Paraíso. Este entorno es completamente espiritual; y vosotros sois casi completamente materiales. Una realidad puramente espiritual es, para un ser puramente material, aparentemente inexistente.
\vs p011 3:2 Aunque no haya materia física en el área del Santísimo, hay abundantes recuerdos de vuestros días materiales en los sectores de la Tierra Santa y los hay incluso más en las áreas históricas recordatorias del Paraíso periférico.
\vs p011 3:3 El Área Santa, la región exterior o de residencia, está dividida en siete zonas concéntricas. Al Paraíso se le llama a veces “la casa del Padre” puesto que es su lugar de eterna residencia, y a estas siete zonas se las denomina con frecuencia “las moradas del Padre en el Paraíso”. La zona interior o primera está ocupada por los ciudadanos del Paraíso y los originarios de Havona que puedan habitar en el Paraíso. La zona siguiente, o segunda, es la zona de residencia de los originarios de los siete suprauniversos del tiempo y del espacio. Esta segunda zona está en parte subdividida en siete inmensas divisiones. Es la residencia habitual en el Paraíso de seres espirituales y de criaturas que ascienden desde los universos evolutivos. Cada uno de estos sectores está dedicado en exclusiva al bien y al progreso de los seres personales de cada suprauniverso, pero estos complejos sobrepasan de manera casi infinita las especificaciones de los actuales siete suprauniversos.
\vs p011 3:4 Cada uno de los siete sectores del Paraíso se subdivide en unidades residenciales capaces de albergar a mil millones de equipos de seres glorificados. Mil de estas unidades constituyen una división. Cien mil divisiones corresponden a una congregación. Diez millones de congregaciones constituyen una asamblea. Mil millones de asambleas componen una gran unidad. Y esta serie ascendente continúa a través de la segunda gran unidad, la tercera y así sucesivamente hasta la séptima gran unidad. Siete de las grandes unidades componen las unidades mayores, y siete unidades mayores constituyen una unidad superior. De este modo, en agrupaciones de siete, las series ascendentes se expanden a través de las unidades superiores, suprasuperiores, celestiales y supracelestiales, hasta las unidades supremas. Pero incluso esto no llega a ocupar todo el espacio disponible. Este asombroso número de residencias del Paraíso, un número que rebasa vuestro entendimiento, ocupa mucho menos del uno por ciento del área asignada de Tierra Santa. Aún hay lugar de sobra para los que están de camino al interior, e incluso para los que no comenzarán la ascensión al Paraíso sino hasta los tiempos del futuro eterno.
\usection{4. EL PARAÍSO PERIFÉRICO}
\vs p011 4:1 La Isla central termina de forma abrupta en la periferia, pero su extensión es tan enorme que su ángulo de terminación es relativamente imperceptible desde cualquier área determinada. La superficie periférica del Paraíso está ocupada, en parte, por los recintos de llegada y salida de distintos grupos de seres personales espirituales. Puesto que las zonas no infundidas del espacio casi tocan la periferia, todos los transportes de seres personales destinados al Paraíso hacen su llegada en estas regiones. Ni el Paraíso superior ni el inferior son accesibles para los supernafines de transporte ni para otros tipos de surcadores del espacio.
\vs p011 4:2 Los siete espíritus mayores tienen su posición personal de potencia y autoridad en las siete esferas del Espíritu, que circundan el Paraíso en el espacio entre los orbes resplandecientes del Hijo y la vía de circulación interna de los mundos de Havona, pero mantienen sedes centrales de convergencia de la fuerza en la periferia del Paraíso. Aquí, las presencias de los siete directores supremos de la potencia, circundando lentamente, indican la ubicación de las siete estaciones de transmisión de determinadas energías que salen del Paraíso para los siete suprauniversos.
\vs p011 4:3 Aquí, en el Paraíso periférico, están las enormes áreas de exposición histórica y profética asignadas a los hijos creadores, y dedicadas a los universos locales del tiempo y del espacio. Hay justamente siete billones de estos enclaves históricos ya establecidos o en reserva, pero dichos emplazamientos en su conjunto ocupan solamente alrededor de un cuatro por ciento de esa porción del área periférica asignada para ello. Deducimos que estas inmensas reservas pertenecen a creaciones que en algún momento se localizarán más allá de las fronteras de los siete suprauniversos habitados actualmente conocidos.
\vs p011 4:4 Tal porción del Paraíso designada para el uso de los universos existentes está ocupada solamente entre un uno y un cuatro por ciento, en tanto que el área asignada a esta actividad es al menos un millón de veces mayor de lo que se requiere para dicho objeto. El Paraíso es lo suficientemente grande como para dar cabida a la actividad de una creación casi infinita.
\vs p011 4:5 Pero no sería de utilidad continuar intentando que imaginéis las glorias del Paraíso. Debéis esperar, y ascender mientras esperáis, porque verdaderamente “el ojo no ha visto, ni el oído ha oído, ni ha penetrado en la mente del hombre mortal, lo que el Padre Universal ha preparado para los que sobreviven a la vida en la carne de los mundos del tiempo y del espacio”.
\usection{5. EL PARAÍSO INFERIOR}
\vs p011 5:1 En cuanto al Paraíso inferior, sabemos tan solo lo que se ha revelado; los seres personales no habitan allí. No tiene nada que ver con los asuntos de las inteligencias espirituales, ni tampoco obra allí el Absoluto de la Deidad. Se nos informa de que todas las vías circulatorias de la energía física y de la fuerza cósmica tienen su origen en el Paraíso inferior, y de que este está constituido de la siguiente manera:
\vs p011 5:2 \li{1.}Directamente bajo la ubicación de la Trinidad, en la porción central del Paraíso Inferior, se encuentra la desconocida y no revelada Zona de la Infinitud.
\vs p011 5:3 \li{2.}Esta zona está inmediatamente rodeada por un área sin identificar.
\vs p011 5:4 \li{3.}En los márgenes externos de la superficie inferior hay una región relacionada mayormente con la potencia del espacio y la fuerza\hyp{}energía. La actividad de este inmenso centro elíptico de fuerza no se identifica con las funciones conocidas de ninguna triunidad, pero la carga primordial de fuerza del espacio parece que tiene su punto de convergencia en esta zona. Este centro consta de tres zonas elípticas concéntricas: la más interior es el punto de actividad de la fuerza\hyp{}energía del Paraíso mismo; la más exterior puede identificarse posiblemente con las funciones del Absoluto Indeterminado; si bien, no estamos seguros respecto a las funciones espaciales de la zona intermedia.
\vs p011 5:5 \pc \bibemph{La zona interior} de este centro de fuerza parece servir de corazón gigantesco cuyas pulsaciones dirigen las corrientes hacia las fronteras más exteriores del espacio físico. Dirige y modifica la fuerza\hyp{}energías, pero realmente no las impulsa. La realidad de la presión\hyp{}presencia de esta fuerza primordial es ciertamente mayor en el extremo norte del centro del Paraíso que en las regiones del sur; esta es una diferencia que se registra con uniformidad. La fuerza matriz del espacio parece fluir hacia dentro en el sur y hacia fuera en el norte, mediante la operación de un sistema circulatorio desconocido que se ocupa de la difusión de esta forma básica de fuerza\hyp{}energía. Periódicamente, también se manifiestan diferencias en las presiones este\hyp{}oeste. Las fuerzas que emanan de esta zona no responden a ninguna gravedad física observable, pero siempre obedecen a la gravedad del Paraíso.
\vs p011 5:6 \pc \bibemph{La zona intermedia} del centro de fuerza rodea directamente a esta área. Esta zona intermedia parece ser estática excepto que se expande y contrae a través de tres ciclos de actividad. La menor de estas pulsaciones lo hace en dirección este\hyp{}oeste; la siguiente, en sentido norte\hyp{}sur; mientras que la fluctuación más grande se hace en todas las direcciones; sigue una expansión y una contracción generalizadas. La acción de esta área intermedia nunca ha sido en verdad identificada; si bien, debe tener algo que ver con los ajustes mutuos entre las zonas interior y exterior del centro de fuerza. Muchos creen que la zona intermedia es el mecanismo de control del espacio intermedio o zonas quietas que separan los niveles espaciales consecutivos del universo matriz, pero no hay evidencia o revelación alguna que lo confirme. Dicha suposición se basa en el conocimiento de que esta área intermedia se relaciona de alguna manera con el funcionamiento de los mecanismos del espacio no infundido del universo matriz.
\vs p011 5:7 \pc \bibemph{La zona exterior} es la más grande y más activa de los tres cinturones concéntricos y elípticos del potencial espacial no identificado. Esta área es el lugar de una inimaginable actividad; es el punto central de la vía circulatoria de las emanaciones que van hacia el espacio en todas direcciones hasta las fronteras más exteriores de los siete suprauniversos y más allá, hasta extenderse a los dominios, enormes e inescrutables, de todo el espacio exterior. Esta presencia espacial es enteramente impersonal a pesar de que, de alguna manera no desvelada, parece responder indirectamente a la voluntad y mandatos de las Deidades infinitas cuando actúan como Trinidad. Se cree que este es el punto central de convergencia, el centro en el Paraíso de la presencia espacial del Absoluto Indeterminado.
\vs p011 5:8 Todas las formas de fuerza y todas las facetas de la energía parecen ser circulatorias; circulan por el universo y regresan por rutas definidas. Pero, en cuanto a las emanaciones de la zona activada del Absoluto Indeterminado, parece que la dirección fuese hacia fuera o hacia dentro ---nunca en ambas direcciones simultáneamente---. Esta zona exterior emite pulsaciones en ciclos temporales de proporciones gigantescas. Durante poco más de mil millones de años de Urantia, la fuerza espacial sale de este centro; luego, durante un período de tiempo semejante, retornará. Y las manifestaciones de la fuerza espacial de dicho centro son universales y se extienden por todas partes del espacio infundible.
\vs p011 5:9 \pc Toda fuerza física, energía y materia son una. Toda la fuerza\hyp{}energía proviene originariamente del Paraíso inferior y acabará por regresar allí tras completar su vía circulatoria en el espacio. Pero no toda organización material y energética del universo de los universos en sus estados fenoménicos actuales se originó en el Paraíso inferior; el espacio es el seno de varias formas de energía y de premateria. Aunque la zona exterior del centro de fuerza en el Paraíso es la fuente de las energías espaciales, el espacio no se origina allí. El espacio no es fuerza ni energía ni potencia. Tampoco las pulsaciones de esta zona explican la respiración del espacio; no obstante, las fases de entrada y de salida de esta zona están sincronizadas con los ciclos espaciales de expansión\hyp{}contracción de dos mil millones de años.
\usection{6. LA RESPIRACIÓN DEL ESPACIO}
\vs p011 6:1 No conocemos el mecanismo real de la respiración espacial; simplemente observamos que el espacio completo se contrae y se expande de forma alternativa. Esta respiración afecta tanto a la extensión horizontal del espacio infundido como a las extensiones verticales del espacio no infundido que existen en los inmensos reservorios espaciales por encima y por debajo del Paraíso. Para intentar imaginar la configuración del volumen de estos reservorios espaciales, podríais pensar en un reloj de arena.
\vs p011 6:2 Cuando los universos de la extensión horizontal del espacio infundido se expanden, los reservorios de la extensión vertical del espacio no infundido se contraen, y viceversa. Hay una confluencia del espacio infundido y no infundido justo bajo el Paraíso inferior. Ambos tipos de espacio confluyen allí a través de canales reguladores y trasmutadores, donde se operan cambios que hacen no infundible el espacio infundible y viceversa en los ciclos de contracción y expansión del cosmos.
\vs p011 6:3 \pc Espacio “no infundido” significa no infundido de aquellas fuerzas, energías, potencias y presencias que se sabe existen en el espacio infundido. No sabemos si el espacio vertical (reservorio) está siempre destinado a obrar como el contrapeso del espacio horizontal (universo); no sabemos si hay una intención creativa respecto del espacio no infundido; realmente sabemos muy poco acerca de los reservorios espaciales, simplemente que existen y que parecen contrabalancear los ciclos espaciales de expansión\hyp{}contracción del universo de los universos.
\vs p011 6:4 \pc Los ciclos de respiración del espacio duran en cada fase poco más de mil millones de años de Urantia. Durante una fase, los universos se expanden; durante la siguiente, se contraen. El espacio infundido se está aproximando ahora a un punto medio de la fase de expansión, en tanto que el espacio no infundido se aproxima al punto medio de la fase de contracción, y se nos ha informado de que los límites externos de ambas extensiones espaciales están ahora, en teoría, aproximadamente equidistantes del Paraíso. Los reservorios de espacio no infundido se extienden ahora en sentido vertical por encima del Paraíso superior y por debajo del Paraíso inferior, al igual que el espacio infundido del universo se extiende horizontalmente hacia fuera desde el Paraíso periférico hasta el cuarto nivel del espacio exterior e incluso más allá.
\vs p011 6:5 Durante mil millones de años en tiempo de Urantia, los reservorios espaciales se contraen mientras que el universo matriz y la actividad de fuerza de todo el espacio horizontal se expanden. Por consiguiente, hacen falta algo más de dos mil millones de años de Urantia para concluir el ciclo completo de expansión\hyp{}contracción.
\usection{7. ACTIVIDAD ESPACIAL DEL PARAÍSO}
\vs p011 7:1 El espacio no existe en ninguna de las superficies del Paraíso. Si uno “mirara” directamente hacia arriba desde la superficie superior del Paraíso, no “vería” nada sino espacio no infundido que entra o que sale, y que en este momento entra. El espacio no toca al Paraíso; solo las \bibemph{zonas quiescentes del espacio intermedio} entran en contacto con la Isla central.
\vs p011 7:2 El Paraíso es, en realidad, el núcleo inmóvil de las zonas relativamente quiescentes que existen entre el espacio infundido y el espacio no infundido. Geográficamente, estas zonas parecen ser una extensión relativa del Paraíso, pero probablemente tengan algún movimiento. Sabemos muy poco acerca de ellas, pero observamos que tales zonas de reducido movimiento espacial separan el espacio infundido del espacio no infundido. Zonas similares existieron en el pasado entre los niveles del espacio infundido, pero ahora son menos quiescentes.
\vs p011 7:3 Una sección vertical del espacio total se asemejaría ligeramente a una cruz de Malta, donde los brazos horizontales representan el espacio infundido (universo) y los brazos verticales representan el espacio no infundido (reservorio). Las áreas entre los cuatro brazos las separarían de forma semejante a como las zonas del espacio intermedio separan el espacio infundido del no infundido. Estas zonas quiescentes del espacio intermedio se van agrandando cada vez más cuanto más aumentan su lejanía del Paraíso y, finalmente, abarcan las fronteras de todo el espacio y encierran completamente tanto los reservorios espaciales como toda la extensión horizontal del espacio infundido.
\vs p011 7:4 \pc El espacio no es ni una condición subabsoluta dentro del Absoluto Indeterminado ni la presencia de este, ni tampoco es la acción del Último. Es concesión del Paraíso, y se cree que el espacio del gran universo y el de todas las regiones exteriores realmente está infundido por la ancestral potencia espacial del Absoluto Indeterminado. Desde un lugar próximo al Paraíso periférico, este espacio infundido se extiende horizontalmente hacia fuera a través del cuarto nivel espacial y más allá de la periferia del universo matriz, pero no sabemos cuánto más.
\vs p011 7:5 Si imagináis un plano en forma de V, finito pero inconcebiblemente grande, ubicado en ángulo recto respecto de las superficies superior e inferior del Paraíso, con la punta casi tangente al Paraíso Periférico, y luego imagináis ese plano en revolución elíptica alrededor del Paraíso, su revolución esbozaría aproximadamente el volumen del espacio infundido.
\vs p011 7:6 Hay un límite superior y un límite inferior del espacio horizontal con referencia a cualquier lugar dado en los universos. Si uno pudiera moverse lo bastante lejos en ángulo recto respecto del plano de Orvontón, ya sea hacia arriba o hacia abajo, se podría encontrar finalmente el límite superior o inferior del espacio infundido. Dentro de las dimensiones conocidas del universo matriz, estos límites se separan cada vez más del Paraíso; el espacio se ensancha, y se ensancha con algo más de rapidez que lo hace el plano de la creación, de los universos.
\vs p011 7:7 \pc Las zonas relativamente quietas entre los niveles del espacio, tal como la que separa a los siete suprauniversos del primer nivel del espacio exterior, son enormes regiones elípticas de quiescente actividad espacial. Estas zonas separan las inmensas galaxias que giran de forma veloz y en ordenada procesión alrededor del Paraíso. Podéis concebir el primer nivel del espacio exterior, donde incalculables universos están ahora en proceso de formación, como una inmensa procesión de galaxias que giran alrededor del Paraíso, limitadas hacia arriba y hacia abajo por las zonas quiescentes del espacio intermedio y limitadas en los márgenes interior y exterior por zonas de espacio relativamente tranquilas.
\vs p011 7:8 Un nivel espacial actúa, pues, como una región elíptica de movimiento rodeada por todas partes de una relativa falta de movimiento. Tales relaciones de movimiento y quiescencia constituyen una senda espacial curva de menor resistencia al movimiento, a la que la siguen de forma universal la fuerza cósmica y la energía emergente conforme circundan por siempre la Isla del Paraíso.
\vs p011 7:9 Esta división del universo matriz en zonas alternas, junto con el flujo alterno de las galaxias en el sentido de las manecillas del reloj y en sentido contrario, constituye un factor estabilizador de la gravedad física concebido para prevenir el aumento de la presión de la gravedad, hasta el punto de producirse actividades disruptivas o de dispersión. Esta disposición ejerce un efecto antigravitatorio y sirve de freno a velocidades, que de otro modo serían problemáticas.
\usection{8. LA GRAVEDAD DEL PARAÍSO}
\vs p011 8:1 La atracción ineludible de la gravedad mantiene realmente sujetos a todos los mundos de todos los universos de todo el espacio. La gravedad es la atracción todopoderosa de la presencia física del Paraíso. La gravedad es el filamento omnipotente en el que se enhebran todas las estrellas fulgurantes, los soles brillantes y las esferas rotatorias que constituyen el ornamento físico y universal del Dios Eterno, que es todas las cosas, llena todas las cosas y en quien consisten todas las cosas.
\vs p011 8:2 El centro y el punto de actividad de la gravedad material absoluta es la Isla del Paraíso, complementada por los cuerpos oscuros de gravedad que circundan Havona y equilibrada por los reservorios del espacio inferior y superior. Todas las emanaciones conocidas del Paraíso inferior invariable e infaliblemente responden a la atracción de la gravedad central que opera en las interminables vías que circulan por los niveles espaciales elípticos del universo matriz. Toda forma conocida de realidad cósmica tiene la curvatura de las eras, la tendencia del círculo, la oscilación de la gran elipse.
\vs p011 8:3 El espacio no responde a la gravedad, pero sirve de equilibrador de la gravedad. Sin la amortiguación del espacio, la acción explosiva sacudiría los cuerpos espaciales circundantes. El espacio infundido también ejerce una influencia antigravitatoria sobre la gravedad física o lineal; el espacio puede realmente neutralizar la acción de la gravedad aunque no puede retardarla. La gravedad absoluta es la gravedad del Paraíso. La gravedad local o lineal pertenece a una etapa eléctrica de la energía o de la materia; opera dentro del universo central, los suprauniversos y los universos exteriores, dondequiera que haya tenido lugar una adecuada materialización.
\vs p011 8:4 \pc Las numerosas formas de la fuerza cósmica, de la energía física, de la potencia universal y de las diversas materializaciones revelan tres etapas generales no perfectamente delineadas de respuesta a la gravedad del Paraíso:
\vs p011 8:5 \li{1.}\bibemph{Etapas pregravitatorias (fuerza)}. Este es el primer paso en la diferenciación de la potencia espacial en formas preenergéticas de fuerza cósmica. Este estado es análogo al concepto de la carga\hyp{}fuerza primordial del espacio, a veces llamada \bibemph{energía pura} o \bibemph{segregata}.
\vs p011 8:6 \li{2.}\bibemph{Etapas gravitatorias (energía}). Esta modificación de la carga\hyp{}fuerza del espacio se produce por la acción de los organizadores de la fuerza del Paraíso. Señala la aparición de los sistemas de energía que responden a la atracción de la gravedad del Paraíso. Esta energía emergente es originariamente neutral, pero después de posteriores metamorfosis mostrará las llamadas cualidades positivas y negativas. Designamos a estas etapas \bibemph{ultimata}.
\vs p011 8:7 \li{3.}\bibemph{Etapas posgravitatorias (potencia del universo o potencia universal)}. En esta etapa, la energía\hyp{}materia revela su respuesta a la acción de la gravedad lineal. En el universo central, estos sistemas físicos son estructuras triples conocidas como \bibemph{triata}. Estos son los sistemas maternos de la absoluta potencia de las creaciones del tiempo y del espacio. Los directores de la potencia del universo y sus colaboradores activan los sistemas físicos de los suprauniversos. Estas estructuras materiales tienen una constitución doble y se conocen como \bibemph{gravita.} Los cuerpos oscuros de gravedad que circundan Havona no son ni triata ni gravita, y su poder de atracción desvela las dos formas de gravedad física, la lineal y la absoluta.
\vs p011 8:8 \pc La potencia del espacio no está sujeta a la interacción de forma alguna de gravitación. Esta dotación primordial del Paraíso no constituye un nivel auténtico de realidad, pero antecede a todas las realidades, operativas y relativas, no espirituales ---todas las manifestaciones de la fuerza\hyp{}energía y la organización de la potencia y la materia---. La potencia espacial es un término difícil de definir. No significa aquello ancestral al espacio; su significado debe transmitir la idea de las potencias y potenciales existentes dentro del espacio. Se podría pensar, de manera aproximada, que incluye todas esas influencias y potenciales absolutos que emanan del Paraíso y constituyen la presencia espacial del Absoluto Indeterminado.
\vs p011 8:9 El Paraíso es la fuente absoluta y el eterno punto de actividad de toda la energía\hyp{}materia del universo de los universos. El Absoluto Indeterminado es el revelador, regulador y depositario de aquello que tiene al Paraíso como su fuente y origen. La presencia universal del Absoluto Indeterminado parece ser equivalente al concepto de una extensión gravitatoria infinitamente potencial, de una tensión elástica de la presencia del Paraíso. Este concepto nos ayuda a captar el hecho de que todas las cosas son atraídas hacia el Paraíso. La ilustración es cruda, pero útil, no obstante. También explica por qué la gravedad siempre actúa de forma preferente en el plano perpendicular de la masa, un fenómeno que indica las dimensiones diferenciales del Paraíso y de las creaciones que lo rodean.
\usection{9. LA SINGULARIDAD DEL PARAÍSO}
\vs p011 9:1 El Paraíso es único en el sentido de que constituye el origen primordial y el destino final de todos los seres personales espirituales. Aunque es cierto que no todos los seres espirituales menores de los universos locales están de inmediato destinados al Paraíso, el Paraíso no deja de ser la meta anhelada por todos los seres personales supramateriales.
\vs p011 9:2 \pc El Paraíso es el centro geográfico de la infinitud; no es parte de la creación universal, ni siquiera parte real del eterno universo de Havona. Por lo común, nos referimos a la Isla central como perteneciente al universo divino, pero realmente no es así. El Paraíso es una existencia eterna y exclusiva.
\vs p011 9:3 \pc En la eternidad del pasado, cuando el Padre Universal dio expresión infinita de su yo espiritual en el ser del Hijo Eterno, simultáneamente reveló la infinitud potencial de su yo no personal como Paraíso. Un Paraíso no personal y no espiritual parece haber sido la consecuencia inevitable de la voluntad y la acción del Padre que eternizó al Hijo Primigenio. De este modo, el Padre proyectó la realidad en dos facetas: la personal y la no personal, la espiritual y la no espiritual. La tensión entre ellas, frente a la voluntad de acción del Padre y del Hijo, dio existencia al Actor Conjunto y al universo central de mundos materiales y seres espirituales.
\vs p011 9:4 Cuando la realidad se diferencia entre el ser personal y lo no personal (el Hijo Eterno y el Paraíso), no es apropiado denominar “Deidad” a aquello que es no personal a menos que se determine de alguna manera. Las consecuencias materiales y energéticas de los actos de la Deidad difícilmente podrían llamarse Deidad. La Deidad puede originar muchas cosas que no son Deidad, el Paraíso no es Deidad; como tampoco es consciente de cómo el hombre mortal puede llegar a comprender dicho término.
\vs p011 9:5 \pc El Paraíso no antecede a ningún ser o entidad viva; no es creador. El ser personal y las relaciones de mente\hyp{}espíritu son \bibemph{trasmisibles,} pero el modelo no lo es. Los modelos nunca son reflejos; nunca son duplicaciones, reproducciones. El Paraíso es el absoluto de los modelos; Havona es una manifestación de estos potenciales en actualidad.
\vs p011 9:6 \pc El lugar de residencia de Dios es central y eterno, glorioso y ejemplar. Su residencia habitual constituye un bello modelo para todos los mundos sedes del universo; el universo central contiguo a su morada constituye un modelo para todos los universos en lo que se refiere a excelencia, organización y destino último.
\vs p011 9:7 El Paraíso es la sede universal de toda la actividad de los seres personales y el origen\hyp{}centro de todas las manifestaciones de fuerza\hyp{}espacio y de energía. Todo lo que ha sido, es ahora o está todavía por ser, ha procedido, procede ahora o procederá de esta morada central de los Dioses eternos. El Paraíso es el centro de toda la creación, la fuente de todas las energías y el origen primordial de todos los seres personales.
\vs p011 9:8 \pc Al fin y al cabo, para los mortales, lo más importante acerca del Paraíso eterno es el hecho de que esta morada perfecta del Padre Universal es el destino real y remoto de las almas inmortales de los hijos mortales y materiales de Dios, de las criaturas que ascienden de los mundos evolutivos del tiempo y del espacio. Todo mortal que conoce a Dios y que ha emprendido la andadura del cumplimiento de la voluntad del Padre ya se ha embarcado en la larga, en la larguísima senda hacia el Paraíso para buscar la divinidad y lograr la perfección. Y cuando un ser de origen animal llega por fin a la presencia de los Dioses del Paraíso, como lo hacen innumerables otros, habiendo ascendido desde las esferas modestas del espacio, ese logro representa la realidad de una transformación espiritual que linda los límites de la supremacía.
\vsetoff
\vs p011 9:9 [Exposición de un perfeccionador de la sabiduría encomendado para esta labor por los ancianos de días de Uversa.]
