\upaper{1}{El Padre Universal}
\author{Consejero divino}
\vs p001 0:1 El Padre Universal es el Dios de toda la creación, la Primera Fuente y Centro de todas las cosas y de todos los seres. Pensad primero en Dios como creador, después como rector y, por último, como sostenedor infinito. La verdad sobre el Padre Universal comenzó a manifestarse a la humanidad cuando el profeta dijo: “Solo tú eres Dios; no hay semejante a ti. Tú has creado el cielo y el cielo de los cielos, con todos sus ejércitos; tú los preservas y mandas. Por los Hijos de Dios fueron hechos los universos. El Creador se cubre de luz como de vestidura y extiende los cielos como una cortina”. Solamente el concepto del Padre Universal ---un solo Dios en lugar de muchos dioses--- permitió al hombre mortal comprender al Padre como creador divino y rector infinito.
\vs p001 0:2 La totalidad de los innumerables sistemas planetarios se hicieron para que, con el tiempo, los habitaran muy distintos tipos de criaturas inteligentes, de seres que pudieran conocer a Dios, recibir su afecto divino y amarlo a su vez. El universo de los universos es la obra de Dios y la morada de sus diversas criaturas. “Dios creó los cielos y formó la tierra; estableció el universo y creó este mundo no en vano; para que fuese habitado lo formó”.
\vs p001 0:3 Todos los mundos de luz reconocen y adoran al Padre Universal, al Hacedor Eterno y Sostenedor Infinito de toda la creación. Universo tras universo, las criaturas dotadas de voluntad emprenden el largo, largo viaje al Paraíso, el fascinante afán, la aventura eterna de llegar a Dios Padre. La suprema meta de los hijos del tiempo consiste en encontrar al Dios Eterno, comprender su naturaleza divina, reconocer al Padre Universal. Las criaturas conocedoras de Dios poseen una única aspiración suprema, un único deseo ardiente, que es llegar a ser, en sus propias esferas, como él es, personalmente perfecto en el Paraíso y en su esfera universal de rectitud suprema. Del Padre Universal que habita la eternidad surge el mandato supremo: “Sed vosotros perfectos, como yo soy perfecto”. Los mensajeros del Paraíso llevan este llamamiento divino con amor y misericordia a través de los tiempos y de los universos, hasta alcanzar incluso a tan modestas criaturas de origen animal como las razas humanas de Urantia.
\vs p001 0:4 Este dictado magnífico y universal por lograr con ahínco la perfección divina es el primer deber, y ha de ser la aspiración más sublime de cualquiera de las tenaces criaturas del Dios de perfección. Esta oportunidad de alcanzar la perfección divina constituye el destino último y seguro del eterno progreso espiritual de todo hombre.
\vs p001 0:5 Los mortales de Urantia tienen escasas esperanzas de ser perfectos en el sentido infinito, pero es enteramente posible para los seres humanos, comenzando como lo hacen en este planeta, alcanzar la meta celestial y divina que el Dios infinito ha dispuesto para el hombre mortal; y, cuando por fin consigan este destino, estarán, en todo lo que constituye de realización propia y de consecución mental, tan pletóricos en su esfera de perfección divina como Dios mismo lo está en su esfera de infinitud y eternidad. Quizás tal perfección no sea universal en el sentido material ni ilimitada en comprensión intelectual ni final en experiencia espiritual, pero es final y completa en todos los aspectos finitos de la voluntad divina, del estímulo hacia la perfección del ser personal y de la conciencia de Dios.
\vs p001 0:6 Ese es el auténtico significado de ese mandato divino, “Sed vosotros perfectos, como yo soy perfecto”, que por siempre alienta al hombre mortal a seguir adelante y lo atrae interiormente hacia ese anhelo fascinante y perdurable por alcanzar niveles cada vez más elevados de valores espirituales y de contenidos auténticos del universo. Esta búsqueda sublime del Dios de los universos constituye la aventura suprema de los habitantes de todos los mundos del tiempo y del espacio.
\usection{1. EL NOMBRE DEL PADRE}
\vs p001 1:1 De todos los nombres con los que se conoce a Dios Padre en el conjunto de los universos, los más frecuentes son los que hacen referencia a él como Primera Fuente y Centro del Universo. Según los universos y los distintos sectores de un mismo universo, al Padre Primero se le conoce con distintos nombres. Y estos nombres que las criaturas asignan al Creador dependen, en gran medida, del concepto que tengan de él. La Primera Fuente y Centro del Universo nunca se ha revelado a sí misma aludiendo a su nombre, sino solo a su naturaleza. Si nos consideramos hijos del Creador, resulta muy natural que acabemos por llamarle Padre. Pero este es el nombre que nosotros escogemos, y que nace de nuestra propia relación personal con dicha Primera Fuente y Centro.
\vs p001 1:2 El Padre Universal nunca impone forma alguna de reconocimiento arbitrario, de adoración ceremonial ni de servilismo a las criaturas de inteligencia y voluntad del universo. Los habitantes evolutivos de los mundos del tiempo y del espacio, por sí mismos ---en sus propios corazones---, han de reconocerlo, amarlo y adorarlo de forma voluntaria. El Creador no desea la sumisión de la libre voluntad espiritual de sus criaturas materiales por coacción o imposición. La ofrenda más especial que el hombre puede hacer a Dios consiste en dedicar, con todo afecto, su voluntad humana a hacer la voluntad del Padre; de hecho, la consagración de la voluntad de las criaturas constituye el único ofrecimiento de auténtico valor que el hombre pueda dispensar al Padre del Paraíso. En Dios, el hombre vive, se mueve y tiene su ser; no hay nada que él le pueda brindar a Dios a no ser su determinación para dejarse guiar por la voluntad del Padre, y esta decisión de las criaturas inteligentes y de voluntad de los universos constituye la realidad de esa adoración auténtica, que tanto satisface a la naturaleza amorosa del Padre Creador.
\vs p001 1:3 Una vez que en verdad os hayáis hecho conscientes de Dios, después de que descubrís realmente al majestuoso Creador y comenzáis a percibir vivencialmente la presencia moradora del divino rector, entonces, de acuerdo con vuestra lucidez y con la manera y método de revelar a Dios de los hijos divinos, encontraréis un nombre para el Padre Universal que expresará idóneamente vuestro concepto de la Primera Gran Fuente y Centro. Y así, en los diferentes mundos y en los diversos universos, el Creador se llega a conocer con numerosos apelativos; todos por afinidad significan lo mismo, pero en cuanto a palabras y signos, cada nombre representa el grado, la profundidad, de su entronización en los corazones de las criaturas de cualquier mundo.
\vs p001 1:4 \pc Cerca del centro del universo de los universos, al Padre Universal se le conoce por lo general con nombres que pueden considerarse representativos de la Primera Fuente. Más allá en los universos del espacio, los términos que se emplean para designar al Padre Universal aluden con mayor frecuencia al Centro Universal. Aún más allá en la creación estelar, tal como en el mundo sede de vuestro universo local, se le conoce como Primera Fuente Creativa y Centro Divino. En una constelación cercana, a Dios se le llama Padre de los Universos. En otra, Sostenedor Infinito y, hacia el este, Rector Divino. También se le ha designado Padre de las Luces, Don de Vida y Todopoderoso.
\vs p001 1:5 En aquellos mundos donde un hijo del Paraíso vivió su vida de gracia, a Dios se le conoce, por lo general, con algún nombre que indica relación personal, tierno afecto y devoción paternal. En la sede de vuestra constelación, se hace referencia a Dios como Padre Universal y, en los diferentes planetas de vuestro sistema local de mundos habitados, se le conoce de diversos modos tales como Padre de Padres, Padre del Paraíso, Padre de Havona y Padre Espíritu. Aquellos que conocen a Dios mediante la revelación de los ministerios de gracia de los hijos del Paraíso acaban por rendirse a la emotiva llamada y al tierno vínculo relacional entre criatura\hyp{}Creador, y hacen referencia a Dios como “Padre nuestro”.
\vs p001 1:6 En un planeta de criaturas sexuadas, en un mundo donde los seres inteligentes sienten en sus corazones un impulso, innato y emotivo, hacia la paternidad, el término Padre aparece como el nombre más elocuente y apropiado para el Dios Eterno. A él se le conoce mejor, se le reconoce más generalizadamente en vuestro planeta, Urantia, con el nombre de \bibemph{Dios}. El nombre que se le dé tiene poca importancia; lo significativo consiste en que debéis conocerlo y aspirar a ser como él. Vuestros profetas de la antigüedad en verdad lo llamaban “Dios Eterno” y hacían referencia a él como “el que habita la eternidad”.
\usection{2. LA REALIDAD DE DIOS}
\vs p001 2:1 Dios es la realidad primordial en el mundo del espíritu; Dios es la fuente de la verdad en las esferas de la mente; Dios cubre con su sombra la totalidad de los reinos materiales. Para todas las inteligencias creadas, Dios es un ser personal y, para el universo de los universos, él es la Primera Fuente y Centro de la realidad eterna. Dios no se asemeja ni a un hombre ni a una máquina. El Padre Primero es espíritu universal, verdad eterna, realidad infinita y ser personal paterno.
\vs p001 2:2 \pc El Dios Eterno es infinitamente más que una realidad idealizada o un universo que ya haya tomado forma personal. Dios no es simplemente el deseo supremo del hombre, el objetivo de la búsqueda de los mortales. Dios tampoco es un mero concepto, la potencia\hyp{}potencial de la rectitud. El Padre Universal no es sinónimo de naturaleza ni es la ley natural en estado personal. Dios es una realidad suprema, no meramente el concepto tradicional que el hombre tiene de los valores supremos. Dios no es un enfoque psicológico de los contenidos espirituales ni es “la obra más noble del hombre”. Dios puede ser cualquiera de estos conceptos o todos ellos en la mente de los hombres, pero él es más que eso. Es una persona que salva y un Padre que ama a todos los que gozan en la tierra de paz espiritual, a todos los que anhelan tener la experiencia de que su ser personal sobreviva a la muerte.
\vs p001 2:3 \pc La realidad de la existencia de Dios se demuestra en la experiencia humana mediante la inhabitación de la presencia divina, el mentor espiritual enviado desde el Paraíso para vivir en la mente humana del hombre y ayudar allí a la evolución y supervivencia eterna del alma inmortal. La presencia de este modelador divino se desvela en la mente humana mediante tres fenómenos experienciales:
\vs p001 2:4 1 La capacidad intelectual para conocer a Dios: la conciencia de Dios.
\vs p001 2:5 2 El impulso espiritual por encontrar a Dios: la búsqueda de Dios.
\vs p001 2:6 3 El anhelo de la persona por semejarse a Dios: el deseo ferviente de hacer la voluntad del Padre.
\vs p001 2:7 \pc Ni los experimentos científicos ni la pura deducción lógica de la razón podrán probar jamás la existencia de Dios. Dios es solamente cognoscible en el reino de la experiencia humana; no obstante, el auténtico concepto de la realidad de Dios resulta razonable para la lógica, plausible para la filosofía, esencial para la religión e indispensable para cualquier esperanza de supervivencia del ser personal.
\vs p001 2:8 Los que conocen a Dios experimentan el hecho de su presencia; estos mortales conocedores de Dios albergan en su experiencia personal la única prueba positiva de la existencia del Dios vivo que un ser humano puede ofrecer a sus semejantes. La existencia de Dios está completamente fuera de cualquier posible demostración, a no ser por el contacto entre la conciencia de Dios de la mente humana y la presencia de Dios en la forma del modelador del pensamiento, que mora en el intelecto de los mortales y se otorga al hombre como don gratuito del Padre Universal.
\vs p001 2:9 \pc En teoría, podéis considerar a Dios como el Creador, y él es el creador personal del Paraíso y del universo central de perfección, pero es el colectivo de los hijos creadores del Paraíso quien crea y organiza todos los universos del tiempo y del espacio. El Padre Universal no es el creador personal del universo local de Nebadón; el universo en el que vivís es creación de su hijo miguel. Pero aunque el Padre no crea personalmente los universos evolutivos, sí rige muchas de sus relaciones universales y ciertas manifestaciones de la energía física, mental y espiritual. Dios Padre es el creador personal del universo del Paraíso y, en vinculación con el Hijo Eterno, el creador de todos los demás creadores personales del universo.
\vs p001 2:10 \pc Como rector de los aspectos físicos del universo de los universos materiales, la Primera Fuente y Centro obra en los modelos de la Isla Eterna del Paraíso y, a través de este centro de gravedad absoluto, el eterno Dios ejerce cósmicamente, y por igual, una acción directiva sobre el nivel físico en el universo central y en todo el universo de los universos. Como mente, Dios obra en la Deidad del Espíritu Infinito; como espíritu, Dios se manifiesta en la persona del Hijo Eterno y en las personas de los hijos divinos del Hijo Eterno. Esta interrelación de la Primera Fuente y Centro con las Personas y los Absolutos del Paraíso de igual rango no excluye, en lo más mínimo, la acción personal \bibemph{directa} del Padre Universal sobre toda la creación y en todos sus niveles. Mediante la presencia de su espíritu fraccionado, el Padre Creador mantiene un contacto directo con sus hijos y con los universos creados.
\usection{3. DIOS ES ESPÍRITU UNIVERSAL}
\vs p001 3:1 “Dios es espíritu”. Es una presencia espiritual universal. El Padre Universal es una realidad espiritual infinita; es “el soberano, el eterno, el inmortal, el invisible y el único Dios verdadero”. Aunque seáis “linaje de Dios”, no debéis pensar que el Padre es semejante a vosotros en forma y constitución porque se diga que sois creados “a su imagen” ---estéis habitados por los mentores misteriosos enviados desde la morada central de su presencia eterna---. Los seres espirituales son reales a pesar de ser invisibles para los humanos y no estar compuestos de carne y hueso.
\vs p001 3:2 Así decía en la antigüedad el profeta: “He aquí que él pasa ante mí y yo no lo veo; se desliza y no lo advierto”. Podremos observar constantemente las obras de Dios, podremos ser extremadamente conscientes de las pruebas materiales de su majestuoso proceder, pero rara vez podremos contemplar la manifestación visible de su divinidad, ni siquiera advertir la presencia de su espíritu delegado que tiene morada humana.
\vs p001 3:3 El Padre Universal no es invisible porque quiera ocultarse de las modestas criaturas impedidas por lo material y limitadas en cuanto a dotes espirituales. Se trata de algo más: “No podrás ver mi rostro, porque ningún mortal puede verme y vivir”. Ningún hombre material puede contemplar a Dios espíritu y preservar su existencia mortal. La gloria y el resplandor espiritual de la presencia del ser personal divino resultan del todo inaccesibles para los grupos menores de seres espirituales o para cualquier orden de seres personales materiales. La luminosidad espiritual de la presencia personal del Padre es una “luz a la que ningún mortal puede acercarse; que ninguna criatura material ha visto ni puede ver”. Pero no es necesario ver a Dios con los ojos de la carne para poder percibirlo con los ojos de la fe de la mente espiritualizada.
\vs p001 3:4 \pc El Padre Universal comparte plenamente su naturaleza espiritual con su ser coexistente, el Hijo Eterno del Paraíso. De igual manera, tanto el Padre como el Hijo comparten el espíritu universal y eterno, plenamente y sin reservas, con su coigual y ser personal conjunto, el Espíritu Infinito. El espíritu de Dios es, en sí mismo y de sí mismo, absoluto; en el Hijo, incondicional; en el Espíritu, universal; y, en todos y mediante todos ellos, infinito.
\vs p001 3:5 \pc Dios es un espíritu universal; Dios es la persona universal. La suprema realidad personal de la creación finita es espíritu; la realidad última del cosmos personal es espíritu absonito. Solamente los niveles de la infinitud son absolutos y, solamente en tales niveles, hay completud de la unicidad entre la materia, la mente y el espíritu.
\vs p001 3:6 \pc En los universos, Dios Padre es, en potencia, el regidor de la materia, la mente y el espíritu. Solamente por medio de su extensa vía por la que circula el ser personal se relaciona Dios, de manera directa, con los seres personales de su inmensa creación de criaturas dotadas de voluntad, pero solo es accesible (fuera del Paraíso) a través de las entidades en las que su presencia se reparte, y que constituyen la voluntad de Dios manifestada en el exterior, en los universos. Este espíritu del Paraíso, que mora en la mente de los mortales del tiempo y alienta allí la evolución del alma inmortal de la criatura superviviente, participa de la naturaleza y divinidad del Padre Universal. Pero la mente de las criaturas evolutivas se origina en los universos locales y ha de adquirir su perfección divina al conseguir esas transformaciones experienciales, espiritualmente válidas, que resultan inevitables cuando la criatura elige hacer la voluntad del Padre que está en los cielos.
\vs p001 3:7 \pc En la experiencia interior del hombre, la mente está ligada a la materia, y una mente arraigada en lo material no puede sobrevivir a la muerte. La supervivencia se logra con esas modificaciones de la voluntad humana y esas transformaciones de la mente de los mortales por las que el intelecto, consciente de Dios, es gradualmente instruido y finalmente guiado por el espíritu. Esta evolución de la mente humana, desde su vinculación con la materia hasta la unión con el espíritu, tiene como resultado la transmutación de las facetas potencialmente espirituales de la mente mortal en las realidades morontiales del alma inmortal. Una mente humana que se subordine a la materia está destinada a hacerse cada vez más material y, por consiguiente, a sufrir finalmente la extinción de su ser personal; una mente que se entregue al espíritu está destinada a hacerse cada vez más espiritual y conseguir, por último, la unicidad con el espíritu divino, subsistente y guía, y alcanzar así la supervivencia y la existencia eterna de su ser personal.
\vs p001 3:8 Vengo del Eterno y, repetidas veces, he regresado a la presencia del Padre Universal. Sé de la realidad y del ser personal de la Primera Fuente y Centro, el Padre Eterno y Universal. Sé que, si bien el gran Dios es absoluto, eterno e infinito, también es bueno, divino y clemente. Conozco la verdad de las magníficas afirmaciones: “Dios es espíritu” y “Dios es amor”, y estos dos atributos se revelan más completamente al universo en el Hijo Eterno.
\usection{4. EL MISTERIO DE DIOS}
\vs p001 4:1 Es tal la infinitud de la perfección de Dios que hace de él un misterio para la eternidad. Y el más grande de todos los misterios impenetrables de Dios es el prodigio de su morada divina en la mente de los mortales. La manera en la que el Padre Universal reside en las criaturas del tiempo es el más profundo de todos los misterios del universo; la presencia divina en la mente del hombre es el misterio de los misterios.
\vs p001 4:2 Los cuerpos físicos de los mortales son “los templos de Dios”. A pesar de que los hijos creadores soberanos se aproximan a las criaturas de sus mundos habitados y “acercan hacia sí a todos los hombres”; aunque “están a la puerta” de la conciencia y “llaman” y se llenan de dicha al entrar en todos los que “abren las puertas de sus corazones”; aunque de cierto exista esta comunión personal e íntima entre los hijos creadores y sus criaturas, mortales, no obstante, los hombres mortales tienen algo del mismo Dios, que en realidad mora en su interior y de quien sus cuerpos son templos.
\vs p001 4:3 Cuando hayas acabado aquí, cuando tu camino haya concluido en la tierra en su forma temporal, cuando tu viaje de tribulación en la carne termine, cuando el polvo del que está hecho el tabernáculo mortal “vuelva a la tierra de dónde provino”, entonces, como se ha revelado, “el espíritu” morador “volverá a Dios que lo dio”. En el interior de cada ser moral de este planeta reside una fracción de Dios, una parte integrante de la divinidad. Aún no es tuyo por derecho propio, pero está concebido y destinado a hacerse uno contigo si sobrevives a la existencia mortal.
\vs p001 4:4 \pc Afrontar este misterio de Dios es una constante en nosotros. Nos confunde el despliegue creciente y el ilimitado panorama de la verdad de su infinita bondad, de su ilimitada misericordia, de su inigualable sabiduría y de su grandioso carácter.
\vs p001 4:5 \pc El misterio divino consiste en la intrínseca diferencia que existe entre lo finito y lo infinito, entre lo temporal y lo eterno, entre la criatura espacio\hyp{}temporal y el Creador Universal, entre lo material y lo espiritual, entre la imperfección del hombre y la perfección de la Deidad del Paraíso. El Dios de amor universal se manifiesta invariablemente en cada una de sus criaturas conforme a la plena capacidad de que estas disponen para aprehender espiritualmente las cualidades de la verdad, la belleza y la bondad divinas.
\vs p001 4:6 El Padre Universal revela a todos los seres espirituales y a todas las criaturas mortales de cualquier esfera y de cualquier mundo del universo de los universos toda la clemencia y divinidad de su propio ser que estos seres espirituales y criaturas mortales sean capaces de percibir y comprender. Dios no hace distinción de personas, ni espirituales ni materiales. La divina presencia de la que un hijo del universo disfruta en cualquier momento dado está determinada solamente por su capacidad para recibir y percibir las manifestaciones espirituales del mundo supramaterial.
\vs p001 4:7 En la experiencia espiritual del ser humano, Dios no es un misterio sino una realidad. Pero, cuando se intenta que las realidades del espíritu queden claras para las mentes físicas de orden material, aparece el misterio: misterios tan sutiles y tan profundos, que solamente la cognición, por la fe, del mortal conocedor de Dios puede lograr el milagro filosófico del reconocimiento del Infinito por parte del finito, de la percepción del Dios Eterno por parte de los mortales evolutivos de los mundos materiales del tiempo y del espacio.
\usection{5. EL SER PERSONAL DEL PADRE UNIVERSAL}
\vs p001 5:1 No permitáis que la magnitud de Dios, su infinitud, oscurezca ni eclipse su ser personal. “El que hizo el oído, ¿no oirá? El que formó el ojo, ¿no verá?”.  El Padre Universal es la cúspide del ser personal divino; él es el origen y el destino del ser personal en toda la creación. Dios es infinito y personal; es un ser personal infinito. El Padre es en verdad un ser personal, a pesar de que la infinitud de su persona lo sitúe por siempre fuera de la plena comprensión de los seres materiales y finitos.
\vs p001 5:2 Dios es mucho más que un ser personal de la manera en que la mente humana entiende el ser personal; es incluso mucho más que cualquier posible concepto de un supraser personal. Pero es completamente inútil debatir conceptos tan incomprensibles del ser personal divino con las mentes de las criaturas materiales cuyo más amplio concepto de la realidad del ser consiste en la idea y en el ideal del ser personal. El concepto más elevado que las criaturas mortales pueden concebir del Creador Universal lo constituyen los ideales espirituales contenidos en la excelsa idea del ser personal divino. Por tanto, aunque sepáis que Dios ha de ser mucho más de lo que el ser humano entiende por ser personal, sabéis, igualmente bien, que el Padre Universal no puede ser, de ningún modo, nada menos que una persona eterna, infinita, verdadera, buena y bella.
\vs p001 5:3 Dios no se oculta de ninguna de sus criaturas. Es inaccesible para tantos órdenes de seres únicamente porque “habita en una luz a la que ninguna criatura material puede acercarse”. La inmensidad y la grandiosidad del ser personal divino están más allá del alcance de la mente no perfeccionada de los mortales evolutivos. Él “mide las aguas con el hueco de su mano, mide un universo con el palmo de su mano. Es él quien está sentado sobre el círculo de la tierra, quien extiende los cielos como una cortina y los despliega como un universo para morar”. “Levantad en alto vuestros ojos, y mirad quién ha creado todas estas cosas, quién cuenta el número de sus mundos y a todos llama por sus nombres”; y, así pues, es verdad que “las cosas invisibles de Dios son en parte entendidas por medio de las cosas hechas”. Hoy día, y tal como sois, debéis discernir al Hacedor invisible mediante su múltiple y diversa creación, así como mediante la revelación y el ministerio de sus hijos del Paraíso y de sus numerosos subordinados.
\vs p001 5:4 Aunque los mortales materiales no puedan ver a la persona de Dios, deben regocijarse en la seguridad de que es una persona. Deben aceptar por la fe esa verdad que declara que tanto amó el Padre Universal al mundo que facilitó a sus modestos habitantes el eterno progreso espiritual y que él “se goza en sus hijos”. Dios no carece de ninguno de esos atributos sobrenaturales y divinos que constituyen el ser personal perfecto, eterno, amoroso e infinito del Creador.
\vs p001 5:5 \pc En las creaciones locales (exceptuando a sus delegados de los suprauniversos), Dios no se manifiesta de forma personal ni tiene residencia al margen de los hijos creadores del Paraíso, que son los padres de los mundos habitados y los soberanos de los universos locales. Si la fe de las criaturas fuera perfecta, estas sabrían con seguridad que aquel que ha visto a un hijo creador, ha visto al Padre Universal; no pretenderían ni esperarían, al buscar al Padre, ver a otro que no fuera al Hijo. El hombre mortal no puede sencillamente ver a Dios hasta que no haya conseguido su plena transformación espiritual y haya verdaderamente llegado al Paraíso.
\vs p001 5:6 La naturaleza de los hijos creadores del Paraíso no abarca todos los potenciales incondicionales propios de la absolutidad universal de la naturaleza infinita de la Primera Gran Fuente y Centro, pero el Padre Universal está en todos los aspectos presente \bibemph{de manera divina} en los hijos creadores. El Padre y sus hijos creadores son uno solo. Estos hijos del Paraíso pertenecientes al orden de Miguel son seres personales perfectos; incluso el modelo para cualquier ser personal del universo local desde la estrella resplandeciente de la mañana hasta las más modestas criaturas humanas procedentes de la evolución animal.
\vs p001 5:7 \pc En la ausencia de Dios, y a no ser por su persona magnífica y central, no habría ser personal alguno en todo el inmenso universo de los universos. \bibemph{Dios es un ser personal}.
\vs p001 5:8 \pc A pesar de que Dios es potencia eterna, presencia majestuosa, ideal supremo y espíritu glorioso, aunque sea todo esto e infinitamente más, no obstante, es verdadera y perpetuamente un Creador personal perfecto, una persona que puede “conocer y ser conocida”, que puede “amar y ser amada”, alguien que puede hacerse amigo nuestro; siempre que puedas ser conocido, tal como otros seres humanos han sido conocidos, como el amigo de Dios. Él es un espíritu real y realidad espiritual.
\vs p001 5:9 Ya que vemos al Padre Universal revelarse por medio de su universo, ya que lo percibimos habitando en sus innumerables criaturas, ya que lo contemplamos en las personas de sus hijos soberanos, ya que continuamente sentimos su divina presencia aquí y allí, cerca y lejos, no pongamos en duda ni en tela de juicio la primacía de su ser personal. A pesar de distribuirse tan extensamente, sigue siendo una verdadera persona y mantiene, en perpetuidad, una relación personal con el incontable número de criaturas suyas diseminadas por todo el universo de los universos.
\vs p001 5:10 \pc La idea del ser personal del Padre Universal constituye el concepto de Dios más profundo y auténtico legado a la humanidad, principalmente a través de la revelación. Mediante la razón, la sabiduría y la experiencia religiosa se deduce y supone el ser personal de Dios, pero no lo corroboran en su totalidad. Incluso el modelador del pensamiento interior es prepersonal. La verdad y el desarrollo de cualquier religión se establecen en proporción directa a su concepto del ser personal infinito de Dios y a su cognición de la unidad absoluta de la Deidad. La idea de una Deidad personal se erige, pues, como medida del desarrollo religioso, después de que la religión haya definido claramente primero el concepto de la unidad de Dios.
\vs p001 5:11 La religión primitiva tenía muchos dioses personales, todos hechos a la imagen del hombre. En la revelación se afirma la validez del concepto del ser personal de Dios que es una mera posibilidad en el postulado científico de una Causa Primera, y solamente se sugiere, de manera provisional, en el enunciado filosófico de la Unidad Universal. Solamente mediante un acercamiento al ser personal, se puede comenzar a comprender la unidad de Dios. Al negar el ser personal de la Primera Fuente y Centro nos quedamos únicamente con la posibilidad de elegir entre dos dilemas filosóficos: el materialismo o el panteísmo.
\vs p001 5:12 Al reflexionar sobre la Deidad, el concepto del ser personal debe despojarse de la idea de corporeidad. El cuerpo material no es imprescindible al ser personal ni en el caso del hombre ni en el de Dios. Este error sobre la corporeidad se manifiesta en los dos polos extremos de la filosofía humana. En el materialismo, puesto que el hombre pierde su cuerpo al morir, deja de existir como ser personal; en el panteísmo, puesto que Dios no tiene cuerpo, no es por tanto una persona. Es en la unión de la mente y el espíritu cuando la forma sobrehumana del ser personal progresivo tiene su razón de ser.
\vs p001 5:13 \pc El ser personal no es simplemente un atributo de Dios sino que, por el contrario, representa la totalidad de la naturaleza infinita coordinada y de la voluntad divina unificada, que se manifiesta con perfección en eternidad y en universalidad. En un sentido supremo, el ser personal constituye la revelación de Dios al universo de los universos.
\vs p001 5:14 \pc Dios, siendo eterno, universal, absoluto e infinito, no crece en conocimiento ni aumenta en sabiduría. Dios no adquiere experiencias, tal como podría el hombre finito conjeturar o comprender, pero ciertamente, dentro del marco de su propio ser personal eterno, él disfruta de una realización de sí en continua expansión en cierto modo comparable, y análoga, al logro de experiencias nuevas de parte de las criaturas finitas de los mundos evolutivos.
\vs p001 5:15 La perfección absoluta del Dios infinito le ocasionaría estar sujeto a las enormes limitaciones de la completud incondicional de la perfección, si no fuera por el hecho de que el Padre Universal, de manera directa, se hace partícipe en el afán del ser personal de toda alma imperfecta del amplio universo que busca, con la ayuda divina, ascender a los mundos espiritualmente perfectos de lo alto. Esta experiencia progresiva de los seres espirituales y de las criaturas mortales de todo el universo de los universos forma parte de la conciencia en cuanto Deidad del Padre, que se expande en el interminable círculo divino en incesante autorrealización.
\vs p001 5:16 Es verdad en el sentido literal: “En toda angustia de vosotros él es angustiado”. “En todos vuestros triunfos, él triunfa en vosotros y con vosotros”. Su espíritu divino y prepersonal es una parte real de vosotros. La Isla del Paraíso responde a todas las metamorfosis físicas del universo de los universos; el Hijo Eterno contiene todos los impulsos espirituales de la creación completa; el Actor Conjunto incluye toda la expresión mental del cosmos en expansión. El Padre Universal hace realidad, en su plena conciencia divina, toda la experiencia individual del progresivo empeño de las mentes en expansión y de los espíritus ascendentes de cualquier entidad, ser y ser personal de la totalidad de la creación evolutiva del tiempo y del espacio. Y todo esto es verdad en el sentido literal, porque “en él vivimos y nos movemos y somos”.
\usection{6. EL SER PERSONAL EN EL UNIVERSO}
\vs p001 6:1 El ser personal humano es la sombra de una imagen espacio\hyp{}temporal proyectada por el ser personal divino del Creador. Y ninguna realidad se puede comprender, de manera adecuada, examinando su sombra. Las sombras deben interpretarse en términos de su auténtica esencia.
\vs p001 6:2 \pc Dios es para la ciencia una causa, para la filosofía una idea, para la religión una persona, al igual que un amoroso Padre celestial. Dios es para el científico una fuerza primordial, para el filósofo una hipótesis de unidad, para el religioso una experiencia espiritual viva. El concepto inapropiado que el hombre tiene respecto al ser personal del Padre Universal solamente podrá mejorarse mediante el progreso espiritual del hombre en el universo, y solamente adquirirá su verdadero sentido cuando los peregrinos del tiempo y del espacio alcancen finalmente el acogimiento divino del Dios vivo del Paraíso.
\vs p001 6:3 Nunca perdáis de vista las perspectivas contrapuestas del ser personal según sea Dios o el hombre quien lo conciba. El hombre ve y comprende el ser personal partiendo de lo finito a lo infinito; Dios parte de lo infinito a lo finito. El hombre posee el tipo más modesto de ser personal, mientras que Dios el más elevado, realmente supremo, último y absoluto. Para que surgieran conceptos más avanzados sobre el ser personal divino hubo que esperar pacientemente hasta que aparecieran ideas perfeccionadas sobre el ser personal humano, en especial con la revelación ampliada de la persona divina y humana conocida mediante la vida de gracia de Miguel, el hijo creador, en Urantia.
\vs p001 6:4 \pc El espíritu divino prepersonal que mora en la mente de los mortales lleva consigo, en el hecho mismo de su presencia, la prueba válida de su existencia real, pero solo es posible aprehender el concepto del ser personal divino mediante la percepción espiritual de la experiencia religiosa personal y genuina. Toda persona, humana o divina, puede ser conocida y comprendida con total independencia de sus reacciones externas o de su presencia material.
\vs p001 6:5 Un cierto grado de afinidad moral y de armonía espiritual resulta esencial para la amistad entre dos personas; una persona amorosa difícilmente podrá revelarse a otra que esté desprovista de amor. Así, para acercarse al conocimiento del ser personal divino, todas las dotes del ser personal humano han de consagrarse totalmente a ese esfuerzo; una dedicación parcial y tibia sería inútil.
\vs p001 6:6 Conforme el hombre tenga un entendimiento más completo de sí mismo y una mayor apreciación de los valores de sus semejantes como seres personales, más anhelará conocer al Ser Personal Primigenio y, con mayor fervor, procurará, ese ser humano conocedor de Dios, semejarse al Ser Personal Primigenio. Podéis expresar vuestras opiniones sobre Dios, pero vuestra vivencia de él y en él está por encima y más allá de cualquier controversia humana y de la mera lógica intelectual. El hombre que conoce a Dios comparte sus experiencias espirituales no para convencer a los no creyentes, sino para la edificación de los creyentes y la mutua satisfacción.
\vs p001 6:7 \pc Suponer que el universo pueda llegar a conocerse, que sea inteligible, es suponer que el universo está creado por una mente y que está dirigido por un ser personal. La mente del hombre solamente puede percibir los fenómenos mentales de otras mentes, ya sean humanas o sobrehumanas. Si la persona del hombre puede experimentar el universo, es que hay una mente divina y una persona reales tras ese universo.
\vs p001 6:8 \pc Dios es espíritu, es un ser personal espiritual; el hombre también es un espíritu: es un ser personal potencialmente espiritual. Jesús de Nazaret alcanzó la plena realización de este potencial del ser personal espiritual en la experiencia humana; por consiguiente, su vida dedicada a realizar la voluntad del Padre se erige, para el hombre, como la más auténtica y ejemplar revelación de la persona de Dios. Aunque solo es posible alcanzar a comprender la persona del Padre Universal mediante una experiencia religiosa real, con la vida terrena de Jesús nos sentimos inspirados por tan perfecto ejemplo de realización y revelación del ser personal de Dios en una vivencia verdaderamente humana.
\usection{7. EL VALOR ESPIRITUAL DEL CONCEPTO DEL SER PERSONAL}
\vs p001 7:1 Cuando Jesús hablaba del “Dios vivo”, se refería a una Deidad personal ---al Padre que está en los cielos---. El concepto del ser personal de la Deidad facilita la coparticipación, favorece la adoración inteligente, suscita una reconfortante confianza. Puede existir una acción recíproca entre las cosas no personales, pero nunca coparticipación. La relación de coparticipación entre padre e hijo, al igual que entre Dios y el hombre, no se da a menos que los dos sean personas. Solamente puede haber comunión entre personas, si bien, tal comunión personal se facilita, en buena medida, gracias a la presencia de una entidad impersonal como el modelador del pensamiento.
\vs p001 7:2 El hombre no consigue su unión con Dios como si se tratara de una gota de agua que se uniera al océano. El hombre alcanza la unión divina mediante una comunión espiritual recíproca y progresiva, mediante una relación personal con el Dios personal, consiguiendo gradualmente una naturaleza divina en conformidad, inteligente e incondicional, con la voluntad divina. Una relación tan sublime solamente puede existir entre personas.
\vs p001 7:3 \pc El concepto de la verdad quizás se pueda contemplar independiente del ser personal, el concepto de la belleza quizás exista sin el ser personal, pero el concepto de la bondad divina es solamente inteligible en relación con el ser personal. Tan solo una \bibemph{persona} puede amar y ser amada. Incluso la belleza y la verdad se desligarían de la esperanza de supervivencia si no fueran atributos de un Dios personal, de un Padre amoroso.
\vs p001 7:4 \pc No podemos entender del todo cómo Dios puede ser primordial, inmutable, todopoderoso y perfecto y estar, al mismo tiempo, circundado de un universo siempre cambiante y aparentemente regulado, de un universo evolutivo y lleno de imperfecciones relativas. Pero podemos \bibemph{conocer} esa verdad en nuestra propia experiencia personal porque todos mantenemos la identidad personal y la unidad de la voluntad, pese al cambio constante tanto de nosotros mismos como de nuestro entorno.
\vs p001 7:5 La realidad última del universo no puede aprehenderse con las matemáticas ni con la lógica ni con la filosofía, sino solo con una experiencia personal en progresiva conformidad con la voluntad divina de un Dios personal. Ni la ciencia ni la filosofía ni la teología pueden corroborar la persona de Dios. Solo la experiencia personal de los hijos de la fe del Padre celestial puede llevar a efecto la auténtica comprensión espiritual del ser personal de Dios.
\vs p001 7:6 \pc En el universo, el ser personal, como concepto superior, significa identidad, conciencia de sí mismo, volición y posibilidad de revelarse a sí mismo. Y estas características conllevan además la coparticipación con seres personales diferentes e iguales, tal como la que existe en las relaciones que se dan, como personas, en las Deidades del Paraíso. Y la unidad absoluta de estas relaciones es tan perfecta que la divinidad se conoce por su indivisibilidad, por su unicidad. “El Señor Dios \bibemph{uno} es”. La indivisibilidad de su persona no interfiere con el hecho de que Dios otorgue su espíritu para que viva en el corazón de los hombres mortales. La indivisibilidad de la persona de un padre humano no impide que los hijos e hijas mortales se reproduzcan.
\vs p001 7:7 Este concepto de indivisibilidad unido al concepto de unidad significa la trascendencia del tiempo y del espacio de la Ultimidad de la Deidad; por ello, ni el espacio ni el tiempo pueden ser absolutos o infinitos. La Primera Fuente y Centro constituye esa infinitud que trasciende de forma incondicional toda mente, toda materia y todo espíritu.
\vs p001 7:8 El hecho en sí de la Trinidad del Paraíso no contraviene, en modo alguno, la veracidad de la unidad divina. Las tres personas de la Deidad del Paraíso son, a todos los efectos de la realidad del universo y en todas las relaciones de las criaturas, como una sola. Tampoco la existencia de estas tres personas eternas contraviene la veracidad de la indivisibilidad de la Deidad. Soy plenamente consciente de que no dispongo de un lenguaje apropiado para hacer comprender con claridad a la mente humana de qué manera percibimos nosotros estos problemas del universo. Pero no debéis desanimaros; ni incluso para los elevados seres personales pertenecientes a mi grupo de seres del Paraíso resultan totalmente claras todas estas cosas. Tened siempre en cuenta que estas profundas verdades referentes a la Deidad tendrán cada vez más sentido a medida que vuestra mente se haga progresivamente más espiritual, durante las sucesivas épocas en el largo ascenso del mortal al Paraíso.
\vsetoff
\vs p001 7:9 [Exposición de un consejero divino, miembro de un grupo de seres personales celestiales designado por los ancianos de días de Uversa, la sede del gobierno del séptimo suprauniverso, para supervisar aquellas partes de la revelación que siguen y que tengan que ver con asuntos que sobrepasan los límites del universo local de Nebadón. Se me ha encargado auspiciar aquellos escritos que describan la naturaleza y atributos de Dios porque represento la fuente de información más exacta disponible, para tal fin, en cualquier mundo habitado. He servido como consejero divino en los siete suprauniversos y he residido mucho tiempo en el Paraíso, el centro de todas las cosas. Muchas veces he tenido el supremo placer de estar en la inmediata presencia personal del Padre Universal. Estoy indiscutiblemente facultado para describir la realidad y la verdad sobre la naturaleza y los atributos del Padre; sé de lo que hablo.]
