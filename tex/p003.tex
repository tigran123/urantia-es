\upaper{3}{Los atributos de Dios}
\author{Consejero divino}
\vs p003 0:1 Dios está presente en todas partes; el Padre Universal gobierna el círculo de la eternidad, pero en los universos locales lo hace a través de las personas de sus hijos creadores del Paraíso. También da vida por medio de estos. “Dios nos ha dado la vida eterna, y esta vida está en sus Hijos”. Estos hijos creadores de Dios son la expresión personal de él mismo en los sectores del tiempo y para los hijos de los planetas giratorios de los universos evolutivos del espacio.
\vs p003 0:2 Los Hijos de Dios, de un más manifiesto estado personal, son claramente perceptibles por los órdenes más modestos de inteligencias creadas, compensando así la invisibilidad del Padre, que es infinito y, por consiguiente, menos perceptible. Los hijos creadores del Paraíso del Padre Universal constituyen la revelación de un ser que, de no ser así, sería invisible, invisible a causa de la absolutidad e infinitud inherentes al círculo de la eternidad y a las personas de las Deidades del Paraíso.
\vs p003 0:3 \pc La facultad de crear no es precisamente un atributo de Dios, sino más bien el conjunto de su naturaleza actuante. Y esta actividad universal creadora se manifiesta eternamente a medida que se condiciona y se regula mediante todos los atributos coordinados de la realidad divina e infinita de la Primera Fuente y Centro. Sinceramente, dudamos de que pudiera considerarse cualquier característica de la naturaleza divina como antecedente a las demás. No obstante, si este fuera el caso, entonces la naturaleza creadora de la Deidad tendría precedencia sobre cualquier otra naturaleza, actividad y atributo. Y la facultad creadora de la Deidad tiene su culminación en la verdad universal de la Paternidad de Dios.
\usection{1. LA PRESENCIA DE DIOS EN TODAS PARTES}
\vs p003 1:1 La capacidad del Padre Universal para estar presente al mismo tiempo en todas partes constituye su omnipresencia. Únicamente Dios puede estar al mismo tiempo en dos lugares o en un sinnúmero de lugares. Dios está presente de forma simultánea “arriba en el cielo y abajo en la tierra”; como exclamó el salmista: “¿Adónde me iré de tu espíritu? ¿O adónde huiré de tu presencia?”.
\vs p003 1:2 “'Yo soy Dios de cerca y también desde muy lejos', dice el Señor. '¿No lleno yo el cielo y la tierra?'”. El Padre Universal está en todo momento presente en todas las partes y en todos los corazones de su extensa creación. Él es “la plenitud de aquel que todo lo llena en todo”, y “que hace todas las cosas en todo” y, además, de tal manera es el concepto de su persona que “los cielos (el universo) y los cielos de los cielos (el universo de los universos) no lo pueden contener”. Es cierto de forma literal que Dios lo es todo y se encuentra en todo, pero ni siquiera esto constituye el \bibemph{todo} de Dios. Solamente en la infinitud se puede finalmente revelar el Infinito; nunca puede comprenderse del todo la causa por medio del análisis de los efectos. El Dios vivo es, de manera inconmensurable, más grandioso que la suma total de la creación surgida como resultado de los actos creativos de su voluntad libre e incoercible. Dios se revela en todo el cosmos, pero el cosmos nunca podrá contener ni englobar enteramente la infinitud de Dios.
\vs p003 1:3 La presencia del Padre ronda sin cesar el universo matriz. “De un extremo de los cielos es su salida, y su curso hasta el término de ellos; y nada hay que se esconda de su luz”.
\vs p003 1:4 \pc La criatura no solo existe en Dios, sino que Dios vive también en la criatura. “Conocemos que permanecemos en él porque él vive en nosotros; él nos ha dado su espíritu. Este don del Padre del Paraíso es el compañero inseparable del hombre”. “Es el Dios siempre presente que todo lo impregna”. “Está oculto el espíritu del Padre perpetuo en la mente de todos los hijos mortales”. “El hombre va en búsqueda de un amigo, aunque ese mismo amigo vive en su propio corazón”. “No está lejano el verdadero Dios; forma parte de nosotros; su espíritu nos habla desde nuestro interior”. “El Padre vive en el Hijo. Siempre está Dios con nosotros. Él es el espíritu que nos guía al destino eterno”.
\vs p003 1:5 Se ha dicho con verdad de la raza humana: “Sois de Dios” porque “aquel que permanece en el amor permanece en Dios, y Dios en él”. De hecho, en la maleficencia atormentáis al don de Dios que mora en vosotros, porque el modelador del pensamiento ha de sufrir las consecuencias de los pensamientos errados junto con la mente humana, su lugar de confinamiento.
\vs p003 1:6 \pc La omnipresencia de Dios en realidad forma parte de su naturaleza infinita; el espacio no constituye un obstáculo para la Deidad. Dios es, en perfección y sin límites, perceptiblemente presente solamente en el Paraíso y en el universo central. No es, pues, ostensiblemente presente en las creaciones que circundan Havona, porque Dios ha limitado su presencia directa y real en reconocimiento de la soberanía y las prerrogativas divinas de sus creadores y gobernantes homólogos de los universos del tiempo y del espacio. Por ello, el concepto de presencia divina debe dar lugar a un amplio espectro de modos y cauces de manifestación, abarcando las vías por donde circula la presencia del Hijo Eterno, del Espíritu Infinito y de la Isla del Paraíso. Tampoco es posible distinguir siempre entre la presencia del Padre Universal y la acción de sus eternos homólogos y de sus instancias intermedias, debido a que cumplen perfectamente todos los imperativos infinitos de su invariable propósito. No sucede lo mismo, sin embargo, con la vía circulatoria del ser personal ni con los modeladores; en este respecto, Dios actúa de manera única, directa y exclusiva.
\vs p003 1:7 \pc El Rector Universal, presente de manera potencial en las vías circulatorias de la gravedad de la Isla del Paraíso, está en todas las partes del universo en todo momento y con la misma intensidad, según la masa, en respuesta a la exigencia física de esta presencia y debido a la propia naturaleza de toda la creación que hace que todas las cosas se adhieran a él y consistan en él. Asimismo, la Primera Fuente y Centro está presente de forma potencial en el Absoluto Indeterminado, el depositario de los universos increados del futuro eterno. Dios, por tanto, se difunde de forma potencial por los universos físicos del pasado, del presente y del futuro. Constituye el fundamento primordial de la cohesión de la llamada creación material. Este potencial no espiritual de la Deidad se actualiza por doquier en el nivel de las existencias físicas mediante la inexplicable intrusión, en el marco de acción del universo, de algunas de sus exclusivas instancias intermedias.
\vs p003 1:8 La presencia de la mente de Dios se correlaciona con la mente absoluta del Actor Conjunto, el Espíritu Infinito, pero en las creaciones finitas se le percibe mejor en la siempre presente acción de la mente cósmica de los espíritus mayores del Paraíso. Así como la Primera Fuente y Centro está de forma potencial presente en las vías del Actor Conjunto por donde circula la mente, también lo está en las tensiones del Absoluto Universal. Pero la mente del género humano constituye el don de las hijas del Actor Conjunto, las benefactoras divinas de los universos en evolución.
\vs p003 1:9 El espíritu del Padre Universal, presente en todas partes, obra en coordinación con la presencia del espíritu universal del Hijo Eterno y con el potencial divino y perpetuo del Absoluto de la Deidad. Pero ni la actividad espiritual del Hijo Eterno y de sus hijos del Paraíso ni el don de la mente que el Espíritu Infinito otorga parecen excluir la acción directa de los modeladores del pensamiento, esas fracciones de Dios que moran en el corazón de los hijos de la creación.
\vs p003 1:10 Con respecto a la presencia de Dios en un planeta, en un sistema, en una constelación o en un universo, el grado de dicha presencia en cualquier elemento creado se mide por el grado de la presencia evolutiva del Ser Supremo. Este grado está determinado por el reconocimiento colectivo de Dios y la lealtad hacia él de parte de la extensa organización del universo, incluyendo a los sistemas y a los planetas mismos. Así pues, sucede a veces que, con la esperanza de preservar y salvaguardar estas facetas de la preciosa presencia de Dios, si algunos planetas (e incluso algunos sistemas) se han sumido en profundas tinieblas espirituales, se les ha puesto, en cierto modo, en cuarentena, o se les ha privado parcialmente de comunicación con unidades mayores de la creación. Y todo esto, tal como se aplica en Urantia, es una forma espiritual de defensa de la mayoría de los mundos para evitar sufrir, en lo posible, las consecuencias del aislamiento por la enajenación de una minoría obstinada, perversa y rebelde.
\vs p003 1:11 \pc A pesar de que el Padre encauza paternalmente hacia sí a todos sus hijos ---a todos los seres personales---, su influencia sobre estos es limitada debido a la lejanía de su origen de la Segunda y Tercera Personas de la Deidad; esta influencia aumenta, no obstante, a medida que alcanzan su destino y se acercan a dichos niveles. El \bibemph{hecho} de la presencia de Dios en las mentes de las criaturas está determinado por la morada o no en ellas de una fracción del Padre, como son los mentores misteriosos. Pero su \bibemph{presencia efectiva} está determinada por el grado de colaboración que las mentes, su lugar de morada, otorguen a estos modeladores.
\vs p003 1:12 La presencia fluctuante del Padre no se debe a que Dios sea variable. El Padre no se recluye porque se le haya menospreciado; no se distancia su afecto en razón de la maleficencia de sus criaturas. Más bien, como ha dotado a sus hijos de la capacidad de elección (en lo que respecta a él mismo), son sus hijos los que, en el ejercicio de dicha facultad, determinan directamente el grado y los límites de la influencia divina del Padre en sus propios corazones y en sus almas. El Padre se nos ha dado gratuitamente sin límites y sin mostrar favoritismo. Él no hace distinción de personas, de planetas, de sistemas ni de universos. En los sectores del tiempo, solo confiere honor diferenciado a los seres personales paradisíacos del Dios Séptuplo, a los creadores homólogos de los universos finitos.
\usection{2. EL PODER INFINITO DE DIOS}
\vs p003 2:1 Todos los universos saben que “el Señor Dios todopoderoso reina”. Los asuntos de este mundo y de otros mundos están dirigidos por medios divinos. “Él hace de acuerdo con su voluntad en el ejército del cielo y en los habitantes de la tierra”. Es eternamente cierto que “no hay autoridad sino de parte de Dios”.
\vs p003 2:2 Dentro de los límites de lo que es consecuente con la naturaleza divina, es literalmente cierto que “para Dios todo es posible”. El prolongado proceso evolutivo de pueblos, planetas y universos se encuentra bajo la perfecta potestad de los creadores y administradores del universo y se despliega en consonancia con el propósito eterno del Padre Universal, avanzando en armonía y orden y en conformidad con el omnisapiente designio de Dios. Existe un único dador de la ley. Él sostiene los mundos en el espacio y hace girar los universos alrededor del interminable círculo eterno.
\vs p003 2:3 De todos los atributos divinos de Dios, su omnipotencia, en especial tal como impera en los universos materiales, es el que mejor se entiende. Considerado como fenómeno no espiritual, Dios es energía. Esta aseveración del hecho físico se basa en la verdad incomprensible de que la Primera Fuente y Centro es la causa primordial de todos los fenómenos físicos universales que se producen en la totalidad del espacio. De esta actividad divina se derivan toda la energía física y las demás manifestaciones materiales. La luz, es decir, la luz sin calor, es otra de las manifestaciones no espirituales de las Deidades. Y existe además otra forma de energía no espiritual que es prácticamente desconocida en Urantia ---aún sin descubrir---.
\vs p003 2:4 Dios dirige toda la potencia; ha trazado “un camino al relámpago”; ha establecido las vías circulatorias de toda la energía. Ha decretado el momento y el modo de manifestación de todas las formas de la energía\hyp{}materia. Y mantiene todas estas cosas a su alcance perpetuo, bajo el control gravitatorio centrado en el Paraíso inferior. La luz y la energía del Dios Eterno giran, por tanto, para siempre alrededor de su majestuoso curso, de la interminable pero ordenada sucesión de multitudes de estrellas de que se compone el universo de los universos. Toda la creación da vueltas eternamente alrededor del centro Ser Personal\hyp{}Paraíso de todas las cosas y de todos los seres.
\vs p003 2:5 La omnipotencia del Padre guarda relación con el predominio, en todo lugar, del nivel absoluto, donde las tres energías, material, mental y espiritual, son indistinguibles en inmediata proximidad a él: la Fuente de todas las cosas. La mente de las criaturas, al no ser monota del Paraíso ni espíritu del Paraíso, no es directamente receptiva al Padre Universal. Dios \bibemph{se acomoda} a la mente imperfecta de los mortales de Urantia mediante los modeladores del pensamiento.
\vs p003 2:6 \pc El Padre Universal no es una fuerza transitoria ni una potencia cambiante ni una energía fluctuante. La potencia y la sabiduría del Padre son totalmente adecuadas para hacer frente a todas las exigencias del universo. Conforme se presentan circunstancias críticas en la experiencia humana, él las tiene todas previstas y, por ello, no reacciona de manera distante ante los asuntos del universo, sino más bien de acuerdo con los dictados de su sabiduría eterna y en consonancia con los requerimientos de su entendimiento infinito. Pese a las apariencias, la potencia de Dios no obra en el universo como una fuerza ciega.
\vs p003 2:7 Se producen situaciones en las que parece que se han realizado decretos urgentes, que se han suspendido leyes naturales, que se han reconocido fallos de adaptación y que se ha hecho un esfuerzo para rectificar la situación; pero estas no son las razones. Tales conceptos de Dios son productos del grado limitado de vuestro limitado criterio, de la finitud de vuestra comprensión y del restringido alcance de vuestro análisis; tal incomprensión de Dios se debe a vuestro profundo desconocimiento de la existencia de leyes superiores en el mundo, de la magnitud del carácter del Padre, de la infinitud de sus atributos y del hecho de su libre voluntad.
\vs p003 2:8 Las criaturas planetarias, moradas del espíritu de Dios y dispersas por doquier en los universos del espacio, son casi tan infinitas en número y orden, tan distintas en inteligencia, tan limitadas y, a veces, tan toscas de mente, tan restringidas y tan delimitadas en visión, que resulta casi imposible desarrollar leyes generales que expresen de forma adecuada los atributos infinitos del Padre y, al mismo tiempo, sean hasta cierto punto comprensibles para esas inteligencias creadas. De esta manera, para vosotros, como criaturas, muchos de los actos del todopoderoso creador parecen arbitrarios, distantes, y no raras veces sin corazón y crueles. Pero de nuevo os aseguro que esto no es verdad. Las acciones de Dios tienen un propósito; son inteligentes, sensatas, benévolas y están eternamente atentas al mayor bien, no siempre de un solo ser, de una sola raza, de un solo planeta o incluso de un solo universo; sino al bienestar y mayor bien de todo aquel ser a quien personalmente le afecta, desde el más modesto al más elevado. En las épocas del tiempo, el bienestar de la parte parece a veces diferir del bienestar de la totalidad. En el círculo de la eternidad, estas aparentes diferencias no existen.
\vs p003 2:9 Todos formamos parte de la familia de Dios y, por esta razón, a veces tenemos que participar en la disciplina familiar. Muchos de los actos de Dios que tanto nos perturban y confunden son el resultado de las decisiones y los decretos finales de plena sabiduría. Y el Actor Conjunto está investido de poder para poner por obra las determinaciones de la voluntad infalible de la mente infinita, hacer cumplir las decisiones de su ser personal perfecto, cuyo análisis, visión y cuidados abarcan el bienestar eterno más elevado de toda su enorme y extensa creación.
\vs p003 2:10 Así pues, vuestro punto de vista aislado, fragmentario, finito, tosco y sumamente materialista y los límites propios de la naturaleza de vuestro ser constituyen un obstáculo tal que os impiden ver, comprender o conocer la sabiduría y la benevolencia de muchos actos divinos que os parecen cargados de una crueldad aplastante, y que parecen caracterizarse por una total indiferencia hacia el consuelo y el bienestar, hacia la felicidad planetaria y la prosperidad personal de vuestros semejantes. Debido a los límites de la visión humana y a vuestro conocimiento restringido y finita cognición malinterpretáis las intenciones de Dios y distorsionáis sus propósitos. Pero en los mundos en evolución suceden muchas cosas que no son la obra personal del Padre Universal.
\vs p003 2:11 \pc La omnipotencia divina está perfectamente coordinada con los demás atributos de la persona de Dios. El poder de Dios está, generalmente, solo limitado en su manifestación espiritual y universal por tres condiciones o situaciones:
\vs p003 2:12 \li{1.}Por la naturaleza de Dios, en particular por su amor infinito, por la verdad, la belleza y la bondad.
\vs p003 2:13 \li{2.}Por la voluntad de Dios, por su ministerio de misericordia y por su relación paternal con los seres personales del universo.
\vs p003 2:14 \li{3.}Por la ley de Dios, por la rectitud y la justicia de la Trinidad eterna del Paraíso.
\vs p003 2:15 \pc Dios es ilimitado en poder, divino en naturaleza, final en voluntad, infinito en atributos, eterno en sabiduría y absoluto en realidad. Todas estas características del Padre Universal se unifican en la Deidad y se expresan universalmente en la Trinidad del Paraíso y en los hijos divinos de esta Trinidad. Aparte de eso, fuera del Paraíso y del universo central de Havona, todo lo referente a Dios está limitado por la presencia evolutiva del Supremo, condicionado por la presencia que deviene del Último y coordinado por los tres Absolutos existenciales: el Deificado, el Universal y el Indeterminado. Y la presencia de Dios está, pues, limitada, porque tal es la voluntad de Dios.
\usection{3. CONOCIMIENTO UNIVERSAL DE DIOS}
\vs p003 3:1 “Dios sabe todas las cosas”. La mente divina es consciente y conocedora de los pensamientos de toda la creación. Su conocimiento de los acontecimientos es universal y perfecto. Las entidades divinas que de él emanan forman parte de él. Aquel que “diferencia las nubes” es también “perfecto en sabiduría”. “Los ojos del Señor están en todo lugar”. Dijo vuestro gran maestro acerca de los pequeños gorriones: “Ni uno de ellos cae a tierra sin el conocimiento de mi Padre”. Y también: “los cabellos de vuestras cabezas están contados”. “Él cuenta el número de las estrellas; a todas llama por su nombre”.
\vs p003 3:2 El Padre Universal es el único ser personal de todo el universo que conoce en realidad el número de estrellas y planetas del espacio. Todos los mundos de todos los universos están constantemente en la conciencia de Dios. Él también nos dice: “Verdaderamente he visto la aflicción de mi pueblo, he oído su clamor y conozco sus angustias”. Porque “desde los cielos mira el Señor; ve a todos los hijos de los hombres; desde el lugar de su morada mira sobre todos los habitantes de la tierra”. Todo hijo de criatura puede en verdad decir: “Él conoce mi camino; me probará, y saldré como oro”. “Dios ha escudriñado nuestro andar y nuestro descanso; ha entendido desde lejos nuestros pensamientos, y todos nuestros caminos le son conocidos”. “Todas las cosas están desnudas y abiertas a los ojos de aquel a quien tenemos que dar cuentas”. Y debería realmente confortar a todo ser humano entender que “Él conoce vuestra condición; se acuerda de que sois polvo”. Hablando del Dios vivo, Jesús dijo: “Vuestro Padre sabe de qué tenéis necesidad, incluso antes de que vosotros le pidáis”.
\vs p003 3:3 Dios posee un poder ilimitado para conocer todas las cosas; su conciencia es universal. Su vía circulatoria del ser personal abarca a todos los seres personales, y su conocimiento incluso de las criaturas más modestas se complementa, de forma indirecta, mediante la serie descendente de hijos divinos, y, de forma directa, mediante los modeladores interiores del pensamiento. Además, el Espíritu Infinito está presente, en todo momento, en todas partes.
\vs p003 3:4 No estamos totalmente seguros de si Dios elige o no conocer de antemano las ocasiones de pecado. Pero aunque Dios conociera de antemano los actos de la libre voluntad de sus hijos, este conocimiento previo no anularía en absoluto la libertad de estos. Una cosa es segura: Dios nunca está sujeto a la sorpresa.
\vs p003 3:5 \pc La omnipotencia no significa tener poder para hacer lo imposible de hacer, actuar de forma no divina. Tampoco significa la omnisciencia tener conocimiento de lo incognoscible. Pero no es fácil hacer entender estas aseveraciones a la mente finita. Las criaturas difícilmente pueden comprender el alcance y los límites de la voluntad del Creador.
\usection{4. LA FACULTAD ILIMITADA DE DIOS}
\vs p003 4:1 Que Dios se otorgue a sí mismo de forma sucesiva a los universos a medida que estos tienen su ser, de ninguna manera disminuye el potencial de energía ni la reserva de sabiduría que continúan residiendo y reposando en el ser personal central de la Deidad. El Padre nunca ha disminuido nada del potencial de fuerza, de sabiduría y de amor que posee, ni tampoco ha sido despojado de atributo alguno de su gloriosa persona por haberse dado a sí mismo sin límite a los hijos del Paraíso, a sus creaciones de menor rango y a las múltiples criaturas de estas.
\vs p003 4:2 La creación de cada nuevo universo requiere un nuevo ajuste de la gravedad; pero aunque la creación continuara indefinidamente, eternamente, incluso hasta la infinitud, de tal manera que, al final, la creación material existiera sin límites, aun así se comprobaría que el poder de control y de coordinación depositado en la Isla del Paraíso resultaría igual y adecuado para el dominio, control y coordinación de ese universo infinito. Y después de esta concesión ilimitada de fuerza y de potencia a un universo sin límites, el Infinito todavía continuaría recargado con el mismo grado de fuerza y de energía. El Absoluto Indeterminado permanecería íntegro. Dios continuaría en posesión del mismo potencial infinito, como si su fuerza, su energía y su potencia nunca se hubieran derramado para dotar a universo tras universo.
\vs p003 4:3 E igual con la sabiduría: el hecho de que la mente se distribuya profusamente a los pensantes de los mundos no empobrece de ningún modo la fuente central de sabiduría divina. A medida que se multiplican los universos y se incrementa el número de los seres de los mundos hasta límites inimaginables, aunque la mente se otorgue sin fin a estos seres de cualquier condición, el ser personal central de Dios seguirá poseyendo la misma mente eterna, infinita y plena en sabiduría.
\vs p003 4:4 El hecho de que él envíe mensajeros espirituales procedentes de sí mismo para morar en los hombres y mujeres de vuestro mundo y de otros mundos, de ninguna manera disminuye su capacidad para obrar como persona divina y todopoderosa espiritualmente; no existe en absoluto ningún límite en cuanto a la amplitud o cantidad de mentores espirituales que Dios envía y puede enviar. Este darse a sus criaturas genera, para tales mortales dotados divinamente, unas posibilidades futuras sin límites y casi inconcebibles de existencias progresivas y consecutivas. Y esta pródiga distribución de sí mismo bajo la forma y el servicio de estas entidades espirituales no disminuye, de ninguna manera, la sabiduría y la perfección de la verdad y del conocimiento que reposan en la persona del Padre omnipotente, omnisciente y de sabiduría plena.
\vs p003 4:5 \pc Para los mortales del tiempo existe un futuro, pero Dios habita la eternidad. Aunque procedo de cerca del lugar mismo donde mora la Deidad, no puedo atreverme a hablar con un entendimiento perfecto respecto a la infinitud de muchos de los atributos divinos. Únicamente una infinitud de mente puede comprender del todo una infinitud existencial y una eternidad de acción.
\vs p003 4:6 \pc Al hombre mortal le es imposible conocer la infinidad del Padre celestial. La mente finita no puede concebir tal verdad o hecho absoluto. Pero este mismo ser humano finito puede en realidad \bibemph{sentir} ---vivenciar en un sentido literal--- el efecto pleno y sin disminución del AMOR de ese Padre infinito. Este amor se puede verdaderamente experimentar y, si bien es cierto que la cualidad de tal experiencia es ilimitada, su cantidad está estrictamente limitada por la capacidad humana para la receptividad espiritual y por la capacidad recíproca para amar al Padre a su vez.
\vs p003 4:7 La apreciación finita de las cualidades infinitas trasciende en mucho la capacidad, lógicamente limitada, de la criatura debido al hecho de que el hombre mortal está creado a imagen de Dios: una fracción de la infinitud vive dentro de él. Por tanto el mayor acercamiento posible y afectuoso del hombre a Dios se realiza por y a través del amor, porque Dios es amor. Y la totalidad de esta relación única constituye una experiencia real en la sociología cósmica, en la relación entre Creador\hyp{}criatura ---el afecto entre Padre e hijo---.
\usection{5. EL GOBIERNO SUPREMO DEL PADRE}
\vs p003 5:1 En su contacto con las creaciones posteriores a Havona, el Padre Universal no ejerce su poder infinito y su autoridad final por transmisión directa sino más bien a través de sus Hijos y de los seres personales a ellos subordinados. Y Dios hace todo esto por su propia libre voluntad. Cualquiera de los poderes delegados o todos ellos, si se presentara la circunstancia, si la mente divina así lo eligiera, podrían ejercerse de forma directa; pero, por regla general, solo se lleva a cabo tal acción a raíz del incumplimiento de la confianza divina por parte de la persona en quien ha delegado. En esos momentos y ante tal omisión y dentro de los límites de la reserva del poder y del potencial divinos, el Padre efectivamente actúa de forma independiente y de acuerdo con los mandatos de su propia elección; y siempre esta elección es indefectiblemente perfecta e infinitamente sabia.
\vs p003 5:2 El Padre gobierna a través de sus Hijos; en la escala descendente de la organización de los universos existe una cadena ininterrumpida de gobernantes que acaba en los príncipes planetarios, que son los que dirigen los destinos de las esferas evolutivas de los inmensos dominios del Padre. Esta exclamación no es simplemente una expresión poética: “Del Señor es la tierra y su plenitud”. “Quita reyes y pone reyes”. “Los Altísimos gobiernan los reinos de los hombres”.
\vs p003 5:3 En el fuero interno de los humanos quizás el Padre Universal no siempre lleve a cabo su intención; pero, en la dirección y destino de un planeta, el plan divino prevalece; el propósito eterno de sabiduría y de amor triunfa.
\vs p003 5:4 Jesús dijo: “Mi Padre, que me las dio, es mayor que todos; y nadie las puede arrebatar de la mano de mi Padre”. Cuando vislumbráis las múltiples obras de Dios y contempláis la asombrosa inmensidad de su casi ilimitada creación, quizás vacile vuestro concepto de su primacía, pero no deberíais dejar de aceptar a Dios como entronizado perpetua y firmemente en el centro paradisíaco de todas las cosas y como Padre benefactor de todos los seres inteligentes. No hay sino “un Dios y Padre de todos, el cual es sobre todos y en todos;” “y él es antes de todas las cosas, y todas las cosas en él subsisten”.
\vs p003 5:5 \pc Las incertidumbres de la vida y las vicisitudes de la existencia no contradicen de ninguna manera el concepto de la soberanía universal de Dios. La vida de cualquier criatura evolutiva está acuciada por ciertas \bibemph{inevitabilidades}. Planteaos lo siguiente:
\vs p003 5:6 \li{1.}¿Es el \bibemph{coraje} ---la fuerza del carácter--- deseable? Entonces, el hombre tiene que crecer en un ambiente que lo haga luchar contra las dificultades y reaccionar ante las decepciones.
\vs p003 5:7 2 ¿Es el \bibemph{altruismo} ---el servicio del prójimo--- deseable? Entonces, la experiencia de vida tiene que proporcionar situaciones donde se encuentre la desigualdad social.
\vs p003 5:8 \li{3.}¿Es la \bibemph{esperanza} ---la grandeza de la confianza--- deseable? Entonces, la existencia humana debe enfrentarse constantemente con inseguridades e incertidumbres periódicas.
\vs p003 5:9 \li{4.}¿Es la \bibemph{fe} ---la afirmación suprema del pensamiento humano--- deseable? Entonces, la mente del hombre tiene que verse en esa situación difícil y problemática en la que siempre sabe menos de lo que cree.
\vs p003 5:10 \li{5.}¿Es deseable el \bibemph{amor a la verdad} y la disposición para seguirla allá donde conduzca? Entonces, el hombre tiene que crecer en un mundo donde el error esté presente y la falsedad sea siempre posible.
\vs p003 5:11 \li{6.}¿Es el \bibemph{idealismo} ---el concepto que aproxima a lo divino--- deseable? Entonces, el hombre tiene que esforzarse en un ambiente de bondad y de belleza relativas, en un entorno que estimule su irreprimible tendencia hacia cosas mejores.
\vs p003 5:12 \li{7.}¿Es la \bibemph{lealtad} ---la devoción al deber supremo--- deseable? Entonces, es preciso que el hombre se mantenga en medio de las posibilidades de incumplimiento y deserción. La valentía de la devoción al deber consiste en el peligro implícito de no cumplirlo.
\vs p003 5:13 \li{8.}¿Es el \bibemph{desinterés} ---la disposición para olvidarse de sí mismo--- deseable? Entonces, el hombre mortal tiene que vivir frente al incesante clamor de un ineludible yo, deseoso de reconocimientos y honores. El hombre no podría elegir con dinamismo la vida divina si no hubiese una vida propia a la que renunciar. El hombre nunca podría utilizar la rectitud como salvación si no existiera el mal potencial que exalta y diferencia el bien por contraposición.
\vs p003 5:14 \li{9.}¿Es el \bibemph{placer} ---la satisfacción de la felicidad--- deseable? Entonces, el hombre tiene que vivir en un mundo donde la alternativa del dolor y la probabilidad del sufrimiento sean unas posibilidades experienciales siempre presentes.
\vs p003 5:15 \pc En todo el universo, se considera cada unidad como parte del todo. La supervivencia de la parte depende de su cooperación con el plan y el propósito del todo, del deseo incondicional y de la disposición perfecta para hacer la voluntad divina del Padre. Un mundo evolutivo sin error (sin la posibilidad de juicio irreflexivo) sería solo un mundo sin inteligencia \bibemph{libre}. En el universo de Havona hay mil millones de mundos perfectos con habitantes perfectos, pero es preciso que el hombre evolutivo sea falible si ha de ser libre. No es posible que una inteligencia libre e inexperta sea en sus comienzos invariablemente sensata. La posibilidad del juicio erróneo (el mal) solo se vuelve pecado cuando la voluntad humana refrenda de forma consciente y adopta a sabiendas un deliberado juicio inmoral.
\vs p003 5:16 \pc La apreciación plena de la verdad, de la belleza y de la bondad es connatural a la perfección del universo divino. Los habitantes de los mundos de Havona no necesitan el potencial de los niveles de valor relativo para estimular sus elecciones; estos seres perfectos son capaces de identificar y de elegir el bien en la ausencia de situaciones morales que les sirvan de contraste y les fuercen a pensar. Pero todos estos seres perfectos son en su naturaleza moral y condición espiritual lo que son por el hecho mismo de existir. Han conseguido avanzar de forma experiencial solo dentro de su inherente condición. Mediante su propia fe y esperanza, el hombre mortal consigue incluso su condición de aspirante a la ascensión. Todo lo divino que la mente humana alcanza a comprender y que el alma humana adquiere constituye un logro experiencial; es una \bibemph{realidad} de la experiencia personal y es, por ello, su singular posesión, en contraste con la bondad y la rectitud intrínsecas a los seres personales libres de error de Havona.
\vs p003 5:17 \pc Las criaturas de Havona son valientes de forma natural, pero no valerosas en el sentido humano. Son de forma innata amables y atentas, pero apenas altruistas a la manera humana. Esperan un futuro placentero, pero no sienten la excelente esperanza de los confiados mortales que vienen de las imprevisibles esferas evolutivas. Tienen fe en la estabilidad del universo, pero son completamente ajenos a esa fe salvadora por la que el hombre mortal asciende desde su condición animal hasta las puertas del Paraíso. Aman la verdad, pero no saben nada de sus cualidades salvadoras para el alma. Son idealistas, pero han nacido así; ignoran totalmente el éxtasis de volverse de ese modo mediante el entusiasmo de la elección. Son leales, pero nunca han experimentado el estremecimiento de la emoción por la devoción inteligente e incondicional al deber frente a la tentación del incumplimiento. Son desinteresados, pero nunca han conseguido tal nivel experiencial mediante la magnífica conquista de un yo beligerante. Disfrutan del placer, pero no comprenden la dulzura del placentero escape de la posibilidad del dolor.
\usection{6. LA PRIMACÍA DEL PADRE}
\vs p003 6:1 Con desprendimiento divino, con generosidad consumada, el Padre Universal cede autoridad y delega poder, pero permanece siendo primordial; tiene su mano sobre la poderosa palanca de los acontecimientos de los reinos universales; se ha reservado todas las decisiones finales y empuña de forma infalible el cetro todopoderoso con poder de veto de su propósito eterno, con autoridad indiscutible a favor del bien y el destino de la extensa, rotatoria y siempre circundante creación.
\vs p003 6:2 La soberanía de Dios es ilimitada; es el hecho fundamental de toda la creación. El universo no era inevitable. El universo no es un accidente ni existe por sí mismo. El universo es una labor de creación y, por ese motivo, está totalmente subordinado a la voluntad del Creador. La voluntad de Dios es verdad divina, amor vivo; por tanto, las creaciones en proceso de perfección de los universos evolutivos se caracterizan por la bondad: acercamiento a la divinidad; y por el mal potencial: el distanciamiento de la divinidad.
\vs p003 6:3 \pc Todas las filosofías religiosas, antes o después, llegan al concepto de un gobierno unificado del universo, de un Dios. Las causas universales no pueden ser menores que los efectos universales. La fuente del flujo de la vida universal y de la mente cósmica tiene que estar por encima de sus niveles de manifestación. La mente humana no puede explicarse congruentemente en términos de los órdenes de existencia más modestos. La mente del hombre puede en verdad comprenderse solo al reconocer la realidad de unos órdenes superiores de pensamiento y de voluntad resolutiva. El hombre como ser moral es inexplicable a menos que se reconozca la realidad del Padre Universal.
\vs p003 6:4 El filósofo mecanicista manifiesta su rechazo a la idea de una voluntad universal y soberana, la misma voluntad soberana cuya actividad es la elaboración de las leyes del universo que él tan profundamente reverencia. ¡Qué involuntario homenaje rinde el mecanicista al creador de las leyes cuando piensa que tales leyes actúan y se explican por sí mismas!
\vs p003 6:5 Es un gran error humanizar a Dios, salvo en cuanto a la noción de los modeladores del pensamiento interiores, pero incluso eso no es tan irracional como \bibemph{mecanizar} por completo la idea de la Primera Gran Fuente y Centro.
\vs p003 6:6 \pc ¿Sufre el Padre del Paraíso? No lo sé. Con toda certeza los hijos creadores pueden sufrir y a veces sufren tal como los mortales. El Hijo Eterno y el Espíritu Infinito sufren en un sentido diferente. Creo que el Padre Universal sufre, pero no puedo entender \bibemph{cómo;} quizás sea a través de la vía circulatoria del ser personal o por medio de la individualidad de los modeladores del pensamiento y de las otras dádivas de su naturaleza eterna. Él ha dicho de las razas mortales: “En toda angustia de vosotros yo soy angustiado”. No hay duda de que siente una clemente comprensión de índole paternal; quizás en verdad sufra, pero no alcanzo a comprender la naturaleza de tal sufrimiento.
\vs p003 6:7 \pc El Gobernante eterno e infinito del universo de los universos es potencia, forma, energía, progreso, modelo, principio, presencia y realidad idealizada. Pero es más que todo esto; es personal; él ejerce una voluntad soberana, experimenta la conciencia de su divinidad, pone por obra los mandatos de una mente creativa, busca la satisfacción de realizar un propósito eterno y manifiesta el amor y el afecto de un Padre por sus hijos del universo. Y todos estos rasgos más personales del Padre se pueden entender mejor al observarlos tal como fueron revelados en la vida de gracia de Miguel, vuestro hijo creador, mientras estuvo encarnado en Urantia.
\vs p003 6:8 \pc El Dios Padre ama a los hombres; el Dios Hijo sirve a los hombres; el Dios Espíritu estimula a los hijos del universo a la aventura siempre ascendente de encontrar al Dios Padre, mediante los caminos ordenados por el Dios Hijo y por medio del ministerio de la gracia del Dios Espíritu.
\vsetoff
\vs p003 6:9 [Siendo el consejero divino encargado de exponer la revelación sobre el Padre Universal, he continuado con este enunciado de los atributos de la Deidad.]
